\chapter{Construction of a miniature epi-fluorescence microscope}

\section{Introduction}\todo{edit microscope intro}


    One of the major technological limitation in neuroscience research is recording neural activity in model animals. Traditional techniques such multi-unit recordings give excellent temporal resolution, however the spatial resolution --- as measured by the number of cells simutaneously recorded --- is limited. Moreover, it is very hard to distinguish cell subpopulations within the same region from the recording. Neural activity can also be inferred by post-hoc staining of neural activity markers, such as cfos or arc. This method give excellent spatial resolution, however the temporal resolution is very poor, where the time window of neural activity lasts from minutes to hours.

    Live calcium imaging gives the best of both method. By labelling the cell of interest with a calcium indicator, neural activity can be inferred in milli-second resolution. Hundreds of cells can be simultaneously recorded, and specific subpopulations can be distinguished by fluorescence in different colour channels. However, traditional live calcium imaging requires the animals' head firmly fixed under a microscope stage. This requirement is incompatible with most well established behaviour assays, and at the same time introduce significant stress to the animal, potentially confounding the behavioural result. Moreover, due to light scattering in the opaque brain tissue, most of the studies have focused only on cortical areas, while techniques to image deep brain tissue on a standard two-photon microscope is still under development and not widely adopted \citep{barretto12}.

    \textit{In vivo} calcium imaging in behaving animals is first demonstrated by Mark Schnitzer's group in Stanford \citep{ghosh11}. The authors constructed a miniature epifluorescence microscope which is chronically implanted in the brain to image the fluorescence from region of interest. In a follow-up paper \citep{ziv13}, the authors demonstrated that the miniature microscope can image GCaMP3 calcium signals from hippocampal \gls{ca1} place cells for more than a month. However, there has been several limitations of their design: first their design incorporates an objective lens of \SI{1}{\mm}  in diameter, which is impractical to reach deep brain tissue; second, their mini-microscope was only able to identify GCaMP signals, and therefore unable to distinguish different cells types within the population \citep{ghosh11,ziv13}. 
    
    In the current project, we aim to tackle the above mentioned limitations by building a head-mount miniature microscope which is able to image calcium signals in deep brain structures, while also able to image a separate fluorescence colour channel, allowing to distinguish difference cell type. The mini-microscope, once developed, will be used for imaging \gls{la} neurons to investigate mechanisms of fear memory encoding.


\section{Material and Methods}

\subsection{Construction of the mini-microscope}

\subsubsection{Microscope design}
Optical design of the microscope is aided with Zemax software (Zemax Development Corporation) to optimize the lens and filter configuration. The casing of the microscope is modelled using OpenSCAD software. 

\subsubsection{Lens configuration}



Potential lenses for emission light path are selected from modelling and calculation to give a working distance of less than \SI{100}{\um} in water, a magnification of about 2--6x and a focal length of less than \SI{6}{\cm}. The lenses are purchased and installed into custom-made mounts on a two-arm stereotaxic frame. The distance of the lenses are then optimized against a fibre bundle light source close to the \gls{grin} lens. A drum lens is used to collect light from the \gls{led}. The drum lens is tested in a similar manner and selected to give diverging light after \gls{grin} lens.

\subsubsection{Filter selection}
The filters are selected to cover the excitation and emission spectrum of the genetic encoded calcium sensor GCaMP6s \citep{chen13}, and further screen for high bandwidth and low overlap. The size of the filters are custom-made to fit the size constraints.

\subsubsection{Image sensor}
The image sensor are selected to have an packaged size of less than \SI{1.5 x 1.5}{\cm}. The sensor with highest sensitivity is then used.

\subsubsection{Casing}
The casing model is produced by 3D printing using PolyJet technology with VeroBlackPlus material (Stratasys). This gives a rigid, opaque and black casing with highest resolution for details.



\subsection{Implantation of the mini-microscope}

Two weeks after viral infusion, animals are anesthesized and head-fixed on a stereotaxic frame. Three screws were placed around the viral injection site for anchoring the microscope. A circular craniology of \SI{2}{\mm} was performed above the viral injection site. The dura was pierced and lifted with a fine tweezer to expose the brain. The brain is then constantly irrigated with artificial cerebral-spinal cord fluid to remove the blood. For \gls{ca1} experiments, a 27 gauge aspiration needle was used to remove cortex, to expose \gls{ca1}. The mini-microscope is then fixed on the stereotaxic frame and gradually lowered to the target coordinates. Opaque black dental acrylic was used to secure the microscope baseplate to the skull. Once the dental acrylic cured, the microscope body was detached from the baseplate and replaced with a cap. Animals were given \SI{5}{\mg\per\kg} ketoprofen for analgesia.

\subsection{In vivo mini-microscope testing}
The animals were kept in the home cage for two weeks before the first image session. This time allows the optical window to clear up. The animals were scruffed, the cap was removed and replaced with the microscope body. A typical imaging session lasts for \SI{5}{\minute}. After the imaging session the microscope body was removed, and the animal was recapped.

\subsection{Image analysis}
Individual cell calcium signals were extracted from the movie as previously described \citep{mukamel09}. Briefly, we first estimate the number of cells in the movie, and reduced the number of temporal dimension to roughly number of cells using principle component analysis. The resulting principle components were then subjected to independent component analysis, where the spatial filter for individual cells were extracted from the components, and the calcium signal of the corresponding cell was extracted from the mixing matrix. The time-course calcium signal was then aligned with behaviour recordings to identify neural activity patterns.


\section{Results}\todo{edit microscope result}

With the design from \citet{ghosh11} as a guide, we started to make our own epifluorescence mini-microscope. Currently we have constructed working prototypes weighing less than \SI{3}{\g}, and can be bounded in a \SI{25 x 16 x 11}{\mm} box. The light source is a high intensity blue \gls{led} (LXML--PB01--0023, Lumileds). The illumination light is collected by a drum lens (45--549, Edmund Optics) and then filtered by a blue bandpass filter (ET470/40x, Chroma). The filtered illumination is then reflected by a dichroic mirror (T495lpxr, Chroma) on to the sample. Fluorescence is collected by a \SI{1.8}{\mm} \gls{grin} lens (64--537, Edmund Optics), filtered with a green bandpass filter (ET525/50m, Chroma), then focused by an achromatic lens (49--277, Edmund Optics) onto a 600 tv-line analogue CMOS camera sensor (ASX340, Aptina). The analogue signal is then converted by a consumer video capture device (MyGica) at resolution of \num{720 x 576} and a frame rate of 25 frames per second.

An image of the mini-microscope is shown in Figure~\ref{f.scope}. The resolution of the microscope is better than \SI{2}{\um}, as shown in Figure~\ref{f.usaf} when it is tested against USAF resolution target (Lines in group 7 element 6 have width of \SI{2.07}{\um} and are clearly visible). We have tested the prototype on a perfused brain with \gls{gfp} signals. And as shown in \ref{f.scope-gfp}, the \gls{gfp} cells are clearly identified, with some of the neural processes visible.

To test \textit{in vivo} imaging capability of the microscope, we first implanted the microscope above the cortex, and injected \SI{150}{\ul} of fluorescein-dextran (molecular weight \SI{120}{\kilo\dalton}). The fluorescein-dextran will fill the blood vessels and have similar excitation and emission wavelength to GCaMP. As expected, after fluorescein-dextran injection, the blood vessels are clearly visible when the microscope is implanted (Figure~\ref{f.bloodvessel}).

We have first tested GCaMP6s fluorescence \textit{in vitro}. HEK--293 cells were transfected with pGP--CMV--gCAMP6s (Addgene), and imaged the next day. During imaging session, we challenged the cell with \SI{10}{\nmol} ATP, which is known to up-regulate intra-cellular \ce{Ca^2+} level \citep[\textit{e.g.}][]{lee04}. The result is shown in Figure~\ref{f.gcampinvitro}. The GCaMP6s gives minimal background but bright fluorescence when the intracellular \ce{Ca^2+} is induced.
    
To test GCaMP6s expression \textit{in vivo}, we infused AAV--syn--GCaMP6s--WPRE into CA1 hippocampus of animals. After AAV expression plateaued, we aspirated cortical tissue above the viral infusion site and implanted the microscope baseplate. After two weeks when the animals recovered and the cranial window was cleared, the microscope was re-attached to the implanted baseplate. The animal were placed in a novel environment to explore for \SI{5}{\minute}, during which GCaMP6 fluorescence were recorded. The maximum projection of the GCaMP6 fluorescence in a 5-minute session is shown in Figure~\ref{f.ca1bw}. More than 200 cells are clearly identifiable.

We used a previously established method to extract \ce{Ca^2+} signals from the movie \citep{mukamel09}. Briefly, we used principle component analysis to reduce the temporal dimension, and then independent component analysis to extract the spatial location of cells and their corresponding \ce{Ca^2+} signals. Figure~\ref{f.analysis} shows a sample independent component that represents a cell and it's activity. The extracted cells are random coloured in Figure~\ref{f.ca1rainbow}.   

The timecourse of the identified cells were mapped back to the behaviour of the animal. Figure~\ref{f.traceplot} shows \ce{Ca^2+} activity of potential place cells as they respond to specific location in the environment the animal is in.

This design of the miniature microscope incorporates an objective lens of \SI{1.8}{\mm} in diameter. This lens is both too thick and too short to reach deep brain structures such as amygdala. We have modified the design and attached a \SI{4.8}{\mm} long \SI{0.5}{\mm} diameter relay \gls{grin} lens (ILW-050-P050, GoFoton) to the objective lens. Attaching the relay lens does not significantly alter the imaging ability of the microscope, however allows the lens to reach deep brain regions without extensive damage. With this configuration, we are able to visualize activity form more than 40 cells in lateral amygdala and track them over time (Figure~\ref{f.amygdala}).


To enable us to identify different cell types in a population, we decided to add a second colour channel in the microscope. We have switched the filter set to a FITC/TRITC dual band set (Chroma 59004), and also from grayscale camera to an RGB camera chip. Figure~\ref{f.twocolour} shows the two-colour microscope against perfused brain expressing \gls{gfp} and TdTomato. Both fluorophore can be clearly seen. We have also tested the red channel \textit{in vivo}, where we infused red retrobeads (LumaFlour) in \gls{nac} and implanted the mini-microscope in \gls{la}. The retrobeads travels retrogradely, and will label amygdala neurons that have connection to \gls{nac}. These cells can be clearly identified under the mini-microscope in the red channel, with no interference to the green channel (Figure~\ref{f.twocolour.g.invivo},\ref{f.twocolour.r.invivo}). 

\section{Discussion}
