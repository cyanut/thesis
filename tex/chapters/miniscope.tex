\chapter{Construction of a miniature epi-fluorescence microscope}

\section{Introduction}

\section{Material and Methods}

\subsection{Construction of the mini-microscope}

\subsubsection{Microscope design}
Optical design of the microscope is aided using Zemax software (Zemax Development Corporation) to optimize the lens and filter configuration. The casing of the microscope is modelled using Rhinoceros 3D software (Robert McNeel \& Associates) and OpenSCAD software. 

\subsubsection{Lens configuration}
Potential lenses for emission light path are selected from modelling and calculation to give a working distance of less than \SI{50}{\um}, a magnification of about 2--5x and a focal length of less than \SI{3}{\cm}. The lenses are purchased and installed into custom-made mounts on a two-arm stereotaxic frame. The distance of the lenses are then optimized against a fibre bundle light source close to the \gls{grin} lens. A drum lens is used to collect light from the \gls{led}. The drum lens is tested in a similar manner and selected to give diverging light after \gls{grin} lens.

\subsubsection{Filter selection}
The filters are selected to cover the excitation and emission spectrum of the genetic encoded calcium sensor GCaMP6s \citep{chen13}, and further screen for high bandwidth and low overlap. The size of the filters are custom-made to fit the size constraints.

\subsubsection{Image sensor}
The image sensor are selected to have an packaged size of less than \SI{1.5 x 1.5}{\cm}. The sensor with highest sensitivity is then used.

\subsubsection{Casing}
The casing model is produced by 3D printing using PolyJet technology with VeroBlackPlus material (Stratasys). This gives a rigid, opaque and black casing with highest resolution for details.



\subsection{Implantation of the mini-microscope}

Two weeks after viral infusion, animals are anesthesized and head-fixed on a stereotaxic frame. Three screws were placed around the viral injection site for anchoring the microscope. A circular craniology of \SI{2}{\mm} was performed above the viral injection site. The dura was pierced and lifted with a fine tweezer to expose the brain. The brain is then constantly irrigated with artificial cerebral-spinal cord fluid to remove the blood. For \gls{ca1} experiments, a 27 gauge aspiration needle was used to remove cortex, to expose \gls{ca1}. The mini-microscope is then fixed on the stereotaxic frame and gradually lowered to the target coordinates. Opaque black dental acrylic was used to secure the microscope baseplate to the skull. Once the dental acrylic cured, the microscope body was detached from the baseplate and replaced with a cap. Animals were given \SI{5}{\mg\per\kg} ketoprofen for analgesia.

\subsection{In vivo mini-microscope testing}
The animals were kept in the home cage for two weeks before the first image session. This time allows the optical window to clear up. The animals were scruffed, the cap was removed and replaced with the microscope body. A typical imaging session lasts for \SI{5}{\minute}. After the imaging session the microscope body was removed, and the animal was recapped.

\subsection{Image analysis}
Individual cell calcium signals were extracted from the movie as previously described \citep{mukamel09}. Briefly, we first estimate the number of cells in the movie, and reduced the number of temporal dimension to roughly number of cells using principle component analysis. The resulting principle components were then subjected to independent component analysis, where the spatial filter for individual cells were extracted from the components, and the calcium signal of the corresponding cell was extracted from the mixing matrix. The time-course calcium signal was then aligned with behaviour recordings to identify neural activity patterns.


\section{Results}
\section{Discussion}
