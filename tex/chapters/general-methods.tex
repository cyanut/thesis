\chapter{General Methods}
\section{Mice}
Wildtype C57/BL6 $\times$ 129 F1 mice of 3-months age were used in the experiments. All animals were caged in groups of 4 or 5, with a 12-hour light/dark cycle. All experiments are performed during the light phase of the cycle. Food and water are provided \textit{ad libitum} to all animals. All procedures are approved by the Animal Care and Use Committee in the Hospital for Sick Children.

\section{Viral Infusion}
Each animal received \gls{ip} injection of atropine (\SI{0.1}{\mg\per\kg}) and chlorohydrate (\SI{400}{\mg\per\kg}) before being placed on a stereotaxic frame. An incision was made on the scalp and the skin was pulled to the side to reveal the skull. Holes were drilled above \gls{la} on the skull for micropipette injection. \SI{1.5}{\ul} of virus were injected on each side at a rate of \SI{0.12}{\ul\per\min}, aiming at \gls{la} (\gls{a/p} \SI{-1.4}{\mm}, \gls{m/l} $\pm$\SI{3.5}{\mm}, \gls{d/v} \SI{5.0}{\mm} from Bregma). Each animal received subcutaneous injection of analgesic (ketoprofen, \SI{5}{\mg\per\kg}) when injection of one side is finished. The animals were allowed recovery in home cage for 3 days before they were subjected to behaviour experiments.

\section{Auditory Fear Conditioning}
Auditory fear conditioning was done in the middle of the animals' light cycle on the 3rd day after surgery. The animals were placed in a training chamber and allowed for a 2-minute habituation. A tone (\SI{2800}{\Hz}, \SI{85}{\dB}) was played for \SI{30}{\second} and co-terminated with a 2-second shock (\SI{0.7}{\mA}). The animals were allowed to rest in their home cage until early in their next light cycle when they were sacrificed for analysis. \todo{complete methods for fear testing}
