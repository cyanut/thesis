\chapter{General Methods}

\section{Mice}

Wildtype C57/BL6 $\times$ 129 F1 mice of 3-months age were used in the experiments. All animals were caged in groups of 4 or 5, with a 12-hour light/dark cycle. All experiments are performed during the light phase of the cycle. Food and water are provided \textit{ad libitum} to all animals. All procedures are approved by the Animal Care and Use Committee in the Hospital for Sick Children.

\section{Viral Infusion}\label{methods.viralinfusion}

Each animal received \gls{ip} injection of atropine (\SI{0.1}{\mg\per\kg}) and chlorohydrate (\SI{400}{\mg\per\kg}) before being secured on a stereotaxic frame. An incision was made on the scalp and the skin was pulled to the side to reveal the skull. Holes were drilled above \gls{la} on the skull for micropipette injection. Virus was loaded into a glass micropipette and gradually lowered to target coordinate. \SI{1.5}{\ul} of virus were injected on each side at a rate of \SI{0.12}{\ul\per\min}, aiming at \gls{la} (\gls{a/p} \SI{-1.4}{\mm}, \gls{m/l} $\pm$\SI{3.5}{\mm}, \gls{d/v} \SI{5.0}{\mm} from Bregma). The micropipette was left in the brain for an extra \SI{10}{\min} before slowly retracted. The incision was sutured and treated with antibiotics. Each animal then received subcutaneous injection of analgesic (ketoprofen, \SI{5}{\mg\per\kg}) before returned to a partially heated clean cage for recovery. All behaviour experiments are conducted at least 3 days after surgery.

\section{Histology}
Placement of implants and extent of viral infections was determined by \gls{gfp} expression. After all experiments, animals were transcardially perfused with first \gls{pbs} then 4\% \gls{pfa}. The brains were disected and kept in 4\% \gls{pfa} overnight, and washed with \gls{pbs}. The brains were then sliced coronally on a vibrotome (\todo{vibrotome info}) to \SI{50}{\um} thickness. Slices containing \gls{la} were then mounted on gelatin-coated glass slices with a hardening mounting media (Permaflour\todo{permaflour info}) and assessed under an epi-flourescence microscope(Nikon\todo{Nikon info}).

\section{Behavioural Experiments}

\subsection{\Acrlong{afc}}\label{methods.afc}

The \gls{afc} chamber (Med Associates) consists of two plexiglass walls and two metal walls, with an overhead camera for recording animal behaviour and a metal grid floor for shock delivery. The chamber was cleaned with water before \gls{afc} training. The animals were placed in the chamber and allowed for a 2-minute habituation. A tone (\SI{2800}{\Hz}, \SI{85}{\dB}) was played for \SI{30}{\second} and co-terminated with a 2-second shock (\SI{0.7}{\mA}). The animals were returned to their home cage \SI{30}{\second} after the shock.

The \gls{afc} chamber is modified for testing by inserting a plexiglass board horizontally to cover the metal grids, and two more boards vertically to meet at the rear of the chamber, creating a triangular space. The testing chamber is washed with 70\% ethanol. During testing, animals were placed in the triangular space. A \SI{1}{\min} tone was played \SI{2}{\min} after the animal was put in the chamber. The amount of freezing was assessed.

\subsection{\Acrlong{cpp}}\label{methods.cpp}

 The \gls{cpp} boxes are composed of two chambers with a removable shutter that can be switched to allow animals to move between the chambers. One chamber has white walls, textured floor and is sprayed with water, the other chamber has striped walls, smooth floor and is sprayed with 2\% acetic acid.\todo{insert cpp image here} Animals were first placed in the box, allowed to travel between two chambers for \SI{10}{\min}, and the amount of time spent in each chamber is recorded as a measure for pre-conditioning preference. During training, animals received saline (\gls{ip}), and confined to one of the chambers for \SI{15}{\min}. In the next day, the animals received cocaine injection (\SI{30}{\mg\per\kg}, \gls{ip}), and paired to the other chamber for \SI{15}{\min}. The animals are kept in home cage for one day and put back to the \gls{cpp} boxes and allowed to move freely between chambers for \SI{10}{\min}. The amount of time spent in each chamber were recorded. The difference in time spent between cocaine and saline conditioned chamber were calculated as cocaine preference.

 \subsection{Animal tracing}
 \todo{Animal tracing method}
