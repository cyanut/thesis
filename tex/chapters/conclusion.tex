\chapter{General Discussion}
\section{Construction of the mini-microscope}
\subsection{advantage of current mini-microscope}
In the current project, we have constructed a miniature microscope which weighs less than \SI{3}{\gram}. The light weight of the mini-microscope allows for its implantation on mice without causing significant alterations to natural behaviour.  This miniature microscope is able to image green fluorescence at a resolution of at less \SI{2}{\um}. Using a tail vein injection of the fluorescent dye dextran-fluorescein, we are able to resolve individual red blood cells in capillary blood vessels, and measure the flow rate of the blood in the capillaries. 

Moreover, we have shown that the mini-microscope is able to record calcium transients from GCaMP-expressing neurons. Using the mini-microscope, we are able to image more than \num{200} \gls{ca1} neurons simultaneously, and identify potential place cells from the recording. Furthermore, with the addition of a thin relay \gls{grin} lens, we are able to extend the imaging capability of the mini-microscope to deep brain regions with minimal damage to the tissue. Here we have demonstrated simultaneous recording of more than \SI{40} neurons in \gls{la} while an animal undergoes auditory fear conditioning. 

In this project, we recorded calcium transients at a rate of \num{20} frames per second, but this does not represent a limit in the frame rate. With the XIMEA camera (MU9PM), the recoding rate can reach up to \num{200} frames per second at a cost of exposure and spatial resolution. A higher frame rate will allow identification of fast brain oscillations such as the \SIrange{7}{11}{\hertz} theta oscillation and potentially the fast \SIrange{40}{100}{\hertz} gamma oscillation, both of which have been shown to be important in learning and memory. Calcium imaging is especially suited for detection of brain oscillations, as these oscillation may not lead to action potentials, but create sub-threshold changes of membrane potential which would be reflected by a fluctuation of internal calcium concentrations. The ability to simultaneously image a dense ensemble of neurons allows brain oscillations to be measured in each cell, and the \gls{snr} in detecting these oscillations can be increased by averaging signals from all cells and even background fluorescence. The brain oscillation signal can then be related to neural activities to study how local neural circuits respond to the brain oscillation. 

In the current project, we have also demonstrated the ability of the mini-microscope to image in both red and green channels. The extra colour channel can be used for the identification of neural subpopulations or for gathering extra information from the brain. In our prototype we used blue light to stimulate red retrobeads and the fluorescent protein tdTomato, both of which have a broad excitation spectrum. However, this requirement is usually not necessary. If the red channel is static, it can be captured by a separate mini-microscope with an efficient excitation \gls{led} and corresponding filters just for the red fluorophore. The resulting image can be later aligned with recording of calcium transients for cell identification. 

Our design of mini-microscope costs \$\,300--\$\,1000 for each unit, which will not represent a significant expense in most neuroscience laboratories. Moreover, all components of the mini-microscope are commercially available and can be assembled with minimal tools, providing access to neuroscientists with no requirement of engineering experience. The microscope casing is 3D printable. We will make both the 3D models and relevant analysis code freely available in order to open this work to the neuroscience community.

\subsection{Advantages of using the mini-microscope to study \gls{ad}}

As a test of the usefulness of the miniature microscope, we used it to image \gls{ca1} neurons in a transgenic mouse model of \gls{ad}, TgCRND8, while the mice encoded and recalled a contextual fear memory. We were able to image more than \num{100} neurons simultaneously for each mouse, and this data enabled us to investigate hippocampal circuitry mechanisms in \gls{ad}.

The importance of the mini-microscope in understanding neural circuitry mechanisms is particularly reflected in the classifier prediction results. We have found that the \gls{nbc}, which predicts mouse behaviour by considering individual neurons, is unable to reveal the pattern completion deficit in TgCRND8 mice. This result suggests that a significant amount of information the neurons are encoding is contained in the synergy between activity of neurons. This information cannot be revealed using traditional methods such as \textit{in vivo} electrophysiological recording, where only a handful neurons can be recorded at the same time. To compensate for the scarcity of simultaneously recorded neurons, electrophysiological recordings often require experimenters to repeat the same behaviour trial many times, which may lead to changes in the animal's behaviour and a potential confound in the interpretation of the data. However in the current study, the mini-microscope has given us the freedom to choose an experimental paradigm which is well-established in the field of behavioural neuroscience. This freedom allowed us to connect the findings from the neural activity recordings to the rich findings of behavioural neuroscience. This stands in contrast to \textit{in vivo} 2-photon imaging, where due to the requirement of head fixation, only specifically designed behavioural paradigms are compatible, whose neural mechanisms are less well understood. 


\section{Examining circuitry deficits in a mouse model of \gls{ad}}

\subsection{\gls{ca1} hyperexcitability}
We found that in TgCRND8 mice, \gls{ca1} neurons are more active than those in \gls{wt} mice, both during context exposure and during contextual memory recall. \tglu{} treatment during training is able to reduce the overall cell activity to \gls{wt} level. Moreover, to control for the behavioural state of the mice, we investigated cell activity when the mice were both freezing and active. We found the TgCRND8 genotype has a significant effect on increasing cell activity during freezing, and \tglu{} treatment reduces cell activity in both \gls{wt} and TgCRND8 mice. When mice were not freezing during the memory test, only TgCRND8 mice treated with \tglu{} has a significant decrease in cell activity compared to the other three groups. 

Our results confirm previous findings that \gls{ca1} cells in \gls{ad} are hyperactive \citep{palop16}. We have also found that while during memory test, the average cell activity in \gls{wt} mice is at the same level as it is during contextual exposure, cell activity in the TgCRND8 mice increases during memory testing compared to the memory encoding session (Figure~\ref{f.ad.actdiff}). This result suggests that in addition to the hyperactivity introduced by \gls{ad} pathology, the TgCRND8 mice show additional activity in response to the memory recall task. This finding parallels evidence from human patients with early \gls{ad}, where \gls{fmri} studies have found hyperactivity in the hippocampus only during memory recall, and that this negatively correlates with memory performance \citep{sperling09, reiman12, kunz15}. It has been hypothesized that this task-dependent increase of hippocampal cell activity may be a compensatory mechanism in response of degraded hippocampal function in \gls{ad} \citep{kunz15}.

While hyperexcitability in \gls{ad} mice is present when the mouse is freezing, cell activity when the mouse is not freezing does not differ between groups. Considering that the majority of \gls{ca1} cells lower their activity during freezing, it is possible that the hyperactivity of \gls{ca1} neurons is caused by an inability to keep silent. This is congruent with the \gls{nmdar} pathology seen in \gls{ad}, where \glspl{nmdar} are found to spontaneously activate in \gls{ad}, and cause excitation in the neuron \citep{danysz12}, \todo{you may want to take out this final part of the sentence (from "though" to the end.  I don't think it necessarily contributes to the point you are making.)}though this spontaneous activity of the neuron can be masked by external input. This result is also hinted at by \citet{cheng13}, who investigated spatial encoding in another mouse model of \gls{ad}. In this study, when a mouse was in the receptive field of a place cell, the cell was normally active. However when the mouse was outside the receptive field of the place cell, the cell was unable to stop firing, leading to enlarged place fields \citep{cheng13}. 

It is somewhat counter-intuitive that the \tglu{} treatment, which increases synaptic \gls{ampar} density and therefore strengthens excitatory synapses, results in a decrease of overall neural activity in \gls{ca1}. However, considering that the hyperexcitability in \gls{ad} is hypothesized to be the result of increased \gls{ltd} and decreased \gls{ltp} in the synapse (discussed in Section~\ref{ad.synaptic}), and \tglu{} has been shown to block \gls{ltd}, it is possible that \tglu{} treatment rescues excitability by reversing the \gls{ltp} -- \gls{ltd} imbalance. It is still unclear how an \gls{ltp}--\gls{ltd} imbalance leads to hyperexcitability. Computational models of \gls{ad} neurons suggest this may be caused by a loss of synaptic spines in hippocampal neurons. This morphological change of neuronal dendrites in \gls{ad} creates less hindrance for the transmission of incoming \glspl{epsp}, therefore allowing the neurons to be more excitable \citep{siskova14}. It is possible that \tglu{} rescues hyperexcitability by restoring the morphology of the neurons. There is a close correlation between synaptic \gls{ampar} density and spine size, and endocytosis of GluA2-containing \glspl{ampar} can trigger changes in dendritic spine size \citep{hanley08}. Indeed, in unpublished data from this project \todo{include spine data?}, we have found \tglu{} treatment protects from spine density decreases in both \gls{ca1} and \gls{dg} after an acute expression of \gls{app} \todo{cite?}, supporting the possibility that \tglu{} rescues hyperexcitability by restoring normal neuronal morphology in \gls{ad}. 

The \tglu{} treatment was applied to TgCRND8 mice only \SI{1}{\hour} before contextual fear training, however the effect of \tglu{} was present both during contextual fear training and during memory testing. This suggests that \tglu{} starts to affect cell excitability within \SI{1}{\hour}, and has long-term effects. A closer investigation of cell activity during training shows that, while \tglu{} is able to correct average cell activity in TgCRND8 mice to \gls{wt} level, the distribution of cell activity between Tg-\tglu{} and WT-Veh is still significantly different. From the cumulative plot (Figure~\ref{f.ad.acttrain}), the distribution of the lower \SI{90}{\percent} of cell activity in Tg-\tglu{} is similar to WT-Veh, however activity of the most active \SI{10}{\percent} of cells are similar to Tg-Veh. This difference is not present \SI{24}{\hour} later during memory testing. This suggests that the rescuing effect of \tglu{} is a gradual process which take hours to complete. Cells that are less hyperactive are first rescued, and cells with very high activity require longer time for the effect of \tglu{} to appear. Given that cells close to amyloid plaques are more excitable \citep{busche12}, it is possible these cells with very high activity are located near amyloid plaques, have more damage from the \gls{ad} pathology, and require longer time for the pathological process to reverse. 


\subsection{Contextual fear memory}

In this project, we subjected \gls{wt} and TgCRND8 mice to contextual fear conditioning, and found that while the TgCRND8 mice show inferior memory during testing, this memory deficit can be rescued by \tglu{} treatment either during training, or during exposure to a brief reminder. This result suggests that TgCRND8 mice are still able to encode the memory during training, however have deficits in recalling the memory during testing. Similar results have been reported previously. In another mouse model of early \gls{ad}, APP/PS1, \citet{roy16} tagged cells activated during contextual fear conditioning, and optically activated these cells during memory recall. They found that while the APP/PS1 mice showed a memory deficit during memory recall, light stimulation of the contextual fear memory trace was able to induce memory expression. 

In order to distinguish whether the memory recall deficit is due to initiation or maintenance of memory expression, we have also analyzed the number of freezing bouts and duration of freezing bouts in TgCRND8 and \gls{wt} mice. We found that while the TgCRND8 mice do not show a significant difference in number of freezing bouts during memory test, the duration of these freezing bouts are significantly shorter than those of the \gls{wt} mice. This result suggests that the TgCRND8 mice are still able initiate the memory recall, however are unable to maintain it. 

\subsection{Encoding}

In the current study, we have found that cell activity in TgCRND8 mice is not as predictive of fear memory expression as is activity in \gls{wt} mice. On the other hand, this deficit is not reflected in the prediction accuracy of the machine learning classifiers: both \gls{nbc} and \gls{gsvm} are able to predict freezing from cell activity in TgCRND8 mice as well as in \gls{wt} mice. This discrepancy of the prediction power suggests that the hippocampal neural circuit is highly redundant, such that even when the prediction power of individual neurons is negatively affected by \gls{ad}, there is no significant information loss if multiple neurons are combined to make a prediction. Similar results are found in place encoding, where while individual neurons in a \gls{ad} mouse model have degraded spatial encoding, the ensemble is still able to accurately represent the mouse's position \citep{cheng13}.  

In this study, we only recorded close to one hundred cells in \gls{ca1} for each mouse on average, and the information in the activity of these neurons is enough for an accurate prediction of the mouse's fear memory recall. There are more than \num{1e4} neurons in \gls{ca1}: therefore even if under \gls{ad} pathology where individual cells' firing patterns are degraded, any downstream brain structure should still able to decode the information. This then suggests that the degradation of individual cell activity in \gls{ad} is secondary to the cognitive deficit, since to an efficient downstream brain structure, redundancy in the brain can protect information loss due to the degradation of cell activity. 

This conclusion also explains why in the preclinical population, significant \gls{ad} pathology often does not impact cognitive performance (Discussed in Section~\ref{preclinical}). Another important implication of this conclusion is that treatment of the \gls{ad} pathology alone, without any restoration of the neural network structure, will not result in a significant improvement in cognitive symptoms. This may explain the failure of recent attempts at treating \gls{ad} with \abeta{} clearance: it is possible that even with the successful \abeta{} removal, additional intervention is still required to restore the neural network at a circuitry level.

The importance of neural network structure is also reflected in the difference of \gls{gsvm} and \gls{nbc} performance. The significantly improved performance of the \gls{gsvm} suggests that the network encodes more information than individual cells. This may be a universal phenomenon across the brain, as it has also recently been found in other brain regions such as \gls{bla} during auditory fear conditioning \citep{grewe17}. Moreover, we find the accuracy of the \gls{gsvm}, which predicts freezing based on the cell assemble as whole, is again similar to \gls{wt} mice. This suggest that when a mouse is freezing, the information contained in the activity of the neural network in \gls{ad} mice is unaffected. 

\subsection{Pattern completion}

Given that the neural network contains enough information about freezing in \gls{ad} mice, it is intriguing t consider the source of the memory deficit. During memory expression, the information content for the cell ensemble in \gls{ad} is not different from that of the \gls{wt} mice, and it is therefore possible the memory deficit can be detected outside of the duration where the memory is expressed. Given that the memory deficit in \gls{ad} is a deficit in memory recall, and that the process of pattern completion in hippocampus is theorized to be important in memory recall, we investigated whether this process is affected in the TgCRND8 mice.

To detect the pattern completion process, we aligned classifier prediction accuracy to the behavioural change point of the mice (Figure~\ref{f.ad.into_f}). We found a significant drop of prediction accuracy just prior to freezing, both in the \gls{nbc} and \gls{gsvm}. Since the drop in accuracy happens at a time where mice are predominantly not freezing, the accuracy drop suggests that the classifiers are (inaccurately) predicting freezing, before mice start to freeze. 

Since the classifiers are trained on each time point shuffled, they are agnostic to the temporal dynamics of cell activity. Therefore, the significant change of prediction accuracy before freezing must reflect changes in the pattern of neural activity itself. Moreover, the fact that the change in neural activity in \gls{ca1} is present before behaviour onset is very important in interpreting findings of the this study. Given that nature of the study is correlational, the causal relationship between neural activity and behaviour is unclear. However, the temporal precedence of \gls{ca1} activity suggests that the neural activity pattern is not a result of the mouse's behaviour,  but likely involved in initiating the behaviour. This confirms that the neural activity difference we have found between treatment and genotype groups is a result of circuitry deficit, instead of a consequence of different behaviour between mice. 

We have found in \gls{nbc} prediction, the temporal signature of freezing behaviour in \gls{wt} animals precedes the other three groups. Given that the \gls{nbc} considers each cell individually, this result suggests that the activity of individual cells start to form a ``cellular signal'' for freezing, which is a measurement of the contextual fear memory recall. The findings that all groups in \gls{nbc} predict freezing before behaviour, and that they have similar temporal precedence over behaviour suggest that in \gls{tg} mice, the ``cellular signal'' for contextual fear memory recall is unaffected. 

On the other hand, we found that the \gls{gsvm} prediction precedes freezing in \gls{wt} mice, however this is not present in the vehicle-treated \gls{ad} mice. Again, given that the \gls{gsvm} detects a network pattern of cell activity, this result suggests that in \gls{wt} mice, a ``network signal'' of freezing appears in the \gls{ca1} network, however this signal is missing from the \gls{ad} mice before freezing. 

As the \gls{gsvm} can accurate classify the behavioural state of mice based on their neural activity, we can use the \gls{gsvm} to approximate the neural activity pattern of recalling a contextual fear memory (considering each freezing bout as an episode of memory recall). The gradual rise of freezing prediction before the freezing behaviour therefore represents the process of pattern completion: where an incomplete freezing pattern first appears in the network and confuses the classifier's attempt to predict freezing. Over time, the pattern gets more complete, which leads to increased classifier prediction of freezing. 

It is worth noting that while computational studies often model pattern completion using a stationary pattern \citep{rolls13}, the difference in prediction accuracy between \gls{nbc} and \gls{gsvm}, as well as the different temporal dynamics of the prediction precedence suggest that the neural correlates of a contextual fear memory recall is not a stationary state, since otherwise the performance of \gls{nbc} and \gls{gsvm} should be very similar, and have similar temporal dynamics in pattern completion. \todo{This is a pretty long sentence!  You should try and break it up} Therefore, our result suggests that even in an apparently ``simple'' memory such as the contextual fear memory, the neural correlates are dynamic: the state of neural activity during memory recall can move between multiple states.

We found that the ``cellular signal'', as detected by the \gls{nbc}, appears significantly earlier than the ``network signal'' from the \gls{gsvm}. This result reveals some details about the pattern completion process. The pattern completion process starts with individual cells changing their firing rate to that representing their activity distribution during memory expression, potentially guided by feed-forward input. However, as the memory pattern is a collection of patterns, at any single point each cell may have the activity from different patterns. Therefore at this time point, \gls{nbc}, which only classifies by examining whether the activity of individual neurons represents \textit{any} single contextual fear memory pattern \todo{This sentence seems like it is lacking a conclusion. You set up an idea but you do not reolve it}. However as the cell activities are not synchronized, no significant global pattern can be detected by \gls{gsvm}. 

However, as the ``cellular signal'' grows stronger, some cell activities become synchronized, and form a partial global pattern. This global pattern then, potentially through recurrent connections, recruits more cells to join the pattern. At this time, the partial global pattern can be detected by the \gls{gsvm}, and this positive feedback loop continues until the pattern is complete. This pattern then forms an attractor of brain states, which maintains itself despite small fluctuation in the feed-forward input, and only disappears when there is a large shift of the feed-forward input which significantly deviates the neural activity pattern from the contextual memory pattern \citep{rolls13}. 

It is easy to fit the specific deficit of \gls{ad} mice within this interpretation. \gls{ad} mice show normal dynamic of ``cellular signal'', however they are missing the ``network signal''. It is therefore possible that the \gls{ad} mice have a deficit in pattern completion: that the positive feedback force of converting an asynchronous ``cellular signal'' to a ``network signal'' is degraded. The lack of a gradual network signal suggest that the ``network signal'' in the \gls{ad} mice is likely formed by chance, that at some point, a global pattern appears from the asynchronous activity of individual cells. 

%This interpretation has several predictions. First, it predicts that the synchronization of cell activity in the \gls{ad} mice is degraded. To test this hypothesis, we calculated correlation of the neural activity of each pair of neurons, and compared the distribution between \gls{wt} and \gls{tg} mice. And indeed, we have found that cells in the \gls{tg} mice have less correlation with each other \todo{ref correlation figure}. 

This result is also supported by reports of neural oscillation deficits in \gls{ad}. In hippocampus, the \SIrange{3}{12}{\hertz} theta oscillation and the faster \SIrange{25}{120}{\hertz} gamma oscillation are considered important for learning and memory, and are critically dependent on the integrity of hippocampal neural networks \citep{buzsaki02, colgin09}. However, both theta and gamma oscillation has been shown to be altered in rodent models of \gls{ad}. It has been reported that the progression of plaque deposition is correlated with a decrease of theta power and frequency in both mouse models of \gls{ad},and acute \abeta{} treatment in rat \citep{scott12, villette10}. Decreased oscillation power is also found in the gamma frequency, and removal of \abeta{} plaques is able to block the deficit \citep{driver07, kurudenkandy14}. The coupling of theta and gamma oscillations has also been reported to be affected in \gls{ad} \citep{goutagny13}. 

Similar oscillation deficits are also found in the human patients. \Gls{eeg} recordings in early \gls{ad} patients have shown a decreased coupling between parietal alpha and prefrontal theta oscillations, and event-related delta, theta and alpha coherences are also significantly decreased \citep{guntekin08, montez09}. Moreover, the prefrontal theta coherence increases in \gls{ad} patients treated with \gls{ache} inhibitors \citep{yener07}, suggesting a close relationship between brain oscillations and \gls{ad}. More recent studies have aimed to enhance brain oscillations in \gls{ad}, and have suggested a causal relationship between the two. It has been found that deep brain stimulation at gamma frequency, both in rodent models and \gls{ad} patients, is able to improve cognitive function \citep{suthana14}. A recent study also showed that in a mouse model of early \gls{ad}, induction of gamma oscillation also reverses \abeta{} deposition \citep{iaccarino16}. 

These results suggest that the pattern completion deficit in \gls{ad} may be closely related to the brain oscillation deficit reported in the literature. However in the current study the cell activity is recorded at \SI{20}{\hertz}. The sampling frequency is unfortunately too slow to allow detection of any brain oscillations above \SI{10}{\hertz}. How a deficit in brain oscillation affects the pattern completion process in \gls{ad} is an important topic for future research. 

Secondly, this interpretation suggests that \gls{ad} mice may still be able to initiate freezing, however due to degraded attractor functions, they are unable to maintain the state of memory recall. This is supported by our findings that the \gls{tg} mice have similar number of freezing bouts, however a significantly shortened bout duration, showing a deficit in maintaining the expression of the memory but not initiating the memory. 

Moreover, the instability of the memory state in \gls{ad} can be inferred from the distance to the boundary of memory states. We measured the signed distance to the \gls{gsvm} classification boundary when the mouse's behaviour transitioned into freezing (Figure~\ref{f.ad.cls-distance}). We have found that while \gls{wt} mice show an acceleration over and away from the boundary, the \gls{tg} mice only barely cross the boundary, and stay close to it during freezing. Therefore, a small perturbation in the brain state is more likely to shift the \gls{tg} mice out of freezing. The freezing state in the \gls{wt} mice, on the other hand, is more robust to small perturbations.  

While the current study is the first to suggest that neural attractor states for memory recall are unstable in \gls{ad}, evidence from human behavioural studies suggest that similar dynamics may exist in \gls{ad} patients for other cognitive functions. Attention has long been considered computationally as a result of neural attractor states, and the strength of the attractor is important for maintaining attention \citep{desimone95, rolls08a, rolls13a}. Interestingly, early \gls{ad} patients have no deficits in focusing the attention, however, their attention is more likely to be disrupted by distractors, and less likely to be maintained over time \citep{perry99}. This result is similar to the memory recall deficit we have found in the current study, suggesting that it is possible that the deficit in attractor state in early \gls{ad} is global, affecting many brain areas' cognitive functions. 

The fragility of the memory attractor in \gls{ad} may also implicate another curious deficit of \gls{ad} called subjective memory impairment \citep{jahn13}. Subjective memory impairment is defined as a sense of memory deterioration with no objective impairment in cognitive test. Subjective memory impairment is correlated with hyperactivation of the \gls{mtl} during memory tasks, and predictive of hippocampal atrophy as well as later development of cognitive impairment and \gls{ad} \citep{jahn13}. It is possible the subjective memory impairment arises from our finding that during memory recall, the network state in \gls{ad} tends to stay close to the boundary of the attractor state. This close distance can be detected by other brain regions, and reduce the confidence of those brain regions in predicting the patient's behaviour, creating a sense of unsureness without a degradation of performance.  

The deterioration of the attractor state in \gls{ad} can be caused by several factors. First, we have shown that the deficit in \gls{ad} mice can be rescued by \tglu{} treatment with a reminder. From this we conclude that the \gls{ad} mice have encoded the contextual fear memory during training, but that this memory may not be encoded in the same way as in \gls{wt} mice. Therefore, the attractor state deficit can be a result of the formation of a weak attractor for the memory. Given that the neurons in \gls{ad} have impaired \gls{ltp}, and the normal memory encoding process may be interfered with by hyperactive neurons, it is unlikely that the memory encoding process is completely spared. 

Secondly, it is also possible that interference exists during the memory recall. The hyperactivity observed in the \gls{ad} mice may generate noise in the network state, so even with a functional memory attractor, the network state is more likely to spontaneously shift out of the attractor field due to noise. In addition, it is also possible that in \gls{ad}, other attractor states exist, and these sporadic attractors may actively pull the network state toward their own center of attraction, and away from that of the memory recall. This is hinted at both by animal and human studies. In a mouse model of taupathy, \citet{cheng13} have found that the \gls{ca1} place cells displayed rigid firing, such that the firing pattern of a familiar environment lingers even when the mice is placed in a novel environment. In humans, \gls{mci} and early \gls{ad} patients have deficits in shifting attention, and often maintain focus on an initial item even when it is no longer relevant \citep{perry99}. These studies suggest that if the memory recall deficit is a result of competing attractor fields, it is possible that the competing attractor represents a strong, familiar memory formed previously. This hypothesis can be tested by artificially creating a strong attractor state, for example, using repeated optical activation of a selected neural ensemble \citep{carrillo-reid16}, before contextual fear conditioning. The hypothesis will be supported if a future experiment shows the similar pattern reappeared during contextual fear memory recall in \gls{ad} and correlates with the memory deficit. 

\subsection{\tglu{} treatment is able to rescue circuitry deficits in \gls{ad}}

In the current project, we gave TgCRND8 mice an acute treatment of \tglu{} during memory formation. This treatment is able to rescue information content in \gls{ca1} neural activity as well as the presence of a network pattern just before memory recall, and rescues the pattern completion deficit in the TgCRND8 mice. Interestingly, while behaviourally \tglu{} had no effect on \gls{wt} mice, \tglu-treated \gls{wt} mice showed an earlier presence of network pattern before memory recall compared to vehicle-treated \gls{wt} mice. 

These results suggest that \tglu{} treatment during memory formation has a long-lasting effect. Moreover, given that the effect of \tglu{} treatment is present only when the memory is activated, either during formation or a reminder, these results suggest that the \tglu{} treatment potentially affects memory by strengthening the underlying neural representation. A robustly-connected memory trace allows it to be reactivated with a degraded pattern, so that a small number of neurons showing the activity of a memory are able to synchronize the network. In fact this is what we have found: at the same level of cellular memory signalling, \tglu{} treatment, both in \gls{wt} and \gls{tg} mice, forms a network pattern with a smaller cellular signal. This suggests that on a circuit level, the \tglu{} treatment results in an enhanced pattern completion process.

Our finding that \tglu{} treatment strengthens the memory network is congruent with the literature, where \tglu{} treatment has been shown to make the memory more resilient. For example, chronic \tglu{} treatment prevents forgetting of contextual fear memory, conditioned place preference, and novel object recognition \citep{dong15, migues16}. Treatment of \tglu{} also protects memory recall from protein synthesis inhibitors in auditory fear memory \citep{lopez15}. 

Our results, together with many other previous studies, found that in \gls{wt} mice, \tglu{} treatment does not affect the magnitude of the memory \citep{dias12, dong15, migues16}.  This suggests that the magnitude of memory, especially several days after memory formation, is not encoded in the strength of functional connections within an engram.

Given that the \tglu{} blocks synaptic \gls{ampar} endocytosis and therefore prevents \gls{ltd} \citep{ahmadian04}, it is likely that the rescuing effect of \tglu{} is mediated through synaptic plasticity. \citet{dong15} showed that \tglu{} treatment prevented \gls{ltp} from delay, and was able to rescue memory deficits in the APP23/PS45 mouse model of \gls{ad}. Moreover, it has been shown that \gls{ltp} may be sufficient to rescue memory recall. \citet{roy16} also found while the optogenetic reactivation of contextual fear memory does not have long-term effect, a train of fast, \gls{ltp}-inducing stimulations of the memory trace is able to rescue the memory deficit in APP/PS1 mice. Our findings that the effect of \tglu{} requires a reminder of the original memory is consistent with these results: in the short-term, the activity of the original memory trace needs to be reactivated by the reminder in order for the associations to be strengthened, and without activation of the original memory, our no-reminder controls did not benefit from the \tglu{}-mediated rescuing of memory recall.

The long term effect of \tglu{} is in accordance with the idea that \gls{ad} pathology forms a positive feedback loop, that the effect of \gls{ad} pathology including synaptic degeneration and aberrant neural activity in turn accelerates the signature pathology. In the current study, we show that an acute \tglu{} treatment is able to have lasting effects at least \SI{24}{\hour} later when the \tglu{} is not longer present. This result suggests that in TgCRND8 mice, correcting the synaptic pathology is not only able to rescue cognitive ability, but may also prevent feedback loop of \gls{ad} pathology in the short term. Previous reports of chronic \tglu{} treatment in a mouse model of \gls{ad} showed a reduction of neuritic plaques, supporting the hypothesis that rescuing circuit function in \gls{ad} can protect neurons from the \gls{ad} pathology.\todo{citation?} While more research is required to investigate long-term effect and outcome of \tglu{} treatment in \gls{ad}, our result and previous studies \citep{roy16, migues16, dong15} suggest that interventions that restore network circuit functions in \gls{ad} can enhance cognitive function and potentially protect neurons from \gls{ad} pathology. 

\chapter{Conclusions and Future Directions}

\section{Conclusion}

In the current project, we sought to explore the circuit mechanisms of hippocampal function, and how deficits in this functioning contribute to the cognitive symptoms of \gls{ad}. To investigate neural circuitry functions, we first designed and built a miniature fluorescent microscope for recording of neural activity in freely behaving mice. The mini-microscope is implantable on a mouse's head, and combined with a calcium indicator such as gCaMP we have shown that the mini-microscope is able to record hundreds of neurons in \gls{ca1}, and tens of neurons in \gls{la} simultaneously in freely behaving mice. In addition, we have extended the ability of mini-microscope to include a second colour channel, which can be potentially used to identify neural subpopulations and other brain structures concurrently with neural activity recording.

In the second part of this project, we used our mini-microscope to investigate circuitry deficits in a mouse model of early \gls{ad}, TgCRND8. We recorded from \gls{ca1} as mice learned and recalled a contextual fear memory. We have found that TgCRND8 mice have a significant memory deficit, showing less freezing during contextual memory test. This deficit is the result of significantly shorter freezing bouts, but not a reduced number of freezing bouts, suggesting the memory deficit may be the result of an inability to maintain the expression of a fear memory.

From the calcium recordings, we have found that \gls{ca1} neurons in TgCRND8 mice are hyperactive, and the cell activity of TgCRND8 mice contains less information about the behavioural state of mice during memory recall.  This effect is independent of spatial location of the mice, suggesting that the deficit reflects impaired recall of the fear memory. 

Next, we took advantage of machine learning methods to investigate how memory information is recalled in the \gls{ca1} circuitry. We differentiated the cellular signal, where individual neurons were independently considered, and a network signal, where the activity of all neurons was considered using a \gls{nbc} and \gls{gsvm}, respectively. The two classifiers were trained on the ensemble neural activity to predict the behaviour of the mice at each time point. We have found that even though individual neurons in \gls{ca1} contain less information about the behavioural state of the mice, the ensemble of neurons is still able to create an accurate prediction of when the mice were freezing. Moreover, we have found that the \gls{gsvm} shows significantly better performance than the \gls{nbc}, suggesting that a significant amount of information is encoded in the synergy between activity of single neurons.

The performance of \gls{nbc} and \gls{gsvm} suggest that these classifiers are able to recognize a neural activity pattern for fear memory recall. We then investigated how this pattern emerges at the beginning of memory recall. We aligned the classifier predictions to the time when the mice started to freeze, and found a gradual decrease of prediction accuracy before the mice froze, where the classifiers start to falsely predict freezing before the behaviour occurs. This result shows that the fear memory pattern emerges gradually before the behavioural change. This process corresponds to the pattern completion process in hippocampus. We found that in the \gls{nbc}, all groups show similar timing in the pattern completion, suggesting that in the TgCRND8 mice, the cellular signal for pattern completion is not impaired. However in \gls{gsvm}, TgCRND8 mice show no pattern completion, suggesting that in TgCRND8 mice, even though individual cells show activity patterns of freezing, these activity are not synchronized, and cannot form a pattern across the network. 

In addition, we found in TgCRND8 mice that the network state tends to linger at the classification boundary during freezing, while the other groups are able to dip into the freezing attractor state. This result suggests that in TgCRND8 mice, the attractor state for the expression of the memory is not robust, and is susceptible to interruption from minor deviation. This result explains the behavioural findings that the TgCRND8 mice are unable to maintain freezing, and more likely to be distracted. 

\Gls{ad} is characterized by significant synaptic loss, and a bias away from \gls{ltp} toward \gls{ltd}. Here we investigated whether rescuing synaptic function in \gls{ad} is able to restore hippocampal circuit function and consequently behaviour. We used the well-characterized peptide \tglu{}: \tglu{} blocks \gls{ampar} endocytosis, therefore increases synaptic strength, and shifts synaptic plasticity towards \gls{ltp}. We have found that \tglu{} treatment before contextual fear memory training is able to rescue the hyperactivity phenotype found in TgCRND8 mice. Moreover, this rescuing effect is long lasting: during memory testing the treated TgCRND8 mice show normal cell activity levels, better information content for behaviour, and also a pattern completion processes similar to \gls{wt} animals. Interestingly, we have found that \tglu{} treated \gls{wt} mice show an earlier pattern completion process, suggesting that effect of \tglu{} may promote \gls{ltp}, and this leads to a strengthening of the memory network, allowing the network pattern to be retrieved with a smaller cellular signal. 

We then investigated whether the memory deficit in TgCRND8 mice is due to impaired memory encoding or impaired memory recall. We trained mice with contextual fear conditioning, gave the mice a brief reminder with \tglu{} treatment next day, then tested their fear memory on the following day. We have found \tglu{} treatment during a brief reminder is able to rescue the fear memory in TgCRND8 mice, while vehicle treated TgCRND8 mice still show the memory deficit. This result suggests that the contextual fear memory is at least encoded one day after training, and the memory deficit can be largely attributed to deficient recall. Moreover, we have found that a memory reminder is necessary for the rescuing effect of \tglu{}. This supports our calcium imaging result that \tglu{} rescues memory recall by strengthening the active memory trace. 

In conclusion, in this project we have developed a miniature fluorescent microscope for calcium imaging in freely behaving mice, and used it to image \gls{ca1} neural activity in \gls{wt} and \gls{ad} model TgCRND8 mice. We have found TgCRND8 mice to have contextual fear memory deficits, hyperactive \gls{ca1} neurons, and less information content about behaviour. We found the memory recalled by TgCRND8 mice is unstable, both at the neural circuit level and behavioural level. \tglu{} treatment which potentially enhances \gls{ltp} can rescue both circuitry dysfunction and behaviour in TgCRND8 mice, suggesting the importance of restoring circuitry function as a potential treatment in restoring cognitive functions in \gls{ad}.

\section{Future directions for mini-microscope development}
\subsection{Technical improvement for the mini-microscope}

The focus mechanism in the mini-microscope can be improved in the future. In the current project, the focus of our mini-microscope is manually adjusted and fixed during the whole imaging session. This only allows us to image a \SI{400x400x50}{\um} rectangular box \SIrange{50}{150}{\um} below the lens. If the focus can be adjust fast enough during recording, it may be possible to provide a scan across the z-axis, and allow recording of more neurons in a 3D volume. In addition, in cases where neurons are heterogeneous across the z-axis such as cortex, a fine-adjustment of focal plane can give the experimenter more control to image specific subpopulations.

One way to improve the focus mechanism is to, instead of manually turning the camera up and down, control movement using a miniature motor. While this can provide an easy auto-focus mechanism, it will not be fast enough to give z-scan during recording. Moreover, the addition of an extra motor on the mini-microscope can add significant weight, making it too heavy for small rodents such as mice. 

Alternatively, a liquid lens can be used to replace the barrel lens we currently use to focus light to the camera. A liquid lens contains two immiscible liquids, and the curved interface between the two liquids provides the function of a lens. The focus of a liquid lens can be changed by applying voltage across the two liquids, which will bend the meniscus of the two liquids accordingly \citep{kuiper04}. A liquid lens allows focus change in several milliseconds, and is fast enough to provide a real time z-scan with the mini-microscope. However, current commercially available liquid lenses are still too large to be implemented on a mini-microscope. Future miniaturized liquid lens can be a great improvement for the mini-microscope.

In the current project, the mini-microscope is connected to a computer through a cable. Although the mini-microscope is compatible with the majority of behavioural paradigms, a connection cable can make some paradigms hard to perform, especially in environments with overhead structures, or during long-term recording sessions. Efforts can be made to create a wireless version of the mini-microscope. Again, the size and weight of the transmitting circuit and battery could be prohibitive on small rodents. However, recent developments have shown that in a resonant cavity, electrical power can be delivered wirelessly using radio frequency and a minimal implant weighing \SI{20}{\mg} \citep{montgomery15}. This could potentially provide the means for a wireless power source, and if combined with minimal transmission circuitry, it would be possible to allow wireless long-term calcium imaging in freely behaving animals. This would be very useful for investigating neural circuitry for slower neural circuitry process such as consolidation and circadian rhythms. 

\subsection{Combining mini-microscope with other techniques for investigating circuitry function}

One important future improvement for the construction of the mini-microscope is to combine it with other methods for investigating neural circuitry mechanisms. This will allow cross-referencing the neural activity to other physiological, anatomical and molecular information to provide a whole picture of the function of neural circuits under investigation. 

First, efforts can be made to allow \textit{post mortem} identification of the field of view from the mini-microscope. This would allow the mini-microscope data to be enriched with information from \gls{ihc} or \gls{fish}, which could potentially provide detailed molecular and cellular information about neurons, and offer an explanation of how subpopulations of neurons with different molecular markers coordinate circuitry behaviour. 

However in order to reach this goal, the optical properties of the mini-microscope will have to be accurately characterized. Efforts will need to be made in quantification of the curvature, range, and depth of the field of view. Measurement of these parameters allows mapping the cells identified in the mini-microscope to their original x--y plane, and also give constrains to the position of z-axis. The next step is to construct a 3D image of the \textit{post mortem} tissue, potentially using tissue clearing techniques such as \gls{clarity}. The reconstructed calcium imaging cells can then be mapped to the 3D image of the \textit{post mortem} tissue, allowing further molecular characterization of the recorded neurons. 

Secondly, it is also possible to combine mini-microscope recording with \gls{lfp} measurement for concurrent identification of brain oscillations. This addition does not pose any theoretical challenge, as \gls{lfp} measurements only require several extracellular metal electrodes, and \gls{lfp} recording techniques are well established. However to shield potential interference from the camera chip, the \gls{lfp} signals may need to be amplified and digitized on the headset before the signal is transmitted. This may require extra electronic circuitry, and may be challenging in small rodents such as mice, where both the weight and size of the implant are restricted. 

Our investigation of pattern completion in \gls{ad} can greatly benefit from a potential combination of \gls{lfp} and mini-microscope. Since we have found that the network signal, not individual cellular signals, are important in mediating the memory recall deficit, it suggests that the neurons in \gls{ad} are not synchronized to form a pattern. In the hippocampus, oscillations in theta and gamma frequency has been implicated in learning and memory, and are hypothesized to provide a reference frame to which neurons can align their activity. Given previous reports that in other mouse models of \gls{ad}, both theta and gamma oscillation are degraded \citep{driver07, villette10, scott12, goutagny13}, it is possible that the pattern completion deficit is a consequence of degraded oscillation. And if so, artificially restoring the oscillation by stimulation of the neurons, either electrically or optogenetically, could potentially restore the pattern completion process in \gls{ad} mice. 


\section{Effect of \tglu{} in strengthening memory trace}

We hypothesized that \tglu{} rescues the circuitry deficits in \gls{ad} by strengthening of the memory trace, however direct evidence is still required to show that the connection of neurons involved in the contextual fear memory are strengthened after \tglu{} treatment. This hypothesis can be directly tested by combining the mini-microscope recording with two-photon imaging. \citet{carrillo-reid16} have shown that a strong, optogenetically imprinted neural ensembles can be recalled by reactivated by stimulating a single neuron in the ensemble. A similar experiment can be conducted for evaluation of the fear memory ensemble. First, an excitatory opsin can be introduced to the cells of interest. After mini-microscope recording of contextual fear conditioning, the mice can be head-fixed on the stage of two-photon microscope, where the same cells can be imaged through the \gls{grin} lens, and neurons involved in the memory trace can be identified. Part of the memory trace can then be optically stimulated, and if the connections between neurons of the fear memory ensemble are strong, a partial activation of the ensemble should be able to activate the rest of the ensemble. Therefore we predict in \tglu{}-treated mice, activation of a smaller part of the ensemble should be able to reactivate the whole memory trace. 

\section{Pattern completion and pattern separation}

In the current project, we recorded from \gls{ca1}, which is considered an output of the hippocampus circuit, after both pattern separation process in \gls{dg} and pattern completion process in \gls{ca3} \citep{rolls13}. We have only tested contextual fear memory using the same context, which would primarily involve the pattern completion process. However, it would also be interesting to see, whether the \tglu{} treatment also improves the pattern-separation process. 

If the pattern completion process is enhanced by the \tglu{} treatment without affecting the pattern separation process, it follows that the patterns are likely able to complete with a smaller partial activation. This suggests that the memory strengthened by \tglu{} treatment is more generalized: since a similar environment may activate part of the trace, and this leads to completion of the fear memory pattern and generates corresponding behaviour. However, this hypothesis is not supported by previous studies. \citet{migues16} gave chronic \tglu{}-treatment in dorsal hippocampus after mice were trained with object recognition, conditioned place preference, or contextual fear conditioning. Interestingly, the authors found \tglu{} treatment not only prevents generalization, but in fact allows the mice to distinguish different memory testing environment more than the vehicle treated group \citep{migues16}. 

\citet{migues16}'s result may not be surprising, given that \tglu{} treatment improves \gls{ltp}, which has also been shown to be important for the pattern separation process in \gls{dg} \citep{rolls13}. However, neither \citet{migues16} nor our current experiment distinguished \gls{ca3} and \gls{dg} in \tglu{} treatment, and therefore are unable to tease out the effects of \tglu{} on pattern separation and pattern completion separately. A future experiment could limit the effect of \tglu{} to either \gls{ca3} or \gls{dg}, and observe both the \gls{ca1} activity and the mouse's behaviour. We hypothesize that \tglu{} treatment in \gls{ca3} will result in an enhancement of pattern completion as well as behavioural generalization, and treatment in \gls{dg} will result in an enhancement pattern separation and behavioural discrimination. 

A more direct way, but technically more difficult approach is to investigate the involvement of the pattern separation process using the mini-microscope to image \gls{ca3} and \gls{dg} simultaneously, and observe the pattern completion and pattern separation process directly. This potentially could be done using a doublet \gls{grin} lens, where a small, thin relay \gls{grin} lens is glued to a large \gls{grin} lens. With proper placement, it allows imaging \gls{ca3} using the rest of the surface of the large \gls{grin} lens, and the thin \gls{grin} lens could be inserted above \gls{dg} to relay an image to the large \gls{grin} lens. If combined with connectivity studies, data from this approach could allow direct validation of the computational model of hippocampus. Furthermore, in a disease model having information from both \gls{dg} and \gls{ca3} cell activity as well as their connectivity could enable researchers to pinpoint the functional impairment in the hippocampus circuitry.

\section{Effect of \tglu{} in correcting \gls{ad} pathology, potential treatment target}

In the current project, we have shown that \tglu{} treatment is not only able to rescue the behavioural deficits seen in TgCRND8 mice, but is also able to rescue circuitry function in these mice. Given that the aberrant circuitry activity in \gls{ad} promotes \gls{ad} pathology, it is worth investigating whether \tglu{}, and its target of \gls{ampar} endocytosis, can be targeted to alleviate \gls{ad} pathology. Results from \citet{dong15} show that chronically \tglu{}-treated \gls{ad} mice have fewer amyloid plaques compared to vehicle treated mice after memory training. However, it is still unknown whether \tglu{} treatment can lead to a reversal of \gls{ad} pathology. Instead of comparing to vehicle-treated controls, chronically treated \gls{ad} mice can be longitudinally compared to mice before treatment to characterize the effect of \tglu{} on \gls{ad} progression.

In addition with the mini-microscope, it may be possible to chronically image both the neural activity changes during the chronic treatment of \tglu{}, as well as simultaneous imaging of amyloid plaques \citep{zhang15}. Data from such experiments can provide details on how plaque deposition affects circuit activity, as well as conversely, how circuit activity can affect \gls{ad} progression. Moreover, \tglu{}, being an interference peptide, is not a good candidate for drug treatment because of its chemical instability, short half life, and inability to be administered orally \citep{fosgerau15}. Future studies can also focus on small molecules which have similar circuitry effects as \tglu{} with better pharmacological properties.

In conclusion, the aims the current project have been two-fold: 
\begin{enumerate*}[label={\alph*)}, font={\bfseries}]
    \item improve the current tools available for neuroscientists to investigate neural circuitry mechanisms, and
    \item contribute to the understanding of circuitry mechanisms underlying the cognitive deficits in \gls{ad}.
\end{enumerate*}
In the end, I hope through this thesis, I help to provide a potential link between studies at cellular level and behavioural level, and contribute to the understanding, and potential treatment of \gls{ad}. In the future, I predict advances in neural recording technology will reveal detailed mechanisms of neural circuitry, and inspire novel treatments for neurological disorders such as \gls{ad}. 
