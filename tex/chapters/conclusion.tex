\chapter{General Discussion}
\section{Summary of result}
\subsection{Summary of result}
\subsubsection{Hyperactivity}
    \citep{busche12} cells close to plaque are more excitable
\subsubsection{Cellular freezing encoding}
\subsubsection{Network freezing encoding}
\subsection{Pattern completion}
\todo{Pattern is not specific to freezing because of \citep{redondo14}}
\subsection{Attractor}
\todo{Freezing encoding: not behavioural}
\todo{Direction of causation}
\todo{\tglu rescue - how?}
\section{improvement of mini-microscope}
\subsection{prisms}
\subsection{realignment with clarity}
\section{activity on \gls{ad}}

The TgCRND8 mice are tested free of \tglu. This suggest that the effect of \tglu is long lasting. Given that the \tglu treatment is also able to rescue the memory deficit, only if a reminder is present, it suggests that the \tglu treatment transforms the underlying neural representation of the memory.  

The finding that the prediction of the classifier preceeds the behaviour of the mice suggest that the neural activity pattern is not a result of the mice's behaviour.  

Resilience of the network: as can be seen from the descrepency of mutual information result vs classifier result. This is also seen in early \gls{ad}, wwhere significant pathology does not impact cognitive performance. 

The difference of \gls{gsvm} and \gls{nbc} performance suggest that the network encode more information than individual cells. This maybe a universal phenomenon across the brain, as it is also found in encoding auditory fear memory in \gls{bla} \todo{cite Schinitzer amygdata paper}. 

\subsection{Other memories}
what about memories which require excitability to recall?
Reward / addiction

\subsection{implication on treatment - network}
\subsection{global noise reduction}
\subsection{Targets for drug development}
Eplepsy drug etc.
\subsection{Long-term monitoring}

20 pages

\chapter{Conclusions and Future Directions}
4-5 pages
