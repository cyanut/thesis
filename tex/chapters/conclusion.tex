\chapter{General Discussion}
\begin{comment}
\section{Summary of result}
\subsection{Summary of result}
\subsubsection{Hyperactivity}
\subsubsection{Cellular freezing encoding}
\subsubsection{Network freezing encoding}
\end{comment}
\section{mini-microscope}
\subsection{prisms}
\subsection{realignment with clarity}
\section{\gls{ad}}

\subsection{\gls{ca1} hyperexcitability}
In the current project, we have found that \gls{ca1} neurons in TgCRND8 miceare more active than those in the \gls{wt} mice, both during context exposure and during contextual memory recall. \tglu{} treatment during training is able to reduce the overall cell activity to \gls{wt} level. Moreover to control for the behavioural state of the mice, we investigated cell activity when the mice were freezing and non-freezing. We found the TgCRND8 genotype has a signficant major effect in increasing the cell activity during freezing, and \tglu treatment reduces the cell activity in both \gls{wt} and TgCRND8 mice. When the mice were not freezing during memory test, only TgCRND8 mice treated with \tglu has a significant decrease of cell activity compared to the other three groups. 

Our results confirm the consistent findings that \gls{ca1} cells in \gls{ad} is hyperactive, although the underlying mechanism is still not fully known, but closely related to \gls{ad} pathology \todo{cite}. We have also found that while during memory test the average cell activity in \gls{wt} mice is at the same level as it is during contextual exposure, the cell activity in the TgCRND8 mice have an increase of cell activity during memory testing compared to the memory encoding session \todo{ref figure}. This result suggest that in addition to the hyperactivity induced by \gls{ad} pathology, the TgCRND8 mice show additional activity in response to the memory recall task. This finding parallels those found in human patients in early \gls{ad}, where \gls{fmri} studies have found hyperactive hippocampus only during memory recall, and negatively correlate with memory performance \citep{sperling09, reiman12, kunz15}. It has been hypothesized that this task-dependent increase of hippocampal cell activity may present as a compensation mechanism in response of degraded hippocampal function in \gls{ad} \citep{kunz15}.

While the hyperexcitability in \gls{ad} is present when the mice is freezing, however the cell activity when the mice is not freezing is not different between groups. Considering that the majority of \gls{ca1} cells lower their activity during freezing, it is possible that the hyperactivity of \gls{ca1} neurons is caused by an innability to keep silent. This is congruent with the \gls{nmdar} pathology in \gls{ad}, where the \gls{nmdar} is found prone to spontaneously activate in \gls{ad}, and excite the neuron \todo{cite}. Under higher input, the spontaneous activities of the neuron can be masked by external input. This is also hinted by \citet{chang13}, who investigated spatial encoding in another mouse model of \gls{ad}. When their mouse is in the receptive field of the place cell, the cell is normally active. However when their mouse is outside the receptive field of the cell, the cell is unable to stop firing, leading to enlarged place fields \citep{chang13}. 

It is somewhat counter-intuitive that the \tglu{} treatment, which increases synaptic \gls{ampar} density and therefore strengthens excitatory synapses, results in a decrease of overall neural activity in \gls{ca1}. However, considering that the hyperexcitability in \gls{ad} is hypothesized to be a result of increased \gls{ltd} and decreased \gls{ltp} in the synapse \todo{cite}, and \tglu{} have been shown to block \gls{ltd}, it is possible that \tglu treatment is able rescue excitability by reversing the \gls{ltp} -- \gls{ltd} imbalance. It is still unclear how an \gls{ltp}--\gls{ltd} imbalance will lead to hyperexcitability. Computational models of \gls{ad} neurons suggest this may be caused by a loss of synaptic spines in the hippocampal neurons. This morphological change of the neuronal dendrites in \gls{ad} creates less hinderance for the transmission of incoming \glspl{epsp}, therefore allowing the neurons to be more excitable \citep{siskova14}. It is possible that the \tglu{} rescues the hyperexcitability by restoring the morphology of the neurons. There is a close correlation between synaptic \gls{ampar} density and spine size, and GluA2-containing \glspl{ampar} can trigger change in dendritic spine size \citep{hanley08}. Indeed, in unpublished data from this project \todo{include spine data?}, we have found \tglu{} treatment protects from spine density decrease in both \gls{ca1} and \gls{dg} after an acute expression of \gls{app} \todo{cite?}, supporting the possibility that the \tglu{ } rescues hyperexcitability by restoring normal neuronal morphology in \gls{ad}. 

The \tglu{} treatment is only applied to TgCRND8 mice \SI{1}{\hour} before contextual fear training, however the effect \tglu{} is present both during contextual fear training and during memory testing. This suggest that \tglu{}
starts to affect cell excitablity within \SI{1}{\hour}, and have a long-term effect. A closer investigation of the cell activity during training show that while the \tglu{} is able to correct mean cell activity in TgCRND8 mice to \gls{wt} level, the distribution of cell activity between Tg-\tglu and WT-Veh is still significantly different. From the cumulative plot \todo{ref figure}, the distribution of the lower \SI{90}{\percent} cell activity in Tg-\tglu is similar to WT-Veh, however the most active \SI{10}{\percent} cells are similar to Tg-Veh. This difference is not present \SI{24}{\hour} later during memory testing. This suggest that the rescuing effect of \tglu{} is a gradual process which take hours to complete. Cells that are less hyperactive are first rescued, and cells with very high activity require longer time for the effect of \tglu{} to appear. Given that cells close to amyloid plaques are more excitable\citep{busche12}, it is possible these cells with very high activity are located near amyloid plaques, have more damage from the \gls{ad} pathology, and require longer time for the process to reverse. 



\subsection{Contextual fear memory}

In the current project, we subjected \gls{wt} and TgCRND8 mice to contextual fear conditioning, and found that while the TgCRND8 mice shows inferior memory during memory testing, this memory deficit can be rescued by \tglu treatment either during training, or during exposure of a brief reminder. This result suggest that the TgCRND8 mice is still able to encode the memory during training, however have deficit in recalling the memory during memory testing. Similar results have been reported previously. In another mouse model of early \gls{ad}, APP/PS1, \citet{roy16} tagged cells activated during contextual fear conditioning, and optically activate the cells during memory recall. They have found that while the APP/PS1 mice shows a memory deficit during memory recall, light stimulation of the contextual fear memory trace is able to induce memory expression. 

In order to distinguish whether the memery recall deficit is due to initiation or maintenance of memeory expression, we have also analyzed the number of freezing bouts and duration of freezing bouts in the TgCRND8 and \gls{wt} mice. We have found while the TgCRND8 mice do not show significant difference in number of freezing bouts during memory test, however the duration of freezing bouts are significant shorter than that of the \gls{wt} mice. This result suggests that the TgCRND8 mice is still able initiate the memory recall, however unable to maintain it. 

\subsection{Encoding}

In the current study, we have found that cell activity in TgCRND8 mice are not predictive of the expression of fear memory. However on the other hand, this deficit is not reflected in the prediction accuracy of the machine learning classifiers: both \gls{nbc} and \gls{gsvm} are able to predict freezing from cell activity in TgCRND8 mice as well as in \gls{wt} mice. This discrepency of the prediction power suggest that the hippocampal neural circuit is highly redundant, such that even the prediction power of individual neurons is negatively affected by \gls{ad}, there is no significant information loss if multiple neurons are combined to make a prediction. Similar result is found in place encoding, where while individual neurons in a \gls{ad} mouse model have degraded spatial encoding, the ensemble is still able to accurately represent the mouse's position \citep{cheng13}.  

In this study, we on average only recorded close to one hundred cells in \gls{ca1} for each mouse, and the information in the activity of these neurons is enough for an acurate prediction of the mice's fear memory recall. There are more than \Num{1e4} neurons in \gls{ca1}: therefore even if under \gls{ad} pathology where individual cells' firing pattern is degraded, any downstream brain structure should still able to decode the information. This then suggest that the degradation of individual cell activity in \gls{ad} is secondary to the cognitive deficit, since to an efficient downstream brain structure, redundancy in the brain can protect information loss due to the degradation of cell activity. 

This conclusion alse explains why in the preclinical population, significant \gls{ad} pathology often does not significantly impact cognitive performance \todo{cite}. Another important implication of this conclusion is that treatment for the \gls{ad} pathology alone, without any restoration of the neural network structure, will not result in a significant improvement in the cognitive symptoms. This may explain the failure of recent attempts at treating \gls{ad} with \abeta{} clearance: it is possible that even with the successful \abeta{} removal, additional intervention is still required to restore the neural network at a circuitry level.

The importance of neural network structure is also reflected in the difference of \gls{gsvm} and \gls{nbc} performance. The significant better performance of the \gls{gsvm} suggest that the network encode more information than individual cells. This may be a universal phenomenon across the brain, as it is also recently found in other brain regions, such as \gls{bla}, during auditory fear conditioning \citep{grewe17}. Moreover, we find the accuracy of \gls{gsvm}, which predicts freezing based on the cell assemble as whole, is again similar to \gls{wt} mice. This suggest that when the mice is freezing, the information contained in the activity of neural network in \gls{ad} mice is unaffected. 

\subsection{pattern completion}

Given that the neural network contains enough information about freezing in \gls{ad} mice, it is then intriguing what can be the source of the memory deficit. During memory expression, the information content for the cell ensemble in \gls{ad} is not different from that of the \gls{wt} mice, it is possible the memory deficit can be detected outside of the duration where the memory is expressed. Given that the memory deficit in \gls{ad} is a deficit in memory recall, and that the process of pattern completion in hippocampus is theorized to be important in memory recall, we investigated whether this process is affected in the TgCRND8 mice.

To detect the pattern completion process, we aligned the classifier prediction accuracy to the behavioural change point of the mice \todo{ref fig}. We found a significant drop of prediction accuracy just before the mice is freezing, both in the \gls{nbc} and \gls{gsvm}. Since the drop of accuracy happens at a time where the mice are predominantly not freezing, the accuracy drop suggest that the classifiers are (inaccurately) predicting freezing, before the mice start to freeze. 

Since the classifiers are trained on each timepoint shuffled, the classifiers are agnostic to the temporal dynamic of the cell activity. Therefore, the siginificant change of prediction accuracy before freezing must reflect changes in the pattern of neural activity itself. Moreover, the fact that the change of neural activity in \gls{ca1} leads behaviour is very important in interpreting findings of the this study. Given that nature of the study is correlational, the causal relationship between the neural activity and behaviour is unclear. However, the tempororal precedance of the \gls{ca1} activity suggest that the neural activity pattern is not a result of the mice's behaviour, and is likely involved in initiating the behaviour. This confirms that the neural activity difference we have found is a result of circuitry deficit between groups, instead of a consequence of different freezing level between groups. 

We have found in \gls{nbc} prediction, \gls{wt} precedes the other three groups in the temporal precedance of freezing behaviour. Given that the \gls{nbc} consider each cell individually, this result suggest that the activity of individual cells start to form a ``cellular signal'' for freezing, which is a measurement of the contextual fear memory recall. The findings that all groups in \gls{nbc} predict freezing before behaviour, and that they have similar temporal precedence over behaviour suggest that in \gls{tg}, the ``cellular signal'' for contextual fear memory recall is unaffected in \gls{tg}. 

On the other hand, we found that the \gls{gsvm} prediction precedes freezing in the \gls{wt} mice, however this is not present in the vehicle-treated \gls{ad} mice. Again, given that the \gls{gsvm} detects a network pattern of cell activity, this result suggest that in \gls{wt} mice, a ``network signal'' of freezing appears in the \gls{ca1} network, however this signal is missing from the \gls{ad} mice before freezing. 

As the \gls{gsvm} can accurate classify the behaviour state of the mice based on the neural activity, we can use the \gls{gsvm} to approximate the neural activity pattern of recalling a contextual fear memory (considering each freezing bouts as an episode of memory recall). The gradual rise of freezing prediction before the freezing behaviour therefore represents the process of pattern completion: where an incomplete freezing pattern first appears in the network and confuses the classifier to predict freezing. Over time, the pattern gets more complete, which leads to increased classifier prediction of freezing. 

It is worth noting that while computational studies often model pattern completion using a stationary pattern \citep{rolls13}, the difference of prediction accuracy between \gls{nbs} and \gls{gsvm}, as well as the different temporal dynamics of the prediction precedence suggested that the neural correlate of a contextual fear memory recall is not a stationary state, since otherwise the performance of \gls{nbs} and \gls{gsvm} should be very similar, and have similar temporal dynamics in pattern completion. Therefore, our result suggest that, even in an apparently ``simple'' memory such as the contextual fear memory, the neural correlates is dynamic: the state of the neural activity during memory recall can move between multiple states.

We found that the ``cellular signal'', as detected by the \gls{nbc}, is siginficantly earlier than the ``network signal'' from the \gls{gsvm}. This result reveals some detail about the pattern completion process. The pattern completion process start with individual cells change their firing rate to that representing its activity distribution during memory pattern, potentially guided by feed-forward information. However, as the memory pattern is a collection of patterns, at any single point each cell may have the activity from different memory patterns. Therefore at this time point, \gls{nbc}, which only classify by looking at whether the activity of individual neuron represent \textit{any} single contextual fear memory pattern. However as the cell activities are not synchronized, no visible global pattern can be detected by \gls{gsvm}. 

However, as the ``cellular signal'' grow stronger, potentially due to an increase of feed-forward input, some cell activities become synchronized, and form a partial global pattern. This global pattern then, potentially through recurrent connections, recruits more cells to join the pattern. At this time, the partial global pattern can be detected by the \gls{gsvm}, and this positive feedback loop continues until the pattern is complete. This pattern then forms an attractor, maintain itself despite small fluctuation in the feed-forward input, and only disappear when there is a large shift of the feed-forward input which significantly deviates the neural activity from the contextual memory pattern \citep{rolls13}. \todo{confirm citation}

It is easy to put the specific deficit of \gls{ad} mice under this interpretation. \gls{ad} mice show normal dynamic of ``cellular signal'', however missing the ``network signal''. It is therefore possible that the \gls{ad} mice have a deficit of pattern completion: that the positive feedback force of converting an asynchronized ``cellular signal'' to a ``network signal'' is degraded. The lack of a gradual network signal suggest that the ``network signal'' in the \gls{ad} mice is likely formed by chance, that at some point, a global pattern is formed by the random asynchonous activity of individual cells. 

This interpretation has several predictions. First, it predicts that the synchronization of cell activity in the \gls{ad} mice is degraded. To test this hypothesis, we calculated correlation of the neural activity of each pair of neurons, and compared the distribution between \gls{wt} and \gls{tg} mice. And indeed, we have found that cells in the \gls{tg} mice have less correlation with each other \todo{ref correlation figure}. 

This result is also supported by reports of neural oscillation deficit in \gls{ad}. In hippocampus, the \SIRange{3}{12}{\hz} theta oscillation and the faster \SIRange{25}{120}{\hz} gamma oscillation are considered important for learning and memory, and are critically dependent on the integrity of hippocampal neural networks \citep{buzsaki02, colgin09}. However, both theta and gamma oscillation has been shown to be altered in rodent models of \gls{ad}. It has been reported that the progression of plaque deposition is correlated with a decrease of theta power and frequency in both mouse models of \gls{ad},and acute \abeta{} treatment in rat \citep{scott12, villette10}. Decreased oscillation power is also found in the gamma frequency, and removal of \abeta{} plaques is able to block the deficit \citep{driver07, kurudenkandy14}. The coupling of theta and gamma oscillations has also been reported to be affected in \gls{ad} \citep{goutagny13}. 

Similar oscillation deficits are also found in the human patients. \Gls{eeg} recordings in early \gls{ad} patients have shown a decreased coupling between parietal alpha and prefrontal theta oscillations, and event-related delta, theta and alpha coherences are also significantly decreased \citep{guntekin08, montez09}. Moreover, the prefrontal theta coherence increases in \gls{ad} patients treated with \gls{ache} inhibitors \citep{yener07}, suggesting a close relationship between brain oscillation and \gls{ad}. More recent studies aim to enhance the brain oscillation in \gls{ad}, and have suggested a causal relationship between the two. It has been found deep brain stimulation at gamma frequency, both in rodent model and \gls{ad} patients, is able to improve cognitive function \citep{suthana14}. A recent study also shows thatin a mouse model of early \gls{ad}, induction of gamma oscillation also reverses \abeta{} deposition \citep{iaccarino16}. 

These results suggest that the pattern completion deficit in \gls{ad} may be closely related to the brain oscillation deficit reported in the literature. However in the current study the cell activity is recorded in \SI{20}{\hz}. The sampling frequency is unfortunately too slow to allow detection of any brain oscillation above \SI{10}{\hz}. How a deficit in brain oscillation affects the pattern completion process in \gls{ad} can be an important topic for future research. 

Secondly, this interpretation suggests that the \gls{ad} mice may still able to initiate freezing, however due to degraded attractor functions, unable to maintain the state of memory recall. This is supported by our findings that the \gls{tg} mice have similar number of freezing bouts, however a significantly shortened bout duration, showing a deficit in maintaining the expression of the memory but not initiating the memory. 

Moreover, the instability of the memory state in the \gls{ad} can be implied from the distance to the memory states. We measured the signed distance to the \gls{gsvm} classification boundary when the mice's behaviour transits into freezing \todo{ref figure}. We have found that while the \gls{wt} mice shows an acceleration over and away from the boundary, the \gls{tg} mice only barely cross the boundary, and stay close to it during freezing. Therefore, a small perturbation in the brain state is more likely to shift the \gls{tg} mice out of freezing, and the freezing states in the \gls{wt} mice is more robust.  

While the current study is the first to suggest that neural attractor states are unstable, evidence from human behaviour studies suggest that similar dynamics may exist in \gls{ad} patients. Attention has long been considered computationally as a result of neural attractor state, and the strength of the attractor is important for maintaining attention \citep{desimone95, rolls08a, rolls13a}. Interestingly, early \gls{ad} patients have no deficits in focusing the attention, however, their attention is more likely to be disrupted by distractors, and less likely to maintain over time \citep{perry99}. This result is similar to the memory recall deficit we have found in the current study, suggesting that it is possible that the deficit in attractor state in early \gls{ad} is global, affecting many brain areas congitive functions. 

The fragility of the memory attractor in \gls{ad} may also implicate another curious prodrome of \gls{ad} called subjective memory impairment \citep{jahn13}. The subjective memory impairment is defined as a sense of memory deterioration with no objective impairment in cognitive test. The subjective memory impairment is correlated with hyperactivation of \gls{mtl} during memory tasks, and predictive of hippocampal atrophy as well as later development of cognitive impairment and \gls{ad} \citep{jahn13}. It is possible the subjective memory impairment arises from our finding that during memory recall, the network state in \gls{ad} tend to stay close to the boundary of the attractor state. This close distance can be detected by other brain regions, and reduce the confidence of those brain regions in predicting the patient's behaviour, creating a sense of forgetting, even though without a degradation of performance.  

The deterioration of the attractor state in \gls{ad} can be caused by several factors. First, we have shown that the \gls{ad} mice can be rescued by \tglu{} treatment with a reminder, however we are only able to conclude that the \gls{ad} mice has encoded the contextual fear memory during training, but does not provide support that the memory was encoded \textit{in the same way} as \gls{wt} mice does. Therefore, the attractor state deficit can be a result of the formation of a weak attractor for the memory. Given that the neurons in \gls{ad} have impaired \gls{ltp}, and the a normal memory encoding process may be interfered by hyperactive neurons, it is unlikely that the memory encoding process is completely spared. 

Secondly, it is also possible that interferance exists during the memory recall. The hyperactivity may generate noise in the network state, so even with a functional memory attractor, the network state however is more likely to be spontaneously bumped out of the attractor field by noise. In addition, it is also possible that in \gls{ad}, other attractor states exists, and these sporadic attractors may actively pull the network state toward their own basin, and away from that of the memory recall. This is hinted by both animal and human studies. In a mouse model of taupathy, \citet{cheng13} have found that the \gls{ca1} place cells displays rigid firing, such that the firing pattern of a familiar environment lingers even when the mice is place in a novel environment; in humans, \gls{mci} and early \gls{ad} patients have deficits in shifting attention, often maintain focus in the original item even when it is no longer relavent \citep{perry99}. These studies suggest that if the memory recall deficit is a result of competing attractor fields, it is possible that the competing attractor represents a strong, familiar memory formed previously. This hypothesis can be tested by artificially creating a strong attractor state, for example, using repeated optical activation of a selected neural ensemble \citep{carrillo-reid16}, before contextual fear conditioning. The hypothesis will be supported if the similar pattern reappeared during centextual fear memory recall in \gls{ad} and correlates with the memory deficit. Experiments like this is yet to be performed and reported.  

\subsection{tglu}

The TgCRND8 mice are tested free of \tglu. This suggest that the effect of \tglu is long lasting. Given that the \tglu treatment is also able to rescue the memory deficit, only if a reminder is present, it suggests that the \tglu treatment transforms the underlying neural representation of the memory.  

Interestingly, \citet{roy16} have also found while the optogenetic reactivation of contextual fear memory does not have long-term effect, a train of fast, \gls{ltp}-inducing stimulation of the memory trace is able to rescue the memory deficit in APP/PS1 mice. 

The long term effect of \tglu is in accordance with the idea that \gls{ad} pathology is a vicious loop, that the effect of the \gls{ad} pathology including synaptic degeneration and aberrent neural activity in turn accelerates the signature pathologies. In the current study, we an acute \tglu treatment is able to have lasting effect at least \SI{24}{\hour} when the \tglu is not longer present, suggesting that in the TgCRND8 mice, which is a model of early \gls{ad}, correcting the synaptic pathology is not only able to rescue the cognitive ability, but may also prevent the vicious loop of \gls{ad} patholgy in the short term. However, experiments examining the \gls{ad} pathology or the outcome of the TgCRND8 mice is still needed to confirm this hypothesis. 
\subsection{Other memories}
what about memories which require excitability to recall?
Reward / addiction


\subsection{Targets for drug development}
Eplepsy drug etc.


\subsection{Conclusion}

\chapter{Future Directions}
4-5 pages
