\chapter{General Discussion}
\begin{comment}
\section{Summary of result}
\subsection{Summary of result}
\subsubsection{Hyperactivity}
\subsubsection{Cellular freezing encoding}
\subsubsection{Network freezing encoding}
\end{comment}
\section{mini-microscope}
\subsection{advantage of current mini-microscope}
In the current project, we constructed a miniature microscope which weights less than \SI{3}{\gram}. This light weight of the mini-microscope allow implantation on mice without significantly alter the mice's natural behaviour.  This miniature microscope is able to image green fluorescence at a resolution of at less \SI{2}{\micrometer}. Using a tail vein injection of the the flourescent dye, dextran-flourescein, we are able to resolve individual red blood cells in capillary blood vessels, and measure the flow rate of the blood the capillaries. 

Moreover, we have shown that the mini-microscope is able to recording calcium transients from GCaMP-expressiong neurons. Using the mini-microscope, we are able to image more than \num{200} \gls{ca1} neurons simutaneously, and identify potential place cells from the recording. Moreover, with an additional thin relay \gls{grin} lens, we are able toextend the imaging capability of the mini-microscope to deep brain region with minimal damage. Here we have demonstrated simutanoeus recording of more than \SI{30} neurons in \gls{la} when the animal is under an auditory fear conditioning paradigm. 

While in this project, all recordings of calcium transients are at a rate of \num{20} frames per second, with our construction of the mini-microscope the recoding rate can reach up to \num{200} frames per second, at a cost of spatial resolution. A higher frame rate will allow the resolution of fast brain oscillations such as the \SIrange{7}{11}{\hertz} theta oscillation and potentially the fast \SIrange{40}{100}{\hertz} gamma oscillation, both of which has been shown to be inportant in learning and memory. Calcium imaging is especially suited for detection of brain oscillation, as these oscillation may not generate action potential, but create sub-threshold changes of membrane potential which is reflected by a fluctuation of internal calcium concentrations. The ability of simutaneous imaging a dense ensemble of neurons allows the brain oscillation to be measured in each cell and average, therefore increase the \gls{snr} in detecting these oscillations. The brain oscillation signal can then be related to neural activities to study how local neural circuit response to the brain oscillation. 

In the current project, we have also demonstrated the ability of the mini-microscope to image in both red and green channels. The extra colour channel can be used for the identification of neural subpopulation or extra inforamation for the brain. In our prototype we used blue light to stimulate red retrobeads and the fluorescent protein tdTomato, both of which have a broad excitation spectrum. However, this requirement is usually not necessary. If the red channel is static, for example, to identify neural subpopulation and stable brain structure, the red channel can be captured by a mini-microscope with an efficient excitation \gls{led} and corresponding filters separately. The resulting image can be later aligned with recording of calcium transients for cell identification. 

Our design of mini-microscope only cost \$\,300--\$\,1000 for each unit, which will not represent a significant expense in most neuroscience laboratories. Moreover, all components of the mini-microscope is commercially available and can be assembled with minimal tools, our design of mini-microscope provides access to neuroscientists with no requirement of engineering experience. The microscope casing is 3D printable. We provide both the 3D models and relavent freely available, in order to allow the neuroscience community to benefit from this work. 

\subsection{advantage of using the mini-microscope in the \gls{ad} project}

As a test of the capabilities of the miniature microscope, we used it to image \gls{ca1} neurons in \gls{wt} and a transgenic mouse model of \gls{ad}, TgCRND8, when the mice are encoding and recalling a contextual fear memory. Our mini-microscope allows us to image more than \num{100} neurons simultaneously, allowing us to investigate hippocampal circuitry mechanisms such as pattern completion. 

We have found that the \gls{nbc}, which predicts the mice's behaviour by considering individual neurons, is unable to reveal the pattern completion deficit in the TgCRND8 mice. This results suggest that a significant amount of information the neurons encoding is contained in the synergy between activity of neurons. This information cannot be revealed using traditional methods such as \textit{in vivo} electrophysiological recording, where only several neurons can be recorded at the same time. 

To compensate for the lack of number of neurons simutaneously recorded, electrophysiological recodings often needs to repeat the same behaviour trial for many times, which may lead to change of the animal's behaviour and confound the interpretation. However in the current study, the mini-microscope gives us the freedom to choose an experimental paradigm which is well-established in the field of behavioural neuroscience. This freedom allows us to connect the findings from the neural activity recording to the rich findings of the behavioural neuroscience. This is in contrast to the \textit{in vivo} 2-photon imaging, where due to the requirement of the animal's head to be fixed, only specifically design behavioural paradigms are compatible, where the neural mechanisms for those are less well understood. 



\section{\gls{ad}}

\subsection{\gls{ca1} hyperexcitability}
In the current project, we have found that \gls{ca1} neurons in TgCRND8 miceare more active than those in the \gls{wt} mice, both during context exposure and during contextual memory recall. \tglu{} treatment during training is able to reduce the overall cell activity to \gls{wt} level. Moreover to control for the behavioural state of the mice, we investigated cell activity when the mice were freezing and non-freezing. We found the TgCRND8 genotype has a signficant major effect in increasing the cell activity during freezing, and \tglu treatment reduces the cell activity in both \gls{wt} and TgCRND8 mice. When the mice were not freezing during memory test, only TgCRND8 mice treated with \tglu has a significant decrease of cell activity compared to the other three groups. 

Our results confirm the consistent findings that \gls{ca1} cells in \gls{ad} is hyperactive, although the underlying mechanism is still not fully known, but closely related to \gls{ad} pathology \todo{cite}. We have also found that while during memory test the average cell activity in \gls{wt} mice is at the same level as it is during contextual exposure, the cell activity in the TgCRND8 mice have an increase of cell activity during memory testing compared to the memory encoding session \todo{ref figure}. This result suggest that in addition to the hyperactivity induced by \gls{ad} pathology, the TgCRND8 mice show additional activity in response to the memory recall task. This finding parallels those found in human patients in early \gls{ad}, where \gls{fmri} studies have found hyperactive hippocampus only during memory recall, and negatively correlate with memory performance \citep{sperling09, reiman12, kunz15}. It has been hypothesized that this task-dependent increase of hippocampal cell activity may present as a compensation mechanism in response of degraded hippocampal function in \gls{ad} \citep{kunz15}.

While the hyperexcitability in \gls{ad} is present when the mice is freezing, however the cell activity when the mice is not freezing is not different between groups. Considering that the majority of \gls{ca1} cells lower their activity during freezing, it is possible that the hyperactivity of \gls{ca1} neurons is caused by an innability to keep silent. This is congruent with the \gls{nmdar} pathology in \gls{ad}, where the \gls{nmdar} is found prone to spontaneously activate in \gls{ad}, and excite the neuron \todo{cite}. Under higher input, the spontaneous activities of the neuron can be masked by external input. This is also hinted by \citet{chang13}, who investigated spatial encoding in another mouse model of \gls{ad}. When their mouse is in the receptive field of the place cell, the cell is normally active. However when their mouse is outside the receptive field of the cell, the cell is unable to stop firing, leading to enlarged place fields \citep{chang13}. 

It is somewhat counter-intuitive that the \tglu{} treatment, which increases synaptic \gls{ampar} density and therefore strengthens excitatory synapses, results in a decrease of overall neural activity in \gls{ca1}. However, considering that the hyperexcitability in \gls{ad} is hypothesized to be a result of increased \gls{ltd} and decreased \gls{ltp} in the synapse \todo{cite}, and \tglu{} have been shown to block \gls{ltd}, it is possible that \tglu treatment is able rescue excitability by reversing the \gls{ltp} -- \gls{ltd} imbalance. It is still unclear how an \gls{ltp}--\gls{ltd} imbalance will lead to hyperexcitability. Computational models of \gls{ad} neurons suggest this may be caused by a loss of synaptic spines in the hippocampal neurons. This morphological change of the neuronal dendrites in \gls{ad} creates less hinderance for the transmission of incoming \glspl{epsp}, therefore allowing the neurons to be more excitable \citep{siskova14}. It is possible that the \tglu{} rescues the hyperexcitability by restoring the morphology of the neurons. There is a close correlation between synaptic \gls{ampar} density and spine size, and GluA2-containing \glspl{ampar} can trigger change in dendritic spine size \citep{hanley08}. Indeed, in unpublished data from this project \todo{include spine data?}, we have found \tglu{} treatment protects from spine density decrease in both \gls{ca1} and \gls{dg} after an acute expression of \gls{app} \todo{cite?}, supporting the possibility that the \tglu{ } rescues hyperexcitability by restoring normal neuronal morphology in \gls{ad}. 

The \tglu{} treatment is only applied to TgCRND8 mice \SI{1}{\hour} before contextual fear training, however the effect \tglu{} is present both during contextual fear training and during memory testing. This suggest that \tglu{}
starts to affect cell excitablity within \SI{1}{\hour}, and have a long-term effect. A closer investigation of the cell activity during training show that while the \tglu{} is able to correct mean cell activity in TgCRND8 mice to \gls{wt} level, the distribution of cell activity between Tg-\tglu and WT-Veh is still significantly different. From the cumulative plot \todo{ref figure}, the distribution of the lower \SI{90}{\percent} cell activity in Tg-\tglu is similar to WT-Veh, however the most active \SI{10}{\percent} cells are similar to Tg-Veh. This difference is not present \SI{24}{\hour} later during memory testing. This suggest that the rescuing effect of \tglu{} is a gradual process which take hours to complete. Cells that are less hyperactive are first rescued, and cells with very high activity require longer time for the effect of \tglu{} to appear. Given that cells close to amyloid plaques are more excitable\citep{busche12}, it is possible these cells with very high activity are located near amyloid plaques, have more damage from the \gls{ad} pathology, and require longer time for the process to reverse. 



\subsection{Contextual fear memory}

In the current project, we subjected \gls{wt} and TgCRND8 mice to contextual fear conditioning, and found that while the TgCRND8 mice shows inferior memory during memory testing, this memory deficit can be rescued by \tglu treatment either during training, or during exposure of a brief reminder. This result suggest that the TgCRND8 mice is still able to encode the memory during training, however have deficit in recalling the memory during memory testing. Similar results have been reported previously. In another mouse model of early \gls{ad}, APP/PS1, \citet{roy16} tagged cells activated during contextual fear conditioning, and optically activate the cells during memory recall. They have found that while the APP/PS1 mice shows a memory deficit during memory recall, light stimulation of the contextual fear memory trace is able to induce memory expression. 

In order to distinguish whether the memery recall deficit is due to initiation or maintenance of memeory expression, we have also analyzed the number of freezing bouts and duration of freezing bouts in the TgCRND8 and \gls{wt} mice. We have found while the TgCRND8 mice do not show significant difference in number of freezing bouts during memory test, however the duration of freezing bouts are significant shorter than that of the \gls{wt} mice. This result suggests that the TgCRND8 mice is still able initiate the memory recall, however unable to maintain it. 

\subsection{Encoding}

In the current study, we have found that cell activity in TgCRND8 mice are not predictive of the expression of fear memory. However on the other hand, this deficit is not reflected in the prediction accuracy of the machine learning classifiers: both \gls{nbc} and \gls{gsvm} are able to predict freezing from cell activity in TgCRND8 mice as well as in \gls{wt} mice. This discrepency of the prediction power suggest that the hippocampal neural circuit is highly redundant, such that even the prediction power of individual neurons is negatively affected by \gls{ad}, there is no significant information loss if multiple neurons are combined to make a prediction. Similar result is found in place encoding, where while individual neurons in a \gls{ad} mouse model have degraded spatial encoding, the ensemble is still able to accurately represent the mouse's position \citep{cheng13}.  

In this study, we on average only recorded close to one hundred cells in \gls{ca1} for each mouse, and the information in the activity of these neurons is enough for an acurate prediction of the mice's fear memory recall. There are more than \num{1e4} neurons in \gls{ca1}: therefore even if under \gls{ad} pathology where individual cells' firing pattern is degraded, any downstream brain structure should still able to decode the information. This then suggest that the degradation of individual cell activity in \gls{ad} is secondary to the cognitive deficit, since to an efficient downstream brain structure, redundancy in the brain can protect information loss due to the degradation of cell activity. 

This conclusion alse explains why in the preclinical population, significant \gls{ad} pathology often does not significantly impact cognitive performance \todo{cite}. Another important implication of this conclusion is that treatment for the \gls{ad} pathology alone, without any restoration of the neural network structure, will not result in a significant improvement in the cognitive symptoms. This may explain the failure of recent attempts at treating \gls{ad} with \abeta{} clearance: it is possible that even with the successful \abeta{} removal, additional intervention is still required to restore the neural network at a circuitry level.

The importance of neural network structure is also reflected in the difference of \gls{gsvm} and \gls{nbc} performance. The significant better performance of the \gls{gsvm} suggest that the network encode more information than individual cells. This may be a universal phenomenon across the brain, as it is also recently found in other brain regions, such as \gls{bla}, during auditory fear conditioning \citep{grewe17}. Moreover, we find the accuracy of \gls{gsvm}, which predicts freezing based on the cell assemble as whole, is again similar to \gls{wt} mice. This suggest that when the mice is freezing, the information contained in the activity of neural network in \gls{ad} mice is unaffected. 

\subsection{pattern completion}

Given that the neural network contains enough information about freezing in \gls{ad} mice, it is then intriguing what can be the source of the memory deficit. During memory expression, the information content for the cell ensemble in \gls{ad} is not different from that of the \gls{wt} mice, it is possible the memory deficit can be detected outside of the duration where the memory is expressed. Given that the memory deficit in \gls{ad} is a deficit in memory recall, and that the process of pattern completion in hippocampus is theorized to be important in memory recall, we investigated whether this process is affected in the TgCRND8 mice.

To detect the pattern completion process, we aligned the classifier prediction accuracy to the behavioural change point of the mice \todo{ref fig}. We found a significant drop of prediction accuracy just before the mice is freezing, both in the \gls{nbc} and \gls{gsvm}. Since the drop of accuracy happens at a time where the mice are predominantly not freezing, the accuracy drop suggest that the classifiers are (inaccurately) predicting freezing, before the mice start to freeze. 

Since the classifiers are trained on each timepoint shuffled, the classifiers are agnostic to the temporal dynamic of the cell activity. Therefore, the siginificant change of prediction accuracy before freezing must reflect changes in the pattern of neural activity itself. Moreover, the fact that the change of neural activity in \gls{ca1} leads behaviour is very important in interpreting findings of the this study. Given that nature of the study is correlational, the causal relationship between the neural activity and behaviour is unclear. However, the tempororal precedance of the \gls{ca1} activity suggest that the neural activity pattern is not a result of the mice's behaviour, and is likely involved in initiating the behaviour. This confirms that the neural activity difference we have found is a result of circuitry deficit between groups, instead of a consequence of different freezing level between groups. 

We have found in \gls{nbc} prediction, \gls{wt} precedes the other three groups in the temporal precedance of freezing behaviour. Given that the \gls{nbc} consider each cell individually, this result suggest that the activity of individual cells start to form a ``cellular signal'' for freezing, which is a measurement of the contextual fear memory recall. The findings that all groups in \gls{nbc} predict freezing before behaviour, and that they have similar temporal precedence over behaviour suggest that in \gls{tg}, the ``cellular signal'' for contextual fear memory recall is unaffected in \gls{tg}. 

On the other hand, we found that the \gls{gsvm} prediction precedes freezing in the \gls{wt} mice, however this is not present in the vehicle-treated \gls{ad} mice. Again, given that the \gls{gsvm} detects a network pattern of cell activity, this result suggest that in \gls{wt} mice, a ``network signal'' of freezing appears in the \gls{ca1} network, however this signal is missing from the \gls{ad} mice before freezing. 

As the \gls{gsvm} can accurate classify the behaviour state of the mice based on the neural activity, we can use the \gls{gsvm} to approximate the neural activity pattern of recalling a contextual fear memory (considering each freezing bouts as an episode of memory recall). The gradual rise of freezing prediction before the freezing behaviour therefore represents the process of pattern completion: where an incomplete freezing pattern first appears in the network and confuses the classifier to predict freezing. Over time, the pattern gets more complete, which leads to increased classifier prediction of freezing. 

It is worth noting that while computational studies often model pattern completion using a stationary pattern \citep{rolls13}, the difference of prediction accuracy between \gls{nbc} and \gls{gsvm}, as well as the different temporal dynamics of the prediction precedence suggested that the neural correlate of a contextual fear memory recall is not a stationary state, since otherwise the performance of \gls{nbc} and \gls{gsvm} should be very similar, and have similar temporal dynamics in pattern completion. Therefore, our result suggest that, even in an apparently ``simple'' memory such as the contextual fear memory, the neural correlates is dynamic: the state of the neural activity during memory recall can move between multiple states.

We found that the ``cellular signal'', as detected by the \gls{nbc}, is siginficantly earlier than the ``network signal'' from the \gls{gsvm}. This result reveals some detail about the pattern completion process. The pattern completion process start with individual cells change their firing rate to that representing its activity distribution during memory pattern, potentially guided by feed-forward information. However, as the memory pattern is a collection of patterns, at any single point each cell may have the activity from different memory patterns. Therefore at this time point, \gls{nbc}, which only classify by looking at whether the activity of individual neuron represent \textit{any} single contextual fear memory pattern. However as the cell activities are not synchronized, no visible global pattern can be detected by \gls{gsvm}. 

However, as the ``cellular signal'' grow stronger, potentially due to an increase of feed-forward input, some cell activities become synchronized, and form a partial global pattern. This global pattern then, potentially through recurrent connections, recruits more cells to join the pattern. At this time, the partial global pattern can be detected by the \gls{gsvm}, and this positive feedback loop continues until the pattern is complete. This pattern then forms an attractor, maintain itself despite small fluctuation in the feed-forward input, and only disappear when there is a large shift of the feed-forward input which significantly deviates the neural activity from the contextual memory pattern \citep{rolls13}. \todo{confirm citation}

It is easy to put the specific deficit of \gls{ad} mice under this interpretation. \gls{ad} mice show normal dynamic of ``cellular signal'', however missing the ``network signal''. It is therefore possible that the \gls{ad} mice have a deficit of pattern completion: that the positive feedback force of converting an asynchronized ``cellular signal'' to a ``network signal'' is degraded. The lack of a gradual network signal suggest that the ``network signal'' in the \gls{ad} mice is likely formed by chance, that at some point, a global pattern is formed by the random asynchonous activity of individual cells. 

This interpretation has several predictions. First, it predicts that the synchronization of cell activity in the \gls{ad} mice is degraded. To test this hypothesis, we calculated correlation of the neural activity of each pair of neurons, and compared the distribution between \gls{wt} and \gls{tg} mice. And indeed, we have found that cells in the \gls{tg} mice have less correlation with each other \todo{ref correlation figure}. 

This result is also supported by reports of neural oscillation deficit in \gls{ad}. In hippocampus, the \SIrange{3}{12}{\hertz} theta oscillation and the faster \SIrange{25}{120}{\hertz} gamma oscillation are considered important for learning and memory, and are critically dependent on the integrity of hippocampal neural networks \citep{buzsaki02, colgin09}. However, both theta and gamma oscillation has been shown to be altered in rodent models of \gls{ad}. It has been reported that the progression of plaque deposition is correlated with a decrease of theta power and frequency in both mouse models of \gls{ad},and acute \abeta{} treatment in rat \citep{scott12, villette10}. Decreased oscillation power is also found in the gamma frequency, and removal of \abeta{} plaques is able to block the deficit \citep{driver07, kurudenkandy14}. The coupling of theta and gamma oscillations has also been reported to be affected in \gls{ad} \citep{goutagny13}. 

Similar oscillation deficits are also found in the human patients. \Gls{eeg} recordings in early \gls{ad} patients have shown a decreased coupling between parietal alpha and prefrontal theta oscillations, and event-related delta, theta and alpha coherences are also significantly decreased \citep{guntekin08, montez09}. Moreover, the prefrontal theta coherence increases in \gls{ad} patients treated with \gls{ache} inhibitors \citep{yener07}, suggesting a close relationship between brain oscillation and \gls{ad}. More recent studies aim to enhance the brain oscillation in \gls{ad}, and have suggested a causal relationship between the two. It has been found deep brain stimulation at gamma frequency, both in rodent model and \gls{ad} patients, is able to improve cognitive function \citep{suthana14}. A recent study also shows thatin a mouse model of early \gls{ad}, induction of gamma oscillation also reverses \abeta{} deposition \citep{iaccarino16}. 

These results suggest that the pattern completion deficit in \gls{ad} may be closely related to the brain oscillation deficit reported in the literature. However in the current study the cell activity is recorded in \SI{20}{\hertz}. The sampling frequency is unfortunately too slow to allow detection of any brain oscillation above \SI{10}{\hertz}. How a deficit in brain oscillation affects the pattern completion process in \gls{ad} can be an important topic for future research. 

Secondly, this interpretation suggests that the \gls{ad} mice may still able to initiate freezing, however due to degraded attractor functions, unable to maintain the state of memory recall. This is supported by our findings that the \gls{tg} mice have similar number of freezing bouts, however a significantly shortened bout duration, showing a deficit in maintaining the expression of the memory but not initiating the memory. 

Moreover, the instability of the memory state in the \gls{ad} can be implied from the distance to the memory states. We measured the signed distance to the \gls{gsvm} classification boundary when the mice's behaviour transits into freezing \todo{ref figure}. We have found that while the \gls{wt} mice shows an acceleration over and away from the boundary, the \gls{tg} mice only barely cross the boundary, and stay close to it during freezing. Therefore, a small perturbation in the brain state is more likely to shift the \gls{tg} mice out of freezing, and the freezing states in the \gls{wt} mice is more robust.  

While the current study is the first to suggest that neural attractor states are unstable, evidence from human behaviour studies suggest that similar dynamics may exist in \gls{ad} patients. Attention has long been considered computationally as a result of neural attractor state, and the strength of the attractor is important for maintaining attention \citep{desimone95, rolls08a, rolls13a}. Interestingly, early \gls{ad} patients have no deficits in focusing the attention, however, their attention is more likely to be disrupted by distractors, and less likely to maintain over time \citep{perry99}. This result is similar to the memory recall deficit we have found in the current study, suggesting that it is possible that the deficit in attractor state in early \gls{ad} is global, affecting many brain areas congitive functions. 

The fragility of the memory attractor in \gls{ad} may also implicate another curious prodrome of \gls{ad} called subjective memory impairment \citep{jahn13}. The subjective memory impairment is defined as a sense of memory deterioration with no objective impairment in cognitive test. The subjective memory impairment is correlated with hyperactivation of \gls{mtl} during memory tasks, and predictive of hippocampal atrophy as well as later development of cognitive impairment and \gls{ad} \citep{jahn13}. It is possible the subjective memory impairment arises from our finding that during memory recall, the network state in \gls{ad} tend to stay close to the boundary of the attractor state. This close distance can be detected by other brain regions, and reduce the confidence of those brain regions in predicting the patient's behaviour, creating a sense of forgetting, even though without a degradation of performance.  

The deterioration of the attractor state in \gls{ad} can be caused by several factors. First, we have shown that the \gls{ad} mice can be rescued by \tglu{} treatment with a reminder, however we are only able to conclude that the \gls{ad} mice has encoded the contextual fear memory during training, but does not provide support that the memory was encoded \textit{in the same way} as \gls{wt} mice does. Therefore, the attractor state deficit can be a result of the formation of a weak attractor for the memory. Given that the neurons in \gls{ad} have impaired \gls{ltp}, and the a normal memory encoding process may be interfered by hyperactive neurons, it is unlikely that the memory encoding process is completely spared. 

Secondly, it is also possible that interferance exists during the memory recall. The hyperactivity may generate noise in the network state, so even with a functional memory attractor, the network state however is more likely to be spontaneously bumped out of the attractor field by noise. In addition, it is also possible that in \gls{ad}, other attractor states exists, and these sporadic attractors may actively pull the network state toward their own basin, and away from that of the memory recall. This is hinted by both animal and human studies. In a mouse model of taupathy, \citet{cheng13} have found that the \gls{ca1} place cells displays rigid firing, such that the firing pattern of a familiar environment lingers even when the mice is place in a novel environment; in humans, \gls{mci} and early \gls{ad} patients have deficits in shifting attention, often maintain focus in the original item even when it is no longer relavent \citep{perry99}. These studies suggest that if the memory recall deficit is a result of competing attractor fields, it is possible that the competing attractor represents a strong, familiar memory formed previously. This hypothesis can be tested by artificially creating a strong attractor state, for example, using repeated optical activation of a selected neural ensemble \citep{carrillo-reid16}, before contextual fear conditioning. The hypothesis will be supported if the similar pattern reappeared during centextual fear memory recall in \gls{ad} and correlates with the memory deficit. Experiments like this is yet to be performed and reported.  

\subsection{\tglu}

In the current project, we gave the TgCRND8 mice an acute treatment of \tglu during memory formation. This treatment is able to rescue information content in \gls{ca1} neural activity as well as the presence of a network pattern just before memory recall, and rescues the pattern completion deficit in the TgCRND8 mice. Interestingly, while behaviourally \tglu have no effect on \gls{wt} mice, \tglu-treated \gls{wt} mice show an even earlier presence of network pattern before memory recall. 

These results suggest that \tglu treatment during memory formation has a long-lasting effect. Moreover, given that the effect of \tglu treatment is only present only when the memory is activated: either during formation or a reminder, these results suggest that the \tglu treatment potentially strengthens the underlying neural representation of the memory. Stronger association within a memory trace allows the memory trace to be reactivated with a more degraded pattern, so that a small number of neurons showing the activity of a memory recall is able to synchronize the network. In fact this is what we have found: at the same level of cellular signal of memory, \tglu treatment mice, both in \gls{wt} and \gls{tg} group, forms a network pattern with a smaller cellular signal. This suggest that on a circuitry level, the cognitive benefit of \tglu treatment includes an enhanced pattern completion process.

Our finding that \tglu treatment strengthens the memory network is congruent with literature, where \tglu treatment has been shown to make the memory more resilient. For example, chronic \tglu treatment prevents forgetting of contextual fear memory, conditional place preference, and novel object recognition \citep{dong14, migues16}. Treatment of \tglu also protects memory recall from protein synthesis inhibitors in auditory fear memory \citep{lopez15}.
While our result, together with many other previous studies found that in \gls{wt} mice, \tglu treatment does not affect the magnitude of the memory \citep{dias12, dong14, migues16}, suggesting that the magnitude of memory, especially several days after memory formation, is not encoded in the strength of memory engram connections. 

Given that the \tglu blocks synaptic \gls{ampar} trafficking, and blocks \gls{ltd} \todo{cite}, it is likely that the rescuing effect of \tglu is mediated by synaptic plasticity. \citet{dong14} shows that \tglu treatment prevents \gls{ltp} from delay, and is able to rescue memory deficit in the APP23/PS45 mouse model of \gls{ad}. Moreover, it has been shown that \gls{ltp} may be sufficient to rescue memory recall. \citet{roy16} have also found while the optogenetic reactivation of contextual fear memory does not have long-term effect, a train of fast, \gls{ltp}-inducing stimulation of the memory trace is able to rescue the memory deficit in APP/PS1 mice. Our findings that the effect of \tglu requires a reminder of the original memory is consistent with these results: in the short-term, the activity of the original memory trace needs to be reactivated by the reminder in order for the associations to be strengthened, and without activation of the original memory, our no-reminder controls did not benefit from the \tglu-mediated rescuing of memory recall.

The long term effect of \tglu is in accordance with the idea that \gls{ad} pathology is a vicious loop, that the effect of the \gls{ad} pathology including synaptic degeneration and aberrent neural activity in turn accelerates the signature pathologies. In the current study, we an acute \tglu treatment is able to have lasting effect at least \SI{24}{\hour} when the \tglu is not longer present, suggesting that in the TgCRND8 mice, which is a model of early \gls{ad}, correcting the synaptic pathology is not only able to rescue the cognitive ability, but may also prevent the vicious loop of \gls{ad} patholgy in the short term. Previous report of chronic \tglu treatment in a mouse model of \gls{ad} results in a reduction of neuritic plaques, supporting the hypothesis that rescuing circuit function in \gls{ad} can protect neurons from the \gls{ad} pathology. While more research is required to investigate long-term effect and outcome of \tglu in \gls{ad}, our result and previous studies \citep{roy16, migues16, dong14} suggest that interventions that restores network circuit functions in \gls{ad} can enhance cognitive function and potentially protect the neurons from \gls{ad} pathology. 



\chapter{Conclusions and Future Directions}

\section{Conclusion}

In the current project, we aim to explore the circuit mechanism of the hippocampus function, and how its deficit contributes to coginitive symptoms in \gls{ad}. As one of the requirement for investigate neural circuitry is recording neural activity from ensemble of neurons simultaneously, we first designed and built a miniature flourescent microscope which is implantable on a mouse's head. Combined with a calcium indicator such as gCaMP, we have shown that the mini-microscope is able to record hundreds of neurons in \gls{ca1}, and tens of neurons in \gls{la} simutaneously in freely behaving mice. In addition, we have extended the ability of mini-microscope to include a second colour channel, which can be potentially used to identify neural subpopulation and other brain structures at the same time of neural activity recording.

In the second part of this project, we used our mini-microscope to investigate circuitry deficit in a mouse model of early \gls{ad}, TgCRND8. We recorded from \gls{ca1} when the mice was learning and recalling a contextual fear memory. We have found that TgCRND8 mice have a significant memory deficit, showing less freezing during contextual memory test. We have found this deficit is a result of significant shorter freezing bouts, but not a reduced number of freezing bouts, suggesting the memory deficit may be a result of maintaining the expression of the fear memory.

From the calcium recording, we have found that \gls{ca1} neurons in TgCRND8 mice are hyperactive, and the cell activity of TgCRND8 mice contain less information about the behavioural state of mice during memory recall, and this effect is independent of spatial location of the mice, suggesting it's a effect reflecting recall of the fear memory. 

Next, we took advantage of machine learning methods to investigate how memory information is recalled in the \gls{ca1} circuitry. We differentiates the cellular signal, where individual neurons are independently considered, and a network signal, where the activity of all neurons are considered using a \gls{nbc} and \gls{gsvm}, respectively. The two classifiers are trained on the ensemble neural activity to predict the behaviour of the mice at each time point. We have found that even though individual neurons in \gls{ca1} contain less information about the behavioural state of the mice, the ensemble of neurons is still able to create an accurate prediction of when the mice were freezing. Moreover, we have found that the \gls{gsvm} shows a significant better performance than the \gls{nbc}, suggesting that a significant amount of information is encoded in the synergy between activity of single neurons.

The performance of \gls{nbc} and \gls{gsvm} suggest that these classifiers are able to recognize a neural activity pattern for fear memory recall. We then investigated how this pattern emerge at the beginning of memory recall. We aligned the classifier predictions to the time when the mice start to freeze. Interestingly, we have found a gradual decrease of prediction accuracy before the mice froze, where the classifiers start to falsely predict freezing before the behaviour occur. This result show that the fear memory pattern emerges gradually before the behavioural change, corresponding to the pattern completion process in hippocampus. We found that in the \gls{nbc}, all groups show similar timing in the pattern completion, suggesting that in the TgCRND8 mice, cellular signal for the pattern completion is not impaired. However in \gls{gsvm}, TgCRND8 mice show no pattern completion, suggesting that in TgCRND8 mice, even though individual cells show activity pattern of freezing, these activity are not synchronized, and cannot form a pattern across the network. 

In addition, we found in TgCRND8 mice, the network state tend to linger at the classification boundary during freezing, while the other groups are able to dip into the freezing attractor state. This result suggest that in TgCRND8 mice, the attractor state for the expression of the memory is not robust, and suseptible to interuption from minor deviation. This result explains the behavioural findings that the TgCRND8 mice are unable to maintain the freezing, and more likely to be distracted. 

\Gls{ad} is characterized by significant synaptic loss, and a bias from \gls{ltp} to \gls{ltd}. Here we investigated whether rescuing synaptic function in \gls{ad} is able to restore hippocampal circuitry function and subsequently behaviour. In order to do this, we used the well-characterized peptide \tglu. \tglu blocks \gls{ampar} endocytosis, therefore increases synaptic strength, and shifts synaptic plasticity towards \gls{ltp}. We have found that \tglu treatment before contextual fear memory training is able to rescue the hyperactivity phenotype found in TgCRND8 mice. Moreover, the rescuing effect is long lasting: during memory testing the treated TgCRND8 mice show normal cell activity level, better information content for behaviour, and also a pattern completion process similar to \gls{wt}. Interestingly, we have found that \tglu treated \gls{wt} mice show an earlier pattern completion process, suggesting that effect of \tglu may promote \gls{ltp}, and this leads to a strengthening of the memory network, allowing the network pattern to be retrieved with a smaller cellular signal. 

We then investigated whether the memory deficit in TgCRND8 mice is due to impaired memory encoding or impaired memory recall. We trained the mice with contextual fear conditioning, gave the mice a brief reminder with \tglu treatment next day, and test the fear memory on the following day. We have found \tglu treatment during a brief reminder is able to rescue the fear memory in TgCRND8 mice, while vehicle treated TgCRND8 mice still show the memory deficit. This result suggest that the contextual fear memory is at least encoded one day after training, and the memory deficit can be largely attributed to memory recall deficit. Moreover, we have found that the memory reminder is necessary for the rescuing effect of \tglu. This supports our calcium imaging result that the \tglu rescues memory recall by strengthening the active memory trace. 

In conclusion, in the project we have developed a miniature flourescent microscope for calcium imaging in freely behaving mice, and use it to image \gls{ca1} neural activity in \gls{wt} and the mouse model for \gls{ad}, TgCRND8 mice. We have found TgCRND8 mice have contextual fear memory deficit, hyperactive \gls{ca1} neurons, and show less information content about behaviour. We found the memory recalled by TgCRND8 mice is unstable, both from neural activity and behaviour. \tglu treatment which potentially enhance \gls{ltp} can rescue both the circuitry disfunction and behaviour in the TgCRND8 mice, suggesting the importance of restoring circuitry funciton in potential treatment in restoring cognitive functions in \gls{ad}.

\section{Future directions for mini-microscope development}
\subsection{Technical improvement for the mini-microscope}

In the current project, the focus of our mini-microscope manually adjusted, and fixed during the whole imaging session. This only allows us to image a \SI{400 \times 400 \times 50}{\um} rectangular box \SIrange{50}{150}{\um} below the lens. If the focus can be adjust fast enough during recording, it may be possible to provide a z-axis scan. This will allow recording of more neurons, or in case of cortex where neurons are heterogeneous across the z-axis, gives the experimenter some control to anatomy. 

The most basic change to this is instead of manually turning the camera up and down, this part can be controlled by a miniature motor. While this can provide a easy auto-focus mechanisms, it will not be fast enough to give z-scan during recording. Moreover, the addition of an extra motor on the mini-microscope can add significant weight, making it too heavy for small rodents such as mice. 

Alternatively, a liquid lens can be used to replace the barrow lens we currently used to focus light to the camera. A liquid lens contains two immiscible liquids, and the curved interface between the two liquid provide the function of a lens. The focus of the liquid lens can be changed by applying voltage across the two liquids, which will bend the meniscus of the two liquids accordingly \citep{kuiper04}. The liquid lens allows focus change in several milliseconds, and is fast enough to provide a real time z-scan with mini-microscope. However, currently commercially available miniaturized liquid lenses are still not easy to find, and they often have long focus length and can often add significant weight to the mini-microscope. 

In the current project, the mini-microscope is connected to the computer through a cable. Although the mini-microscope is compatible with the majority of the behavioural paradigms, the connection cable can make some paradigm hard to perform, especially in environments with overhead structures, or during long-term recording sessions. Efforts can be put to create a wireless version of the mini-microscope. And again, the size and weight of the transmiting circuit and battery can be prohibitive on small rodents. However, recent development have showed that in a resonent cavity, electrical power can be delivered wirelessly using radio frequency and a minimal implant weighing \SI{20}{\mg} \citep{montgomery15}. This can potential provide a way for a wireless power source, and if combined with a minimal transmission circuitry, it is possible to allow wireless long-term calcium imaging in freely behaving animals. This will be very useful for investigating neural circuitry for slower neural circuitry process such as consolidation. 

\subsection{Combining mini-microscope with other techniques for investigating circuitry function}

One important improvement for the construction of the mini-microscope is to combine it with other methods for investigating neural circuitry mechanisms. This will allow relating the neural activity to other physiological, anatomical and molecular information to provide a whole picture of the function of neural circuit. 

First, efforts can be put for \textit{post mortem} identification of the field of view from the mini-microscope. This would allow the mini-microscope data to be enriched with those from \gls{ihc} or \gls{fish}, which can potentially provide detailed molecular and cellular information about the neurons, and offer an explanation how subpopulations of neurons with different molecular markers coordinate for circuitry behaviour. 

However in order to reach this goal, the optical property of the mini-microscope has to be accurately characterized. Efforts will need to be put on quantification of the curvature, range, and depth of the field of view. Measurement of these parameters allows mapping the cells identified in the mini-microscope to their original x--y plane, and provide constrains to the z-axis. The next step is to construct a 3D image of the \textit{post mortem} tissue, potentially using tissue clearing techniques such as \gls{clarity}. The reconstructed calcium imaging cells can then be mapped to the 3D image of the \textit{post mortem} tissue, allowing further molecular characterization of the recorded neurons. 

Secondly, it is also possible to combine mini-microscope recording with \gls{lfp} measurement for concurrent identification of brain oscillations. This addition does not pose any theoretical challenge, as the \gls{ltp} measurement only involves several extracellular metal eletrodes, and \gls{ltp} recording techniques are well established. However to shield potential interference from the camera chip, the \gls{lfp} signals may need to be amplified and digitized on the headset before the signal is transmitted. This may require extra electronic circuitry, and provide a miniaturization challenge for small rodents such as mice, where both the weight and size of the implant are restricted. 

However, our investigation of pattern completion in \gls{ad} can greatly benefit from a potential combination of \gls{lfp} and mini-microscope. Since we have found that the network signal, not individual cellular signal, is important in mediating the memory recall deficit, it suggest that the neurons in \gls{ad} are not synchronized to form a pattern. In the hippocampus, oscillations in theta and gamma frequency has been implicated in learning and memory, and hypothesized to provide a reference frame to which neurons can align their activity. Given previous reports that in other mouse model of \gls{ad}, both theta and gamma oscillation are degraded \todo{cite}, it is possible the pattern completion deficit is a consequence of degraded oscillation. And if so, artificially restoring the oscillation by stimulation of the neurons, either electrically or optogentically, should be able to restore the pattern completion process in the \gls{ad} mice. 


\section{Effect of \tglu in strengthening memory trace}

We hypothesized that the rescuing effect from \tglu is because of strengthening of the memory trace, however direct evidence is still required to show that the connection of neurons involved in the contextual fear memory are strengthend after \tglu treatment. This hypothesis can be directly tested by combining the mini-microscope recording with two-photon imaging. 

\citet{carrillo-reid16} have shown that a strong, optogenetically imprinted neural ensembles can be recalled by reactivated by stimulating a single neuron in the ensemble. A similar experiment can be conducted for evaluation of the fear memory ensemble. First, an excitatory opsin can be introduced to the cells of interest. After mini-microscope recording of contextual fear conditioning, the mice can be head-fixed on the stage of two-photon microscope, where the same cells can be imaged through the \gls{grin} lens, and neurons involved in the memory trace can be identified. Part of the memory trace can then be optically stimulated, and if the connections between the fear memory ensemble is strong, a partial activation of the ensemble should be able to activate the rest of the ensemble. Therefore we predict in \tglu-treated mice, activation of a smaller part of the ensemble is able to reactivate the whole memory trace. 

\section{pattern completion and pattern separation}

In the current project, we recorded from \gls{ca1}, which is considered an output of the hippocampus circuit, after both pattern separation process in \gls{dg} and pattern completion process in \gls{ca3} \todo{cite}. We have only tested the contextual fear memory using the same context, which would primarily involve the pattern completion process. However, it would also be interesting to see, whether the \tglu treatment also improves the pattern-separation process. 

If the pattern completion process is enhanced by the \tglu treatment without affecting the pattern separation process, it follows that the patterns are likely to complete with a smaller partial activation. This suggest that memory strengthened by \tglu treatment is more generalized: since a similar environment may activate part of the trace, and this leads to completion of the fear memory pattern and generates corresponding behaviour. However, this hypothesis is not supported. \citet{migues16} gave chronic \tglu-treatment in dorsal hippocampus after mice were trained with object recognition, conditioned place preference, or contextual fear conditioning. Interestingly, the authors found \tglu treatment not only prevents generalization, but in fact allow the mice to distinguish different memory testing environment more than the vehicle treated group \citep{migues16}. 

\citet{migues16}'s result may not be surprising, given that \tglu treatment improves \gls{ltp}, which has also been cardinal for the pattern separation process in \gls{dg} \citep{rolls13}. However, neither \citet{migues16} nor us distinguished \gls{ca3} and \gls{dg} in \tglu treatment, and therefore are unable to tease out the effect of \tglu on pattern separation and pattern completion separately. A future experiment will be limiting effect of \tglu in either \gls{ca3} or \gls{dg}, and observe both the \gls{ca1} activity and the mice's behaviour. We hypothesized that if \tglu treatment in \gls{ca3} will result in an enhancement of pattern completion as well as behavioural generalization, and treatment in \gls{dg} will result in an enhance pattern separation and behavioural discrimination. 

A more direct way, but potentially harder approach to investigate the involvement of pattern separation process in to use the mini-microscope to image \gls{ca3} and \gls{dg} simutaneously, and obsereve the pattern completion and pattern separation process directly. This potentially can be done using a doublet \gls{grin} lenses, where a small, thin relay \gls{grin} lens is glued to a large \gls{grin} lens. With proper placement, it allows imaging \gls{ca3} using the rest of the surface of the large \gls{grin} lens, and the thin \gls{grin} lens will be inserted above \gls{dg} to relay image to the large \gls{grin} lens. If combined with connectivity studies, data from this approach allows direct validation of the computational model of hippocampus. Further more in a disease model, having information of both \gls{dg} and \gls{ca3} cell activity as well as their connectivity enables researchers to pin-point the functional impairment in the hippocampus circuitry.


\section{Effect of \tglu in correcting \gls{ad} pathology, potential treatment target}

In the current project, we have shown that the \tglu treatment is not only able to rescue the behavioural deficit of the TgCRND8 mice, but also able to rescue circuitry function in these mice. Given that the aberrent circuitry activity in \gls{ad} promotes \gls{ad} pathology, it is worth investigate whether \tglu, and its target of \gls{ampar} endocytosis process, can be targeted to alleviate the \gls{ad} pathology. Results from \citet{dong14} show that chronically \tglu-treated \gls{ad} mice have fewer number of amyloid plaques compared to vehicle treated mice after memory training. However, it is still unknown whether \tglu treatment can lead to reverse of the \gls{ad} pathology. Instead of comparing to vehicle-treated controls, chronically treated \gls{ad} mice can be longitudinally compared to mice before treatment to characterize the effect of \tglu on \gls{ad} progression.

Moreover, \tglu, being an interfere peptide, is not a good candidate for drug treatment because of the inferior \todo{drug property}. Future studies can also focus on small molecules having the same effect of \tglu with better 

- understanding the link between neural activity and ad pathology:
    long-term, co-image with plaques







