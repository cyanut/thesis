\chapter{Memory traces in amygdala}

\section{Introduction}

\section{Methods}

\subsection{Animals}
Wildtype C57/BL6 $\times$ 129 F1 mice of 3-month age were used in the experiment. All animals were caged in groups of 4 or 5, with a 12-hour light\slash dark cycle. Food and water are provided \textit{ad libitum} to all animals.

\subsection{Viral vectors}
 p1005 plasmid expresses \gls{gfp} under \gls{cmv} promoter. CREB--p1005--\gls{gfp} plasmid was constructed by inserting \gls{creb} \gls{cdna} into p1005 plasmid, under the control of \gls{hsv} IE4/5 promoter. Plasmids were packaged with a replication-defective \gls{hsv} helper virus as previously described \citep{neve05}. All virus was concentrated by sucrose gradient, diluted in 10\% sucrose and stored in \SI{-80}{\celsius}. A typical titer of the virus is \num{1e7} infectious unit per \si{ml}. 

\subsection{Surgery}
See general methods.\todo{refer to general methods}

\subsection{Auditory fear conditioning}
See general methods.\todo{refer to general methods}

\subsection{\Acrlong{cpp}}
See general methods.\todo{refer to general methods}

\subsection{\Acrlong{fish}}
\todo{description why catfish}
\subsubsection{Tissue preparation}
After behavioural experiments, animals were anaesthetized with isoflurane and decapitulated. The brains were quickly dissected, frozen on dry ice, and stored in \SI{-80}{\celsius}. At least 1 day after, the brains were taken out for slicing and equilibrated in the cryostat at the optimal cutting temperature (typically \SI{-20}{\celsius}, with blades, brain matrix, mold, and mounting base. The brains were then cut on the brain matrix into \SI{1}{\cm} trunks containing the structure of interest. A thin layer of embedding medium (\todo{info about OCT compound}) was added to the mold, and 4-5 brain trunks from different conditions were quickly mounted in the mold. The brains are sliced into \SI{20}{\um} coronal slices and melt-mounted on Superfrost Plus slides (VWR\todo{right way to put this}). The slices were kept in \SI{-80}{\celsius} until hybridization.

\subsubsection{Riboprobe synthesis}
Template \gls{dna} plasmids were linearized by restriction enzyme digestion at the 5\' end of the target gene and purified. Riboprobes were synthesized using commercially available \textit{in vitro} transcription kit (\todo{in vitro transcription detial}) with \gls{dig} or flourescein labeling mix (\todo{labeling mix detial}). Turbo DNase were then added to the mixture, and incubated at \SI{37}{\celsius} for \SI{15}{\min} to remove the \gls{dna} template. The DNase was inactived by \SI{0.05}{\Molar} \gls{edta}. The riboprobes were purified by a \gls{rna} purification column (\todo{RNA column info}). A small sample of the riboprobes were taken out for quantification of the yield on a spectrophotometer (Absorbance at A\textsubscript{260}) and run on a denaturing gel for confirming the integrity of the probe. The rest of the probeswere stored in \SI{-80}{\celsius}.

\subsubsection{Hybridization and staining}
A selection of slides evenly distributed in \gls{a/p} were selected to cover \gls{la}. The slides were thawed on a filter paper and loaded on an autoclaved metal rack. The slides were lightly fixed in cold 4\% \gls{pfa} for \SI{10}{\min}, then carried through fresh acetic anhydride solution (\SI{10}{\min}), 2$\times$\gls{ssc} (\SI{5}{\min}), 1:1 methanol\slash acetone (\SI{5}{\min}), and 2$\times$\gls{ssc} (\SI{5}{\min}). The slides were then treated with prehybridization buffer (Sigma\todo{prehyb info}) for \SI{30}{\min}. The riboprobes were diluted to \SI{0.67}{\ng\per\ul}, heated to \SI{90}{\celsius} for \SI{7}{\min} and cooled on ice. Each slide was applied with \SI{150}{\ul} riboprobe solutions, coversliped, and incubated in a humid chamber overnight at \SI{56}{\celsius} for hybridization.

After hybridization, slides were treated with \SI{1}{\ng\per\ml} RNase A at \SI{37}{\celsius} for \SI{30}{\min}, washed with 2$\times$ \gls{ssc}, incubated 2 times with 0.5$\times$\gls{ssc} at \SI{56}{\celsius} for \SI{30}{\min} each, and washed with 2$\times$\gls{ssc}. the slides were then incubated with blocking solution at room temperature for \SI{30}{\min} and then treated with Alexa-488 conjugated anti-\gls{gfp} (1:100, \todo{Ab info}) and \gls{hrp}-conjugated anti-\gls{dig} antibodies (1:300, \todo{Ab info}) for \SI{2}{\hr} to detect hybridized \gls{gfp} and \gls{arc} \gls{rna} probes. The \gls{arc} signals were amplified using \gls{tsa}-biotin kit and stained by Alexa 568-conjugated streptavidin (1:300). The nuclei were counter-stained with Hoechst 33258.

\subsubsection{Analysis}
Images were taken on a laser-scanning microscope. The parameters were optimized for \gls{arc} signal and kept constant across slides. Confocal stacks of 1um were taken by 40x objectives. Small-uniform nuclei from glial cells and nuclei on the top or bottom stacks were excluded from the counting. Nucleus \gls{ieg} signals are characterized by a double puncta that overlaps with nulei counter-staining. %\gls{arc} and \gls{gfp} transcription foci  were examined for each included nuclei. Total number of cells, \gls{arc} positive cells (\gls{arc}+), \gls{arc} and \gls{gfp} double positive cells (\gls{arc}+/\gls{gfp}+) were counted. \gls{arc} ratio was calculated by dividing \gls{arc}+/\gls{gfp}+ with \gls{gfp}+.

\section{Results}
\section{Discussion}
