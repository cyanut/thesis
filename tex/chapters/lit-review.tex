\chapter{Literature Review}

\section{Clinical presentation of \gls{ad}}
\subsection{Prevalence}
It is estimated that 35.6 million people has dementia worldwide, costing more than US\$ 604 billion each year, and create heavy burden to their family members and caregivers \citep{who13}. \gls{ad} is the most common form of dementia, contributes to \SI{60}{\percent} -- \SI{80}{\percent} of the cases \citep{ad16}. It is estimated to affect \SI{11}{\percent} of population at age 65 and older in the North America, and the risk triples to \SI{33}{\percent} for individuals beyond 85 \citep{hebert13}. The real number of patients affected by \gls{ad} is much larger, as instances of \gls{ad} are often under-diagnosed and under-reported \citep{barrett06, zaleta12}. While the rate of \gls{ad} in population beyond 65 is stable over years, however as the population ages, the burden of \gls{ad} is expect to continue to rise in the future. It is estimated the instances of \gls{ad} will double every 20 years \citep{who13, hebert13}. This will translates to more than 8 million instances in the United States in 2030, and as many as 16 million in 2050. \gls{who} estimated in 2040, cases of dementia worldwide will reach 81.1 million, most of which are contributed by \gls{ad} \citep{who13}. 

\subsection{Progression}
The \gls{ad} is named after Alois Alzheimer, who described the disease in 1906 \citep{goedert06}. One of his patients, Auguste D., was admitted with progressive memory loss, hallucinations and focal symptoms. After her death, Dr. Alzheimer examined her post-mortem brain tissue with silver staining, and made the crucial observation of plaques and neurofibrilary tangles, which become the definitive biomarkers of \gls{ad} \citep{goedert06, dubois16}. \gls{ad} is characterized by progressive decline of memory, learning ability and other cognitive functions. Post-mortem examination of patient's brain is characterized with amyloid plagues, neurofibrilary tangles and significant loss of neural tissue.

\subsubsection{Preclinical stage}

It is now known the biology of \gls{ad} starts well before clinical symptoms appear \citep{dubois16}.  Neural tissue loss can be detected by \gls{mri} before the onset of \gls{ad} \citep{jack92, scheltens92, chetelat03}, and changes in brain structure from \gls{mri} imaging can predict whether aging participants with \gls{mci} will develop \gls{ad} \citep{jack99}. Moreover, \gls{pet} imaging agents which binds to amyloid plaques have been developed in 2003 \citep{mathis03}, and more recently, \atau tracers have also been discovered \citep{maruyama13, okamura13}. These radiopharmaceuticals allows \textit{in vivo} imaging of amyloid plaques and \atau tangles, and the results consensually show that presence of plaque deposition and \atau tangles predicts \gls{ad} risk in \gls{mci} or asymptomatic participants \citep{klunk04, chien14, sepulcre16}.

Another line of evidence for preclinical \gls{ad} comes from biomarkers detectable from \gls{csf}. As \gls{csf} is circulating extra-cellular space in the brain, its molecular composition reflects that of the \gls{cns}. Core biomarkers such as \abeta\tsb{42}, \atau and hyperphosphorylated \atau are directly related to \gls{ad}, and their levels in \gls{csf} has been consistently shown to differentiate \gls{ad} from other aging-related cognitive disorders \citep{blennow10}. Moreover, longitudinal studies of \gls{csf} biomarker levels in families with autosomal-dominant \gls{ad}, where the participants have a predictable age of \gls{ad} onset, have shown detectable changes of \abeta\tsb{42} and \atau concentrations in \gls{csf} 15--20 years before symptom onset \citep{bateman12, fagan14}. Studies in asymptomatic and \gls{mci} patients have shown a similar timeline, where the biomarker changes are detectable 5--10 years before \gls{ad} symptoms onset \citep{buchhave12, vos13}. 
\todo{MCI} 

In conclusion, research in preclinical \gls{ad} suggest that the pathology of \gls{ad} exists as a continuum. In response to recent research result, biomarker evidence has been proposed to be included to increase accuracy of diagnosis, together with the recognition of preclinical \gls{ad} stage \citep{ad16}. Recognizing \gls{ad} before the onset of clinical symptoms also provides an optimal window for early intervention and optimization of treatment. 

\subsubsection{Clinical stage}
As a result of the broad clinical \gls{ad} pathology spectrum, the diagnosis procedure for living humans is complex. The current \gls{ad} diagnosis recommendation from the \gls{niaaa} requires the patient to meet the diagnosis of dementia, where the patient need to show cognitive or behaviour decline that:
\begin{enumerate*}[label={\alph*)}, font={\bfseries}]
    \item disables usual activity or work, 
    \item reported both from personal history and an objective cognitive assessment, and
    \item cannot explained by other major mental disorder.
\end{enumerate*} \citep{mckhann11}.
Diagnosis of \gls{ad} is further split into 3 categories, probable \gls{ad}dementia, possible \gls{ad} dementia and probable \gls{ad} dementia with evidence of the \gls{ad} pathophysiological process. To meet the probable \gls{ad} dementia, the cognitive decline from the patients needs to be gradual over months and years, and the initial symptoms needs to be a deterioration of memory, language, visuospatial or execution function. In possible \gls{ad} dementia, the patients may have a sudden onset or mixed aetiological origin of the symptoms compared to probable dementia category. The last category incorporates biomarker evidence in the diagnosis of \gls{ad}, which can increase the confidence that the symptoms from probable \gls{ad} dememtia is a result of \gls{ad} pathophysiological process \citep{mckhann11}. However a definitive diagnosis of \gls{ad} can only be obtained post-mortem, where the brain tissue is immunohistochemically stained against \abeta, \atau for \gls{ad} pathophysiology, and neuritic plaques for neuronal damage. A diagnosis of \gls{ad} requires significant presence of amyloid plaques, neurofibrillary tangles, and neuritic plaques in various brain regions related to memory, language and other cognitive functions \citep{hyman12}.

Clinical assessment of \gls{ad} progression can be accomplished using \gls{gds} scales \citep{reisberg82} and \gls{fast} \citep{sclan92}. \gls{ad} patients generally begins with decline of memory, language, and spatial navigation performace. The symptoms then progress to difficulties in routine tasks, change in emotion and personality, and disturbance in diurnal rhythms. In the end, the patient shows completely loss of verbal ability, problems in toileting and eating, as well as psychomotor problems, and require full time assistance to survive \citep{reisberg82, sclan92}.

While the stage of \gls{ad} is consistent, the rate of progression is highly variable from patient to patient \citep{komarova11, tschanz11}. This suggests that \gls{ad} patients are a heterogeneous population, and the progression of \gls{ad} may under influence of various latent factors. This is systemetically investigated in the Cache County Study on Memory in Aging, where a population of \num{5000} eldly residents in Cache County, Utah, USA are followed for up to \num{12} years, and most of the participants are followed until their deaths. The study has found several factors that are predictive of rate of cognitive decline in \gls{ad}. Cardiovascular disease history predicts a faster rate of decline \citep{mielke07}, and better general health predicts otherwise \citep{leoutsakos12}. Moreover, aspects of the caregiving environment is also influential for the decline rate of congitive functions in \gls{ad} patients. For example, \gls{ad} patients will benefit from more engagement in cognitive stimulating activities \citep{treiber11} as well as a closer relationship to the caregiver \citep{norton09}.

As \gls{ad} is a progressive disorder with no effective treatment, the prognosis for an \gls{ad} patient is not bright. The average survival duration after diagnosis is 4--8 years \citep{larson04, helzner08}. Moreover, patients will spend \SI{40}{\percent} of the time after diagnosis with the most severe stage of \gls{ad} \citep{arrighi10}. Given the prevalence of \gls{ad} and amount of burden it creates, delaying the progression of \gls{ad} by only months can reduce more than \$\num{2000} per year even if no effective treatment is available \citep{zhu06}. This will translate to more than \$15 billion total reduction of cost over the a decade in Canada \citep{adc10}.

\subsection{intervention}
\subsubsection{drug treatment}
Currently only four drugs are approved for \gls{ad}: three \gls{ache} inhibitors donepezil, rivastigmine and galantamine, and one \gls{nmdar} antagonist memantine \citep{nelson15}. Rivastigmine and galantamine are approved for treatment of mild-moderate \gls{ad}, and donepezil is available for \gls{ad} patients at all stages \citep{bassil09, smith09}. Among these, memantine is only available to moderate-severe \gls{ad} patients, used either by itself or combined with donepezil \citep{nelson15}.

\Gls{ache} inhibitors aims to increase amount of \gls{ach} concentration, which has been found depleted in the \gls{ad} brain (discussed in \ref{ach-hypo}). The \gls{ache} inhibitors act by blocking the \gls{ache}, and therefore inhibits the reuptake of \gls{ach} in the synapse, allowing its concentration to increase. \Gls{ache} inhibitors are effective on the cognitive function of the \gls{ad} patients. For mild-moderate \gls{ad} patients, large-scale double-blind studies have found significant cognitive protection effects with donepezil \citep{rogers98}, rivastigmine \citep{farlow00}, and galantamine treatment \citep{wilkinson01}. A more recent meta-analysis of double-blind studies has confirmed the effectiveness of \gls{ache} inhibitors on the cognitive outcome in mild-moderate \gls{ad} patients, and found no difference in the effectiveness between \gls{ache} inhibitors at their prescription dose \citep{tan14}.

While the effect of \gls{ache} inhibitors is consistent, it however only provides a small improvement of the cognitive function in \gls{ad} patients, which is not likely to be clinical meaningful on individual patients. Moreover, not all patients respond to \gls{ache} inhibitor treatment, and the factors influnecing drug responsiveness is still unknown \citep{putt2006}. The \gls{ache} inhibitors unfortunately also lose their effectiveness as the patient progresses beyond moderate stage \citep{gillette-guyonnet11}, except donezepil, which has been found at high dose mildly effective in moderate-severe \gls{ad} patients \citep{sabbagh13}. 

Memantine aims to reduce neuronal excitotoxicity.  


\subsubsection{treatment targeting the \gls{ad} pathophysiology}
other: clearing amyloid plagues
immunotherapy
\subsubsection{treatment targeting restoring circuit activity}
\subsubsection{preventive intervention}
\subsubsection{conclusion}

preclinical: \citep{malek-ahmadi16} \citep{reiman16}

\section{neuropathology in \gls{ad}}
\subsection{Cholinergic hypothesis\label{ach-hypo}}
\subsection{Amyloid hypothesis}
\subsection{Tau hypothesis}


\section{GluA2, synapses}
\subsection{Synaptic effect of A\textbeta{} accumulation}

\section{synaptic deficits in \gls{ad}}

\section{Circuitry mechanism for information processing in hippocampus}
\subsection{anatomy and information flow}
\subsection{pattern separation and pattern completion}
    Kinerim J, neuron 2014

\section{Hippocampal deficits in \gls{ad}}
\subsection{synaptic function}
\subsection{Network}


\section{Methods and tools for investigating neural population activity}
\subsection{\textit{In vivo} electrophysiology recording}
\subsection{Immunohistochemistry and \textit{In situ} hybridization against \glspl{ieg}}
\subsection{Two-photon calcium imaging}
\subsection{Miniature endoscopes}
\subsubsection{Fiber bundle based fluorescence endoscopes}
\subsubsection{\gls{grin} lens based fluorescence endoscopes}
\subsection{Conclusion}


\section{Hypothesis and Research Aims}
2-5 pages
