\chapter{Literature Review}

\section{Clinical presentation of \gls{ad}}
\subsection{Prevalence}
It is estimated that 35.6 million people have dementia worldwide \citep{who13}. This illness costs more than \$604 billion USD each year and creates substantial burden to family members and caregivers of the patients \citep{who13}. \gls{ad} is the most common form of dementia, contributing to \SIrange{60}{80}{\percent}  of all clinical cases \citep{ad16}. More than \SI{11}{\percent} of population older than 65 are affected by \gls{ad} in North America, and the risk triples to \SI{33}{\percent} for individuals beyond 85 \citep{hebert13}. The real number of patients affected by \gls{ad} is likely larger than reported, as instances of \gls{ad} are often under-diagnosed and under-reported \citep{barrett06, zaleta12}. While the prevalence of \gls{ad} in individuals beyond 65 is stable over the years, as the population ages, the burden of \gls{ad} is expect to continue to rise. The instances of \gls{ad} is expected to double every 20 years \citep{who13, hebert13}. This will translate into more than 8 million \gls{ad} patients in the United States in 2030, and as many as 16 million in 2050. \Gls{who} estimated in 2040, cases of dementia worldwide will reach 81.1 million, most of which are \gls{ad} \citep{who13}. 

\subsection{Progression}
\subsubsection{Introduction}

\gls{ad} is named after Dr. Alois Alzheimer, who first described the disease in 1906 \citep{goedert06}. A patient in her early fifties named Auguste D. was admitted with progressive memory loss, hallucinations and focal neurologic symptoms. After her death, Dr. Alzheimer examined her post-mortem brain tissue with silver staining, and made the crucial observation of senile plaques and \glspl{nft}, both of which became the pathologic hallmarks of \gls{ad} \citep{goedert06, dubois16}. The importance of senile plaques in the early pathogenesis of AD was further cemented by the discovery of autosomal dominant AD. Patients with familial AD have disordered \abeta{} metabolism, namely mutations in genes presenilin 1, presenilin 2, and amyloid precursor protein (APP). For these patients, the onset of clinical symptoms is predictable and early, before the age of 65 \citep{goedert06, dubois16}.

\subsubsection{Preclinical stage \label{preclinical}}

The pathogenesis of \gls{ad} starts well before clinical symptoms appear \citep{dubois16}. Atrophy of brain structures particularly the medial temporal lobe can be detected by \gls{mri} before the onset of \gls{ad} \citep{jack92, scheltens92, chetelat03}, and is considered a diagnostic biomarker for \gls{ad} at the mild cognitive impairment stage \citep{jack99}. Moreover, \gls{pet} imaging allows detection of amyloid plaques (the senile plaques originally described by Dr. Alois Alzheimer), and more recently hyperphosphorylated \atau{} protein (which forms the neurofibrillary tangles) \textit{in vivo} \citep{mathis03, maruyama13, okamura13}. These results from \gls{pet} studies have shown to reliably predict \gls{ad} risk in \gls{mci} or asymptomatic participants \citep{klunk04, chien14, sepulcre16}.

Cerebrospinal fluid (\gls{csf}) protein levels are another biomarker for \gls{ad} in the preclinical stage. As \gls{csf} circulates through extra-cellular space in the brain, its molecular composition reflects that of the \gls{cns}. Core biomarkers such as \abeta\tsb{42}, \atau{} and hyper-phosphorylated \atau{} have consistently been shown to differentiate \gls{ad} from other aging-related cognitive disorders \citep{blennow10}. Moreover, longitudinal studies of \gls{csf} biomarker levels in families with autosomal-dominant (familial) \gls{ad}, where the participants have a predictable earlier age of \gls{ad} onset, have shown detectable changes of \abeta\tsb{42} and \atau{} concentrations in \gls{csf} \SIrange{5}{20}{\year} before symptom onset \citep{bateman12, fagan14}. Studies in asymptomatic and \gls{mci} patients display a similar timeline, where the biomarker changes are detectable \SIrange{5}{10}{\year} before \gls{ad} symptoms onset \citep{buchhave12, vos13}. 

In conclusion, there is robust evidence to support that the pathology of \gls{ad} exists as a continuum, and is present long before clinical diagnosis. As a result, neuroimaging and \gls{csf} biomarkers have been proposed to be included in the clinician's toolbox to increase the accuracy of diagnosis and earlier recognition of preclinical \gls{ad} \citep{ad16}. Recognizing \gls{ad} before the onset of clinical symptoms also provides an optimal window for intervention and optimization of treatment. 


\subsubsection{Clinical stage}

As a result of the broad clinical \gls{ad} pathology spectrum, the algorithm for diagnosis of \gls{ad} is complex and nuanced. Because an autopsy with neuropathological examination showing amyloid plaques and \gls{nft}s is the gold-standard of diagnoising \gls{ad}, ``probable \gls{ad}'' is the terminology for clinical diagnosis of \gls{ad} \citep{hyman12}. The most recent guidelines from the \gls{niaaa} describes the following criteria for probable \gls{ad}: the patient's presentation 
\begin{enumerate*}[label={\alph*)}, font={\bfseries}]
    \item satisfies the criteria for dementia (as opposed to delirium), 
    \item is insidious in onset over months to years, 
    \item is progressive,
    \item consists of cognitive deficits that are evident in domains of learning, language, visuospatial or executive functions, and 
    \item is not explained by other causes such as stroke, drug abuse, vitamin deficity, or other dementia types
\end{enumerate*} \citep{mckhann11}.
The diagnosis of possible \gls{ad} is given to patients who would otherwise meet the criteria for probable \gls{ad} but present with an atypical course or concurrent diseases (\textit{e.g.}, cerebrovascular) that may impact the person's cognition or be difficult to disentangle. The diagnostic category of probable AD with evidence of AD pathophysiological process incorporates biomarker evidence, which can increase the confidence that the symptoms from probable \gls{ad} is a result of the \gls{ad} pathophysiological process \citep{mckhann11}. 
The \gls{mmse} is the most commonly used test for detecting and tracking cognitive decline. The functional decline of \gls{ad} patients can be further measured by \gls{gds} scales \citep{reisberg82} and \gls{fast} \citep{sclan92}. \gls{ad} patients' disease progression generally begins with a decline in memory, language, and spatial navigation performance. The symptoms then progress to difficulties in routine tasks, changes in emotion and personality, and disturbances in diurnal rhythms. In the end, the patient shows completely loss of verbal ability, problems in eating, incontinence, as well as psychomotor problems. At this stage the patient will require full time assistance for basic activities of daily living \citep{reisberg82, sclan92}.

While the stages of \gls{ad} are well described, the rate of progression is highly variable from patient to patient \citep{komarova11, tschanz11}. This suggests that \gls{ad} patients are a heterogeneous population, and the progression of \gls{ad} may be under the influence of various latent factors. The Cache County Study on Memory in Aging systematically followed \num{5000} elderly residents of Cache County, Utah, USA to map the natural course for up to \num{12} years. The study found several factors that are predictive of rate of cognitive decline in \gls{ad}. Cardiovascular disease history predicts a faster rate of decline \citep{mielke07}, and better general health was protective \citep{leoutsakos12}. Moreover, various aspects of the care-giving environment are also influence the rate of cognitive decline in \gls{ad} patients. For example, \gls{ad} patients benefit from more engagement in cognitive stimulating activities \citep{treiber11} as well as closer relationships to the caregivers \citep{norton09}.

As \gls{ad} is a progressive disorder with no effective disease-modifying treatment, the prognosis for an \gls{ad} patient is poor. The average survival after diagnosis is \SIrange{4}{8}{\year} \citep{larson04, helzner08}. Moreover, patients will spend more than \SI{40}{\percent} of the time after diagnosis at the most severe stage of dementia \citep{arrighi10}. Given the prevalence of \gls{ad} and the burden it creates, delaying the progression of the disease by only a few months can reduce the cost of care by more than \$\,\num{2000} per patient per year even if no effective treatment is available \citep{zhu06}. This will translate to more than \$\,15 billion total reduction of cost over a decade in Canada \citep{adc10}.

\subsection{Pharmacological intervention \label{treatment}}
Currently only four drugs are approved for \gls{ad}: three \gls{ache} inhibitors - donepezil, rivastigmine and galantamine, and one \gls{nmdar} antagonist - memantine \citep{nelson15}. Rivastigmine and galantamine are approved for treatment of mild-moderate \gls{ad}, and donepezil is available for \gls{ad} patients at all stages \citep{bassil09, smith09}. Memantine is only available to moderate-severe \gls{ad} patients, used either by itself or combined with donepezil \citep{nelson15}.

\Gls{ache} inhibitors aim to increase the amount of \gls{ach}, which has been found be depleted in circuits implicated in memory (\textit{e.g.}, hippocampus) in the \gls{ad} brain (discussed in Section \ref{ach-hypo}). The \gls{ache} inhibitors act by blocking \gls{ache}, and therefore inhibiting the reuptake of \gls{ach} in the synapse, allowing its concentration to increase. \Gls{ache} inhibitors are effective in improving the cognitive function of  \gls{ad} patients. For mild--moderate \gls{ad} patients, large-scale double-blind studies have found significant cognitive protection effects with donepezil \citep{rogers98}, rivastigmine \citep{farlow00}, and galantamine treatment \citep{wilkinson01}. A more recent meta-analysis of double-blind studies has confirmed the effectiveness of \gls{ache} inhibitors on the cognitive outcome in mild--moderate \gls{ad} patients, and found no difference in the effectiveness between \gls{ache} inhibitors at their prescription dose \citep{tan14}.

While the effects of \gls{ache} inhibitors have been demonstrated in multiple late phase clinical trials, the effect size is small. Therefore, the cognitive benefit to individual patients may not be clinically meaningful, particularly in terms of overall function and independent living \citep{lin13}. Moreover, not all patients respond to \gls{ache} inhibitor treatment, and the factors influencing drug responsiveness are still unclear \citep{vanderputt06}. \gls{ache} inhibitors unfortunately also lose their effectiveness as the patient progresses beyond moderate stage \citep{gillette-guyonnet11}, except donepezil, which has been found to be mildly effective at high dose in moderate -- severe \gls{ad} patients \citep{sabbagh13}. 

Memantine aims to protect neurons from \gls{nmdar}-mediated excitotoxicity in \gls{ad}. \Gls{nmdar} are calcium-permeable glutamate receptors which are central for learning and memory. The \Gls{nmdar} is blocked by a \ce{Mg^2+} ion in a voltage-dependent manner. In \gls{ad}, the \ce{Mg^2+} block is compromised by \abeta{} accumulation, leading to a pathological influx of \ce{Ca^2+} into the post-synaptic neuron, and eventually cell death \citep{danysz12}. Memantine replaces the \ce{Mg^2+} and blocks the \gls{nmdar} in a \abeta{}-independent manner, therefore preventing the pathological influx of \ce{Ca^2+}. Moreover, as the memantine block of \gls{nmdar} is also voltage dependent, it does not affect physiological \ce{Ca^2+} influx during learning and memory \citep{danysz12}.

Memantine only shows significant effects in moderate -- severe \gls{ad} patients \citep{reisberg03, tariot04, schneider11}, and is often prescribed with \gls{ache} inhibitor for a positive additive effect \citep{rountree09}. Double blind studies have shown that memantine is able to improve cognitive function in moderate -- severe \gls{ad} patients \citep{reisberg03, tariot04}, although the improvement is not consistent \citep{porsteinsson08}. 

In conclusion, the currently approved pharmacological treatments of \gls{ad} are only able to provide relief from the cognitive symptoms of \gls{ad}, and their effectiveness varies from patient to patient. None of these therapies have effects on \gls{ad}-related behavioural symptoms, or show any benefits for the general functioning of  \gls{ad} patients \citep{tan14}. The drugs are, at best, able to delay the progression of \gls{ad}, but unable to alter the course or improve the outcome of the disease. Together with a high drop-off rate and reports of adverse effect in many randomized clinical trials, more effective or disease-modifying therapies are urgently needed \citep{bond12}. 

\section{Neuropathology in \gls{ad}}
\subsection{Introduction}

Understanding the aetiology of \gls{ad} will facilitate the discovery of more effective interventions. The most widely accepted cause of \gls{ad} is synapse and neuronal loss, which are the best pathological correlates of disease progression and have been the focus of clinical investigation for over a decade \citep{selkoe02, coleman04}. Unfortunately given the progressive nature of \gls{ad}, slowing or reversing neurodegeneration is difficult to achieve without a thorough understanding of its upstream players. In this section, I will first discuss the cholinergic hypothesis, upon which the currently available pharmaceutical \gls{ad} treatments are based. This is followed by a discussion of the amyloid hypothesis and the tau hypothesis. These hypotheses are based on the hallmarks of \gls{ad} pathology: amyloid plaques and \glspl{nft}.

It should be noted that although not discussed, other theories for the pathophysiology of late-onset \gls{ad} are also supported by empirical evidence, and have gained various levels of attention over the years. The glutamate hypothesis, the basis for the memantine (Section \ref{treatment}), suggests that  neuronal death is a result of excess glutamate activity. Excess glutamate activity results in a toxic overload of \ce{Ca^2+} in neurons and initiates neuronal death \citep{greenamyre88, parsons07}. The oxidative stress hypothesis proposes that heavy metal ions and other molecules accumulate in the \gls{ad} brain and induce the generation of free radicals, which cause neuronal damage either directly or indirectly \citep{markesbery97, smith10}. 

Another group of hypotheses propose that central to \gls{ad} pathophysiology is abnormal neuronal metabolism and insufficient energy supply. The mitochondrial hypothesis suggests \gls{ad} is caused by decreased mitochondrial function, leading to insufficient energy in the cell \citep{zhu06a, swerdlow14}. The vascular hypothesis states that \gls{ad} is caused by the abnormal constriction and hardening of blood vessels in the brain, making the brain deficient of energy \citep{luchsinger05, mamelak17}. The inflammatory hypothesis proposes that chronic inflammation leads to \gls{ad}. A chronic elevated inflammatory response impairs the metabolic balance of neurons, resulting in pathological changes to axons, plaque deposition and tangle formation \citep{krstic13}. 

It is important to bear in mind that although each of the hypotheses for \gls{ad} aetiology is presented as separate, linear sequences of causes and effects, there is significant overlap between them. Moreover, given the heterogeneity of late-onset \gls{ad} patients \citep{komarova11, tschanz11}, it is likely that the aetiology of \gls{ad} is multifactorial and mixed \citep{schneider07}. Grouping \gls{ad} aetiologies into multiple hypotheses is an artificial construct. In reality, a patients can show multiple aetiologies simultaneously where the symptoms are the result of their interactions. 

\subsection{Cholinergic hypothesis\label{ach-hypo}}

First proposed in \citeyear{bartus82} by \citeauthor{bartus82}, the cholinergic hypothesis is the oldest theory of \gls{ad} pathogenesis. \citet{bartus82} proposed that cognitive deficits seen in \gls{ad} are the result of impairments in cholinergic system function. The cholinergic system is important in learning and memory \citep{deutsch71}. Young human experimental participants administered scopolamine, a nicotinic \gls{ach} receptor antagonist, showed impaired memory performance at the same level as elderly participants. This impairment could be rescued by \gls{ach} agonists such as physostigmine \citep{drachman74}. Together with the evidence that cholinergic synapses in \gls{ad} undergo profound degeneration \citep{whitehouse82}, these results led to the development of \gls{ache} inhibitors donepezil, rivastigmine and galantamine. Currently, these are the only three drugs available to mild-moderate \gls{ad} patients \citep[][ also see Section \ref{treatment}]{bartus00}. 

The reception of the cholinergic hypothesis was mixed. Studies of the cholinergic neural projections in \gls{ad} found cholinergic neurons have impaired synthesis pathway for \gls{ach} \citep{milner87}. \gls{ad} is also found to correlate with a decrease of nicotinic \gls{ach} receptor density, however the muscarinic \gls{ach} receptors are not affected \citep{nordberg92, burghaus00}. Moreover, the degeneration of cholinergic synapses is most prominent in areas involved in learning and memory \citep{geula96}. 

On the other hand, the cholinergic hypothesis was challenged primarily by paucity of evidence for early development of \gls{ad}. For example, degeneration of cholinergic neurons is only found in late-stage \gls{ad}, but not present in preclinical and early-stage patients \citep{davis99}. In fact, cholinergic neurons in hippocampus and frontal cortex are increased in early \gls{ad} patients \citep{dekosky02}. Moreover, the limited clinical effect of \gls{ache} inhibitors suggests that cholinergic degeneration may be a secondary effect. This has generated significant doubt on the role of cholinergic system in \gls{ad} aetiology, and the popularity of cholinergic hypothesis has declined. The cholinergic hypothesis is now often viewed as either a result of the amyloid cascade (Discussed in Section \ref{amy-hypo}), or a cofactor for other risk factors for neuronal degeneration \citep{roberson97, contestabile11}.

\subsection{Amyloid cascade hypothesis\label{amy-hypo}}
The amyloid cascade hypothesis, first proposed by \citeauthor{hardy92} in \citeyear{hardy92}, states that the abnormal accumulation of the protein \abeta{} is a primary trigger for \gls{ad}. Accumulation of \abeta{} initiates a series of pathological events in the brain, including tangle formation, neuritic injury, and finally dysfunction and death of neurons. This hypothesis is based on the discovery that the ``senile plaques'', which characterized \gls{ad}, are composed of \abeta{} \citep{masters85}. \abeta{} is the result of sequential cleavage of the \gls{app} on chromosome 21. \Gls{app} is over-expressed in Down syndrome patients, whose genome contains an extra chromosome 21. As a result, all Down syndrome patients have early onset of \gls{ad}-like neural pathology, including amyloid plaque deposition and \gls{nft} formation \citep{wisniewski85, hardy02}. Moreover, mutation of \gls{app} (Dutch type) causes cerebral hemorrhage and deposition of \abeta{} \citep{hardy02}.

Subsequent genetic and molecular studies described how \abeta{} is processed from \gls{app} in detail. \Gls{app} can be processed in two pathways: an amyloidgenic pathway and a non-amyloidgenic pathway. In the non-amyloidgenic pathway, \gls{app} is cleaved by \textalpha-secretase into a soluble N-terminal fragments, \gls{app}\textalpha, and a 83-amino-acid C-terminal peptide, C83. C83 is then cleaved by \textgamma-secretase into a small p3 peptide and the \gls{aicd}. However in the amyloidgenic pathway, the \gls{app} is first processed by \textbeta-secretase into N-terminal fragments, \gls{app}\textbeta, and a 99-amino-acid C-terminal peptide, C99. The C99 is then cleaved by \textgamma-secretase to form an extracellular peptide containing 38-43 amino acids, \abeta{} and the \gls{aicd} \citep{barage15}. The most toxic forms of \abeta{} are \abeta[40] and \abeta[42]. While \abeta[40] is more abundant, \abeta[42] is more hydrophobic and aggregates faster \citep{walsh07}. 

Evidence from animal models and \gls{fad} patients supports a causal role of \abeta{} in \gls{ad} symptoms. Autosomal dominant mutations in genes encoding the \textgamma-secretase (\textit{PSEN1} and \textit{PSEN2}) renders the non-amyloidgenic pathway for \gls{app} processing defective, allowing more \abeta{} production through the amyloidgenic pathway \citep{suzuki94, levy-lahad95, rogaev95}. The effect of mutation in \gls{fad} has also been confirmed in mouse models in which the human mutant gene is expressed. These mouse models develop amyloid plaques in the brain, together with learning and memory deficits \citep{hsiao96, dodart99, chishti01, westerman02}. Moreover, acute treatment of \abeta{} is toxic to hippocampal and neocortical neurons, both in cultured cells and \textit{in vivo} \citep{lacor04, shankar08}. Additionally, genome-wide association studies have found a link between \gls{ad} and regions associated with increased \abeta{} levels \citep{kehoe99, myers00}.

However, more recent evidence suggests the aetiology of \gls{ad}, especially the late-onset type, is more complicated than stated by amyloid cascade hypothesis. Longitudinal \gls{pet} studies have shown that while amyloid plaque load in humans poses a significant risk of cognitive decline in both cognitively normal and \gls{mci} subjects, it is only weakly associated with cognitive decline and not predictive of disease progression in \gls{ad} patients \citep{chen14}. Moreover, amyloid plaque prevalence in elderly subjects is more than twice of \gls{ad} prevalence \citep{rowe10, ad16}, suggesting a large population carrying significant plaque load without cognitive impairment. Recent clinical trials of immunotherapies aimed at amyloid plaque removal in \gls{ad} patients have been successful in reducing the plaque load, however were unable to demonstrate clinical effect \citep{farlow15, siemers16, sevigny16}. These results suggest that the role of \abeta{} in the development of \gls{ad} is more nuanced than initially stated in the amyloid cascade hypothesis. 

Amendments to the amyloid cascade hypothesis have attempted to reconcile the relationship between plaque load and cognitive performance. It has been proposed that the soluble fraction of \abeta{} oligomers, which is most toxic to neurons, is cardinal to \gls{ad} and insoluble plaques are irrelevant \citep{ferreira15}. \abeta{} toxicity on synapses and neurons is evident \citep{ferreira15},  and synaptic degeneration correlates with the cognitive decline in \gls{ad} \citep{selkoe02, coleman04}. Moreover, \citet{canter16} proposed \abeta{} destabilizes synapses and impairs normal function of neural networks, resulting in cognitive deficits \gls{ad}. In conclusion, the amyloid cascade hypothesis is supported by the most compelling evidence to date, however evidence suggest the reality of \gls{ad} aetiology is more complex than what the initial hypothesis suggested.

\subsection{Tau hypothesis}

\Glspl{nft} are another key pathological signature of \gls{ad}. \Glspl{nft} are composed of hyper-phosphorylated \atau{} proteins \citep{grundke-iqbal86}. The \atau{} hypothesis states that abnormal function of these \atau{} proteins leads to a series of pathological events that are responsible for neurodegeneration and clinical symptoms \citep{goedert11}.

Under physiological conditions, \atau{} proteins, expressed by the \gls{mapt} gene, are widely expressed in the brain. The \atau{} proteins assemble and stabilize microtubules in the cell, which is important for cell structure maintenance and intracellular transport. However, in \gls{ad}, \atau{} is found hyper-phosphorylated by multiple kinases, including \gls{gsk3}, \gls{cdk5} and \gls{mapk} \citep{singh94}. This hyper-phosphorylated \atau{} is unable to bind to the microtubule building block tubulin, and is thought to result in destabilization of microtubules \citep{bramblett93, yoshida93, alonso94}. Unbound \atau{} proteins, under high concentrations, are likely to aggregate and form twisted paired helical filaments, which are the main component of \glspl{nft} \citep{kidd63, kuret05}. Destabilized microtubules and the toxicity of \glspl{nft} leads to failure in axonal transportation of vesicles and mitochondria, and finally the death of the neuron.

Importantly, the presence of \glspl{nft} is significantly correlated with the cognitive deficits of patients with \gls{ad} \citep{hyman12}. Moreover, the spread of \atau{} pathology is stereotypical and  corresponds to the progression of \gls{ad} symptoms. \atau{} affects brain regions important for learning and memory, such as hippocampus and amygdala before spreading out to the neocortex \citep{braak91}. These early results from post-mortem studies have recently been confirmed by \gls{pet} studies using \atau{} tracers. \gls{pet} scans allow detection of the amount and distribution of the \atau{} protein in patients \citep{ossenkoppele16, scholl16}. \atau{} is likely to be of key importance in the development of \gls{ad}.
 
\atau{} is implicated in many other neurodegenerative disorders where amyloid plaques are not present \citep[e.g.][]{williams09, mckee16}. For instance, primary \atau{} pathology, such as mutation in the \gls{mapt} gene which encodes the \atau{} protein, causes a familial form of frontotemporal dementia, but not \gls{ad} \citep{hutton98, poorkaj98}. These findings suggest that \atau{} pathology is sufficient for causing neurodegeneration. However, as there are no amyloid plaques in these disorders or mouse models of \atau{} pathology \citep{gotz04}, \atau{} pathology is unlikely to be the primary cause for \gls{ad}.

\atau{} is now viewed as a downstream element in the pathophysiology of \gls{ad}, causing direct neurotoxicity. Further evidence suggests that \atau{} may be downstream of the amyloid cascade. Plaque formation in \gls{ad} proceeds tangle formation \citep{price99}. Furthermore, \abeta{} binds to the \atau{} protein \textit{in vitro}, and accelerates \atau{} hyper-phosphorylation \citep{guo06, zempel10}. In mouse models of \atau{} pathology, \abeta{} accelerates \gls{nft} formation and neural degeneration \citep{lewis01, terwel08}.

In conclusion, evidence suggests \atau{} pathology is secondary to the cause of \gls{ad}. However, given its strong correlation with the cognitive deficits in \gls{ad} and important role in neurodegeneration, further studies aimed at understanding the underlying mechanism are necessary. 

\section{Hippocampus and memory}
\subsection{History}
The hippocampus first came into the spotlight of learning and memory research following reports of the patient, \gls{hm} \citep{scoville57, squire09}. \Gls{hm} suffered from severe epilepsy which was not responsive to pharmaceutical treatments. As a last resort, \gls{hm} received bilateral resection of the medial temporal lobe including hippocampus and brain areas surrounding it. While this surgery cured his epilepsy, it created profound anterograde amnesia, to the point that \gls{hm} forgot daily events minutes after they occurred \citep{scoville57, squire09}. The amnesic effects of \gls{hm}'s surgery were unanticipated, primarily due to the earlier work of Karl Lashley showing that memory was not stored in a localized fashion \citep{bruce01}. Further studies of \gls{hm}'s memory deficit and animal models of hippocampus lesion have suggested that hippocampus is important for governing the formation of long-term memories, which form the basis for \citet{squire91}'s theory of hippocampal function (discussed in Section \ref{hpc-squire}).

Another breakthrough in understanding hippocampal function stemmed from the discovery of place cells \citep{o'keefe71}. \citet{o'keefe71} recorded single unit signals from \gls{ca1} and \gls{ca3} subregions of hippocampus in rats, and found cells that fired only when the rat was in a specific location within the environment. These cells were consequently named ``place cells''. This discovery forms the basis for the cognitive mapping theory \citep{o'keefe76}, and has inspired further research that confirms the important role of hippocampus and surrounding areas in processing location-related information (discussed in Section \ref{hpc-spatial}).

\subsection{Anatomy of hippocampus}
The anatomy and cellular structure of hippocampus was first described by \citet{cajal93}, and more recently reviewed in detail by \citet{strien09}. While the anatomy described here is mostly based on rodent studies, the organization, connectivity and function of the hippocampus are very similar across mice, rats, monkeys and humans \citep{clark13}. 

The hippocampus is a large brain area in the medial temporal cortex consisting of multiple subregions including \gls{dg}, \gls{ca1}, \gls{ca2}, \gls{ca3} and subiculum. While original studies regarded hippocampus as a single brain region, it is now recognized that the hippocampus is heterogeneous along the dorsal-ventral axis \citep{moser98, fanselow10}. The \gls{dh} (septal pole) has strong connections to cingulate and retrosplenial cortex as well as thalamus, and is important for cognition and spatial navigation. The ventral hippocampus (temporal pole), as well as the subregion \gls{ca2}, preferentially connect to regions such as hypothalamus and amygdala, and have been implicated in emotion and stress regulation \citep{fanselow10, chevaleyre16}. In this thesis, the discussion will be limited to \gls{dh}, as it is the focus of all of the experiments.

The standard view regards the \gls{dh} as a tri-synaptic loop \citep{strien09}. The \gls{ec} collects multimodal sensory information and sends projections from layer II to \gls{dg} and \gls{ca3}, through axons called the perforant path. Axons from granular cells in \gls{dg}, called mossy fibers, then synapse onto \gls{ca3}. Pyramidal cells in \gls{ca3} then project to the ipsilateral \gls{ca1} through axons called Schaffer colaterals, and to contralateral \gls{ca1} through commissural pathway. \gls{ca1} pyramidal cells in turn synapses on the deep layer of \gls{ec} either directly or through subiculum \citep{strien09}.

While the standard view of hippocampus connectivity is unidirectional, there are also significant recurrent projections and back projections in the hippocampus. The subregion \gls{ca3} is well known to have recurrent connections, and this recurrent connection is proposed to be critical for encoding and retrieving episodic memories \citep{nakazawa02, rolls07}. Weaker recurrent connections in \gls{ca1} and \gls{dg} are also reported \citep{swanson81, ishizuka90, buckmaster93}. Moreover, outside the standard view there are also back projections from \gls{ca3} back to \gls{dg}, as well as back projections from \gls{ca1} to \gls{ca3} \citep{swanson81, li94}. Back projections from subiculum to \gls{ca1} are also reported \citep{finch83}. 

\subsection{Theories of hippocampus function}
\subsubsection{Introduction}
While it has been established that hippocampus plays a cardinal role in learning and memory, the exact role of hippocampus, and the mechanism by which hippocampus enables these cognitive processes are still unclear. Many theories have been proposed to answer these questions, and some of the most prominent theories are discussed here. I will start with the earlier Marr model \citep{marr71} which has been hugely influential on later theories of hippocampal function. The standard model of memory consolidation \citep{squire92} and the competing multiple trace theory \citep{nadel97}, are mostly influenced by the studies of the famous patient \gls{hm}, who received bilateral medial temporal lobe lesions and displayed anterograde amnesia and other cognitive symptoms. The cognitive mapping theory and its extension, relational theory, stem from the discovery of place cells in hippocampus \citep{o'keefe71}. 

Readers are also reminded that these theories should be regarded as multiple facets, instead of competing hypothesis, of hippocampal functions. There is significant overlap between the theories, and each theory aims to provide a perspective of hippocampal function. In light of this, there has been a recent push for unification of many of these theories, and each of the theories are continually supported or challenged by emerging experimental evidence.

\subsubsection{Marr Model \label{hpc-marr}}
In his seminal paper, \citet{marr71} first used mathematical terms to describe the computational aspect of brain function. He proposed that the neocortex, being a complex structure, must perform complex computational tasks, while the archicortex, with a simpler structure, must perform simpler computational functions. In terms of memory function, and based on the symptoms of patient \gls{hm}, he proposed that the role of hippocampus is to provide short-term storage for ``simple memories'', where sensory inputs are stored with minimal modification. The representation of ``simple memories'' are then transferred to neocortex, in a process where statistical regularities of the ``simple memories'' are extracted, and stored in the neocortex to form long-term memory. 

In his model, \citet{marr71} also described how the hippocampus stores ``simple memories''. He argued this task could be completed with a three-layer neural network: an input layer, a codon layer and an output layer. The input layer represents sensory patterns. These patterns are then encoded with the codon layer, which has a larger number of nodes with sparse activity in order to store patterns reliably. He also proposed that the output layer should have a recurrent connection, which will allow the system to reconstruct a pattern with only a partial input pattern. 

The Marr model is remarkable even viewed from today's perspective. While in the original paper Marr \citep{marr71} regarded the neocortex, entorhinal cortex and the whole hippocampus as the input, codon and output layer respectively, under modern interpretation the entorhinal cortex is better fit as the input layer, \gls{dg} as the codon layer and both \gls{ca3} and \gls{ca1} as the output layer \citep{willshaw15}. Not only do the large number of sparsely active granular cells in \gls{dg} and the recurrent connections in \gls{ca3} fit the anatomy described in \citet{marr71}'s model, the function of \gls{dg} and \gls{ca3} is also congruent with the model. For example, the \gls{dg} has been shown to perform pattern separation, and the \gls{ca3} is shown to perform pattern completion through its recurrent connections \citep{knierim16}. Moreover, the idea of dissociating the memory system into a fast, simple system and a slow, complex system also forms the basis of the modern theory of systems memory consolidation \citep{squire92, mcclelland13}. 

One of the most important predictions of the Marr model \citep{marr71} is the concept of pattern separation and pattern completion. The Marr model predicts that the \gls{dg}, with its large number of neurons, performs the function of pattern separation, where similar inputs are mapped onto more different representations in \gls{dg}. On the other hand in \gls{ca3}, the strong recurrent connections are able to perform the function of pattern completion, which maps partial or noisy input patterns back into their complete representation \citep{rolls13, knierim16}. 

While the processes of pattern separation and pattern completion have been inferred from \gls{dg} and \gls{ca1} ensemble activity or even behavioural performance \citep{santoro13, rolls13}, direct evidence for  pattern separation and pattern completion has not been available until recently, mostly due to the difficulty of simultaneously recording large number of neurons in behaving animals \citep{knierim16}. Using multi-tetrode arrays, \citet{neunuebel14} simultaneously recorded neural activity from both \gls{dg} and \gls{ca3} in behaving rats. The rats were placed in an environment where local spatial cues and global spatial cues were presented in conflict. The authors found while \gls{dg} responses are highly different from the degree of environmental conflict, \gls{ca3} responses are more coherent \citep{neunuebel14}. This result, therefore, directly confirms the long-standing prediction from Marr's model \citep{rolls13, knierim16}. 

However, the \citet{marr71} model is unfortunately limited by the data available at its time. The discovery of place cells \citep{o'keefe71} and other specialized position-related cells in entorhinal cortex and hippocampus, such as grid cells and head-direction cells \citep{taube90, fyhn04, hafting05}, suggest hippocampus also processes contextual information about the environment (discussed in \ref{hpc-spatial}), and is not a mere pattern store. Modern theoretical development has also shown that \citet{marr71}'s proposal of an overnight transfer from the fast system to the slow system is overly simplified \citep{squire09}. 

In conclusion, while \citet{marr71}'s model is sometimes regarded as a ``noble failure'' \citep{willshaw15}, the model and its ideas have inspired modern theoretical development and still exerts a strong influence on the learning and memory literature today. 


\subsubsection{Standard model of memory consolidation \label{hpc-squire}}
The standard model of memory extends the Marr model to take into account studies of medial temporal lobe lesion in human and animals, as well as neural imaging studies of hippocampal and cortical activity after memory encoding \citep{squire92, squire09}. First of all, it has been found that in humans with \gls{mtl} lesion, anterograde amnesia does not affect all forms of memory. Declarative memory, which includes the ability to remember facts and events, is impaired by \gls{mtl} lesion, while non-declarative memory, which encompasses learning skills and habits, is spared \citep{cohen80, squire04}. This result suggests multiple memory systems exist, and \gls{mtl} is only necessary for the formation of declarative memory. Moreover, it was also found that while \gls{mtl} lesion patients such as \gls{hm} suffer from anterograde amnesia, they also display temporally graded retrograde amnesia, as they are unable to recall events years before the lesion, but are able to recall older memories \citep{marslen-wilson75}. This result suggest that \gls{mtl} does not disengage once the memory forms, but perform an important role in a slow memory consolidation process, where memories are transformed into a more robust form and become resistant to disruption \citep{squire92}. 

As a result, \citet{squire92} proposes that for declarative memories, the hippocampus (and its surrounding regions in \gls{mtl}) serve as a fast system to quickly encode the memory. Then over a period of time, hippocampus will actively engage with cortical regions in the process of memory consolidation, which allow a representation of the memory to gradually form in the neocortex and become independent of hippocampal activity. This model was proposed by \citet{squire92} as the standard model of memory consolidation. 

The standard model received support from a wide range of animal studies \citep{squire09}. First of all, the temporally graded role of hippocampus in memory formation has been consistently demonstrated in rodents and monkeys, as shown by more than 30 studies compiled by \citet{frankland05} with a variety of lesion techniques and memory tests. Moreover, studies tracking the activity and structural change of hippocampus have supported the prediction that hippocampus will gradually disengage after initial memory encoding. For example, it has been found that over time after memory encoding, hippocampus shows gradually decreased activity and the cortex show increased activity \citep[e.g.,][]{bontempi99, frankland04a, takehara-nishiuchi06}, as well as a similar gradient of neuronal structural change \citep[e.g.][]{routtenberg00, maviel04, restivo09}. Furthermore, there is also evidence that the transferring process occurs during sleep. For example, task-related coordinated activity between hippocampus and visual cortex has been found when rats are asleep \citep{ji07}. 

The standard model is not without criticism. As the standard model is similar to the original \citet{marr71} model, the line of criticism involving place cells to \citet{marr71}'s model also applies to the standard model, as the spatial orientation of the animal in an environment critically depends on hippocampus, even long after encoding when memory consolidation is thought to be complete \citep[e.g.][]{mumby99, sutherland01, clark05}. Moreover, close examination of \gls{mtl} lesion patients reveals that they have global deficits in recalling details of episodic memory, and this deficit is not temporally graded \citep{cipolotti01, viskontas02}. Other evidence suggests that for some more demanding memory tasks, the hippocampus is also active during memory recall after consolidation \citep{ryan01, wheeler13}. These criticisms of the standard model led to the development of multiple trace theory (discussed in \ref{hpc-mtt}).

\subsubsection{Multiple trace theory (memory transformation theory) \label{hpc-mtt}}
Multiple trace theory is an extension of the standard model. Similar to the standard model,  multiple trace theory also acknowledges that the hippocampus automatically encodes all information during memory formation, and that the hippocampus initiates a representation of memory in the cortex \citep{nadel97}. However multiple trace theory differentiates itself from the standard model on several accounts. First, multiple trace theory differentiates declarative memory into episodic memory (memory of events) and semantic memory (memory of facts), and states that the two kinds of memory are encoded differently. Multiple trace theory then proposes that for episodic memories, there is no consolidation process, and the memory is represented by both hippocampus and cortex. However, upon reactivation of an old episodic memory, a new hippocampal representation is formed to link part of the cortical representation of the memory. Multiple trace theory argues that the multiple hippocampal representation provide robustness to the underlying memory, so they are less likely to be disrupted \citep{nadel97}. 

As a result of this, multiple trace theory have several different predictions, especially regarding the encoding of episodic memory, compared to the standard model. First, it predicts that reactivation of memory, no matter how old the memory is, will depend on the hippocampus, while for the standard model, older memories are independent of hippocampus. Secondly, multiple trace theory predicts that as memories get older, hippocampal activation will be stronger and more distributed among to the multiple traces, while the standard model predicts the opposite. Thirdly, multiple trace theory suggests that reactivation of a memory as new traces are formed in hippocampus and cortex provides a window to update the memory, while in the standard model reactivation of the memory should not have any effects on the underlying neural representation. 

The dissociation of semantic and episodic memories which multiple trace theory identified is well supported. For example in the retrograde amnesia caused by \gls{mtl} lesion, episodic memories are usually severely impaired, often dating back for decades or even lifetime, while semantic memories are less affected, and this impairemnt is temporally-graded \citep{kapur97, vargha-khadem97, moscovitch05}. Moreover, hippocampal activation has been consistently reported during recall of episodic memory \citep{maguire01, svoboda06}. In addition, it has been established, both in animal models and humans, that reactivation of memory creates a window to alter the original memory representation \citep{wang10, dunbar16}.

On the other hand, many predictions of multiple trace theory have not been validated. For example, hippocampal activation during the recall of a remote memory is at a level similar to that of a recent memory \citep{addis04, steinvorth06, wheeler13}, which stands in contrast to the predictions of multiple trace theory. Moreover, while multiple trace theory predicts that partial lesioning of hippocampus leads to temporal-graded amnesia, this is not found in rodent models \citep{sutherland10}, and inconsistently in humans \citep{yassa13}. The biological relevance of multiple trace theory has also been questioned, as it seems to necessitate a huge amount of storage for the multiple traces, which the hippocampus may not be able to provide \citep{yassa13}. 

In conclusion, multiple trace theory has been influential in recent studies of reconsolidation, and could potentially provide novel treatments for many psychiatric conditions \citep{dunbar16}, however evidence exists against many of its core predictions. Efforts are still being made to amend multiple trace theory to account for the available evidence \citep{moscovitch05, yassa13}.

\subsubsection{Cognitive mapping theory \label{hpc-spatial}}
Cognitive mapping theory was proposed in response of the discovery of place cells in the hippocampus \citep{o'keefe71, o'keefe78}. Place cells are neurons that respond to the position of the animal in a certain environment, independent of the behavioural state of the animal \citep{o'keefe78}. In rats, it has been found that \SIrange{20}{30}{\percent} of \gls{ca1} cells show place preference, and the ensemble of place cells is able to accurately determine the position of the animal \citep{guzowski99, o'keefe05, ziv13}. Place cell responses form rapidly when the animals is in a novel environment, and the place representation of each place cell remains stable, at least over months \citep{wilson93, ziv13}. Place cells remain the most robust neural representation of higher cognitive function observed to date.

Based on the discovery of place cells, cognitive mapping theory proposes that the main function of hippocampus is to maintain a map of the environment, in order to allow navigation and the formation of spatial memory. Further research has revealed cells responding to different features of the spatial map in the hippocampus and the adjacent entorhinal cortex, supporting the idea that hippocampus is important in maintaining a spatial map. For example, cells in the \gls{ec} have been found to encode a hexagonal lattice of the current environment at different frequencies, and have been proposed to be important for distance measurement \citep{fyhn04, hafting05, moser15}; cells encoding the border of the current environment are also reported \citep{solstad08}. In addition, cells in the hippocampus do not only respond to allocentric cues. For example, cells responding to the head direction of the animals are also reported, supporting the idea that the function of hippocampus is navigation \citep{sargolini06}. 

Cognitive mapping theory also predicts that the hippocampus should be active during spatial navigation tasks, and that lesioning the hippocampus will result in a performance deficit in these tasks. These two predictions are supported by imaging and lesion studies. Functional neuroimaging studies have shown that hippocampal activity correlates with performance of spatial navigation tasks in humans \citep{burgess02, hartley07}. In addition, patients with hippocampal lesions have deficits in learning to navigate new environments \citep{hartley07}. These human studies are paralleled by animal studies, which have shown that rats with hippocampal lesions are unable to navigate the environment using allocentric cues, however are still able to utilize egocentric cues \citep{morris06}. There is also a correlation of hippocampal damage in \gls{ad} with this symptom. While the hippocampus is one of the earliest brain structures affected in  \gls{ad} and its precursor \gls{mci}, at the same time one of the early symptoms of \gls{ad} is the inability to learn and explore new environments \citep{vlcek14}.

Cognitive mapping theory, although having received wide support, also has several weaknesses. The first criticism is that only a small proportion of cells show properties of spatial coding in hippocampus, and in fact, cells may also encode time \citep{hampson93}, odour \citep{wood99}, tactile input \citep{young94}, or reward and punishment \citep{moser08}, which have no relationship with space. Moreover, it is not clear how episodic memory is encoded by the spatial map \citep{konkel09}. In response, proponents of the cognitive mapping theory suggest a generalization of the cognitive mapping theory, where the hippocampus not only encodes the three-dimensional space, but also dimensions along time, language, and other factors \citep{burgess02}. This generalization is similar to the relational theory, which will be discussed in the next section.

\subsubsection{Relational theory}

Relational theory was proposed to reconcile two seemingly incompatible views of hippocampal function: episodic memory function and spatial navigation, and has aimed to provide an explanation of hippocampal function which applies both to animal models and to humans. First proposed by \citet{eichenbaum93}, relational theory states that the hippocampus is fundamentally a ``relation processor''. Given that multimodal sensory information converges in hippocampus, the function of hippocampus is to represent the output of the sensory system in a common multi-dimensional space, and captures the relationship between items in these sensory experiences \citep{eichenbaum93}.

Relational theory, therefore, suggests that the hippocampus is not specialized in encoding episodic memory, nor does it support for spatial navigation. Both episodic memory and spatial information are just examples of ``relationships'' captured by hippocampus. Spatial memories are relational mappings of specific events and objects to a spatial context, and episodic memories are relationship of concepts, events, and objects to time and space. The hippocampus creates these relationships in dimensions of space, time and concepts, which underlie both spatial and episodic memory \citep{eichenbaum14}.

This view of hippocampal function is able to explain the findings that hippocampus encodes time and sensory aspects of particular events other than space \citep{hampson93, young94, wood99, moser08}. Relational theory is also able to explain why the hippocampus is necessary for many memory tasks which do not involve encoding space, for example, object recognition \citep{eacott04, langston10}, trace fear conditioning (where animals associate a conditioned stimulus with a delayed unconditioned stimulus) \citep{crestani02, mcechron98} and taste aversion (where the animals associate a particular taste with a punishment) \citep{best73, gallo95}, as these tasks can all be regarded as forming an association between specific sensory outcomes and either time or context. 

There are also neural correlates of the relationship formed in the hippocampus. In animals, recalling a specific event also reactivates the representation of the place where the event occurred \citep{moita03, itskov11}, and new learning in hippocampus leads to reorganization of the cellular activity which was related to a previous, similar learning episode \citep{mckenzie13}. In humans, hippocampal activation during sequential learning only carries information about objects in learned temporal contexts, but not about objects in random contexts \citep{hsieh14}, and again, new learning of object pairings in humans activates hippocampal representations of related previously learned pairings \citep{zeithamova12}. More recent studies of hippocampal lesion patients have also revealed that these patients' cognitive deficits are not limited to episodic memory and spatial navigation, but also language, imagination and creative thinking \citep{duff09, duff13}.


On the other hand, it is unclear why certain kinds of associations are mediated by hippocampus while others are not, especially given that many of these associative learning tasks involve very similar sensory and motivation pairings but do not require the hippocampus (e.g. auditory fear conditioning, \citep{phillips92}). Relational theory also does not take the unique anatomical structure of hippocampus into account or explain how the anatomy of hippocampus gives rise to its function. Nevertheless, relational theory is able to explain a wide range of functions of hippocampus, and has the advantage of being applicable across species.


\section{Synaptic mechanisms for memory}
\subsection{Introduction}
The brain's ability to learn and remember stems from its ability to change with experience. How do neurons change, and how do changes on the cellular level give rise to the ability to learn and remember? These questions were first comprehensively tackled by Donald Hebb \citeyear{hebb49}, whose model is now famously paraphrased as ''neurons that fire together, wire together''. However, evidence for \citet{hebb49}'s proposition only started to appear more than 20 years after its publication, when \gls{ltp} - where a high frequency stimulation results in a persistent strengthening of synapses, was discovered \citep{bliss73}. After the discovery of \gls{ltp}, a variety of mechanisms for neural plasticity were also found to be important for learning and memory. On the synaptic level, \gls{ltp} and \gls{ltd}, and more formally, spike-timing-dependent plasticity have been found to regulate strength of a single synapse. Moreover, synapses themselves are plastic, and synaptogenesis (the growth of new synapses) and synaptic elimination also contributes to neuroplasticity. The discovery of neurogenesis in \gls{dg} and olfactory bulbs has also shed light on neural plasticity at the population level.

In this section, I will briefly review the evidence and mechanisms underlying \gls{ltp} and \gls{ltd}, as these mechanisms are important for synaptic plasticity (changes in synaptic strength), and critical for learning in the hippocampus. These mechanisms are also the first to be compromised in \gls{ad}. How they are affected by \gls{ad} is discussed in Section \ref{ad.synaptic}. 

\subsection{The \gls{ampa} receptor}

\glspl{ampar} mediate the majority of fast excitatory transmission in the brain. There are four types of subunit in the composition of \glspl{ampar}, namely GluA1--GluA4. The \gls{ampar} forms a heteromeric ``dimer of dimers'', consisting of two pairs of different subunits \citep{ayalon01}. In hippocampus, the majority of \glspl{ampar} are composed of GluA1/2 or GluA2/3, and synaptic \glspl{ampar} are predominantly of type GluA1/2 \citep{wenthold96, lu09}. The GluA2 subunit undergoes post-transcriptional editing. Residue 607, which lies on the channel wall, is modified from glutamine to arginine in the majority of adult neurons \citep{greger03}. This change renders GluA2-containing \gls{ampar}s impermeable to \ce{Ca^2+} ions, and predominantly mediates excitatory synaptic transmission \citep{sommer91,swanson97}. Although limited in expression, the GluA2-lacking, and therefore \ce{Ca^2+}-permeable \glspl{ampar} are implicated in mediating excitotoxicity in neurological disorders such as ischemia and \gls{ad} \citep{kwak06, whitehead17}, however the exact role these receptors play in physiological conditions is still under research \citep{whitehead17}.

\Glspl{ampar} are first synthesized in the \gls{er} of the cell as homo-dimers, and then these homo-dimers form tetramers. The teramerized receptors are then transported to the cell surface \citep{henley13}. It is now widely acknowledged that the mobilized \glspl{ampar} are inserted near the synapse, but not directly onto the synapse, potentially due to inaccessibility of the \gls{psd} to transport vesicles \citep{henley11, chater14}. Cell-surface \glspl{ampar} are highly mobile and diffuse along the cell membrane laterally. However, \glspl{ampar}, especially GluA2 containing ones, can become immobilized on the synaptic membrane \citep{borgdorff02, groc04}. The density of \glspl{ampar} on the synaptic membrane is highly correlated with the synaptic strength, and is under tight control with constant receptor exocytosis and endocytosis at extrasynaptic sites \citep{malinow02, henley11}. Therefore, the strength of a synapse reflects the balance of receptor exocytosis and endocytosis processes. 

\subsection{\gls{ltp}}

First discovered by \citet{bliss73} in the rabbit hippocampus, \gls{ltp} refers to the phenomenon of synaptic strengthening after a brief high-frequency stimulation. Since its initial discovery, \gls{ltp} has been found to be not limited to hippocampus, but prevalent across the brain \citep[e.g.][]{clugnet90}. \gls{ltp} is therefore proposed to be a general phenomenon of neurons \citep{malenka04}. 

While early studies debated whether \gls{ltp} was a pre-synaptic or post-synaptic phenomenon \citep{malinow90, bekkers90, isaac95, liao95}, more recent glutamate uncaging studies have shown that direct stimulation of dendritic spines with glutamate is sufficient to induce \gls{ltp}, therefore establishing that \gls{ltp} is primarily mediated by post-synaptic mechanisms \citep{kerchner08}. The post-synaptic molecular mechanisms of \gls{ltp} have been extensively studied, especially in the Shaffer colateral -- \gls{ca1} synapse. The initiation of \gls{ltp} starts with the activation of \glspl{nmdar} after high-frequency stimulation \citep{collingridge83}. The opening of \glspl{nmdar} results in an influx of \ce{Ca^2+} and activation of \gls{camkii}. While a rapid increase of synaptic \glspl{ampar} is observed after \gls{camkii} activation \citep{patterson10}, the exact mechanism of how \gls{camkii} activation leads to accumulation of \gls{ampar} on synapse is still under debate \citep{herring16}. 

The currently accepted hypothesis is that \gls{camkii} directly phosphorylates the GluA1 subunits of \glspl{ampar} or their associated proteins, \glspl{tarp}, leading to their increased synaptic trafficking. It has been found that GluA1 is indeed phophorylated by \gls{camkii} after \gls{ltp} induction \citep{mcglade-mcculloh93, barria97, lee03}, and that GluA1 with a mutated C-terminal which blocks the \gls{camkii} phosphorylation, failed to traffic to synapse, while mutation on other \gls{ampar} subunits does not affect their synaptic trafficking \citep{hayashi00, shi01}. Moreover, \gls{ltp} in GluA1 knockout mice is impaired, while unaffected in GluA2 or GluA3 knockout mice \citep{zamanillo99, meng03}. Similar results are also found for \glspl{tarp}, such that \gls{tarp} is phosphorylated during \gls{ltp} induction, and blocking the \gls{camkii}-mediated phosphorylation of \glspl{tarp} also prevents \gls{ampar} translocation to the synapse \citep{tomita05, sumioka10}.

While ample evidence has indicated \gls{ampar}\slash\gls{tarp} as a central player in \gls{ltp} and a direct downstream target of \gls{camkii}, this view is not unchallenged. The main criticism of the \gls{ampar}\slash\gls{tarp}-centric hypothesis is that \gls{ltp} is never completely blocked in the absence of \gls{ampar} phosphorylation, only attenuated \citep{herring16}. This is further demonstrated by molecular replacement studies. Under strong stimulation, \gls{ltp} is not affected by replacing endogeneous \glspl{ampar} with either a mutant GluA1 subunit where all \gls{ltp}-related residues are mutated, or even completely replaced by kainate receptors, which do not interact with \gls{tarp} \citep{granger13, chen03}. These results suggest alternative mechanisms for \gls{ltp} might exist. Other hypotheses have been proposed to explain these results. For example it has been hypothesized that \gls{camkii} may create holes in the \gls{psd} which capture \glspl{ampar}, and \gls{camkii} may phosphorylate recycling endosomes, which are responsible for inserting \glspl{ampar} onto synaptic membrane. Both of these hypotheses still require further evidence for validation \citep{herring16}.

\subsection{\gls{ltd}}

Opposite to \gls{ltp}, \gls{ltd} is a process whereby synaptic strength is persistently decreased after receiving low-frequency stimulation. There are several types of \gls{ltd}, and they can be classified in different ways. First \gls{ltd} can be either homosynaptic, and only happens in the synapse which receives low-frequency stimulation, or heterosynaptic, where \gls{ltd} also affects synapses which are not stimulated. Second, both homo- and heterosynaptic \gls{ltd} can then be classified into \textit{de novo} \gls{ltd} or depotentiation. In \textit{de novo} \gls{ltd}, baseline synaptic strength is persistently decreased, however in depotentiation, \gls{ltd} only reduces the strength of a previously potentiated synapse (e.g. by \gls{ltp}) to baseline, however does not affect the synaptic strength if it is already at baseline. \citep{collingridge10}. Different type of \gls{ltd} may be mediated by different mechanisms, and can co-exist in the same population of synapses \citep{collingridge10}. Here I will review the two best-studied \gls{ltd} mechanisms in the hippocampus: \gls{nmdar}-dependent \gls{ltd} and \gls{mglur}-dependent \gls{ltd}.

\textit{De novo} \gls{ltd} and depotentiation at many synapses are dependent on \glspl{nmdar} \citep{collingridge83, dudek92}. In \gls{nmdar}-dependent \gls{ltd}, low levels of \ce{Ca^2+} ions enter from \glspl{nmdar}, and activate calcineurin. Calcineurin then disinhibits \gls{pp1} by dephosphorylating inhibitor-1 \citep{mulkey93}. \Gls{pp1} then dephosphorylates its target (such as GluA1), and facilitates the removal of the receptor from synaptic membrane \citep{collingridge04}. In addition, the low level of \ce{Ca^2+} entry also activates \ce{Ca^2+} sensors such as hippocalcin \citep{palmer05}. Hippocalcin, upon activation, translocates to the plasma membrane, then binds and activates the clathrin adaptor protein \gls{ap2}. \Gls{ap2} then replaces the \gls{nsf}, de-stabilizes the \gls{ampar} on the synaptic membrane, and initiates clathrin-mediated endocytosis of the \gls{ampar} \citep{collingridge04, palmer05}. Another pathway involves a different low-affinity \ce{Ca^2+} sensor, \gls{pick1}. \Gls{pick1} mediates the dephosphorylation of GluA2, and may also play a role in endocytosis by inducing membrane curvature \citep{collingridge04, lin07}. However the exact role of \gls{pick1} in \gls{ampar} endocytosis is still under debate \citep{collingridge10}. As a result of multiple converging pathways, synaptic \gls{ampar} density is reduced, and the result is a persistent decrease of synaptic strength.

\Gls{ltd} mediated by \glspl{mglur} recruits proteins in different pathways from \gls{nmdar}-dependent \gls{ltd} \citep{gladding09}. While all types of \gls{mglur} can potentially mediate \gls{ltd}, in \gls{ca1} hippocampus mGluR5, and to a small extent mGluR1, initiate most \gls{mglur}-mediated \gls{ltd} \citep{luscher10}. Both mGluR5 and mGluR1 activate the \gls{pkc} pathway through \gls{ip3} and \gls{dag} \citep{oliet97}. \Gls{pkc} then recruits \gls{pick1} and \gls{ncs1}, phosphorylates the C-terminal of GluA2, and mediates its endocytosis \citep{bellone06, jo08}. In addition, \gls{mglur} activation in \gls{ltd} also enables rapid protein synthesis through the activation of \gls{eef2} and the \gls{mtor} pathway, which leads to the translation of proteins such as \gls{arc} and \gls{ptp} \citep{park08, zhang08}. These \textit{de novo} synthesized proteins, again, are implicated in the process of \gls{ampar} endocytosis \citep{collingridge10}.

\subsection{Homeostatic plasticity}
Hebbian plasticity, while implicated in learning and memory, is computationally unstable. Cells undergoing \gls{ltp} are involved in a positive feedback loop, and tend to become hyperactive, while cells under \gls{ltd} are excessively silenced, leading to a pathological deletion of synaptic connections. It is therefore hypothesized that besides Hebbian plasticity, additional mechanisms exist in the cell to maintain overall synaptic strength within a physiological dynamic range \citep{bienenstock82, cooper12}. Indeed, a number of mechanisms, both pre-synaptic and post-synaptic, has been found to maintain either excitability of the cell or amplitude of \gls{epsc} \citep{turrigiano98, frank06, collingridge10, chater14, wang16}. 

One of the most studied mechanisms for homoeostatic plasticity is synaptic scaling, where global synaptic strength is regulated to control the excitability of cells. It was first found in rat cortical neuron cultures, where chronic inhibition of presynaptic terminals resulted in an increase of \gls{ampar}-mediated post-synaptic currents, while chronic block of \gls{gaba}-mediated inhibition decreased post-synaptic currents \citep{turrigiano98}. This effect was later confirmed \textit{in vivo} by several studies \citep{whitt14}. In synaptic scaling, the strength of synapses across the whole cell are scaled multiplicatively by a uniform amount, opposing the direction of external changes in neural activity. Importantly, relative strength between synapses is maintained in this process, and therefore information stored in individual synapse is not lost \citep{turrigiano08}. 

The molecular pathway for synaptic scaling partially overlaps with those of \gls{ltp} and \gls{ltd}. Not only are \glspl{ampar} central to both mechanisms, but regulatory pathways and proteins involved in \gls{ltp} and \gls{ltd} are also important for synaptic scaling. Blocking \ce{Ca^2+}, \gls{camkii} or \gls{pka} prevents synaptic upscaling, while blocking calcineurin or \gls{camkiv} pathway induces synaptic upscaling \citep{goel11, kim14, ibata08}. Proteins important for modulating synaptic \gls{ampar} density such as \gls{pick1}, \gls{grip1} and the \gls{tarp} protein stargazin, have all been shown to be necessary for synaptic upscaling \citep{anggono11, gainey15, louros14}. Moreover, immediate early genes such as homer1a or \gls{arc}, which are responsive to \gls{ltp} and \gls{ltd}, are also necessary for synaptic scaling \citep{hu10, gao10}. The commonality between synaptic scaling and activity-dependent synaptic plasticity suggests that while conceptually separate, these regulatory processes are closely associated, and may fail or become compromised under similar pathological conditions \citep{fernandes16}.

\subsection{Synaptic plasticity and memory hypothesis}
\Gls{ltp} and \gls{ltd}, being consistent with the hebbian model, have been considered as mechanisms for learning and memory since they were discovered \citep{morris90, bliss93, shors97, martin00}. Moreover, \gls{ltp} and \gls{ltd} are correlated with memory formation, and have similar time-scales: both have a early phase independent of protein synthesis, and a late phase dependent of protein synthesis \citep{abel01, reymann07}. The molecular mechanisms of \gls{ltp} and \gls{ltd} also involve many key proteins necessary for memory formation, such as \glspl{nmdar} and \ce{Ca^2+} signaling pathways \citep{martin00}. This evidence has been formalized in the synaptic plasticity and memory hypothesis, which proposes that activity-dependent synaptic plasticity is induced during memory formation, and is both necessary and sufficient for information storage in the brain \citep{martin00}.  

However, as critiqued by \citet{neves08}, direct empirical evidence showing necessity and sufficiency of synaptic plasticity in learning and memory is hard to obtain. To show necessity of, for example, \gls{ltp} in memory, the ideal evidence would show the abolition of \gls{ltp}, but nothing else, results in a learning and memory deficit \citep{neves08}. Since most findings are based on molecular manipulation of key proteins in synaptic plasticity, these results are often confounded by other synaptic and cellular functions of the targeting protein \citep{neves08}. However, further evidence can be obtained from occlusion studies, where necessity can be implied if saturation of either memory formation or \gls{ltp} blocks the other. Indeed, it has been shown that watermaze training in rats results in a failure to induce certain forms of \gls{ltp} in \gls{ca1} cells \citep{habib14}. Conversely, saturating \gls{ltp} in \gls{dg} prevents spatial learning \citep{moser98}. These results suggest at a minimum that \gls{ltp} and memory formation share some necessary mechanisms \citep{takeuchi14}. The evidence however, is mixed. It is also reported that in some cases, abolishing \gls{ltp}, for example by knocking-out GluA1 in hippocampus, does not affect spatial learning \citep{zamanillo99}. This discrepancy in evidence suggests that \gls{ltp} many involve various different mechanisms, which may be specific to different induction protocols \citep{neves08}.

To show synaptic plasticity is sufficient for memory, one will need to show that activating synaptic plasticity mechanisms creates an artificial memory \citep{neves08}, and this has not been possible until recent development of optogenetics, which allows activation and deactivation of neural populations with light  \citep{zhang07, rajasethupathy16}. In an auditory fear conditioning paradigm, \citet{nabavi14} trained rats to associate a foot shock with optogentic stimulation of auditory thalamus and auditory cortex, hence creating an artificial memory. The authors further showed that optically induced \gls{ltd} abolishes the fear memory, and subsequent \gls{ltp} induction reactivates the fear memory \citep{nabavi14}. This result is the first direct evidence that \gls{ltp} and \gls{ltd} are sufficient to induce and inhibit memory, respectively. 

In conclusion, the synaptic plasticity and memory hypothesis is well supported by empirical evidence. While the \gls{ltp} and \gls{ltd} are sometimes considered as an artificially constructed experimental phenomena \citep{stevens98}, the mechanism underlying \gls{ltp} and \gls{ltd} are both necessary and sufficient to support memory formation.

\section{Hippocampal deficits in \gls{ad}}
\subsection{Initiation of \gls{ad} pathology in hippocampus}
Converging evidence suggests that the hippocampus and its surrounding brain areas in the \gls{mtl} are the first brain areas compromised in the development \gls{ad} \citep{palmer11, zhou16}. For example, post-mortem studies of \gls{ad} patients have reported initial \glspl{nft} and neuronal loss in the \gls{ec}, which then spread to other regions of \gls{mtl} in a predictable network-dependent manner \citep{braak91,  hoesen93, zhan09}. In fact, longitudinal studies have shown significant neuronal loss and atrophy of \gls{ec} in patients with \gls{mci} and early \gls{ad}, and in addition, the degree of neurodegeneration in \gls{ec} and hippocampus correlates with cognitive performance \citep{kordower01, jack02, pennanen04}. These results suggest that hippocampal neuronal degeneration is a sensitive biomarker for \gls{ad} progression and cognitive outcome \citep{jack02, zhou16}. 

While the \gls{ad} pathology is consistently found to originate in the \gls{ec} and hippocampus, it is still unclear why this circuit is particularly vulnerable to \gls{ad}. Several hypotheses exists. It has been hypothesized that \gls{ec} and hippocampal neurons are formed early in development, and the vulnerability is simply due to the neurons' old age \citep{rakic81}. The complicated morphology, larger surface area, and higher energy consumption in the \gls{ec} neurons may contribute to the vulnerability \citep{hevner92, buckmaster04}. Moreover, the specific molecular environment of \gls{ec} neurons renders them with reduced neurotropic support \citep{narisawa-saito96, peterson96} and increased inflammation \citep{janelsins05, okun10}, both of which have been found to aggravate \abeta{} toxicity \citep{tang08, stranahan10}.

The neuronal degeneration of \gls{ec} then propagates to the \gls{dg} and \gls{ca3} through the perforant pathway. Firstly, presynaptic markers are reduced at the perforant-\gls{dg} synapse, correlating with a spatial learning deficit in animal models of \gls{ad} \citep{smith00}. While initially the post-synaptic cells are unaffected, over time they show reduced amplitude of \glspl{epsc}, and an increase in \gls{ltp} threshold \citep{calhoun08, barnes80, barnes00}. \Gls{dg} neurons then, in turn, show an increase of \abeta{} deposition and synaptic degeneration \citep{reilly03, dong07}, leading to an imbalance of excitation and inhibition through the hippocampal network and reduction of both short-term and long-term plasticity \citep{palop07}. Changes in cytoskeletal architecture and atrophy of the hippocampus also potentially contribute by the prion-like axonal trafficking of \atau{} protein \citep{clavaguera09}. However, as \atau{} cannot cross the synapse, the mechanism of how it affects post-synaptic cells is still unclear \citep{stranahan10}.  

\subsection{Synaptic deficits \label{ad.synaptic}}

There is converging evidence that the main pathological marker of \gls{ad}, \abeta{}, inhibits \gls{ltp} and promotes \gls{ltd}. In mouse models of \gls{ad} where \abeta{} production is transgenically enhanced, hippocampal \gls{ltp} is severely impaired in advance of plaque formation and neural degeneration, and these \gls{ltp} deficits are correlated with learning impairment in these mice \citep{hsia99, chapman99, roberson11}. Moreover, acute \abeta{} induction leads to severe \gls{ltp} impairment, both \textit{in vitro} \citep{lambert98, shankar08} and \textit{in vivo} \citep{walsh02, hu08}. In human \gls{ad} patients, similar impairment of \gls{ltp}-like cortical plasticity is also observed \citep{inghilleri06, koch12}. On the other hand, the \gls{ltp} impairment created by \abeta{} is accompanied by an enhancement of \gls{ltd}, such that sub-threshold stimulation which does not have an effect in wild-type mice can create \gls{ltd} in mouse models of \gls{ad} \citep{hsia99, fitzjohn01, jacobsen06}. 

The molecular pathways by which \abeta{} modulates \gls{ltp} and \gls{ltd} are still under investigation. Current evidence suggests that \abeta{} leads to more excitable synapses, and that attenuated \gls{ltp} and enhanced \gls{ltd} are a result of neuronal homeostasis processes trying to maintain overall cell activity \citep{guntupalli16, jang16}. For example, \abeta{} blocks synaptic glutamate reuptake by astrocytes, and therefore lead to chronic increase of synaptic glutamate levels \citep{matos08, li09}. The increased amount of glutamate may spill over to the extra-synaptic membrane, and activate GluN2B-containing \glspl{nmdar}, which are abundantly present at extra-synaptic sites in mature hippocampal neurons \citep{citri08, li11, shipton14}. Activation of extra-synaptic \glspl{nmdar} then leads to an excess influx of \ce{Ca^2+}, which activates \gls{pp1}, calcineurin and \gls{mapk} pathway \citep{hsieh06, shankar07, zhao10}. As a result, synaptic \glspl{ampar} are dephosphorylated and endocytosed, and consequently synaptic strength is reduced \citep{hsieh06, liu10, minanomolina11}. 

\abeta{} also affects \gls{mglur}-dependent \gls{ltd}. It has been found that \abeta{} clusters and stabilizes membrane mGluR5s, potentially through the membrane protein \gls{prpc}, which upon \abeta{} activation, binds to mGluR5 \citep{renner10, um13}. The activated \gls{prpc}-mGluR5 complex then interacts with homer1b/c, Fyn and \gls{camkii}, all of which are implicated in hippocampal \gls{ltp} and \gls{ltd} \citep{raka15, haas16}. Moreover, inhibiting mGluR5s pharmacologically or genetically in mouse \gls{ad} models rescues deficits in \gls{ltp}, dendritic spine density and spatial learning \citep{rammes11, hu14, um13, hamilton14}. The \gls{mglur}- and \gls{nmdar}-mediated pathways of \abeta{} are not independent. Activation of \gls{mglur} can lead to \gls{nmdar} activation through \gls{camkii} and \gls{pkc} pathways \citep{chen11, jin15}, and conversely \gls{nmdar} activation can enhance \gls{mglur} activity through calcineurin \citep{alagarsamy99, alagarsamy05}. 

In conclusion, \abeta{} has been consistently shown to shift hippocampal synaptic plasticity from \gls{ltp} to \gls{ltd}. Although further investigation is still required, multiple molecular pathways are implicated and converges on the dysregulation of synaptic \gls{ampar} density. 

\subsection{Hyperexcitability}

While individual hippocampal synapses shift from \gls{ltp} to \gls{ltd} in the presence of \abeta{}, synaptic changes are not predictive to changes in neural networks. With degeneration of excitatory synapses, counter-intuitively a hyperactive hippocampus has been consistently found across animal models and human patients \citep{palop16}. Mouse models of \gls{ad} show larger discharges on \gls{eeg}, increased hippocampal \gls{ieg} expression \textit{in vivo}, paradoxically accompanied by decreased surface \glspl{ampar} \citep{palop07, harris10, born14}. Pharmacologically-induced hyperactivity \textit{in vivo} in mouse hippocampus creates molecular, synaptic and anatomical deficits similar to that of mouse models of \gls{ad} \citep{palop07}. This has also been confirmed by more recent \textit{in vivo} calcium imaging studies, where hippocampal neurons in a mouse model of \gls{ad} were found to have increased spontaneous activity, especially those neurons located close to plaques \citep{busche12}. 

Similarly in humans, hyperactivation of hippocampus has been reported in asymptomatic individuals with high risk of developing \gls{ad}, suggesting hippocampal hyperactivity is an alternation of neural networks early in the development of \gls{ad} pathology \citep{sperling09, reiman12}. While \abeta{} disrupts glutamate reuptake and may directly create hyperexcitability, it also promotes neural degradation \citep{spires04, koffie09}. Neurons treated with \abeta{} have shorter dendrites and lower dendritic spine density, and these morphological changes also promote hyperexcitability \citep{siskova14}.

Given the hyperexcitability of neural networks in \gls{ad}, it is therefore not surprising that epilepsy often accompanies \gls{ad}. Spontaneous epileptic activity has been reported in many mouse models of \gls{ad}, and this is usually accompanied by behavioural seizures and a lower threshold for pharmacologically induced seizures \citep{palop07, um12, ittner10}. Incidences of epileptic seizures are also common in \gls{ad} patients. While prevalence varies from study to study due to different detection methods and difficulty of seizure assessment in \gls{ad} patients, a recent review suggests that on average epileptic seizures affect \SI{17}{\percent} of late-onset \gls{ad} patients, which is 7--8 times higher than the general incidental rate of epilepsy \citep{amatniek06, horvath16}. Moreover, populations with a genetic disposition toward \gls{ad} have an even higher rate of epilepsy, and the epileptic episodes often precede any sign of cognitive decline \citep{moehlmann02,cabrejo06,mcnaughton12}. 

Moreover, \gls{ad} patients with concomitant epilepsy have faster progression of cognitive decline \citep{vossel13, bakker15}. Anti-epileptic treatment in \gls{mci} restores cognitive ability in the short-term \citep{bakker15}. In animal models of \gls{ad}, anti-epileptic treatment restores synaptic function, long-range network coherence, as well as cognitive and behavioural deficits \citep{sanchez12, busche15}. These findings suggest that network hyperexcitability is not only a symptom of \gls{ad}, but is also important in the initial development of \gls{ad} progression, and can be a potential therapeutic target for \gls{ad}.

\subsection{Circuit function deficits}
\begin{comment}
\subsubsection{synchrony}
\citep{goutagny13}
\end{comment}
\subsubsection{Place encoding}
Given the discovery of place encoding in hippocampus and the finding that deficits in spatial navigation are one of the first symptoms of \gls{ad}, it is surprising that only a few studies have investigated how hippocampal place encoding is compromised in \gls{ad}. This idea was first studied by \citet{cacucci08}, where the authors recorded from both wild-type and an \gls{app} transgenic mouse model of \gls{ad}, Tg2576. They have found that while at a young age the transgenic mice have similar place encoding, place fields in aged Tg2576 mice are less defined, as measured by a larger field size and less spatial information encoded by cells. The degradation of place fields also correlated with deficits in spatial memory and amyloid plaque burden in the hippocampus \citep{cacucci08}.

More recent studies suggest that the place field deficit in \gls{ad} mouse models is a learning deficit instead of an encoding deficit. \citet{cheng13} recorded \gls{ca1} place cells in aged mice with \atau{} pathology on a linear track. The authors found while the transgenic mice have degraded place fields, potentially due to the hyperexcitability of the \gls{ca1} neurons, on a familiar track these mice are still able to maintain the correct sequence of firing, from which the mice's position can be decoded. However, the transgenic mice failed to form new place cell trajectories when placed on a novel track \citep{cheng13}. In a later study, \citet{zhao14} investigated \gls{ca1} place encoding in an \gls{app} transgenic mouse model, and found the transgenic mice were able to initiate place fields in a novel environment, however over time, they were unable to refine these place fields. Furthermore this deficit correlated with a deficit in spatial memory performance; the transgenic mice learned slower and less effectively than the wild-type mice.

While neuroimaging techniques in humans are unable to resolve single cells activity and identify place cells, the hexagonal symmetry of grid cells in \gls{ec} creates a similarly symmetric \gls{bold} signal, which can be detected under \gls{fmri} \citep{doeller10}. \citet{kunz15} recently investigated this grid-cell-like representation in individuals with genetic depositions for \gls{ad} before cognitive decline. Similar to the place field results found in mouse models, \citet{kunz15} found while the spatial memory performance is similar between the \gls{ad}-risk group and the control group, grid-cell-like representations in the \gls{ad}-risk group were less stable over time, and the degradation of grid-cell-like representation was also correlated with hippocampal hyperactivation. 

In conclusion, evidence from both mouse models and human studies suggest that the spatial encoding in hippocampus and \gls{ec} is affected in \gls{ad}, which potentially leads to cognitive deficits in spatial memory. However, future study is needed to understand the circuit mechanisms of this hippocampal spatial encoding deficit in \gls{ad}.

\subsubsection{Pattern separation and completion}
The two vital circuit mechanisms of hippocampus, as first predicted by the Marr model, are pattern separation and pattern completion (Discussed in \ref{hpc-marr}). It is thought that the \gls{dg} functions as a pattern separator, which involves pulling similar patterns apart to reduce memory interference, and that the role of \gls{ca1} is pattern completion, where noisy or incomplete patterns are mapped to the original, complete pattern \citep{rolls13}. Given the hippocampal deficits seen in \gls{ad}, it is hypothesized that the process of pattern separation and pattern completion are compromised. However, only a few studies have investigated this hypothesis explicitly \citep{maruszak14}. 

\citet{palmer11} used c-Fos expression to study hippocampal cell activity in the \gls{app} transgenic mice, Tg2576. Mice were exposed to either a familiar environment or a novel environment, and hippocampal cell activity was examined by analyzing the expression of the \gls{ieg}, c-Fos. The authors found that while wild-type mice show increased \gls{dg} activation during novel environment exposure, potentially supporting a pattern separation process, transgenic mice show similar \gls{dg} activation, but an increased \gls{ca3} activation. The authors concluded that the Tg2576 mice have a deficit in pattern separation, and that this leads to an overactive pattern completion, resulting in memory deficits \citep{palmer11}.

However, this theory was not supported by a later human study. \citet{ally13} studied behavioural pattern separation and pattern completion processes in human patients in \gls{mci} and mild \gls{ad}. In their study, \gls{mci} patients, \gls{ad} patients and age-matched controls were presented with a lag-based memory task, where the individuals were presented with a continuous list of pictures. The list of pictures contained repeated, similar, and novel items, and the participants were asked to label the items accordingly. The authors found that for similar items, the \gls{ad} patients performed significantly worse in telling them apart during testing, and the performance of the \gls{mci} group was negatively correlated with the lag, with the performance of low-lag items similar to control group and that of high-lag comparable to \gls{ad}.  In addition, \citet{ally13} also examined the pattern completion process. Interestingly, while the \gls{mci} patients showed intact behavioural pattern completion, the \gls{ad} patients showed a significant deficit \citep{ally13}. These results suggest that in \gls{ad}, both behavioural pattern completion and behavioral pattern separation are impaired, while in \gls{mci}, only behavioural pattern separation is impaired in a temporal-graded manner \citep{ally13}.

However, it is worth bearing in mind that neither the \citet{palmer11} nor \citet{ally13} studies directly address the pattern separation and pattern completion process originally defined computationally, and conflicting results are not uncommon once a definition is loosed.  The findings of these studies may be influenced by processes other than pattern separation or pattern completion \citep{santoro13}. While computational studies have long proposed that pattern separation and pattern completion processes can be compromised in \gls{ad} \citep{horn93, hasselmo94, hasselmo97}, validating these hypothesis requires simultaneous recording of large ensemble of hippocampal neurons \textit{in vivo} during behaviour, which has not been possible until recently (discussed in Section \ref{tech}).  

\section{Methods and tools for investigating neural population activity \label{tech}}
\subsection{\textit{In vivo} electrophysiological recording}
\textit{In vivo} electrophysiological recording is one of the oldest techniques for recording neural activity from a living brain. In electrophysiological recording, one or multiple fine-tipped microeletrodes are inserted into the brain region of interest. If the tips of the electrodes are placed close to or within the cell membrane of neurons, changes of the membrane potential when these neuron generate action potentials can be detected by the microeletrode. The signal is then amplified, sampled and recorded. 

Since the electrical activity of the cell is measured directly, cell activity can be measured in a time resolution of less than \SI{1}{\ms}, which allows the shape of the action potential to be resolved \citep{lutcke13}. Moreover, since the insertion of microelectrodes induces minimal tissue damage, it is possible to reach deep in the brain, and therefore this method is preferred for recording in larger model animals such as cat and monkeys \citep{lutcke13}. A chronic implantation is stable for months, allowing repeated recordings from the same animal. With an electrode array, it is possible to record tens of neurons simultaneously \citep{lutcke13}. Another unique advantage of single unit recording is the possibility to deliver electrical neural stimulation. A small electric current can be precisely delivered through the same recording electrode, which is useful for brain-computer interfaces \citep{hatsopoulos09}. 

With the flexibility of single unit recording, however, comes less stability. Firstly, \textit{in vivo} electrophysiological recording requires extraordinary mechanical stability, experimenters often need to wait for months before getting a stable recording, possibly because of the formation of scar tissue which de-stabilizes the electrodes \citep{jackson07}. Secondly, while neuron identity across recording sessions can be inferred by the unique signature in the action potential waveform, this method cannot reliably distinguish whether the sample neuron is recorded over time \citep{rousche98, schmitzer-torbert04, tolias07}. In fact, recordings tend to become unstable quickly over time, such that half of the stable units can drop out in the first three days of recording \citep{fraser12}. Thirdly, while it is possible to cluster units into regular spiking (often representing excitatory cells) and fast spiking (often representing inhibitory cells), \textit{in vivo} electrophysiological recording is otherwise unable to differentiate subpopulation of neurons \citep{connors90}. And lastly, recordings from single units are heavily biased towards neurons which fire across the whole experimental session, neurons which stay silent for long periods of time will be missed in the recording. Therefore the recorded population is often not representative of the true underlying neural population \citep{lutcke13}. 

Many of the limitations of \textit{in vivo} electrophysiological recording can be overcome by combination with the recent development of optogenetics, where light-sensitive ion channels can be expressed in neurons, and allow neural activity to be controlled by light \citep{yizhar11}. Combined with molecular constructs, opsins can be expressed in specific neural subpopulations. A pattern of light stimulation can be used to identify opsin-expressing neurons during \textit{in vivo} electrophysiological recording \citep{zhao11}. 

In conclusion, \textit{in vivo} recording is useful to identify circuit mechanisms which rely on the activity of individual neurons. However, the small number of neurons able to be simultaneously recorded by this technique limits its usefulness in identifying circuit mechanisms where synergy amongst multiple neurons is crucial. 

\subsection{Immunohistochemistry and \textit{in situ} hybridization against \glspl{ieg}}
Neuronal activity leads to \textit{de novo} protein synthesis. \Glspl{ieg} are a group of genes that are immediately expressed after neural activity. For example, it has been shown that \glspl{ieg} such as \gls{arc}, c-fos, zif268 are selectively up-regulated after maximum electroconvulsive shock, \gls{ltp} induction, and behavioural experiences \citep{guzowski99, vann00, hall01}. After exposure to a novel environment, the \gls{arc}-expressing population of neurons is similar in size to electrophysiological estimates. Repeated exposure to the same environment reactivates similar populations of neurons \citep{guzowski06, niibori12}. Moreover, given many of the \glspl{ieg} are crucially involved in the synaptic plasticity process, \glspl{ieg} are widely used as a molecular marker of neural activity \citep{minatohara15}. 

The crucial role of \gls{ieg}-expressing neurons in learning and memory has been directly confirmed recently using optogenetic techniques. Inhibiting the hippocampal neural population that expresses c-fos or \gls{arc} after contextual fear conditioning prevents the expression of the fear memory during memory recall \citep{denny14, tanaka14}. Moreover, reactivating the c-fos ensemble formed during contextual fear conditioning is able to reactivate memory response \citep{liu12, cowansage14, ohkawa15}, or even able to modify the original memory in a new environment \citep{ramirez13, redondo14}. These two lines of evidence confirm that the \gls{ieg}-expressing neurons are both necessary and sufficient for learning and memory. 

The expression of \glspl{ieg} is detected in \textit{post mortem} tissue by \gls{ihc} against proteins or \gls{fish} against \gls{mrna}, and therefore the size of the neural population to be analyzed is theoretically unlimited. And in fact, \glspl{ieg} staining allows neural activity across the whole-brain to be studied with cellular precision \citep{wheeler13}. Recent developments in tissue clearing and microscopy allow fast whole-brain imaging at the scale of a mouse brain \citep{chung13, tomer14}, and combining this technique with the detection of \glspl{ieg}, whole-brain analysis of active neural ensembles is much less labour intensive, and more accessible to researchers.   

Recent developments have also allowed genetic access to active neurons by taking advantage of the molecular mechanisms of \gls{ieg} expression. This idea was first developed and demonstrated by \citet{reijmers07}, where the authors expressed the \gls{tta} under the Fos promoter. In their procedure, experimental mice are placed under a doxycyline diet, which suppresses the activation of \gls{tta}. During the experimental intervention, doxycyline is removed from the diet, allowing activation of \gls{tta} only in c-fos expressing neurons. The \gls{tta} then binds to a tetracyline operator for expression of the transgene of interest. A more recent, similar system, \gls{trap}, has been developed by \citet{guenthner13}. In \gls{trap} transgenic mice, the promoters of \gls{arc} and c-fos drive a tamoxifen-dependent recombinase, CreER(T2). When tamoxifen is given, CreER(T2) is expressed only in active cells, allowing Cre-dependent recombination of any gene of interest \citep{guenthner13}. Both systems give genetic access to active neurons within a certain period of time, and can be combined with molecular tools for labelling, tracing and manipulating neurons. These tools provide powerful ways to understand circuitry mechanisms.

The major disadvantage of \gls{ieg}-based studies is the time resolution. Fist, \gls{ieg} expression needs to be examined \textit{post mortem}, and can only represent neural activity at a single time point. However, several techniques exist allowing comparison of \gls{ieg}-expression at two time points. The \gls{catfish} method, developed by \citet{guzowski99}, takes advantage of the temporally different cellular localization of \gls{arc} \gls{mrna}. \Gls{arc} \gls{mrna} can be detected in the nucleus \SI{5}{\min} after expression, and is transported into the cytoplasm after \SI{30}{\min}. If the two experimental procedures of interest are performed at \SI{5}{\min} and \SI{30}{\min} before the animal is sacrificed for \gls{arc} detection, the cell population with cytoplasmic \gls{arc} signal will represent the active ensemble during the first procedure, and the population with nuclear \gls{arc} signal will represent the population active during the second \citep{guzowski99}. \Glspl{ieg} with different expression rates can be used to identify two active cell ensembles in a similar manner. For example, homer1a's signal cannot be detected in the cell nucleus until \SI{30}{\min} after cell activation, and it can be co-stained with \gls{arc} nuclear signal to identify active ensembles from \SI{5}{\min} and \SI{30}{\min} prior \citep{vazdarjanova04}. Moreover, longer intervals between two procedures can be achieved by using the \gls{trap} or Fos-\gls{tta} transgenic mice described above, in combination with a fluorescent marker which marks the first period of activation. The second period can be detected by \gls{ihc} against \gls{ieg} such as c-fos \citep{reijmers07, guenthner13}. However, in this procedure, temporal precision can be as long as \SI{12}{\hour} compared to minutes for direct \gls{fish} against \glspl{ieg}.

In conclusion, while \gls{ieg} detection provides almost perfect spatial range and resolution and allows analysis of the activation pattern of the whole brain at a cellular level, it can only detect activation of a few timepoints imprecisely. While this prevents it being used for investigating fast dynamics of neural activity, the genetic access it provides nevertheless provides a powerful way for manipulating neuronal activity at the circuit level.


\subsection{\textit{In vivo} calcium imaging}

\textit{In vivo} imaging records neural activity using molecular sensors which convert physiological events in the neuron into detectable optical signals. While traditional imaging techniques use small-molecule dyes which respond to membrane potential, pH or \ce{Ca^2+} concentration changes in the cell, recent development of \gls{geci} \ce{Ca^2+} sensors have surpassed the performance of small-molecule \ce{Ca^2+} dyes in both sensitivity and responsiveness \citep{lutcke13}. \Glspl{geci} can be expressed in cell-type specific manner, non-invasively, and most importantly, allow long-term expression for weeks to months. Given this, \glspl{geci} have largely replaced traditional dyes. 

In \textit{in vivo} calcium imaging, fluorescent calcium sensors such as gCaMP are introduced to the neural population of interest, either by viral delivery or by a transgene. To allow optical access, a cranial window is created above the brain region of interest. During imaging, the animal's head is firmly fixed on the microscope stage, and fluorescence from the sensors are directly imaged through the cranial window \citep{lutcke13, yang17}. 

\textit{In vivo} calcium imaging has overcome many of the shortcomings of electrical recording. First, cell identity can be defined. The unique distribution of neurons and blood vessels in the field of view allows the same population of neurons to be imaged (if chronically expressing a \gls{geci}) for weeks to months. Second, \textit{in vivo} calcium imaging is able to capture a large and dense set of neurons, often able to record hundreds of neurons simultaneously. Thirdly, subpopulations of neurons are easily identifiable, either through the use of specific promoters which drive the expression of \glspl{geci} in specific neural populations, or by using fluorescent markers in a separate colour channel. These advantage make \textit{in vivo} calcium imaging ideal for studying fast dynamics of neural ensembles \citep{lutcke13}. 

Moreover, the spatial resolution of calcium imaging is only limited by that of the microscope, which can usually reach, or with recent development of super-resolution microscopy, surpass the diffraction limit \citep{dudok15}. This allows researchers to track subcellular features over time. Moreover, combined with optogenetics or caged glutamate, the excitation light from the microscope can also be used to induce activity, either at cellular or subcellular level \citep{kantevari10, noguchi11, prakash12}. These advantage of \textit{in vivo} imaging also make it a powerful tool for the study of dynamic subcellular processes. 

One of the major disadvantages of \textit{in vivo} calcium imaging is limitation to a single region of interest. Since the brain is opaque, excitation light and emission fluorescence can only penetrate the surface of the tissue. Even with the recent development of multi-photon imaging, where short bursts of intense laser at long wavelengths are used to stimulate fluorescent molecules, the depth of imaging is often limited to about \SI{1}{\mm} from the top of the brain, restricting \textit{in vivo} calcium imaging to cortical regions in small rodents \citep{horton13, yang17}. This limitation can be overcome by physically inserting a \gls{grin} lens into the region of interest, which relays an image to the surface. This allows the imaging of deep brain regions such as hippocampus, thalamus and hypothalamus without significantly affecting imaging quality \citep{attardo15}.

A second disadvantage of \textit{in vivo} imaging technique is the requirement of firm head fixation of the animal, which limits the behavioural paradigms available to the animal, and may induce stress. To allow complex behavioural paradigms with \textit{in vivo} imaging, the microscope stage can be modified to include a virtual environment setup \citep{harvey09}. The mouse is placed on a omni-direction treadmill, which measures the speed and direction of the mouse's movement. A toroidal screen occupying the entire visual field is used to display the virtual environment and update with the mouse's movement. This system allows tasks such as spatial exploration and decision making to be performed during \textit{in vivo} imaging \citep{harvey09, harvey12}. 

In conclusion, \textit{in vivo} calcium imaging is well-suited for studies of the dynamic neural circuit. However, the behavioural and spatial limitations of the microscope restricts its application to cortical activity with a few behavioural paradigms, and the result may not reflect that of a naturally behaving animal. In addition, setting up \textit{in vivo} imaging system usually requires significant effort and time of the researchers, as well as a significant financial cost to the laboratory. These restrictions limit its application in neuroscience research. 

\subsection{Fluorescent endoscopy}

The fluorescent endoscopy approach allows \textit{in vivo} imaging in freely behaving animals. In fluorescent endoscopy, a thin relay \gls{grin} lens is inserted into the brain region of interest and fixed in this position. The relay lens forms an image of the targeted cells at the surface of the brain. This image is then transferred away from the animal, while allowing it to behave freely. Two approaches are available to achieve the goal of recording at a cellular level. The advantages and disadvantages of each of approach are discussed below. 

\subsubsection{Fiber based fluorescence endoscopes}
In the fiber based fluorescence endoscopy, the \gls{grin} lens is attached to an optic fiber bundle. The flexibility of the fiber bundle allows an animal to behave freely, while traditional fluorescence microscopy techniques can be used to capture the image of cells at the end of the fiber bundle \citep{flusberg08}. 

The fiber bundle approach has several advantages. First, since it can be coupled with traditional microscopy techniques, strong stimulation light can be applied, and faint fluorescent signals can be detected. It is also easy to combine optogenetics with the fiber bundle approach. With an arbitrary-patterned illuminator, individual cells in the field of view can be selected and stimulated, at the same time as calcium imaging. This advantage allows traditional electrophysiology to be performed in freely-behaving animals, which can be very powerful for the observation and control of local neural circuits \citep{szabo14}.

The main disadvantages of the fiber bundle based approach are resolution and field-of-view. Currently single fibers in the fiber bundle can only be as small as \SI{4}{\um} in diameter, which is barely able to resolve individual cells, and therefore multiple overlapping cells may be detected as a single cell, and strong fluorescence in neuronal processes can also be falsely detected as cellular signal. The field of view is determined by the diameter of the fiber bundle. However, thicker fiber bundles are also significantly more rigid, potentially limiting the behaviour of the animal \citep{yang17}. 


\subsubsection{Miniature integrated fluorescence endoscopes}
A second approach to \textit{in vivo} calcium imaging involves integrating all components of the fluorescent microscope, including light source, filters and sensors, on the animal's head, and only transmitting the image in electrical signals. Earlier attempts focused on miniature two-photon microscopes with an external pulse laser source \citep{flusberg05, piyawattanametha09}. While these mini-microscopes were able to produce optical slices in freely behaving rats \citep{sawinski09}, their complexity in engineering and slow frame rates prevented their popularity in neuroscience laboratories \citep{hamel15, yang17}.

Later development of miniature endoscopes used camera-based single-photon imaging, with an integrated high-intensity \gls{led} as a light source \citep{ghosh11}. Single photon imaging provides a larger \gls{fov} and faster frame rates than the two-photon approach. Even though it cannot provide optical slices, recent developments in calcium indicators make indicators that are sufficiently bright to provide a satisfying \gls{snr}. The original development from \citet{ghosh11} was able to record calcium transients from hundreds of neurons in \gls{ca1} at more than 20 frames per second. A later paper demonstrated how this technique could be used to record brain activity in sensory, cognitive and motor tasks \citep{ziv13}.

The integrated single-photon endoscopes developed by \citet{ghosh11} is primarily designed to record the green fluorescence from gCaMP calcium indicators. It is therefore unable to detect a second image channel, which can be useful for distinguishing neural subpopulations. In addition, the design described by \citet{ghosh11} is still not accessible to the majority of neuroscience laboratories which have little expertise in microfabrication, optics and electronic engineering. I attempt to address these two limitations in Section \ref{chap-scope} in this thesis. 

\subsection{Conclusion}

In conclusion, the ideal tool for examine neural circuits needs to have high spatial resolution, a large field of view, good temporal resolution, and offer minimal disturbance to animal's natural behaviour. Table \ref{tech-compare} provides a summary of the advantages and disadvantages of the technologies discussed in this section. The choice of technology to investigate neuronal activity depends on a trade-off between advantages and disadvantages. It can be predicted that in the foreseeable future, studies of neural circuit mechanisms will continue to take a hybrid approach, where multiple techniques will be used for the best performance. 

\begin{table}[h]
    \centering
    \renewcommand{\arraystretch}{2.5}
    \begin{tabular}{| c | c | c | c | c |}
        \hline
        & \thead{\textit{In vivo} \\ electro- \\ physiological \\ recording} & \thead{\Gls{ieg} \\ detection} & \thead{\textit{In vivo} \\ \ce{Ca^2+} imaging} & \thead{Miniature \\ microscopes} \\ \hline
        \thead{Number of cells} & ~10 & unlimited & hundreds & hundreds \\ \hline
        \thead{Distinguishing \\ subpopulations} & hard & easy & easy & possible \\ \hline
        \thead{Temporal \\ resolution} & <\SI{1}{\ms} & minutes to hours & \SIrange{10}{100}{\ms} & \SIrange{10}{100}{\ms} \\ \hline
        \thead{Brain region} & unlimited & unlimited & surface & unlimited \\ \hline
        \thead{Behaviour \\ compatibility} & unlimited & \makecell{at most \\ 2 events} & head-fixed & unlimited \\ \hline
    \end{tabular}
    \caption{Comparison of techniques for studying neural activity at cellular level. \label{tech-compare}} 
\end{table}

\section{Hypothesis and Research Aims}

The cognitive impairment seen in \gls{ad} suggests that normal neural circuitry functions are compromised. The abnormal function of neural circuitry is potentially mediated by, and in turn aggravates, the underlying molecular and cellular pathophysiology of \gls{ad}. The failure of current interventions targeting the removal of \abeta{} suggest that the cognitive impairment in \gls{ad} is a consequence of complex pathophysiology \citep{canter16}. Understanding the neural circuitry mechanisms that directly contribute to the cognitive symptoms in \gls{ad} may yield novel treatment targets for preventing memory loss in \gls{ad}. 

In this thesis, I examine the deficits of hippocampal circuitry that underlie memory loss in \gls{ad}. I focus on a mouse model of early \gls{ad}, TgCRND8. TgCRND8 mice carry a double mutated human \gls{app} gene. \abeta{} plaques are first observed at the age of \SIrange{9}{11}{\week}, together with neurite dystrophy \citep{chishti01}. At this age, these mice display significant memory deficits in hippocampal-related memory tasks such as Morris watermaze and contextual fear conditioning \citep{hyde05, yiu11}. These cellular and cognitive symptoms parallel the early development of \gls{ad} in human patients. 

Early computational models of \gls{ad} have suggested that while synaptic loss in \gls{ad} creates memory deficits, enhancing the strength of the remaining synapses can compensate for this deficit \citep{horn93}. Considering that aberrant synaptic \gls{ampa} trafficking is a cardinal signal of synaptic degeneration in \gls{ad}, I decided to use the well-characterized interference peptide \tglu to rescue synaptic deficits, and investigate whether this rescue improves neural circuit functions and ultimately memory performance of TgCRND8 mice. \tglu is a membrane permeable peptide which selectively inhibits activity-dependent \gls{ampar} endocytosis \citep{ahmadian04}, and blocks \gls{ad}-related \gls{ca1} \gls{ltd} \citep{dong15}. Chronically infusing \tglu in a mouse model of \gls{ad} also prevents memory degradation \citep{dong15}. However, how hippocampal circuit function is affected by \tglu-mediated synaptic strengthening is still unknown. 

To investigate circuitry deficit in TgCRND8 mice, the first aim of this thesis was to design and assemble an integrated miniature microscope for imaging calcium transients in freely behaving mice. This approach allows us to record neural activity from hundreds of neurons at the cellular level while mice are perform memory tasks. Compared to the original \citet{ghosh11} publication, I intended to create a version of miniature microscope which is readily available and can be easily produced in laboratories with little engineering expertise. At the same time, I attempted to add a separate colour channel, which potentially allows the identification of neural subpopulations using a separate fluorophore. 

The second aim of this thesis was to use this tool to investigate the circuitry mechanisms of TgCRND8 mice in contextual fear conditioning, which is a memory task dependent on hippocampal function. My hypothesis was threefold. First, given that a hyperactive hippocampus has been consistently found across mouse models of \gls{ad} and human patients, I hypothesized that TgCRND8 mice would show the same hyperactive symptom. Secondly, I hypothesized the \gls{tg} mice would show a deficit in memory encoding efficiency of hippocampal neurons. Previously it has been found in a taupathy mouse model, which is related to \gls{ad}, hippocampal place cells are less efficient in encoding space \citep{cheng13, ciupek15}. Therefore, I hypothesized that hippocampal neurons in TgCRND8 mice are equally affected in encoding a fear memory. Thirdly, I hypothesized that pattern completion, which is important for memory recall, is compromised in these mice. As predicted by \citet{horn93}, I hypothesized that TgCRND8 mice would have a deficit in pattern completion. In addition, I also investigated whether \tglu{} treatment affects the above mentioned circuit function. I hypothesized that \tglu{} treatment would be able to rescue the circuit deficits in TgCRND8 mice, and as a consequence, also able to rescue behavioural deficit. 

Last, I investigated whether the memory deficits seen in TgCRND8 mice were due to memory encoding or memory recall. While it was originally believed that \gls{ad} was characterized by either an inability to form new memories or a very fast forgetting process, a recent study by \citet{roy16} showed that in a mouse model of early stage \gls{ad}, optogenetically reactivating the neural ensemble encoding a memory is sufficient to allow animals to express the memory, suggesting that the memory deficit in their mouse model of \gls{ad} may be mediated by a deficit in memory recall, but not memory encoding. I hypothesized that our TgCRND8 mice, which are also a model of early stage of \gls{ad}, may be similarly affected by deficits in memory recall, but not memory encoding. If this were the case, I expected that after memory training, \tglu treatment along with a reminder would be able to bring back the original memory.

Through this study, I hope to highlight deficits in \gls{ad} circuit function, which is an under-studied area when compared with the cellular pathophysiology and behavioural symptoms of \gls{ad}. I hope that this study, as well as future studies in this area, are able to create a necessary link at the circuit level between cellular pathology and the cognitive symptoms of \gls{ad}, and make the tools for studying circuitry mechanisms more accessible to the neuroscience community. A better understanding of the circuitry deficits in \gls{ad} will hopefully inspire novel treatment targets for \gls{ad} in the future. 
