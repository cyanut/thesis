\chapter{Literature Review}

\section{Clinical presentation of \gls{ad}}
\subsection{Prevalence}
It is estimated that 35.6 million people has dementia worldwide, costing more than US\$ 604 billion each year, and create heavy burden to their family members and caregivers \citep{who13}. \gls{ad} is the most common form of dementia, contributes to \SI{60}{\percent} -- \SI{80}{\percent} of the cases \citep{ad16}. It is estimated to affect \SI{11}{\percent} of population at age 65 and older in the North America, and the risk triples to \SI{33}{\percent} for individuals beyond 85 \citep{hebert13}. The real number of patients affected by \gls{ad} is much larger, as instances of \gls{ad} are often under-diagnosed and under-reported \citep{barrett06, zaleta12}. While the rate of \gls{ad} in population beyond 65 is stable over years, however as the population ages, the burden of \gls{ad} is expect to continue to rise in the future. It is estimated the instances of \gls{ad} will double every 20 years \citep{who13, hebert13}. This will translates to more than 8 million instances in the United States in 2030, and as many as 16 million in 2050. \gls{who} estimated in 2040, cases of dementia worldwide will reach 81.1 million, most of which are contributed by \gls{ad} \citep{who13}. 

\subsection{Progression}
The \gls{ad} is named after Alois Alzheimer, who described the disease in 1906 \citep{goedert06}. One of his patients, Auguste D., was admitted with progressive memory loss, hallucinations and focal symptoms. After her death, Dr. Alzheimer examined her post-mortem brain tissue with silver staining, and made the crucial observation of plaques and neurofibrilary tangles, which become the definitive biomarkers of \gls{ad} \citep{goedert06, dubois16}. \gls{ad} is characterized by progressive decline of memory, learning ability and other cognitive functions. Post-mortem examination of patient's brain is characterized with amyloid plagues, neurofibrilary tangles and significant loss of neural tissue.

\subsubsection{Preclinical}
It is now known the biology of \gls{ad} starts well before clinical symptoms appear \citep{dubois16}.  Neural tissue loss can be detected by \gls{mri} before the onset of \gls{ad} \citep{jack92, scheltens92, chetelat03}, and changes in brain structure from \gls{mri} imaging can predict whether aging participants with \gls{mci} will develop \gls{ad} \citep{jack99}. Moreover, \gls{pet} imaging agents which binds to amyloid plaques have been developed in 2003 \citep{mathis03}, and more recently, \atau tracers have also been discovered \citep{maruyama13, okamura13}. These radiopharmaceuticals allows \textit{in vivo} imaging of amyloid plaques and \atau tangles, and the results consensually show that presence of plaque deposition and \atau tangles predicts \gls{ad} risk in \gls{mci} or asymptomatic participants \citep{klunk04, chien14, sepulcre16}.


\gls{mri} studies in aging participants have shown that 

\citep{vos13} preclinical ad outcome
\citep{sepulcre16} in vivo imaging tau amyloid preclinical

\gls{ad} is characterized by progressive decline of memory, learning ability and other cognitive functions. \gls{ad} starts before clinical symptoms appear \citep{hubbard90}, \citep{dubois16}. \citep{pandya16} \citep{reiman16}

In the early stage of \gls{ad}



\subsection{intervention}
the drugs: donepezil, rivastigmine, galantamine, mementine
other: clearing amyloid plagues
immunotherapy
preclinical: \citep{malek-ahmadi16} \citep{reiman16}

\section{neuropathology in \gls{ad}}
\subsection{A\textbeta{} hypothesis}
\subsection{Synaptic effect of A\textbeta{} accumulation}

\section{GluA2, synapses}

\section{synaptic deficits in \gls{ad}}

\section{Circuitry mechanism for information processing in hippocampus}
\subsection{anatomy and information flow}
\subsection{pattern separation and pattern completion}
    Kinerim J, neuron 2014

\section{Hippocampal deficits in \gls{ad}}
\subsection{synaptic function}
\subsection{Network}


\section{Methods and tools for investigating neural population activity}
\subsection{\textit{In vivo} electrophysiology recording}
\subsection{Immunohistochemistry and \textit{In situ} hybridization against \glspl{ieg}}
\subsection{Two-photon calcium imaging}
\subsection{Miniature endoscopes}
\subsubsection{Fiber bundle based fluorescence endoscopes}
\subsubsection{\gls{grin} lens based fluorescence endoscopes}
\subsection{Conclusion}


\section{Hypothesis and Research Aims}
2-5 pages
