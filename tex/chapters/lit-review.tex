\chapter{Literature Review}

\section{Clinical presentation of \gls{ad}}
\subsection{Prevalence}
It is estimated that 35.6 million people has dementia worldwide, costing more than US\$ 604 billion each year, and create heavy burden to their family members and caregivers \citep{who13}. \gls{ad} is the most common form of dementia, contributes to \SI{60}{\percent} -- \SI{80}{\percent} of the cases \citep{ad16}. It is estimated to affect \SI{11}{\percent} of population at age 65 and older in the North America, and the risk triples to \SI{33}{\percent} for individuals beyond 85 \citep{hebert13}. The real number of patients affected by \gls{ad} is much larger, as instances of \gls{ad} are often under-diagnosed and under-reported \citep{barrett06, zaleta12}. While the rate of \gls{ad} in population beyond 65 is stable over years, however as the population ages, the burden of \gls{ad} is expect to continue to rise in the future. It is estimated the instances of \gls{ad} will double every 20 years \citep{who13, hebert13}. This will translates to more than 8 million instances in the United States in 2030, and as many as 16 million in 2050. \gls{who} estimated in 2040, cases of dementia worldwide will reach 81.1 million, most of which are contributed by \gls{ad} \citep{who13}. 

\subsection{Progression}
\subsubsection{History}
The \gls{ad} is named after Alois Alzheimer, who described the disease in 1906 \citep{goedert06}. One of his patients, Auguste D., was admitted with progressive memory loss, hallucinations and focal symptoms. After her death, Dr. Alzheimer examined her post-mortem brain tissue with silver staining, and made the crucial observation of plaques and \glspl{nft}, which become the definitive biomarkers of \gls{ad} \citep{goedert06, dubois16}. \gls{ad} is characterized by progressive decline of memory, learning ability and other cognitive functions. Post-mortem examination of patient's brain is characterized with amyloid plagues, \glspl{nft} and significant loss of neural tissue.

\subsubsection{Preclinical stage}

It is now known the biology of \gls{ad} starts well before clinical symptoms appear \citep{dubois16}.  Neural tissue loss can be detected by \gls{mri} before the onset of \gls{ad} \citep{jack92, scheltens92, chetelat03}, and changes in brain structure from \gls{mri} imaging can predict whether aging participants with \gls{mci} will develop \gls{ad} \citep{jack99}. Moreover, \gls{pet} imaging agents which binds to amyloid plaques have been developed in 2003 \citep{mathis03}, and more recently, \atau tracers have also been discovered \citep{maruyama13, okamura13}. These radiopharmaceuticals allows \textit{in vivo} imaging of amyloid plaques and \atau tangles, and the results consensually show that presence of plaque deposition and \atau tangles predicts \gls{ad} risk in \gls{mci} or asymptomatic participants \citep{klunk04, chien14, sepulcre16}.

Another line of evidence for preclinical \gls{ad} comes from biomarkers detectable from \gls{csf}. As \gls{csf} is circulating extra-cellular space in the brain, its molecular composition reflects that of the \gls{cns}. Core biomarkers such as \abeta\tsb{42}, \atau and hyperphosphorylated \atau are directly related to \gls{ad}, and their levels in \gls{csf} has been consistently shown to differentiate \gls{ad} from other aging-related cognitive disorders \citep{blennow10}. Moreover, longitudinal studies of \gls{csf} biomarker levels in families with autosomal-dominant \gls{ad}, where the participants have a predictable age of \gls{ad} onset, have shown detectable changes of \abeta\tsb{42} and \atau concentrations in \gls{csf} 15--20 years before symptom onset \citep{bateman12, fagan14}. Studies in asymptomatic and \gls{mci} patients have shown a similar timeline, where the biomarker changes are detectable 5--10 years before \gls{ad} symptoms onset \citep{buchhave12, vos13}. 
\todo{MCI?} 

In conclusion, research in preclinical \gls{ad} suggest that the pathology of \gls{ad} exists as a continuum. In response to recent research result, biomarker evidence has been proposed to be included to increase accuracy of diagnosis, together with the recognition of preclinical \gls{ad} stage \citep{ad16}. Recognizing \gls{ad} before the onset of clinical symptoms also provides an optimal window for early intervention and optimization of treatment. 

\subsubsection{Clinical stage}
As a result of the broad clinical \gls{ad} pathology spectrum, the diagnosis procedure for living humans is complex. The current \gls{ad} diagnosis recommendation from the \gls{niaaa} requires the patient to meet the diagnosis of dementia, where the patient need to show cognitive or behaviour decline that:
\begin{enumerate*}[label={\alph*)}, font={\bfseries}]
    \item disables usual activity or work, 
    \item reported both from personal history and an objective cognitive assessment, and
    \item cannot explained by other major mental disorder.
\end{enumerate*} \citep{mckhann11}.
Diagnosis of \gls{ad} is further split into 3 categories, probable \gls{ad}dementia, possible \gls{ad} dementia and probable \gls{ad} dementia with evidence of the \gls{ad} pathophysiological process. To meet the probable \gls{ad} dementia, the cognitive decline from the patients needs to be gradual over months and years, and the initial symptoms needs to be a deterioration of memory, language, visuospatial or execution function. In possible \gls{ad} dementia, the patients may have a sudden onset or mixed aetiological origin of the symptoms compared to probable dementia category. The last category incorporates biomarker evidence in the diagnosis of \gls{ad}, which can increase the confidence that the symptoms from probable \gls{ad} dememtia is a result of \gls{ad} pathophysiological process \citep{mckhann11}. However a definitive diagnosis of \gls{ad} can only be obtained post-mortem, where the brain tissue is immunohistochemically stained against \abeta, \atau for \gls{ad} pathophysiology, and neuritic plaques for neuronal damage. A diagnosis of \gls{ad} requires significant presence of amyloid plaques, neurofibrillary tangles, and neuritic plaques in various brain regions related to memory, language and other cognitive functions \citep{hyman12}.

Clinical assessment of \gls{ad} progression can be accomplished using \gls{gds} scales \citep{reisberg82} and \gls{fast} \citep{sclan92}. \gls{ad} patients generally begins with decline of memory, language, and spatial navigation performace. The symptoms then progress to difficulties in routine tasks, change in emotion and personality, and disturbance in diurnal rhythms. In the end, the patient shows completely loss of verbal ability, problems in toileting and eating, as well as psychomotor problems, and require full time assistance to survive \citep{reisberg82, sclan92}.

While the stage of \gls{ad} is consistent, the rate of progression is highly variable from patient to patient \citep{komarova11, tschanz11}. This suggests that \gls{ad} patients are a heterogeneous population, and the progression of \gls{ad} may under influence of various latent factors. This is systemetically investigated in the Cache County Study on Memory in Aging, where a population of \num{5000} eldly residents in Cache County, Utah, USA are followed for up to \num{12} years, and most of the participants are followed until their deaths. The study has found several factors that are predictive of rate of cognitive decline in \gls{ad}. Cardiovascular disease history predicts a faster rate of decline \citep{mielke07}, and better general health predicts otherwise \citep{leoutsakos12}. Moreover, aspects of the caregiving environment is also influential for the decline rate of congitive functions in \gls{ad} patients. For example, \gls{ad} patients will benefit from more engagement in cognitive stimulating activities \citep{treiber11} as well as a closer relationship to the caregiver \citep{norton09}.

As \gls{ad} is a progressive disorder with no effective treatment, the prognosis for an \gls{ad} patient is not bright. The average survival duration after diagnosis is 4--8 years \citep{larson04, helzner08}. Moreover, patients will spend \SI{40}{\percent} of the time after diagnosis with the most severe stage of \gls{ad} \citep{arrighi10}. Given the prevalence of \gls{ad} and amount of burden it creates, delaying the progression of \gls{ad} by only months can reduce more than \$\num{2000} per year even if no effective treatment is available \citep{zhu06}. This will translate to more than \$15 billion total reduction of cost over the a decade in Canada \citep{adc10}.

\subsection{intervention}
\subsubsection{drug treatment \label{treatment}}
Currently only four drugs are approved for \gls{ad}: three \gls{ache} inhibitors donepezil, rivastigmine and galantamine, and one \gls{nmdar} antagonist memantine \citep{nelson15}. Rivastigmine and galantamine are approved for treatment of mild-moderate \gls{ad}, and donepezil is available for \gls{ad} patients at all stages \citep{bassil09, smith09}. Among these, memantine is only available to moderate-severe \gls{ad} patients, used either by itself or combined with donepezil \citep{nelson15}.

\Gls{ache} inhibitors aims to increase amount of \gls{ach} concentration, which has been found depleted in the \gls{ad} brain (discussed in \ref{ach-hypo}). The \gls{ache} inhibitors act by blocking the \gls{ache}, and therefore inhibits the reuptake of \gls{ach} in the synapse, allowing its concentration to increase. \Gls{ache} inhibitors are effective on the cognitive function of the \gls{ad} patients. For mild-moderate \gls{ad} patients, large-scale double-blind studies have found significant cognitive protection effects with donepezil \citep{rogers98}, rivastigmine \citep{farlow00}, and galantamine treatment \citep{wilkinson01}. A more recent meta-analysis of double-blind studies has confirmed the effectiveness of \gls{ache} inhibitors on the cognitive outcome in mild-moderate \gls{ad} patients, and found no difference in the effectiveness between \gls{ache} inhibitors at their prescription dose \citep{tan14}.

While the effect of \gls{ache} inhibitors is consistent, it however only provides a small improvement of the cognitive function in \gls{ad} patients, which is not likely to be clinical meaningful on individual patients \citep{lin13}. Moreover, not all patients respond to \gls{ache} inhibitor treatment, and the factors influnecing drug responsiveness is still unknown \citep{vanderputt06}. The \gls{ache} inhibitors unfortunately also lose their effectiveness as the patient progresses beyond moderate stage \citep{gillette-guyonnet11}, except donezepil, which has been found at high dose mildly effective in moderate-severe \gls{ad} patients \citep{sabbagh13}. 

Memantine aims to protect neurons from excitotoxicity acting on \gls{nmdar}. \Gls{nmdar} are calcium-permeable glutamate receptors which is central for learning and memory. The \Gls{nmdar} is blocked by a \ce{Mg^2+} ion in a voltage-dependent manner. In \gls{ad}, the \ce{Mg^2+} block is compromised by \abeta accumulation, leading to a pathological influx of \ce{Ca^2+} into the post-synaptic neuron, and eventually cell death \citep{danysz12}. Memantine replaces the \ce{Mg^2+} and blocks the \gls{nmdar} in a \abeta-independent manner, therefore prevents the pathological influx of \ce{Ca^2+}. Moreover, as the memantine block of \gls{nmdar} is also voltage dependent, it still allows physiological \ce{Ca^2+} influx for learning and memory \citep{danysz12}.

Memantine only have a significant effect in moderate-severe \gls{ad} patients \citep{reisberg03, tariot04, schneider11}, and often prescribed with \gls{ache} inhibitor for a positive additive effect \citep{rountree09}. Double blind studies have shown memantine is able to improve cognitive function in moderate-severe \gls{ad} patients \citep{reisberg03, tariot04}, although the improvement is not consistent \citep{porsteinsson08}. 

In conclusion, the currently approved treatments of \gls{ad} are only able to provide relief on the cognitive symptoms of \gls{ad}, and the effectiveness of the drug varies from patient to patient. None of the drugs has effect on \gls{ad}-related behaviour symptoms, or show any improvement for the general function of the \gls{ad} patients \citep{tan14}. The drugs are, at best, able to delay the progression of \gls{ad}, but unable to alter the course or improve the outcome of the disease. Together with a high drop-off rate and reports of adverse effect in many randomized clinical trials, the role of currently available drugs is still under debate \citep{bond12}. 



\subsubsection{treatment targeting the \gls{ad} pathophysiology}
other: clearing amyloid plagues
immunotherapy
\subsubsection{treatment targeting restoring circuit activity}
\subsubsection{preventive intervention}
\subsubsection{conclusion}

preclinical: \citep{malek-ahmadi16} \citep{reiman16}

\section{neuropathology in \gls{ad}}
\subsection{Introduction}
An effective intervention for \gls{ad} will be much easier to find if the aetiology of \gls{ad} is known. The most direct, and acceptable cause of \gls{ad} symptoms is synapse and neuronal loss, which is the best pathological correlate of the progression of the \gls{ad} symptoms, and has been a focus of clinical investigation for over a decade \citep{selkoe02, coleman04}. 

Unfortunately given the progressive nature of \gls{ad}, rescuing synaptic and neuronal loss is not effective without knowing its upstream pathological cause. In this section, I will discuss the cholinergic hypothesis which leads to most of the currently available pharmaceutical \gls{ad} treatments, as well as the amyloid and tau hypothesis, which hypothesize the hallmarks of \gls{ad}, amyloid plaques and \glspl{nft}, are core of the \gls{ad} aetiology. 

It should be noted that although not discussed, there are many other hypotheses for aetiology of late-onset \gls{ad} supported by empirical evidence, and has gain various attention over the years. The glutamate hypothesis, which is the basis for the memantine treatment (Section \ref{treatment}), suggests that the neuronal death is a result of excess glutamate activity, which leads to a toxic overload of \ce{Ca^2+} in the neurons which initiates neuronal death \citep{greenamyre88, parsons07}. The oxidative stress hypothesis proposes that heavy metal ions and other molecules accumulates in the \gls{ad} brain, and induces the generation of free radicals, which causes neuronal damage either directly or indirectly \citep{markesbery97, smith10}. 

A group of hypotheses proposes the symptoms of \gls{ad} is a result of abnoraml metabolism in the neurons or insufficient energy supply to the neurons. The mitochondria hypothesis suggests \gls{ad} is caused by a decreased mitochondria function which leads to insufficient energy in the cell \citep{zhu06a, swerdlow14}. The vascular hypothesis suggest that in \gls{ad}, blood vessels in the brain constricts and hardens, therefore unable to provide the brain with sufficient energy \citep{luchsinger05, mamelak17}. The inflammatory hypothesis proposes that chronic inflammation leads to \gls{ad}. An chronic elevated inflammatory response impairs metabolic balance of the neurons, which leads to pathological changes in the axons, then plaque deposition and tangle formation \citep{krstic13}. 

Here each of the hypothesis for \gls{ad} aetiology is presented as separate, linear sequences of causes and effects, however there are significant overlap between them. Moreover, given the heterogeneity of late-onset \gls{ad} patients \citep{komarova11, tschanz11}, it is unlikely that there is a single cause for \gls{ad}. In fact, the aetiology of the \gls{ad} patients is often mixed \citep{schneider07}. The readers are reminded here the grouping of \gls{ad} aetiology into multiple hypothesis is artificial, and in reality a patients can show multiple aetiology simutaneously, and the symptoms can be a result of interactions between them. 

\subsection{Cholinergic hypothesis\label{ach-hypo}}
First proposed in \citeyear{bartus82} by \citeauthor{bartus82}, the cholinergic hypothesis the oldest theory explaining the aetiology of the \gls{ad}. \citeauthor{bartus82} proposed that the cognitive symptoms of \gls{ad} is a result of impairments in the cholinergic system function. This claim is based on  the importance of cholinergic system in learning and memory \citep{deutsch71}. Moreover, it has been found young experimental participants with scopolamine, a nicotinic \gls{ach} receptor antagonist, shows impaired memory performance at the same level of elderly participants. This impairment can be rescued by \gls{ach} agonist such as physostigmine \citep{drachman74}. Together with the evidence that cholinergic synapses in \gls{ad} undergo profound degeneration \citep{whitehouse82}, these results led to the proposal of the cholinergic hypothesis, and development of the \gls{ache} inhibitors donepezil, rivastigmine and galantamine, the only three drugs available to mild-moderate \gls{ad} patients \citep[][; also see Section \ref{treatment}]{bartus00}. 

The cholinergic hypothesis received mixed support after its publication, at which time details of the cholinergic projections are studied in the context of \gls{ad}. It has been found in \gls{ad}, cholinergic neurons have impaired synthesis pathway for \gls{ach} \citep{milner87}. \gls{ad} is also found to correlate with a decrease of nicotinic \gls{ach} receptor density, however the muscarinic \gls{ach} receptors are not affected \citep{nordberg92, burghaus00}. Moreover, the degeneration of cholinergic synapses is most prominent in areas involved in learning and memory \citep{geula96}. 

While the cholinergic hypothesis is supported by these evidence, it is however also challenged, primarily by an absent of evidence on early development of \gls{ad}. For example, degeneration of cholinergic neurons is only found in late-stage \gls{ad}, but not present in preclinical and early-stage patients \citep{davis99}, and cholinergic neurons in hippocampus and frontal cortex is found increased in early \gls{ad} patients \citep{dekosky02}. Moreover, the limited effect of \gls{ache} inhibitors on treating \gls{ad} suggests that the cholinergic degeneration may be a secondary effect. This has generated significant doubt on the role of cholinergic system in \gls{ad} aetiology and the popularity of cholinergic hypothesis has declined. The cholinergic hypothesis is now often viewed as either a effect of amyloid cascade (Discussed in Section \ref{amy-hypo}), or a cofactor for other risk factors for neuronal degeneration \citep{roberson97, contestabile11}.


\subsection{Amyloid cascade hypothesis\label{amy-hypo}}
The amyloid cascade hypothesis, first proposed by \citeauthor{hardy92} in \citeyear{hardy92}, states that abnormal accumumulation of the protein \abeta is a primary trigger for the \gls{ad}. Accumulation of \abeta initiates a series of pathological events in the brain, including tangle formation, neuritic injury, and finally dysfunction and death of neurons. This claim is based on the discovery that the ``senial plaques'', which characterized the \gls{ad}, is composed of \abeta \citep{masters85}. \abeta is the result of sequential cleavage of the \gls{app} on chromosome 21. In the Down syndrome patients, \gls{app} is over-expressed, and all Down syndrome patients have early onset of \gls{ad} like neural pathology, including amyloid plaque and \gls{nft} formation\citep{wisniewski85, hardy02}. Moreover, the findings that mutation in the \gls{app} (Dutch type) causes cerebral hemorrhage and deposition of \abeta further supports a causal role of \gls{app} in \abeta accumulation.

Genetic and molecular studies after the original discovery have described how \abeta is processed from \gls{app} in detail. \gls{app} can be processed in two pathways: a amyloidgenic pathway and a non-amyloidgenic pathway. In the non-amyloidgenic pathway, \gls{app} is cleaved by \textalpha-secretase into a soluble N-terminal fragment s\gls{app}\textalpha, and a 83-amino-acid C-terminal peptide, C83. The C83 is then cleaved by \textgamma-secretase into a small p3 peptide and the \gls{aicd}. However in the amyloidgenic pathway, the \gls{app} is first processed by \textbeta-secretase into N-terminal fragment s\gls{app}\textbeta, and a 99-amino-acid C-terminal peptide, C99. The C99 is then cleaved by \textgamma-secretase to form a extracellular peptide containing 38-43 amino acids, \abeta and the \gls{aicd} \citep{barage15}. The most toxic forms of \abeta are \abeta[40] and \abeta[42]. While \abeta[40] is more abundent, \abeta[42] is more hydrophobic and aggregates faster \citep{walsh07}. 

Evidence from animal models and \gls{fad} supports a causal role of \abeta in \gls{ad} symptoms. Autosomal dominent mutations in genes encoding the \textgamma-secretase (\textit{PSEN1} and \textit{PSEN2}) renders the non-amyloidgenic pathway for \gls{app} processing defective, allowing more \abeta production through the amyloidgenic pathway, resulting in many instances of \gls{fad} \citep{suzuki94, levy-lahad95, rogaev95}. The effect of mutation in \gls{fad} is also confirmed in mouse models where the mutant human genes found in \gls{fad} is expressed. These mouse models develop amyloid plaques in the brain, together with learning and memory deficits \citep{hsiao96, dodart99, chishti01, westerman02}. Moreover, acute treatment of \gls{abeta} is toxic to hippocampal and neocortical neurons, both in cultured cells and \textit{in vivo} \citep{lacor04, shankar08}. In addition to these evidence, genome-wide association studies have found regions associated with increased \abeta levels \citep{kehoe99, myers00}.

However, there is more evidence suggesting the aetiology of \gls{ad}, especilally late-onset \gls{ad}, is more complicated than stated by amyloid cascade hypothesis. Longitudinal \gls{pet} studies have shown that while amyloid plaque load in humans poses a significant risk of cognitive decline in both cognitively normal and \gls{mci} subjects, it is only weakly associated with cognition, and not predictive of the disease progression in \gls{ad} patients \citep{chen14}. Moreover, amyloid plaque prevalence in elderly subjects is more than twice of \gls{ad} prevalence \citep{rowe10, ad16}, suggesting a large population carrying significant plaque load, yet the cognition is unaffected. On the other hand, recent clinical trials aiming to remove plaques from \gls{ad} patients using immunotherapy shows success in reducing the plaque load, however unable to prevent the cognitive decline of the patients \citep{farlow15, siemers16, sevigny16}. These results suggest that the amyloid palque load does not correlate with the progression of \gls{ad} or the cognitive symptoms, and the role of \abeta in the development of \gls{ad} is more nuanced. 

In light of the negative evidence, amendament to the amyloid cascade hypothesis is proposed to explain the missing relationship between amyloid plaques and cognitive performance. Given the consistent evidence of \abeta toxicity on synapse and neurons\citep{ferreira15}, and that synaptic degeneration correlates with the cognitive decline in \gls{ad} \citep{selkoe02, coleman04}, it has been proposed that the soluble fractionof \abeta oligomers, which is most toxic to neurons, are cardinal to \gls{ad} and insoluable plaques are irrelavent \citep{ferreira15}. Moreover, \citet{canter16} proposed \abeta destabilizes synapses and impairs normal function of neural networks, and its this destructive effect creates the cognitive symptoms of \gls{ad}. Evidence for this hypothesis is descussed in Section \ref{network-hypo}. In conclusion, the amyloid cascade hypothesis is supported by the most compelling evidence to date, the reality of \gls{ad} aetiology is more complex than what the hypothesis suggested.

\subsection{Tau hypothesis}

The \atau hypothesis is based on the findings that the other neural pathological signature of \gls{ad}, the \glspl{nft}, are composed of the hyperphosphorylated \atau proteins \citep{grundke-iqbal86}. It is therefore hypothesized that abnormal functions of the \atau proteins leads to a series of pathological events which is responsible for neurodegeneration and symptoms of the \gls{ad} \citep{goedert11}.

In physiological conditions, the \atau proteins, expressed by the \gls{mapt} gene, are widely expressed in the brain. In physiological conditions, the \atau proteins assemble and stabilize microtubules in the cells, which is important for maintaining cell structure and intracellular transport. However, in \gls{ad}, \atau is found hyperphosphorylated by multiple kinases, including \gls{gsk3}, \gls{cdk5} and \gls{mapk} \citep{singh94}. The hyperphosphorylated \atau is unable to bind to the microtubule building block, tubulin, and therefore hypothesized to cause the destabilization of microtubules \citep{bramblett93, yoshida93, alonso94}. The unbound \atau proteins under high concentrations, are likely to aggregate and sebsequently from twisted paired helical filaments, which are the main component of the \glspl{nft} \citep{kidd63, kuret05}. Destabilized microtubules and the toxicity of the \glspl{nft} will lead to failure in axonal transportation of vesicles and mitochondria, and finally the death of the neuron.

In \gls{ad}, the presence of \glspl{nft} is significantly correlated with the cognitive deficits of the patients \citep{hyman12}. Moreover, the spread of \atau pathology also follows the progression of \gls{ad} symptoms, first affecting brain regions important for learning and memory, such as hippocampus and amygdala, then spread out to neocortex \citep{braak91}. These earlier results from post-mortem studies are confirmed by recent \gls{pet} studies using \atau tracers, where the amount and distribution of the \atau protein in the brain can be measured \textit{in vivo} \citep{ossenkoppele16, scholl16}. These results suggest the importance of \atau pathology in the development of \gls{ad}.

Despite the correlation with cognitive symptoms of \gls{ad}, \atau pathology is now considered secondary to the cause of \gls{ad}. This claim is first supported by the discovery that mutation in the MAPT gene, which encodes the \atau protein, causes a familial form of frontaltemporal dementia \citep{hutton98, poorkaj98}. Moreover, \atau pathology has been implicated in many other neurodegenerative disorders \citep[e.g.][]{williams09, mckee16}. These evidence suggest that \atau pathology is sufficient for causing neurodegeneration. However, as there is no amyloid plaque formation in these disorders or mouse model of \atau pathology \citep{gotz04}, it is unlikely to be the primary cause for \gls{ad}.

There is also evidence suggesting \atau pathology may be downstream of the amyloid cascade. It has been found in \gls{ad}, plaque formation preceeds tangle formation \citep{price99}, and \abeta binds to the \atau protein \textit{in vitro}, and accelarates \atau hyperphosphorylation \citep{guo06, zempel10}. In mouse models of \atau pathology, \abeta accelerates \gls{nft} formation and neural degeneration \citep{lewis01, terwel08}. These evidence further supports the amyloid cascade hypothesis. 

In conclusion, evidence suggest \atau pathology is secondary to the cause of \gls{ad}. However given its strong correlation with the cognitive deficits in \gls{ad} and important role in neurodegeneration, lots of efforts are still put to understand mechanism underlies it. 

\section{Hippocampus and memory}
In the early studies of memory by 
The hippocampus is not a focus for memory studies until the patient H.M..
\citep{opitz14} hpc and memory
\citep{shohamy13} hpc in cognition
\citep{huijgen15} hpc node of memory

\citep{lee16} lesion studies review
\citep{meister15} hpc to motor
\citep{mcdonald13} memory to behaviour
\citep{middei14} hpc in learning disorder
\citep{eichenbaum14} hpc where vs episodic
\citep{sasaki15} hpc memory circuit


\section{GluA2, synapses}
\subsection{Synaptic effect of A\textbeta{} accumulation}

\section{synaptic deficits in \gls{ad}}

\section{Circuitry mechanism for information processing in hippocampus}
    network laterality, place coding, goal-directed review\citep{kitanishi2016}
    \citep{smith14} cotext representations review
    homeostatic regulation \citep{mizumori13}
    \citep{draguhn14} hpc oscillation
    \citep{oh14} excitability

\subsection{anatomy and information flow}
\citep{igarashi14} CA1 heterogeneity

\subsection{pattern separation and pattern completion}
\citep{knierim16} review!
\citep{rolls13} review
\citep{kesner15} computational
    Kinerim J, neuron 2014
    \citep{mehta15} spatial map plasticity in hpc

\section{Network deficits in \gls{ad} \label{network-hypo}}
\citep{moodley14} hpc in \gls{ad}
\citep{mufson15} hpc plasticity in \gls{ad}
\citep{saura15} gene expression diff in \gls{ad} hpc
    \subsection{synaptic function}
    \subsection{Network}


\section{Methods and tools for investigating neural population activity}
\subsection{\textit{In vivo} electrophysiology recording}
\subsection{Immunohistochemistry and \textit{In situ} hybridization against \glspl{ieg}}
\subsection{Two-photon calcium imaging}
\subsection{Miniature endoscopes}
\subsubsection{Fiber bundle based fluorescence endoscopes}
\subsubsection{\gls{grin} lens based fluorescence endoscopes}
\subsection{Conclusion}


\section{Hypothesis and Research Aims}
\citep{rozeske14} contextual fear conditioning
2-5 pages
