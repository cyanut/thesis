\chapter{Memory formation in Alzheimer's Disease}
\section{Introduction}

\section{Material and Methods}\todo{edit methods}

\subsection{Animals and viral vectors}

Heterozygous and wild-type from the Tg-CRND8 Alzheimer model mouse strain were used in the experiment. All animals were caged in groups of 4 or 5, with a 12-hour light/dark cycle. Food and water are provided \textit{ad libitum} to all animals. AAV--DJ--syn--GCaMP6f virus was purchased from Stanford University Gene and Viral Vector Core. Additionally, GCaMP6f was introduced by crossing Tg-CRND8 with GP5.17 GCaMP6f strain. Results from F1 of Tg-CRND8 and GP5.17 were pooled with the AAV experiments.

\subsection{Contextual fear conditioning}

Animals underwent contextual fear conditioning three weeks after mini-microscope baseplate implantation. One hour before training, animals received either TAT-GluA2 peptide (i.p., \SI{15}{\mmol\per\kg}) or vehicle injection. A mini-microscope is attached to the animal to record calcium activities during both training and testing of contextual fear conditioning. During training, animals were confined in the chamber for \SI{5}{\minute}. A footshock of \SI{0.5}{\mA} was delivered at \SI{4}{\minute} time point. During testing session \SI{24}{\hour} later, animals were placed back in the training environment for \SI{10}{\minute}. 

\subsection{Analysis}

All traces were normalized to have zero median and unit noise standard deviation. The noise standard deviation was estimated from median absolute deviation of the trace. The signal to noise ratio (SNR) were calculated as the ratio of maximum signal intensity and noise standard deviation. Only traces with more than 10 SNR and animals with more than 20 cells are included in the analysis. The average activity of a cell was calculated by the area under the calcium trace above 3 standard deviation of the noise divided by duration.

Freezing information and spatial information (below) were calculated according to \citet{skaggs93}. The information measurement represents how much cell activity at a single time can predict about freezing or location of the animal.  It was calculated using the following formula:

$Information = \displaystyle\sum_{i}^{}P_i  \frac{R_i}{R} log_2 \frac{R_i}{R}$

where $P_i$ represents the probability of the animal being in state $i$,  $R_i$ represents the average cell activity when the animal is in state $i$, and $R$ is the average cell activity during the session. For freezing, the states are freezing and not freezing. For spatial information, environment is divided in \num{12 x 9} grids, and the states are when animal is in one of the grids.

\section{Results}
First, we compared the average activity between groups, and have found the Tg animals are hyper-active. We have then investigated the average cell activity during freezing, and found CA1 cells decrease activity to encode freezing. However, Tg animals have higher activity during freezing than WT animals. This suggest Tg animals may have 

We then investigated that how cell Tg cells encodes freezing. First we looked at cells individually, and calculated the mutual information between cell firing and freezing (Skaggs et al., 1993). Then using machine learning methods, we investigated how freezing is encoded at a network level by training general classifiers to predict animals' behaviour from recorded cell activity. Both approach suggest that Tg animals have consistent worse freezing encoding both at a cellular level but also at a network level. 

We have also investigated how the animals freeze, by A detailed analysis of the animals' freezing behaviour suggests that the Tg animals often start to freeze, however each freezing period is significantly 
\subsection{Overall cell activity}
\subsection{Freezing Behaviour}
\subsection{Freezing information encoding}
\subsection{Decoding freezing using machine learning}
\section{Discussion}
