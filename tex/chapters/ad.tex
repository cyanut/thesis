\chapter{Memory formation in Alzheimer's Disease}
\section{Introduction}
\section{Material and Methods}
\subsection{Animals}
\subsection{Contextual fear conditioning}
\subsection{Analysis}
\section{Results}
First, we compared the average activity between groups, and have found the Tg animals are hyper-active. We have then investigated the average cell activity during freezing, and found CA1 cells decrease activity to encode freezing. However, Tg animals have higher activity during freezing than WT animals. This suggest Tg animals may have 

We then investigated that how cell Tg cells encodes freezing. First we looked at cells individually, and calculated the mutual information between cell firing and freezing (Skaggs et al., 1993). Then using machine learning methods, we investigated how freezing is encoded at a network level by training general classifiers to predict animals' behaviour from recorded cell activity. Both approach suggest that Tg animals have consistent worse freezing encoding both at a cellular level but also at a network level. 

We have also investigated how the animals freeze, by A detailed analysis of the animals' freezing behaviour suggests that the Tg animals often start to freeze, however each freezing period is significantly 
\subsection{Overall cell activity}
\subsection{Freezing Behaviour}
\subsection{Freezing information encoding}
\subsection{Decoding freezing using machine learning}
\section{Discussion}
