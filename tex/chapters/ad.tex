\chapter{Memory formation in Alzheimer's Disease \label{chap-ad}}
\section{Introduction}

Progressive memory loss is the defining feature of Alzheimer’s disease (AD).  The early stages of AD are characterized by the inability to acquire new memories, often observed as difficulty in recalling recent events \citep{albert96, storandt89, wilson83}. Over time, memory loss gradually worsens and additional cognitive symptoms (including aphasia, apraxia, agnosia, dyscalculia, executive function impairment) develop \citep{shah06}. Although the precise cause(s) of AD remain elusive, \abeta{}, derived from \gls{app}, is widely implicated \citep{selkoe02, tanzi01}. Mutations in APP cause \gls{fad} \citep{hardy02,price98} and increase \abeta{} levels \citep{cai93, citron92}.  While high \abeta{} may eventually trigger cell death, memory deficits are observed in AD patients before extensive neurodegeneration \citep{selkoe02}, suggesting that \abeta{} itself interferes with the ability to acquire memories, at least in the early stages of the disease.  

Memory formation is thought to be mediated by the strengthening of excitatory synaptic transmission between neurons \citep{bailey93, lamprecht04}.  The majority of fast excitatory neurotransmission is mediated by \glspl{ampar}.  \Glspl{ampar} are localized in dendritic spines and composed of four types of subunits (GluA1-4), which combine to form tetramers \citep{hollmann94}.  In mature hippocampal pyramidal neurons, most AMPARs consist of two identical heterodimers comprised of GluA1/2 and GluA2/3 dimers \citep{wenthold96}.  \Glspl{ampar} are highly dynamic and undergo rapid shuttling between the plasma membrane and internal recycling pools. For instance, the activity-dependent removal of \gls{ampar} from the synapse (endocytosis) rapidly and persistently reduces surface expression of GluA2-containing \glspl{ampar}. This endocytosis is critical for the expression of some forms of \gls{ltd}, a type of synaptic plasticity that is characterized by a decrease in synaptic strength and dendritic spine size/density \citep{collingridge04, malinow02, zhou04}. In addition, endocytosis of GluA2-containing \gls{ampar} is implicated in homeostatic synaptic scaling \citep{gainey09}. Therefore, changes in surface expression of GluA2s may critically modulate memory formation.  

Intriguingly, a recent finding shows that \abeta{} potentiates endocytosis of GluA2-containing \glspl{ampar} in hippocampal organotypic slice cultures.  Hsieh and colleagues showed that high levels of \abeta{} (produced by transiently expressing human \gls{app} or incubating slices directly in synthetic \abeta) decreases both synaptic strength and spine density by internalizing GluA2-containing \glspl{ampar} \citep{hsieh06}. This finding was replicated by applying synthetic \abeta{} peptides to neuroblastoma N2A cells \citep{zhao10} or dissociated hippocampal neurons \citep{liu10, zhao10}. However, whether a decrease in synaptic GluA2 levels is responsible for the memory deficits observed in \gls{ad} patients or even in mice designed to model \gls{ad} is not known. Here using a mini-microscope, we directly investigated whether interfering with GluA2-containing \gls{ampar} endocytosis is sufficient to reverse the circuitry and memory deficits observed in a mouse models of early \gls{ad}.
\section{Material and Methods}

\subsection{Animals and vectors}
All mice were housed in groups of 3--5 on a 12-hour light/dark cycle. Food and water are provided \textit{ad libitum} to all mice. Experiments were performed during the light phase of the circadian cycle. Mice were at least 8 weeks old at the beginning of all experiments. All experiments were conducted in accordance to the Hospital for Sick Children Animal Care and Use Committee.

\subsubsection{TgCRND8 mice}
TgCRND8 mice were developed at the Centre of Research for Neurodegenerative Diseases (CRND) and carry a human APP695 transgene with the Swedish (K670N-M671L) and Indiana (V717F) FAD mutations under the regulation of the Syrian hamster prion promoter \citep{chishti01}. Transgenic mice were maintained on a 129S6/SvEvTac background. TgCRND8s were then crossed with either \gls{wt} C57BL/6NTac or GP5.17. \Gls{tg} and \gls{wt} litter-mates of F1 generation were used in the experiments.


\subsubsection{GP5.17 mice}
GP5.17 mice transgenically express the fluorescent calcium indicator GCaMP6f under the Thy1 promoter \citep{dana14}. Offspring of TgCRND8 $\times$ GP5.17 positive for GCaMP6f and negative for \gls{app} were included in the \gls{wt} group. Double positive offspring were included in the \gls{tg} group.


\subsubsection{Viral vectors}
In some TgCRND8 mice, GCaMP6f was delivered using \gls{aav}. GCaMP6f expression was controlled by the \gls{hsyn} promoter. AAV--DJ--syn--GCaMP6f virus was purchased from Stanford University Gene and Viral Vector Core and used undiluted. 

\subsubsection{\tglu{} peptide}
To deliver the \glu{} construct (\texttt{YKEGYNVYG}) to our target region, we attached it to the protein transduction domain of the \gls{hiv} \textit{tat} gene (TAT peptide). The TAT peptide is able to be transported across cell membrane and \gls{bbb}. \tglu{} was synthesized from the sequence \texttt{YGRKKRRQRRRYKEGYNVYG}. It was injected in saline solution.


\subsection{Viral infusion}

Each mouse received \gls{ip} injection of atropine (\SI{0.1}{\mg\per\kg}) and chloral hydrate (\SI{400}{\mg\per\kg}) before being secured on a stereotaxic frame. An incision was made on the scalp and the skin was pulled to the side to reveal the skull. Holes were drilled in the skull above \gls{ca1} for micropipette injection. Virus was loaded into a glass micropipette and gradually lowered to target coordinates. \SI{1.5}{\ul} of virus was injected on each side at a rate of \SI{0.12}{\ul\per\min}, targeting \gls{ca1} (\gls{a/p} \SI{-1.8}{\mm}, \gls{m/l} \SI{1.5}{\mm}, \gls{d/v} \SI{3.5}{\mm} from Bregma). The micropipette was left in the brain for an additional \SI{10}{\min} before slowly retracted. The incision was sutured and treated with antibiotics. Each mouse then received subcutaneous injection of analgesic (ketoprofen, \SI{5}{\mg\per\kg}) before returned to a partially heated clean cage for recovery.

\subsection{Histology}
Placement of lens implants and extent of viral infection was determined by gCaMP6f fluorescence expression \textit{post-mortem}. After all experiments, mice were transcardially perfused with first \gls{pbs} then \SI{4}{\percent} \gls{pfa}. The brains were dissected,  kept in \SI{4}{\percent} \gls{pfa} overnight, and washed with \gls{pbs}. The brains were then sliced coronally on a vibrotome (VT1200S, Leica) to \SI{50}{\um} thickness. Slices containing \gls{la} were then mounted on gelatin-coated glass slides with a hardening mounting media (Permaflour, ThermoScientific). The mounted brain slices were assessed under an epi-fluorescence microscope(Eclipse 80i, Nikon) for histology.


\subsection{Contextual fear conditioning}
Fear conditioning chambers (\SI{31 x 24 x 21}{\cm}; MED Associates, St. Albans, VT) consisted of 2 stainless steel and 2 clear acrylic walls with a stainless steel shock-grid floor (bars \SI{3.2}{\mm} diameter, spaced \SI{7.9}{\mm} apart). A plastic drop-pan containing a \SI{70}{\percent} ethanol solution was placed below the grid floor. A fan provided low-level white noise during training and testing in the context. Behaviour was monitored by overhead cameras, which recorded video images of the chambers at \SI{15}{\Hz}. 

Mice underwent contextual fear conditioning three weeks after mini-microscope baseplate implantation. One hour before training, mice received either \tglu{} peptide (\SI{15}{\mmol\per\kg}, \gls{ip}) or vehicle injection. A mini-microscope was attached to the mouse to record calcium transients during both training and testing of contextual fear conditioning. 

During training, mice were confined in the chamber for \SI{5}{\minute}. A 2-second foot-shock of \SI{0.5}{\mA} was delivered at \SI{4}{\minute} time point. During testing session \SI{24}{\hour} later, mice were placed back in the training environment for \SI{10}{\minute}. 

\subsection{Motion tracking}
Videos of mouse behaviour were encoded as grey-scale images. Due to a dark background, occlusion by mini-microscope wires and commutators, and changing shadows, no simple feature was able to reliably identify the mouse from the background. Instead, we calculated the distribution of multiple features from a training set of videos where mice were correctly tracked, and used all the features together to estimate the position of the mouse for each frame in a new video. During model fitting, distribution of every feature was estimated with a Gaussian kernel with a bandwidth calculated using Silverman's rule of thumb \citep{silverman86}: $1.06\sigma n^{-\frac{1}{5}}$, where $\sigma$ is the standard deviation of the samples, and $n$ is the number of samples.

\subsubsection{Features} 

\paragraph{Pixel intensity.} Pixel intensity at mouse's position. This feature tries to capture the mouse's fur colour.

\paragraph{Normalized pixel intensity.} Normalized pixel intensity of mouse's position, where the frame is normalized to a zero mean and unit standard deviation. This feature tries to capture mouse's fur colour when the illumination in the chamber varies across frames.

\paragraph{Foreground pixel intensity.} Pixel intensity difference between foreground and background images at the mouse's position. The background image of the environment was generated by taking the mean pixel density across time. This feature tries to separate mice from any background pixels with similar colour.

\paragraph{Difference to low pass intensity.} Pixel intensity difference between foreground image and low-pass filtered foreground image (\SI{10}{\mm} window) at the mouse's position. This feature tries to capture the fact that the mouse's colour is close to uniform, therefore eliminates sporadic noise pixels with similar colour.

\paragraph{Speed.} Distance between the mouse's positions in two consecutive frames. 

\paragraph{Change in intensity.} Difference of pixel intensity at the mouse's positions in two consecutive frames. This feature captures the fact that the mouse's colour is relative consistent between frames.

\paragraph{Change in intensity (blurred).} Similar to change in intensity, but using Gaussian blurred images (\SI{10}{\mm} window) to remove effect of random noise.

\paragraph{Magnitude of acceleration.} Magnitude of acceleration vector, which is calculated as the vector difference of two consecutive velocity calculations. 

\paragraph{Segmentation area.} Edges in the frame are detected (Canny, lower threshold=100, higher threshold=200). The resulting image is then morphologically closed (6 iterations) to remove sporadic edges. The area enclosing the position of the mouse is then calculated. This feature tries to capture the size of the mouse.

\subsubsection{Tracking}

The mouse's position over time was modelled as a \gls{hmm}, as shown in Figure \ref{f.ad.hmm}. The mouse's true position at time $i$ was represented by the latent variables $z_i$. At each time point, the value of the all the feature measurements are represented as $x_i$. Under this model, the goal of tracking the mice is equivalent to estimating the latent variables $z_i$ using $x_i$. 

\begin{figure}[h]
    
\begin{tikzpicture}[biglatent/.style={latent, minimum width={width("$z_{n+1}$")+6pt}}, bigobs/.style={biglatent, fill=gray!25}]
    \node[biglatent] (z0) {$z_0$};
    \node[biglatent, right=of z0] (z1) {$z_{1}$};
    \node[biglatent, right=of z1] (z2) {$z_2$};
    \node[right=of z2] (z25) {\Large{$\cdots$}};
    \node[biglatent, right=of z25] (z3) {$z_{n-1}$};
    \node[biglatent, right=of z3] (z4) {$z_{n}$};
    \node[biglatent, right=of z4] (z5) {$z_{n+1}$};
    \node[right=of z5] (z55) {};

    \foreach \t [count=\i from 0] in {0,1,2,n-1,n,n+1}{
        \node[bigobs, below=of z\i] (x\i) {$x_{\t}$};
        \edge {z\i} {x\i};
    };
    \node[right=of x2] (x25) {\Large{$\cdots$}};
    \edge {z0} {z1};
    \edge {z1} {z2};
    \edge {z2} {z25};
    \edge {z25} {z3};
    \edge {z3} {z4};
    \edge {z4} {z5};
    \edge {z5} {z55};

\end{tikzpicture}

    \caption[\gls{hmm} model for tracking mice.]{\gls{hmm} model for tracking mice. The mouse's true position at time $i$ is represented by the latent variables $z_i$, measurements of the mouse's position is represented by $x_i$. \label{f.ad.hmm}}
\end{figure}

Using particle filter, we approximated the posterior distribution $f_{z_n}(z_n|X_n)$ with $L$ particles at $\{a^i\}_{i=1}^L$ with weights $\{w^i\}_{i=1}^L$, such that the posterior distribution can be approximated with a mixture of Dirac delta function: 
\begin{equation} \label{zn_approx}
    f_{z_n}(z_n|X_n) \approx \sum_{i=1}^Lw_n^i\delta(z_n - a_n^i)
\end{equation}

On the other hand, using Bayes' rule and the \gls{hmm} structure, we have:
\begin{align*}
    f_{z_n}(z_n|X_n) &\propto f_{x_n}(x_n|z_n, X_{n-1}) \cdot f_{z_n}(z_n|X_{n-1}) \\
                     &= f_{x_n}(x_n|z_n) \int f_{z_n}(z_n|z_{n-1})f_{z_{n-1}}(z_{n-1}|X_{n-1})dz_{n-1}  \\
                     &\approx f_{x_n}(x_n|z_n) \int f_{z_n}(z_n|z_{n-1})\cdot \sum_{i=1}^Lw_{n-1}^i\delta(z_{n-1}-a_{n-1}^i)dz_{n-1} \\
                     &= f_{x_n}(x_n|z_n)  \sum_{i=1}^Lw_{n-1}^i \int f_{z_n}(z_n|z_{n-1})\delta(z_{n-1} - a_{n-1}^i)dz_{n-1} \\
                     &= f_{x_n}(x_n|z_n) \sum_{i=1}^Lw_{n-1}^if_{z_n}(z_n|a_{n-1}^i) 
\end{align*}
where the probability distribution $f_{x_n}(x_n|z_n)$ is the measurement model, which we approximate with the normalized product of all feature distributions. The motion model $f_{z_n}(z_n|z_{n-1})$ is approximated with the distribution of speed at all directions. Combined with (\ref{zn_approx}), we can find $\{a_n^i\}_{i=1}^L$ by sampling from the mixture $\sum_{i=1}^Lw_{n-1}^if_{z_n}(z_n|a_{n-1}^i)$, and calculate $w_n^i = f_{x_n}(x_n|a_n^i)$. The centre of mass of all particles is used as an estimate of the mouse's position $z_n$. This process is then iterated for all frames to track the mouse's position in one behavioural video.

\subsection{Analysis}

\subsubsection{Freezing behaviour}
Freezing behaviour of the mice during first \SI{10}{\min} of contextual fear memory testing were assessed by an experimenter blind to the genotype and treatment of the mice. Frame-to-frame timestamps for the beginning and end of freezing bouts were recorded.

\subsubsection{Preprocessing for calcium transients}

All calcium transients were normalized to have zero median and unit noise standard deviation. The noise standard deviation was estimated from the median absolute deviation of the transient. The \gls{snr} was calculated as the ratio of maximum signal intensity and noise standard deviation. Only transients with more than 10 \gls{snr} and mice with more than 20 cells were included in the analysis. The average activity of a cell was calculated by the area under the calcium transients above 3 standard deviation of the noise divided by duration.

\subsubsection{Mutual information}

The mutual information measurement of two random variables represents the degree of relatedness between them. The mutual information between the calcium transient for each cell $C$, a continuous random variable and the freezing state of the mouse $F$, a discrete random variable, is defined as:
\begin{equation} \label{mutual_info_formula}
    I(C, F) = \int\limits_{c \in C} \sum_{f \in F} P(c,f)\log\frac{P(c,f)}{P(c)P(f)}dc
\end{equation}

Here we chose the \gls{kl} to estimate the mutual information. While electrophysiological studies have commonly adapted from \citet{skaggs93} to calculate mutual information between cell activity and position of the animal for characterization of spatial encoding in hippocampus \citet[e.g.]{knierim95, skaggs96, ji07, cheng13, roux17}, unfortunately this method cannot be applied to raw calcium imaging traces. \citet{skaggs93}'s method made an important assumption: that during a sufficient short time, the electrophysiological recording of a neuron is essentially binary, with the neuron either firing or not. With this assumption, the firing rate of a single neuron can be modelled as a Poisson distribution, and the integral in Formula \ref{mutual_info_formula} is then trackable.

Unfortunately, the calcium trace we have collected is a continuous variable which violate \citet{skaggs93}'s assumption. A different way to estimate the mutual information then is to quantize the cell activity into finite bins. Mutual information between the binned cell activity and behavioural states can then be calculated, as the integral of Formula \ref{mutual_info_formula} is then replaced by a summation over all bins of cell activity. This binning method is however negatively biased, and only represents a lower bound of the actual information content measurement. Correction methods exists, however they tend to be sensitive to specific distributions of data \citep{paninski03}.

Another method for entropy estimation exploits the nearest neighbour distance within the dataset, which contains information about the underlying probability distribution of the continuous variable. \citet{kozachenko87} first showed that entropy estimation using nearest neighbour distance is unbiased as long as the probabilistic density function of the continuous random variable is smooth. Here we preferred \gls{kl} for mutual information measurement, as it is non-parametric and robust, while estimates from binning-based methods, even with correction for their asymptotic bias, is sensitively dependent on the choice of binning size and method to correct bias \citep{victor02, ross14}. 

To estimate the mutual information between freezing and calcium transients, we used the \gls{kl} twice to estimate the entropy of calcium transient $H(C)$ and its conditional entropy on freezing $H(C|F)$ \citep{ross14, victor02}. The mutual information was then calculated using the identity:
\begin{equation} \label{cond_h}
    I(C, F) = H(C) - H(C|F)
\end{equation}
The \gls{kl} is a nearest-neighbour entropy estimator. Given a data point $C_{i,0}$ from a continuous random variable and its m\textsuperscript{th} nearest neighbour $C_{i,m}$, we define $V_{i,m}$ as the volume of sphere centred at $C_{i,0}$ with a radius equal to the distance between $C_{i,0}$ and $C_{i,m}$. The entropy of $C$ can be estimated as:
\begin{equation} \label{est_h}
    H(C) \approx \langle \log V_{i,m}^F\rangle + \varphi(N) - \varphi(m)
\end{equation}
where $N$ is the number of samples, and $\langle\cdot\rangle$ denotes the average over $1\ldots N$, and $\varphi$ is the digamma function. Similarly, we can calculate the entropy for the conditional entropy $H(C|F)$:
\begin{equation} \label{est_cond_h}
    H(C|F) \approx \langle \log V_{i,k}^{F_i} \rangle + \langle \varphi(N_{F_i}) \rangle - \varphi(k)
\end{equation}
where $N_{F_i}$ is the number of samples in the freezing state of $i$\textsuperscript{th} sample, and here we use the $k$\textsuperscript{th} nearest neighbour conditioned on $F_i$ for the calculation.

To avoid sampling error, we fix $k$ for each sample, but change $m$ to the total number of samples between the sample $C_{i,0}$ and its $k$\textsuperscript{th} nearest neighbour conditioned on $F_i$, $C_{i,k}^{F_i}$. Therefore we have:
\begin{equation*}
\log V_{i, m_i} = \log V_{i, k}^{F_i}, \forall i 
\end{equation*}
Plug (\ref{est_h}) and (\ref{est_cond_h}) to (\ref{cond_h}), we have:
\begin{equation*}
    I(C, F) \approx \varphi(N) - \langle\varphi(m_i)\rangle - \langle\varphi(N_{F_i})\rangle + \varphi(k)
\end{equation*}




\subsubsection{Machine learning}

Time-course data for the calcium transients and mouse behavioural states were paired and shuffled across time. The calcium transient for each cell was regarded as a single feature. Classifiers were then trained and validated using 5-fold cross validation. Specifically, the shuffled data was divided into 5 equal blocks. Each block was presented one at a time, and the classifier was trained using the remaining 4 blocks. The presented block was then used to test the performance of the classifier. The classifier prediction from each block was then concatenated and sorted into the original order. In the analysis, we used two classifiers, a \gls{nbc} and a \gls{gsvm}. 

\paragraph{Naive Bayes classifier (\gls{nbc}).}
In an \gls{nbc}, the classifier tries to infer the likelihood of the $i$th target class $T_i$ given the features $\mathbf{x} = (x_1, x_2, \dots, x_n)$. Therefore, we have:
\begin{equation*}
    P(T_i|\mathbf{x}) = \frac{P(T_i, \mathbf{x})}{P(\mathbf{x})}
\end{equation*}
The denominator is not relevant for classification purposes, since it does not depend on the target class $T_i$. Therefore, repeatedly applying the chain rule, we have:
\begin{align*}
    P(T_i|x_1, \dots, x_n) &\propto P(x_1, \dots, x_n, T_i) \\
                           &= P(x_1|x_2 \dots, x_n, T_i) P(x_2, \dots, x_n, T_i) \\
                           &\vdots \\
                           &= P(x_1|x_2, \dots, x_n, T_i)  P(x_2|x_3, \dots, x_n, T_i) \ldots  P(x_{n-1}|x_n, T_i)  P(x_n|T_i)  P(T_i) \\
                           &= P(T_i)  \prod_{k=1}^{n-1} P(x_k|x_{k+1}, \dots, x_n, T_i)
\end{align*}
To make the product tractable, \gls{nbc} discards the interaction between all the features, and assumes the features are conditionally independent. Therefore, each of the terms in the product can be reduced to:
\begin{equation*}
    P(x_k|x_{k+1}, \dots, x_n, T_i) = P(x_k|T_i)
\end{equation*}
Therefore:
\begin{equation*}
    P(T_i|\mathbf{x}) = C\cdot P(T_i) \cdot \prod_{k=1}^n P(x_k|T_i)
\end{equation*}
where $C$ is a constant independent of $T_i$. The conditional probability for each feature $P(x_k|T_i)$ is estimated from the training data assuming a Gaussian distribution. During testing, the target class with maximum likelihood $\hat{T}=\underset{i}{\operatorname{arg\,max}} P(T_i|\mathbf{x})$ is predicted.

\paragraph{Gaussian support vector machine (\gls{gsvm}).}
A \gls{svm} aims to find a hyperplane $\mathbf{w}^T\mathbf{x}+b=0$ which separates the target classes with maximum margin. Concretely, for features $\mathbf{x}$ and targets $y$, the \gls{svm} aims to maximize the minimum distance for each data point, transformed by a function $\phi$, to the classifying hyperplane:
\begin{equation} \label{dist}
    \us{\mathbf{w},b}{arg\,max}\Big(\min_{i=1}^{n}\frac{y_i(\mathbf{w}^T\phi(\mathbf{x}_i)+b)}{|\mathbf{w}|}\Big)
\end{equation}
 
Given that $\mathbf{w}$ and $b$ can be scaled without changing the value of this term, we scale $\mathbf{w}$ and $b$ such that the minimum distance to classifying plane $\min_{i=1}^n(y_i(\mathbf{w}^T\phi(\mathbf{x}_i + b))$ is equal to 1. The original objective function therefore becomes:
\begin{align*}
    &\us{\mathbf{w},b}{arg\,max}\Big(\min_{i=1}^{n}\frac{y_i(\mathbf{w}^T\phi(\mathbf{x}_i)+b)}{|\mathbf{w}|}\Big) \\
    =&\us{\mathbf{w},b}{arg\,max}\frac{1}{|\mathbf{w}|}\min_{i=1}^{n}(y_i(\mathbf{w}^T\phi(\mathbf{x}_i)+b)) \\
    =&\us{\mathbf{w},b}{arg\,max}\frac{1}{|\mathbf{w}|} \cdot 1 \\
    =&\us{\mathbf{w},b}{arg\,min}\frac{|\mathbf{w}|^2}{2}
\end{align*}
To allow the classifier to perform with data sets that are not linearly separable, an error term can be added for each data point $\varepsilon_i$.
If the distance from the data point to the classifying hyperplane is at least 1, the value of $\varepsilon_i$ will be $0$. If the data point lies within the margin but is still correctly classified, $\varepsilon_i$ represents the distance from the data point to the boundary of the its correct class, and will have a value between $0$ and $1$. If the data point is incorrectly classified, $\varepsilon_i$ will have a value greater than $1$. The error is controlled by a parameter $C$, which dictates how relaxed the classifying boundary is. Therefore the objective function becomes:
\begin{equation*}
    \us{\mathbf{w},b}{arg\,min}\Big(\frac{|\mathbf{w}|^2}{2} + C\sum_{i=1}^n \varepsilon_i\Big)
\end{equation*}
under the constraints that data are classified for all $i$:
\begin{align*}
    y_i (\mathbf{w}^T\phi(\mathbf{x}_i) + b) &\geq 1 - \varepsilon_i \\ 
    \varepsilon_i &\geq 0
\end{align*}
To find the extrema of the objective function under these constraints, we create the Lagrangian:
\begin{equation} \label{lagr}
    \mathcal{L}(\mathbf{w}, b, \mathbf{\lambda}, \mathbf{\mu})=\frac{|\mathbf{w}|^2}{2} + C\sum_{i=1}^n\varepsilon_i - \sum_{i=1}^n \lambda_i(y_i(\mathbf{w}^T\phi(\mathbf{x}_i) + b) - 1 + \varepsilon_i) - \sum_{i=1}^n \mu_i\varepsilon_i
\end{equation}
where $\mathbf{\lambda}=(\lambda_1, \lambda_2, \ldots, \lambda_n)$, and $\mathbf{\mu}=(\mu_1, \mu_2, \ldots, \mu_n)$. We can optimize it by setting its partial derivatives regarding to $\mathbf{w}$, $b$, $\mathbf{\varepsilon}$ to 0:
\begin{align}\label{dw}
    \frac{\partial\mathcal{L}}{\partial\mathbf{w}} &= \mathbf{w} - \sum_{i=1}^n \lambda_i  y_i  \phi(\mathbf{x}_i) = 0 \nonumber\\ 
    \frac{\partial\mathcal{L}}{\partial{b}} &= - \sum_{i=1}^n \lambda_i y_i = 0 \\
    \frac{\partial\mathcal{L}}{\partial{\mathbf{\varepsilon}_i}} &= C - \lambda_i - \mu_i = 0, \forall i  \nonumber 
\end{align}
Using (\ref{dw}) to eliminate $\mathbf{w}$, $b$, $\varepsilon$ from the Lagrangian (\ref{lagr}), equivalently we aim to minimize the Lagrangian of $\mathbf{\lambda}$:
\begin{equation} \label{lagr2}
    \mathcal{L}^*(\mathbf{\lambda}) = \sum_{i=1}^{n} \lambda_i - \frac{1}{2}\sum_{i=1}^n \sum_{j=1}^n \lambda_i \lambda_j y_i y_j \phi(\mathbf{x}_i)^T\phi(\mathbf{x}_j)
\end{equation}
subject to:
\begin{align*}
    0 \leq \lambda_i \leq C, i &= 1,\ldots,n \\
    \sum_{i=1}^n \lambda_i y_i &= 0
\end{align*}
The parameters $\lambda_i$ can then be solved numerically through quadratic programming. Since Equation \ref{lagr2} does not depend on $\phi$, but only the inner product of $\phi(\mathbf{x}_i)^T\phi(\mathbf{x}_j)$, we can replace the term with a function $K(\chi, \chi) \to \mathbb{R}$, which represents a measure of similarity between $\mathbf{x}_i$ and $ \mathbf{x}_j$. In this way, the data $\mathbf{x}$ can be projected into a high dimensional space (or even infinite-dimension space) to allow a linear separation between the classes. Particularly, here we used the Gaussian kernel:
\begin{equation*}
    K(\mathbf{x}_i, \mathbf{x}_j) = e^{-\frac{|\mathbf{x}_i - \mathbf{x}_j|^2}{2\sigma^2}}
\end{equation*}
The Gaussian kernel is a universal kernel \citep{park91}, meaning that with proper regulation, an \gls{svm} with it can create a classification boundary approaching any function at arbitrary precision. Therefore using a Gaussian kernel, we would be able to capture any structure present in the training data which helps in prediction.

The parameters of the model, $\lambda_i$ and $\mu_i$ are fitted with training data. To predict a new data point $\mathbf{x}'$, we calculate the sign of the distance from it to the classification boundary (\ref{dist}), using (\ref{dw}) to eliminate $\mathbf{w}$, and $b$:
\begin{equation*}
    \operatorname{sgn}\Big(\sum_{i=1}^n\lambda_iy_iK(\mathbf{x}', \mathbf{x}_i) + b\Big)
\end{equation*}
as the prediction of whether the mouse is freezing or not freezing.

We used the \texttt{SVC} module from \texttt{scikit-learn} Python package for \gls{gsvm} classification. The regularization hyperparameter $C=1$ and Gaussian kernel variance hyperparameter $2\sigma^2=\operatorname{dim}(\mathbf{x})$ were set to the default provided by the Python package \texttt{scikit-learn}. Since these sets of parameters gives satisfactory performance, they are not optimized to prevent over-fitting of the classifier to this particular data set.



\subsubsection{Statistics}

Influence of \textit{Genotype} (\gls{tg}, \gls{wt}) and \textit{Treatment} (Vehicle, \tglu{}) was evaluated with two-way \gls{anova}. Covariates were controlled with \gls{glm}. Comparisons between single levels were contrasted with T-tests, and multiple levels with F-tests. Bonferroni correction was performed for multiple comparisons. For analysis where the number of samples was large, if comparison of means was not significant, a two-sample \gls{kstest} was performed to detect any difference in the distributions of the two samples. All statistical tests were performed using \texttt{statsmodels} package in the Python programming language.

\subsubsection{Bayesian modelling}
\Gls{mcmc} sampling was performed on \texttt{pymc} package in Python. Samples were drawn using Metropolis-Hasting steps. \num{540000} samples were drawn with \num{40000} in the burn-in period. The rest were thinned \num{10} times, resulting in \num{50000} total samples for each model. Bayes factors of alternative hypothesis versus null hypothesis (BF\tsb{10}) were calculated. Interpretation of Bayes factors is reported according to \citet{kass95}, where there is positive evidence if the Bayes factor is between 3 and 20, strong evidence if it is between 20 and 150, and very strong evidence if it is greater than 150.

\section{Results}


\subsection{\tglu{} rescues memory deficits in \gls{tg} mice}

To record \gls{ca1} activity, we first expressed the \gls{geci} GCaMP6f in \gls{wt} and \gls{tg} mice. GCaMP6f was expressed by crossing the TgCRND8 mouse line with a GP5.17 mouse line. F1 offspring positive for GCaMP6f were used in the experiment ($N=17$). We also crossed TgCRND8 mice with a background C57BL/6 line, and infused an \gls{aav} in the \gls{ca1} of dorsal hippocampus to express GCaMP6f ($N=14$). We then implanted the mini-microscope above \gls{ca1} for recording of calcium activity. We found no difference between the two groups of mice, and pooled results from both groups. 

First, we examined the behavioural performance between the \gls{wt} and the \gls{tg} mice. We hypothesized that the \gls{tg} mice would have deficits in contextual fear memory, and that this deficit could be rescued by \tglu{} treatment before training.  The experimental paradigm is summarized in Figure~\ref{f.ad.paradigm}. Mini-microscope base-plates were surgically implanted unilaterally in the \gls{ca1} of dorsal hippocampus in gCaMP6f expressing \gls{wt} and TgCRND8 (\gls{tg}) mice. The mice received either vehicle (Veh) or \tglu{} (Glu) peptide one hour before training. Twenty-four hours later, the mice were put back into the training chamber for the memory test of contextual fear. Calcium transients were recorded during both the training and testing sessions of contextual fear conditioning. Figure~\ref{f.ad.trace} shows cells recorded from a single mouse and calcium transients from a sample of cells.
\begin{figure}[h]
    %LaTeX with PSTricks extensions
%%Creator: inkscape 0.92.1
%%Please note this file requires PSTricks extensions
\psset{xunit=.5pt,yunit=.5pt,runit=.5pt}
\begin{pspicture}(739.09666443,176.07633591)
{
\newrgbcolor{curcolor}{1 1 1}
\pscustom[linestyle=none,fillstyle=solid,fillcolor=curcolor]
{
\newpath
\moveto(119.78684671,112.54269195)
\curveto(119.18881499,112.51134096)(118.59577094,112.38588276)(118.03803179,112.16767337)
\lineto(125.66360725,106.58471312)
\curveto(125.77427055,108.20248924)(125.13802174,109.85648714)(123.96998272,110.98327857)
\curveto(123.20416308,111.7220554)(122.22124386,112.23460611)(121.17698387,112.44310645)
\curveto(120.91588627,112.49528578)(120.65114843,112.52858943)(120.38534569,112.54269195)
\curveto(120.18599092,112.55354004)(119.98619062,112.55354004)(119.78684671,112.54269195)
\closepath
\moveto(125.66360725,106.58471312)
\curveto(125.68349274,103.59982777)(123.74101703,102.57452025)(121.55687283,101.78994277)
\lineto(121.73727662,92.93772526)
\curveto(119.39832984,89.80712832)(116.50577313,87.95990419)(111.93842335,88.48408404)
\curveto(114.79866341,86.73479649)(114.41617738,86.77603009)(116.69034962,85.5432311)
\curveto(118.96452185,84.31043211)(120.20876687,84.75231833)(120.80344088,84.78895234)
\curveto(121.39811488,84.8256189)(119.46000746,79.12124679)(119.46000746,79.12124679)
\lineto(113.76788861,75.41339896)
\lineto(117.72392795,76.78847937)
\lineto(122.63502509,76.31175558)
\lineto(125.26673412,76.97917149)
\curveto(125.4957324,76.93892506)(125.72651277,76.90755238)(125.95861885,76.89865694)
\curveto(126.71096416,76.86969253)(127.46441784,76.99976117)(128.17645972,77.24401683)
\curveto(129.60053261,77.7325824)(130.83434589,78.65891191)(131.93300302,79.68695315)
\curveto(133.14266032,80.81885403)(134.22832125,82.10528599)(134.98280724,83.57912961)
\curveto(135.73112112,85.04088846)(136.1404254,86.67323356)(136.1691996,88.31458259)
\curveto(137.12685447,88.21543102)(138.09815752,88.23408974)(139.05133544,88.3696909)
\curveto(140.03992691,88.5102388)(141.00983907,88.77601708)(141.93135233,89.15998533)
\curveto(141.63662326,91.22437744)(141.19531747,93.10633717)(141.50688944,95.92732953)
\lineto(141.93135233,89.15998533)
\curveto(142.15729716,88.51608592)(142.44079038,87.89303654)(142.77817003,87.29971094)
\curveto(143.87759868,85.36623362)(145.55350013,83.76811424)(147.5237286,82.73164404)
\curveto(149.76612751,81.55198985)(152.38748088,81.11792511)(154.89461763,81.47944865)
\lineto(153.84830389,80.31836641)
\curveto(153.84830389,80.31836641)(151.58681273,79.72067987)(150.5544297,79.30135768)
\curveto(149.52204668,78.88203549)(148.63583832,78.04493154)(148.63583832,78.04493154)
\lineto(145.94046311,77.38811205)
\lineto(149.80099775,76.29271067)
\lineto(151.69201024,77.56608587)
\curveto(151.64376348,77.66198301)(156.96212201,77.98246822)(160.13892172,77.62748607)
\curveto(160.8728485,78.36044833)(162.04130045,81.23293659)(163.32668562,83.75708173)
\curveto(164.25045908,85.57112155)(165.4163031,87.10293738)(165.67823741,88.01157365)
\curveto(166.11023071,89.03604587)(163.83004936,88.73390478)(163.95914206,89.83794775)
\curveto(164.20346195,91.92755083)(163.88112139,94.06219519)(163.16750388,96.04170098)
\curveto(162.45394071,98.02121761)(161.35782631,99.85080274)(160.06252014,101.51024639)
\curveto(156.82840023,105.65353451)(152.26910243,108.78825372)(147.18415394,110.13786498)
\curveto(143.82741769,111.02878632)(140.28532258,111.14474159)(136.84834892,110.64425397)
\curveto(132.89580856,110.06868669)(129.06325487,108.67815643)(125.66360725,106.58469143)
\closepath
\moveto(111.93842335,88.48408404)
\lineto(113.64903205,106.27537975)
\curveto(113.66891754,108.0895172)(115.85732137,111.19726805)(118.03803179,112.16767337)
\curveto(117.14650977,112.50235874)(116.15633179,112.56924808)(115.22805964,112.35623493)
\curveto(114.18533181,112.11696938)(113.22826373,111.52302544)(112.54965773,110.6972469)
\curveto(111.87105174,109.87146836)(111.47520005,108.81769545)(111.44391571,107.75003698)
\curveto(109.62844636,108.08044819)(107.69895599,107.75055768)(106.09773922,106.83472912)
\curveto(104.77199629,106.07645827)(103.70071159,104.95653368)(102.61922336,103.87904684)
\curveto(100.46675468,101.73454156)(98.20267732,99.69945015)(95.8404656,97.78759308)
\lineto(96.51749597,96.4358122)
\curveto(97.55371483,95.32315584)(98.68971974,94.30099427)(99.90687485,93.38902763)
\curveto(103.40972056,90.76447254)(107.59668153,89.05826276)(111.93842335,88.48408404)
\closepath
}
}
{
\newrgbcolor{curcolor}{0.52156866 0.37254903 0.23137255}
\pscustom[linestyle=none,fillstyle=solid,fillcolor=curcolor]
{
\newpath
\moveto(117.90260398,82.11230471)
\lineto(115.98549043,80.6170795)
\lineto(112.21116971,80.91612888)
\lineto(108.67648548,79.66013666)
\lineto(115.20666372,78.76299937)
\lineto(119.46026826,79.12185428)
\lineto(121.55711189,78.22471699)
\lineto(123.29449537,79.54051474)
\lineto(123.53414272,84.08600676)
\curveto(123.53414272,84.08600676)(120.77828519,85.88027051)(120.89809799,85.58123197)
\curveto(121.01792166,85.28218259)(117.90260398,82.11230471)(117.90260398,82.11230471)
\closepath
}
}
{
\newrgbcolor{curcolor}{0 0 0}
\pscustom[linewidth=1.33333336,linecolor=curcolor]
{
\newpath
\moveto(117.90260398,82.11230471)
\lineto(115.98549043,80.6170795)
\lineto(112.21116971,80.91612888)
\lineto(108.67648548,79.66013666)
\lineto(115.20666372,78.76299937)
\lineto(119.46026826,79.12185428)
\lineto(121.55711189,78.22471699)
\lineto(123.29449537,79.54051474)
\lineto(123.53414272,84.08600676)
\curveto(123.53414272,84.08600676)(120.77828519,85.88027051)(120.89809799,85.58123197)
\curveto(121.01792166,85.28218259)(117.90260398,82.11230471)(117.90260398,82.11230471)
\closepath
}
}
{
\newrgbcolor{curcolor}{0.90588236 0.73333335 0.75294119}
\pscustom[linestyle=none,fillstyle=solid,fillcolor=curcolor]
{
\newpath
\moveto(163.11245476,88.79288586)
\curveto(163.11245476,88.79288586)(164.04443235,88.6237207)(164.7222451,88.53913812)
\curveto(165.40004698,88.45452299)(171.07664863,87.86248831)(173.6184057,85.07125227)
\curveto(176.16016276,82.28001623)(176.24489887,79.99628652)(176.16017363,77.88171113)
\curveto(176.0754158,75.76714442)(174.04203187,72.55299873)(171.92390099,71.453423)
\curveto(169.8057701,70.35384726)(164.46805852,67.56261665)(157.85947928,67.73178181)
\curveto(151.25090005,67.90094806)(138.00085987,69.64167972)(130.3237234,70.10009952)
\curveto(122.64658692,70.55852041)(112.68626998,70.1232602)(110.24379911,65.7017966)
\curveto(109.5109155,64.37509759)(110.49797481,60.71141452)(112.22501349,59.80811008)
\curveto(110.14170938,68.01260856)(117.5312886,69.20971191)(135.90022387,67.29620701)
\curveto(148.77212457,66.50058725)(164.21643642,61.8802781)(175.53015126,70.71608896)
\curveto(181.6846567,75.82069712)(179.27252479,87.9825767)(171.79152663,89.88842393)
\curveto(168.93940376,90.56759049)(165.05725587,91.86987152)(161.82457032,91.37909294)
\closepath
}
}
{
\newrgbcolor{curcolor}{0 0 0}
\pscustom[linewidth=1.33543295,linecolor=curcolor]
{
\newpath
\moveto(163.11245476,88.79288586)
\curveto(163.11245476,88.79288586)(164.04443235,88.6237207)(164.7222451,88.53913812)
\curveto(165.40004698,88.45452299)(171.07664863,87.86248831)(173.6184057,85.07125227)
\curveto(176.16016276,82.28001623)(176.24489887,79.99628652)(176.16017363,77.88171113)
\curveto(176.0754158,75.76714442)(174.04203187,72.55299873)(171.92390099,71.453423)
\curveto(169.8057701,70.35384726)(164.46805852,67.56261665)(157.85947928,67.73178181)
\curveto(151.25090005,67.90094806)(138.00085987,69.64167972)(130.3237234,70.10009952)
\curveto(122.64658692,70.55852041)(112.68626998,70.1232602)(110.24379911,65.7017966)
\curveto(109.5109155,64.37509759)(110.49797481,60.71141452)(112.22501349,59.80811008)
\curveto(110.14170938,68.01260856)(117.5312886,69.20971191)(135.90022387,67.29620701)
\curveto(148.77212457,66.50058725)(164.21643642,61.8802781)(175.53015126,70.71608896)
\curveto(181.6846567,75.82069712)(179.27252479,87.9825767)(171.79152663,89.88842393)
\curveto(168.93940376,90.56759049)(165.05725587,91.86987152)(161.82457032,91.37909294)
\closepath
}
}
{
\newrgbcolor{curcolor}{0.52156866 0.37254903 0.23137255}
\pscustom[linestyle=none,fillstyle=solid,fillcolor=curcolor]
{
\newpath
\moveto(145.6408985,83.60752992)
\lineto(144.68234172,82.41135409)
\lineto(139.93198021,82.67910672)
\lineto(138.01485578,80.52597721)
\lineto(144.2629643,80.97593441)
\lineto(148.45666243,80.6170795)
\lineto(149.05575906,81.93287725)
\curveto(150.00026559,86.64968235)(147.61310465,84.71433916)(145.6408985,83.60752992)
\closepath
}
}
{
\newrgbcolor{curcolor}{0 0 0}
\pscustom[linewidth=1.33333336,linecolor=curcolor]
{
\newpath
\moveto(145.6408985,83.60752992)
\lineto(144.68234172,82.41135409)
\lineto(139.93198021,82.67910672)
\lineto(138.01485578,80.52597721)
\lineto(144.2629643,80.97593441)
\lineto(148.45666243,80.6170795)
\lineto(149.05575906,81.93287725)
\curveto(150.00026559,86.64968235)(147.61310465,84.71433916)(145.6408985,83.60752992)
\closepath
}
}
{
\newrgbcolor{curcolor}{0.94901961 0.94901961 0.94901961}
\pscustom[linestyle=none,fillstyle=solid,fillcolor=curcolor]
{
\newpath
\moveto(133.08152601,83.38415684)
\curveto(136.64277857,83.28467982)(140.01468429,82.5506653)(145.6408985,83.60752992)
\curveto(146.28888301,84.61890848)(145.92744518,85.70269807)(146.22865061,86.90549124)
\curveto(146.7328402,88.91885392)(148.07894674,91.14061931)(146.01487614,93.20121459)
\curveto(145.49098579,93.72421201)(144.11445336,93.87718097)(143.44958248,94.37499996)
\curveto(142.45994782,95.11596811)(140.94948715,96.33008668)(139.70853465,96.08230539)
\curveto(138.68345384,95.87764527)(135.80553416,95.32081265)(135.11237806,94.80182903)
\curveto(134.10086937,94.04446941)(134.21584663,93.22518888)(133.72284943,92.24085461)
\curveto(131.92764589,88.65648194)(129.58085358,86.19101413)(133.08152601,83.38415684)
\closepath
}
}
{
\newrgbcolor{curcolor}{0 0 0}
\pscustom[linewidth=1.33333336,linecolor=curcolor]
{
\newpath
\moveto(133.08152601,83.38415684)
\curveto(136.64277857,83.28467982)(140.01468429,82.5506653)(145.6408985,83.60752992)
\curveto(146.28888301,84.61890848)(145.92744518,85.70269807)(146.22865061,86.90549124)
\curveto(146.7328402,88.91885392)(148.07894674,91.14061931)(146.01487614,93.20121459)
\curveto(145.49098579,93.72421201)(144.11445336,93.87718097)(143.44958248,94.37499996)
\curveto(142.45994782,95.11596811)(140.94948715,96.33008668)(139.70853465,96.08230539)
\curveto(138.68345384,95.87764527)(135.80553416,95.32081265)(135.11237806,94.80182903)
\curveto(134.10086937,94.04446941)(134.21584663,93.22518888)(133.72284943,92.24085461)
\curveto(131.92764589,88.65648194)(129.58085358,86.19101413)(133.08152601,83.38415684)
\closepath
}
}
{
\newrgbcolor{curcolor}{0.60000002 0.39607844 0.2}
\pscustom[linestyle=none,fillstyle=solid,fillcolor=curcolor]
{
\newpath
\moveto(119.78684671,112.54269195)
\curveto(119.18881499,112.51134096)(118.59577094,112.38588276)(118.03803179,112.16767337)
\curveto(117.20904583,112.25369875)(118.03803179,112.16767337)(115.22805964,112.35623493)
\curveto(114.18533181,112.11696938)(113.22826373,111.52302544)(112.54965773,110.6972469)
\curveto(111.87105174,109.87146836)(111.47520005,108.81769545)(111.44391571,107.75003698)
\curveto(109.62844636,108.08044819)(107.69895599,107.75055768)(106.09773922,106.83472912)
\curveto(104.77199629,106.07645827)(103.70071159,104.95653368)(102.61922336,103.87904684)
\curveto(101.92179677,103.18419393)(101.20879865,102.50584098)(100.48838966,101.83444165)
\lineto(100.47567598,101.82174938)
\curveto(100.44568475,101.7937613)(100.41449821,101.76707499)(100.38439831,101.73908691)
\curveto(100.30148777,101.6629333)(100.21895754,101.58646509)(100.13607959,101.51025724)
\curveto(99.43850088,100.86712804)(98.73168574,100.23432623)(98.01374341,99.61395633)
\curveto(97.99635719,99.59887748)(97.98027493,99.58238838)(97.96278005,99.56730953)
\curveto(97.95952013,99.5640551)(97.95517357,99.56188548)(97.95191366,99.55863105)
\lineto(97.95191366,99.51621501)
\curveto(98.40586793,99.33940194)(99.43779456,98.88433528)(99.69011212,98.38054984)
\curveto(99.96708551,97.82753575)(98.5034481,96.02457184)(100.22918281,95.45029549)
\curveto(101.95491753,94.87601914)(105.64272005,97.60565971)(107.47697739,96.58596065)
\curveto(110.32421039,95.42693954)(108.59270562,94.06666461)(109.03478293,93.1175324)
\curveto(109.69876276,91.79180865)(110.38676819,90.32815138)(111.91055106,89.94786063)
\curveto(112.93281748,89.40374198)(114.17964869,89.73771137)(115.36784487,90.04321537)
\lineto(121.73699409,92.93744321)
\curveto(121.48036258,92.59397089)(121.21351667,92.27040482)(120.94111803,91.96069176)
\curveto(120.79911605,91.79792697)(120.65058338,91.64788699)(120.50390886,91.4945709)
\curveto(120.3759028,91.36211568)(120.24737515,91.2297147)(120.11552238,91.10471213)
\curveto(119.92355675,90.92076101)(119.72639699,90.74661657)(119.52551006,90.57926304)
\curveto(119.44455546,90.51243879)(119.36178617,90.44848928)(119.27932115,90.38433366)
\curveto(119.04266206,90.19884211)(118.80164555,90.02053201)(118.55136001,89.85887373)
\curveto(118.54918673,89.85778892)(118.54701345,89.8556193)(118.54484018,89.85453449)
\curveto(119.54836207,88.00429458)(119.04211874,86.05392678)(120.59714248,85.76107166)
\curveto(120.98839768,82.57211198)(121.71459847,82.61035151)(119.3895498,80.81370118)
\lineto(119.68700633,79.95187443)
\lineto(114.14578902,76.46977867)
\lineto(117.72355849,76.78804544)
\lineto(122.63465563,76.31132165)
\lineto(125.26636466,76.97873756)
\curveto(125.49536295,76.93852368)(125.72614332,76.90716184)(125.95824939,76.89822302)
\curveto(126.7105947,76.86923691)(127.46404839,76.9992947)(128.17609026,77.24358291)
\curveto(128.53210577,77.36572159)(128.8765703,77.51479608)(129.20966859,77.68640207)
\curveto(129.54223443,77.85773685)(129.86391214,78.05252521)(130.17533199,78.26270701)
\curveto(130.48722995,78.47329019)(130.78915257,78.70166425)(131.08157797,78.94071283)
\curveto(131.3739925,79.17976141)(131.65797471,79.4295062)(131.93263356,79.68651922)
\curveto(132.23543636,79.96986057)(132.52939392,80.26258551)(132.81340873,80.56581141)
\curveto(133.09708668,80.86861423)(133.36991998,81.18075726)(133.62838791,81.50442096)
\curveto(133.88723616,81.82856199)(134.13140392,82.16262901)(134.35846799,82.50871572)
\curveto(134.58524953,82.85427088)(134.793819,83.21024021)(134.98243778,83.57869569)
\curveto(135.35660016,84.30957511)(135.64677621,85.08370588)(135.84623965,85.87968471)
\curveto(136.04569222,86.67565269)(136.15444304,87.49346873)(136.16883014,88.31414867)
\curveto(136.64765758,88.26457288)(137.13034258,88.24352758)(137.61202787,88.25274846)
\curveto(138.0937023,88.26142693)(138.57437702,88.3014564)(139.05096598,88.36927867)
\curveto(140.03955746,88.50982657)(141.00946961,88.77560485)(141.93098287,89.1595731)
\curveto(142.1569277,88.51567369)(142.44042092,87.89262431)(142.77780057,87.29929871)
\curveto(143.32752033,86.33256005)(144.02105674,85.44915644)(144.82373517,84.67837773)
\curveto(145.22508525,84.29299922)(145.65427502,83.93631392)(146.10563222,83.61052399)
\curveto(146.55698943,83.28473406)(147.03079659,82.99034936)(147.52335914,82.73123181)
\curveto(148.92486253,81.99395201)(150.47315998,81.5474662)(152.05030769,81.40488971)
\curveto(152.36573723,81.37635923)(152.68232948,81.36030405)(152.99899779,81.35618177)
\curveto(152.9997041,81.35617092)(153.00008443,81.35619262)(153.00117107,81.35618177)
\curveto(153.00877754,81.35608414)(153.01671,81.35626856)(153.0245338,81.35618177)
\curveto(153.65011182,81.34967292)(154.27523345,81.38981086)(154.89431337,81.47906897)
\lineto(153.84799963,80.31798672)
\curveto(153.84799963,80.31798672)(153.2817521,80.16810947)(152.58732464,79.96626985)
\curveto(151.89290804,79.76444108)(151.07032239,79.51063909)(150.55412545,79.300978)
\curveto(149.52174242,78.88165581)(148.63553406,78.04455186)(148.63553406,78.04455186)
\lineto(145.94015885,77.38773237)
\lineto(150.01505475,75.65246381)
\lineto(151.69170598,77.56570618)
\curveto(151.69170598,77.56570618)(157.42273738,77.06609726)(160.13861746,77.62710639)
\curveto(160.26413512,77.7151929)(160.38426305,77.82666791)(160.50154399,77.95975231)
\curveto(160.736095,78.22594282)(160.95537873,78.57515379)(161.15734144,78.98100264)
\curveto(163.27788466,82.44836778)(164.00094506,84.65309083)(165.67761802,88.01098786)
\curveto(166.10961132,89.03546007)(163.8297451,88.7335251)(163.9588378,89.83756807)
\curveto(164.08099775,90.88236961)(164.06195983,91.9374009)(163.92276139,92.97970738)
\curveto(163.78350861,94.02200301)(163.52400838,95.05157383)(163.16719962,96.04132129)
\curveto(162.45363645,98.02083792)(161.35752205,99.85042306)(160.06221588,101.50986671)
\curveto(157.6370117,104.6168365)(154.46574298,107.1574165)(150.8873324,108.80265999)
\curveto(150.88673475,108.80294204)(150.88624576,108.80238879)(150.88515912,108.80265999)
\curveto(149.69211653,109.35107449)(148.45503247,109.80008791)(147.18379535,110.1374853)
\curveto(146.34527957,110.36004478)(145.49410445,110.53463399)(144.63698541,110.66294523)
\curveto(144.63628996,110.66304286)(144.63589877,110.66283675)(144.63481213,110.66294523)
\curveto(142.06135781,111.0479549)(139.42566081,111.01924)(136.84793599,110.64385259)
\curveto(135.36573879,110.42801893)(133.89937392,110.09775958)(132.4674229,109.65862877)
\curveto(131.51278889,109.36587129)(130.57328089,109.02500251)(129.65320199,108.63737844)
\curveto(128.73311223,108.24975438)(127.83357081,107.81545098)(126.95782679,107.33645343)
\curveto(126.51995477,107.09696008)(126.08815707,106.8459786)(125.66319433,106.58429005)
\curveto(125.66862752,105.83807142)(125.54979269,105.21400232)(125.33635508,104.68587376)
\curveto(125.22964713,104.42180948)(125.099207,104.18298871)(124.9479686,103.96337991)
\curveto(124.34303672,103.08404434)(123.40985296,102.53491387)(122.36083349,102.09886392)
\curveto(122.09872532,101.98990567)(121.82947793,101.88759731)(121.55645991,101.7895197)
\curveto(124.42354579,102.49038245)(126.17593576,103.85894532)(125.66319433,106.58429005)
\curveto(125.6771033,106.78721448)(125.67873326,106.98943378)(125.66971416,107.1923799)
\curveto(125.66102105,107.39463175)(125.64146155,107.59614593)(125.61027501,107.79622816)
\curveto(125.54790194,108.19639261)(125.44179165,108.58962514)(125.29617117,108.96790899)
\curveto(125.15035509,109.3467244)(124.96483323,109.70927852)(124.74223525,110.04847669)
\curveto(124.63114816,110.21788052)(124.51041171,110.37957134)(124.38143854,110.53579473)
\curveto(124.25247623,110.69201812)(124.11571186,110.84200385)(123.96971106,110.98285549)
\curveto(123.20389142,111.72163233)(122.2209722,112.23418303)(121.17671221,112.44268338)
\curveto(120.91561461,112.49486271)(120.65087677,112.52816635)(120.38507403,112.54226887)
\curveto(120.18571926,112.55311697)(119.98591896,112.55311697)(119.78657505,112.54226887)
\closepath
}
}
{
\newrgbcolor{curcolor}{0 0 0}
\pscustom[linestyle=none,fillstyle=solid,fillcolor=curcolor]
{
\newpath
\moveto(108.42424871,99.66160165)
\curveto(108.08550412,98.48714966)(107.01186215,97.76475675)(106.02620224,98.04809192)
\curveto(105.04054233,98.33142709)(104.51611384,99.51319709)(104.85485843,100.68764908)
\curveto(105.19360302,101.86210108)(106.267245,102.58449399)(107.25290491,102.30115882)
\curveto(108.23856482,102.01782364)(108.7629933,100.83605365)(108.42424871,99.66160165)
\closepath
}
}
{
\newrgbcolor{curcolor}{0.96862745 0.71764708 0.74117649}
\pscustom[linestyle=none,fillstyle=solid,fillcolor=curcolor]
{
\newpath
\moveto(115.62601941,105.67705318)
\curveto(115.48749468,109.30389613)(120.33433886,112.19788531)(122.09629125,110.34217797)
\curveto(125.66382458,106.58479991)(123.74949281,103.68344487)(119.40035099,102.80622467)
\curveto(118.76237442,103.58160297)(116.11227945,104.45937643)(115.62601941,105.67705318)
\closepath
}
}
{
\newrgbcolor{curcolor}{0.2 0.2 0.2}
\pscustom[linestyle=none,fillstyle=solid,fillcolor=curcolor]
{
\newpath
\moveto(97.59351841,99.24530558)
\curveto(97.01544825,98.75215127)(96.43102124,98.26554921)(95.8404656,97.78759308)
\lineto(96.51749597,96.4358122)
\curveto(96.98692398,95.93175555)(97.48386568,95.45166235)(97.99252135,94.9865829)
\curveto(98.58732052,95.34646117)(98.97072148,96.14933402)(98.97092016,97.03542969)
\curveto(98.96998252,98.13121028)(98.38755417,99.06564672)(97.59351841,99.24530558)
\closepath
}
}
{
\newrgbcolor{curcolor}{0 0 0}
\pscustom[linewidth=1.33333336,linecolor=curcolor]
{
\newpath
\moveto(141.50710677,95.92819738)
\curveto(141.19553479,93.10720502)(141.63637333,91.22514766)(141.9311024,89.16075555)
\moveto(121.73727662,92.93772526)
\curveto(119.39832984,89.80712832)(116.50577313,87.95990419)(111.93842335,88.48408404)
\curveto(114.79866341,86.73479649)(114.41597092,86.7765725)(116.69014316,85.54377351)
\curveto(118.96431539,84.31097451)(120.73800348,85.4311703)(121.33267749,85.46783686)
\curveto(121.9273515,85.50450341)(119.68700633,79.95187443)(119.68700633,79.95187443)
\lineto(116.53849183,78.39945807)
\lineto(114.14578902,76.46977867)
\lineto(117.72392795,76.78851625)
\lineto(122.63502509,76.31178595)
\lineto(125.26673412,76.97920403)
\curveto(125.4957324,76.93895761)(125.72651277,76.90760662)(125.95861885,76.89871118)
\curveto(126.71096416,76.86974677)(127.46441784,76.99981541)(128.17645972,77.24407107)
\curveto(129.60053261,77.73263664)(130.83434589,78.65896615)(131.93300302,79.68700739)
\curveto(133.14266032,80.81890827)(134.22832125,82.10534023)(134.98280724,83.57918385)
\curveto(135.73112112,85.0409427)(136.1404254,86.6732878)(136.1691996,88.31463683)
\curveto(137.12685447,88.21548526)(138.09815752,88.23414398)(139.05133544,88.36974514)
\curveto(140.03992691,88.51029304)(141.00983907,88.77607132)(141.93135233,89.16003957)
\curveto(142.15729716,88.51614016)(142.44079038,87.89309078)(142.77817003,87.29976518)
\curveto(143.87759868,85.36628786)(145.55350013,83.76816848)(147.5237286,82.73169828)
\curveto(149.76612751,81.55204409)(152.38748088,81.11797935)(154.89461763,81.4795029)
\lineto(153.84830389,80.31842065)
\curveto(153.84830389,80.31842065)(151.58681273,79.72073411)(150.5544297,79.30141192)
\curveto(149.52204668,78.88208973)(148.63583832,78.04498578)(148.63583832,78.04498578)
\lineto(145.94015885,77.38776491)
\lineto(149.80158454,76.29313808)
\lineto(151.69201024,77.56614011)
\curveto(151.64376348,77.66203725)(156.96212201,77.98247907)(160.13892172,77.62754031)
\curveto(160.8728485,78.36050257)(162.04162644,81.23354408)(163.32701161,83.75768922)
\curveto(164.25078507,85.57172904)(165.41568372,87.10238413)(165.67761802,88.0110204)
\curveto(166.10961132,89.03549261)(163.83004936,88.73395902)(163.95914206,89.838002)
\curveto(164.20346195,91.92760507)(163.88112139,94.06224944)(163.16750388,96.04175522)
\curveto(162.45394071,98.02127185)(161.35782631,99.85085698)(160.06252014,101.51030063)
\curveto(156.82840023,105.65358875)(152.26910243,108.78830796)(147.18415394,110.13791922)
\curveto(143.82741769,111.02884056)(140.28532258,111.14479583)(136.84834892,110.64430821)
\curveto(132.89580856,110.06874093)(129.06325487,108.67821067)(125.66360725,106.58474567)
\curveto(125.68349274,103.59986032)(123.74101703,102.5745528)(121.55687283,101.78997532)
\moveto(125.66360725,106.58474567)
\curveto(125.77427055,108.20252179)(125.13802174,109.85651969)(123.96998272,110.98331111)
\curveto(123.20416308,111.72208795)(122.22124386,112.23463865)(121.17698387,112.443139)
\curveto(120.91589714,112.49531833)(120.65114843,112.52862197)(120.38534569,112.54272449)
\curveto(120.18599092,112.55357259)(119.98619062,112.55357259)(119.78684671,112.54272449)
\lineto(119.78684671,112.54274619)
\curveto(119.18881499,112.5113952)(118.59577094,112.385937)(118.03803179,112.16772761)
\moveto(113.64903205,106.27543399)
\curveto(113.66891754,108.08957144)(115.85732137,111.19732229)(118.03803179,112.16772761)
\curveto(117.14650977,112.50241298)(116.15633179,112.56930232)(115.22805964,112.35628917)
\curveto(114.18533181,112.11702362)(113.22826373,111.52307968)(112.54965773,110.69730114)
\curveto(111.87105174,109.8715226)(111.47520005,108.81774969)(111.44391571,107.75009122)
\curveto(109.62844636,108.08050244)(107.69895599,107.75061193)(106.09773922,106.83478336)
\curveto(104.77199629,106.07651251)(103.70071159,104.95658792)(102.61922336,103.87910108)
\curveto(100.46675468,101.7345958)(98.20267732,99.69950439)(95.8404656,97.78764732)
\lineto(96.51749597,96.43586644)
\curveto(97.55371483,95.32321008)(98.44192261,93.80053919)(99.90687485,93.38908187)
\curveto(103.19679367,92.46505215)(108.7724357,90.23209152)(111.93842335,88.48413828)
}
}
{
\newrgbcolor{curcolor}{0 0 0}
\pscustom[linewidth=1.33333336,linecolor=curcolor]
{
\newpath
\moveto(154.89431337,81.47906897)
\curveto(159.67971898,81.89468111)(161.93347327,85.24535331)(162.41769044,87.95710538)
}
}
{
\newrgbcolor{curcolor}{0 0 0}
\pscustom[linestyle=none,fillstyle=solid,fillcolor=curcolor,opacity=0]
{
\newpath
\moveto(-148.44390667,-516.61285076)
\curveto(-147.26241333,-516.64685076)(-146.10185333,-516.72271743)(-144.80205333,-516.52005076)
}
}
{
\newrgbcolor{curcolor}{0 0 0}
\pscustom[linewidth=1.33333333,linecolor=curcolor]
{
\newpath
\moveto(-148.44390667,-516.61285076)
\curveto(-147.26241333,-516.64685076)(-146.10185333,-516.72271743)(-144.80205333,-516.52005076)
}
}
{
\newrgbcolor{curcolor}{0.56862748 0.57254905 0.78039217}
\pscustom[linestyle=none,fillstyle=solid,fillcolor=curcolor]
{
\newpath
\moveto(349.430056,139.94763322)
\curveto(349.17312555,139.8793558)(348.96248561,139.74727483)(348.81420749,139.56146774)
\curveto(348.27824796,138.88635449)(348.67221175,137.68018145)(349.69416369,136.86736369)
\curveto(350.71617875,136.05461942)(351.97911003,135.94307274)(352.5150386,136.61821378)
\curveto(352.66236245,136.80478043)(352.74330442,137.04000963)(352.75168883,137.30595391)
\curveto(352.13388656,137.58964274)(351.51305431,137.97502898)(350.93672057,138.43260895)
\curveto(350.36114168,138.89113979)(349.8457685,139.40937251)(349.430056,139.94763322)
\closepath
}
}
{
\newrgbcolor{curcolor}{0 0 0}
\pscustom[linewidth=1.33543299,linecolor=curcolor]
{
\newpath
\moveto(349.430056,139.94763322)
\curveto(349.17312555,139.8793558)(348.96248561,139.74727483)(348.81420749,139.56146774)
\curveto(348.27824796,138.88635449)(348.67221175,137.68018145)(349.69416369,136.86736369)
\curveto(350.71617875,136.05461942)(351.97911003,135.94307274)(352.5150386,136.61821378)
\curveto(352.66236245,136.80478043)(352.74330442,137.04000963)(352.75168883,137.30595391)
\curveto(352.13388656,137.58964274)(351.51305431,137.97502898)(350.93672057,138.43260895)
\curveto(350.36114168,138.89113979)(349.8457685,139.40937251)(349.430056,139.94763322)
\closepath
}
}
{
\newrgbcolor{curcolor}{0.00784314 0.00784314 0.00784314}
\pscustom[linewidth=1.33599994,linecolor=curcolor]
{
\newpath
\moveto(368.54958052,168.52970101)
\lineto(348.90573931,143.78441197)
\lineto(349.06887637,143.65466982)
\curveto(348.90616951,143.56888425)(348.76972508,143.45775552)(348.66188313,143.32318953)
\curveto(347.83422272,142.27974653)(348.85255716,140.09047068)(350.93672057,138.43260895)
\curveto(353.02146354,136.77429367)(355.38327742,136.27554781)(356.21198586,137.31862755)
\curveto(356.31719269,137.45541904)(356.39303255,137.61474786)(356.43820316,137.79387811)
\lineto(356.60429007,137.66178996)
\lineto(376.24813128,162.407079)
\closepath
}
}
{
\newrgbcolor{curcolor}{0 0 0}
\pscustom[linewidth=1.33333331,linecolor=curcolor]
{
\newpath
\moveto(366.66702142,166.45252263)
\lineto(366.38003589,166.41918908)
\curveto(366.38003589,166.41918908)(357.83293967,166.15422168)(360.55219179,170.40011716)
\curveto(363.27145268,174.64600699)(368.22569003,168.74413751)(368.22569003,168.74413751)
\lineto(368.22569003,168.74413751)
\lineto(369.99621876,168.513116)
\curveto(373.16925289,168.73611796)(368.96608963,173.77909117)(368.96608963,173.77909117)
\lineto(370.26010704,175.40916109)
\lineto(382.55696933,165.62952307)
\lineto(381.26295192,163.99945314)
\curveto(381.26295192,163.99945314)(375.40507038,166.95798259)(375.90510063,163.81380942)
\lineto(376.53005645,162.13971309)
\lineto(376.53005645,162.13971309)
\curveto(376.53005645,162.13971309)(383.39358285,158.6429694)(379.87583253,155.03211454)
\curveto(376.35807474,151.42126695)(374.68441854,159.81475175)(374.68441854,159.81475175)
\lineto(374.65200505,160.10210187)
}
}
{
\newrgbcolor{curcolor}{0 0 0}
\pscustom[linewidth=1.33333333,linecolor=curcolor]
{
\newpath
\moveto(349.7749012,136.96903757)
\lineto(341.7173066,126.81890934)
}
}
{
\newrgbcolor{curcolor}{0 0 0}
\pscustom[linewidth=1.33543299,linecolor=curcolor]
{
\newpath
\moveto(346.66192557,148.10676239)
\lineto(361.31804019,136.45082183)
\lineto(359.73004709,134.45043159)
\lineto(345.07393247,146.10637215)
\closepath
}
}
{
\newrgbcolor{curcolor}{0 0 0}
\pscustom[linestyle=none,fillstyle=solid,fillcolor=curcolor,opacity=0]
{
\newpath
\moveto(365.26156392,152.85296563)
\curveto(365.91313278,152.30344342)(366.53411246,151.73966667)(367.35858287,151.27073539)
}
}
{
\newrgbcolor{curcolor}{0 0 0}
\pscustom[linewidth=0.96157707,linecolor=curcolor]
{
\newpath
\moveto(365.26156392,152.85296563)
\curveto(365.91313278,152.30344342)(366.53411246,151.73966667)(367.35858287,151.27073539)
}
}
{
\newrgbcolor{curcolor}{0 0 0}
\pscustom[linestyle=none,fillstyle=solid,fillcolor=curcolor,opacity=0]
{
\newpath
\moveto(363.76675879,150.969964)
\curveto(364.41832765,150.42044178)(365.03930732,149.85666504)(365.86377773,149.38773376)
}
}
{
\newrgbcolor{curcolor}{0 0 0}
\pscustom[linewidth=0.96157707,linecolor=curcolor]
{
\newpath
\moveto(363.76675879,150.969964)
\curveto(364.41832765,150.42044178)(365.03930732,149.85666504)(365.86377773,149.38773376)
}
}
{
\newrgbcolor{curcolor}{0 0 0}
\pscustom[linestyle=none,fillstyle=solid,fillcolor=curcolor,opacity=0]
{
\newpath
\moveto(361.32650201,149.83887571)
\curveto(362.24233554,149.07918502)(363.12289772,148.3089635)(364.23809445,147.60881892)
}
}
{
\newrgbcolor{curcolor}{0 0 0}
\pscustom[linewidth=1.13625225,linecolor=curcolor]
{
\newpath
\moveto(361.32650201,149.83887571)
\curveto(362.24233554,149.07918502)(363.12289772,148.3089635)(364.23809445,147.60881892)
}
}
{
\newrgbcolor{curcolor}{0 0 0}
\pscustom[linestyle=none,fillstyle=solid,fillcolor=curcolor,opacity=0]
{
\newpath
\moveto(360.77714852,147.20396073)
\curveto(361.42871738,146.65443852)(362.04969706,146.09066177)(362.87416747,145.62173049)
}
}
{
\newrgbcolor{curcolor}{0 0 0}
\pscustom[linewidth=0.96157707,linecolor=curcolor]
{
\newpath
\moveto(360.77714852,147.20396073)
\curveto(361.42871738,146.65443852)(362.04969706,146.09066177)(362.87416747,145.62173049)
}
}
{
\newrgbcolor{curcolor}{0 0 0}
\pscustom[linestyle=none,fillstyle=solid,fillcolor=curcolor,opacity=0]
{
\newpath
\moveto(359.28234339,145.3209591)
\curveto(359.93391225,144.77143688)(360.55489192,144.20766014)(361.37936233,143.73872886)
}
}
{
\newrgbcolor{curcolor}{0 0 0}
\pscustom[linewidth=0.96157707,linecolor=curcolor]
{
\newpath
\moveto(359.28234339,145.3209591)
\curveto(359.93391225,144.77143688)(360.55489192,144.20766014)(361.37936233,143.73872886)
}
}
{
\newrgbcolor{curcolor}{0 0 0}
\pscustom[linestyle=none,fillstyle=solid,fillcolor=curcolor,opacity=0]
{
\newpath
\moveto(357.78753826,143.43795747)
\curveto(358.43910712,142.88843525)(359.06008679,142.3246585)(359.8845572,141.85572722)
}
}
{
\newrgbcolor{curcolor}{0 0 0}
\pscustom[linewidth=0.96157707,linecolor=curcolor]
{
\newpath
\moveto(357.78753826,143.43795747)
\curveto(358.43910712,142.88843525)(359.06008679,142.3246585)(359.8845572,141.85572722)
}
}
{
\newrgbcolor{curcolor}{0 0 0}
\pscustom[linewidth=2.66666667,linecolor=curcolor]
{
\newpath
\moveto(352.09758813,93.98949591)
\lineto(352.09758813,47.29050924)
\lineto(379.6550548,59.95969591)
\lineto(379.6550548,106.65868257)
\closepath
}
}
{
\newrgbcolor{curcolor}{0 0 0}
\pscustom[linewidth=2.66666667,linecolor=curcolor]
{
\newpath
\moveto(379.6550548,106.65868257)
\lineto(467.2176948,106.65868257)
\lineto(467.2176948,59.95969591)
\lineto(379.6550548,59.95969591)
\closepath
}
}
{
\newrgbcolor{curcolor}{0 0 0}
\pscustom[linewidth=2.66666667,linecolor=curcolor]
{
\newpath
\moveto(352.09758813,47.29050924)
\lineto(439.66022813,47.29050924)
\lineto(467.2176948,59.95969591)
\lineto(379.6550548,59.95969591)
\closepath
}
}
{
\newrgbcolor{curcolor}{0 0 0}
\pscustom[linewidth=2.66666667,linecolor=curcolor]
{
\newpath
\moveto(352.09758813,93.98949591)
\lineto(439.66022813,93.98949591)
\lineto(467.2176948,106.65868257)
\lineto(379.6550548,106.65868257)
\closepath
}
}
{
\newrgbcolor{curcolor}{0 0 0}
\pscustom[linewidth=2.66666667,linecolor=curcolor]
{
\newpath
\moveto(439.66022813,93.98949591)
\lineto(439.66022813,47.29050924)
\lineto(467.2176948,59.95969591)
\lineto(467.2176948,106.65868257)
\closepath
}
}
{
\newrgbcolor{curcolor}{0 0 0}
\pscustom[linewidth=2.66666667,linecolor=curcolor]
{
\newpath
\moveto(352.09758813,93.98949591)
\lineto(439.66022813,93.98949591)
\lineto(439.66022813,47.29050924)
\lineto(352.09758813,47.29050924)
\closepath
}
}
{
\newrgbcolor{curcolor}{0 0 0}
\pscustom[linewidth=1.33333325,linecolor=curcolor]
{
\newpath
\moveto(384.80344147,60.20698924)
\lineto(357.18590813,47.64137591)
}
}
{
\newrgbcolor{curcolor}{0 0 0}
\pscustom[linewidth=1.33333325,linecolor=curcolor]
{
\newpath
\moveto(384.80344433,60.20698924)
\lineto(357.185911,47.64137591)
}
}
{
\newrgbcolor{curcolor}{0 0 0}
\pscustom[linewidth=1.33333325,linecolor=curcolor]
{
\newpath
\moveto(401.64580413,60.20698924)
\lineto(374.0282708,47.64137591)
}
}
{
\newrgbcolor{curcolor}{0 0 0}
\pscustom[linewidth=1.33333325,linecolor=curcolor]
{
\newpath
\moveto(396.03165493,60.20698924)
\lineto(368.4141216,47.64137591)
}
}
{
\newrgbcolor{curcolor}{0 0 0}
\pscustom[linewidth=1.33333325,linecolor=curcolor]
{
\newpath
\moveto(390.41754413,60.20698924)
\lineto(362.8000108,47.64137591)
}
}
{
\newrgbcolor{curcolor}{0 0 0}
\pscustom[linewidth=1.33333325,linecolor=curcolor]
{
\newpath
\moveto(407.25995347,60.20698924)
\lineto(379.64242013,47.64137591)
}
}
{
\newrgbcolor{curcolor}{0 0 0}
\pscustom[linewidth=1.33333325,linecolor=curcolor]
{
\newpath
\moveto(412.87410147,60.20698924)
\lineto(385.25656813,47.64137591)
}
}
{
\newrgbcolor{curcolor}{0 0 0}
\pscustom[linewidth=1.33333325,linecolor=curcolor]
{
\newpath
\moveto(469.01538947,60.20698924)
\lineto(441.39785613,47.64137591)
}
}
{
\newrgbcolor{curcolor}{0 0 0}
\pscustom[linewidth=1.33333325,linecolor=curcolor]
{
\newpath
\moveto(463.40124147,60.20698924)
\lineto(435.78370813,47.64137591)
}
}
{
\newrgbcolor{curcolor}{0 0 0}
\pscustom[linewidth=1.33333325,linecolor=curcolor]
{
\newpath
\moveto(457.7871588,60.20698924)
\lineto(430.16962547,47.64137591)
}
}
{
\newrgbcolor{curcolor}{0 0 0}
\pscustom[linewidth=1.33333325,linecolor=curcolor]
{
\newpath
\moveto(452.1730108,60.20698924)
\lineto(424.55547747,47.64137591)
}
}
{
\newrgbcolor{curcolor}{0 0 0}
\pscustom[linewidth=1.33333325,linecolor=curcolor]
{
\newpath
\moveto(446.55886147,60.20698924)
\lineto(418.94132813,47.64137591)
}
}
{
\newrgbcolor{curcolor}{0 0 0}
\pscustom[linewidth=1.33333325,linecolor=curcolor]
{
\newpath
\moveto(440.94471347,60.20698924)
\lineto(413.32718013,47.64137591)
}
}
{
\newrgbcolor{curcolor}{0 0 0}
\pscustom[linewidth=1.33333325,linecolor=curcolor]
{
\newpath
\moveto(435.3306308,60.20698924)
\lineto(407.71309747,47.64137591)
}
}
{
\newrgbcolor{curcolor}{0 0 0}
\pscustom[linewidth=1.33333325,linecolor=curcolor]
{
\newpath
\moveto(429.71648147,60.20698924)
\lineto(402.09894813,47.64137591)
}
}
{
\newrgbcolor{curcolor}{0 0 0}
\pscustom[linewidth=1.33333325,linecolor=curcolor]
{
\newpath
\moveto(424.10233347,60.20698924)
\lineto(396.48480013,47.64137591)
}
}
{
\newrgbcolor{curcolor}{0 0 0}
\pscustom[linewidth=1.33333325,linecolor=curcolor]
{
\newpath
\moveto(418.48818413,60.20698924)
\lineto(390.8706508,47.64137591)
}
}
{
\newrgbcolor{curcolor}{0 0 0}
\pscustom[linewidth=2.66666667,linecolor=curcolor]
{
\newpath
\moveto(550.62497333,93.98949591)
\lineto(550.62497333,47.29050924)
\lineto(578.18244,59.95969591)
\lineto(578.18244,106.65868257)
\closepath
}
}
{
\newrgbcolor{curcolor}{0 0 0}
\pscustom[linewidth=2.66666667,linecolor=curcolor]
{
\newpath
\moveto(578.18244,106.65868257)
\lineto(665.74508,106.65868257)
\lineto(665.74508,59.95969591)
\lineto(578.18244,59.95969591)
\closepath
}
}
{
\newrgbcolor{curcolor}{0 0 0}
\pscustom[linewidth=2.66666667,linecolor=curcolor]
{
\newpath
\moveto(550.62497333,47.29050924)
\lineto(638.18761333,47.29050924)
\lineto(665.74508,59.95969591)
\lineto(578.18244,59.95969591)
\closepath
}
}
{
\newrgbcolor{curcolor}{0 0 0}
\pscustom[linewidth=2.66666667,linecolor=curcolor]
{
\newpath
\moveto(550.62497333,93.98949591)
\lineto(638.18761333,93.98949591)
\lineto(665.74508,106.65868257)
\lineto(578.18244,106.65868257)
\closepath
}
}
{
\newrgbcolor{curcolor}{0 0 0}
\pscustom[linewidth=2.66666667,linecolor=curcolor]
{
\newpath
\moveto(638.18761333,93.98949591)
\lineto(638.18761333,47.29050924)
\lineto(665.74508,59.95969591)
\lineto(665.74508,106.65868257)
\closepath
}
}
{
\newrgbcolor{curcolor}{0 0 0}
\pscustom[linewidth=2.66666667,linecolor=curcolor]
{
\newpath
\moveto(550.62497333,93.98949591)
\lineto(638.18761333,93.98949591)
\lineto(638.18761333,47.29050924)
\lineto(550.62497333,47.29050924)
\closepath
}
}
{
\newrgbcolor{curcolor}{0 0 0}
\pscustom[linewidth=1.33333325,linecolor=curcolor]
{
\newpath
\moveto(583.33082667,60.20698924)
\lineto(555.71329333,47.64137591)
}
}
{
\newrgbcolor{curcolor}{0 0 0}
\pscustom[linewidth=1.33333325,linecolor=curcolor]
{
\newpath
\moveto(583.33082953,60.20698924)
\lineto(555.7132962,47.64137591)
}
}
{
\newrgbcolor{curcolor}{0 0 0}
\pscustom[linewidth=1.33333325,linecolor=curcolor]
{
\newpath
\moveto(600.17318933,60.20698924)
\lineto(572.555656,47.64137591)
}
}
{
\newrgbcolor{curcolor}{0 0 0}
\pscustom[linewidth=1.33333325,linecolor=curcolor]
{
\newpath
\moveto(594.55904013,60.20698924)
\lineto(566.9415068,47.64137591)
}
}
{
\newrgbcolor{curcolor}{0 0 0}
\pscustom[linewidth=1.33333325,linecolor=curcolor]
{
\newpath
\moveto(588.94492933,60.20698924)
\lineto(561.327396,47.64137591)
}
}
{
\newrgbcolor{curcolor}{0 0 0}
\pscustom[linewidth=1.33333325,linecolor=curcolor]
{
\newpath
\moveto(605.78733867,60.20698924)
\lineto(578.16980533,47.64137591)
}
}
{
\newrgbcolor{curcolor}{0 0 0}
\pscustom[linewidth=1.33333325,linecolor=curcolor]
{
\newpath
\moveto(611.40148667,60.20698924)
\lineto(583.78395333,47.64137591)
}
}
{
\newrgbcolor{curcolor}{0 0 0}
\pscustom[linewidth=1.33333325,linecolor=curcolor]
{
\newpath
\moveto(667.54277467,60.20698924)
\lineto(639.92524133,47.64137591)
}
}
{
\newrgbcolor{curcolor}{0 0 0}
\pscustom[linewidth=1.33333325,linecolor=curcolor]
{
\newpath
\moveto(661.92862667,60.20698924)
\lineto(634.31109333,47.64137591)
}
}
{
\newrgbcolor{curcolor}{0 0 0}
\pscustom[linewidth=1.33333325,linecolor=curcolor]
{
\newpath
\moveto(656.314544,60.20698924)
\lineto(628.69701067,47.64137591)
}
}
{
\newrgbcolor{curcolor}{0 0 0}
\pscustom[linewidth=1.33333325,linecolor=curcolor]
{
\newpath
\moveto(650.700396,60.20698924)
\lineto(623.08286267,47.64137591)
}
}
{
\newrgbcolor{curcolor}{0 0 0}
\pscustom[linewidth=1.33333325,linecolor=curcolor]
{
\newpath
\moveto(645.08624667,60.20698924)
\lineto(617.46871333,47.64137591)
}
}
{
\newrgbcolor{curcolor}{0 0 0}
\pscustom[linewidth=1.33333325,linecolor=curcolor]
{
\newpath
\moveto(639.47209867,60.20698924)
\lineto(611.85456533,47.64137591)
}
}
{
\newrgbcolor{curcolor}{0 0 0}
\pscustom[linewidth=1.33333325,linecolor=curcolor]
{
\newpath
\moveto(633.858016,60.20698924)
\lineto(606.24048267,47.64137591)
}
}
{
\newrgbcolor{curcolor}{0 0 0}
\pscustom[linewidth=1.33333325,linecolor=curcolor]
{
\newpath
\moveto(628.24386667,60.20698924)
\lineto(600.62633333,47.64137591)
}
}
{
\newrgbcolor{curcolor}{0 0 0}
\pscustom[linewidth=1.33333325,linecolor=curcolor]
{
\newpath
\moveto(622.62971867,60.20698924)
\lineto(595.01218533,47.64137591)
}
}
{
\newrgbcolor{curcolor}{0 0 0}
\pscustom[linewidth=1.33333325,linecolor=curcolor]
{
\newpath
\moveto(617.01556933,60.20698924)
\lineto(589.398036,47.64137591)
}
}
{
\newrgbcolor{curcolor}{1 1 1}
\pscustom[linestyle=none,fillstyle=solid,fillcolor=curcolor]
{
\newpath
\moveto(119.7872135,53.40259664)
\curveto(119.18917258,53.37124509)(118.59611941,53.24578468)(118.03837169,53.02757144)
\lineto(125.66406438,47.44451258)
\curveto(125.77472939,49.06231727)(125.1384708,50.71634439)(123.97041381,51.84315572)
\curveto(123.2045824,52.5819456)(122.22164807,53.09450535)(121.17737202,53.30300938)
\curveto(120.91627041,53.35518963)(120.6515285,53.38849387)(120.38572168,53.40259664)
\curveto(120.18636384,53.41344492)(119.98656047,53.41344492)(119.7872135,53.40259664)
\closepath
\moveto(125.66406438,47.44451258)
\curveto(125.68395018,44.45957451)(123.74144461,43.43424888)(121.55726683,42.64965754)
\lineto(121.73767339,33.79728368)
\curveto(119.39869065,30.66663145)(116.50608947,28.81937469)(111.93866946,29.34356381)
\curveto(114.7989535,27.59424536)(114.41646159,27.63547969)(116.69066879,26.40265892)
\curveto(118.96487599,25.16983816)(120.20914014,25.61173218)(120.80382329,25.64836684)
\curveto(121.39850644,25.68503404)(119.46036922,19.98056119)(119.46036922,19.98056119)
\lineto(113.76816285,16.27264786)
\lineto(117.72426301,17.64775256)
\lineto(122.63543566,17.17102035)
\lineto(125.26718515,17.83844805)
\curveto(125.49618696,17.79820091)(125.72697088,17.76682767)(125.95908052,17.75793208)
\curveto(126.7114374,17.72896716)(127.46490267,17.85903809)(128.17695549,18.10329807)
\curveto(129.60105027,18.59187227)(130.83488252,19.51821814)(131.93355655,20.54627754)
\curveto(133.14323244,21.67819841)(134.22891006,22.96465309)(134.98340766,24.43852274)
\curveto(135.73173304,25.9003074)(136.14104361,27.53268134)(136.16981825,29.17405936)
\curveto(137.12748785,29.07490604)(138.09880583,29.09356509)(139.0519984,29.22916864)
\curveto(140.04060508,29.36971902)(141.01053215,29.63550199)(141.93205958,30.01947703)
\curveto(141.63732597,32.0839056)(141.1960134,33.96589857)(141.50759016,36.78694076)
\lineto(141.93205958,30.01947703)
\curveto(142.15800788,29.37556625)(142.44150546,28.75250586)(142.77889029,28.15916978)
\curveto(143.87833584,26.22565831)(145.55426306,24.62751071)(147.52452182,23.5910222)
\curveto(149.76695521,22.41134718)(152.38834888,21.97727477)(154.89552418,22.3388047)
\lineto(153.84919436,21.17770194)
\curveto(153.84919436,21.17770194)(151.58766843,20.58000485)(150.55526953,20.16067526)
\curveto(149.52287063,19.74134566)(148.63664864,18.90422693)(148.63664864,18.90422693)
\lineto(145.94123199,18.24739583)
\lineto(149.80182599,17.1519751)
\lineto(151.69286756,18.42537279)
\curveto(151.64462005,18.52127163)(156.96306035,18.8417625)(160.1399089,18.48677408)
\curveto(160.87384697,19.21974929)(162.04231688,22.09228828)(163.32772181,24.616478)
\curveto(164.25150948,26.43054986)(165.41737142,27.96239275)(165.67930975,28.87104507)
\curveto(166.11130969,29.89553538)(163.83109329,29.59338895)(163.96018797,30.69745143)
\curveto(164.20451162,32.78709141)(163.8821661,34.92177348)(163.16853763,36.90131422)
\curveto(162.45496349,38.88086582)(161.35883223,40.71048327)(160.06350614,42.36995622)
\curveto(156.82933651,46.51331752)(152.26996862,49.6480921)(147.18494194,50.99772719)
\curveto(143.82815409,51.88866427)(140.28600452,52.00462158)(136.84897801,51.50412512)
\curveto(132.89637688,50.92854768)(129.06376427,49.53799286)(125.66406438,47.44449088)
\closepath
\moveto(111.93866946,29.34356381)
\lineto(113.64930446,47.13517374)
\curveto(113.66919026,48.94934324)(115.85762773,52.05714898)(118.03837169,53.02757144)
\curveto(117.14683595,53.36226271)(116.15664276,53.42915324)(115.22835634,53.21613632)
\curveto(114.18561248,52.97686655)(113.22852968,52.38291212)(112.54991325,51.55711899)
\curveto(111.87129682,50.73132587)(111.47543904,49.67753434)(111.44415423,48.60985702)
\curveto(109.62865696,48.94027407)(107.69913692,48.61037773)(106.09789554,47.694533)
\curveto(104.77213223,46.93624875)(103.70083106,45.81630438)(102.61932619,44.73879851)
\curveto(100.46682442,42.59425535)(98.20271225,40.559128)(95.84046421,38.64723716)
\lineto(96.51750499,37.2954324)
\curveto(97.55373979,36.18275639)(98.68976216,35.16057677)(99.90693598,34.24859402)
\curveto(103.40983555,31.62399257)(107.5968609,29.91775267)(111.93866946,29.34356381)
\closepath
}
}
{
\newrgbcolor{curcolor}{0.41960785 0.41960785 0.41960785}
\pscustom[linestyle=none,fillstyle=solid,fillcolor=curcolor]
{
\newpath
\moveto(117.9029418,22.97167193)
\lineto(115.98579876,21.47642031)
\lineto(112.21142002,21.77547497)
\lineto(108.67668144,20.51946058)
\lineto(115.20696009,19.62230743)
\lineto(119.46063002,19.98116869)
\lineto(121.55750589,19.08401555)
\lineto(123.29491608,20.39983654)
\lineto(123.53456711,24.94540884)
\curveto(123.53456711,24.94540884)(120.77866721,26.73970428)(120.89848186,26.44066046)
\curveto(121.01830737,26.1416058)(117.9029418,22.97167193)(117.9029418,22.97167193)
\closepath
}
}
{
\newrgbcolor{curcolor}{0 0 0}
\pscustom[linewidth=1.33333338,linecolor=curcolor]
{
\newpath
\moveto(117.9029418,22.97167193)
\lineto(115.98579876,21.47642031)
\lineto(112.21142002,21.77547497)
\lineto(108.67668144,20.51946058)
\lineto(115.20696009,19.62230743)
\lineto(119.46063002,19.98116869)
\lineto(121.55750589,19.08401555)
\lineto(123.29491608,20.39983654)
\lineto(123.53456711,24.94540884)
\curveto(123.53456711,24.94540884)(120.77866721,26.73970428)(120.89848186,26.44066046)
\curveto(121.01830737,26.1416058)(117.9029418,22.97167193)(117.9029418,22.97167193)
\closepath
}
}
{
\newrgbcolor{curcolor}{0.90588236 0.73333335 0.75294119}
\pscustom[linestyle=none,fillstyle=solid,fillcolor=curcolor]
{
\newpath
\moveto(163.11348765,29.65237108)
\curveto(163.11348765,29.65237108)(164.04547957,29.48320293)(164.72330275,29.39861885)
\curveto(165.40111505,29.31400223)(171.07780397,28.72195709)(173.61960012,25.93067175)
\curveto(176.16139626,23.13938641)(176.24613367,20.85561637)(176.16140713,18.74100364)
\curveto(176.07664799,16.62639958)(174.04323281,13.41219711)(171.92506936,12.31260196)
\curveto(169.8069059,11.21300681)(164.46911226,8.42172689)(157.86043142,8.59089504)
\curveto(151.25175058,8.76006428)(138.00150669,10.50082668)(130.32425218,10.95925458)
\curveto(122.64699767,11.41768357)(112.68652759,10.98241567)(110.24401917,6.56087398)
\curveto(109.51112429,5.23415154)(110.49819878,1.57040375)(112.22526402,0.66708336)
\curveto(110.14192787,8.87172676)(117.53162071,10.06885124)(135.90083839,8.15531255)
\curveto(148.77293699,7.35967874)(164.21748629,2.73928799)(175.53137507,11.5752549)
\curveto(181.68597514,16.67995322)(179.27380614,28.84204761)(171.79269297,30.7479285)
\curveto(168.94052625,31.42710706)(165.05831867,32.72941109)(161.82558342,32.23862384)
\closepath
}
}
{
\newrgbcolor{curcolor}{0 0 0}
\pscustom[linewidth=1.33543301,linecolor=curcolor]
{
\newpath
\moveto(163.11348765,29.65237108)
\curveto(163.11348765,29.65237108)(164.04547957,29.48320293)(164.72330275,29.39861885)
\curveto(165.40111505,29.31400223)(171.07780397,28.72195709)(173.61960012,25.93067175)
\curveto(176.16139626,23.13938641)(176.24613367,20.85561637)(176.16140713,18.74100364)
\curveto(176.07664799,16.62639958)(174.04323281,13.41219711)(171.92506936,12.31260196)
\curveto(169.8069059,11.21300681)(164.46911226,8.42172689)(157.86043142,8.59089504)
\curveto(151.25175058,8.76006428)(138.00150669,10.50082668)(130.32425218,10.95925458)
\curveto(122.64699767,11.41768357)(112.68652759,10.98241567)(110.24401917,6.56087398)
\curveto(109.51112429,5.23415154)(110.49819878,1.57040375)(112.22526402,0.66708336)
\curveto(110.14192787,8.87172676)(117.53162071,10.06885124)(135.90083839,8.15531255)
\curveto(148.77293699,7.35967874)(164.21748629,2.73928799)(175.53137507,11.5752549)
\curveto(181.68597514,16.67995322)(179.27380614,28.84204761)(171.79269297,30.7479285)
\curveto(168.94052625,31.42710706)(165.05831867,32.72941109)(161.82558342,32.23862384)
\closepath
}
}
{
\newrgbcolor{curcolor}{0.41960785 0.41960785 0.41960785}
\pscustom[linestyle=none,fillstyle=solid,fillcolor=curcolor]
{
\newpath
\moveto(145.64166277,24.46692355)
\lineto(144.68309126,23.27072659)
\lineto(139.93265671,23.53848396)
\lineto(138.01550281,21.38531642)
\lineto(144.26370739,21.83528157)
\lineto(148.45747,21.47642031)
\lineto(149.05657584,22.7922413)
\curveto(150.0010969,27.50912972)(147.61389925,25.57375234)(145.64166277,24.46692355)
\closepath
}
}
{
\newrgbcolor{curcolor}{0 0 0}
\pscustom[linewidth=1.33333338,linecolor=curcolor]
{
\newpath
\moveto(145.64166277,24.46692355)
\lineto(144.68309126,23.27072659)
\lineto(139.93265671,23.53848396)
\lineto(138.01550281,21.38531642)
\lineto(144.26370739,21.83528157)
\lineto(148.45747,21.47642031)
\lineto(149.05657584,22.7922413)
\curveto(150.0010969,27.50912972)(147.61389925,25.57375234)(145.64166277,24.46692355)
\closepath
}
}
{
\newrgbcolor{curcolor}{0.94901961 0.94901961 0.94901961}
\pscustom[linestyle=none,fillstyle=solid,fillcolor=curcolor]
{
\newpath
\moveto(133.0820972,24.24354652)
\curveto(136.6434045,24.14406775)(140.01536207,23.41004027)(145.64166277,24.46692355)
\curveto(146.28965725,25.47831998)(145.92821386,26.56212871)(146.22942393,27.76494312)
\curveto(146.73362127,29.77834136)(148.07974851,32.000146)(146.01564617,34.06077767)
\curveto(145.49174777,34.58378432)(144.11519417,34.73675598)(143.45031306,35.23458377)
\curveto(142.4606632,35.975565)(140.9501793,37.18970502)(139.70920772,36.94191935)
\curveto(138.68411115,36.73725561)(135.80614722,36.18041316)(135.11298047,35.66142038)
\curveto(134.10145623,34.90404738)(134.21643526,34.08475238)(133.72343047,33.10040072)
\curveto(131.92819933,29.51596475)(129.58137094,27.05045339)(133.0820972,24.24354652)
\closepath
}
}
{
\newrgbcolor{curcolor}{0 0 0}
\pscustom[linewidth=1.33333338,linecolor=curcolor]
{
\newpath
\moveto(133.0820972,24.24354652)
\curveto(136.6434045,24.14406775)(140.01536207,23.41004027)(145.64166277,24.46692355)
\curveto(146.28965725,25.47831998)(145.92821386,26.56212871)(146.22942393,27.76494312)
\curveto(146.73362127,29.77834136)(148.07974851,32.000146)(146.01564617,34.06077767)
\curveto(145.49174777,34.58378432)(144.11519417,34.73675598)(143.45031306,35.23458377)
\curveto(142.4606632,35.975565)(140.9501793,37.18970502)(139.70920772,36.94191935)
\curveto(138.68411115,36.73725561)(135.80614722,36.18041316)(135.11298047,35.66142038)
\curveto(134.10145623,34.90404738)(134.21643526,34.08475238)(133.72343047,33.10040072)
\curveto(131.92819933,29.51596475)(129.58137094,27.05045339)(133.0820972,24.24354652)
\closepath
}
}
{
\newrgbcolor{curcolor}{0.51372552 0.51372552 0.51372552}
\pscustom[linestyle=none,fillstyle=solid,fillcolor=curcolor]
{
\newpath
\moveto(119.7872135,53.40259664)
\curveto(119.18917258,53.37124509)(118.59611941,53.24578468)(118.03837169,53.02757144)
\curveto(117.62507309,52.77399278)(116.88402829,53.47305625)(115.22835634,53.21613632)
\curveto(114.18561248,52.97686655)(113.22852968,52.38291212)(112.54991325,51.55711899)
\curveto(111.87129682,50.73132587)(111.47543904,49.67753434)(111.44415423,48.60985702)
\curveto(109.62865696,48.94027407)(107.69913692,48.61037773)(106.09789554,47.694533)
\curveto(104.77213223,46.93624875)(103.70083106,45.81630438)(102.61932619,44.73879851)
\curveto(101.92188889,44.04393333)(101.20887981,43.36556839)(100.48845973,42.69415721)
\lineto(100.47574586,42.68146471)
\curveto(100.44575417,42.65347614)(100.41456715,42.62678936)(100.38446679,42.59880078)
\curveto(100.30155497,42.52264583)(100.21902348,42.44617627)(100.13614425,42.36996707)
\curveto(99.43855481,41.72682651)(98.7317288,41.09401352)(98.01377544,40.47363267)
\curveto(97.99638895,40.45855355)(97.98030645,40.44206416)(97.96281129,40.42698504)
\curveto(97.95955132,40.42373056)(97.9552047,40.4215609)(97.95194474,40.41830642)
\lineto(97.95194474,40.37588962)
\curveto(98.40590598,40.19907343)(99.43784849,39.74399874)(99.69016992,39.2402044)
\curveto(99.96714757,38.68718054)(98.50348766,36.88418478)(100.2292489,36.30989829)
\curveto(101.95501015,35.73561179)(105.64286937,38.46530057)(107.47715491,37.44558351)
\curveto(110.32443168,36.28654193)(108.5929003,34.92624297)(109.0349844,33.977094)
\curveto(109.69897445,32.65134683)(110.38699045,31.18766372)(111.91079675,30.80736625)
\curveto(112.93307888,30.26323799)(114.17992927,30.59721328)(115.36814371,30.90272267)
\lineto(121.73739086,33.79700163)
\curveto(121.4807554,33.45352324)(121.21390538,33.12995145)(120.94150256,32.82023292)
\curveto(120.7994984,32.65746526)(120.65096344,32.50742264)(120.50428667,32.35410383)
\curveto(120.37627864,32.22164627)(120.24774901,32.08924296)(120.11589422,31.96423817)
\curveto(119.92392564,31.78028381)(119.72676284,31.6061363)(119.52587282,31.43877981)
\curveto(119.44491698,31.37195438)(119.36214642,31.30800374)(119.27968013,31.24384698)
\curveto(119.0430174,31.05835216)(118.80199718,30.88003891)(118.5517078,30.71837777)
\curveto(118.54953449,30.71729294)(118.54736117,30.71512328)(118.54518786,30.71403846)
\curveto(119.54872519,28.86376587)(119.04247407,26.91336362)(120.59752172,26.62050333)
\curveto(120.98878294,23.43148733)(121.71499489,23.46972753)(119.38991047,21.67304547)
\lineto(119.68737158,20.81120349)
\lineto(114.14606908,17.32904623)
\lineto(117.72389355,17.64731863)
\lineto(122.6350662,17.17058642)
\lineto(125.26681569,17.83801412)
\curveto(125.4958175,17.79779953)(125.72660142,17.76643713)(125.95871105,17.75749815)
\curveto(126.71106793,17.72851153)(127.4645332,17.85857162)(128.17658603,18.10286414)
\curveto(128.53260701,18.22500498)(128.87707684,18.3740821)(129.21018025,18.54569112)
\curveto(129.5427512,18.71702893)(129.86443386,18.91182073)(130.17585849,19.12200624)
\curveto(130.48776125,19.33259314)(130.78968851,19.56097123)(131.08211841,19.80002403)
\curveto(131.37453743,20.03907683)(131.65852401,20.28882604)(131.93318708,20.5458436)
\curveto(132.23599454,20.82918995)(132.52995661,21.12192006)(132.81397579,21.42515132)
\curveto(133.0976581,21.72795949)(133.3704956,22.04010803)(133.6289675,22.36377745)
\curveto(133.88781973,22.6879242)(134.13199125,23.02199712)(134.35905881,23.36808995)
\curveto(134.58584383,23.71365121)(134.79441651,24.06962682)(134.98303819,24.43808881)
\curveto(135.35720632,25.16898114)(135.64738684,25.94312558)(135.84685334,26.73911847)
\curveto(136.04630898,27.53510051)(136.15506147,28.352931)(136.16944879,29.17362543)
\curveto(136.64828359,29.12404877)(137.13097601,29.10300309)(137.61266871,29.11222414)
\curveto(138.09435055,29.12090276)(138.57503266,29.16093293)(139.05162894,29.22875641)
\curveto(140.04023562,29.36930679)(141.01016269,29.63508976)(141.93169011,30.0190648)
\curveto(142.15763841,29.37515401)(142.441136,28.75209363)(142.77852083,28.15875755)
\curveto(143.32824904,27.19200181)(144.02179612,26.3085826)(144.82448689,25.53779028)
\curveto(145.22584314,25.15240496)(145.65503951,24.79571336)(146.10640365,24.46991768)
\curveto(146.55776779,24.14412199)(147.03158224,23.84973209)(147.52415236,23.59060996)
\curveto(148.9256773,22.85331715)(150.47399855,22.40682345)(152.05117051,22.26424444)
\curveto(152.3666049,22.23571345)(152.68320202,22.21965799)(152.99987519,22.21553564)
\curveto(153.00058152,22.21552479)(153.00096185,22.21554649)(153.00204851,22.21553564)
\curveto(153.0096551,22.21543801)(153.01758768,22.21562243)(153.0254116,22.21553564)
\curveto(153.65099924,22.20902667)(154.27613048,22.24916532)(154.89521992,22.33842501)
\lineto(153.84889009,21.17732225)
\curveto(153.84889009,21.17732225)(153.28263385,21.02744235)(152.58819572,20.82559917)
\curveto(151.89376845,20.62376683)(151.07117015,20.36996036)(150.55496526,20.16029557)
\curveto(149.52256637,19.74096597)(148.63634438,18.90384724)(148.63634438,18.90384724)
\lineto(145.94092773,18.24701614)
\lineto(150.01588628,16.51171694)
\lineto(151.69256329,18.4249931)
\curveto(151.69256329,18.4249931)(157.4236828,17.92537535)(160.13960464,18.48639439)
\curveto(160.26512423,18.57448247)(160.385254,18.68595944)(160.50253675,18.8190462)
\curveto(160.73709136,19.0852414)(160.95637846,19.43445854)(161.15834427,19.84031456)
\curveto(163.2789201,23.30774094)(164.00199162,25.51250292)(165.67869036,28.87045926)
\curveto(166.1106903,29.89494957)(163.83078902,29.59300926)(163.95988371,30.69707174)
\curveto(164.08204553,31.74189173)(164.06300733,32.79694166)(163.92380674,33.83926654)
\curveto(163.78455183,34.88158058)(163.5250476,35.91116958)(163.16823336,36.90093453)
\curveto(162.45465922,38.88048613)(161.35852797,40.71010358)(160.06320188,42.36957653)
\curveto(157.63796042,45.4766012)(154.46664294,48.01722607)(150.88817734,49.66249862)
\curveto(150.88757968,49.66278068)(150.88709069,49.66222741)(150.88600403,49.66249862)
\curveto(149.6929431,50.2109228)(148.45584002,50.65994416)(147.18458335,50.9973475)
\curveto(146.34605468,51.21991091)(145.49486647,51.39450321)(144.63773425,51.52281672)
\curveto(144.63703879,51.52291435)(144.6366476,51.52270823)(144.63556094,51.52281672)
\curveto(142.06206705,51.90783319)(139.42632953,51.87911778)(136.84856508,51.50372374)
\curveto(135.36634509,51.28788626)(133.89995768,50.95762109)(132.46798464,50.51848252)
\curveto(131.51333595,50.22571986)(130.57381351,49.88484506)(129.65372047,49.49721416)
\curveto(128.73361656,49.10958325)(127.83406131,48.67527217)(126.95830382,48.19626616)
\curveto(126.52042507,47.95676858)(126.08862073,47.70578267)(125.66365146,47.4440895)
\curveto(125.66908473,46.69785769)(125.55024808,46.07377757)(125.33680718,45.54563968)
\curveto(125.2300976,45.28157073)(125.09965546,45.04274575)(124.94841473,44.82313307)
\curveto(124.34347355,43.94378197)(123.41027544,43.3946418)(122.36123985,42.95858415)
\curveto(122.09912765,42.84962397)(121.82987612,42.7473138)(121.5568539,42.64923446)
\curveto(124.42398387,43.35010959)(126.17640077,44.71869664)(125.66365146,47.4440895)
\curveto(125.67756065,47.64701751)(125.67919063,47.84924038)(125.67017139,48.05219009)
\curveto(125.66147814,48.25444551)(125.64191834,48.45596325)(125.61073133,48.65604901)
\curveto(125.54835729,49.05622054)(125.44224537,49.44946001)(125.29662266,49.82775054)
\curveto(125.15080434,50.20657264)(124.96527963,50.56913316)(124.74267822,50.90833733)
\curveto(124.63158942,51.07774414)(124.51085111,51.23943783)(124.38187596,51.39566397)
\curveto(124.25291167,51.55189012)(124.1161452,51.7018785)(123.97014215,51.84273263)
\curveto(123.20431074,52.58152251)(122.2213764,53.09408227)(121.17710036,53.3025863)
\curveto(120.91599875,53.35476655)(120.65125684,53.38807078)(120.38545001,53.40217355)
\curveto(120.18609217,53.41302184)(119.9862888,53.41302184)(119.78694183,53.40217355)
\closepath
}
}
{
\newrgbcolor{curcolor}{0 0 0}
\pscustom[linestyle=none,fillstyle=solid,fillcolor=curcolor]
{
\newpath
\moveto(108.4244408,40.52127883)
\curveto(108.085691,39.3468061)(107.01203252,38.62440042)(106.02635745,38.9077406)
\curveto(105.04068239,39.19108078)(104.51624584,40.37287165)(104.85499564,41.54734439)
\curveto(105.19374544,42.72181712)(106.26740392,43.4442228)(107.25307899,43.16088262)
\curveto(108.23875405,42.87754244)(108.7631906,41.69575157)(108.4244408,40.52127883)
\closepath
}
}
{
\newrgbcolor{curcolor}{0.96862745 0.71764708 0.74117649}
\pscustom[linestyle=none,fillstyle=solid,fillcolor=curcolor]
{
\newpath
\moveto(115.62632222,46.5368366)
\curveto(115.48779536,50.16374361)(120.33471406,53.05778391)(122.09669354,51.20204379)
\curveto(125.66428172,47.44459937)(123.74992052,44.54319309)(119.40071183,43.6659574)
\curveto(118.76272545,44.44134939)(116.11258974,45.31913834)(115.62632222,46.5368366)
\closepath
}
}
{
\newrgbcolor{curcolor}{0.2 0.2 0.2}
\pscustom[linestyle=none,fillstyle=solid,fillcolor=curcolor]
{
\newpath
\moveto(97.59354398,40.10497541)
\curveto(97.01546493,39.61181239)(96.43102894,39.12520173)(95.84046421,38.64723716)
\lineto(96.51750499,37.2954324)
\curveto(96.98694022,36.79136686)(97.48388956,36.31126517)(97.99255306,35.84617751)
\curveto(98.58736136,36.20606214)(98.97076822,37.00894917)(98.9709669,37.89506049)
\curveto(98.97002926,38.99086043)(98.38759195,39.92531337)(97.59354398,40.10497541)
\closepath
}
}
{
\newrgbcolor{curcolor}{0 0 0}
\pscustom[linewidth=1.33333338,linecolor=curcolor]
{
\newpath
\moveto(141.50780749,36.78780862)
\curveto(141.19623073,33.96676643)(141.63707604,32.08467583)(141.93180964,30.02024726)
\moveto(121.73767339,33.79728368)
\curveto(119.39869065,30.66663145)(116.50608947,28.81937469)(111.93866946,29.34356381)
\curveto(114.7989535,27.59424536)(114.41625513,27.6360221)(116.69046233,26.40320134)
\curveto(118.96466953,25.17038057)(120.73838489,26.29059614)(121.33306804,26.32726334)
\curveto(121.92775119,26.36393055)(119.68737158,20.81120349)(119.68737158,20.81120349)
\lineto(116.53880867,19.25875972)
\lineto(114.14606908,17.32904623)
\lineto(117.72426301,17.64778944)
\lineto(122.63543566,17.17105073)
\lineto(125.26718515,17.83848059)
\curveto(125.49618696,17.79823346)(125.72697088,17.76688191)(125.95908052,17.75798632)
\curveto(126.7114374,17.7290214)(127.46490267,17.85909233)(128.17695549,18.10335231)
\curveto(129.60105027,18.59192651)(130.83488252,19.51827238)(131.93355655,20.54633178)
\curveto(133.14323244,21.67825265)(134.22891006,22.96470733)(134.98340766,24.43857698)
\curveto(135.73173304,25.90036165)(136.14104361,27.53273558)(136.16981825,29.1741136)
\curveto(137.12748785,29.07496028)(138.09880583,29.09361933)(139.0519984,29.22922289)
\curveto(140.04060508,29.36977326)(141.01053215,29.63555624)(141.93205958,30.01953127)
\curveto(142.15800788,29.37562049)(142.44150546,28.75256011)(142.77889029,28.15922402)
\curveto(143.87833584,26.22571255)(145.55426306,24.62756495)(147.52452182,23.59107644)
\curveto(149.76695521,22.41140142)(152.38834888,21.97732901)(154.89552418,22.33885894)
\lineto(153.84919436,21.17775619)
\curveto(153.84919436,21.17775619)(151.58766843,20.58005909)(150.55526953,20.1607295)
\curveto(149.52287063,19.7413999)(148.63664864,18.90428117)(148.63664864,18.90428117)
\lineto(145.94092773,18.24704869)
\lineto(149.80241279,17.15240253)
\lineto(151.69286756,18.42542703)
\curveto(151.64462005,18.52132587)(156.96306035,18.84177335)(160.1399089,18.48682833)
\curveto(160.87384697,19.21980353)(162.04264288,22.09289578)(163.3280478,24.61708551)
\curveto(164.25183547,26.43115737)(165.41675202,27.96183948)(165.67869036,28.87049181)
\curveto(166.1106903,29.89498212)(163.83109329,29.5934432)(163.96018797,30.69750567)
\curveto(164.20451162,32.78714566)(163.8821661,34.92182772)(163.16853763,36.90136846)
\curveto(162.45496349,38.88092006)(161.35883223,40.71053751)(160.06350614,42.37001046)
\curveto(156.82933651,46.51337176)(152.26996862,49.64814634)(147.18494194,50.99778144)
\curveto(143.82815409,51.88871851)(140.28600452,52.00467583)(136.84897801,51.50417936)
\curveto(132.89637688,50.92860192)(129.06376427,49.5380471)(125.66406438,47.44454512)
\curveto(125.68395018,44.45960705)(123.74144461,43.43428143)(121.55726683,42.64969009)
\moveto(125.66406438,47.44454512)
\curveto(125.77472939,49.06234982)(125.1384708,50.71637693)(123.97041381,51.84318826)
\curveto(123.2045824,52.58197814)(122.22164807,53.0945379)(121.17737202,53.30304193)
\curveto(120.91628128,53.35522218)(120.6515285,53.38852641)(120.38572168,53.40262918)
\curveto(120.18636384,53.41347747)(119.98656047,53.41347747)(119.7872135,53.40262918)
\lineto(119.7872135,53.40265088)
\curveto(119.18917258,53.37129933)(118.59611941,53.24583892)(118.03837169,53.02762568)
\moveto(113.64930446,47.13522798)
\curveto(113.66919026,48.94939748)(115.85762773,52.05720322)(118.03837169,53.02762568)
\curveto(117.14683595,53.36231696)(116.15664276,53.42920748)(115.22835634,53.21619056)
\curveto(114.18561248,52.97692079)(113.22852968,52.38296636)(112.54991325,51.55717323)
\curveto(111.87129682,50.73138011)(111.47543904,49.67758859)(111.44415423,48.60991126)
\curveto(109.62865696,48.94032831)(107.69913692,48.61043197)(106.09789554,47.69458724)
\curveto(104.77213223,46.93630299)(103.70083106,45.81635862)(102.61932619,44.73885275)
\curveto(100.46682442,42.5943096)(98.20271225,40.55918224)(95.84046421,38.6472914)
\lineto(96.51750499,37.29548665)
\curveto(97.55373979,36.18281063)(98.44196122,34.66011285)(99.90693598,34.24864826)
\curveto(103.19690538,33.32460222)(108.77263314,31.09160216)(111.93866946,29.34361805)
}
}
{
\newrgbcolor{curcolor}{0 0 0}
\pscustom[linewidth=1.33333338,linecolor=curcolor]
{
\newpath
\moveto(154.89521992,22.33842501)
\curveto(159.6806991,22.75404449)(161.93448805,26.10477587)(162.41871265,28.81657583)
}
}
{
\newrgbcolor{curcolor}{0 0 0}
\pscustom[linestyle=none,fillstyle=solid,fillcolor=curcolor]
{
\newpath
\moveto(251.08202181,115.84161263)
\lineto(246.24413187,115.80841263)
\curveto(242.01139192,115.77881263)(241.03248193,115.68151263)(238.42773196,115.02911264)
\curveto(232.88657202,113.64129266)(228.64375207,112.17280268)(226.3593721,110.85333269)
\curveto(225.56758211,110.3960127)(224.75593212,110.0232527)(224.55468212,110.0232527)
\curveto(223.99625213,110.0232527)(220.58712217,107.93016273)(219.39648218,106.85723274)
\curveto(217.14476221,104.82819276)(215.89621222,102.83102279)(213.64257225,97.63848285)
\curveto(211.28702228,92.21111291)(211.27932228,92.18951291)(211.01757228,90.43341293)
\curveto(210.88955228,89.57460294)(210.61303228,87.78605296)(210.40234229,86.45880298)
\curveto(209.83645229,82.89400302)(209.89820229,78.17303308)(210.54491228,75.46075311)
\curveto(210.83388228,74.24889312)(211.21287228,72.93594314)(211.38866227,72.54278314)
\curveto(211.56446227,72.14961315)(211.65240227,71.49335315)(211.58398227,71.08380316)
\curveto(211.34344228,69.64397318)(212.20621227,66.00034322)(213.21679225,64.18341324)
\curveto(213.74097225,63.24097325)(214.59588224,61.63781327)(215.11718223,60.62286328)
\curveto(216.37678222,58.17042331)(218.72210219,55.42480334)(220.56445217,54.24591336)
\curveto(222.03608215,53.30423337)(222.10604215,53.28827337)(225.03320211,53.30450337)
\curveto(226.6686921,53.31350337)(228.56572207,53.43139337)(229.24804206,53.56427336)
\lineto(229.24804206,53.56227336)
\curveto(233.68773201,54.42684335)(237.20832197,55.40204334)(238.01757196,55.99001334)
\curveto(238.51983196,56.35491333)(238.58681196,56.59828333)(238.51366196,57.78884331)
\curveto(238.43586196,59.0557633)(238.47626196,59.1755133)(239.03124195,59.2966533)
\curveto(239.46417194,59.3911533)(240.02375194,59.0973933)(241.01562193,58.25368331)
\curveto(241.77562192,57.60719332)(242.45325191,57.01007332)(242.52148191,56.92751332)
\curveto(242.86118191,56.51644333)(244.11514189,55.74786334)(245.29687188,55.22634334)
\curveto(246.49223186,54.69878335)(251.9742518,53.82036336)(254.12109177,53.81227336)
\lineto(254.92773176,53.81027336)
\lineto(254.92773176,82.98019302)
\curveto(254.92731176,116.47642263)(255.14445176,113.51726266)(252.57030179,114.98996264)
\closepath
\moveto(249.08593183,115.22052264)
\curveto(251.12805181,115.20062264)(251.4100318,115.05788264)(252.33788179,114.50567265)
\curveto(254.27565177,113.35247266)(254.29679177,113.29629266)(254.39452177,108.81622272)
\curveto(254.47612177,105.07593276)(254.45462177,104.86475276)(254.02148177,105.00567276)
\curveto(253.76928178,105.08777276)(253.07286179,105.24233276)(252.47265179,105.34942276)
\curveto(250.99097181,105.61386275)(249.71679182,106.25463275)(249.71679182,106.73614274)
\curveto(249.71679182,106.95439274)(249.37494183,107.46274273)(248.95507183,107.86700273)
\curveto(248.01188184,108.77510272)(247.10342186,108.80603272)(244.54687189,108.01348272)
\curveto(243.5010419,107.68929273)(242.25498191,107.30596273)(241.77734192,107.16192273)
\curveto(237.39939197,105.84161275)(235.39328199,105.13296276)(233.58984201,104.27130277)
\curveto(232.42989203,103.71708277)(231.31406204,103.18941278)(231.10937204,103.09942278)
\curveto(230.90467205,103.00942278)(230.29000205,102.61642279)(229.74413206,102.22442279)
\curveto(229.19827207,101.8324028)(228.49903207,101.3833728)(228.18945208,101.2283328)
\curveto(227.46393209,100.86497281)(225.00278211,98.00886284)(224.01366213,96.38067286)
\curveto(220.40259217,90.43651293)(218.53662219,81.34489304)(219.23827218,73.11700313)
\lineto(219.51757218,69.84942317)
\lineto(218.80077219,68.97247318)
\curveto(217.19453221,67.00695321)(217.10862221,65.74046322)(218.51562219,64.70880323)
\curveto(219.33817218,64.10565324)(220.43945217,64.19986324)(220.43945217,64.87286323)
\curveto(220.43945217,65.03403323)(220.55102217,65.16583323)(220.68749217,65.16583323)
\curveto(220.82395216,65.16583323)(220.93554216,64.78776323)(220.93554216,64.32598324)
\curveto(220.93554216,63.70672324)(221.16417216,63.27793325)(221.80468215,62.69317326)
\curveto(222.40211215,62.14772326)(222.67187214,61.66681327)(222.67187214,61.15216327)
\curveto(222.67187214,59.87684329)(224.10165213,59.1474733)(227.41601209,58.7302833)
\curveto(228.28213208,58.6212633)(228.81905207,58.7352833)(229.61327206,59.1990333)
\curveto(230.55331205,59.74777329)(230.65197205,59.90157329)(230.55468205,60.65997328)
\curveto(230.48068205,61.23673327)(230.58008205,61.61732327)(230.86327205,61.84747327)
\curveto(231.09217204,62.03350326)(231.32321204,62.32949326)(231.37890204,62.50567326)
\curveto(231.45080204,62.73323326)(231.66095204,62.74135326)(232.10156203,62.53687326)
\lineto(232.72070202,62.24976326)
\lineto(232.10156203,62.06226326)
\curveto(231.28273204,61.81500327)(231.23068204,60.75424328)(232.03906203,60.78687328)
\curveto(232.55853203,60.80787328)(232.59802203,60.68977328)(232.60156203,59.0837533)
\curveto(232.60356203,58.13448331)(232.70762202,57.17171332)(232.83398202,56.94312332)
\curveto(232.98784202,56.66479333)(233.62703201,56.49134333)(234.765622,56.41773333)
\curveto(236.92740197,56.27796333)(237.13776197,56.52062333)(237.02343197,59.0232033)
\curveto(236.97743197,60.03036329)(237.00993197,60.80504328)(237.09573197,60.74390328)
\curveto(237.47385197,60.47434328)(238.05666196,58.6018533)(238.05666196,57.65601332)
\curveto(238.05666196,56.75330333)(237.95251196,56.57128333)(237.25002197,56.24390333)
\curveto(236.80650198,56.03723334)(235.82871199,55.71653334)(235.078152,55.53101334)
\curveto(234.32760201,55.34550334)(232.98862202,55.01338335)(232.10159203,54.79273335)
\curveto(229.38349206,54.11656336)(228.45396207,53.99134336)(229.37112206,54.42554335)
\curveto(230.58728205,55.00129335)(230.73692205,55.15340335)(230.81448205,55.89625334)
\curveto(230.86828205,56.41207333)(230.77778205,56.64820333)(230.52541205,56.64820333)
\curveto(230.32301205,56.64820333)(230.11366205,56.49338333)(230.05862206,56.30445333)
\curveto(229.95972206,55.96508334)(226.5797321,55.26546334)(224.92971212,55.24195334)
\curveto(222.93545214,55.21355334)(220.65960217,57.23378332)(218.63088219,60.83570328)
\curveto(218.25712219,61.49932327)(217.8699522,62.04244326)(217.7695522,62.04273326)
\curveto(217.6691322,62.04301326)(217.19639221,62.71349326)(216.71877221,63.53297325)
\curveto(216.24115222,64.35243324)(215.78968222,65.08770323)(215.71682222,65.16578323)
\curveto(215.64392222,65.24388323)(215.15638223,65.91432322)(214.63283224,66.65601321)
\curveto(214.10928224,67.3977032)(213.52616225,68.00562319)(213.33791225,68.00562319)
\curveto(212.71798226,68.00562319)(212.98178226,66.89016321)(213.86526225,65.77125322)
\curveto(214.34289224,65.16632323)(214.73244224,64.51936324)(214.73244224,64.33375324)
\curveto(214.73244224,64.14814324)(215.26678223,63.33361325)(215.91994222,62.52320326)
\curveto(216.57311221,61.71277327)(217.4339822,60.49302328)(217.8320522,59.81422329)
\curveto(218.67888219,58.37023331)(221.27085216,55.22828334)(221.61526215,55.22828334)
\curveto(221.94577215,55.22828334)(223.41799213,54.37032335)(223.41799213,54.17750336)
\curveto(223.41799213,54.09050336)(222.94327214,54.08050336)(222.36330215,54.15600336)
\curveto(220.41803217,54.40934335)(217.4405022,57.33955332)(215.72073222,60.69116328)
\curveto(215.13877223,61.82533327)(214.28291224,63.36866325)(213.81838225,64.12085324)
\curveto(212.81466226,65.74611322)(211.74283227,70.31110317)(212.18166227,71.09350316)
\curveto(212.37098226,71.43102315)(212.31707226,71.82816315)(211.98049227,72.58960314)
\curveto(210.55371228,75.8174631)(210.12832229,81.09183304)(210.88283228,86.17358298)
\curveto(211.07991228,87.50082297)(211.34963228,89.32307294)(211.48440227,90.22241293)
\curveto(211.61916227,91.12176292)(211.89773227,92.20857291)(212.10159227,92.63647291)
\curveto(212.30544226,93.0643929)(212.77991226,94.24458289)(213.15627225,95.25952287)
\curveto(213.53263225,96.27448286)(214.07856224,97.55089285)(214.36916224,98.09741284)
\curveto(214.65976224,98.64392283)(215.27526223,99.92231282)(215.73830222,100.93725281)
\curveto(216.20134222,101.9522028)(216.86226221,103.15598278)(217.20705221,103.61303278)
\curveto(218.20216219,104.93220276)(220.29525217,107.17204273)(220.53713217,107.17749273)
\curveto(220.65850217,107.18049273)(221.32344216,107.58273273)(222.01565215,108.07006272)
\curveto(222.70786214,108.55742272)(223.83593213,109.18004271)(224.52346212,109.45483271)
\curveto(225.21100211,109.7296327)(226.71906209,110.4174627)(227.87307208,110.98217269)
\curveto(230.09794205,112.07094268)(232.85144202,112.95176267)(238.30471196,114.32397265)
\curveto(241.30870192,115.07986264)(242.13936191,115.17277264)(246.36721186,115.21069264)
\curveto(247.52791185,115.22109264)(248.40525184,115.22709264)(249.08596183,115.22069264)
\closepath
\moveto(247.38866185,108.01739272)
\curveto(248.28223184,107.95919273)(248.83002183,107.40859273)(249.32812183,106.26544275)
\curveto(249.60124183,105.63858275)(251.32724181,104.90568276)(253.50195178,104.49395277)
\curveto(254.40073177,104.32379277)(254.43163177,104.27803277)(254.43163177,103.15411278)
\lineto(254.43163177,101.9920028)
\lineto(252.63280179,102.17559279)
\curveto(251.6434318,102.27717279)(250.49923182,102.53888279)(250.08984182,102.75567279)
\curveto(249.68044182,102.97247278)(248.61394184,103.54628278)(247.71874185,104.03106277)
\curveto(245.37704188,105.29918276)(243.71462189,105.24766276)(238.42773196,103.74591277)
\curveto(234.527832,102.63813279)(231.86984203,101.4795228)(230.60741205,100.33770281)
\curveto(229.05916207,98.93738283)(227.28413209,96.42225286)(226.3515621,94.30841289)
\curveto(225.65516211,92.7299029)(225.61026211,92.44534291)(225.69335211,90.03497294)
\curveto(225.79255211,87.15763297)(226.4688521,85.14161299)(227.67968208,84.10723301)
\lineto(228.33007208,83.55059301)
\lineto(229.77343206,84.308413)
\curveto(230.56822205,84.724763)(231.41874204,85.27127299)(231.66210204,85.52325299)
\curveto(231.90545203,85.77525299)(232.17388203,85.90080298)(232.25976203,85.80255299)
\curveto(232.34566203,85.70425299)(232.04849203,85.39707299)(231.59960204,85.11895299)
\curveto(231.15072204,84.840833)(230.21496205,84.09524301)(229.51952206,83.46270301)
\curveto(227.67830208,81.78798303)(227.67394208,81.01762304)(229.49612206,79.03302306)
\curveto(230.26246205,78.19837307)(231.11498204,77.06780309)(231.39065204,76.52130309)
\curveto(231.66632204,75.9747931)(232.60351203,74.88992311)(233.47268202,74.10919312)
\curveto(234.34183201,73.32845313)(235.121092,72.47347314)(235.20315199,72.20880315)
\curveto(235.33499199,71.78364315)(236.82968198,70.52940316)(238.44143196,69.49395318)
\curveto(239.55084194,68.78122319)(241.77737192,66.01780322)(241.77737192,65.35333323)
\curveto(241.77737192,64.49705324)(240.21350194,62.92501325)(239.33791195,62.90216325)
\curveto(239.15595195,62.89716325)(238.60799195,63.36365325)(238.12112196,63.93731324)
\curveto(236.88417198,65.39471323)(235.32566199,66.71776321)(233.79494201,67.6052832)
\curveto(233.07595202,68.02217319)(231.60782204,69.36110318)(230.52541205,70.58966316)
\curveto(229.32141206,71.95624315)(228.35937208,72.81863314)(228.03713208,72.82012314)
\curveto(227.74807208,72.82212314)(227.09184209,72.90252314)(226.5801021,72.99981314)
\curveto(225.79610211,73.14882313)(225.65041211,73.28835313)(225.65041211,73.88262313)
\curveto(225.65041211,74.33463312)(225.48319211,74.63869312)(225.18362211,74.72833312)
\curveto(224.88174212,74.81863311)(224.75561212,75.04694311)(224.83010212,75.37286311)
\curveto(224.94725212,75.8854031)(223.88112213,77.08966309)(223.31057213,77.08966309)
\curveto(223.15270214,77.08966309)(222.52637214,77.68280308)(221.91799215,78.40802307)
\curveto(221.30960216,79.13323306)(220.96374216,79.62690306)(221.15041216,79.50567306)
\curveto(222.01014215,78.94739307)(223.14510214,78.85850307)(223.49612213,79.32208306)
\curveto(224.08448213,80.09911305)(224.22394212,81.33751304)(223.80666213,82.06622303)
\curveto(223.59939213,82.42816303)(223.38297213,83.31357301)(223.32424213,84.03302301)
\curveto(223.26554214,84.752463)(223.07925214,85.53041299)(222.91018214,85.76348299)
\curveto(222.50892214,86.31671298)(222.79869214,86.64351298)(223.28323213,86.18341298)
\curveto(223.91860213,85.58004299)(224.41491212,86.01848298)(224.17580212,86.97247297)
\curveto(223.91180213,88.02573296)(223.41801213,89.10098295)(223.08401214,89.34942294)
\curveto(222.70040214,89.63474294)(221.92776215,89.06815295)(221.92776215,88.50177295)
\curveto(221.92776215,88.23170296)(221.70459215,87.53560297)(221.43166216,86.95294297)
\curveto(221.15872216,86.37026298)(220.93367216,85.66816299)(220.93166216,85.39434299)
\curveto(220.92966216,85.12052299)(220.70729217,84.13005301)(220.43557217,83.19317302)
\curveto(219.94285217,81.49432304)(219.94114217,81.49325304)(220.06643217,82.76739302)
\curveto(220.13553217,83.47005301)(220.47416217,85.19411299)(220.82033216,86.59942298)
\curveto(221.16652216,88.00473296)(221.54080216,89.53919294)(221.65041215,90.00762294)
\curveto(221.92416215,91.17742292)(223.84427213,95.24336287)(224.53323212,96.11114286)
\curveto(224.84313212,96.50150286)(225.27790211,97.22884285)(225.49807211,97.72833285)
\curveto(225.71826211,98.22784284)(226.6376321,99.24071283)(227.54104208,99.97833282)
\curveto(230.94163205,102.75490279)(235.51303199,104.84322276)(242.15041191,106.65020274)
\curveto(243.2421519,106.94741274)(244.67794188,107.38068273)(245.34182188,107.61309273)
\curveto(246.19195187,107.91073273)(246.85256186,108.05233272)(247.38869185,108.01739272)
\closepath
\moveto(244.16015189,104.31231277)
\curveto(245.75939187,104.34061277)(246.14574187,104.22501277)(248.32030184,103.06231278)
\curveto(250.22909182,102.04169279)(251.09478181,101.7444828)(252.57421179,101.6013828)
\lineto(254.43163177,101.4216928)
\lineto(254.43163177,100.06427282)
\curveto(254.43163177,98.71538283)(254.42763177,98.70711283)(253.74999178,98.93927283)
\curveto(253.37472178,99.06782283)(252.75443179,99.31367283)(252.37109179,99.48614282)
\curveto(251.9877618,99.65862282)(251.06294181,100.12081282)(250.31640182,100.51153281)
\curveto(249.56985183,100.90227281)(248.81872184,101.2205228)(248.64843184,101.2205228)
\curveto(248.47814184,101.2205228)(248.11878184,101.3977628)(247.84960185,101.6130928)
\curveto(246.65738186,102.56686279)(244.33670189,103.84223277)(243.4101519,104.05255277)
\curveto(242.49136191,104.26108277)(242.55893191,104.28406277)(244.16015189,104.31231277)
\closepath
\moveto(240.66210193,103.45098278)
\lineto(242.53320191,103.37678278)
\curveto(243.5631319,103.33558278)(244.62412188,103.14337278)(244.89062188,102.95099278)
\curveto(245.33481188,102.63035279)(245.32453188,102.61707279)(244.75585188,102.78889279)
\curveto(244.41470189,102.89196278)(243.3536819,103.08420278)(242.39843191,103.21467278)
\closepath
\moveto(239.56640194,103.17364278)
\curveto(239.65260194,103.17764278)(239.75028194,103.15994278)(239.83984194,103.11894278)
\curveto(240.03778194,103.02834278)(239.97848194,102.96069278)(239.68945194,102.94707278)
\curveto(239.42788195,102.93477278)(239.28083195,103.00117278)(239.36327195,103.09551278)
\curveto(239.40447195,103.14271278)(239.48020194,103.16961278)(239.56640194,103.17361278)
\closepath
\moveto(238.74413195,102.90802278)
\curveto(238.92762195,102.89412278)(238.71073195,102.67200279)(237.93163196,102.25567279)
\curveto(237.24931197,101.8910528)(236.35665198,101.2835128)(235.94726199,100.90606281)
\curveto(235.53786199,100.52861281)(234.567992,99.73313282)(233.79101201,99.13653283)
\curveto(232.19560203,97.91149284)(230.86132205,96.26497286)(230.86132205,95.52130287)
\curveto(230.86132205,95.24718287)(231.16768204,94.75716288)(231.54296204,94.43145288)
\curveto(232.03596203,94.00358289)(232.08705203,93.89009289)(231.72851204,94.02325289)
\curveto(231.45558204,94.12460289)(230.92604205,94.53085288)(230.55077205,94.92755288)
\curveto(230.17549205,95.32423287)(229.86913206,95.88124287)(229.86913206,96.16388286)
\curveto(229.86913206,97.47328285)(233.30698202,100.37346281)(237.15820197,102.31427279)
\curveto(237.97819196,102.72750279)(238.56064196,102.92190278)(238.74413195,102.90802278)
\closepath
\moveto(236.01171199,102.50372279)
\curveto(236.07461198,102.49272279)(235.54308199,102.12055279)(234.707022,101.5974728)
\curveto(231.61294204,99.66160282)(229.85056206,97.97811284)(229.35156206,96.47638286)
\curveto(229.13432207,95.82259287)(229.21112207,95.60740287)(229.98632206,94.69903288)
\curveto(230.97286204,93.54303289)(232.49001203,92.8210029)(234.707022,92.45294291)
\curveto(235.52581199,92.31699291)(235.91537199,92.18358291)(235.57421199,92.15606291)
\curveto(235.23305199,92.12846291)(234.22881201,91.62853292)(233.34179202,91.04473292)
\curveto(231.81194203,90.03785294)(231.72314204,89.92023294)(231.60546204,88.76348295)
\curveto(231.51146204,87.83900296)(231.33158204,87.44834297)(230.86132205,87.15216297)
\curveto(230.11049205,86.67925298)(228.24329208,86.62790298)(227.18359209,87.05059297)
\curveto(226.2499521,87.42301297)(226.0226621,88.39268295)(226.2031221,91.23809292)
\curveto(226.3842021,94.09327289)(228.13797208,97.30089285)(230.79687205,99.63653282)
\curveto(232.15661203,100.83096281)(233.01372202,101.3055228)(235.94726199,102.48614279)
\curveto(235.98136199,102.49984279)(236.00276199,102.50534279)(236.01166199,102.50374279)
\closepath
\moveto(242.09570191,102.07012279)
\curveto(243.4174619,102.07512279)(245.68152187,101.14050281)(247.21874185,99.95684282)
\curveto(247.93793185,99.40307283)(248.62346184,98.94903283)(248.74218184,98.94903283)
\curveto(248.86088183,98.94903283)(249.60387183,98.60979283)(250.39257182,98.19317284)
\curveto(251.18129181,97.77656284)(252.41246179,97.37761285)(253.12890178,97.30645285)
\lineto(254.43163177,97.17559285)
\lineto(254.43163177,95.32208287)
\curveto(254.43163177,92.58547291)(253.63995178,91.54590292)(250.46288182,90.11700293)
\curveto(249.18420183,89.54191294)(246.35878186,89.55405294)(244.50780189,90.14240293)
\curveto(242.79112191,90.68806293)(240.05891194,91.87364291)(239.91796194,92.13459291)
\curveto(239.84856194,92.26308291)(240.71361193,92.33772291)(241.83984192,92.30060291)
\curveto(243.80676189,92.23580291)(245.44067187,91.77459292)(247.02734186,90.83576293)
\curveto(247.90092185,90.31886293)(249.60952183,90.62058293)(251.3925718,91.60529292)
\curveto(252.89563179,92.43539291)(252.80186179,93.0796729)(251.07226181,93.82013289)
\curveto(250.31657182,94.14364289)(249.48000183,94.40803288)(249.21288183,94.40803288)
\curveto(248.94576183,94.40803288)(248.16379184,95.04017288)(247.47460185,95.81232287)
\curveto(245.93530187,97.53693285)(244.04986189,98.41895284)(241.77734192,98.48029284)
\curveto(240.50603193,98.51459284)(240.16601194,98.43649284)(240.16601194,98.10724284)
\curveto(240.16601194,97.79734284)(240.71267193,97.60546285)(242.28515191,97.36701285)
\curveto(244.73997188,96.99476285)(245.24167188,96.73479286)(246.67773186,95.09162288)
\curveto(247.25859185,94.42698288)(248.15259184,93.72980289)(248.66406184,93.54279289)
\curveto(249.17551183,93.3557829)(249.70043182,93.0902429)(249.83007182,92.9529529)
\curveto(249.95970182,92.8156629)(250.29445182,92.6943929)(250.57421181,92.6834229)
\curveto(251.07776181,92.6636229)(251.07723181,92.6613229)(250.58591181,92.41974291)
\curveto(249.69414182,91.98128291)(246.92318186,92.10115291)(246.11911187,92.61310291)
\curveto(245.58229187,92.9548929)(244.54566189,93.1245129)(242.39841191,93.2185729)
\lineto(239.41989195,93.3494329)
\lineto(241.03318193,93.87678289)
\curveto(243.5405119,94.69824288)(244.53261189,94.53008288)(246.53122186,92.9431829)
\curveto(247.23976185,92.38059291)(247.98044184,92.25189291)(247.98044184,92.6912329)
\curveto(247.98044184,92.9624329)(247.71875185,93.2022529)(245.87107187,94.62287288)
\curveto(245.03046188,95.26918287)(244.61360188,95.38614287)(243.1425519,95.38849287)
\curveto(242.03013191,95.39049287)(240.98446193,95.20142287)(240.23239194,94.86310288)
\curveto(238.18625196,93.94268289)(232.10154203,94.58997288)(232.10154203,95.72834287)
\curveto(232.10154203,96.23048286)(234.662732,98.50842284)(237.49411197,100.52326281)
\curveto(238.70180195,101.3826528)(240.73250193,102.06525279)(242.09568191,102.07013279)
\closepath
\moveto(249.22265183,100.25958282)
\curveto(249.27175183,100.25558282)(249.35298183,100.23448282)(249.46874183,100.19898282)
\curveto(250.13248182,99.99572282)(251.23062181,99.22601283)(250.82812181,99.24586283)
\curveto(250.69540181,99.25286283)(250.19442182,99.51287282)(249.71679182,99.82398282)
\curveto(249.23059183,100.14067282)(249.07538183,100.27276282)(249.22265183,100.25953282)
\closepath
\moveto(251.5312418,99.08184283)
\curveto(251.5530418,99.08884283)(251.6044418,99.07884283)(251.6874918,99.05454283)
\curveto(252.59584179,98.78824283)(253.86471178,98.11935284)(253.43945178,98.13071284)
\curveto(253.23475178,98.13571284)(252.62008179,98.38699284)(252.0742118,98.68735283)
\curveto(251.6648118,98.91262283)(251.4657518,99.06221283)(251.5312418,99.08188283)
\closepath
\moveto(232.84374202,93.67169289)
\curveto(233.08222202,93.68919289)(233.47363202,93.67569289)(233.98827201,93.63069289)
\curveto(234.928652,93.54819289)(235.74586199,93.4342029)(235.80273199,93.3767929)
\curveto(236.02916199,93.1481529)(232.94695202,93.2977929)(232.62499203,93.53108289)
\curveto(232.52111203,93.60638289)(232.60529203,93.65423289)(232.84374202,93.67171289)
\closepath
\moveto(254.24999177,92.00567291)
\curveto(254.40034177,92.08727291)(254.43163177,91.93497291)(254.43163177,91.46075292)
\curveto(254.43163177,90.61322293)(254.35913177,90.55892293)(253.96288177,91.10528292)
\curveto(253.74873178,91.40054292)(253.77122178,91.57839292)(254.05658177,91.84942291)
\curveto(254.13628177,91.92512291)(254.19983177,91.97848291)(254.24994177,92.00567291)
\closepath
\moveto(238.42773196,91.87872291)
\lineto(237.43554197,91.23809292)
\curveto(236.02287199,90.32614293)(233.59432201,89.34269294)(232.72070202,89.32794294)
\curveto(231.99860203,89.31574294)(231.99126203,89.32994294)(232.47265203,89.76544294)
\curveto(233.48646202,90.68308293)(235.61604199,91.61302292)(236.99999197,91.74395292)
\closepath
\moveto(239.27148195,91.68341292)
\lineto(240.36523193,91.05841292)
\curveto(240.96712193,90.71446293)(241.59202192,90.43341293)(241.75390192,90.43341293)
\curveto(241.91576192,90.43341293)(242.9095719,90.12165293)(243.96093189,89.74005294)
\curveto(245.01230188,89.35844294)(246.11379187,89.00659295)(246.41015186,88.95880295)
\curveto(246.94856186,88.87200295)(246.94872186,88.87160295)(246.48825186,87.79473296)
\curveto(246.23481187,87.20200297)(245.99282187,86.50317298)(245.94919187,86.24395298)
\curveto(245.78535187,85.27051299)(244.15070189,85.14255299)(239.79489194,85.75958299)
\curveto(236.77660198,86.18714298)(232.89100202,87.57958296)(232.67966202,88.30841296)
\curveto(232.63506203,88.46239295)(233.29572202,88.81706295)(234.14841201,89.09747295)
\curveto(235.001122,89.37787294)(236.50193198,90.07528294)(237.48435197,90.64630293)
\closepath
\moveto(253.59374178,90.62091293)
\curveto(253.77075178,90.55641293)(253.93554177,90.15301293)(253.93554177,89.58770294)
\curveto(253.93554177,88.57597295)(253.34584178,88.27060296)(253.23437178,89.22442295)
\curveto(253.19007178,89.60344294)(253.21747178,90.10288293)(253.29487178,90.33380293)
\curveto(253.37667178,90.57782293)(253.48750178,90.65960293)(253.59370178,90.62091293)
\closepath
\moveto(252.49218179,90.43341293)
\curveto(252.61914179,90.43341293)(252.68807179,89.83350294)(252.64648179,89.10138295)
\curveto(252.57688179,87.87607296)(252.49167179,87.72052296)(251.5781218,87.14434297)
\curveto(250.49771182,86.46292298)(249.78170182,86.33011298)(249.25390183,86.71270297)
\curveto(248.97653183,86.91376297)(248.97072183,87.14874297)(249.21680183,88.00177296)
\curveto(249.46922183,88.87694295)(249.73762182,89.15563295)(250.88867181,89.73809294)
\curveto(251.6431618,90.11989293)(252.36523179,90.43341293)(252.49219179,90.43341293)
\closepath
\moveto(248.89648183,88.91192295)
\curveto(248.96118183,88.83782295)(248.89348183,88.52028295)(248.74609184,88.20489296)
\curveto(248.33821184,87.33284297)(248.42353184,86.51204298)(248.95702183,86.18536298)
\curveto(249.54547183,85.82500299)(250.52743181,85.80215299)(250.84570181,86.14236298)
\curveto(250.97534181,86.28096298)(251.32410181,86.47809298)(251.6191318,86.57986298)
\curveto(252.0647818,86.73361297)(252.1824118,86.62516298)(252.31445179,85.93728298)
\curveto(252.40185179,85.48194299)(252.74605179,84.685773)(253.08007178,84.167753)
\curveto(253.60165178,83.35883301)(253.68749178,82.91379302)(253.68749178,81.01931304)
\lineto(253.68749178,78.81424307)
\lineto(252.76171179,78.09353308)
\curveto(251.7059018,77.27262309)(250.67104181,77.18384309)(249.37695183,77.80252308)
\curveto(247.95857185,78.48062307)(247.13702185,80.16038305)(247.02929186,82.60135302)
\curveto(246.94789186,84.445283)(247.01849186,84.804703)(247.85937185,86.83181297)
\curveto(248.36478184,88.05017296)(248.83173183,88.98597295)(248.89648183,88.91189295)
\closepath
\moveto(247.78124185,88.72052295)
\curveto(247.81734185,88.71052295)(247.85154185,88.68822295)(247.88476185,88.65022295)
\curveto(247.95066185,88.57482295)(247.66415185,87.70733296)(247.24804185,86.72444297)
\curveto(246.63536186,85.27725299)(246.49218186,84.575513)(246.49218186,83.03499302)
\curveto(246.49218186,80.78615304)(246.89719186,79.54185306)(248.02343184,78.32991307)
\curveto(249.12039183,77.14948309)(249.75375182,76.80647309)(250.82812181,76.80647309)
\curveto(251.7091918,76.80647309)(253.50476178,77.65805308)(253.80859178,78.22054307)
\curveto(254.18963177,78.92604307)(254.43163177,78.37846307)(254.43163177,76.80647309)
\lineto(254.43163177,75.10140311)
\lineto(252.67187179,75.10140311)
\curveto(251.7035018,75.10140311)(250.27937182,75.31165311)(249.50780183,75.56819311)
\curveto(248.73625184,75.8247331)(247.85629185,76.2410831)(247.55273185,76.49202309)
\curveto(246.03439187,77.74722308)(245.57879187,83.26397302)(246.71093186,86.67757298)
\curveto(247.20216185,88.15871296)(247.52887185,88.78952295)(247.78124185,88.72054295)
\closepath
\moveto(222.74609214,88.68932295)
\curveto(222.83089214,88.67092295)(222.95133214,88.50010295)(223.17968214,88.16002296)
\curveto(223.90242213,87.08364297)(223.81469213,86.71209297)(222.92187214,87.06822297)
\curveto(222.51263214,87.23145297)(222.21468215,87.44874297)(222.26171215,87.55064296)
\curveto(222.30871215,87.65255296)(222.42763215,87.99132296)(222.52538214,88.30260296)
\curveto(222.61078214,88.57448295)(222.66125214,88.70776295)(222.74609214,88.68932295)
\closepath
\moveto(253.57616178,87.59361296)
\curveto(254.33351177,87.59361296)(254.51554177,87.20429297)(254.32030177,86.01353298)
\curveto(254.14712177,84.957163)(253.71070178,84.769383)(253.08593178,85.48424299)
\curveto(252.38463179,86.28669298)(252.68827179,87.59361296)(253.57616178,87.59361296)
\closepath
\moveto(232.40234203,87.46471297)
\curveto(232.45314203,87.52291297)(233.02205202,87.28076297)(233.66601201,86.92564297)
\lineto(234.835932,86.27916298)
\lineto(234.21288201,85.60924299)
\curveto(232.95236202,84.254263)(233.35163202,83.35608301)(236.75780198,79.86510306)
\curveto(238.04794196,78.54283307)(238.53964196,77.46656308)(238.71874195,75.58385311)
\curveto(238.90156195,73.66200313)(239.27604195,73.02330314)(240.48632193,72.56627314)
\curveto(241.31606192,72.25295314)(241.65667192,72.25544314)(242.55859191,72.58187314)
\curveto(243.5587219,72.94391314)(243.6446719,73.05338314)(243.71679189,74.05453312)
\curveto(243.78329189,74.97814311)(243.6784919,75.24484311)(242.9960919,75.8982831)
\curveto(240.87479193,77.92957308)(239.79338194,79.40180306)(238.94921195,81.41000304)
\curveto(238.45757196,82.57957302)(238.05663196,83.73084301)(238.05663196,83.96859301)
\curveto(238.05663196,84.206343)(237.88829196,84.561313)(237.68359197,84.755703)
\curveto(237.47889197,84.950103)(237.31249197,85.19670299)(237.31249197,85.30453299)
\curveto(237.31249197,85.41235299)(238.62309195,85.32333299)(240.22656194,85.10726299)
\curveto(241.83001192,84.891213)(243.7045619,84.775223)(244.39062189,84.849453)
\lineto(245.63671187,84.984223)
\lineto(245.53911187,82.99984302)
\curveto(245.39314188,80.04411305)(245.88377187,77.81454308)(246.96684186,76.51547309)
\curveto(248.05667184,75.20826311)(249.16324183,74.80206311)(252.1367618,74.61312312)
\lineto(254.43168177,74.46664312)
\lineto(254.41798177,70.59750316)
\curveto(254.40998177,68.46912319)(254.32698177,66.28073321)(254.23243177,65.73422322)
\lineto(254.06056177,64.74008323)
\lineto(252.42384179,64.74008323)
\curveto(250.34928182,64.74008323)(249.96486182,64.40696324)(249.96486182,62.59945326)
\curveto(249.96486182,60.39765328)(250.02066182,60.33969328)(252.1015818,60.33969328)
\curveto(253.48113178,60.33969328)(253.93556177,60.24279329)(253.93556177,59.95297329)
\curveto(253.93556177,59.68993329)(253.45118178,59.51107329)(252.41017179,59.3865633)
\curveto(251.5712418,59.2862433)(250.75090181,59.0765033)(250.58790181,58.9217233)
\curveto(250.42492182,58.7669533)(250.23446182,58.7063933)(250.16408182,58.7869533)
\curveto(249.89447182,59.0954333)(247.98048184,56.57039333)(247.98048184,55.90609334)
\curveto(247.98048184,55.07491335)(247.36670185,55.01918335)(245.63282187,55.69515334)
\curveto(244.52166189,56.12837333)(241.62516192,58.18667331)(240.36134193,59.4412533)
\curveto(240.06002194,59.74037329)(239.76296194,60.28935329)(239.70118194,60.66000328)
\curveto(239.63938194,61.03064328)(239.51630194,61.54079327)(239.42775195,61.79476327)
\curveto(239.31165195,62.12774326)(239.43935195,62.30551326)(239.88478194,62.43344326)
\curveto(240.78719193,62.69260326)(242.27345191,64.40284324)(242.27345191,65.18148323)
\curveto(242.27345191,66.06306322)(241.44191192,67.4034432)(240.26564194,68.41586319)
\curveto(239.73641194,68.87134318)(239.34419195,69.28990318)(239.39454195,69.34750318)
\curveto(239.44494195,69.40510318)(240.19759194,69.16520318)(241.06642193,68.81430318)
\curveto(243.1280619,67.98162319)(247.97206184,67.7502732)(248.58790184,68.45492319)
\curveto(248.80036184,68.69804319)(248.97267183,69.06209318)(248.97267183,69.26351318)
\curveto(248.97267183,69.49575318)(249.18931183,69.58147318)(249.56251183,69.49984318)
\curveto(250.37229182,69.32272318)(251.7031418,70.15613317)(251.7031418,70.83969316)
\curveto(251.7031418,71.38965315)(250.14720182,73.23874313)(249.13868183,73.88656313)
\curveto(248.33903184,74.40023312)(245.72573187,74.34234312)(245.04493188,73.79676313)
\curveto(244.12978189,73.06331314)(244.44667189,72.27757314)(246.05470187,71.29090316)
\curveto(246.83727186,70.81071316)(247.47836185,70.30602317)(247.48048185,70.16980317)
\curveto(247.48748185,69.73723317)(243.2062619,70.12070317)(242.39845191,70.62488316)
\curveto(241.33106192,71.29109316)(238.86638195,72.54747314)(238.62892195,72.54676314)
\curveto(238.51813196,72.54647314)(237.05684197,73.33096313)(235.38087199,74.29090312)
\curveto(233.03487202,75.6346031)(232.15001203,76.3030231)(231.53517204,77.19715309)
\curveto(231.09593204,77.83591308)(230.26163205,78.88240307)(229.68165206,79.52137306)
\curveto(229.06287207,80.20308305)(228.62697207,80.94267304)(228.62697207,81.31238304)
\curveto(228.62697207,82.05491303)(229.60615206,83.16888302)(230.95314204,83.95691301)
\curveto(231.48211204,84.266383)(232.18234203,84.848053)(232.50782203,85.24988299)
\curveto(233.06558202,85.93851298)(233.07723202,86.01996298)(232.70509202,86.66980298)
\curveto(232.48795203,87.04901297)(232.35155203,87.40659297)(232.40236203,87.46473297)
\closepath
\moveto(232.02538203,86.70299297)
\curveto(232.09348203,86.67539298)(232.02138203,86.57162298)(231.85351203,86.37096298)
\curveto(231.45981204,85.90075298)(228.72446207,84.187363)(228.36718208,84.187363)
\curveto(228.25153208,84.187363)(227.87154208,84.614373)(227.52343209,85.13658299)
\curveto(227.17532209,85.65880299)(226.89062209,86.14067298)(226.89062209,86.20689298)
\curveto(226.89062209,86.27309298)(227.70007208,86.32762298)(228.68945207,86.32799298)
\curveto(229.67881206,86.32826298)(230.87979205,86.43840298)(231.35741204,86.57213298)
\curveto(231.74943204,86.68189298)(231.95733203,86.73060297)(232.02538203,86.70299297)
\closepath
\moveto(221.81835215,86.58580298)
\lineto(222.37304215,85.56041299)
\curveto(222.67762214,84.99651299)(222.86253214,84.341893)(222.78320214,84.10533301)
\curveto(222.70390214,83.86877301)(222.85880214,83.12014302)(223.12890214,82.44127302)
\curveto(223.76311213,80.84745304)(223.60461213,79.64439306)(222.75976214,79.64439306)
\curveto(222.43011215,79.64439306)(221.77402215,79.83025306)(221.30077216,80.05650305)
\curveto(220.82753216,80.28277305)(220.43945217,80.58852305)(220.43945217,80.73619304)
\curveto(220.43945217,80.88385304)(220.64872217,81.81610303)(220.90429216,82.80846302)
\curveto(221.15985216,83.80082301)(221.46994216,85.05702299)(221.59374215,85.59947299)
\closepath
\moveto(235.55273199,85.60728299)
\curveto(236.28616198,85.60728299)(236.58549198,85.43420299)(237.01757197,84.765493)
\curveto(237.31617197,84.303393)(237.56054197,83.79142301)(237.56054197,83.62682301)
\curveto(237.56054197,83.46223301)(237.65614197,83.10497302)(237.77343196,82.83385302)
\curveto(237.89069196,82.56271302)(238.30862196,81.59445303)(238.70312195,80.68150305)
\curveto(239.38830195,79.09581306)(240.75558193,77.26914309)(242.49609191,75.61314311)
\curveto(243.5685519,74.59278312)(243.5209119,73.35045313)(242.39839191,73.07799313)
\curveto(241.17010192,72.77984314)(240.94204193,72.80978314)(240.14058194,73.36900313)
\curveto(239.45499195,73.84736313)(239.35674195,74.09240312)(239.21480195,75.6463531)
\curveto(239.12730195,76.60404309)(238.89772195,77.73650308)(238.70503195,78.16197308)
\curveto(238.17999196,79.32128306)(236.75659198,80.90952304)(235.61714199,81.60728303)
\curveto(235.34562199,81.77355303)(234.890022,82.34014303)(234.605422,82.86705302)
\curveto(234.16837201,83.67619301)(234.13008201,83.96417301)(234.359332,84.716663)
\curveto(234.597882,85.49963299)(234.742822,85.60728299)(235.55269199,85.60728299)
\closepath
\moveto(254.15820177,84.595573)
\curveto(254.39134177,84.591573)(254.42254177,84.04227301)(254.39062177,81.83580303)
\lineto(254.34762177,78.93541307)
\lineto(254.18942177,81.20689304)
\curveto(254.10242177,82.45607302)(253.94813177,83.68426301)(253.84567178,83.93736301)
\curveto(253.72363178,84.238823)(253.79177178,84.457083)(254.04488177,84.568223)
\curveto(254.08688177,84.586623)(254.12488177,84.596123)(254.15817177,84.595623)
\closepath
\moveto(219.76562218,80.91588304)
\curveto(219.79732218,80.96848304)(219.84582218,80.86968304)(219.92577217,80.63853305)
\curveto(220.03378217,80.32624305)(220.06979217,79.91185305)(220.00587217,79.71666306)
\curveto(219.80634218,79.10765306)(219.69476218,79.32354306)(219.71290218,80.28307305)
\curveto(219.71990218,80.65875305)(219.73400218,80.86328304)(219.76560218,80.91588304)
\closepath
\moveto(220.19140217,79.41783306)
\lineto(221.65234215,77.96861308)
\curveto(222.45553214,77.17243309)(223.22402214,76.52135309)(223.35937213,76.52135309)
\curveto(223.80517213,76.52135309)(224.28171212,75.60365311)(224.20116212,74.90025311)
\curveto(224.14456212,74.40603312)(224.26096212,74.16719312)(224.61132212,74.06236312)
\curveto(224.90963212,73.97316312)(225.15954211,73.58381313)(225.24999211,73.06627314)
\curveto(225.38423211,72.29836314)(225.46774211,72.23410314)(226.1210921,72.39830314)
\curveto(227.08233209,72.63987314)(227.21118209,72.33140314)(227.26366209,69.66588317)
\curveto(227.32066209,66.77259321)(227.79592208,65.59193322)(229.41601206,64.32603324)
\curveto(230.17893205,63.72990324)(230.53883205,63.29404325)(230.33202205,63.21861325)
\curveto(230.14466205,63.15031325)(229.90209206,62.92213325)(229.79296206,62.71080326)
\curveto(229.53211206,62.20569326)(229.12304207,62.21899326)(229.12304207,62.73230326)
\curveto(229.12304207,62.95520325)(228.87345207,63.37504325)(228.56640207,63.66590325)
\curveto(228.25936208,63.95676324)(227.88350208,64.44581324)(227.73241208,64.75184323)
\curveto(227.58133208,65.05785323)(227.24623209,65.35602323)(226.98827209,65.41395322)
\curveto(226.4186921,65.54185322)(225.8369921,66.26302321)(224.53320212,68.46082319)
\curveto(223.98733213,69.38097318)(222.78833214,70.98146316)(221.86718215,72.01746315)
\lineto(220.19140217,73.90223313)
\lineto(220.19140217,76.66004309)
\closepath
\moveto(234.673822,73.96666312)
\curveto(234.732122,73.96666312)(235.45414199,73.58353313)(236.27929198,73.11510313)
\curveto(237.10442197,72.64665314)(237.84896196,72.26353314)(237.93359196,72.26353314)
\curveto(238.17033196,72.26353314)(240.79239193,70.92128316)(241.87890192,70.24400317)
\curveto(242.55920191,69.81992317)(243.4655019,69.59573318)(244.91796188,69.49205318)
\curveto(246.53364186,69.37672318)(246.98827186,69.25168318)(246.98827186,68.91978318)
\curveto(246.98827186,68.28702319)(243.1111119,68.62007319)(241.03320193,69.43150318)
\curveto(240.14617194,69.77789317)(239.25351195,70.17094317)(239.04882195,70.30455317)
\curveto(238.84412195,70.43815317)(237.98753196,70.95263316)(237.14452197,71.44908315)
\curveto(236.30151198,71.94556315)(235.37675199,72.71597314)(235.089842,73.16002313)
\curveto(234.802922,73.60405313)(234.615512,73.96666312)(234.673822,73.96666312)
\closepath
\moveto(247.02148186,73.68346313)
\curveto(248.47694184,73.68346313)(249.28298183,73.25158313)(250.46288182,71.84166315)
\curveto(251.31405181,70.82451316)(251.32037181,70.79658316)(250.83398181,70.37877317)
\curveto(250.10865182,69.75575317)(248.94818183,70.06497317)(246.83398186,71.44908315)
\curveto(244.36995189,73.06223314)(244.42241189,73.68346313)(247.02148186,73.68346313)
\closepath
\moveto(220.05077217,73.25963313)
\curveto(221.95584215,71.36713315)(223.54302213,69.03376318)(225.8593721,65.70689322)
\curveto(223.80647213,65.65859322)(221.23756216,66.21750322)(220.86913216,66.59947321)
\curveto(220.50071217,66.98144321)(220.03802217,71.62169315)(220.05077217,73.25963313)
\closepath
\moveto(228.40820207,71.81822315)
\curveto(228.46050207,71.87492315)(229.34174206,71.03289316)(230.36523205,69.94713317)
\curveto(231.38871204,68.86135318)(232.91806202,67.5247232)(233.76366201,66.97838321)
\curveto(235.90966199,65.59185322)(238.49709196,62.75713326)(238.91601195,61.33385327)
\curveto(239.21225195,60.32733328)(239.13040195,59.85259329)(238.64843195,59.77525329)
\curveto(238.63353195,59.77225329)(237.44043197,61.08102328)(235.99609199,62.68150326)
\curveto(234.551732,64.28200324)(232.83369202,66.18268322)(232.17773203,66.90416321)
\curveto(230.68763205,68.54314319)(228.28093208,71.68032315)(228.40820207,71.81822315)
\closepath
\moveto(227.86132208,70.64049316)
\curveto(227.96602208,70.70189316)(228.13808208,70.52033316)(228.37304208,70.08580317)
\curveto(229.07276207,68.79176319)(231.79520204,65.53820322)(232.73046202,64.87682323)
\curveto(233.23633202,64.51910324)(233.58984201,64.04013324)(233.58984201,63.71471324)
\curveto(233.58984201,63.41024325)(233.78462201,63.08341325)(234.02343201,62.98814325)
\curveto(234.395732,62.83962326)(236.31835198,60.72221328)(236.31835198,60.46080328)
\curveto(236.31835198,60.36053328)(233.86465201,60.14200329)(233.46679202,60.20689329)
\curveto(233.33032202,60.22919329)(233.25349202,60.58262328)(233.29687202,60.99400328)
\curveto(233.38187202,61.79980327)(232.67261202,63.15523325)(232.15820203,63.16978325)
\curveto(231.69902204,63.18278325)(230.09037206,64.39534324)(229.07812207,65.49205322)
\curveto(228.34115208,66.29053321)(228.15313208,66.75218321)(227.90820208,68.37096319)
\curveto(227.69742208,69.76401317)(227.68682208,70.53817316)(227.86130208,70.64049316)
\closepath
\moveto(229.17187207,69.70885317)
\curveto(229.24007207,69.70885317)(229.49714206,69.45278318)(229.74413206,69.14049318)
\curveto(229.99113206,68.82820318)(230.13658205,68.57213319)(230.06835206,68.57213319)
\curveto(230.00015206,68.57213319)(229.74309206,68.82820318)(229.49609206,69.14049318)
\curveto(229.24909206,69.45278318)(229.10364207,69.70885317)(229.17187207,69.70885317)
\closepath
\moveto(247.84179185,69.70885317)
\curveto(248.25631184,69.70885317)(248.53623184,69.30387318)(248.30468184,69.03893318)
\curveto(248.00815184,68.69962319)(247.48437185,68.85544318)(247.48437185,69.28307318)
\curveto(247.48437185,69.51729318)(247.64514185,69.70885317)(247.84179185,69.70885317)
\closepath
\moveto(219.46288218,68.85728318)
\curveto(219.74648218,68.85728318)(220.23794217,65.92122322)(220.04882217,65.35728323)
\curveto(219.86514218,64.80956323)(219.71054218,64.78252323)(218.86523219,65.15025323)
\curveto(217.9027322,65.56897322)(217.8080422,66.44927321)(218.58984219,67.7401032)
\curveto(218.96171219,68.35409319)(219.35465218,68.85728318)(219.46288218,68.85728318)
\closepath
\moveto(252.0624918,68.29283319)
\curveto(251.6958718,68.31953319)(251.20680181,68.17738319)(250.57812181,67.8768232)
\curveto(249.60166183,67.4099932)(249.48633183,66.78232321)(250.19921182,65.80455322)
\curveto(250.77237181,65.01841323)(252.54936179,64.88872323)(252.79296179,65.61510322)
\curveto(252.87576179,65.86209322)(252.94335179,66.36159321)(252.94335179,66.72643321)
\curveto(252.94335179,67.7368932)(252.67353179,68.24832319)(252.0624918,68.29283319)
\closepath
\moveto(252.0858918,67.7205732)
\curveto(252.45452179,67.7205732)(252.60913179,66.29891321)(252.28120179,65.92369322)
\curveto(251.9021618,65.48997322)(250.72073181,65.74960322)(250.45112182,66.32603321)
\curveto(250.29880182,66.65170321)(250.21065182,66.94529321)(250.25581182,66.97838321)
\curveto(250.61751181,67.2432632)(251.7948718,67.7205732)(252.0858918,67.7205732)
\closepath
\moveto(221.02730216,65.93150322)
\curveto(221.08000216,65.99180322)(221.83315215,65.89010322)(222.70308214,65.70494322)
\curveto(223.57300213,65.51983322)(224.92384212,65.31126323)(225.70308211,65.24205323)
\curveto(226.87768209,65.13773323)(227.14830209,65.01375323)(227.28706209,64.51353324)
\curveto(227.37916209,64.18169324)(227.71817208,63.67124325)(228.04097208,63.38072325)
\curveto(229.19543207,62.34165326)(228.75418207,62.13594326)(225.27730211,62.08971326)
\curveto(223.24037214,62.06261326)(222.98477214,62.12211326)(222.35347215,62.75767326)
\curveto(221.73395215,63.38137325)(220.78453216,65.65372322)(221.02730216,65.93150322)
\closepath
\moveto(214.73237224,65.44908322)
\curveto(214.74517224,65.46378322)(214.84616223,65.37758323)(215.04292223,65.20103323)
\curveto(215.28173223,64.98674323)(215.47652223,64.76388323)(215.47652223,64.70494323)
\curveto(215.47652223,64.47128324)(215.27330223,64.62264323)(214.94917223,65.09557323)
\curveto(214.79486223,65.32071323)(214.71957224,65.43443322)(214.73237224,65.44908322)
\closepath
\moveto(252.36519179,64.29088324)
\curveto(253.77413178,64.32148324)(254.18355177,64.09258324)(254.18355177,63.27330325)
\curveto(254.18355177,62.56437326)(254.00158177,62.57362326)(253.19136178,63.32600325)
\curveto(252.98666179,63.51610325)(252.48357179,63.80634324)(252.0741718,63.97054324)
\lineto(251.33003181,64.26936324)
\closepath
\moveto(215.49019223,64.11900324)
\curveto(215.55139223,64.17780324)(215.92858222,63.76620324)(216.32808222,63.20494325)
\lineto(217.05464221,62.18541326)
\lineto(216.21675222,63.09752325)
\curveto(215.75604222,63.59998325)(215.42895223,64.06023324)(215.49019223,64.11900324)
\closepath
\moveto(234.19526201,64.02920324)
\curveto(234.25546201,64.02920324)(234.374592,63.90213324)(234.458942,63.74600324)
\curveto(234.543242,63.58985325)(234.494042,63.46279325)(234.34956201,63.46279325)
\curveto(234.20504201,63.46279325)(234.08589201,63.58985325)(234.08589201,63.74600324)
\curveto(234.08589201,63.90213324)(234.13509201,64.02920324)(234.19526201,64.02920324)
\closepath
\moveto(250.42769182,63.59756325)
\lineto(251.00386181,63.01553325)
\curveto(251.32035181,62.69470326)(251.9128218,62.19024326)(252.32222179,61.89639327)
\lineto(253.06636179,61.36318327)
\lineto(252.44722179,61.15811327)
\curveto(252.1060518,61.04502328)(251.5481418,60.97313328)(251.20698181,60.99990328)
\curveto(250.65687181,61.04310328)(250.57800181,61.19382327)(250.50776182,62.32412326)
\closepath
\moveto(251.34175181,63.50776325)
\curveto(251.5016018,63.51076325)(251.7944918,63.43636325)(252.1874518,63.25972325)
\curveto(253.34586178,62.73906326)(254.18355177,61.91375327)(254.18355177,61.29292327)
\curveto(254.18355177,60.97436328)(253.92979177,61.10424328)(253.28511178,61.75191327)
\curveto(252.79068179,62.24863326)(252.1211818,62.79374326)(251.7968318,62.96284325)
\curveto(251.17864181,63.28513325)(251.07533181,63.50341325)(251.34175181,63.50776325)
\closepath
\moveto(235.62300199,62.32612326)
\curveto(235.69120199,62.32612326)(235.94826199,62.07200326)(236.19526198,61.75972327)
\curveto(236.44226198,61.44741327)(236.58771198,61.19136327)(236.51948198,61.19136327)
\curveto(236.45128198,61.19136327)(236.19422198,61.44741327)(235.94722199,61.75972327)
\curveto(235.70022199,62.07200326)(235.55476199,62.32612326)(235.62300199,62.32612326)
\closepath
\moveto(248.23042184,62.26952326)
\curveto(247.72256185,62.17622326)(247.20614185,61.43501327)(247.28511185,60.80468328)
\curveto(247.38081185,60.04107329)(248.04515184,59.88545329)(248.58394184,60.50195328)
\curveto(249.08713183,61.07772328)(249.00342183,62.14594326)(248.44526184,62.26757326)
\curveto(248.37476184,62.28297326)(248.30297184,62.28287326)(248.23042184,62.26957326)
\closepath
\moveto(228.60152207,61.94726327)
\curveto(228.85520207,61.94526327)(229.05703207,61.87476327)(229.28316206,61.73632327)
\curveto(230.14118205,61.21090327)(230.25701205,60.42720328)(229.57612206,59.75195329)
\curveto(229.05776207,59.2378833)(228.77304207,59.1962133)(226.76753209,59.3476533)
\curveto(225.53935211,59.4403533)(224.49661212,59.63679329)(224.45112212,59.78515329)
\curveto(224.40562212,59.93352329)(224.25058212,60.05468329)(224.10737213,60.05468329)
\curveto(223.80642213,60.05468329)(222.92183214,61.15729327)(222.92183214,61.53124327)
\curveto(222.92183214,61.67221327)(223.71559213,61.70113327)(224.72066212,61.59764327)
\curveto(225.8496321,61.48138327)(226.92618209,61.53744327)(227.61323208,61.74998327)
\curveto(228.04306208,61.88295327)(228.34783208,61.94938327)(228.60152207,61.94725327)
\closepath
\moveto(248.34956184,61.65038327)
\curveto(248.38056184,61.65538327)(248.40676184,61.65037327)(248.42376184,61.63088327)
\curveto(248.49196184,61.55278327)(248.40376184,61.29062327)(248.22845184,61.04885328)
\curveto(248.05309184,60.80709328)(247.85332185,60.67390328)(247.78509185,60.75198328)
\curveto(247.71689185,60.83008328)(247.80509185,61.09224328)(247.98040184,61.33401327)
\curveto(248.11192184,61.51533327)(248.25648184,61.63435327)(248.34954184,61.65041327)
\closepath
\moveto(217.5644122,61.45507327)
\curveto(217.6436122,61.41457327)(217.8317122,61.16410327)(218.0800322,60.73046328)
\curveto(218.31458219,60.32087328)(218.38423219,60.07173329)(218.23433219,60.17773329)
\curveto(218.0844522,60.28373329)(217.8450822,60.62022328)(217.7030822,60.92382328)
\curveto(217.5145622,61.32687327)(217.4851622,61.49560327)(217.5644122,61.45507327)
\closepath
\moveto(236.56636198,59.62890329)
\lineto(236.56636198,58.47070331)
\curveto(236.56650198,57.83311331)(236.49326198,57.22517332)(236.40230198,57.12109332)
\curveto(236.19718198,56.88639333)(234.405942,56.88021333)(233.65230201,57.11109332)
\curveto(233.19487202,57.25136332)(233.09370202,57.48120332)(233.09370202,58.38453331)
\curveto(233.09370202,59.59492329)(233.04250202,59.56870329)(235.51362199,59.61109329)
\closepath
\moveto(253.25972178,58.8847633)
\curveto(253.51984178,58.8797633)(253.68516178,58.8369633)(253.68355178,58.7460933)
\curveto(253.67555178,58.29314331)(251.7320718,56.63191333)(250.40034182,55.93945334)
\curveto(248.80426184,55.10951335)(248.47652184,55.07965335)(248.47652184,55.75976334)
\curveto(248.47652184,56.46848333)(249.34999183,57.69922332)(250.21480182,58.21093331)
\curveto(250.83652181,58.57881331)(252.47935179,58.9001933)(253.25972178,58.8847633)
\closepath
\moveto(219.19917218,58.6367133)
\curveto(219.21197218,58.6514133)(219.31296218,58.56521331)(219.50972218,58.38866331)
\curveto(219.74853218,58.17437331)(219.94331217,57.95152331)(219.94331217,57.89257331)
\curveto(219.94331217,57.65891332)(219.74010218,57.80837331)(219.41597218,58.28124331)
\curveto(219.26166218,58.50638331)(219.18636218,58.6220633)(219.19917218,58.6367133)
\closepath
\moveto(254.18355177,57.56640332)
\lineto(254.18355177,56.84570333)
\curveto(254.18355177,55.43333334)(253.97361177,55.22851334)(252.52925179,55.22851334)
\curveto(251.7515818,55.22851334)(251.23768181,55.34322334)(251.32026181,55.49609334)
\curveto(251.3992618,55.64235334)(251.7699618,55.93667334)(252.1425318,56.15038333)
\curveto(252.51509179,56.36411333)(253.12663178,56.77015333)(253.50191178,57.05273332)
\closepath
\moveto(250.33784182,55.22851334)
\curveto(250.54254181,55.22851334)(250.71089181,55.10145335)(250.71089181,54.94530335)
\curveto(250.71089181,54.78915335)(250.54254181,54.66210335)(250.33784182,54.66210335)
\curveto(250.13314182,54.66210335)(249.96480182,54.78915335)(249.96480182,54.94530335)
\curveto(249.96480182,55.10145335)(250.13314182,55.22851334)(250.33784182,55.22851334)
\closepath
\moveto(228.40034207,55.19331334)
\curveto(228.48654207,55.19731334)(228.58422207,55.17961335)(228.67378207,55.13861335)
\curveto(228.87172207,55.04801335)(228.81244207,54.98231335)(228.52339207,54.96869335)
\curveto(228.26182208,54.95639335)(228.11673208,55.02279335)(228.19917208,55.11713335)
\curveto(228.24037208,55.16433335)(228.31415208,55.18923335)(228.40034207,55.19333334)
\closepath
\moveto(226.5194821,54.33003336)
\curveto(228.34654208,54.29873336)(228.54826207,54.25623336)(227.63472208,54.09370336)
\curveto(225.83324211,53.77323336)(223.48063213,53.77323336)(223.91402213,54.09370336)
\curveto(224.11872213,54.24507336)(225.29130211,54.35111335)(226.5194821,54.33003336)
\closepath
}
}
{
\newrgbcolor{curcolor}{0 0 0}
\pscustom[linewidth=0.4,linecolor=curcolor]
{
\newpath
\moveto(251.08202181,115.84161263)
\lineto(246.24413187,115.80841263)
\curveto(242.01139192,115.77881263)(241.03248193,115.68151263)(238.42773196,115.02911264)
\curveto(232.88657202,113.64129266)(228.64375207,112.17280268)(226.3593721,110.85333269)
\curveto(225.56758211,110.3960127)(224.75593212,110.0232527)(224.55468212,110.0232527)
\curveto(223.99625213,110.0232527)(220.58712217,107.93016273)(219.39648218,106.85723274)
\curveto(217.14476221,104.82819276)(215.89621222,102.83102279)(213.64257225,97.63848285)
\curveto(211.28702228,92.21111291)(211.27932228,92.18951291)(211.01757228,90.43341293)
\curveto(210.88955228,89.57460294)(210.61303228,87.78605296)(210.40234229,86.45880298)
\curveto(209.83645229,82.89400302)(209.89820229,78.17303308)(210.54491228,75.46075311)
\curveto(210.83388228,74.24889312)(211.21287228,72.93594314)(211.38866227,72.54278314)
\curveto(211.56446227,72.14961315)(211.65240227,71.49335315)(211.58398227,71.08380316)
\curveto(211.34344228,69.64397318)(212.20621227,66.00034322)(213.21679225,64.18341324)
\curveto(213.74097225,63.24097325)(214.59588224,61.63781327)(215.11718223,60.62286328)
\curveto(216.37678222,58.17042331)(218.72210219,55.42480334)(220.56445217,54.24591336)
\curveto(222.03608215,53.30423337)(222.10604215,53.28827337)(225.03320211,53.30450337)
\curveto(226.6686921,53.31350337)(228.56572207,53.43139337)(229.24804206,53.56427336)
\lineto(229.24804206,53.56227336)
\curveto(233.68773201,54.42684335)(237.20832197,55.40204334)(238.01757196,55.99001334)
\curveto(238.51983196,56.35491333)(238.58681196,56.59828333)(238.51366196,57.78884331)
\curveto(238.43586196,59.0557633)(238.47626196,59.1755133)(239.03124195,59.2966533)
\curveto(239.46417194,59.3911533)(240.02375194,59.0973933)(241.01562193,58.25368331)
\curveto(241.77562192,57.60719332)(242.45325191,57.01007332)(242.52148191,56.92751332)
\curveto(242.86118191,56.51644333)(244.11514189,55.74786334)(245.29687188,55.22634334)
\curveto(246.49223186,54.69878335)(251.9742518,53.82036336)(254.12109177,53.81227336)
\lineto(254.92773176,53.81027336)
\lineto(254.92773176,82.98019302)
\curveto(254.92731176,116.47642263)(255.14445176,113.51726266)(252.57030179,114.98996264)
\closepath
\moveto(249.08593183,115.22052264)
\curveto(251.12805181,115.20062264)(251.4100318,115.05788264)(252.33788179,114.50567265)
\curveto(254.27565177,113.35247266)(254.29679177,113.29629266)(254.39452177,108.81622272)
\curveto(254.47612177,105.07593276)(254.45462177,104.86475276)(254.02148177,105.00567276)
\curveto(253.76928178,105.08777276)(253.07286179,105.24233276)(252.47265179,105.34942276)
\curveto(250.99097181,105.61386275)(249.71679182,106.25463275)(249.71679182,106.73614274)
\curveto(249.71679182,106.95439274)(249.37494183,107.46274273)(248.95507183,107.86700273)
\curveto(248.01188184,108.77510272)(247.10342186,108.80603272)(244.54687189,108.01348272)
\curveto(243.5010419,107.68929273)(242.25498191,107.30596273)(241.77734192,107.16192273)
\curveto(237.39939197,105.84161275)(235.39328199,105.13296276)(233.58984201,104.27130277)
\curveto(232.42989203,103.71708277)(231.31406204,103.18941278)(231.10937204,103.09942278)
\curveto(230.90467205,103.00942278)(230.29000205,102.61642279)(229.74413206,102.22442279)
\curveto(229.19827207,101.8324028)(228.49903207,101.3833728)(228.18945208,101.2283328)
\curveto(227.46393209,100.86497281)(225.00278211,98.00886284)(224.01366213,96.38067286)
\curveto(220.40259217,90.43651293)(218.53662219,81.34489304)(219.23827218,73.11700313)
\lineto(219.51757218,69.84942317)
\lineto(218.80077219,68.97247318)
\curveto(217.19453221,67.00695321)(217.10862221,65.74046322)(218.51562219,64.70880323)
\curveto(219.33817218,64.10565324)(220.43945217,64.19986324)(220.43945217,64.87286323)
\curveto(220.43945217,65.03403323)(220.55102217,65.16583323)(220.68749217,65.16583323)
\curveto(220.82395216,65.16583323)(220.93554216,64.78776323)(220.93554216,64.32598324)
\curveto(220.93554216,63.70672324)(221.16417216,63.27793325)(221.80468215,62.69317326)
\curveto(222.40211215,62.14772326)(222.67187214,61.66681327)(222.67187214,61.15216327)
\curveto(222.67187214,59.87684329)(224.10165213,59.1474733)(227.41601209,58.7302833)
\curveto(228.28213208,58.6212633)(228.81905207,58.7352833)(229.61327206,59.1990333)
\curveto(230.55331205,59.74777329)(230.65197205,59.90157329)(230.55468205,60.65997328)
\curveto(230.48068205,61.23673327)(230.58008205,61.61732327)(230.86327205,61.84747327)
\curveto(231.09217204,62.03350326)(231.32321204,62.32949326)(231.37890204,62.50567326)
\curveto(231.45080204,62.73323326)(231.66095204,62.74135326)(232.10156203,62.53687326)
\lineto(232.72070202,62.24976326)
\lineto(232.10156203,62.06226326)
\curveto(231.28273204,61.81500327)(231.23068204,60.75424328)(232.03906203,60.78687328)
\curveto(232.55853203,60.80787328)(232.59802203,60.68977328)(232.60156203,59.0837533)
\curveto(232.60356203,58.13448331)(232.70762202,57.17171332)(232.83398202,56.94312332)
\curveto(232.98784202,56.66479333)(233.62703201,56.49134333)(234.765622,56.41773333)
\curveto(236.92740197,56.27796333)(237.13776197,56.52062333)(237.02343197,59.0232033)
\curveto(236.97743197,60.03036329)(237.00993197,60.80504328)(237.09573197,60.74390328)
\curveto(237.47385197,60.47434328)(238.05666196,58.6018533)(238.05666196,57.65601332)
\curveto(238.05666196,56.75330333)(237.95251196,56.57128333)(237.25002197,56.24390333)
\curveto(236.80650198,56.03723334)(235.82871199,55.71653334)(235.078152,55.53101334)
\curveto(234.32760201,55.34550334)(232.98862202,55.01338335)(232.10159203,54.79273335)
\curveto(229.38349206,54.11656336)(228.45396207,53.99134336)(229.37112206,54.42554335)
\curveto(230.58728205,55.00129335)(230.73692205,55.15340335)(230.81448205,55.89625334)
\curveto(230.86828205,56.41207333)(230.77778205,56.64820333)(230.52541205,56.64820333)
\curveto(230.32301205,56.64820333)(230.11366205,56.49338333)(230.05862206,56.30445333)
\curveto(229.95972206,55.96508334)(226.5797321,55.26546334)(224.92971212,55.24195334)
\curveto(222.93545214,55.21355334)(220.65960217,57.23378332)(218.63088219,60.83570328)
\curveto(218.25712219,61.49932327)(217.8699522,62.04244326)(217.7695522,62.04273326)
\curveto(217.6691322,62.04301326)(217.19639221,62.71349326)(216.71877221,63.53297325)
\curveto(216.24115222,64.35243324)(215.78968222,65.08770323)(215.71682222,65.16578323)
\curveto(215.64392222,65.24388323)(215.15638223,65.91432322)(214.63283224,66.65601321)
\curveto(214.10928224,67.3977032)(213.52616225,68.00562319)(213.33791225,68.00562319)
\curveto(212.71798226,68.00562319)(212.98178226,66.89016321)(213.86526225,65.77125322)
\curveto(214.34289224,65.16632323)(214.73244224,64.51936324)(214.73244224,64.33375324)
\curveto(214.73244224,64.14814324)(215.26678223,63.33361325)(215.91994222,62.52320326)
\curveto(216.57311221,61.71277327)(217.4339822,60.49302328)(217.8320522,59.81422329)
\curveto(218.67888219,58.37023331)(221.27085216,55.22828334)(221.61526215,55.22828334)
\curveto(221.94577215,55.22828334)(223.41799213,54.37032335)(223.41799213,54.17750336)
\curveto(223.41799213,54.09050336)(222.94327214,54.08050336)(222.36330215,54.15600336)
\curveto(220.41803217,54.40934335)(217.4405022,57.33955332)(215.72073222,60.69116328)
\curveto(215.13877223,61.82533327)(214.28291224,63.36866325)(213.81838225,64.12085324)
\curveto(212.81466226,65.74611322)(211.74283227,70.31110317)(212.18166227,71.09350316)
\curveto(212.37098226,71.43102315)(212.31707226,71.82816315)(211.98049227,72.58960314)
\curveto(210.55371228,75.8174631)(210.12832229,81.09183304)(210.88283228,86.17358298)
\curveto(211.07991228,87.50082297)(211.34963228,89.32307294)(211.48440227,90.22241293)
\curveto(211.61916227,91.12176292)(211.89773227,92.20857291)(212.10159227,92.63647291)
\curveto(212.30544226,93.0643929)(212.77991226,94.24458289)(213.15627225,95.25952287)
\curveto(213.53263225,96.27448286)(214.07856224,97.55089285)(214.36916224,98.09741284)
\curveto(214.65976224,98.64392283)(215.27526223,99.92231282)(215.73830222,100.93725281)
\curveto(216.20134222,101.9522028)(216.86226221,103.15598278)(217.20705221,103.61303278)
\curveto(218.20216219,104.93220276)(220.29525217,107.17204273)(220.53713217,107.17749273)
\curveto(220.65850217,107.18049273)(221.32344216,107.58273273)(222.01565215,108.07006272)
\curveto(222.70786214,108.55742272)(223.83593213,109.18004271)(224.52346212,109.45483271)
\curveto(225.21100211,109.7296327)(226.71906209,110.4174627)(227.87307208,110.98217269)
\curveto(230.09794205,112.07094268)(232.85144202,112.95176267)(238.30471196,114.32397265)
\curveto(241.30870192,115.07986264)(242.13936191,115.17277264)(246.36721186,115.21069264)
\curveto(247.52791185,115.22109264)(248.40525184,115.22709264)(249.08596183,115.22069264)
\closepath
\moveto(247.38866185,108.01739272)
\curveto(248.28223184,107.95919273)(248.83002183,107.40859273)(249.32812183,106.26544275)
\curveto(249.60124183,105.63858275)(251.32724181,104.90568276)(253.50195178,104.49395277)
\curveto(254.40073177,104.32379277)(254.43163177,104.27803277)(254.43163177,103.15411278)
\lineto(254.43163177,101.9920028)
\lineto(252.63280179,102.17559279)
\curveto(251.6434318,102.27717279)(250.49923182,102.53888279)(250.08984182,102.75567279)
\curveto(249.68044182,102.97247278)(248.61394184,103.54628278)(247.71874185,104.03106277)
\curveto(245.37704188,105.29918276)(243.71462189,105.24766276)(238.42773196,103.74591277)
\curveto(234.527832,102.63813279)(231.86984203,101.4795228)(230.60741205,100.33770281)
\curveto(229.05916207,98.93738283)(227.28413209,96.42225286)(226.3515621,94.30841289)
\curveto(225.65516211,92.7299029)(225.61026211,92.44534291)(225.69335211,90.03497294)
\curveto(225.79255211,87.15763297)(226.4688521,85.14161299)(227.67968208,84.10723301)
\lineto(228.33007208,83.55059301)
\lineto(229.77343206,84.308413)
\curveto(230.56822205,84.724763)(231.41874204,85.27127299)(231.66210204,85.52325299)
\curveto(231.90545203,85.77525299)(232.17388203,85.90080298)(232.25976203,85.80255299)
\curveto(232.34566203,85.70425299)(232.04849203,85.39707299)(231.59960204,85.11895299)
\curveto(231.15072204,84.840833)(230.21496205,84.09524301)(229.51952206,83.46270301)
\curveto(227.67830208,81.78798303)(227.67394208,81.01762304)(229.49612206,79.03302306)
\curveto(230.26246205,78.19837307)(231.11498204,77.06780309)(231.39065204,76.52130309)
\curveto(231.66632204,75.9747931)(232.60351203,74.88992311)(233.47268202,74.10919312)
\curveto(234.34183201,73.32845313)(235.121092,72.47347314)(235.20315199,72.20880315)
\curveto(235.33499199,71.78364315)(236.82968198,70.52940316)(238.44143196,69.49395318)
\curveto(239.55084194,68.78122319)(241.77737192,66.01780322)(241.77737192,65.35333323)
\curveto(241.77737192,64.49705324)(240.21350194,62.92501325)(239.33791195,62.90216325)
\curveto(239.15595195,62.89716325)(238.60799195,63.36365325)(238.12112196,63.93731324)
\curveto(236.88417198,65.39471323)(235.32566199,66.71776321)(233.79494201,67.6052832)
\curveto(233.07595202,68.02217319)(231.60782204,69.36110318)(230.52541205,70.58966316)
\curveto(229.32141206,71.95624315)(228.35937208,72.81863314)(228.03713208,72.82012314)
\curveto(227.74807208,72.82212314)(227.09184209,72.90252314)(226.5801021,72.99981314)
\curveto(225.79610211,73.14882313)(225.65041211,73.28835313)(225.65041211,73.88262313)
\curveto(225.65041211,74.33463312)(225.48319211,74.63869312)(225.18362211,74.72833312)
\curveto(224.88174212,74.81863311)(224.75561212,75.04694311)(224.83010212,75.37286311)
\curveto(224.94725212,75.8854031)(223.88112213,77.08966309)(223.31057213,77.08966309)
\curveto(223.15270214,77.08966309)(222.52637214,77.68280308)(221.91799215,78.40802307)
\curveto(221.30960216,79.13323306)(220.96374216,79.62690306)(221.15041216,79.50567306)
\curveto(222.01014215,78.94739307)(223.14510214,78.85850307)(223.49612213,79.32208306)
\curveto(224.08448213,80.09911305)(224.22394212,81.33751304)(223.80666213,82.06622303)
\curveto(223.59939213,82.42816303)(223.38297213,83.31357301)(223.32424213,84.03302301)
\curveto(223.26554214,84.752463)(223.07925214,85.53041299)(222.91018214,85.76348299)
\curveto(222.50892214,86.31671298)(222.79869214,86.64351298)(223.28323213,86.18341298)
\curveto(223.91860213,85.58004299)(224.41491212,86.01848298)(224.17580212,86.97247297)
\curveto(223.91180213,88.02573296)(223.41801213,89.10098295)(223.08401214,89.34942294)
\curveto(222.70040214,89.63474294)(221.92776215,89.06815295)(221.92776215,88.50177295)
\curveto(221.92776215,88.23170296)(221.70459215,87.53560297)(221.43166216,86.95294297)
\curveto(221.15872216,86.37026298)(220.93367216,85.66816299)(220.93166216,85.39434299)
\curveto(220.92966216,85.12052299)(220.70729217,84.13005301)(220.43557217,83.19317302)
\curveto(219.94285217,81.49432304)(219.94114217,81.49325304)(220.06643217,82.76739302)
\curveto(220.13553217,83.47005301)(220.47416217,85.19411299)(220.82033216,86.59942298)
\curveto(221.16652216,88.00473296)(221.54080216,89.53919294)(221.65041215,90.00762294)
\curveto(221.92416215,91.17742292)(223.84427213,95.24336287)(224.53323212,96.11114286)
\curveto(224.84313212,96.50150286)(225.27790211,97.22884285)(225.49807211,97.72833285)
\curveto(225.71826211,98.22784284)(226.6376321,99.24071283)(227.54104208,99.97833282)
\curveto(230.94163205,102.75490279)(235.51303199,104.84322276)(242.15041191,106.65020274)
\curveto(243.2421519,106.94741274)(244.67794188,107.38068273)(245.34182188,107.61309273)
\curveto(246.19195187,107.91073273)(246.85256186,108.05233272)(247.38869185,108.01739272)
\closepath
\moveto(244.16015189,104.31231277)
\curveto(245.75939187,104.34061277)(246.14574187,104.22501277)(248.32030184,103.06231278)
\curveto(250.22909182,102.04169279)(251.09478181,101.7444828)(252.57421179,101.6013828)
\lineto(254.43163177,101.4216928)
\lineto(254.43163177,100.06427282)
\curveto(254.43163177,98.71538283)(254.42763177,98.70711283)(253.74999178,98.93927283)
\curveto(253.37472178,99.06782283)(252.75443179,99.31367283)(252.37109179,99.48614282)
\curveto(251.9877618,99.65862282)(251.06294181,100.12081282)(250.31640182,100.51153281)
\curveto(249.56985183,100.90227281)(248.81872184,101.2205228)(248.64843184,101.2205228)
\curveto(248.47814184,101.2205228)(248.11878184,101.3977628)(247.84960185,101.6130928)
\curveto(246.65738186,102.56686279)(244.33670189,103.84223277)(243.4101519,104.05255277)
\curveto(242.49136191,104.26108277)(242.55893191,104.28406277)(244.16015189,104.31231277)
\closepath
\moveto(240.66210193,103.45098278)
\lineto(242.53320191,103.37678278)
\curveto(243.5631319,103.33558278)(244.62412188,103.14337278)(244.89062188,102.95099278)
\curveto(245.33481188,102.63035279)(245.32453188,102.61707279)(244.75585188,102.78889279)
\curveto(244.41470189,102.89196278)(243.3536819,103.08420278)(242.39843191,103.21467278)
\closepath
\moveto(239.56640194,103.17364278)
\curveto(239.65260194,103.17764278)(239.75028194,103.15994278)(239.83984194,103.11894278)
\curveto(240.03778194,103.02834278)(239.97848194,102.96069278)(239.68945194,102.94707278)
\curveto(239.42788195,102.93477278)(239.28083195,103.00117278)(239.36327195,103.09551278)
\curveto(239.40447195,103.14271278)(239.48020194,103.16961278)(239.56640194,103.17361278)
\closepath
\moveto(238.74413195,102.90802278)
\curveto(238.92762195,102.89412278)(238.71073195,102.67200279)(237.93163196,102.25567279)
\curveto(237.24931197,101.8910528)(236.35665198,101.2835128)(235.94726199,100.90606281)
\curveto(235.53786199,100.52861281)(234.567992,99.73313282)(233.79101201,99.13653283)
\curveto(232.19560203,97.91149284)(230.86132205,96.26497286)(230.86132205,95.52130287)
\curveto(230.86132205,95.24718287)(231.16768204,94.75716288)(231.54296204,94.43145288)
\curveto(232.03596203,94.00358289)(232.08705203,93.89009289)(231.72851204,94.02325289)
\curveto(231.45558204,94.12460289)(230.92604205,94.53085288)(230.55077205,94.92755288)
\curveto(230.17549205,95.32423287)(229.86913206,95.88124287)(229.86913206,96.16388286)
\curveto(229.86913206,97.47328285)(233.30698202,100.37346281)(237.15820197,102.31427279)
\curveto(237.97819196,102.72750279)(238.56064196,102.92190278)(238.74413195,102.90802278)
\closepath
\moveto(236.01171199,102.50372279)
\curveto(236.07461198,102.49272279)(235.54308199,102.12055279)(234.707022,101.5974728)
\curveto(231.61294204,99.66160282)(229.85056206,97.97811284)(229.35156206,96.47638286)
\curveto(229.13432207,95.82259287)(229.21112207,95.60740287)(229.98632206,94.69903288)
\curveto(230.97286204,93.54303289)(232.49001203,92.8210029)(234.707022,92.45294291)
\curveto(235.52581199,92.31699291)(235.91537199,92.18358291)(235.57421199,92.15606291)
\curveto(235.23305199,92.12846291)(234.22881201,91.62853292)(233.34179202,91.04473292)
\curveto(231.81194203,90.03785294)(231.72314204,89.92023294)(231.60546204,88.76348295)
\curveto(231.51146204,87.83900296)(231.33158204,87.44834297)(230.86132205,87.15216297)
\curveto(230.11049205,86.67925298)(228.24329208,86.62790298)(227.18359209,87.05059297)
\curveto(226.2499521,87.42301297)(226.0226621,88.39268295)(226.2031221,91.23809292)
\curveto(226.3842021,94.09327289)(228.13797208,97.30089285)(230.79687205,99.63653282)
\curveto(232.15661203,100.83096281)(233.01372202,101.3055228)(235.94726199,102.48614279)
\curveto(235.98136199,102.49984279)(236.00276199,102.50534279)(236.01166199,102.50374279)
\closepath
\moveto(242.09570191,102.07012279)
\curveto(243.4174619,102.07512279)(245.68152187,101.14050281)(247.21874185,99.95684282)
\curveto(247.93793185,99.40307283)(248.62346184,98.94903283)(248.74218184,98.94903283)
\curveto(248.86088183,98.94903283)(249.60387183,98.60979283)(250.39257182,98.19317284)
\curveto(251.18129181,97.77656284)(252.41246179,97.37761285)(253.12890178,97.30645285)
\lineto(254.43163177,97.17559285)
\lineto(254.43163177,95.32208287)
\curveto(254.43163177,92.58547291)(253.63995178,91.54590292)(250.46288182,90.11700293)
\curveto(249.18420183,89.54191294)(246.35878186,89.55405294)(244.50780189,90.14240293)
\curveto(242.79112191,90.68806293)(240.05891194,91.87364291)(239.91796194,92.13459291)
\curveto(239.84856194,92.26308291)(240.71361193,92.33772291)(241.83984192,92.30060291)
\curveto(243.80676189,92.23580291)(245.44067187,91.77459292)(247.02734186,90.83576293)
\curveto(247.90092185,90.31886293)(249.60952183,90.62058293)(251.3925718,91.60529292)
\curveto(252.89563179,92.43539291)(252.80186179,93.0796729)(251.07226181,93.82013289)
\curveto(250.31657182,94.14364289)(249.48000183,94.40803288)(249.21288183,94.40803288)
\curveto(248.94576183,94.40803288)(248.16379184,95.04017288)(247.47460185,95.81232287)
\curveto(245.93530187,97.53693285)(244.04986189,98.41895284)(241.77734192,98.48029284)
\curveto(240.50603193,98.51459284)(240.16601194,98.43649284)(240.16601194,98.10724284)
\curveto(240.16601194,97.79734284)(240.71267193,97.60546285)(242.28515191,97.36701285)
\curveto(244.73997188,96.99476285)(245.24167188,96.73479286)(246.67773186,95.09162288)
\curveto(247.25859185,94.42698288)(248.15259184,93.72980289)(248.66406184,93.54279289)
\curveto(249.17551183,93.3557829)(249.70043182,93.0902429)(249.83007182,92.9529529)
\curveto(249.95970182,92.8156629)(250.29445182,92.6943929)(250.57421181,92.6834229)
\curveto(251.07776181,92.6636229)(251.07723181,92.6613229)(250.58591181,92.41974291)
\curveto(249.69414182,91.98128291)(246.92318186,92.10115291)(246.11911187,92.61310291)
\curveto(245.58229187,92.9548929)(244.54566189,93.1245129)(242.39841191,93.2185729)
\lineto(239.41989195,93.3494329)
\lineto(241.03318193,93.87678289)
\curveto(243.5405119,94.69824288)(244.53261189,94.53008288)(246.53122186,92.9431829)
\curveto(247.23976185,92.38059291)(247.98044184,92.25189291)(247.98044184,92.6912329)
\curveto(247.98044184,92.9624329)(247.71875185,93.2022529)(245.87107187,94.62287288)
\curveto(245.03046188,95.26918287)(244.61360188,95.38614287)(243.1425519,95.38849287)
\curveto(242.03013191,95.39049287)(240.98446193,95.20142287)(240.23239194,94.86310288)
\curveto(238.18625196,93.94268289)(232.10154203,94.58997288)(232.10154203,95.72834287)
\curveto(232.10154203,96.23048286)(234.662732,98.50842284)(237.49411197,100.52326281)
\curveto(238.70180195,101.3826528)(240.73250193,102.06525279)(242.09568191,102.07013279)
\closepath
\moveto(249.22265183,100.25958282)
\curveto(249.27175183,100.25558282)(249.35298183,100.23448282)(249.46874183,100.19898282)
\curveto(250.13248182,99.99572282)(251.23062181,99.22601283)(250.82812181,99.24586283)
\curveto(250.69540181,99.25286283)(250.19442182,99.51287282)(249.71679182,99.82398282)
\curveto(249.23059183,100.14067282)(249.07538183,100.27276282)(249.22265183,100.25953282)
\closepath
\moveto(251.5312418,99.08184283)
\curveto(251.5530418,99.08884283)(251.6044418,99.07884283)(251.6874918,99.05454283)
\curveto(252.59584179,98.78824283)(253.86471178,98.11935284)(253.43945178,98.13071284)
\curveto(253.23475178,98.13571284)(252.62008179,98.38699284)(252.0742118,98.68735283)
\curveto(251.6648118,98.91262283)(251.4657518,99.06221283)(251.5312418,99.08188283)
\closepath
\moveto(232.84374202,93.67169289)
\curveto(233.08222202,93.68919289)(233.47363202,93.67569289)(233.98827201,93.63069289)
\curveto(234.928652,93.54819289)(235.74586199,93.4342029)(235.80273199,93.3767929)
\curveto(236.02916199,93.1481529)(232.94695202,93.2977929)(232.62499203,93.53108289)
\curveto(232.52111203,93.60638289)(232.60529203,93.65423289)(232.84374202,93.67171289)
\closepath
\moveto(254.24999177,92.00567291)
\curveto(254.40034177,92.08727291)(254.43163177,91.93497291)(254.43163177,91.46075292)
\curveto(254.43163177,90.61322293)(254.35913177,90.55892293)(253.96288177,91.10528292)
\curveto(253.74873178,91.40054292)(253.77122178,91.57839292)(254.05658177,91.84942291)
\curveto(254.13628177,91.92512291)(254.19983177,91.97848291)(254.24994177,92.00567291)
\closepath
\moveto(238.42773196,91.87872291)
\lineto(237.43554197,91.23809292)
\curveto(236.02287199,90.32614293)(233.59432201,89.34269294)(232.72070202,89.32794294)
\curveto(231.99860203,89.31574294)(231.99126203,89.32994294)(232.47265203,89.76544294)
\curveto(233.48646202,90.68308293)(235.61604199,91.61302292)(236.99999197,91.74395292)
\closepath
\moveto(239.27148195,91.68341292)
\lineto(240.36523193,91.05841292)
\curveto(240.96712193,90.71446293)(241.59202192,90.43341293)(241.75390192,90.43341293)
\curveto(241.91576192,90.43341293)(242.9095719,90.12165293)(243.96093189,89.74005294)
\curveto(245.01230188,89.35844294)(246.11379187,89.00659295)(246.41015186,88.95880295)
\curveto(246.94856186,88.87200295)(246.94872186,88.87160295)(246.48825186,87.79473296)
\curveto(246.23481187,87.20200297)(245.99282187,86.50317298)(245.94919187,86.24395298)
\curveto(245.78535187,85.27051299)(244.15070189,85.14255299)(239.79489194,85.75958299)
\curveto(236.77660198,86.18714298)(232.89100202,87.57958296)(232.67966202,88.30841296)
\curveto(232.63506203,88.46239295)(233.29572202,88.81706295)(234.14841201,89.09747295)
\curveto(235.001122,89.37787294)(236.50193198,90.07528294)(237.48435197,90.64630293)
\closepath
\moveto(253.59374178,90.62091293)
\curveto(253.77075178,90.55641293)(253.93554177,90.15301293)(253.93554177,89.58770294)
\curveto(253.93554177,88.57597295)(253.34584178,88.27060296)(253.23437178,89.22442295)
\curveto(253.19007178,89.60344294)(253.21747178,90.10288293)(253.29487178,90.33380293)
\curveto(253.37667178,90.57782293)(253.48750178,90.65960293)(253.59370178,90.62091293)
\closepath
\moveto(252.49218179,90.43341293)
\curveto(252.61914179,90.43341293)(252.68807179,89.83350294)(252.64648179,89.10138295)
\curveto(252.57688179,87.87607296)(252.49167179,87.72052296)(251.5781218,87.14434297)
\curveto(250.49771182,86.46292298)(249.78170182,86.33011298)(249.25390183,86.71270297)
\curveto(248.97653183,86.91376297)(248.97072183,87.14874297)(249.21680183,88.00177296)
\curveto(249.46922183,88.87694295)(249.73762182,89.15563295)(250.88867181,89.73809294)
\curveto(251.6431618,90.11989293)(252.36523179,90.43341293)(252.49219179,90.43341293)
\closepath
\moveto(248.89648183,88.91192295)
\curveto(248.96118183,88.83782295)(248.89348183,88.52028295)(248.74609184,88.20489296)
\curveto(248.33821184,87.33284297)(248.42353184,86.51204298)(248.95702183,86.18536298)
\curveto(249.54547183,85.82500299)(250.52743181,85.80215299)(250.84570181,86.14236298)
\curveto(250.97534181,86.28096298)(251.32410181,86.47809298)(251.6191318,86.57986298)
\curveto(252.0647818,86.73361297)(252.1824118,86.62516298)(252.31445179,85.93728298)
\curveto(252.40185179,85.48194299)(252.74605179,84.685773)(253.08007178,84.167753)
\curveto(253.60165178,83.35883301)(253.68749178,82.91379302)(253.68749178,81.01931304)
\lineto(253.68749178,78.81424307)
\lineto(252.76171179,78.09353308)
\curveto(251.7059018,77.27262309)(250.67104181,77.18384309)(249.37695183,77.80252308)
\curveto(247.95857185,78.48062307)(247.13702185,80.16038305)(247.02929186,82.60135302)
\curveto(246.94789186,84.445283)(247.01849186,84.804703)(247.85937185,86.83181297)
\curveto(248.36478184,88.05017296)(248.83173183,88.98597295)(248.89648183,88.91189295)
\closepath
\moveto(247.78124185,88.72052295)
\curveto(247.81734185,88.71052295)(247.85154185,88.68822295)(247.88476185,88.65022295)
\curveto(247.95066185,88.57482295)(247.66415185,87.70733296)(247.24804185,86.72444297)
\curveto(246.63536186,85.27725299)(246.49218186,84.575513)(246.49218186,83.03499302)
\curveto(246.49218186,80.78615304)(246.89719186,79.54185306)(248.02343184,78.32991307)
\curveto(249.12039183,77.14948309)(249.75375182,76.80647309)(250.82812181,76.80647309)
\curveto(251.7091918,76.80647309)(253.50476178,77.65805308)(253.80859178,78.22054307)
\curveto(254.18963177,78.92604307)(254.43163177,78.37846307)(254.43163177,76.80647309)
\lineto(254.43163177,75.10140311)
\lineto(252.67187179,75.10140311)
\curveto(251.7035018,75.10140311)(250.27937182,75.31165311)(249.50780183,75.56819311)
\curveto(248.73625184,75.8247331)(247.85629185,76.2410831)(247.55273185,76.49202309)
\curveto(246.03439187,77.74722308)(245.57879187,83.26397302)(246.71093186,86.67757298)
\curveto(247.20216185,88.15871296)(247.52887185,88.78952295)(247.78124185,88.72054295)
\closepath
\moveto(222.74609214,88.68932295)
\curveto(222.83089214,88.67092295)(222.95133214,88.50010295)(223.17968214,88.16002296)
\curveto(223.90242213,87.08364297)(223.81469213,86.71209297)(222.92187214,87.06822297)
\curveto(222.51263214,87.23145297)(222.21468215,87.44874297)(222.26171215,87.55064296)
\curveto(222.30871215,87.65255296)(222.42763215,87.99132296)(222.52538214,88.30260296)
\curveto(222.61078214,88.57448295)(222.66125214,88.70776295)(222.74609214,88.68932295)
\closepath
\moveto(253.57616178,87.59361296)
\curveto(254.33351177,87.59361296)(254.51554177,87.20429297)(254.32030177,86.01353298)
\curveto(254.14712177,84.957163)(253.71070178,84.769383)(253.08593178,85.48424299)
\curveto(252.38463179,86.28669298)(252.68827179,87.59361296)(253.57616178,87.59361296)
\closepath
\moveto(232.40234203,87.46471297)
\curveto(232.45314203,87.52291297)(233.02205202,87.28076297)(233.66601201,86.92564297)
\lineto(234.835932,86.27916298)
\lineto(234.21288201,85.60924299)
\curveto(232.95236202,84.254263)(233.35163202,83.35608301)(236.75780198,79.86510306)
\curveto(238.04794196,78.54283307)(238.53964196,77.46656308)(238.71874195,75.58385311)
\curveto(238.90156195,73.66200313)(239.27604195,73.02330314)(240.48632193,72.56627314)
\curveto(241.31606192,72.25295314)(241.65667192,72.25544314)(242.55859191,72.58187314)
\curveto(243.5587219,72.94391314)(243.6446719,73.05338314)(243.71679189,74.05453312)
\curveto(243.78329189,74.97814311)(243.6784919,75.24484311)(242.9960919,75.8982831)
\curveto(240.87479193,77.92957308)(239.79338194,79.40180306)(238.94921195,81.41000304)
\curveto(238.45757196,82.57957302)(238.05663196,83.73084301)(238.05663196,83.96859301)
\curveto(238.05663196,84.206343)(237.88829196,84.561313)(237.68359197,84.755703)
\curveto(237.47889197,84.950103)(237.31249197,85.19670299)(237.31249197,85.30453299)
\curveto(237.31249197,85.41235299)(238.62309195,85.32333299)(240.22656194,85.10726299)
\curveto(241.83001192,84.891213)(243.7045619,84.775223)(244.39062189,84.849453)
\lineto(245.63671187,84.984223)
\lineto(245.53911187,82.99984302)
\curveto(245.39314188,80.04411305)(245.88377187,77.81454308)(246.96684186,76.51547309)
\curveto(248.05667184,75.20826311)(249.16324183,74.80206311)(252.1367618,74.61312312)
\lineto(254.43168177,74.46664312)
\lineto(254.41798177,70.59750316)
\curveto(254.40998177,68.46912319)(254.32698177,66.28073321)(254.23243177,65.73422322)
\lineto(254.06056177,64.74008323)
\lineto(252.42384179,64.74008323)
\curveto(250.34928182,64.74008323)(249.96486182,64.40696324)(249.96486182,62.59945326)
\curveto(249.96486182,60.39765328)(250.02066182,60.33969328)(252.1015818,60.33969328)
\curveto(253.48113178,60.33969328)(253.93556177,60.24279329)(253.93556177,59.95297329)
\curveto(253.93556177,59.68993329)(253.45118178,59.51107329)(252.41017179,59.3865633)
\curveto(251.5712418,59.2862433)(250.75090181,59.0765033)(250.58790181,58.9217233)
\curveto(250.42492182,58.7669533)(250.23446182,58.7063933)(250.16408182,58.7869533)
\curveto(249.89447182,59.0954333)(247.98048184,56.57039333)(247.98048184,55.90609334)
\curveto(247.98048184,55.07491335)(247.36670185,55.01918335)(245.63282187,55.69515334)
\curveto(244.52166189,56.12837333)(241.62516192,58.18667331)(240.36134193,59.4412533)
\curveto(240.06002194,59.74037329)(239.76296194,60.28935329)(239.70118194,60.66000328)
\curveto(239.63938194,61.03064328)(239.51630194,61.54079327)(239.42775195,61.79476327)
\curveto(239.31165195,62.12774326)(239.43935195,62.30551326)(239.88478194,62.43344326)
\curveto(240.78719193,62.69260326)(242.27345191,64.40284324)(242.27345191,65.18148323)
\curveto(242.27345191,66.06306322)(241.44191192,67.4034432)(240.26564194,68.41586319)
\curveto(239.73641194,68.87134318)(239.34419195,69.28990318)(239.39454195,69.34750318)
\curveto(239.44494195,69.40510318)(240.19759194,69.16520318)(241.06642193,68.81430318)
\curveto(243.1280619,67.98162319)(247.97206184,67.7502732)(248.58790184,68.45492319)
\curveto(248.80036184,68.69804319)(248.97267183,69.06209318)(248.97267183,69.26351318)
\curveto(248.97267183,69.49575318)(249.18931183,69.58147318)(249.56251183,69.49984318)
\curveto(250.37229182,69.32272318)(251.7031418,70.15613317)(251.7031418,70.83969316)
\curveto(251.7031418,71.38965315)(250.14720182,73.23874313)(249.13868183,73.88656313)
\curveto(248.33903184,74.40023312)(245.72573187,74.34234312)(245.04493188,73.79676313)
\curveto(244.12978189,73.06331314)(244.44667189,72.27757314)(246.05470187,71.29090316)
\curveto(246.83727186,70.81071316)(247.47836185,70.30602317)(247.48048185,70.16980317)
\curveto(247.48748185,69.73723317)(243.2062619,70.12070317)(242.39845191,70.62488316)
\curveto(241.33106192,71.29109316)(238.86638195,72.54747314)(238.62892195,72.54676314)
\curveto(238.51813196,72.54647314)(237.05684197,73.33096313)(235.38087199,74.29090312)
\curveto(233.03487202,75.6346031)(232.15001203,76.3030231)(231.53517204,77.19715309)
\curveto(231.09593204,77.83591308)(230.26163205,78.88240307)(229.68165206,79.52137306)
\curveto(229.06287207,80.20308305)(228.62697207,80.94267304)(228.62697207,81.31238304)
\curveto(228.62697207,82.05491303)(229.60615206,83.16888302)(230.95314204,83.95691301)
\curveto(231.48211204,84.266383)(232.18234203,84.848053)(232.50782203,85.24988299)
\curveto(233.06558202,85.93851298)(233.07723202,86.01996298)(232.70509202,86.66980298)
\curveto(232.48795203,87.04901297)(232.35155203,87.40659297)(232.40236203,87.46473297)
\closepath
\moveto(232.02538203,86.70299297)
\curveto(232.09348203,86.67539298)(232.02138203,86.57162298)(231.85351203,86.37096298)
\curveto(231.45981204,85.90075298)(228.72446207,84.187363)(228.36718208,84.187363)
\curveto(228.25153208,84.187363)(227.87154208,84.614373)(227.52343209,85.13658299)
\curveto(227.17532209,85.65880299)(226.89062209,86.14067298)(226.89062209,86.20689298)
\curveto(226.89062209,86.27309298)(227.70007208,86.32762298)(228.68945207,86.32799298)
\curveto(229.67881206,86.32826298)(230.87979205,86.43840298)(231.35741204,86.57213298)
\curveto(231.74943204,86.68189298)(231.95733203,86.73060297)(232.02538203,86.70299297)
\closepath
\moveto(221.81835215,86.58580298)
\lineto(222.37304215,85.56041299)
\curveto(222.67762214,84.99651299)(222.86253214,84.341893)(222.78320214,84.10533301)
\curveto(222.70390214,83.86877301)(222.85880214,83.12014302)(223.12890214,82.44127302)
\curveto(223.76311213,80.84745304)(223.60461213,79.64439306)(222.75976214,79.64439306)
\curveto(222.43011215,79.64439306)(221.77402215,79.83025306)(221.30077216,80.05650305)
\curveto(220.82753216,80.28277305)(220.43945217,80.58852305)(220.43945217,80.73619304)
\curveto(220.43945217,80.88385304)(220.64872217,81.81610303)(220.90429216,82.80846302)
\curveto(221.15985216,83.80082301)(221.46994216,85.05702299)(221.59374215,85.59947299)
\closepath
\moveto(235.55273199,85.60728299)
\curveto(236.28616198,85.60728299)(236.58549198,85.43420299)(237.01757197,84.765493)
\curveto(237.31617197,84.303393)(237.56054197,83.79142301)(237.56054197,83.62682301)
\curveto(237.56054197,83.46223301)(237.65614197,83.10497302)(237.77343196,82.83385302)
\curveto(237.89069196,82.56271302)(238.30862196,81.59445303)(238.70312195,80.68150305)
\curveto(239.38830195,79.09581306)(240.75558193,77.26914309)(242.49609191,75.61314311)
\curveto(243.5685519,74.59278312)(243.5209119,73.35045313)(242.39839191,73.07799313)
\curveto(241.17010192,72.77984314)(240.94204193,72.80978314)(240.14058194,73.36900313)
\curveto(239.45499195,73.84736313)(239.35674195,74.09240312)(239.21480195,75.6463531)
\curveto(239.12730195,76.60404309)(238.89772195,77.73650308)(238.70503195,78.16197308)
\curveto(238.17999196,79.32128306)(236.75659198,80.90952304)(235.61714199,81.60728303)
\curveto(235.34562199,81.77355303)(234.890022,82.34014303)(234.605422,82.86705302)
\curveto(234.16837201,83.67619301)(234.13008201,83.96417301)(234.359332,84.716663)
\curveto(234.597882,85.49963299)(234.742822,85.60728299)(235.55269199,85.60728299)
\closepath
\moveto(254.15820177,84.595573)
\curveto(254.39134177,84.591573)(254.42254177,84.04227301)(254.39062177,81.83580303)
\lineto(254.34762177,78.93541307)
\lineto(254.18942177,81.20689304)
\curveto(254.10242177,82.45607302)(253.94813177,83.68426301)(253.84567178,83.93736301)
\curveto(253.72363178,84.238823)(253.79177178,84.457083)(254.04488177,84.568223)
\curveto(254.08688177,84.586623)(254.12488177,84.596123)(254.15817177,84.595623)
\closepath
\moveto(219.76562218,80.91588304)
\curveto(219.79732218,80.96848304)(219.84582218,80.86968304)(219.92577217,80.63853305)
\curveto(220.03378217,80.32624305)(220.06979217,79.91185305)(220.00587217,79.71666306)
\curveto(219.80634218,79.10765306)(219.69476218,79.32354306)(219.71290218,80.28307305)
\curveto(219.71990218,80.65875305)(219.73400218,80.86328304)(219.76560218,80.91588304)
\closepath
\moveto(220.19140217,79.41783306)
\lineto(221.65234215,77.96861308)
\curveto(222.45553214,77.17243309)(223.22402214,76.52135309)(223.35937213,76.52135309)
\curveto(223.80517213,76.52135309)(224.28171212,75.60365311)(224.20116212,74.90025311)
\curveto(224.14456212,74.40603312)(224.26096212,74.16719312)(224.61132212,74.06236312)
\curveto(224.90963212,73.97316312)(225.15954211,73.58381313)(225.24999211,73.06627314)
\curveto(225.38423211,72.29836314)(225.46774211,72.23410314)(226.1210921,72.39830314)
\curveto(227.08233209,72.63987314)(227.21118209,72.33140314)(227.26366209,69.66588317)
\curveto(227.32066209,66.77259321)(227.79592208,65.59193322)(229.41601206,64.32603324)
\curveto(230.17893205,63.72990324)(230.53883205,63.29404325)(230.33202205,63.21861325)
\curveto(230.14466205,63.15031325)(229.90209206,62.92213325)(229.79296206,62.71080326)
\curveto(229.53211206,62.20569326)(229.12304207,62.21899326)(229.12304207,62.73230326)
\curveto(229.12304207,62.95520325)(228.87345207,63.37504325)(228.56640207,63.66590325)
\curveto(228.25936208,63.95676324)(227.88350208,64.44581324)(227.73241208,64.75184323)
\curveto(227.58133208,65.05785323)(227.24623209,65.35602323)(226.98827209,65.41395322)
\curveto(226.4186921,65.54185322)(225.8369921,66.26302321)(224.53320212,68.46082319)
\curveto(223.98733213,69.38097318)(222.78833214,70.98146316)(221.86718215,72.01746315)
\lineto(220.19140217,73.90223313)
\lineto(220.19140217,76.66004309)
\closepath
\moveto(234.673822,73.96666312)
\curveto(234.732122,73.96666312)(235.45414199,73.58353313)(236.27929198,73.11510313)
\curveto(237.10442197,72.64665314)(237.84896196,72.26353314)(237.93359196,72.26353314)
\curveto(238.17033196,72.26353314)(240.79239193,70.92128316)(241.87890192,70.24400317)
\curveto(242.55920191,69.81992317)(243.4655019,69.59573318)(244.91796188,69.49205318)
\curveto(246.53364186,69.37672318)(246.98827186,69.25168318)(246.98827186,68.91978318)
\curveto(246.98827186,68.28702319)(243.1111119,68.62007319)(241.03320193,69.43150318)
\curveto(240.14617194,69.77789317)(239.25351195,70.17094317)(239.04882195,70.30455317)
\curveto(238.84412195,70.43815317)(237.98753196,70.95263316)(237.14452197,71.44908315)
\curveto(236.30151198,71.94556315)(235.37675199,72.71597314)(235.089842,73.16002313)
\curveto(234.802922,73.60405313)(234.615512,73.96666312)(234.673822,73.96666312)
\closepath
\moveto(247.02148186,73.68346313)
\curveto(248.47694184,73.68346313)(249.28298183,73.25158313)(250.46288182,71.84166315)
\curveto(251.31405181,70.82451316)(251.32037181,70.79658316)(250.83398181,70.37877317)
\curveto(250.10865182,69.75575317)(248.94818183,70.06497317)(246.83398186,71.44908315)
\curveto(244.36995189,73.06223314)(244.42241189,73.68346313)(247.02148186,73.68346313)
\closepath
\moveto(220.05077217,73.25963313)
\curveto(221.95584215,71.36713315)(223.54302213,69.03376318)(225.8593721,65.70689322)
\curveto(223.80647213,65.65859322)(221.23756216,66.21750322)(220.86913216,66.59947321)
\curveto(220.50071217,66.98144321)(220.03802217,71.62169315)(220.05077217,73.25963313)
\closepath
\moveto(228.40820207,71.81822315)
\curveto(228.46050207,71.87492315)(229.34174206,71.03289316)(230.36523205,69.94713317)
\curveto(231.38871204,68.86135318)(232.91806202,67.5247232)(233.76366201,66.97838321)
\curveto(235.90966199,65.59185322)(238.49709196,62.75713326)(238.91601195,61.33385327)
\curveto(239.21225195,60.32733328)(239.13040195,59.85259329)(238.64843195,59.77525329)
\curveto(238.63353195,59.77225329)(237.44043197,61.08102328)(235.99609199,62.68150326)
\curveto(234.551732,64.28200324)(232.83369202,66.18268322)(232.17773203,66.90416321)
\curveto(230.68763205,68.54314319)(228.28093208,71.68032315)(228.40820207,71.81822315)
\closepath
\moveto(227.86132208,70.64049316)
\curveto(227.96602208,70.70189316)(228.13808208,70.52033316)(228.37304208,70.08580317)
\curveto(229.07276207,68.79176319)(231.79520204,65.53820322)(232.73046202,64.87682323)
\curveto(233.23633202,64.51910324)(233.58984201,64.04013324)(233.58984201,63.71471324)
\curveto(233.58984201,63.41024325)(233.78462201,63.08341325)(234.02343201,62.98814325)
\curveto(234.395732,62.83962326)(236.31835198,60.72221328)(236.31835198,60.46080328)
\curveto(236.31835198,60.36053328)(233.86465201,60.14200329)(233.46679202,60.20689329)
\curveto(233.33032202,60.22919329)(233.25349202,60.58262328)(233.29687202,60.99400328)
\curveto(233.38187202,61.79980327)(232.67261202,63.15523325)(232.15820203,63.16978325)
\curveto(231.69902204,63.18278325)(230.09037206,64.39534324)(229.07812207,65.49205322)
\curveto(228.34115208,66.29053321)(228.15313208,66.75218321)(227.90820208,68.37096319)
\curveto(227.69742208,69.76401317)(227.68682208,70.53817316)(227.86130208,70.64049316)
\closepath
\moveto(229.17187207,69.70885317)
\curveto(229.24007207,69.70885317)(229.49714206,69.45278318)(229.74413206,69.14049318)
\curveto(229.99113206,68.82820318)(230.13658205,68.57213319)(230.06835206,68.57213319)
\curveto(230.00015206,68.57213319)(229.74309206,68.82820318)(229.49609206,69.14049318)
\curveto(229.24909206,69.45278318)(229.10364207,69.70885317)(229.17187207,69.70885317)
\closepath
\moveto(247.84179185,69.70885317)
\curveto(248.25631184,69.70885317)(248.53623184,69.30387318)(248.30468184,69.03893318)
\curveto(248.00815184,68.69962319)(247.48437185,68.85544318)(247.48437185,69.28307318)
\curveto(247.48437185,69.51729318)(247.64514185,69.70885317)(247.84179185,69.70885317)
\closepath
\moveto(219.46288218,68.85728318)
\curveto(219.74648218,68.85728318)(220.23794217,65.92122322)(220.04882217,65.35728323)
\curveto(219.86514218,64.80956323)(219.71054218,64.78252323)(218.86523219,65.15025323)
\curveto(217.9027322,65.56897322)(217.8080422,66.44927321)(218.58984219,67.7401032)
\curveto(218.96171219,68.35409319)(219.35465218,68.85728318)(219.46288218,68.85728318)
\closepath
\moveto(252.0624918,68.29283319)
\curveto(251.6958718,68.31953319)(251.20680181,68.17738319)(250.57812181,67.8768232)
\curveto(249.60166183,67.4099932)(249.48633183,66.78232321)(250.19921182,65.80455322)
\curveto(250.77237181,65.01841323)(252.54936179,64.88872323)(252.79296179,65.61510322)
\curveto(252.87576179,65.86209322)(252.94335179,66.36159321)(252.94335179,66.72643321)
\curveto(252.94335179,67.7368932)(252.67353179,68.24832319)(252.0624918,68.29283319)
\closepath
\moveto(252.0858918,67.7205732)
\curveto(252.45452179,67.7205732)(252.60913179,66.29891321)(252.28120179,65.92369322)
\curveto(251.9021618,65.48997322)(250.72073181,65.74960322)(250.45112182,66.32603321)
\curveto(250.29880182,66.65170321)(250.21065182,66.94529321)(250.25581182,66.97838321)
\curveto(250.61751181,67.2432632)(251.7948718,67.7205732)(252.0858918,67.7205732)
\closepath
\moveto(221.02730216,65.93150322)
\curveto(221.08000216,65.99180322)(221.83315215,65.89010322)(222.70308214,65.70494322)
\curveto(223.57300213,65.51983322)(224.92384212,65.31126323)(225.70308211,65.24205323)
\curveto(226.87768209,65.13773323)(227.14830209,65.01375323)(227.28706209,64.51353324)
\curveto(227.37916209,64.18169324)(227.71817208,63.67124325)(228.04097208,63.38072325)
\curveto(229.19543207,62.34165326)(228.75418207,62.13594326)(225.27730211,62.08971326)
\curveto(223.24037214,62.06261326)(222.98477214,62.12211326)(222.35347215,62.75767326)
\curveto(221.73395215,63.38137325)(220.78453216,65.65372322)(221.02730216,65.93150322)
\closepath
\moveto(214.73237224,65.44908322)
\curveto(214.74517224,65.46378322)(214.84616223,65.37758323)(215.04292223,65.20103323)
\curveto(215.28173223,64.98674323)(215.47652223,64.76388323)(215.47652223,64.70494323)
\curveto(215.47652223,64.47128324)(215.27330223,64.62264323)(214.94917223,65.09557323)
\curveto(214.79486223,65.32071323)(214.71957224,65.43443322)(214.73237224,65.44908322)
\closepath
\moveto(252.36519179,64.29088324)
\curveto(253.77413178,64.32148324)(254.18355177,64.09258324)(254.18355177,63.27330325)
\curveto(254.18355177,62.56437326)(254.00158177,62.57362326)(253.19136178,63.32600325)
\curveto(252.98666179,63.51610325)(252.48357179,63.80634324)(252.0741718,63.97054324)
\lineto(251.33003181,64.26936324)
\closepath
\moveto(215.49019223,64.11900324)
\curveto(215.55139223,64.17780324)(215.92858222,63.76620324)(216.32808222,63.20494325)
\lineto(217.05464221,62.18541326)
\lineto(216.21675222,63.09752325)
\curveto(215.75604222,63.59998325)(215.42895223,64.06023324)(215.49019223,64.11900324)
\closepath
\moveto(234.19526201,64.02920324)
\curveto(234.25546201,64.02920324)(234.374592,63.90213324)(234.458942,63.74600324)
\curveto(234.543242,63.58985325)(234.494042,63.46279325)(234.34956201,63.46279325)
\curveto(234.20504201,63.46279325)(234.08589201,63.58985325)(234.08589201,63.74600324)
\curveto(234.08589201,63.90213324)(234.13509201,64.02920324)(234.19526201,64.02920324)
\closepath
\moveto(250.42769182,63.59756325)
\lineto(251.00386181,63.01553325)
\curveto(251.32035181,62.69470326)(251.9128218,62.19024326)(252.32222179,61.89639327)
\lineto(253.06636179,61.36318327)
\lineto(252.44722179,61.15811327)
\curveto(252.1060518,61.04502328)(251.5481418,60.97313328)(251.20698181,60.99990328)
\curveto(250.65687181,61.04310328)(250.57800181,61.19382327)(250.50776182,62.32412326)
\closepath
\moveto(251.34175181,63.50776325)
\curveto(251.5016018,63.51076325)(251.7944918,63.43636325)(252.1874518,63.25972325)
\curveto(253.34586178,62.73906326)(254.18355177,61.91375327)(254.18355177,61.29292327)
\curveto(254.18355177,60.97436328)(253.92979177,61.10424328)(253.28511178,61.75191327)
\curveto(252.79068179,62.24863326)(252.1211818,62.79374326)(251.7968318,62.96284325)
\curveto(251.17864181,63.28513325)(251.07533181,63.50341325)(251.34175181,63.50776325)
\closepath
\moveto(235.62300199,62.32612326)
\curveto(235.69120199,62.32612326)(235.94826199,62.07200326)(236.19526198,61.75972327)
\curveto(236.44226198,61.44741327)(236.58771198,61.19136327)(236.51948198,61.19136327)
\curveto(236.45128198,61.19136327)(236.19422198,61.44741327)(235.94722199,61.75972327)
\curveto(235.70022199,62.07200326)(235.55476199,62.32612326)(235.62300199,62.32612326)
\closepath
\moveto(248.23042184,62.26952326)
\curveto(247.72256185,62.17622326)(247.20614185,61.43501327)(247.28511185,60.80468328)
\curveto(247.38081185,60.04107329)(248.04515184,59.88545329)(248.58394184,60.50195328)
\curveto(249.08713183,61.07772328)(249.00342183,62.14594326)(248.44526184,62.26757326)
\curveto(248.37476184,62.28297326)(248.30297184,62.28287326)(248.23042184,62.26957326)
\closepath
\moveto(228.60152207,61.94726327)
\curveto(228.85520207,61.94526327)(229.05703207,61.87476327)(229.28316206,61.73632327)
\curveto(230.14118205,61.21090327)(230.25701205,60.42720328)(229.57612206,59.75195329)
\curveto(229.05776207,59.2378833)(228.77304207,59.1962133)(226.76753209,59.3476533)
\curveto(225.53935211,59.4403533)(224.49661212,59.63679329)(224.45112212,59.78515329)
\curveto(224.40562212,59.93352329)(224.25058212,60.05468329)(224.10737213,60.05468329)
\curveto(223.80642213,60.05468329)(222.92183214,61.15729327)(222.92183214,61.53124327)
\curveto(222.92183214,61.67221327)(223.71559213,61.70113327)(224.72066212,61.59764327)
\curveto(225.8496321,61.48138327)(226.92618209,61.53744327)(227.61323208,61.74998327)
\curveto(228.04306208,61.88295327)(228.34783208,61.94938327)(228.60152207,61.94725327)
\closepath
\moveto(248.34956184,61.65038327)
\curveto(248.38056184,61.65538327)(248.40676184,61.65037327)(248.42376184,61.63088327)
\curveto(248.49196184,61.55278327)(248.40376184,61.29062327)(248.22845184,61.04885328)
\curveto(248.05309184,60.80709328)(247.85332185,60.67390328)(247.78509185,60.75198328)
\curveto(247.71689185,60.83008328)(247.80509185,61.09224328)(247.98040184,61.33401327)
\curveto(248.11192184,61.51533327)(248.25648184,61.63435327)(248.34954184,61.65041327)
\closepath
\moveto(217.5644122,61.45507327)
\curveto(217.6436122,61.41457327)(217.8317122,61.16410327)(218.0800322,60.73046328)
\curveto(218.31458219,60.32087328)(218.38423219,60.07173329)(218.23433219,60.17773329)
\curveto(218.0844522,60.28373329)(217.8450822,60.62022328)(217.7030822,60.92382328)
\curveto(217.5145622,61.32687327)(217.4851622,61.49560327)(217.5644122,61.45507327)
\closepath
\moveto(236.56636198,59.62890329)
\lineto(236.56636198,58.47070331)
\curveto(236.56650198,57.83311331)(236.49326198,57.22517332)(236.40230198,57.12109332)
\curveto(236.19718198,56.88639333)(234.405942,56.88021333)(233.65230201,57.11109332)
\curveto(233.19487202,57.25136332)(233.09370202,57.48120332)(233.09370202,58.38453331)
\curveto(233.09370202,59.59492329)(233.04250202,59.56870329)(235.51362199,59.61109329)
\closepath
\moveto(253.25972178,58.8847633)
\curveto(253.51984178,58.8797633)(253.68516178,58.8369633)(253.68355178,58.7460933)
\curveto(253.67555178,58.29314331)(251.7320718,56.63191333)(250.40034182,55.93945334)
\curveto(248.80426184,55.10951335)(248.47652184,55.07965335)(248.47652184,55.75976334)
\curveto(248.47652184,56.46848333)(249.34999183,57.69922332)(250.21480182,58.21093331)
\curveto(250.83652181,58.57881331)(252.47935179,58.9001933)(253.25972178,58.8847633)
\closepath
\moveto(219.19917218,58.6367133)
\curveto(219.21197218,58.6514133)(219.31296218,58.56521331)(219.50972218,58.38866331)
\curveto(219.74853218,58.17437331)(219.94331217,57.95152331)(219.94331217,57.89257331)
\curveto(219.94331217,57.65891332)(219.74010218,57.80837331)(219.41597218,58.28124331)
\curveto(219.26166218,58.50638331)(219.18636218,58.6220633)(219.19917218,58.6367133)
\closepath
\moveto(254.18355177,57.56640332)
\lineto(254.18355177,56.84570333)
\curveto(254.18355177,55.43333334)(253.97361177,55.22851334)(252.52925179,55.22851334)
\curveto(251.7515818,55.22851334)(251.23768181,55.34322334)(251.32026181,55.49609334)
\curveto(251.3992618,55.64235334)(251.7699618,55.93667334)(252.1425318,56.15038333)
\curveto(252.51509179,56.36411333)(253.12663178,56.77015333)(253.50191178,57.05273332)
\closepath
\moveto(250.33784182,55.22851334)
\curveto(250.54254181,55.22851334)(250.71089181,55.10145335)(250.71089181,54.94530335)
\curveto(250.71089181,54.78915335)(250.54254181,54.66210335)(250.33784182,54.66210335)
\curveto(250.13314182,54.66210335)(249.96480182,54.78915335)(249.96480182,54.94530335)
\curveto(249.96480182,55.10145335)(250.13314182,55.22851334)(250.33784182,55.22851334)
\closepath
\moveto(228.40034207,55.19331334)
\curveto(228.48654207,55.19731334)(228.58422207,55.17961335)(228.67378207,55.13861335)
\curveto(228.87172207,55.04801335)(228.81244207,54.98231335)(228.52339207,54.96869335)
\curveto(228.26182208,54.95639335)(228.11673208,55.02279335)(228.19917208,55.11713335)
\curveto(228.24037208,55.16433335)(228.31415208,55.18923335)(228.40034207,55.19333334)
\closepath
\moveto(226.5194821,54.33003336)
\curveto(228.34654208,54.29873336)(228.54826207,54.25623336)(227.63472208,54.09370336)
\curveto(225.83324211,53.77323336)(223.48063213,53.77323336)(223.91402213,54.09370336)
\curveto(224.11872213,54.24507336)(225.29130211,54.35111335)(226.5194821,54.33003336)
\closepath
}
}
{
\newrgbcolor{curcolor}{0 0 0}
\pscustom[linestyle=none,fillstyle=solid,fillcolor=curcolor]
{
\newpath
\moveto(258.15234508,115.84161263)
\lineto(256.6640651,114.99005264)
\curveto(254.08991513,113.51735266)(254.30705513,116.47651263)(254.30663513,82.98028302)
\lineto(254.30663513,53.81036336)
\lineto(255.11327512,53.81236336)
\curveto(257.26011509,53.82036336)(262.74408503,54.69887335)(263.93945501,55.22643334)
\curveto(265.121185,55.74795334)(266.37318498,56.51653333)(266.71288498,56.92760332)
\curveto(266.78108498,57.01020332)(267.45874497,57.60728332)(268.21874496,58.25377331)
\curveto(269.21061495,59.0974833)(269.77019494,59.3912433)(270.20312494,59.2967433)
\curveto(270.75814493,59.1756033)(270.80050493,59.0558533)(270.72265493,57.78893331)
\curveto(270.64955493,56.59837333)(270.71665493,56.35500333)(271.21874493,55.99010334)
\curveto(272.02799492,55.40213334)(275.54663488,54.42693335)(279.98632482,53.56236336)
\lineto(279.98632482,53.56436336)
\curveto(280.66864482,53.43148337)(282.56567479,53.31365337)(284.20116478,53.30459337)
\curveto(287.12832474,53.28839337)(287.19828474,53.30432337)(288.66991472,54.24600336)
\curveto(290.5122647,55.42489334)(292.85953467,58.17051331)(294.11913466,60.62295328)
\curveto(294.64043465,61.63790327)(295.49534464,63.24106325)(296.01952464,64.18350324)
\curveto(297.03011462,66.00043322)(297.89092461,69.64406318)(297.65038462,71.08389316)
\curveto(297.58198462,71.49344315)(297.66988462,72.14970315)(297.84570462,72.54287314)
\curveto(298.02149461,72.93603314)(298.40243461,74.24898312)(298.69140461,75.46084311)
\curveto(299.3381146,78.17312308)(299.3979146,82.89409302)(298.8320246,86.45889298)
\curveto(298.62133461,87.78614296)(298.34481461,89.57469294)(298.21679461,90.43350293)
\curveto(297.95504461,92.18960291)(297.94734461,92.21120291)(295.59179464,97.63857285)
\curveto(293.33815467,102.83111279)(292.09155468,104.82828276)(289.83984471,106.85732274)
\curveto(288.64919472,107.93025273)(285.24007476,110.0233427)(284.68163477,110.0233427)
\curveto(284.48039477,110.0233427)(283.66678478,110.3961027)(282.87499479,110.85342269)
\curveto(280.59061482,112.17289268)(276.34779487,113.64138266)(270.80663493,115.02920264)
\curveto(268.20188496,115.68160263)(267.22297497,115.77893263)(262.99023502,115.80850263)
\closepath
\moveto(260.14843506,115.22052264)
\curveto(260.82914505,115.22752264)(261.70648504,115.22116264)(262.86718503,115.21052264)
\curveto(267.09503498,115.17262264)(267.92764497,115.07969264)(270.93163493,114.32380265)
\curveto(276.38491487,112.95159267)(279.13840483,112.07077268)(281.36327481,110.98200269)
\curveto(282.5172848,110.4172927)(284.02339478,109.7294627)(284.71093477,109.45466271)
\curveto(285.39846476,109.17987271)(286.52653475,108.55725272)(287.21874474,108.06989272)
\curveto(287.91095473,107.58256273)(288.57784472,107.17984273)(288.69921472,107.17732273)
\curveto(288.94109472,107.17232273)(291.0322347,104.93203276)(292.02734468,103.61286278)
\curveto(292.37213468,103.15581278)(293.03305467,101.9520328)(293.49609467,100.93708281)
\curveto(293.95913466,99.92214282)(294.57659465,98.64375283)(294.86718465,98.09724284)
\curveto(295.15779465,97.55072285)(295.70371464,96.27431286)(296.08007464,95.25935287)
\curveto(296.45644463,94.24441289)(296.93090463,93.0642229)(297.13476462,92.63630291)
\curveto(297.33861462,92.20840291)(297.61523462,91.12159292)(297.74999462,90.22224293)
\curveto(297.88476461,89.32290294)(298.15644461,87.50065297)(298.35351461,86.17341298)
\curveto(299.1080346,81.09166304)(298.68068461,75.8172931)(297.25390462,72.58943314)
\curveto(296.91732463,71.82799315)(296.86536463,71.43085315)(297.05468462,71.09333316)
\curveto(297.49352462,70.31093317)(296.42169463,65.74594322)(295.41796464,64.12068324)
\curveto(294.95343465,63.36849325)(294.09757466,61.82516327)(293.51562467,60.69099328)
\curveto(291.79584469,57.33938332)(288.81831472,54.40917335)(286.87304474,54.15583336)
\curveto(286.29307475,54.08023336)(285.81835476,54.09033336)(285.81835476,54.17733336)
\curveto(285.81835476,54.37015335)(287.29058474,55.22811334)(287.62109474,55.22811334)
\curveto(287.96550473,55.22811334)(290.5555147,58.37006331)(291.40234469,59.81405329)
\curveto(291.80041469,60.49285328)(292.66128468,61.71260327)(293.31445467,62.52303326)
\curveto(293.96761466,63.33344325)(294.50195465,64.14797324)(294.50195465,64.33358324)
\curveto(294.50195465,64.51919324)(294.89345465,65.16615323)(295.37109464,65.77108322)
\curveto(296.25456463,66.88999321)(296.51641463,68.00545319)(295.89648464,68.00545319)
\curveto(295.70823464,68.00545319)(295.12511465,67.3975332)(294.60156465,66.65584321)
\curveto(294.07801466,65.91415322)(293.59044467,65.24368323)(293.51757467,65.16561323)
\curveto(293.44467467,65.08751323)(292.99520467,64.35226324)(292.51757468,63.53280325)
\curveto(292.03995468,62.71332326)(291.56721469,62.04284326)(291.46679469,62.04256326)
\curveto(291.36639469,62.04227326)(290.9772747,61.49915327)(290.6035147,60.83553328)
\curveto(288.57479472,57.23361332)(286.29894475,55.21338334)(284.30468477,55.24178334)
\curveto(282.65466479,55.26528334)(279.27659483,55.96491334)(279.17773483,56.30428333)
\curveto(279.12273484,56.49321333)(278.91138484,56.64803333)(278.70898484,56.64803333)
\curveto(278.45661484,56.64803333)(278.36607484,56.41190333)(278.41991484,55.89608334)
\curveto(278.49751484,55.15323335)(278.64711484,55.00112335)(279.86327483,54.42537335)
\curveto(280.78043482,53.99117336)(279.85287483,54.11639336)(277.13476486,54.79256335)
\curveto(276.24772487,55.01321335)(274.90679488,55.34533334)(274.15624489,55.53084334)
\curveto(273.4056849,55.71636334)(272.42789491,56.03706334)(271.98437492,56.24373333)
\curveto(271.28188493,56.57111333)(271.17968493,56.75313333)(271.17968493,57.65584332)
\curveto(271.17968493,58.6016833)(271.76054492,60.47417328)(272.13866492,60.74373328)
\curveto(272.22436492,60.80483328)(272.25693492,60.03019329)(272.21096492,59.0230333)
\curveto(272.09663492,56.52045333)(272.30896492,56.27779333)(274.47073489,56.41756333)
\curveto(275.60931488,56.49116333)(276.24655487,56.66462333)(276.40041487,56.94295332)
\curveto(276.52677487,57.17154332)(276.63082486,58.13431331)(276.63283486,59.0835833)
\curveto(276.63683486,60.68964328)(276.67583486,60.80768328)(277.19533486,60.78670328)
\curveto(278.00371485,60.75410328)(277.95361485,61.81483327)(277.13483486,62.06209326)
\lineto(276.51373487,62.24959326)
\lineto(277.13483486,62.53670326)
\curveto(277.57543485,62.74113326)(277.78362485,62.73301326)(277.85553485,62.50550326)
\curveto(277.91123485,62.32932326)(278.14422485,62.03333326)(278.37311484,61.84730327)
\curveto(278.65625484,61.61715327)(278.75374484,61.23656327)(278.67975484,60.65980328)
\curveto(278.58245484,59.90140329)(278.68275484,59.74760329)(279.62311483,59.1988633)
\curveto(280.41733482,58.7352533)(280.95230481,58.6210933)(281.8184248,58.7301133)
\curveto(285.13278476,59.1473033)(286.56256475,59.87667329)(286.56256475,61.15199327)
\curveto(286.56256475,61.66664327)(286.83427474,62.14755326)(287.43170474,62.69300326)
\curveto(288.07222473,63.27776325)(288.29889473,63.70655324)(288.29889473,64.32581324)
\curveto(288.29889473,64.78759323)(288.41048473,65.16566323)(288.54694472,65.16566323)
\curveto(288.68341472,65.16566323)(288.79498472,65.03386323)(288.79498472,64.87269323)
\curveto(288.79498472,64.19969324)(289.89626471,64.10548324)(290.7188147,64.70863323)
\curveto(292.12581468,65.74029322)(292.04186468,67.00678321)(290.4356147,68.97230318)
\lineto(289.71881471,69.84925317)
\lineto(289.99616471,73.11683313)
\curveto(290.6978147,81.34472304)(288.83184472,90.43634293)(285.22077476,96.38050286)
\curveto(284.23165478,98.00869284)(281.7724648,100.86480281)(281.04694481,101.2281628)
\curveto(280.73735482,101.3832028)(280.03616482,101.8322328)(279.49030483,102.22425279)
\curveto(278.94443484,102.61625279)(278.33172484,103.00926278)(278.12702485,103.09925278)
\curveto(277.92232485,103.18925278)(276.80454486,103.71691277)(275.64459488,104.27113277)
\curveto(273.8411549,105.13279276)(271.83504492,105.84144275)(267.45709497,107.16175273)
\curveto(266.97945498,107.30579273)(265.73339499,107.68912273)(264.687565,108.01331272)
\curveto(262.13101503,108.80586272)(261.22451505,108.77493272)(260.28131506,107.86683273)
\curveto(259.86144506,107.46257273)(259.51764507,106.95422274)(259.51764507,106.73597274)
\curveto(259.51764507,106.25446275)(258.24541508,105.61369275)(256.7637351,105.34925276)
\curveto(256.1635351,105.24216276)(255.46515511,105.08756276)(255.21295512,105.00550276)
\curveto(254.77978512,104.86458276)(254.76026512,105.07580276)(254.84186512,108.81605272)
\curveto(254.93956512,113.29612266)(254.95878512,113.35230266)(256.8965551,114.50550265)
\curveto(257.82440509,115.05771264)(258.10638508,115.20040264)(260.14850506,115.22035264)
\closepath
\moveto(261.84570504,108.01739272)
\curveto(262.38183503,108.05229272)(263.04244502,107.91073273)(263.89257501,107.61309273)
\curveto(264.55645501,107.38068273)(265.99420499,106.94741274)(267.08593498,106.65020274)
\curveto(273.7233149,104.84322276)(278.29471484,102.75490279)(281.6953048,99.97833282)
\curveto(282.59871479,99.24071283)(283.51808478,98.22784284)(283.73827478,97.72833285)
\curveto(283.95845478,97.22884285)(284.39126477,96.50150286)(284.70116477,96.11114286)
\curveto(285.39012476,95.24336287)(287.31219474,91.17742292)(287.58593474,90.00762294)
\curveto(287.69554473,89.53919294)(288.06787473,88.00473296)(288.41406473,86.59942298)
\curveto(288.76023472,85.19411299)(289.09888472,83.47005301)(289.16796472,82.76739302)
\curveto(289.29325472,81.49325304)(289.29349472,81.49432304)(288.80077472,83.19317302)
\curveto(288.52906472,84.13005301)(288.30474473,85.12052299)(288.30273473,85.39434299)
\curveto(288.30073473,85.66816299)(288.07567473,86.37026298)(287.80273473,86.95294297)
\curveto(287.52980474,87.53560297)(287.30663474,88.23170296)(287.30663474,88.50177295)
\curveto(287.30663474,89.06815295)(286.53595475,89.63474294)(286.15234475,89.34942294)
\curveto(285.81833476,89.10098295)(285.32454476,88.02573296)(285.06054477,86.97247297)
\curveto(284.82143477,86.01848298)(285.31579476,85.58004299)(285.95116475,86.18341298)
\curveto(286.43570475,86.64351298)(286.72743475,86.31671298)(286.32616475,85.76348299)
\curveto(286.15710475,85.53041299)(285.97083475,84.752463)(285.91210476,84.03302301)
\curveto(285.85340476,83.31357301)(285.63500476,82.42816303)(285.42773476,82.06622303)
\curveto(285.01045477,81.33751304)(285.15186476,80.09911305)(285.74023476,79.32208306)
\curveto(286.09124475,78.85850307)(287.22425474,78.94739307)(288.08398473,79.50567306)
\curveto(288.27065473,79.62690306)(287.92479473,79.13323306)(287.31640474,78.40802307)
\curveto(286.70802475,77.68280308)(286.08169475,77.08966309)(285.92382476,77.08966309)
\curveto(285.35327476,77.08966309)(284.28714477,75.8854031)(284.40429477,75.37286311)
\curveto(284.47879477,75.04694311)(284.35459477,74.81866311)(284.05273478,74.72833312)
\curveto(283.75315478,74.63873312)(283.58398478,74.33463312)(283.58398478,73.88262313)
\curveto(283.58398478,73.28835313)(283.43829478,73.14882313)(282.65429479,72.99981314)
\curveto(282.1425548,72.90251314)(281.48828481,72.82164314)(281.19921481,72.82012314)
\curveto(280.87697481,72.81912314)(279.91298483,71.95624315)(278.70898484,70.58966316)
\curveto(277.62657485,69.36110318)(276.16039487,68.02217319)(275.44140488,67.6052832)
\curveto(273.9106949,66.71776321)(272.35022491,65.39471323)(271.11327493,63.93731324)
\curveto(270.62640493,63.36365325)(270.07844494,62.89760325)(269.89648494,62.90216325)
\curveto(269.02089495,62.92506325)(267.45702497,64.49705324)(267.45702497,65.35333323)
\curveto(267.45702497,66.01780322)(269.68355495,68.78122319)(270.79296493,69.49395318)
\curveto(272.40471491,70.52940316)(273.9013649,71.78364315)(274.03320489,72.20880315)
\curveto(274.11530489,72.47347314)(274.89256488,73.32845313)(275.76171487,74.10919312)
\curveto(276.63088486,74.88992311)(277.56807485,75.9747931)(277.84374485,76.52130309)
\curveto(278.11941485,77.06780309)(278.97193484,78.19837307)(279.73827483,79.03302306)
\curveto(281.56042481,81.01762304)(281.55606481,81.78798303)(279.71487483,83.46270301)
\curveto(279.01943484,84.09524301)(278.08367485,84.840833)(277.63479485,85.11895299)
\curveto(277.18590486,85.39707299)(276.88875486,85.70428299)(276.97463486,85.80255299)
\curveto(277.06053486,85.90085298)(277.32894486,85.77525299)(277.57229485,85.52325299)
\curveto(277.81565485,85.27127299)(278.66617484,84.724763)(279.46096483,84.308413)
\lineto(280.90627481,83.55059301)
\lineto(281.55666481,84.10723301)
\curveto(282.76749479,85.14161299)(283.44185478,87.15763297)(283.54104478,90.03497294)
\curveto(283.62414478,92.44534291)(283.58114478,92.7299029)(282.88479479,94.30841289)
\curveto(281.9522148,96.42225286)(280.17719482,98.93738283)(278.62893484,100.33770281)
\curveto(277.36651486,101.4795228)(274.70656489,102.63813279)(270.80666493,103.74591277)
\curveto(265.51977499,105.24766276)(263.85930501,105.29918276)(261.51760504,104.03106277)
\curveto(260.62241505,103.54628278)(259.55395507,102.97247278)(259.14455507,102.75567279)
\curveto(258.73516507,102.53888279)(257.59096509,102.27717279)(256.6015951,102.17559279)
\lineto(254.80276512,101.9920028)
\lineto(254.80276512,103.15411278)
\curveto(254.80276512,104.27803277)(254.83556512,104.32379277)(255.73440511,104.49395277)
\curveto(257.90911508,104.90568276)(259.63510506,105.63858275)(259.90823506,106.26544275)
\curveto(260.40632505,107.40859273)(260.95216505,107.95915273)(261.84573504,108.01739272)
\closepath
\moveto(265.074215,104.31231277)
\curveto(266.67543498,104.28401277)(266.74495498,104.26111277)(265.82616499,104.05255277)
\curveto(264.899615,103.84223277)(262.57698503,102.56686279)(261.38476504,101.6130928)
\curveto(261.11558505,101.3977628)(260.75622505,101.2205228)(260.58593505,101.2205228)
\curveto(260.41564505,101.2205228)(259.66646506,100.90227281)(258.91991507,100.51153281)
\curveto(258.17337508,100.12081282)(257.24856509,99.65862282)(256.8652351,99.48614282)
\curveto(256.4818851,99.31367283)(255.86160511,99.06782283)(255.48632511,98.93927283)
\curveto(254.80853512,98.70711283)(254.80273512,98.71538283)(254.80273512,100.06427282)
\lineto(254.80273512,101.4216928)
\lineto(256.6601551,101.6013828)
\curveto(258.13958508,101.7444828)(259.00527507,102.04169279)(260.91406505,103.06231278)
\curveto(263.08862502,104.22502277)(263.47497502,104.34057277)(265.074215,104.31231277)
\closepath
\moveto(268.57421496,103.45098278)
\lineto(266.83593498,103.21466278)
\curveto(265.88068499,103.08419278)(264.821615,102.89195278)(264.48046501,102.78888279)
\curveto(263.91178501,102.61706279)(263.89955501,102.63034279)(264.34374501,102.95098278)
\curveto(264.61024501,103.14336278)(265.67123499,103.33562278)(266.70116498,103.37677278)
\closepath
\moveto(269.66991495,103.17364278)
\curveto(269.75611495,103.16964278)(269.82986494,103.14264278)(269.87109494,103.09554278)
\curveto(269.95349494,103.00124278)(269.80649494,102.93476278)(269.54491495,102.94710278)
\curveto(269.25588495,102.96070278)(269.19854495,103.02840278)(269.39648495,103.11897278)
\curveto(269.48608495,103.15997278)(269.58372495,103.17777278)(269.66991495,103.17367278)
\closepath
\moveto(270.49023494,102.90802278)
\curveto(270.67372493,102.92192278)(271.25812493,102.72750279)(272.07812492,102.31427279)
\curveto(275.92933487,100.37346281)(279.36718483,97.47328285)(279.36718483,96.16388286)
\curveto(279.36718483,95.88124287)(279.05887484,95.32423287)(278.68359484,94.92755288)
\curveto(278.30832484,94.53085288)(277.77878485,94.12460289)(277.50585485,94.02325289)
\curveto(277.14731486,93.89009289)(277.19840486,94.00355289)(277.69140485,94.43145288)
\curveto(278.06668485,94.75716288)(278.37499484,95.24718287)(278.37499484,95.52130287)
\curveto(278.37499484,96.26497286)(277.04071486,97.91149284)(275.44530488,99.13653283)
\curveto(274.66833489,99.73313282)(273.6965049,100.52861281)(273.2871049,100.90606281)
\curveto(272.87771491,101.2835128)(271.98505492,101.8910528)(271.30273493,102.25567279)
\curveto(270.52361494,102.67200279)(270.30674494,102.89413278)(270.49023494,102.90802278)
\closepath
\moveto(273.2246049,102.50372279)
\curveto(273.2336049,102.50572279)(273.2530049,102.49972279)(273.2871049,102.48612279)
\curveto(276.22064487,101.3055028)(277.07775486,100.83094281)(278.43749484,99.63651282)
\curveto(281.09639481,97.30087285)(282.85211479,94.09325289)(283.03320479,91.23807292)
\curveto(283.21366479,88.39266295)(282.98640479,87.42299297)(282.0527348,87.05057297)
\curveto(280.99302481,86.62788298)(279.12582484,86.67923298)(278.37499484,87.15214297)
\curveto(277.90473485,87.44832297)(277.72491485,87.83898296)(277.63085485,88.76346295)
\curveto(277.51318485,89.92021294)(277.42242486,90.03783294)(275.89257487,91.04471292)
\curveto(275.00555488,91.62851292)(274.0013149,92.12849291)(273.6601549,92.15604291)
\curveto(273.3189949,92.18354291)(273.7104549,92.31697291)(274.52929489,92.45292291)
\curveto(276.74630486,92.8209829)(278.26150485,93.54301289)(279.24804483,94.69901288)
\curveto(280.02324482,95.60738287)(280.10004482,95.82257287)(279.88280483,96.47636286)
\curveto(279.38380483,97.97809284)(277.62337485,99.66158282)(274.52929489,101.5974528)
\curveto(273.6932449,102.12053279)(273.16172491,102.49269279)(273.2246049,102.50370279)
\closepath
\moveto(267.13866498,102.07012279)
\curveto(268.50184496,102.06512279)(270.53254494,101.3826428)(271.74023492,100.52325281)
\curveto(274.57161489,98.50841284)(277.13476486,96.23047286)(277.13476486,95.72833287)
\curveto(277.13476486,94.58996288)(271.05004493,93.94267289)(269.00390495,94.86309288)
\curveto(268.25183496,95.20141287)(267.20421498,95.39050287)(266.09179499,95.38848287)
\curveto(264.62074501,95.38648287)(264.20388501,95.26917287)(263.36327502,94.62286288)
\curveto(261.51559504,93.2022429)(261.25390505,92.9624229)(261.25390505,92.6912229)
\curveto(261.25390505,92.25188291)(261.99653504,92.38058291)(262.70507503,92.9431729)
\curveto(264.703685,94.53007288)(265.69383499,94.69823288)(268.20116496,93.87677289)
\lineto(269.81445494,93.3494229)
\lineto(266.83788498,93.2185629)
\curveto(264.690635,93.1244629)(263.65205502,92.9548829)(263.11523502,92.61309291)
\curveto(262.31116503,92.10114291)(259.54020507,91.98127291)(258.64843508,92.41973291)
\curveto(258.15713508,92.6612929)(258.15856508,92.6636429)(258.66213508,92.6834129)
\curveto(258.94190507,92.6944129)(259.27664507,92.8156529)(259.40627507,92.9529429)
\curveto(259.53592507,93.0902329)(260.06083506,93.3557729)(260.57229505,93.54278289)
\curveto(261.08375505,93.72979289)(261.97580504,94.42697288)(262.55666503,95.09161288)
\curveto(263.99272501,96.73478286)(264.49637501,96.99475285)(266.95119498,97.36700285)
\curveto(268.52367496,97.60545285)(269.07033495,97.79733284)(269.07033495,98.10723284)
\curveto(269.07033495,98.43643284)(268.72836496,98.51461284)(267.45705497,98.48028284)
\curveto(265.184535,98.41898284)(263.29909502,97.53692285)(261.75979504,95.81231287)
\curveto(261.07060505,95.04016288)(260.28863506,94.40802288)(260.02151506,94.40802288)
\curveto(259.75439506,94.40802288)(258.91782507,94.14363289)(258.16213508,93.82012289)
\curveto(256.4325351,93.0796629)(256.3387651,92.43538291)(257.84182509,91.60528292)
\curveto(259.62487506,90.62057293)(261.33543504,90.31885293)(262.20901503,90.83575293)
\curveto(263.79568502,91.77458292)(265.429585,92.23579291)(267.39651497,92.30059291)
\curveto(268.52273496,92.33769291)(269.38584495,92.26309291)(269.31643495,92.13458291)
\curveto(269.17548495,91.87363291)(266.44523498,90.68805293)(264.728545,90.14239293)
\curveto(262.87756503,89.55405294)(260.05214506,89.54191294)(258.77346507,90.11699293)
\curveto(255.59639511,91.54589292)(254.80276512,92.58546291)(254.80276512,95.32207287)
\lineto(254.80276512,97.17558285)
\lineto(256.10549511,97.30644285)
\curveto(256.8219351,97.37764285)(258.05310508,97.77655284)(258.84182507,98.19316284)
\curveto(259.63052506,98.60978283)(260.37351506,98.94902283)(260.49221505,98.94902283)
\curveto(260.61093505,98.94902283)(261.29646504,99.40306283)(262.01565504,99.95683282)
\curveto(263.55287502,101.14049281)(265.81693499,102.07466279)(267.13869498,102.07011279)
\closepath
\moveto(260.01366506,100.25958282)
\curveto(260.16093506,100.27278282)(260.00366506,100.14072282)(259.51757507,99.82403282)
\curveto(259.03994507,99.51292282)(258.54092508,99.25248283)(258.40820508,99.24591283)
\curveto(258.00569508,99.22611283)(259.10188507,99.99577282)(259.76562506,100.19903282)
\curveto(259.88138506,100.23443282)(259.96458506,100.25513282)(260.01366506,100.25963282)
\closepath
\moveto(257.70507509,99.08184283)
\curveto(257.77057509,99.06214283)(257.56955509,98.91258283)(257.16015509,98.68731283)
\curveto(256.6142851,98.38695284)(255.99961511,98.13613284)(255.79491511,98.13067284)
\curveto(255.36965511,98.11927284)(256.6404751,98.78820283)(257.54882509,99.05450283)
\curveto(257.63192509,99.07890283)(257.68324509,99.08840283)(257.70507509,99.08180283)
\closepath
\moveto(276.39257487,93.67169289)
\curveto(276.63105486,93.65419289)(276.71521486,93.60639289)(276.61132486,93.53106289)
\curveto(276.28936487,93.2977929)(273.2071549,93.1481329)(273.4335949,93.3767729)
\curveto(273.4904949,93.4341729)(274.30571489,93.54815289)(275.24609488,93.63067289)
\curveto(275.76073487,93.67587289)(276.15410487,93.68917289)(276.39257487,93.67167289)
\closepath
\moveto(254.87109512,92.00567291)
\curveto(254.92509512,92.06147291)(255.01839512,92.00067291)(255.17773512,91.84942291)
\curveto(255.46314511,91.57839292)(255.48563511,91.40054292)(255.27143512,91.10528292)
\curveto(254.87513512,90.55892293)(254.80268512,90.61322293)(254.80268512,91.46075292)
\curveto(254.80268512,91.77690292)(254.81708512,91.94984291)(254.87108512,92.00567291)
\closepath
\moveto(270.80663493,91.87872291)
\lineto(272.23437492,91.74395292)
\curveto(273.6183249,91.61302292)(275.74790487,90.68308293)(276.76171486,89.76544294)
\curveto(277.24310486,89.32984294)(277.23576486,89.31575294)(276.51366487,89.32794294)
\curveto(275.64004488,89.34264294)(273.2114949,90.32614293)(271.79882492,91.23809292)
\closepath
\moveto(269.96288494,91.68341292)
\lineto(271.74999492,90.64630293)
\curveto(272.73241491,90.07528294)(274.23322489,89.37787294)(275.08593488,89.09747295)
\curveto(275.93862487,88.81706295)(276.59931486,88.46239295)(276.55468487,88.30841296)
\curveto(276.34334487,87.57958296)(272.45774491,86.18714298)(269.43945495,85.75958299)
\curveto(265.083645,85.14255299)(263.45095502,85.27051299)(263.28710502,86.24395298)
\curveto(263.24350502,86.50317298)(262.99953502,87.20200297)(262.74609503,87.79473296)
\curveto(262.28564503,88.87159295)(262.28580503,88.87195295)(262.82419503,88.95880295)
\curveto(263.12055502,89.00660295)(264.22399501,89.35844294)(265.275365,89.74005294)
\curveto(266.32673499,90.12165293)(267.31858497,90.43341293)(267.48044497,90.43341293)
\curveto(267.64232497,90.43341293)(268.26722496,90.71446293)(268.86911496,91.05841292)
\closepath
\moveto(255.64062511,90.62091293)
\curveto(255.74682511,90.65961293)(255.85761511,90.57781293)(255.93945511,90.33380293)
\curveto(256.01685511,90.10288293)(256.04427511,89.60344294)(255.99995511,89.22442295)
\curveto(255.88848511,88.27060296)(255.29878511,88.57597295)(255.29878511,89.58770294)
\curveto(255.29878511,90.15301293)(255.46357511,90.55641293)(255.64058511,90.62091293)
\closepath
\moveto(256.7441351,90.43341293)
\curveto(256.8711051,90.43341293)(257.59121509,90.11989293)(258.34570508,89.73809294)
\curveto(259.49675507,89.15563295)(259.76710506,88.87694295)(260.01952506,88.00177296)
\curveto(260.26559506,87.14874297)(260.25783506,86.91376297)(259.98042506,86.71270297)
\curveto(259.45262507,86.33011298)(258.73661507,86.46292298)(257.65620509,87.14434297)
\curveto(256.7426551,87.72052296)(256.6574651,87.87607296)(256.5878451,89.10138295)
\curveto(256.5462451,89.83350294)(256.6171451,90.43341293)(256.7440951,90.43341293)
\closepath
\moveto(260.33984506,88.91192295)
\curveto(260.40454506,88.98602295)(260.86958505,88.05020296)(261.37499504,86.83184297)
\curveto(262.21591503,84.804733)(262.28645503,84.445313)(262.20507503,82.60138302)
\curveto(262.09734504,80.16041305)(261.27579504,78.48065307)(259.85741506,77.80255308)
\curveto(258.56332508,77.18387309)(257.52846509,77.27265309)(256.4726551,78.09356308)
\lineto(255.54687511,78.81427307)
\lineto(255.54687511,81.01934304)
\curveto(255.54687511,82.91382302)(255.63267511,83.35886301)(256.1542951,84.167783)
\curveto(256.4883151,84.685803)(256.8344651,85.48197299)(256.9218751,85.93731298)
\curveto(257.05391509,86.62514298)(257.17154509,86.73364297)(257.61718509,86.57989298)
\curveto(257.91221508,86.47812298)(258.25902508,86.28099298)(258.38866508,86.14239298)
\curveto(258.70693507,85.80215299)(259.68889506,85.82500299)(260.27734506,86.18539298)
\curveto(260.81083505,86.51207298)(260.89810505,87.33287297)(260.49023505,88.20492296)
\curveto(260.34264506,88.52031295)(260.27509506,88.83786295)(260.33984506,88.91195295)
\closepath
\moveto(261.45312504,88.72052295)
\curveto(261.70549504,88.78952295)(262.03415504,88.15869296)(262.52538503,86.67755298)
\curveto(263.65752502,83.26395302)(263.19997502,77.74720308)(261.68163504,76.49200309)
\curveto(261.37807504,76.2410631)(260.49811505,75.8247131)(259.72656506,75.56817311)
\curveto(258.95499507,75.31163311)(257.53281509,75.10138311)(256.5644551,75.10138311)
\lineto(254.80273512,75.10138311)
\lineto(254.80273512,76.80645309)
\curveto(254.80273512,78.37844307)(255.04473512,78.92602307)(255.42577511,78.22052307)
\curveto(255.72960511,77.65803308)(257.52713509,76.80645309)(258.40820508,76.80645309)
\curveto(259.48256507,76.80645309)(260.11397506,77.14946309)(261.21093505,78.32989307)
\curveto(262.33717503,79.54183306)(262.74218503,80.78613304)(262.74218503,83.03497302)
\curveto(262.74218503,84.575493)(262.59900503,85.27723299)(261.98632504,86.72442297)
\curveto(261.57021504,87.70731296)(261.28373504,88.57484295)(261.34960504,88.65020295)
\curveto(261.38280504,88.68820295)(261.41710504,88.71070295)(261.45312504,88.72050295)
\closepath
\moveto(286.49023475,88.68932295)
\curveto(286.57503475,88.70772295)(286.62361475,88.57448295)(286.70898475,88.30260296)
\curveto(286.80678474,87.99132296)(286.92564474,87.65255296)(286.97265474,87.55064296)
\curveto(287.01965474,87.44874297)(286.72368475,87.23145297)(286.31445475,87.06822297)
\curveto(285.42163476,86.71209297)(285.33389476,87.08362297)(286.05663475,88.16002296)
\curveto(286.28498475,88.50010295)(286.40540475,88.67088295)(286.49023475,88.68932295)
\closepath
\moveto(255.66015511,87.59361296)
\curveto(256.5480551,87.59361296)(256.8516851,86.28669298)(256.1503851,85.48424299)
\curveto(255.52562511,84.769383)(255.08724512,84.957163)(254.91406512,86.01353298)
\curveto(254.71882512,87.20429297)(254.90286512,87.59361296)(255.66015511,87.59361296)
\closepath
\moveto(276.83202486,87.46471297)
\curveto(276.88282486,87.40661297)(276.74842486,87.04899297)(276.53124487,86.66978298)
\curveto(276.15911487,86.01994298)(276.16880487,85.93849298)(276.72656486,85.24986299)
\curveto(277.05204486,84.848033)(277.75227485,84.266363)(278.28124485,83.95689301)
\curveto(279.62823483,83.16886302)(280.60741482,82.05489303)(280.60741482,81.31236304)
\curveto(280.60741482,80.94265304)(280.17151482,80.20306305)(279.55273483,79.52135306)
\curveto(278.97275484,78.88238307)(278.13845485,77.83589308)(277.69921485,77.19713309)
\curveto(277.08437486,76.3030031)(276.19951487,75.6345831)(273.8535149,74.29088312)
\curveto(272.17754492,73.33094313)(270.71625493,72.54645314)(270.60546494,72.54674314)
\curveto(270.36800494,72.54745314)(267.90528497,71.29107316)(266.83788498,70.62486316)
\curveto(266.03007499,70.12066317)(261.74719504,69.73721317)(261.75390504,70.16978317)
\curveto(261.75590504,70.30600317)(262.39906503,70.81069316)(263.18163502,71.29088316)
\curveto(264.789665,72.27755314)(265.106565,73.06329314)(264.19140501,73.79674313)
\curveto(263.51060502,74.34236312)(260.89731505,74.40025312)(260.09765506,73.88654313)
\curveto(259.08914507,73.23872313)(257.53320509,71.38963315)(257.53320509,70.83967316)
\curveto(257.53320509,70.15611317)(258.86405507,69.32270318)(259.67382506,69.49982318)
\curveto(260.04702506,69.58142318)(260.26171506,69.49582318)(260.26171506,69.26349318)
\curveto(260.26171506,69.06207318)(260.43597505,68.69802319)(260.64843505,68.45490319)
\curveto(261.26427504,67.7502532)(266.10632499,67.98160319)(268.16796496,68.81428318)
\curveto(269.03679495,69.16518318)(269.78949494,69.40508318)(269.83984494,69.34748318)
\curveto(269.89014494,69.28988318)(269.49797495,68.87132318)(268.96874495,68.41584319)
\curveto(267.79247497,67.4034232)(266.96093498,66.06304322)(266.96093498,65.18146323)
\curveto(266.96093498,64.40282324)(268.44719496,62.69258326)(269.34960495,62.43342326)
\curveto(269.79506494,62.30549326)(269.92273494,62.12772326)(269.80663494,61.79474327)
\curveto(269.71813495,61.54077327)(269.59498495,61.03062328)(269.53320495,60.65998328)
\curveto(269.47140495,60.28933329)(269.17436495,59.74035329)(268.87304496,59.4412333)
\curveto(267.60922497,58.18665331)(264.712725,56.12835333)(263.60156502,55.69513334)
\curveto(261.86768504,55.01916335)(261.25390505,55.07489335)(261.25390505,55.90607334)
\curveto(261.25390505,56.57037333)(259.33991507,59.0954133)(259.07030507,58.7869333)
\curveto(258.99990507,58.7063333)(258.80946507,58.7669333)(258.64648508,58.9217033)
\curveto(258.48348508,59.0764833)(257.66314509,59.2862233)(256.8242151,59.3865433)
\curveto(255.78320511,59.51105329)(255.29882511,59.68991329)(255.29882511,59.95295329)
\curveto(255.29882511,60.24280329)(255.75325511,60.33967328)(257.13280509,60.33967328)
\curveto(259.21367507,60.33967328)(259.26952507,60.39767328)(259.26952507,62.59943326)
\curveto(259.26952507,64.40694324)(258.88705507,64.74006323)(256.8124951,64.74006323)
\lineto(255.17577512,64.74006323)
\lineto(255.00390512,65.73420322)
\curveto(254.90940512,66.28071321)(254.82620512,68.46910319)(254.81835512,70.59748316)
\lineto(254.80275512,74.46662312)
\lineto(257.09767509,74.61310312)
\curveto(260.07119506,74.80204311)(261.17776505,75.20824311)(262.26759503,76.51545309)
\curveto(263.35066502,77.81452308)(263.84129501,80.04409305)(263.69532502,82.99982302)
\lineto(263.59772502,84.984203)
\lineto(264.845775,84.849433)
\curveto(265.53182499,84.775233)(267.40442497,84.891233)(269.00787495,85.10724299)
\curveto(270.61134494,85.32329299)(271.92389492,85.41233299)(271.92389492,85.30451299)
\curveto(271.92389492,85.19668299)(271.75554492,84.950083)(271.55084492,84.755683)
\curveto(271.34614493,84.561293)(271.17975493,84.206323)(271.17975493,83.96857301)
\curveto(271.17975493,83.73082301)(270.77686493,82.57955302)(270.28522494,81.40998304)
\curveto(269.44105495,79.40178306)(268.35964496,77.92955308)(266.23834499,75.8982631)
\curveto(265.55595499,75.24482311)(265.451115,74.97812311)(265.51764499,74.05451312)
\curveto(265.58974499,73.05336314)(265.67571499,72.94389314)(266.67584498,72.58185314)
\curveto(267.57776497,72.25540314)(267.91837497,72.25291314)(268.74811496,72.56625314)
\curveto(269.95839494,73.02328314)(270.33287494,73.66198313)(270.51569494,75.58383311)
\curveto(270.69479493,77.46654308)(271.18844493,78.54281307)(272.47858491,79.86508306)
\curveto(275.88476487,83.35606301)(276.28207487,84.254243)(275.02155488,85.60922299)
\lineto(274.39850489,86.27914298)
\lineto(275.57037488,86.92562297)
\curveto(276.21433487,87.28074297)(276.78128486,87.52284297)(276.83209486,87.46469297)
\closepath
\moveto(277.21093486,86.70299297)
\curveto(277.27903486,86.73059297)(277.48688485,86.68189298)(277.87890485,86.57213298)
\curveto(278.35652484,86.43840298)(279.55555483,86.32826298)(280.54491482,86.32799298)
\curveto(281.53429481,86.32762298)(282.3437448,86.27309298)(282.3437448,86.20689298)
\curveto(282.3437448,86.14069298)(282.0590448,85.65880299)(281.7109348,85.13658299)
\curveto(281.36282481,84.614373)(280.98283481,84.187363)(280.86718481,84.187363)
\curveto(280.50990482,84.187363)(277.77650485,85.90075298)(277.38280486,86.37096298)
\curveto(277.21481486,86.57162298)(277.14287486,86.67537298)(277.21093486,86.70299297)
\closepath
\moveto(287.41601474,86.58580298)
\lineto(287.64062474,85.59947299)
\curveto(287.76442473,85.05702299)(288.07451473,83.80082301)(288.33007473,82.80846302)
\curveto(288.58564472,81.81610303)(288.79491472,80.88385304)(288.79491472,80.73619304)
\curveto(288.79491472,80.58852305)(288.40878473,80.28277305)(287.93554473,80.05650305)
\curveto(287.46230474,79.83025306)(286.80425474,79.64439306)(286.47460475,79.64439306)
\curveto(285.62975476,79.64439306)(285.47125476,80.84745304)(286.10546475,82.44127302)
\curveto(286.37559475,83.12014302)(286.53245475,83.86877301)(286.45312475,84.10533301)
\curveto(286.37382475,84.341893)(286.55674475,84.99651299)(286.86132474,85.56041299)
\closepath
\moveto(273.6816349,85.60728299)
\curveto(274.49150489,85.60728299)(274.63644489,85.49963299)(274.87499489,84.716663)
\curveto(275.10424488,83.96417301)(275.06595488,83.67619301)(274.62890489,82.86705302)
\curveto(274.34430489,82.34014303)(273.8906549,81.77355303)(273.6191349,81.60728303)
\curveto(272.47968491,80.90952304)(271.05628493,79.32128306)(270.53124494,78.16197308)
\curveto(270.33855494,77.73650308)(270.10896494,76.60404309)(270.02148494,75.6463531)
\curveto(269.87954494,74.09240312)(269.77933494,73.84736313)(269.09374495,73.36900313)
\curveto(268.29228496,72.80978314)(268.06422497,72.77984314)(266.83593498,73.07799313)
\curveto(265.71345499,73.35045313)(265.66581499,74.59278312)(266.73823498,75.61314311)
\curveto(268.47874496,77.26914309)(269.84602494,79.09581306)(270.53120494,80.68150305)
\curveto(270.92570493,81.59445303)(271.34559493,82.56271302)(271.46284493,82.83385302)
\curveto(271.58010492,83.10497302)(271.67573492,83.46223301)(271.67573492,83.62682301)
\curveto(271.67573492,83.79142301)(271.92010492,84.303393)(272.21870492,84.765493)
\curveto(272.65078491,85.43420299)(272.94816491,85.60728299)(273.6815949,85.60728299)
\closepath
\moveto(255.07616512,84.595573)
\curveto(255.10946512,84.596143)(255.14746512,84.586573)(255.18945512,84.568173)
\curveto(255.44259511,84.457033)(255.51265511,84.238773)(255.39062511,83.93731301)
\curveto(255.28815512,83.68421301)(255.13189512,82.45602302)(255.04491512,81.20684304)
\lineto(254.88671512,78.93536307)
\lineto(254.84571512,81.83575303)
\curveto(254.81381512,84.04222301)(254.84271512,84.591483)(255.07617512,84.595523)
\closepath
\moveto(289.47070471,80.91588304)
\curveto(289.50240471,80.86328304)(289.51440471,80.65875305)(289.52150471,80.28307305)
\curveto(289.53960471,79.32354306)(289.42810471,79.10765306)(289.22853472,79.71666306)
\curveto(289.16463472,79.91185305)(289.20063472,80.32624305)(289.30863472,80.63853305)
\curveto(289.38853471,80.86973304)(289.43905471,80.96847304)(289.47074471,80.91588304)
\closepath
\moveto(289.04296472,79.41783306)
\lineto(289.04296472,76.66002309)
\lineto(289.04296472,73.90221313)
\lineto(287.36913474,72.01744315)
\curveto(286.44799475,70.98144316)(285.24703476,69.38095318)(284.70116477,68.46080319)
\curveto(283.39737478,66.26300321)(282.81567479,65.54183322)(282.2460948,65.41393322)
\curveto(281.9881348,65.35603323)(281.65303481,65.05783323)(281.50195481,64.75182323)
\curveto(281.35086481,64.44579324)(280.97695481,63.95674324)(280.66991482,63.66588325)
\curveto(280.36286482,63.37502325)(280.11132482,62.95518325)(280.11132482,62.73228326)
\curveto(280.11132482,62.21899326)(279.70225483,62.20569326)(279.44140483,62.71078326)
\curveto(279.33227483,62.92211325)(279.08970484,63.15026325)(278.90234484,63.21859325)
\curveto(278.69553484,63.29399325)(279.05543484,63.72988324)(279.81835483,64.32601324)
\curveto(281.43844481,65.59191322)(281.9156748,66.77257321)(281.9726548,69.66586317)
\curveto(282.0251548,72.33138314)(282.1539848,72.63985314)(283.11523479,72.39828314)
\curveto(283.76857478,72.23408314)(283.85208478,72.29838314)(283.98632478,73.06625314)
\curveto(284.07682478,73.58379313)(284.32669477,73.97309312)(284.62499477,74.06234312)
\curveto(284.97535477,74.16717312)(285.09176477,74.40601312)(285.03515477,74.90023311)
\curveto(284.95465477,75.60363311)(285.43115476,76.52133309)(285.87695476,76.52133309)
\curveto(286.01229475,76.52133309)(286.78078475,77.17241309)(287.58398474,77.96859308)
\closepath
\moveto(274.56054489,73.96666312)
\curveto(274.61884489,73.96666312)(274.43339489,73.60405313)(274.14648489,73.16002313)
\curveto(273.8595649,72.71597314)(272.93481491,71.94556315)(272.09179492,71.44908315)
\curveto(271.24879493,70.95263316)(270.39024494,70.43815317)(270.18554494,70.30455317)
\curveto(269.98085494,70.17094317)(269.08819495,69.77789317)(268.20116496,69.43150318)
\curveto(266.12325499,68.62007319)(262.24609503,68.28702319)(262.24609503,68.91978318)
\curveto(262.24609503,69.25168318)(262.70072503,69.37672318)(264.31640501,69.49205318)
\curveto(265.76886499,69.59573318)(266.67516498,69.81992317)(267.35546497,70.24400317)
\curveto(268.44197496,70.92128316)(271.06403493,72.26353314)(271.30077493,72.26353314)
\curveto(271.38537493,72.26353314)(272.12994492,72.64665314)(272.95507491,73.11510313)
\curveto(273.7802249,73.58353313)(274.50223489,73.96666312)(274.56054489,73.96666312)
\closepath
\moveto(262.21484503,73.68346313)
\curveto(264.813905,73.68346313)(264.864415,73.06223314)(262.40038503,71.44908315)
\curveto(260.28618506,70.06497317)(259.12571507,69.75575317)(258.40038508,70.37877317)
\curveto(257.91399508,70.79658316)(257.92226508,70.82451316)(258.77343507,71.84166315)
\curveto(259.95333506,73.25158313)(260.75937505,73.68346313)(262.21484503,73.68346313)
\closepath
\moveto(289.18554472,73.25963313)
\curveto(289.19834472,71.62169315)(288.73561472,66.98144321)(288.36718473,66.59947321)
\curveto(287.99875473,66.21750322)(285.42789476,65.65857322)(283.37499479,65.70689322)
\curveto(285.69134476,69.03376318)(287.28048474,71.36713315)(289.18554472,73.25963313)
\closepath
\moveto(280.82616482,71.81822315)
\curveto(280.95343481,71.68032315)(278.54868484,68.54314319)(277.05859486,66.90416321)
\curveto(276.40263487,66.18268322)(274.68263489,64.28200324)(273.2382749,62.68150326)
\curveto(271.79393492,61.08102328)(270.60079494,59.77274329)(270.58593494,59.77525329)
\curveto(270.10396494,59.85255329)(270.02211494,60.32733328)(270.31835494,61.33385327)
\curveto(270.73727493,62.75713326)(273.3266549,65.59185322)(275.47265488,66.97838321)
\curveto(276.31825487,67.5247232)(277.84760485,68.86135318)(278.87109484,69.94713317)
\curveto(279.89458483,71.03289316)(280.77383482,71.87494315)(280.82616482,71.81822315)
\closepath
\moveto(281.37304481,70.64049316)
\curveto(281.54754481,70.53817316)(281.53694481,69.76401317)(281.32614481,68.37096319)
\curveto(281.08121481,66.75218321)(280.89514481,66.29053321)(280.15818482,65.49205322)
\curveto(279.14592483,64.39534324)(277.53532485,63.18278325)(277.07614486,63.16978325)
\curveto(276.56173487,63.15528325)(275.85250487,61.79980327)(275.93747487,60.99400328)
\curveto(275.98087487,60.58262328)(275.90597487,60.22917329)(275.76950487,60.20689329)
\curveto(275.37164488,60.14199329)(272.91599491,60.36053328)(272.91599491,60.46080328)
\curveto(272.91599491,60.72221328)(274.83861489,62.83962326)(275.21091488,62.98814325)
\curveto(275.44972488,63.08344325)(275.64450488,63.41024325)(275.64450488,63.71471324)
\curveto(275.64450488,64.04013324)(275.99996487,64.51910324)(276.50583487,64.87682323)
\curveto(277.44109485,65.53820322)(280.16158482,68.79176319)(280.86130481,70.08580317)
\curveto(281.09626481,70.52033316)(281.26832481,70.70188316)(281.37302481,70.64049316)
\closepath
\moveto(261.39257504,69.70885317)
\curveto(261.58922504,69.70885317)(261.74999504,69.51729318)(261.74999504,69.28307318)
\curveto(261.74999504,68.85544318)(261.22621505,68.69962319)(260.92968505,69.03893318)
\curveto(260.69813505,69.30387318)(260.97808505,69.70885317)(261.39257504,69.70885317)
\closepath
\moveto(280.06445482,69.70885317)
\curveto(280.13265482,69.70885317)(279.98525483,69.45278318)(279.73827483,69.14049318)
\curveto(279.49127483,68.82820318)(279.23423483,68.57213319)(279.16601483,68.57213319)
\curveto(279.09781484,68.57213319)(279.24321483,68.82820318)(279.49023483,69.14049318)
\curveto(279.73722483,69.45278318)(279.99620482,69.70885317)(280.06445482,69.70885317)
\closepath
\moveto(289.77148471,68.85728318)
\curveto(289.87971471,68.85728318)(290.2726547,68.35409319)(290.6445247,67.7401032)
\curveto(291.42632469,66.44927321)(291.33359469,65.56897322)(290.3710947,65.15025323)
\curveto(289.52577471,64.78252323)(289.36922471,64.80956323)(289.18554472,65.35728323)
\curveto(288.99642472,65.92122322)(289.48788471,68.85728318)(289.77148471,68.85728318)
\closepath
\moveto(257.17187509,68.29283319)
\curveto(256.5608351,68.24833319)(256.2910151,67.7368932)(256.2910151,66.72643321)
\curveto(256.2910151,66.36159321)(256.3605151,65.86209322)(256.4433551,65.61510322)
\curveto(256.6869551,64.88872323)(258.46199508,65.01841323)(259.03515507,65.80455322)
\curveto(259.74803506,66.78232321)(259.63270506,67.4099932)(258.65624508,67.8768232)
\curveto(258.02756508,68.17738319)(257.53849509,68.31954319)(257.17187509,68.29283319)
\closepath
\moveto(257.14847509,67.7205732)
\curveto(257.43949509,67.7205732)(258.61685508,67.2432632)(258.97855507,66.97838321)
\curveto(259.02375507,66.94528321)(258.93755507,66.65170321)(258.78519507,66.32603321)
\curveto(258.51558508,65.74960322)(257.33220509,65.48997322)(256.9531651,65.92369322)
\curveto(256.6252351,66.29891321)(256.7798451,67.7205732)(257.14847509,67.7205732)
\closepath
\moveto(288.20902473,65.93150322)
\curveto(288.45179473,65.65372322)(287.50236474,63.38137325)(286.88284474,62.75767326)
\curveto(286.25155475,62.12213326)(285.99399475,62.06261326)(283.95706478,62.08971326)
\curveto(280.48018482,62.13591326)(280.04088482,62.34165326)(281.19534481,63.38072325)
\curveto(281.51814481,63.67124325)(281.8572048,64.18169324)(281.9492548,64.51353324)
\curveto(282.0880148,65.01375323)(282.3586348,65.13773323)(283.53324478,65.24205323)
\curveto(284.31248477,65.31125323)(285.66136476,65.51983322)(286.53128475,65.70494322)
\curveto(287.40121474,65.89005322)(288.15635473,65.99178322)(288.20902473,65.93150322)
\closepath
\moveto(294.50199465,65.44908322)
\curveto(294.51479465,65.43448322)(294.43949466,65.32071323)(294.28519466,65.09557323)
\curveto(293.96106466,64.62266323)(293.75784466,64.47128324)(293.75784466,64.70494323)
\curveto(293.75784466,64.76384323)(293.95263466,64.98674323)(294.19144466,65.20103323)
\curveto(294.38820466,65.37761323)(294.48918465,65.46374322)(294.50199465,65.44908322)
\closepath
\moveto(256.8711351,64.29088324)
\lineto(257.90433508,64.26938324)
\lineto(257.16019509,63.97056324)
\curveto(256.7507951,63.80636324)(256.2477051,63.51612325)(256.04300511,63.32602325)
\curveto(255.23278512,62.57361326)(255.05081512,62.56436326)(255.05081512,63.27332325)
\curveto(255.05081512,64.09260324)(255.46218511,64.32148324)(256.8711351,64.29090324)
\closepath
\moveto(293.74417466,64.11900324)
\curveto(293.80537466,64.06020324)(293.48028467,63.59998325)(293.01956467,63.09752325)
\lineto(292.18167468,62.18541326)
\lineto(292.90824467,63.20494325)
\curveto(293.30773467,63.76620324)(293.68294466,64.17780324)(293.74417466,64.11900324)
\closepath
\moveto(275.03910488,64.02920324)
\curveto(275.09930488,64.02920324)(275.14847488,63.90213324)(275.14847488,63.74600324)
\curveto(275.14847488,63.58985325)(275.03127488,63.46279325)(274.88675488,63.46279325)
\curveto(274.74225489,63.46279325)(274.69304489,63.58985325)(274.77738489,63.74600324)
\curveto(274.86168489,63.90213324)(274.97892488,64.02920324)(275.03910488,64.02920324)
\closepath
\moveto(258.80667507,63.59756325)
\lineto(258.72857507,62.32412326)
\curveto(258.65837508,61.19382327)(258.57946508,61.04307328)(258.02935508,60.99990328)
\curveto(257.68819509,60.97310328)(257.13028509,61.04500328)(256.7891251,61.15811327)
\lineto(256.1680251,61.36318327)
\lineto(256.9121651,61.89639327)
\curveto(257.32156509,62.19024326)(257.91598508,62.69470326)(258.23247508,63.01553325)
\closepath
\moveto(257.89261508,63.50776325)
\curveto(258.15903508,63.50376325)(258.05767508,63.28513325)(257.43949509,62.96284325)
\curveto(257.11513509,62.79374326)(256.4456451,62.24863326)(255.95120511,61.75191327)
\curveto(255.30652511,61.10424328)(255.05081512,60.97436328)(255.05081512,61.29292327)
\curveto(255.05081512,61.91375327)(255.88850511,62.73906326)(257.04691509,63.25972325)
\curveto(257.43987509,63.43634325)(257.73276509,63.51038325)(257.89261508,63.50776325)
\closepath
\moveto(273.6133149,62.32612326)
\curveto(273.6815149,62.32612326)(273.5341149,62.07200326)(273.2871449,61.75972327)
\curveto(273.04014491,61.44741327)(272.78311491,61.19136327)(272.71488491,61.19136327)
\curveto(272.64668491,61.19136327)(272.79208491,61.44741327)(273.03910491,61.75972327)
\curveto(273.2861049,62.07200326)(273.5450949,62.32612326)(273.6133149,62.32612326)
\closepath
\moveto(261.00589505,62.26952326)
\curveto(260.93339505,62.28282326)(260.85958505,62.28292326)(260.78910505,62.26752326)
\curveto(260.23094506,62.14589326)(260.14723506,61.07767328)(260.65042505,60.50190328)
\curveto(261.18921505,59.88540329)(261.85358504,60.04102329)(261.94925504,60.80463328)
\curveto(262.02825504,61.43496327)(261.51375504,62.17622326)(261.00589505,62.26947326)
\closepath
\moveto(280.63284482,61.94726327)
\curveto(280.88653481,61.94926327)(281.19130481,61.88296327)(281.62113481,61.74999327)
\curveto(282.3081848,61.53744327)(283.38669479,61.48139327)(284.51566477,61.59765327)
\curveto(285.52072476,61.70113327)(286.31449475,61.67225327)(286.31449475,61.53125327)
\curveto(286.31449475,61.15730327)(285.42794476,60.05469329)(285.12699476,60.05469329)
\curveto(284.98378477,60.05469329)(284.82873477,59.93353329)(284.78324477,59.78516329)
\curveto(284.73774477,59.63680329)(283.69696478,59.4403833)(282.4687848,59.3476633)
\curveto(280.46328482,59.1962233)(280.17660482,59.2378933)(279.65824483,59.75196329)
\curveto(278.97735484,60.42721328)(279.09318484,61.21091327)(279.95120483,61.73633327)
\curveto(280.17733482,61.87480327)(280.37916482,61.94514327)(280.63284482,61.94727327)
\closepath
\moveto(260.88480505,61.65038327)
\curveto(260.97790505,61.63428327)(261.12242505,61.51530327)(261.25394505,61.33398327)
\curveto(261.42929504,61.09221328)(261.51749504,60.83002328)(261.44925504,60.75195328)
\curveto(261.38105504,60.67385328)(261.18125505,60.80705328)(261.00589505,61.04882328)
\curveto(260.83055505,61.29059327)(260.74235505,61.55279327)(260.81058505,61.63085327)
\curveto(260.82768505,61.65035327)(260.85378505,61.65575327)(260.88478505,61.65035327)
\closepath
\moveto(291.67191469,61.45507327)
\curveto(291.75111469,61.49557327)(291.71981469,61.32687327)(291.53128469,60.92382328)
\curveto(291.38928469,60.62022328)(291.15187469,60.28373329)(291.0019947,60.17773329)
\curveto(290.8520847,60.07173329)(290.9216947,60.32087328)(291.15628469,60.73046328)
\curveto(291.40460469,61.16410327)(291.59266469,61.41454327)(291.67191469,61.45507327)
\closepath
\moveto(272.66800491,59.62890329)
\lineto(273.7226949,59.61130329)
\curveto(276.19378487,59.56890329)(276.14066487,59.59510329)(276.14066487,58.38474331)
\curveto(276.14066487,57.48141332)(276.04146487,57.25157332)(275.58402488,57.11130332)
\curveto(274.83038489,56.88019333)(273.03913491,56.88637333)(272.83402491,57.12130332)
\curveto(272.74302491,57.22538332)(272.66786491,57.83332331)(272.66800491,58.47091331)
\closepath
\moveto(255.97464511,58.8847633)
\curveto(256.7550151,58.9001633)(258.39980508,58.57881331)(259.02152507,58.21093331)
\curveto(259.88633506,57.69922332)(260.75784505,56.46848333)(260.75784505,55.75976334)
\curveto(260.75785505,55.07965335)(260.43205505,55.10951335)(258.83597507,55.93945334)
\curveto(257.50425509,56.63191333)(255.56078511,58.29314331)(255.55277511,58.7460933)
\curveto(255.55077511,58.8369933)(255.71452511,58.8796133)(255.97464511,58.8847633)
\closepath
\moveto(290.03519471,58.6367133)
\curveto(290.04799471,58.6221133)(289.97469471,58.50638331)(289.82034471,58.28124331)
\curveto(289.49621471,57.80834331)(289.29105472,57.65891332)(289.29105472,57.89257331)
\curveto(289.29105472,57.95157331)(289.48778471,58.17437331)(289.72660471,58.38866331)
\curveto(289.92336471,58.56524331)(290.02238471,58.6513733)(290.03519471,58.6367133)
\closepath
\moveto(255.05081512,57.56640332)
\lineto(255.73441511,57.05273332)
\curveto(256.10969511,56.77015333)(256.7212251,56.36411333)(257.09378509,56.15038333)
\curveto(257.46635509,55.93667334)(257.83511509,55.64235334)(257.91410508,55.49609334)
\curveto(257.99670508,55.34322334)(257.48278509,55.22851334)(256.7051151,55.22851334)
\curveto(255.26075512,55.22851334)(255.05081512,55.43333334)(255.05081512,56.84570333)
\closepath
\moveto(258.89652507,55.22851334)
\curveto(259.10122507,55.22851334)(259.26956507,55.10145335)(259.26956507,54.94530335)
\curveto(259.26956507,54.78915335)(259.10122507,54.66210335)(258.89652507,54.66210335)
\curveto(258.69182508,54.66210335)(258.52542508,54.78915335)(258.52542508,54.94530335)
\curveto(258.52542508,55.10145335)(258.69182508,55.22851334)(258.89652507,55.22851334)
\closepath
\moveto(280.83597482,55.19331334)
\curveto(280.92217481,55.18931335)(280.99591481,55.16431335)(281.03714481,55.11711335)
\curveto(281.11954481,55.02281335)(280.97254481,54.95638335)(280.71097482,54.96867335)
\curveto(280.42192482,54.98227335)(280.36264482,55.04797335)(280.56058482,55.13859335)
\curveto(280.65018482,55.17959335)(280.74978482,55.19739334)(280.83597482,55.19329334)
\closepath
\moveto(282.71683479,54.33003336)
\curveto(283.94502478,54.35113335)(285.11760476,54.24503336)(285.32230476,54.09370336)
\curveto(285.75568476,53.77323336)(283.40112478,53.77323336)(281.59964481,54.09370336)
\curveto(280.68610482,54.25621336)(280.88978481,54.29874336)(282.71683479,54.33003336)
\closepath
}
}
{
\newrgbcolor{curcolor}{0 0 0}
\pscustom[linewidth=0.4,linecolor=curcolor]
{
\newpath
\moveto(258.15234508,115.84161263)
\lineto(256.6640651,114.99005264)
\curveto(254.08991513,113.51735266)(254.30705513,116.47651263)(254.30663513,82.98028302)
\lineto(254.30663513,53.81036336)
\lineto(255.11327512,53.81236336)
\curveto(257.26011509,53.82036336)(262.74408503,54.69887335)(263.93945501,55.22643334)
\curveto(265.121185,55.74795334)(266.37318498,56.51653333)(266.71288498,56.92760332)
\curveto(266.78108498,57.01020332)(267.45874497,57.60728332)(268.21874496,58.25377331)
\curveto(269.21061495,59.0974833)(269.77019494,59.3912433)(270.20312494,59.2967433)
\curveto(270.75814493,59.1756033)(270.80050493,59.0558533)(270.72265493,57.78893331)
\curveto(270.64955493,56.59837333)(270.71665493,56.35500333)(271.21874493,55.99010334)
\curveto(272.02799492,55.40213334)(275.54663488,54.42693335)(279.98632482,53.56236336)
\lineto(279.98632482,53.56436336)
\curveto(280.66864482,53.43148337)(282.56567479,53.31365337)(284.20116478,53.30459337)
\curveto(287.12832474,53.28839337)(287.19828474,53.30432337)(288.66991472,54.24600336)
\curveto(290.5122647,55.42489334)(292.85953467,58.17051331)(294.11913466,60.62295328)
\curveto(294.64043465,61.63790327)(295.49534464,63.24106325)(296.01952464,64.18350324)
\curveto(297.03011462,66.00043322)(297.89092461,69.64406318)(297.65038462,71.08389316)
\curveto(297.58198462,71.49344315)(297.66988462,72.14970315)(297.84570462,72.54287314)
\curveto(298.02149461,72.93603314)(298.40243461,74.24898312)(298.69140461,75.46084311)
\curveto(299.3381146,78.17312308)(299.3979146,82.89409302)(298.8320246,86.45889298)
\curveto(298.62133461,87.78614296)(298.34481461,89.57469294)(298.21679461,90.43350293)
\curveto(297.95504461,92.18960291)(297.94734461,92.21120291)(295.59179464,97.63857285)
\curveto(293.33815467,102.83111279)(292.09155468,104.82828276)(289.83984471,106.85732274)
\curveto(288.64919472,107.93025273)(285.24007476,110.0233427)(284.68163477,110.0233427)
\curveto(284.48039477,110.0233427)(283.66678478,110.3961027)(282.87499479,110.85342269)
\curveto(280.59061482,112.17289268)(276.34779487,113.64138266)(270.80663493,115.02920264)
\curveto(268.20188496,115.68160263)(267.22297497,115.77893263)(262.99023502,115.80850263)
\closepath
\moveto(260.14843506,115.22052264)
\curveto(260.82914505,115.22752264)(261.70648504,115.22116264)(262.86718503,115.21052264)
\curveto(267.09503498,115.17262264)(267.92764497,115.07969264)(270.93163493,114.32380265)
\curveto(276.38491487,112.95159267)(279.13840483,112.07077268)(281.36327481,110.98200269)
\curveto(282.5172848,110.4172927)(284.02339478,109.7294627)(284.71093477,109.45466271)
\curveto(285.39846476,109.17987271)(286.52653475,108.55725272)(287.21874474,108.06989272)
\curveto(287.91095473,107.58256273)(288.57784472,107.17984273)(288.69921472,107.17732273)
\curveto(288.94109472,107.17232273)(291.0322347,104.93203276)(292.02734468,103.61286278)
\curveto(292.37213468,103.15581278)(293.03305467,101.9520328)(293.49609467,100.93708281)
\curveto(293.95913466,99.92214282)(294.57659465,98.64375283)(294.86718465,98.09724284)
\curveto(295.15779465,97.55072285)(295.70371464,96.27431286)(296.08007464,95.25935287)
\curveto(296.45644463,94.24441289)(296.93090463,93.0642229)(297.13476462,92.63630291)
\curveto(297.33861462,92.20840291)(297.61523462,91.12159292)(297.74999462,90.22224293)
\curveto(297.88476461,89.32290294)(298.15644461,87.50065297)(298.35351461,86.17341298)
\curveto(299.1080346,81.09166304)(298.68068461,75.8172931)(297.25390462,72.58943314)
\curveto(296.91732463,71.82799315)(296.86536463,71.43085315)(297.05468462,71.09333316)
\curveto(297.49352462,70.31093317)(296.42169463,65.74594322)(295.41796464,64.12068324)
\curveto(294.95343465,63.36849325)(294.09757466,61.82516327)(293.51562467,60.69099328)
\curveto(291.79584469,57.33938332)(288.81831472,54.40917335)(286.87304474,54.15583336)
\curveto(286.29307475,54.08023336)(285.81835476,54.09033336)(285.81835476,54.17733336)
\curveto(285.81835476,54.37015335)(287.29058474,55.22811334)(287.62109474,55.22811334)
\curveto(287.96550473,55.22811334)(290.5555147,58.37006331)(291.40234469,59.81405329)
\curveto(291.80041469,60.49285328)(292.66128468,61.71260327)(293.31445467,62.52303326)
\curveto(293.96761466,63.33344325)(294.50195465,64.14797324)(294.50195465,64.33358324)
\curveto(294.50195465,64.51919324)(294.89345465,65.16615323)(295.37109464,65.77108322)
\curveto(296.25456463,66.88999321)(296.51641463,68.00545319)(295.89648464,68.00545319)
\curveto(295.70823464,68.00545319)(295.12511465,67.3975332)(294.60156465,66.65584321)
\curveto(294.07801466,65.91415322)(293.59044467,65.24368323)(293.51757467,65.16561323)
\curveto(293.44467467,65.08751323)(292.99520467,64.35226324)(292.51757468,63.53280325)
\curveto(292.03995468,62.71332326)(291.56721469,62.04284326)(291.46679469,62.04256326)
\curveto(291.36639469,62.04227326)(290.9772747,61.49915327)(290.6035147,60.83553328)
\curveto(288.57479472,57.23361332)(286.29894475,55.21338334)(284.30468477,55.24178334)
\curveto(282.65466479,55.26528334)(279.27659483,55.96491334)(279.17773483,56.30428333)
\curveto(279.12273484,56.49321333)(278.91138484,56.64803333)(278.70898484,56.64803333)
\curveto(278.45661484,56.64803333)(278.36607484,56.41190333)(278.41991484,55.89608334)
\curveto(278.49751484,55.15323335)(278.64711484,55.00112335)(279.86327483,54.42537335)
\curveto(280.78043482,53.99117336)(279.85287483,54.11639336)(277.13476486,54.79256335)
\curveto(276.24772487,55.01321335)(274.90679488,55.34533334)(274.15624489,55.53084334)
\curveto(273.4056849,55.71636334)(272.42789491,56.03706334)(271.98437492,56.24373333)
\curveto(271.28188493,56.57111333)(271.17968493,56.75313333)(271.17968493,57.65584332)
\curveto(271.17968493,58.6016833)(271.76054492,60.47417328)(272.13866492,60.74373328)
\curveto(272.22436492,60.80483328)(272.25693492,60.03019329)(272.21096492,59.0230333)
\curveto(272.09663492,56.52045333)(272.30896492,56.27779333)(274.47073489,56.41756333)
\curveto(275.60931488,56.49116333)(276.24655487,56.66462333)(276.40041487,56.94295332)
\curveto(276.52677487,57.17154332)(276.63082486,58.13431331)(276.63283486,59.0835833)
\curveto(276.63683486,60.68964328)(276.67583486,60.80768328)(277.19533486,60.78670328)
\curveto(278.00371485,60.75410328)(277.95361485,61.81483327)(277.13483486,62.06209326)
\lineto(276.51373487,62.24959326)
\lineto(277.13483486,62.53670326)
\curveto(277.57543485,62.74113326)(277.78362485,62.73301326)(277.85553485,62.50550326)
\curveto(277.91123485,62.32932326)(278.14422485,62.03333326)(278.37311484,61.84730327)
\curveto(278.65625484,61.61715327)(278.75374484,61.23656327)(278.67975484,60.65980328)
\curveto(278.58245484,59.90140329)(278.68275484,59.74760329)(279.62311483,59.1988633)
\curveto(280.41733482,58.7352533)(280.95230481,58.6210933)(281.8184248,58.7301133)
\curveto(285.13278476,59.1473033)(286.56256475,59.87667329)(286.56256475,61.15199327)
\curveto(286.56256475,61.66664327)(286.83427474,62.14755326)(287.43170474,62.69300326)
\curveto(288.07222473,63.27776325)(288.29889473,63.70655324)(288.29889473,64.32581324)
\curveto(288.29889473,64.78759323)(288.41048473,65.16566323)(288.54694472,65.16566323)
\curveto(288.68341472,65.16566323)(288.79498472,65.03386323)(288.79498472,64.87269323)
\curveto(288.79498472,64.19969324)(289.89626471,64.10548324)(290.7188147,64.70863323)
\curveto(292.12581468,65.74029322)(292.04186468,67.00678321)(290.4356147,68.97230318)
\lineto(289.71881471,69.84925317)
\lineto(289.99616471,73.11683313)
\curveto(290.6978147,81.34472304)(288.83184472,90.43634293)(285.22077476,96.38050286)
\curveto(284.23165478,98.00869284)(281.7724648,100.86480281)(281.04694481,101.2281628)
\curveto(280.73735482,101.3832028)(280.03616482,101.8322328)(279.49030483,102.22425279)
\curveto(278.94443484,102.61625279)(278.33172484,103.00926278)(278.12702485,103.09925278)
\curveto(277.92232485,103.18925278)(276.80454486,103.71691277)(275.64459488,104.27113277)
\curveto(273.8411549,105.13279276)(271.83504492,105.84144275)(267.45709497,107.16175273)
\curveto(266.97945498,107.30579273)(265.73339499,107.68912273)(264.687565,108.01331272)
\curveto(262.13101503,108.80586272)(261.22451505,108.77493272)(260.28131506,107.86683273)
\curveto(259.86144506,107.46257273)(259.51764507,106.95422274)(259.51764507,106.73597274)
\curveto(259.51764507,106.25446275)(258.24541508,105.61369275)(256.7637351,105.34925276)
\curveto(256.1635351,105.24216276)(255.46515511,105.08756276)(255.21295512,105.00550276)
\curveto(254.77978512,104.86458276)(254.76026512,105.07580276)(254.84186512,108.81605272)
\curveto(254.93956512,113.29612266)(254.95878512,113.35230266)(256.8965551,114.50550265)
\curveto(257.82440509,115.05771264)(258.10638508,115.20040264)(260.14850506,115.22035264)
\closepath
\moveto(261.84570504,108.01739272)
\curveto(262.38183503,108.05229272)(263.04244502,107.91073273)(263.89257501,107.61309273)
\curveto(264.55645501,107.38068273)(265.99420499,106.94741274)(267.08593498,106.65020274)
\curveto(273.7233149,104.84322276)(278.29471484,102.75490279)(281.6953048,99.97833282)
\curveto(282.59871479,99.24071283)(283.51808478,98.22784284)(283.73827478,97.72833285)
\curveto(283.95845478,97.22884285)(284.39126477,96.50150286)(284.70116477,96.11114286)
\curveto(285.39012476,95.24336287)(287.31219474,91.17742292)(287.58593474,90.00762294)
\curveto(287.69554473,89.53919294)(288.06787473,88.00473296)(288.41406473,86.59942298)
\curveto(288.76023472,85.19411299)(289.09888472,83.47005301)(289.16796472,82.76739302)
\curveto(289.29325472,81.49325304)(289.29349472,81.49432304)(288.80077472,83.19317302)
\curveto(288.52906472,84.13005301)(288.30474473,85.12052299)(288.30273473,85.39434299)
\curveto(288.30073473,85.66816299)(288.07567473,86.37026298)(287.80273473,86.95294297)
\curveto(287.52980474,87.53560297)(287.30663474,88.23170296)(287.30663474,88.50177295)
\curveto(287.30663474,89.06815295)(286.53595475,89.63474294)(286.15234475,89.34942294)
\curveto(285.81833476,89.10098295)(285.32454476,88.02573296)(285.06054477,86.97247297)
\curveto(284.82143477,86.01848298)(285.31579476,85.58004299)(285.95116475,86.18341298)
\curveto(286.43570475,86.64351298)(286.72743475,86.31671298)(286.32616475,85.76348299)
\curveto(286.15710475,85.53041299)(285.97083475,84.752463)(285.91210476,84.03302301)
\curveto(285.85340476,83.31357301)(285.63500476,82.42816303)(285.42773476,82.06622303)
\curveto(285.01045477,81.33751304)(285.15186476,80.09911305)(285.74023476,79.32208306)
\curveto(286.09124475,78.85850307)(287.22425474,78.94739307)(288.08398473,79.50567306)
\curveto(288.27065473,79.62690306)(287.92479473,79.13323306)(287.31640474,78.40802307)
\curveto(286.70802475,77.68280308)(286.08169475,77.08966309)(285.92382476,77.08966309)
\curveto(285.35327476,77.08966309)(284.28714477,75.8854031)(284.40429477,75.37286311)
\curveto(284.47879477,75.04694311)(284.35459477,74.81866311)(284.05273478,74.72833312)
\curveto(283.75315478,74.63873312)(283.58398478,74.33463312)(283.58398478,73.88262313)
\curveto(283.58398478,73.28835313)(283.43829478,73.14882313)(282.65429479,72.99981314)
\curveto(282.1425548,72.90251314)(281.48828481,72.82164314)(281.19921481,72.82012314)
\curveto(280.87697481,72.81912314)(279.91298483,71.95624315)(278.70898484,70.58966316)
\curveto(277.62657485,69.36110318)(276.16039487,68.02217319)(275.44140488,67.6052832)
\curveto(273.9106949,66.71776321)(272.35022491,65.39471323)(271.11327493,63.93731324)
\curveto(270.62640493,63.36365325)(270.07844494,62.89760325)(269.89648494,62.90216325)
\curveto(269.02089495,62.92506325)(267.45702497,64.49705324)(267.45702497,65.35333323)
\curveto(267.45702497,66.01780322)(269.68355495,68.78122319)(270.79296493,69.49395318)
\curveto(272.40471491,70.52940316)(273.9013649,71.78364315)(274.03320489,72.20880315)
\curveto(274.11530489,72.47347314)(274.89256488,73.32845313)(275.76171487,74.10919312)
\curveto(276.63088486,74.88992311)(277.56807485,75.9747931)(277.84374485,76.52130309)
\curveto(278.11941485,77.06780309)(278.97193484,78.19837307)(279.73827483,79.03302306)
\curveto(281.56042481,81.01762304)(281.55606481,81.78798303)(279.71487483,83.46270301)
\curveto(279.01943484,84.09524301)(278.08367485,84.840833)(277.63479485,85.11895299)
\curveto(277.18590486,85.39707299)(276.88875486,85.70428299)(276.97463486,85.80255299)
\curveto(277.06053486,85.90085298)(277.32894486,85.77525299)(277.57229485,85.52325299)
\curveto(277.81565485,85.27127299)(278.66617484,84.724763)(279.46096483,84.308413)
\lineto(280.90627481,83.55059301)
\lineto(281.55666481,84.10723301)
\curveto(282.76749479,85.14161299)(283.44185478,87.15763297)(283.54104478,90.03497294)
\curveto(283.62414478,92.44534291)(283.58114478,92.7299029)(282.88479479,94.30841289)
\curveto(281.9522148,96.42225286)(280.17719482,98.93738283)(278.62893484,100.33770281)
\curveto(277.36651486,101.4795228)(274.70656489,102.63813279)(270.80666493,103.74591277)
\curveto(265.51977499,105.24766276)(263.85930501,105.29918276)(261.51760504,104.03106277)
\curveto(260.62241505,103.54628278)(259.55395507,102.97247278)(259.14455507,102.75567279)
\curveto(258.73516507,102.53888279)(257.59096509,102.27717279)(256.6015951,102.17559279)
\lineto(254.80276512,101.9920028)
\lineto(254.80276512,103.15411278)
\curveto(254.80276512,104.27803277)(254.83556512,104.32379277)(255.73440511,104.49395277)
\curveto(257.90911508,104.90568276)(259.63510506,105.63858275)(259.90823506,106.26544275)
\curveto(260.40632505,107.40859273)(260.95216505,107.95915273)(261.84573504,108.01739272)
\closepath
\moveto(265.074215,104.31231277)
\curveto(266.67543498,104.28401277)(266.74495498,104.26111277)(265.82616499,104.05255277)
\curveto(264.899615,103.84223277)(262.57698503,102.56686279)(261.38476504,101.6130928)
\curveto(261.11558505,101.3977628)(260.75622505,101.2205228)(260.58593505,101.2205228)
\curveto(260.41564505,101.2205228)(259.66646506,100.90227281)(258.91991507,100.51153281)
\curveto(258.17337508,100.12081282)(257.24856509,99.65862282)(256.8652351,99.48614282)
\curveto(256.4818851,99.31367283)(255.86160511,99.06782283)(255.48632511,98.93927283)
\curveto(254.80853512,98.70711283)(254.80273512,98.71538283)(254.80273512,100.06427282)
\lineto(254.80273512,101.4216928)
\lineto(256.6601551,101.6013828)
\curveto(258.13958508,101.7444828)(259.00527507,102.04169279)(260.91406505,103.06231278)
\curveto(263.08862502,104.22502277)(263.47497502,104.34057277)(265.074215,104.31231277)
\closepath
\moveto(268.57421496,103.45098278)
\lineto(266.83593498,103.21466278)
\curveto(265.88068499,103.08419278)(264.821615,102.89195278)(264.48046501,102.78888279)
\curveto(263.91178501,102.61706279)(263.89955501,102.63034279)(264.34374501,102.95098278)
\curveto(264.61024501,103.14336278)(265.67123499,103.33562278)(266.70116498,103.37677278)
\closepath
\moveto(269.66991495,103.17364278)
\curveto(269.75611495,103.16964278)(269.82986494,103.14264278)(269.87109494,103.09554278)
\curveto(269.95349494,103.00124278)(269.80649494,102.93476278)(269.54491495,102.94710278)
\curveto(269.25588495,102.96070278)(269.19854495,103.02840278)(269.39648495,103.11897278)
\curveto(269.48608495,103.15997278)(269.58372495,103.17777278)(269.66991495,103.17367278)
\closepath
\moveto(270.49023494,102.90802278)
\curveto(270.67372493,102.92192278)(271.25812493,102.72750279)(272.07812492,102.31427279)
\curveto(275.92933487,100.37346281)(279.36718483,97.47328285)(279.36718483,96.16388286)
\curveto(279.36718483,95.88124287)(279.05887484,95.32423287)(278.68359484,94.92755288)
\curveto(278.30832484,94.53085288)(277.77878485,94.12460289)(277.50585485,94.02325289)
\curveto(277.14731486,93.89009289)(277.19840486,94.00355289)(277.69140485,94.43145288)
\curveto(278.06668485,94.75716288)(278.37499484,95.24718287)(278.37499484,95.52130287)
\curveto(278.37499484,96.26497286)(277.04071486,97.91149284)(275.44530488,99.13653283)
\curveto(274.66833489,99.73313282)(273.6965049,100.52861281)(273.2871049,100.90606281)
\curveto(272.87771491,101.2835128)(271.98505492,101.8910528)(271.30273493,102.25567279)
\curveto(270.52361494,102.67200279)(270.30674494,102.89413278)(270.49023494,102.90802278)
\closepath
\moveto(273.2246049,102.50372279)
\curveto(273.2336049,102.50572279)(273.2530049,102.49972279)(273.2871049,102.48612279)
\curveto(276.22064487,101.3055028)(277.07775486,100.83094281)(278.43749484,99.63651282)
\curveto(281.09639481,97.30087285)(282.85211479,94.09325289)(283.03320479,91.23807292)
\curveto(283.21366479,88.39266295)(282.98640479,87.42299297)(282.0527348,87.05057297)
\curveto(280.99302481,86.62788298)(279.12582484,86.67923298)(278.37499484,87.15214297)
\curveto(277.90473485,87.44832297)(277.72491485,87.83898296)(277.63085485,88.76346295)
\curveto(277.51318485,89.92021294)(277.42242486,90.03783294)(275.89257487,91.04471292)
\curveto(275.00555488,91.62851292)(274.0013149,92.12849291)(273.6601549,92.15604291)
\curveto(273.3189949,92.18354291)(273.7104549,92.31697291)(274.52929489,92.45292291)
\curveto(276.74630486,92.8209829)(278.26150485,93.54301289)(279.24804483,94.69901288)
\curveto(280.02324482,95.60738287)(280.10004482,95.82257287)(279.88280483,96.47636286)
\curveto(279.38380483,97.97809284)(277.62337485,99.66158282)(274.52929489,101.5974528)
\curveto(273.6932449,102.12053279)(273.16172491,102.49269279)(273.2246049,102.50370279)
\closepath
\moveto(267.13866498,102.07012279)
\curveto(268.50184496,102.06512279)(270.53254494,101.3826428)(271.74023492,100.52325281)
\curveto(274.57161489,98.50841284)(277.13476486,96.23047286)(277.13476486,95.72833287)
\curveto(277.13476486,94.58996288)(271.05004493,93.94267289)(269.00390495,94.86309288)
\curveto(268.25183496,95.20141287)(267.20421498,95.39050287)(266.09179499,95.38848287)
\curveto(264.62074501,95.38648287)(264.20388501,95.26917287)(263.36327502,94.62286288)
\curveto(261.51559504,93.2022429)(261.25390505,92.9624229)(261.25390505,92.6912229)
\curveto(261.25390505,92.25188291)(261.99653504,92.38058291)(262.70507503,92.9431729)
\curveto(264.703685,94.53007288)(265.69383499,94.69823288)(268.20116496,93.87677289)
\lineto(269.81445494,93.3494229)
\lineto(266.83788498,93.2185629)
\curveto(264.690635,93.1244629)(263.65205502,92.9548829)(263.11523502,92.61309291)
\curveto(262.31116503,92.10114291)(259.54020507,91.98127291)(258.64843508,92.41973291)
\curveto(258.15713508,92.6612929)(258.15856508,92.6636429)(258.66213508,92.6834129)
\curveto(258.94190507,92.6944129)(259.27664507,92.8156529)(259.40627507,92.9529429)
\curveto(259.53592507,93.0902329)(260.06083506,93.3557729)(260.57229505,93.54278289)
\curveto(261.08375505,93.72979289)(261.97580504,94.42697288)(262.55666503,95.09161288)
\curveto(263.99272501,96.73478286)(264.49637501,96.99475285)(266.95119498,97.36700285)
\curveto(268.52367496,97.60545285)(269.07033495,97.79733284)(269.07033495,98.10723284)
\curveto(269.07033495,98.43643284)(268.72836496,98.51461284)(267.45705497,98.48028284)
\curveto(265.184535,98.41898284)(263.29909502,97.53692285)(261.75979504,95.81231287)
\curveto(261.07060505,95.04016288)(260.28863506,94.40802288)(260.02151506,94.40802288)
\curveto(259.75439506,94.40802288)(258.91782507,94.14363289)(258.16213508,93.82012289)
\curveto(256.4325351,93.0796629)(256.3387651,92.43538291)(257.84182509,91.60528292)
\curveto(259.62487506,90.62057293)(261.33543504,90.31885293)(262.20901503,90.83575293)
\curveto(263.79568502,91.77458292)(265.429585,92.23579291)(267.39651497,92.30059291)
\curveto(268.52273496,92.33769291)(269.38584495,92.26309291)(269.31643495,92.13458291)
\curveto(269.17548495,91.87363291)(266.44523498,90.68805293)(264.728545,90.14239293)
\curveto(262.87756503,89.55405294)(260.05214506,89.54191294)(258.77346507,90.11699293)
\curveto(255.59639511,91.54589292)(254.80276512,92.58546291)(254.80276512,95.32207287)
\lineto(254.80276512,97.17558285)
\lineto(256.10549511,97.30644285)
\curveto(256.8219351,97.37764285)(258.05310508,97.77655284)(258.84182507,98.19316284)
\curveto(259.63052506,98.60978283)(260.37351506,98.94902283)(260.49221505,98.94902283)
\curveto(260.61093505,98.94902283)(261.29646504,99.40306283)(262.01565504,99.95683282)
\curveto(263.55287502,101.14049281)(265.81693499,102.07466279)(267.13869498,102.07011279)
\closepath
\moveto(260.01366506,100.25958282)
\curveto(260.16093506,100.27278282)(260.00366506,100.14072282)(259.51757507,99.82403282)
\curveto(259.03994507,99.51292282)(258.54092508,99.25248283)(258.40820508,99.24591283)
\curveto(258.00569508,99.22611283)(259.10188507,99.99577282)(259.76562506,100.19903282)
\curveto(259.88138506,100.23443282)(259.96458506,100.25513282)(260.01366506,100.25963282)
\closepath
\moveto(257.70507509,99.08184283)
\curveto(257.77057509,99.06214283)(257.56955509,98.91258283)(257.16015509,98.68731283)
\curveto(256.6142851,98.38695284)(255.99961511,98.13613284)(255.79491511,98.13067284)
\curveto(255.36965511,98.11927284)(256.6404751,98.78820283)(257.54882509,99.05450283)
\curveto(257.63192509,99.07890283)(257.68324509,99.08840283)(257.70507509,99.08180283)
\closepath
\moveto(276.39257487,93.67169289)
\curveto(276.63105486,93.65419289)(276.71521486,93.60639289)(276.61132486,93.53106289)
\curveto(276.28936487,93.2977929)(273.2071549,93.1481329)(273.4335949,93.3767729)
\curveto(273.4904949,93.4341729)(274.30571489,93.54815289)(275.24609488,93.63067289)
\curveto(275.76073487,93.67587289)(276.15410487,93.68917289)(276.39257487,93.67167289)
\closepath
\moveto(254.87109512,92.00567291)
\curveto(254.92509512,92.06147291)(255.01839512,92.00067291)(255.17773512,91.84942291)
\curveto(255.46314511,91.57839292)(255.48563511,91.40054292)(255.27143512,91.10528292)
\curveto(254.87513512,90.55892293)(254.80268512,90.61322293)(254.80268512,91.46075292)
\curveto(254.80268512,91.77690292)(254.81708512,91.94984291)(254.87108512,92.00567291)
\closepath
\moveto(270.80663493,91.87872291)
\lineto(272.23437492,91.74395292)
\curveto(273.6183249,91.61302292)(275.74790487,90.68308293)(276.76171486,89.76544294)
\curveto(277.24310486,89.32984294)(277.23576486,89.31575294)(276.51366487,89.32794294)
\curveto(275.64004488,89.34264294)(273.2114949,90.32614293)(271.79882492,91.23809292)
\closepath
\moveto(269.96288494,91.68341292)
\lineto(271.74999492,90.64630293)
\curveto(272.73241491,90.07528294)(274.23322489,89.37787294)(275.08593488,89.09747295)
\curveto(275.93862487,88.81706295)(276.59931486,88.46239295)(276.55468487,88.30841296)
\curveto(276.34334487,87.57958296)(272.45774491,86.18714298)(269.43945495,85.75958299)
\curveto(265.083645,85.14255299)(263.45095502,85.27051299)(263.28710502,86.24395298)
\curveto(263.24350502,86.50317298)(262.99953502,87.20200297)(262.74609503,87.79473296)
\curveto(262.28564503,88.87159295)(262.28580503,88.87195295)(262.82419503,88.95880295)
\curveto(263.12055502,89.00660295)(264.22399501,89.35844294)(265.275365,89.74005294)
\curveto(266.32673499,90.12165293)(267.31858497,90.43341293)(267.48044497,90.43341293)
\curveto(267.64232497,90.43341293)(268.26722496,90.71446293)(268.86911496,91.05841292)
\closepath
\moveto(255.64062511,90.62091293)
\curveto(255.74682511,90.65961293)(255.85761511,90.57781293)(255.93945511,90.33380293)
\curveto(256.01685511,90.10288293)(256.04427511,89.60344294)(255.99995511,89.22442295)
\curveto(255.88848511,88.27060296)(255.29878511,88.57597295)(255.29878511,89.58770294)
\curveto(255.29878511,90.15301293)(255.46357511,90.55641293)(255.64058511,90.62091293)
\closepath
\moveto(256.7441351,90.43341293)
\curveto(256.8711051,90.43341293)(257.59121509,90.11989293)(258.34570508,89.73809294)
\curveto(259.49675507,89.15563295)(259.76710506,88.87694295)(260.01952506,88.00177296)
\curveto(260.26559506,87.14874297)(260.25783506,86.91376297)(259.98042506,86.71270297)
\curveto(259.45262507,86.33011298)(258.73661507,86.46292298)(257.65620509,87.14434297)
\curveto(256.7426551,87.72052296)(256.6574651,87.87607296)(256.5878451,89.10138295)
\curveto(256.5462451,89.83350294)(256.6171451,90.43341293)(256.7440951,90.43341293)
\closepath
\moveto(260.33984506,88.91192295)
\curveto(260.40454506,88.98602295)(260.86958505,88.05020296)(261.37499504,86.83184297)
\curveto(262.21591503,84.804733)(262.28645503,84.445313)(262.20507503,82.60138302)
\curveto(262.09734504,80.16041305)(261.27579504,78.48065307)(259.85741506,77.80255308)
\curveto(258.56332508,77.18387309)(257.52846509,77.27265309)(256.4726551,78.09356308)
\lineto(255.54687511,78.81427307)
\lineto(255.54687511,81.01934304)
\curveto(255.54687511,82.91382302)(255.63267511,83.35886301)(256.1542951,84.167783)
\curveto(256.4883151,84.685803)(256.8344651,85.48197299)(256.9218751,85.93731298)
\curveto(257.05391509,86.62514298)(257.17154509,86.73364297)(257.61718509,86.57989298)
\curveto(257.91221508,86.47812298)(258.25902508,86.28099298)(258.38866508,86.14239298)
\curveto(258.70693507,85.80215299)(259.68889506,85.82500299)(260.27734506,86.18539298)
\curveto(260.81083505,86.51207298)(260.89810505,87.33287297)(260.49023505,88.20492296)
\curveto(260.34264506,88.52031295)(260.27509506,88.83786295)(260.33984506,88.91195295)
\closepath
\moveto(261.45312504,88.72052295)
\curveto(261.70549504,88.78952295)(262.03415504,88.15869296)(262.52538503,86.67755298)
\curveto(263.65752502,83.26395302)(263.19997502,77.74720308)(261.68163504,76.49200309)
\curveto(261.37807504,76.2410631)(260.49811505,75.8247131)(259.72656506,75.56817311)
\curveto(258.95499507,75.31163311)(257.53281509,75.10138311)(256.5644551,75.10138311)
\lineto(254.80273512,75.10138311)
\lineto(254.80273512,76.80645309)
\curveto(254.80273512,78.37844307)(255.04473512,78.92602307)(255.42577511,78.22052307)
\curveto(255.72960511,77.65803308)(257.52713509,76.80645309)(258.40820508,76.80645309)
\curveto(259.48256507,76.80645309)(260.11397506,77.14946309)(261.21093505,78.32989307)
\curveto(262.33717503,79.54183306)(262.74218503,80.78613304)(262.74218503,83.03497302)
\curveto(262.74218503,84.575493)(262.59900503,85.27723299)(261.98632504,86.72442297)
\curveto(261.57021504,87.70731296)(261.28373504,88.57484295)(261.34960504,88.65020295)
\curveto(261.38280504,88.68820295)(261.41710504,88.71070295)(261.45312504,88.72050295)
\closepath
\moveto(286.49023475,88.68932295)
\curveto(286.57503475,88.70772295)(286.62361475,88.57448295)(286.70898475,88.30260296)
\curveto(286.80678474,87.99132296)(286.92564474,87.65255296)(286.97265474,87.55064296)
\curveto(287.01965474,87.44874297)(286.72368475,87.23145297)(286.31445475,87.06822297)
\curveto(285.42163476,86.71209297)(285.33389476,87.08362297)(286.05663475,88.16002296)
\curveto(286.28498475,88.50010295)(286.40540475,88.67088295)(286.49023475,88.68932295)
\closepath
\moveto(255.66015511,87.59361296)
\curveto(256.5480551,87.59361296)(256.8516851,86.28669298)(256.1503851,85.48424299)
\curveto(255.52562511,84.769383)(255.08724512,84.957163)(254.91406512,86.01353298)
\curveto(254.71882512,87.20429297)(254.90286512,87.59361296)(255.66015511,87.59361296)
\closepath
\moveto(276.83202486,87.46471297)
\curveto(276.88282486,87.40661297)(276.74842486,87.04899297)(276.53124487,86.66978298)
\curveto(276.15911487,86.01994298)(276.16880487,85.93849298)(276.72656486,85.24986299)
\curveto(277.05204486,84.848033)(277.75227485,84.266363)(278.28124485,83.95689301)
\curveto(279.62823483,83.16886302)(280.60741482,82.05489303)(280.60741482,81.31236304)
\curveto(280.60741482,80.94265304)(280.17151482,80.20306305)(279.55273483,79.52135306)
\curveto(278.97275484,78.88238307)(278.13845485,77.83589308)(277.69921485,77.19713309)
\curveto(277.08437486,76.3030031)(276.19951487,75.6345831)(273.8535149,74.29088312)
\curveto(272.17754492,73.33094313)(270.71625493,72.54645314)(270.60546494,72.54674314)
\curveto(270.36800494,72.54745314)(267.90528497,71.29107316)(266.83788498,70.62486316)
\curveto(266.03007499,70.12066317)(261.74719504,69.73721317)(261.75390504,70.16978317)
\curveto(261.75590504,70.30600317)(262.39906503,70.81069316)(263.18163502,71.29088316)
\curveto(264.789665,72.27755314)(265.106565,73.06329314)(264.19140501,73.79674313)
\curveto(263.51060502,74.34236312)(260.89731505,74.40025312)(260.09765506,73.88654313)
\curveto(259.08914507,73.23872313)(257.53320509,71.38963315)(257.53320509,70.83967316)
\curveto(257.53320509,70.15611317)(258.86405507,69.32270318)(259.67382506,69.49982318)
\curveto(260.04702506,69.58142318)(260.26171506,69.49582318)(260.26171506,69.26349318)
\curveto(260.26171506,69.06207318)(260.43597505,68.69802319)(260.64843505,68.45490319)
\curveto(261.26427504,67.7502532)(266.10632499,67.98160319)(268.16796496,68.81428318)
\curveto(269.03679495,69.16518318)(269.78949494,69.40508318)(269.83984494,69.34748318)
\curveto(269.89014494,69.28988318)(269.49797495,68.87132318)(268.96874495,68.41584319)
\curveto(267.79247497,67.4034232)(266.96093498,66.06304322)(266.96093498,65.18146323)
\curveto(266.96093498,64.40282324)(268.44719496,62.69258326)(269.34960495,62.43342326)
\curveto(269.79506494,62.30549326)(269.92273494,62.12772326)(269.80663494,61.79474327)
\curveto(269.71813495,61.54077327)(269.59498495,61.03062328)(269.53320495,60.65998328)
\curveto(269.47140495,60.28933329)(269.17436495,59.74035329)(268.87304496,59.4412333)
\curveto(267.60922497,58.18665331)(264.712725,56.12835333)(263.60156502,55.69513334)
\curveto(261.86768504,55.01916335)(261.25390505,55.07489335)(261.25390505,55.90607334)
\curveto(261.25390505,56.57037333)(259.33991507,59.0954133)(259.07030507,58.7869333)
\curveto(258.99990507,58.7063333)(258.80946507,58.7669333)(258.64648508,58.9217033)
\curveto(258.48348508,59.0764833)(257.66314509,59.2862233)(256.8242151,59.3865433)
\curveto(255.78320511,59.51105329)(255.29882511,59.68991329)(255.29882511,59.95295329)
\curveto(255.29882511,60.24280329)(255.75325511,60.33967328)(257.13280509,60.33967328)
\curveto(259.21367507,60.33967328)(259.26952507,60.39767328)(259.26952507,62.59943326)
\curveto(259.26952507,64.40694324)(258.88705507,64.74006323)(256.8124951,64.74006323)
\lineto(255.17577512,64.74006323)
\lineto(255.00390512,65.73420322)
\curveto(254.90940512,66.28071321)(254.82620512,68.46910319)(254.81835512,70.59748316)
\lineto(254.80275512,74.46662312)
\lineto(257.09767509,74.61310312)
\curveto(260.07119506,74.80204311)(261.17776505,75.20824311)(262.26759503,76.51545309)
\curveto(263.35066502,77.81452308)(263.84129501,80.04409305)(263.69532502,82.99982302)
\lineto(263.59772502,84.984203)
\lineto(264.845775,84.849433)
\curveto(265.53182499,84.775233)(267.40442497,84.891233)(269.00787495,85.10724299)
\curveto(270.61134494,85.32329299)(271.92389492,85.41233299)(271.92389492,85.30451299)
\curveto(271.92389492,85.19668299)(271.75554492,84.950083)(271.55084492,84.755683)
\curveto(271.34614493,84.561293)(271.17975493,84.206323)(271.17975493,83.96857301)
\curveto(271.17975493,83.73082301)(270.77686493,82.57955302)(270.28522494,81.40998304)
\curveto(269.44105495,79.40178306)(268.35964496,77.92955308)(266.23834499,75.8982631)
\curveto(265.55595499,75.24482311)(265.451115,74.97812311)(265.51764499,74.05451312)
\curveto(265.58974499,73.05336314)(265.67571499,72.94389314)(266.67584498,72.58185314)
\curveto(267.57776497,72.25540314)(267.91837497,72.25291314)(268.74811496,72.56625314)
\curveto(269.95839494,73.02328314)(270.33287494,73.66198313)(270.51569494,75.58383311)
\curveto(270.69479493,77.46654308)(271.18844493,78.54281307)(272.47858491,79.86508306)
\curveto(275.88476487,83.35606301)(276.28207487,84.254243)(275.02155488,85.60922299)
\lineto(274.39850489,86.27914298)
\lineto(275.57037488,86.92562297)
\curveto(276.21433487,87.28074297)(276.78128486,87.52284297)(276.83209486,87.46469297)
\closepath
\moveto(277.21093486,86.70299297)
\curveto(277.27903486,86.73059297)(277.48688485,86.68189298)(277.87890485,86.57213298)
\curveto(278.35652484,86.43840298)(279.55555483,86.32826298)(280.54491482,86.32799298)
\curveto(281.53429481,86.32762298)(282.3437448,86.27309298)(282.3437448,86.20689298)
\curveto(282.3437448,86.14069298)(282.0590448,85.65880299)(281.7109348,85.13658299)
\curveto(281.36282481,84.614373)(280.98283481,84.187363)(280.86718481,84.187363)
\curveto(280.50990482,84.187363)(277.77650485,85.90075298)(277.38280486,86.37096298)
\curveto(277.21481486,86.57162298)(277.14287486,86.67537298)(277.21093486,86.70299297)
\closepath
\moveto(287.41601474,86.58580298)
\lineto(287.64062474,85.59947299)
\curveto(287.76442473,85.05702299)(288.07451473,83.80082301)(288.33007473,82.80846302)
\curveto(288.58564472,81.81610303)(288.79491472,80.88385304)(288.79491472,80.73619304)
\curveto(288.79491472,80.58852305)(288.40878473,80.28277305)(287.93554473,80.05650305)
\curveto(287.46230474,79.83025306)(286.80425474,79.64439306)(286.47460475,79.64439306)
\curveto(285.62975476,79.64439306)(285.47125476,80.84745304)(286.10546475,82.44127302)
\curveto(286.37559475,83.12014302)(286.53245475,83.86877301)(286.45312475,84.10533301)
\curveto(286.37382475,84.341893)(286.55674475,84.99651299)(286.86132474,85.56041299)
\closepath
\moveto(273.6816349,85.60728299)
\curveto(274.49150489,85.60728299)(274.63644489,85.49963299)(274.87499489,84.716663)
\curveto(275.10424488,83.96417301)(275.06595488,83.67619301)(274.62890489,82.86705302)
\curveto(274.34430489,82.34014303)(273.8906549,81.77355303)(273.6191349,81.60728303)
\curveto(272.47968491,80.90952304)(271.05628493,79.32128306)(270.53124494,78.16197308)
\curveto(270.33855494,77.73650308)(270.10896494,76.60404309)(270.02148494,75.6463531)
\curveto(269.87954494,74.09240312)(269.77933494,73.84736313)(269.09374495,73.36900313)
\curveto(268.29228496,72.80978314)(268.06422497,72.77984314)(266.83593498,73.07799313)
\curveto(265.71345499,73.35045313)(265.66581499,74.59278312)(266.73823498,75.61314311)
\curveto(268.47874496,77.26914309)(269.84602494,79.09581306)(270.53120494,80.68150305)
\curveto(270.92570493,81.59445303)(271.34559493,82.56271302)(271.46284493,82.83385302)
\curveto(271.58010492,83.10497302)(271.67573492,83.46223301)(271.67573492,83.62682301)
\curveto(271.67573492,83.79142301)(271.92010492,84.303393)(272.21870492,84.765493)
\curveto(272.65078491,85.43420299)(272.94816491,85.60728299)(273.6815949,85.60728299)
\closepath
\moveto(255.07616512,84.595573)
\curveto(255.10946512,84.596143)(255.14746512,84.586573)(255.18945512,84.568173)
\curveto(255.44259511,84.457033)(255.51265511,84.238773)(255.39062511,83.93731301)
\curveto(255.28815512,83.68421301)(255.13189512,82.45602302)(255.04491512,81.20684304)
\lineto(254.88671512,78.93536307)
\lineto(254.84571512,81.83575303)
\curveto(254.81381512,84.04222301)(254.84271512,84.591483)(255.07617512,84.595523)
\closepath
\moveto(289.47070471,80.91588304)
\curveto(289.50240471,80.86328304)(289.51440471,80.65875305)(289.52150471,80.28307305)
\curveto(289.53960471,79.32354306)(289.42810471,79.10765306)(289.22853472,79.71666306)
\curveto(289.16463472,79.91185305)(289.20063472,80.32624305)(289.30863472,80.63853305)
\curveto(289.38853471,80.86973304)(289.43905471,80.96847304)(289.47074471,80.91588304)
\closepath
\moveto(289.04296472,79.41783306)
\lineto(289.04296472,76.66002309)
\lineto(289.04296472,73.90221313)
\lineto(287.36913474,72.01744315)
\curveto(286.44799475,70.98144316)(285.24703476,69.38095318)(284.70116477,68.46080319)
\curveto(283.39737478,66.26300321)(282.81567479,65.54183322)(282.2460948,65.41393322)
\curveto(281.9881348,65.35603323)(281.65303481,65.05783323)(281.50195481,64.75182323)
\curveto(281.35086481,64.44579324)(280.97695481,63.95674324)(280.66991482,63.66588325)
\curveto(280.36286482,63.37502325)(280.11132482,62.95518325)(280.11132482,62.73228326)
\curveto(280.11132482,62.21899326)(279.70225483,62.20569326)(279.44140483,62.71078326)
\curveto(279.33227483,62.92211325)(279.08970484,63.15026325)(278.90234484,63.21859325)
\curveto(278.69553484,63.29399325)(279.05543484,63.72988324)(279.81835483,64.32601324)
\curveto(281.43844481,65.59191322)(281.9156748,66.77257321)(281.9726548,69.66586317)
\curveto(282.0251548,72.33138314)(282.1539848,72.63985314)(283.11523479,72.39828314)
\curveto(283.76857478,72.23408314)(283.85208478,72.29838314)(283.98632478,73.06625314)
\curveto(284.07682478,73.58379313)(284.32669477,73.97309312)(284.62499477,74.06234312)
\curveto(284.97535477,74.16717312)(285.09176477,74.40601312)(285.03515477,74.90023311)
\curveto(284.95465477,75.60363311)(285.43115476,76.52133309)(285.87695476,76.52133309)
\curveto(286.01229475,76.52133309)(286.78078475,77.17241309)(287.58398474,77.96859308)
\closepath
\moveto(274.56054489,73.96666312)
\curveto(274.61884489,73.96666312)(274.43339489,73.60405313)(274.14648489,73.16002313)
\curveto(273.8595649,72.71597314)(272.93481491,71.94556315)(272.09179492,71.44908315)
\curveto(271.24879493,70.95263316)(270.39024494,70.43815317)(270.18554494,70.30455317)
\curveto(269.98085494,70.17094317)(269.08819495,69.77789317)(268.20116496,69.43150318)
\curveto(266.12325499,68.62007319)(262.24609503,68.28702319)(262.24609503,68.91978318)
\curveto(262.24609503,69.25168318)(262.70072503,69.37672318)(264.31640501,69.49205318)
\curveto(265.76886499,69.59573318)(266.67516498,69.81992317)(267.35546497,70.24400317)
\curveto(268.44197496,70.92128316)(271.06403493,72.26353314)(271.30077493,72.26353314)
\curveto(271.38537493,72.26353314)(272.12994492,72.64665314)(272.95507491,73.11510313)
\curveto(273.7802249,73.58353313)(274.50223489,73.96666312)(274.56054489,73.96666312)
\closepath
\moveto(262.21484503,73.68346313)
\curveto(264.813905,73.68346313)(264.864415,73.06223314)(262.40038503,71.44908315)
\curveto(260.28618506,70.06497317)(259.12571507,69.75575317)(258.40038508,70.37877317)
\curveto(257.91399508,70.79658316)(257.92226508,70.82451316)(258.77343507,71.84166315)
\curveto(259.95333506,73.25158313)(260.75937505,73.68346313)(262.21484503,73.68346313)
\closepath
\moveto(289.18554472,73.25963313)
\curveto(289.19834472,71.62169315)(288.73561472,66.98144321)(288.36718473,66.59947321)
\curveto(287.99875473,66.21750322)(285.42789476,65.65857322)(283.37499479,65.70689322)
\curveto(285.69134476,69.03376318)(287.28048474,71.36713315)(289.18554472,73.25963313)
\closepath
\moveto(280.82616482,71.81822315)
\curveto(280.95343481,71.68032315)(278.54868484,68.54314319)(277.05859486,66.90416321)
\curveto(276.40263487,66.18268322)(274.68263489,64.28200324)(273.2382749,62.68150326)
\curveto(271.79393492,61.08102328)(270.60079494,59.77274329)(270.58593494,59.77525329)
\curveto(270.10396494,59.85255329)(270.02211494,60.32733328)(270.31835494,61.33385327)
\curveto(270.73727493,62.75713326)(273.3266549,65.59185322)(275.47265488,66.97838321)
\curveto(276.31825487,67.5247232)(277.84760485,68.86135318)(278.87109484,69.94713317)
\curveto(279.89458483,71.03289316)(280.77383482,71.87494315)(280.82616482,71.81822315)
\closepath
\moveto(281.37304481,70.64049316)
\curveto(281.54754481,70.53817316)(281.53694481,69.76401317)(281.32614481,68.37096319)
\curveto(281.08121481,66.75218321)(280.89514481,66.29053321)(280.15818482,65.49205322)
\curveto(279.14592483,64.39534324)(277.53532485,63.18278325)(277.07614486,63.16978325)
\curveto(276.56173487,63.15528325)(275.85250487,61.79980327)(275.93747487,60.99400328)
\curveto(275.98087487,60.58262328)(275.90597487,60.22917329)(275.76950487,60.20689329)
\curveto(275.37164488,60.14199329)(272.91599491,60.36053328)(272.91599491,60.46080328)
\curveto(272.91599491,60.72221328)(274.83861489,62.83962326)(275.21091488,62.98814325)
\curveto(275.44972488,63.08344325)(275.64450488,63.41024325)(275.64450488,63.71471324)
\curveto(275.64450488,64.04013324)(275.99996487,64.51910324)(276.50583487,64.87682323)
\curveto(277.44109485,65.53820322)(280.16158482,68.79176319)(280.86130481,70.08580317)
\curveto(281.09626481,70.52033316)(281.26832481,70.70188316)(281.37302481,70.64049316)
\closepath
\moveto(261.39257504,69.70885317)
\curveto(261.58922504,69.70885317)(261.74999504,69.51729318)(261.74999504,69.28307318)
\curveto(261.74999504,68.85544318)(261.22621505,68.69962319)(260.92968505,69.03893318)
\curveto(260.69813505,69.30387318)(260.97808505,69.70885317)(261.39257504,69.70885317)
\closepath
\moveto(280.06445482,69.70885317)
\curveto(280.13265482,69.70885317)(279.98525483,69.45278318)(279.73827483,69.14049318)
\curveto(279.49127483,68.82820318)(279.23423483,68.57213319)(279.16601483,68.57213319)
\curveto(279.09781484,68.57213319)(279.24321483,68.82820318)(279.49023483,69.14049318)
\curveto(279.73722483,69.45278318)(279.99620482,69.70885317)(280.06445482,69.70885317)
\closepath
\moveto(289.77148471,68.85728318)
\curveto(289.87971471,68.85728318)(290.2726547,68.35409319)(290.6445247,67.7401032)
\curveto(291.42632469,66.44927321)(291.33359469,65.56897322)(290.3710947,65.15025323)
\curveto(289.52577471,64.78252323)(289.36922471,64.80956323)(289.18554472,65.35728323)
\curveto(288.99642472,65.92122322)(289.48788471,68.85728318)(289.77148471,68.85728318)
\closepath
\moveto(257.17187509,68.29283319)
\curveto(256.5608351,68.24833319)(256.2910151,67.7368932)(256.2910151,66.72643321)
\curveto(256.2910151,66.36159321)(256.3605151,65.86209322)(256.4433551,65.61510322)
\curveto(256.6869551,64.88872323)(258.46199508,65.01841323)(259.03515507,65.80455322)
\curveto(259.74803506,66.78232321)(259.63270506,67.4099932)(258.65624508,67.8768232)
\curveto(258.02756508,68.17738319)(257.53849509,68.31954319)(257.17187509,68.29283319)
\closepath
\moveto(257.14847509,67.7205732)
\curveto(257.43949509,67.7205732)(258.61685508,67.2432632)(258.97855507,66.97838321)
\curveto(259.02375507,66.94528321)(258.93755507,66.65170321)(258.78519507,66.32603321)
\curveto(258.51558508,65.74960322)(257.33220509,65.48997322)(256.9531651,65.92369322)
\curveto(256.6252351,66.29891321)(256.7798451,67.7205732)(257.14847509,67.7205732)
\closepath
\moveto(288.20902473,65.93150322)
\curveto(288.45179473,65.65372322)(287.50236474,63.38137325)(286.88284474,62.75767326)
\curveto(286.25155475,62.12213326)(285.99399475,62.06261326)(283.95706478,62.08971326)
\curveto(280.48018482,62.13591326)(280.04088482,62.34165326)(281.19534481,63.38072325)
\curveto(281.51814481,63.67124325)(281.8572048,64.18169324)(281.9492548,64.51353324)
\curveto(282.0880148,65.01375323)(282.3586348,65.13773323)(283.53324478,65.24205323)
\curveto(284.31248477,65.31125323)(285.66136476,65.51983322)(286.53128475,65.70494322)
\curveto(287.40121474,65.89005322)(288.15635473,65.99178322)(288.20902473,65.93150322)
\closepath
\moveto(294.50199465,65.44908322)
\curveto(294.51479465,65.43448322)(294.43949466,65.32071323)(294.28519466,65.09557323)
\curveto(293.96106466,64.62266323)(293.75784466,64.47128324)(293.75784466,64.70494323)
\curveto(293.75784466,64.76384323)(293.95263466,64.98674323)(294.19144466,65.20103323)
\curveto(294.38820466,65.37761323)(294.48918465,65.46374322)(294.50199465,65.44908322)
\closepath
\moveto(256.8711351,64.29088324)
\lineto(257.90433508,64.26938324)
\lineto(257.16019509,63.97056324)
\curveto(256.7507951,63.80636324)(256.2477051,63.51612325)(256.04300511,63.32602325)
\curveto(255.23278512,62.57361326)(255.05081512,62.56436326)(255.05081512,63.27332325)
\curveto(255.05081512,64.09260324)(255.46218511,64.32148324)(256.8711351,64.29090324)
\closepath
\moveto(293.74417466,64.11900324)
\curveto(293.80537466,64.06020324)(293.48028467,63.59998325)(293.01956467,63.09752325)
\lineto(292.18167468,62.18541326)
\lineto(292.90824467,63.20494325)
\curveto(293.30773467,63.76620324)(293.68294466,64.17780324)(293.74417466,64.11900324)
\closepath
\moveto(275.03910488,64.02920324)
\curveto(275.09930488,64.02920324)(275.14847488,63.90213324)(275.14847488,63.74600324)
\curveto(275.14847488,63.58985325)(275.03127488,63.46279325)(274.88675488,63.46279325)
\curveto(274.74225489,63.46279325)(274.69304489,63.58985325)(274.77738489,63.74600324)
\curveto(274.86168489,63.90213324)(274.97892488,64.02920324)(275.03910488,64.02920324)
\closepath
\moveto(258.80667507,63.59756325)
\lineto(258.72857507,62.32412326)
\curveto(258.65837508,61.19382327)(258.57946508,61.04307328)(258.02935508,60.99990328)
\curveto(257.68819509,60.97310328)(257.13028509,61.04500328)(256.7891251,61.15811327)
\lineto(256.1680251,61.36318327)
\lineto(256.9121651,61.89639327)
\curveto(257.32156509,62.19024326)(257.91598508,62.69470326)(258.23247508,63.01553325)
\closepath
\moveto(257.89261508,63.50776325)
\curveto(258.15903508,63.50376325)(258.05767508,63.28513325)(257.43949509,62.96284325)
\curveto(257.11513509,62.79374326)(256.4456451,62.24863326)(255.95120511,61.75191327)
\curveto(255.30652511,61.10424328)(255.05081512,60.97436328)(255.05081512,61.29292327)
\curveto(255.05081512,61.91375327)(255.88850511,62.73906326)(257.04691509,63.25972325)
\curveto(257.43987509,63.43634325)(257.73276509,63.51038325)(257.89261508,63.50776325)
\closepath
\moveto(273.6133149,62.32612326)
\curveto(273.6815149,62.32612326)(273.5341149,62.07200326)(273.2871449,61.75972327)
\curveto(273.04014491,61.44741327)(272.78311491,61.19136327)(272.71488491,61.19136327)
\curveto(272.64668491,61.19136327)(272.79208491,61.44741327)(273.03910491,61.75972327)
\curveto(273.2861049,62.07200326)(273.5450949,62.32612326)(273.6133149,62.32612326)
\closepath
\moveto(261.00589505,62.26952326)
\curveto(260.93339505,62.28282326)(260.85958505,62.28292326)(260.78910505,62.26752326)
\curveto(260.23094506,62.14589326)(260.14723506,61.07767328)(260.65042505,60.50190328)
\curveto(261.18921505,59.88540329)(261.85358504,60.04102329)(261.94925504,60.80463328)
\curveto(262.02825504,61.43496327)(261.51375504,62.17622326)(261.00589505,62.26947326)
\closepath
\moveto(280.63284482,61.94726327)
\curveto(280.88653481,61.94926327)(281.19130481,61.88296327)(281.62113481,61.74999327)
\curveto(282.3081848,61.53744327)(283.38669479,61.48139327)(284.51566477,61.59765327)
\curveto(285.52072476,61.70113327)(286.31449475,61.67225327)(286.31449475,61.53125327)
\curveto(286.31449475,61.15730327)(285.42794476,60.05469329)(285.12699476,60.05469329)
\curveto(284.98378477,60.05469329)(284.82873477,59.93353329)(284.78324477,59.78516329)
\curveto(284.73774477,59.63680329)(283.69696478,59.4403833)(282.4687848,59.3476633)
\curveto(280.46328482,59.1962233)(280.17660482,59.2378933)(279.65824483,59.75196329)
\curveto(278.97735484,60.42721328)(279.09318484,61.21091327)(279.95120483,61.73633327)
\curveto(280.17733482,61.87480327)(280.37916482,61.94514327)(280.63284482,61.94727327)
\closepath
\moveto(260.88480505,61.65038327)
\curveto(260.97790505,61.63428327)(261.12242505,61.51530327)(261.25394505,61.33398327)
\curveto(261.42929504,61.09221328)(261.51749504,60.83002328)(261.44925504,60.75195328)
\curveto(261.38105504,60.67385328)(261.18125505,60.80705328)(261.00589505,61.04882328)
\curveto(260.83055505,61.29059327)(260.74235505,61.55279327)(260.81058505,61.63085327)
\curveto(260.82768505,61.65035327)(260.85378505,61.65575327)(260.88478505,61.65035327)
\closepath
\moveto(291.67191469,61.45507327)
\curveto(291.75111469,61.49557327)(291.71981469,61.32687327)(291.53128469,60.92382328)
\curveto(291.38928469,60.62022328)(291.15187469,60.28373329)(291.0019947,60.17773329)
\curveto(290.8520847,60.07173329)(290.9216947,60.32087328)(291.15628469,60.73046328)
\curveto(291.40460469,61.16410327)(291.59266469,61.41454327)(291.67191469,61.45507327)
\closepath
\moveto(272.66800491,59.62890329)
\lineto(273.7226949,59.61130329)
\curveto(276.19378487,59.56890329)(276.14066487,59.59510329)(276.14066487,58.38474331)
\curveto(276.14066487,57.48141332)(276.04146487,57.25157332)(275.58402488,57.11130332)
\curveto(274.83038489,56.88019333)(273.03913491,56.88637333)(272.83402491,57.12130332)
\curveto(272.74302491,57.22538332)(272.66786491,57.83332331)(272.66800491,58.47091331)
\closepath
\moveto(255.97464511,58.8847633)
\curveto(256.7550151,58.9001633)(258.39980508,58.57881331)(259.02152507,58.21093331)
\curveto(259.88633506,57.69922332)(260.75784505,56.46848333)(260.75784505,55.75976334)
\curveto(260.75785505,55.07965335)(260.43205505,55.10951335)(258.83597507,55.93945334)
\curveto(257.50425509,56.63191333)(255.56078511,58.29314331)(255.55277511,58.7460933)
\curveto(255.55077511,58.8369933)(255.71452511,58.8796133)(255.97464511,58.8847633)
\closepath
\moveto(290.03519471,58.6367133)
\curveto(290.04799471,58.6221133)(289.97469471,58.50638331)(289.82034471,58.28124331)
\curveto(289.49621471,57.80834331)(289.29105472,57.65891332)(289.29105472,57.89257331)
\curveto(289.29105472,57.95157331)(289.48778471,58.17437331)(289.72660471,58.38866331)
\curveto(289.92336471,58.56524331)(290.02238471,58.6513733)(290.03519471,58.6367133)
\closepath
\moveto(255.05081512,57.56640332)
\lineto(255.73441511,57.05273332)
\curveto(256.10969511,56.77015333)(256.7212251,56.36411333)(257.09378509,56.15038333)
\curveto(257.46635509,55.93667334)(257.83511509,55.64235334)(257.91410508,55.49609334)
\curveto(257.99670508,55.34322334)(257.48278509,55.22851334)(256.7051151,55.22851334)
\curveto(255.26075512,55.22851334)(255.05081512,55.43333334)(255.05081512,56.84570333)
\closepath
\moveto(258.89652507,55.22851334)
\curveto(259.10122507,55.22851334)(259.26956507,55.10145335)(259.26956507,54.94530335)
\curveto(259.26956507,54.78915335)(259.10122507,54.66210335)(258.89652507,54.66210335)
\curveto(258.69182508,54.66210335)(258.52542508,54.78915335)(258.52542508,54.94530335)
\curveto(258.52542508,55.10145335)(258.69182508,55.22851334)(258.89652507,55.22851334)
\closepath
\moveto(280.83597482,55.19331334)
\curveto(280.92217481,55.18931335)(280.99591481,55.16431335)(281.03714481,55.11711335)
\curveto(281.11954481,55.02281335)(280.97254481,54.95638335)(280.71097482,54.96867335)
\curveto(280.42192482,54.98227335)(280.36264482,55.04797335)(280.56058482,55.13859335)
\curveto(280.65018482,55.17959335)(280.74978482,55.19739334)(280.83597482,55.19329334)
\closepath
\moveto(282.71683479,54.33003336)
\curveto(283.94502478,54.35113335)(285.11760476,54.24503336)(285.32230476,54.09370336)
\curveto(285.75568476,53.77323336)(283.40112478,53.77323336)(281.59964481,54.09370336)
\curveto(280.68610482,54.25621336)(280.88978481,54.29874336)(282.71683479,54.33003336)
\closepath
}
}
{
\newrgbcolor{curcolor}{0 1 1}
\pscustom[linestyle=none,fillstyle=solid,fillcolor=curcolor]
{
\newpath
\moveto(239.97993811,109.1391867)
\lineto(239.97993811,111.18701218)
\lineto(241.47261024,111.18701218)
\lineto(242.96528237,111.18701218)
\lineto(242.96528237,109.1391867)
\lineto(242.96528237,105.09136103)
\lineto(241.47261024,105.09136103)
\lineto(239.97993811,105.09136103)
\closepath
}
}
{
\newrgbcolor{curcolor}{0.53333336 0.70980394 0.85882354}
\pscustom[linewidth=0,linecolor=curcolor]
{
\newpath
\moveto(239.97993811,109.1391867)
\lineto(239.97993811,111.18701218)
\lineto(241.47261024,111.18701218)
\lineto(242.96528237,111.18701218)
\lineto(242.96528237,109.1391867)
\lineto(242.96528237,105.09136103)
\lineto(241.47261024,105.09136103)
\lineto(239.97993811,105.09136103)
\closepath
}
}
{
\newrgbcolor{curcolor}{0 1 1}
\pscustom[linestyle=none,fillstyle=solid,fillcolor=curcolor]
{
\newpath
\moveto(240.79360117,104.79385268)
\lineto(240.23588769,104.79385268)
\lineto(239.67820107,104.79385268)
\lineto(239.67820107,110.60364499)
\lineto(239.67820107,111.46594967)
\lineto(237.77274343,111.46594967)
\lineto(235.86728131,111.46594967)
\lineto(235.86728131,113.59942117)
\lineto(235.86728131,115.73292097)
\lineto(234.61555741,115.73292097)
\curveto(233.66266966,115.73292097)(233.35940607,115.73802229)(233.34526425,115.75422885)
\curveto(233.33075981,115.77083508)(233.32668604,117.92013369)(233.32668604,125.3194106)
\lineto(233.32668604,134.86327638)
\lineto(232.4436658,134.86327638)
\lineto(231.56064109,134.86327638)
\lineto(231.56064109,134.4721292)
\lineto(231.56064109,134.0809874)
\lineto(231.7000538,134.0809874)
\lineto(231.83949784,134.0809874)
\lineto(231.83949784,133.1919228)
\lineto(231.83949784,132.30285192)
\lineto(230.69698697,132.30738742)
\lineto(229.55450744,132.31192292)
\lineto(229.55450744,133.19197714)
\lineto(229.55450744,134.07205965)
\lineto(229.69003886,134.07716096)
\lineto(229.82558819,134.08226228)
\lineto(229.82558819,134.47271252)
\lineto(229.82558819,134.86316815)
\lineto(229.54673145,134.86316815)
\lineto(229.26790605,134.86316815)
\lineto(229.26790605,136.39218327)
\lineto(229.26790605,137.92117773)
\lineto(229.54673145,137.92117773)
\lineto(229.82558819,137.92117773)
\lineto(229.82558819,138.09829586)
\lineto(229.82558819,138.27538706)
\lineto(229.69003886,138.28050634)
\lineto(229.55450744,138.28558071)
\lineto(229.55450744,139.17453933)
\lineto(229.55450744,140.06349346)
\lineto(230.04633766,140.06802897)
\lineto(230.53820368,140.07256447)
\lineto(230.53820368,144.83704319)
\lineto(230.53820368,149.60152641)
\lineto(229.14783259,149.60606191)
\lineto(227.75746598,149.61059741)
\lineto(227.75343697,153.91760104)
\lineto(227.74940796,158.22461364)
\lineto(239.87929344,158.22461364)
\lineto(252.00917892,158.22461364)
\lineto(252.00917892,153.91315535)
\lineto(252.00917892,149.60169256)
\lineto(250.61495341,149.60169256)
\lineto(249.22071895,149.60169256)
\lineto(249.22071895,144.83702074)
\lineto(249.22071895,140.07231749)
\lineto(250.30123678,140.06778198)
\lineto(251.38178147,140.06324648)
\lineto(251.38178147,139.17429684)
\lineto(251.38178147,138.28533822)
\lineto(251.24568151,138.28021894)
\lineto(251.10957259,138.27509966)
\lineto(251.11409404,138.10246314)
\lineto(251.11861548,137.92982213)
\lineto(252.49349727,137.92528663)
\lineto(253.86835219,137.92075112)
\lineto(253.86851783,137.16967674)
\curveto(253.86851783,136.7566087)(253.86851783,134.67248752)(253.86835219,132.5383532)
\lineto(253.86816417,128.65807037)
\lineto(257.32674252,128.65353487)
\lineto(260.78532536,128.64899937)
\lineto(260.78935437,122.84858118)
\lineto(260.79338338,117.04816839)
\lineto(257.3348856,117.04363288)
\lineto(253.87638783,117.03909738)
\lineto(253.86868794,116.39015894)
\lineto(253.86098805,115.74122049)
\lineto(250.48003977,115.73668499)
\lineto(247.09908253,115.73214948)
\lineto(247.09505352,113.60312681)
\lineto(247.09102451,111.47410998)
\lineto(245.18958245,111.46957448)
\lineto(243.28814486,111.46503898)
\lineto(243.28411585,110.60727429)
\lineto(243.28008684,104.80200985)
\lineto(242.72631283,104.79747435)
\lineto(242.17250749,104.79293885)
\curveto(240.79360117,104.79383696)(243.28008684,104.80200985)(240.79360117,104.79383696)
\closepath
\moveto(242.10982057,104.79380194)
\lineto(241.48240074,104.79380194)
\closepath
\moveto(243.22518934,110.63915079)
\lineto(243.22518934,111.46589893)
\lineto(241.48240074,111.46589893)
\lineto(239.73960766,111.46589893)
\lineto(239.73960766,110.63915079)
\lineto(239.73960766,104.86493704)
\lineto(241.48240074,104.86493704)
\lineto(243.22518934,104.86493704)
\closepath
\moveto(247.03610463,113.63495482)
\lineto(247.03610463,115.73287022)
\lineto(241.48240074,115.73287022)
\lineto(235.9286879,115.73287022)
\lineto(235.9286879,113.63495482)
\lineto(235.9286879,111.53701112)
\lineto(241.48240074,111.53701112)
\lineto(247.03610463,111.53701112)
\closepath
\moveto(241.36180799,113.40392437)
\curveto(241.1362461,113.44926189)(240.90364687,113.67692619)(240.83511788,113.91940496)
\curveto(240.71477583,114.34516163)(240.93294673,114.81809291)(241.30238457,114.93225377)
\curveto(241.46850065,114.98359881)(241.69636356,114.94579741)(241.83694915,114.84339159)
\curveto(242.18757151,114.58819544)(242.26701464,114.07324205)(242.01549249,113.68632815)
\curveto(241.94367315,113.57586799)(241.80283686,113.46571993)(241.68377514,113.42692745)
\curveto(241.5929523,113.3973434)(241.44652912,113.38686235)(241.36180799,113.40397601)
\closepath
\moveto(241.47046591,113.4323723)
\curveto(241.46540727,113.4381517)(241.4518877,113.43900492)(241.44074077,113.43520138)
\curveto(241.42842991,113.42942198)(241.43196648,113.42522776)(241.45005226,113.42443293)
\curveto(241.46648167,113.42443293)(241.47547979,113.42784578)(241.47028685,113.43384073)
\closepath
\moveto(241.74500266,113.54426631)
\curveto(241.88061914,113.62096661)(241.96726076,113.72047196)(242.03392745,113.87616231)
\curveto(242.08319329,113.99117908)(242.08658214,114.01061752)(242.08658214,114.17744411)
\curveto(242.08658214,114.34325627)(242.08300079,114.36415685)(242.03550772,114.47486085)
\curveto(241.93157716,114.71717482)(241.74784535,114.85892951)(241.51823207,114.87397031)
\curveto(241.41389414,114.88088583)(241.37184022,114.87397031)(241.2965246,114.84501045)
\curveto(241.03594613,114.74024572)(240.8705105,114.48090744)(240.8705105,114.1772115)
\curveto(240.8705105,114.02981797)(240.90475709,113.90775141)(240.98314372,113.77542699)
\curveto(241.05695518,113.6508668)(241.131371,113.58332463)(241.26615929,113.51851408)
\curveto(241.34544125,113.48037544)(241.37824187,113.47482057)(241.49831533,113.47895641)
\curveto(241.62131653,113.48290814)(241.65102376,113.4911394)(241.74502504,113.54423128)
\closepath
\moveto(242.13556594,114.1905809)
\curveto(242.1310445,114.20276389)(242.12764223,114.19340998)(242.12764223,114.16847818)
\curveto(242.12764223,114.14416609)(242.13122357,114.13402183)(242.13556594,114.14637546)
\curveto(242.14008739,114.15855845)(242.14008739,114.17862244)(242.13556594,114.19086381)
\closepath
\moveto(240.83427179,114.20826038)
\curveto(240.82975035,114.22044337)(240.82634807,114.21108946)(240.82634807,114.18615766)
\curveto(240.82634807,114.16184557)(240.82992942,114.15170131)(240.83427179,114.16405494)
\curveto(240.83879324,114.17623793)(240.83879324,114.19630192)(240.83427179,114.20854329)
\closepath
\moveto(235.25984984,115.78171265)
\curveto(234.92179351,115.78454173)(234.36410689,115.78454173)(234.02052186,115.78171265)
\curveto(233.67693684,115.77888358)(233.9535239,115.77661133)(234.63516526,115.77661133)
\curveto(235.31677975,115.77661133)(235.59788379,115.77944041)(235.25984984,115.78171265)
\closepath
\moveto(244.25799473,115.78171265)
\curveto(242.72577563,115.78454173)(240.22312916,115.78454173)(238.69654172,115.78171265)
\curveto(237.16995875,115.77888358)(238.42359419,115.77717715)(241.48240074,115.77717715)
\curveto(244.54119833,115.77717715)(245.79021384,115.78000623)(244.25799473,115.78171265)
\closepath
\moveto(252.1379237,115.78171265)
\curveto(251.21627421,115.78454173)(249.70353313,115.78454173)(248.77627437,115.78171265)
\curveto(247.84901561,115.77888358)(248.6030717,115.77717715)(250.45198443,115.77717715)
\curveto(252.30089268,115.77717715)(253.05957766,115.78000623)(252.1379237,115.78171265)
\closepath
\moveto(233.36872652,130.10282931)
\curveto(233.36604051,132.72590883)(233.36469751,130.57973882)(233.36469751,125.33356496)
\curveto(233.36469751,120.08736281)(233.36738351,117.94120223)(233.36872652,120.56430061)
\curveto(233.37141252,123.18740438)(233.37141252,127.47973093)(233.36872652,130.10282931)
\closepath
\moveto(253.80589358,125.34243973)
\lineto(253.80589358,134.8631942)
\lineto(252.50462629,134.8631942)
\lineto(251.20333213,134.8631942)
\lineto(251.20333213,134.47204747)
\lineto(251.20333213,134.08090567)
\lineto(251.29626797,134.08090567)
\lineto(251.38923066,134.08090567)
\lineto(251.38923066,133.1928326)
\lineto(251.38923066,132.30476491)
\lineto(251.30014925,132.29966359)
\lineto(251.21107679,132.29456227)
\lineto(251.20704778,124.66281164)
\lineto(251.20301877,117.03109289)
\lineto(243.19781445,117.0356284)
\lineto(235.19258775,117.0401639)
\lineto(235.18855874,125.95198082)
\lineto(235.18452973,134.86377394)
\lineto(234.29376485,134.86377394)
\lineto(233.4030313,134.86377394)
\lineto(233.4030313,125.34301901)
\lineto(233.4030313,115.82229687)
\lineto(243.60419384,115.82229687)
\lineto(253.80534295,115.82229687)
\lineto(253.80534295,125.34301901)
\closepath
\moveto(235.22770729,130.42139587)
\curveto(235.22502129,132.86926172)(235.22367828,130.87176595)(235.22367828,125.98250341)
\curveto(235.22367828,121.09321841)(235.22636429,119.0904161)(235.22770729,121.53180561)
\curveto(235.2303933,123.97320096)(235.2303933,127.97350622)(235.22770729,130.42139587)
\closepath
\moveto(251.14136596,124.70240479)
\lineto(251.14136596,132.30299112)
\lineto(251.07939978,132.30299112)
\lineto(251.01743809,132.30299112)
\lineto(251.01743809,124.78239175)
\lineto(251.01743809,117.26179687)
\lineto(243.1980741,117.26633238)
\lineto(235.37874145,117.27086788)
\lineto(235.37874145,125.99152367)
\lineto(235.37874145,134.71220595)
\lineto(242.26860386,134.71674145)
\lineto(249.15846179,134.72127696)
\lineto(249.15846179,134.40124242)
\lineto(249.15846179,134.08118499)
\lineto(249.26687797,134.08118499)
\lineto(249.37532549,134.08118499)
\lineto(249.37532549,134.47232633)
\lineto(249.37532549,134.86347351)
\lineto(242.31893962,134.86347351)
\lineto(235.26255823,134.86347351)
\lineto(235.26255823,125.98278272)
\lineto(235.26255823,117.10209552)
\lineto(243.20196433,117.10209552)
\lineto(251.14136596,117.10209552)
\lineto(251.14136596,124.70268365)
\closepath
\moveto(260.73060692,122.85338298)
\lineto(260.73060692,128.5871378)
\lineto(257.29924229,128.5871378)
\lineto(253.86785975,128.5871378)
\lineto(253.86785975,122.85338298)
\lineto(253.86785975,117.11959987)
\lineto(257.29924229,117.11959987)
\lineto(260.73060692,117.11959987)
\closepath
\moveto(235.41406244,130.325126)
\curveto(235.41137643,132.70862901)(235.41003343,130.75848839)(235.41003343,125.99143431)
\curveto(235.41003343,121.22441435)(235.41271944,119.27424948)(235.41406244,121.65777674)
\curveto(235.41674845,124.04130356)(235.41674845,127.94159918)(235.41406244,130.325126)
\closepath
\moveto(250.95593749,124.81803273)
\lineto(250.95593749,132.30304726)
\lineto(250.02643143,132.30304726)
\lineto(249.09692985,132.30304726)
\lineto(249.09692985,133.47648297)
\lineto(249.09692985,134.64991284)
\lineto(242.27292385,134.64991284)
\lineto(235.44891785,134.64991284)
\lineto(235.44891785,125.99143431)
\lineto(235.44891785,117.33298811)
\lineto(243.20242543,117.33298811)
\lineto(250.95593749,117.33298811)
\closepath
\moveto(231.79292248,133.19200453)
\lineto(231.79292248,134.00984961)
\lineto(230.70077872,134.00984961)
\lineto(229.60860809,134.00984961)
\lineto(229.60860809,133.19200453)
\lineto(229.60860809,132.3741599)
\lineto(230.70077872,132.3741599)
\lineto(231.79292248,132.3741599)
\closepath
\moveto(231.82000638,133.60537882)
\curveto(231.81732037,133.83273147)(231.81548494,133.64669232)(231.81548494,133.19200453)
\curveto(231.81548494,132.73729923)(231.81817094,132.55130633)(231.82000638,132.77863069)
\curveto(231.82269239,133.00598873)(231.82269239,133.37802617)(231.82000638,133.60537882)
\closepath
\moveto(251.33948582,133.1964592)
\lineto(251.34351483,134.00984961)
\lineto(250.25120095,134.00984961)
\lineto(249.15889155,134.00984961)
\lineto(249.15889155,133.19189676)
\lineto(249.15889155,132.37393897)
\lineto(250.24715403,132.37847447)
\lineto(251.3354389,132.38300997)
\lineto(251.33946791,133.19640621)
\closepath
\moveto(251.37033013,133.60537882)
\curveto(251.36764412,133.83273147)(251.36580868,133.64669232)(251.36580868,133.19200453)
\curveto(251.36580868,132.73729923)(251.36849469,132.55130633)(251.37033013,132.77863069)
\curveto(251.37301613,133.00598873)(251.37301613,133.37802617)(251.37033013,133.60537882)
\closepath
\moveto(231.51052021,134.47655782)
\lineto(231.51454922,134.86325033)
\lineto(230.70095778,134.86325033)
\lineto(229.88734397,134.86325033)
\lineto(229.88734397,134.47193924)
\lineto(229.88734397,134.080656)
\lineto(230.69677209,134.0851915)
\lineto(231.50620469,134.089727)
\lineto(231.5102337,134.47641996)
\closepath
\moveto(231.54114069,134.67211749)
\curveto(231.53845468,134.78212949)(231.53603727,134.69201084)(231.53603727,134.4721036)
\curveto(231.53603727,134.252107)(231.53872328,134.16208265)(231.54114069,134.27209509)
\curveto(231.54382669,134.38210754)(231.54382669,134.5621275)(231.54114069,134.67211749)
\closepath
\moveto(251.04888675,134.4721036)
\lineto(251.04888675,134.86325033)
\lineto(250.2433399,134.86325033)
\lineto(249.43777067,134.86325033)
\lineto(249.43777067,134.4721036)
\lineto(249.43777067,134.08096181)
\lineto(250.2433399,134.08096181)
\lineto(251.04888675,134.08096181)
\closepath
\moveto(251.09141519,134.67211749)
\curveto(251.08872918,134.78212949)(251.08631178,134.69201084)(251.08631178,134.4721036)
\curveto(251.08631178,134.252107)(251.08899778,134.16208265)(251.09141519,134.27209509)
\curveto(251.09410119,134.38210754)(251.09410119,134.5621275)(251.09141519,134.67211749)
\closepath
\moveto(251.14180915,134.4721036)
\curveto(251.14180915,134.72102548)(251.13603424,134.86325033)(251.12623031,134.86325033)
\curveto(251.11638162,134.86325033)(251.11065147,134.72102548)(251.11065147,134.4721036)
\curveto(251.11065147,134.22321539)(251.11642639,134.08096181)(251.12623031,134.08096181)
\curveto(251.136079,134.08096181)(251.14180915,134.22321539)(251.14180915,134.4721036)
\closepath
\moveto(253.80633677,135.64553931)
\lineto(253.80633677,136.35671519)
\lineto(241.56806197,136.35671519)
\lineto(229.32976031,136.35671519)
\lineto(229.32976031,135.64553931)
\lineto(229.32976031,134.93436298)
\lineto(241.56806197,134.93436298)
\lineto(253.80633677,134.93436298)
\closepath
\moveto(253.80633677,137.13897542)
\lineto(253.80633677,137.85014995)
\lineto(241.56806197,137.85014995)
\lineto(229.32976031,137.85014995)
\lineto(229.32976031,137.13897542)
\lineto(229.32976031,136.42782783)
\lineto(241.56806197,136.42782783)
\lineto(253.80633677,136.42782783)
\closepath
\moveto(233.15195235,137.89645699)
\curveto(233.14684893,137.90229477)(233.13337413,137.90310308)(233.12227197,137.89645699)
\curveto(233.10996111,137.89079884)(233.11349768,137.88648787)(233.13158346,137.88567956)
\curveto(233.14801287,137.88567956)(233.15701099,137.88927204)(233.15181805,137.89506491)
\closepath
\moveto(233.32236157,137.89645699)
\curveto(233.31725815,137.90229477)(233.30378335,137.90310308)(233.29268119,137.89645699)
\curveto(233.28037033,137.89079884)(233.2839069,137.88648787)(233.30199268,137.88567956)
\curveto(233.31837732,137.88567956)(233.32742021,137.88927204)(233.32222727,137.89506491)
\closepath
\moveto(231.51404783,138.09989901)
\lineto(231.51404783,138.27685099)
\lineto(230.70076081,138.27685099)
\lineto(229.88744693,138.27685099)
\lineto(229.88744693,138.09965202)
\lineto(229.88744693,137.92244857)
\lineto(230.70076081,137.92270004)
\lineto(231.51404783,137.92294702)
\closepath
\moveto(232.39706806,137.98073652)
\curveto(232.39706806,138.01495487)(232.40852836,138.05198434)(232.42415197,138.06796637)
\curveto(232.44438655,138.08871293)(233.1961282,138.59223461)(233.65959419,138.89554302)
\curveto(233.68314151,138.91103558)(233.46609426,138.91602014)(232.76533302,138.91628958)
\lineto(231.84006637,138.91628958)
\lineto(231.83603736,138.60070837)
\lineto(231.83200835,138.28512716)
\lineto(231.69645007,138.28000788)
\lineto(231.56089627,138.2748886)
\lineto(231.56089627,138.10963013)
\curveto(231.56089627,138.01874943)(231.56541771,137.93904133)(231.57128216,137.93253894)
\curveto(231.57705707,137.92589286)(231.7651626,137.92063886)(231.9895516,137.92063886)
\lineto(232.39747992,137.92063886)
\lineto(232.39747992,137.98011233)
\closepath
\moveto(233.16423187,137.93753698)
\curveto(233.35442353,137.96156167)(233.85689929,138.00273596)(234.59687175,138.05484934)
\curveto(235.28316438,138.10319062)(237.06997663,138.19240531)(240.64630057,138.35691385)
\curveto(244.13105853,138.51718775)(245.67343074,138.60729608)(246.58730872,138.70401007)
\curveto(247.13878619,138.76237436)(247.29989735,138.80257868)(247.29989735,138.88176138)
\curveto(247.29989735,138.91566539)(247.28812368,138.91687785)(246.96236031,138.91687785)
\curveto(246.776708,138.91687785)(246.50310688,138.9052472)(246.35433345,138.89087729)
\curveto(245.39567079,138.79895027)(243.93540085,138.71763454)(239.95694142,138.53460229)
\curveto(237.94495227,138.44200617)(237.49070377,138.42024923)(236.23895749,138.35648724)
\curveto(234.27297963,138.25631549)(233.02903173,138.15758073)(232.62169881,138.06941234)
\curveto(232.48811474,138.0404929)(232.45903424,138.0233927)(232.45903424,137.97380304)
\curveto(232.45903424,137.95642443)(232.46780853,137.93832732)(232.47846302,137.93397144)
\curveto(232.51772349,137.91794001)(233.02966294,137.92040984)(233.16429455,137.93756392)
\closepath
\moveto(233.44628496,137.93187882)
\curveto(233.44118155,137.9377166)(233.42770675,137.93852491)(233.41660459,137.93457318)
\curveto(233.40429372,137.92878031)(233.4078303,137.92460406)(233.42591608,137.92379575)
\curveto(233.44230072,137.92379575)(233.45134361,137.92738823)(233.44615066,137.9331811)
\closepath
\moveto(233.51457221,137.95289482)
\curveto(233.50333575,137.95693636)(233.48690634,137.95289482)(233.47790821,137.94184795)
\curveto(233.46644792,137.92860069)(233.4722676,137.92635539)(233.4981428,137.93407922)
\curveto(233.51922795,137.94041096)(233.52612204,137.94849403)(233.51452744,137.95289482)
\closepath
\moveto(245.96519822,137.95558918)
\curveto(245.99725124,137.97440478)(246.24966424,138.14119454)(246.52613939,138.32630145)
\curveto(246.80261901,138.51143979)(247.04971373,138.67610999)(247.07528899,138.69228512)
\curveto(247.11795173,138.71909398)(247.11862323,138.72098003)(247.08298887,138.71303168)
\curveto(246.92778693,138.6785888)(246.43650287,138.62621496)(245.83592072,138.58012347)
\curveto(244.79944892,138.50060847)(243.49276485,138.43040696)(240.35966336,138.28594445)
\curveto(237.71122044,138.16382266)(237.52445344,138.15503007)(236.75788951,138.11648727)
\curveto(235.49148211,138.05286898)(234.31550807,137.98200287)(233.83773463,137.94059058)
\curveto(233.7482727,137.93282185)(236.4272197,137.92541236)(239.79097754,137.92397537)
\curveto(245.792967,137.92128101)(245.90795494,137.92128101)(245.96515793,137.95558918)
\closepath
\moveto(249.37576869,138.09858775)
\lineto(249.37576869,138.27567445)
\lineto(249.2402104,138.28079373)
\lineto(249.10467898,138.28591302)
\lineto(249.10064997,138.60149422)
\lineto(249.09662096,138.91707543)
\lineto(248.23039722,138.91707543)
\lineto(247.36417349,138.91707543)
\lineto(247.35907007,138.85367269)
\lineto(247.35396666,138.79027893)
\lineto(246.72662741,138.37206512)
\curveto(246.3816188,138.14205673)(246.08881273,137.94680108)(246.07601391,137.93817913)
\curveto(246.06236004,137.92906322)(246.74712164,137.92241713)(247.71415569,137.92187826)
\lineto(249.375612,137.92187826)
\lineto(249.375612,138.09899191)
\closepath
\moveto(251.04885093,138.09858775)
\lineto(251.04885093,138.27601125)
\lineto(250.24330409,138.27601125)
\lineto(249.43773486,138.27601125)
\lineto(249.43773486,138.1085928)
\curveto(249.43773486,137.99111878)(249.44350978,137.93858777)(249.45716364,137.93252996)
\curveto(249.4678629,137.92799446)(249.83031713,137.92314461)(250.26270601,137.92256083)
\lineto(251.04891361,137.92256083)
\lineto(251.04891361,138.09997984)
\closepath
\moveto(251.09128984,138.19153862)
\curveto(251.08860383,138.24288411)(251.08551493,138.20092397)(251.08551493,138.09822851)
\curveto(251.08551493,137.99554651)(251.08820093,137.95354595)(251.09128984,138.00489144)
\curveto(251.09397585,138.05617855)(251.09397585,138.14022906)(251.09128984,138.19153862)
\closepath
\moveto(231.54102429,138.19153862)
\curveto(231.53744295,138.24316702)(231.53524938,138.20536966)(231.53524938,138.10768121)
\curveto(231.53524938,138.00989397)(231.53793538,137.967615)(231.54102429,138.01373792)
\curveto(231.54460563,138.05986982)(231.54460563,138.13986532)(231.54118545,138.19153862)
\closepath
\moveto(233.72517304,137.96728718)
\curveto(233.72011439,137.97308005)(233.70659483,137.97393327)(233.69553743,137.96998154)
\curveto(233.68322657,137.96432339)(233.68676314,137.96001241)(233.70484892,137.95920411)
\curveto(233.72127833,137.95920411)(233.73027645,137.96279658)(233.72508351,137.96858945)
\closepath
\moveto(233.86459022,137.96728718)
\curveto(233.85953157,137.97308005)(233.84601201,137.97393327)(233.83495461,137.96998154)
\curveto(233.82264375,137.96432339)(233.82618032,137.96001241)(233.8442661,137.95920411)
\curveto(233.86069551,137.95920411)(233.86969363,137.96279658)(233.86450069,137.96858945)
\closepath
\moveto(233.95755291,137.98498013)
\curveto(233.9524495,137.99077301)(233.9389747,137.99162622)(233.92787254,137.98767449)
\curveto(233.91556167,137.98183672)(233.91909825,137.97770537)(233.93718403,137.97689706)
\curveto(233.95361343,137.97689706)(233.96261156,137.98048954)(233.95741861,137.98632731)
\closepath
\moveto(234.20541313,138.00267309)
\curveto(234.20030972,138.00846596)(234.18683492,138.00931917)(234.17577753,138.00536745)
\curveto(234.16346666,137.99957458)(234.16700324,137.99539832)(234.18508902,137.99459001)
\curveto(234.20151843,137.99459001)(234.21051655,137.99818249)(234.2053236,138.00397536)
\closepath
\moveto(234.34482584,138.00267309)
\curveto(234.33972242,138.00846596)(234.32624762,138.00931917)(234.31514546,138.00536745)
\curveto(234.3028346,137.99957458)(234.30637117,137.99539832)(234.32445695,137.99459001)
\curveto(234.34084159,137.99459001)(234.34988448,137.99818249)(234.34469154,138.00397536)
\closepath
\moveto(234.46134481,138.02148869)
\curveto(234.45064555,138.02660797)(234.43327604,138.02660797)(234.42257678,138.02148869)
\curveto(234.41187752,138.01636941)(234.41989077,138.01264221)(234.44200556,138.01264221)
\curveto(234.46335931,138.01264221)(234.4719993,138.01668375)(234.46143434,138.02148869)
\closepath
\moveto(234.74020155,138.03918164)
\curveto(234.72954706,138.04425602)(234.71213278,138.04425602)(234.70143352,138.03918164)
\curveto(234.69077903,138.03406236)(234.69874751,138.03033517)(234.7208623,138.03033517)
\curveto(234.74221605,138.03033517)(234.75085604,138.0343767)(234.74029108,138.03918164)
\closepath
\moveto(234.89509908,138.03918164)
\curveto(234.88439982,138.04425602)(234.86703031,138.04425602)(234.85642058,138.03918164)
\curveto(234.84572132,138.03406236)(234.85373458,138.03033517)(234.87584937,138.03033517)
\curveto(234.89720312,138.03033517)(234.90584311,138.0343767)(234.89527815,138.03918164)
\closepath
\moveto(235.01902695,138.0568746)
\curveto(235.00832769,138.06199388)(234.99095818,138.06199388)(234.98034846,138.0568746)
\curveto(234.9696492,138.05175532)(234.97766245,138.04802812)(234.99977724,138.04802812)
\curveto(235.02113099,138.04802812)(235.02977098,138.05206966)(235.01920602,138.0568746)
\closepath
\moveto(235.10392267,138.0568746)
\curveto(235.09881926,138.06266747)(235.08534446,138.06352068)(235.0742423,138.0568746)
\curveto(235.06193143,138.05108173)(235.06546801,138.04690547)(235.08355379,138.04609716)
\curveto(235.0999832,138.04609716)(235.10898132,138.04968964)(235.10378837,138.05548251)
\closepath
\moveto(235.18944065,138.0568746)
\curveto(235.17878616,138.06199388)(235.16137188,138.06199388)(235.15076215,138.0568746)
\curveto(235.14006289,138.05175532)(235.14807615,138.04802812)(235.17019094,138.04802812)
\curveto(235.19154469,138.04802812)(235.20022944,138.05206966)(235.18961972,138.0568746)
\closepath
\moveto(235.32888021,138.07456755)
\curveto(235.31818095,138.07964192)(235.30081144,138.07964192)(235.29011218,138.07456755)
\curveto(235.27945769,138.06944827)(235.28742618,138.06572107)(235.30954096,138.06572107)
\curveto(235.33089472,138.06572107)(235.33953471,138.06976261)(235.32896975,138.07456755)
\closepath
\moveto(235.41374459,138.07456755)
\curveto(235.40864118,138.08036042)(235.39516638,138.08121363)(235.38406422,138.07726191)
\curveto(235.37175336,138.07146904)(235.37528993,138.06729278)(235.39337571,138.06648447)
\curveto(235.40976035,138.06648447)(235.41880324,138.07007695)(235.41361029,138.07586982)
\closepath
\moveto(235.49926705,138.07456755)
\curveto(235.48856779,138.07964192)(235.47119828,138.07964192)(235.46054379,138.07456755)
\curveto(235.44988929,138.06944827)(235.45785778,138.06572107)(235.47997257,138.06572107)
\curveto(235.50132632,138.06572107)(235.50996631,138.06976261)(235.49940135,138.07456755)
\closepath
\moveto(235.63870661,138.0922605)
\curveto(235.62805212,138.09737978)(235.61063784,138.09737978)(235.59993858,138.0922605)
\curveto(235.58923932,138.08714122)(235.59725257,138.08341403)(235.61936736,138.08341403)
\curveto(235.64072112,138.08341403)(235.6493611,138.08745556)(235.63879614,138.0922605)
\closepath
\moveto(235.82490507,138.09209884)
\curveto(235.81451918,138.09717322)(235.79347879,138.09717322)(235.77840582,138.09209884)
\curveto(235.76336418,138.08756334)(235.77186987,138.0835218)(235.797566,138.08325236)
\curveto(235.82299353,138.08325236)(235.83548346,138.08684484)(235.82496327,138.09151506)
\closepath
\moveto(235.96401336,138.11006123)
\curveto(235.9533141,138.11518051)(235.93594459,138.11518051)(235.92533486,138.11006123)
\curveto(235.9146356,138.10494195)(235.92264886,138.10121475)(235.94476365,138.10121475)
\curveto(235.9661174,138.10121475)(235.97480215,138.10525629)(235.96419243,138.11006123)
\closepath
\moveto(236.05698053,138.11006123)
\curveto(236.04628127,138.11518051)(236.02891176,138.11518051)(236.0182125,138.11006123)
\curveto(236.00751324,138.10494195)(236.01552649,138.10121475)(236.03764128,138.10121475)
\curveto(236.05899504,138.10121475)(236.06763502,138.10525629)(236.05707006,138.11006123)
\closepath
\moveto(236.14991189,138.11006123)
\curveto(236.13925739,138.11518051)(236.12184312,138.11518051)(236.11123339,138.11006123)
\curveto(236.10053413,138.10494195)(236.10854738,138.10121475)(236.13066217,138.10121475)
\curveto(236.15201592,138.10121475)(236.16065591,138.10525629)(236.15009095,138.11006123)
\closepath
\moveto(236.30513621,138.12775418)
\curveto(236.29475032,138.13287346)(236.27370994,138.13287346)(236.25864144,138.12775418)
\curveto(236.2435998,138.12321868)(236.25210549,138.11917714)(236.27780162,138.11890771)
\curveto(236.30322915,138.11890771)(236.31571908,138.12250018)(236.30519889,138.1271704)
\closepath
\moveto(236.49103474,138.12775418)
\curveto(236.48064885,138.13287346)(236.45960846,138.13287346)(236.44453549,138.12775418)
\curveto(236.42949385,138.12321868)(236.43799954,138.11917714)(236.46369567,138.11890771)
\curveto(236.4891232,138.11890771)(236.50161313,138.12250018)(236.49109294,138.1271704)
\closepath
\moveto(232.93156102,138.17768064)
\curveto(233.64217543,138.27141288)(235.36605902,138.37858098)(238.51623904,138.52486667)
\curveto(238.93800925,138.54449058)(239.37023697,138.56469827)(239.47673714,138.56992084)
\curveto(239.58324178,138.57504012)(240.19668095,138.60324107)(240.8399706,138.63246138)
\curveto(243.21893542,138.74040187)(245.64041077,138.87190003)(245.85921735,138.90501818)
\curveto(245.8805711,138.90861066)(243.18194463,138.91220314)(239.8623716,138.91359522)
\lineto(233.82678467,138.91628958)
\lineto(233.4101537,138.64754979)
\curveto(232.96408408,138.35984172)(232.71808616,138.19693183)(232.66821597,138.15620661)
\curveto(232.63911756,138.13245135)(232.64010243,138.13159813)(232.6837948,138.14102839)
\curveto(232.70922233,138.14668654)(232.82088411,138.16312213)(232.93165503,138.17767166)
\closepath
\moveto(236.64593227,138.14561778)
\curveto(236.63554638,138.15073706)(236.61455076,138.15073706)(236.59949121,138.14561778)
\curveto(236.58444958,138.14108227)(236.59295526,138.13704074)(236.6186514,138.1367713)
\curveto(236.64407893,138.1367713)(236.65656886,138.14036378)(236.64604866,138.145034)
\closepath
\moveto(236.73859055,138.14581087)
\curveto(236.72793606,138.15093015)(236.71052178,138.15093015)(236.69991205,138.14581087)
\curveto(236.68921279,138.14069159)(236.69722605,138.1369644)(236.71934083,138.1369644)
\curveto(236.74069459,138.1369644)(236.74937934,138.14100593)(236.73876962,138.14581087)
\closepath
\moveto(236.84734248,138.14561778)
\curveto(236.83695659,138.15073706)(236.81591621,138.15073706)(236.80084323,138.14561778)
\curveto(236.78580159,138.14108227)(236.79430728,138.13704074)(236.82000341,138.1367713)
\curveto(236.84543094,138.1367713)(236.85792087,138.14036378)(236.84740068,138.145034)
\closepath
\moveto(237.00224002,138.16331073)
\curveto(236.99185412,138.16843001)(236.9708585,138.16843001)(236.95579896,138.16331073)
\curveto(236.94075732,138.15877523)(236.94926301,138.15473369)(236.97495914,138.15446425)
\curveto(237.00038667,138.15446425)(237.0128766,138.15805673)(237.00235641,138.16272695)
\closepath
\moveto(237.09489829,138.16347688)
\curveto(237.08419903,138.16859616)(237.06682952,138.16859616)(237.05617503,138.16347688)
\curveto(237.04547577,138.1583576)(237.05348902,138.15463041)(237.07560381,138.15463041)
\curveto(237.09695757,138.15463041)(237.10559755,138.15867194)(237.09503259,138.16347688)
\closepath
\moveto(237.20364575,138.16331073)
\curveto(237.19325986,138.16843001)(237.17221947,138.16843001)(237.1571465,138.16331073)
\curveto(237.14210486,138.15877523)(237.15061055,138.15473369)(237.17630668,138.15446425)
\curveto(237.20173421,138.15446425)(237.21422414,138.15805673)(237.20370395,138.16272695)
\closepath
\moveto(237.37405497,138.18100368)
\curveto(237.36366908,138.18612296)(237.34262869,138.18612296)(237.3275602,138.18100368)
\curveto(237.31251856,138.17646818)(237.32102425,138.17242664)(237.34672038,138.17215721)
\curveto(237.37214791,138.17215721)(237.38463784,138.17574968)(237.37411765,138.18041991)
\closepath
\moveto(237.46669087,138.18116984)
\curveto(237.45603637,138.18628912)(237.4386221,138.18628912)(237.42801237,138.18116984)
\curveto(237.41731311,138.17605055)(237.42532636,138.17232336)(237.44744115,138.17232336)
\curveto(237.46879491,138.17232336)(237.47743489,138.1763649)(237.46686993,138.18116984)
\closepath
\moveto(237.57543833,138.18100368)
\curveto(237.56505243,138.18612296)(237.54401205,138.18612296)(237.52893907,138.18100368)
\curveto(237.51389744,138.17646818)(237.52240312,138.17242664)(237.54809925,138.17215721)
\curveto(237.57352678,138.17215721)(237.58601672,138.17574968)(237.57549652,138.18041991)
\closepath
\moveto(237.73780296,138.19923551)
\curveto(237.72298515,138.20377101)(237.69849772,138.20377101)(237.683568,138.19923551)
\curveto(237.66852637,138.19470001)(237.680882,138.19119734)(237.7106519,138.19119734)
\curveto(237.74046658,138.19119734)(237.75268791,138.19478982)(237.73773581,138.19923551)
\closepath
\moveto(237.8385103,138.19923551)
\curveto(237.82781104,138.20435479)(237.81044153,138.20435479)(237.79974227,138.19923551)
\curveto(237.78904301,138.19411623)(237.79705626,138.1901196)(237.81917105,138.1901196)
\curveto(237.84052481,138.1901196)(237.84916479,138.19416113)(237.83859983,138.19923551)
\closepath
\moveto(237.94722642,138.19904241)
\curveto(237.93684053,138.20416169)(237.91580014,138.20416169)(237.90072717,138.19904241)
\curveto(237.88568553,138.19450691)(237.89419122,138.19046537)(237.91988735,138.19019594)
\curveto(237.94531488,138.19019594)(237.95780481,138.19378841)(237.94728462,138.19845863)
\closepath
\moveto(238.11763564,138.21673537)
\curveto(238.10724975,138.22185465)(238.08620936,138.22185465)(238.07114087,138.21673537)
\curveto(238.05609923,138.21219986)(238.06460492,138.20820323)(238.09030105,138.20788889)
\curveto(238.11572858,138.20788889)(238.12821851,138.21148137)(238.11769832,138.21615159)
\closepath
\moveto(238.2102984,138.21692846)
\curveto(238.19959914,138.22204774)(238.18222963,138.22204774)(238.17153037,138.21692846)
\curveto(238.16083111,138.21180918)(238.16884436,138.20781255)(238.19095915,138.20781255)
\curveto(238.2123129,138.20781255)(238.22095289,138.21185409)(238.21038793,138.21692846)
\closepath
\moveto(238.3264861,138.21692846)
\curveto(238.31166829,138.22146396)(238.28718086,138.22146396)(238.27225114,138.21692846)
\curveto(238.25743334,138.21239296)(238.26956514,138.20889029)(238.29933504,138.20889029)
\curveto(238.32914972,138.20889029)(238.34137105,138.21248277)(238.32641895,138.21692846)
\closepath
\moveto(233.72517304,138.23322933)
\curveto(233.72011439,138.23906711)(233.70659483,138.23987541)(233.69553743,138.23592369)
\curveto(233.68322657,138.23013082)(233.68676314,138.22595456)(233.70484892,138.22514625)
\curveto(233.72127833,138.22514625)(233.73027645,138.22873873)(233.72508351,138.23457651)
\closepath
\moveto(233.85718132,138.23322933)
\curveto(233.84648206,138.23834861)(233.82911255,138.23834861)(233.81841329,138.23322933)
\curveto(233.80771403,138.22811005)(233.81572728,138.22411342)(233.83784207,138.22411342)
\curveto(233.85919582,138.22411342)(233.86783581,138.22815495)(233.85727085,138.23322933)
\closepath
\moveto(238.50463997,138.23322933)
\curveto(238.49394071,138.23834861)(238.4765712,138.23834861)(238.46587194,138.23322933)
\curveto(238.45517268,138.22811005)(238.46318593,138.22411342)(238.48530072,138.22411342)
\curveto(238.50665447,138.22411342)(238.51529446,138.22815495)(238.5047295,138.23322933)
\closepath
\moveto(238.59757132,138.23322933)
\curveto(238.58691683,138.23834861)(238.56950255,138.23834861)(238.55889283,138.23322933)
\curveto(238.54819357,138.22811005)(238.55620682,138.22411342)(238.57832161,138.22411342)
\curveto(238.59967536,138.22411342)(238.60831535,138.22815495)(238.59775039,138.23322933)
\closepath
\moveto(238.70631878,138.23303623)
\curveto(238.69593289,138.23815551)(238.6748925,138.23815551)(238.65982401,138.23303623)
\curveto(238.64478237,138.22850073)(238.65328806,138.22445919)(238.67898419,138.22418976)
\curveto(238.70441172,138.22418976)(238.71690165,138.22778223)(238.70638146,138.23245246)
\closepath
\moveto(233.95755291,138.24992088)
\curveto(233.9524495,138.25571375)(233.9389747,138.25656696)(233.92787254,138.25261524)
\curveto(233.91556167,138.24682237)(233.91909825,138.24264611)(233.93718403,138.2418378)
\curveto(233.95361343,138.2418378)(233.96261156,138.24543028)(233.95741861,138.25122315)
\closepath
\moveto(234.09697009,138.24992088)
\curveto(234.09186668,138.25571375)(234.07839188,138.25656696)(234.06728972,138.25261524)
\curveto(234.05497885,138.24682237)(234.05851543,138.24264611)(234.07660121,138.2418378)
\curveto(234.09303061,138.2418378)(234.10202874,138.24543028)(234.09683579,138.25122315)
\closepath
\moveto(238.87673248,138.24992088)
\curveto(238.86634659,138.25504016)(238.8453062,138.25504016)(238.83023323,138.24992088)
\curveto(238.81519159,138.24538538)(238.82369728,138.24134384)(238.84939341,138.2410744)
\curveto(238.87482094,138.2410744)(238.88731087,138.24466688)(238.87679068,138.2493371)
\closepath
\moveto(238.96939076,138.25009152)
\curveto(238.95873626,138.2552108)(238.94132199,138.2552108)(238.93062273,138.25009152)
\curveto(238.91996823,138.24497224)(238.92793672,138.24124505)(238.95005151,138.24124505)
\curveto(238.97140526,138.24124505)(238.98004525,138.24528658)(238.96948029,138.25009152)
\closepath
\moveto(239.08557845,138.25009152)
\curveto(239.07053682,138.25462702)(239.04627322,138.25462702)(239.03133903,138.25009152)
\curveto(239.01652122,138.24555602)(239.02865302,138.24205335)(239.05842293,138.24205335)
\curveto(239.0882376,138.24205335)(239.10045893,138.24564583)(239.08550683,138.25009152)
\closepath
\moveto(234.19799976,138.26747013)
\curveto(234.1873005,138.27258941)(234.16993099,138.27258941)(234.15923173,138.26747013)
\curveto(234.14853247,138.26235085)(234.15654572,138.25862366)(234.17866051,138.25862366)
\curveto(234.20001426,138.25862366)(234.20865425,138.26266519)(234.19808929,138.26747013)
\closepath
\moveto(234.34482584,138.26747013)
\curveto(234.33972242,138.27330791)(234.32624762,138.27411622)(234.31514546,138.27016449)
\curveto(234.3028346,138.26437162)(234.30637117,138.26019537)(234.32445695,138.25938706)
\curveto(234.34084159,138.25938706)(234.34988448,138.26297954)(234.34469154,138.26877241)
\closepath
\moveto(239.2640054,138.26747013)
\curveto(239.25361951,138.27258941)(239.23257913,138.27258941)(239.21751063,138.26747013)
\curveto(239.20246899,138.26293463)(239.21097468,138.25889309)(239.23667081,138.25862366)
\curveto(239.26209834,138.25862366)(239.27458827,138.26221613)(239.26406808,138.26688636)
\closepath
\moveto(239.35666816,138.26766323)
\curveto(239.3459689,138.27278251)(239.32859939,138.27278251)(239.31790013,138.26766323)
\curveto(239.30720087,138.26254395)(239.31521412,138.25881675)(239.33732891,138.25881675)
\curveto(239.35868266,138.25881675)(239.36732265,138.26285829)(239.35675769,138.26766323)
\closepath
\moveto(239.47285138,138.26766323)
\curveto(239.45803358,138.27219873)(239.43354615,138.27219873)(239.41861643,138.26766323)
\curveto(239.40379862,138.26312773)(239.41593042,138.25962506)(239.44570033,138.25962506)
\curveto(239.475515,138.25962506)(239.48773633,138.26321754)(239.47273946,138.26766323)
\closepath
\moveto(234.45326888,138.2839641)
\curveto(234.44816547,138.28975697)(234.43469067,138.29061018)(234.42358851,138.28665845)
\curveto(234.41127764,138.2810003)(234.41481422,138.27668933)(234.4329,138.27588102)
\curveto(234.44928464,138.27588102)(234.45832753,138.2794735)(234.45313458,138.28526637)
\closepath
\moveto(234.53075122,138.2839641)
\curveto(234.52564781,138.28975697)(234.51217301,138.29061018)(234.50102608,138.28665845)
\curveto(234.48871522,138.2810003)(234.49225179,138.27668933)(234.51033757,138.27588102)
\curveto(234.52672221,138.27588102)(234.5357651,138.2794735)(234.53057216,138.28526637)
\closepath
\moveto(239.6587499,138.28665845)
\curveto(239.6439321,138.29119396)(239.61944467,138.29119396)(239.60451048,138.28665845)
\curveto(239.58978221,138.28212295)(239.60182447,138.27862029)(239.63159438,138.27862029)
\curveto(239.66140905,138.27862029)(239.67363038,138.28221276)(239.65867828,138.28665845)
\closepath
\moveto(239.75945277,138.28665845)
\curveto(239.74875351,138.29177774)(239.731384,138.29177774)(239.72068474,138.28665845)
\curveto(239.70998548,138.28153917)(239.71799874,138.27781198)(239.74011353,138.27781198)
\curveto(239.76146728,138.27781198)(239.77010727,138.28185352)(239.75954231,138.28665845)
\closepath
\moveto(239.84431716,138.28665845)
\curveto(239.83921374,138.29245132)(239.82573894,138.29330454)(239.81463678,138.28935281)
\curveto(239.80232592,138.28369466)(239.80586249,138.27938369)(239.82394827,138.27857538)
\curveto(239.84033291,138.27857538)(239.8493758,138.28216786)(239.84418286,138.28796073)
\closepath
\moveto(234.72471672,138.30547406)
\curveto(234.71401746,138.31059334)(234.69664795,138.31059334)(234.68594869,138.30547406)
\curveto(234.67524943,138.30035478)(234.68326268,138.29662758)(234.70537747,138.29662758)
\curveto(234.72673123,138.29662758)(234.73537121,138.30066912)(234.72480625,138.30547406)
\closepath
\moveto(234.80957663,138.30547406)
\curveto(234.80447321,138.31126693)(234.79099841,138.31212014)(234.77994102,138.30816842)
\curveto(234.76763015,138.30237554)(234.77116673,138.29819929)(234.78925251,138.29739098)
\curveto(234.80563715,138.29739098)(234.81468004,138.30098346)(234.80948709,138.30677633)
\closepath
\moveto(234.89509908,138.30547406)
\curveto(234.88439982,138.31059334)(234.86703031,138.31059334)(234.85642058,138.30547406)
\curveto(234.84572132,138.30035478)(234.85373458,138.29662758)(234.87584937,138.29662758)
\curveto(234.89720312,138.29662758)(234.90584311,138.30066912)(234.89527815,138.30547406)
\closepath
\moveto(240.04604969,138.30547406)
\curveto(240.03123189,138.31000956)(240.00674446,138.31000956)(239.99181474,138.30547406)
\curveto(239.97708647,138.30093855)(239.98912873,138.29743589)(240.01889864,138.29743589)
\curveto(240.04871331,138.29743589)(240.06093464,138.30102837)(240.04598254,138.30547406)
\closepath
\moveto(240.1467257,138.30547406)
\curveto(240.13607121,138.31059334)(240.11865693,138.31059334)(240.1080472,138.30547406)
\curveto(240.09734794,138.30035478)(240.1053612,138.29662758)(240.12747598,138.29662758)
\curveto(240.14882974,138.29662758)(240.15751449,138.30066912)(240.14690477,138.30547406)
\closepath
\moveto(240.25547316,138.3053079)
\curveto(240.24508727,138.31042719)(240.22404688,138.31042719)(240.2089739,138.3053079)
\curveto(240.19393227,138.3007724)(240.20243796,138.29673086)(240.22813409,138.29646143)
\curveto(240.25356162,138.29646143)(240.26605155,138.30005391)(240.25553135,138.30472413)
\closepath
\moveto(229.79065668,138.32439294)
\curveto(229.7586932,138.32843448)(229.70642351,138.32843448)(229.67446898,138.32439294)
\curveto(229.6425055,138.32035141)(229.66869406,138.31747742)(229.73257626,138.31747742)
\curveto(229.79645397,138.31747742)(229.82260673,138.32017178)(229.79065668,138.32439294)
\closepath
\moveto(230.10048307,138.32458604)
\curveto(230.06001391,138.32817852)(229.99378593,138.32817852)(229.95332572,138.32458604)
\curveto(229.91281179,138.32099356)(229.9459392,138.31793995)(230.02688649,138.31793995)
\curveto(230.10784273,138.31793995)(230.14097015,138.32063431)(230.10048307,138.32458604)
\closepath
\moveto(230.82083874,138.32477913)
\curveto(230.70795483,138.32747349)(230.52321129,138.32747349)(230.41030947,138.32477913)
\curveto(230.29743004,138.32208478)(230.38980629,138.31912098)(230.61557411,138.31912098)
\curveto(230.84137774,138.31912098)(230.93374056,138.32181534)(230.82083874,138.32477913)
\closepath
\moveto(231.19262684,138.32477913)
\curveto(231.16496097,138.32882067)(231.1196346,138.32882067)(231.09194635,138.32477913)
\curveto(231.06428048,138.3207376)(231.08684294,138.31759418)(231.14228659,138.31759418)
\curveto(231.19765415,138.31759418)(231.22034195,138.32118666)(231.19262684,138.32477913)
\closepath
\moveto(231.36303606,138.32477913)
\curveto(231.34387588,138.32882067)(231.3125257,138.32882067)(231.29332971,138.32477913)
\curveto(231.27416952,138.3207376)(231.28974836,138.3170104)(231.32820303,138.3170104)
\curveto(231.36656816,138.3170104)(231.38221414,138.32060288)(231.36307635,138.32477913)
\closepath
\moveto(231.50275317,138.32477913)
\curveto(231.49236728,138.32989841)(231.47141643,138.32989841)(231.45631212,138.32477913)
\curveto(231.44127048,138.32024363)(231.44977617,138.31620209)(231.4754723,138.31593266)
\curveto(231.50089983,138.31593266)(231.51338976,138.31952514)(231.50286957,138.32419536)
\closepath
\moveto(231.74300304,138.32477913)
\curveto(231.7112634,138.32882067)(231.65549295,138.32882067)(231.61907965,138.32494978)
\curveto(231.58263949,138.32090824)(231.60869375,138.31803426)(231.67674374,138.31803426)
\curveto(231.74491906,138.31803426)(231.77471583,138.32072862)(231.74300304,138.32494978)
\closepath
\moveto(235.10392267,138.32208478)
\curveto(235.09881926,138.32787765)(235.08534446,138.32873086)(235.0742423,138.32477913)
\curveto(235.06193143,138.31898626)(235.06546801,138.31481001)(235.08355379,138.3140017)
\curveto(235.0999832,138.3140017)(235.10898132,138.31759418)(235.10378837,138.32338705)
\closepath
\moveto(235.18944065,138.32208478)
\curveto(235.17878616,138.32720406)(235.16137188,138.32720406)(235.15076215,138.32208478)
\curveto(235.14006289,138.3169655)(235.14807615,138.3132383)(235.17019094,138.3132383)
\curveto(235.19154469,138.3132383)(235.20022944,138.31727984)(235.18961972,138.32208478)
\closepath
\moveto(240.43332262,138.32208478)
\curveto(240.41850481,138.32662028)(240.39401738,138.32662028)(240.37908766,138.32208478)
\curveto(240.36426986,138.31754927)(240.37640166,138.31404661)(240.40617156,138.31404661)
\curveto(240.43598624,138.31404661)(240.44820757,138.31763908)(240.4332107,138.32208478)
\closepath
\moveto(240.53402996,138.32208478)
\curveto(240.52337547,138.32720406)(240.50596119,138.32720406)(240.49526193,138.32208478)
\curveto(240.48460744,138.3169655)(240.49257592,138.3132383)(240.51469071,138.3132383)
\curveto(240.53604447,138.3132383)(240.54468445,138.31727984)(240.5341195,138.32208478)
\closepath
\moveto(249.32545978,138.32208478)
\curveto(249.2934963,138.32612631)(249.24122213,138.32612631)(249.20926761,138.32208478)
\curveto(249.17730413,138.31804324)(249.20349269,138.31516926)(249.2673525,138.31516926)
\curveto(249.33125708,138.31516926)(249.35741431,138.31786361)(249.32545978,138.32208478)
\closepath
\moveto(250.44860006,138.32208478)
\curveto(250.25901275,138.32477913)(249.94880584,138.32477913)(249.75921405,138.32208478)
\curveto(249.56962674,138.31939042)(249.72474363,138.3169655)(250.10389586,138.3169655)
\curveto(250.48307496,138.3169655)(250.63816051,138.31965985)(250.44860006,138.32208478)
\closepath
\moveto(250.88265875,138.31939042)
\curveto(250.87227285,138.3245097)(250.85123247,138.3245097)(250.83615949,138.31939042)
\curveto(250.82111785,138.31485491)(250.82962354,138.31081338)(250.85531967,138.31054394)
\curveto(250.8807472,138.31054394)(250.89323714,138.31413642)(250.88271694,138.31880664)
\closepath
\moveto(250.97529016,138.31958351)
\curveto(250.96463567,138.32470279)(250.94722139,138.32470279)(250.93661167,138.31958351)
\curveto(250.92591241,138.31446423)(250.93392566,138.31073704)(250.95604045,138.31073704)
\curveto(250.9773942,138.31073704)(250.98603419,138.31477857)(250.97546923,138.31958351)
\closepath
\moveto(251.29329098,138.31958351)
\curveto(251.26155133,138.32362505)(251.20580774,138.32362505)(251.16936758,138.31974517)
\curveto(251.13292742,138.31570364)(251.15898169,138.31282966)(251.22703167,138.31282966)
\curveto(251.29520699,138.31282966)(251.32500376,138.31552401)(251.29329098,138.31974517)
\closepath
\moveto(235.313373,138.33619872)
\curveto(235.30267374,138.341318)(235.28530423,138.341318)(235.2746945,138.33619872)
\curveto(235.26399524,138.33107944)(235.2720085,138.32735225)(235.29412329,138.32735225)
\curveto(235.31547704,138.32735225)(235.32411703,138.33139378)(235.31355207,138.33619872)
\closepath
\moveto(235.49956699,138.33603706)
\curveto(235.48918109,138.34115634)(235.46814071,138.34115634)(235.45307221,138.33603706)
\curveto(235.43803057,138.33150156)(235.44653626,138.32746002)(235.47223239,138.32719058)
\curveto(235.49765992,138.32719058)(235.51014985,138.33078306)(235.49962966,138.33545328)
\closepath
\moveto(240.82062688,138.33603706)
\curveto(240.80580907,138.34057256)(240.78132165,138.34057256)(240.76638745,138.33603706)
\curveto(240.75156965,138.33150156)(240.76370144,138.32799889)(240.79347135,138.32799889)
\curveto(240.82328603,138.32799889)(240.83550736,138.33159137)(240.82055525,138.33603706)
\closepath
\moveto(240.92130289,138.33603706)
\curveto(240.91060363,138.34115634)(240.89323412,138.34115634)(240.88257962,138.33603706)
\curveto(240.87192513,138.33091778)(240.87989362,138.32719058)(240.90200841,138.32719058)
\curveto(240.92336216,138.32719058)(240.93204691,138.33123212)(240.92143719,138.33603706)
\closepath
\moveto(241.03004587,138.33586642)
\curveto(241.01965998,138.3409857)(240.99861959,138.3409857)(240.98355109,138.33586642)
\curveto(240.96850946,138.33133092)(240.97701514,138.32728938)(241.00271127,138.32701994)
\curveto(241.0281388,138.32701994)(241.04062874,138.33061242)(241.03010854,138.33528264)
\closepath
\moveto(231.81996609,138.77326855)
\curveto(231.81728008,138.85394662)(231.81486268,138.7879528)(231.81486268,138.62659666)
\curveto(231.81486268,138.46523603)(231.81754868,138.39923773)(231.81996609,138.4799158)
\curveto(231.8226521,138.56059386)(231.8226521,138.69259048)(231.81996609,138.77326855)
\closepath
\moveto(235.63900655,138.35378839)
\curveto(235.62862066,138.35886277)(235.60758027,138.35886277)(235.59251177,138.35378839)
\curveto(235.57747013,138.34925289)(235.58597582,138.34521135)(235.61167195,138.34494192)
\curveto(235.63709948,138.34494192)(235.64958941,138.34853439)(235.63906922,138.35320461)
\closepath
\moveto(235.72357099,138.35378839)
\curveto(235.71846758,138.35958126)(235.70499278,138.36043448)(235.69389062,138.35648275)
\curveto(235.68157975,138.35064497)(235.68511633,138.34651362)(235.70320211,138.34570532)
\curveto(235.71963152,138.34570532)(235.72862964,138.34929779)(235.72343669,138.35509067)
\closepath
\moveto(235.82490507,138.35378839)
\curveto(235.81451918,138.35886277)(235.79347879,138.35886277)(235.77840582,138.35378839)
\curveto(235.76336418,138.34925289)(235.77186987,138.34521135)(235.797566,138.34494192)
\curveto(235.82299353,138.34494192)(235.83548346,138.34853439)(235.82496327,138.35320461)
\closepath
\moveto(241.20045957,138.35378839)
\curveto(241.19007367,138.35886277)(241.16903329,138.35886277)(241.15396031,138.35378839)
\curveto(241.13891867,138.34925289)(241.14742436,138.34521135)(241.17312049,138.34494192)
\curveto(241.19854802,138.34494192)(241.21103796,138.34853439)(241.20051776,138.35320461)
\closepath
\moveto(241.29309098,138.35395005)
\curveto(241.28239172,138.35906933)(241.26502221,138.35906933)(241.25436772,138.35395005)
\curveto(241.24371323,138.34883077)(241.25168171,138.34510358)(241.2737965,138.34510358)
\curveto(241.29515026,138.34510358)(241.30379024,138.34914511)(241.29322528,138.35395005)
\closepath
\moveto(241.42446357,138.35395005)
\curveto(241.41936016,138.35974292)(241.40588536,138.36059614)(241.3947832,138.35664441)
\curveto(241.38247233,138.35098626)(241.38600891,138.34667529)(241.40409469,138.34586698)
\curveto(241.42047933,138.34586698)(241.42952222,138.34945946)(241.42432927,138.35525233)
\closepath
\moveto(251.37030774,139.59343571)
\curveto(251.36762174,139.82009861)(251.3657863,139.62954011)(251.3657863,139.16994096)
\curveto(251.3657863,138.71034181)(251.3684723,138.5249026)(251.37030774,138.7578164)
\curveto(251.37299375,138.99072572)(251.37299375,139.36674138)(251.37028088,139.59343571)
\closepath
\moveto(231.79293591,139.16988707)
\lineto(231.79293591,139.97882774)
\lineto(230.70079215,139.97882774)
\lineto(229.60862152,139.97882774)
\lineto(229.60862152,139.16988707)
\lineto(229.60862152,138.36091946)
\lineto(230.70079215,138.36091946)
\lineto(231.79293591,138.36091946)
\closepath
\moveto(235.96432225,138.37258603)
\curveto(235.95393636,138.37766041)(235.93289597,138.37766041)(235.917823,138.37258603)
\curveto(235.90278136,138.36805053)(235.91128705,138.36400899)(235.93698318,138.36373956)
\curveto(235.96241071,138.36373956)(235.97490064,138.36733203)(235.96438045,138.37200225)
\closepath
\moveto(236.04890908,138.37258603)
\curveto(236.04380567,138.3783789)(236.03033087,138.37923212)(236.01922871,138.37528039)
\curveto(236.00691784,138.36948752)(236.01045442,138.36531126)(236.0285402,138.36450296)
\curveto(236.0449696,138.36450296)(236.05396773,138.36809543)(236.04877478,138.37388831)
\closepath
\moveto(236.15021182,138.37258603)
\curveto(236.13982593,138.37766041)(236.11878555,138.37766041)(236.10371705,138.37258603)
\curveto(236.08867541,138.36805053)(236.0971811,138.36400899)(236.12287723,138.36373956)
\curveto(236.14830476,138.36373956)(236.16079469,138.36733203)(236.1502745,138.37200225)
\closepath
\moveto(241.57969238,138.37258603)
\curveto(241.56487457,138.37712154)(241.54038715,138.37712154)(241.52545743,138.37258603)
\curveto(241.51072916,138.36805053)(241.52277142,138.36454786)(241.55254133,138.36454786)
\curveto(241.582356,138.36454786)(241.59457733,138.36814034)(241.57962523,138.37258603)
\closepath
\moveto(241.68039077,138.37258603)
\curveto(241.66969151,138.37766041)(241.652322,138.37766041)(241.64162274,138.37258603)
\curveto(241.63092348,138.36746675)(241.63893673,138.36373956)(241.66105152,138.36373956)
\curveto(241.68240527,138.36373956)(241.69104526,138.36778109)(241.6804803,138.37258603)
\closepath
\moveto(241.78911584,138.37239294)
\curveto(241.77872995,138.37751222)(241.7577791,138.37751222)(241.74267479,138.37239294)
\curveto(241.72763315,138.36785743)(241.73613884,138.3638159)(241.76183497,138.36354646)
\curveto(241.7872625,138.36354646)(241.79975243,138.36713894)(241.78923224,138.37180916)
\closepath
\moveto(251.34322384,139.1697793)
\lineto(251.34322384,139.97871997)
\lineto(250.25108008,139.97871997)
\lineto(249.15893632,139.97871997)
\lineto(249.15893632,139.1697793)
\lineto(249.15893632,138.3608072)
\lineto(250.25108008,138.3608072)
\lineto(251.34322384,138.3608072)
\closepath
\moveto(236.30513621,138.39025653)
\curveto(236.29475032,138.39537581)(236.27370994,138.39537581)(236.25864144,138.39025653)
\curveto(236.2435998,138.38572103)(236.25210549,138.38167949)(236.27780162,138.38141006)
\curveto(236.30322915,138.38141006)(236.31571908,138.38500253)(236.30519889,138.38967275)
\closepath
\moveto(236.38972752,138.39025653)
\curveto(236.38462411,138.3960494)(236.37114931,138.39690262)(236.36000238,138.39295089)
\curveto(236.34769151,138.38711311)(236.35122809,138.38298176)(236.36931387,138.38217346)
\curveto(236.38569851,138.38217346)(236.3947414,138.38576593)(236.38954845,138.39155881)
\closepath
\moveto(236.49103474,138.39025653)
\curveto(236.48064885,138.39537581)(236.45960846,138.39537581)(236.44453549,138.39025653)
\curveto(236.42949385,138.38572103)(236.43799954,138.38167949)(236.46369567,138.38141006)
\curveto(236.4891232,138.38141006)(236.50161313,138.38500253)(236.49109294,138.38967275)
\closepath
\moveto(241.95952506,138.39025653)
\curveto(241.94913917,138.39537581)(241.92818832,138.39537581)(241.91308401,138.39025653)
\curveto(241.89804237,138.38572103)(241.90654806,138.38167949)(241.93224419,138.38141006)
\curveto(241.95767172,138.38141006)(241.97016165,138.38500253)(241.95964146,138.38967275)
\closepath
\moveto(242.05218782,138.39044963)
\curveto(242.04148856,138.39556891)(242.02411905,138.39556891)(242.01341979,138.39044963)
\curveto(242.00272053,138.38533035)(242.01073378,138.38160315)(242.03284857,138.38160315)
\curveto(242.05420232,138.38160315)(242.06284231,138.38564469)(242.05227735,138.39044963)
\closepath
\moveto(242.16837104,138.39044963)
\curveto(242.1533294,138.39498513)(242.12906581,138.39498513)(242.11413609,138.39044963)
\curveto(242.09931828,138.38591412)(242.11145008,138.38241146)(242.14121999,138.38241146)
\curveto(242.17103466,138.38241146)(242.18325599,138.38600394)(242.16830389,138.39044963)
\closepath
\moveto(236.65307257,138.4067505)
\curveto(236.64796916,138.41258827)(236.63449436,138.41339658)(236.6233922,138.40944485)
\curveto(236.61108133,138.4037867)(236.61461791,138.39947573)(236.63270369,138.39866742)
\curveto(236.6491331,138.39866742)(236.65813122,138.4022599)(236.65293827,138.40805277)
\closepath
\moveto(236.73859055,138.4067505)
\curveto(236.72793606,138.41186978)(236.71052178,138.41186978)(236.69991205,138.4067505)
\curveto(236.68921279,138.40163121)(236.69722605,138.39790402)(236.71934083,138.39790402)
\curveto(236.74069459,138.39790402)(236.74937934,138.40194556)(236.73876962,138.4067505)
\closepath
\moveto(236.84734248,138.40658883)
\curveto(236.83695659,138.41166321)(236.81591621,138.41166321)(236.80084323,138.40658883)
\curveto(236.78580159,138.40205333)(236.79430728,138.39801179)(236.82000341,138.39774236)
\curveto(236.84543094,138.39774236)(236.85792087,138.40133484)(236.84740068,138.40600506)
\closepath
\moveto(242.3313445,138.40658883)
\curveto(242.32095861,138.41166321)(242.29991822,138.41166321)(242.28484525,138.40658883)
\curveto(242.26980361,138.40205333)(242.2783093,138.39801179)(242.30400543,138.39774236)
\curveto(242.32943296,138.39774236)(242.34192289,138.40133484)(242.33140269,138.40600506)
\closepath
\moveto(242.42397592,138.4067505)
\curveto(242.41327666,138.41186978)(242.39590715,138.41186978)(242.38525265,138.4067505)
\curveto(242.37459816,138.40163121)(242.38256665,138.39790402)(242.40468143,138.39790402)
\curveto(242.42603519,138.39790402)(242.43467518,138.40194556)(242.42411022,138.4067505)
\closepath
\moveto(242.53272338,138.40658883)
\curveto(242.52233748,138.41166321)(242.5012971,138.41166321)(242.4862286,138.40658883)
\curveto(242.47118696,138.40205333)(242.47969265,138.39801179)(242.50538878,138.39774236)
\curveto(242.53108491,138.39774236)(242.54330624,138.40133484)(242.53278605,138.40600506)
\closepath
\moveto(237.00224002,138.42428179)
\curveto(236.99185412,138.42940107)(236.9708585,138.42940107)(236.95579896,138.42428179)
\curveto(236.94075732,138.41974628)(236.94926301,138.41570475)(236.97495914,138.41543531)
\curveto(237.00038667,138.41543531)(237.0128766,138.41902779)(237.00235641,138.42369801)
\closepath
\moveto(237.09489829,138.42447937)
\curveto(237.08419903,138.42959865)(237.06682952,138.42959865)(237.05617503,138.42447937)
\curveto(237.04547577,138.41936009)(237.05348902,138.4156329)(237.07560381,138.4156329)
\curveto(237.09695757,138.4156329)(237.10559755,138.41967443)(237.09503259,138.42447937)
\closepath
\moveto(237.20364575,138.42428179)
\curveto(237.19325986,138.42940107)(237.17221947,138.42940107)(237.1571465,138.42428179)
\curveto(237.14210486,138.41974628)(237.15061055,138.41570475)(237.17630668,138.41543531)
\curveto(237.20173421,138.41543531)(237.21422414,138.41902779)(237.20370395,138.42369801)
\closepath
\moveto(242.69509248,138.42428179)
\curveto(242.68027468,138.42881729)(242.65578725,138.42881729)(242.64085305,138.42428179)
\curveto(242.62612478,138.41974628)(242.63816705,138.41628852)(242.66793695,138.41628852)
\curveto(242.69775163,138.41628852)(242.70997296,138.419881)(242.69502085,138.42428179)
\closepath
\moveto(242.88902664,138.42428179)
\curveto(242.87864075,138.42940107)(242.85760036,138.42940107)(242.84253187,138.42428179)
\curveto(242.82749023,138.41974628)(242.83599592,138.41570475)(242.86169205,138.41543531)
\curveto(242.88711958,138.41543531)(242.89960951,138.41902779)(242.88908932,138.42369801)
\closepath
\moveto(237.35854776,138.44197474)
\curveto(237.34816187,138.44704911)(237.32721102,138.44704911)(237.3121067,138.44197474)
\curveto(237.29706507,138.43743924)(237.30557075,138.4333977)(237.33126689,138.43312826)
\curveto(237.35669442,138.43312826)(237.36918435,138.43672074)(237.35866415,138.44139096)
\closepath
\moveto(237.45120604,138.44216784)
\curveto(237.44050678,138.44728712)(237.42313727,138.44728712)(237.41248278,138.44216784)
\curveto(237.40182828,138.43704855)(237.40979677,138.43332136)(237.43191156,138.43332136)
\curveto(237.45326531,138.43332136)(237.4619053,138.4373629)(237.45134034,138.44216784)
\closepath
\moveto(237.56739374,138.44216784)
\curveto(237.55257593,138.44670334)(237.5280885,138.44670334)(237.51315431,138.44216784)
\curveto(237.4983365,138.43763233)(237.5104683,138.43412967)(237.54023821,138.43412967)
\curveto(237.57005288,138.43412967)(237.58227421,138.43772214)(237.56732211,138.44216784)
\closepath
\moveto(243.04392865,138.44216784)
\curveto(243.03354276,138.44728712)(243.01259191,138.44728712)(242.99748759,138.44216784)
\curveto(242.98244596,138.43763233)(242.99095164,138.43359079)(243.01664778,138.43332136)
\curveto(243.04207531,138.43332136)(243.05456524,138.43691384)(243.04404504,138.44158406)
\closepath
\moveto(243.13659141,138.4423295)
\curveto(243.12589215,138.44744878)(243.10852264,138.44744878)(243.09791291,138.4423295)
\curveto(243.08721365,138.43721022)(243.0952269,138.43348302)(243.11734169,138.43348302)
\curveto(243.13869544,138.43348302)(243.1473802,138.43752456)(243.13677047,138.4423295)
\closepath
\moveto(243.24533439,138.44216784)
\curveto(243.2349485,138.44728712)(243.21390811,138.44728712)(243.19883961,138.44216784)
\curveto(243.18379797,138.43763233)(243.19230366,138.43359079)(243.21799979,138.43332136)
\curveto(243.24342732,138.43332136)(243.25591725,138.43691384)(243.24539706,138.44158406)
\closepath
\moveto(237.73036272,138.45986079)
\curveto(237.71997682,138.46498007)(237.69893644,138.46498007)(237.68386794,138.45986079)
\curveto(237.6688263,138.45532529)(237.67733199,138.45128375)(237.70302812,138.45101431)
\curveto(237.72845565,138.45101431)(237.74094558,138.45460679)(237.73042539,138.45927701)
\closepath
\moveto(237.82299414,138.46003143)
\curveto(237.81229488,138.46515071)(237.79492537,138.46515071)(237.78427087,138.46003143)
\curveto(237.77361638,138.45491215)(237.78158487,138.45118495)(237.80369965,138.45118495)
\curveto(237.82505341,138.45118495)(237.8336934,138.45522649)(237.82312844,138.46003143)
\closepath
\moveto(237.93918631,138.46003143)
\curveto(237.92436851,138.46456693)(237.89988108,138.46456693)(237.88494688,138.46003143)
\curveto(237.87012908,138.45549593)(237.88226087,138.45176873)(237.91203078,138.45176873)
\curveto(237.94184546,138.45176873)(237.95406679,138.45536121)(237.93911468,138.46003143)
\closepath
\moveto(243.40023192,138.46003143)
\curveto(243.38984603,138.46515071)(243.36889517,138.46515071)(243.35379086,138.46003143)
\curveto(243.33874923,138.45549593)(243.34725491,138.45145439)(243.37295104,138.45118495)
\curveto(243.39837857,138.45118495)(243.41086851,138.45477743)(243.40034831,138.45944765)
\closepath
\moveto(243.48482322,138.46003143)
\curveto(243.47971981,138.4658243)(243.46624501,138.46667752)(243.45518762,138.46272579)
\curveto(243.44287675,138.45693292)(243.44641333,138.45275666)(243.46449911,138.45194836)
\curveto(243.48092851,138.45194836)(243.48992664,138.45554083)(243.48473369,138.4613337)
\closepath
\moveto(243.5861573,138.46003143)
\curveto(243.57577141,138.46515071)(243.55473103,138.46515071)(243.53966253,138.46003143)
\curveto(243.52462089,138.45549593)(243.53312658,138.45145439)(243.55882271,138.45118495)
\curveto(243.58425024,138.45118495)(243.59674017,138.45477743)(243.58621998,138.45944765)
\closepath
\moveto(238.10959553,138.47826326)
\curveto(238.09477773,138.48279876)(238.0702903,138.48279876)(238.05536058,138.47826326)
\curveto(238.04054277,138.47372775)(238.05267457,138.47026999)(238.08244448,138.47026999)
\curveto(238.11225915,138.47026999)(238.12448048,138.47386247)(238.10952838,138.47826326)
\closepath
\moveto(238.20222695,138.4755689)
\curveto(238.1971683,138.48136177)(238.18364873,138.48221498)(238.17259134,138.47826326)
\curveto(238.16028048,138.47247039)(238.16381705,138.46829413)(238.18190283,138.46748582)
\curveto(238.19833224,138.46748582)(238.20733036,138.4710783)(238.20213741,138.47687117)
\closepath
\moveto(238.31100127,138.47826326)
\curveto(238.296273,138.48279876)(238.27169603,138.48279876)(238.25676631,138.47826326)
\curveto(238.24172468,138.47372775)(238.25408031,138.47026999)(238.28385022,138.47026999)
\curveto(238.31366489,138.47026999)(238.32588622,138.47386247)(238.31093412,138.47826326)
\closepath
\moveto(243.74105483,138.47826326)
\curveto(243.73066894,138.48333763)(243.70962856,138.48333763)(243.69455558,138.47826326)
\curveto(243.67951394,138.47372775)(243.68801963,138.46968622)(243.71371576,138.46941678)
\curveto(243.73941189,138.46941678)(243.75163322,138.47300926)(243.74111303,138.47767948)
\closepath
\moveto(243.82564614,138.47826326)
\curveto(243.82054273,138.48405613)(243.80706793,138.48490934)(243.79601053,138.48095761)
\curveto(243.78369967,138.47516474)(243.78723624,138.47098849)(243.80532202,138.47018018)
\curveto(243.82175143,138.47018018)(243.83074955,138.47377266)(243.82555661,138.47956553)
\closepath
\moveto(243.92694888,138.47826326)
\curveto(243.91656299,138.48333763)(243.8955226,138.48333763)(243.88045411,138.47826326)
\curveto(243.86541247,138.47372775)(243.87391816,138.46968622)(243.89961429,138.46941678)
\curveto(243.92504182,138.46941678)(243.93753175,138.47300926)(243.92701156,138.47767948)
\closepath
\moveto(238.48942822,138.49595621)
\curveto(238.47904232,138.50107549)(238.45800194,138.50107549)(238.44299164,138.49595621)
\curveto(238.42795,138.49142071)(238.43645569,138.48737917)(238.46215182,138.48710973)
\curveto(238.48757935,138.48710973)(238.50006928,138.49070221)(238.48954909,138.49537243)
\closepath
\moveto(238.58209097,138.49612236)
\curveto(238.57143648,138.50124164)(238.5540222,138.50124164)(238.54332294,138.49612236)
\curveto(238.53266845,138.49100308)(238.54063693,138.48727588)(238.56275172,138.48727588)
\curveto(238.58410548,138.48727588)(238.59274546,138.49131742)(238.5821805,138.49612236)
\closepath
\moveto(238.66697774,138.49612236)
\curveto(238.66187432,138.50191523)(238.64839952,138.50276844)(238.6372526,138.49881672)
\curveto(238.62494173,138.49302385)(238.62847831,138.48884759)(238.64656409,138.48803929)
\curveto(238.66299349,138.48803929)(238.67199162,138.49163176)(238.66679867,138.49742463)
\closepath
\moveto(244.06638845,138.49612236)
\curveto(244.05600255,138.50124164)(244.03496217,138.50124164)(244.01989367,138.49612236)
\curveto(244.00485203,138.49158686)(244.01335772,138.48754532)(244.03905385,138.48727588)
\curveto(244.06448138,138.48727588)(244.07697131,138.49086836)(244.06645112,138.49553858)
\closepath
\moveto(244.15095289,138.49612236)
\curveto(244.14584948,138.50191523)(244.13237468,138.50276844)(244.12131728,138.49881672)
\curveto(244.10900642,138.49302385)(244.11254299,138.48884759)(244.13062877,138.48803929)
\curveto(244.14705818,138.48803929)(244.1560563,138.49163176)(244.15086336,138.49742463)
\closepath
\moveto(244.2365022,138.49612236)
\curveto(244.22584771,138.50124164)(244.20843343,138.50124164)(244.19773417,138.49612236)
\curveto(244.18703491,138.49100308)(244.19504817,138.48727588)(244.21716296,138.48727588)
\curveto(244.23851671,138.48727588)(244.2471567,138.49131742)(244.23659174,138.49612236)
\closepath
\moveto(238.86868789,138.51435419)
\curveto(238.85387008,138.51888969)(238.82938266,138.51888969)(238.81444846,138.51435419)
\curveto(238.79972019,138.50981868)(238.81176245,138.50631602)(238.84153236,138.50631602)
\curveto(238.87134704,138.50631602)(238.88356837,138.50990849)(238.86861626,138.51435419)
\closepath
\moveto(238.96939076,138.51435419)
\curveto(238.95873626,138.51947347)(238.94132199,138.51947347)(238.93062273,138.51435419)
\curveto(238.91996823,138.50927981)(238.92793672,138.50550771)(238.95005151,138.50550771)
\curveto(238.97140526,138.50550771)(238.98004525,138.50954925)(238.96948029,138.51435419)
\closepath
\moveto(239.08557845,138.51435419)
\curveto(239.07053682,138.51888969)(239.04627322,138.51888969)(239.03133903,138.51435419)
\curveto(239.01652122,138.50981868)(239.02865302,138.50631602)(239.05842293,138.50631602)
\curveto(239.0882376,138.50631602)(239.10045893,138.50990849)(239.08550683,138.51435419)
\closepath
\moveto(244.37591491,138.51435419)
\curveto(244.36521565,138.51947347)(244.34784614,138.51947347)(244.33714688,138.51435419)
\curveto(244.32649238,138.50927981)(244.33446087,138.50550771)(244.35657566,138.50550771)
\curveto(244.37792941,138.50550771)(244.3865694,138.50954925)(244.37600444,138.51435419)
\closepath
\moveto(244.46080615,138.51435419)
\curveto(244.45570274,138.52014706)(244.44222794,138.52100027)(244.43108101,138.51704854)
\curveto(244.41877014,138.51121077)(244.42230672,138.50707942)(244.4403925,138.50627111)
\curveto(244.45682191,138.50627111)(244.46582003,138.50986359)(244.46062708,138.51565646)
\closepath
\moveto(244.54632413,138.51435419)
\curveto(244.53566963,138.51947347)(244.51825536,138.51947347)(244.5075561,138.51435419)
\curveto(244.49685684,138.50927981)(244.50487009,138.50550771)(244.52698488,138.50550771)
\curveto(244.54833863,138.50550771)(244.55697862,138.50954925)(244.54641366,138.51435419)
\closepath
\moveto(239.25596081,138.53258601)
\curveto(239.24114301,138.53712151)(239.21665558,138.53712151)(239.20172586,138.53258601)
\curveto(239.18690806,138.52805051)(239.19903985,138.52432331)(239.22880976,138.52432331)
\curveto(239.25862444,138.52432331)(239.27084577,138.52791579)(239.25589366,138.53258601)
\closepath
\moveto(239.35666816,138.53258601)
\curveto(239.3459689,138.53770529)(239.32859939,138.53770529)(239.31790013,138.53258601)
\curveto(239.30720087,138.52746673)(239.31521412,138.52373953)(239.33732891,138.52373953)
\curveto(239.35868266,138.52373953)(239.36732265,138.52778107)(239.35675769,138.53258601)
\closepath
\moveto(244.69318154,138.53258601)
\curveto(244.68807813,138.53837888)(244.67460333,138.53923209)(244.6634564,138.53528037)
\curveto(244.65114554,138.52944259)(244.65468211,138.52531124)(244.67276789,138.52450294)
\curveto(244.6891973,138.52450294)(244.69819542,138.52809541)(244.69300248,138.53388828)
\closepath
\moveto(244.8406657,138.53258601)
\curveto(244.8300112,138.53770529)(244.81259693,138.53770529)(244.80189767,138.53258601)
\curveto(244.79124317,138.52746673)(244.79921166,138.52373953)(244.82132645,138.52373953)
\curveto(244.8426802,138.52373953)(244.85132019,138.52778107)(244.84075523,138.53258601)
\closepath
\moveto(239.6432606,138.55081783)
\curveto(239.6284428,138.55535334)(239.60395537,138.55535334)(239.58902565,138.55081783)
\curveto(239.57420784,138.54628233)(239.58633964,138.54277967)(239.61610955,138.54277967)
\curveto(239.64592422,138.54277967)(239.65814555,138.54637214)(239.64319345,138.55081783)
\closepath
\moveto(239.74396794,138.55081783)
\curveto(239.73326868,138.55593712)(239.71589917,138.55593712)(239.70519991,138.55081783)
\curveto(239.69450065,138.54569855)(239.70251391,138.54197136)(239.7246287,138.54197136)
\curveto(239.74598245,138.54197136)(239.75462244,138.5460129)(239.74405748,138.55081783)
\closepath
\moveto(239.8368993,138.55081783)
\curveto(239.82620004,138.55593712)(239.80883053,138.55593712)(239.79813127,138.55081783)
\curveto(239.78743201,138.54569855)(239.79544526,138.54197136)(239.81756005,138.54197136)
\curveto(239.83891381,138.54197136)(239.84755379,138.5460129)(239.83698883,138.55081783)
\closepath
\moveto(244.96459805,138.55081783)
\curveto(244.95389879,138.55593712)(244.93652928,138.55593712)(244.92583002,138.55081783)
\curveto(244.91513076,138.54569855)(244.92314401,138.54197136)(244.9452588,138.54197136)
\curveto(244.96661255,138.54197136)(244.97525254,138.5460129)(244.96468758,138.55081783)
\closepath
\moveto(245.04946243,138.55081783)
\curveto(245.04435902,138.55661071)(245.03088422,138.55746392)(245.01978205,138.55351219)
\curveto(245.00747119,138.54771932)(245.01100777,138.54354307)(245.02909354,138.54273476)
\curveto(245.04552295,138.54273476)(245.05452107,138.54632724)(245.04932813,138.55212011)
\closepath
\moveto(245.12693582,138.55081783)
\curveto(245.1218324,138.55661071)(245.1083576,138.55746392)(245.09721067,138.55351219)
\curveto(245.08489981,138.54771932)(245.08843639,138.54354307)(245.10652216,138.54273476)
\curveto(245.12290681,138.54273476)(245.13194969,138.54632724)(245.12675675,138.55212011)
\closepath
\moveto(240.030538,138.57021721)
\curveto(240.01549636,138.57475272)(239.99123277,138.57475272)(239.97629857,138.57021721)
\curveto(239.96148077,138.56568171)(239.97361257,138.56217905)(240.00338247,138.56217905)
\curveto(240.03319715,138.56217905)(240.04541848,138.56577152)(240.03046637,138.57021721)
\closepath
\moveto(240.13124087,138.57021721)
\curveto(240.12054161,138.5753365)(240.1031721,138.5753365)(240.09247284,138.57021721)
\curveto(240.08177358,138.56509793)(240.08978683,138.56137074)(240.11190162,138.56137074)
\curveto(240.13325538,138.56137074)(240.14189536,138.56541228)(240.1313304,138.57021721)
\closepath
\moveto(240.24742857,138.57021721)
\curveto(240.2327003,138.57475272)(240.20812334,138.57475272)(240.19318914,138.57021721)
\curveto(240.17837133,138.56568171)(240.19050313,138.56217905)(240.22027304,138.56217905)
\curveto(240.25008771,138.56217905)(240.26230904,138.56577152)(240.24735694,138.57021721)
\closepath
\moveto(245.39835232,138.57021721)
\curveto(245.38765306,138.5753365)(245.37028355,138.5753365)(245.35958429,138.57021721)
\curveto(245.34888503,138.56509793)(245.35689828,138.56137074)(245.37901307,138.56137074)
\curveto(245.40036682,138.56137074)(245.40900681,138.56541228)(245.39844185,138.57021721)
\closepath
\moveto(240.41783779,138.58844904)
\curveto(240.40310952,138.59298454)(240.37853256,138.59298454)(240.36360284,138.58844904)
\curveto(240.34878503,138.58391354)(240.36091683,138.58041087)(240.39068674,138.58041087)
\curveto(240.42050141,138.58041087)(240.43272274,138.58400335)(240.41777064,138.58844904)
\closepath
\moveto(240.5185138,138.58844904)
\curveto(240.5078593,138.59356832)(240.49044503,138.59356832)(240.4798353,138.58844904)
\curveto(240.46913604,138.58332976)(240.47714929,138.57960256)(240.49926408,138.57960256)
\curveto(240.52061783,138.57960256)(240.52925782,138.5836441)(240.51869286,138.58844904)
\closepath
\moveto(240.63470597,138.58844904)
\curveto(240.61966433,138.59298454)(240.59544551,138.59298454)(240.58052474,138.58844904)
\curveto(240.5654831,138.58391354)(240.57783873,138.58041087)(240.60760864,138.58041087)
\curveto(240.63742331,138.58041087)(240.64964464,138.58400335)(240.63469254,138.58844904)
\closepath
\moveto(245.51421322,138.58575468)
\curveto(245.50915457,138.59154755)(245.495635,138.59240076)(245.48457761,138.58844904)
\curveto(245.47226675,138.58261126)(245.47580332,138.57847991)(245.4938891,138.57767161)
\curveto(245.51027374,138.57767161)(245.51931663,138.58126408)(245.51412368,138.58705695)
\closepath
\moveto(245.65365278,138.58575468)
\curveto(245.64854937,138.59154755)(245.63507457,138.59240076)(245.62392764,138.58844904)
\curveto(245.61161677,138.58261126)(245.61515335,138.57847991)(245.63323913,138.57767161)
\curveto(245.64962377,138.57767161)(245.65866666,138.58126408)(245.65347371,138.58705695)
\closepath
\moveto(240.80511967,138.60515406)
\curveto(240.79030186,138.60968956)(240.76581443,138.60968956)(240.75088024,138.60515406)
\curveto(240.7358386,138.60061856)(240.74819423,138.59711589)(240.77791937,138.59711589)
\curveto(240.80773405,138.59711589)(240.81995538,138.60070837)(240.80500327,138.60515406)
\closepath
\moveto(241.02201023,138.60515406)
\curveto(241.00696859,138.60968956)(240.982705,138.60968956)(240.9677708,138.60515406)
\curveto(240.952953,138.60061856)(240.9650848,138.59711589)(240.9948547,138.59711589)
\curveto(241.02466938,138.59711589)(241.03689071,138.60070837)(241.02193861,138.60515406)
\closepath
\moveto(245.76207344,138.6024597)
\curveto(245.75697003,138.60825257)(245.74349523,138.60910579)(245.73239307,138.60515406)
\curveto(245.7200822,138.59931628)(245.72361878,138.59518494)(245.74170456,138.59437663)
\curveto(245.75813396,138.59437663)(245.76713209,138.59796911)(245.76193914,138.60376198)
\closepath
\moveto(241.19241945,138.62176927)
\curveto(241.17760165,138.62630477)(241.15311422,138.62630477)(241.1381845,138.62176927)
\curveto(241.1233667,138.61723377)(241.13549849,138.6137311)(241.1652684,138.6137311)
\curveto(241.19508308,138.6137311)(241.20730441,138.61732358)(241.1923523,138.62176927)
\closepath
\moveto(241.29309098,138.62176927)
\curveto(241.28239172,138.62688855)(241.26502221,138.62688855)(241.25436772,138.62176927)
\curveto(241.24371323,138.61664999)(241.25168171,138.61292279)(241.2737965,138.61292279)
\curveto(241.29515026,138.61292279)(241.30379024,138.61696433)(241.29322528,138.62176927)
\closepath
\moveto(241.40184292,138.62157617)
\curveto(241.39145703,138.62669546)(241.37041664,138.62669546)(241.35534367,138.62157617)
\curveto(241.34030203,138.61704067)(241.34880772,138.61299913)(241.37450385,138.6127297)
\curveto(241.39993138,138.6127297)(241.41242131,138.61632218)(241.40190112,138.6209924)
\closepath
\moveto(245.99444884,138.62157617)
\curveto(245.98934542,138.62736905)(245.97587062,138.62822226)(245.96476846,138.62427053)
\curveto(245.9524576,138.61847766)(245.95599417,138.61430141)(245.97407995,138.6134931)
\curveto(245.99046459,138.6134931)(245.99950748,138.61708558)(245.99431454,138.62287845)
\closepath
\moveto(241.57969238,138.64097555)
\curveto(241.56487457,138.64551106)(241.54038715,138.64551106)(241.52545743,138.64097555)
\curveto(241.51072916,138.63644005)(241.52277142,138.63293739)(241.55254133,138.63293739)
\curveto(241.582356,138.63293739)(241.59457733,138.63652986)(241.57962523,138.64097555)
\closepath
\moveto(241.78109811,138.64097555)
\curveto(241.76628031,138.64551106)(241.74179288,138.64551106)(241.72686316,138.64097555)
\curveto(241.71204536,138.63644005)(241.72417716,138.63293739)(241.75394706,138.63293739)
\curveto(241.78376174,138.63293739)(241.79598307,138.63652986)(241.7809862,138.64097555)
\closepath
\moveto(241.95148047,138.65866851)
\curveto(241.93666267,138.66320401)(241.91222001,138.66320401)(241.89729924,138.65866851)
\curveto(241.88257097,138.654133)(241.89461324,138.65040581)(241.92438314,138.65040581)
\curveto(241.95419782,138.65040581)(241.96641915,138.65399829)(241.95146704,138.65866851)
\closepath
\moveto(242.04411189,138.65597415)
\curveto(242.03900848,138.66176702)(242.02553368,138.66262023)(242.01447629,138.65866851)
\curveto(242.00216542,138.65287564)(242.005702,138.64869938)(242.02378778,138.64789107)
\curveto(242.04021718,138.64789107)(242.04921531,138.65148355)(242.04402236,138.65727642)
\closepath
\moveto(242.15289069,138.65866851)
\curveto(242.13807288,138.66320401)(242.11358546,138.66320401)(242.09865574,138.65866851)
\curveto(242.08383793,138.654133)(242.09596973,138.65040581)(242.12569487,138.65040581)
\curveto(242.15550954,138.65040581)(242.16773088,138.65399829)(242.15277877,138.65866851)
\closepath
\moveto(242.31583281,138.67577768)
\curveto(242.30544692,138.68089696)(242.28440653,138.68089696)(242.26939175,138.67577768)
\curveto(242.25435012,138.67124218)(242.2628558,138.66720064)(242.28855193,138.66693121)
\curveto(242.31397946,138.66693121)(242.3264694,138.67052368)(242.3159492,138.67519391)
\closepath
\moveto(242.41590894,138.67577768)
\curveto(242.41080553,138.68161546)(242.39733073,138.68242377)(242.38622857,138.67847204)
\curveto(242.3739177,138.67267917)(242.37745428,138.66850292)(242.39554006,138.66769461)
\curveto(242.41196947,138.66769461)(242.42096759,138.67128709)(242.41577464,138.67712486)
\closepath
\moveto(242.52467879,138.67847204)
\curveto(242.50963715,138.68300754)(242.48537355,138.68300754)(242.47044383,138.67847204)
\curveto(242.45571556,138.67393654)(242.46775783,138.67043387)(242.49752773,138.67043387)
\curveto(242.52734241,138.67043387)(242.53956374,138.67402635)(242.52461163,138.67847204)
\closepath
\moveto(242.68765224,138.69558122)
\curveto(242.67726635,138.7007005)(242.65622596,138.7007005)(242.64115299,138.69558122)
\curveto(242.62611135,138.69104571)(242.63461704,138.68700418)(242.66031317,138.68673474)
\curveto(242.6857407,138.68673474)(242.69823063,138.69032722)(242.68771044,138.69499744)
\closepath
\moveto(242.78028366,138.69574288)
\curveto(242.7695844,138.70086216)(242.75221489,138.70086216)(242.7415604,138.69574288)
\curveto(242.7309059,138.6906236)(242.73887439,138.6868964)(242.76098918,138.6868964)
\curveto(242.78234293,138.6868964)(242.79102769,138.69093794)(242.78041796,138.69574288)
\closepath
\moveto(242.88902664,138.69558122)
\curveto(242.87864075,138.7007005)(242.85760036,138.7007005)(242.84253187,138.69558122)
\curveto(242.82749023,138.69104571)(242.83599592,138.68700418)(242.86169205,138.68673474)
\curveto(242.88711958,138.68673474)(242.89960951,138.69032722)(242.88908932,138.69499744)
\closepath
\moveto(243.13659141,138.71354361)
\curveto(243.12589215,138.71866289)(243.10852264,138.71866289)(243.09791291,138.71354361)
\curveto(243.08721365,138.70842432)(243.0952269,138.70469713)(243.11734169,138.70469713)
\curveto(243.13869544,138.70469713)(243.1473802,138.70873867)(243.13677047,138.71354361)
\closepath
\moveto(243.24533439,138.71338194)
\curveto(243.2349485,138.71845632)(243.21390811,138.71845632)(243.19883961,138.71338194)
\curveto(243.18379797,138.70884644)(243.19230366,138.7048049)(243.21799979,138.70453547)
\curveto(243.24342732,138.70453547)(243.25591725,138.70812795)(243.24539706,138.71279817)
\closepath
\moveto(243.38444715,138.73134433)
\curveto(243.37374789,138.73646361)(243.35637838,138.73646361)(243.34572389,138.73134433)
\curveto(243.3350694,138.72622505)(243.34303788,138.72222842)(243.36515267,138.72222842)
\curveto(243.38650642,138.72222842)(243.39514641,138.72626996)(243.38458145,138.73134433)
\closepath
\moveto(243.47740537,138.73134433)
\curveto(243.46670611,138.73646361)(243.4493366,138.73646361)(243.43863734,138.73134433)
\curveto(243.42793808,138.72622505)(243.43595133,138.72222842)(243.45806612,138.72222842)
\curveto(243.47941987,138.72222842)(243.48805986,138.72626996)(243.4774949,138.73134433)
\closepath
\moveto(243.5861573,138.73115124)
\curveto(243.57577141,138.73627052)(243.55473103,138.73627052)(243.53966253,138.73115124)
\curveto(243.52462089,138.72661573)(243.53312658,138.7226191)(243.55882271,138.72230476)
\curveto(243.58425024,138.72230476)(243.59674017,138.72589724)(243.58621998,138.73056746)
\closepath
\moveto(243.73268345,138.74803588)
\curveto(243.72758003,138.75382875)(243.71410523,138.75468197)(243.70300307,138.75073024)
\curveto(243.69069221,138.74489246)(243.69422878,138.74076112)(243.71231456,138.73995281)
\curveto(243.72874397,138.73995281)(243.73774209,138.74354529)(243.73254915,138.74933816)
\closepath
\moveto(243.81016131,138.74803588)
\curveto(243.8050579,138.75382875)(243.7915831,138.75468197)(243.78043617,138.75073024)
\curveto(243.76812531,138.74489246)(243.77166188,138.74076112)(243.78974766,138.73995281)
\curveto(243.80617707,138.73995281)(243.81517519,138.74354529)(243.80998224,138.74933816)
\closepath
\moveto(243.91146405,138.74803588)
\curveto(243.90107816,138.75315516)(243.88003778,138.75315516)(243.86496928,138.74803588)
\curveto(243.84992764,138.74350038)(243.85843333,138.73950375)(243.88412946,138.73918941)
\curveto(243.90955699,138.73918941)(243.92204692,138.74278188)(243.91152673,138.74745211)
\closepath
\moveto(244.05060368,138.76599827)
\curveto(244.03990442,138.77111755)(244.02253491,138.77111755)(244.01183565,138.76599827)
\curveto(244.00113639,138.76087899)(244.00914964,138.75715179)(244.03126443,138.75715179)
\curveto(244.05261818,138.75715179)(244.06125817,138.76119333)(244.05069321,138.76599827)
\closepath
\moveto(244.13546806,138.76599827)
\curveto(244.13036465,138.77179114)(244.11688985,138.77264436)(244.10583245,138.76869263)
\curveto(244.09352159,138.76289976)(244.09705817,138.7587235)(244.11514394,138.7579152)
\curveto(244.13157335,138.7579152)(244.14057147,138.76150767)(244.13537853,138.76730054)
\closepath
\moveto(244.23677528,138.76599827)
\curveto(244.22638939,138.77111755)(244.20539377,138.77111755)(244.19033423,138.76599827)
\curveto(244.17529259,138.76146277)(244.18379828,138.75742123)(244.20949441,138.75715179)
\curveto(244.23492194,138.75715179)(244.24741187,138.76074427)(244.23689168,138.76541449)
\closepath
\moveto(244.37621932,138.78369122)
\curveto(244.36583343,138.78881051)(244.34479304,138.78881051)(244.32972007,138.78369122)
\curveto(244.31467843,138.77915572)(244.32318412,138.77515909)(244.34888025,138.77484475)
\curveto(244.37430778,138.77484475)(244.38679771,138.77843723)(244.37627752,138.78310745)
\closepath
\moveto(244.46080615,138.78369122)
\curveto(244.45570274,138.789529)(244.44222794,138.79033731)(244.43108101,138.78638558)
\curveto(244.41877014,138.78054781)(244.42230672,138.77641646)(244.4403925,138.77560815)
\curveto(244.45682191,138.77560815)(244.46582003,138.77920063)(244.46062708,138.7849935)
\closepath
\moveto(244.54632413,138.78369122)
\curveto(244.53566963,138.78881051)(244.51825536,138.78881051)(244.5075561,138.78369122)
\curveto(244.49685684,138.77857194)(244.50487009,138.77484475)(244.52698488,138.77484475)
\curveto(244.54833863,138.77484475)(244.55697862,138.77888629)(244.54641366,138.78369122)
\closepath
\moveto(244.67025648,138.80138418)
\curveto(244.65955722,138.80650346)(244.64218771,138.80650346)(244.63148845,138.80138418)
\curveto(244.62078919,138.7962649)(244.62880244,138.7925377)(244.65091723,138.7925377)
\curveto(244.67227098,138.7925377)(244.68091097,138.79657924)(244.67034601,138.80138418)
\closepath
\moveto(244.75514772,138.80138418)
\curveto(244.75004431,138.80717705)(244.73656951,138.80803026)(244.72542258,138.80407854)
\curveto(244.71311171,138.79828567)(244.71664829,138.79410941)(244.73473407,138.7933011)
\curveto(244.75116348,138.7933011)(244.7601616,138.79689358)(244.75496865,138.80268645)
\closepath
\moveto(244.8406657,138.80138418)
\curveto(244.8300112,138.80650346)(244.81259693,138.80650346)(244.80189767,138.80138418)
\curveto(244.79124317,138.7962649)(244.79921166,138.7925377)(244.82132645,138.7925377)
\curveto(244.8426802,138.7925377)(244.85132019,138.79657924)(244.84075523,138.80138418)
\closepath
\moveto(244.96459805,138.81907713)
\curveto(244.95389879,138.82419641)(244.93652928,138.82419641)(244.92583002,138.81907713)
\curveto(244.91513076,138.81400276)(244.92314401,138.81023065)(244.9452588,138.81023065)
\curveto(244.96661255,138.81023065)(244.97525254,138.81427219)(244.96468758,138.81907713)
\closepath
\moveto(245.04946243,138.81907713)
\curveto(245.04435902,138.82487)(245.03088422,138.82572321)(245.01978205,138.82177149)
\curveto(245.00747119,138.81597862)(245.01100777,138.81180236)(245.02909354,138.81099406)
\curveto(245.04552295,138.81099406)(245.05452107,138.81458653)(245.04932813,138.8203794)
\closepath
\moveto(245.12693582,138.81907713)
\curveto(245.1218324,138.82487)(245.1083576,138.82572321)(245.09721067,138.82177149)
\curveto(245.08489981,138.81597862)(245.08843639,138.81180236)(245.10652216,138.81099406)
\curveto(245.12290681,138.81099406)(245.13194969,138.81458653)(245.12675675,138.8203794)
\closepath
\moveto(245.24345479,138.83789273)
\curveto(245.23275553,138.84301201)(245.21538602,138.84301201)(245.20468676,138.83789273)
\curveto(245.1939875,138.83277345)(245.20200075,138.82904626)(245.22411554,138.82904626)
\curveto(245.24546929,138.82904626)(245.25410928,138.83308779)(245.24354432,138.83789273)
\closepath
\moveto(245.39835232,138.83789273)
\curveto(245.38765306,138.84301201)(245.37028355,138.84301201)(245.35958429,138.83789273)
\curveto(245.34888503,138.83277345)(245.35689828,138.82904626)(245.37901307,138.82904626)
\curveto(245.40036682,138.82904626)(245.40900681,138.83308779)(245.39844185,138.83789273)
\closepath
\moveto(245.49872839,138.85477738)
\curveto(245.49362498,138.86061515)(245.48015018,138.86142346)(245.46909278,138.85747174)
\curveto(245.45678192,138.85167887)(245.46031849,138.84750261)(245.47840427,138.8466943)
\curveto(245.49483368,138.8466943)(245.5038318,138.85028678)(245.49863886,138.85607965)
\closepath
\moveto(245.57617939,138.85477738)
\curveto(245.57107598,138.86061515)(245.55760118,138.86142346)(245.54654379,138.85747174)
\curveto(245.53423292,138.85167887)(245.5377695,138.84750261)(245.55585528,138.8466943)
\curveto(245.57223992,138.8466943)(245.58128281,138.85028678)(245.57608986,138.85607965)
\closepath
\moveto(245.65365278,138.85477738)
\curveto(245.64854937,138.86061515)(245.63507457,138.86142346)(245.62392764,138.85747174)
\curveto(245.61161677,138.85167887)(245.61515335,138.84750261)(245.63323913,138.8466943)
\curveto(245.64962377,138.8466943)(245.65866666,138.85028678)(245.65347371,138.85607965)
\closepath
\moveto(245.75466006,138.87359298)
\curveto(245.7439608,138.87871226)(245.72659129,138.87871226)(245.71589203,138.87359298)
\curveto(245.70519277,138.8684737)(245.71320603,138.8647465)(245.73532081,138.8647465)
\curveto(245.75667457,138.8647465)(245.76531456,138.86878804)(245.7547496,138.87359298)
\closepath
\moveto(245.97896401,138.89047763)
\curveto(245.97386059,138.8962705)(245.96038579,138.89712371)(245.94928363,138.89317198)
\curveto(245.93697277,138.88733421)(245.94050934,138.88320286)(245.95859512,138.88239455)
\curveto(245.97497976,138.88239455)(245.98402265,138.88598703)(245.97882971,138.8917799)
\closepath
\moveto(246.17681526,138.91535553)
\curveto(246.15984865,138.92361823)(246.12489475,138.92999488)(246.09934188,138.92945601)
\curveto(246.05397522,138.92945601)(246.05414086,138.92676165)(246.10704176,138.91535553)
\curveto(246.18309604,138.89712371)(246.21401198,138.89712371)(246.17677945,138.91535553)
\closepath
\moveto(246.35496466,138.91535553)
\lineto(246.39368792,138.92945601)
\lineto(246.35496466,138.93215037)
\curveto(246.3336109,138.93215037)(246.30575701,138.92581863)(246.29300744,138.91638837)
\curveto(246.27250425,138.90121015)(246.27250425,138.8992792)(246.29300744,138.90062638)
\curveto(246.30590027,138.90062638)(246.33370044,138.90866454)(246.35496466,138.91638837)
\closepath
\moveto(246.44790049,138.95087616)
\curveto(246.44790049,138.96277624)(245.96886462,138.96717703)(245.0343044,138.96443776)
\lineto(243.6207083,138.95990226)
\lineto(245.02267846,138.95078635)
\curveto(245.79378175,138.94566707)(246.42990425,138.93973948)(246.43627456,138.93722474)
\curveto(246.44254191,138.93453038)(246.44800345,138.94081722)(246.44800345,138.95078635)
\closepath
\moveto(233.28397853,138.96412342)
\curveto(233.01773259,138.96681778)(232.58203544,138.96681778)(232.31576712,138.96412342)
\curveto(232.04952118,138.96142906)(232.26735185,138.95900414)(232.79988849,138.95900414)
\curveto(233.33240275,138.95900414)(233.55025133,138.9616985)(233.28397853,138.96412342)
\closepath
\moveto(235.56125561,138.96412342)
\curveto(235.24385019,138.96681778)(234.72449289,138.96681778)(234.40711881,138.96412342)
\curveto(234.08974026,138.96142906)(234.34942339,138.95900414)(234.98417602,138.95900414)
\curveto(235.61895551,138.95900414)(235.87863416,138.9616985)(235.56125561,138.96412342)
\closepath
\moveto(241.7281032,138.96412342)
\curveto(240.71275685,138.96681778)(239.0466224,138.96681778)(238.02563096,138.96412342)
\curveto(237.00464399,138.96142906)(237.83539006,138.95958792)(239.87175024,138.95958792)
\curveto(241.90811937,138.95958792)(242.74348537,138.96228228)(241.7281032,138.96412342)
\closepath
\moveto(247.16521649,138.96412342)
\curveto(247.07835999,138.96681778)(246.93196814,138.96681778)(246.83990974,138.96412342)
\curveto(246.74784686,138.96142906)(246.81882459,138.95833055)(246.99784694,138.95833055)
\curveto(247.1767887,138.95830361)(247.25210433,138.96102491)(247.16521649,138.96412342)
\closepath
\moveto(247.66785788,138.96412342)
\curveto(247.61035496,138.9677159)(247.51624176,138.9677159)(247.45874331,138.96412342)
\curveto(247.40120457,138.96053094)(247.44835741,138.95779168)(247.56330059,138.95779168)
\curveto(247.67831093,138.95779168)(247.72536529,138.96048604)(247.66785788,138.96412342)
\closepath
\moveto(248.78430105,138.96429406)
\curveto(248.60805872,138.96698842)(248.31527504,138.96698842)(248.13365174,138.96431652)
\curveto(247.95202396,138.96162216)(248.09618194,138.95919723)(248.45406548,138.95919723)
\curveto(248.81189977,138.95919723)(248.96050757,138.96189159)(248.78430105,138.96431652)
\closepath
\moveto(231.82001981,139.74530771)
\curveto(231.8173338,139.89015641)(231.81549837,139.77665208)(231.81549837,139.49304392)
\curveto(231.81549837,139.20948067)(231.81818437,139.0909783)(231.82001981,139.22971979)
\curveto(231.82270582,139.36845679)(231.82270582,139.60048595)(231.82004219,139.74530771)
\closepath
\moveto(236.13018764,144.30231186)
\lineto(236.13002648,149.60049357)
\lineto(233.36504221,149.60049357)
\lineto(230.60005794,149.60049357)
\lineto(230.60005794,144.83590258)
\lineto(230.60005794,140.07128016)
\lineto(231.21585184,140.06674466)
\lineto(231.83161888,140.06220915)
\lineto(231.83564789,139.53326171)
\lineto(231.8396769,139.0043457)
\lineto(233.98490094,139.0043457)
\lineto(236.13012049,139.0043457)
\lineto(236.12995933,144.30252741)
\closepath
\moveto(243.56648678,144.31115834)
\lineto(243.56648678,149.61823592)
\lineto(239.87949937,149.61823592)
\lineto(236.19253434,149.61823592)
\lineto(236.19253434,144.31115834)
\lineto(236.19253434,139.00407177)
\lineto(239.87949937,139.00407177)
\lineto(243.56648678,139.00407177)
\closepath
\moveto(239.69360084,139.57302146)
\curveto(239.26687495,139.6226695)(238.85968529,139.8429288)(238.55749163,140.18751029)
\curveto(238.28191629,140.50173534)(238.12847368,140.81535415)(238.03195203,141.26165665)
\curveto(238.00513673,141.38563305)(237.99900368,141.46962967)(237.99828741,141.72427348)
\curveto(237.99828741,141.9219361)(238.00482336,142.07433349)(238.01798479,142.1421011)
\curveto(238.10806898,142.60791972)(238.28195658,142.97650345)(238.55602775,143.28243437)
\curveto(238.8325029,143.5910776)(239.1300453,143.77419069)(239.50768889,143.8681699)
\curveto(239.67793247,143.91056114)(240.06499051,143.91168379)(240.23578473,143.87086426)
\curveto(240.84580371,143.72322241)(241.32158056,143.30227832)(241.58632234,142.67597922)
\curveto(241.70866547,142.38655574)(241.76945428,142.07185917)(241.76945428,141.72772226)
\curveto(241.76945428,141.05804399)(241.50925633,140.44919535)(241.03892312,140.0182956)
\curveto(240.78556106,139.78615418)(240.3981225,139.60721735)(240.09636755,139.58290975)
\curveto(240.02395281,139.57711688)(239.92984409,139.56966249)(239.88721716,139.56602511)
\curveto(239.84464395,139.56243263)(239.75749199,139.56602511)(239.69358294,139.5735244)
\closepath
\moveto(239.86790029,139.60252468)
\curveto(239.8572458,139.60764396)(239.83983152,139.60764396)(239.82913226,139.60252468)
\curveto(239.81847777,139.59745031)(239.82644626,139.5936782)(239.84856105,139.5936782)
\curveto(239.8699148,139.5936782)(239.87855479,139.59771974)(239.86798983,139.60252468)
\closepath
\moveto(239.67393032,139.61940933)
\curveto(239.66882691,139.6252471)(239.65535211,139.62605541)(239.64420518,139.62210368)
\curveto(239.63189431,139.61631081)(239.63543089,139.61213456)(239.65351667,139.61132625)
\curveto(239.66994608,139.61132625)(239.6789442,139.61491873)(239.67375125,139.6207116)
\closepath
\moveto(240.15320793,139.66229453)
\curveto(240.42126245,139.71222099)(240.75920686,139.87491083)(240.97166999,140.056313)
\curveto(241.59684701,140.59010131)(241.85890725,141.54164987)(241.61298543,142.38487625)
\curveto(241.43138452,143.00757389)(241.02709126,143.48810372)(240.50308716,143.7041239)
\curveto(239.89224447,143.95591617)(239.22085918,143.82293163)(238.71764478,143.35039058)
\curveto(238.58025553,143.22138021)(238.50386998,143.12757163)(238.39980513,142.96001847)
\curveto(238.13003156,142.52566097)(238.01107728,141.99577948)(238.06746104,141.47957635)
\curveto(238.12871542,140.91903306)(238.42117231,140.35281815)(238.81375457,140.03474915)
\curveto(239.02968713,139.85981345)(239.30695465,139.72593976)(239.56343248,139.67277109)
\curveto(239.70843208,139.642729)(240.01860766,139.63725047)(240.15320793,139.662308)
\closepath
\moveto(241.15755956,140.17760852)
\curveto(241.1728698,140.19723243)(241.18218129,140.21312914)(241.17779415,140.21312914)
\curveto(241.17376514,140.21312914)(241.15755956,140.19709771)(241.14202549,140.17760852)
\curveto(241.12671525,140.15798461)(241.11740376,140.14199808)(241.12179091,140.14199808)
\curveto(241.12581992,140.14199808)(241.14202549,140.15802952)(241.15755956,140.17760852)
\closepath
\moveto(241.71809335,141.40769536)
\curveto(241.71298994,141.42179583)(241.70931906,141.41766448)(241.70878186,141.39691792)
\curveto(241.70878186,141.37810232)(241.71146787,141.36768414)(241.71697418,141.37370154)
\curveto(241.72203283,141.37949441)(241.7227491,141.39498697)(241.71697418,141.40774026)
\closepath
\moveto(241.73367219,141.5321343)
\curveto(241.72856878,141.54623477)(241.7248979,141.54210342)(241.7243607,141.52135686)
\curveto(241.7243607,141.50254126)(241.72704671,141.49212308)(241.73255302,141.49814048)
\curveto(241.73765643,141.50393335)(241.73832793,141.51942591)(241.73255302,141.5321792)
\closepath
\moveto(238.03119995,141.94106155)
\curveto(238.02609653,141.95516203)(238.02242566,141.95103068)(238.02188846,141.93028412)
\curveto(238.02188846,141.91146851)(238.02457446,141.90105033)(238.03008078,141.90706773)
\curveto(238.03518419,141.91290551)(238.03585569,141.92835316)(238.03008078,141.94110646)
\closepath
\moveto(240.02289184,143.87043317)
\curveto(240.00373165,143.8744747)(239.97238148,143.8744747)(239.95318548,143.87043317)
\curveto(239.9340253,143.86639163)(239.94960414,143.86266443)(239.98805881,143.86266443)
\curveto(240.02642393,143.86266443)(240.04206992,143.86625691)(240.02293213,143.87043317)
\closepath
\moveto(249.09703729,139.53032486)
\curveto(249.09703729,140.04152097)(249.09703729,140.05672613)(249.12806067,140.0657702)
\curveto(249.15894975,140.07515554)(249.15908405,140.09010923)(249.15908405,144.83779761)
\lineto(249.15908405,149.60053848)
\lineto(246.39407292,149.60053848)
\lineto(243.62909312,149.60053848)
\lineto(243.62895434,144.30235677)
\lineto(243.62878871,139.00417506)
\lineto(246.36291524,139.00417506)
\lineto(249.09703281,139.00417506)
\lineto(249.09703281,139.53032486)
\closepath
\moveto(230.30972748,140.03087826)
\curveto(230.17977847,140.03357261)(229.96715418,140.03357261)(229.83723203,140.03087826)
\curveto(229.70728303,140.0281839)(229.81372947,140.0252201)(230.07346632,140.0252201)
\curveto(230.33333747,140.0252201)(230.43964514,140.02791446)(230.30972748,140.03087826)
\closepath
\moveto(231.48033397,140.03087826)
\curveto(231.32114331,140.03357261)(231.05623589,140.03357261)(230.89167769,140.0309052)
\curveto(230.72709711,140.02821084)(230.85734157,140.02578592)(231.18112178,140.02578592)
\curveto(231.50490198,140.02578592)(231.63955597,140.02848028)(231.48033397,140.0309052)
\closepath
\moveto(250.79838528,140.03087826)
\curveto(250.51142576,140.03357261)(250.03740377,140.03357261)(249.74494688,140.03087826)
\curveto(249.4525258,140.0281839)(249.68728279,140.02575898)(250.26665892,140.02575898)
\curveto(250.84603952,140.02575898)(251.085309,140.02845333)(250.79838528,140.03087826)
\closepath
\moveto(241.7269706,149.64944557)
\curveto(240.71090798,149.65213993)(239.0482743,149.65213993)(238.03223854,149.64944557)
\curveto(237.01618039,149.64675121)(237.84749499,149.64491007)(239.87959338,149.64491007)
\curveto(241.91168728,149.64491007)(242.74301084,149.64760443)(241.7269706,149.64944557)
\closepath
\moveto(251.94748134,151.77849608)
\lineto(251.94748134,153.88531364)
\lineto(239.87959338,153.88531364)
\lineto(227.81170094,153.88531364)
\lineto(227.81170094,151.77849608)
\lineto(227.81170094,149.67167852)
\lineto(239.87959338,149.67167852)
\lineto(251.94748134,149.67167852)
\closepath
\moveto(228.88450545,153.91641552)
\curveto(228.6864035,153.91910988)(228.36224278,153.91910988)(228.16414978,153.91641552)
\curveto(227.96604783,153.91372116)(228.12811253,153.91129624)(228.52431195,153.91129624)
\curveto(228.92050689,153.91129624)(229.08259844,153.9139906)(228.88450545,153.91641552)
\closepath
\moveto(251.59506383,153.91641552)
\curveto(251.39696636,153.91910988)(251.07280564,153.91910988)(250.87471264,153.91641552)
\curveto(250.67661069,153.91372116)(250.83867539,153.91129624)(251.23487481,153.91129624)
\curveto(251.63106527,153.91129624)(251.7931613,153.9139906)(251.59506383,153.91641552)
\closepath
\moveto(229.23692744,156.04549746)
\lineto(229.23692744,158.15231503)
\lineto(228.52431195,158.15231503)
\lineto(227.81170094,158.15231503)
\lineto(227.81170094,156.04549746)
\lineto(227.81170094,153.9386799)
\lineto(228.52431195,153.9386799)
\lineto(229.23692744,153.9386799)
\closepath
\moveto(251.94748134,156.04549746)
\lineto(251.94748134,158.15231503)
\lineto(251.23487481,158.15231503)
\lineto(250.52226379,158.15231503)
\lineto(250.52226379,156.04549746)
\lineto(250.52226379,153.9386799)
\lineto(251.23487481,153.9386799)
\lineto(251.94748134,153.9386799)
\closepath
\moveto(229.42281701,156.05434394)
\lineto(229.42281701,158.15225665)
\lineto(229.36085083,158.15225665)
\lineto(229.29889361,158.15225665)
\lineto(229.29889361,156.05434394)
\lineto(229.29889361,153.9563998)
\lineto(229.36085083,153.9563998)
\lineto(229.42281701,153.9563998)
\closepath
\moveto(250.27439909,156.05434394)
\lineto(250.27439909,158.15225665)
\lineto(239.87959338,158.15225665)
\lineto(229.48478318,158.15225665)
\lineto(229.48478318,156.05434394)
\lineto(229.48478318,153.9563998)
\lineto(239.87959338,153.9563998)
\lineto(250.27439909,153.9563998)
\closepath
\moveto(250.46029762,156.05434394)
\lineto(250.46029762,158.15225665)
\lineto(250.39833144,158.15225665)
\lineto(250.33636527,158.15225665)
\lineto(250.33636527,156.05434394)
\lineto(250.33636527,153.9563998)
\lineto(250.39833144,153.9563998)
\lineto(250.46029762,153.9563998)
\closepath
}
}
{
\newrgbcolor{curcolor}{0 0 0}
\pscustom[linewidth=1.42222219,linecolor=curcolor]
{
\newpath
\moveto(240.79360117,104.79385268)
\lineto(240.23588769,104.79385268)
\lineto(239.67820107,104.79385268)
\lineto(239.67820107,110.60364499)
\lineto(239.67820107,111.46594967)
\lineto(237.77274343,111.46594967)
\lineto(235.86728131,111.46594967)
\lineto(235.86728131,113.59942117)
\lineto(235.86728131,115.73292097)
\lineto(234.61555741,115.73292097)
\curveto(233.66266966,115.73292097)(233.35940607,115.73802229)(233.34526425,115.75422885)
\curveto(233.33075981,115.77083508)(233.32668604,117.92013369)(233.32668604,125.3194106)
\lineto(233.32668604,134.86327638)
\lineto(232.4436658,134.86327638)
\lineto(231.56064109,134.86327638)
\lineto(231.56064109,134.4721292)
\lineto(231.56064109,134.0809874)
\lineto(231.7000538,134.0809874)
\lineto(231.83949784,134.0809874)
\lineto(231.83949784,133.1919228)
\lineto(231.83949784,132.30285192)
\lineto(230.69698697,132.30738742)
\lineto(229.55450744,132.31192292)
\lineto(229.55450744,133.19197714)
\lineto(229.55450744,134.07205965)
\lineto(229.69003886,134.07716096)
\lineto(229.82558819,134.08226228)
\lineto(229.82558819,134.47271252)
\lineto(229.82558819,134.86316815)
\lineto(229.54673145,134.86316815)
\lineto(229.26790605,134.86316815)
\lineto(229.26790605,136.39218327)
\lineto(229.26790605,137.92117773)
\lineto(229.54673145,137.92117773)
\lineto(229.82558819,137.92117773)
\lineto(229.82558819,138.09829586)
\lineto(229.82558819,138.27538706)
\lineto(229.69003886,138.28050634)
\lineto(229.55450744,138.28558071)
\lineto(229.55450744,139.17453933)
\lineto(229.55450744,140.06349346)
\lineto(230.04633766,140.06802897)
\lineto(230.53820368,140.07256447)
\lineto(230.53820368,144.83704319)
\lineto(230.53820368,149.60152641)
\lineto(229.14783259,149.60606191)
\lineto(227.75746598,149.61059741)
\lineto(227.75343697,153.91760104)
\lineto(227.74940796,158.22461364)
\lineto(239.87929344,158.22461364)
\lineto(252.00917892,158.22461364)
\lineto(252.00917892,153.91315535)
\lineto(252.00917892,149.60169256)
\lineto(250.61495341,149.60169256)
\lineto(249.22071895,149.60169256)
\lineto(249.22071895,144.83702074)
\lineto(249.22071895,140.07231749)
\lineto(250.30123678,140.06778198)
\lineto(251.38178147,140.06324648)
\lineto(251.38178147,139.17429684)
\lineto(251.38178147,138.28533822)
\lineto(251.24568151,138.28021894)
\lineto(251.10957259,138.27509966)
\lineto(251.11409404,138.10246314)
\lineto(251.11861548,137.92982213)
\lineto(252.49349727,137.92528663)
\lineto(253.86835219,137.92075112)
\lineto(253.86851783,137.16967674)
\curveto(253.86851783,136.7566087)(253.86851783,134.67248752)(253.86835219,132.5383532)
\lineto(253.86816417,128.65807037)
\lineto(257.32674252,128.65353487)
\lineto(260.78532536,128.64899937)
\lineto(260.78935437,122.84858118)
\lineto(260.79338338,117.04816839)
\lineto(257.3348856,117.04363288)
\lineto(253.87638783,117.03909738)
\lineto(253.86868794,116.39015894)
\lineto(253.86098805,115.74122049)
\lineto(250.48003977,115.73668499)
\lineto(247.09908253,115.73214948)
\lineto(247.09505352,113.60312681)
\lineto(247.09102451,111.47410998)
\lineto(245.18958245,111.46957448)
\lineto(243.28814486,111.46503898)
\lineto(243.28411585,110.60727429)
\lineto(243.28008684,104.80200985)
\lineto(242.72631283,104.79747435)
\lineto(242.17250749,104.79293885)
\curveto(240.79360117,104.79383696)(243.28008684,104.80200985)(240.79360117,104.79383696)
\closepath
\moveto(242.10982057,104.79380194)
\lineto(241.48240074,104.79380194)
\closepath
\moveto(243.22518934,110.63915079)
\lineto(243.22518934,111.46589893)
\lineto(241.48240074,111.46589893)
\lineto(239.73960766,111.46589893)
\lineto(239.73960766,110.63915079)
\lineto(239.73960766,104.86493704)
\lineto(241.48240074,104.86493704)
\lineto(243.22518934,104.86493704)
\closepath
\moveto(247.03610463,113.63495482)
\lineto(247.03610463,115.73287022)
\lineto(241.48240074,115.73287022)
\lineto(235.9286879,115.73287022)
\lineto(235.9286879,113.63495482)
\lineto(235.9286879,111.53701112)
\lineto(241.48240074,111.53701112)
\lineto(247.03610463,111.53701112)
\closepath
\moveto(241.36180799,113.40392437)
\curveto(241.1362461,113.44926189)(240.90364687,113.67692619)(240.83511788,113.91940496)
\curveto(240.71477583,114.34516163)(240.93294673,114.81809291)(241.30238457,114.93225377)
\curveto(241.46850065,114.98359881)(241.69636356,114.94579741)(241.83694915,114.84339159)
\curveto(242.18757151,114.58819544)(242.26701464,114.07324205)(242.01549249,113.68632815)
\curveto(241.94367315,113.57586799)(241.80283686,113.46571993)(241.68377514,113.42692745)
\curveto(241.5929523,113.3973434)(241.44652912,113.38686235)(241.36180799,113.40397601)
\closepath
\moveto(241.47046591,113.4323723)
\curveto(241.46540727,113.4381517)(241.4518877,113.43900492)(241.44074077,113.43520138)
\curveto(241.42842991,113.42942198)(241.43196648,113.42522776)(241.45005226,113.42443293)
\curveto(241.46648167,113.42443293)(241.47547979,113.42784578)(241.47028685,113.43384073)
\closepath
\moveto(241.74500266,113.54426631)
\curveto(241.88061914,113.62096661)(241.96726076,113.72047196)(242.03392745,113.87616231)
\curveto(242.08319329,113.99117908)(242.08658214,114.01061752)(242.08658214,114.17744411)
\curveto(242.08658214,114.34325627)(242.08300079,114.36415685)(242.03550772,114.47486085)
\curveto(241.93157716,114.71717482)(241.74784535,114.85892951)(241.51823207,114.87397031)
\curveto(241.41389414,114.88088583)(241.37184022,114.87397031)(241.2965246,114.84501045)
\curveto(241.03594613,114.74024572)(240.8705105,114.48090744)(240.8705105,114.1772115)
\curveto(240.8705105,114.02981797)(240.90475709,113.90775141)(240.98314372,113.77542699)
\curveto(241.05695518,113.6508668)(241.131371,113.58332463)(241.26615929,113.51851408)
\curveto(241.34544125,113.48037544)(241.37824187,113.47482057)(241.49831533,113.47895641)
\curveto(241.62131653,113.48290814)(241.65102376,113.4911394)(241.74502504,113.54423128)
\closepath
\moveto(242.13556594,114.1905809)
\curveto(242.1310445,114.20276389)(242.12764223,114.19340998)(242.12764223,114.16847818)
\curveto(242.12764223,114.14416609)(242.13122357,114.13402183)(242.13556594,114.14637546)
\curveto(242.14008739,114.15855845)(242.14008739,114.17862244)(242.13556594,114.19086381)
\closepath
\moveto(240.83427179,114.20826038)
\curveto(240.82975035,114.22044337)(240.82634807,114.21108946)(240.82634807,114.18615766)
\curveto(240.82634807,114.16184557)(240.82992942,114.15170131)(240.83427179,114.16405494)
\curveto(240.83879324,114.17623793)(240.83879324,114.19630192)(240.83427179,114.20854329)
\closepath
\moveto(235.25984984,115.78171265)
\curveto(234.92179351,115.78454173)(234.36410689,115.78454173)(234.02052186,115.78171265)
\curveto(233.67693684,115.77888358)(233.9535239,115.77661133)(234.63516526,115.77661133)
\curveto(235.31677975,115.77661133)(235.59788379,115.77944041)(235.25984984,115.78171265)
\closepath
\moveto(244.25799473,115.78171265)
\curveto(242.72577563,115.78454173)(240.22312916,115.78454173)(238.69654172,115.78171265)
\curveto(237.16995875,115.77888358)(238.42359419,115.77717715)(241.48240074,115.77717715)
\curveto(244.54119833,115.77717715)(245.79021384,115.78000623)(244.25799473,115.78171265)
\closepath
\moveto(252.1379237,115.78171265)
\curveto(251.21627421,115.78454173)(249.70353313,115.78454173)(248.77627437,115.78171265)
\curveto(247.84901561,115.77888358)(248.6030717,115.77717715)(250.45198443,115.77717715)
\curveto(252.30089268,115.77717715)(253.05957766,115.78000623)(252.1379237,115.78171265)
\closepath
\moveto(233.36872652,130.10282931)
\curveto(233.36604051,132.72590883)(233.36469751,130.57973882)(233.36469751,125.33356496)
\curveto(233.36469751,120.08736281)(233.36738351,117.94120223)(233.36872652,120.56430061)
\curveto(233.37141252,123.18740438)(233.37141252,127.47973093)(233.36872652,130.10282931)
\closepath
\moveto(253.80589358,125.34243973)
\lineto(253.80589358,134.8631942)
\lineto(252.50462629,134.8631942)
\lineto(251.20333213,134.8631942)
\lineto(251.20333213,134.47204747)
\lineto(251.20333213,134.08090567)
\lineto(251.29626797,134.08090567)
\lineto(251.38923066,134.08090567)
\lineto(251.38923066,133.1928326)
\lineto(251.38923066,132.30476491)
\lineto(251.30014925,132.29966359)
\lineto(251.21107679,132.29456227)
\lineto(251.20704778,124.66281164)
\lineto(251.20301877,117.03109289)
\lineto(243.19781445,117.0356284)
\lineto(235.19258775,117.0401639)
\lineto(235.18855874,125.95198082)
\lineto(235.18452973,134.86377394)
\lineto(234.29376485,134.86377394)
\lineto(233.4030313,134.86377394)
\lineto(233.4030313,125.34301901)
\lineto(233.4030313,115.82229687)
\lineto(243.60419384,115.82229687)
\lineto(253.80534295,115.82229687)
\lineto(253.80534295,125.34301901)
\closepath
\moveto(235.22770729,130.42139587)
\curveto(235.22502129,132.86926172)(235.22367828,130.87176595)(235.22367828,125.98250341)
\curveto(235.22367828,121.09321841)(235.22636429,119.0904161)(235.22770729,121.53180561)
\curveto(235.2303933,123.97320096)(235.2303933,127.97350622)(235.22770729,130.42139587)
\closepath
\moveto(251.14136596,124.70240479)
\lineto(251.14136596,132.30299112)
\lineto(251.07939978,132.30299112)
\lineto(251.01743809,132.30299112)
\lineto(251.01743809,124.78239175)
\lineto(251.01743809,117.26179687)
\lineto(243.1980741,117.26633238)
\lineto(235.37874145,117.27086788)
\lineto(235.37874145,125.99152367)
\lineto(235.37874145,134.71220595)
\lineto(242.26860386,134.71674145)
\lineto(249.15846179,134.72127696)
\lineto(249.15846179,134.40124242)
\lineto(249.15846179,134.08118499)
\lineto(249.26687797,134.08118499)
\lineto(249.37532549,134.08118499)
\lineto(249.37532549,134.47232633)
\lineto(249.37532549,134.86347351)
\lineto(242.31893962,134.86347351)
\lineto(235.26255823,134.86347351)
\lineto(235.26255823,125.98278272)
\lineto(235.26255823,117.10209552)
\lineto(243.20196433,117.10209552)
\lineto(251.14136596,117.10209552)
\lineto(251.14136596,124.70268365)
\closepath
\moveto(260.73060692,122.85338298)
\lineto(260.73060692,128.5871378)
\lineto(257.29924229,128.5871378)
\lineto(253.86785975,128.5871378)
\lineto(253.86785975,122.85338298)
\lineto(253.86785975,117.11959987)
\lineto(257.29924229,117.11959987)
\lineto(260.73060692,117.11959987)
\closepath
\moveto(235.41406244,130.325126)
\curveto(235.41137643,132.70862901)(235.41003343,130.75848839)(235.41003343,125.99143431)
\curveto(235.41003343,121.22441435)(235.41271944,119.27424948)(235.41406244,121.65777674)
\curveto(235.41674845,124.04130356)(235.41674845,127.94159918)(235.41406244,130.325126)
\closepath
\moveto(250.95593749,124.81803273)
\lineto(250.95593749,132.30304726)
\lineto(250.02643143,132.30304726)
\lineto(249.09692985,132.30304726)
\lineto(249.09692985,133.47648297)
\lineto(249.09692985,134.64991284)
\lineto(242.27292385,134.64991284)
\lineto(235.44891785,134.64991284)
\lineto(235.44891785,125.99143431)
\lineto(235.44891785,117.33298811)
\lineto(243.20242543,117.33298811)
\lineto(250.95593749,117.33298811)
\closepath
\moveto(231.79292248,133.19200453)
\lineto(231.79292248,134.00984961)
\lineto(230.70077872,134.00984961)
\lineto(229.60860809,134.00984961)
\lineto(229.60860809,133.19200453)
\lineto(229.60860809,132.3741599)
\lineto(230.70077872,132.3741599)
\lineto(231.79292248,132.3741599)
\closepath
\moveto(231.82000638,133.60537882)
\curveto(231.81732037,133.83273147)(231.81548494,133.64669232)(231.81548494,133.19200453)
\curveto(231.81548494,132.73729923)(231.81817094,132.55130633)(231.82000638,132.77863069)
\curveto(231.82269239,133.00598873)(231.82269239,133.37802617)(231.82000638,133.60537882)
\closepath
\moveto(251.33948582,133.1964592)
\lineto(251.34351483,134.00984961)
\lineto(250.25120095,134.00984961)
\lineto(249.15889155,134.00984961)
\lineto(249.15889155,133.19189676)
\lineto(249.15889155,132.37393897)
\lineto(250.24715403,132.37847447)
\lineto(251.3354389,132.38300997)
\lineto(251.33946791,133.19640621)
\closepath
\moveto(251.37033013,133.60537882)
\curveto(251.36764412,133.83273147)(251.36580868,133.64669232)(251.36580868,133.19200453)
\curveto(251.36580868,132.73729923)(251.36849469,132.55130633)(251.37033013,132.77863069)
\curveto(251.37301613,133.00598873)(251.37301613,133.37802617)(251.37033013,133.60537882)
\closepath
\moveto(231.51052021,134.47655782)
\lineto(231.51454922,134.86325033)
\lineto(230.70095778,134.86325033)
\lineto(229.88734397,134.86325033)
\lineto(229.88734397,134.47193924)
\lineto(229.88734397,134.080656)
\lineto(230.69677209,134.0851915)
\lineto(231.50620469,134.089727)
\lineto(231.5102337,134.47641996)
\closepath
\moveto(231.54114069,134.67211749)
\curveto(231.53845468,134.78212949)(231.53603727,134.69201084)(231.53603727,134.4721036)
\curveto(231.53603727,134.252107)(231.53872328,134.16208265)(231.54114069,134.27209509)
\curveto(231.54382669,134.38210754)(231.54382669,134.5621275)(231.54114069,134.67211749)
\closepath
\moveto(251.04888675,134.4721036)
\lineto(251.04888675,134.86325033)
\lineto(250.2433399,134.86325033)
\lineto(249.43777067,134.86325033)
\lineto(249.43777067,134.4721036)
\lineto(249.43777067,134.08096181)
\lineto(250.2433399,134.08096181)
\lineto(251.04888675,134.08096181)
\closepath
\moveto(251.09141519,134.67211749)
\curveto(251.08872918,134.78212949)(251.08631178,134.69201084)(251.08631178,134.4721036)
\curveto(251.08631178,134.252107)(251.08899778,134.16208265)(251.09141519,134.27209509)
\curveto(251.09410119,134.38210754)(251.09410119,134.5621275)(251.09141519,134.67211749)
\closepath
\moveto(251.14180915,134.4721036)
\curveto(251.14180915,134.72102548)(251.13603424,134.86325033)(251.12623031,134.86325033)
\curveto(251.11638162,134.86325033)(251.11065147,134.72102548)(251.11065147,134.4721036)
\curveto(251.11065147,134.22321539)(251.11642639,134.08096181)(251.12623031,134.08096181)
\curveto(251.136079,134.08096181)(251.14180915,134.22321539)(251.14180915,134.4721036)
\closepath
\moveto(253.80633677,135.64553931)
\lineto(253.80633677,136.35671519)
\lineto(241.56806197,136.35671519)
\lineto(229.32976031,136.35671519)
\lineto(229.32976031,135.64553931)
\lineto(229.32976031,134.93436298)
\lineto(241.56806197,134.93436298)
\lineto(253.80633677,134.93436298)
\closepath
\moveto(253.80633677,137.13897542)
\lineto(253.80633677,137.85014995)
\lineto(241.56806197,137.85014995)
\lineto(229.32976031,137.85014995)
\lineto(229.32976031,137.13897542)
\lineto(229.32976031,136.42782783)
\lineto(241.56806197,136.42782783)
\lineto(253.80633677,136.42782783)
\closepath
\moveto(233.15195235,137.89645699)
\curveto(233.14684893,137.90229477)(233.13337413,137.90310308)(233.12227197,137.89645699)
\curveto(233.10996111,137.89079884)(233.11349768,137.88648787)(233.13158346,137.88567956)
\curveto(233.14801287,137.88567956)(233.15701099,137.88927204)(233.15181805,137.89506491)
\closepath
\moveto(233.32236157,137.89645699)
\curveto(233.31725815,137.90229477)(233.30378335,137.90310308)(233.29268119,137.89645699)
\curveto(233.28037033,137.89079884)(233.2839069,137.88648787)(233.30199268,137.88567956)
\curveto(233.31837732,137.88567956)(233.32742021,137.88927204)(233.32222727,137.89506491)
\closepath
\moveto(231.51404783,138.09989901)
\lineto(231.51404783,138.27685099)
\lineto(230.70076081,138.27685099)
\lineto(229.88744693,138.27685099)
\lineto(229.88744693,138.09965202)
\lineto(229.88744693,137.92244857)
\lineto(230.70076081,137.92270004)
\lineto(231.51404783,137.92294702)
\closepath
\moveto(232.39706806,137.98073652)
\curveto(232.39706806,138.01495487)(232.40852836,138.05198434)(232.42415197,138.06796637)
\curveto(232.44438655,138.08871293)(233.1961282,138.59223461)(233.65959419,138.89554302)
\curveto(233.68314151,138.91103558)(233.46609426,138.91602014)(232.76533302,138.91628958)
\lineto(231.84006637,138.91628958)
\lineto(231.83603736,138.60070837)
\lineto(231.83200835,138.28512716)
\lineto(231.69645007,138.28000788)
\lineto(231.56089627,138.2748886)
\lineto(231.56089627,138.10963013)
\curveto(231.56089627,138.01874943)(231.56541771,137.93904133)(231.57128216,137.93253894)
\curveto(231.57705707,137.92589286)(231.7651626,137.92063886)(231.9895516,137.92063886)
\lineto(232.39747992,137.92063886)
\lineto(232.39747992,137.98011233)
\closepath
\moveto(233.16423187,137.93753698)
\curveto(233.35442353,137.96156167)(233.85689929,138.00273596)(234.59687175,138.05484934)
\curveto(235.28316438,138.10319062)(237.06997663,138.19240531)(240.64630057,138.35691385)
\curveto(244.13105853,138.51718775)(245.67343074,138.60729608)(246.58730872,138.70401007)
\curveto(247.13878619,138.76237436)(247.29989735,138.80257868)(247.29989735,138.88176138)
\curveto(247.29989735,138.91566539)(247.28812368,138.91687785)(246.96236031,138.91687785)
\curveto(246.776708,138.91687785)(246.50310688,138.9052472)(246.35433345,138.89087729)
\curveto(245.39567079,138.79895027)(243.93540085,138.71763454)(239.95694142,138.53460229)
\curveto(237.94495227,138.44200617)(237.49070377,138.42024923)(236.23895749,138.35648724)
\curveto(234.27297963,138.25631549)(233.02903173,138.15758073)(232.62169881,138.06941234)
\curveto(232.48811474,138.0404929)(232.45903424,138.0233927)(232.45903424,137.97380304)
\curveto(232.45903424,137.95642443)(232.46780853,137.93832732)(232.47846302,137.93397144)
\curveto(232.51772349,137.91794001)(233.02966294,137.92040984)(233.16429455,137.93756392)
\closepath
\moveto(233.44628496,137.93187882)
\curveto(233.44118155,137.9377166)(233.42770675,137.93852491)(233.41660459,137.93457318)
\curveto(233.40429372,137.92878031)(233.4078303,137.92460406)(233.42591608,137.92379575)
\curveto(233.44230072,137.92379575)(233.45134361,137.92738823)(233.44615066,137.9331811)
\closepath
\moveto(233.51457221,137.95289482)
\curveto(233.50333575,137.95693636)(233.48690634,137.95289482)(233.47790821,137.94184795)
\curveto(233.46644792,137.92860069)(233.4722676,137.92635539)(233.4981428,137.93407922)
\curveto(233.51922795,137.94041096)(233.52612204,137.94849403)(233.51452744,137.95289482)
\closepath
\moveto(245.96519822,137.95558918)
\curveto(245.99725124,137.97440478)(246.24966424,138.14119454)(246.52613939,138.32630145)
\curveto(246.80261901,138.51143979)(247.04971373,138.67610999)(247.07528899,138.69228512)
\curveto(247.11795173,138.71909398)(247.11862323,138.72098003)(247.08298887,138.71303168)
\curveto(246.92778693,138.6785888)(246.43650287,138.62621496)(245.83592072,138.58012347)
\curveto(244.79944892,138.50060847)(243.49276485,138.43040696)(240.35966336,138.28594445)
\curveto(237.71122044,138.16382266)(237.52445344,138.15503007)(236.75788951,138.11648727)
\curveto(235.49148211,138.05286898)(234.31550807,137.98200287)(233.83773463,137.94059058)
\curveto(233.7482727,137.93282185)(236.4272197,137.92541236)(239.79097754,137.92397537)
\curveto(245.792967,137.92128101)(245.90795494,137.92128101)(245.96515793,137.95558918)
\closepath
\moveto(249.37576869,138.09858775)
\lineto(249.37576869,138.27567445)
\lineto(249.2402104,138.28079373)
\lineto(249.10467898,138.28591302)
\lineto(249.10064997,138.60149422)
\lineto(249.09662096,138.91707543)
\lineto(248.23039722,138.91707543)
\lineto(247.36417349,138.91707543)
\lineto(247.35907007,138.85367269)
\lineto(247.35396666,138.79027893)
\lineto(246.72662741,138.37206512)
\curveto(246.3816188,138.14205673)(246.08881273,137.94680108)(246.07601391,137.93817913)
\curveto(246.06236004,137.92906322)(246.74712164,137.92241713)(247.71415569,137.92187826)
\lineto(249.375612,137.92187826)
\lineto(249.375612,138.09899191)
\closepath
\moveto(251.04885093,138.09858775)
\lineto(251.04885093,138.27601125)
\lineto(250.24330409,138.27601125)
\lineto(249.43773486,138.27601125)
\lineto(249.43773486,138.1085928)
\curveto(249.43773486,137.99111878)(249.44350978,137.93858777)(249.45716364,137.93252996)
\curveto(249.4678629,137.92799446)(249.83031713,137.92314461)(250.26270601,137.92256083)
\lineto(251.04891361,137.92256083)
\lineto(251.04891361,138.09997984)
\closepath
\moveto(251.09128984,138.19153862)
\curveto(251.08860383,138.24288411)(251.08551493,138.20092397)(251.08551493,138.09822851)
\curveto(251.08551493,137.99554651)(251.08820093,137.95354595)(251.09128984,138.00489144)
\curveto(251.09397585,138.05617855)(251.09397585,138.14022906)(251.09128984,138.19153862)
\closepath
\moveto(231.54102429,138.19153862)
\curveto(231.53744295,138.24316702)(231.53524938,138.20536966)(231.53524938,138.10768121)
\curveto(231.53524938,138.00989397)(231.53793538,137.967615)(231.54102429,138.01373792)
\curveto(231.54460563,138.05986982)(231.54460563,138.13986532)(231.54118545,138.19153862)
\closepath
\moveto(233.72517304,137.96728718)
\curveto(233.72011439,137.97308005)(233.70659483,137.97393327)(233.69553743,137.96998154)
\curveto(233.68322657,137.96432339)(233.68676314,137.96001241)(233.70484892,137.95920411)
\curveto(233.72127833,137.95920411)(233.73027645,137.96279658)(233.72508351,137.96858945)
\closepath
\moveto(233.86459022,137.96728718)
\curveto(233.85953157,137.97308005)(233.84601201,137.97393327)(233.83495461,137.96998154)
\curveto(233.82264375,137.96432339)(233.82618032,137.96001241)(233.8442661,137.95920411)
\curveto(233.86069551,137.95920411)(233.86969363,137.96279658)(233.86450069,137.96858945)
\closepath
\moveto(233.95755291,137.98498013)
\curveto(233.9524495,137.99077301)(233.9389747,137.99162622)(233.92787254,137.98767449)
\curveto(233.91556167,137.98183672)(233.91909825,137.97770537)(233.93718403,137.97689706)
\curveto(233.95361343,137.97689706)(233.96261156,137.98048954)(233.95741861,137.98632731)
\closepath
\moveto(234.20541313,138.00267309)
\curveto(234.20030972,138.00846596)(234.18683492,138.00931917)(234.17577753,138.00536745)
\curveto(234.16346666,137.99957458)(234.16700324,137.99539832)(234.18508902,137.99459001)
\curveto(234.20151843,137.99459001)(234.21051655,137.99818249)(234.2053236,138.00397536)
\closepath
\moveto(234.34482584,138.00267309)
\curveto(234.33972242,138.00846596)(234.32624762,138.00931917)(234.31514546,138.00536745)
\curveto(234.3028346,137.99957458)(234.30637117,137.99539832)(234.32445695,137.99459001)
\curveto(234.34084159,137.99459001)(234.34988448,137.99818249)(234.34469154,138.00397536)
\closepath
\moveto(234.46134481,138.02148869)
\curveto(234.45064555,138.02660797)(234.43327604,138.02660797)(234.42257678,138.02148869)
\curveto(234.41187752,138.01636941)(234.41989077,138.01264221)(234.44200556,138.01264221)
\curveto(234.46335931,138.01264221)(234.4719993,138.01668375)(234.46143434,138.02148869)
\closepath
\moveto(234.74020155,138.03918164)
\curveto(234.72954706,138.04425602)(234.71213278,138.04425602)(234.70143352,138.03918164)
\curveto(234.69077903,138.03406236)(234.69874751,138.03033517)(234.7208623,138.03033517)
\curveto(234.74221605,138.03033517)(234.75085604,138.0343767)(234.74029108,138.03918164)
\closepath
\moveto(234.89509908,138.03918164)
\curveto(234.88439982,138.04425602)(234.86703031,138.04425602)(234.85642058,138.03918164)
\curveto(234.84572132,138.03406236)(234.85373458,138.03033517)(234.87584937,138.03033517)
\curveto(234.89720312,138.03033517)(234.90584311,138.0343767)(234.89527815,138.03918164)
\closepath
\moveto(235.01902695,138.0568746)
\curveto(235.00832769,138.06199388)(234.99095818,138.06199388)(234.98034846,138.0568746)
\curveto(234.9696492,138.05175532)(234.97766245,138.04802812)(234.99977724,138.04802812)
\curveto(235.02113099,138.04802812)(235.02977098,138.05206966)(235.01920602,138.0568746)
\closepath
\moveto(235.10392267,138.0568746)
\curveto(235.09881926,138.06266747)(235.08534446,138.06352068)(235.0742423,138.0568746)
\curveto(235.06193143,138.05108173)(235.06546801,138.04690547)(235.08355379,138.04609716)
\curveto(235.0999832,138.04609716)(235.10898132,138.04968964)(235.10378837,138.05548251)
\closepath
\moveto(235.18944065,138.0568746)
\curveto(235.17878616,138.06199388)(235.16137188,138.06199388)(235.15076215,138.0568746)
\curveto(235.14006289,138.05175532)(235.14807615,138.04802812)(235.17019094,138.04802812)
\curveto(235.19154469,138.04802812)(235.20022944,138.05206966)(235.18961972,138.0568746)
\closepath
\moveto(235.32888021,138.07456755)
\curveto(235.31818095,138.07964192)(235.30081144,138.07964192)(235.29011218,138.07456755)
\curveto(235.27945769,138.06944827)(235.28742618,138.06572107)(235.30954096,138.06572107)
\curveto(235.33089472,138.06572107)(235.33953471,138.06976261)(235.32896975,138.07456755)
\closepath
\moveto(235.41374459,138.07456755)
\curveto(235.40864118,138.08036042)(235.39516638,138.08121363)(235.38406422,138.07726191)
\curveto(235.37175336,138.07146904)(235.37528993,138.06729278)(235.39337571,138.06648447)
\curveto(235.40976035,138.06648447)(235.41880324,138.07007695)(235.41361029,138.07586982)
\closepath
\moveto(235.49926705,138.07456755)
\curveto(235.48856779,138.07964192)(235.47119828,138.07964192)(235.46054379,138.07456755)
\curveto(235.44988929,138.06944827)(235.45785778,138.06572107)(235.47997257,138.06572107)
\curveto(235.50132632,138.06572107)(235.50996631,138.06976261)(235.49940135,138.07456755)
\closepath
\moveto(235.63870661,138.0922605)
\curveto(235.62805212,138.09737978)(235.61063784,138.09737978)(235.59993858,138.0922605)
\curveto(235.58923932,138.08714122)(235.59725257,138.08341403)(235.61936736,138.08341403)
\curveto(235.64072112,138.08341403)(235.6493611,138.08745556)(235.63879614,138.0922605)
\closepath
\moveto(235.82490507,138.09209884)
\curveto(235.81451918,138.09717322)(235.79347879,138.09717322)(235.77840582,138.09209884)
\curveto(235.76336418,138.08756334)(235.77186987,138.0835218)(235.797566,138.08325236)
\curveto(235.82299353,138.08325236)(235.83548346,138.08684484)(235.82496327,138.09151506)
\closepath
\moveto(235.96401336,138.11006123)
\curveto(235.9533141,138.11518051)(235.93594459,138.11518051)(235.92533486,138.11006123)
\curveto(235.9146356,138.10494195)(235.92264886,138.10121475)(235.94476365,138.10121475)
\curveto(235.9661174,138.10121475)(235.97480215,138.10525629)(235.96419243,138.11006123)
\closepath
\moveto(236.05698053,138.11006123)
\curveto(236.04628127,138.11518051)(236.02891176,138.11518051)(236.0182125,138.11006123)
\curveto(236.00751324,138.10494195)(236.01552649,138.10121475)(236.03764128,138.10121475)
\curveto(236.05899504,138.10121475)(236.06763502,138.10525629)(236.05707006,138.11006123)
\closepath
\moveto(236.14991189,138.11006123)
\curveto(236.13925739,138.11518051)(236.12184312,138.11518051)(236.11123339,138.11006123)
\curveto(236.10053413,138.10494195)(236.10854738,138.10121475)(236.13066217,138.10121475)
\curveto(236.15201592,138.10121475)(236.16065591,138.10525629)(236.15009095,138.11006123)
\closepath
\moveto(236.30513621,138.12775418)
\curveto(236.29475032,138.13287346)(236.27370994,138.13287346)(236.25864144,138.12775418)
\curveto(236.2435998,138.12321868)(236.25210549,138.11917714)(236.27780162,138.11890771)
\curveto(236.30322915,138.11890771)(236.31571908,138.12250018)(236.30519889,138.1271704)
\closepath
\moveto(236.49103474,138.12775418)
\curveto(236.48064885,138.13287346)(236.45960846,138.13287346)(236.44453549,138.12775418)
\curveto(236.42949385,138.12321868)(236.43799954,138.11917714)(236.46369567,138.11890771)
\curveto(236.4891232,138.11890771)(236.50161313,138.12250018)(236.49109294,138.1271704)
\closepath
\moveto(232.93156102,138.17768064)
\curveto(233.64217543,138.27141288)(235.36605902,138.37858098)(238.51623904,138.52486667)
\curveto(238.93800925,138.54449058)(239.37023697,138.56469827)(239.47673714,138.56992084)
\curveto(239.58324178,138.57504012)(240.19668095,138.60324107)(240.8399706,138.63246138)
\curveto(243.21893542,138.74040187)(245.64041077,138.87190003)(245.85921735,138.90501818)
\curveto(245.8805711,138.90861066)(243.18194463,138.91220314)(239.8623716,138.91359522)
\lineto(233.82678467,138.91628958)
\lineto(233.4101537,138.64754979)
\curveto(232.96408408,138.35984172)(232.71808616,138.19693183)(232.66821597,138.15620661)
\curveto(232.63911756,138.13245135)(232.64010243,138.13159813)(232.6837948,138.14102839)
\curveto(232.70922233,138.14668654)(232.82088411,138.16312213)(232.93165503,138.17767166)
\closepath
\moveto(236.64593227,138.14561778)
\curveto(236.63554638,138.15073706)(236.61455076,138.15073706)(236.59949121,138.14561778)
\curveto(236.58444958,138.14108227)(236.59295526,138.13704074)(236.6186514,138.1367713)
\curveto(236.64407893,138.1367713)(236.65656886,138.14036378)(236.64604866,138.145034)
\closepath
\moveto(236.73859055,138.14581087)
\curveto(236.72793606,138.15093015)(236.71052178,138.15093015)(236.69991205,138.14581087)
\curveto(236.68921279,138.14069159)(236.69722605,138.1369644)(236.71934083,138.1369644)
\curveto(236.74069459,138.1369644)(236.74937934,138.14100593)(236.73876962,138.14581087)
\closepath
\moveto(236.84734248,138.14561778)
\curveto(236.83695659,138.15073706)(236.81591621,138.15073706)(236.80084323,138.14561778)
\curveto(236.78580159,138.14108227)(236.79430728,138.13704074)(236.82000341,138.1367713)
\curveto(236.84543094,138.1367713)(236.85792087,138.14036378)(236.84740068,138.145034)
\closepath
\moveto(237.00224002,138.16331073)
\curveto(236.99185412,138.16843001)(236.9708585,138.16843001)(236.95579896,138.16331073)
\curveto(236.94075732,138.15877523)(236.94926301,138.15473369)(236.97495914,138.15446425)
\curveto(237.00038667,138.15446425)(237.0128766,138.15805673)(237.00235641,138.16272695)
\closepath
\moveto(237.09489829,138.16347688)
\curveto(237.08419903,138.16859616)(237.06682952,138.16859616)(237.05617503,138.16347688)
\curveto(237.04547577,138.1583576)(237.05348902,138.15463041)(237.07560381,138.15463041)
\curveto(237.09695757,138.15463041)(237.10559755,138.15867194)(237.09503259,138.16347688)
\closepath
\moveto(237.20364575,138.16331073)
\curveto(237.19325986,138.16843001)(237.17221947,138.16843001)(237.1571465,138.16331073)
\curveto(237.14210486,138.15877523)(237.15061055,138.15473369)(237.17630668,138.15446425)
\curveto(237.20173421,138.15446425)(237.21422414,138.15805673)(237.20370395,138.16272695)
\closepath
\moveto(237.37405497,138.18100368)
\curveto(237.36366908,138.18612296)(237.34262869,138.18612296)(237.3275602,138.18100368)
\curveto(237.31251856,138.17646818)(237.32102425,138.17242664)(237.34672038,138.17215721)
\curveto(237.37214791,138.17215721)(237.38463784,138.17574968)(237.37411765,138.18041991)
\closepath
\moveto(237.46669087,138.18116984)
\curveto(237.45603637,138.18628912)(237.4386221,138.18628912)(237.42801237,138.18116984)
\curveto(237.41731311,138.17605055)(237.42532636,138.17232336)(237.44744115,138.17232336)
\curveto(237.46879491,138.17232336)(237.47743489,138.1763649)(237.46686993,138.18116984)
\closepath
\moveto(237.57543833,138.18100368)
\curveto(237.56505243,138.18612296)(237.54401205,138.18612296)(237.52893907,138.18100368)
\curveto(237.51389744,138.17646818)(237.52240312,138.17242664)(237.54809925,138.17215721)
\curveto(237.57352678,138.17215721)(237.58601672,138.17574968)(237.57549652,138.18041991)
\closepath
\moveto(237.73780296,138.19923551)
\curveto(237.72298515,138.20377101)(237.69849772,138.20377101)(237.683568,138.19923551)
\curveto(237.66852637,138.19470001)(237.680882,138.19119734)(237.7106519,138.19119734)
\curveto(237.74046658,138.19119734)(237.75268791,138.19478982)(237.73773581,138.19923551)
\closepath
\moveto(237.8385103,138.19923551)
\curveto(237.82781104,138.20435479)(237.81044153,138.20435479)(237.79974227,138.19923551)
\curveto(237.78904301,138.19411623)(237.79705626,138.1901196)(237.81917105,138.1901196)
\curveto(237.84052481,138.1901196)(237.84916479,138.19416113)(237.83859983,138.19923551)
\closepath
\moveto(237.94722642,138.19904241)
\curveto(237.93684053,138.20416169)(237.91580014,138.20416169)(237.90072717,138.19904241)
\curveto(237.88568553,138.19450691)(237.89419122,138.19046537)(237.91988735,138.19019594)
\curveto(237.94531488,138.19019594)(237.95780481,138.19378841)(237.94728462,138.19845863)
\closepath
\moveto(238.11763564,138.21673537)
\curveto(238.10724975,138.22185465)(238.08620936,138.22185465)(238.07114087,138.21673537)
\curveto(238.05609923,138.21219986)(238.06460492,138.20820323)(238.09030105,138.20788889)
\curveto(238.11572858,138.20788889)(238.12821851,138.21148137)(238.11769832,138.21615159)
\closepath
\moveto(238.2102984,138.21692846)
\curveto(238.19959914,138.22204774)(238.18222963,138.22204774)(238.17153037,138.21692846)
\curveto(238.16083111,138.21180918)(238.16884436,138.20781255)(238.19095915,138.20781255)
\curveto(238.2123129,138.20781255)(238.22095289,138.21185409)(238.21038793,138.21692846)
\closepath
\moveto(238.3264861,138.21692846)
\curveto(238.31166829,138.22146396)(238.28718086,138.22146396)(238.27225114,138.21692846)
\curveto(238.25743334,138.21239296)(238.26956514,138.20889029)(238.29933504,138.20889029)
\curveto(238.32914972,138.20889029)(238.34137105,138.21248277)(238.32641895,138.21692846)
\closepath
\moveto(233.72517304,138.23322933)
\curveto(233.72011439,138.23906711)(233.70659483,138.23987541)(233.69553743,138.23592369)
\curveto(233.68322657,138.23013082)(233.68676314,138.22595456)(233.70484892,138.22514625)
\curveto(233.72127833,138.22514625)(233.73027645,138.22873873)(233.72508351,138.23457651)
\closepath
\moveto(233.85718132,138.23322933)
\curveto(233.84648206,138.23834861)(233.82911255,138.23834861)(233.81841329,138.23322933)
\curveto(233.80771403,138.22811005)(233.81572728,138.22411342)(233.83784207,138.22411342)
\curveto(233.85919582,138.22411342)(233.86783581,138.22815495)(233.85727085,138.23322933)
\closepath
\moveto(238.50463997,138.23322933)
\curveto(238.49394071,138.23834861)(238.4765712,138.23834861)(238.46587194,138.23322933)
\curveto(238.45517268,138.22811005)(238.46318593,138.22411342)(238.48530072,138.22411342)
\curveto(238.50665447,138.22411342)(238.51529446,138.22815495)(238.5047295,138.23322933)
\closepath
\moveto(238.59757132,138.23322933)
\curveto(238.58691683,138.23834861)(238.56950255,138.23834861)(238.55889283,138.23322933)
\curveto(238.54819357,138.22811005)(238.55620682,138.22411342)(238.57832161,138.22411342)
\curveto(238.59967536,138.22411342)(238.60831535,138.22815495)(238.59775039,138.23322933)
\closepath
\moveto(238.70631878,138.23303623)
\curveto(238.69593289,138.23815551)(238.6748925,138.23815551)(238.65982401,138.23303623)
\curveto(238.64478237,138.22850073)(238.65328806,138.22445919)(238.67898419,138.22418976)
\curveto(238.70441172,138.22418976)(238.71690165,138.22778223)(238.70638146,138.23245246)
\closepath
\moveto(233.95755291,138.24992088)
\curveto(233.9524495,138.25571375)(233.9389747,138.25656696)(233.92787254,138.25261524)
\curveto(233.91556167,138.24682237)(233.91909825,138.24264611)(233.93718403,138.2418378)
\curveto(233.95361343,138.2418378)(233.96261156,138.24543028)(233.95741861,138.25122315)
\closepath
\moveto(234.09697009,138.24992088)
\curveto(234.09186668,138.25571375)(234.07839188,138.25656696)(234.06728972,138.25261524)
\curveto(234.05497885,138.24682237)(234.05851543,138.24264611)(234.07660121,138.2418378)
\curveto(234.09303061,138.2418378)(234.10202874,138.24543028)(234.09683579,138.25122315)
\closepath
\moveto(238.87673248,138.24992088)
\curveto(238.86634659,138.25504016)(238.8453062,138.25504016)(238.83023323,138.24992088)
\curveto(238.81519159,138.24538538)(238.82369728,138.24134384)(238.84939341,138.2410744)
\curveto(238.87482094,138.2410744)(238.88731087,138.24466688)(238.87679068,138.2493371)
\closepath
\moveto(238.96939076,138.25009152)
\curveto(238.95873626,138.2552108)(238.94132199,138.2552108)(238.93062273,138.25009152)
\curveto(238.91996823,138.24497224)(238.92793672,138.24124505)(238.95005151,138.24124505)
\curveto(238.97140526,138.24124505)(238.98004525,138.24528658)(238.96948029,138.25009152)
\closepath
\moveto(239.08557845,138.25009152)
\curveto(239.07053682,138.25462702)(239.04627322,138.25462702)(239.03133903,138.25009152)
\curveto(239.01652122,138.24555602)(239.02865302,138.24205335)(239.05842293,138.24205335)
\curveto(239.0882376,138.24205335)(239.10045893,138.24564583)(239.08550683,138.25009152)
\closepath
\moveto(234.19799976,138.26747013)
\curveto(234.1873005,138.27258941)(234.16993099,138.27258941)(234.15923173,138.26747013)
\curveto(234.14853247,138.26235085)(234.15654572,138.25862366)(234.17866051,138.25862366)
\curveto(234.20001426,138.25862366)(234.20865425,138.26266519)(234.19808929,138.26747013)
\closepath
\moveto(234.34482584,138.26747013)
\curveto(234.33972242,138.27330791)(234.32624762,138.27411622)(234.31514546,138.27016449)
\curveto(234.3028346,138.26437162)(234.30637117,138.26019537)(234.32445695,138.25938706)
\curveto(234.34084159,138.25938706)(234.34988448,138.26297954)(234.34469154,138.26877241)
\closepath
\moveto(239.2640054,138.26747013)
\curveto(239.25361951,138.27258941)(239.23257913,138.27258941)(239.21751063,138.26747013)
\curveto(239.20246899,138.26293463)(239.21097468,138.25889309)(239.23667081,138.25862366)
\curveto(239.26209834,138.25862366)(239.27458827,138.26221613)(239.26406808,138.26688636)
\closepath
\moveto(239.35666816,138.26766323)
\curveto(239.3459689,138.27278251)(239.32859939,138.27278251)(239.31790013,138.26766323)
\curveto(239.30720087,138.26254395)(239.31521412,138.25881675)(239.33732891,138.25881675)
\curveto(239.35868266,138.25881675)(239.36732265,138.26285829)(239.35675769,138.26766323)
\closepath
\moveto(239.47285138,138.26766323)
\curveto(239.45803358,138.27219873)(239.43354615,138.27219873)(239.41861643,138.26766323)
\curveto(239.40379862,138.26312773)(239.41593042,138.25962506)(239.44570033,138.25962506)
\curveto(239.475515,138.25962506)(239.48773633,138.26321754)(239.47273946,138.26766323)
\closepath
\moveto(234.45326888,138.2839641)
\curveto(234.44816547,138.28975697)(234.43469067,138.29061018)(234.42358851,138.28665845)
\curveto(234.41127764,138.2810003)(234.41481422,138.27668933)(234.4329,138.27588102)
\curveto(234.44928464,138.27588102)(234.45832753,138.2794735)(234.45313458,138.28526637)
\closepath
\moveto(234.53075122,138.2839641)
\curveto(234.52564781,138.28975697)(234.51217301,138.29061018)(234.50102608,138.28665845)
\curveto(234.48871522,138.2810003)(234.49225179,138.27668933)(234.51033757,138.27588102)
\curveto(234.52672221,138.27588102)(234.5357651,138.2794735)(234.53057216,138.28526637)
\closepath
\moveto(239.6587499,138.28665845)
\curveto(239.6439321,138.29119396)(239.61944467,138.29119396)(239.60451048,138.28665845)
\curveto(239.58978221,138.28212295)(239.60182447,138.27862029)(239.63159438,138.27862029)
\curveto(239.66140905,138.27862029)(239.67363038,138.28221276)(239.65867828,138.28665845)
\closepath
\moveto(239.75945277,138.28665845)
\curveto(239.74875351,138.29177774)(239.731384,138.29177774)(239.72068474,138.28665845)
\curveto(239.70998548,138.28153917)(239.71799874,138.27781198)(239.74011353,138.27781198)
\curveto(239.76146728,138.27781198)(239.77010727,138.28185352)(239.75954231,138.28665845)
\closepath
\moveto(239.84431716,138.28665845)
\curveto(239.83921374,138.29245132)(239.82573894,138.29330454)(239.81463678,138.28935281)
\curveto(239.80232592,138.28369466)(239.80586249,138.27938369)(239.82394827,138.27857538)
\curveto(239.84033291,138.27857538)(239.8493758,138.28216786)(239.84418286,138.28796073)
\closepath
\moveto(234.72471672,138.30547406)
\curveto(234.71401746,138.31059334)(234.69664795,138.31059334)(234.68594869,138.30547406)
\curveto(234.67524943,138.30035478)(234.68326268,138.29662758)(234.70537747,138.29662758)
\curveto(234.72673123,138.29662758)(234.73537121,138.30066912)(234.72480625,138.30547406)
\closepath
\moveto(234.80957663,138.30547406)
\curveto(234.80447321,138.31126693)(234.79099841,138.31212014)(234.77994102,138.30816842)
\curveto(234.76763015,138.30237554)(234.77116673,138.29819929)(234.78925251,138.29739098)
\curveto(234.80563715,138.29739098)(234.81468004,138.30098346)(234.80948709,138.30677633)
\closepath
\moveto(234.89509908,138.30547406)
\curveto(234.88439982,138.31059334)(234.86703031,138.31059334)(234.85642058,138.30547406)
\curveto(234.84572132,138.30035478)(234.85373458,138.29662758)(234.87584937,138.29662758)
\curveto(234.89720312,138.29662758)(234.90584311,138.30066912)(234.89527815,138.30547406)
\closepath
\moveto(240.04604969,138.30547406)
\curveto(240.03123189,138.31000956)(240.00674446,138.31000956)(239.99181474,138.30547406)
\curveto(239.97708647,138.30093855)(239.98912873,138.29743589)(240.01889864,138.29743589)
\curveto(240.04871331,138.29743589)(240.06093464,138.30102837)(240.04598254,138.30547406)
\closepath
\moveto(240.1467257,138.30547406)
\curveto(240.13607121,138.31059334)(240.11865693,138.31059334)(240.1080472,138.30547406)
\curveto(240.09734794,138.30035478)(240.1053612,138.29662758)(240.12747598,138.29662758)
\curveto(240.14882974,138.29662758)(240.15751449,138.30066912)(240.14690477,138.30547406)
\closepath
\moveto(240.25547316,138.3053079)
\curveto(240.24508727,138.31042719)(240.22404688,138.31042719)(240.2089739,138.3053079)
\curveto(240.19393227,138.3007724)(240.20243796,138.29673086)(240.22813409,138.29646143)
\curveto(240.25356162,138.29646143)(240.26605155,138.30005391)(240.25553135,138.30472413)
\closepath
\moveto(229.79065668,138.32439294)
\curveto(229.7586932,138.32843448)(229.70642351,138.32843448)(229.67446898,138.32439294)
\curveto(229.6425055,138.32035141)(229.66869406,138.31747742)(229.73257626,138.31747742)
\curveto(229.79645397,138.31747742)(229.82260673,138.32017178)(229.79065668,138.32439294)
\closepath
\moveto(230.10048307,138.32458604)
\curveto(230.06001391,138.32817852)(229.99378593,138.32817852)(229.95332572,138.32458604)
\curveto(229.91281179,138.32099356)(229.9459392,138.31793995)(230.02688649,138.31793995)
\curveto(230.10784273,138.31793995)(230.14097015,138.32063431)(230.10048307,138.32458604)
\closepath
\moveto(230.82083874,138.32477913)
\curveto(230.70795483,138.32747349)(230.52321129,138.32747349)(230.41030947,138.32477913)
\curveto(230.29743004,138.32208478)(230.38980629,138.31912098)(230.61557411,138.31912098)
\curveto(230.84137774,138.31912098)(230.93374056,138.32181534)(230.82083874,138.32477913)
\closepath
\moveto(231.19262684,138.32477913)
\curveto(231.16496097,138.32882067)(231.1196346,138.32882067)(231.09194635,138.32477913)
\curveto(231.06428048,138.3207376)(231.08684294,138.31759418)(231.14228659,138.31759418)
\curveto(231.19765415,138.31759418)(231.22034195,138.32118666)(231.19262684,138.32477913)
\closepath
\moveto(231.36303606,138.32477913)
\curveto(231.34387588,138.32882067)(231.3125257,138.32882067)(231.29332971,138.32477913)
\curveto(231.27416952,138.3207376)(231.28974836,138.3170104)(231.32820303,138.3170104)
\curveto(231.36656816,138.3170104)(231.38221414,138.32060288)(231.36307635,138.32477913)
\closepath
\moveto(231.50275317,138.32477913)
\curveto(231.49236728,138.32989841)(231.47141643,138.32989841)(231.45631212,138.32477913)
\curveto(231.44127048,138.32024363)(231.44977617,138.31620209)(231.4754723,138.31593266)
\curveto(231.50089983,138.31593266)(231.51338976,138.31952514)(231.50286957,138.32419536)
\closepath
\moveto(231.74300304,138.32477913)
\curveto(231.7112634,138.32882067)(231.65549295,138.32882067)(231.61907965,138.32494978)
\curveto(231.58263949,138.32090824)(231.60869375,138.31803426)(231.67674374,138.31803426)
\curveto(231.74491906,138.31803426)(231.77471583,138.32072862)(231.74300304,138.32494978)
\closepath
\moveto(235.10392267,138.32208478)
\curveto(235.09881926,138.32787765)(235.08534446,138.32873086)(235.0742423,138.32477913)
\curveto(235.06193143,138.31898626)(235.06546801,138.31481001)(235.08355379,138.3140017)
\curveto(235.0999832,138.3140017)(235.10898132,138.31759418)(235.10378837,138.32338705)
\closepath
\moveto(235.18944065,138.32208478)
\curveto(235.17878616,138.32720406)(235.16137188,138.32720406)(235.15076215,138.32208478)
\curveto(235.14006289,138.3169655)(235.14807615,138.3132383)(235.17019094,138.3132383)
\curveto(235.19154469,138.3132383)(235.20022944,138.31727984)(235.18961972,138.32208478)
\closepath
\moveto(240.43332262,138.32208478)
\curveto(240.41850481,138.32662028)(240.39401738,138.32662028)(240.37908766,138.32208478)
\curveto(240.36426986,138.31754927)(240.37640166,138.31404661)(240.40617156,138.31404661)
\curveto(240.43598624,138.31404661)(240.44820757,138.31763908)(240.4332107,138.32208478)
\closepath
\moveto(240.53402996,138.32208478)
\curveto(240.52337547,138.32720406)(240.50596119,138.32720406)(240.49526193,138.32208478)
\curveto(240.48460744,138.3169655)(240.49257592,138.3132383)(240.51469071,138.3132383)
\curveto(240.53604447,138.3132383)(240.54468445,138.31727984)(240.5341195,138.32208478)
\closepath
\moveto(249.32545978,138.32208478)
\curveto(249.2934963,138.32612631)(249.24122213,138.32612631)(249.20926761,138.32208478)
\curveto(249.17730413,138.31804324)(249.20349269,138.31516926)(249.2673525,138.31516926)
\curveto(249.33125708,138.31516926)(249.35741431,138.31786361)(249.32545978,138.32208478)
\closepath
\moveto(250.44860006,138.32208478)
\curveto(250.25901275,138.32477913)(249.94880584,138.32477913)(249.75921405,138.32208478)
\curveto(249.56962674,138.31939042)(249.72474363,138.3169655)(250.10389586,138.3169655)
\curveto(250.48307496,138.3169655)(250.63816051,138.31965985)(250.44860006,138.32208478)
\closepath
\moveto(250.88265875,138.31939042)
\curveto(250.87227285,138.3245097)(250.85123247,138.3245097)(250.83615949,138.31939042)
\curveto(250.82111785,138.31485491)(250.82962354,138.31081338)(250.85531967,138.31054394)
\curveto(250.8807472,138.31054394)(250.89323714,138.31413642)(250.88271694,138.31880664)
\closepath
\moveto(250.97529016,138.31958351)
\curveto(250.96463567,138.32470279)(250.94722139,138.32470279)(250.93661167,138.31958351)
\curveto(250.92591241,138.31446423)(250.93392566,138.31073704)(250.95604045,138.31073704)
\curveto(250.9773942,138.31073704)(250.98603419,138.31477857)(250.97546923,138.31958351)
\closepath
\moveto(251.29329098,138.31958351)
\curveto(251.26155133,138.32362505)(251.20580774,138.32362505)(251.16936758,138.31974517)
\curveto(251.13292742,138.31570364)(251.15898169,138.31282966)(251.22703167,138.31282966)
\curveto(251.29520699,138.31282966)(251.32500376,138.31552401)(251.29329098,138.31974517)
\closepath
\moveto(235.313373,138.33619872)
\curveto(235.30267374,138.341318)(235.28530423,138.341318)(235.2746945,138.33619872)
\curveto(235.26399524,138.33107944)(235.2720085,138.32735225)(235.29412329,138.32735225)
\curveto(235.31547704,138.32735225)(235.32411703,138.33139378)(235.31355207,138.33619872)
\closepath
\moveto(235.49956699,138.33603706)
\curveto(235.48918109,138.34115634)(235.46814071,138.34115634)(235.45307221,138.33603706)
\curveto(235.43803057,138.33150156)(235.44653626,138.32746002)(235.47223239,138.32719058)
\curveto(235.49765992,138.32719058)(235.51014985,138.33078306)(235.49962966,138.33545328)
\closepath
\moveto(240.82062688,138.33603706)
\curveto(240.80580907,138.34057256)(240.78132165,138.34057256)(240.76638745,138.33603706)
\curveto(240.75156965,138.33150156)(240.76370144,138.32799889)(240.79347135,138.32799889)
\curveto(240.82328603,138.32799889)(240.83550736,138.33159137)(240.82055525,138.33603706)
\closepath
\moveto(240.92130289,138.33603706)
\curveto(240.91060363,138.34115634)(240.89323412,138.34115634)(240.88257962,138.33603706)
\curveto(240.87192513,138.33091778)(240.87989362,138.32719058)(240.90200841,138.32719058)
\curveto(240.92336216,138.32719058)(240.93204691,138.33123212)(240.92143719,138.33603706)
\closepath
\moveto(241.03004587,138.33586642)
\curveto(241.01965998,138.3409857)(240.99861959,138.3409857)(240.98355109,138.33586642)
\curveto(240.96850946,138.33133092)(240.97701514,138.32728938)(241.00271127,138.32701994)
\curveto(241.0281388,138.32701994)(241.04062874,138.33061242)(241.03010854,138.33528264)
\closepath
\moveto(231.81996609,138.77326855)
\curveto(231.81728008,138.85394662)(231.81486268,138.7879528)(231.81486268,138.62659666)
\curveto(231.81486268,138.46523603)(231.81754868,138.39923773)(231.81996609,138.4799158)
\curveto(231.8226521,138.56059386)(231.8226521,138.69259048)(231.81996609,138.77326855)
\closepath
\moveto(235.63900655,138.35378839)
\curveto(235.62862066,138.35886277)(235.60758027,138.35886277)(235.59251177,138.35378839)
\curveto(235.57747013,138.34925289)(235.58597582,138.34521135)(235.61167195,138.34494192)
\curveto(235.63709948,138.34494192)(235.64958941,138.34853439)(235.63906922,138.35320461)
\closepath
\moveto(235.72357099,138.35378839)
\curveto(235.71846758,138.35958126)(235.70499278,138.36043448)(235.69389062,138.35648275)
\curveto(235.68157975,138.35064497)(235.68511633,138.34651362)(235.70320211,138.34570532)
\curveto(235.71963152,138.34570532)(235.72862964,138.34929779)(235.72343669,138.35509067)
\closepath
\moveto(235.82490507,138.35378839)
\curveto(235.81451918,138.35886277)(235.79347879,138.35886277)(235.77840582,138.35378839)
\curveto(235.76336418,138.34925289)(235.77186987,138.34521135)(235.797566,138.34494192)
\curveto(235.82299353,138.34494192)(235.83548346,138.34853439)(235.82496327,138.35320461)
\closepath
\moveto(241.20045957,138.35378839)
\curveto(241.19007367,138.35886277)(241.16903329,138.35886277)(241.15396031,138.35378839)
\curveto(241.13891867,138.34925289)(241.14742436,138.34521135)(241.17312049,138.34494192)
\curveto(241.19854802,138.34494192)(241.21103796,138.34853439)(241.20051776,138.35320461)
\closepath
\moveto(241.29309098,138.35395005)
\curveto(241.28239172,138.35906933)(241.26502221,138.35906933)(241.25436772,138.35395005)
\curveto(241.24371323,138.34883077)(241.25168171,138.34510358)(241.2737965,138.34510358)
\curveto(241.29515026,138.34510358)(241.30379024,138.34914511)(241.29322528,138.35395005)
\closepath
\moveto(241.42446357,138.35395005)
\curveto(241.41936016,138.35974292)(241.40588536,138.36059614)(241.3947832,138.35664441)
\curveto(241.38247233,138.35098626)(241.38600891,138.34667529)(241.40409469,138.34586698)
\curveto(241.42047933,138.34586698)(241.42952222,138.34945946)(241.42432927,138.35525233)
\closepath
\moveto(251.37030774,139.59343571)
\curveto(251.36762174,139.82009861)(251.3657863,139.62954011)(251.3657863,139.16994096)
\curveto(251.3657863,138.71034181)(251.3684723,138.5249026)(251.37030774,138.7578164)
\curveto(251.37299375,138.99072572)(251.37299375,139.36674138)(251.37028088,139.59343571)
\closepath
\moveto(231.79293591,139.16988707)
\lineto(231.79293591,139.97882774)
\lineto(230.70079215,139.97882774)
\lineto(229.60862152,139.97882774)
\lineto(229.60862152,139.16988707)
\lineto(229.60862152,138.36091946)
\lineto(230.70079215,138.36091946)
\lineto(231.79293591,138.36091946)
\closepath
\moveto(235.96432225,138.37258603)
\curveto(235.95393636,138.37766041)(235.93289597,138.37766041)(235.917823,138.37258603)
\curveto(235.90278136,138.36805053)(235.91128705,138.36400899)(235.93698318,138.36373956)
\curveto(235.96241071,138.36373956)(235.97490064,138.36733203)(235.96438045,138.37200225)
\closepath
\moveto(236.04890908,138.37258603)
\curveto(236.04380567,138.3783789)(236.03033087,138.37923212)(236.01922871,138.37528039)
\curveto(236.00691784,138.36948752)(236.01045442,138.36531126)(236.0285402,138.36450296)
\curveto(236.0449696,138.36450296)(236.05396773,138.36809543)(236.04877478,138.37388831)
\closepath
\moveto(236.15021182,138.37258603)
\curveto(236.13982593,138.37766041)(236.11878555,138.37766041)(236.10371705,138.37258603)
\curveto(236.08867541,138.36805053)(236.0971811,138.36400899)(236.12287723,138.36373956)
\curveto(236.14830476,138.36373956)(236.16079469,138.36733203)(236.1502745,138.37200225)
\closepath
\moveto(241.57969238,138.37258603)
\curveto(241.56487457,138.37712154)(241.54038715,138.37712154)(241.52545743,138.37258603)
\curveto(241.51072916,138.36805053)(241.52277142,138.36454786)(241.55254133,138.36454786)
\curveto(241.582356,138.36454786)(241.59457733,138.36814034)(241.57962523,138.37258603)
\closepath
\moveto(241.68039077,138.37258603)
\curveto(241.66969151,138.37766041)(241.652322,138.37766041)(241.64162274,138.37258603)
\curveto(241.63092348,138.36746675)(241.63893673,138.36373956)(241.66105152,138.36373956)
\curveto(241.68240527,138.36373956)(241.69104526,138.36778109)(241.6804803,138.37258603)
\closepath
\moveto(241.78911584,138.37239294)
\curveto(241.77872995,138.37751222)(241.7577791,138.37751222)(241.74267479,138.37239294)
\curveto(241.72763315,138.36785743)(241.73613884,138.3638159)(241.76183497,138.36354646)
\curveto(241.7872625,138.36354646)(241.79975243,138.36713894)(241.78923224,138.37180916)
\closepath
\moveto(251.34322384,139.1697793)
\lineto(251.34322384,139.97871997)
\lineto(250.25108008,139.97871997)
\lineto(249.15893632,139.97871997)
\lineto(249.15893632,139.1697793)
\lineto(249.15893632,138.3608072)
\lineto(250.25108008,138.3608072)
\lineto(251.34322384,138.3608072)
\closepath
\moveto(236.30513621,138.39025653)
\curveto(236.29475032,138.39537581)(236.27370994,138.39537581)(236.25864144,138.39025653)
\curveto(236.2435998,138.38572103)(236.25210549,138.38167949)(236.27780162,138.38141006)
\curveto(236.30322915,138.38141006)(236.31571908,138.38500253)(236.30519889,138.38967275)
\closepath
\moveto(236.38972752,138.39025653)
\curveto(236.38462411,138.3960494)(236.37114931,138.39690262)(236.36000238,138.39295089)
\curveto(236.34769151,138.38711311)(236.35122809,138.38298176)(236.36931387,138.38217346)
\curveto(236.38569851,138.38217346)(236.3947414,138.38576593)(236.38954845,138.39155881)
\closepath
\moveto(236.49103474,138.39025653)
\curveto(236.48064885,138.39537581)(236.45960846,138.39537581)(236.44453549,138.39025653)
\curveto(236.42949385,138.38572103)(236.43799954,138.38167949)(236.46369567,138.38141006)
\curveto(236.4891232,138.38141006)(236.50161313,138.38500253)(236.49109294,138.38967275)
\closepath
\moveto(241.95952506,138.39025653)
\curveto(241.94913917,138.39537581)(241.92818832,138.39537581)(241.91308401,138.39025653)
\curveto(241.89804237,138.38572103)(241.90654806,138.38167949)(241.93224419,138.38141006)
\curveto(241.95767172,138.38141006)(241.97016165,138.38500253)(241.95964146,138.38967275)
\closepath
\moveto(242.05218782,138.39044963)
\curveto(242.04148856,138.39556891)(242.02411905,138.39556891)(242.01341979,138.39044963)
\curveto(242.00272053,138.38533035)(242.01073378,138.38160315)(242.03284857,138.38160315)
\curveto(242.05420232,138.38160315)(242.06284231,138.38564469)(242.05227735,138.39044963)
\closepath
\moveto(242.16837104,138.39044963)
\curveto(242.1533294,138.39498513)(242.12906581,138.39498513)(242.11413609,138.39044963)
\curveto(242.09931828,138.38591412)(242.11145008,138.38241146)(242.14121999,138.38241146)
\curveto(242.17103466,138.38241146)(242.18325599,138.38600394)(242.16830389,138.39044963)
\closepath
\moveto(236.65307257,138.4067505)
\curveto(236.64796916,138.41258827)(236.63449436,138.41339658)(236.6233922,138.40944485)
\curveto(236.61108133,138.4037867)(236.61461791,138.39947573)(236.63270369,138.39866742)
\curveto(236.6491331,138.39866742)(236.65813122,138.4022599)(236.65293827,138.40805277)
\closepath
\moveto(236.73859055,138.4067505)
\curveto(236.72793606,138.41186978)(236.71052178,138.41186978)(236.69991205,138.4067505)
\curveto(236.68921279,138.40163121)(236.69722605,138.39790402)(236.71934083,138.39790402)
\curveto(236.74069459,138.39790402)(236.74937934,138.40194556)(236.73876962,138.4067505)
\closepath
\moveto(236.84734248,138.40658883)
\curveto(236.83695659,138.41166321)(236.81591621,138.41166321)(236.80084323,138.40658883)
\curveto(236.78580159,138.40205333)(236.79430728,138.39801179)(236.82000341,138.39774236)
\curveto(236.84543094,138.39774236)(236.85792087,138.40133484)(236.84740068,138.40600506)
\closepath
\moveto(242.3313445,138.40658883)
\curveto(242.32095861,138.41166321)(242.29991822,138.41166321)(242.28484525,138.40658883)
\curveto(242.26980361,138.40205333)(242.2783093,138.39801179)(242.30400543,138.39774236)
\curveto(242.32943296,138.39774236)(242.34192289,138.40133484)(242.33140269,138.40600506)
\closepath
\moveto(242.42397592,138.4067505)
\curveto(242.41327666,138.41186978)(242.39590715,138.41186978)(242.38525265,138.4067505)
\curveto(242.37459816,138.40163121)(242.38256665,138.39790402)(242.40468143,138.39790402)
\curveto(242.42603519,138.39790402)(242.43467518,138.40194556)(242.42411022,138.4067505)
\closepath
\moveto(242.53272338,138.40658883)
\curveto(242.52233748,138.41166321)(242.5012971,138.41166321)(242.4862286,138.40658883)
\curveto(242.47118696,138.40205333)(242.47969265,138.39801179)(242.50538878,138.39774236)
\curveto(242.53108491,138.39774236)(242.54330624,138.40133484)(242.53278605,138.40600506)
\closepath
\moveto(237.00224002,138.42428179)
\curveto(236.99185412,138.42940107)(236.9708585,138.42940107)(236.95579896,138.42428179)
\curveto(236.94075732,138.41974628)(236.94926301,138.41570475)(236.97495914,138.41543531)
\curveto(237.00038667,138.41543531)(237.0128766,138.41902779)(237.00235641,138.42369801)
\closepath
\moveto(237.09489829,138.42447937)
\curveto(237.08419903,138.42959865)(237.06682952,138.42959865)(237.05617503,138.42447937)
\curveto(237.04547577,138.41936009)(237.05348902,138.4156329)(237.07560381,138.4156329)
\curveto(237.09695757,138.4156329)(237.10559755,138.41967443)(237.09503259,138.42447937)
\closepath
\moveto(237.20364575,138.42428179)
\curveto(237.19325986,138.42940107)(237.17221947,138.42940107)(237.1571465,138.42428179)
\curveto(237.14210486,138.41974628)(237.15061055,138.41570475)(237.17630668,138.41543531)
\curveto(237.20173421,138.41543531)(237.21422414,138.41902779)(237.20370395,138.42369801)
\closepath
\moveto(242.69509248,138.42428179)
\curveto(242.68027468,138.42881729)(242.65578725,138.42881729)(242.64085305,138.42428179)
\curveto(242.62612478,138.41974628)(242.63816705,138.41628852)(242.66793695,138.41628852)
\curveto(242.69775163,138.41628852)(242.70997296,138.419881)(242.69502085,138.42428179)
\closepath
\moveto(242.88902664,138.42428179)
\curveto(242.87864075,138.42940107)(242.85760036,138.42940107)(242.84253187,138.42428179)
\curveto(242.82749023,138.41974628)(242.83599592,138.41570475)(242.86169205,138.41543531)
\curveto(242.88711958,138.41543531)(242.89960951,138.41902779)(242.88908932,138.42369801)
\closepath
\moveto(237.35854776,138.44197474)
\curveto(237.34816187,138.44704911)(237.32721102,138.44704911)(237.3121067,138.44197474)
\curveto(237.29706507,138.43743924)(237.30557075,138.4333977)(237.33126689,138.43312826)
\curveto(237.35669442,138.43312826)(237.36918435,138.43672074)(237.35866415,138.44139096)
\closepath
\moveto(237.45120604,138.44216784)
\curveto(237.44050678,138.44728712)(237.42313727,138.44728712)(237.41248278,138.44216784)
\curveto(237.40182828,138.43704855)(237.40979677,138.43332136)(237.43191156,138.43332136)
\curveto(237.45326531,138.43332136)(237.4619053,138.4373629)(237.45134034,138.44216784)
\closepath
\moveto(237.56739374,138.44216784)
\curveto(237.55257593,138.44670334)(237.5280885,138.44670334)(237.51315431,138.44216784)
\curveto(237.4983365,138.43763233)(237.5104683,138.43412967)(237.54023821,138.43412967)
\curveto(237.57005288,138.43412967)(237.58227421,138.43772214)(237.56732211,138.44216784)
\closepath
\moveto(243.04392865,138.44216784)
\curveto(243.03354276,138.44728712)(243.01259191,138.44728712)(242.99748759,138.44216784)
\curveto(242.98244596,138.43763233)(242.99095164,138.43359079)(243.01664778,138.43332136)
\curveto(243.04207531,138.43332136)(243.05456524,138.43691384)(243.04404504,138.44158406)
\closepath
\moveto(243.13659141,138.4423295)
\curveto(243.12589215,138.44744878)(243.10852264,138.44744878)(243.09791291,138.4423295)
\curveto(243.08721365,138.43721022)(243.0952269,138.43348302)(243.11734169,138.43348302)
\curveto(243.13869544,138.43348302)(243.1473802,138.43752456)(243.13677047,138.4423295)
\closepath
\moveto(243.24533439,138.44216784)
\curveto(243.2349485,138.44728712)(243.21390811,138.44728712)(243.19883961,138.44216784)
\curveto(243.18379797,138.43763233)(243.19230366,138.43359079)(243.21799979,138.43332136)
\curveto(243.24342732,138.43332136)(243.25591725,138.43691384)(243.24539706,138.44158406)
\closepath
\moveto(237.73036272,138.45986079)
\curveto(237.71997682,138.46498007)(237.69893644,138.46498007)(237.68386794,138.45986079)
\curveto(237.6688263,138.45532529)(237.67733199,138.45128375)(237.70302812,138.45101431)
\curveto(237.72845565,138.45101431)(237.74094558,138.45460679)(237.73042539,138.45927701)
\closepath
\moveto(237.82299414,138.46003143)
\curveto(237.81229488,138.46515071)(237.79492537,138.46515071)(237.78427087,138.46003143)
\curveto(237.77361638,138.45491215)(237.78158487,138.45118495)(237.80369965,138.45118495)
\curveto(237.82505341,138.45118495)(237.8336934,138.45522649)(237.82312844,138.46003143)
\closepath
\moveto(237.93918631,138.46003143)
\curveto(237.92436851,138.46456693)(237.89988108,138.46456693)(237.88494688,138.46003143)
\curveto(237.87012908,138.45549593)(237.88226087,138.45176873)(237.91203078,138.45176873)
\curveto(237.94184546,138.45176873)(237.95406679,138.45536121)(237.93911468,138.46003143)
\closepath
\moveto(243.40023192,138.46003143)
\curveto(243.38984603,138.46515071)(243.36889517,138.46515071)(243.35379086,138.46003143)
\curveto(243.33874923,138.45549593)(243.34725491,138.45145439)(243.37295104,138.45118495)
\curveto(243.39837857,138.45118495)(243.41086851,138.45477743)(243.40034831,138.45944765)
\closepath
\moveto(243.48482322,138.46003143)
\curveto(243.47971981,138.4658243)(243.46624501,138.46667752)(243.45518762,138.46272579)
\curveto(243.44287675,138.45693292)(243.44641333,138.45275666)(243.46449911,138.45194836)
\curveto(243.48092851,138.45194836)(243.48992664,138.45554083)(243.48473369,138.4613337)
\closepath
\moveto(243.5861573,138.46003143)
\curveto(243.57577141,138.46515071)(243.55473103,138.46515071)(243.53966253,138.46003143)
\curveto(243.52462089,138.45549593)(243.53312658,138.45145439)(243.55882271,138.45118495)
\curveto(243.58425024,138.45118495)(243.59674017,138.45477743)(243.58621998,138.45944765)
\closepath
\moveto(238.10959553,138.47826326)
\curveto(238.09477773,138.48279876)(238.0702903,138.48279876)(238.05536058,138.47826326)
\curveto(238.04054277,138.47372775)(238.05267457,138.47026999)(238.08244448,138.47026999)
\curveto(238.11225915,138.47026999)(238.12448048,138.47386247)(238.10952838,138.47826326)
\closepath
\moveto(238.20222695,138.4755689)
\curveto(238.1971683,138.48136177)(238.18364873,138.48221498)(238.17259134,138.47826326)
\curveto(238.16028048,138.47247039)(238.16381705,138.46829413)(238.18190283,138.46748582)
\curveto(238.19833224,138.46748582)(238.20733036,138.4710783)(238.20213741,138.47687117)
\closepath
\moveto(238.31100127,138.47826326)
\curveto(238.296273,138.48279876)(238.27169603,138.48279876)(238.25676631,138.47826326)
\curveto(238.24172468,138.47372775)(238.25408031,138.47026999)(238.28385022,138.47026999)
\curveto(238.31366489,138.47026999)(238.32588622,138.47386247)(238.31093412,138.47826326)
\closepath
\moveto(243.74105483,138.47826326)
\curveto(243.73066894,138.48333763)(243.70962856,138.48333763)(243.69455558,138.47826326)
\curveto(243.67951394,138.47372775)(243.68801963,138.46968622)(243.71371576,138.46941678)
\curveto(243.73941189,138.46941678)(243.75163322,138.47300926)(243.74111303,138.47767948)
\closepath
\moveto(243.82564614,138.47826326)
\curveto(243.82054273,138.48405613)(243.80706793,138.48490934)(243.79601053,138.48095761)
\curveto(243.78369967,138.47516474)(243.78723624,138.47098849)(243.80532202,138.47018018)
\curveto(243.82175143,138.47018018)(243.83074955,138.47377266)(243.82555661,138.47956553)
\closepath
\moveto(243.92694888,138.47826326)
\curveto(243.91656299,138.48333763)(243.8955226,138.48333763)(243.88045411,138.47826326)
\curveto(243.86541247,138.47372775)(243.87391816,138.46968622)(243.89961429,138.46941678)
\curveto(243.92504182,138.46941678)(243.93753175,138.47300926)(243.92701156,138.47767948)
\closepath
\moveto(238.48942822,138.49595621)
\curveto(238.47904232,138.50107549)(238.45800194,138.50107549)(238.44299164,138.49595621)
\curveto(238.42795,138.49142071)(238.43645569,138.48737917)(238.46215182,138.48710973)
\curveto(238.48757935,138.48710973)(238.50006928,138.49070221)(238.48954909,138.49537243)
\closepath
\moveto(238.58209097,138.49612236)
\curveto(238.57143648,138.50124164)(238.5540222,138.50124164)(238.54332294,138.49612236)
\curveto(238.53266845,138.49100308)(238.54063693,138.48727588)(238.56275172,138.48727588)
\curveto(238.58410548,138.48727588)(238.59274546,138.49131742)(238.5821805,138.49612236)
\closepath
\moveto(238.66697774,138.49612236)
\curveto(238.66187432,138.50191523)(238.64839952,138.50276844)(238.6372526,138.49881672)
\curveto(238.62494173,138.49302385)(238.62847831,138.48884759)(238.64656409,138.48803929)
\curveto(238.66299349,138.48803929)(238.67199162,138.49163176)(238.66679867,138.49742463)
\closepath
\moveto(244.06638845,138.49612236)
\curveto(244.05600255,138.50124164)(244.03496217,138.50124164)(244.01989367,138.49612236)
\curveto(244.00485203,138.49158686)(244.01335772,138.48754532)(244.03905385,138.48727588)
\curveto(244.06448138,138.48727588)(244.07697131,138.49086836)(244.06645112,138.49553858)
\closepath
\moveto(244.15095289,138.49612236)
\curveto(244.14584948,138.50191523)(244.13237468,138.50276844)(244.12131728,138.49881672)
\curveto(244.10900642,138.49302385)(244.11254299,138.48884759)(244.13062877,138.48803929)
\curveto(244.14705818,138.48803929)(244.1560563,138.49163176)(244.15086336,138.49742463)
\closepath
\moveto(244.2365022,138.49612236)
\curveto(244.22584771,138.50124164)(244.20843343,138.50124164)(244.19773417,138.49612236)
\curveto(244.18703491,138.49100308)(244.19504817,138.48727588)(244.21716296,138.48727588)
\curveto(244.23851671,138.48727588)(244.2471567,138.49131742)(244.23659174,138.49612236)
\closepath
\moveto(238.86868789,138.51435419)
\curveto(238.85387008,138.51888969)(238.82938266,138.51888969)(238.81444846,138.51435419)
\curveto(238.79972019,138.50981868)(238.81176245,138.50631602)(238.84153236,138.50631602)
\curveto(238.87134704,138.50631602)(238.88356837,138.50990849)(238.86861626,138.51435419)
\closepath
\moveto(238.96939076,138.51435419)
\curveto(238.95873626,138.51947347)(238.94132199,138.51947347)(238.93062273,138.51435419)
\curveto(238.91996823,138.50927981)(238.92793672,138.50550771)(238.95005151,138.50550771)
\curveto(238.97140526,138.50550771)(238.98004525,138.50954925)(238.96948029,138.51435419)
\closepath
\moveto(239.08557845,138.51435419)
\curveto(239.07053682,138.51888969)(239.04627322,138.51888969)(239.03133903,138.51435419)
\curveto(239.01652122,138.50981868)(239.02865302,138.50631602)(239.05842293,138.50631602)
\curveto(239.0882376,138.50631602)(239.10045893,138.50990849)(239.08550683,138.51435419)
\closepath
\moveto(244.37591491,138.51435419)
\curveto(244.36521565,138.51947347)(244.34784614,138.51947347)(244.33714688,138.51435419)
\curveto(244.32649238,138.50927981)(244.33446087,138.50550771)(244.35657566,138.50550771)
\curveto(244.37792941,138.50550771)(244.3865694,138.50954925)(244.37600444,138.51435419)
\closepath
\moveto(244.46080615,138.51435419)
\curveto(244.45570274,138.52014706)(244.44222794,138.52100027)(244.43108101,138.51704854)
\curveto(244.41877014,138.51121077)(244.42230672,138.50707942)(244.4403925,138.50627111)
\curveto(244.45682191,138.50627111)(244.46582003,138.50986359)(244.46062708,138.51565646)
\closepath
\moveto(244.54632413,138.51435419)
\curveto(244.53566963,138.51947347)(244.51825536,138.51947347)(244.5075561,138.51435419)
\curveto(244.49685684,138.50927981)(244.50487009,138.50550771)(244.52698488,138.50550771)
\curveto(244.54833863,138.50550771)(244.55697862,138.50954925)(244.54641366,138.51435419)
\closepath
\moveto(239.25596081,138.53258601)
\curveto(239.24114301,138.53712151)(239.21665558,138.53712151)(239.20172586,138.53258601)
\curveto(239.18690806,138.52805051)(239.19903985,138.52432331)(239.22880976,138.52432331)
\curveto(239.25862444,138.52432331)(239.27084577,138.52791579)(239.25589366,138.53258601)
\closepath
\moveto(239.35666816,138.53258601)
\curveto(239.3459689,138.53770529)(239.32859939,138.53770529)(239.31790013,138.53258601)
\curveto(239.30720087,138.52746673)(239.31521412,138.52373953)(239.33732891,138.52373953)
\curveto(239.35868266,138.52373953)(239.36732265,138.52778107)(239.35675769,138.53258601)
\closepath
\moveto(244.69318154,138.53258601)
\curveto(244.68807813,138.53837888)(244.67460333,138.53923209)(244.6634564,138.53528037)
\curveto(244.65114554,138.52944259)(244.65468211,138.52531124)(244.67276789,138.52450294)
\curveto(244.6891973,138.52450294)(244.69819542,138.52809541)(244.69300248,138.53388828)
\closepath
\moveto(244.8406657,138.53258601)
\curveto(244.8300112,138.53770529)(244.81259693,138.53770529)(244.80189767,138.53258601)
\curveto(244.79124317,138.52746673)(244.79921166,138.52373953)(244.82132645,138.52373953)
\curveto(244.8426802,138.52373953)(244.85132019,138.52778107)(244.84075523,138.53258601)
\closepath
\moveto(239.6432606,138.55081783)
\curveto(239.6284428,138.55535334)(239.60395537,138.55535334)(239.58902565,138.55081783)
\curveto(239.57420784,138.54628233)(239.58633964,138.54277967)(239.61610955,138.54277967)
\curveto(239.64592422,138.54277967)(239.65814555,138.54637214)(239.64319345,138.55081783)
\closepath
\moveto(239.74396794,138.55081783)
\curveto(239.73326868,138.55593712)(239.71589917,138.55593712)(239.70519991,138.55081783)
\curveto(239.69450065,138.54569855)(239.70251391,138.54197136)(239.7246287,138.54197136)
\curveto(239.74598245,138.54197136)(239.75462244,138.5460129)(239.74405748,138.55081783)
\closepath
\moveto(239.8368993,138.55081783)
\curveto(239.82620004,138.55593712)(239.80883053,138.55593712)(239.79813127,138.55081783)
\curveto(239.78743201,138.54569855)(239.79544526,138.54197136)(239.81756005,138.54197136)
\curveto(239.83891381,138.54197136)(239.84755379,138.5460129)(239.83698883,138.55081783)
\closepath
\moveto(244.96459805,138.55081783)
\curveto(244.95389879,138.55593712)(244.93652928,138.55593712)(244.92583002,138.55081783)
\curveto(244.91513076,138.54569855)(244.92314401,138.54197136)(244.9452588,138.54197136)
\curveto(244.96661255,138.54197136)(244.97525254,138.5460129)(244.96468758,138.55081783)
\closepath
\moveto(245.04946243,138.55081783)
\curveto(245.04435902,138.55661071)(245.03088422,138.55746392)(245.01978205,138.55351219)
\curveto(245.00747119,138.54771932)(245.01100777,138.54354307)(245.02909354,138.54273476)
\curveto(245.04552295,138.54273476)(245.05452107,138.54632724)(245.04932813,138.55212011)
\closepath
\moveto(245.12693582,138.55081783)
\curveto(245.1218324,138.55661071)(245.1083576,138.55746392)(245.09721067,138.55351219)
\curveto(245.08489981,138.54771932)(245.08843639,138.54354307)(245.10652216,138.54273476)
\curveto(245.12290681,138.54273476)(245.13194969,138.54632724)(245.12675675,138.55212011)
\closepath
\moveto(240.030538,138.57021721)
\curveto(240.01549636,138.57475272)(239.99123277,138.57475272)(239.97629857,138.57021721)
\curveto(239.96148077,138.56568171)(239.97361257,138.56217905)(240.00338247,138.56217905)
\curveto(240.03319715,138.56217905)(240.04541848,138.56577152)(240.03046637,138.57021721)
\closepath
\moveto(240.13124087,138.57021721)
\curveto(240.12054161,138.5753365)(240.1031721,138.5753365)(240.09247284,138.57021721)
\curveto(240.08177358,138.56509793)(240.08978683,138.56137074)(240.11190162,138.56137074)
\curveto(240.13325538,138.56137074)(240.14189536,138.56541228)(240.1313304,138.57021721)
\closepath
\moveto(240.24742857,138.57021721)
\curveto(240.2327003,138.57475272)(240.20812334,138.57475272)(240.19318914,138.57021721)
\curveto(240.17837133,138.56568171)(240.19050313,138.56217905)(240.22027304,138.56217905)
\curveto(240.25008771,138.56217905)(240.26230904,138.56577152)(240.24735694,138.57021721)
\closepath
\moveto(245.39835232,138.57021721)
\curveto(245.38765306,138.5753365)(245.37028355,138.5753365)(245.35958429,138.57021721)
\curveto(245.34888503,138.56509793)(245.35689828,138.56137074)(245.37901307,138.56137074)
\curveto(245.40036682,138.56137074)(245.40900681,138.56541228)(245.39844185,138.57021721)
\closepath
\moveto(240.41783779,138.58844904)
\curveto(240.40310952,138.59298454)(240.37853256,138.59298454)(240.36360284,138.58844904)
\curveto(240.34878503,138.58391354)(240.36091683,138.58041087)(240.39068674,138.58041087)
\curveto(240.42050141,138.58041087)(240.43272274,138.58400335)(240.41777064,138.58844904)
\closepath
\moveto(240.5185138,138.58844904)
\curveto(240.5078593,138.59356832)(240.49044503,138.59356832)(240.4798353,138.58844904)
\curveto(240.46913604,138.58332976)(240.47714929,138.57960256)(240.49926408,138.57960256)
\curveto(240.52061783,138.57960256)(240.52925782,138.5836441)(240.51869286,138.58844904)
\closepath
\moveto(240.63470597,138.58844904)
\curveto(240.61966433,138.59298454)(240.59544551,138.59298454)(240.58052474,138.58844904)
\curveto(240.5654831,138.58391354)(240.57783873,138.58041087)(240.60760864,138.58041087)
\curveto(240.63742331,138.58041087)(240.64964464,138.58400335)(240.63469254,138.58844904)
\closepath
\moveto(245.51421322,138.58575468)
\curveto(245.50915457,138.59154755)(245.495635,138.59240076)(245.48457761,138.58844904)
\curveto(245.47226675,138.58261126)(245.47580332,138.57847991)(245.4938891,138.57767161)
\curveto(245.51027374,138.57767161)(245.51931663,138.58126408)(245.51412368,138.58705695)
\closepath
\moveto(245.65365278,138.58575468)
\curveto(245.64854937,138.59154755)(245.63507457,138.59240076)(245.62392764,138.58844904)
\curveto(245.61161677,138.58261126)(245.61515335,138.57847991)(245.63323913,138.57767161)
\curveto(245.64962377,138.57767161)(245.65866666,138.58126408)(245.65347371,138.58705695)
\closepath
\moveto(240.80511967,138.60515406)
\curveto(240.79030186,138.60968956)(240.76581443,138.60968956)(240.75088024,138.60515406)
\curveto(240.7358386,138.60061856)(240.74819423,138.59711589)(240.77791937,138.59711589)
\curveto(240.80773405,138.59711589)(240.81995538,138.60070837)(240.80500327,138.60515406)
\closepath
\moveto(241.02201023,138.60515406)
\curveto(241.00696859,138.60968956)(240.982705,138.60968956)(240.9677708,138.60515406)
\curveto(240.952953,138.60061856)(240.9650848,138.59711589)(240.9948547,138.59711589)
\curveto(241.02466938,138.59711589)(241.03689071,138.60070837)(241.02193861,138.60515406)
\closepath
\moveto(245.76207344,138.6024597)
\curveto(245.75697003,138.60825257)(245.74349523,138.60910579)(245.73239307,138.60515406)
\curveto(245.7200822,138.59931628)(245.72361878,138.59518494)(245.74170456,138.59437663)
\curveto(245.75813396,138.59437663)(245.76713209,138.59796911)(245.76193914,138.60376198)
\closepath
\moveto(241.19241945,138.62176927)
\curveto(241.17760165,138.62630477)(241.15311422,138.62630477)(241.1381845,138.62176927)
\curveto(241.1233667,138.61723377)(241.13549849,138.6137311)(241.1652684,138.6137311)
\curveto(241.19508308,138.6137311)(241.20730441,138.61732358)(241.1923523,138.62176927)
\closepath
\moveto(241.29309098,138.62176927)
\curveto(241.28239172,138.62688855)(241.26502221,138.62688855)(241.25436772,138.62176927)
\curveto(241.24371323,138.61664999)(241.25168171,138.61292279)(241.2737965,138.61292279)
\curveto(241.29515026,138.61292279)(241.30379024,138.61696433)(241.29322528,138.62176927)
\closepath
\moveto(241.40184292,138.62157617)
\curveto(241.39145703,138.62669546)(241.37041664,138.62669546)(241.35534367,138.62157617)
\curveto(241.34030203,138.61704067)(241.34880772,138.61299913)(241.37450385,138.6127297)
\curveto(241.39993138,138.6127297)(241.41242131,138.61632218)(241.40190112,138.6209924)
\closepath
\moveto(245.99444884,138.62157617)
\curveto(245.98934542,138.62736905)(245.97587062,138.62822226)(245.96476846,138.62427053)
\curveto(245.9524576,138.61847766)(245.95599417,138.61430141)(245.97407995,138.6134931)
\curveto(245.99046459,138.6134931)(245.99950748,138.61708558)(245.99431454,138.62287845)
\closepath
\moveto(241.57969238,138.64097555)
\curveto(241.56487457,138.64551106)(241.54038715,138.64551106)(241.52545743,138.64097555)
\curveto(241.51072916,138.63644005)(241.52277142,138.63293739)(241.55254133,138.63293739)
\curveto(241.582356,138.63293739)(241.59457733,138.63652986)(241.57962523,138.64097555)
\closepath
\moveto(241.78109811,138.64097555)
\curveto(241.76628031,138.64551106)(241.74179288,138.64551106)(241.72686316,138.64097555)
\curveto(241.71204536,138.63644005)(241.72417716,138.63293739)(241.75394706,138.63293739)
\curveto(241.78376174,138.63293739)(241.79598307,138.63652986)(241.7809862,138.64097555)
\closepath
\moveto(241.95148047,138.65866851)
\curveto(241.93666267,138.66320401)(241.91222001,138.66320401)(241.89729924,138.65866851)
\curveto(241.88257097,138.654133)(241.89461324,138.65040581)(241.92438314,138.65040581)
\curveto(241.95419782,138.65040581)(241.96641915,138.65399829)(241.95146704,138.65866851)
\closepath
\moveto(242.04411189,138.65597415)
\curveto(242.03900848,138.66176702)(242.02553368,138.66262023)(242.01447629,138.65866851)
\curveto(242.00216542,138.65287564)(242.005702,138.64869938)(242.02378778,138.64789107)
\curveto(242.04021718,138.64789107)(242.04921531,138.65148355)(242.04402236,138.65727642)
\closepath
\moveto(242.15289069,138.65866851)
\curveto(242.13807288,138.66320401)(242.11358546,138.66320401)(242.09865574,138.65866851)
\curveto(242.08383793,138.654133)(242.09596973,138.65040581)(242.12569487,138.65040581)
\curveto(242.15550954,138.65040581)(242.16773088,138.65399829)(242.15277877,138.65866851)
\closepath
\moveto(242.31583281,138.67577768)
\curveto(242.30544692,138.68089696)(242.28440653,138.68089696)(242.26939175,138.67577768)
\curveto(242.25435012,138.67124218)(242.2628558,138.66720064)(242.28855193,138.66693121)
\curveto(242.31397946,138.66693121)(242.3264694,138.67052368)(242.3159492,138.67519391)
\closepath
\moveto(242.41590894,138.67577768)
\curveto(242.41080553,138.68161546)(242.39733073,138.68242377)(242.38622857,138.67847204)
\curveto(242.3739177,138.67267917)(242.37745428,138.66850292)(242.39554006,138.66769461)
\curveto(242.41196947,138.66769461)(242.42096759,138.67128709)(242.41577464,138.67712486)
\closepath
\moveto(242.52467879,138.67847204)
\curveto(242.50963715,138.68300754)(242.48537355,138.68300754)(242.47044383,138.67847204)
\curveto(242.45571556,138.67393654)(242.46775783,138.67043387)(242.49752773,138.67043387)
\curveto(242.52734241,138.67043387)(242.53956374,138.67402635)(242.52461163,138.67847204)
\closepath
\moveto(242.68765224,138.69558122)
\curveto(242.67726635,138.7007005)(242.65622596,138.7007005)(242.64115299,138.69558122)
\curveto(242.62611135,138.69104571)(242.63461704,138.68700418)(242.66031317,138.68673474)
\curveto(242.6857407,138.68673474)(242.69823063,138.69032722)(242.68771044,138.69499744)
\closepath
\moveto(242.78028366,138.69574288)
\curveto(242.7695844,138.70086216)(242.75221489,138.70086216)(242.7415604,138.69574288)
\curveto(242.7309059,138.6906236)(242.73887439,138.6868964)(242.76098918,138.6868964)
\curveto(242.78234293,138.6868964)(242.79102769,138.69093794)(242.78041796,138.69574288)
\closepath
\moveto(242.88902664,138.69558122)
\curveto(242.87864075,138.7007005)(242.85760036,138.7007005)(242.84253187,138.69558122)
\curveto(242.82749023,138.69104571)(242.83599592,138.68700418)(242.86169205,138.68673474)
\curveto(242.88711958,138.68673474)(242.89960951,138.69032722)(242.88908932,138.69499744)
\closepath
\moveto(243.13659141,138.71354361)
\curveto(243.12589215,138.71866289)(243.10852264,138.71866289)(243.09791291,138.71354361)
\curveto(243.08721365,138.70842432)(243.0952269,138.70469713)(243.11734169,138.70469713)
\curveto(243.13869544,138.70469713)(243.1473802,138.70873867)(243.13677047,138.71354361)
\closepath
\moveto(243.24533439,138.71338194)
\curveto(243.2349485,138.71845632)(243.21390811,138.71845632)(243.19883961,138.71338194)
\curveto(243.18379797,138.70884644)(243.19230366,138.7048049)(243.21799979,138.70453547)
\curveto(243.24342732,138.70453547)(243.25591725,138.70812795)(243.24539706,138.71279817)
\closepath
\moveto(243.38444715,138.73134433)
\curveto(243.37374789,138.73646361)(243.35637838,138.73646361)(243.34572389,138.73134433)
\curveto(243.3350694,138.72622505)(243.34303788,138.72222842)(243.36515267,138.72222842)
\curveto(243.38650642,138.72222842)(243.39514641,138.72626996)(243.38458145,138.73134433)
\closepath
\moveto(243.47740537,138.73134433)
\curveto(243.46670611,138.73646361)(243.4493366,138.73646361)(243.43863734,138.73134433)
\curveto(243.42793808,138.72622505)(243.43595133,138.72222842)(243.45806612,138.72222842)
\curveto(243.47941987,138.72222842)(243.48805986,138.72626996)(243.4774949,138.73134433)
\closepath
\moveto(243.5861573,138.73115124)
\curveto(243.57577141,138.73627052)(243.55473103,138.73627052)(243.53966253,138.73115124)
\curveto(243.52462089,138.72661573)(243.53312658,138.7226191)(243.55882271,138.72230476)
\curveto(243.58425024,138.72230476)(243.59674017,138.72589724)(243.58621998,138.73056746)
\closepath
\moveto(243.73268345,138.74803588)
\curveto(243.72758003,138.75382875)(243.71410523,138.75468197)(243.70300307,138.75073024)
\curveto(243.69069221,138.74489246)(243.69422878,138.74076112)(243.71231456,138.73995281)
\curveto(243.72874397,138.73995281)(243.73774209,138.74354529)(243.73254915,138.74933816)
\closepath
\moveto(243.81016131,138.74803588)
\curveto(243.8050579,138.75382875)(243.7915831,138.75468197)(243.78043617,138.75073024)
\curveto(243.76812531,138.74489246)(243.77166188,138.74076112)(243.78974766,138.73995281)
\curveto(243.80617707,138.73995281)(243.81517519,138.74354529)(243.80998224,138.74933816)
\closepath
\moveto(243.91146405,138.74803588)
\curveto(243.90107816,138.75315516)(243.88003778,138.75315516)(243.86496928,138.74803588)
\curveto(243.84992764,138.74350038)(243.85843333,138.73950375)(243.88412946,138.73918941)
\curveto(243.90955699,138.73918941)(243.92204692,138.74278188)(243.91152673,138.74745211)
\closepath
\moveto(244.05060368,138.76599827)
\curveto(244.03990442,138.77111755)(244.02253491,138.77111755)(244.01183565,138.76599827)
\curveto(244.00113639,138.76087899)(244.00914964,138.75715179)(244.03126443,138.75715179)
\curveto(244.05261818,138.75715179)(244.06125817,138.76119333)(244.05069321,138.76599827)
\closepath
\moveto(244.13546806,138.76599827)
\curveto(244.13036465,138.77179114)(244.11688985,138.77264436)(244.10583245,138.76869263)
\curveto(244.09352159,138.76289976)(244.09705817,138.7587235)(244.11514394,138.7579152)
\curveto(244.13157335,138.7579152)(244.14057147,138.76150767)(244.13537853,138.76730054)
\closepath
\moveto(244.23677528,138.76599827)
\curveto(244.22638939,138.77111755)(244.20539377,138.77111755)(244.19033423,138.76599827)
\curveto(244.17529259,138.76146277)(244.18379828,138.75742123)(244.20949441,138.75715179)
\curveto(244.23492194,138.75715179)(244.24741187,138.76074427)(244.23689168,138.76541449)
\closepath
\moveto(244.37621932,138.78369122)
\curveto(244.36583343,138.78881051)(244.34479304,138.78881051)(244.32972007,138.78369122)
\curveto(244.31467843,138.77915572)(244.32318412,138.77515909)(244.34888025,138.77484475)
\curveto(244.37430778,138.77484475)(244.38679771,138.77843723)(244.37627752,138.78310745)
\closepath
\moveto(244.46080615,138.78369122)
\curveto(244.45570274,138.789529)(244.44222794,138.79033731)(244.43108101,138.78638558)
\curveto(244.41877014,138.78054781)(244.42230672,138.77641646)(244.4403925,138.77560815)
\curveto(244.45682191,138.77560815)(244.46582003,138.77920063)(244.46062708,138.7849935)
\closepath
\moveto(244.54632413,138.78369122)
\curveto(244.53566963,138.78881051)(244.51825536,138.78881051)(244.5075561,138.78369122)
\curveto(244.49685684,138.77857194)(244.50487009,138.77484475)(244.52698488,138.77484475)
\curveto(244.54833863,138.77484475)(244.55697862,138.77888629)(244.54641366,138.78369122)
\closepath
\moveto(244.67025648,138.80138418)
\curveto(244.65955722,138.80650346)(244.64218771,138.80650346)(244.63148845,138.80138418)
\curveto(244.62078919,138.7962649)(244.62880244,138.7925377)(244.65091723,138.7925377)
\curveto(244.67227098,138.7925377)(244.68091097,138.79657924)(244.67034601,138.80138418)
\closepath
\moveto(244.75514772,138.80138418)
\curveto(244.75004431,138.80717705)(244.73656951,138.80803026)(244.72542258,138.80407854)
\curveto(244.71311171,138.79828567)(244.71664829,138.79410941)(244.73473407,138.7933011)
\curveto(244.75116348,138.7933011)(244.7601616,138.79689358)(244.75496865,138.80268645)
\closepath
\moveto(244.8406657,138.80138418)
\curveto(244.8300112,138.80650346)(244.81259693,138.80650346)(244.80189767,138.80138418)
\curveto(244.79124317,138.7962649)(244.79921166,138.7925377)(244.82132645,138.7925377)
\curveto(244.8426802,138.7925377)(244.85132019,138.79657924)(244.84075523,138.80138418)
\closepath
\moveto(244.96459805,138.81907713)
\curveto(244.95389879,138.82419641)(244.93652928,138.82419641)(244.92583002,138.81907713)
\curveto(244.91513076,138.81400276)(244.92314401,138.81023065)(244.9452588,138.81023065)
\curveto(244.96661255,138.81023065)(244.97525254,138.81427219)(244.96468758,138.81907713)
\closepath
\moveto(245.04946243,138.81907713)
\curveto(245.04435902,138.82487)(245.03088422,138.82572321)(245.01978205,138.82177149)
\curveto(245.00747119,138.81597862)(245.01100777,138.81180236)(245.02909354,138.81099406)
\curveto(245.04552295,138.81099406)(245.05452107,138.81458653)(245.04932813,138.8203794)
\closepath
\moveto(245.12693582,138.81907713)
\curveto(245.1218324,138.82487)(245.1083576,138.82572321)(245.09721067,138.82177149)
\curveto(245.08489981,138.81597862)(245.08843639,138.81180236)(245.10652216,138.81099406)
\curveto(245.12290681,138.81099406)(245.13194969,138.81458653)(245.12675675,138.8203794)
\closepath
\moveto(245.24345479,138.83789273)
\curveto(245.23275553,138.84301201)(245.21538602,138.84301201)(245.20468676,138.83789273)
\curveto(245.1939875,138.83277345)(245.20200075,138.82904626)(245.22411554,138.82904626)
\curveto(245.24546929,138.82904626)(245.25410928,138.83308779)(245.24354432,138.83789273)
\closepath
\moveto(245.39835232,138.83789273)
\curveto(245.38765306,138.84301201)(245.37028355,138.84301201)(245.35958429,138.83789273)
\curveto(245.34888503,138.83277345)(245.35689828,138.82904626)(245.37901307,138.82904626)
\curveto(245.40036682,138.82904626)(245.40900681,138.83308779)(245.39844185,138.83789273)
\closepath
\moveto(245.49872839,138.85477738)
\curveto(245.49362498,138.86061515)(245.48015018,138.86142346)(245.46909278,138.85747174)
\curveto(245.45678192,138.85167887)(245.46031849,138.84750261)(245.47840427,138.8466943)
\curveto(245.49483368,138.8466943)(245.5038318,138.85028678)(245.49863886,138.85607965)
\closepath
\moveto(245.57617939,138.85477738)
\curveto(245.57107598,138.86061515)(245.55760118,138.86142346)(245.54654379,138.85747174)
\curveto(245.53423292,138.85167887)(245.5377695,138.84750261)(245.55585528,138.8466943)
\curveto(245.57223992,138.8466943)(245.58128281,138.85028678)(245.57608986,138.85607965)
\closepath
\moveto(245.65365278,138.85477738)
\curveto(245.64854937,138.86061515)(245.63507457,138.86142346)(245.62392764,138.85747174)
\curveto(245.61161677,138.85167887)(245.61515335,138.84750261)(245.63323913,138.8466943)
\curveto(245.64962377,138.8466943)(245.65866666,138.85028678)(245.65347371,138.85607965)
\closepath
\moveto(245.75466006,138.87359298)
\curveto(245.7439608,138.87871226)(245.72659129,138.87871226)(245.71589203,138.87359298)
\curveto(245.70519277,138.8684737)(245.71320603,138.8647465)(245.73532081,138.8647465)
\curveto(245.75667457,138.8647465)(245.76531456,138.86878804)(245.7547496,138.87359298)
\closepath
\moveto(245.97896401,138.89047763)
\curveto(245.97386059,138.8962705)(245.96038579,138.89712371)(245.94928363,138.89317198)
\curveto(245.93697277,138.88733421)(245.94050934,138.88320286)(245.95859512,138.88239455)
\curveto(245.97497976,138.88239455)(245.98402265,138.88598703)(245.97882971,138.8917799)
\closepath
\moveto(246.17681526,138.91535553)
\curveto(246.15984865,138.92361823)(246.12489475,138.92999488)(246.09934188,138.92945601)
\curveto(246.05397522,138.92945601)(246.05414086,138.92676165)(246.10704176,138.91535553)
\curveto(246.18309604,138.89712371)(246.21401198,138.89712371)(246.17677945,138.91535553)
\closepath
\moveto(246.35496466,138.91535553)
\lineto(246.39368792,138.92945601)
\lineto(246.35496466,138.93215037)
\curveto(246.3336109,138.93215037)(246.30575701,138.92581863)(246.29300744,138.91638837)
\curveto(246.27250425,138.90121015)(246.27250425,138.8992792)(246.29300744,138.90062638)
\curveto(246.30590027,138.90062638)(246.33370044,138.90866454)(246.35496466,138.91638837)
\closepath
\moveto(246.44790049,138.95087616)
\curveto(246.44790049,138.96277624)(245.96886462,138.96717703)(245.0343044,138.96443776)
\lineto(243.6207083,138.95990226)
\lineto(245.02267846,138.95078635)
\curveto(245.79378175,138.94566707)(246.42990425,138.93973948)(246.43627456,138.93722474)
\curveto(246.44254191,138.93453038)(246.44800345,138.94081722)(246.44800345,138.95078635)
\closepath
\moveto(233.28397853,138.96412342)
\curveto(233.01773259,138.96681778)(232.58203544,138.96681778)(232.31576712,138.96412342)
\curveto(232.04952118,138.96142906)(232.26735185,138.95900414)(232.79988849,138.95900414)
\curveto(233.33240275,138.95900414)(233.55025133,138.9616985)(233.28397853,138.96412342)
\closepath
\moveto(235.56125561,138.96412342)
\curveto(235.24385019,138.96681778)(234.72449289,138.96681778)(234.40711881,138.96412342)
\curveto(234.08974026,138.96142906)(234.34942339,138.95900414)(234.98417602,138.95900414)
\curveto(235.61895551,138.95900414)(235.87863416,138.9616985)(235.56125561,138.96412342)
\closepath
\moveto(241.7281032,138.96412342)
\curveto(240.71275685,138.96681778)(239.0466224,138.96681778)(238.02563096,138.96412342)
\curveto(237.00464399,138.96142906)(237.83539006,138.95958792)(239.87175024,138.95958792)
\curveto(241.90811937,138.95958792)(242.74348537,138.96228228)(241.7281032,138.96412342)
\closepath
\moveto(247.16521649,138.96412342)
\curveto(247.07835999,138.96681778)(246.93196814,138.96681778)(246.83990974,138.96412342)
\curveto(246.74784686,138.96142906)(246.81882459,138.95833055)(246.99784694,138.95833055)
\curveto(247.1767887,138.95830361)(247.25210433,138.96102491)(247.16521649,138.96412342)
\closepath
\moveto(247.66785788,138.96412342)
\curveto(247.61035496,138.9677159)(247.51624176,138.9677159)(247.45874331,138.96412342)
\curveto(247.40120457,138.96053094)(247.44835741,138.95779168)(247.56330059,138.95779168)
\curveto(247.67831093,138.95779168)(247.72536529,138.96048604)(247.66785788,138.96412342)
\closepath
\moveto(248.78430105,138.96429406)
\curveto(248.60805872,138.96698842)(248.31527504,138.96698842)(248.13365174,138.96431652)
\curveto(247.95202396,138.96162216)(248.09618194,138.95919723)(248.45406548,138.95919723)
\curveto(248.81189977,138.95919723)(248.96050757,138.96189159)(248.78430105,138.96431652)
\closepath
\moveto(231.82001981,139.74530771)
\curveto(231.8173338,139.89015641)(231.81549837,139.77665208)(231.81549837,139.49304392)
\curveto(231.81549837,139.20948067)(231.81818437,139.0909783)(231.82001981,139.22971979)
\curveto(231.82270582,139.36845679)(231.82270582,139.60048595)(231.82004219,139.74530771)
\closepath
\moveto(236.13018764,144.30231186)
\lineto(236.13002648,149.60049357)
\lineto(233.36504221,149.60049357)
\lineto(230.60005794,149.60049357)
\lineto(230.60005794,144.83590258)
\lineto(230.60005794,140.07128016)
\lineto(231.21585184,140.06674466)
\lineto(231.83161888,140.06220915)
\lineto(231.83564789,139.53326171)
\lineto(231.8396769,139.0043457)
\lineto(233.98490094,139.0043457)
\lineto(236.13012049,139.0043457)
\lineto(236.12995933,144.30252741)
\closepath
\moveto(243.56648678,144.31115834)
\lineto(243.56648678,149.61823592)
\lineto(239.87949937,149.61823592)
\lineto(236.19253434,149.61823592)
\lineto(236.19253434,144.31115834)
\lineto(236.19253434,139.00407177)
\lineto(239.87949937,139.00407177)
\lineto(243.56648678,139.00407177)
\closepath
\moveto(239.69360084,139.57302146)
\curveto(239.26687495,139.6226695)(238.85968529,139.8429288)(238.55749163,140.18751029)
\curveto(238.28191629,140.50173534)(238.12847368,140.81535415)(238.03195203,141.26165665)
\curveto(238.00513673,141.38563305)(237.99900368,141.46962967)(237.99828741,141.72427348)
\curveto(237.99828741,141.9219361)(238.00482336,142.07433349)(238.01798479,142.1421011)
\curveto(238.10806898,142.60791972)(238.28195658,142.97650345)(238.55602775,143.28243437)
\curveto(238.8325029,143.5910776)(239.1300453,143.77419069)(239.50768889,143.8681699)
\curveto(239.67793247,143.91056114)(240.06499051,143.91168379)(240.23578473,143.87086426)
\curveto(240.84580371,143.72322241)(241.32158056,143.30227832)(241.58632234,142.67597922)
\curveto(241.70866547,142.38655574)(241.76945428,142.07185917)(241.76945428,141.72772226)
\curveto(241.76945428,141.05804399)(241.50925633,140.44919535)(241.03892312,140.0182956)
\curveto(240.78556106,139.78615418)(240.3981225,139.60721735)(240.09636755,139.58290975)
\curveto(240.02395281,139.57711688)(239.92984409,139.56966249)(239.88721716,139.56602511)
\curveto(239.84464395,139.56243263)(239.75749199,139.56602511)(239.69358294,139.5735244)
\closepath
\moveto(239.86790029,139.60252468)
\curveto(239.8572458,139.60764396)(239.83983152,139.60764396)(239.82913226,139.60252468)
\curveto(239.81847777,139.59745031)(239.82644626,139.5936782)(239.84856105,139.5936782)
\curveto(239.8699148,139.5936782)(239.87855479,139.59771974)(239.86798983,139.60252468)
\closepath
\moveto(239.67393032,139.61940933)
\curveto(239.66882691,139.6252471)(239.65535211,139.62605541)(239.64420518,139.62210368)
\curveto(239.63189431,139.61631081)(239.63543089,139.61213456)(239.65351667,139.61132625)
\curveto(239.66994608,139.61132625)(239.6789442,139.61491873)(239.67375125,139.6207116)
\closepath
\moveto(240.15320793,139.66229453)
\curveto(240.42126245,139.71222099)(240.75920686,139.87491083)(240.97166999,140.056313)
\curveto(241.59684701,140.59010131)(241.85890725,141.54164987)(241.61298543,142.38487625)
\curveto(241.43138452,143.00757389)(241.02709126,143.48810372)(240.50308716,143.7041239)
\curveto(239.89224447,143.95591617)(239.22085918,143.82293163)(238.71764478,143.35039058)
\curveto(238.58025553,143.22138021)(238.50386998,143.12757163)(238.39980513,142.96001847)
\curveto(238.13003156,142.52566097)(238.01107728,141.99577948)(238.06746104,141.47957635)
\curveto(238.12871542,140.91903306)(238.42117231,140.35281815)(238.81375457,140.03474915)
\curveto(239.02968713,139.85981345)(239.30695465,139.72593976)(239.56343248,139.67277109)
\curveto(239.70843208,139.642729)(240.01860766,139.63725047)(240.15320793,139.662308)
\closepath
\moveto(241.15755956,140.17760852)
\curveto(241.1728698,140.19723243)(241.18218129,140.21312914)(241.17779415,140.21312914)
\curveto(241.17376514,140.21312914)(241.15755956,140.19709771)(241.14202549,140.17760852)
\curveto(241.12671525,140.15798461)(241.11740376,140.14199808)(241.12179091,140.14199808)
\curveto(241.12581992,140.14199808)(241.14202549,140.15802952)(241.15755956,140.17760852)
\closepath
\moveto(241.71809335,141.40769536)
\curveto(241.71298994,141.42179583)(241.70931906,141.41766448)(241.70878186,141.39691792)
\curveto(241.70878186,141.37810232)(241.71146787,141.36768414)(241.71697418,141.37370154)
\curveto(241.72203283,141.37949441)(241.7227491,141.39498697)(241.71697418,141.40774026)
\closepath
\moveto(241.73367219,141.5321343)
\curveto(241.72856878,141.54623477)(241.7248979,141.54210342)(241.7243607,141.52135686)
\curveto(241.7243607,141.50254126)(241.72704671,141.49212308)(241.73255302,141.49814048)
\curveto(241.73765643,141.50393335)(241.73832793,141.51942591)(241.73255302,141.5321792)
\closepath
\moveto(238.03119995,141.94106155)
\curveto(238.02609653,141.95516203)(238.02242566,141.95103068)(238.02188846,141.93028412)
\curveto(238.02188846,141.91146851)(238.02457446,141.90105033)(238.03008078,141.90706773)
\curveto(238.03518419,141.91290551)(238.03585569,141.92835316)(238.03008078,141.94110646)
\closepath
\moveto(240.02289184,143.87043317)
\curveto(240.00373165,143.8744747)(239.97238148,143.8744747)(239.95318548,143.87043317)
\curveto(239.9340253,143.86639163)(239.94960414,143.86266443)(239.98805881,143.86266443)
\curveto(240.02642393,143.86266443)(240.04206992,143.86625691)(240.02293213,143.87043317)
\closepath
\moveto(249.09703729,139.53032486)
\curveto(249.09703729,140.04152097)(249.09703729,140.05672613)(249.12806067,140.0657702)
\curveto(249.15894975,140.07515554)(249.15908405,140.09010923)(249.15908405,144.83779761)
\lineto(249.15908405,149.60053848)
\lineto(246.39407292,149.60053848)
\lineto(243.62909312,149.60053848)
\lineto(243.62895434,144.30235677)
\lineto(243.62878871,139.00417506)
\lineto(246.36291524,139.00417506)
\lineto(249.09703281,139.00417506)
\lineto(249.09703281,139.53032486)
\closepath
\moveto(230.30972748,140.03087826)
\curveto(230.17977847,140.03357261)(229.96715418,140.03357261)(229.83723203,140.03087826)
\curveto(229.70728303,140.0281839)(229.81372947,140.0252201)(230.07346632,140.0252201)
\curveto(230.33333747,140.0252201)(230.43964514,140.02791446)(230.30972748,140.03087826)
\closepath
\moveto(231.48033397,140.03087826)
\curveto(231.32114331,140.03357261)(231.05623589,140.03357261)(230.89167769,140.0309052)
\curveto(230.72709711,140.02821084)(230.85734157,140.02578592)(231.18112178,140.02578592)
\curveto(231.50490198,140.02578592)(231.63955597,140.02848028)(231.48033397,140.0309052)
\closepath
\moveto(250.79838528,140.03087826)
\curveto(250.51142576,140.03357261)(250.03740377,140.03357261)(249.74494688,140.03087826)
\curveto(249.4525258,140.0281839)(249.68728279,140.02575898)(250.26665892,140.02575898)
\curveto(250.84603952,140.02575898)(251.085309,140.02845333)(250.79838528,140.03087826)
\closepath
\moveto(241.7269706,149.64944557)
\curveto(240.71090798,149.65213993)(239.0482743,149.65213993)(238.03223854,149.64944557)
\curveto(237.01618039,149.64675121)(237.84749499,149.64491007)(239.87959338,149.64491007)
\curveto(241.91168728,149.64491007)(242.74301084,149.64760443)(241.7269706,149.64944557)
\closepath
\moveto(251.94748134,151.77849608)
\lineto(251.94748134,153.88531364)
\lineto(239.87959338,153.88531364)
\lineto(227.81170094,153.88531364)
\lineto(227.81170094,151.77849608)
\lineto(227.81170094,149.67167852)
\lineto(239.87959338,149.67167852)
\lineto(251.94748134,149.67167852)
\closepath
\moveto(228.88450545,153.91641552)
\curveto(228.6864035,153.91910988)(228.36224278,153.91910988)(228.16414978,153.91641552)
\curveto(227.96604783,153.91372116)(228.12811253,153.91129624)(228.52431195,153.91129624)
\curveto(228.92050689,153.91129624)(229.08259844,153.9139906)(228.88450545,153.91641552)
\closepath
\moveto(251.59506383,153.91641552)
\curveto(251.39696636,153.91910988)(251.07280564,153.91910988)(250.87471264,153.91641552)
\curveto(250.67661069,153.91372116)(250.83867539,153.91129624)(251.23487481,153.91129624)
\curveto(251.63106527,153.91129624)(251.7931613,153.9139906)(251.59506383,153.91641552)
\closepath
\moveto(229.23692744,156.04549746)
\lineto(229.23692744,158.15231503)
\lineto(228.52431195,158.15231503)
\lineto(227.81170094,158.15231503)
\lineto(227.81170094,156.04549746)
\lineto(227.81170094,153.9386799)
\lineto(228.52431195,153.9386799)
\lineto(229.23692744,153.9386799)
\closepath
\moveto(251.94748134,156.04549746)
\lineto(251.94748134,158.15231503)
\lineto(251.23487481,158.15231503)
\lineto(250.52226379,158.15231503)
\lineto(250.52226379,156.04549746)
\lineto(250.52226379,153.9386799)
\lineto(251.23487481,153.9386799)
\lineto(251.94748134,153.9386799)
\closepath
\moveto(229.42281701,156.05434394)
\lineto(229.42281701,158.15225665)
\lineto(229.36085083,158.15225665)
\lineto(229.29889361,158.15225665)
\lineto(229.29889361,156.05434394)
\lineto(229.29889361,153.9563998)
\lineto(229.36085083,153.9563998)
\lineto(229.42281701,153.9563998)
\closepath
\moveto(250.27439909,156.05434394)
\lineto(250.27439909,158.15225665)
\lineto(239.87959338,158.15225665)
\lineto(229.48478318,158.15225665)
\lineto(229.48478318,156.05434394)
\lineto(229.48478318,153.9563998)
\lineto(239.87959338,153.9563998)
\lineto(250.27439909,153.9563998)
\closepath
\moveto(250.46029762,156.05434394)
\lineto(250.46029762,158.15225665)
\lineto(250.39833144,158.15225665)
\lineto(250.33636527,158.15225665)
\lineto(250.33636527,156.05434394)
\lineto(250.33636527,153.9563998)
\lineto(250.39833144,153.9563998)
\lineto(250.46029762,153.9563998)
\closepath
}
}
{
\newrgbcolor{curcolor}{0.68627453 0.7764706 0.9137255}
\pscustom[linestyle=none,fillstyle=solid,fillcolor=curcolor]
{
\newpath
\moveto(229.78001114,156.04403353)
\lineto(229.78001114,157.85329065)
\lineto(239.85554914,157.85329065)
\lineto(249.93108714,157.85329065)
\lineto(249.93108714,156.04403353)
\lineto(249.93108714,154.2347809)
\lineto(239.85554914,154.2347809)
\lineto(229.78001114,154.2347809)
\closepath
}
}
{
\newrgbcolor{curcolor}{0.53333336 0.70980394 0.85882354}
\pscustom[linewidth=0,linecolor=curcolor]
{
\newpath
\moveto(229.78001114,156.04403353)
\lineto(229.78001114,157.85329065)
\lineto(239.85554914,157.85329065)
\lineto(249.93108714,157.85329065)
\lineto(249.93108714,156.04403353)
\lineto(249.93108714,154.2347809)
\lineto(239.85554914,154.2347809)
\lineto(229.78001114,154.2347809)
\closepath
}
}
{
\newrgbcolor{curcolor}{0.68627453 0.7764706 0.9137255}
\pscustom[linestyle=none,fillstyle=solid,fillcolor=curcolor]
{
\newpath
\moveto(228.03855659,151.80164511)
\lineto(228.03855659,153.61090223)
\lineto(239.85554914,153.61090223)
\lineto(251.6725417,153.61090223)
\lineto(251.6725417,151.80164511)
\lineto(251.6725417,149.99239248)
\lineto(239.85554914,149.99239248)
\lineto(228.03855659,149.99239248)
\closepath
}
}
{
\newrgbcolor{curcolor}{0.53333336 0.70980394 0.85882354}
\pscustom[linewidth=0,linecolor=curcolor]
{
\newpath
\moveto(228.03855659,151.80164511)
\lineto(228.03855659,153.61090223)
\lineto(239.85554914,153.61090223)
\lineto(251.6725417,153.61090223)
\lineto(251.6725417,151.80164511)
\lineto(251.6725417,149.99239248)
\lineto(239.85554914,149.99239248)
\lineto(228.03855659,149.99239248)
\closepath
}
}
{
\newrgbcolor{curcolor}{0.68627453 0.7764706 0.9137255}
\pscustom[linestyle=none,fillstyle=solid,fillcolor=curcolor]
{
\newpath
\moveto(250.80181442,156.04403353)
\lineto(250.80181442,157.85329065)
\lineto(251.23717582,157.85329065)
\lineto(251.6725417,157.85329065)
\lineto(251.6725417,156.04403353)
\lineto(251.6725417,154.2347809)
\lineto(251.23717582,154.2347809)
\lineto(250.80181442,154.2347809)
\closepath
}
}
{
\newrgbcolor{curcolor}{0.53333336 0.70980394 0.85882354}
\pscustom[linewidth=0,linecolor=curcolor]
{
\newpath
\moveto(250.80181442,156.04403353)
\lineto(250.80181442,157.85329065)
\lineto(251.23717582,157.85329065)
\lineto(251.6725417,157.85329065)
\lineto(251.6725417,156.04403353)
\lineto(251.6725417,154.2347809)
\lineto(251.23717582,154.2347809)
\lineto(250.80181442,154.2347809)
\closepath
}
}
{
\newrgbcolor{curcolor}{0.68627453 0.7764706 0.9137255}
\pscustom[linestyle=none,fillstyle=solid,fillcolor=curcolor]
{
\newpath
\moveto(228.03855659,156.04403353)
\lineto(228.03855659,157.85329065)
\lineto(228.47392246,157.85329065)
\lineto(228.90928386,157.85329065)
\lineto(228.90928386,156.04403353)
\lineto(228.90928386,154.2347809)
\lineto(228.47392246,154.2347809)
\lineto(228.03855659,154.2347809)
\closepath
}
}
{
\newrgbcolor{curcolor}{0.53333336 0.70980394 0.85882354}
\pscustom[linewidth=0,linecolor=curcolor]
{
\newpath
\moveto(228.03855659,156.04403353)
\lineto(228.03855659,157.85329065)
\lineto(228.47392246,157.85329065)
\lineto(228.90928386,157.85329065)
\lineto(228.90928386,156.04403353)
\lineto(228.90928386,154.2347809)
\lineto(228.47392246,154.2347809)
\lineto(228.03855659,154.2347809)
\closepath
}
}
{
\newrgbcolor{curcolor}{0.68627453 0.7764706 0.9137255}
\pscustom[linestyle=none,fillstyle=solid,fillcolor=curcolor]
{
\newpath
\moveto(232.14340607,139.67340427)
\curveto(232.14340607,140.20863855)(231.95257424,140.38462954)(231.37219534,140.38462954)
\lineto(230.89951188,140.38462954)
\lineto(230.89951188,144.87656943)
\lineto(230.89951188,149.36850932)
\lineto(233.38730025,149.36850932)
\lineto(235.87508863,149.36850932)
\lineto(235.87508863,144.31507863)
\lineto(235.87508863,139.26164345)
\lineto(234.00924959,139.26164345)
\lineto(232.14340607,139.26164345)
\closepath
}
}
{
\newrgbcolor{curcolor}{0.53333336 0.70980394 0.85882354}
\pscustom[linewidth=0,linecolor=curcolor]
{
\newpath
\moveto(232.14340607,139.67340427)
\curveto(232.14340607,140.20863855)(231.95257424,140.38462954)(231.37219534,140.38462954)
\lineto(230.89951188,140.38462954)
\lineto(230.89951188,144.87656943)
\lineto(230.89951188,149.36850932)
\lineto(233.38730025,149.36850932)
\lineto(235.87508863,149.36850932)
\lineto(235.87508863,144.31507863)
\lineto(235.87508863,139.26164345)
\lineto(234.00924959,139.26164345)
\lineto(232.14340607,139.26164345)
\closepath
}
}
{
\newrgbcolor{curcolor}{0.68627453 0.7764706 0.9137255}
\pscustom[linestyle=none,fillstyle=solid,fillcolor=curcolor]
{
\newpath
\moveto(236.49703796,144.31507863)
\lineto(236.49703796,149.36850932)
\lineto(239.91774363,149.36850932)
\lineto(243.33845376,149.36850932)
\lineto(243.33845376,144.31507863)
\lineto(243.33845376,139.26164345)
\lineto(241.75248767,139.2786179)
\curveto(240.88020698,139.28795835)(240.29603634,139.32042536)(240.45432719,139.35072791)
\curveto(240.61262252,139.38103945)(240.98433899,139.640102)(241.28037274,139.92642248)
\curveto(241.71642355,140.34816589)(241.84445206,140.57728963)(241.9547395,141.13326598)
\curveto(242.17673347,142.25240813)(241.80096561,143.29024352)(240.96833039,143.85765742)
\curveto(240.60937693,144.10226923)(240.41786465,144.14224003)(239.79034185,144.10351312)
\curveto(238.95397308,144.0518982)(238.57020539,143.83083508)(238.10078095,143.13027497)
\curveto(237.85796594,142.76789725)(237.80736158,142.53115746)(237.80544108,141.74855414)
\curveto(237.80365041,140.93208274)(237.84810382,140.73591549)(238.13502754,140.30052066)
\curveto(238.32962872,140.00524593)(238.68984013,139.68622043)(239.00575481,139.52935938)
\lineto(239.54457671,139.26182307)
\lineto(238.0208051,139.26172428)
\lineto(236.49703796,139.26162548)
\lineto(236.49703796,144.31506067)
\closepath
}
}
{
\newrgbcolor{curcolor}{0.53333336 0.70980394 0.85882354}
\pscustom[linewidth=0,linecolor=curcolor]
{
\newpath
\moveto(236.49703796,144.31507863)
\lineto(236.49703796,149.36850932)
\lineto(239.91774363,149.36850932)
\lineto(243.33845376,149.36850932)
\lineto(243.33845376,144.31507863)
\lineto(243.33845376,139.26164345)
\lineto(241.75248767,139.2786179)
\curveto(240.88020698,139.28795835)(240.29603634,139.32042536)(240.45432719,139.35072791)
\curveto(240.61262252,139.38103945)(240.98433899,139.640102)(241.28037274,139.92642248)
\curveto(241.71642355,140.34816589)(241.84445206,140.57728963)(241.9547395,141.13326598)
\curveto(242.17673347,142.25240813)(241.80096561,143.29024352)(240.96833039,143.85765742)
\curveto(240.60937693,144.10226923)(240.41786465,144.14224003)(239.79034185,144.10351312)
\curveto(238.95397308,144.0518982)(238.57020539,143.83083508)(238.10078095,143.13027497)
\curveto(237.85796594,142.76789725)(237.80736158,142.53115746)(237.80544108,141.74855414)
\curveto(237.80365041,140.93208274)(237.84810382,140.73591549)(238.13502754,140.30052066)
\curveto(238.32962872,140.00524593)(238.68984013,139.68622043)(239.00575481,139.52935938)
\lineto(239.54457671,139.26182307)
\lineto(238.0208051,139.26172428)
\lineto(236.49703796,139.26162548)
\lineto(236.49703796,144.31506067)
\closepath
}
}
{
\newrgbcolor{curcolor}{0.68627453 0.7764706 0.9137255}
\pscustom[linestyle=none,fillstyle=solid,fillcolor=curcolor]
{
\newpath
\moveto(243.96039862,144.31507863)
\lineto(243.96039862,149.36850932)
\lineto(246.46821118,149.36850932)
\lineto(248.97602374,149.36850932)
\lineto(248.89380507,145.79208993)
\curveto(248.84858167,143.82506017)(248.8115864,141.55101277)(248.8115864,140.73865924)
\lineto(248.8115864,139.26164345)
\lineto(246.38599251,139.26164345)
\lineto(243.96039862,139.26164345)
\closepath
}
}
{
\newrgbcolor{curcolor}{0.53333336 0.70980394 0.85882354}
\pscustom[linewidth=0,linecolor=curcolor]
{
\newpath
\moveto(243.96039862,144.31507863)
\lineto(243.96039862,149.36850932)
\lineto(246.46821118,149.36850932)
\lineto(248.97602374,149.36850932)
\lineto(248.89380507,145.79208993)
\curveto(248.84858167,143.82506017)(248.8115864,141.55101277)(248.8115864,140.73865924)
\lineto(248.8115864,139.26164345)
\lineto(246.38599251,139.26164345)
\lineto(243.96039862,139.26164345)
\closepath
}
}
{
\newrgbcolor{curcolor}{0.68627453 0.7764706 0.9137255}
\pscustom[linestyle=none,fillstyle=solid,fillcolor=curcolor]
{
\newpath
\moveto(229.65562217,137.14044924)
\lineto(229.65562217,137.51477868)
\lineto(241.59699921,137.51477868)
\lineto(253.53838074,137.51477868)
\lineto(253.53838074,137.14044924)
\lineto(253.53838074,136.76612069)
\lineto(241.59699921,136.76612069)
\lineto(229.65562217,136.76612069)
\closepath
}
}
{
\newrgbcolor{curcolor}{0.53333336 0.70980394 0.85882354}
\pscustom[linewidth=0,linecolor=curcolor]
{
\newpath
\moveto(229.65562217,137.14044924)
\lineto(229.65562217,137.51477868)
\lineto(241.59699921,137.51477868)
\lineto(253.53838074,137.51477868)
\lineto(253.53838074,137.14044924)
\lineto(253.53838074,136.76612069)
\lineto(241.59699921,136.76612069)
\lineto(229.65562217,136.76612069)
\closepath
}
}
{
\newrgbcolor{curcolor}{0.68627453 0.7764706 0.9137255}
\pscustom[linestyle=none,fillstyle=solid,fillcolor=curcolor]
{
\newpath
\moveto(229.65562217,135.64313549)
\lineto(229.65562217,136.01746404)
\lineto(241.59699921,136.01746404)
\lineto(253.53838074,136.01746404)
\lineto(253.53838074,135.64313549)
\lineto(253.53838074,135.26880694)
\lineto(241.59699921,135.26880694)
\lineto(229.65562217,135.26880694)
\closepath
}
}
{
\newrgbcolor{curcolor}{0.53333336 0.70980394 0.85882354}
\pscustom[linewidth=0,linecolor=curcolor]
{
\newpath
\moveto(229.65562217,135.64313549)
\lineto(229.65562217,136.01746404)
\lineto(241.59699921,136.01746404)
\lineto(253.53838074,136.01746404)
\lineto(253.53838074,135.64313549)
\lineto(253.53838074,135.26880694)
\lineto(241.59699921,135.26880694)
\lineto(229.65562217,135.26880694)
\closepath
}
}
{
\newrgbcolor{curcolor}{0.68627453 0.7764706 0.9137255}
\pscustom[linestyle=none,fillstyle=solid,fillcolor=curcolor]
{
\newpath
\moveto(235.75069966,126.03537255)
\lineto(235.75069966,134.39537411)
\lineto(242.27647386,134.39537411)
\lineto(248.80224805,134.39537411)
\lineto(248.83801671,133.24119453)
\lineto(248.87378537,132.0870154)
\lineto(249.77560765,132.05035417)
\lineto(250.67742993,132.01369293)
\lineto(250.67742993,124.84453084)
\lineto(250.67742993,117.67536875)
\lineto(243.21406927,117.67536875)
\lineto(235.75070414,117.67536875)
\lineto(235.75070414,126.03537076)
\closepath
}
}
{
\newrgbcolor{curcolor}{0.53333336 0.70980394 0.85882354}
\pscustom[linewidth=0,linecolor=curcolor]
{
\newpath
\moveto(235.75069966,126.03537255)
\lineto(235.75069966,134.39537411)
\lineto(242.27647386,134.39537411)
\lineto(248.80224805,134.39537411)
\lineto(248.83801671,133.24119453)
\lineto(248.87378537,132.0870154)
\lineto(249.77560765,132.05035417)
\lineto(250.67742993,132.01369293)
\lineto(250.67742993,124.84453084)
\lineto(250.67742993,117.67536875)
\lineto(243.21406927,117.67536875)
\lineto(235.75070414,117.67536875)
\lineto(235.75070414,126.03537076)
\closepath
}
}
{
\newrgbcolor{curcolor}{0.68627453 0.7764706 0.9137255}
\pscustom[linestyle=none,fillstyle=solid,fillcolor=curcolor]
{
\newpath
\moveto(233.6360782,125.28671546)
\lineto(233.6360782,134.52015029)
\lineto(234.25719487,134.52015029)
\lineto(234.87830706,134.52015029)
\lineto(234.91022577,125.69223794)
\lineto(234.94214449,116.86432603)
\lineto(243.16929354,116.83224521)
\curveto(249.96828386,116.80573272)(251.40758952,116.82928141)(251.4606471,116.96798922)
\curveto(251.49596809,117.0602939)(251.54217188,121.04728948)(251.56334657,125.82798241)
\lineto(251.601846,134.52014985)
\lineto(252.5700977,134.52014985)
\lineto(253.53835388,134.52014985)
\lineto(253.53835388,125.28671501)
\lineto(253.53835388,116.05328062)
\lineto(243.58720485,116.05328062)
\lineto(233.63605134,116.05328062)
\lineto(233.63605134,125.28671501)
\closepath
}
}
{
\newrgbcolor{curcolor}{0.53333336 0.70980394 0.85882354}
\pscustom[linewidth=0,linecolor=curcolor]
{
\newpath
\moveto(233.6360782,125.28671546)
\lineto(233.6360782,134.52015029)
\lineto(234.25719487,134.52015029)
\lineto(234.87830706,134.52015029)
\lineto(234.91022577,125.69223794)
\lineto(234.94214449,116.86432603)
\lineto(243.16929354,116.83224521)
\curveto(249.96828386,116.80573272)(251.40758952,116.82928141)(251.4606471,116.96798922)
\curveto(251.49596809,117.0602939)(251.54217188,121.04728948)(251.56334657,125.82798241)
\lineto(251.601846,134.52014985)
\lineto(252.5700977,134.52014985)
\lineto(253.53835388,134.52014985)
\lineto(253.53835388,125.28671501)
\lineto(253.53835388,116.05328062)
\lineto(243.58720485,116.05328062)
\lineto(233.63605134,116.05328062)
\lineto(233.63605134,125.28671501)
\closepath
}
}
{
\newrgbcolor{curcolor}{0.68627453 0.7764706 0.9137255}
\pscustom[linestyle=none,fillstyle=solid,fillcolor=curcolor]
{
\newpath
\moveto(254.16033007,122.85358057)
\lineto(254.16033007,128.28134295)
\lineto(257.33225779,128.28134295)
\lineto(260.50418551,128.28134295)
\lineto(260.50418551,122.85358057)
\lineto(260.50418551,117.42581863)
\lineto(257.33225779,117.42581863)
\lineto(254.16033007,117.42581863)
\closepath
}
}
{
\newrgbcolor{curcolor}{0.53333336 0.70980394 0.85882354}
\pscustom[linewidth=0,linecolor=curcolor]
{
\newpath
\moveto(254.16033007,122.85358057)
\lineto(254.16033007,128.28134295)
\lineto(257.33225779,128.28134295)
\lineto(260.50418551,128.28134295)
\lineto(260.50418551,122.85358057)
\lineto(260.50418551,117.42581863)
\lineto(257.33225779,117.42581863)
\lineto(254.16033007,117.42581863)
\closepath
}
}
{
\newrgbcolor{curcolor}{0.68627453 0.7764706 0.9137255}
\pscustom[linestyle=none,fillstyle=solid,fillcolor=curcolor]
{
\newpath
\moveto(236.24825555,113.62014618)
\lineto(236.24825555,115.42940015)
\lineto(241.47261024,115.42940015)
\lineto(246.69696494,115.42940015)
\lineto(246.69696494,113.62014618)
\lineto(246.69696494,111.8108922)
\lineto(241.47261024,111.8108922)
\lineto(236.24825555,111.8108922)
\closepath
\moveto(242.23298293,113.54160788)
\curveto(242.7509659,114.20216485)(242.29718745,115.17984779)(241.47261024,115.17984779)
\curveto(240.71119896,115.17984779)(240.33288835,114.52079113)(240.6530693,113.75211136)
\curveto(240.95810566,113.01978252)(241.74189585,112.91535234)(242.23298293,113.54160788)
\closepath
}
}
{
\newrgbcolor{curcolor}{0.53333336 0.70980394 0.85882354}
\pscustom[linewidth=0,linecolor=curcolor]
{
\newpath
\moveto(236.24825555,113.62014618)
\lineto(236.24825555,115.42940015)
\lineto(241.47261024,115.42940015)
\lineto(246.69696494,115.42940015)
\lineto(246.69696494,113.62014618)
\lineto(246.69696494,111.8108922)
\lineto(241.47261024,111.8108922)
\lineto(236.24825555,111.8108922)
\closepath
\moveto(242.23298293,113.54160788)
\curveto(242.7509659,114.20216485)(242.29718745,115.17984779)(241.47261024,115.17984779)
\curveto(240.71119896,115.17984779)(240.33288835,114.52079113)(240.6530693,113.75211136)
\curveto(240.95810566,113.01978252)(241.74189585,112.91535234)(242.23298293,113.54160788)
\closepath
}
}
{
\newrgbcolor{curcolor}{0 0 0}
\pscustom[linewidth=2.66666667,linecolor=curcolor]
{
\newpath
\moveto(325.56037467,60.02361991)
\curveto(325.56037467,60.02361991)(313.559708,56.38831324)(321.689548,43.89407324)
\curveto(329.81940133,31.39983324)(314.838348,28.37361991)(314.838348,28.37361991)
}
}
{
\newrgbcolor{curcolor}{0 0 0}
\pscustom[linewidth=2.66666667,linecolor=curcolor]
{
\newpath
\moveto(335.56037467,60.02361991)
\curveto(335.56037467,60.02361991)(323.559708,56.38831324)(331.689548,43.89407324)
\curveto(339.81940133,31.39983324)(324.838348,28.37361991)(324.838348,28.37361991)
}
}
{
\newrgbcolor{curcolor}{0 0 0}
\pscustom[linestyle=none,fillstyle=solid,fillcolor=curcolor]
{
\newpath
\moveto(1.3281255,104.01102695)
\lineto(5.3906255,90.86258945)
\lineto(4.0625005,90.86258945)
\lineto(0.0000005,104.01102695)
\closepath
}
}
{
\newrgbcolor{curcolor}{0 0 0}
\pscustom[linestyle=none,fillstyle=solid,fillcolor=curcolor]
{
\newpath
\moveto(8.335938,103.58133945)
\lineto(8.335938,101.09696445)
\lineto(11.2968755,101.09696445)
\lineto(11.2968755,99.97977695)
\lineto(8.335938,99.97977695)
\lineto(8.335938,95.22977695)
\curveto(8.335938,94.51623528)(8.43229217,94.05790195)(8.6250005,93.85477695)
\curveto(8.82291717,93.65165195)(9.22135467,93.55008945)(9.820313,93.55008945)
\lineto(11.2968755,93.55008945)
\lineto(11.2968755,92.34696445)
\lineto(9.820313,92.34696445)
\curveto(8.710938,92.34696445)(7.945313,92.55269362)(7.523438,92.96415195)
\curveto(7.101563,93.38081862)(6.8906255,94.13602695)(6.8906255,95.22977695)
\lineto(6.8906255,99.97977695)
\lineto(5.835938,99.97977695)
\lineto(5.835938,101.09696445)
\lineto(6.8906255,101.09696445)
\lineto(6.8906255,103.58133945)
\closepath
}
}
{
\newrgbcolor{curcolor}{0 0 0}
\pscustom[linestyle=none,fillstyle=solid,fillcolor=curcolor]
{
\newpath
\moveto(20.679688,97.08133945)
\lineto(20.679688,96.37821445)
\lineto(14.070313,96.37821445)
\curveto(14.132813,95.38863112)(14.429688,94.63342278)(14.960938,94.11258945)
\curveto(15.49739633,93.59696445)(16.242188,93.33915195)(17.195313,93.33915195)
\curveto(17.74739633,93.33915195)(18.2812505,93.40686028)(18.7968755,93.54227695)
\curveto(19.31770883,93.67769362)(19.83333383,93.88081862)(20.3437505,94.15165195)
\lineto(20.3437505,92.79227695)
\curveto(19.8281255,92.57352695)(19.29947967,92.40686028)(18.757813,92.29227695)
\curveto(18.21614633,92.17769362)(17.66666717,92.12040195)(17.1093755,92.12040195)
\curveto(15.71354217,92.12040195)(14.60677133,92.52665195)(13.789063,93.33915195)
\curveto(12.976563,94.15165195)(12.570313,95.25061028)(12.570313,96.63602695)
\curveto(12.570313,98.06831862)(12.95572967,99.20373528)(13.726563,100.04227695)
\curveto(14.50260467,100.88602695)(15.5468755,101.30790195)(16.8593755,101.30790195)
\curveto(18.03645883,101.30790195)(18.96614633,100.92769362)(19.648438,100.16727695)
\curveto(20.335938,99.41206862)(20.679688,98.38342278)(20.679688,97.08133945)
\closepath
\moveto(19.242188,97.50321445)
\curveto(19.23177133,98.28967278)(19.01041717,98.91727695)(18.5781255,99.38602695)
\curveto(18.15104217,99.85477695)(17.58333383,100.08915195)(16.8750005,100.08915195)
\curveto(16.07291717,100.08915195)(15.429688,99.86258945)(14.945313,99.40946445)
\curveto(14.46614633,98.95633945)(14.19010467,98.31831862)(14.117188,97.49540195)
\closepath
}
}
{
\newrgbcolor{curcolor}{0 0 0}
\pscustom[linestyle=none,fillstyle=solid,fillcolor=curcolor]
{
\newpath
\moveto(30.0312505,101.09696445)
\lineto(26.867188,96.83915195)
\lineto(30.195313,92.34696445)
\lineto(28.5000005,92.34696445)
\lineto(25.9531255,95.78446445)
\lineto(23.4062505,92.34696445)
\lineto(21.710938,92.34696445)
\lineto(25.1093755,96.92508945)
\lineto(22.0000005,101.09696445)
\lineto(23.695313,101.09696445)
\lineto(26.0156255,97.97977695)
\lineto(28.335938,101.09696445)
\closepath
}
}
{
\newrgbcolor{curcolor}{0 0 0}
\pscustom[linestyle=none,fillstyle=solid,fillcolor=curcolor]
{
\newpath
\moveto(33.648438,103.58133945)
\lineto(33.648438,101.09696445)
\lineto(36.6093755,101.09696445)
\lineto(36.6093755,99.97977695)
\lineto(33.648438,99.97977695)
\lineto(33.648438,95.22977695)
\curveto(33.648438,94.51623528)(33.74479217,94.05790195)(33.9375005,93.85477695)
\curveto(34.13541717,93.65165195)(34.53385467,93.55008945)(35.132813,93.55008945)
\lineto(36.6093755,93.55008945)
\lineto(36.6093755,92.34696445)
\lineto(35.132813,92.34696445)
\curveto(34.023438,92.34696445)(33.257813,92.55269362)(32.835938,92.96415195)
\curveto(32.414063,93.38081862)(32.2031255,94.13602695)(32.2031255,95.22977695)
\lineto(32.2031255,99.97977695)
\lineto(31.148438,99.97977695)
\lineto(31.148438,101.09696445)
\lineto(32.2031255,101.09696445)
\lineto(32.2031255,103.58133945)
\closepath
}
}
{
\newrgbcolor{curcolor}{0 0 0}
\pscustom[linestyle=none,fillstyle=solid,fillcolor=curcolor]
{
\newpath
\moveto(44.789063,96.71415195)
\curveto(44.789063,97.77144362)(44.570313,98.59956862)(44.132813,99.19852695)
\curveto(43.70052133,99.80269362)(43.10416717,100.10477695)(42.3437505,100.10477695)
\curveto(41.58333383,100.10477695)(40.9843755,99.80269362)(40.5468755,99.19852695)
\curveto(40.11458383,98.59956862)(39.898438,97.77144362)(39.898438,96.71415195)
\curveto(39.898438,95.65686028)(40.11458383,94.82613112)(40.5468755,94.22196445)
\curveto(40.9843755,93.62300612)(41.58333383,93.32352695)(42.3437505,93.32352695)
\curveto(43.10416717,93.32352695)(43.70052133,93.62300612)(44.132813,94.22196445)
\curveto(44.570313,94.82613112)(44.789063,95.65686028)(44.789063,96.71415195)
\closepath
\moveto(39.898438,99.76883945)
\curveto(40.20052133,100.28967278)(40.58072967,100.67508945)(41.039063,100.92508945)
\curveto(41.50260467,101.18029778)(42.054688,101.30790195)(42.695313,101.30790195)
\curveto(43.757813,101.30790195)(44.61979217,100.88602695)(45.2812505,100.04227695)
\curveto(45.94791717,99.19852695)(46.2812505,98.08915195)(46.2812505,96.71415195)
\curveto(46.2812505,95.33915195)(45.94791717,94.22977695)(45.2812505,93.38602695)
\curveto(44.61979217,92.54227695)(43.757813,92.12040195)(42.695313,92.12040195)
\curveto(42.054688,92.12040195)(41.50260467,92.24540195)(41.039063,92.49540195)
\curveto(40.58072967,92.75061028)(40.20052133,93.13863112)(39.898438,93.65946445)
\lineto(39.898438,92.34696445)
\lineto(38.4531255,92.34696445)
\lineto(38.4531255,104.50321445)
\lineto(39.898438,104.50321445)
\closepath
}
}
{
\newrgbcolor{curcolor}{0 0 0}
\pscustom[linestyle=none,fillstyle=solid,fillcolor=curcolor]
{
\newpath
\moveto(53.0937505,104.50321445)
\lineto(53.0937505,103.30790195)
\lineto(51.7187505,103.30790195)
\curveto(51.2031255,103.30790195)(50.8437505,103.20373528)(50.6406255,102.99540195)
\curveto(50.44270883,102.78706862)(50.3437505,102.41206862)(50.3437505,101.87040195)
\lineto(50.3437505,101.09696445)
\lineto(52.710938,101.09696445)
\lineto(52.710938,99.97977695)
\lineto(50.3437505,99.97977695)
\lineto(50.3437505,92.34696445)
\lineto(48.898438,92.34696445)
\lineto(48.898438,99.97977695)
\lineto(47.523438,99.97977695)
\lineto(47.523438,101.09696445)
\lineto(48.898438,101.09696445)
\lineto(48.898438,101.70633945)
\curveto(48.898438,102.68029778)(49.1250005,103.38863112)(49.5781255,103.83133945)
\curveto(50.0312505,104.27925612)(50.7500005,104.50321445)(51.7343755,104.50321445)
\closepath
}
}
{
\newrgbcolor{curcolor}{0 0 0}
\pscustom[linestyle=none,fillstyle=solid,fillcolor=curcolor]
{
\newpath
\moveto(60.960938,90.86258945)
\lineto(60.960938,89.73758945)
\lineto(60.476563,89.73758945)
\curveto(59.179688,89.73758945)(58.30989633,89.93029778)(57.867188,90.31571445)
\curveto(57.429688,90.70113112)(57.210938,91.46936028)(57.210938,92.62040195)
\lineto(57.210938,94.48758945)
\curveto(57.210938,95.27404778)(57.070313,95.81831862)(56.789063,96.12040195)
\curveto(56.507813,96.42248528)(55.99739633,96.57352695)(55.257813,96.57352695)
\lineto(54.7812505,96.57352695)
\lineto(54.7812505,97.69071445)
\lineto(55.257813,97.69071445)
\curveto(56.00260467,97.69071445)(56.51302133,97.83915195)(56.789063,98.13602695)
\curveto(57.070313,98.43811028)(57.210938,98.97717278)(57.210938,99.75321445)
\lineto(57.210938,101.62821445)
\curveto(57.210938,102.77925612)(57.429688,103.54488112)(57.867188,103.92508945)
\curveto(58.30989633,104.31050612)(59.179688,104.50321445)(60.476563,104.50321445)
\lineto(60.960938,104.50321445)
\lineto(60.960938,103.38602695)
\lineto(60.429688,103.38602695)
\curveto(59.695313,103.38602695)(59.21614633,103.27144362)(58.992188,103.04227695)
\curveto(58.76822967,102.81311028)(58.6562505,102.33133945)(58.6562505,101.59696445)
\lineto(58.6562505,99.65946445)
\curveto(58.6562505,98.84175612)(58.53645883,98.24800612)(58.2968755,97.87821445)
\curveto(58.0625005,97.50842278)(57.65885467,97.25842278)(57.085938,97.12821445)
\curveto(57.664063,96.98758945)(58.070313,96.73238112)(58.304688,96.36258945)
\curveto(58.539063,95.99279778)(58.6562505,95.40165195)(58.6562505,94.58915195)
\lineto(58.6562505,92.65165195)
\curveto(58.6562505,91.91727695)(58.76822967,91.43550612)(58.992188,91.20633945)
\curveto(59.21614633,90.97717278)(59.695313,90.86258945)(60.429688,90.86258945)
\closepath
}
}
{
\newrgbcolor{curcolor}{0 0 0}
\pscustom[linestyle=none,fillstyle=solid,fillcolor=curcolor]
{
\newpath
\moveto(63.5000005,104.01102695)
\lineto(65.0937505,104.01102695)
\lineto(67.5468755,94.15165195)
\lineto(69.992188,104.01102695)
\lineto(71.7656255,104.01102695)
\lineto(74.2187505,94.15165195)
\lineto(76.664063,104.01102695)
\lineto(78.2656255,104.01102695)
\lineto(75.335938,92.34696445)
\lineto(73.351563,92.34696445)
\lineto(70.8906255,102.47196445)
\lineto(68.4062505,92.34696445)
\lineto(66.4218755,92.34696445)
\closepath
}
}
{
\newrgbcolor{curcolor}{0 0 0}
\pscustom[linestyle=none,fillstyle=solid,fillcolor=curcolor]
{
\newpath
\moveto(78.7343755,104.01102695)
\lineto(88.601563,104.01102695)
\lineto(88.601563,102.68290195)
\lineto(84.460938,102.68290195)
\lineto(84.460938,92.34696445)
\lineto(82.8750005,92.34696445)
\lineto(82.8750005,102.68290195)
\lineto(78.7343755,102.68290195)
\closepath
}
}
{
\newrgbcolor{curcolor}{0 0 0}
\pscustom[linestyle=none,fillstyle=solid,fillcolor=curcolor]
{
\newpath
\moveto(90.5625005,90.86258945)
\lineto(91.1093755,90.86258945)
\curveto(91.83854217,90.86258945)(92.3125005,90.97456862)(92.5312505,91.19852695)
\curveto(92.75520883,91.42248528)(92.867188,91.90686028)(92.867188,92.65165195)
\lineto(92.867188,94.58915195)
\curveto(92.867188,95.40165195)(92.9843755,95.99279778)(93.2187505,96.36258945)
\curveto(93.4531255,96.73238112)(93.8593755,96.98758945)(94.4375005,97.12821445)
\curveto(93.8593755,97.25842278)(93.4531255,97.50842278)(93.2187505,97.87821445)
\curveto(92.9843755,98.24800612)(92.867188,98.84175612)(92.867188,99.65946445)
\lineto(92.867188,101.59696445)
\curveto(92.867188,102.33654778)(92.75520883,102.81831862)(92.5312505,103.04227695)
\curveto(92.3125005,103.27144362)(91.83854217,103.38602695)(91.1093755,103.38602695)
\lineto(90.5625005,103.38602695)
\lineto(90.5625005,104.50321445)
\lineto(91.054688,104.50321445)
\curveto(92.351563,104.50321445)(93.21614633,104.31050612)(93.648438,103.92508945)
\curveto(94.085938,103.54488112)(94.304688,102.77925612)(94.304688,101.62821445)
\lineto(94.304688,99.75321445)
\curveto(94.304688,98.97717278)(94.445313,98.43811028)(94.726563,98.13602695)
\curveto(95.007813,97.83915195)(95.51822967,97.69071445)(96.257813,97.69071445)
\lineto(96.742188,97.69071445)
\lineto(96.742188,96.57352695)
\lineto(96.257813,96.57352695)
\curveto(95.51822967,96.57352695)(95.007813,96.42248528)(94.726563,96.12040195)
\curveto(94.445313,95.81831862)(94.304688,95.27404778)(94.304688,94.48758945)
\lineto(94.304688,92.62040195)
\curveto(94.304688,91.46936028)(94.085938,90.70113112)(93.648438,90.31571445)
\curveto(93.21614633,89.93029778)(92.351563,89.73758945)(91.054688,89.73758945)
\lineto(90.5625005,89.73758945)
\closepath
}
}
{
\newrgbcolor{curcolor}{0 0 0}
\pscustom[linestyle=none,fillstyle=solid,fillcolor=curcolor]
{
\newpath
\moveto(6.98242237,46.53452864)
\lineto(11.04492237,33.38609114)
\lineto(9.71679737,33.38609114)
\lineto(5.65429737,46.53452864)
\closepath
}
}
{
\newrgbcolor{curcolor}{0 0 0}
\pscustom[linestyle=none,fillstyle=solid,fillcolor=curcolor]
{
\newpath
\moveto(13.99023487,46.10484114)
\lineto(13.99023487,43.62046614)
\lineto(16.95117237,43.62046614)
\lineto(16.95117237,42.50327864)
\lineto(13.99023487,42.50327864)
\lineto(13.99023487,37.75327864)
\curveto(13.99023487,37.03973698)(14.08658904,36.58140364)(14.27929737,36.37827864)
\curveto(14.47721404,36.17515364)(14.87565154,36.07359114)(15.47460987,36.07359114)
\lineto(16.95117237,36.07359114)
\lineto(16.95117237,34.87046614)
\lineto(15.47460987,34.87046614)
\curveto(14.36523487,34.87046614)(13.59960987,35.07619531)(13.17773487,35.48765364)
\curveto(12.75585987,35.90432031)(12.54492237,36.65952864)(12.54492237,37.75327864)
\lineto(12.54492237,42.50327864)
\lineto(11.49023487,42.50327864)
\lineto(11.49023487,43.62046614)
\lineto(12.54492237,43.62046614)
\lineto(12.54492237,46.10484114)
\closepath
}
}
{
\newrgbcolor{curcolor}{0 0 0}
\pscustom[linestyle=none,fillstyle=solid,fillcolor=curcolor]
{
\newpath
\moveto(26.33398487,39.60484114)
\lineto(26.33398487,38.90171614)
\lineto(19.72460987,38.90171614)
\curveto(19.78710987,37.91213281)(20.08398487,37.15692448)(20.61523487,36.63609114)
\curveto(21.15169321,36.12046614)(21.89648487,35.86265364)(22.84960987,35.86265364)
\curveto(23.40169321,35.86265364)(23.93554737,35.93036198)(24.45117237,36.06577864)
\curveto(24.97200571,36.20119531)(25.48763071,36.40432031)(25.99804737,36.67515364)
\lineto(25.99804737,35.31577864)
\curveto(25.48242237,35.09702864)(24.95377654,34.93036198)(24.41210987,34.81577864)
\curveto(23.87044321,34.70119531)(23.32096404,34.64390364)(22.76367237,34.64390364)
\curveto(21.36783904,34.64390364)(20.26106821,35.05015364)(19.44335987,35.86265364)
\curveto(18.63085987,36.67515364)(18.22460987,37.77411198)(18.22460987,39.15952864)
\curveto(18.22460987,40.59182031)(18.61002654,41.72723698)(19.38085987,42.56577864)
\curveto(20.15690154,43.40952864)(21.20117237,43.83140364)(22.51367237,43.83140364)
\curveto(23.69075571,43.83140364)(24.62044321,43.45119531)(25.30273487,42.69077864)
\curveto(25.99023487,41.93557031)(26.33398487,40.90692448)(26.33398487,39.60484114)
\closepath
\moveto(24.89648487,40.02671614)
\curveto(24.88606821,40.81317448)(24.66471404,41.44077864)(24.23242237,41.90952864)
\curveto(23.80533904,42.37827864)(23.23763071,42.61265364)(22.52929737,42.61265364)
\curveto(21.72721404,42.61265364)(21.08398487,42.38609114)(20.59960987,41.93296614)
\curveto(20.12044321,41.47984114)(19.84440154,40.84182031)(19.77148487,40.01890364)
\closepath
}
}
{
\newrgbcolor{curcolor}{0 0 0}
\pscustom[linestyle=none,fillstyle=solid,fillcolor=curcolor]
{
\newpath
\moveto(35.68554737,43.62046614)
\lineto(32.52148487,39.36265364)
\lineto(35.84960987,34.87046614)
\lineto(34.15429737,34.87046614)
\lineto(31.60742237,38.30796614)
\lineto(29.06054737,34.87046614)
\lineto(27.36523487,34.87046614)
\lineto(30.76367237,39.44859114)
\lineto(27.65429737,43.62046614)
\lineto(29.34960987,43.62046614)
\lineto(31.66992237,40.50327864)
\lineto(33.99023487,43.62046614)
\closepath
}
}
{
\newrgbcolor{curcolor}{0 0 0}
\pscustom[linestyle=none,fillstyle=solid,fillcolor=curcolor]
{
\newpath
\moveto(39.30273487,46.10484114)
\lineto(39.30273487,43.62046614)
\lineto(42.26367237,43.62046614)
\lineto(42.26367237,42.50327864)
\lineto(39.30273487,42.50327864)
\lineto(39.30273487,37.75327864)
\curveto(39.30273487,37.03973698)(39.39908904,36.58140364)(39.59179737,36.37827864)
\curveto(39.78971404,36.17515364)(40.18815154,36.07359114)(40.78710987,36.07359114)
\lineto(42.26367237,36.07359114)
\lineto(42.26367237,34.87046614)
\lineto(40.78710987,34.87046614)
\curveto(39.67773487,34.87046614)(38.91210987,35.07619531)(38.49023487,35.48765364)
\curveto(38.06835987,35.90432031)(37.85742237,36.65952864)(37.85742237,37.75327864)
\lineto(37.85742237,42.50327864)
\lineto(36.80273487,42.50327864)
\lineto(36.80273487,43.62046614)
\lineto(37.85742237,43.62046614)
\lineto(37.85742237,46.10484114)
\closepath
}
}
{
\newrgbcolor{curcolor}{0 0 0}
\pscustom[linestyle=none,fillstyle=solid,fillcolor=curcolor]
{
\newpath
\moveto(50.44335987,39.23765364)
\curveto(50.44335987,40.29494531)(50.22460987,41.12307031)(49.78710987,41.72202864)
\curveto(49.35481821,42.32619531)(48.75846404,42.62827864)(47.99804737,42.62827864)
\curveto(47.23763071,42.62827864)(46.63867237,42.32619531)(46.20117237,41.72202864)
\curveto(45.76888071,41.12307031)(45.55273487,40.29494531)(45.55273487,39.23765364)
\curveto(45.55273487,38.18036198)(45.76888071,37.34963281)(46.20117237,36.74546614)
\curveto(46.63867237,36.14650781)(47.23763071,35.84702864)(47.99804737,35.84702864)
\curveto(48.75846404,35.84702864)(49.35481821,36.14650781)(49.78710987,36.74546614)
\curveto(50.22460987,37.34963281)(50.44335987,38.18036198)(50.44335987,39.23765364)
\closepath
\moveto(45.55273487,42.29234114)
\curveto(45.85481821,42.81317448)(46.23502654,43.19859114)(46.69335987,43.44859114)
\curveto(47.15690154,43.70379948)(47.70898487,43.83140364)(48.34960987,43.83140364)
\curveto(49.41210987,43.83140364)(50.27408904,43.40952864)(50.93554737,42.56577864)
\curveto(51.60221404,41.72202864)(51.93554737,40.61265364)(51.93554737,39.23765364)
\curveto(51.93554737,37.86265364)(51.60221404,36.75327864)(50.93554737,35.90952864)
\curveto(50.27408904,35.06577864)(49.41210987,34.64390364)(48.34960987,34.64390364)
\curveto(47.70898487,34.64390364)(47.15690154,34.76890364)(46.69335987,35.01890364)
\curveto(46.23502654,35.27411198)(45.85481821,35.66213281)(45.55273487,36.18296614)
\lineto(45.55273487,34.87046614)
\lineto(44.10742237,34.87046614)
\lineto(44.10742237,47.02671614)
\lineto(45.55273487,47.02671614)
\closepath
}
}
{
\newrgbcolor{curcolor}{0 0 0}
\pscustom[linestyle=none,fillstyle=solid,fillcolor=curcolor]
{
\newpath
\moveto(58.74804737,47.02671614)
\lineto(58.74804737,45.83140364)
\lineto(57.37304737,45.83140364)
\curveto(56.85742237,45.83140364)(56.49804737,45.72723698)(56.29492237,45.51890364)
\curveto(56.09700571,45.31057031)(55.99804737,44.93557031)(55.99804737,44.39390364)
\lineto(55.99804737,43.62046614)
\lineto(58.36523487,43.62046614)
\lineto(58.36523487,42.50327864)
\lineto(55.99804737,42.50327864)
\lineto(55.99804737,34.87046614)
\lineto(54.55273487,34.87046614)
\lineto(54.55273487,42.50327864)
\lineto(53.17773487,42.50327864)
\lineto(53.17773487,43.62046614)
\lineto(54.55273487,43.62046614)
\lineto(54.55273487,44.22984114)
\curveto(54.55273487,45.20379948)(54.77929737,45.91213281)(55.23242237,46.35484114)
\curveto(55.68554737,46.80275781)(56.40429737,47.02671614)(57.38867237,47.02671614)
\closepath
}
}
{
\newrgbcolor{curcolor}{0 0 0}
\pscustom[linestyle=none,fillstyle=solid,fillcolor=curcolor]
{
\newpath
\moveto(66.61523487,33.38609114)
\lineto(66.61523487,32.26109114)
\lineto(66.13085987,32.26109114)
\curveto(64.83398487,32.26109114)(63.96419321,32.45379948)(63.52148487,32.83921614)
\curveto(63.08398487,33.22463281)(62.86523487,33.99286198)(62.86523487,35.14390364)
\lineto(62.86523487,37.01109114)
\curveto(62.86523487,37.79754948)(62.72460987,38.34182031)(62.44335987,38.64390364)
\curveto(62.16210987,38.94598698)(61.65169321,39.09702864)(60.91210987,39.09702864)
\lineto(60.43554737,39.09702864)
\lineto(60.43554737,40.21421614)
\lineto(60.91210987,40.21421614)
\curveto(61.65690154,40.21421614)(62.16731821,40.36265364)(62.44335987,40.65952864)
\curveto(62.72460987,40.96161198)(62.86523487,41.50067448)(62.86523487,42.27671614)
\lineto(62.86523487,44.15171614)
\curveto(62.86523487,45.30275781)(63.08398487,46.06838281)(63.52148487,46.44859114)
\curveto(63.96419321,46.83400781)(64.83398487,47.02671614)(66.13085987,47.02671614)
\lineto(66.61523487,47.02671614)
\lineto(66.61523487,45.90952864)
\lineto(66.08398487,45.90952864)
\curveto(65.34960987,45.90952864)(64.87044321,45.79494531)(64.64648487,45.56577864)
\curveto(64.42252654,45.33661198)(64.31054737,44.85484114)(64.31054737,44.12046614)
\lineto(64.31054737,42.18296614)
\curveto(64.31054737,41.36525781)(64.19075571,40.77150781)(63.95117237,40.40171614)
\curveto(63.71679737,40.03192448)(63.31315154,39.78192448)(62.74023487,39.65171614)
\curveto(63.31835987,39.51109114)(63.72460987,39.25588281)(63.95898487,38.88609114)
\curveto(64.19335987,38.51629948)(64.31054737,37.92515364)(64.31054737,37.11265364)
\lineto(64.31054737,35.17515364)
\curveto(64.31054737,34.44077864)(64.42252654,33.95900781)(64.64648487,33.72984114)
\curveto(64.87044321,33.50067448)(65.34960987,33.38609114)(66.08398487,33.38609114)
\closepath
}
}
{
\newrgbcolor{curcolor}{0 0 0}
\pscustom[linestyle=none,fillstyle=solid,fillcolor=curcolor]
{
\newpath
\moveto(68.57617237,46.53452864)
\lineto(78.44335987,46.53452864)
\lineto(78.44335987,45.20640364)
\lineto(74.30273487,45.20640364)
\lineto(74.30273487,34.87046614)
\lineto(72.71679737,34.87046614)
\lineto(72.71679737,45.20640364)
\lineto(68.57617237,45.20640364)
\closepath
}
}
{
\newrgbcolor{curcolor}{0 0 0}
\pscustom[linestyle=none,fillstyle=solid,fillcolor=curcolor]
{
\newpath
\moveto(85.66992237,39.34702864)
\curveto(85.66992237,40.38869531)(85.45377654,41.19598698)(85.02148487,41.76890364)
\curveto(84.59440154,42.34182031)(83.99283904,42.62827864)(83.21679737,42.62827864)
\curveto(82.44596404,42.62827864)(81.84440154,42.34182031)(81.41210987,41.76890364)
\curveto(80.98502654,41.19598698)(80.77148487,40.38869531)(80.77148487,39.34702864)
\curveto(80.77148487,38.31057031)(80.98502654,37.50588281)(81.41210987,36.93296614)
\curveto(81.84440154,36.36004948)(82.44596404,36.07359114)(83.21679737,36.07359114)
\curveto(83.99283904,36.07359114)(84.59440154,36.36004948)(85.02148487,36.93296614)
\curveto(85.45377654,37.50588281)(85.66992237,38.31057031)(85.66992237,39.34702864)
\closepath
\moveto(87.10742237,35.95640364)
\curveto(87.10742237,34.46682031)(86.77669321,33.36004948)(86.11523487,32.63609114)
\curveto(85.45377654,31.90692448)(84.44075571,31.54234114)(83.07617237,31.54234114)
\curveto(82.57096404,31.54234114)(82.09440154,31.58140364)(81.64648487,31.65952864)
\curveto(81.19856821,31.73244531)(80.76367237,31.84702864)(80.34179737,32.00327864)
\lineto(80.34179737,33.40171614)
\curveto(80.76367237,33.17254948)(81.18033904,33.00327864)(81.59179737,32.89390364)
\curveto(82.00325571,32.78452864)(82.42252654,32.72984114)(82.84960987,32.72984114)
\curveto(83.79231821,32.72984114)(84.49804737,32.97723698)(84.96679737,33.47202864)
\curveto(85.43554737,33.96161198)(85.66992237,34.70379948)(85.66992237,35.69859114)
\lineto(85.66992237,36.40952864)
\curveto(85.37304737,35.89390364)(84.99283904,35.50848698)(84.52929737,35.25327864)
\curveto(84.06575571,34.99807031)(83.51106821,34.87046614)(82.86523487,34.87046614)
\curveto(81.79231821,34.87046614)(80.92773487,35.27932031)(80.27148487,36.09702864)
\curveto(79.61523487,36.91473698)(79.28710987,37.99807031)(79.28710987,39.34702864)
\curveto(79.28710987,40.70119531)(79.61523487,41.78713281)(80.27148487,42.60484114)
\curveto(80.92773487,43.42254948)(81.79231821,43.83140364)(82.86523487,43.83140364)
\curveto(83.51106821,43.83140364)(84.06575571,43.70379948)(84.52929737,43.44859114)
\curveto(84.99283904,43.19338281)(85.37304737,42.80796614)(85.66992237,42.29234114)
\lineto(85.66992237,43.62046614)
\lineto(87.10742237,43.62046614)
\closepath
}
}
{
\newrgbcolor{curcolor}{0 0 0}
\pscustom[linestyle=none,fillstyle=solid,fillcolor=curcolor]
{
\newpath
\moveto(90.56054737,33.38609114)
\lineto(91.10742237,33.38609114)
\curveto(91.83658904,33.38609114)(92.31054737,33.49807031)(92.52929737,33.72202864)
\curveto(92.75325571,33.94598698)(92.86523487,34.43036198)(92.86523487,35.17515364)
\lineto(92.86523487,37.11265364)
\curveto(92.86523487,37.92515364)(92.98242237,38.51629948)(93.21679737,38.88609114)
\curveto(93.45117237,39.25588281)(93.85742237,39.51109114)(94.43554737,39.65171614)
\curveto(93.85742237,39.78192448)(93.45117237,40.03192448)(93.21679737,40.40171614)
\curveto(92.98242237,40.77150781)(92.86523487,41.36525781)(92.86523487,42.18296614)
\lineto(92.86523487,44.12046614)
\curveto(92.86523487,44.86004948)(92.75325571,45.34182031)(92.52929737,45.56577864)
\curveto(92.31054737,45.79494531)(91.83658904,45.90952864)(91.10742237,45.90952864)
\lineto(90.56054737,45.90952864)
\lineto(90.56054737,47.02671614)
\lineto(91.05273487,47.02671614)
\curveto(92.34960987,47.02671614)(93.21419321,46.83400781)(93.64648487,46.44859114)
\curveto(94.08398487,46.06838281)(94.30273487,45.30275781)(94.30273487,44.15171614)
\lineto(94.30273487,42.27671614)
\curveto(94.30273487,41.50067448)(94.44335987,40.96161198)(94.72460987,40.65952864)
\curveto(95.00585987,40.36265364)(95.51627654,40.21421614)(96.25585987,40.21421614)
\lineto(96.74023487,40.21421614)
\lineto(96.74023487,39.09702864)
\lineto(96.25585987,39.09702864)
\curveto(95.51627654,39.09702864)(95.00585987,38.94598698)(94.72460987,38.64390364)
\curveto(94.44335987,38.34182031)(94.30273487,37.79754948)(94.30273487,37.01109114)
\lineto(94.30273487,35.14390364)
\curveto(94.30273487,33.99286198)(94.08398487,33.22463281)(93.64648487,32.83921614)
\curveto(93.21419321,32.45379948)(92.34960987,32.26109114)(91.05273487,32.26109114)
\lineto(90.56054737,32.26109114)
\closepath
}
}
{
\newrgbcolor{curcolor}{0 0 0}
\pscustom[linestyle=none,fillstyle=solid,fillcolor=curcolor]
{
\newpath
\moveto(395.72558644,173.00176553)
\lineto(405.59277394,173.00176553)
\lineto(405.59277394,171.67364053)
\lineto(401.45214894,171.67364053)
\lineto(401.45214894,161.33770303)
\lineto(399.86621144,161.33770303)
\lineto(399.86621144,171.67364053)
\lineto(395.72558644,171.67364053)
\closepath
}
}
{
\newrgbcolor{curcolor}{0 0 0}
\pscustom[linestyle=none,fillstyle=solid,fillcolor=curcolor]
{
\newpath
\moveto(409.77246144,171.44707803)
\lineto(407.63183644,165.64239053)
\lineto(411.92089894,165.64239053)
\closepath
\moveto(408.88183644,173.00176553)
\lineto(410.67089894,173.00176553)
\lineto(415.11621144,161.33770303)
\lineto(413.47558644,161.33770303)
\lineto(412.41308644,164.32989053)
\lineto(407.15527394,164.32989053)
\lineto(406.09277394,161.33770303)
\lineto(404.42871144,161.33770303)
\closepath
}
}
{
\newrgbcolor{curcolor}{0 0 0}
\pscustom[linestyle=none,fillstyle=solid,fillcolor=curcolor]
{
\newpath
\moveto(413.94433644,173.00176553)
\lineto(423.81152394,173.00176553)
\lineto(423.81152394,171.67364053)
\lineto(419.67089894,171.67364053)
\lineto(419.67089894,161.33770303)
\lineto(418.08496144,161.33770303)
\lineto(418.08496144,171.67364053)
\lineto(413.94433644,171.67364053)
\closepath
}
}
{
\newrgbcolor{curcolor}{0 0 0}
\pscustom[linestyle=none,fillstyle=solid,fillcolor=curcolor]
{
\newpath
\moveto(423.08496144,166.36114053)
\lineto(427.29589894,166.36114053)
\lineto(427.29589894,165.07989053)
\lineto(423.08496144,165.07989053)
\closepath
}
}
{
\newrgbcolor{curcolor}{0 0 0}
\pscustom[linestyle=none,fillstyle=solid,fillcolor=curcolor]
{
\newpath
\moveto(438.20214894,163.00176553)
\lineto(438.20214894,166.13457803)
\lineto(435.62402394,166.13457803)
\lineto(435.62402394,167.43145303)
\lineto(439.76464894,167.43145303)
\lineto(439.76464894,162.42364053)
\curveto(439.15527394,161.99134887)(438.48339894,161.66322387)(437.74902394,161.43926553)
\curveto(437.01464894,161.22051553)(436.23079477,161.11114053)(435.39746144,161.11114053)
\curveto(433.57454477,161.11114053)(432.14746144,161.64239053)(431.11621144,162.70489053)
\curveto(430.09016977,163.77259887)(429.57714894,165.25697387)(429.57714894,167.15801553)
\curveto(429.57714894,169.06426553)(430.09016977,170.54864053)(431.11621144,171.61114053)
\curveto(432.14746144,172.67884887)(433.57454477,173.21270303)(435.39746144,173.21270303)
\curveto(436.1578781,173.21270303)(436.87923227,173.11895303)(437.56152394,172.93145303)
\curveto(438.24902394,172.74395303)(438.88183644,172.46791137)(439.45996144,172.10332803)
\lineto(439.45996144,170.42364053)
\curveto(438.8766281,170.9184322)(438.25683644,171.29082803)(437.60058644,171.54082803)
\curveto(436.94433644,171.79082803)(436.25423227,171.91582803)(435.53027394,171.91582803)
\curveto(434.1031906,171.91582803)(433.03027394,171.51739053)(432.31152394,170.72051553)
\curveto(431.59798227,169.92364053)(431.24121144,168.73614053)(431.24121144,167.15801553)
\curveto(431.24121144,165.58509887)(431.59798227,164.40020303)(432.31152394,163.60332803)
\curveto(433.03027394,162.80645303)(434.1031906,162.40801553)(435.53027394,162.40801553)
\curveto(436.0875656,162.40801553)(436.58496144,162.45489053)(437.02246144,162.54864053)
\curveto(437.45996144,162.64759887)(437.8531906,162.79864053)(438.20214894,163.00176553)
\closepath
}
}
{
\newrgbcolor{curcolor}{0 0 0}
\pscustom[linestyle=none,fillstyle=solid,fillcolor=curcolor]
{
\newpath
\moveto(442.59277394,173.49395303)
\lineto(444.03027394,173.49395303)
\lineto(444.03027394,161.33770303)
\lineto(442.59277394,161.33770303)
\closepath
}
}
{
\newrgbcolor{curcolor}{0 0 0}
\pscustom[linestyle=none,fillstyle=solid,fillcolor=curcolor]
{
\newpath
\moveto(446.88183644,164.79082803)
\lineto(446.88183644,170.08770303)
\lineto(448.31933644,170.08770303)
\lineto(448.31933644,164.84551553)
\curveto(448.31933644,164.01739053)(448.48079477,163.3949947)(448.80371144,162.97832803)
\curveto(449.1266281,162.5668697)(449.6110031,162.36114053)(450.25683644,162.36114053)
\curveto(451.0328781,162.36114053)(451.64485727,162.60853637)(452.09277394,163.10332803)
\curveto(452.54589894,163.5981197)(452.77246144,164.27259887)(452.77246144,165.12676553)
\lineto(452.77246144,170.08770303)
\lineto(454.20996144,170.08770303)
\lineto(454.20996144,161.33770303)
\lineto(452.77246144,161.33770303)
\lineto(452.77246144,162.68145303)
\curveto(452.4235031,162.15020303)(452.0172531,161.7543697)(451.55371144,161.49395303)
\curveto(451.0953781,161.2387447)(450.56152394,161.11114053)(449.95214894,161.11114053)
\curveto(448.9469406,161.11114053)(448.18391977,161.42364053)(447.66308644,162.04864053)
\curveto(447.1422531,162.67364053)(446.88183644,163.58770303)(446.88183644,164.79082803)
\closepath
\moveto(450.49902394,170.29864053)
\closepath
}
}
{
\newrgbcolor{curcolor}{0 0 0}
\pscustom[linestyle=none,fillstyle=solid,fillcolor=curcolor]
{
\newpath
\moveto(457.00683644,173.00176553)
\lineto(461.06933644,159.85332803)
\lineto(459.74121144,159.85332803)
\lineto(455.67871144,173.00176553)
\closepath
}
}
{
\newrgbcolor{curcolor}{0 0 0}
\pscustom[linestyle=none,fillstyle=solid,fillcolor=curcolor]
{
\newpath
\moveto(464.01464894,172.57207803)
\lineto(464.01464894,170.08770303)
\lineto(466.97558644,170.08770303)
\lineto(466.97558644,168.97051553)
\lineto(464.01464894,168.97051553)
\lineto(464.01464894,164.22051553)
\curveto(464.01464894,163.50697387)(464.1110031,163.04864053)(464.30371144,162.84551553)
\curveto(464.5016281,162.64239053)(464.9000656,162.54082803)(465.49902394,162.54082803)
\lineto(466.97558644,162.54082803)
\lineto(466.97558644,161.33770303)
\lineto(465.49902394,161.33770303)
\curveto(464.38964894,161.33770303)(463.62402394,161.5434322)(463.20214894,161.95489053)
\curveto(462.78027394,162.3715572)(462.56933644,163.12676553)(462.56933644,164.22051553)
\lineto(462.56933644,168.97051553)
\lineto(461.51464894,168.97051553)
\lineto(461.51464894,170.08770303)
\lineto(462.56933644,170.08770303)
\lineto(462.56933644,172.57207803)
\closepath
}
}
{
\newrgbcolor{curcolor}{0 0 0}
\pscustom[linestyle=none,fillstyle=solid,fillcolor=curcolor]
{
\newpath
\moveto(476.35839894,166.07207803)
\lineto(476.35839894,165.36895303)
\lineto(469.74902394,165.36895303)
\curveto(469.81152394,164.3793697)(470.10839894,163.62416137)(470.63964894,163.10332803)
\curveto(471.17610727,162.58770303)(471.92089894,162.32989053)(472.87402394,162.32989053)
\curveto(473.42610727,162.32989053)(473.95996144,162.39759887)(474.47558644,162.53301553)
\curveto(474.99641977,162.6684322)(475.51204477,162.8715572)(476.02246144,163.14239053)
\lineto(476.02246144,161.78301553)
\curveto(475.50683644,161.56426553)(474.9781906,161.39759887)(474.43652394,161.28301553)
\curveto(473.89485727,161.1684322)(473.3453781,161.11114053)(472.78808644,161.11114053)
\curveto(471.3922531,161.11114053)(470.28548227,161.51739053)(469.46777394,162.32989053)
\curveto(468.65527394,163.14239053)(468.24902394,164.24134887)(468.24902394,165.62676553)
\curveto(468.24902394,167.0590572)(468.6344406,168.19447387)(469.40527394,169.03301553)
\curveto(470.1813156,169.87676553)(471.22558644,170.29864053)(472.53808644,170.29864053)
\curveto(473.71516977,170.29864053)(474.64485727,169.9184322)(475.32714894,169.15801553)
\curveto(476.01464894,168.4028072)(476.35839894,167.37416137)(476.35839894,166.07207803)
\closepath
\moveto(474.92089894,166.49395303)
\curveto(474.91048227,167.28041137)(474.6891281,167.90801553)(474.25683644,168.37676553)
\curveto(473.8297531,168.84551553)(473.26204477,169.07989053)(472.55371144,169.07989053)
\curveto(471.7516281,169.07989053)(471.10839894,168.85332803)(470.62402394,168.40020303)
\curveto(470.14485727,167.94707803)(469.8688156,167.3090572)(469.79589894,166.48614053)
\closepath
}
}
{
\newrgbcolor{curcolor}{0 0 0}
\pscustom[linestyle=none,fillstyle=solid,fillcolor=curcolor]
{
\newpath
\moveto(485.70996144,170.08770303)
\lineto(482.54589894,165.82989053)
\lineto(485.87402394,161.33770303)
\lineto(484.17871144,161.33770303)
\lineto(481.63183644,164.77520303)
\lineto(479.08496144,161.33770303)
\lineto(477.38964894,161.33770303)
\lineto(480.78808644,165.91582803)
\lineto(477.67871144,170.08770303)
\lineto(479.37402394,170.08770303)
\lineto(481.69433644,166.97051553)
\lineto(484.01464894,170.08770303)
\closepath
}
}
{
\newrgbcolor{curcolor}{0 0 0}
\pscustom[linestyle=none,fillstyle=solid,fillcolor=curcolor]
{
\newpath
\moveto(489.32714894,172.57207803)
\lineto(489.32714894,170.08770303)
\lineto(492.28808644,170.08770303)
\lineto(492.28808644,168.97051553)
\lineto(489.32714894,168.97051553)
\lineto(489.32714894,164.22051553)
\curveto(489.32714894,163.50697387)(489.4235031,163.04864053)(489.61621144,162.84551553)
\curveto(489.8141281,162.64239053)(490.2125656,162.54082803)(490.81152394,162.54082803)
\lineto(492.28808644,162.54082803)
\lineto(492.28808644,161.33770303)
\lineto(490.81152394,161.33770303)
\curveto(489.70214894,161.33770303)(488.93652394,161.5434322)(488.51464894,161.95489053)
\curveto(488.09277394,162.3715572)(487.88183644,163.12676553)(487.88183644,164.22051553)
\lineto(487.88183644,168.97051553)
\lineto(486.82714894,168.97051553)
\lineto(486.82714894,170.08770303)
\lineto(487.88183644,170.08770303)
\lineto(487.88183644,172.57207803)
\closepath
}
}
{
\newrgbcolor{curcolor}{0 0 0}
\pscustom[linestyle=none,fillstyle=solid,fillcolor=curcolor]
{
\newpath
\moveto(499.76464894,169.82989053)
\lineto(499.76464894,168.47051553)
\curveto(499.35839894,168.67884887)(498.93652394,168.83509887)(498.49902394,168.93926553)
\curveto(498.06152394,169.0434322)(497.60839894,169.09551553)(497.13964894,169.09551553)
\curveto(496.42610727,169.09551553)(495.88964894,168.98614053)(495.53027394,168.76739053)
\curveto(495.17610727,168.54864053)(494.99902394,168.22051553)(494.99902394,167.78301553)
\curveto(494.99902394,167.4496822)(495.1266281,167.18666137)(495.38183644,166.99395303)
\curveto(495.63704477,166.80645303)(496.1500656,166.62676553)(496.92089894,166.45489053)
\lineto(497.41308644,166.34551553)
\curveto(498.43391977,166.12676553)(499.1578781,165.8168697)(499.58496144,165.41582803)
\curveto(500.0172531,165.0199947)(500.23339894,164.4653072)(500.23339894,163.75176553)
\curveto(500.23339894,162.93926553)(499.91048227,162.29603637)(499.26464894,161.82207803)
\curveto(498.62402394,161.3481197)(497.74121144,161.11114053)(496.61621144,161.11114053)
\curveto(496.14746144,161.11114053)(495.6578781,161.15801553)(495.14746144,161.25176553)
\curveto(494.6422531,161.3403072)(494.10839894,161.47572387)(493.54589894,161.65801553)
\lineto(493.54589894,163.14239053)
\curveto(494.07714894,162.86634887)(494.60058644,162.65801553)(495.11621144,162.51739053)
\curveto(495.63183644,162.38197387)(496.1422531,162.31426553)(496.64746144,162.31426553)
\curveto(497.32454477,162.31426553)(497.8453781,162.42884887)(498.20996144,162.65801553)
\curveto(498.57454477,162.89239053)(498.75683644,163.22051553)(498.75683644,163.64239053)
\curveto(498.75683644,164.03301553)(498.62402394,164.3324947)(498.35839894,164.54082803)
\curveto(498.09798227,164.74916137)(497.52246144,164.9496822)(496.63183644,165.14239053)
\lineto(496.13183644,165.25957803)
\curveto(495.24121144,165.44707803)(494.59798227,165.73353637)(494.20214894,166.11895303)
\curveto(493.8063156,166.50957803)(493.60839894,167.0434322)(493.60839894,167.72051553)
\curveto(493.60839894,168.5434322)(493.9000656,169.17884887)(494.48339894,169.62676553)
\curveto(495.06673227,170.0746822)(495.89485727,170.29864053)(496.96777394,170.29864053)
\curveto(497.49902394,170.29864053)(497.99902394,170.25957803)(498.46777394,170.18145303)
\curveto(498.93652394,170.10332803)(499.3688156,169.98614053)(499.76464894,169.82989053)
\closepath
}
}
{
\newrgbcolor{curcolor}{0 0 0}
\pscustom[linestyle=none,fillstyle=solid,fillcolor=curcolor]
{
\newpath
\moveto(502.38183644,164.79082803)
\lineto(502.38183644,170.08770303)
\lineto(503.81933644,170.08770303)
\lineto(503.81933644,164.84551553)
\curveto(503.81933644,164.01739053)(503.98079477,163.3949947)(504.30371144,162.97832803)
\curveto(504.6266281,162.5668697)(505.1110031,162.36114053)(505.75683644,162.36114053)
\curveto(506.5328781,162.36114053)(507.14485727,162.60853637)(507.59277394,163.10332803)
\curveto(508.04589894,163.5981197)(508.27246144,164.27259887)(508.27246144,165.12676553)
\lineto(508.27246144,170.08770303)
\lineto(509.70996144,170.08770303)
\lineto(509.70996144,161.33770303)
\lineto(508.27246144,161.33770303)
\lineto(508.27246144,162.68145303)
\curveto(507.9235031,162.15020303)(507.5172531,161.7543697)(507.05371144,161.49395303)
\curveto(506.5953781,161.2387447)(506.06152394,161.11114053)(505.45214894,161.11114053)
\curveto(504.4469406,161.11114053)(503.68391977,161.42364053)(503.16308644,162.04864053)
\curveto(502.6422531,162.67364053)(502.38183644,163.58770303)(502.38183644,164.79082803)
\closepath
\moveto(505.99902394,170.29864053)
\closepath
}
}
{
\newrgbcolor{curcolor}{0 0 0}
\pscustom[linestyle=none,fillstyle=solid,fillcolor=curcolor]
{
\newpath
\moveto(518.96777394,165.70489053)
\curveto(518.96777394,166.7621822)(518.74902394,167.5903072)(518.31152394,168.18926553)
\curveto(517.87923227,168.7934322)(517.2828781,169.09551553)(516.52246144,169.09551553)
\curveto(515.76204477,169.09551553)(515.16308644,168.7934322)(514.72558644,168.18926553)
\curveto(514.29329477,167.5903072)(514.07714894,166.7621822)(514.07714894,165.70489053)
\curveto(514.07714894,164.64759887)(514.29329477,163.8168697)(514.72558644,163.21270303)
\curveto(515.16308644,162.6137447)(515.76204477,162.31426553)(516.52246144,162.31426553)
\curveto(517.2828781,162.31426553)(517.87923227,162.6137447)(518.31152394,163.21270303)
\curveto(518.74902394,163.8168697)(518.96777394,164.64759887)(518.96777394,165.70489053)
\closepath
\moveto(514.07714894,168.75957803)
\curveto(514.37923227,169.28041137)(514.7594406,169.66582803)(515.21777394,169.91582803)
\curveto(515.6813156,170.17103637)(516.23339894,170.29864053)(516.87402394,170.29864053)
\curveto(517.93652394,170.29864053)(518.7985031,169.87676553)(519.45996144,169.03301553)
\curveto(520.1266281,168.18926553)(520.45996144,167.07989053)(520.45996144,165.70489053)
\curveto(520.45996144,164.32989053)(520.1266281,163.22051553)(519.45996144,162.37676553)
\curveto(518.7985031,161.53301553)(517.93652394,161.11114053)(516.87402394,161.11114053)
\curveto(516.23339894,161.11114053)(515.6813156,161.23614053)(515.21777394,161.48614053)
\curveto(514.7594406,161.74134887)(514.37923227,162.1293697)(514.07714894,162.65020303)
\lineto(514.07714894,161.33770303)
\lineto(512.63183644,161.33770303)
\lineto(512.63183644,173.49395303)
\lineto(514.07714894,173.49395303)
\closepath
}
}
{
\newrgbcolor{curcolor}{0 0 0}
\pscustom[linestyle=none,fillstyle=solid,fillcolor=curcolor]
{
\newpath
\moveto(528.42089894,169.82989053)
\lineto(528.42089894,168.47051553)
\curveto(528.01464894,168.67884887)(527.59277394,168.83509887)(527.15527394,168.93926553)
\curveto(526.71777394,169.0434322)(526.26464894,169.09551553)(525.79589894,169.09551553)
\curveto(525.08235727,169.09551553)(524.54589894,168.98614053)(524.18652394,168.76739053)
\curveto(523.83235727,168.54864053)(523.65527394,168.22051553)(523.65527394,167.78301553)
\curveto(523.65527394,167.4496822)(523.7828781,167.18666137)(524.03808644,166.99395303)
\curveto(524.29329477,166.80645303)(524.8063156,166.62676553)(525.57714894,166.45489053)
\lineto(526.06933644,166.34551553)
\curveto(527.09016977,166.12676553)(527.8141281,165.8168697)(528.24121144,165.41582803)
\curveto(528.6735031,165.0199947)(528.88964894,164.4653072)(528.88964894,163.75176553)
\curveto(528.88964894,162.93926553)(528.56673227,162.29603637)(527.92089894,161.82207803)
\curveto(527.28027394,161.3481197)(526.39746144,161.11114053)(525.27246144,161.11114053)
\curveto(524.80371144,161.11114053)(524.3141281,161.15801553)(523.80371144,161.25176553)
\curveto(523.2985031,161.3403072)(522.76464894,161.47572387)(522.20214894,161.65801553)
\lineto(522.20214894,163.14239053)
\curveto(522.73339894,162.86634887)(523.25683644,162.65801553)(523.77246144,162.51739053)
\curveto(524.28808644,162.38197387)(524.7985031,162.31426553)(525.30371144,162.31426553)
\curveto(525.98079477,162.31426553)(526.5016281,162.42884887)(526.86621144,162.65801553)
\curveto(527.23079477,162.89239053)(527.41308644,163.22051553)(527.41308644,163.64239053)
\curveto(527.41308644,164.03301553)(527.28027394,164.3324947)(527.01464894,164.54082803)
\curveto(526.75423227,164.74916137)(526.17871144,164.9496822)(525.28808644,165.14239053)
\lineto(524.78808644,165.25957803)
\curveto(523.89746144,165.44707803)(523.25423227,165.73353637)(522.85839894,166.11895303)
\curveto(522.4625656,166.50957803)(522.26464894,167.0434322)(522.26464894,167.72051553)
\curveto(522.26464894,168.5434322)(522.5563156,169.17884887)(523.13964894,169.62676553)
\curveto(523.72298227,170.0746822)(524.55110727,170.29864053)(525.62402394,170.29864053)
\curveto(526.15527394,170.29864053)(526.65527394,170.25957803)(527.12402394,170.18145303)
\curveto(527.59277394,170.10332803)(528.0250656,169.98614053)(528.42089894,169.82989053)
\closepath
}
}
{
\newrgbcolor{curcolor}{0 0 0}
\pscustom[linestyle=none,fillstyle=solid,fillcolor=curcolor]
{
\newpath
\moveto(537.48339894,169.75176553)
\lineto(537.48339894,168.40801553)
\curveto(537.07714894,168.63197387)(536.66829477,168.79864053)(536.25683644,168.90801553)
\curveto(535.85058644,169.02259887)(535.4391281,169.07989053)(535.02246144,169.07989053)
\curveto(534.09016977,169.07989053)(533.36621144,168.78301553)(532.85058644,168.18926553)
\curveto(532.33496144,167.60072387)(532.07714894,166.77259887)(532.07714894,165.70489053)
\curveto(532.07714894,164.6371822)(532.33496144,163.80645303)(532.85058644,163.21270303)
\curveto(533.36621144,162.62416137)(534.09016977,162.32989053)(535.02246144,162.32989053)
\curveto(535.4391281,162.32989053)(535.85058644,162.38457803)(536.25683644,162.49395303)
\curveto(536.66829477,162.60853637)(537.07714894,162.7778072)(537.48339894,163.00176553)
\lineto(537.48339894,161.67364053)
\curveto(537.08235727,161.48614053)(536.6656906,161.34551553)(536.23339894,161.25176553)
\curveto(535.8063156,161.15801553)(535.35058644,161.11114053)(534.86621144,161.11114053)
\curveto(533.5485031,161.11114053)(532.5016281,161.52520303)(531.72558644,162.35332803)
\curveto(530.94954477,163.18145303)(530.56152394,164.29864053)(530.56152394,165.70489053)
\curveto(530.56152394,167.13197387)(530.95214894,168.2543697)(531.73339894,169.07207803)
\curveto(532.51985727,169.88978637)(533.5953781,170.29864053)(534.95996144,170.29864053)
\curveto(535.40266977,170.29864053)(535.83496144,170.25176553)(536.25683644,170.15801553)
\curveto(536.67871144,170.06947387)(537.0875656,169.9340572)(537.48339894,169.75176553)
\closepath
}
}
{
\newrgbcolor{curcolor}{0 0 0}
\pscustom[linestyle=none,fillstyle=solid,fillcolor=curcolor]
{
\newpath
\moveto(545.06933644,168.74395303)
\curveto(544.9078781,168.83770303)(544.73079477,168.90541137)(544.53808644,168.94707803)
\curveto(544.35058644,168.99395303)(544.1422531,169.01739053)(543.91308644,169.01739053)
\curveto(543.10058644,169.01739053)(542.47558644,168.75176553)(542.03808644,168.22051553)
\curveto(541.60579477,167.69447387)(541.38964894,166.93666137)(541.38964894,165.94707803)
\lineto(541.38964894,161.33770303)
\lineto(539.94433644,161.33770303)
\lineto(539.94433644,170.08770303)
\lineto(541.38964894,170.08770303)
\lineto(541.38964894,168.72832803)
\curveto(541.69173227,169.25957803)(542.08496144,169.6528072)(542.56933644,169.90801553)
\curveto(543.05371144,170.1684322)(543.6422531,170.29864053)(544.33496144,170.29864053)
\curveto(544.43391977,170.29864053)(544.54329477,170.29082803)(544.66308644,170.27520303)
\curveto(544.7828781,170.26478637)(544.9156906,170.2465572)(545.06152394,170.22051553)
\closepath
}
}
{
\newrgbcolor{curcolor}{0 0 0}
\pscustom[linestyle=none,fillstyle=solid,fillcolor=curcolor]
{
\newpath
\moveto(546.59277394,170.08770303)
\lineto(548.03027394,170.08770303)
\lineto(548.03027394,161.33770303)
\lineto(546.59277394,161.33770303)
\closepath
\moveto(546.59277394,173.49395303)
\lineto(548.03027394,173.49395303)
\lineto(548.03027394,171.67364053)
\lineto(546.59277394,171.67364053)
\closepath
}
}
{
\newrgbcolor{curcolor}{0 0 0}
\pscustom[linestyle=none,fillstyle=solid,fillcolor=curcolor]
{
\newpath
\moveto(552.42089894,162.65020303)
\lineto(552.42089894,158.00957803)
\lineto(550.97558644,158.00957803)
\lineto(550.97558644,170.08770303)
\lineto(552.42089894,170.08770303)
\lineto(552.42089894,168.75957803)
\curveto(552.72298227,169.28041137)(553.1031906,169.66582803)(553.56152394,169.91582803)
\curveto(554.0250656,170.17103637)(554.57714894,170.29864053)(555.21777394,170.29864053)
\curveto(556.28027394,170.29864053)(557.1422531,169.87676553)(557.80371144,169.03301553)
\curveto(558.4703781,168.18926553)(558.80371144,167.07989053)(558.80371144,165.70489053)
\curveto(558.80371144,164.32989053)(558.4703781,163.22051553)(557.80371144,162.37676553)
\curveto(557.1422531,161.53301553)(556.28027394,161.11114053)(555.21777394,161.11114053)
\curveto(554.57714894,161.11114053)(554.0250656,161.23614053)(553.56152394,161.48614053)
\curveto(553.1031906,161.74134887)(552.72298227,162.1293697)(552.42089894,162.65020303)
\closepath
\moveto(557.31152394,165.70489053)
\curveto(557.31152394,166.7621822)(557.09277394,167.5903072)(556.65527394,168.18926553)
\curveto(556.22298227,168.7934322)(555.6266281,169.09551553)(554.86621144,169.09551553)
\curveto(554.10579477,169.09551553)(553.50683644,168.7934322)(553.06933644,168.18926553)
\curveto(552.63704477,167.5903072)(552.42089894,166.7621822)(552.42089894,165.70489053)
\curveto(552.42089894,164.64759887)(552.63704477,163.8168697)(553.06933644,163.21270303)
\curveto(553.50683644,162.6137447)(554.10579477,162.31426553)(554.86621144,162.31426553)
\curveto(555.6266281,162.31426553)(556.22298227,162.6137447)(556.65527394,163.21270303)
\curveto(557.09277394,163.8168697)(557.31152394,164.64759887)(557.31152394,165.70489053)
\closepath
}
}
{
\newrgbcolor{curcolor}{0 0 0}
\pscustom[linestyle=none,fillstyle=solid,fillcolor=curcolor]
{
\newpath
\moveto(562.60839894,172.57207803)
\lineto(562.60839894,170.08770303)
\lineto(565.56933644,170.08770303)
\lineto(565.56933644,168.97051553)
\lineto(562.60839894,168.97051553)
\lineto(562.60839894,164.22051553)
\curveto(562.60839894,163.50697387)(562.7047531,163.04864053)(562.89746144,162.84551553)
\curveto(563.0953781,162.64239053)(563.4938156,162.54082803)(564.09277394,162.54082803)
\lineto(565.56933644,162.54082803)
\lineto(565.56933644,161.33770303)
\lineto(564.09277394,161.33770303)
\curveto(562.98339894,161.33770303)(562.21777394,161.5434322)(561.79589894,161.95489053)
\curveto(561.37402394,162.3715572)(561.16308644,163.12676553)(561.16308644,164.22051553)
\lineto(561.16308644,168.97051553)
\lineto(560.10839894,168.97051553)
\lineto(560.10839894,170.08770303)
\lineto(561.16308644,170.08770303)
\lineto(561.16308644,172.57207803)
\closepath
}
}
{
\newrgbcolor{curcolor}{0 0 0}
\pscustom[linestyle=none,fillstyle=solid,fillcolor=curcolor]
{
\newpath
\moveto(574.13964894,159.85332803)
\lineto(574.13964894,158.72832803)
\lineto(573.65527394,158.72832803)
\curveto(572.35839894,158.72832803)(571.48860727,158.92103637)(571.04589894,159.30645303)
\curveto(570.60839894,159.6918697)(570.38964894,160.46009887)(570.38964894,161.61114053)
\lineto(570.38964894,163.47832803)
\curveto(570.38964894,164.26478637)(570.24902394,164.8090572)(569.96777394,165.11114053)
\curveto(569.68652394,165.41322387)(569.17610727,165.56426553)(568.43652394,165.56426553)
\lineto(567.95996144,165.56426553)
\lineto(567.95996144,166.68145303)
\lineto(568.43652394,166.68145303)
\curveto(569.1813156,166.68145303)(569.69173227,166.82989053)(569.96777394,167.12676553)
\curveto(570.24902394,167.42884887)(570.38964894,167.96791137)(570.38964894,168.74395303)
\lineto(570.38964894,170.61895303)
\curveto(570.38964894,171.7699947)(570.60839894,172.5356197)(571.04589894,172.91582803)
\curveto(571.48860727,173.3012447)(572.35839894,173.49395303)(573.65527394,173.49395303)
\lineto(574.13964894,173.49395303)
\lineto(574.13964894,172.37676553)
\lineto(573.60839894,172.37676553)
\curveto(572.87402394,172.37676553)(572.39485727,172.2621822)(572.17089894,172.03301553)
\curveto(571.9469406,171.80384887)(571.83496144,171.32207803)(571.83496144,170.58770303)
\lineto(571.83496144,168.65020303)
\curveto(571.83496144,167.8324947)(571.71516977,167.2387447)(571.47558644,166.86895303)
\curveto(571.24121144,166.49916137)(570.8375656,166.24916137)(570.26464894,166.11895303)
\curveto(570.84277394,165.97832803)(571.24902394,165.7231197)(571.48339894,165.35332803)
\curveto(571.71777394,164.98353637)(571.83496144,164.39239053)(571.83496144,163.57989053)
\lineto(571.83496144,161.64239053)
\curveto(571.83496144,160.90801553)(571.9469406,160.4262447)(572.17089894,160.19707803)
\curveto(572.39485727,159.96791137)(572.87402394,159.85332803)(573.60839894,159.85332803)
\closepath
}
}
{
\newrgbcolor{curcolor}{0 0 0}
\pscustom[linestyle=none,fillstyle=solid,fillcolor=curcolor]
{
\newpath
\moveto(582.63964894,167.62676553)
\curveto(583.39485727,167.4653072)(583.98339894,167.1293697)(584.40527394,166.61895303)
\curveto(584.83235727,166.10853637)(585.04589894,165.47832803)(585.04589894,164.72832803)
\curveto(585.04589894,163.57728637)(584.6500656,162.68666137)(583.85839894,162.05645303)
\curveto(583.06673227,161.4262447)(581.94173227,161.11114053)(580.48339894,161.11114053)
\curveto(579.9938156,161.11114053)(579.48860727,161.1606197)(578.96777394,161.25957803)
\curveto(578.45214894,161.35332803)(577.91829477,161.4965572)(577.36621144,161.68926553)
\lineto(577.36621144,163.21270303)
\curveto(577.80371144,162.9574947)(578.2828781,162.76478637)(578.80371144,162.63457803)
\curveto(579.32454477,162.5043697)(579.8688156,162.43926553)(580.43652394,162.43926553)
\curveto(581.42610727,162.43926553)(582.17871144,162.63457803)(582.69433644,163.02520303)
\curveto(583.21516977,163.41582803)(583.47558644,163.98353637)(583.47558644,164.72832803)
\curveto(583.47558644,165.41582803)(583.23339894,165.95228637)(582.74902394,166.33770303)
\curveto(582.26985727,166.72832803)(581.60058644,166.92364053)(580.74121144,166.92364053)
\lineto(579.38183644,166.92364053)
\lineto(579.38183644,168.22051553)
\lineto(580.80371144,168.22051553)
\curveto(581.5797531,168.22051553)(582.1735031,168.37416137)(582.58496144,168.68145303)
\curveto(582.99641977,168.99395303)(583.20214894,169.4418697)(583.20214894,170.02520303)
\curveto(583.20214894,170.62416137)(582.98860727,171.0824947)(582.56152394,171.40020303)
\curveto(582.13964894,171.7231197)(581.5328781,171.88457803)(580.74121144,171.88457803)
\curveto(580.30891977,171.88457803)(579.8453781,171.83770303)(579.35058644,171.74395303)
\curveto(578.85579477,171.65020303)(578.31152394,171.5043697)(577.71777394,171.30645303)
\lineto(577.71777394,172.71270303)
\curveto(578.31673227,172.8793697)(578.8766281,173.0043697)(579.39746144,173.08770303)
\curveto(579.9235031,173.17103637)(580.41829477,173.21270303)(580.88183644,173.21270303)
\curveto(582.0797531,173.21270303)(583.02766977,172.93926553)(583.72558644,172.39239053)
\curveto(584.4235031,171.85072387)(584.77246144,171.11634887)(584.77246144,170.18926553)
\curveto(584.77246144,169.5434322)(584.5875656,168.9965572)(584.21777394,168.54864053)
\curveto(583.84798227,168.1059322)(583.3219406,167.79864053)(582.63964894,167.62676553)
\closepath
}
}
{
\newrgbcolor{curcolor}{0 0 0}
\pscustom[linestyle=none,fillstyle=solid,fillcolor=curcolor]
{
\newpath
\moveto(586.30371144,173.00176553)
\lineto(587.99902394,173.00176553)
\lineto(591.23339894,168.20489053)
\lineto(594.44433644,173.00176553)
\lineto(596.13964894,173.00176553)
\lineto(592.01464894,166.89239053)
\lineto(592.01464894,161.33770303)
\lineto(590.42871144,161.33770303)
\lineto(590.42871144,166.89239053)
\closepath
}
}
{
\newrgbcolor{curcolor}{0 0 0}
\pscustom[linestyle=none,fillstyle=solid,fillcolor=curcolor]
{
\newpath
\moveto(598.11621144,159.85332803)
\lineto(598.66308644,159.85332803)
\curveto(599.3922531,159.85332803)(599.86621144,159.9653072)(600.08496144,160.18926553)
\curveto(600.30891977,160.41322387)(600.42089894,160.89759887)(600.42089894,161.64239053)
\lineto(600.42089894,163.57989053)
\curveto(600.42089894,164.39239053)(600.53808644,164.98353637)(600.77246144,165.35332803)
\curveto(601.00683644,165.7231197)(601.41308644,165.97832803)(601.99121144,166.11895303)
\curveto(601.41308644,166.24916137)(601.00683644,166.49916137)(600.77246144,166.86895303)
\curveto(600.53808644,167.2387447)(600.42089894,167.8324947)(600.42089894,168.65020303)
\lineto(600.42089894,170.58770303)
\curveto(600.42089894,171.32728637)(600.30891977,171.8090572)(600.08496144,172.03301553)
\curveto(599.86621144,172.2621822)(599.3922531,172.37676553)(598.66308644,172.37676553)
\lineto(598.11621144,172.37676553)
\lineto(598.11621144,173.49395303)
\lineto(598.60839894,173.49395303)
\curveto(599.90527394,173.49395303)(600.76985727,173.3012447)(601.20214894,172.91582803)
\curveto(601.63964894,172.5356197)(601.85839894,171.7699947)(601.85839894,170.61895303)
\lineto(601.85839894,168.74395303)
\curveto(601.85839894,167.96791137)(601.99902394,167.42884887)(602.28027394,167.12676553)
\curveto(602.56152394,166.82989053)(603.0719406,166.68145303)(603.81152394,166.68145303)
\lineto(604.29589894,166.68145303)
\lineto(604.29589894,165.56426553)
\lineto(603.81152394,165.56426553)
\curveto(603.0719406,165.56426553)(602.56152394,165.41322387)(602.28027394,165.11114053)
\curveto(601.99902394,164.8090572)(601.85839894,164.26478637)(601.85839894,163.47832803)
\lineto(601.85839894,161.61114053)
\curveto(601.85839894,160.46009887)(601.63964894,159.6918697)(601.20214894,159.30645303)
\curveto(600.76985727,158.92103637)(599.90527394,158.72832803)(598.60839894,158.72832803)
\lineto(598.11621144,158.72832803)
\closepath
}
}
{
\newrgbcolor{curcolor}{0 0 0}
\pscustom[linestyle=none,fillstyle=solid,fillcolor=curcolor]
{
\newpath
\moveto(614.29589894,162.65020303)
\lineto(614.29589894,158.00957803)
\lineto(612.85058644,158.00957803)
\lineto(612.85058644,170.08770303)
\lineto(614.29589894,170.08770303)
\lineto(614.29589894,168.75957803)
\curveto(614.59798227,169.28041137)(614.9781906,169.66582803)(615.43652394,169.91582803)
\curveto(615.9000656,170.17103637)(616.45214894,170.29864053)(617.09277394,170.29864053)
\curveto(618.15527394,170.29864053)(619.0172531,169.87676553)(619.67871144,169.03301553)
\curveto(620.3453781,168.18926553)(620.67871144,167.07989053)(620.67871144,165.70489053)
\curveto(620.67871144,164.32989053)(620.3453781,163.22051553)(619.67871144,162.37676553)
\curveto(619.0172531,161.53301553)(618.15527394,161.11114053)(617.09277394,161.11114053)
\curveto(616.45214894,161.11114053)(615.9000656,161.23614053)(615.43652394,161.48614053)
\curveto(614.9781906,161.74134887)(614.59798227,162.1293697)(614.29589894,162.65020303)
\closepath
\moveto(619.18652394,165.70489053)
\curveto(619.18652394,166.7621822)(618.96777394,167.5903072)(618.53027394,168.18926553)
\curveto(618.09798227,168.7934322)(617.5016281,169.09551553)(616.74121144,169.09551553)
\curveto(615.98079477,169.09551553)(615.38183644,168.7934322)(614.94433644,168.18926553)
\curveto(614.51204477,167.5903072)(614.29589894,166.7621822)(614.29589894,165.70489053)
\curveto(614.29589894,164.64759887)(614.51204477,163.8168697)(614.94433644,163.21270303)
\curveto(615.38183644,162.6137447)(615.98079477,162.31426553)(616.74121144,162.31426553)
\curveto(617.5016281,162.31426553)(618.09798227,162.6137447)(618.53027394,163.21270303)
\curveto(618.96777394,163.8168697)(619.18652394,164.64759887)(619.18652394,165.70489053)
\closepath
}
}
{
\newrgbcolor{curcolor}{0 0 0}
\pscustom[linestyle=none,fillstyle=solid,fillcolor=curcolor]
{
\newpath
\moveto(630.54589894,166.07207803)
\lineto(630.54589894,165.36895303)
\lineto(623.93652394,165.36895303)
\curveto(623.99902394,164.3793697)(624.29589894,163.62416137)(624.82714894,163.10332803)
\curveto(625.36360727,162.58770303)(626.10839894,162.32989053)(627.06152394,162.32989053)
\curveto(627.61360727,162.32989053)(628.14746144,162.39759887)(628.66308644,162.53301553)
\curveto(629.18391977,162.6684322)(629.69954477,162.8715572)(630.20996144,163.14239053)
\lineto(630.20996144,161.78301553)
\curveto(629.69433644,161.56426553)(629.1656906,161.39759887)(628.62402394,161.28301553)
\curveto(628.08235727,161.1684322)(627.5328781,161.11114053)(626.97558644,161.11114053)
\curveto(625.5797531,161.11114053)(624.47298227,161.51739053)(623.65527394,162.32989053)
\curveto(622.84277394,163.14239053)(622.43652394,164.24134887)(622.43652394,165.62676553)
\curveto(622.43652394,167.0590572)(622.8219406,168.19447387)(623.59277394,169.03301553)
\curveto(624.3688156,169.87676553)(625.41308644,170.29864053)(626.72558644,170.29864053)
\curveto(627.90266977,170.29864053)(628.83235727,169.9184322)(629.51464894,169.15801553)
\curveto(630.20214894,168.4028072)(630.54589894,167.37416137)(630.54589894,166.07207803)
\closepath
\moveto(629.10839894,166.49395303)
\curveto(629.09798227,167.28041137)(628.8766281,167.90801553)(628.44433644,168.37676553)
\curveto(628.0172531,168.84551553)(627.44954477,169.07989053)(626.74121144,169.07989053)
\curveto(625.9391281,169.07989053)(625.29589894,168.85332803)(624.81152394,168.40020303)
\curveto(624.33235727,167.94707803)(624.0563156,167.3090572)(623.98339894,166.48614053)
\closepath
}
}
{
\newrgbcolor{curcolor}{0 0 0}
\pscustom[linestyle=none,fillstyle=solid,fillcolor=curcolor]
{
\newpath
\moveto(634.29589894,162.65020303)
\lineto(634.29589894,158.00957803)
\lineto(632.85058644,158.00957803)
\lineto(632.85058644,170.08770303)
\lineto(634.29589894,170.08770303)
\lineto(634.29589894,168.75957803)
\curveto(634.59798227,169.28041137)(634.9781906,169.66582803)(635.43652394,169.91582803)
\curveto(635.9000656,170.17103637)(636.45214894,170.29864053)(637.09277394,170.29864053)
\curveto(638.15527394,170.29864053)(639.0172531,169.87676553)(639.67871144,169.03301553)
\curveto(640.3453781,168.18926553)(640.67871144,167.07989053)(640.67871144,165.70489053)
\curveto(640.67871144,164.32989053)(640.3453781,163.22051553)(639.67871144,162.37676553)
\curveto(639.0172531,161.53301553)(638.15527394,161.11114053)(637.09277394,161.11114053)
\curveto(636.45214894,161.11114053)(635.9000656,161.23614053)(635.43652394,161.48614053)
\curveto(634.9781906,161.74134887)(634.59798227,162.1293697)(634.29589894,162.65020303)
\closepath
\moveto(639.18652394,165.70489053)
\curveto(639.18652394,166.7621822)(638.96777394,167.5903072)(638.53027394,168.18926553)
\curveto(638.09798227,168.7934322)(637.5016281,169.09551553)(636.74121144,169.09551553)
\curveto(635.98079477,169.09551553)(635.38183644,168.7934322)(634.94433644,168.18926553)
\curveto(634.51204477,167.5903072)(634.29589894,166.7621822)(634.29589894,165.70489053)
\curveto(634.29589894,164.64759887)(634.51204477,163.8168697)(634.94433644,163.21270303)
\curveto(635.38183644,162.6137447)(635.98079477,162.31426553)(636.74121144,162.31426553)
\curveto(637.5016281,162.31426553)(638.09798227,162.6137447)(638.53027394,163.21270303)
\curveto(638.96777394,163.8168697)(639.18652394,164.64759887)(639.18652394,165.70489053)
\closepath
}
}
{
\newrgbcolor{curcolor}{0 0 0}
\pscustom[linestyle=none,fillstyle=solid,fillcolor=curcolor]
{
\newpath
\moveto(644.48339894,172.57207803)
\lineto(644.48339894,170.08770303)
\lineto(647.44433644,170.08770303)
\lineto(647.44433644,168.97051553)
\lineto(644.48339894,168.97051553)
\lineto(644.48339894,164.22051553)
\curveto(644.48339894,163.50697387)(644.5797531,163.04864053)(644.77246144,162.84551553)
\curveto(644.9703781,162.64239053)(645.3688156,162.54082803)(645.96777394,162.54082803)
\lineto(647.44433644,162.54082803)
\lineto(647.44433644,161.33770303)
\lineto(645.96777394,161.33770303)
\curveto(644.85839894,161.33770303)(644.09277394,161.5434322)(643.67089894,161.95489053)
\curveto(643.24902394,162.3715572)(643.03808644,163.12676553)(643.03808644,164.22051553)
\lineto(643.03808644,168.97051553)
\lineto(641.98339894,168.97051553)
\lineto(641.98339894,170.08770303)
\lineto(643.03808644,170.08770303)
\lineto(643.03808644,172.57207803)
\closepath
}
}
{
\newrgbcolor{curcolor}{0 0 0}
\pscustom[linestyle=none,fillstyle=solid,fillcolor=curcolor]
{
\newpath
\moveto(649.34277394,170.08770303)
\lineto(650.78027394,170.08770303)
\lineto(650.78027394,161.33770303)
\lineto(649.34277394,161.33770303)
\closepath
\moveto(649.34277394,173.49395303)
\lineto(650.78027394,173.49395303)
\lineto(650.78027394,171.67364053)
\lineto(649.34277394,171.67364053)
\closepath
}
}
{
\newrgbcolor{curcolor}{0 0 0}
\pscustom[linestyle=none,fillstyle=solid,fillcolor=curcolor]
{
\newpath
\moveto(659.53808644,168.75957803)
\lineto(659.53808644,173.49395303)
\lineto(660.97558644,173.49395303)
\lineto(660.97558644,161.33770303)
\lineto(659.53808644,161.33770303)
\lineto(659.53808644,162.65020303)
\curveto(659.2360031,162.1293697)(658.8531906,161.74134887)(658.38964894,161.48614053)
\curveto(657.9313156,161.23614053)(657.37923227,161.11114053)(656.73339894,161.11114053)
\curveto(655.67610727,161.11114053)(654.8141281,161.53301553)(654.14746144,162.37676553)
\curveto(653.4860031,163.22051553)(653.15527394,164.32989053)(653.15527394,165.70489053)
\curveto(653.15527394,167.07989053)(653.4860031,168.18926553)(654.14746144,169.03301553)
\curveto(654.8141281,169.87676553)(655.67610727,170.29864053)(656.73339894,170.29864053)
\curveto(657.37923227,170.29864053)(657.9313156,170.17103637)(658.38964894,169.91582803)
\curveto(658.8531906,169.66582803)(659.2360031,169.28041137)(659.53808644,168.75957803)
\closepath
\moveto(654.63964894,165.70489053)
\curveto(654.63964894,164.64759887)(654.85579477,163.8168697)(655.28808644,163.21270303)
\curveto(655.72558644,162.6137447)(656.32454477,162.31426553)(657.08496144,162.31426553)
\curveto(657.8453781,162.31426553)(658.44433644,162.6137447)(658.88183644,163.21270303)
\curveto(659.31933644,163.8168697)(659.53808644,164.64759887)(659.53808644,165.70489053)
\curveto(659.53808644,166.7621822)(659.31933644,167.5903072)(658.88183644,168.18926553)
\curveto(658.44433644,168.7934322)(657.8453781,169.09551553)(657.08496144,169.09551553)
\curveto(656.32454477,169.09551553)(655.72558644,168.7934322)(655.28808644,168.18926553)
\curveto(654.85579477,167.5903072)(654.63964894,166.7621822)(654.63964894,165.70489053)
\closepath
}
}
{
\newrgbcolor{curcolor}{0 0 0}
\pscustom[linestyle=none,fillstyle=solid,fillcolor=curcolor]
{
\newpath
\moveto(671.42089894,166.07207803)
\lineto(671.42089894,165.36895303)
\lineto(664.81152394,165.36895303)
\curveto(664.87402394,164.3793697)(665.17089894,163.62416137)(665.70214894,163.10332803)
\curveto(666.23860727,162.58770303)(666.98339894,162.32989053)(667.93652394,162.32989053)
\curveto(668.48860727,162.32989053)(669.02246144,162.39759887)(669.53808644,162.53301553)
\curveto(670.05891977,162.6684322)(670.57454477,162.8715572)(671.08496144,163.14239053)
\lineto(671.08496144,161.78301553)
\curveto(670.56933644,161.56426553)(670.0406906,161.39759887)(669.49902394,161.28301553)
\curveto(668.95735727,161.1684322)(668.4078781,161.11114053)(667.85058644,161.11114053)
\curveto(666.4547531,161.11114053)(665.34798227,161.51739053)(664.53027394,162.32989053)
\curveto(663.71777394,163.14239053)(663.31152394,164.24134887)(663.31152394,165.62676553)
\curveto(663.31152394,167.0590572)(663.6969406,168.19447387)(664.46777394,169.03301553)
\curveto(665.2438156,169.87676553)(666.28808644,170.29864053)(667.60058644,170.29864053)
\curveto(668.77766977,170.29864053)(669.70735727,169.9184322)(670.38964894,169.15801553)
\curveto(671.07714894,168.4028072)(671.42089894,167.37416137)(671.42089894,166.07207803)
\closepath
\moveto(669.98339894,166.49395303)
\curveto(669.97298227,167.28041137)(669.7516281,167.90801553)(669.31933644,168.37676553)
\curveto(668.8922531,168.84551553)(668.32454477,169.07989053)(667.61621144,169.07989053)
\curveto(666.8141281,169.07989053)(666.17089894,168.85332803)(665.68652394,168.40020303)
\curveto(665.20735727,167.94707803)(664.9313156,167.3090572)(664.85839894,166.48614053)
\closepath
}
}
{
\newrgbcolor{curcolor}{0 0 0}
\pscustom[linewidth=5.33333333,linecolor=curcolor]
{
\newpath
\moveto(200.69564,44.20912657)
\lineto(298.27364,44.20912657)
}
}
{
\newrgbcolor{curcolor}{0 0 0}
\pscustom[linewidth=5.33333333,linecolor=curcolor]
{
\newpath
\moveto(347.42028,44.20911324)
\lineto(659.29210667,44.20911324)
}
}
{
\newrgbcolor{curcolor}{0 0 0}
\pscustom[linestyle=none,fillstyle=solid,fillcolor=curcolor]
{
\newpath
\moveto(226.601563,20.47051553)
\lineto(230.664063,7.32207803)
\lineto(229.335938,7.32207803)
\lineto(225.273438,20.47051553)
\closepath
}
}
{
\newrgbcolor{curcolor}{0 0 0}
\pscustom[linestyle=none,fillstyle=solid,fillcolor=curcolor]
{
\newpath
\moveto(233.6093755,20.04082803)
\lineto(233.6093755,17.55645303)
\lineto(236.570313,17.55645303)
\lineto(236.570313,16.43926553)
\lineto(233.6093755,16.43926553)
\lineto(233.6093755,11.68926553)
\curveto(233.6093755,10.97572387)(233.70572967,10.51739053)(233.898438,10.31426553)
\curveto(234.09635467,10.11114053)(234.49479217,10.00957803)(235.0937505,10.00957803)
\lineto(236.570313,10.00957803)
\lineto(236.570313,8.80645303)
\lineto(235.0937505,8.80645303)
\curveto(233.9843755,8.80645303)(233.2187505,9.0121822)(232.7968755,9.42364053)
\curveto(232.3750005,9.8403072)(232.164063,10.59551553)(232.164063,11.68926553)
\lineto(232.164063,16.43926553)
\lineto(231.1093755,16.43926553)
\lineto(231.1093755,17.55645303)
\lineto(232.164063,17.55645303)
\lineto(232.164063,20.04082803)
\closepath
}
}
{
\newrgbcolor{curcolor}{0 0 0}
\pscustom[linestyle=none,fillstyle=solid,fillcolor=curcolor]
{
\newpath
\moveto(245.9531255,13.54082803)
\lineto(245.9531255,12.83770303)
\lineto(239.3437505,12.83770303)
\curveto(239.4062505,11.8481197)(239.7031255,11.09291137)(240.2343755,10.57207803)
\curveto(240.77083383,10.05645303)(241.5156255,9.79864053)(242.4687505,9.79864053)
\curveto(243.02083383,9.79864053)(243.554688,9.86634887)(244.070313,10.00176553)
\curveto(244.59114633,10.1371822)(245.10677133,10.3403072)(245.617188,10.61114053)
\lineto(245.617188,9.25176553)
\curveto(245.101563,9.03301553)(244.57291717,8.86634887)(244.0312505,8.75176553)
\curveto(243.48958383,8.6371822)(242.94010467,8.57989053)(242.382813,8.57989053)
\curveto(240.98697967,8.57989053)(239.88020883,8.98614053)(239.0625005,9.79864053)
\curveto(238.2500005,10.61114053)(237.8437505,11.71009887)(237.8437505,13.09551553)
\curveto(237.8437505,14.5278072)(238.22916717,15.66322387)(239.0000005,16.50176553)
\curveto(239.77604217,17.34551553)(240.820313,17.76739053)(242.132813,17.76739053)
\curveto(243.30989633,17.76739053)(244.23958383,17.3871822)(244.9218755,16.62676553)
\curveto(245.6093755,15.8715572)(245.9531255,14.84291137)(245.9531255,13.54082803)
\closepath
\moveto(244.5156255,13.96270303)
\curveto(244.50520883,14.74916137)(244.28385467,15.37676553)(243.851563,15.84551553)
\curveto(243.42447967,16.31426553)(242.85677133,16.54864053)(242.148438,16.54864053)
\curveto(241.34635467,16.54864053)(240.7031255,16.32207803)(240.2187505,15.86895303)
\curveto(239.73958383,15.41582803)(239.46354217,14.7778072)(239.3906255,13.95489053)
\closepath
}
}
{
\newrgbcolor{curcolor}{0 0 0}
\pscustom[linestyle=none,fillstyle=solid,fillcolor=curcolor]
{
\newpath
\moveto(255.304688,17.55645303)
\lineto(252.1406255,13.29864053)
\lineto(255.4687505,8.80645303)
\lineto(253.773438,8.80645303)
\lineto(251.226563,12.24395303)
\lineto(248.679688,8.80645303)
\lineto(246.9843755,8.80645303)
\lineto(250.382813,13.38457803)
\lineto(247.273438,17.55645303)
\lineto(248.9687505,17.55645303)
\lineto(251.289063,14.43926553)
\lineto(253.6093755,17.55645303)
\closepath
}
}
{
\newrgbcolor{curcolor}{0 0 0}
\pscustom[linestyle=none,fillstyle=solid,fillcolor=curcolor]
{
\newpath
\moveto(258.9218755,20.04082803)
\lineto(258.9218755,17.55645303)
\lineto(261.882813,17.55645303)
\lineto(261.882813,16.43926553)
\lineto(258.9218755,16.43926553)
\lineto(258.9218755,11.68926553)
\curveto(258.9218755,10.97572387)(259.01822967,10.51739053)(259.210938,10.31426553)
\curveto(259.40885467,10.11114053)(259.80729217,10.00957803)(260.4062505,10.00957803)
\lineto(261.882813,10.00957803)
\lineto(261.882813,8.80645303)
\lineto(260.4062505,8.80645303)
\curveto(259.2968755,8.80645303)(258.5312505,9.0121822)(258.1093755,9.42364053)
\curveto(257.6875005,9.8403072)(257.476563,10.59551553)(257.476563,11.68926553)
\lineto(257.476563,16.43926553)
\lineto(256.4218755,16.43926553)
\lineto(256.4218755,17.55645303)
\lineto(257.476563,17.55645303)
\lineto(257.476563,20.04082803)
\closepath
}
}
{
\newrgbcolor{curcolor}{0 0 0}
\pscustom[linestyle=none,fillstyle=solid,fillcolor=curcolor]
{
\newpath
\moveto(270.0625005,13.17364053)
\curveto(270.0625005,14.2309322)(269.8437505,15.0590572)(269.4062505,15.65801553)
\curveto(268.97395883,16.2621822)(268.37760467,16.56426553)(267.617188,16.56426553)
\curveto(266.85677133,16.56426553)(266.257813,16.2621822)(265.820313,15.65801553)
\curveto(265.38802133,15.0590572)(265.1718755,14.2309322)(265.1718755,13.17364053)
\curveto(265.1718755,12.11634887)(265.38802133,11.2856197)(265.820313,10.68145303)
\curveto(266.257813,10.0824947)(266.85677133,9.78301553)(267.617188,9.78301553)
\curveto(268.37760467,9.78301553)(268.97395883,10.0824947)(269.4062505,10.68145303)
\curveto(269.8437505,11.2856197)(270.0625005,12.11634887)(270.0625005,13.17364053)
\closepath
\moveto(265.1718755,16.22832803)
\curveto(265.47395883,16.74916137)(265.85416717,17.13457803)(266.3125005,17.38457803)
\curveto(266.77604217,17.63978637)(267.3281255,17.76739053)(267.9687505,17.76739053)
\curveto(269.0312505,17.76739053)(269.89322967,17.34551553)(270.554688,16.50176553)
\curveto(271.22135467,15.65801553)(271.554688,14.54864053)(271.554688,13.17364053)
\curveto(271.554688,11.79864053)(271.22135467,10.68926553)(270.554688,9.84551553)
\curveto(269.89322967,9.00176553)(269.0312505,8.57989053)(267.9687505,8.57989053)
\curveto(267.3281255,8.57989053)(266.77604217,8.70489053)(266.3125005,8.95489053)
\curveto(265.85416717,9.21009887)(265.47395883,9.5981197)(265.1718755,10.11895303)
\lineto(265.1718755,8.80645303)
\lineto(263.726563,8.80645303)
\lineto(263.726563,20.96270303)
\lineto(265.1718755,20.96270303)
\closepath
}
}
{
\newrgbcolor{curcolor}{0 0 0}
\pscustom[linestyle=none,fillstyle=solid,fillcolor=curcolor]
{
\newpath
\moveto(278.367188,20.96270303)
\lineto(278.367188,19.76739053)
\lineto(276.992188,19.76739053)
\curveto(276.476563,19.76739053)(276.117188,19.66322387)(275.914063,19.45489053)
\curveto(275.71614633,19.2465572)(275.617188,18.8715572)(275.617188,18.32989053)
\lineto(275.617188,17.55645303)
\lineto(277.9843755,17.55645303)
\lineto(277.9843755,16.43926553)
\lineto(275.617188,16.43926553)
\lineto(275.617188,8.80645303)
\lineto(274.1718755,8.80645303)
\lineto(274.1718755,16.43926553)
\lineto(272.7968755,16.43926553)
\lineto(272.7968755,17.55645303)
\lineto(274.1718755,17.55645303)
\lineto(274.1718755,18.16582803)
\curveto(274.1718755,19.13978637)(274.398438,19.8481197)(274.851563,20.29082803)
\curveto(275.304688,20.7387447)(276.023438,20.96270303)(277.007813,20.96270303)
\closepath
}
}
{
\newrgbcolor{curcolor}{0 0 0}
\pscustom[linestyle=none,fillstyle=solid,fillcolor=curcolor]
{
\newpath
\moveto(286.2343755,7.32207803)
\lineto(286.2343755,6.19707803)
\lineto(285.7500005,6.19707803)
\curveto(284.4531255,6.19707803)(283.58333383,6.38978637)(283.1406255,6.77520303)
\curveto(282.7031255,7.1606197)(282.4843755,7.92884887)(282.4843755,9.07989053)
\lineto(282.4843755,10.94707803)
\curveto(282.4843755,11.73353637)(282.3437505,12.2778072)(282.0625005,12.57989053)
\curveto(281.7812505,12.88197387)(281.27083383,13.03301553)(280.5312505,13.03301553)
\lineto(280.054688,13.03301553)
\lineto(280.054688,14.15020303)
\lineto(280.5312505,14.15020303)
\curveto(281.27604217,14.15020303)(281.78645883,14.29864053)(282.0625005,14.59551553)
\curveto(282.3437505,14.89759887)(282.4843755,15.43666137)(282.4843755,16.21270303)
\lineto(282.4843755,18.08770303)
\curveto(282.4843755,19.2387447)(282.7031255,20.0043697)(283.1406255,20.38457803)
\curveto(283.58333383,20.7699947)(284.4531255,20.96270303)(285.7500005,20.96270303)
\lineto(286.2343755,20.96270303)
\lineto(286.2343755,19.84551553)
\lineto(285.7031255,19.84551553)
\curveto(284.9687505,19.84551553)(284.48958383,19.7309322)(284.2656255,19.50176553)
\curveto(284.04166717,19.27259887)(283.929688,18.79082803)(283.929688,18.05645303)
\lineto(283.929688,16.11895303)
\curveto(283.929688,15.3012447)(283.80989633,14.7074947)(283.570313,14.33770303)
\curveto(283.335938,13.96791137)(282.93229217,13.71791137)(282.3593755,13.58770303)
\curveto(282.9375005,13.44707803)(283.3437505,13.1918697)(283.5781255,12.82207803)
\curveto(283.8125005,12.45228637)(283.929688,11.86114053)(283.929688,11.04864053)
\lineto(283.929688,9.11114053)
\curveto(283.929688,8.37676553)(284.04166717,7.8949947)(284.2656255,7.66582803)
\curveto(284.48958383,7.43666137)(284.9687505,7.32207803)(285.7031255,7.32207803)
\closepath
}
}
{
\newrgbcolor{curcolor}{0 0 0}
\pscustom[linestyle=none,fillstyle=solid,fillcolor=curcolor]
{
\newpath
\moveto(291.3906255,19.17364053)
\lineto(291.3906255,10.10332803)
\lineto(293.2968755,10.10332803)
\curveto(294.9062505,10.10332803)(296.08333383,10.46791137)(296.8281255,11.19707803)
\curveto(297.5781255,11.9262447)(297.9531255,13.07728637)(297.9531255,14.65020303)
\curveto(297.9531255,16.21270303)(297.5781255,17.3559322)(296.8281255,18.07989053)
\curveto(296.08333383,18.8090572)(294.9062505,19.17364053)(293.2968755,19.17364053)
\closepath
\moveto(289.8125005,20.47051553)
\lineto(293.054688,20.47051553)
\curveto(295.31510467,20.47051553)(296.97395883,19.99916137)(298.0312505,19.05645303)
\curveto(299.08854217,18.11895303)(299.617188,16.65020303)(299.617188,14.65020303)
\curveto(299.617188,12.63978637)(299.085938,11.16322387)(298.023438,10.22051553)
\curveto(296.960938,9.2778072)(295.304688,8.80645303)(293.054688,8.80645303)
\lineto(289.8125005,8.80645303)
\closepath
}
}
{
\newrgbcolor{curcolor}{0 0 0}
\pscustom[linestyle=none,fillstyle=solid,fillcolor=curcolor]
{
\newpath
\moveto(306.039063,13.20489053)
\curveto(304.87760467,13.20489053)(304.07291717,13.07207803)(303.6250005,12.80645303)
\curveto(303.17708383,12.54082803)(302.9531255,12.08770303)(302.9531255,11.44707803)
\curveto(302.9531255,10.93666137)(303.11979217,10.53041137)(303.4531255,10.22832803)
\curveto(303.79166717,9.93145303)(304.2500005,9.78301553)(304.8281255,9.78301553)
\curveto(305.6250005,9.78301553)(306.26302133,10.06426553)(306.742188,10.62676553)
\curveto(307.226563,11.19447387)(307.4687505,11.94707803)(307.4687505,12.88457803)
\lineto(307.4687505,13.20489053)
\closepath
\moveto(308.9062505,13.79864053)
\lineto(308.9062505,8.80645303)
\lineto(307.4687505,8.80645303)
\lineto(307.4687505,10.13457803)
\curveto(307.1406255,9.60332803)(306.73177133,9.21009887)(306.242188,8.95489053)
\curveto(305.75260467,8.70489053)(305.15364633,8.57989053)(304.445313,8.57989053)
\curveto(303.54947967,8.57989053)(302.835938,8.82989053)(302.304688,9.32989053)
\curveto(301.77864633,9.83509887)(301.5156255,10.50957803)(301.5156255,11.35332803)
\curveto(301.5156255,12.33770303)(301.8437505,13.07989053)(302.5000005,13.57989053)
\curveto(303.16145883,14.07989053)(304.14583383,14.32989053)(305.4531255,14.32989053)
\lineto(307.4687505,14.32989053)
\lineto(307.4687505,14.47051553)
\curveto(307.4687505,15.13197387)(307.2500005,15.64239053)(306.8125005,16.00176553)
\curveto(306.38020883,16.36634887)(305.77083383,16.54864053)(304.9843755,16.54864053)
\curveto(304.4843755,16.54864053)(303.99739633,16.4887447)(303.523438,16.36895303)
\curveto(303.04947967,16.24916137)(302.5937505,16.06947387)(302.1562505,15.82989053)
\lineto(302.1562505,17.15801553)
\curveto(302.68229217,17.36114053)(303.19270883,17.5121822)(303.6875005,17.61114053)
\curveto(304.18229217,17.7153072)(304.664063,17.76739053)(305.132813,17.76739053)
\curveto(306.398438,17.76739053)(307.3437505,17.43926553)(307.9687505,16.78301553)
\curveto(308.5937505,16.12676553)(308.9062505,15.13197387)(308.9062505,13.79864053)
\closepath
}
}
{
\newrgbcolor{curcolor}{0 0 0}
\pscustom[linestyle=none,fillstyle=solid,fillcolor=curcolor]
{
\newpath
\moveto(315.5156255,7.99395303)
\curveto(315.1093755,6.95228637)(314.71354217,6.27259887)(314.3281255,5.95489053)
\curveto(313.94270883,5.6371822)(313.42708383,5.47832803)(312.7812505,5.47832803)
\lineto(311.632813,5.47832803)
\lineto(311.632813,6.68145303)
\lineto(312.476563,6.68145303)
\curveto(312.87239633,6.68145303)(313.179688,6.77520303)(313.398438,6.96270303)
\curveto(313.617188,7.15020303)(313.8593755,7.59291137)(314.1250005,8.29082803)
\lineto(314.382813,8.94707803)
\lineto(310.8437505,17.55645303)
\lineto(312.367188,17.55645303)
\lineto(315.101563,10.71270303)
\lineto(317.835938,17.55645303)
\lineto(319.3593755,17.55645303)
\closepath
}
}
{
\newrgbcolor{curcolor}{0 0 0}
\pscustom[linestyle=none,fillstyle=solid,fillcolor=curcolor]
{
\newpath
\moveto(330.0156255,19.43145303)
\curveto(329.2031255,19.43145303)(328.59114633,19.03041137)(328.179688,18.22832803)
\curveto(327.773438,17.43145303)(327.570313,16.2309322)(327.570313,14.62676553)
\curveto(327.570313,13.0278072)(327.773438,11.82728637)(328.179688,11.02520303)
\curveto(328.59114633,10.22832803)(329.2031255,9.82989053)(330.0156255,9.82989053)
\curveto(330.83333383,9.82989053)(331.445313,10.22832803)(331.851563,11.02520303)
\curveto(332.26302133,11.82728637)(332.4687505,13.0278072)(332.4687505,14.62676553)
\curveto(332.4687505,16.2309322)(332.26302133,17.43145303)(331.851563,18.22832803)
\curveto(331.445313,19.03041137)(330.83333383,19.43145303)(330.0156255,19.43145303)
\closepath
\moveto(330.0156255,20.68145303)
\curveto(331.32291717,20.68145303)(332.320313,20.16322387)(333.007813,19.12676553)
\curveto(333.70052133,18.09551553)(334.0468755,16.59551553)(334.0468755,14.62676553)
\curveto(334.0468755,12.66322387)(333.70052133,11.16322387)(333.007813,10.12676553)
\curveto(332.320313,9.09551553)(331.32291717,8.57989053)(330.0156255,8.57989053)
\curveto(328.70833383,8.57989053)(327.70833383,9.09551553)(327.0156255,10.12676553)
\curveto(326.3281255,11.16322387)(325.9843755,12.66322387)(325.9843755,14.62676553)
\curveto(325.9843755,16.59551553)(326.3281255,18.09551553)(327.0156255,19.12676553)
\curveto(327.70833383,20.16322387)(328.70833383,20.68145303)(330.0156255,20.68145303)
\closepath
}
}
{
\newrgbcolor{curcolor}{0 0 0}
\pscustom[linestyle=none,fillstyle=solid,fillcolor=curcolor]
{
\newpath
\moveto(337.117188,7.32207803)
\lineto(337.664063,7.32207803)
\curveto(338.39322967,7.32207803)(338.867188,7.4340572)(339.085938,7.65801553)
\curveto(339.30989633,7.88197387)(339.4218755,8.36634887)(339.4218755,9.11114053)
\lineto(339.4218755,11.04864053)
\curveto(339.4218755,11.86114053)(339.539063,12.45228637)(339.773438,12.82207803)
\curveto(340.007813,13.1918697)(340.414063,13.44707803)(340.992188,13.58770303)
\curveto(340.414063,13.71791137)(340.007813,13.96791137)(339.773438,14.33770303)
\curveto(339.539063,14.7074947)(339.4218755,15.3012447)(339.4218755,16.11895303)
\lineto(339.4218755,18.05645303)
\curveto(339.4218755,18.79603637)(339.30989633,19.2778072)(339.085938,19.50176553)
\curveto(338.867188,19.7309322)(338.39322967,19.84551553)(337.664063,19.84551553)
\lineto(337.117188,19.84551553)
\lineto(337.117188,20.96270303)
\lineto(337.6093755,20.96270303)
\curveto(338.9062505,20.96270303)(339.77083383,20.7699947)(340.2031255,20.38457803)
\curveto(340.6406255,20.0043697)(340.8593755,19.2387447)(340.8593755,18.08770303)
\lineto(340.8593755,16.21270303)
\curveto(340.8593755,15.43666137)(341.0000005,14.89759887)(341.2812505,14.59551553)
\curveto(341.5625005,14.29864053)(342.07291717,14.15020303)(342.8125005,14.15020303)
\lineto(343.2968755,14.15020303)
\lineto(343.2968755,13.03301553)
\lineto(342.8125005,13.03301553)
\curveto(342.07291717,13.03301553)(341.5625005,12.88197387)(341.2812505,12.57989053)
\curveto(341.0000005,12.2778072)(340.8593755,11.73353637)(340.8593755,10.94707803)
\lineto(340.8593755,9.07989053)
\curveto(340.8593755,7.92884887)(340.6406255,7.1606197)(340.2031255,6.77520303)
\curveto(339.77083383,6.38978637)(338.9062505,6.19707803)(337.6093755,6.19707803)
\lineto(337.117188,6.19707803)
\closepath
}
}
{
\newrgbcolor{curcolor}{0 0 0}
\pscustom[linestyle=none,fillstyle=solid,fillcolor=curcolor]
{
\newpath
\moveto(389.90918019,22.83184366)
\lineto(393.97168019,9.68340616)
\lineto(392.64355519,9.68340616)
\lineto(388.58105519,22.83184366)
\closepath
}
}
{
\newrgbcolor{curcolor}{0 0 0}
\pscustom[linestyle=none,fillstyle=solid,fillcolor=curcolor]
{
\newpath
\moveto(396.91699269,22.40215616)
\lineto(396.91699269,19.91778116)
\lineto(399.87793019,19.91778116)
\lineto(399.87793019,18.80059366)
\lineto(396.91699269,18.80059366)
\lineto(396.91699269,14.05059366)
\curveto(396.91699269,13.33705199)(397.01334685,12.87871866)(397.20605519,12.67559366)
\curveto(397.40397185,12.47246866)(397.80240935,12.37090616)(398.40136769,12.37090616)
\lineto(399.87793019,12.37090616)
\lineto(399.87793019,11.16778116)
\lineto(398.40136769,11.16778116)
\curveto(397.29199269,11.16778116)(396.52636769,11.37351032)(396.10449269,11.78496866)
\curveto(395.68261769,12.20163532)(395.47168019,12.95684366)(395.47168019,14.05059366)
\lineto(395.47168019,18.80059366)
\lineto(394.41699269,18.80059366)
\lineto(394.41699269,19.91778116)
\lineto(395.47168019,19.91778116)
\lineto(395.47168019,22.40215616)
\closepath
}
}
{
\newrgbcolor{curcolor}{0 0 0}
\pscustom[linestyle=none,fillstyle=solid,fillcolor=curcolor]
{
\newpath
\moveto(409.26074269,15.90215616)
\lineto(409.26074269,15.19903116)
\lineto(402.65136769,15.19903116)
\curveto(402.71386769,14.20944782)(403.01074269,13.45423949)(403.54199269,12.93340616)
\curveto(404.07845102,12.41778116)(404.82324269,12.15996866)(405.77636769,12.15996866)
\curveto(406.32845102,12.15996866)(406.86230519,12.22767699)(407.37793019,12.36309366)
\curveto(407.89876352,12.49851032)(408.41438852,12.70163532)(408.92480519,12.97246866)
\lineto(408.92480519,11.61309366)
\curveto(408.40918019,11.39434366)(407.88053435,11.22767699)(407.33886769,11.11309366)
\curveto(406.79720102,10.99851032)(406.24772185,10.94121866)(405.69043019,10.94121866)
\curveto(404.29459685,10.94121866)(403.18782602,11.34746866)(402.37011769,12.15996866)
\curveto(401.55761769,12.97246866)(401.15136769,14.07142699)(401.15136769,15.45684366)
\curveto(401.15136769,16.88913532)(401.53678435,18.02455199)(402.30761769,18.86309366)
\curveto(403.08365935,19.70684366)(404.12793019,20.12871866)(405.44043019,20.12871866)
\curveto(406.61751352,20.12871866)(407.54720102,19.74851032)(408.22949269,18.98809366)
\curveto(408.91699269,18.23288532)(409.26074269,17.20423949)(409.26074269,15.90215616)
\closepath
\moveto(407.82324269,16.32403116)
\curveto(407.81282602,17.11048949)(407.59147185,17.73809366)(407.15918019,18.20684366)
\curveto(406.73209685,18.67559366)(406.16438852,18.90996866)(405.45605519,18.90996866)
\curveto(404.65397185,18.90996866)(404.01074269,18.68340616)(403.52636769,18.23028116)
\curveto(403.04720102,17.77715616)(402.77115935,17.13913532)(402.69824269,16.31621866)
\closepath
}
}
{
\newrgbcolor{curcolor}{0 0 0}
\pscustom[linestyle=none,fillstyle=solid,fillcolor=curcolor]
{
\newpath
\moveto(418.61230519,19.91778116)
\lineto(415.44824269,15.65996866)
\lineto(418.77636769,11.16778116)
\lineto(417.08105519,11.16778116)
\lineto(414.53418019,14.60528116)
\lineto(411.98730519,11.16778116)
\lineto(410.29199269,11.16778116)
\lineto(413.69043019,15.74590616)
\lineto(410.58105519,19.91778116)
\lineto(412.27636769,19.91778116)
\lineto(414.59668019,16.80059366)
\lineto(416.91699269,19.91778116)
\closepath
}
}
{
\newrgbcolor{curcolor}{0 0 0}
\pscustom[linestyle=none,fillstyle=solid,fillcolor=curcolor]
{
\newpath
\moveto(422.22949269,22.40215616)
\lineto(422.22949269,19.91778116)
\lineto(425.19043019,19.91778116)
\lineto(425.19043019,18.80059366)
\lineto(422.22949269,18.80059366)
\lineto(422.22949269,14.05059366)
\curveto(422.22949269,13.33705199)(422.32584685,12.87871866)(422.51855519,12.67559366)
\curveto(422.71647185,12.47246866)(423.11490935,12.37090616)(423.71386769,12.37090616)
\lineto(425.19043019,12.37090616)
\lineto(425.19043019,11.16778116)
\lineto(423.71386769,11.16778116)
\curveto(422.60449269,11.16778116)(421.83886769,11.37351032)(421.41699269,11.78496866)
\curveto(420.99511769,12.20163532)(420.78418019,12.95684366)(420.78418019,14.05059366)
\lineto(420.78418019,18.80059366)
\lineto(419.72949269,18.80059366)
\lineto(419.72949269,19.91778116)
\lineto(420.78418019,19.91778116)
\lineto(420.78418019,22.40215616)
\closepath
}
}
{
\newrgbcolor{curcolor}{0 0 0}
\pscustom[linestyle=none,fillstyle=solid,fillcolor=curcolor]
{
\newpath
\moveto(433.37011769,15.53496866)
\curveto(433.37011769,16.59226032)(433.15136769,17.42038532)(432.71386769,18.01934366)
\curveto(432.28157602,18.62351032)(431.68522185,18.92559366)(430.92480519,18.92559366)
\curveto(430.16438852,18.92559366)(429.56543019,18.62351032)(429.12793019,18.01934366)
\curveto(428.69563852,17.42038532)(428.47949269,16.59226032)(428.47949269,15.53496866)
\curveto(428.47949269,14.47767699)(428.69563852,13.64694782)(429.12793019,13.04278116)
\curveto(429.56543019,12.44382282)(430.16438852,12.14434366)(430.92480519,12.14434366)
\curveto(431.68522185,12.14434366)(432.28157602,12.44382282)(432.71386769,13.04278116)
\curveto(433.15136769,13.64694782)(433.37011769,14.47767699)(433.37011769,15.53496866)
\closepath
\moveto(428.47949269,18.58965616)
\curveto(428.78157602,19.11048949)(429.16178435,19.49590616)(429.62011769,19.74590616)
\curveto(430.08365935,20.00111449)(430.63574269,20.12871866)(431.27636769,20.12871866)
\curveto(432.33886769,20.12871866)(433.20084685,19.70684366)(433.86230519,18.86309366)
\curveto(434.52897185,18.01934366)(434.86230519,16.90996866)(434.86230519,15.53496866)
\curveto(434.86230519,14.15996866)(434.52897185,13.05059366)(433.86230519,12.20684366)
\curveto(433.20084685,11.36309366)(432.33886769,10.94121866)(431.27636769,10.94121866)
\curveto(430.63574269,10.94121866)(430.08365935,11.06621866)(429.62011769,11.31621866)
\curveto(429.16178435,11.57142699)(428.78157602,11.95944782)(428.47949269,12.48028116)
\lineto(428.47949269,11.16778116)
\lineto(427.03418019,11.16778116)
\lineto(427.03418019,23.32403116)
\lineto(428.47949269,23.32403116)
\closepath
}
}
{
\newrgbcolor{curcolor}{0 0 0}
\pscustom[linestyle=none,fillstyle=solid,fillcolor=curcolor]
{
\newpath
\moveto(441.67480519,23.32403116)
\lineto(441.67480519,22.12871866)
\lineto(440.29980519,22.12871866)
\curveto(439.78418019,22.12871866)(439.42480519,22.02455199)(439.22168019,21.81621866)
\curveto(439.02376352,21.60788532)(438.92480519,21.23288532)(438.92480519,20.69121866)
\lineto(438.92480519,19.91778116)
\lineto(441.29199269,19.91778116)
\lineto(441.29199269,18.80059366)
\lineto(438.92480519,18.80059366)
\lineto(438.92480519,11.16778116)
\lineto(437.47949269,11.16778116)
\lineto(437.47949269,18.80059366)
\lineto(436.10449269,18.80059366)
\lineto(436.10449269,19.91778116)
\lineto(437.47949269,19.91778116)
\lineto(437.47949269,20.52715616)
\curveto(437.47949269,21.50111449)(437.70605519,22.20944782)(438.15918019,22.65215616)
\curveto(438.61230519,23.10007282)(439.33105519,23.32403116)(440.31543019,23.32403116)
\closepath
}
}
{
\newrgbcolor{curcolor}{0 0 0}
\pscustom[linestyle=none,fillstyle=solid,fillcolor=curcolor]
{
\newpath
\moveto(449.54199269,9.68340616)
\lineto(449.54199269,8.55840616)
\lineto(449.05761769,8.55840616)
\curveto(447.76074269,8.55840616)(446.89095102,8.75111449)(446.44824269,9.13653116)
\curveto(446.01074269,9.52194782)(445.79199269,10.29017699)(445.79199269,11.44121866)
\lineto(445.79199269,13.30840616)
\curveto(445.79199269,14.09486449)(445.65136769,14.63913532)(445.37011769,14.94121866)
\curveto(445.08886769,15.24330199)(444.57845102,15.39434366)(443.83886769,15.39434366)
\lineto(443.36230519,15.39434366)
\lineto(443.36230519,16.51153116)
\lineto(443.83886769,16.51153116)
\curveto(444.58365935,16.51153116)(445.09407602,16.65996866)(445.37011769,16.95684366)
\curveto(445.65136769,17.25892699)(445.79199269,17.79798949)(445.79199269,18.57403116)
\lineto(445.79199269,20.44903116)
\curveto(445.79199269,21.60007282)(446.01074269,22.36569782)(446.44824269,22.74590616)
\curveto(446.89095102,23.13132282)(447.76074269,23.32403116)(449.05761769,23.32403116)
\lineto(449.54199269,23.32403116)
\lineto(449.54199269,22.20684366)
\lineto(449.01074269,22.20684366)
\curveto(448.27636769,22.20684366)(447.79720102,22.09226032)(447.57324269,21.86309366)
\curveto(447.34928435,21.63392699)(447.23730519,21.15215616)(447.23730519,20.41778116)
\lineto(447.23730519,18.48028116)
\curveto(447.23730519,17.66257282)(447.11751352,17.06882282)(446.87793019,16.69903116)
\curveto(446.64355519,16.32923949)(446.23990935,16.07923949)(445.66699269,15.94903116)
\curveto(446.24511769,15.80840616)(446.65136769,15.55319782)(446.88574269,15.18340616)
\curveto(447.12011769,14.81361449)(447.23730519,14.22246866)(447.23730519,13.40996866)
\lineto(447.23730519,11.47246866)
\curveto(447.23730519,10.73809366)(447.34928435,10.25632282)(447.57324269,10.02715616)
\curveto(447.79720102,9.79798949)(448.27636769,9.68340616)(449.01074269,9.68340616)
\closepath
}
}
{
\newrgbcolor{curcolor}{0 0 0}
\pscustom[linestyle=none,fillstyle=solid,fillcolor=curcolor]
{
\newpath
\moveto(454.69824269,21.53496866)
\lineto(454.69824269,12.46465616)
\lineto(456.60449269,12.46465616)
\curveto(458.21386769,12.46465616)(459.39095102,12.82923949)(460.13574269,13.55840616)
\curveto(460.88574269,14.28757282)(461.26074269,15.43861449)(461.26074269,17.01153116)
\curveto(461.26074269,18.57403116)(460.88574269,19.71726032)(460.13574269,20.44121866)
\curveto(459.39095102,21.17038532)(458.21386769,21.53496866)(456.60449269,21.53496866)
\closepath
\moveto(453.12011769,22.83184366)
\lineto(456.36230519,22.83184366)
\curveto(458.62272185,22.83184366)(460.28157602,22.36048949)(461.33886769,21.41778116)
\curveto(462.39615935,20.48028116)(462.92480519,19.01153116)(462.92480519,17.01153116)
\curveto(462.92480519,15.00111449)(462.39355519,13.52455199)(461.33105519,12.58184366)
\curveto(460.26855519,11.63913532)(458.61230519,11.16778116)(456.36230519,11.16778116)
\lineto(453.12011769,11.16778116)
\closepath
}
}
{
\newrgbcolor{curcolor}{0 0 0}
\pscustom[linestyle=none,fillstyle=solid,fillcolor=curcolor]
{
\newpath
\moveto(469.34668019,15.56621866)
\curveto(468.18522185,15.56621866)(467.38053435,15.43340616)(466.93261769,15.16778116)
\curveto(466.48470102,14.90215616)(466.26074269,14.44903116)(466.26074269,13.80840616)
\curveto(466.26074269,13.29798949)(466.42740935,12.89173949)(466.76074269,12.58965616)
\curveto(467.09928435,12.29278116)(467.55761769,12.14434366)(468.13574269,12.14434366)
\curveto(468.93261769,12.14434366)(469.57063852,12.42559366)(470.04980519,12.98809366)
\curveto(470.53418019,13.55580199)(470.77636769,14.30840616)(470.77636769,15.24590616)
\lineto(470.77636769,15.56621866)
\closepath
\moveto(472.21386769,16.15996866)
\lineto(472.21386769,11.16778116)
\lineto(470.77636769,11.16778116)
\lineto(470.77636769,12.49590616)
\curveto(470.44824269,11.96465616)(470.03938852,11.57142699)(469.54980519,11.31621866)
\curveto(469.06022185,11.06621866)(468.46126352,10.94121866)(467.75293019,10.94121866)
\curveto(466.85709685,10.94121866)(466.14355519,11.19121866)(465.61230519,11.69121866)
\curveto(465.08626352,12.19642699)(464.82324269,12.87090616)(464.82324269,13.71465616)
\curveto(464.82324269,14.69903116)(465.15136769,15.44121866)(465.80761769,15.94121866)
\curveto(466.46907602,16.44121866)(467.45345102,16.69121866)(468.76074269,16.69121866)
\lineto(470.77636769,16.69121866)
\lineto(470.77636769,16.83184366)
\curveto(470.77636769,17.49330199)(470.55761769,18.00371866)(470.12011769,18.36309366)
\curveto(469.68782602,18.72767699)(469.07845102,18.90996866)(468.29199269,18.90996866)
\curveto(467.79199269,18.90996866)(467.30501352,18.85007282)(466.83105519,18.73028116)
\curveto(466.35709685,18.61048949)(465.90136769,18.43080199)(465.46386769,18.19121866)
\lineto(465.46386769,19.51934366)
\curveto(465.98990935,19.72246866)(466.50032602,19.87351032)(466.99511769,19.97246866)
\curveto(467.48990935,20.07663532)(467.97168019,20.12871866)(468.44043019,20.12871866)
\curveto(469.70605519,20.12871866)(470.65136769,19.80059366)(471.27636769,19.14434366)
\curveto(471.90136769,18.48809366)(472.21386769,17.49330199)(472.21386769,16.15996866)
\closepath
}
}
{
\newrgbcolor{curcolor}{0 0 0}
\pscustom[linestyle=none,fillstyle=solid,fillcolor=curcolor]
{
\newpath
\moveto(478.82324269,10.35528116)
\curveto(478.41699269,9.31361449)(478.02115935,8.63392699)(477.63574269,8.31621866)
\curveto(477.25032602,7.99851032)(476.73470102,7.83965616)(476.08886769,7.83965616)
\lineto(474.94043019,7.83965616)
\lineto(474.94043019,9.04278116)
\lineto(475.78418019,9.04278116)
\curveto(476.18001352,9.04278116)(476.48730519,9.13653116)(476.70605519,9.32403116)
\curveto(476.92480519,9.51153116)(477.16699269,9.95423949)(477.43261769,10.65215616)
\lineto(477.69043019,11.30840616)
\lineto(474.15136769,19.91778116)
\lineto(475.67480519,19.91778116)
\lineto(478.40918019,13.07403116)
\lineto(481.14355519,19.91778116)
\lineto(482.66699269,19.91778116)
\closepath
}
}
{
\newrgbcolor{curcolor}{0 0 0}
\pscustom[linestyle=none,fillstyle=solid,fillcolor=curcolor]
{
\newpath
\moveto(491.30761769,12.49590616)
\lineto(496.81543019,12.49590616)
\lineto(496.81543019,11.16778116)
\lineto(489.40918019,11.16778116)
\lineto(489.40918019,12.49590616)
\curveto(490.00813852,13.11569782)(490.82324269,13.94642699)(491.85449269,14.98809366)
\curveto(492.89095102,16.03496866)(493.54199269,16.70944782)(493.80761769,17.01153116)
\curveto(494.31282602,17.57923949)(494.66438852,18.05840616)(494.86230519,18.44903116)
\curveto(495.06543019,18.84486449)(495.16699269,19.23288532)(495.16699269,19.61309366)
\curveto(495.16699269,20.23288532)(494.94824269,20.73809366)(494.51074269,21.12871866)
\curveto(494.07845102,21.51934366)(493.51334685,21.71465616)(492.81543019,21.71465616)
\curveto(492.32063852,21.71465616)(491.79720102,21.62871866)(491.24511769,21.45684366)
\curveto(490.69824269,21.28496866)(490.11230519,21.02455199)(489.48730519,20.67559366)
\lineto(489.48730519,22.26934366)
\curveto(490.12272185,22.52455199)(490.71647185,22.71726032)(491.26855519,22.84746866)
\curveto(491.82063852,22.97767699)(492.32584685,23.04278116)(492.78418019,23.04278116)
\curveto(493.99251352,23.04278116)(494.95605519,22.74069782)(495.67480519,22.13653116)
\curveto(496.39355519,21.53236449)(496.75293019,20.72507282)(496.75293019,19.71465616)
\curveto(496.75293019,19.23548949)(496.66178435,18.77976032)(496.47949269,18.34746866)
\curveto(496.30240935,17.92038532)(495.97688852,17.41517699)(495.50293019,16.83184366)
\curveto(495.37272185,16.68080199)(494.95865935,16.24330199)(494.26074269,15.51934366)
\curveto(493.56282602,14.80059366)(492.57845102,13.79278116)(491.30761769,12.49590616)
\closepath
}
}
{
\newrgbcolor{curcolor}{0 0 0}
\pscustom[linestyle=none,fillstyle=solid,fillcolor=curcolor]
{
\newpath
\moveto(500.40918019,12.49590616)
\lineto(502.98730519,12.49590616)
\lineto(502.98730519,21.39434366)
\lineto(500.18261769,20.83184366)
\lineto(500.18261769,22.26934366)
\lineto(502.97168019,22.83184366)
\lineto(504.54980519,22.83184366)
\lineto(504.54980519,12.49590616)
\lineto(507.12793019,12.49590616)
\lineto(507.12793019,11.16778116)
\lineto(500.40918019,11.16778116)
\closepath
}
}
{
\newrgbcolor{curcolor}{0 0 0}
\pscustom[linestyle=none,fillstyle=solid,fillcolor=curcolor]
{
\newpath
\moveto(510.61230519,9.68340616)
\lineto(511.15918019,9.68340616)
\curveto(511.88834685,9.68340616)(512.36230519,9.79538532)(512.58105519,10.01934366)
\curveto(512.80501352,10.24330199)(512.91699269,10.72767699)(512.91699269,11.47246866)
\lineto(512.91699269,13.40996866)
\curveto(512.91699269,14.22246866)(513.03418019,14.81361449)(513.26855519,15.18340616)
\curveto(513.50293019,15.55319782)(513.90918019,15.80840616)(514.48730519,15.94903116)
\curveto(513.90918019,16.07923949)(513.50293019,16.32923949)(513.26855519,16.69903116)
\curveto(513.03418019,17.06882282)(512.91699269,17.66257282)(512.91699269,18.48028116)
\lineto(512.91699269,20.41778116)
\curveto(512.91699269,21.15736449)(512.80501352,21.63913532)(512.58105519,21.86309366)
\curveto(512.36230519,22.09226032)(511.88834685,22.20684366)(511.15918019,22.20684366)
\lineto(510.61230519,22.20684366)
\lineto(510.61230519,23.32403116)
\lineto(511.10449269,23.32403116)
\curveto(512.40136769,23.32403116)(513.26595102,23.13132282)(513.69824269,22.74590616)
\curveto(514.13574269,22.36569782)(514.35449269,21.60007282)(514.35449269,20.44903116)
\lineto(514.35449269,18.57403116)
\curveto(514.35449269,17.79798949)(514.49511769,17.25892699)(514.77636769,16.95684366)
\curveto(515.05761769,16.65996866)(515.56803435,16.51153116)(516.30761769,16.51153116)
\lineto(516.79199269,16.51153116)
\lineto(516.79199269,15.39434366)
\lineto(516.30761769,15.39434366)
\curveto(515.56803435,15.39434366)(515.05761769,15.24330199)(514.77636769,14.94121866)
\curveto(514.49511769,14.63913532)(514.35449269,14.09486449)(514.35449269,13.30840616)
\lineto(514.35449269,11.44121866)
\curveto(514.35449269,10.29017699)(514.13574269,9.52194782)(513.69824269,9.13653116)
\curveto(513.26595102,8.75111449)(512.40136769,8.55840616)(511.10449269,8.55840616)
\lineto(510.61230519,8.55840616)
\closepath
}
}
{
\newrgbcolor{curcolor}{0 0 0}
\pscustom[linestyle=none,fillstyle=solid,fillcolor=curcolor]
{
\newpath
\moveto(612.21386769,20.47051553)
\lineto(616.27636769,7.32207803)
\lineto(614.94824269,7.32207803)
\lineto(610.88574269,20.47051553)
\closepath
}
}
{
\newrgbcolor{curcolor}{0 0 0}
\pscustom[linestyle=none,fillstyle=solid,fillcolor=curcolor]
{
\newpath
\moveto(619.22168019,20.04082803)
\lineto(619.22168019,17.55645303)
\lineto(622.18261769,17.55645303)
\lineto(622.18261769,16.43926553)
\lineto(619.22168019,16.43926553)
\lineto(619.22168019,11.68926553)
\curveto(619.22168019,10.97572387)(619.31803435,10.51739053)(619.51074269,10.31426553)
\curveto(619.70865935,10.11114053)(620.10709685,10.00957803)(620.70605519,10.00957803)
\lineto(622.18261769,10.00957803)
\lineto(622.18261769,8.80645303)
\lineto(620.70605519,8.80645303)
\curveto(619.59668019,8.80645303)(618.83105519,9.0121822)(618.40918019,9.42364053)
\curveto(617.98730519,9.8403072)(617.77636769,10.59551553)(617.77636769,11.68926553)
\lineto(617.77636769,16.43926553)
\lineto(616.72168019,16.43926553)
\lineto(616.72168019,17.55645303)
\lineto(617.77636769,17.55645303)
\lineto(617.77636769,20.04082803)
\closepath
}
}
{
\newrgbcolor{curcolor}{0 0 0}
\pscustom[linestyle=none,fillstyle=solid,fillcolor=curcolor]
{
\newpath
\moveto(631.56543019,13.54082803)
\lineto(631.56543019,12.83770303)
\lineto(624.95605519,12.83770303)
\curveto(625.01855519,11.8481197)(625.31543019,11.09291137)(625.84668019,10.57207803)
\curveto(626.38313852,10.05645303)(627.12793019,9.79864053)(628.08105519,9.79864053)
\curveto(628.63313852,9.79864053)(629.16699269,9.86634887)(629.68261769,10.00176553)
\curveto(630.20345102,10.1371822)(630.71907602,10.3403072)(631.22949269,10.61114053)
\lineto(631.22949269,9.25176553)
\curveto(630.71386769,9.03301553)(630.18522185,8.86634887)(629.64355519,8.75176553)
\curveto(629.10188852,8.6371822)(628.55240935,8.57989053)(627.99511769,8.57989053)
\curveto(626.59928435,8.57989053)(625.49251352,8.98614053)(624.67480519,9.79864053)
\curveto(623.86230519,10.61114053)(623.45605519,11.71009887)(623.45605519,13.09551553)
\curveto(623.45605519,14.5278072)(623.84147185,15.66322387)(624.61230519,16.50176553)
\curveto(625.38834685,17.34551553)(626.43261769,17.76739053)(627.74511769,17.76739053)
\curveto(628.92220102,17.76739053)(629.85188852,17.3871822)(630.53418019,16.62676553)
\curveto(631.22168019,15.8715572)(631.56543019,14.84291137)(631.56543019,13.54082803)
\closepath
\moveto(630.12793019,13.96270303)
\curveto(630.11751352,14.74916137)(629.89615935,15.37676553)(629.46386769,15.84551553)
\curveto(629.03678435,16.31426553)(628.46907602,16.54864053)(627.76074269,16.54864053)
\curveto(626.95865935,16.54864053)(626.31543019,16.32207803)(625.83105519,15.86895303)
\curveto(625.35188852,15.41582803)(625.07584685,14.7778072)(625.00293019,13.95489053)
\closepath
}
}
{
\newrgbcolor{curcolor}{0 0 0}
\pscustom[linestyle=none,fillstyle=solid,fillcolor=curcolor]
{
\newpath
\moveto(640.91699269,17.55645303)
\lineto(637.75293019,13.29864053)
\lineto(641.08105519,8.80645303)
\lineto(639.38574269,8.80645303)
\lineto(636.83886769,12.24395303)
\lineto(634.29199269,8.80645303)
\lineto(632.59668019,8.80645303)
\lineto(635.99511769,13.38457803)
\lineto(632.88574269,17.55645303)
\lineto(634.58105519,17.55645303)
\lineto(636.90136769,14.43926553)
\lineto(639.22168019,17.55645303)
\closepath
}
}
{
\newrgbcolor{curcolor}{0 0 0}
\pscustom[linestyle=none,fillstyle=solid,fillcolor=curcolor]
{
\newpath
\moveto(644.53418019,20.04082803)
\lineto(644.53418019,17.55645303)
\lineto(647.49511769,17.55645303)
\lineto(647.49511769,16.43926553)
\lineto(644.53418019,16.43926553)
\lineto(644.53418019,11.68926553)
\curveto(644.53418019,10.97572387)(644.63053435,10.51739053)(644.82324269,10.31426553)
\curveto(645.02115935,10.11114053)(645.41959685,10.00957803)(646.01855519,10.00957803)
\lineto(647.49511769,10.00957803)
\lineto(647.49511769,8.80645303)
\lineto(646.01855519,8.80645303)
\curveto(644.90918019,8.80645303)(644.14355519,9.0121822)(643.72168019,9.42364053)
\curveto(643.29980519,9.8403072)(643.08886769,10.59551553)(643.08886769,11.68926553)
\lineto(643.08886769,16.43926553)
\lineto(642.03418019,16.43926553)
\lineto(642.03418019,17.55645303)
\lineto(643.08886769,17.55645303)
\lineto(643.08886769,20.04082803)
\closepath
}
}
{
\newrgbcolor{curcolor}{0 0 0}
\pscustom[linestyle=none,fillstyle=solid,fillcolor=curcolor]
{
\newpath
\moveto(655.67480519,13.17364053)
\curveto(655.67480519,14.2309322)(655.45605519,15.0590572)(655.01855519,15.65801553)
\curveto(654.58626352,16.2621822)(653.98990935,16.56426553)(653.22949269,16.56426553)
\curveto(652.46907602,16.56426553)(651.87011769,16.2621822)(651.43261769,15.65801553)
\curveto(651.00032602,15.0590572)(650.78418019,14.2309322)(650.78418019,13.17364053)
\curveto(650.78418019,12.11634887)(651.00032602,11.2856197)(651.43261769,10.68145303)
\curveto(651.87011769,10.0824947)(652.46907602,9.78301553)(653.22949269,9.78301553)
\curveto(653.98990935,9.78301553)(654.58626352,10.0824947)(655.01855519,10.68145303)
\curveto(655.45605519,11.2856197)(655.67480519,12.11634887)(655.67480519,13.17364053)
\closepath
\moveto(650.78418019,16.22832803)
\curveto(651.08626352,16.74916137)(651.46647185,17.13457803)(651.92480519,17.38457803)
\curveto(652.38834685,17.63978637)(652.94043019,17.76739053)(653.58105519,17.76739053)
\curveto(654.64355519,17.76739053)(655.50553435,17.34551553)(656.16699269,16.50176553)
\curveto(656.83365935,15.65801553)(657.16699269,14.54864053)(657.16699269,13.17364053)
\curveto(657.16699269,11.79864053)(656.83365935,10.68926553)(656.16699269,9.84551553)
\curveto(655.50553435,9.00176553)(654.64355519,8.57989053)(653.58105519,8.57989053)
\curveto(652.94043019,8.57989053)(652.38834685,8.70489053)(651.92480519,8.95489053)
\curveto(651.46647185,9.21009887)(651.08626352,9.5981197)(650.78418019,10.11895303)
\lineto(650.78418019,8.80645303)
\lineto(649.33886769,8.80645303)
\lineto(649.33886769,20.96270303)
\lineto(650.78418019,20.96270303)
\closepath
}
}
{
\newrgbcolor{curcolor}{0 0 0}
\pscustom[linestyle=none,fillstyle=solid,fillcolor=curcolor]
{
\newpath
\moveto(663.97949269,20.96270303)
\lineto(663.97949269,19.76739053)
\lineto(662.60449269,19.76739053)
\curveto(662.08886769,19.76739053)(661.72949269,19.66322387)(661.52636769,19.45489053)
\curveto(661.32845102,19.2465572)(661.22949269,18.8715572)(661.22949269,18.32989053)
\lineto(661.22949269,17.55645303)
\lineto(663.59668019,17.55645303)
\lineto(663.59668019,16.43926553)
\lineto(661.22949269,16.43926553)
\lineto(661.22949269,8.80645303)
\lineto(659.78418019,8.80645303)
\lineto(659.78418019,16.43926553)
\lineto(658.40918019,16.43926553)
\lineto(658.40918019,17.55645303)
\lineto(659.78418019,17.55645303)
\lineto(659.78418019,18.16582803)
\curveto(659.78418019,19.13978637)(660.01074269,19.8481197)(660.46386769,20.29082803)
\curveto(660.91699269,20.7387447)(661.63574269,20.96270303)(662.62011769,20.96270303)
\closepath
}
}
{
\newrgbcolor{curcolor}{0 0 0}
\pscustom[linestyle=none,fillstyle=solid,fillcolor=curcolor]
{
\newpath
\moveto(671.84668019,7.32207803)
\lineto(671.84668019,6.19707803)
\lineto(671.36230519,6.19707803)
\curveto(670.06543019,6.19707803)(669.19563852,6.38978637)(668.75293019,6.77520303)
\curveto(668.31543019,7.1606197)(668.09668019,7.92884887)(668.09668019,9.07989053)
\lineto(668.09668019,10.94707803)
\curveto(668.09668019,11.73353637)(667.95605519,12.2778072)(667.67480519,12.57989053)
\curveto(667.39355519,12.88197387)(666.88313852,13.03301553)(666.14355519,13.03301553)
\lineto(665.66699269,13.03301553)
\lineto(665.66699269,14.15020303)
\lineto(666.14355519,14.15020303)
\curveto(666.88834685,14.15020303)(667.39876352,14.29864053)(667.67480519,14.59551553)
\curveto(667.95605519,14.89759887)(668.09668019,15.43666137)(668.09668019,16.21270303)
\lineto(668.09668019,18.08770303)
\curveto(668.09668019,19.2387447)(668.31543019,20.0043697)(668.75293019,20.38457803)
\curveto(669.19563852,20.7699947)(670.06543019,20.96270303)(671.36230519,20.96270303)
\lineto(671.84668019,20.96270303)
\lineto(671.84668019,19.84551553)
\lineto(671.31543019,19.84551553)
\curveto(670.58105519,19.84551553)(670.10188852,19.7309322)(669.87793019,19.50176553)
\curveto(669.65397185,19.27259887)(669.54199269,18.79082803)(669.54199269,18.05645303)
\lineto(669.54199269,16.11895303)
\curveto(669.54199269,15.3012447)(669.42220102,14.7074947)(669.18261769,14.33770303)
\curveto(668.94824269,13.96791137)(668.54459685,13.71791137)(667.97168019,13.58770303)
\curveto(668.54980519,13.44707803)(668.95605519,13.1918697)(669.19043019,12.82207803)
\curveto(669.42480519,12.45228637)(669.54199269,11.86114053)(669.54199269,11.04864053)
\lineto(669.54199269,9.11114053)
\curveto(669.54199269,8.37676553)(669.65397185,7.8949947)(669.87793019,7.66582803)
\curveto(670.10188852,7.43666137)(670.58105519,7.32207803)(671.31543019,7.32207803)
\closepath
}
}
{
\newrgbcolor{curcolor}{0 0 0}
\pscustom[linestyle=none,fillstyle=solid,fillcolor=curcolor]
{
\newpath
\moveto(677.00293019,19.17364053)
\lineto(677.00293019,10.10332803)
\lineto(678.90918019,10.10332803)
\curveto(680.51855519,10.10332803)(681.69563852,10.46791137)(682.44043019,11.19707803)
\curveto(683.19043019,11.9262447)(683.56543019,13.07728637)(683.56543019,14.65020303)
\curveto(683.56543019,16.21270303)(683.19043019,17.3559322)(682.44043019,18.07989053)
\curveto(681.69563852,18.8090572)(680.51855519,19.17364053)(678.90918019,19.17364053)
\closepath
\moveto(675.42480519,20.47051553)
\lineto(678.66699269,20.47051553)
\curveto(680.92740935,20.47051553)(682.58626352,19.99916137)(683.64355519,19.05645303)
\curveto(684.70084685,18.11895303)(685.22949269,16.65020303)(685.22949269,14.65020303)
\curveto(685.22949269,12.63978637)(684.69824269,11.16322387)(683.63574269,10.22051553)
\curveto(682.57324269,9.2778072)(680.91699269,8.80645303)(678.66699269,8.80645303)
\lineto(675.42480519,8.80645303)
\closepath
}
}
{
\newrgbcolor{curcolor}{0 0 0}
\pscustom[linestyle=none,fillstyle=solid,fillcolor=curcolor]
{
\newpath
\moveto(691.65136769,13.20489053)
\curveto(690.48990935,13.20489053)(689.68522185,13.07207803)(689.23730519,12.80645303)
\curveto(688.78938852,12.54082803)(688.56543019,12.08770303)(688.56543019,11.44707803)
\curveto(688.56543019,10.93666137)(688.73209685,10.53041137)(689.06543019,10.22832803)
\curveto(689.40397185,9.93145303)(689.86230519,9.78301553)(690.44043019,9.78301553)
\curveto(691.23730519,9.78301553)(691.87532602,10.06426553)(692.35449269,10.62676553)
\curveto(692.83886769,11.19447387)(693.08105519,11.94707803)(693.08105519,12.88457803)
\lineto(693.08105519,13.20489053)
\closepath
\moveto(694.51855519,13.79864053)
\lineto(694.51855519,8.80645303)
\lineto(693.08105519,8.80645303)
\lineto(693.08105519,10.13457803)
\curveto(692.75293019,9.60332803)(692.34407602,9.21009887)(691.85449269,8.95489053)
\curveto(691.36490935,8.70489053)(690.76595102,8.57989053)(690.05761769,8.57989053)
\curveto(689.16178435,8.57989053)(688.44824269,8.82989053)(687.91699269,9.32989053)
\curveto(687.39095102,9.83509887)(687.12793019,10.50957803)(687.12793019,11.35332803)
\curveto(687.12793019,12.33770303)(687.45605519,13.07989053)(688.11230519,13.57989053)
\curveto(688.77376352,14.07989053)(689.75813852,14.32989053)(691.06543019,14.32989053)
\lineto(693.08105519,14.32989053)
\lineto(693.08105519,14.47051553)
\curveto(693.08105519,15.13197387)(692.86230519,15.64239053)(692.42480519,16.00176553)
\curveto(691.99251352,16.36634887)(691.38313852,16.54864053)(690.59668019,16.54864053)
\curveto(690.09668019,16.54864053)(689.60970102,16.4887447)(689.13574269,16.36895303)
\curveto(688.66178435,16.24916137)(688.20605519,16.06947387)(687.76855519,15.82989053)
\lineto(687.76855519,17.15801553)
\curveto(688.29459685,17.36114053)(688.80501352,17.5121822)(689.29980519,17.61114053)
\curveto(689.79459685,17.7153072)(690.27636769,17.76739053)(690.74511769,17.76739053)
\curveto(692.01074269,17.76739053)(692.95605519,17.43926553)(693.58105519,16.78301553)
\curveto(694.20605519,16.12676553)(694.51855519,15.13197387)(694.51855519,13.79864053)
\closepath
}
}
{
\newrgbcolor{curcolor}{0 0 0}
\pscustom[linestyle=none,fillstyle=solid,fillcolor=curcolor]
{
\newpath
\moveto(701.12793019,7.99395303)
\curveto(700.72168019,6.95228637)(700.32584685,6.27259887)(699.94043019,5.95489053)
\curveto(699.55501352,5.6371822)(699.03938852,5.47832803)(698.39355519,5.47832803)
\lineto(697.24511769,5.47832803)
\lineto(697.24511769,6.68145303)
\lineto(698.08886769,6.68145303)
\curveto(698.48470102,6.68145303)(698.79199269,6.77520303)(699.01074269,6.96270303)
\curveto(699.22949269,7.15020303)(699.47168019,7.59291137)(699.73730519,8.29082803)
\lineto(699.99511769,8.94707803)
\lineto(696.45605519,17.55645303)
\lineto(697.97949269,17.55645303)
\lineto(700.71386769,10.71270303)
\lineto(703.44824269,17.55645303)
\lineto(704.97168019,17.55645303)
\closepath
}
}
{
\newrgbcolor{curcolor}{0 0 0}
\pscustom[linestyle=none,fillstyle=solid,fillcolor=curcolor]
{
\newpath
\moveto(713.61230519,10.13457803)
\lineto(719.12011769,10.13457803)
\lineto(719.12011769,8.80645303)
\lineto(711.71386769,8.80645303)
\lineto(711.71386769,10.13457803)
\curveto(712.31282602,10.7543697)(713.12793019,11.58509887)(714.15918019,12.62676553)
\curveto(715.19563852,13.67364053)(715.84668019,14.3481197)(716.11230519,14.65020303)
\curveto(716.61751352,15.21791137)(716.96907602,15.69707803)(717.16699269,16.08770303)
\curveto(717.37011769,16.48353637)(717.47168019,16.8715572)(717.47168019,17.25176553)
\curveto(717.47168019,17.8715572)(717.25293019,18.37676553)(716.81543019,18.76739053)
\curveto(716.38313852,19.15801553)(715.81803435,19.35332803)(715.12011769,19.35332803)
\curveto(714.62532602,19.35332803)(714.10188852,19.26739053)(713.54980519,19.09551553)
\curveto(713.00293019,18.92364053)(712.41699269,18.66322387)(711.79199269,18.31426553)
\lineto(711.79199269,19.90801553)
\curveto(712.42740935,20.16322387)(713.02115935,20.3559322)(713.57324269,20.48614053)
\curveto(714.12532602,20.61634887)(714.63053435,20.68145303)(715.08886769,20.68145303)
\curveto(716.29720102,20.68145303)(717.26074269,20.3793697)(717.97949269,19.77520303)
\curveto(718.69824269,19.17103637)(719.05761769,18.3637447)(719.05761769,17.35332803)
\curveto(719.05761769,16.87416137)(718.96647185,16.4184322)(718.78418019,15.98614053)
\curveto(718.60709685,15.5590572)(718.28157602,15.05384887)(717.80761769,14.47051553)
\curveto(717.67740935,14.31947387)(717.26334685,13.88197387)(716.56543019,13.15801553)
\curveto(715.86751352,12.43926553)(714.88313852,11.43145303)(713.61230519,10.13457803)
\closepath
}
}
{
\newrgbcolor{curcolor}{0 0 0}
\pscustom[linestyle=none,fillstyle=solid,fillcolor=curcolor]
{
\newpath
\moveto(723.79980519,10.13457803)
\lineto(729.30761769,10.13457803)
\lineto(729.30761769,8.80645303)
\lineto(721.90136769,8.80645303)
\lineto(721.90136769,10.13457803)
\curveto(722.50032602,10.7543697)(723.31543019,11.58509887)(724.34668019,12.62676553)
\curveto(725.38313852,13.67364053)(726.03418019,14.3481197)(726.29980519,14.65020303)
\curveto(726.80501352,15.21791137)(727.15657602,15.69707803)(727.35449269,16.08770303)
\curveto(727.55761769,16.48353637)(727.65918019,16.8715572)(727.65918019,17.25176553)
\curveto(727.65918019,17.8715572)(727.44043019,18.37676553)(727.00293019,18.76739053)
\curveto(726.57063852,19.15801553)(726.00553435,19.35332803)(725.30761769,19.35332803)
\curveto(724.81282602,19.35332803)(724.28938852,19.26739053)(723.73730519,19.09551553)
\curveto(723.19043019,18.92364053)(722.60449269,18.66322387)(721.97949269,18.31426553)
\lineto(721.97949269,19.90801553)
\curveto(722.61490935,20.16322387)(723.20865935,20.3559322)(723.76074269,20.48614053)
\curveto(724.31282602,20.61634887)(724.81803435,20.68145303)(725.27636769,20.68145303)
\curveto(726.48470102,20.68145303)(727.44824269,20.3793697)(728.16699269,19.77520303)
\curveto(728.88574269,19.17103637)(729.24511769,18.3637447)(729.24511769,17.35332803)
\curveto(729.24511769,16.87416137)(729.15397185,16.4184322)(728.97168019,15.98614053)
\curveto(728.79459685,15.5590572)(728.46907602,15.05384887)(727.99511769,14.47051553)
\curveto(727.86490935,14.31947387)(727.45084685,13.88197387)(726.75293019,13.15801553)
\curveto(726.05501352,12.43926553)(725.07063852,11.43145303)(723.79980519,10.13457803)
\closepath
}
}
{
\newrgbcolor{curcolor}{0 0 0}
\pscustom[linestyle=none,fillstyle=solid,fillcolor=curcolor]
{
\newpath
\moveto(732.91699269,7.32207803)
\lineto(733.46386769,7.32207803)
\curveto(734.19303435,7.32207803)(734.66699269,7.4340572)(734.88574269,7.65801553)
\curveto(735.10970102,7.88197387)(735.22168019,8.36634887)(735.22168019,9.11114053)
\lineto(735.22168019,11.04864053)
\curveto(735.22168019,11.86114053)(735.33886769,12.45228637)(735.57324269,12.82207803)
\curveto(735.80761769,13.1918697)(736.21386769,13.44707803)(736.79199269,13.58770303)
\curveto(736.21386769,13.71791137)(735.80761769,13.96791137)(735.57324269,14.33770303)
\curveto(735.33886769,14.7074947)(735.22168019,15.3012447)(735.22168019,16.11895303)
\lineto(735.22168019,18.05645303)
\curveto(735.22168019,18.79603637)(735.10970102,19.2778072)(734.88574269,19.50176553)
\curveto(734.66699269,19.7309322)(734.19303435,19.84551553)(733.46386769,19.84551553)
\lineto(732.91699269,19.84551553)
\lineto(732.91699269,20.96270303)
\lineto(733.40918019,20.96270303)
\curveto(734.70605519,20.96270303)(735.57063852,20.7699947)(736.00293019,20.38457803)
\curveto(736.44043019,20.0043697)(736.65918019,19.2387447)(736.65918019,18.08770303)
\lineto(736.65918019,16.21270303)
\curveto(736.65918019,15.43666137)(736.79980519,14.89759887)(737.08105519,14.59551553)
\curveto(737.36230519,14.29864053)(737.87272185,14.15020303)(738.61230519,14.15020303)
\lineto(739.09668019,14.15020303)
\lineto(739.09668019,13.03301553)
\lineto(738.61230519,13.03301553)
\curveto(737.87272185,13.03301553)(737.36230519,12.88197387)(737.08105519,12.57989053)
\curveto(736.79980519,12.2778072)(736.65918019,11.73353637)(736.65918019,10.94707803)
\lineto(736.65918019,9.07989053)
\curveto(736.65918019,7.92884887)(736.44043019,7.1606197)(736.00293019,6.77520303)
\curveto(735.57063852,6.38978637)(734.70605519,6.19707803)(733.40918019,6.19707803)
\lineto(732.91699269,6.19707803)
\closepath
}
}
{
\newrgbcolor{curcolor}{0 0 0}
\pscustom[linestyle=none,fillstyle=solid,fillcolor=curcolor]
{
\newpath
\moveto(389.27734425,92.25957827)
\lineto(395.37109425,72.53692202)
\lineto(393.37890675,72.53692202)
\lineto(387.28515675,92.25957827)
\closepath
}
}
{
\newrgbcolor{curcolor}{0 0 0}
\pscustom[linestyle=none,fillstyle=solid,fillcolor=curcolor]
{
\newpath
\moveto(399.789063,91.61504702)
\lineto(399.789063,87.88848452)
\lineto(404.23046925,87.88848452)
\lineto(404.23046925,86.21270327)
\lineto(399.789063,86.21270327)
\lineto(399.789063,79.08770327)
\curveto(399.789063,78.01739077)(399.93359425,77.32989077)(400.22265675,77.02520327)
\curveto(400.51953175,76.72051577)(401.117188,76.56817202)(402.0156255,76.56817202)
\lineto(404.23046925,76.56817202)
\lineto(404.23046925,74.76348452)
\lineto(402.0156255,74.76348452)
\curveto(400.351563,74.76348452)(399.2031255,75.07207827)(398.570313,75.68926577)
\curveto(397.9375005,76.31426577)(397.62109425,77.44707827)(397.62109425,79.08770327)
\lineto(397.62109425,86.21270327)
\lineto(396.039063,86.21270327)
\lineto(396.039063,87.88848452)
\lineto(397.62109425,87.88848452)
\lineto(397.62109425,91.61504702)
\closepath
}
}
{
\newrgbcolor{curcolor}{0 0 0}
\pscustom[linestyle=none,fillstyle=solid,fillcolor=curcolor]
{
\newpath
\moveto(418.304688,81.86504702)
\lineto(418.304688,80.81035952)
\lineto(408.3906255,80.81035952)
\curveto(408.4843755,79.32598452)(408.929688,78.19317202)(409.726563,77.41192202)
\curveto(410.5312505,76.63848452)(411.648438,76.25176577)(413.0781255,76.25176577)
\curveto(413.9062505,76.25176577)(414.70703175,76.35332827)(415.48046925,76.55645327)
\curveto(416.26171925,76.75957827)(417.03515675,77.06426577)(417.80078175,77.47051577)
\lineto(417.80078175,75.43145327)
\curveto(417.02734425,75.10332827)(416.2343755,74.85332827)(415.4218755,74.68145327)
\curveto(414.6093755,74.50957827)(413.78515675,74.42364077)(412.94921925,74.42364077)
\curveto(410.85546925,74.42364077)(409.195313,75.03301577)(407.9687505,76.25176577)
\curveto(406.7500005,77.47051577)(406.1406255,79.11895327)(406.1406255,81.19707827)
\curveto(406.1406255,83.34551577)(406.7187505,85.04864077)(407.8750005,86.30645327)
\curveto(409.039063,87.57207827)(410.60546925,88.20489077)(412.57421925,88.20489077)
\curveto(414.33984425,88.20489077)(415.7343755,87.63457827)(416.757813,86.49395327)
\curveto(417.789063,85.36114077)(418.304688,83.81817202)(418.304688,81.86504702)
\closepath
\moveto(416.148438,82.49785952)
\curveto(416.132813,83.67754702)(415.80078175,84.61895327)(415.15234425,85.32207827)
\curveto(414.51171925,86.02520327)(413.66015675,86.37676577)(412.59765675,86.37676577)
\curveto(411.39453175,86.37676577)(410.429688,86.03692202)(409.7031255,85.35723452)
\curveto(408.9843755,84.67754702)(408.570313,83.72051577)(408.460938,82.48614077)
\closepath
}
}
{
\newrgbcolor{curcolor}{0 0 0}
\pscustom[linestyle=none,fillstyle=solid,fillcolor=curcolor]
{
\newpath
\moveto(432.33203175,87.88848452)
\lineto(427.585938,81.50176577)
\lineto(432.5781255,74.76348452)
\lineto(430.03515675,74.76348452)
\lineto(426.21484425,79.91973452)
\lineto(422.39453175,74.76348452)
\lineto(419.851563,74.76348452)
\lineto(424.94921925,81.63067202)
\lineto(420.28515675,87.88848452)
\lineto(422.8281255,87.88848452)
\lineto(426.30859425,83.21270327)
\lineto(429.789063,87.88848452)
\closepath
}
}
{
\newrgbcolor{curcolor}{0 0 0}
\pscustom[linestyle=none,fillstyle=solid,fillcolor=curcolor]
{
\newpath
\moveto(437.757813,91.61504702)
\lineto(437.757813,87.88848452)
\lineto(442.19921925,87.88848452)
\lineto(442.19921925,86.21270327)
\lineto(437.757813,86.21270327)
\lineto(437.757813,79.08770327)
\curveto(437.757813,78.01739077)(437.90234425,77.32989077)(438.19140675,77.02520327)
\curveto(438.48828175,76.72051577)(439.085938,76.56817202)(439.9843755,76.56817202)
\lineto(442.19921925,76.56817202)
\lineto(442.19921925,74.76348452)
\lineto(439.9843755,74.76348452)
\curveto(438.320313,74.76348452)(437.1718755,75.07207827)(436.539063,75.68926577)
\curveto(435.9062505,76.31426577)(435.58984425,77.44707827)(435.58984425,79.08770327)
\lineto(435.58984425,86.21270327)
\lineto(434.007813,86.21270327)
\lineto(434.007813,87.88848452)
\lineto(435.58984425,87.88848452)
\lineto(435.58984425,91.61504702)
\closepath
}
}
{
\newrgbcolor{curcolor}{0 0 0}
\pscustom[linestyle=none,fillstyle=solid,fillcolor=curcolor]
{
\newpath
\moveto(454.4687505,81.31426577)
\curveto(454.4687505,82.90020327)(454.1406255,84.14239077)(453.4843755,85.04082827)
\curveto(452.835938,85.94707827)(451.94140675,86.40020327)(450.80078175,86.40020327)
\curveto(449.66015675,86.40020327)(448.76171925,85.94707827)(448.10546925,85.04082827)
\curveto(447.45703175,84.14239077)(447.132813,82.90020327)(447.132813,81.31426577)
\curveto(447.132813,79.72832827)(447.45703175,78.48223452)(448.10546925,77.57598452)
\curveto(448.76171925,76.67754702)(449.66015675,76.22832827)(450.80078175,76.22832827)
\curveto(451.94140675,76.22832827)(452.835938,76.67754702)(453.4843755,77.57598452)
\curveto(454.1406255,78.48223452)(454.4687505,79.72832827)(454.4687505,81.31426577)
\closepath
\moveto(447.132813,85.89629702)
\curveto(447.585938,86.67754702)(448.1562505,87.25567202)(448.8437505,87.63067202)
\curveto(449.539063,88.01348452)(450.367188,88.20489077)(451.3281255,88.20489077)
\curveto(452.9218755,88.20489077)(454.21484425,87.57207827)(455.20703175,86.30645327)
\curveto(456.20703175,85.04082827)(456.70703175,83.37676577)(456.70703175,81.31426577)
\curveto(456.70703175,79.25176577)(456.20703175,77.58770327)(455.20703175,76.32207827)
\curveto(454.21484425,75.05645327)(452.9218755,74.42364077)(451.3281255,74.42364077)
\curveto(450.367188,74.42364077)(449.539063,74.61114077)(448.8437505,74.98614077)
\curveto(448.1562505,75.36895327)(447.585938,75.95098452)(447.132813,76.73223452)
\lineto(447.132813,74.76348452)
\lineto(444.96484425,74.76348452)
\lineto(444.96484425,92.99785952)
\lineto(447.132813,92.99785952)
\closepath
}
}
{
\newrgbcolor{curcolor}{0 0 0}
\pscustom[linestyle=none,fillstyle=solid,fillcolor=curcolor]
{
\newpath
\moveto(466.92578175,92.99785952)
\lineto(466.92578175,91.20489077)
\lineto(464.86328175,91.20489077)
\curveto(464.08984425,91.20489077)(463.55078175,91.04864077)(463.24609425,90.73614077)
\curveto(462.94921925,90.42364077)(462.80078175,89.86114077)(462.80078175,89.04864077)
\lineto(462.80078175,87.88848452)
\lineto(466.351563,87.88848452)
\lineto(466.351563,86.21270327)
\lineto(462.80078175,86.21270327)
\lineto(462.80078175,74.76348452)
\lineto(460.632813,74.76348452)
\lineto(460.632813,86.21270327)
\lineto(458.570313,86.21270327)
\lineto(458.570313,87.88848452)
\lineto(460.632813,87.88848452)
\lineto(460.632813,88.80254702)
\curveto(460.632813,90.26348452)(460.97265675,91.32598452)(461.65234425,91.99004702)
\curveto(462.33203175,92.66192202)(463.41015675,92.99785952)(464.88671925,92.99785952)
\closepath
}
}
{
\newrgbcolor{curcolor}{0 0 0}
\pscustom[linestyle=none,fillstyle=solid,fillcolor=curcolor]
{
\newpath
\moveto(478.726563,72.53692202)
\lineto(478.726563,70.84942202)
\lineto(478.0000005,70.84942202)
\curveto(476.054688,70.84942202)(474.7500005,71.13848452)(474.085938,71.71660952)
\curveto(473.429688,72.29473452)(473.101563,73.44707827)(473.101563,75.17364077)
\lineto(473.101563,77.97442202)
\curveto(473.101563,79.15410952)(472.8906255,79.97051577)(472.4687505,80.42364077)
\curveto(472.0468755,80.87676577)(471.2812505,81.10332827)(470.1718755,81.10332827)
\lineto(469.45703175,81.10332827)
\lineto(469.45703175,82.77910952)
\lineto(470.1718755,82.77910952)
\curveto(471.289063,82.77910952)(472.054688,83.00176577)(472.4687505,83.44707827)
\curveto(472.8906255,83.90020327)(473.101563,84.70879702)(473.101563,85.87285952)
\lineto(473.101563,88.68535952)
\curveto(473.101563,90.41192202)(473.429688,91.56035952)(474.085938,92.13067202)
\curveto(474.7500005,92.70879702)(476.054688,92.99785952)(478.0000005,92.99785952)
\lineto(478.726563,92.99785952)
\lineto(478.726563,91.32207827)
\lineto(477.929688,91.32207827)
\curveto(476.8281255,91.32207827)(476.1093755,91.15020327)(475.773438,90.80645327)
\curveto(475.4375005,90.46270327)(475.26953175,89.74004702)(475.26953175,88.63848452)
\lineto(475.26953175,85.73223452)
\curveto(475.26953175,84.50567202)(475.08984425,83.61504702)(474.73046925,83.06035952)
\curveto(474.37890675,82.50567202)(473.773438,82.13067202)(472.914063,81.93535952)
\curveto(473.7812505,81.72442202)(474.3906255,81.34160952)(474.742188,80.78692202)
\curveto(475.0937505,80.23223452)(475.26953175,79.34551577)(475.26953175,78.12676577)
\lineto(475.26953175,75.22051577)
\curveto(475.26953175,74.11895327)(475.4375005,73.39629702)(475.773438,73.05254702)
\curveto(476.1093755,72.70879702)(476.8281255,72.53692202)(477.929688,72.53692202)
\closepath
}
}
{
\newrgbcolor{curcolor}{0 0 0}
\pscustom[linestyle=none,fillstyle=solid,fillcolor=curcolor]
{
\newpath
\moveto(486.132813,91.61504702)
\lineto(486.132813,87.88848452)
\lineto(490.57421925,87.88848452)
\lineto(490.57421925,86.21270327)
\lineto(486.132813,86.21270327)
\lineto(486.132813,79.08770327)
\curveto(486.132813,78.01739077)(486.27734425,77.32989077)(486.56640675,77.02520327)
\curveto(486.86328175,76.72051577)(487.460938,76.56817202)(488.3593755,76.56817202)
\lineto(490.57421925,76.56817202)
\lineto(490.57421925,74.76348452)
\lineto(488.3593755,74.76348452)
\curveto(486.695313,74.76348452)(485.5468755,75.07207827)(484.914063,75.68926577)
\curveto(484.2812505,76.31426577)(483.96484425,77.44707827)(483.96484425,79.08770327)
\lineto(483.96484425,86.21270327)
\lineto(482.382813,86.21270327)
\lineto(482.382813,87.88848452)
\lineto(483.96484425,87.88848452)
\lineto(483.96484425,91.61504702)
\closepath
}
}
{
\newrgbcolor{curcolor}{0 0 0}
\pscustom[linestyle=none,fillstyle=solid,fillcolor=curcolor]
{
\newpath
\moveto(501.02734425,85.87285952)
\curveto(500.78515675,86.01348452)(500.51953175,86.11504702)(500.23046925,86.17754702)
\curveto(499.94921925,86.24785952)(499.63671925,86.28301577)(499.29296925,86.28301577)
\curveto(498.07421925,86.28301577)(497.13671925,85.88457827)(496.48046925,85.08770327)
\curveto(495.83203175,84.29864077)(495.507813,83.16192202)(495.507813,81.67754702)
\lineto(495.507813,74.76348452)
\lineto(493.33984425,74.76348452)
\lineto(493.33984425,87.88848452)
\lineto(495.507813,87.88848452)
\lineto(495.507813,85.84942202)
\curveto(495.960938,86.64629702)(496.55078175,87.23614077)(497.27734425,87.61895327)
\curveto(498.00390675,88.00957827)(498.88671925,88.20489077)(499.92578175,88.20489077)
\curveto(500.07421925,88.20489077)(500.23828175,88.19317202)(500.41796925,88.16973452)
\curveto(500.59765675,88.15410952)(500.7968755,88.12676577)(501.0156255,88.08770327)
\closepath
}
}
{
\newrgbcolor{curcolor}{0 0 0}
\pscustom[linestyle=none,fillstyle=solid,fillcolor=curcolor]
{
\newpath
\moveto(509.27734425,81.36114077)
\curveto(507.53515675,81.36114077)(506.3281255,81.16192202)(505.6562505,80.76348452)
\curveto(504.9843755,80.36504702)(504.648438,79.68535952)(504.648438,78.72442202)
\curveto(504.648438,77.95879702)(504.898438,77.34942202)(505.398438,76.89629702)
\curveto(505.9062505,76.45098452)(506.5937505,76.22832827)(507.460938,76.22832827)
\curveto(508.6562505,76.22832827)(509.61328175,76.65020327)(510.33203175,77.49395327)
\curveto(511.05859425,78.34551577)(511.4218755,79.47442202)(511.4218755,80.88067202)
\lineto(511.4218755,81.36114077)
\closepath
\moveto(513.5781255,82.25176577)
\lineto(513.5781255,74.76348452)
\lineto(511.4218755,74.76348452)
\lineto(511.4218755,76.75567202)
\curveto(510.929688,75.95879702)(510.31640675,75.36895327)(509.58203175,74.98614077)
\curveto(508.84765675,74.61114077)(507.94921925,74.42364077)(506.88671925,74.42364077)
\curveto(505.54296925,74.42364077)(504.47265675,74.79864077)(503.67578175,75.54864077)
\curveto(502.88671925,76.30645327)(502.492188,77.31817202)(502.492188,78.58379702)
\curveto(502.492188,80.06035952)(502.9843755,81.17364077)(503.9687505,81.92364077)
\curveto(504.960938,82.67364077)(506.4375005,83.04864077)(508.398438,83.04864077)
\lineto(511.4218755,83.04864077)
\lineto(511.4218755,83.25957827)
\curveto(511.4218755,84.25176577)(511.0937505,85.01739077)(510.4375005,85.55645327)
\curveto(509.789063,86.10332827)(508.8750005,86.37676577)(507.695313,86.37676577)
\curveto(506.945313,86.37676577)(506.21484425,86.28692202)(505.50390675,86.10723452)
\curveto(504.79296925,85.92754702)(504.1093755,85.65801577)(503.4531255,85.29864077)
\lineto(503.4531255,87.29082827)
\curveto(504.242188,87.59551577)(505.007813,87.82207827)(505.7500005,87.97051577)
\curveto(506.492188,88.12676577)(507.21484425,88.20489077)(507.91796925,88.20489077)
\curveto(509.81640675,88.20489077)(511.2343755,87.71270327)(512.1718755,86.72832827)
\curveto(513.1093755,85.74395327)(513.5781255,84.25176577)(513.5781255,82.25176577)
\closepath
}
}
{
\newrgbcolor{curcolor}{0 0 0}
\pscustom[linestyle=none,fillstyle=solid,fillcolor=curcolor]
{
\newpath
\moveto(518.0312505,87.88848452)
\lineto(520.1875005,87.88848452)
\lineto(520.1875005,74.76348452)
\lineto(518.0312505,74.76348452)
\closepath
\moveto(518.0312505,92.99785952)
\lineto(520.1875005,92.99785952)
\lineto(520.1875005,90.26739077)
\lineto(518.0312505,90.26739077)
\closepath
}
}
{
\newrgbcolor{curcolor}{0 0 0}
\pscustom[linestyle=none,fillstyle=solid,fillcolor=curcolor]
{
\newpath
\moveto(535.59765675,82.68535952)
\lineto(535.59765675,74.76348452)
\lineto(533.44140675,74.76348452)
\lineto(533.44140675,82.61504702)
\curveto(533.44140675,83.85723452)(533.19921925,84.78692202)(532.71484425,85.40410952)
\curveto(532.23046925,86.02129702)(531.50390675,86.32989077)(530.53515675,86.32989077)
\curveto(529.37109425,86.32989077)(528.4531255,85.95879702)(527.7812505,85.21660952)
\curveto(527.1093755,84.47442202)(526.773438,83.46270327)(526.773438,82.18145327)
\lineto(526.773438,74.76348452)
\lineto(524.60546925,74.76348452)
\lineto(524.60546925,87.88848452)
\lineto(526.773438,87.88848452)
\lineto(526.773438,85.84942202)
\curveto(527.289063,86.63848452)(527.89453175,87.22832827)(528.58984425,87.61895327)
\curveto(529.29296925,88.00957827)(530.101563,88.20489077)(531.0156255,88.20489077)
\curveto(532.523438,88.20489077)(533.664063,87.73614077)(534.4375005,86.79864077)
\curveto(535.210938,85.86895327)(535.59765675,84.49785952)(535.59765675,82.68535952)
\closepath
}
}
{
\newrgbcolor{curcolor}{0 0 0}
\pscustom[linestyle=none,fillstyle=solid,fillcolor=curcolor]
{
\newpath
\moveto(540.66015675,72.53692202)
\lineto(541.48046925,72.53692202)
\curveto(542.57421925,72.53692202)(543.28515675,72.70489077)(543.61328175,73.04082827)
\curveto(543.94921925,73.37676577)(544.117188,74.10332827)(544.117188,75.22051577)
\lineto(544.117188,78.12676577)
\curveto(544.117188,79.34551577)(544.29296925,80.23223452)(544.64453175,80.78692202)
\curveto(544.99609425,81.34160952)(545.60546925,81.72442202)(546.47265675,81.93535952)
\curveto(545.60546925,82.13067202)(544.99609425,82.50567202)(544.64453175,83.06035952)
\curveto(544.29296925,83.61504702)(544.117188,84.50567202)(544.117188,85.73223452)
\lineto(544.117188,88.63848452)
\curveto(544.117188,89.74785952)(543.94921925,90.47051577)(543.61328175,90.80645327)
\curveto(543.28515675,91.15020327)(542.57421925,91.32207827)(541.48046925,91.32207827)
\lineto(540.66015675,91.32207827)
\lineto(540.66015675,92.99785952)
\lineto(541.398438,92.99785952)
\curveto(543.3437505,92.99785952)(544.6406255,92.70879702)(545.289063,92.13067202)
\curveto(545.945313,91.56035952)(546.273438,90.41192202)(546.273438,88.68535952)
\lineto(546.273438,85.87285952)
\curveto(546.273438,84.70879702)(546.4843755,83.90020327)(546.9062505,83.44707827)
\curveto(547.3281255,83.00176577)(548.0937505,82.77910952)(549.2031255,82.77910952)
\lineto(549.929688,82.77910952)
\lineto(549.929688,81.10332827)
\lineto(549.2031255,81.10332827)
\curveto(548.0937505,81.10332827)(547.3281255,80.87676577)(546.9062505,80.42364077)
\curveto(546.4843755,79.97051577)(546.273438,79.15410952)(546.273438,77.97442202)
\lineto(546.273438,75.17364077)
\curveto(546.273438,73.44707827)(545.945313,72.29473452)(545.289063,71.71660952)
\curveto(544.6406255,71.13848452)(543.3437505,70.84942202)(541.398438,70.84942202)
\lineto(540.66015675,70.84942202)
\closepath
}
}
{
\newrgbcolor{curcolor}{0 0 0}
\pscustom[linestyle=none,fillstyle=solid,fillcolor=curcolor]
{
\newpath
\moveto(585.81738331,91.89825015)
\lineto(591.91113331,72.1755939)
\lineto(589.91894581,72.1755939)
\lineto(583.82519581,91.89825015)
\closepath
}
}
{
\newrgbcolor{curcolor}{0 0 0}
\pscustom[linestyle=none,fillstyle=solid,fillcolor=curcolor]
{
\newpath
\moveto(596.32910206,91.2537189)
\lineto(596.32910206,87.5271564)
\lineto(600.77050831,87.5271564)
\lineto(600.77050831,85.85137515)
\lineto(596.32910206,85.85137515)
\lineto(596.32910206,78.72637515)
\curveto(596.32910206,77.65606265)(596.47363331,76.96856265)(596.76269581,76.66387515)
\curveto(597.05957081,76.35918765)(597.65722706,76.2068439)(598.55566456,76.2068439)
\lineto(600.77050831,76.2068439)
\lineto(600.77050831,74.4021564)
\lineto(598.55566456,74.4021564)
\curveto(596.89160206,74.4021564)(595.74316456,74.71075015)(595.11035206,75.32793765)
\curveto(594.47753956,75.95293765)(594.16113331,77.08575015)(594.16113331,78.72637515)
\lineto(594.16113331,85.85137515)
\lineto(592.57910206,85.85137515)
\lineto(592.57910206,87.5271564)
\lineto(594.16113331,87.5271564)
\lineto(594.16113331,91.2537189)
\closepath
}
}
{
\newrgbcolor{curcolor}{0 0 0}
\pscustom[linestyle=none,fillstyle=solid,fillcolor=curcolor]
{
\newpath
\moveto(614.84472706,81.5037189)
\lineto(614.84472706,80.4490314)
\lineto(604.93066456,80.4490314)
\curveto(605.02441456,78.9646564)(605.46972706,77.8318439)(606.26660206,77.0505939)
\curveto(607.07128956,76.2771564)(608.18847706,75.89043765)(609.61816456,75.89043765)
\curveto(610.44628956,75.89043765)(611.24707081,75.99200015)(612.02050831,76.19512515)
\curveto(612.80175831,76.39825015)(613.57519581,76.70293765)(614.34082081,77.10918765)
\lineto(614.34082081,75.07012515)
\curveto(613.56738331,74.74200015)(612.77441456,74.49200015)(611.96191456,74.32012515)
\curveto(611.14941456,74.14825015)(610.32519581,74.06231265)(609.48925831,74.06231265)
\curveto(607.39550831,74.06231265)(605.73535206,74.67168765)(604.50878956,75.89043765)
\curveto(603.29003956,77.10918765)(602.68066456,78.75762515)(602.68066456,80.83575015)
\curveto(602.68066456,82.98418765)(603.25878956,84.68731265)(604.41503956,85.94512515)
\curveto(605.57910206,87.21075015)(607.14550831,87.84356265)(609.11425831,87.84356265)
\curveto(610.87988331,87.84356265)(612.27441456,87.27325015)(613.29785206,86.13262515)
\curveto(614.32910206,84.99981265)(614.84472706,83.4568439)(614.84472706,81.5037189)
\closepath
\moveto(612.68847706,82.1365314)
\curveto(612.67285206,83.3162189)(612.34082081,84.25762515)(611.69238331,84.96075015)
\curveto(611.05175831,85.66387515)(610.20019581,86.01543765)(609.13769581,86.01543765)
\curveto(607.93457081,86.01543765)(606.96972706,85.6755939)(606.24316456,84.9959064)
\curveto(605.52441456,84.3162189)(605.11035206,83.35918765)(605.00097706,82.12481265)
\closepath
}
}
{
\newrgbcolor{curcolor}{0 0 0}
\pscustom[linestyle=none,fillstyle=solid,fillcolor=curcolor]
{
\newpath
\moveto(628.87207081,87.5271564)
\lineto(624.12597706,81.14043765)
\lineto(629.11816456,74.4021564)
\lineto(626.57519581,74.4021564)
\lineto(622.75488331,79.5584064)
\lineto(618.93457081,74.4021564)
\lineto(616.39160206,74.4021564)
\lineto(621.48925831,81.2693439)
\lineto(616.82519581,87.5271564)
\lineto(619.36816456,87.5271564)
\lineto(622.84863331,82.85137515)
\lineto(626.32910206,87.5271564)
\closepath
}
}
{
\newrgbcolor{curcolor}{0 0 0}
\pscustom[linestyle=none,fillstyle=solid,fillcolor=curcolor]
{
\newpath
\moveto(634.29785206,91.2537189)
\lineto(634.29785206,87.5271564)
\lineto(638.73925831,87.5271564)
\lineto(638.73925831,85.85137515)
\lineto(634.29785206,85.85137515)
\lineto(634.29785206,78.72637515)
\curveto(634.29785206,77.65606265)(634.44238331,76.96856265)(634.73144581,76.66387515)
\curveto(635.02832081,76.35918765)(635.62597706,76.2068439)(636.52441456,76.2068439)
\lineto(638.73925831,76.2068439)
\lineto(638.73925831,74.4021564)
\lineto(636.52441456,74.4021564)
\curveto(634.86035206,74.4021564)(633.71191456,74.71075015)(633.07910206,75.32793765)
\curveto(632.44628956,75.95293765)(632.12988331,77.08575015)(632.12988331,78.72637515)
\lineto(632.12988331,85.85137515)
\lineto(630.54785206,85.85137515)
\lineto(630.54785206,87.5271564)
\lineto(632.12988331,87.5271564)
\lineto(632.12988331,91.2537189)
\closepath
}
}
{
\newrgbcolor{curcolor}{0 0 0}
\pscustom[linestyle=none,fillstyle=solid,fillcolor=curcolor]
{
\newpath
\moveto(651.00878956,80.95293765)
\curveto(651.00878956,82.53887515)(650.68066456,83.78106265)(650.02441456,84.67950015)
\curveto(649.37597706,85.58575015)(648.48144581,86.03887515)(647.34082081,86.03887515)
\curveto(646.20019581,86.03887515)(645.30175831,85.58575015)(644.64550831,84.67950015)
\curveto(643.99707081,83.78106265)(643.67285206,82.53887515)(643.67285206,80.95293765)
\curveto(643.67285206,79.36700015)(643.99707081,78.1209064)(644.64550831,77.2146564)
\curveto(645.30175831,76.3162189)(646.20019581,75.86700015)(647.34082081,75.86700015)
\curveto(648.48144581,75.86700015)(649.37597706,76.3162189)(650.02441456,77.2146564)
\curveto(650.68066456,78.1209064)(651.00878956,79.36700015)(651.00878956,80.95293765)
\closepath
\moveto(643.67285206,85.5349689)
\curveto(644.12597706,86.3162189)(644.69628956,86.8943439)(645.38378956,87.2693439)
\curveto(646.07910206,87.6521564)(646.90722706,87.84356265)(647.86816456,87.84356265)
\curveto(649.46191456,87.84356265)(650.75488331,87.21075015)(651.74707081,85.94512515)
\curveto(652.74707081,84.67950015)(653.24707081,83.01543765)(653.24707081,80.95293765)
\curveto(653.24707081,78.89043765)(652.74707081,77.22637515)(651.74707081,75.96075015)
\curveto(650.75488331,74.69512515)(649.46191456,74.06231265)(647.86816456,74.06231265)
\curveto(646.90722706,74.06231265)(646.07910206,74.24981265)(645.38378956,74.62481265)
\curveto(644.69628956,75.00762515)(644.12597706,75.5896564)(643.67285206,76.3709064)
\lineto(643.67285206,74.4021564)
\lineto(641.50488331,74.4021564)
\lineto(641.50488331,92.6365314)
\lineto(643.67285206,92.6365314)
\closepath
}
}
{
\newrgbcolor{curcolor}{0 0 0}
\pscustom[linestyle=none,fillstyle=solid,fillcolor=curcolor]
{
\newpath
\moveto(663.46582081,92.6365314)
\lineto(663.46582081,90.84356265)
\lineto(661.40332081,90.84356265)
\curveto(660.62988331,90.84356265)(660.09082081,90.68731265)(659.78613331,90.37481265)
\curveto(659.48925831,90.06231265)(659.34082081,89.49981265)(659.34082081,88.68731265)
\lineto(659.34082081,87.5271564)
\lineto(662.89160206,87.5271564)
\lineto(662.89160206,85.85137515)
\lineto(659.34082081,85.85137515)
\lineto(659.34082081,74.4021564)
\lineto(657.17285206,74.4021564)
\lineto(657.17285206,85.85137515)
\lineto(655.11035206,85.85137515)
\lineto(655.11035206,87.5271564)
\lineto(657.17285206,87.5271564)
\lineto(657.17285206,88.4412189)
\curveto(657.17285206,89.9021564)(657.51269581,90.9646564)(658.19238331,91.6287189)
\curveto(658.87207081,92.3005939)(659.95019581,92.6365314)(661.42675831,92.6365314)
\closepath
}
}
{
\newrgbcolor{curcolor}{0 0 0}
\pscustom[linestyle=none,fillstyle=solid,fillcolor=curcolor]
{
\newpath
\moveto(675.26660206,72.1755939)
\lineto(675.26660206,70.4880939)
\lineto(674.54003956,70.4880939)
\curveto(672.59472706,70.4880939)(671.29003956,70.7771564)(670.62597706,71.3552814)
\curveto(669.96972706,71.9334064)(669.64160206,73.08575015)(669.64160206,74.81231265)
\lineto(669.64160206,77.6130939)
\curveto(669.64160206,78.7927814)(669.43066456,79.60918765)(669.00878956,80.06231265)
\curveto(668.58691456,80.51543765)(667.82128956,80.74200015)(666.71191456,80.74200015)
\lineto(665.99707081,80.74200015)
\lineto(665.99707081,82.4177814)
\lineto(666.71191456,82.4177814)
\curveto(667.82910206,82.4177814)(668.59472706,82.64043765)(669.00878956,83.08575015)
\curveto(669.43066456,83.53887515)(669.64160206,84.3474689)(669.64160206,85.5115314)
\lineto(669.64160206,88.3240314)
\curveto(669.64160206,90.0505939)(669.96972706,91.1990314)(670.62597706,91.7693439)
\curveto(671.29003956,92.3474689)(672.59472706,92.6365314)(674.54003956,92.6365314)
\lineto(675.26660206,92.6365314)
\lineto(675.26660206,90.96075015)
\lineto(674.46972706,90.96075015)
\curveto(673.36816456,90.96075015)(672.64941456,90.78887515)(672.31347706,90.44512515)
\curveto(671.97753956,90.10137515)(671.80957081,89.3787189)(671.80957081,88.2771564)
\lineto(671.80957081,85.3709064)
\curveto(671.80957081,84.1443439)(671.62988331,83.2537189)(671.27050831,82.6990314)
\curveto(670.91894581,82.1443439)(670.31347706,81.7693439)(669.45410206,81.5740314)
\curveto(670.32128956,81.3630939)(670.93066456,80.9802814)(671.28222706,80.4255939)
\curveto(671.63378956,79.8709064)(671.80957081,78.98418765)(671.80957081,77.76543765)
\lineto(671.80957081,74.85918765)
\curveto(671.80957081,73.75762515)(671.97753956,73.0349689)(672.31347706,72.6912189)
\curveto(672.64941456,72.3474689)(673.36816456,72.1755939)(674.46972706,72.1755939)
\closepath
}
}
{
\newrgbcolor{curcolor}{0 0 0}
\pscustom[linestyle=none,fillstyle=solid,fillcolor=curcolor]
{
\newpath
\moveto(682.67285206,91.2537189)
\lineto(682.67285206,87.5271564)
\lineto(687.11425831,87.5271564)
\lineto(687.11425831,85.85137515)
\lineto(682.67285206,85.85137515)
\lineto(682.67285206,78.72637515)
\curveto(682.67285206,77.65606265)(682.81738331,76.96856265)(683.10644581,76.66387515)
\curveto(683.40332081,76.35918765)(684.00097706,76.2068439)(684.89941456,76.2068439)
\lineto(687.11425831,76.2068439)
\lineto(687.11425831,74.4021564)
\lineto(684.89941456,74.4021564)
\curveto(683.23535206,74.4021564)(682.08691456,74.71075015)(681.45410206,75.32793765)
\curveto(680.82128956,75.95293765)(680.50488331,77.08575015)(680.50488331,78.72637515)
\lineto(680.50488331,85.85137515)
\lineto(678.92285206,85.85137515)
\lineto(678.92285206,87.5271564)
\lineto(680.50488331,87.5271564)
\lineto(680.50488331,91.2537189)
\closepath
}
}
{
\newrgbcolor{curcolor}{0 0 0}
\pscustom[linestyle=none,fillstyle=solid,fillcolor=curcolor]
{
\newpath
\moveto(701.18847706,81.5037189)
\lineto(701.18847706,80.4490314)
\lineto(691.27441456,80.4490314)
\curveto(691.36816456,78.9646564)(691.81347706,77.8318439)(692.61035206,77.0505939)
\curveto(693.41503956,76.2771564)(694.53222706,75.89043765)(695.96191456,75.89043765)
\curveto(696.79003956,75.89043765)(697.59082081,75.99200015)(698.36425831,76.19512515)
\curveto(699.14550831,76.39825015)(699.91894581,76.70293765)(700.68457081,77.10918765)
\lineto(700.68457081,75.07012515)
\curveto(699.91113331,74.74200015)(699.11816456,74.49200015)(698.30566456,74.32012515)
\curveto(697.49316456,74.14825015)(696.66894581,74.06231265)(695.83300831,74.06231265)
\curveto(693.73925831,74.06231265)(692.07910206,74.67168765)(690.85253956,75.89043765)
\curveto(689.63378956,77.10918765)(689.02441456,78.75762515)(689.02441456,80.83575015)
\curveto(689.02441456,82.98418765)(689.60253956,84.68731265)(690.75878956,85.94512515)
\curveto(691.92285206,87.21075015)(693.48925831,87.84356265)(695.45800831,87.84356265)
\curveto(697.22363331,87.84356265)(698.61816456,87.27325015)(699.64160206,86.13262515)
\curveto(700.67285206,84.99981265)(701.18847706,83.4568439)(701.18847706,81.5037189)
\closepath
\moveto(699.03222706,82.1365314)
\curveto(699.01660206,83.3162189)(698.68457081,84.25762515)(698.03613331,84.96075015)
\curveto(697.39550831,85.66387515)(696.54394581,86.01543765)(695.48144581,86.01543765)
\curveto(694.27832081,86.01543765)(693.31347706,85.6755939)(692.58691456,84.9959064)
\curveto(691.86816456,84.3162189)(691.45410206,83.35918765)(691.34472706,82.12481265)
\closepath
}
}
{
\newrgbcolor{curcolor}{0 0 0}
\pscustom[linestyle=none,fillstyle=solid,fillcolor=curcolor]
{
\newpath
\moveto(713.09472706,87.14043765)
\lineto(713.09472706,85.10137515)
\curveto(712.48535206,85.41387515)(711.85253956,85.64825015)(711.19628956,85.80450015)
\curveto(710.54003956,85.96075015)(709.86035206,86.03887515)(709.15722706,86.03887515)
\curveto(708.08691456,86.03887515)(707.28222706,85.87481265)(706.74316456,85.54668765)
\curveto(706.21191456,85.21856265)(705.94628956,84.72637515)(705.94628956,84.07012515)
\curveto(705.94628956,83.57012515)(706.13769581,83.1755939)(706.52050831,82.8865314)
\curveto(706.90332081,82.6052814)(707.67285206,82.33575015)(708.82910206,82.07793765)
\lineto(709.56738331,81.91387515)
\curveto(711.09863331,81.58575015)(712.18457081,81.1209064)(712.82519581,80.5193439)
\curveto(713.47363331,79.9255939)(713.79785206,79.09356265)(713.79785206,78.02325015)
\curveto(713.79785206,76.80450015)(713.31347706,75.8396564)(712.34472706,75.1287189)
\curveto(711.38378956,74.4177814)(710.05957081,74.06231265)(708.37207081,74.06231265)
\curveto(707.66894581,74.06231265)(706.93457081,74.13262515)(706.16894581,74.27325015)
\curveto(705.41113331,74.40606265)(704.61035206,74.60918765)(703.76660206,74.88262515)
\lineto(703.76660206,77.10918765)
\curveto(704.56347706,76.69512515)(705.34863331,76.38262515)(706.12207081,76.17168765)
\curveto(706.89550831,75.96856265)(707.66113331,75.86700015)(708.41894581,75.86700015)
\curveto(709.43457081,75.86700015)(710.21582081,76.03887515)(710.76269581,76.38262515)
\curveto(711.30957081,76.73418765)(711.58300831,77.22637515)(711.58300831,77.85918765)
\curveto(711.58300831,78.44512515)(711.38378956,78.8943439)(710.98535206,79.2068439)
\curveto(710.59472706,79.5193439)(709.73144581,79.82012515)(708.39550831,80.10918765)
\lineto(707.64550831,80.2849689)
\curveto(706.30957081,80.5662189)(705.34472706,80.9959064)(704.75097706,81.5740314)
\curveto(704.15722706,82.1599689)(703.86035206,82.96075015)(703.86035206,83.97637515)
\curveto(703.86035206,85.21075015)(704.29785206,86.16387515)(705.17285206,86.83575015)
\curveto(706.04785206,87.50762515)(707.29003956,87.84356265)(708.89941456,87.84356265)
\curveto(709.69628956,87.84356265)(710.44628956,87.7849689)(711.14941456,87.6677814)
\curveto(711.85253956,87.5505939)(712.50097706,87.37481265)(713.09472706,87.14043765)
\closepath
}
}
{
\newrgbcolor{curcolor}{0 0 0}
\pscustom[linestyle=none,fillstyle=solid,fillcolor=curcolor]
{
\newpath
\moveto(719.37597706,91.2537189)
\lineto(719.37597706,87.5271564)
\lineto(723.81738331,87.5271564)
\lineto(723.81738331,85.85137515)
\lineto(719.37597706,85.85137515)
\lineto(719.37597706,78.72637515)
\curveto(719.37597706,77.65606265)(719.52050831,76.96856265)(719.80957081,76.66387515)
\curveto(720.10644581,76.35918765)(720.70410206,76.2068439)(721.60253956,76.2068439)
\lineto(723.81738331,76.2068439)
\lineto(723.81738331,74.4021564)
\lineto(721.60253956,74.4021564)
\curveto(719.93847706,74.4021564)(718.79003956,74.71075015)(718.15722706,75.32793765)
\curveto(717.52441456,75.95293765)(717.20800831,77.08575015)(717.20800831,78.72637515)
\lineto(717.20800831,85.85137515)
\lineto(715.62597706,85.85137515)
\lineto(715.62597706,87.5271564)
\lineto(717.20800831,87.5271564)
\lineto(717.20800831,91.2537189)
\closepath
}
}
{
\newrgbcolor{curcolor}{0 0 0}
\pscustom[linestyle=none,fillstyle=solid,fillcolor=curcolor]
{
\newpath
\moveto(727.40332081,72.1755939)
\lineto(728.22363331,72.1755939)
\curveto(729.31738331,72.1755939)(730.02832081,72.34356265)(730.35644581,72.67950015)
\curveto(730.69238331,73.01543765)(730.86035206,73.74200015)(730.86035206,74.85918765)
\lineto(730.86035206,77.76543765)
\curveto(730.86035206,78.98418765)(731.03613331,79.8709064)(731.38769581,80.4255939)
\curveto(731.73925831,80.9802814)(732.34863331,81.3630939)(733.21582081,81.5740314)
\curveto(732.34863331,81.7693439)(731.73925831,82.1443439)(731.38769581,82.6990314)
\curveto(731.03613331,83.2537189)(730.86035206,84.1443439)(730.86035206,85.3709064)
\lineto(730.86035206,88.2771564)
\curveto(730.86035206,89.3865314)(730.69238331,90.10918765)(730.35644581,90.44512515)
\curveto(730.02832081,90.78887515)(729.31738331,90.96075015)(728.22363331,90.96075015)
\lineto(727.40332081,90.96075015)
\lineto(727.40332081,92.6365314)
\lineto(728.14160206,92.6365314)
\curveto(730.08691456,92.6365314)(731.38378956,92.3474689)(732.03222706,91.7693439)
\curveto(732.68847706,91.1990314)(733.01660206,90.0505939)(733.01660206,88.3240314)
\lineto(733.01660206,85.5115314)
\curveto(733.01660206,84.3474689)(733.22753956,83.53887515)(733.64941456,83.08575015)
\curveto(734.07128956,82.64043765)(734.83691456,82.4177814)(735.94628956,82.4177814)
\lineto(736.67285206,82.4177814)
\lineto(736.67285206,80.74200015)
\lineto(735.94628956,80.74200015)
\curveto(734.83691456,80.74200015)(734.07128956,80.51543765)(733.64941456,80.06231265)
\curveto(733.22753956,79.60918765)(733.01660206,78.7927814)(733.01660206,77.6130939)
\lineto(733.01660206,74.81231265)
\curveto(733.01660206,73.08575015)(732.68847706,71.9334064)(732.03222706,71.3552814)
\curveto(731.38378956,70.7771564)(730.08691456,70.4880939)(728.14160206,70.4880939)
\lineto(727.40332081,70.4880939)
\closepath
}
}
{
\newrgbcolor{curcolor}{0 0 0}
\pscustom[linewidth=5.33333302,linecolor=curcolor]
{
\newpath
\moveto(250.38490667,43.93332657)
\lineto(250.38490667,33.29190924)
}
}
{
\newrgbcolor{curcolor}{0 0 0}
\pscustom[linewidth=5.33333302,linecolor=curcolor]
{
\newpath
\moveto(404.38490667,43.93332657)
\lineto(404.38490667,33.29190924)
}
}
{
\newrgbcolor{curcolor}{0 0 0}
\pscustom[linewidth=5.33333302,linecolor=curcolor]
{
\newpath
\moveto(606.38490667,43.93332657)
\lineto(606.38490667,33.29190924)
}
}
\end{pspicture}

    \caption[Experimental paradigm for contextual fear conditioning.]{Experimental paradigm. Adult \gls{wt} and \gls{tg} mice (with gCaMP6f expression) were implanted with a mini-microscope base-plate targeting CA1 hippocampus on day 0. Fluorescent cells were visible three weeks later. Mice received \tglu{} peptide or vehicle (i.p.) \SI{1}{\hour} before contextual fear conditioning. Mice were tested \SI{24}{\hour} later for freezing behaviour. Calcium transients were recorded for both training and testing session. \label{f.ad.paradigm}}
\end{figure}

\begin{figure}[h]
    %% Creator: Matplotlib, PGF backend
%%
%% To include the figure in your LaTeX document, write
%%   \input{<filename>.pgf}
%%
%% Make sure the required packages are loaded in your preamble
%%   \usepackage{pgf}
%%
%% Figures using additional raster images can only be included by \input if
%% they are in the same directory as the main LaTeX file. For loading figures
%% from other directories you can use the `import` package
%%   \usepackage{import}
%% and then include the figures with
%%   \import{<path to file>}{<filename>.pgf}
%%
%% Matplotlib used the following preamble
%%   \usepackage[utf8]{inputenc}
%%   \usepackage[T1]{fontenc}
%%   \usepackage{siunitx}
%%
\begingroup%
\makeatletter%
\begin{pgfpicture}%
\pgfpathrectangle{\pgfpointorigin}{\pgfqpoint{6.590038in}{2.443496in}}%
\pgfusepath{use as bounding box, clip}%
\begin{pgfscope}%
\pgfsetbuttcap%
\pgfsetmiterjoin%
\definecolor{currentfill}{rgb}{1.000000,1.000000,1.000000}%
\pgfsetfillcolor{currentfill}%
\pgfsetlinewidth{0.000000pt}%
\definecolor{currentstroke}{rgb}{1.000000,1.000000,1.000000}%
\pgfsetstrokecolor{currentstroke}%
\pgfsetdash{}{0pt}%
\pgfpathmoveto{\pgfqpoint{0.000000in}{0.000000in}}%
\pgfpathlineto{\pgfqpoint{6.590038in}{0.000000in}}%
\pgfpathlineto{\pgfqpoint{6.590038in}{2.443496in}}%
\pgfpathlineto{\pgfqpoint{0.000000in}{2.443496in}}%
\pgfpathclose%
\pgfusepath{fill}%
\end{pgfscope}%
\begin{pgfscope}%
\pgfpathrectangle{\pgfqpoint{0.405556in}{0.414428in}}{\pgfqpoint{1.993438in}{1.804337in}} %
\pgfusepath{clip}%
\pgftext[at=\pgfqpoint{0.405556in}{0.414428in},left,bottom]{\pgfimage[interpolate=true,width=1.996667in,height=1.810000in]{../figs/ad/sample_trace-img0.png}}%
\end{pgfscope}%
\begin{pgfscope}%
\pgfpathrectangle{\pgfqpoint{0.405556in}{0.414428in}}{\pgfqpoint{1.993438in}{1.804337in}} %
\pgfusepath{clip}%
\pgfsetbuttcap%
\pgfsetmiterjoin%
\pgfsetlinewidth{1.505625pt}%
\definecolor{currentstroke}{rgb}{1.000000,0.000000,0.000000}%
\pgfsetstrokecolor{currentstroke}%
\pgfsetdash{{6.000000pt}{6.000000pt}}{0.000000pt}%
\pgfpathmoveto{\pgfqpoint{1.390457in}{1.627825in}}%
\pgfpathlineto{\pgfqpoint{1.784417in}{1.627825in}}%
\pgfpathlineto{\pgfqpoint{1.784417in}{1.233865in}}%
\pgfpathlineto{\pgfqpoint{1.390457in}{1.233865in}}%
\pgfpathlineto{\pgfqpoint{1.390457in}{1.627825in}}%
\pgfpathclose%
\pgfusepath{stroke}%
\end{pgfscope}%
\begin{pgfscope}%
\pgfpathrectangle{\pgfqpoint{0.405556in}{0.414428in}}{\pgfqpoint{1.993438in}{1.804337in}} %
\pgfusepath{clip}%
\pgfsetbuttcap%
\pgfsetroundjoin%
\pgfsetlinewidth{1.505625pt}%
\definecolor{currentstroke}{rgb}{1.000000,0.000000,0.000000}%
\pgfsetstrokecolor{currentstroke}%
\pgfsetdash{{6.000000pt}{6.000000pt}}{0.000000pt}%
\pgfpathmoveto{\pgfqpoint{1.784417in}{1.627825in}}%
\pgfpathlineto{\pgfqpoint{2.398995in}{2.218765in}}%
\pgfusepath{stroke}%
\end{pgfscope}%
\begin{pgfscope}%
\pgfpathrectangle{\pgfqpoint{0.405556in}{0.414428in}}{\pgfqpoint{1.993438in}{1.804337in}} %
\pgfusepath{clip}%
\pgfsetbuttcap%
\pgfsetroundjoin%
\pgfsetlinewidth{1.505625pt}%
\definecolor{currentstroke}{rgb}{1.000000,0.000000,0.000000}%
\pgfsetstrokecolor{currentstroke}%
\pgfsetdash{{6.000000pt}{6.000000pt}}{0.000000pt}%
\pgfpathmoveto{\pgfqpoint{1.784417in}{1.233865in}}%
\pgfpathlineto{\pgfqpoint{2.398995in}{0.414428in}}%
\pgfusepath{stroke}%
\end{pgfscope}%
\begin{pgfscope}%
\pgfpathrectangle{\pgfqpoint{2.398995in}{0.319877in}}{\pgfqpoint{3.986877in}{1.993438in}} %
\pgfusepath{clip}%
\pgfsetbuttcap%
\pgfsetroundjoin%
\definecolor{currentfill}{rgb}{1.000000,1.000000,1.000000}%
\pgfsetfillcolor{currentfill}%
\pgfsetlinewidth{1.003750pt}%
\definecolor{currentstroke}{rgb}{0.220720,0.660996,0.784811}%
\pgfsetstrokecolor{currentstroke}%
\pgfsetdash{}{0pt}%
\pgfpathmoveto{\pgfqpoint{2.398995in}{0.413320in}}%
\pgfpathlineto{\pgfqpoint{2.398995in}{2.164637in}}%
\pgfpathlineto{\pgfqpoint{2.401675in}{2.163858in}}%
\pgfpathlineto{\pgfqpoint{2.404352in}{2.162254in}}%
\pgfpathlineto{\pgfqpoint{2.407024in}{2.159427in}}%
\pgfpathlineto{\pgfqpoint{2.409699in}{2.159404in}}%
\pgfpathlineto{\pgfqpoint{2.412389in}{2.160766in}}%
\pgfpathlineto{\pgfqpoint{2.415184in}{2.159688in}}%
\pgfpathlineto{\pgfqpoint{2.417747in}{2.161401in}}%
\pgfpathlineto{\pgfqpoint{2.420528in}{2.160843in}}%
\pgfpathlineto{\pgfqpoint{2.423098in}{2.154903in}}%
\pgfpathlineto{\pgfqpoint{2.425878in}{2.156966in}}%
\pgfpathlineto{\pgfqpoint{2.428453in}{2.156877in}}%
\pgfpathlineto{\pgfqpoint{2.431251in}{2.158187in}}%
\pgfpathlineto{\pgfqpoint{2.433815in}{2.158488in}}%
\pgfpathlineto{\pgfqpoint{2.436518in}{2.159472in}}%
\pgfpathlineto{\pgfqpoint{2.439167in}{2.156066in}}%
\pgfpathlineto{\pgfqpoint{2.441876in}{2.154664in}}%
\pgfpathlineto{\pgfqpoint{2.444677in}{2.154664in}}%
\pgfpathlineto{\pgfqpoint{2.447209in}{2.155688in}}%
\pgfpathlineto{\pgfqpoint{2.450032in}{2.156972in}}%
\pgfpathlineto{\pgfqpoint{2.452562in}{2.160353in}}%
\pgfpathlineto{\pgfqpoint{2.455353in}{2.161214in}}%
\pgfpathlineto{\pgfqpoint{2.457917in}{2.161742in}}%
\pgfpathlineto{\pgfqpoint{2.460711in}{2.163127in}}%
\pgfpathlineto{\pgfqpoint{2.463280in}{2.159077in}}%
\pgfpathlineto{\pgfqpoint{2.465957in}{2.163820in}}%
\pgfpathlineto{\pgfqpoint{2.468635in}{2.160561in}}%
\pgfpathlineto{\pgfqpoint{2.471311in}{2.158860in}}%
\pgfpathlineto{\pgfqpoint{2.473989in}{2.157347in}}%
\pgfpathlineto{\pgfqpoint{2.476671in}{2.154826in}}%
\pgfpathlineto{\pgfqpoint{2.479420in}{2.156566in}}%
\pgfpathlineto{\pgfqpoint{2.482026in}{2.156594in}}%
\pgfpathlineto{\pgfqpoint{2.484870in}{2.154664in}}%
\pgfpathlineto{\pgfqpoint{2.487384in}{2.154664in}}%
\pgfpathlineto{\pgfqpoint{2.490183in}{2.154664in}}%
\pgfpathlineto{\pgfqpoint{2.492729in}{2.154664in}}%
\pgfpathlineto{\pgfqpoint{2.495542in}{2.158658in}}%
\pgfpathlineto{\pgfqpoint{2.498085in}{2.155477in}}%
\pgfpathlineto{\pgfqpoint{2.500801in}{2.159769in}}%
\pgfpathlineto{\pgfqpoint{2.503454in}{2.157890in}}%
\pgfpathlineto{\pgfqpoint{2.506163in}{2.159482in}}%
\pgfpathlineto{\pgfqpoint{2.508917in}{2.161534in}}%
\pgfpathlineto{\pgfqpoint{2.511478in}{2.164777in}}%
\pgfpathlineto{\pgfqpoint{2.514268in}{2.161577in}}%
\pgfpathlineto{\pgfqpoint{2.516845in}{2.163915in}}%
\pgfpathlineto{\pgfqpoint{2.519607in}{2.163171in}}%
\pgfpathlineto{\pgfqpoint{2.522197in}{2.166856in}}%
\pgfpathlineto{\pgfqpoint{2.524988in}{2.167930in}}%
\pgfpathlineto{\pgfqpoint{2.527560in}{2.165523in}}%
\pgfpathlineto{\pgfqpoint{2.530234in}{2.170564in}}%
\pgfpathlineto{\pgfqpoint{2.532917in}{2.163172in}}%
\pgfpathlineto{\pgfqpoint{2.535624in}{2.159227in}}%
\pgfpathlineto{\pgfqpoint{2.538274in}{2.155947in}}%
\pgfpathlineto{\pgfqpoint{2.540949in}{2.160622in}}%
\pgfpathlineto{\pgfqpoint{2.543765in}{2.163468in}}%
\pgfpathlineto{\pgfqpoint{2.546310in}{2.163699in}}%
\pgfpathlineto{\pgfqpoint{2.549114in}{2.160587in}}%
\pgfpathlineto{\pgfqpoint{2.551664in}{2.158052in}}%
\pgfpathlineto{\pgfqpoint{2.554493in}{2.154664in}}%
\pgfpathlineto{\pgfqpoint{2.557009in}{2.154664in}}%
\pgfpathlineto{\pgfqpoint{2.559790in}{2.160117in}}%
\pgfpathlineto{\pgfqpoint{2.562375in}{2.164118in}}%
\pgfpathlineto{\pgfqpoint{2.565045in}{2.158192in}}%
\pgfpathlineto{\pgfqpoint{2.567730in}{2.156462in}}%
\pgfpathlineto{\pgfqpoint{2.570411in}{2.157179in}}%
\pgfpathlineto{\pgfqpoint{2.573082in}{2.156366in}}%
\pgfpathlineto{\pgfqpoint{2.575779in}{2.154728in}}%
\pgfpathlineto{\pgfqpoint{2.578567in}{2.155659in}}%
\pgfpathlineto{\pgfqpoint{2.581129in}{2.156672in}}%
\pgfpathlineto{\pgfqpoint{2.583913in}{2.161517in}}%
\pgfpathlineto{\pgfqpoint{2.586484in}{2.157570in}}%
\pgfpathlineto{\pgfqpoint{2.589248in}{2.158806in}}%
\pgfpathlineto{\pgfqpoint{2.591842in}{2.157644in}}%
\pgfpathlineto{\pgfqpoint{2.594630in}{2.155484in}}%
\pgfpathlineto{\pgfqpoint{2.597196in}{2.155658in}}%
\pgfpathlineto{\pgfqpoint{2.599920in}{2.157297in}}%
\pgfpathlineto{\pgfqpoint{2.602557in}{2.159033in}}%
\pgfpathlineto{\pgfqpoint{2.605232in}{2.154664in}}%
\pgfpathlineto{\pgfqpoint{2.608004in}{2.157913in}}%
\pgfpathlineto{\pgfqpoint{2.610588in}{2.158557in}}%
\pgfpathlineto{\pgfqpoint{2.613393in}{2.156004in}}%
\pgfpathlineto{\pgfqpoint{2.615934in}{2.154664in}}%
\pgfpathlineto{\pgfqpoint{2.618773in}{2.158217in}}%
\pgfpathlineto{\pgfqpoint{2.621304in}{2.167138in}}%
\pgfpathlineto{\pgfqpoint{2.624077in}{2.166573in}}%
\pgfpathlineto{\pgfqpoint{2.626653in}{2.168187in}}%
\pgfpathlineto{\pgfqpoint{2.629340in}{2.165152in}}%
\pgfpathlineto{\pgfqpoint{2.632018in}{2.167198in}}%
\pgfpathlineto{\pgfqpoint{2.634700in}{2.165017in}}%
\pgfpathlineto{\pgfqpoint{2.637369in}{2.167106in}}%
\pgfpathlineto{\pgfqpoint{2.640053in}{2.161652in}}%
\pgfpathlineto{\pgfqpoint{2.642827in}{2.165875in}}%
\pgfpathlineto{\pgfqpoint{2.645408in}{2.168099in}}%
\pgfpathlineto{\pgfqpoint{2.648196in}{2.172713in}}%
\pgfpathlineto{\pgfqpoint{2.650767in}{2.176711in}}%
\pgfpathlineto{\pgfqpoint{2.653567in}{2.173142in}}%
\pgfpathlineto{\pgfqpoint{2.656124in}{2.165576in}}%
\pgfpathlineto{\pgfqpoint{2.658942in}{2.159568in}}%
\pgfpathlineto{\pgfqpoint{2.661481in}{2.154665in}}%
\pgfpathlineto{\pgfqpoint{2.664151in}{2.154664in}}%
\pgfpathlineto{\pgfqpoint{2.666836in}{2.154664in}}%
\pgfpathlineto{\pgfqpoint{2.669506in}{2.157137in}}%
\pgfpathlineto{\pgfqpoint{2.672301in}{2.161914in}}%
\pgfpathlineto{\pgfqpoint{2.674873in}{2.169759in}}%
\pgfpathlineto{\pgfqpoint{2.677650in}{2.169298in}}%
\pgfpathlineto{\pgfqpoint{2.680224in}{2.174032in}}%
\pgfpathlineto{\pgfqpoint{2.683009in}{2.174194in}}%
\pgfpathlineto{\pgfqpoint{2.685586in}{2.167989in}}%
\pgfpathlineto{\pgfqpoint{2.688328in}{2.162785in}}%
\pgfpathlineto{\pgfqpoint{2.690940in}{2.163910in}}%
\pgfpathlineto{\pgfqpoint{2.693611in}{2.166936in}}%
\pgfpathlineto{\pgfqpoint{2.696293in}{2.168052in}}%
\pgfpathlineto{\pgfqpoint{2.698968in}{2.164651in}}%
\pgfpathlineto{\pgfqpoint{2.701657in}{2.163833in}}%
\pgfpathlineto{\pgfqpoint{2.704326in}{2.164643in}}%
\pgfpathlineto{\pgfqpoint{2.707125in}{2.158302in}}%
\pgfpathlineto{\pgfqpoint{2.709683in}{2.163222in}}%
\pgfpathlineto{\pgfqpoint{2.712477in}{2.161240in}}%
\pgfpathlineto{\pgfqpoint{2.715036in}{2.164334in}}%
\pgfpathlineto{\pgfqpoint{2.717773in}{2.165026in}}%
\pgfpathlineto{\pgfqpoint{2.720404in}{2.160411in}}%
\pgfpathlineto{\pgfqpoint{2.723211in}{2.164136in}}%
\pgfpathlineto{\pgfqpoint{2.725760in}{2.164131in}}%
\pgfpathlineto{\pgfqpoint{2.728439in}{2.166779in}}%
\pgfpathlineto{\pgfqpoint{2.731119in}{2.169670in}}%
\pgfpathlineto{\pgfqpoint{2.733798in}{2.163597in}}%
\pgfpathlineto{\pgfqpoint{2.736476in}{2.160666in}}%
\pgfpathlineto{\pgfqpoint{2.739155in}{2.156156in}}%
\pgfpathlineto{\pgfqpoint{2.741928in}{2.161916in}}%
\pgfpathlineto{\pgfqpoint{2.744510in}{2.160849in}}%
\pgfpathlineto{\pgfqpoint{2.747260in}{2.159580in}}%
\pgfpathlineto{\pgfqpoint{2.749868in}{2.161273in}}%
\pgfpathlineto{\pgfqpoint{2.752614in}{2.158975in}}%
\pgfpathlineto{\pgfqpoint{2.755224in}{2.159966in}}%
\pgfpathlineto{\pgfqpoint{2.758028in}{2.158411in}}%
\pgfpathlineto{\pgfqpoint{2.760581in}{2.158341in}}%
\pgfpathlineto{\pgfqpoint{2.763253in}{2.158250in}}%
\pgfpathlineto{\pgfqpoint{2.765935in}{2.156807in}}%
\pgfpathlineto{\pgfqpoint{2.768617in}{2.156839in}}%
\pgfpathlineto{\pgfqpoint{2.771373in}{2.155257in}}%
\pgfpathlineto{\pgfqpoint{2.773972in}{2.161082in}}%
\pgfpathlineto{\pgfqpoint{2.776767in}{2.159166in}}%
\pgfpathlineto{\pgfqpoint{2.779330in}{2.159881in}}%
\pgfpathlineto{\pgfqpoint{2.782113in}{2.161079in}}%
\pgfpathlineto{\pgfqpoint{2.784687in}{2.164759in}}%
\pgfpathlineto{\pgfqpoint{2.787468in}{2.156655in}}%
\pgfpathlineto{\pgfqpoint{2.790044in}{2.155584in}}%
\pgfpathlineto{\pgfqpoint{2.792721in}{2.161608in}}%
\pgfpathlineto{\pgfqpoint{2.795398in}{2.157213in}}%
\pgfpathlineto{\pgfqpoint{2.798070in}{2.161245in}}%
\pgfpathlineto{\pgfqpoint{2.800756in}{2.158307in}}%
\pgfpathlineto{\pgfqpoint{2.803435in}{2.157811in}}%
\pgfpathlineto{\pgfqpoint{2.806175in}{2.154664in}}%
\pgfpathlineto{\pgfqpoint{2.808792in}{2.158673in}}%
\pgfpathlineto{\pgfqpoint{2.811597in}{2.160934in}}%
\pgfpathlineto{\pgfqpoint{2.814141in}{2.161642in}}%
\pgfpathlineto{\pgfqpoint{2.816867in}{2.165515in}}%
\pgfpathlineto{\pgfqpoint{2.819506in}{2.168420in}}%
\pgfpathlineto{\pgfqpoint{2.822303in}{2.167592in}}%
\pgfpathlineto{\pgfqpoint{2.824851in}{2.165056in}}%
\pgfpathlineto{\pgfqpoint{2.827567in}{2.161225in}}%
\pgfpathlineto{\pgfqpoint{2.830219in}{2.165025in}}%
\pgfpathlineto{\pgfqpoint{2.832894in}{2.163600in}}%
\pgfpathlineto{\pgfqpoint{2.835698in}{2.158903in}}%
\pgfpathlineto{\pgfqpoint{2.838254in}{2.158239in}}%
\pgfpathlineto{\pgfqpoint{2.841055in}{2.154664in}}%
\pgfpathlineto{\pgfqpoint{2.843611in}{2.154664in}}%
\pgfpathlineto{\pgfqpoint{2.846408in}{2.154664in}}%
\pgfpathlineto{\pgfqpoint{2.848960in}{2.156550in}}%
\pgfpathlineto{\pgfqpoint{2.851793in}{2.162966in}}%
\pgfpathlineto{\pgfqpoint{2.854325in}{2.163415in}}%
\pgfpathlineto{\pgfqpoint{2.857003in}{2.163998in}}%
\pgfpathlineto{\pgfqpoint{2.859668in}{2.165043in}}%
\pgfpathlineto{\pgfqpoint{2.862402in}{2.164642in}}%
\pgfpathlineto{\pgfqpoint{2.865031in}{2.164531in}}%
\pgfpathlineto{\pgfqpoint{2.867713in}{2.166304in}}%
\pgfpathlineto{\pgfqpoint{2.870475in}{2.167969in}}%
\pgfpathlineto{\pgfqpoint{2.873074in}{2.162419in}}%
\pgfpathlineto{\pgfqpoint{2.875882in}{2.162105in}}%
\pgfpathlineto{\pgfqpoint{2.878431in}{2.156269in}}%
\pgfpathlineto{\pgfqpoint{2.881254in}{2.158607in}}%
\pgfpathlineto{\pgfqpoint{2.883780in}{2.161131in}}%
\pgfpathlineto{\pgfqpoint{2.886578in}{2.160358in}}%
\pgfpathlineto{\pgfqpoint{2.889145in}{2.158236in}}%
\pgfpathlineto{\pgfqpoint{2.891809in}{2.161175in}}%
\pgfpathlineto{\pgfqpoint{2.894487in}{2.165803in}}%
\pgfpathlineto{\pgfqpoint{2.897179in}{2.162150in}}%
\pgfpathlineto{\pgfqpoint{2.899858in}{2.162985in}}%
\pgfpathlineto{\pgfqpoint{2.902535in}{2.164718in}}%
\pgfpathlineto{\pgfqpoint{2.905341in}{2.166126in}}%
\pgfpathlineto{\pgfqpoint{2.907882in}{2.163442in}}%
\pgfpathlineto{\pgfqpoint{2.910631in}{2.163024in}}%
\pgfpathlineto{\pgfqpoint{2.913243in}{2.161600in}}%
\pgfpathlineto{\pgfqpoint{2.916061in}{2.171333in}}%
\pgfpathlineto{\pgfqpoint{2.918606in}{2.176692in}}%
\pgfpathlineto{\pgfqpoint{2.921363in}{2.172831in}}%
\pgfpathlineto{\pgfqpoint{2.923963in}{2.172628in}}%
\pgfpathlineto{\pgfqpoint{2.926655in}{2.170216in}}%
\pgfpathlineto{\pgfqpoint{2.929321in}{2.162124in}}%
\pgfpathlineto{\pgfqpoint{2.932033in}{2.164100in}}%
\pgfpathlineto{\pgfqpoint{2.934759in}{2.164949in}}%
\pgfpathlineto{\pgfqpoint{2.937352in}{2.171955in}}%
\pgfpathlineto{\pgfqpoint{2.940120in}{2.176220in}}%
\pgfpathlineto{\pgfqpoint{2.942711in}{2.178350in}}%
\pgfpathlineto{\pgfqpoint{2.945461in}{2.179278in}}%
\pgfpathlineto{\pgfqpoint{2.948068in}{2.168007in}}%
\pgfpathlineto{\pgfqpoint{2.950884in}{2.165649in}}%
\pgfpathlineto{\pgfqpoint{2.953422in}{2.161037in}}%
\pgfpathlineto{\pgfqpoint{2.956103in}{2.158370in}}%
\pgfpathlineto{\pgfqpoint{2.958782in}{2.160599in}}%
\pgfpathlineto{\pgfqpoint{2.961460in}{2.163959in}}%
\pgfpathlineto{\pgfqpoint{2.964127in}{2.162318in}}%
\pgfpathlineto{\pgfqpoint{2.966812in}{2.165142in}}%
\pgfpathlineto{\pgfqpoint{2.969599in}{2.158094in}}%
\pgfpathlineto{\pgfqpoint{2.972177in}{2.161087in}}%
\pgfpathlineto{\pgfqpoint{2.974972in}{2.164926in}}%
\pgfpathlineto{\pgfqpoint{2.977517in}{2.159427in}}%
\pgfpathlineto{\pgfqpoint{2.980341in}{2.159129in}}%
\pgfpathlineto{\pgfqpoint{2.982885in}{2.161749in}}%
\pgfpathlineto{\pgfqpoint{2.985666in}{2.162414in}}%
\pgfpathlineto{\pgfqpoint{2.988238in}{2.161047in}}%
\pgfpathlineto{\pgfqpoint{2.990978in}{2.165962in}}%
\pgfpathlineto{\pgfqpoint{2.993595in}{2.169246in}}%
\pgfpathlineto{\pgfqpoint{2.996300in}{2.166055in}}%
\pgfpathlineto{\pgfqpoint{2.999103in}{2.174545in}}%
\pgfpathlineto{\pgfqpoint{3.001635in}{2.172680in}}%
\pgfpathlineto{\pgfqpoint{3.004419in}{2.166091in}}%
\pgfpathlineto{\pgfqpoint{3.006993in}{2.161244in}}%
\pgfpathlineto{\pgfqpoint{3.009784in}{2.161563in}}%
\pgfpathlineto{\pgfqpoint{3.012351in}{2.155458in}}%
\pgfpathlineto{\pgfqpoint{3.015097in}{2.157418in}}%
\pgfpathlineto{\pgfqpoint{3.017707in}{2.158487in}}%
\pgfpathlineto{\pgfqpoint{3.020382in}{2.160312in}}%
\pgfpathlineto{\pgfqpoint{3.023058in}{2.162389in}}%
\pgfpathlineto{\pgfqpoint{3.025803in}{2.167907in}}%
\pgfpathlineto{\pgfqpoint{3.028412in}{2.168656in}}%
\pgfpathlineto{\pgfqpoint{3.031091in}{2.165617in}}%
\pgfpathlineto{\pgfqpoint{3.033921in}{2.167956in}}%
\pgfpathlineto{\pgfqpoint{3.036456in}{2.167585in}}%
\pgfpathlineto{\pgfqpoint{3.039262in}{2.166599in}}%
\pgfpathlineto{\pgfqpoint{3.041813in}{2.161056in}}%
\pgfpathlineto{\pgfqpoint{3.044568in}{2.156961in}}%
\pgfpathlineto{\pgfqpoint{3.047157in}{2.161132in}}%
\pgfpathlineto{\pgfqpoint{3.049988in}{2.171688in}}%
\pgfpathlineto{\pgfqpoint{3.052526in}{2.172892in}}%
\pgfpathlineto{\pgfqpoint{3.055202in}{2.168756in}}%
\pgfpathlineto{\pgfqpoint{3.057884in}{2.164861in}}%
\pgfpathlineto{\pgfqpoint{3.060561in}{2.165215in}}%
\pgfpathlineto{\pgfqpoint{3.063230in}{2.162909in}}%
\pgfpathlineto{\pgfqpoint{3.065916in}{2.157506in}}%
\pgfpathlineto{\pgfqpoint{3.068709in}{2.160065in}}%
\pgfpathlineto{\pgfqpoint{3.071266in}{2.158767in}}%
\pgfpathlineto{\pgfqpoint{3.074056in}{2.158723in}}%
\pgfpathlineto{\pgfqpoint{3.076631in}{2.157670in}}%
\pgfpathlineto{\pgfqpoint{3.079381in}{2.161336in}}%
\pgfpathlineto{\pgfqpoint{3.081990in}{2.156358in}}%
\pgfpathlineto{\pgfqpoint{3.084671in}{2.158230in}}%
\pgfpathlineto{\pgfqpoint{3.087343in}{2.160309in}}%
\pgfpathlineto{\pgfqpoint{3.090023in}{2.160157in}}%
\pgfpathlineto{\pgfqpoint{3.092699in}{2.163992in}}%
\pgfpathlineto{\pgfqpoint{3.095388in}{2.160570in}}%
\pgfpathlineto{\pgfqpoint{3.098163in}{2.157560in}}%
\pgfpathlineto{\pgfqpoint{3.100737in}{2.154664in}}%
\pgfpathlineto{\pgfqpoint{3.103508in}{2.155343in}}%
\pgfpathlineto{\pgfqpoint{3.106094in}{2.154664in}}%
\pgfpathlineto{\pgfqpoint{3.108896in}{2.154664in}}%
\pgfpathlineto{\pgfqpoint{3.111451in}{2.156792in}}%
\pgfpathlineto{\pgfqpoint{3.114242in}{2.155467in}}%
\pgfpathlineto{\pgfqpoint{3.116807in}{2.154664in}}%
\pgfpathlineto{\pgfqpoint{3.119487in}{2.154664in}}%
\pgfpathlineto{\pgfqpoint{3.122163in}{2.159409in}}%
\pgfpathlineto{\pgfqpoint{3.124842in}{2.162849in}}%
\pgfpathlineto{\pgfqpoint{3.127512in}{2.164985in}}%
\pgfpathlineto{\pgfqpoint{3.130199in}{2.160401in}}%
\pgfpathlineto{\pgfqpoint{3.132946in}{2.161019in}}%
\pgfpathlineto{\pgfqpoint{3.135550in}{2.157673in}}%
\pgfpathlineto{\pgfqpoint{3.138375in}{2.163624in}}%
\pgfpathlineto{\pgfqpoint{3.140913in}{2.161834in}}%
\pgfpathlineto{\pgfqpoint{3.143740in}{2.159014in}}%
\pgfpathlineto{\pgfqpoint{3.146271in}{2.157480in}}%
\pgfpathlineto{\pgfqpoint{3.149057in}{2.158149in}}%
\pgfpathlineto{\pgfqpoint{3.151612in}{2.159535in}}%
\pgfpathlineto{\pgfqpoint{3.154327in}{2.160875in}}%
\pgfpathlineto{\pgfqpoint{3.156981in}{2.161494in}}%
\pgfpathlineto{\pgfqpoint{3.159675in}{2.156369in}}%
\pgfpathlineto{\pgfqpoint{3.162474in}{2.160137in}}%
\pgfpathlineto{\pgfqpoint{3.165019in}{2.157804in}}%
\pgfpathlineto{\pgfqpoint{3.167776in}{2.154664in}}%
\pgfpathlineto{\pgfqpoint{3.170375in}{2.154664in}}%
\pgfpathlineto{\pgfqpoint{3.173142in}{2.154664in}}%
\pgfpathlineto{\pgfqpoint{3.175724in}{2.154664in}}%
\pgfpathlineto{\pgfqpoint{3.178525in}{2.160201in}}%
\pgfpathlineto{\pgfqpoint{3.181089in}{2.157499in}}%
\pgfpathlineto{\pgfqpoint{3.183760in}{2.160939in}}%
\pgfpathlineto{\pgfqpoint{3.186440in}{2.163204in}}%
\pgfpathlineto{\pgfqpoint{3.189117in}{2.162797in}}%
\pgfpathlineto{\pgfqpoint{3.191796in}{2.159861in}}%
\pgfpathlineto{\pgfqpoint{3.194508in}{2.159791in}}%
\pgfpathlineto{\pgfqpoint{3.197226in}{2.157579in}}%
\pgfpathlineto{\pgfqpoint{3.199823in}{2.154664in}}%
\pgfpathlineto{\pgfqpoint{3.202562in}{2.154664in}}%
\pgfpathlineto{\pgfqpoint{3.205195in}{2.158245in}}%
\pgfpathlineto{\pgfqpoint{3.207984in}{2.162730in}}%
\pgfpathlineto{\pgfqpoint{3.210545in}{2.162801in}}%
\pgfpathlineto{\pgfqpoint{3.213342in}{2.161339in}}%
\pgfpathlineto{\pgfqpoint{3.215908in}{2.162085in}}%
\pgfpathlineto{\pgfqpoint{3.218586in}{2.161989in}}%
\pgfpathlineto{\pgfqpoint{3.221255in}{2.154664in}}%
\pgfpathlineto{\pgfqpoint{3.223942in}{2.154664in}}%
\pgfpathlineto{\pgfqpoint{3.226609in}{2.154664in}}%
\pgfpathlineto{\pgfqpoint{3.229310in}{2.156930in}}%
\pgfpathlineto{\pgfqpoint{3.232069in}{2.162770in}}%
\pgfpathlineto{\pgfqpoint{3.234658in}{2.160149in}}%
\pgfpathlineto{\pgfqpoint{3.237411in}{2.157780in}}%
\pgfpathlineto{\pgfqpoint{3.240010in}{2.156834in}}%
\pgfpathlineto{\pgfqpoint{3.242807in}{2.155787in}}%
\pgfpathlineto{\pgfqpoint{3.245363in}{2.159091in}}%
\pgfpathlineto{\pgfqpoint{3.248049in}{2.162045in}}%
\pgfpathlineto{\pgfqpoint{3.250716in}{2.163997in}}%
\pgfpathlineto{\pgfqpoint{3.253404in}{2.166280in}}%
\pgfpathlineto{\pgfqpoint{3.256083in}{2.167844in}}%
\pgfpathlineto{\pgfqpoint{3.258784in}{2.163751in}}%
\pgfpathlineto{\pgfqpoint{3.261594in}{2.165417in}}%
\pgfpathlineto{\pgfqpoint{3.264119in}{2.167677in}}%
\pgfpathlineto{\pgfqpoint{3.266849in}{2.165188in}}%
\pgfpathlineto{\pgfqpoint{3.269478in}{2.163350in}}%
\pgfpathlineto{\pgfqpoint{3.272254in}{2.165029in}}%
\pgfpathlineto{\pgfqpoint{3.274831in}{2.158286in}}%
\pgfpathlineto{\pgfqpoint{3.277603in}{2.155745in}}%
\pgfpathlineto{\pgfqpoint{3.280189in}{2.158386in}}%
\pgfpathlineto{\pgfqpoint{3.282870in}{2.160494in}}%
\pgfpathlineto{\pgfqpoint{3.285534in}{2.162398in}}%
\pgfpathlineto{\pgfqpoint{3.288225in}{2.164438in}}%
\pgfpathlineto{\pgfqpoint{3.290890in}{2.158974in}}%
\pgfpathlineto{\pgfqpoint{3.293574in}{2.166563in}}%
\pgfpathlineto{\pgfqpoint{3.296376in}{2.171770in}}%
\pgfpathlineto{\pgfqpoint{3.298937in}{2.172354in}}%
\pgfpathlineto{\pgfqpoint{3.301719in}{2.172110in}}%
\pgfpathlineto{\pgfqpoint{3.304295in}{2.168597in}}%
\pgfpathlineto{\pgfqpoint{3.307104in}{2.168006in}}%
\pgfpathlineto{\pgfqpoint{3.309652in}{2.172736in}}%
\pgfpathlineto{\pgfqpoint{3.312480in}{2.170049in}}%
\pgfpathlineto{\pgfqpoint{3.315008in}{2.176143in}}%
\pgfpathlineto{\pgfqpoint{3.317688in}{2.166272in}}%
\pgfpathlineto{\pgfqpoint{3.320366in}{2.169262in}}%
\pgfpathlineto{\pgfqpoint{3.323049in}{2.169158in}}%
\pgfpathlineto{\pgfqpoint{3.325860in}{2.169640in}}%
\pgfpathlineto{\pgfqpoint{3.328401in}{2.163867in}}%
\pgfpathlineto{\pgfqpoint{3.331183in}{2.164154in}}%
\pgfpathlineto{\pgfqpoint{3.333758in}{2.166461in}}%
\pgfpathlineto{\pgfqpoint{3.336541in}{2.164612in}}%
\pgfpathlineto{\pgfqpoint{3.339101in}{2.168236in}}%
\pgfpathlineto{\pgfqpoint{3.341893in}{2.169772in}}%
\pgfpathlineto{\pgfqpoint{3.344468in}{2.170018in}}%
\pgfpathlineto{\pgfqpoint{3.347139in}{2.170074in}}%
\pgfpathlineto{\pgfqpoint{3.349828in}{2.175921in}}%
\pgfpathlineto{\pgfqpoint{3.352505in}{2.168178in}}%
\pgfpathlineto{\pgfqpoint{3.355177in}{2.163873in}}%
\pgfpathlineto{\pgfqpoint{3.357862in}{2.162097in}}%
\pgfpathlineto{\pgfqpoint{3.360620in}{2.163141in}}%
\pgfpathlineto{\pgfqpoint{3.363221in}{2.170094in}}%
\pgfpathlineto{\pgfqpoint{3.365996in}{2.169597in}}%
\pgfpathlineto{\pgfqpoint{3.368577in}{2.167227in}}%
\pgfpathlineto{\pgfqpoint{3.371357in}{2.166912in}}%
\pgfpathlineto{\pgfqpoint{3.373921in}{2.162151in}}%
\pgfpathlineto{\pgfqpoint{3.376735in}{2.166542in}}%
\pgfpathlineto{\pgfqpoint{3.379290in}{2.165542in}}%
\pgfpathlineto{\pgfqpoint{3.381959in}{2.165481in}}%
\pgfpathlineto{\pgfqpoint{3.384647in}{2.166715in}}%
\pgfpathlineto{\pgfqpoint{3.387309in}{2.165871in}}%
\pgfpathlineto{\pgfqpoint{3.390102in}{2.168544in}}%
\pgfpathlineto{\pgfqpoint{3.392681in}{2.168293in}}%
\pgfpathlineto{\pgfqpoint{3.395461in}{2.165171in}}%
\pgfpathlineto{\pgfqpoint{3.398037in}{2.164773in}}%
\pgfpathlineto{\pgfqpoint{3.400783in}{2.160442in}}%
\pgfpathlineto{\pgfqpoint{3.403394in}{2.165691in}}%
\pgfpathlineto{\pgfqpoint{3.406202in}{2.163111in}}%
\pgfpathlineto{\pgfqpoint{3.408752in}{2.167876in}}%
\pgfpathlineto{\pgfqpoint{3.411431in}{2.161644in}}%
\pgfpathlineto{\pgfqpoint{3.414109in}{2.164255in}}%
\pgfpathlineto{\pgfqpoint{3.416780in}{2.162947in}}%
\pgfpathlineto{\pgfqpoint{3.419455in}{2.159409in}}%
\pgfpathlineto{\pgfqpoint{3.422142in}{2.160461in}}%
\pgfpathlineto{\pgfqpoint{3.424887in}{2.164358in}}%
\pgfpathlineto{\pgfqpoint{3.427501in}{2.160074in}}%
\pgfpathlineto{\pgfqpoint{3.430313in}{2.162156in}}%
\pgfpathlineto{\pgfqpoint{3.432851in}{2.160660in}}%
\pgfpathlineto{\pgfqpoint{3.435635in}{2.162642in}}%
\pgfpathlineto{\pgfqpoint{3.438210in}{2.160533in}}%
\pgfpathlineto{\pgfqpoint{3.440996in}{2.163973in}}%
\pgfpathlineto{\pgfqpoint{3.443574in}{2.164235in}}%
\pgfpathlineto{\pgfqpoint{3.446257in}{2.164069in}}%
\pgfpathlineto{\pgfqpoint{3.448926in}{2.165061in}}%
\pgfpathlineto{\pgfqpoint{3.451597in}{2.163857in}}%
\pgfpathlineto{\pgfqpoint{3.454285in}{2.163604in}}%
\pgfpathlineto{\pgfqpoint{3.456960in}{2.162303in}}%
\pgfpathlineto{\pgfqpoint{3.459695in}{2.163786in}}%
\pgfpathlineto{\pgfqpoint{3.462321in}{2.163213in}}%
\pgfpathlineto{\pgfqpoint{3.465072in}{2.159817in}}%
\pgfpathlineto{\pgfqpoint{3.467678in}{2.157742in}}%
\pgfpathlineto{\pgfqpoint{3.470466in}{2.154664in}}%
\pgfpathlineto{\pgfqpoint{3.473021in}{2.165735in}}%
\pgfpathlineto{\pgfqpoint{3.475821in}{2.164962in}}%
\pgfpathlineto{\pgfqpoint{3.478378in}{2.172937in}}%
\pgfpathlineto{\pgfqpoint{3.481072in}{2.169306in}}%
\pgfpathlineto{\pgfqpoint{3.483744in}{2.172066in}}%
\pgfpathlineto{\pgfqpoint{3.486442in}{2.168279in}}%
\pgfpathlineto{\pgfqpoint{3.489223in}{2.165539in}}%
\pgfpathlineto{\pgfqpoint{3.491783in}{2.167396in}}%
\pgfpathlineto{\pgfqpoint{3.494581in}{2.168344in}}%
\pgfpathlineto{\pgfqpoint{3.497139in}{2.164885in}}%
\pgfpathlineto{\pgfqpoint{3.499909in}{2.162452in}}%
\pgfpathlineto{\pgfqpoint{3.502488in}{2.161052in}}%
\pgfpathlineto{\pgfqpoint{3.505262in}{2.162099in}}%
\pgfpathlineto{\pgfqpoint{3.507840in}{2.170074in}}%
\pgfpathlineto{\pgfqpoint{3.510533in}{2.172971in}}%
\pgfpathlineto{\pgfqpoint{3.513209in}{2.166595in}}%
\pgfpathlineto{\pgfqpoint{3.515884in}{2.163711in}}%
\pgfpathlineto{\pgfqpoint{3.518565in}{2.163055in}}%
\pgfpathlineto{\pgfqpoint{3.521244in}{2.166112in}}%
\pgfpathlineto{\pgfqpoint{3.524041in}{2.162587in}}%
\pgfpathlineto{\pgfqpoint{3.526601in}{2.164012in}}%
\pgfpathlineto{\pgfqpoint{3.529327in}{2.161998in}}%
\pgfpathlineto{\pgfqpoint{3.531955in}{2.160828in}}%
\pgfpathlineto{\pgfqpoint{3.534783in}{2.159960in}}%
\pgfpathlineto{\pgfqpoint{3.537309in}{2.165579in}}%
\pgfpathlineto{\pgfqpoint{3.540093in}{2.163571in}}%
\pgfpathlineto{\pgfqpoint{3.542656in}{2.163367in}}%
\pgfpathlineto{\pgfqpoint{3.545349in}{2.160419in}}%
\pgfpathlineto{\pgfqpoint{3.548029in}{2.159947in}}%
\pgfpathlineto{\pgfqpoint{3.550713in}{2.164099in}}%
\pgfpathlineto{\pgfqpoint{3.553498in}{2.168255in}}%
\pgfpathlineto{\pgfqpoint{3.556061in}{2.162885in}}%
\pgfpathlineto{\pgfqpoint{3.558853in}{2.160685in}}%
\pgfpathlineto{\pgfqpoint{3.561420in}{2.164739in}}%
\pgfpathlineto{\pgfqpoint{3.564188in}{2.161899in}}%
\pgfpathlineto{\pgfqpoint{3.566774in}{2.161777in}}%
\pgfpathlineto{\pgfqpoint{3.569584in}{2.157398in}}%
\pgfpathlineto{\pgfqpoint{3.572126in}{2.158083in}}%
\pgfpathlineto{\pgfqpoint{3.574814in}{2.158012in}}%
\pgfpathlineto{\pgfqpoint{3.577487in}{2.159317in}}%
\pgfpathlineto{\pgfqpoint{3.580191in}{2.159718in}}%
\pgfpathlineto{\pgfqpoint{3.582851in}{2.161432in}}%
\pgfpathlineto{\pgfqpoint{3.585532in}{2.164800in}}%
\pgfpathlineto{\pgfqpoint{3.588258in}{2.164396in}}%
\pgfpathlineto{\pgfqpoint{3.590883in}{2.163482in}}%
\pgfpathlineto{\pgfqpoint{3.593620in}{2.166656in}}%
\pgfpathlineto{\pgfqpoint{3.596240in}{2.168936in}}%
\pgfpathlineto{\pgfqpoint{3.598998in}{2.167243in}}%
\pgfpathlineto{\pgfqpoint{3.601590in}{2.164899in}}%
\pgfpathlineto{\pgfqpoint{3.604387in}{2.163563in}}%
\pgfpathlineto{\pgfqpoint{3.606951in}{2.167634in}}%
\pgfpathlineto{\pgfqpoint{3.609632in}{2.162001in}}%
\pgfpathlineto{\pgfqpoint{3.612311in}{2.161189in}}%
\pgfpathlineto{\pgfqpoint{3.614982in}{2.162214in}}%
\pgfpathlineto{\pgfqpoint{3.617667in}{2.164698in}}%
\pgfpathlineto{\pgfqpoint{3.620345in}{2.163989in}}%
\pgfpathlineto{\pgfqpoint{3.623165in}{2.164642in}}%
\pgfpathlineto{\pgfqpoint{3.625689in}{2.165690in}}%
\pgfpathlineto{\pgfqpoint{3.628460in}{2.163858in}}%
\pgfpathlineto{\pgfqpoint{3.631058in}{2.165437in}}%
\pgfpathlineto{\pgfqpoint{3.633858in}{2.161244in}}%
\pgfpathlineto{\pgfqpoint{3.636413in}{2.162938in}}%
\pgfpathlineto{\pgfqpoint{3.639207in}{2.164634in}}%
\pgfpathlineto{\pgfqpoint{3.641773in}{2.161554in}}%
\pgfpathlineto{\pgfqpoint{3.644452in}{2.154664in}}%
\pgfpathlineto{\pgfqpoint{3.647130in}{2.154664in}}%
\pgfpathlineto{\pgfqpoint{3.649837in}{2.154664in}}%
\pgfpathlineto{\pgfqpoint{3.652628in}{2.154664in}}%
\pgfpathlineto{\pgfqpoint{3.655165in}{2.159738in}}%
\pgfpathlineto{\pgfqpoint{3.657917in}{2.162454in}}%
\pgfpathlineto{\pgfqpoint{3.660515in}{2.166504in}}%
\pgfpathlineto{\pgfqpoint{3.663276in}{2.169176in}}%
\pgfpathlineto{\pgfqpoint{3.665864in}{2.171995in}}%
\pgfpathlineto{\pgfqpoint{3.668665in}{2.165357in}}%
\pgfpathlineto{\pgfqpoint{3.671232in}{2.167260in}}%
\pgfpathlineto{\pgfqpoint{3.673911in}{2.161651in}}%
\pgfpathlineto{\pgfqpoint{3.676591in}{2.159294in}}%
\pgfpathlineto{\pgfqpoint{3.679273in}{2.160234in}}%
\pgfpathlineto{\pgfqpoint{3.681948in}{2.160241in}}%
\pgfpathlineto{\pgfqpoint{3.684620in}{2.159530in}}%
\pgfpathlineto{\pgfqpoint{3.687442in}{2.158737in}}%
\pgfpathlineto{\pgfqpoint{3.689983in}{2.157733in}}%
\pgfpathlineto{\pgfqpoint{3.692765in}{2.154729in}}%
\pgfpathlineto{\pgfqpoint{3.695331in}{2.160770in}}%
\pgfpathlineto{\pgfqpoint{3.698125in}{2.157326in}}%
\pgfpathlineto{\pgfqpoint{3.700684in}{2.160360in}}%
\pgfpathlineto{\pgfqpoint{3.703460in}{2.159414in}}%
\pgfpathlineto{\pgfqpoint{3.706053in}{2.161711in}}%
\pgfpathlineto{\pgfqpoint{3.708729in}{2.162863in}}%
\pgfpathlineto{\pgfqpoint{3.711410in}{2.163685in}}%
\pgfpathlineto{\pgfqpoint{3.714086in}{2.163474in}}%
\pgfpathlineto{\pgfqpoint{3.716875in}{2.161124in}}%
\pgfpathlineto{\pgfqpoint{3.719446in}{2.166048in}}%
\pgfpathlineto{\pgfqpoint{3.722228in}{2.165449in}}%
\pgfpathlineto{\pgfqpoint{3.724804in}{2.170798in}}%
\pgfpathlineto{\pgfqpoint{3.727581in}{2.170830in}}%
\pgfpathlineto{\pgfqpoint{3.730158in}{2.171055in}}%
\pgfpathlineto{\pgfqpoint{3.732950in}{2.170890in}}%
\pgfpathlineto{\pgfqpoint{3.735509in}{2.173564in}}%
\pgfpathlineto{\pgfqpoint{3.738194in}{2.172959in}}%
\pgfpathlineto{\pgfqpoint{3.740874in}{2.177364in}}%
\pgfpathlineto{\pgfqpoint{3.743548in}{2.176025in}}%
\pgfpathlineto{\pgfqpoint{3.746229in}{2.180439in}}%
\pgfpathlineto{\pgfqpoint{3.748903in}{2.176575in}}%
\pgfpathlineto{\pgfqpoint{3.751728in}{2.177325in}}%
\pgfpathlineto{\pgfqpoint{3.754265in}{2.172957in}}%
\pgfpathlineto{\pgfqpoint{3.757065in}{2.173171in}}%
\pgfpathlineto{\pgfqpoint{3.759622in}{2.170529in}}%
\pgfpathlineto{\pgfqpoint{3.762389in}{2.170083in}}%
\pgfpathlineto{\pgfqpoint{3.764966in}{2.164138in}}%
\pgfpathlineto{\pgfqpoint{3.767782in}{2.165136in}}%
\pgfpathlineto{\pgfqpoint{3.770323in}{2.164629in}}%
\pgfpathlineto{\pgfqpoint{3.773014in}{2.169347in}}%
\pgfpathlineto{\pgfqpoint{3.775691in}{2.186739in}}%
\pgfpathlineto{\pgfqpoint{3.778370in}{2.183099in}}%
\pgfpathlineto{\pgfqpoint{3.781046in}{2.177155in}}%
\pgfpathlineto{\pgfqpoint{3.783725in}{2.170641in}}%
\pgfpathlineto{\pgfqpoint{3.786504in}{2.167225in}}%
\pgfpathlineto{\pgfqpoint{3.789084in}{2.164263in}}%
\pgfpathlineto{\pgfqpoint{3.791897in}{2.160181in}}%
\pgfpathlineto{\pgfqpoint{3.794435in}{2.162510in}}%
\pgfpathlineto{\pgfqpoint{3.797265in}{2.156749in}}%
\pgfpathlineto{\pgfqpoint{3.799797in}{2.156195in}}%
\pgfpathlineto{\pgfqpoint{3.802569in}{2.162219in}}%
\pgfpathlineto{\pgfqpoint{3.805145in}{2.166249in}}%
\pgfpathlineto{\pgfqpoint{3.807832in}{2.165753in}}%
\pgfpathlineto{\pgfqpoint{3.810510in}{2.166576in}}%
\pgfpathlineto{\pgfqpoint{3.813172in}{2.166451in}}%
\pgfpathlineto{\pgfqpoint{3.815983in}{2.171203in}}%
\pgfpathlineto{\pgfqpoint{3.818546in}{2.169101in}}%
\pgfpathlineto{\pgfqpoint{3.821315in}{2.168205in}}%
\pgfpathlineto{\pgfqpoint{3.823903in}{2.170012in}}%
\pgfpathlineto{\pgfqpoint{3.826679in}{2.169585in}}%
\pgfpathlineto{\pgfqpoint{3.829252in}{2.165862in}}%
\pgfpathlineto{\pgfqpoint{3.832053in}{2.168712in}}%
\pgfpathlineto{\pgfqpoint{3.834616in}{2.167870in}}%
\pgfpathlineto{\pgfqpoint{3.837286in}{2.167566in}}%
\pgfpathlineto{\pgfqpoint{3.839960in}{2.167766in}}%
\pgfpathlineto{\pgfqpoint{3.842641in}{2.168554in}}%
\pgfpathlineto{\pgfqpoint{3.845329in}{2.166643in}}%
\pgfpathlineto{\pgfqpoint{3.848005in}{2.169549in}}%
\pgfpathlineto{\pgfqpoint{3.850814in}{2.175469in}}%
\pgfpathlineto{\pgfqpoint{3.853358in}{2.170625in}}%
\pgfpathlineto{\pgfqpoint{3.856100in}{2.172542in}}%
\pgfpathlineto{\pgfqpoint{3.858720in}{2.174015in}}%
\pgfpathlineto{\pgfqpoint{3.861561in}{2.172561in}}%
\pgfpathlineto{\pgfqpoint{3.864073in}{2.168228in}}%
\pgfpathlineto{\pgfqpoint{3.866815in}{2.171637in}}%
\pgfpathlineto{\pgfqpoint{3.869435in}{2.172826in}}%
\pgfpathlineto{\pgfqpoint{3.872114in}{2.173743in}}%
\pgfpathlineto{\pgfqpoint{3.874790in}{2.171845in}}%
\pgfpathlineto{\pgfqpoint{3.877466in}{2.167125in}}%
\pgfpathlineto{\pgfqpoint{3.880237in}{2.172568in}}%
\pgfpathlineto{\pgfqpoint{3.882850in}{2.167556in}}%
\pgfpathlineto{\pgfqpoint{3.885621in}{2.166015in}}%
\pgfpathlineto{\pgfqpoint{3.888188in}{2.161938in}}%
\pgfpathlineto{\pgfqpoint{3.890926in}{2.159849in}}%
\pgfpathlineto{\pgfqpoint{3.893541in}{2.164849in}}%
\pgfpathlineto{\pgfqpoint{3.896345in}{2.166938in}}%
\pgfpathlineto{\pgfqpoint{3.898891in}{2.171061in}}%
\pgfpathlineto{\pgfqpoint{3.901573in}{2.169385in}}%
\pgfpathlineto{\pgfqpoint{3.904252in}{2.167275in}}%
\pgfpathlineto{\pgfqpoint{3.906918in}{2.165615in}}%
\pgfpathlineto{\pgfqpoint{3.909602in}{2.168317in}}%
\pgfpathlineto{\pgfqpoint{3.912296in}{2.169230in}}%
\pgfpathlineto{\pgfqpoint{3.915107in}{2.169741in}}%
\pgfpathlineto{\pgfqpoint{3.917646in}{2.176667in}}%
\pgfpathlineto{\pgfqpoint{3.920412in}{2.182866in}}%
\pgfpathlineto{\pgfqpoint{3.923005in}{2.174068in}}%
\pgfpathlineto{\pgfqpoint{3.925778in}{2.174729in}}%
\pgfpathlineto{\pgfqpoint{3.928347in}{2.174144in}}%
\pgfpathlineto{\pgfqpoint{3.931202in}{2.172743in}}%
\pgfpathlineto{\pgfqpoint{3.933714in}{2.169436in}}%
\pgfpathlineto{\pgfqpoint{3.936395in}{2.169330in}}%
\pgfpathlineto{\pgfqpoint{3.939075in}{2.164862in}}%
\pgfpathlineto{\pgfqpoint{3.941778in}{2.159252in}}%
\pgfpathlineto{\pgfqpoint{3.944431in}{2.160685in}}%
\pgfpathlineto{\pgfqpoint{3.947101in}{2.161387in}}%
\pgfpathlineto{\pgfqpoint{3.949894in}{2.160649in}}%
\pgfpathlineto{\pgfqpoint{3.952464in}{2.174106in}}%
\pgfpathlineto{\pgfqpoint{3.955211in}{2.174695in}}%
\pgfpathlineto{\pgfqpoint{3.957823in}{2.175706in}}%
\pgfpathlineto{\pgfqpoint{3.960635in}{2.171674in}}%
\pgfpathlineto{\pgfqpoint{3.963176in}{2.172700in}}%
\pgfpathlineto{\pgfqpoint{3.966013in}{2.173476in}}%
\pgfpathlineto{\pgfqpoint{3.968523in}{2.173550in}}%
\pgfpathlineto{\pgfqpoint{3.971250in}{2.170624in}}%
\pgfpathlineto{\pgfqpoint{3.973885in}{2.168372in}}%
\pgfpathlineto{\pgfqpoint{3.976563in}{2.170339in}}%
\pgfpathlineto{\pgfqpoint{3.979389in}{2.171541in}}%
\pgfpathlineto{\pgfqpoint{3.981929in}{2.171543in}}%
\pgfpathlineto{\pgfqpoint{3.984714in}{2.173329in}}%
\pgfpathlineto{\pgfqpoint{3.987270in}{2.175786in}}%
\pgfpathlineto{\pgfqpoint{3.990055in}{2.174828in}}%
\pgfpathlineto{\pgfqpoint{3.992642in}{2.172337in}}%
\pgfpathlineto{\pgfqpoint{3.995417in}{2.172371in}}%
\pgfpathlineto{\pgfqpoint{3.997990in}{2.174233in}}%
\pgfpathlineto{\pgfqpoint{4.000674in}{2.172811in}}%
\pgfpathlineto{\pgfqpoint{4.003348in}{2.172482in}}%
\pgfpathlineto{\pgfqpoint{4.006034in}{2.171578in}}%
\pgfpathlineto{\pgfqpoint{4.008699in}{2.172540in}}%
\pgfpathlineto{\pgfqpoint{4.011394in}{2.169509in}}%
\pgfpathlineto{\pgfqpoint{4.014186in}{2.174021in}}%
\pgfpathlineto{\pgfqpoint{4.016744in}{2.172927in}}%
\pgfpathlineto{\pgfqpoint{4.019518in}{2.173465in}}%
\pgfpathlineto{\pgfqpoint{4.022097in}{2.170545in}}%
\pgfpathlineto{\pgfqpoint{4.024868in}{2.173626in}}%
\pgfpathlineto{\pgfqpoint{4.027447in}{2.173246in}}%
\pgfpathlineto{\pgfqpoint{4.030229in}{2.171463in}}%
\pgfpathlineto{\pgfqpoint{4.032817in}{2.172317in}}%
\pgfpathlineto{\pgfqpoint{4.035492in}{2.172651in}}%
\pgfpathlineto{\pgfqpoint{4.038174in}{2.170599in}}%
\pgfpathlineto{\pgfqpoint{4.040852in}{2.171198in}}%
\pgfpathlineto{\pgfqpoint{4.043667in}{2.172219in}}%
\pgfpathlineto{\pgfqpoint{4.046210in}{2.171462in}}%
\pgfpathlineto{\pgfqpoint{4.049006in}{2.168053in}}%
\pgfpathlineto{\pgfqpoint{4.051557in}{2.172182in}}%
\pgfpathlineto{\pgfqpoint{4.054326in}{2.174043in}}%
\pgfpathlineto{\pgfqpoint{4.056911in}{2.170498in}}%
\pgfpathlineto{\pgfqpoint{4.059702in}{2.162263in}}%
\pgfpathlineto{\pgfqpoint{4.062266in}{2.167908in}}%
\pgfpathlineto{\pgfqpoint{4.064957in}{2.165270in}}%
\pgfpathlineto{\pgfqpoint{4.067636in}{2.169186in}}%
\pgfpathlineto{\pgfqpoint{4.070313in}{2.165381in}}%
\pgfpathlineto{\pgfqpoint{4.072985in}{2.169573in}}%
\pgfpathlineto{\pgfqpoint{4.075705in}{2.167346in}}%
\pgfpathlineto{\pgfqpoint{4.078471in}{2.167139in}}%
\pgfpathlineto{\pgfqpoint{4.081018in}{2.165457in}}%
\pgfpathlineto{\pgfqpoint{4.083870in}{2.165663in}}%
\pgfpathlineto{\pgfqpoint{4.086385in}{2.166906in}}%
\pgfpathlineto{\pgfqpoint{4.089159in}{2.166229in}}%
\pgfpathlineto{\pgfqpoint{4.091729in}{2.165257in}}%
\pgfpathlineto{\pgfqpoint{4.094527in}{2.169272in}}%
\pgfpathlineto{\pgfqpoint{4.097092in}{2.172032in}}%
\pgfpathlineto{\pgfqpoint{4.099777in}{2.172370in}}%
\pgfpathlineto{\pgfqpoint{4.102456in}{2.173675in}}%
\pgfpathlineto{\pgfqpoint{4.105185in}{2.176933in}}%
\pgfpathlineto{\pgfqpoint{4.107814in}{2.174596in}}%
\pgfpathlineto{\pgfqpoint{4.110488in}{2.175281in}}%
\pgfpathlineto{\pgfqpoint{4.113252in}{2.178524in}}%
\pgfpathlineto{\pgfqpoint{4.115844in}{2.173860in}}%
\pgfpathlineto{\pgfqpoint{4.118554in}{2.171238in}}%
\pgfpathlineto{\pgfqpoint{4.121205in}{2.173090in}}%
\pgfpathlineto{\pgfqpoint{4.124019in}{2.171128in}}%
\pgfpathlineto{\pgfqpoint{4.126553in}{2.170655in}}%
\pgfpathlineto{\pgfqpoint{4.129349in}{2.170668in}}%
\pgfpathlineto{\pgfqpoint{4.131920in}{2.173316in}}%
\pgfpathlineto{\pgfqpoint{4.134615in}{2.171426in}}%
\pgfpathlineto{\pgfqpoint{4.137272in}{2.169689in}}%
\pgfpathlineto{\pgfqpoint{4.139963in}{2.171031in}}%
\pgfpathlineto{\pgfqpoint{4.142713in}{2.176161in}}%
\pgfpathlineto{\pgfqpoint{4.145310in}{2.172555in}}%
\pgfpathlineto{\pgfqpoint{4.148082in}{2.174753in}}%
\pgfpathlineto{\pgfqpoint{4.150665in}{2.172451in}}%
\pgfpathlineto{\pgfqpoint{4.153423in}{2.171055in}}%
\pgfpathlineto{\pgfqpoint{4.156016in}{2.170601in}}%
\pgfpathlineto{\pgfqpoint{4.158806in}{2.170029in}}%
\pgfpathlineto{\pgfqpoint{4.161380in}{2.168875in}}%
\pgfpathlineto{\pgfqpoint{4.164059in}{2.168637in}}%
\pgfpathlineto{\pgfqpoint{4.166737in}{2.171618in}}%
\pgfpathlineto{\pgfqpoint{4.169415in}{2.170417in}}%
\pgfpathlineto{\pgfqpoint{4.172093in}{2.165678in}}%
\pgfpathlineto{\pgfqpoint{4.174770in}{2.167112in}}%
\pgfpathlineto{\pgfqpoint{4.177593in}{2.173350in}}%
\pgfpathlineto{\pgfqpoint{4.180129in}{2.172967in}}%
\pgfpathlineto{\pgfqpoint{4.182899in}{2.170469in}}%
\pgfpathlineto{\pgfqpoint{4.185481in}{2.167264in}}%
\pgfpathlineto{\pgfqpoint{4.188318in}{2.164683in}}%
\pgfpathlineto{\pgfqpoint{4.190842in}{2.169099in}}%
\pgfpathlineto{\pgfqpoint{4.193638in}{2.174700in}}%
\pgfpathlineto{\pgfqpoint{4.196186in}{2.170826in}}%
\pgfpathlineto{\pgfqpoint{4.198878in}{2.172968in}}%
\pgfpathlineto{\pgfqpoint{4.201542in}{2.168825in}}%
\pgfpathlineto{\pgfqpoint{4.204240in}{2.168494in}}%
\pgfpathlineto{\pgfqpoint{4.207076in}{2.156026in}}%
\pgfpathlineto{\pgfqpoint{4.209597in}{2.156191in}}%
\pgfpathlineto{\pgfqpoint{4.212383in}{2.158991in}}%
\pgfpathlineto{\pgfqpoint{4.214948in}{2.162956in}}%
\pgfpathlineto{\pgfqpoint{4.217694in}{2.168579in}}%
\pgfpathlineto{\pgfqpoint{4.220304in}{2.171043in}}%
\pgfpathlineto{\pgfqpoint{4.223082in}{2.171259in}}%
\pgfpathlineto{\pgfqpoint{4.225654in}{2.169321in}}%
\pgfpathlineto{\pgfqpoint{4.228331in}{2.167622in}}%
\pgfpathlineto{\pgfqpoint{4.231013in}{2.166915in}}%
\pgfpathlineto{\pgfqpoint{4.233691in}{2.164424in}}%
\pgfpathlineto{\pgfqpoint{4.236375in}{2.166985in}}%
\pgfpathlineto{\pgfqpoint{4.239084in}{2.167517in}}%
\pgfpathlineto{\pgfqpoint{4.241900in}{2.173381in}}%
\pgfpathlineto{\pgfqpoint{4.244394in}{2.170285in}}%
\pgfpathlineto{\pgfqpoint{4.247225in}{2.172271in}}%
\pgfpathlineto{\pgfqpoint{4.249767in}{2.171246in}}%
\pgfpathlineto{\pgfqpoint{4.252581in}{2.173344in}}%
\pgfpathlineto{\pgfqpoint{4.255120in}{2.169643in}}%
\pgfpathlineto{\pgfqpoint{4.257958in}{2.166857in}}%
\pgfpathlineto{\pgfqpoint{4.260477in}{2.172138in}}%
\pgfpathlineto{\pgfqpoint{4.263157in}{2.171493in}}%
\pgfpathlineto{\pgfqpoint{4.265824in}{2.165458in}}%
\pgfpathlineto{\pgfqpoint{4.268590in}{2.166642in}}%
\pgfpathlineto{\pgfqpoint{4.271187in}{2.167010in}}%
\pgfpathlineto{\pgfqpoint{4.273874in}{2.160679in}}%
\pgfpathlineto{\pgfqpoint{4.276635in}{2.158514in}}%
\pgfpathlineto{\pgfqpoint{4.279212in}{2.163877in}}%
\pgfpathlineto{\pgfqpoint{4.282000in}{2.168329in}}%
\pgfpathlineto{\pgfqpoint{4.284586in}{2.172281in}}%
\pgfpathlineto{\pgfqpoint{4.287399in}{2.172385in}}%
\pgfpathlineto{\pgfqpoint{4.289936in}{2.171578in}}%
\pgfpathlineto{\pgfqpoint{4.292786in}{2.168092in}}%
\pgfpathlineto{\pgfqpoint{4.295299in}{2.165126in}}%
\pgfpathlineto{\pgfqpoint{4.297977in}{2.160236in}}%
\pgfpathlineto{\pgfqpoint{4.300656in}{2.156192in}}%
\pgfpathlineto{\pgfqpoint{4.303357in}{2.164112in}}%
\pgfpathlineto{\pgfqpoint{4.306118in}{2.163081in}}%
\pgfpathlineto{\pgfqpoint{4.308691in}{2.166346in}}%
\pgfpathlineto{\pgfqpoint{4.311494in}{2.162753in}}%
\pgfpathlineto{\pgfqpoint{4.314032in}{2.161997in}}%
\pgfpathlineto{\pgfqpoint{4.316856in}{2.167218in}}%
\pgfpathlineto{\pgfqpoint{4.319405in}{2.167786in}}%
\pgfpathlineto{\pgfqpoint{4.322181in}{2.166962in}}%
\pgfpathlineto{\pgfqpoint{4.324760in}{2.160541in}}%
\pgfpathlineto{\pgfqpoint{4.327440in}{2.162415in}}%
\pgfpathlineto{\pgfqpoint{4.330118in}{2.166677in}}%
\pgfpathlineto{\pgfqpoint{4.332796in}{2.166355in}}%
\pgfpathlineto{\pgfqpoint{4.335463in}{2.167686in}}%
\pgfpathlineto{\pgfqpoint{4.338154in}{2.167038in}}%
\pgfpathlineto{\pgfqpoint{4.340976in}{2.166260in}}%
\pgfpathlineto{\pgfqpoint{4.343510in}{2.164503in}}%
\pgfpathlineto{\pgfqpoint{4.346263in}{2.163820in}}%
\pgfpathlineto{\pgfqpoint{4.348868in}{2.159572in}}%
\pgfpathlineto{\pgfqpoint{4.351645in}{2.159807in}}%
\pgfpathlineto{\pgfqpoint{4.354224in}{2.163343in}}%
\pgfpathlineto{\pgfqpoint{4.357014in}{2.160968in}}%
\pgfpathlineto{\pgfqpoint{4.359582in}{2.160911in}}%
\pgfpathlineto{\pgfqpoint{4.362270in}{2.158557in}}%
\pgfpathlineto{\pgfqpoint{4.364936in}{2.168815in}}%
\pgfpathlineto{\pgfqpoint{4.367646in}{2.163344in}}%
\pgfpathlineto{\pgfqpoint{4.370437in}{2.157546in}}%
\pgfpathlineto{\pgfqpoint{4.372976in}{2.160957in}}%
\pgfpathlineto{\pgfqpoint{4.375761in}{2.163166in}}%
\pgfpathlineto{\pgfqpoint{4.378329in}{2.166417in}}%
\pgfpathlineto{\pgfqpoint{4.381097in}{2.166665in}}%
\pgfpathlineto{\pgfqpoint{4.383674in}{2.162024in}}%
\pgfpathlineto{\pgfqpoint{4.386431in}{2.160005in}}%
\pgfpathlineto{\pgfqpoint{4.389044in}{2.165073in}}%
\pgfpathlineto{\pgfqpoint{4.391721in}{2.163466in}}%
\pgfpathlineto{\pgfqpoint{4.394400in}{2.164510in}}%
\pgfpathlineto{\pgfqpoint{4.397076in}{2.162033in}}%
\pgfpathlineto{\pgfqpoint{4.399745in}{2.165500in}}%
\pgfpathlineto{\pgfqpoint{4.402468in}{2.170859in}}%
\pgfpathlineto{\pgfqpoint{4.405234in}{2.168254in}}%
\pgfpathlineto{\pgfqpoint{4.407788in}{2.163714in}}%
\pgfpathlineto{\pgfqpoint{4.410587in}{2.167602in}}%
\pgfpathlineto{\pgfqpoint{4.413149in}{2.158771in}}%
\pgfpathlineto{\pgfqpoint{4.415932in}{2.155064in}}%
\pgfpathlineto{\pgfqpoint{4.418506in}{2.155142in}}%
\pgfpathlineto{\pgfqpoint{4.421292in}{2.157508in}}%
\pgfpathlineto{\pgfqpoint{4.423863in}{2.156672in}}%
\pgfpathlineto{\pgfqpoint{4.426534in}{2.154664in}}%
\pgfpathlineto{\pgfqpoint{4.429220in}{2.154664in}}%
\pgfpathlineto{\pgfqpoint{4.431901in}{2.154664in}}%
\pgfpathlineto{\pgfqpoint{4.434569in}{2.154664in}}%
\pgfpathlineto{\pgfqpoint{4.437253in}{2.156742in}}%
\pgfpathlineto{\pgfqpoint{4.440041in}{2.159721in}}%
\pgfpathlineto{\pgfqpoint{4.442611in}{2.158086in}}%
\pgfpathlineto{\pgfqpoint{4.445423in}{2.159083in}}%
\pgfpathlineto{\pgfqpoint{4.447965in}{2.163988in}}%
\pgfpathlineto{\pgfqpoint{4.450767in}{2.163417in}}%
\pgfpathlineto{\pgfqpoint{4.453312in}{2.164288in}}%
\pgfpathlineto{\pgfqpoint{4.456138in}{2.163629in}}%
\pgfpathlineto{\pgfqpoint{4.458681in}{2.157155in}}%
\pgfpathlineto{\pgfqpoint{4.461367in}{2.157360in}}%
\pgfpathlineto{\pgfqpoint{4.464029in}{2.162343in}}%
\pgfpathlineto{\pgfqpoint{4.466717in}{2.164567in}}%
\pgfpathlineto{\pgfqpoint{4.469492in}{2.165659in}}%
\pgfpathlineto{\pgfqpoint{4.472059in}{2.166397in}}%
\pgfpathlineto{\pgfqpoint{4.474861in}{2.168567in}}%
\pgfpathlineto{\pgfqpoint{4.477430in}{2.163580in}}%
\pgfpathlineto{\pgfqpoint{4.480201in}{2.162651in}}%
\pgfpathlineto{\pgfqpoint{4.482778in}{2.160702in}}%
\pgfpathlineto{\pgfqpoint{4.485581in}{2.163755in}}%
\pgfpathlineto{\pgfqpoint{4.488130in}{2.164204in}}%
\pgfpathlineto{\pgfqpoint{4.490822in}{2.164298in}}%
\pgfpathlineto{\pgfqpoint{4.493492in}{2.165233in}}%
\pgfpathlineto{\pgfqpoint{4.496167in}{2.163145in}}%
\pgfpathlineto{\pgfqpoint{4.498850in}{2.169111in}}%
\pgfpathlineto{\pgfqpoint{4.501529in}{2.161911in}}%
\pgfpathlineto{\pgfqpoint{4.504305in}{2.168429in}}%
\pgfpathlineto{\pgfqpoint{4.506893in}{2.168302in}}%
\pgfpathlineto{\pgfqpoint{4.509643in}{2.162469in}}%
\pgfpathlineto{\pgfqpoint{4.512246in}{2.161378in}}%
\pgfpathlineto{\pgfqpoint{4.515080in}{2.162296in}}%
\pgfpathlineto{\pgfqpoint{4.517598in}{2.165363in}}%
\pgfpathlineto{\pgfqpoint{4.520345in}{2.166636in}}%
\pgfpathlineto{\pgfqpoint{4.522962in}{2.169978in}}%
\pgfpathlineto{\pgfqpoint{4.525640in}{2.158891in}}%
\pgfpathlineto{\pgfqpoint{4.528307in}{2.165085in}}%
\pgfpathlineto{\pgfqpoint{4.530990in}{2.164602in}}%
\pgfpathlineto{\pgfqpoint{4.533764in}{2.166623in}}%
\pgfpathlineto{\pgfqpoint{4.536400in}{2.168737in}}%
\pgfpathlineto{\pgfqpoint{4.539144in}{2.169567in}}%
\pgfpathlineto{\pgfqpoint{4.541711in}{2.167390in}}%
\pgfpathlineto{\pgfqpoint{4.544464in}{2.160149in}}%
\pgfpathlineto{\pgfqpoint{4.547064in}{2.165282in}}%
\pgfpathlineto{\pgfqpoint{4.549822in}{2.162043in}}%
\pgfpathlineto{\pgfqpoint{4.552425in}{2.160854in}}%
\pgfpathlineto{\pgfqpoint{4.555106in}{2.159777in}}%
\pgfpathlineto{\pgfqpoint{4.557777in}{2.160837in}}%
\pgfpathlineto{\pgfqpoint{4.560448in}{2.161349in}}%
\pgfpathlineto{\pgfqpoint{4.563125in}{2.159073in}}%
\pgfpathlineto{\pgfqpoint{4.565820in}{2.160337in}}%
\pgfpathlineto{\pgfqpoint{4.568612in}{2.158206in}}%
\pgfpathlineto{\pgfqpoint{4.571171in}{2.159900in}}%
\pgfpathlineto{\pgfqpoint{4.573947in}{2.157935in}}%
\pgfpathlineto{\pgfqpoint{4.576531in}{2.160197in}}%
\pgfpathlineto{\pgfqpoint{4.579305in}{2.163747in}}%
\pgfpathlineto{\pgfqpoint{4.581888in}{2.161412in}}%
\pgfpathlineto{\pgfqpoint{4.584672in}{2.161202in}}%
\pgfpathlineto{\pgfqpoint{4.587244in}{2.159897in}}%
\pgfpathlineto{\pgfqpoint{4.589920in}{2.156940in}}%
\pgfpathlineto{\pgfqpoint{4.592589in}{2.162090in}}%
\pgfpathlineto{\pgfqpoint{4.595281in}{2.165428in}}%
\pgfpathlineto{\pgfqpoint{4.597951in}{2.165358in}}%
\pgfpathlineto{\pgfqpoint{4.600633in}{2.167880in}}%
\pgfpathlineto{\pgfqpoint{4.603430in}{2.169434in}}%
\pgfpathlineto{\pgfqpoint{4.605990in}{2.171387in}}%
\pgfpathlineto{\pgfqpoint{4.608808in}{2.170757in}}%
\pgfpathlineto{\pgfqpoint{4.611350in}{2.166444in}}%
\pgfpathlineto{\pgfqpoint{4.614134in}{2.166369in}}%
\pgfpathlineto{\pgfqpoint{4.616702in}{2.163210in}}%
\pgfpathlineto{\pgfqpoint{4.619529in}{2.164409in}}%
\pgfpathlineto{\pgfqpoint{4.622056in}{2.162599in}}%
\pgfpathlineto{\pgfqpoint{4.624741in}{2.171942in}}%
\pgfpathlineto{\pgfqpoint{4.627411in}{2.174276in}}%
\pgfpathlineto{\pgfqpoint{4.630096in}{2.175401in}}%
\pgfpathlineto{\pgfqpoint{4.632902in}{2.173704in}}%
\pgfpathlineto{\pgfqpoint{4.635445in}{2.169023in}}%
\pgfpathlineto{\pgfqpoint{4.638204in}{2.166656in}}%
\pgfpathlineto{\pgfqpoint{4.640809in}{2.168564in}}%
\pgfpathlineto{\pgfqpoint{4.643628in}{2.169127in}}%
\pgfpathlineto{\pgfqpoint{4.646169in}{2.170251in}}%
\pgfpathlineto{\pgfqpoint{4.648922in}{2.169425in}}%
\pgfpathlineto{\pgfqpoint{4.651524in}{2.168691in}}%
\pgfpathlineto{\pgfqpoint{4.654203in}{2.171096in}}%
\pgfpathlineto{\pgfqpoint{4.656873in}{2.167710in}}%
\pgfpathlineto{\pgfqpoint{4.659590in}{2.171170in}}%
\pgfpathlineto{\pgfqpoint{4.662237in}{2.170156in}}%
\pgfpathlineto{\pgfqpoint{4.664923in}{2.167315in}}%
\pgfpathlineto{\pgfqpoint{4.667764in}{2.162426in}}%
\pgfpathlineto{\pgfqpoint{4.670261in}{2.168960in}}%
\pgfpathlineto{\pgfqpoint{4.673068in}{2.167465in}}%
\pgfpathlineto{\pgfqpoint{4.675619in}{2.159892in}}%
\pgfpathlineto{\pgfqpoint{4.678448in}{2.164894in}}%
\pgfpathlineto{\pgfqpoint{4.680988in}{2.159684in}}%
\pgfpathlineto{\pgfqpoint{4.683799in}{2.166675in}}%
\pgfpathlineto{\pgfqpoint{4.686337in}{2.160908in}}%
\pgfpathlineto{\pgfqpoint{4.689051in}{2.156862in}}%
\pgfpathlineto{\pgfqpoint{4.691694in}{2.162318in}}%
\pgfpathlineto{\pgfqpoint{4.694381in}{2.163414in}}%
\pgfpathlineto{\pgfqpoint{4.697170in}{2.162020in}}%
\pgfpathlineto{\pgfqpoint{4.699734in}{2.164547in}}%
\pgfpathlineto{\pgfqpoint{4.702517in}{2.166572in}}%
\pgfpathlineto{\pgfqpoint{4.705094in}{2.162633in}}%
\pgfpathlineto{\pgfqpoint{4.707824in}{2.165012in}}%
\pgfpathlineto{\pgfqpoint{4.710437in}{2.165437in}}%
\pgfpathlineto{\pgfqpoint{4.713275in}{2.167935in}}%
\pgfpathlineto{\pgfqpoint{4.715806in}{2.171024in}}%
\pgfpathlineto{\pgfqpoint{4.718486in}{2.165118in}}%
\pgfpathlineto{\pgfqpoint{4.721160in}{2.164976in}}%
\pgfpathlineto{\pgfqpoint{4.723873in}{2.171032in}}%
\pgfpathlineto{\pgfqpoint{4.726508in}{2.165220in}}%
\pgfpathlineto{\pgfqpoint{4.729233in}{2.165942in}}%
\pgfpathlineto{\pgfqpoint{4.731901in}{2.169915in}}%
\pgfpathlineto{\pgfqpoint{4.734552in}{2.165796in}}%
\pgfpathlineto{\pgfqpoint{4.737348in}{2.164812in}}%
\pgfpathlineto{\pgfqpoint{4.739912in}{2.169594in}}%
\pgfpathlineto{\pgfqpoint{4.742696in}{2.160550in}}%
\pgfpathlineto{\pgfqpoint{4.745256in}{2.163906in}}%
\pgfpathlineto{\pgfqpoint{4.748081in}{2.164783in}}%
\pgfpathlineto{\pgfqpoint{4.750627in}{2.165495in}}%
\pgfpathlineto{\pgfqpoint{4.753298in}{2.162197in}}%
\pgfpathlineto{\pgfqpoint{4.755983in}{2.164330in}}%
\pgfpathlineto{\pgfqpoint{4.758653in}{2.160265in}}%
\pgfpathlineto{\pgfqpoint{4.761337in}{2.162410in}}%
\pgfpathlineto{\pgfqpoint{4.764018in}{2.158178in}}%
\pgfpathlineto{\pgfqpoint{4.766783in}{2.157861in}}%
\pgfpathlineto{\pgfqpoint{4.769367in}{2.160896in}}%
\pgfpathlineto{\pgfqpoint{4.772198in}{2.166291in}}%
\pgfpathlineto{\pgfqpoint{4.774732in}{2.165602in}}%
\pgfpathlineto{\pgfqpoint{4.777535in}{2.165115in}}%
\pgfpathlineto{\pgfqpoint{4.780083in}{2.162923in}}%
\pgfpathlineto{\pgfqpoint{4.782872in}{2.164040in}}%
\pgfpathlineto{\pgfqpoint{4.785445in}{2.160390in}}%
\pgfpathlineto{\pgfqpoint{4.788116in}{2.162937in}}%
\pgfpathlineto{\pgfqpoint{4.790798in}{2.170122in}}%
\pgfpathlineto{\pgfqpoint{4.793512in}{2.170489in}}%
\pgfpathlineto{\pgfqpoint{4.796274in}{2.170375in}}%
\pgfpathlineto{\pgfqpoint{4.798830in}{2.169604in}}%
\pgfpathlineto{\pgfqpoint{4.801586in}{2.167381in}}%
\pgfpathlineto{\pgfqpoint{4.804193in}{2.169565in}}%
\pgfpathlineto{\pgfqpoint{4.807017in}{2.165132in}}%
\pgfpathlineto{\pgfqpoint{4.809538in}{2.166467in}}%
\pgfpathlineto{\pgfqpoint{4.812377in}{2.165306in}}%
\pgfpathlineto{\pgfqpoint{4.814907in}{2.160955in}}%
\pgfpathlineto{\pgfqpoint{4.817587in}{2.161802in}}%
\pgfpathlineto{\pgfqpoint{4.820265in}{2.163640in}}%
\pgfpathlineto{\pgfqpoint{4.822945in}{2.167875in}}%
\pgfpathlineto{\pgfqpoint{4.825619in}{2.163723in}}%
\pgfpathlineto{\pgfqpoint{4.828291in}{2.157849in}}%
\pgfpathlineto{\pgfqpoint{4.831045in}{2.158741in}}%
\pgfpathlineto{\pgfqpoint{4.833657in}{2.158911in}}%
\pgfpathlineto{\pgfqpoint{4.837992in}{2.158983in}}%
\pgfpathlineto{\pgfqpoint{4.839922in}{2.160264in}}%
\pgfpathlineto{\pgfqpoint{4.842380in}{2.163395in}}%
\pgfpathlineto{\pgfqpoint{4.844361in}{2.161749in}}%
\pgfpathlineto{\pgfqpoint{4.847127in}{2.167164in}}%
\pgfpathlineto{\pgfqpoint{4.849715in}{2.163883in}}%
\pgfpathlineto{\pgfqpoint{4.852404in}{2.172381in}}%
\pgfpathlineto{\pgfqpoint{4.855070in}{2.166028in}}%
\pgfpathlineto{\pgfqpoint{4.857807in}{2.165894in}}%
\pgfpathlineto{\pgfqpoint{4.860544in}{2.166444in}}%
\pgfpathlineto{\pgfqpoint{4.863116in}{2.163630in}}%
\pgfpathlineto{\pgfqpoint{4.865910in}{2.164840in}}%
\pgfpathlineto{\pgfqpoint{4.868474in}{2.157333in}}%
\pgfpathlineto{\pgfqpoint{4.871209in}{2.162580in}}%
\pgfpathlineto{\pgfqpoint{4.873832in}{2.158268in}}%
\pgfpathlineto{\pgfqpoint{4.876636in}{2.165778in}}%
\pgfpathlineto{\pgfqpoint{4.879180in}{2.160250in}}%
\pgfpathlineto{\pgfqpoint{4.881864in}{2.160883in}}%
\pgfpathlineto{\pgfqpoint{4.884540in}{2.155017in}}%
\pgfpathlineto{\pgfqpoint{4.887211in}{2.157966in}}%
\pgfpathlineto{\pgfqpoint{4.889902in}{2.157993in}}%
\pgfpathlineto{\pgfqpoint{4.892611in}{2.162722in}}%
\pgfpathlineto{\pgfqpoint{4.895399in}{2.162805in}}%
\pgfpathlineto{\pgfqpoint{4.897938in}{2.161489in}}%
\pgfpathlineto{\pgfqpoint{4.900712in}{2.165473in}}%
\pgfpathlineto{\pgfqpoint{4.903295in}{2.165906in}}%
\pgfpathlineto{\pgfqpoint{4.906096in}{2.165755in}}%
\pgfpathlineto{\pgfqpoint{4.908648in}{2.159830in}}%
\pgfpathlineto{\pgfqpoint{4.911435in}{2.164506in}}%
\pgfpathlineto{\pgfqpoint{4.914009in}{2.169217in}}%
\pgfpathlineto{\pgfqpoint{4.916681in}{2.173027in}}%
\pgfpathlineto{\pgfqpoint{4.919352in}{2.173179in}}%
\pgfpathlineto{\pgfqpoint{4.922041in}{2.169063in}}%
\pgfpathlineto{\pgfqpoint{4.924708in}{2.168011in}}%
\pgfpathlineto{\pgfqpoint{4.927400in}{2.165854in}}%
\pgfpathlineto{\pgfqpoint{4.930170in}{2.173955in}}%
\pgfpathlineto{\pgfqpoint{4.932742in}{2.170389in}}%
\pgfpathlineto{\pgfqpoint{4.935515in}{2.166265in}}%
\pgfpathlineto{\pgfqpoint{4.938112in}{2.166149in}}%
\pgfpathlineto{\pgfqpoint{4.940881in}{2.163199in}}%
\pgfpathlineto{\pgfqpoint{4.943466in}{2.162570in}}%
\pgfpathlineto{\pgfqpoint{4.946151in}{2.158779in}}%
\pgfpathlineto{\pgfqpoint{4.948827in}{2.163492in}}%
\pgfpathlineto{\pgfqpoint{4.951504in}{2.166216in}}%
\pgfpathlineto{\pgfqpoint{4.954182in}{2.161850in}}%
\pgfpathlineto{\pgfqpoint{4.956862in}{2.168857in}}%
\pgfpathlineto{\pgfqpoint{4.959689in}{2.165268in}}%
\pgfpathlineto{\pgfqpoint{4.962219in}{2.165894in}}%
\pgfpathlineto{\pgfqpoint{4.965002in}{2.165677in}}%
\pgfpathlineto{\pgfqpoint{4.967575in}{2.163787in}}%
\pgfpathlineto{\pgfqpoint{4.970314in}{2.163171in}}%
\pgfpathlineto{\pgfqpoint{4.972933in}{2.164869in}}%
\pgfpathlineto{\pgfqpoint{4.975703in}{2.160057in}}%
\pgfpathlineto{\pgfqpoint{4.978287in}{2.161618in}}%
\pgfpathlineto{\pgfqpoint{4.980967in}{2.162043in}}%
\pgfpathlineto{\pgfqpoint{4.983637in}{2.164519in}}%
\pgfpathlineto{\pgfqpoint{4.986325in}{2.162063in}}%
\pgfpathlineto{\pgfqpoint{4.989001in}{2.159893in}}%
\pgfpathlineto{\pgfqpoint{4.991683in}{2.161337in}}%
\pgfpathlineto{\pgfqpoint{4.994390in}{2.163662in}}%
\pgfpathlineto{\pgfqpoint{4.997028in}{2.166180in}}%
\pgfpathlineto{\pgfqpoint{4.999780in}{2.165011in}}%
\pgfpathlineto{\pgfqpoint{5.002384in}{2.167743in}}%
\pgfpathlineto{\pgfqpoint{5.005178in}{2.165584in}}%
\pgfpathlineto{\pgfqpoint{5.007751in}{2.168617in}}%
\pgfpathlineto{\pgfqpoint{5.010562in}{2.169576in}}%
\pgfpathlineto{\pgfqpoint{5.013104in}{2.166755in}}%
\pgfpathlineto{\pgfqpoint{5.015820in}{2.167343in}}%
\pgfpathlineto{\pgfqpoint{5.018466in}{2.160024in}}%
\pgfpathlineto{\pgfqpoint{5.021147in}{2.162218in}}%
\pgfpathlineto{\pgfqpoint{5.023927in}{2.163704in}}%
\pgfpathlineto{\pgfqpoint{5.026501in}{2.166827in}}%
\pgfpathlineto{\pgfqpoint{5.029275in}{2.165141in}}%
\pgfpathlineto{\pgfqpoint{5.031849in}{2.167295in}}%
\pgfpathlineto{\pgfqpoint{5.034649in}{2.160537in}}%
\pgfpathlineto{\pgfqpoint{5.037214in}{2.156332in}}%
\pgfpathlineto{\pgfqpoint{5.039962in}{2.157204in}}%
\pgfpathlineto{\pgfqpoint{5.042572in}{2.156260in}}%
\pgfpathlineto{\pgfqpoint{5.045249in}{2.158405in}}%
\pgfpathlineto{\pgfqpoint{5.047924in}{2.157705in}}%
\pgfpathlineto{\pgfqpoint{5.050606in}{2.161990in}}%
\pgfpathlineto{\pgfqpoint{5.053284in}{2.166263in}}%
\pgfpathlineto{\pgfqpoint{5.055952in}{2.169488in}}%
\pgfpathlineto{\pgfqpoint{5.058711in}{2.166247in}}%
\pgfpathlineto{\pgfqpoint{5.061315in}{2.171374in}}%
\pgfpathlineto{\pgfqpoint{5.064144in}{2.169048in}}%
\pgfpathlineto{\pgfqpoint{5.066677in}{2.169536in}}%
\pgfpathlineto{\pgfqpoint{5.069463in}{2.171326in}}%
\pgfpathlineto{\pgfqpoint{5.072030in}{2.169112in}}%
\pgfpathlineto{\pgfqpoint{5.074851in}{2.165527in}}%
\pgfpathlineto{\pgfqpoint{5.077390in}{2.170353in}}%
\pgfpathlineto{\pgfqpoint{5.080067in}{2.170982in}}%
\pgfpathlineto{\pgfqpoint{5.082746in}{2.168834in}}%
\pgfpathlineto{\pgfqpoint{5.085426in}{2.168746in}}%
\pgfpathlineto{\pgfqpoint{5.088103in}{2.168317in}}%
\pgfpathlineto{\pgfqpoint{5.090788in}{2.169172in}}%
\pgfpathlineto{\pgfqpoint{5.093579in}{2.166640in}}%
\pgfpathlineto{\pgfqpoint{5.096142in}{2.169012in}}%
\pgfpathlineto{\pgfqpoint{5.098948in}{2.167243in}}%
\pgfpathlineto{\pgfqpoint{5.101496in}{2.171713in}}%
\pgfpathlineto{\pgfqpoint{5.104312in}{2.166170in}}%
\pgfpathlineto{\pgfqpoint{5.106842in}{2.168733in}}%
\pgfpathlineto{\pgfqpoint{5.109530in}{2.168876in}}%
\pgfpathlineto{\pgfqpoint{5.112209in}{2.166494in}}%
\pgfpathlineto{\pgfqpoint{5.114887in}{2.166311in}}%
\pgfpathlineto{\pgfqpoint{5.117550in}{2.165470in}}%
\pgfpathlineto{\pgfqpoint{5.120243in}{2.166321in}}%
\pgfpathlineto{\pgfqpoint{5.123042in}{2.169707in}}%
\pgfpathlineto{\pgfqpoint{5.125599in}{2.165493in}}%
\pgfpathlineto{\pgfqpoint{5.128421in}{2.167137in}}%
\pgfpathlineto{\pgfqpoint{5.130953in}{2.166969in}}%
\pgfpathlineto{\pgfqpoint{5.133716in}{2.167420in}}%
\pgfpathlineto{\pgfqpoint{5.136311in}{2.172562in}}%
\pgfpathlineto{\pgfqpoint{5.139072in}{2.174051in}}%
\pgfpathlineto{\pgfqpoint{5.141660in}{2.166202in}}%
\pgfpathlineto{\pgfqpoint{5.144349in}{2.173004in}}%
\pgfpathlineto{\pgfqpoint{5.147029in}{2.171681in}}%
\pgfpathlineto{\pgfqpoint{5.149734in}{2.163835in}}%
\pgfpathlineto{\pgfqpoint{5.152382in}{2.154664in}}%
\pgfpathlineto{\pgfqpoint{5.155059in}{2.157637in}}%
\pgfpathlineto{\pgfqpoint{5.157815in}{2.163119in}}%
\pgfpathlineto{\pgfqpoint{5.160420in}{2.166338in}}%
\pgfpathlineto{\pgfqpoint{5.163243in}{2.168001in}}%
\pgfpathlineto{\pgfqpoint{5.165775in}{2.167014in}}%
\pgfpathlineto{\pgfqpoint{5.168591in}{2.166063in}}%
\pgfpathlineto{\pgfqpoint{5.171133in}{2.167516in}}%
\pgfpathlineto{\pgfqpoint{5.173925in}{2.170705in}}%
\pgfpathlineto{\pgfqpoint{5.176477in}{2.168693in}}%
\pgfpathlineto{\pgfqpoint{5.179188in}{2.168262in}}%
\pgfpathlineto{\pgfqpoint{5.181848in}{2.164108in}}%
\pgfpathlineto{\pgfqpoint{5.184522in}{2.164699in}}%
\pgfpathlineto{\pgfqpoint{5.187294in}{2.169832in}}%
\pgfpathlineto{\pgfqpoint{5.189880in}{2.173589in}}%
\pgfpathlineto{\pgfqpoint{5.192680in}{2.166526in}}%
\pgfpathlineto{\pgfqpoint{5.195239in}{2.165883in}}%
\pgfpathlineto{\pgfqpoint{5.198008in}{2.166811in}}%
\pgfpathlineto{\pgfqpoint{5.200594in}{2.166592in}}%
\pgfpathlineto{\pgfqpoint{5.203388in}{2.170998in}}%
\pgfpathlineto{\pgfqpoint{5.205952in}{2.169125in}}%
\pgfpathlineto{\pgfqpoint{5.208630in}{2.167601in}}%
\pgfpathlineto{\pgfqpoint{5.211299in}{2.167662in}}%
\pgfpathlineto{\pgfqpoint{5.214027in}{2.166599in}}%
\pgfpathlineto{\pgfqpoint{5.216667in}{2.161628in}}%
\pgfpathlineto{\pgfqpoint{5.219345in}{2.171512in}}%
\pgfpathlineto{\pgfqpoint{5.222151in}{2.164580in}}%
\pgfpathlineto{\pgfqpoint{5.224695in}{2.168265in}}%
\pgfpathlineto{\pgfqpoint{5.227470in}{2.173885in}}%
\pgfpathlineto{\pgfqpoint{5.230059in}{2.171543in}}%
\pgfpathlineto{\pgfqpoint{5.232855in}{2.174680in}}%
\pgfpathlineto{\pgfqpoint{5.235409in}{2.172939in}}%
\pgfpathlineto{\pgfqpoint{5.238173in}{2.169904in}}%
\pgfpathlineto{\pgfqpoint{5.240777in}{2.172512in}}%
\pgfpathlineto{\pgfqpoint{5.243445in}{2.171352in}}%
\pgfpathlineto{\pgfqpoint{5.246130in}{2.170588in}}%
\pgfpathlineto{\pgfqpoint{5.248816in}{2.172622in}}%
\pgfpathlineto{\pgfqpoint{5.251590in}{2.162881in}}%
\pgfpathlineto{\pgfqpoint{5.254236in}{2.171741in}}%
\pgfpathlineto{\pgfqpoint{5.256973in}{2.174942in}}%
\pgfpathlineto{\pgfqpoint{5.259511in}{2.175821in}}%
\pgfpathlineto{\pgfqpoint{5.262264in}{2.171269in}}%
\pgfpathlineto{\pgfqpoint{5.264876in}{2.163505in}}%
\pgfpathlineto{\pgfqpoint{5.267691in}{2.171624in}}%
\pgfpathlineto{\pgfqpoint{5.270238in}{2.176027in}}%
\pgfpathlineto{\pgfqpoint{5.272913in}{2.175949in}}%
\pgfpathlineto{\pgfqpoint{5.275589in}{2.180975in}}%
\pgfpathlineto{\pgfqpoint{5.278322in}{2.161719in}}%
\pgfpathlineto{\pgfqpoint{5.280947in}{2.162100in}}%
\pgfpathlineto{\pgfqpoint{5.283631in}{2.160738in}}%
\pgfpathlineto{\pgfqpoint{5.286436in}{2.158226in}}%
\pgfpathlineto{\pgfqpoint{5.288984in}{2.161282in}}%
\pgfpathlineto{\pgfqpoint{5.291794in}{2.162192in}}%
\pgfpathlineto{\pgfqpoint{5.294339in}{2.159460in}}%
\pgfpathlineto{\pgfqpoint{5.297140in}{2.159070in}}%
\pgfpathlineto{\pgfqpoint{5.299696in}{2.161373in}}%
\pgfpathlineto{\pgfqpoint{5.302443in}{2.162653in}}%
\pgfpathlineto{\pgfqpoint{5.305054in}{2.162961in}}%
\pgfpathlineto{\pgfqpoint{5.307731in}{2.165267in}}%
\pgfpathlineto{\pgfqpoint{5.310411in}{2.162012in}}%
\pgfpathlineto{\pgfqpoint{5.313089in}{2.164530in}}%
\pgfpathlineto{\pgfqpoint{5.315754in}{2.165690in}}%
\pgfpathlineto{\pgfqpoint{5.318430in}{2.165994in}}%
\pgfpathlineto{\pgfqpoint{5.321256in}{2.165313in}}%
\pgfpathlineto{\pgfqpoint{5.323802in}{2.163994in}}%
\pgfpathlineto{\pgfqpoint{5.326564in}{2.166306in}}%
\pgfpathlineto{\pgfqpoint{5.329159in}{2.166954in}}%
\pgfpathlineto{\pgfqpoint{5.331973in}{2.163435in}}%
\pgfpathlineto{\pgfqpoint{5.334510in}{2.163071in}}%
\pgfpathlineto{\pgfqpoint{5.337353in}{2.159811in}}%
\pgfpathlineto{\pgfqpoint{5.339872in}{2.163207in}}%
\pgfpathlineto{\pgfqpoint{5.342549in}{2.161943in}}%
\pgfpathlineto{\pgfqpoint{5.345224in}{2.154664in}}%
\pgfpathlineto{\pgfqpoint{5.347905in}{2.154664in}}%
\pgfpathlineto{\pgfqpoint{5.350723in}{2.154664in}}%
\pgfpathlineto{\pgfqpoint{5.353262in}{2.156360in}}%
\pgfpathlineto{\pgfqpoint{5.356056in}{2.156869in}}%
\pgfpathlineto{\pgfqpoint{5.358612in}{2.159776in}}%
\pgfpathlineto{\pgfqpoint{5.361370in}{2.162653in}}%
\pgfpathlineto{\pgfqpoint{5.363966in}{2.162150in}}%
\pgfpathlineto{\pgfqpoint{5.366727in}{2.156214in}}%
\pgfpathlineto{\pgfqpoint{5.369335in}{2.162869in}}%
\pgfpathlineto{\pgfqpoint{5.372013in}{2.162043in}}%
\pgfpathlineto{\pgfqpoint{5.374692in}{2.162248in}}%
\pgfpathlineto{\pgfqpoint{5.377370in}{2.161355in}}%
\pgfpathlineto{\pgfqpoint{5.380048in}{2.164832in}}%
\pgfpathlineto{\pgfqpoint{5.382725in}{2.159620in}}%
\pgfpathlineto{\pgfqpoint{5.385550in}{2.164889in}}%
\pgfpathlineto{\pgfqpoint{5.388083in}{2.163669in}}%
\pgfpathlineto{\pgfqpoint{5.390900in}{2.166311in}}%
\pgfpathlineto{\pgfqpoint{5.393441in}{2.162965in}}%
\pgfpathlineto{\pgfqpoint{5.396219in}{2.162751in}}%
\pgfpathlineto{\pgfqpoint{5.398784in}{2.162725in}}%
\pgfpathlineto{\pgfqpoint{5.401576in}{2.163334in}}%
\pgfpathlineto{\pgfqpoint{5.404154in}{2.165261in}}%
\pgfpathlineto{\pgfqpoint{5.406832in}{2.161527in}}%
\pgfpathlineto{\pgfqpoint{5.409507in}{2.166260in}}%
\pgfpathlineto{\pgfqpoint{5.412190in}{2.164214in}}%
\pgfpathlineto{\pgfqpoint{5.414954in}{2.161390in}}%
\pgfpathlineto{\pgfqpoint{5.417547in}{2.161354in}}%
\pgfpathlineto{\pgfqpoint{5.420304in}{2.158865in}}%
\pgfpathlineto{\pgfqpoint{5.422897in}{2.158223in}}%
\pgfpathlineto{\pgfqpoint{5.425661in}{2.157475in}}%
\pgfpathlineto{\pgfqpoint{5.428259in}{2.160159in}}%
\pgfpathlineto{\pgfqpoint{5.431015in}{2.160415in}}%
\pgfpathlineto{\pgfqpoint{5.433616in}{2.163597in}}%
\pgfpathlineto{\pgfqpoint{5.436295in}{2.162417in}}%
\pgfpathlineto{\pgfqpoint{5.438974in}{2.163932in}}%
\pgfpathlineto{\pgfqpoint{5.441698in}{2.164608in}}%
\pgfpathlineto{\pgfqpoint{5.444328in}{2.163199in}}%
\pgfpathlineto{\pgfqpoint{5.447021in}{2.160248in}}%
\pgfpathlineto{\pgfqpoint{5.449769in}{2.163119in}}%
\pgfpathlineto{\pgfqpoint{5.452365in}{2.163953in}}%
\pgfpathlineto{\pgfqpoint{5.455168in}{2.157173in}}%
\pgfpathlineto{\pgfqpoint{5.457721in}{2.162264in}}%
\pgfpathlineto{\pgfqpoint{5.460489in}{2.161605in}}%
\pgfpathlineto{\pgfqpoint{5.463079in}{2.162503in}}%
\pgfpathlineto{\pgfqpoint{5.465888in}{2.161014in}}%
\pgfpathlineto{\pgfqpoint{5.468425in}{2.156621in}}%
\pgfpathlineto{\pgfqpoint{5.471113in}{2.160014in}}%
\pgfpathlineto{\pgfqpoint{5.473792in}{2.161949in}}%
\pgfpathlineto{\pgfqpoint{5.476458in}{2.166710in}}%
\pgfpathlineto{\pgfqpoint{5.479152in}{2.160469in}}%
\pgfpathlineto{\pgfqpoint{5.481825in}{2.166426in}}%
\pgfpathlineto{\pgfqpoint{5.484641in}{2.166684in}}%
\pgfpathlineto{\pgfqpoint{5.487176in}{2.166924in}}%
\pgfpathlineto{\pgfqpoint{5.490000in}{2.163493in}}%
\pgfpathlineto{\pgfqpoint{5.492541in}{2.166283in}}%
\pgfpathlineto{\pgfqpoint{5.495346in}{2.165391in}}%
\pgfpathlineto{\pgfqpoint{5.497898in}{2.163679in}}%
\pgfpathlineto{\pgfqpoint{5.500687in}{2.168347in}}%
\pgfpathlineto{\pgfqpoint{5.503255in}{2.169060in}}%
\pgfpathlineto{\pgfqpoint{5.505933in}{2.170062in}}%
\pgfpathlineto{\pgfqpoint{5.508612in}{2.174985in}}%
\pgfpathlineto{\pgfqpoint{5.511290in}{2.172835in}}%
\pgfpathlineto{\pgfqpoint{5.514080in}{2.176566in}}%
\pgfpathlineto{\pgfqpoint{5.516646in}{2.184335in}}%
\pgfpathlineto{\pgfqpoint{5.519433in}{2.183765in}}%
\pgfpathlineto{\pgfqpoint{5.522003in}{2.178826in}}%
\pgfpathlineto{\pgfqpoint{5.524756in}{2.178752in}}%
\pgfpathlineto{\pgfqpoint{5.527360in}{2.174511in}}%
\pgfpathlineto{\pgfqpoint{5.530148in}{2.171310in}}%
\pgfpathlineto{\pgfqpoint{5.532717in}{2.173899in}}%
\pgfpathlineto{\pgfqpoint{5.535395in}{2.177240in}}%
\pgfpathlineto{\pgfqpoint{5.538074in}{2.172103in}}%
\pgfpathlineto{\pgfqpoint{5.540750in}{2.171823in}}%
\pgfpathlineto{\pgfqpoint{5.543421in}{2.165848in}}%
\pgfpathlineto{\pgfqpoint{5.546110in}{2.167196in}}%
\pgfpathlineto{\pgfqpoint{5.548921in}{2.165834in}}%
\pgfpathlineto{\pgfqpoint{5.551457in}{2.165689in}}%
\pgfpathlineto{\pgfqpoint{5.554198in}{2.162617in}}%
\pgfpathlineto{\pgfqpoint{5.556822in}{2.167649in}}%
\pgfpathlineto{\pgfqpoint{5.559612in}{2.170639in}}%
\pgfpathlineto{\pgfqpoint{5.562180in}{2.168904in}}%
\pgfpathlineto{\pgfqpoint{5.564940in}{2.168357in}}%
\pgfpathlineto{\pgfqpoint{5.567536in}{2.172713in}}%
\pgfpathlineto{\pgfqpoint{5.570215in}{2.168996in}}%
\pgfpathlineto{\pgfqpoint{5.572893in}{2.166558in}}%
\pgfpathlineto{\pgfqpoint{5.575596in}{2.163344in}}%
\pgfpathlineto{\pgfqpoint{5.578342in}{2.164735in}}%
\pgfpathlineto{\pgfqpoint{5.580914in}{2.165595in}}%
\pgfpathlineto{\pgfqpoint{5.583709in}{2.171093in}}%
\pgfpathlineto{\pgfqpoint{5.586269in}{2.166642in}}%
\pgfpathlineto{\pgfqpoint{5.589040in}{2.159851in}}%
\pgfpathlineto{\pgfqpoint{5.591641in}{2.157709in}}%
\pgfpathlineto{\pgfqpoint{5.594368in}{2.162464in}}%
\pgfpathlineto{\pgfqpoint{5.596999in}{2.156807in}}%
\pgfpathlineto{\pgfqpoint{5.599674in}{2.156755in}}%
\pgfpathlineto{\pgfqpoint{5.602352in}{2.157194in}}%
\pgfpathlineto{\pgfqpoint{5.605073in}{2.158520in}}%
\pgfpathlineto{\pgfqpoint{5.607698in}{2.159838in}}%
\pgfpathlineto{\pgfqpoint{5.610389in}{2.160195in}}%
\pgfpathlineto{\pgfqpoint{5.613235in}{2.158862in}}%
\pgfpathlineto{\pgfqpoint{5.615743in}{2.159037in}}%
\pgfpathlineto{\pgfqpoint{5.618526in}{2.159220in}}%
\pgfpathlineto{\pgfqpoint{5.621102in}{2.163744in}}%
\pgfpathlineto{\pgfqpoint{5.623868in}{2.159675in}}%
\pgfpathlineto{\pgfqpoint{5.626460in}{2.162128in}}%
\pgfpathlineto{\pgfqpoint{5.629232in}{2.162489in}}%
\pgfpathlineto{\pgfqpoint{5.631815in}{2.164477in}}%
\pgfpathlineto{\pgfqpoint{5.634496in}{2.162474in}}%
\pgfpathlineto{\pgfqpoint{5.637172in}{2.165278in}}%
\pgfpathlineto{\pgfqpoint{5.639852in}{2.172429in}}%
\pgfpathlineto{\pgfqpoint{5.642518in}{2.184634in}}%
\pgfpathlineto{\pgfqpoint{5.645243in}{2.177539in}}%
\pgfpathlineto{\pgfqpoint{5.648008in}{2.196314in}}%
\pgfpathlineto{\pgfqpoint{5.650563in}{2.200481in}}%
\pgfpathlineto{\pgfqpoint{5.653376in}{2.186042in}}%
\pgfpathlineto{\pgfqpoint{5.655919in}{2.171582in}}%
\pgfpathlineto{\pgfqpoint{5.658723in}{2.169372in}}%
\pgfpathlineto{\pgfqpoint{5.661273in}{2.163972in}}%
\pgfpathlineto{\pgfqpoint{5.664099in}{2.162558in}}%
\pgfpathlineto{\pgfqpoint{5.666632in}{2.172968in}}%
\pgfpathlineto{\pgfqpoint{5.669313in}{2.180760in}}%
\pgfpathlineto{\pgfqpoint{5.671991in}{2.181639in}}%
\pgfpathlineto{\pgfqpoint{5.674667in}{2.176387in}}%
\pgfpathlineto{\pgfqpoint{5.677486in}{2.173011in}}%
\pgfpathlineto{\pgfqpoint{5.680027in}{2.170272in}}%
\pgfpathlineto{\pgfqpoint{5.682836in}{2.172970in}}%
\pgfpathlineto{\pgfqpoint{5.685385in}{2.173985in}}%
\pgfpathlineto{\pgfqpoint{5.688159in}{2.171750in}}%
\pgfpathlineto{\pgfqpoint{5.690730in}{2.168138in}}%
\pgfpathlineto{\pgfqpoint{5.693473in}{2.171678in}}%
\pgfpathlineto{\pgfqpoint{5.696101in}{2.171932in}}%
\pgfpathlineto{\pgfqpoint{5.698775in}{2.170995in}}%
\pgfpathlineto{\pgfqpoint{5.701453in}{2.172470in}}%
\pgfpathlineto{\pgfqpoint{5.704130in}{2.199668in}}%
\pgfpathlineto{\pgfqpoint{5.706800in}{2.197124in}}%
\pgfpathlineto{\pgfqpoint{5.709490in}{2.182613in}}%
\pgfpathlineto{\pgfqpoint{5.712291in}{2.178112in}}%
\pgfpathlineto{\pgfqpoint{5.714834in}{2.176880in}}%
\pgfpathlineto{\pgfqpoint{5.717671in}{2.170236in}}%
\pgfpathlineto{\pgfqpoint{5.720201in}{2.168109in}}%
\pgfpathlineto{\pgfqpoint{5.722950in}{2.164823in}}%
\pgfpathlineto{\pgfqpoint{5.725548in}{2.162988in}}%
\pgfpathlineto{\pgfqpoint{5.728339in}{2.162588in}}%
\pgfpathlineto{\pgfqpoint{5.730919in}{2.166261in}}%
\pgfpathlineto{\pgfqpoint{5.733594in}{2.165175in}}%
\pgfpathlineto{\pgfqpoint{5.736276in}{2.164486in}}%
\pgfpathlineto{\pgfqpoint{5.738974in}{2.170129in}}%
\pgfpathlineto{\pgfqpoint{5.741745in}{2.170410in}}%
\pgfpathlineto{\pgfqpoint{5.744310in}{2.166713in}}%
\pgfpathlineto{\pgfqpoint{5.744310in}{0.413320in}}%
\pgfpathlineto{\pgfqpoint{5.744310in}{0.413320in}}%
\pgfpathlineto{\pgfqpoint{5.741745in}{0.413320in}}%
\pgfpathlineto{\pgfqpoint{5.738974in}{0.413320in}}%
\pgfpathlineto{\pgfqpoint{5.736276in}{0.413320in}}%
\pgfpathlineto{\pgfqpoint{5.733594in}{0.413320in}}%
\pgfpathlineto{\pgfqpoint{5.730919in}{0.413320in}}%
\pgfpathlineto{\pgfqpoint{5.728339in}{0.413320in}}%
\pgfpathlineto{\pgfqpoint{5.725548in}{0.413320in}}%
\pgfpathlineto{\pgfqpoint{5.722950in}{0.413320in}}%
\pgfpathlineto{\pgfqpoint{5.720201in}{0.413320in}}%
\pgfpathlineto{\pgfqpoint{5.717671in}{0.413320in}}%
\pgfpathlineto{\pgfqpoint{5.714834in}{0.413320in}}%
\pgfpathlineto{\pgfqpoint{5.712291in}{0.413320in}}%
\pgfpathlineto{\pgfqpoint{5.709490in}{0.413320in}}%
\pgfpathlineto{\pgfqpoint{5.706800in}{0.413320in}}%
\pgfpathlineto{\pgfqpoint{5.704130in}{0.413320in}}%
\pgfpathlineto{\pgfqpoint{5.701453in}{0.413320in}}%
\pgfpathlineto{\pgfqpoint{5.698775in}{0.413320in}}%
\pgfpathlineto{\pgfqpoint{5.696101in}{0.413320in}}%
\pgfpathlineto{\pgfqpoint{5.693473in}{0.413320in}}%
\pgfpathlineto{\pgfqpoint{5.690730in}{0.413320in}}%
\pgfpathlineto{\pgfqpoint{5.688159in}{0.413320in}}%
\pgfpathlineto{\pgfqpoint{5.685385in}{0.413320in}}%
\pgfpathlineto{\pgfqpoint{5.682836in}{0.413320in}}%
\pgfpathlineto{\pgfqpoint{5.680027in}{0.413320in}}%
\pgfpathlineto{\pgfqpoint{5.677486in}{0.413320in}}%
\pgfpathlineto{\pgfqpoint{5.674667in}{0.413320in}}%
\pgfpathlineto{\pgfqpoint{5.671991in}{0.413320in}}%
\pgfpathlineto{\pgfqpoint{5.669313in}{0.413320in}}%
\pgfpathlineto{\pgfqpoint{5.666632in}{0.413320in}}%
\pgfpathlineto{\pgfqpoint{5.664099in}{0.413320in}}%
\pgfpathlineto{\pgfqpoint{5.661273in}{0.413320in}}%
\pgfpathlineto{\pgfqpoint{5.658723in}{0.413320in}}%
\pgfpathlineto{\pgfqpoint{5.655919in}{0.413320in}}%
\pgfpathlineto{\pgfqpoint{5.653376in}{0.413320in}}%
\pgfpathlineto{\pgfqpoint{5.650563in}{0.413320in}}%
\pgfpathlineto{\pgfqpoint{5.648008in}{0.413320in}}%
\pgfpathlineto{\pgfqpoint{5.645243in}{0.413320in}}%
\pgfpathlineto{\pgfqpoint{5.642518in}{0.413320in}}%
\pgfpathlineto{\pgfqpoint{5.639852in}{0.413320in}}%
\pgfpathlineto{\pgfqpoint{5.637172in}{0.413320in}}%
\pgfpathlineto{\pgfqpoint{5.634496in}{0.413320in}}%
\pgfpathlineto{\pgfqpoint{5.631815in}{0.413320in}}%
\pgfpathlineto{\pgfqpoint{5.629232in}{0.413320in}}%
\pgfpathlineto{\pgfqpoint{5.626460in}{0.413320in}}%
\pgfpathlineto{\pgfqpoint{5.623868in}{0.413320in}}%
\pgfpathlineto{\pgfqpoint{5.621102in}{0.413320in}}%
\pgfpathlineto{\pgfqpoint{5.618526in}{0.413320in}}%
\pgfpathlineto{\pgfqpoint{5.615743in}{0.413320in}}%
\pgfpathlineto{\pgfqpoint{5.613235in}{0.413320in}}%
\pgfpathlineto{\pgfqpoint{5.610389in}{0.413320in}}%
\pgfpathlineto{\pgfqpoint{5.607698in}{0.413320in}}%
\pgfpathlineto{\pgfqpoint{5.605073in}{0.413320in}}%
\pgfpathlineto{\pgfqpoint{5.602352in}{0.413320in}}%
\pgfpathlineto{\pgfqpoint{5.599674in}{0.413320in}}%
\pgfpathlineto{\pgfqpoint{5.596999in}{0.413320in}}%
\pgfpathlineto{\pgfqpoint{5.594368in}{0.413320in}}%
\pgfpathlineto{\pgfqpoint{5.591641in}{0.413320in}}%
\pgfpathlineto{\pgfqpoint{5.589040in}{0.413320in}}%
\pgfpathlineto{\pgfqpoint{5.586269in}{0.413320in}}%
\pgfpathlineto{\pgfqpoint{5.583709in}{0.413320in}}%
\pgfpathlineto{\pgfqpoint{5.580914in}{0.413320in}}%
\pgfpathlineto{\pgfqpoint{5.578342in}{0.413320in}}%
\pgfpathlineto{\pgfqpoint{5.575596in}{0.413320in}}%
\pgfpathlineto{\pgfqpoint{5.572893in}{0.413320in}}%
\pgfpathlineto{\pgfqpoint{5.570215in}{0.413320in}}%
\pgfpathlineto{\pgfqpoint{5.567536in}{0.413320in}}%
\pgfpathlineto{\pgfqpoint{5.564940in}{0.413320in}}%
\pgfpathlineto{\pgfqpoint{5.562180in}{0.413320in}}%
\pgfpathlineto{\pgfqpoint{5.559612in}{0.413320in}}%
\pgfpathlineto{\pgfqpoint{5.556822in}{0.413320in}}%
\pgfpathlineto{\pgfqpoint{5.554198in}{0.413320in}}%
\pgfpathlineto{\pgfqpoint{5.551457in}{0.413320in}}%
\pgfpathlineto{\pgfqpoint{5.548921in}{0.413320in}}%
\pgfpathlineto{\pgfqpoint{5.546110in}{0.413320in}}%
\pgfpathlineto{\pgfqpoint{5.543421in}{0.413320in}}%
\pgfpathlineto{\pgfqpoint{5.540750in}{0.413320in}}%
\pgfpathlineto{\pgfqpoint{5.538074in}{0.413320in}}%
\pgfpathlineto{\pgfqpoint{5.535395in}{0.413320in}}%
\pgfpathlineto{\pgfqpoint{5.532717in}{0.413320in}}%
\pgfpathlineto{\pgfqpoint{5.530148in}{0.413320in}}%
\pgfpathlineto{\pgfqpoint{5.527360in}{0.413320in}}%
\pgfpathlineto{\pgfqpoint{5.524756in}{0.413320in}}%
\pgfpathlineto{\pgfqpoint{5.522003in}{0.413320in}}%
\pgfpathlineto{\pgfqpoint{5.519433in}{0.413320in}}%
\pgfpathlineto{\pgfqpoint{5.516646in}{0.413320in}}%
\pgfpathlineto{\pgfqpoint{5.514080in}{0.413320in}}%
\pgfpathlineto{\pgfqpoint{5.511290in}{0.413320in}}%
\pgfpathlineto{\pgfqpoint{5.508612in}{0.413320in}}%
\pgfpathlineto{\pgfqpoint{5.505933in}{0.413320in}}%
\pgfpathlineto{\pgfqpoint{5.503255in}{0.413320in}}%
\pgfpathlineto{\pgfqpoint{5.500687in}{0.413320in}}%
\pgfpathlineto{\pgfqpoint{5.497898in}{0.413320in}}%
\pgfpathlineto{\pgfqpoint{5.495346in}{0.413320in}}%
\pgfpathlineto{\pgfqpoint{5.492541in}{0.413320in}}%
\pgfpathlineto{\pgfqpoint{5.490000in}{0.413320in}}%
\pgfpathlineto{\pgfqpoint{5.487176in}{0.413320in}}%
\pgfpathlineto{\pgfqpoint{5.484641in}{0.413320in}}%
\pgfpathlineto{\pgfqpoint{5.481825in}{0.413320in}}%
\pgfpathlineto{\pgfqpoint{5.479152in}{0.413320in}}%
\pgfpathlineto{\pgfqpoint{5.476458in}{0.413320in}}%
\pgfpathlineto{\pgfqpoint{5.473792in}{0.413320in}}%
\pgfpathlineto{\pgfqpoint{5.471113in}{0.413320in}}%
\pgfpathlineto{\pgfqpoint{5.468425in}{0.413320in}}%
\pgfpathlineto{\pgfqpoint{5.465888in}{0.413320in}}%
\pgfpathlineto{\pgfqpoint{5.463079in}{0.413320in}}%
\pgfpathlineto{\pgfqpoint{5.460489in}{0.413320in}}%
\pgfpathlineto{\pgfqpoint{5.457721in}{0.413320in}}%
\pgfpathlineto{\pgfqpoint{5.455168in}{0.413320in}}%
\pgfpathlineto{\pgfqpoint{5.452365in}{0.413320in}}%
\pgfpathlineto{\pgfqpoint{5.449769in}{0.413320in}}%
\pgfpathlineto{\pgfqpoint{5.447021in}{0.413320in}}%
\pgfpathlineto{\pgfqpoint{5.444328in}{0.413320in}}%
\pgfpathlineto{\pgfqpoint{5.441698in}{0.413320in}}%
\pgfpathlineto{\pgfqpoint{5.438974in}{0.413320in}}%
\pgfpathlineto{\pgfqpoint{5.436295in}{0.413320in}}%
\pgfpathlineto{\pgfqpoint{5.433616in}{0.413320in}}%
\pgfpathlineto{\pgfqpoint{5.431015in}{0.413320in}}%
\pgfpathlineto{\pgfqpoint{5.428259in}{0.413320in}}%
\pgfpathlineto{\pgfqpoint{5.425661in}{0.413320in}}%
\pgfpathlineto{\pgfqpoint{5.422897in}{0.413320in}}%
\pgfpathlineto{\pgfqpoint{5.420304in}{0.413320in}}%
\pgfpathlineto{\pgfqpoint{5.417547in}{0.413320in}}%
\pgfpathlineto{\pgfqpoint{5.414954in}{0.413320in}}%
\pgfpathlineto{\pgfqpoint{5.412190in}{0.413320in}}%
\pgfpathlineto{\pgfqpoint{5.409507in}{0.413320in}}%
\pgfpathlineto{\pgfqpoint{5.406832in}{0.413320in}}%
\pgfpathlineto{\pgfqpoint{5.404154in}{0.413320in}}%
\pgfpathlineto{\pgfqpoint{5.401576in}{0.413320in}}%
\pgfpathlineto{\pgfqpoint{5.398784in}{0.413320in}}%
\pgfpathlineto{\pgfqpoint{5.396219in}{0.413320in}}%
\pgfpathlineto{\pgfqpoint{5.393441in}{0.413320in}}%
\pgfpathlineto{\pgfqpoint{5.390900in}{0.413320in}}%
\pgfpathlineto{\pgfqpoint{5.388083in}{0.413320in}}%
\pgfpathlineto{\pgfqpoint{5.385550in}{0.413320in}}%
\pgfpathlineto{\pgfqpoint{5.382725in}{0.413320in}}%
\pgfpathlineto{\pgfqpoint{5.380048in}{0.413320in}}%
\pgfpathlineto{\pgfqpoint{5.377370in}{0.413320in}}%
\pgfpathlineto{\pgfqpoint{5.374692in}{0.413320in}}%
\pgfpathlineto{\pgfqpoint{5.372013in}{0.413320in}}%
\pgfpathlineto{\pgfqpoint{5.369335in}{0.413320in}}%
\pgfpathlineto{\pgfqpoint{5.366727in}{0.413320in}}%
\pgfpathlineto{\pgfqpoint{5.363966in}{0.413320in}}%
\pgfpathlineto{\pgfqpoint{5.361370in}{0.413320in}}%
\pgfpathlineto{\pgfqpoint{5.358612in}{0.413320in}}%
\pgfpathlineto{\pgfqpoint{5.356056in}{0.413320in}}%
\pgfpathlineto{\pgfqpoint{5.353262in}{0.413320in}}%
\pgfpathlineto{\pgfqpoint{5.350723in}{0.413320in}}%
\pgfpathlineto{\pgfqpoint{5.347905in}{0.413320in}}%
\pgfpathlineto{\pgfqpoint{5.345224in}{0.413320in}}%
\pgfpathlineto{\pgfqpoint{5.342549in}{0.413320in}}%
\pgfpathlineto{\pgfqpoint{5.339872in}{0.413320in}}%
\pgfpathlineto{\pgfqpoint{5.337353in}{0.413320in}}%
\pgfpathlineto{\pgfqpoint{5.334510in}{0.413320in}}%
\pgfpathlineto{\pgfqpoint{5.331973in}{0.413320in}}%
\pgfpathlineto{\pgfqpoint{5.329159in}{0.413320in}}%
\pgfpathlineto{\pgfqpoint{5.326564in}{0.413320in}}%
\pgfpathlineto{\pgfqpoint{5.323802in}{0.413320in}}%
\pgfpathlineto{\pgfqpoint{5.321256in}{0.413320in}}%
\pgfpathlineto{\pgfqpoint{5.318430in}{0.413320in}}%
\pgfpathlineto{\pgfqpoint{5.315754in}{0.413320in}}%
\pgfpathlineto{\pgfqpoint{5.313089in}{0.413320in}}%
\pgfpathlineto{\pgfqpoint{5.310411in}{0.413320in}}%
\pgfpathlineto{\pgfqpoint{5.307731in}{0.413320in}}%
\pgfpathlineto{\pgfqpoint{5.305054in}{0.413320in}}%
\pgfpathlineto{\pgfqpoint{5.302443in}{0.413320in}}%
\pgfpathlineto{\pgfqpoint{5.299696in}{0.413320in}}%
\pgfpathlineto{\pgfqpoint{5.297140in}{0.413320in}}%
\pgfpathlineto{\pgfqpoint{5.294339in}{0.413320in}}%
\pgfpathlineto{\pgfqpoint{5.291794in}{0.413320in}}%
\pgfpathlineto{\pgfqpoint{5.288984in}{0.413320in}}%
\pgfpathlineto{\pgfqpoint{5.286436in}{0.413320in}}%
\pgfpathlineto{\pgfqpoint{5.283631in}{0.413320in}}%
\pgfpathlineto{\pgfqpoint{5.280947in}{0.413320in}}%
\pgfpathlineto{\pgfqpoint{5.278322in}{0.413320in}}%
\pgfpathlineto{\pgfqpoint{5.275589in}{0.413320in}}%
\pgfpathlineto{\pgfqpoint{5.272913in}{0.413320in}}%
\pgfpathlineto{\pgfqpoint{5.270238in}{0.413320in}}%
\pgfpathlineto{\pgfqpoint{5.267691in}{0.413320in}}%
\pgfpathlineto{\pgfqpoint{5.264876in}{0.413320in}}%
\pgfpathlineto{\pgfqpoint{5.262264in}{0.413320in}}%
\pgfpathlineto{\pgfqpoint{5.259511in}{0.413320in}}%
\pgfpathlineto{\pgfqpoint{5.256973in}{0.413320in}}%
\pgfpathlineto{\pgfqpoint{5.254236in}{0.413320in}}%
\pgfpathlineto{\pgfqpoint{5.251590in}{0.413320in}}%
\pgfpathlineto{\pgfqpoint{5.248816in}{0.413320in}}%
\pgfpathlineto{\pgfqpoint{5.246130in}{0.413320in}}%
\pgfpathlineto{\pgfqpoint{5.243445in}{0.413320in}}%
\pgfpathlineto{\pgfqpoint{5.240777in}{0.413320in}}%
\pgfpathlineto{\pgfqpoint{5.238173in}{0.413320in}}%
\pgfpathlineto{\pgfqpoint{5.235409in}{0.413320in}}%
\pgfpathlineto{\pgfqpoint{5.232855in}{0.413320in}}%
\pgfpathlineto{\pgfqpoint{5.230059in}{0.413320in}}%
\pgfpathlineto{\pgfqpoint{5.227470in}{0.413320in}}%
\pgfpathlineto{\pgfqpoint{5.224695in}{0.413320in}}%
\pgfpathlineto{\pgfqpoint{5.222151in}{0.413320in}}%
\pgfpathlineto{\pgfqpoint{5.219345in}{0.413320in}}%
\pgfpathlineto{\pgfqpoint{5.216667in}{0.413320in}}%
\pgfpathlineto{\pgfqpoint{5.214027in}{0.413320in}}%
\pgfpathlineto{\pgfqpoint{5.211299in}{0.413320in}}%
\pgfpathlineto{\pgfqpoint{5.208630in}{0.413320in}}%
\pgfpathlineto{\pgfqpoint{5.205952in}{0.413320in}}%
\pgfpathlineto{\pgfqpoint{5.203388in}{0.413320in}}%
\pgfpathlineto{\pgfqpoint{5.200594in}{0.413320in}}%
\pgfpathlineto{\pgfqpoint{5.198008in}{0.413320in}}%
\pgfpathlineto{\pgfqpoint{5.195239in}{0.413320in}}%
\pgfpathlineto{\pgfqpoint{5.192680in}{0.413320in}}%
\pgfpathlineto{\pgfqpoint{5.189880in}{0.413320in}}%
\pgfpathlineto{\pgfqpoint{5.187294in}{0.413320in}}%
\pgfpathlineto{\pgfqpoint{5.184522in}{0.413320in}}%
\pgfpathlineto{\pgfqpoint{5.181848in}{0.413320in}}%
\pgfpathlineto{\pgfqpoint{5.179188in}{0.413320in}}%
\pgfpathlineto{\pgfqpoint{5.176477in}{0.413320in}}%
\pgfpathlineto{\pgfqpoint{5.173925in}{0.413320in}}%
\pgfpathlineto{\pgfqpoint{5.171133in}{0.413320in}}%
\pgfpathlineto{\pgfqpoint{5.168591in}{0.413320in}}%
\pgfpathlineto{\pgfqpoint{5.165775in}{0.413320in}}%
\pgfpathlineto{\pgfqpoint{5.163243in}{0.413320in}}%
\pgfpathlineto{\pgfqpoint{5.160420in}{0.413320in}}%
\pgfpathlineto{\pgfqpoint{5.157815in}{0.413320in}}%
\pgfpathlineto{\pgfqpoint{5.155059in}{0.413320in}}%
\pgfpathlineto{\pgfqpoint{5.152382in}{0.413320in}}%
\pgfpathlineto{\pgfqpoint{5.149734in}{0.413320in}}%
\pgfpathlineto{\pgfqpoint{5.147029in}{0.413320in}}%
\pgfpathlineto{\pgfqpoint{5.144349in}{0.413320in}}%
\pgfpathlineto{\pgfqpoint{5.141660in}{0.413320in}}%
\pgfpathlineto{\pgfqpoint{5.139072in}{0.413320in}}%
\pgfpathlineto{\pgfqpoint{5.136311in}{0.413320in}}%
\pgfpathlineto{\pgfqpoint{5.133716in}{0.413320in}}%
\pgfpathlineto{\pgfqpoint{5.130953in}{0.413320in}}%
\pgfpathlineto{\pgfqpoint{5.128421in}{0.413320in}}%
\pgfpathlineto{\pgfqpoint{5.125599in}{0.413320in}}%
\pgfpathlineto{\pgfqpoint{5.123042in}{0.413320in}}%
\pgfpathlineto{\pgfqpoint{5.120243in}{0.413320in}}%
\pgfpathlineto{\pgfqpoint{5.117550in}{0.413320in}}%
\pgfpathlineto{\pgfqpoint{5.114887in}{0.413320in}}%
\pgfpathlineto{\pgfqpoint{5.112209in}{0.413320in}}%
\pgfpathlineto{\pgfqpoint{5.109530in}{0.413320in}}%
\pgfpathlineto{\pgfqpoint{5.106842in}{0.413320in}}%
\pgfpathlineto{\pgfqpoint{5.104312in}{0.413320in}}%
\pgfpathlineto{\pgfqpoint{5.101496in}{0.413320in}}%
\pgfpathlineto{\pgfqpoint{5.098948in}{0.413320in}}%
\pgfpathlineto{\pgfqpoint{5.096142in}{0.413320in}}%
\pgfpathlineto{\pgfqpoint{5.093579in}{0.413320in}}%
\pgfpathlineto{\pgfqpoint{5.090788in}{0.413320in}}%
\pgfpathlineto{\pgfqpoint{5.088103in}{0.413320in}}%
\pgfpathlineto{\pgfqpoint{5.085426in}{0.413320in}}%
\pgfpathlineto{\pgfqpoint{5.082746in}{0.413320in}}%
\pgfpathlineto{\pgfqpoint{5.080067in}{0.413320in}}%
\pgfpathlineto{\pgfqpoint{5.077390in}{0.413320in}}%
\pgfpathlineto{\pgfqpoint{5.074851in}{0.413320in}}%
\pgfpathlineto{\pgfqpoint{5.072030in}{0.413320in}}%
\pgfpathlineto{\pgfqpoint{5.069463in}{0.413320in}}%
\pgfpathlineto{\pgfqpoint{5.066677in}{0.413320in}}%
\pgfpathlineto{\pgfqpoint{5.064144in}{0.413320in}}%
\pgfpathlineto{\pgfqpoint{5.061315in}{0.413320in}}%
\pgfpathlineto{\pgfqpoint{5.058711in}{0.413320in}}%
\pgfpathlineto{\pgfqpoint{5.055952in}{0.413320in}}%
\pgfpathlineto{\pgfqpoint{5.053284in}{0.413320in}}%
\pgfpathlineto{\pgfqpoint{5.050606in}{0.413320in}}%
\pgfpathlineto{\pgfqpoint{5.047924in}{0.413320in}}%
\pgfpathlineto{\pgfqpoint{5.045249in}{0.413320in}}%
\pgfpathlineto{\pgfqpoint{5.042572in}{0.413320in}}%
\pgfpathlineto{\pgfqpoint{5.039962in}{0.413320in}}%
\pgfpathlineto{\pgfqpoint{5.037214in}{0.413320in}}%
\pgfpathlineto{\pgfqpoint{5.034649in}{0.413320in}}%
\pgfpathlineto{\pgfqpoint{5.031849in}{0.413320in}}%
\pgfpathlineto{\pgfqpoint{5.029275in}{0.413320in}}%
\pgfpathlineto{\pgfqpoint{5.026501in}{0.413320in}}%
\pgfpathlineto{\pgfqpoint{5.023927in}{0.413320in}}%
\pgfpathlineto{\pgfqpoint{5.021147in}{0.413320in}}%
\pgfpathlineto{\pgfqpoint{5.018466in}{0.413320in}}%
\pgfpathlineto{\pgfqpoint{5.015820in}{0.413320in}}%
\pgfpathlineto{\pgfqpoint{5.013104in}{0.413320in}}%
\pgfpathlineto{\pgfqpoint{5.010562in}{0.413320in}}%
\pgfpathlineto{\pgfqpoint{5.007751in}{0.413320in}}%
\pgfpathlineto{\pgfqpoint{5.005178in}{0.413320in}}%
\pgfpathlineto{\pgfqpoint{5.002384in}{0.413320in}}%
\pgfpathlineto{\pgfqpoint{4.999780in}{0.413320in}}%
\pgfpathlineto{\pgfqpoint{4.997028in}{0.413320in}}%
\pgfpathlineto{\pgfqpoint{4.994390in}{0.413320in}}%
\pgfpathlineto{\pgfqpoint{4.991683in}{0.413320in}}%
\pgfpathlineto{\pgfqpoint{4.989001in}{0.413320in}}%
\pgfpathlineto{\pgfqpoint{4.986325in}{0.413320in}}%
\pgfpathlineto{\pgfqpoint{4.983637in}{0.413320in}}%
\pgfpathlineto{\pgfqpoint{4.980967in}{0.413320in}}%
\pgfpathlineto{\pgfqpoint{4.978287in}{0.413320in}}%
\pgfpathlineto{\pgfqpoint{4.975703in}{0.413320in}}%
\pgfpathlineto{\pgfqpoint{4.972933in}{0.413320in}}%
\pgfpathlineto{\pgfqpoint{4.970314in}{0.413320in}}%
\pgfpathlineto{\pgfqpoint{4.967575in}{0.413320in}}%
\pgfpathlineto{\pgfqpoint{4.965002in}{0.413320in}}%
\pgfpathlineto{\pgfqpoint{4.962219in}{0.413320in}}%
\pgfpathlineto{\pgfqpoint{4.959689in}{0.413320in}}%
\pgfpathlineto{\pgfqpoint{4.956862in}{0.413320in}}%
\pgfpathlineto{\pgfqpoint{4.954182in}{0.413320in}}%
\pgfpathlineto{\pgfqpoint{4.951504in}{0.413320in}}%
\pgfpathlineto{\pgfqpoint{4.948827in}{0.413320in}}%
\pgfpathlineto{\pgfqpoint{4.946151in}{0.413320in}}%
\pgfpathlineto{\pgfqpoint{4.943466in}{0.413320in}}%
\pgfpathlineto{\pgfqpoint{4.940881in}{0.413320in}}%
\pgfpathlineto{\pgfqpoint{4.938112in}{0.413320in}}%
\pgfpathlineto{\pgfqpoint{4.935515in}{0.413320in}}%
\pgfpathlineto{\pgfqpoint{4.932742in}{0.413320in}}%
\pgfpathlineto{\pgfqpoint{4.930170in}{0.413320in}}%
\pgfpathlineto{\pgfqpoint{4.927400in}{0.413320in}}%
\pgfpathlineto{\pgfqpoint{4.924708in}{0.413320in}}%
\pgfpathlineto{\pgfqpoint{4.922041in}{0.413320in}}%
\pgfpathlineto{\pgfqpoint{4.919352in}{0.413320in}}%
\pgfpathlineto{\pgfqpoint{4.916681in}{0.413320in}}%
\pgfpathlineto{\pgfqpoint{4.914009in}{0.413320in}}%
\pgfpathlineto{\pgfqpoint{4.911435in}{0.413320in}}%
\pgfpathlineto{\pgfqpoint{4.908648in}{0.413320in}}%
\pgfpathlineto{\pgfqpoint{4.906096in}{0.413320in}}%
\pgfpathlineto{\pgfqpoint{4.903295in}{0.413320in}}%
\pgfpathlineto{\pgfqpoint{4.900712in}{0.413320in}}%
\pgfpathlineto{\pgfqpoint{4.897938in}{0.413320in}}%
\pgfpathlineto{\pgfqpoint{4.895399in}{0.413320in}}%
\pgfpathlineto{\pgfqpoint{4.892611in}{0.413320in}}%
\pgfpathlineto{\pgfqpoint{4.889902in}{0.413320in}}%
\pgfpathlineto{\pgfqpoint{4.887211in}{0.413320in}}%
\pgfpathlineto{\pgfqpoint{4.884540in}{0.413320in}}%
\pgfpathlineto{\pgfqpoint{4.881864in}{0.413320in}}%
\pgfpathlineto{\pgfqpoint{4.879180in}{0.413320in}}%
\pgfpathlineto{\pgfqpoint{4.876636in}{0.413320in}}%
\pgfpathlineto{\pgfqpoint{4.873832in}{0.413320in}}%
\pgfpathlineto{\pgfqpoint{4.871209in}{0.413320in}}%
\pgfpathlineto{\pgfqpoint{4.868474in}{0.413320in}}%
\pgfpathlineto{\pgfqpoint{4.865910in}{0.413320in}}%
\pgfpathlineto{\pgfqpoint{4.863116in}{0.413320in}}%
\pgfpathlineto{\pgfqpoint{4.860544in}{0.413320in}}%
\pgfpathlineto{\pgfqpoint{4.857807in}{0.413320in}}%
\pgfpathlineto{\pgfqpoint{4.855070in}{0.413320in}}%
\pgfpathlineto{\pgfqpoint{4.852404in}{0.413320in}}%
\pgfpathlineto{\pgfqpoint{4.849715in}{0.413320in}}%
\pgfpathlineto{\pgfqpoint{4.847127in}{0.413320in}}%
\pgfpathlineto{\pgfqpoint{4.844361in}{0.413320in}}%
\pgfpathlineto{\pgfqpoint{4.842380in}{0.413320in}}%
\pgfpathlineto{\pgfqpoint{4.839922in}{0.413320in}}%
\pgfpathlineto{\pgfqpoint{4.837992in}{0.413320in}}%
\pgfpathlineto{\pgfqpoint{4.833657in}{0.413320in}}%
\pgfpathlineto{\pgfqpoint{4.831045in}{0.413320in}}%
\pgfpathlineto{\pgfqpoint{4.828291in}{0.413320in}}%
\pgfpathlineto{\pgfqpoint{4.825619in}{0.413320in}}%
\pgfpathlineto{\pgfqpoint{4.822945in}{0.413320in}}%
\pgfpathlineto{\pgfqpoint{4.820265in}{0.413320in}}%
\pgfpathlineto{\pgfqpoint{4.817587in}{0.413320in}}%
\pgfpathlineto{\pgfqpoint{4.814907in}{0.413320in}}%
\pgfpathlineto{\pgfqpoint{4.812377in}{0.413320in}}%
\pgfpathlineto{\pgfqpoint{4.809538in}{0.413320in}}%
\pgfpathlineto{\pgfqpoint{4.807017in}{0.413320in}}%
\pgfpathlineto{\pgfqpoint{4.804193in}{0.413320in}}%
\pgfpathlineto{\pgfqpoint{4.801586in}{0.413320in}}%
\pgfpathlineto{\pgfqpoint{4.798830in}{0.413320in}}%
\pgfpathlineto{\pgfqpoint{4.796274in}{0.413320in}}%
\pgfpathlineto{\pgfqpoint{4.793512in}{0.413320in}}%
\pgfpathlineto{\pgfqpoint{4.790798in}{0.413320in}}%
\pgfpathlineto{\pgfqpoint{4.788116in}{0.413320in}}%
\pgfpathlineto{\pgfqpoint{4.785445in}{0.413320in}}%
\pgfpathlineto{\pgfqpoint{4.782872in}{0.413320in}}%
\pgfpathlineto{\pgfqpoint{4.780083in}{0.413320in}}%
\pgfpathlineto{\pgfqpoint{4.777535in}{0.413320in}}%
\pgfpathlineto{\pgfqpoint{4.774732in}{0.413320in}}%
\pgfpathlineto{\pgfqpoint{4.772198in}{0.413320in}}%
\pgfpathlineto{\pgfqpoint{4.769367in}{0.413320in}}%
\pgfpathlineto{\pgfqpoint{4.766783in}{0.413320in}}%
\pgfpathlineto{\pgfqpoint{4.764018in}{0.413320in}}%
\pgfpathlineto{\pgfqpoint{4.761337in}{0.413320in}}%
\pgfpathlineto{\pgfqpoint{4.758653in}{0.413320in}}%
\pgfpathlineto{\pgfqpoint{4.755983in}{0.413320in}}%
\pgfpathlineto{\pgfqpoint{4.753298in}{0.413320in}}%
\pgfpathlineto{\pgfqpoint{4.750627in}{0.413320in}}%
\pgfpathlineto{\pgfqpoint{4.748081in}{0.413320in}}%
\pgfpathlineto{\pgfqpoint{4.745256in}{0.413320in}}%
\pgfpathlineto{\pgfqpoint{4.742696in}{0.413320in}}%
\pgfpathlineto{\pgfqpoint{4.739912in}{0.413320in}}%
\pgfpathlineto{\pgfqpoint{4.737348in}{0.413320in}}%
\pgfpathlineto{\pgfqpoint{4.734552in}{0.413320in}}%
\pgfpathlineto{\pgfqpoint{4.731901in}{0.413320in}}%
\pgfpathlineto{\pgfqpoint{4.729233in}{0.413320in}}%
\pgfpathlineto{\pgfqpoint{4.726508in}{0.413320in}}%
\pgfpathlineto{\pgfqpoint{4.723873in}{0.413320in}}%
\pgfpathlineto{\pgfqpoint{4.721160in}{0.413320in}}%
\pgfpathlineto{\pgfqpoint{4.718486in}{0.413320in}}%
\pgfpathlineto{\pgfqpoint{4.715806in}{0.413320in}}%
\pgfpathlineto{\pgfqpoint{4.713275in}{0.413320in}}%
\pgfpathlineto{\pgfqpoint{4.710437in}{0.413320in}}%
\pgfpathlineto{\pgfqpoint{4.707824in}{0.413320in}}%
\pgfpathlineto{\pgfqpoint{4.705094in}{0.413320in}}%
\pgfpathlineto{\pgfqpoint{4.702517in}{0.413320in}}%
\pgfpathlineto{\pgfqpoint{4.699734in}{0.413320in}}%
\pgfpathlineto{\pgfqpoint{4.697170in}{0.413320in}}%
\pgfpathlineto{\pgfqpoint{4.694381in}{0.413320in}}%
\pgfpathlineto{\pgfqpoint{4.691694in}{0.413320in}}%
\pgfpathlineto{\pgfqpoint{4.689051in}{0.413320in}}%
\pgfpathlineto{\pgfqpoint{4.686337in}{0.413320in}}%
\pgfpathlineto{\pgfqpoint{4.683799in}{0.413320in}}%
\pgfpathlineto{\pgfqpoint{4.680988in}{0.413320in}}%
\pgfpathlineto{\pgfqpoint{4.678448in}{0.413320in}}%
\pgfpathlineto{\pgfqpoint{4.675619in}{0.413320in}}%
\pgfpathlineto{\pgfqpoint{4.673068in}{0.413320in}}%
\pgfpathlineto{\pgfqpoint{4.670261in}{0.413320in}}%
\pgfpathlineto{\pgfqpoint{4.667764in}{0.413320in}}%
\pgfpathlineto{\pgfqpoint{4.664923in}{0.413320in}}%
\pgfpathlineto{\pgfqpoint{4.662237in}{0.413320in}}%
\pgfpathlineto{\pgfqpoint{4.659590in}{0.413320in}}%
\pgfpathlineto{\pgfqpoint{4.656873in}{0.413320in}}%
\pgfpathlineto{\pgfqpoint{4.654203in}{0.413320in}}%
\pgfpathlineto{\pgfqpoint{4.651524in}{0.413320in}}%
\pgfpathlineto{\pgfqpoint{4.648922in}{0.413320in}}%
\pgfpathlineto{\pgfqpoint{4.646169in}{0.413320in}}%
\pgfpathlineto{\pgfqpoint{4.643628in}{0.413320in}}%
\pgfpathlineto{\pgfqpoint{4.640809in}{0.413320in}}%
\pgfpathlineto{\pgfqpoint{4.638204in}{0.413320in}}%
\pgfpathlineto{\pgfqpoint{4.635445in}{0.413320in}}%
\pgfpathlineto{\pgfqpoint{4.632902in}{0.413320in}}%
\pgfpathlineto{\pgfqpoint{4.630096in}{0.413320in}}%
\pgfpathlineto{\pgfqpoint{4.627411in}{0.413320in}}%
\pgfpathlineto{\pgfqpoint{4.624741in}{0.413320in}}%
\pgfpathlineto{\pgfqpoint{4.622056in}{0.413320in}}%
\pgfpathlineto{\pgfqpoint{4.619529in}{0.413320in}}%
\pgfpathlineto{\pgfqpoint{4.616702in}{0.413320in}}%
\pgfpathlineto{\pgfqpoint{4.614134in}{0.413320in}}%
\pgfpathlineto{\pgfqpoint{4.611350in}{0.413320in}}%
\pgfpathlineto{\pgfqpoint{4.608808in}{0.413320in}}%
\pgfpathlineto{\pgfqpoint{4.605990in}{0.413320in}}%
\pgfpathlineto{\pgfqpoint{4.603430in}{0.413320in}}%
\pgfpathlineto{\pgfqpoint{4.600633in}{0.413320in}}%
\pgfpathlineto{\pgfqpoint{4.597951in}{0.413320in}}%
\pgfpathlineto{\pgfqpoint{4.595281in}{0.413320in}}%
\pgfpathlineto{\pgfqpoint{4.592589in}{0.413320in}}%
\pgfpathlineto{\pgfqpoint{4.589920in}{0.413320in}}%
\pgfpathlineto{\pgfqpoint{4.587244in}{0.413320in}}%
\pgfpathlineto{\pgfqpoint{4.584672in}{0.413320in}}%
\pgfpathlineto{\pgfqpoint{4.581888in}{0.413320in}}%
\pgfpathlineto{\pgfqpoint{4.579305in}{0.413320in}}%
\pgfpathlineto{\pgfqpoint{4.576531in}{0.413320in}}%
\pgfpathlineto{\pgfqpoint{4.573947in}{0.413320in}}%
\pgfpathlineto{\pgfqpoint{4.571171in}{0.413320in}}%
\pgfpathlineto{\pgfqpoint{4.568612in}{0.413320in}}%
\pgfpathlineto{\pgfqpoint{4.565820in}{0.413320in}}%
\pgfpathlineto{\pgfqpoint{4.563125in}{0.413320in}}%
\pgfpathlineto{\pgfqpoint{4.560448in}{0.413320in}}%
\pgfpathlineto{\pgfqpoint{4.557777in}{0.413320in}}%
\pgfpathlineto{\pgfqpoint{4.555106in}{0.413320in}}%
\pgfpathlineto{\pgfqpoint{4.552425in}{0.413320in}}%
\pgfpathlineto{\pgfqpoint{4.549822in}{0.413320in}}%
\pgfpathlineto{\pgfqpoint{4.547064in}{0.413320in}}%
\pgfpathlineto{\pgfqpoint{4.544464in}{0.413320in}}%
\pgfpathlineto{\pgfqpoint{4.541711in}{0.413320in}}%
\pgfpathlineto{\pgfqpoint{4.539144in}{0.413320in}}%
\pgfpathlineto{\pgfqpoint{4.536400in}{0.413320in}}%
\pgfpathlineto{\pgfqpoint{4.533764in}{0.413320in}}%
\pgfpathlineto{\pgfqpoint{4.530990in}{0.413320in}}%
\pgfpathlineto{\pgfqpoint{4.528307in}{0.413320in}}%
\pgfpathlineto{\pgfqpoint{4.525640in}{0.413320in}}%
\pgfpathlineto{\pgfqpoint{4.522962in}{0.413320in}}%
\pgfpathlineto{\pgfqpoint{4.520345in}{0.413320in}}%
\pgfpathlineto{\pgfqpoint{4.517598in}{0.413320in}}%
\pgfpathlineto{\pgfqpoint{4.515080in}{0.413320in}}%
\pgfpathlineto{\pgfqpoint{4.512246in}{0.413320in}}%
\pgfpathlineto{\pgfqpoint{4.509643in}{0.413320in}}%
\pgfpathlineto{\pgfqpoint{4.506893in}{0.413320in}}%
\pgfpathlineto{\pgfqpoint{4.504305in}{0.413320in}}%
\pgfpathlineto{\pgfqpoint{4.501529in}{0.413320in}}%
\pgfpathlineto{\pgfqpoint{4.498850in}{0.413320in}}%
\pgfpathlineto{\pgfqpoint{4.496167in}{0.413320in}}%
\pgfpathlineto{\pgfqpoint{4.493492in}{0.413320in}}%
\pgfpathlineto{\pgfqpoint{4.490822in}{0.413320in}}%
\pgfpathlineto{\pgfqpoint{4.488130in}{0.413320in}}%
\pgfpathlineto{\pgfqpoint{4.485581in}{0.413320in}}%
\pgfpathlineto{\pgfqpoint{4.482778in}{0.413320in}}%
\pgfpathlineto{\pgfqpoint{4.480201in}{0.413320in}}%
\pgfpathlineto{\pgfqpoint{4.477430in}{0.413320in}}%
\pgfpathlineto{\pgfqpoint{4.474861in}{0.413320in}}%
\pgfpathlineto{\pgfqpoint{4.472059in}{0.413320in}}%
\pgfpathlineto{\pgfqpoint{4.469492in}{0.413320in}}%
\pgfpathlineto{\pgfqpoint{4.466717in}{0.413320in}}%
\pgfpathlineto{\pgfqpoint{4.464029in}{0.413320in}}%
\pgfpathlineto{\pgfqpoint{4.461367in}{0.413320in}}%
\pgfpathlineto{\pgfqpoint{4.458681in}{0.413320in}}%
\pgfpathlineto{\pgfqpoint{4.456138in}{0.413320in}}%
\pgfpathlineto{\pgfqpoint{4.453312in}{0.413320in}}%
\pgfpathlineto{\pgfqpoint{4.450767in}{0.413320in}}%
\pgfpathlineto{\pgfqpoint{4.447965in}{0.413320in}}%
\pgfpathlineto{\pgfqpoint{4.445423in}{0.413320in}}%
\pgfpathlineto{\pgfqpoint{4.442611in}{0.413320in}}%
\pgfpathlineto{\pgfqpoint{4.440041in}{0.413320in}}%
\pgfpathlineto{\pgfqpoint{4.437253in}{0.413320in}}%
\pgfpathlineto{\pgfqpoint{4.434569in}{0.413320in}}%
\pgfpathlineto{\pgfqpoint{4.431901in}{0.413320in}}%
\pgfpathlineto{\pgfqpoint{4.429220in}{0.413320in}}%
\pgfpathlineto{\pgfqpoint{4.426534in}{0.413320in}}%
\pgfpathlineto{\pgfqpoint{4.423863in}{0.413320in}}%
\pgfpathlineto{\pgfqpoint{4.421292in}{0.413320in}}%
\pgfpathlineto{\pgfqpoint{4.418506in}{0.413320in}}%
\pgfpathlineto{\pgfqpoint{4.415932in}{0.413320in}}%
\pgfpathlineto{\pgfqpoint{4.413149in}{0.413320in}}%
\pgfpathlineto{\pgfqpoint{4.410587in}{0.413320in}}%
\pgfpathlineto{\pgfqpoint{4.407788in}{0.413320in}}%
\pgfpathlineto{\pgfqpoint{4.405234in}{0.413320in}}%
\pgfpathlineto{\pgfqpoint{4.402468in}{0.413320in}}%
\pgfpathlineto{\pgfqpoint{4.399745in}{0.413320in}}%
\pgfpathlineto{\pgfqpoint{4.397076in}{0.413320in}}%
\pgfpathlineto{\pgfqpoint{4.394400in}{0.413320in}}%
\pgfpathlineto{\pgfqpoint{4.391721in}{0.413320in}}%
\pgfpathlineto{\pgfqpoint{4.389044in}{0.413320in}}%
\pgfpathlineto{\pgfqpoint{4.386431in}{0.413320in}}%
\pgfpathlineto{\pgfqpoint{4.383674in}{0.413320in}}%
\pgfpathlineto{\pgfqpoint{4.381097in}{0.413320in}}%
\pgfpathlineto{\pgfqpoint{4.378329in}{0.413320in}}%
\pgfpathlineto{\pgfqpoint{4.375761in}{0.413320in}}%
\pgfpathlineto{\pgfqpoint{4.372976in}{0.413320in}}%
\pgfpathlineto{\pgfqpoint{4.370437in}{0.413320in}}%
\pgfpathlineto{\pgfqpoint{4.367646in}{0.413320in}}%
\pgfpathlineto{\pgfqpoint{4.364936in}{0.413320in}}%
\pgfpathlineto{\pgfqpoint{4.362270in}{0.413320in}}%
\pgfpathlineto{\pgfqpoint{4.359582in}{0.413320in}}%
\pgfpathlineto{\pgfqpoint{4.357014in}{0.413320in}}%
\pgfpathlineto{\pgfqpoint{4.354224in}{0.413320in}}%
\pgfpathlineto{\pgfqpoint{4.351645in}{0.413320in}}%
\pgfpathlineto{\pgfqpoint{4.348868in}{0.413320in}}%
\pgfpathlineto{\pgfqpoint{4.346263in}{0.413320in}}%
\pgfpathlineto{\pgfqpoint{4.343510in}{0.413320in}}%
\pgfpathlineto{\pgfqpoint{4.340976in}{0.413320in}}%
\pgfpathlineto{\pgfqpoint{4.338154in}{0.413320in}}%
\pgfpathlineto{\pgfqpoint{4.335463in}{0.413320in}}%
\pgfpathlineto{\pgfqpoint{4.332796in}{0.413320in}}%
\pgfpathlineto{\pgfqpoint{4.330118in}{0.413320in}}%
\pgfpathlineto{\pgfqpoint{4.327440in}{0.413320in}}%
\pgfpathlineto{\pgfqpoint{4.324760in}{0.413320in}}%
\pgfpathlineto{\pgfqpoint{4.322181in}{0.413320in}}%
\pgfpathlineto{\pgfqpoint{4.319405in}{0.413320in}}%
\pgfpathlineto{\pgfqpoint{4.316856in}{0.413320in}}%
\pgfpathlineto{\pgfqpoint{4.314032in}{0.413320in}}%
\pgfpathlineto{\pgfqpoint{4.311494in}{0.413320in}}%
\pgfpathlineto{\pgfqpoint{4.308691in}{0.413320in}}%
\pgfpathlineto{\pgfqpoint{4.306118in}{0.413320in}}%
\pgfpathlineto{\pgfqpoint{4.303357in}{0.413320in}}%
\pgfpathlineto{\pgfqpoint{4.300656in}{0.413320in}}%
\pgfpathlineto{\pgfqpoint{4.297977in}{0.413320in}}%
\pgfpathlineto{\pgfqpoint{4.295299in}{0.413320in}}%
\pgfpathlineto{\pgfqpoint{4.292786in}{0.413320in}}%
\pgfpathlineto{\pgfqpoint{4.289936in}{0.413320in}}%
\pgfpathlineto{\pgfqpoint{4.287399in}{0.413320in}}%
\pgfpathlineto{\pgfqpoint{4.284586in}{0.413320in}}%
\pgfpathlineto{\pgfqpoint{4.282000in}{0.413320in}}%
\pgfpathlineto{\pgfqpoint{4.279212in}{0.413320in}}%
\pgfpathlineto{\pgfqpoint{4.276635in}{0.413320in}}%
\pgfpathlineto{\pgfqpoint{4.273874in}{0.413320in}}%
\pgfpathlineto{\pgfqpoint{4.271187in}{0.413320in}}%
\pgfpathlineto{\pgfqpoint{4.268590in}{0.413320in}}%
\pgfpathlineto{\pgfqpoint{4.265824in}{0.413320in}}%
\pgfpathlineto{\pgfqpoint{4.263157in}{0.413320in}}%
\pgfpathlineto{\pgfqpoint{4.260477in}{0.413320in}}%
\pgfpathlineto{\pgfqpoint{4.257958in}{0.413320in}}%
\pgfpathlineto{\pgfqpoint{4.255120in}{0.413320in}}%
\pgfpathlineto{\pgfqpoint{4.252581in}{0.413320in}}%
\pgfpathlineto{\pgfqpoint{4.249767in}{0.413320in}}%
\pgfpathlineto{\pgfqpoint{4.247225in}{0.413320in}}%
\pgfpathlineto{\pgfqpoint{4.244394in}{0.413320in}}%
\pgfpathlineto{\pgfqpoint{4.241900in}{0.413320in}}%
\pgfpathlineto{\pgfqpoint{4.239084in}{0.413320in}}%
\pgfpathlineto{\pgfqpoint{4.236375in}{0.413320in}}%
\pgfpathlineto{\pgfqpoint{4.233691in}{0.413320in}}%
\pgfpathlineto{\pgfqpoint{4.231013in}{0.413320in}}%
\pgfpathlineto{\pgfqpoint{4.228331in}{0.413320in}}%
\pgfpathlineto{\pgfqpoint{4.225654in}{0.413320in}}%
\pgfpathlineto{\pgfqpoint{4.223082in}{0.413320in}}%
\pgfpathlineto{\pgfqpoint{4.220304in}{0.413320in}}%
\pgfpathlineto{\pgfqpoint{4.217694in}{0.413320in}}%
\pgfpathlineto{\pgfqpoint{4.214948in}{0.413320in}}%
\pgfpathlineto{\pgfqpoint{4.212383in}{0.413320in}}%
\pgfpathlineto{\pgfqpoint{4.209597in}{0.413320in}}%
\pgfpathlineto{\pgfqpoint{4.207076in}{0.413320in}}%
\pgfpathlineto{\pgfqpoint{4.204240in}{0.413320in}}%
\pgfpathlineto{\pgfqpoint{4.201542in}{0.413320in}}%
\pgfpathlineto{\pgfqpoint{4.198878in}{0.413320in}}%
\pgfpathlineto{\pgfqpoint{4.196186in}{0.413320in}}%
\pgfpathlineto{\pgfqpoint{4.193638in}{0.413320in}}%
\pgfpathlineto{\pgfqpoint{4.190842in}{0.413320in}}%
\pgfpathlineto{\pgfqpoint{4.188318in}{0.413320in}}%
\pgfpathlineto{\pgfqpoint{4.185481in}{0.413320in}}%
\pgfpathlineto{\pgfqpoint{4.182899in}{0.413320in}}%
\pgfpathlineto{\pgfqpoint{4.180129in}{0.413320in}}%
\pgfpathlineto{\pgfqpoint{4.177593in}{0.413320in}}%
\pgfpathlineto{\pgfqpoint{4.174770in}{0.413320in}}%
\pgfpathlineto{\pgfqpoint{4.172093in}{0.413320in}}%
\pgfpathlineto{\pgfqpoint{4.169415in}{0.413320in}}%
\pgfpathlineto{\pgfqpoint{4.166737in}{0.413320in}}%
\pgfpathlineto{\pgfqpoint{4.164059in}{0.413320in}}%
\pgfpathlineto{\pgfqpoint{4.161380in}{0.413320in}}%
\pgfpathlineto{\pgfqpoint{4.158806in}{0.413320in}}%
\pgfpathlineto{\pgfqpoint{4.156016in}{0.413320in}}%
\pgfpathlineto{\pgfqpoint{4.153423in}{0.413320in}}%
\pgfpathlineto{\pgfqpoint{4.150665in}{0.413320in}}%
\pgfpathlineto{\pgfqpoint{4.148082in}{0.413320in}}%
\pgfpathlineto{\pgfqpoint{4.145310in}{0.413320in}}%
\pgfpathlineto{\pgfqpoint{4.142713in}{0.413320in}}%
\pgfpathlineto{\pgfqpoint{4.139963in}{0.413320in}}%
\pgfpathlineto{\pgfqpoint{4.137272in}{0.413320in}}%
\pgfpathlineto{\pgfqpoint{4.134615in}{0.413320in}}%
\pgfpathlineto{\pgfqpoint{4.131920in}{0.413320in}}%
\pgfpathlineto{\pgfqpoint{4.129349in}{0.413320in}}%
\pgfpathlineto{\pgfqpoint{4.126553in}{0.413320in}}%
\pgfpathlineto{\pgfqpoint{4.124019in}{0.413320in}}%
\pgfpathlineto{\pgfqpoint{4.121205in}{0.413320in}}%
\pgfpathlineto{\pgfqpoint{4.118554in}{0.413320in}}%
\pgfpathlineto{\pgfqpoint{4.115844in}{0.413320in}}%
\pgfpathlineto{\pgfqpoint{4.113252in}{0.413320in}}%
\pgfpathlineto{\pgfqpoint{4.110488in}{0.413320in}}%
\pgfpathlineto{\pgfqpoint{4.107814in}{0.413320in}}%
\pgfpathlineto{\pgfqpoint{4.105185in}{0.413320in}}%
\pgfpathlineto{\pgfqpoint{4.102456in}{0.413320in}}%
\pgfpathlineto{\pgfqpoint{4.099777in}{0.413320in}}%
\pgfpathlineto{\pgfqpoint{4.097092in}{0.413320in}}%
\pgfpathlineto{\pgfqpoint{4.094527in}{0.413320in}}%
\pgfpathlineto{\pgfqpoint{4.091729in}{0.413320in}}%
\pgfpathlineto{\pgfqpoint{4.089159in}{0.413320in}}%
\pgfpathlineto{\pgfqpoint{4.086385in}{0.413320in}}%
\pgfpathlineto{\pgfqpoint{4.083870in}{0.413320in}}%
\pgfpathlineto{\pgfqpoint{4.081018in}{0.413320in}}%
\pgfpathlineto{\pgfqpoint{4.078471in}{0.413320in}}%
\pgfpathlineto{\pgfqpoint{4.075705in}{0.413320in}}%
\pgfpathlineto{\pgfqpoint{4.072985in}{0.413320in}}%
\pgfpathlineto{\pgfqpoint{4.070313in}{0.413320in}}%
\pgfpathlineto{\pgfqpoint{4.067636in}{0.413320in}}%
\pgfpathlineto{\pgfqpoint{4.064957in}{0.413320in}}%
\pgfpathlineto{\pgfqpoint{4.062266in}{0.413320in}}%
\pgfpathlineto{\pgfqpoint{4.059702in}{0.413320in}}%
\pgfpathlineto{\pgfqpoint{4.056911in}{0.413320in}}%
\pgfpathlineto{\pgfqpoint{4.054326in}{0.413320in}}%
\pgfpathlineto{\pgfqpoint{4.051557in}{0.413320in}}%
\pgfpathlineto{\pgfqpoint{4.049006in}{0.413320in}}%
\pgfpathlineto{\pgfqpoint{4.046210in}{0.413320in}}%
\pgfpathlineto{\pgfqpoint{4.043667in}{0.413320in}}%
\pgfpathlineto{\pgfqpoint{4.040852in}{0.413320in}}%
\pgfpathlineto{\pgfqpoint{4.038174in}{0.413320in}}%
\pgfpathlineto{\pgfqpoint{4.035492in}{0.413320in}}%
\pgfpathlineto{\pgfqpoint{4.032817in}{0.413320in}}%
\pgfpathlineto{\pgfqpoint{4.030229in}{0.413320in}}%
\pgfpathlineto{\pgfqpoint{4.027447in}{0.413320in}}%
\pgfpathlineto{\pgfqpoint{4.024868in}{0.413320in}}%
\pgfpathlineto{\pgfqpoint{4.022097in}{0.413320in}}%
\pgfpathlineto{\pgfqpoint{4.019518in}{0.413320in}}%
\pgfpathlineto{\pgfqpoint{4.016744in}{0.413320in}}%
\pgfpathlineto{\pgfqpoint{4.014186in}{0.413320in}}%
\pgfpathlineto{\pgfqpoint{4.011394in}{0.413320in}}%
\pgfpathlineto{\pgfqpoint{4.008699in}{0.413320in}}%
\pgfpathlineto{\pgfqpoint{4.006034in}{0.413320in}}%
\pgfpathlineto{\pgfqpoint{4.003348in}{0.413320in}}%
\pgfpathlineto{\pgfqpoint{4.000674in}{0.413320in}}%
\pgfpathlineto{\pgfqpoint{3.997990in}{0.413320in}}%
\pgfpathlineto{\pgfqpoint{3.995417in}{0.413320in}}%
\pgfpathlineto{\pgfqpoint{3.992642in}{0.413320in}}%
\pgfpathlineto{\pgfqpoint{3.990055in}{0.413320in}}%
\pgfpathlineto{\pgfqpoint{3.987270in}{0.413320in}}%
\pgfpathlineto{\pgfqpoint{3.984714in}{0.413320in}}%
\pgfpathlineto{\pgfqpoint{3.981929in}{0.413320in}}%
\pgfpathlineto{\pgfqpoint{3.979389in}{0.413320in}}%
\pgfpathlineto{\pgfqpoint{3.976563in}{0.413320in}}%
\pgfpathlineto{\pgfqpoint{3.973885in}{0.413320in}}%
\pgfpathlineto{\pgfqpoint{3.971250in}{0.413320in}}%
\pgfpathlineto{\pgfqpoint{3.968523in}{0.413320in}}%
\pgfpathlineto{\pgfqpoint{3.966013in}{0.413320in}}%
\pgfpathlineto{\pgfqpoint{3.963176in}{0.413320in}}%
\pgfpathlineto{\pgfqpoint{3.960635in}{0.413320in}}%
\pgfpathlineto{\pgfqpoint{3.957823in}{0.413320in}}%
\pgfpathlineto{\pgfqpoint{3.955211in}{0.413320in}}%
\pgfpathlineto{\pgfqpoint{3.952464in}{0.413320in}}%
\pgfpathlineto{\pgfqpoint{3.949894in}{0.413320in}}%
\pgfpathlineto{\pgfqpoint{3.947101in}{0.413320in}}%
\pgfpathlineto{\pgfqpoint{3.944431in}{0.413320in}}%
\pgfpathlineto{\pgfqpoint{3.941778in}{0.413320in}}%
\pgfpathlineto{\pgfqpoint{3.939075in}{0.413320in}}%
\pgfpathlineto{\pgfqpoint{3.936395in}{0.413320in}}%
\pgfpathlineto{\pgfqpoint{3.933714in}{0.413320in}}%
\pgfpathlineto{\pgfqpoint{3.931202in}{0.413320in}}%
\pgfpathlineto{\pgfqpoint{3.928347in}{0.413320in}}%
\pgfpathlineto{\pgfqpoint{3.925778in}{0.413320in}}%
\pgfpathlineto{\pgfqpoint{3.923005in}{0.413320in}}%
\pgfpathlineto{\pgfqpoint{3.920412in}{0.413320in}}%
\pgfpathlineto{\pgfqpoint{3.917646in}{0.413320in}}%
\pgfpathlineto{\pgfqpoint{3.915107in}{0.413320in}}%
\pgfpathlineto{\pgfqpoint{3.912296in}{0.413320in}}%
\pgfpathlineto{\pgfqpoint{3.909602in}{0.413320in}}%
\pgfpathlineto{\pgfqpoint{3.906918in}{0.413320in}}%
\pgfpathlineto{\pgfqpoint{3.904252in}{0.413320in}}%
\pgfpathlineto{\pgfqpoint{3.901573in}{0.413320in}}%
\pgfpathlineto{\pgfqpoint{3.898891in}{0.413320in}}%
\pgfpathlineto{\pgfqpoint{3.896345in}{0.413320in}}%
\pgfpathlineto{\pgfqpoint{3.893541in}{0.413320in}}%
\pgfpathlineto{\pgfqpoint{3.890926in}{0.413320in}}%
\pgfpathlineto{\pgfqpoint{3.888188in}{0.413320in}}%
\pgfpathlineto{\pgfqpoint{3.885621in}{0.413320in}}%
\pgfpathlineto{\pgfqpoint{3.882850in}{0.413320in}}%
\pgfpathlineto{\pgfqpoint{3.880237in}{0.413320in}}%
\pgfpathlineto{\pgfqpoint{3.877466in}{0.413320in}}%
\pgfpathlineto{\pgfqpoint{3.874790in}{0.413320in}}%
\pgfpathlineto{\pgfqpoint{3.872114in}{0.413320in}}%
\pgfpathlineto{\pgfqpoint{3.869435in}{0.413320in}}%
\pgfpathlineto{\pgfqpoint{3.866815in}{0.413320in}}%
\pgfpathlineto{\pgfqpoint{3.864073in}{0.413320in}}%
\pgfpathlineto{\pgfqpoint{3.861561in}{0.413320in}}%
\pgfpathlineto{\pgfqpoint{3.858720in}{0.413320in}}%
\pgfpathlineto{\pgfqpoint{3.856100in}{0.413320in}}%
\pgfpathlineto{\pgfqpoint{3.853358in}{0.413320in}}%
\pgfpathlineto{\pgfqpoint{3.850814in}{0.413320in}}%
\pgfpathlineto{\pgfqpoint{3.848005in}{0.413320in}}%
\pgfpathlineto{\pgfqpoint{3.845329in}{0.413320in}}%
\pgfpathlineto{\pgfqpoint{3.842641in}{0.413320in}}%
\pgfpathlineto{\pgfqpoint{3.839960in}{0.413320in}}%
\pgfpathlineto{\pgfqpoint{3.837286in}{0.413320in}}%
\pgfpathlineto{\pgfqpoint{3.834616in}{0.413320in}}%
\pgfpathlineto{\pgfqpoint{3.832053in}{0.413320in}}%
\pgfpathlineto{\pgfqpoint{3.829252in}{0.413320in}}%
\pgfpathlineto{\pgfqpoint{3.826679in}{0.413320in}}%
\pgfpathlineto{\pgfqpoint{3.823903in}{0.413320in}}%
\pgfpathlineto{\pgfqpoint{3.821315in}{0.413320in}}%
\pgfpathlineto{\pgfqpoint{3.818546in}{0.413320in}}%
\pgfpathlineto{\pgfqpoint{3.815983in}{0.413320in}}%
\pgfpathlineto{\pgfqpoint{3.813172in}{0.413320in}}%
\pgfpathlineto{\pgfqpoint{3.810510in}{0.413320in}}%
\pgfpathlineto{\pgfqpoint{3.807832in}{0.413320in}}%
\pgfpathlineto{\pgfqpoint{3.805145in}{0.413320in}}%
\pgfpathlineto{\pgfqpoint{3.802569in}{0.413320in}}%
\pgfpathlineto{\pgfqpoint{3.799797in}{0.413320in}}%
\pgfpathlineto{\pgfqpoint{3.797265in}{0.413320in}}%
\pgfpathlineto{\pgfqpoint{3.794435in}{0.413320in}}%
\pgfpathlineto{\pgfqpoint{3.791897in}{0.413320in}}%
\pgfpathlineto{\pgfqpoint{3.789084in}{0.413320in}}%
\pgfpathlineto{\pgfqpoint{3.786504in}{0.413320in}}%
\pgfpathlineto{\pgfqpoint{3.783725in}{0.413320in}}%
\pgfpathlineto{\pgfqpoint{3.781046in}{0.413320in}}%
\pgfpathlineto{\pgfqpoint{3.778370in}{0.413320in}}%
\pgfpathlineto{\pgfqpoint{3.775691in}{0.413320in}}%
\pgfpathlineto{\pgfqpoint{3.773014in}{0.413320in}}%
\pgfpathlineto{\pgfqpoint{3.770323in}{0.413320in}}%
\pgfpathlineto{\pgfqpoint{3.767782in}{0.413320in}}%
\pgfpathlineto{\pgfqpoint{3.764966in}{0.413320in}}%
\pgfpathlineto{\pgfqpoint{3.762389in}{0.413320in}}%
\pgfpathlineto{\pgfqpoint{3.759622in}{0.413320in}}%
\pgfpathlineto{\pgfqpoint{3.757065in}{0.413320in}}%
\pgfpathlineto{\pgfqpoint{3.754265in}{0.413320in}}%
\pgfpathlineto{\pgfqpoint{3.751728in}{0.413320in}}%
\pgfpathlineto{\pgfqpoint{3.748903in}{0.413320in}}%
\pgfpathlineto{\pgfqpoint{3.746229in}{0.413320in}}%
\pgfpathlineto{\pgfqpoint{3.743548in}{0.413320in}}%
\pgfpathlineto{\pgfqpoint{3.740874in}{0.413320in}}%
\pgfpathlineto{\pgfqpoint{3.738194in}{0.413320in}}%
\pgfpathlineto{\pgfqpoint{3.735509in}{0.413320in}}%
\pgfpathlineto{\pgfqpoint{3.732950in}{0.413320in}}%
\pgfpathlineto{\pgfqpoint{3.730158in}{0.413320in}}%
\pgfpathlineto{\pgfqpoint{3.727581in}{0.413320in}}%
\pgfpathlineto{\pgfqpoint{3.724804in}{0.413320in}}%
\pgfpathlineto{\pgfqpoint{3.722228in}{0.413320in}}%
\pgfpathlineto{\pgfqpoint{3.719446in}{0.413320in}}%
\pgfpathlineto{\pgfqpoint{3.716875in}{0.413320in}}%
\pgfpathlineto{\pgfqpoint{3.714086in}{0.413320in}}%
\pgfpathlineto{\pgfqpoint{3.711410in}{0.413320in}}%
\pgfpathlineto{\pgfqpoint{3.708729in}{0.413320in}}%
\pgfpathlineto{\pgfqpoint{3.706053in}{0.413320in}}%
\pgfpathlineto{\pgfqpoint{3.703460in}{0.413320in}}%
\pgfpathlineto{\pgfqpoint{3.700684in}{0.413320in}}%
\pgfpathlineto{\pgfqpoint{3.698125in}{0.413320in}}%
\pgfpathlineto{\pgfqpoint{3.695331in}{0.413320in}}%
\pgfpathlineto{\pgfqpoint{3.692765in}{0.413320in}}%
\pgfpathlineto{\pgfqpoint{3.689983in}{0.413320in}}%
\pgfpathlineto{\pgfqpoint{3.687442in}{0.413320in}}%
\pgfpathlineto{\pgfqpoint{3.684620in}{0.413320in}}%
\pgfpathlineto{\pgfqpoint{3.681948in}{0.413320in}}%
\pgfpathlineto{\pgfqpoint{3.679273in}{0.413320in}}%
\pgfpathlineto{\pgfqpoint{3.676591in}{0.413320in}}%
\pgfpathlineto{\pgfqpoint{3.673911in}{0.413320in}}%
\pgfpathlineto{\pgfqpoint{3.671232in}{0.413320in}}%
\pgfpathlineto{\pgfqpoint{3.668665in}{0.413320in}}%
\pgfpathlineto{\pgfqpoint{3.665864in}{0.413320in}}%
\pgfpathlineto{\pgfqpoint{3.663276in}{0.413320in}}%
\pgfpathlineto{\pgfqpoint{3.660515in}{0.413320in}}%
\pgfpathlineto{\pgfqpoint{3.657917in}{0.413320in}}%
\pgfpathlineto{\pgfqpoint{3.655165in}{0.413320in}}%
\pgfpathlineto{\pgfqpoint{3.652628in}{0.413320in}}%
\pgfpathlineto{\pgfqpoint{3.649837in}{0.413320in}}%
\pgfpathlineto{\pgfqpoint{3.647130in}{0.413320in}}%
\pgfpathlineto{\pgfqpoint{3.644452in}{0.413320in}}%
\pgfpathlineto{\pgfqpoint{3.641773in}{0.413320in}}%
\pgfpathlineto{\pgfqpoint{3.639207in}{0.413320in}}%
\pgfpathlineto{\pgfqpoint{3.636413in}{0.413320in}}%
\pgfpathlineto{\pgfqpoint{3.633858in}{0.413320in}}%
\pgfpathlineto{\pgfqpoint{3.631058in}{0.413320in}}%
\pgfpathlineto{\pgfqpoint{3.628460in}{0.413320in}}%
\pgfpathlineto{\pgfqpoint{3.625689in}{0.413320in}}%
\pgfpathlineto{\pgfqpoint{3.623165in}{0.413320in}}%
\pgfpathlineto{\pgfqpoint{3.620345in}{0.413320in}}%
\pgfpathlineto{\pgfqpoint{3.617667in}{0.413320in}}%
\pgfpathlineto{\pgfqpoint{3.614982in}{0.413320in}}%
\pgfpathlineto{\pgfqpoint{3.612311in}{0.413320in}}%
\pgfpathlineto{\pgfqpoint{3.609632in}{0.413320in}}%
\pgfpathlineto{\pgfqpoint{3.606951in}{0.413320in}}%
\pgfpathlineto{\pgfqpoint{3.604387in}{0.413320in}}%
\pgfpathlineto{\pgfqpoint{3.601590in}{0.413320in}}%
\pgfpathlineto{\pgfqpoint{3.598998in}{0.413320in}}%
\pgfpathlineto{\pgfqpoint{3.596240in}{0.413320in}}%
\pgfpathlineto{\pgfqpoint{3.593620in}{0.413320in}}%
\pgfpathlineto{\pgfqpoint{3.590883in}{0.413320in}}%
\pgfpathlineto{\pgfqpoint{3.588258in}{0.413320in}}%
\pgfpathlineto{\pgfqpoint{3.585532in}{0.413320in}}%
\pgfpathlineto{\pgfqpoint{3.582851in}{0.413320in}}%
\pgfpathlineto{\pgfqpoint{3.580191in}{0.413320in}}%
\pgfpathlineto{\pgfqpoint{3.577487in}{0.413320in}}%
\pgfpathlineto{\pgfqpoint{3.574814in}{0.413320in}}%
\pgfpathlineto{\pgfqpoint{3.572126in}{0.413320in}}%
\pgfpathlineto{\pgfqpoint{3.569584in}{0.413320in}}%
\pgfpathlineto{\pgfqpoint{3.566774in}{0.413320in}}%
\pgfpathlineto{\pgfqpoint{3.564188in}{0.413320in}}%
\pgfpathlineto{\pgfqpoint{3.561420in}{0.413320in}}%
\pgfpathlineto{\pgfqpoint{3.558853in}{0.413320in}}%
\pgfpathlineto{\pgfqpoint{3.556061in}{0.413320in}}%
\pgfpathlineto{\pgfqpoint{3.553498in}{0.413320in}}%
\pgfpathlineto{\pgfqpoint{3.550713in}{0.413320in}}%
\pgfpathlineto{\pgfqpoint{3.548029in}{0.413320in}}%
\pgfpathlineto{\pgfqpoint{3.545349in}{0.413320in}}%
\pgfpathlineto{\pgfqpoint{3.542656in}{0.413320in}}%
\pgfpathlineto{\pgfqpoint{3.540093in}{0.413320in}}%
\pgfpathlineto{\pgfqpoint{3.537309in}{0.413320in}}%
\pgfpathlineto{\pgfqpoint{3.534783in}{0.413320in}}%
\pgfpathlineto{\pgfqpoint{3.531955in}{0.413320in}}%
\pgfpathlineto{\pgfqpoint{3.529327in}{0.413320in}}%
\pgfpathlineto{\pgfqpoint{3.526601in}{0.413320in}}%
\pgfpathlineto{\pgfqpoint{3.524041in}{0.413320in}}%
\pgfpathlineto{\pgfqpoint{3.521244in}{0.413320in}}%
\pgfpathlineto{\pgfqpoint{3.518565in}{0.413320in}}%
\pgfpathlineto{\pgfqpoint{3.515884in}{0.413320in}}%
\pgfpathlineto{\pgfqpoint{3.513209in}{0.413320in}}%
\pgfpathlineto{\pgfqpoint{3.510533in}{0.413320in}}%
\pgfpathlineto{\pgfqpoint{3.507840in}{0.413320in}}%
\pgfpathlineto{\pgfqpoint{3.505262in}{0.413320in}}%
\pgfpathlineto{\pgfqpoint{3.502488in}{0.413320in}}%
\pgfpathlineto{\pgfqpoint{3.499909in}{0.413320in}}%
\pgfpathlineto{\pgfqpoint{3.497139in}{0.413320in}}%
\pgfpathlineto{\pgfqpoint{3.494581in}{0.413320in}}%
\pgfpathlineto{\pgfqpoint{3.491783in}{0.413320in}}%
\pgfpathlineto{\pgfqpoint{3.489223in}{0.413320in}}%
\pgfpathlineto{\pgfqpoint{3.486442in}{0.413320in}}%
\pgfpathlineto{\pgfqpoint{3.483744in}{0.413320in}}%
\pgfpathlineto{\pgfqpoint{3.481072in}{0.413320in}}%
\pgfpathlineto{\pgfqpoint{3.478378in}{0.413320in}}%
\pgfpathlineto{\pgfqpoint{3.475821in}{0.413320in}}%
\pgfpathlineto{\pgfqpoint{3.473021in}{0.413320in}}%
\pgfpathlineto{\pgfqpoint{3.470466in}{0.413320in}}%
\pgfpathlineto{\pgfqpoint{3.467678in}{0.413320in}}%
\pgfpathlineto{\pgfqpoint{3.465072in}{0.413320in}}%
\pgfpathlineto{\pgfqpoint{3.462321in}{0.413320in}}%
\pgfpathlineto{\pgfqpoint{3.459695in}{0.413320in}}%
\pgfpathlineto{\pgfqpoint{3.456960in}{0.413320in}}%
\pgfpathlineto{\pgfqpoint{3.454285in}{0.413320in}}%
\pgfpathlineto{\pgfqpoint{3.451597in}{0.413320in}}%
\pgfpathlineto{\pgfqpoint{3.448926in}{0.413320in}}%
\pgfpathlineto{\pgfqpoint{3.446257in}{0.413320in}}%
\pgfpathlineto{\pgfqpoint{3.443574in}{0.413320in}}%
\pgfpathlineto{\pgfqpoint{3.440996in}{0.413320in}}%
\pgfpathlineto{\pgfqpoint{3.438210in}{0.413320in}}%
\pgfpathlineto{\pgfqpoint{3.435635in}{0.413320in}}%
\pgfpathlineto{\pgfqpoint{3.432851in}{0.413320in}}%
\pgfpathlineto{\pgfqpoint{3.430313in}{0.413320in}}%
\pgfpathlineto{\pgfqpoint{3.427501in}{0.413320in}}%
\pgfpathlineto{\pgfqpoint{3.424887in}{0.413320in}}%
\pgfpathlineto{\pgfqpoint{3.422142in}{0.413320in}}%
\pgfpathlineto{\pgfqpoint{3.419455in}{0.413320in}}%
\pgfpathlineto{\pgfqpoint{3.416780in}{0.413320in}}%
\pgfpathlineto{\pgfqpoint{3.414109in}{0.413320in}}%
\pgfpathlineto{\pgfqpoint{3.411431in}{0.413320in}}%
\pgfpathlineto{\pgfqpoint{3.408752in}{0.413320in}}%
\pgfpathlineto{\pgfqpoint{3.406202in}{0.413320in}}%
\pgfpathlineto{\pgfqpoint{3.403394in}{0.413320in}}%
\pgfpathlineto{\pgfqpoint{3.400783in}{0.413320in}}%
\pgfpathlineto{\pgfqpoint{3.398037in}{0.413320in}}%
\pgfpathlineto{\pgfqpoint{3.395461in}{0.413320in}}%
\pgfpathlineto{\pgfqpoint{3.392681in}{0.413320in}}%
\pgfpathlineto{\pgfqpoint{3.390102in}{0.413320in}}%
\pgfpathlineto{\pgfqpoint{3.387309in}{0.413320in}}%
\pgfpathlineto{\pgfqpoint{3.384647in}{0.413320in}}%
\pgfpathlineto{\pgfqpoint{3.381959in}{0.413320in}}%
\pgfpathlineto{\pgfqpoint{3.379290in}{0.413320in}}%
\pgfpathlineto{\pgfqpoint{3.376735in}{0.413320in}}%
\pgfpathlineto{\pgfqpoint{3.373921in}{0.413320in}}%
\pgfpathlineto{\pgfqpoint{3.371357in}{0.413320in}}%
\pgfpathlineto{\pgfqpoint{3.368577in}{0.413320in}}%
\pgfpathlineto{\pgfqpoint{3.365996in}{0.413320in}}%
\pgfpathlineto{\pgfqpoint{3.363221in}{0.413320in}}%
\pgfpathlineto{\pgfqpoint{3.360620in}{0.413320in}}%
\pgfpathlineto{\pgfqpoint{3.357862in}{0.413320in}}%
\pgfpathlineto{\pgfqpoint{3.355177in}{0.413320in}}%
\pgfpathlineto{\pgfqpoint{3.352505in}{0.413320in}}%
\pgfpathlineto{\pgfqpoint{3.349828in}{0.413320in}}%
\pgfpathlineto{\pgfqpoint{3.347139in}{0.413320in}}%
\pgfpathlineto{\pgfqpoint{3.344468in}{0.413320in}}%
\pgfpathlineto{\pgfqpoint{3.341893in}{0.413320in}}%
\pgfpathlineto{\pgfqpoint{3.339101in}{0.413320in}}%
\pgfpathlineto{\pgfqpoint{3.336541in}{0.413320in}}%
\pgfpathlineto{\pgfqpoint{3.333758in}{0.413320in}}%
\pgfpathlineto{\pgfqpoint{3.331183in}{0.413320in}}%
\pgfpathlineto{\pgfqpoint{3.328401in}{0.413320in}}%
\pgfpathlineto{\pgfqpoint{3.325860in}{0.413320in}}%
\pgfpathlineto{\pgfqpoint{3.323049in}{0.413320in}}%
\pgfpathlineto{\pgfqpoint{3.320366in}{0.413320in}}%
\pgfpathlineto{\pgfqpoint{3.317688in}{0.413320in}}%
\pgfpathlineto{\pgfqpoint{3.315008in}{0.413320in}}%
\pgfpathlineto{\pgfqpoint{3.312480in}{0.413320in}}%
\pgfpathlineto{\pgfqpoint{3.309652in}{0.413320in}}%
\pgfpathlineto{\pgfqpoint{3.307104in}{0.413320in}}%
\pgfpathlineto{\pgfqpoint{3.304295in}{0.413320in}}%
\pgfpathlineto{\pgfqpoint{3.301719in}{0.413320in}}%
\pgfpathlineto{\pgfqpoint{3.298937in}{0.413320in}}%
\pgfpathlineto{\pgfqpoint{3.296376in}{0.413320in}}%
\pgfpathlineto{\pgfqpoint{3.293574in}{0.413320in}}%
\pgfpathlineto{\pgfqpoint{3.290890in}{0.413320in}}%
\pgfpathlineto{\pgfqpoint{3.288225in}{0.413320in}}%
\pgfpathlineto{\pgfqpoint{3.285534in}{0.413320in}}%
\pgfpathlineto{\pgfqpoint{3.282870in}{0.413320in}}%
\pgfpathlineto{\pgfqpoint{3.280189in}{0.413320in}}%
\pgfpathlineto{\pgfqpoint{3.277603in}{0.413320in}}%
\pgfpathlineto{\pgfqpoint{3.274831in}{0.413320in}}%
\pgfpathlineto{\pgfqpoint{3.272254in}{0.413320in}}%
\pgfpathlineto{\pgfqpoint{3.269478in}{0.413320in}}%
\pgfpathlineto{\pgfqpoint{3.266849in}{0.413320in}}%
\pgfpathlineto{\pgfqpoint{3.264119in}{0.413320in}}%
\pgfpathlineto{\pgfqpoint{3.261594in}{0.413320in}}%
\pgfpathlineto{\pgfqpoint{3.258784in}{0.413320in}}%
\pgfpathlineto{\pgfqpoint{3.256083in}{0.413320in}}%
\pgfpathlineto{\pgfqpoint{3.253404in}{0.413320in}}%
\pgfpathlineto{\pgfqpoint{3.250716in}{0.413320in}}%
\pgfpathlineto{\pgfqpoint{3.248049in}{0.413320in}}%
\pgfpathlineto{\pgfqpoint{3.245363in}{0.413320in}}%
\pgfpathlineto{\pgfqpoint{3.242807in}{0.413320in}}%
\pgfpathlineto{\pgfqpoint{3.240010in}{0.413320in}}%
\pgfpathlineto{\pgfqpoint{3.237411in}{0.413320in}}%
\pgfpathlineto{\pgfqpoint{3.234658in}{0.413320in}}%
\pgfpathlineto{\pgfqpoint{3.232069in}{0.413320in}}%
\pgfpathlineto{\pgfqpoint{3.229310in}{0.413320in}}%
\pgfpathlineto{\pgfqpoint{3.226609in}{0.413320in}}%
\pgfpathlineto{\pgfqpoint{3.223942in}{0.413320in}}%
\pgfpathlineto{\pgfqpoint{3.221255in}{0.413320in}}%
\pgfpathlineto{\pgfqpoint{3.218586in}{0.413320in}}%
\pgfpathlineto{\pgfqpoint{3.215908in}{0.413320in}}%
\pgfpathlineto{\pgfqpoint{3.213342in}{0.413320in}}%
\pgfpathlineto{\pgfqpoint{3.210545in}{0.413320in}}%
\pgfpathlineto{\pgfqpoint{3.207984in}{0.413320in}}%
\pgfpathlineto{\pgfqpoint{3.205195in}{0.413320in}}%
\pgfpathlineto{\pgfqpoint{3.202562in}{0.413320in}}%
\pgfpathlineto{\pgfqpoint{3.199823in}{0.413320in}}%
\pgfpathlineto{\pgfqpoint{3.197226in}{0.413320in}}%
\pgfpathlineto{\pgfqpoint{3.194508in}{0.413320in}}%
\pgfpathlineto{\pgfqpoint{3.191796in}{0.413320in}}%
\pgfpathlineto{\pgfqpoint{3.189117in}{0.413320in}}%
\pgfpathlineto{\pgfqpoint{3.186440in}{0.413320in}}%
\pgfpathlineto{\pgfqpoint{3.183760in}{0.413320in}}%
\pgfpathlineto{\pgfqpoint{3.181089in}{0.413320in}}%
\pgfpathlineto{\pgfqpoint{3.178525in}{0.413320in}}%
\pgfpathlineto{\pgfqpoint{3.175724in}{0.413320in}}%
\pgfpathlineto{\pgfqpoint{3.173142in}{0.413320in}}%
\pgfpathlineto{\pgfqpoint{3.170375in}{0.413320in}}%
\pgfpathlineto{\pgfqpoint{3.167776in}{0.413320in}}%
\pgfpathlineto{\pgfqpoint{3.165019in}{0.413320in}}%
\pgfpathlineto{\pgfqpoint{3.162474in}{0.413320in}}%
\pgfpathlineto{\pgfqpoint{3.159675in}{0.413320in}}%
\pgfpathlineto{\pgfqpoint{3.156981in}{0.413320in}}%
\pgfpathlineto{\pgfqpoint{3.154327in}{0.413320in}}%
\pgfpathlineto{\pgfqpoint{3.151612in}{0.413320in}}%
\pgfpathlineto{\pgfqpoint{3.149057in}{0.413320in}}%
\pgfpathlineto{\pgfqpoint{3.146271in}{0.413320in}}%
\pgfpathlineto{\pgfqpoint{3.143740in}{0.413320in}}%
\pgfpathlineto{\pgfqpoint{3.140913in}{0.413320in}}%
\pgfpathlineto{\pgfqpoint{3.138375in}{0.413320in}}%
\pgfpathlineto{\pgfqpoint{3.135550in}{0.413320in}}%
\pgfpathlineto{\pgfqpoint{3.132946in}{0.413320in}}%
\pgfpathlineto{\pgfqpoint{3.130199in}{0.413320in}}%
\pgfpathlineto{\pgfqpoint{3.127512in}{0.413320in}}%
\pgfpathlineto{\pgfqpoint{3.124842in}{0.413320in}}%
\pgfpathlineto{\pgfqpoint{3.122163in}{0.413320in}}%
\pgfpathlineto{\pgfqpoint{3.119487in}{0.413320in}}%
\pgfpathlineto{\pgfqpoint{3.116807in}{0.413320in}}%
\pgfpathlineto{\pgfqpoint{3.114242in}{0.413320in}}%
\pgfpathlineto{\pgfqpoint{3.111451in}{0.413320in}}%
\pgfpathlineto{\pgfqpoint{3.108896in}{0.413320in}}%
\pgfpathlineto{\pgfqpoint{3.106094in}{0.413320in}}%
\pgfpathlineto{\pgfqpoint{3.103508in}{0.413320in}}%
\pgfpathlineto{\pgfqpoint{3.100737in}{0.413320in}}%
\pgfpathlineto{\pgfqpoint{3.098163in}{0.413320in}}%
\pgfpathlineto{\pgfqpoint{3.095388in}{0.413320in}}%
\pgfpathlineto{\pgfqpoint{3.092699in}{0.413320in}}%
\pgfpathlineto{\pgfqpoint{3.090023in}{0.413320in}}%
\pgfpathlineto{\pgfqpoint{3.087343in}{0.413320in}}%
\pgfpathlineto{\pgfqpoint{3.084671in}{0.413320in}}%
\pgfpathlineto{\pgfqpoint{3.081990in}{0.413320in}}%
\pgfpathlineto{\pgfqpoint{3.079381in}{0.413320in}}%
\pgfpathlineto{\pgfqpoint{3.076631in}{0.413320in}}%
\pgfpathlineto{\pgfqpoint{3.074056in}{0.413320in}}%
\pgfpathlineto{\pgfqpoint{3.071266in}{0.413320in}}%
\pgfpathlineto{\pgfqpoint{3.068709in}{0.413320in}}%
\pgfpathlineto{\pgfqpoint{3.065916in}{0.413320in}}%
\pgfpathlineto{\pgfqpoint{3.063230in}{0.413320in}}%
\pgfpathlineto{\pgfqpoint{3.060561in}{0.413320in}}%
\pgfpathlineto{\pgfqpoint{3.057884in}{0.413320in}}%
\pgfpathlineto{\pgfqpoint{3.055202in}{0.413320in}}%
\pgfpathlineto{\pgfqpoint{3.052526in}{0.413320in}}%
\pgfpathlineto{\pgfqpoint{3.049988in}{0.413320in}}%
\pgfpathlineto{\pgfqpoint{3.047157in}{0.413320in}}%
\pgfpathlineto{\pgfqpoint{3.044568in}{0.413320in}}%
\pgfpathlineto{\pgfqpoint{3.041813in}{0.413320in}}%
\pgfpathlineto{\pgfqpoint{3.039262in}{0.413320in}}%
\pgfpathlineto{\pgfqpoint{3.036456in}{0.413320in}}%
\pgfpathlineto{\pgfqpoint{3.033921in}{0.413320in}}%
\pgfpathlineto{\pgfqpoint{3.031091in}{0.413320in}}%
\pgfpathlineto{\pgfqpoint{3.028412in}{0.413320in}}%
\pgfpathlineto{\pgfqpoint{3.025803in}{0.413320in}}%
\pgfpathlineto{\pgfqpoint{3.023058in}{0.413320in}}%
\pgfpathlineto{\pgfqpoint{3.020382in}{0.413320in}}%
\pgfpathlineto{\pgfqpoint{3.017707in}{0.413320in}}%
\pgfpathlineto{\pgfqpoint{3.015097in}{0.413320in}}%
\pgfpathlineto{\pgfqpoint{3.012351in}{0.413320in}}%
\pgfpathlineto{\pgfqpoint{3.009784in}{0.413320in}}%
\pgfpathlineto{\pgfqpoint{3.006993in}{0.413320in}}%
\pgfpathlineto{\pgfqpoint{3.004419in}{0.413320in}}%
\pgfpathlineto{\pgfqpoint{3.001635in}{0.413320in}}%
\pgfpathlineto{\pgfqpoint{2.999103in}{0.413320in}}%
\pgfpathlineto{\pgfqpoint{2.996300in}{0.413320in}}%
\pgfpathlineto{\pgfqpoint{2.993595in}{0.413320in}}%
\pgfpathlineto{\pgfqpoint{2.990978in}{0.413320in}}%
\pgfpathlineto{\pgfqpoint{2.988238in}{0.413320in}}%
\pgfpathlineto{\pgfqpoint{2.985666in}{0.413320in}}%
\pgfpathlineto{\pgfqpoint{2.982885in}{0.413320in}}%
\pgfpathlineto{\pgfqpoint{2.980341in}{0.413320in}}%
\pgfpathlineto{\pgfqpoint{2.977517in}{0.413320in}}%
\pgfpathlineto{\pgfqpoint{2.974972in}{0.413320in}}%
\pgfpathlineto{\pgfqpoint{2.972177in}{0.413320in}}%
\pgfpathlineto{\pgfqpoint{2.969599in}{0.413320in}}%
\pgfpathlineto{\pgfqpoint{2.966812in}{0.413320in}}%
\pgfpathlineto{\pgfqpoint{2.964127in}{0.413320in}}%
\pgfpathlineto{\pgfqpoint{2.961460in}{0.413320in}}%
\pgfpathlineto{\pgfqpoint{2.958782in}{0.413320in}}%
\pgfpathlineto{\pgfqpoint{2.956103in}{0.413320in}}%
\pgfpathlineto{\pgfqpoint{2.953422in}{0.413320in}}%
\pgfpathlineto{\pgfqpoint{2.950884in}{0.413320in}}%
\pgfpathlineto{\pgfqpoint{2.948068in}{0.413320in}}%
\pgfpathlineto{\pgfqpoint{2.945461in}{0.413320in}}%
\pgfpathlineto{\pgfqpoint{2.942711in}{0.413320in}}%
\pgfpathlineto{\pgfqpoint{2.940120in}{0.413320in}}%
\pgfpathlineto{\pgfqpoint{2.937352in}{0.413320in}}%
\pgfpathlineto{\pgfqpoint{2.934759in}{0.413320in}}%
\pgfpathlineto{\pgfqpoint{2.932033in}{0.413320in}}%
\pgfpathlineto{\pgfqpoint{2.929321in}{0.413320in}}%
\pgfpathlineto{\pgfqpoint{2.926655in}{0.413320in}}%
\pgfpathlineto{\pgfqpoint{2.923963in}{0.413320in}}%
\pgfpathlineto{\pgfqpoint{2.921363in}{0.413320in}}%
\pgfpathlineto{\pgfqpoint{2.918606in}{0.413320in}}%
\pgfpathlineto{\pgfqpoint{2.916061in}{0.413320in}}%
\pgfpathlineto{\pgfqpoint{2.913243in}{0.413320in}}%
\pgfpathlineto{\pgfqpoint{2.910631in}{0.413320in}}%
\pgfpathlineto{\pgfqpoint{2.907882in}{0.413320in}}%
\pgfpathlineto{\pgfqpoint{2.905341in}{0.413320in}}%
\pgfpathlineto{\pgfqpoint{2.902535in}{0.413320in}}%
\pgfpathlineto{\pgfqpoint{2.899858in}{0.413320in}}%
\pgfpathlineto{\pgfqpoint{2.897179in}{0.413320in}}%
\pgfpathlineto{\pgfqpoint{2.894487in}{0.413320in}}%
\pgfpathlineto{\pgfqpoint{2.891809in}{0.413320in}}%
\pgfpathlineto{\pgfqpoint{2.889145in}{0.413320in}}%
\pgfpathlineto{\pgfqpoint{2.886578in}{0.413320in}}%
\pgfpathlineto{\pgfqpoint{2.883780in}{0.413320in}}%
\pgfpathlineto{\pgfqpoint{2.881254in}{0.413320in}}%
\pgfpathlineto{\pgfqpoint{2.878431in}{0.413320in}}%
\pgfpathlineto{\pgfqpoint{2.875882in}{0.413320in}}%
\pgfpathlineto{\pgfqpoint{2.873074in}{0.413320in}}%
\pgfpathlineto{\pgfqpoint{2.870475in}{0.413320in}}%
\pgfpathlineto{\pgfqpoint{2.867713in}{0.413320in}}%
\pgfpathlineto{\pgfqpoint{2.865031in}{0.413320in}}%
\pgfpathlineto{\pgfqpoint{2.862402in}{0.413320in}}%
\pgfpathlineto{\pgfqpoint{2.859668in}{0.413320in}}%
\pgfpathlineto{\pgfqpoint{2.857003in}{0.413320in}}%
\pgfpathlineto{\pgfqpoint{2.854325in}{0.413320in}}%
\pgfpathlineto{\pgfqpoint{2.851793in}{0.413320in}}%
\pgfpathlineto{\pgfqpoint{2.848960in}{0.413320in}}%
\pgfpathlineto{\pgfqpoint{2.846408in}{0.413320in}}%
\pgfpathlineto{\pgfqpoint{2.843611in}{0.413320in}}%
\pgfpathlineto{\pgfqpoint{2.841055in}{0.413320in}}%
\pgfpathlineto{\pgfqpoint{2.838254in}{0.413320in}}%
\pgfpathlineto{\pgfqpoint{2.835698in}{0.413320in}}%
\pgfpathlineto{\pgfqpoint{2.832894in}{0.413320in}}%
\pgfpathlineto{\pgfqpoint{2.830219in}{0.413320in}}%
\pgfpathlineto{\pgfqpoint{2.827567in}{0.413320in}}%
\pgfpathlineto{\pgfqpoint{2.824851in}{0.413320in}}%
\pgfpathlineto{\pgfqpoint{2.822303in}{0.413320in}}%
\pgfpathlineto{\pgfqpoint{2.819506in}{0.413320in}}%
\pgfpathlineto{\pgfqpoint{2.816867in}{0.413320in}}%
\pgfpathlineto{\pgfqpoint{2.814141in}{0.413320in}}%
\pgfpathlineto{\pgfqpoint{2.811597in}{0.413320in}}%
\pgfpathlineto{\pgfqpoint{2.808792in}{0.413320in}}%
\pgfpathlineto{\pgfqpoint{2.806175in}{0.413320in}}%
\pgfpathlineto{\pgfqpoint{2.803435in}{0.413320in}}%
\pgfpathlineto{\pgfqpoint{2.800756in}{0.413320in}}%
\pgfpathlineto{\pgfqpoint{2.798070in}{0.413320in}}%
\pgfpathlineto{\pgfqpoint{2.795398in}{0.413320in}}%
\pgfpathlineto{\pgfqpoint{2.792721in}{0.413320in}}%
\pgfpathlineto{\pgfqpoint{2.790044in}{0.413320in}}%
\pgfpathlineto{\pgfqpoint{2.787468in}{0.413320in}}%
\pgfpathlineto{\pgfqpoint{2.784687in}{0.413320in}}%
\pgfpathlineto{\pgfqpoint{2.782113in}{0.413320in}}%
\pgfpathlineto{\pgfqpoint{2.779330in}{0.413320in}}%
\pgfpathlineto{\pgfqpoint{2.776767in}{0.413320in}}%
\pgfpathlineto{\pgfqpoint{2.773972in}{0.413320in}}%
\pgfpathlineto{\pgfqpoint{2.771373in}{0.413320in}}%
\pgfpathlineto{\pgfqpoint{2.768617in}{0.413320in}}%
\pgfpathlineto{\pgfqpoint{2.765935in}{0.413320in}}%
\pgfpathlineto{\pgfqpoint{2.763253in}{0.413320in}}%
\pgfpathlineto{\pgfqpoint{2.760581in}{0.413320in}}%
\pgfpathlineto{\pgfqpoint{2.758028in}{0.413320in}}%
\pgfpathlineto{\pgfqpoint{2.755224in}{0.413320in}}%
\pgfpathlineto{\pgfqpoint{2.752614in}{0.413320in}}%
\pgfpathlineto{\pgfqpoint{2.749868in}{0.413320in}}%
\pgfpathlineto{\pgfqpoint{2.747260in}{0.413320in}}%
\pgfpathlineto{\pgfqpoint{2.744510in}{0.413320in}}%
\pgfpathlineto{\pgfqpoint{2.741928in}{0.413320in}}%
\pgfpathlineto{\pgfqpoint{2.739155in}{0.413320in}}%
\pgfpathlineto{\pgfqpoint{2.736476in}{0.413320in}}%
\pgfpathlineto{\pgfqpoint{2.733798in}{0.413320in}}%
\pgfpathlineto{\pgfqpoint{2.731119in}{0.413320in}}%
\pgfpathlineto{\pgfqpoint{2.728439in}{0.413320in}}%
\pgfpathlineto{\pgfqpoint{2.725760in}{0.413320in}}%
\pgfpathlineto{\pgfqpoint{2.723211in}{0.413320in}}%
\pgfpathlineto{\pgfqpoint{2.720404in}{0.413320in}}%
\pgfpathlineto{\pgfqpoint{2.717773in}{0.413320in}}%
\pgfpathlineto{\pgfqpoint{2.715036in}{0.413320in}}%
\pgfpathlineto{\pgfqpoint{2.712477in}{0.413320in}}%
\pgfpathlineto{\pgfqpoint{2.709683in}{0.413320in}}%
\pgfpathlineto{\pgfqpoint{2.707125in}{0.413320in}}%
\pgfpathlineto{\pgfqpoint{2.704326in}{0.413320in}}%
\pgfpathlineto{\pgfqpoint{2.701657in}{0.413320in}}%
\pgfpathlineto{\pgfqpoint{2.698968in}{0.413320in}}%
\pgfpathlineto{\pgfqpoint{2.696293in}{0.413320in}}%
\pgfpathlineto{\pgfqpoint{2.693611in}{0.413320in}}%
\pgfpathlineto{\pgfqpoint{2.690940in}{0.413320in}}%
\pgfpathlineto{\pgfqpoint{2.688328in}{0.413320in}}%
\pgfpathlineto{\pgfqpoint{2.685586in}{0.413320in}}%
\pgfpathlineto{\pgfqpoint{2.683009in}{0.413320in}}%
\pgfpathlineto{\pgfqpoint{2.680224in}{0.413320in}}%
\pgfpathlineto{\pgfqpoint{2.677650in}{0.413320in}}%
\pgfpathlineto{\pgfqpoint{2.674873in}{0.413320in}}%
\pgfpathlineto{\pgfqpoint{2.672301in}{0.413320in}}%
\pgfpathlineto{\pgfqpoint{2.669506in}{0.413320in}}%
\pgfpathlineto{\pgfqpoint{2.666836in}{0.413320in}}%
\pgfpathlineto{\pgfqpoint{2.664151in}{0.413320in}}%
\pgfpathlineto{\pgfqpoint{2.661481in}{0.413320in}}%
\pgfpathlineto{\pgfqpoint{2.658942in}{0.413320in}}%
\pgfpathlineto{\pgfqpoint{2.656124in}{0.413320in}}%
\pgfpathlineto{\pgfqpoint{2.653567in}{0.413320in}}%
\pgfpathlineto{\pgfqpoint{2.650767in}{0.413320in}}%
\pgfpathlineto{\pgfqpoint{2.648196in}{0.413320in}}%
\pgfpathlineto{\pgfqpoint{2.645408in}{0.413320in}}%
\pgfpathlineto{\pgfqpoint{2.642827in}{0.413320in}}%
\pgfpathlineto{\pgfqpoint{2.640053in}{0.413320in}}%
\pgfpathlineto{\pgfqpoint{2.637369in}{0.413320in}}%
\pgfpathlineto{\pgfqpoint{2.634700in}{0.413320in}}%
\pgfpathlineto{\pgfqpoint{2.632018in}{0.413320in}}%
\pgfpathlineto{\pgfqpoint{2.629340in}{0.413320in}}%
\pgfpathlineto{\pgfqpoint{2.626653in}{0.413320in}}%
\pgfpathlineto{\pgfqpoint{2.624077in}{0.413320in}}%
\pgfpathlineto{\pgfqpoint{2.621304in}{0.413320in}}%
\pgfpathlineto{\pgfqpoint{2.618773in}{0.413320in}}%
\pgfpathlineto{\pgfqpoint{2.615934in}{0.413320in}}%
\pgfpathlineto{\pgfqpoint{2.613393in}{0.413320in}}%
\pgfpathlineto{\pgfqpoint{2.610588in}{0.413320in}}%
\pgfpathlineto{\pgfqpoint{2.608004in}{0.413320in}}%
\pgfpathlineto{\pgfqpoint{2.605232in}{0.413320in}}%
\pgfpathlineto{\pgfqpoint{2.602557in}{0.413320in}}%
\pgfpathlineto{\pgfqpoint{2.599920in}{0.413320in}}%
\pgfpathlineto{\pgfqpoint{2.597196in}{0.413320in}}%
\pgfpathlineto{\pgfqpoint{2.594630in}{0.413320in}}%
\pgfpathlineto{\pgfqpoint{2.591842in}{0.413320in}}%
\pgfpathlineto{\pgfqpoint{2.589248in}{0.413320in}}%
\pgfpathlineto{\pgfqpoint{2.586484in}{0.413320in}}%
\pgfpathlineto{\pgfqpoint{2.583913in}{0.413320in}}%
\pgfpathlineto{\pgfqpoint{2.581129in}{0.413320in}}%
\pgfpathlineto{\pgfqpoint{2.578567in}{0.413320in}}%
\pgfpathlineto{\pgfqpoint{2.575779in}{0.413320in}}%
\pgfpathlineto{\pgfqpoint{2.573082in}{0.413320in}}%
\pgfpathlineto{\pgfqpoint{2.570411in}{0.413320in}}%
\pgfpathlineto{\pgfqpoint{2.567730in}{0.413320in}}%
\pgfpathlineto{\pgfqpoint{2.565045in}{0.413320in}}%
\pgfpathlineto{\pgfqpoint{2.562375in}{0.413320in}}%
\pgfpathlineto{\pgfqpoint{2.559790in}{0.413320in}}%
\pgfpathlineto{\pgfqpoint{2.557009in}{0.413320in}}%
\pgfpathlineto{\pgfqpoint{2.554493in}{0.413320in}}%
\pgfpathlineto{\pgfqpoint{2.551664in}{0.413320in}}%
\pgfpathlineto{\pgfqpoint{2.549114in}{0.413320in}}%
\pgfpathlineto{\pgfqpoint{2.546310in}{0.413320in}}%
\pgfpathlineto{\pgfqpoint{2.543765in}{0.413320in}}%
\pgfpathlineto{\pgfqpoint{2.540949in}{0.413320in}}%
\pgfpathlineto{\pgfqpoint{2.538274in}{0.413320in}}%
\pgfpathlineto{\pgfqpoint{2.535624in}{0.413320in}}%
\pgfpathlineto{\pgfqpoint{2.532917in}{0.413320in}}%
\pgfpathlineto{\pgfqpoint{2.530234in}{0.413320in}}%
\pgfpathlineto{\pgfqpoint{2.527560in}{0.413320in}}%
\pgfpathlineto{\pgfqpoint{2.524988in}{0.413320in}}%
\pgfpathlineto{\pgfqpoint{2.522197in}{0.413320in}}%
\pgfpathlineto{\pgfqpoint{2.519607in}{0.413320in}}%
\pgfpathlineto{\pgfqpoint{2.516845in}{0.413320in}}%
\pgfpathlineto{\pgfqpoint{2.514268in}{0.413320in}}%
\pgfpathlineto{\pgfqpoint{2.511478in}{0.413320in}}%
\pgfpathlineto{\pgfqpoint{2.508917in}{0.413320in}}%
\pgfpathlineto{\pgfqpoint{2.506163in}{0.413320in}}%
\pgfpathlineto{\pgfqpoint{2.503454in}{0.413320in}}%
\pgfpathlineto{\pgfqpoint{2.500801in}{0.413320in}}%
\pgfpathlineto{\pgfqpoint{2.498085in}{0.413320in}}%
\pgfpathlineto{\pgfqpoint{2.495542in}{0.413320in}}%
\pgfpathlineto{\pgfqpoint{2.492729in}{0.413320in}}%
\pgfpathlineto{\pgfqpoint{2.490183in}{0.413320in}}%
\pgfpathlineto{\pgfqpoint{2.487384in}{0.413320in}}%
\pgfpathlineto{\pgfqpoint{2.484870in}{0.413320in}}%
\pgfpathlineto{\pgfqpoint{2.482026in}{0.413320in}}%
\pgfpathlineto{\pgfqpoint{2.479420in}{0.413320in}}%
\pgfpathlineto{\pgfqpoint{2.476671in}{0.413320in}}%
\pgfpathlineto{\pgfqpoint{2.473989in}{0.413320in}}%
\pgfpathlineto{\pgfqpoint{2.471311in}{0.413320in}}%
\pgfpathlineto{\pgfqpoint{2.468635in}{0.413320in}}%
\pgfpathlineto{\pgfqpoint{2.465957in}{0.413320in}}%
\pgfpathlineto{\pgfqpoint{2.463280in}{0.413320in}}%
\pgfpathlineto{\pgfqpoint{2.460711in}{0.413320in}}%
\pgfpathlineto{\pgfqpoint{2.457917in}{0.413320in}}%
\pgfpathlineto{\pgfqpoint{2.455353in}{0.413320in}}%
\pgfpathlineto{\pgfqpoint{2.452562in}{0.413320in}}%
\pgfpathlineto{\pgfqpoint{2.450032in}{0.413320in}}%
\pgfpathlineto{\pgfqpoint{2.447209in}{0.413320in}}%
\pgfpathlineto{\pgfqpoint{2.444677in}{0.413320in}}%
\pgfpathlineto{\pgfqpoint{2.441876in}{0.413320in}}%
\pgfpathlineto{\pgfqpoint{2.439167in}{0.413320in}}%
\pgfpathlineto{\pgfqpoint{2.436518in}{0.413320in}}%
\pgfpathlineto{\pgfqpoint{2.433815in}{0.413320in}}%
\pgfpathlineto{\pgfqpoint{2.431251in}{0.413320in}}%
\pgfpathlineto{\pgfqpoint{2.428453in}{0.413320in}}%
\pgfpathlineto{\pgfqpoint{2.425878in}{0.413320in}}%
\pgfpathlineto{\pgfqpoint{2.423098in}{0.413320in}}%
\pgfpathlineto{\pgfqpoint{2.420528in}{0.413320in}}%
\pgfpathlineto{\pgfqpoint{2.417747in}{0.413320in}}%
\pgfpathlineto{\pgfqpoint{2.415184in}{0.413320in}}%
\pgfpathlineto{\pgfqpoint{2.412389in}{0.413320in}}%
\pgfpathlineto{\pgfqpoint{2.409699in}{0.413320in}}%
\pgfpathlineto{\pgfqpoint{2.407024in}{0.413320in}}%
\pgfpathlineto{\pgfqpoint{2.404352in}{0.413320in}}%
\pgfpathlineto{\pgfqpoint{2.401675in}{0.413320in}}%
\pgfpathlineto{\pgfqpoint{2.398995in}{0.413320in}}%
\pgfpathclose%
\pgfusepath{stroke,fill}%
\end{pgfscope}%
\begin{pgfscope}%
\pgfpathrectangle{\pgfqpoint{2.398995in}{0.319877in}}{\pgfqpoint{3.986877in}{1.993438in}} %
\pgfusepath{clip}%
\pgfsetbuttcap%
\pgfsetroundjoin%
\definecolor{currentfill}{rgb}{1.000000,1.000000,1.000000}%
\pgfsetfillcolor{currentfill}%
\pgfsetlinewidth{1.003750pt}%
\definecolor{currentstroke}{rgb}{0.207541,0.681682,0.596984}%
\pgfsetstrokecolor{currentstroke}%
\pgfsetdash{}{0pt}%
\pgfpathmoveto{\pgfqpoint{2.398995in}{0.413320in}}%
\pgfpathlineto{\pgfqpoint{2.398995in}{2.099659in}}%
\pgfpathlineto{\pgfqpoint{2.401675in}{2.103971in}}%
\pgfpathlineto{\pgfqpoint{2.404352in}{2.104096in}}%
\pgfpathlineto{\pgfqpoint{2.407024in}{2.105862in}}%
\pgfpathlineto{\pgfqpoint{2.409699in}{2.107011in}}%
\pgfpathlineto{\pgfqpoint{2.412389in}{2.104143in}}%
\pgfpathlineto{\pgfqpoint{2.415184in}{2.104919in}}%
\pgfpathlineto{\pgfqpoint{2.417747in}{2.106684in}}%
\pgfpathlineto{\pgfqpoint{2.420528in}{2.109130in}}%
\pgfpathlineto{\pgfqpoint{2.423098in}{2.104092in}}%
\pgfpathlineto{\pgfqpoint{2.425878in}{2.105601in}}%
\pgfpathlineto{\pgfqpoint{2.428453in}{2.107056in}}%
\pgfpathlineto{\pgfqpoint{2.431251in}{2.108938in}}%
\pgfpathlineto{\pgfqpoint{2.433815in}{2.125633in}}%
\pgfpathlineto{\pgfqpoint{2.436518in}{2.120093in}}%
\pgfpathlineto{\pgfqpoint{2.439167in}{2.113691in}}%
\pgfpathlineto{\pgfqpoint{2.441876in}{2.113471in}}%
\pgfpathlineto{\pgfqpoint{2.444677in}{2.108234in}}%
\pgfpathlineto{\pgfqpoint{2.447209in}{2.106090in}}%
\pgfpathlineto{\pgfqpoint{2.450032in}{2.109271in}}%
\pgfpathlineto{\pgfqpoint{2.452562in}{2.107868in}}%
\pgfpathlineto{\pgfqpoint{2.455353in}{2.100728in}}%
\pgfpathlineto{\pgfqpoint{2.457917in}{2.099633in}}%
\pgfpathlineto{\pgfqpoint{2.460711in}{2.099358in}}%
\pgfpathlineto{\pgfqpoint{2.463280in}{2.101017in}}%
\pgfpathlineto{\pgfqpoint{2.465957in}{2.105590in}}%
\pgfpathlineto{\pgfqpoint{2.468635in}{2.098724in}}%
\pgfpathlineto{\pgfqpoint{2.471311in}{2.099116in}}%
\pgfpathlineto{\pgfqpoint{2.473989in}{2.099851in}}%
\pgfpathlineto{\pgfqpoint{2.476671in}{2.103045in}}%
\pgfpathlineto{\pgfqpoint{2.479420in}{2.107716in}}%
\pgfpathlineto{\pgfqpoint{2.482026in}{2.122761in}}%
\pgfpathlineto{\pgfqpoint{2.484870in}{2.122557in}}%
\pgfpathlineto{\pgfqpoint{2.487384in}{2.120506in}}%
\pgfpathlineto{\pgfqpoint{2.490183in}{2.110228in}}%
\pgfpathlineto{\pgfqpoint{2.492729in}{2.106803in}}%
\pgfpathlineto{\pgfqpoint{2.495542in}{2.110635in}}%
\pgfpathlineto{\pgfqpoint{2.498085in}{2.113781in}}%
\pgfpathlineto{\pgfqpoint{2.500801in}{2.109980in}}%
\pgfpathlineto{\pgfqpoint{2.503454in}{2.111236in}}%
\pgfpathlineto{\pgfqpoint{2.506163in}{2.110637in}}%
\pgfpathlineto{\pgfqpoint{2.508917in}{2.112372in}}%
\pgfpathlineto{\pgfqpoint{2.511478in}{2.110172in}}%
\pgfpathlineto{\pgfqpoint{2.514268in}{2.112671in}}%
\pgfpathlineto{\pgfqpoint{2.516845in}{2.108703in}}%
\pgfpathlineto{\pgfqpoint{2.519607in}{2.102372in}}%
\pgfpathlineto{\pgfqpoint{2.522197in}{2.106813in}}%
\pgfpathlineto{\pgfqpoint{2.524988in}{2.121511in}}%
\pgfpathlineto{\pgfqpoint{2.527560in}{2.123766in}}%
\pgfpathlineto{\pgfqpoint{2.530234in}{2.122595in}}%
\pgfpathlineto{\pgfqpoint{2.532917in}{2.112270in}}%
\pgfpathlineto{\pgfqpoint{2.535624in}{2.113849in}}%
\pgfpathlineto{\pgfqpoint{2.538274in}{2.113885in}}%
\pgfpathlineto{\pgfqpoint{2.540949in}{2.115654in}}%
\pgfpathlineto{\pgfqpoint{2.543765in}{2.111114in}}%
\pgfpathlineto{\pgfqpoint{2.546310in}{2.110677in}}%
\pgfpathlineto{\pgfqpoint{2.549114in}{2.108354in}}%
\pgfpathlineto{\pgfqpoint{2.551664in}{2.107436in}}%
\pgfpathlineto{\pgfqpoint{2.554493in}{2.105677in}}%
\pgfpathlineto{\pgfqpoint{2.557009in}{2.101994in}}%
\pgfpathlineto{\pgfqpoint{2.559790in}{2.104121in}}%
\pgfpathlineto{\pgfqpoint{2.562375in}{2.111939in}}%
\pgfpathlineto{\pgfqpoint{2.565045in}{2.109990in}}%
\pgfpathlineto{\pgfqpoint{2.567730in}{2.110106in}}%
\pgfpathlineto{\pgfqpoint{2.570411in}{2.107015in}}%
\pgfpathlineto{\pgfqpoint{2.573082in}{2.103515in}}%
\pgfpathlineto{\pgfqpoint{2.575779in}{2.108979in}}%
\pgfpathlineto{\pgfqpoint{2.578567in}{2.103190in}}%
\pgfpathlineto{\pgfqpoint{2.581129in}{2.103401in}}%
\pgfpathlineto{\pgfqpoint{2.583913in}{2.111662in}}%
\pgfpathlineto{\pgfqpoint{2.586484in}{2.107176in}}%
\pgfpathlineto{\pgfqpoint{2.589248in}{2.109425in}}%
\pgfpathlineto{\pgfqpoint{2.591842in}{2.105382in}}%
\pgfpathlineto{\pgfqpoint{2.594630in}{2.102407in}}%
\pgfpathlineto{\pgfqpoint{2.597196in}{2.105332in}}%
\pgfpathlineto{\pgfqpoint{2.599920in}{2.108568in}}%
\pgfpathlineto{\pgfqpoint{2.602557in}{2.100472in}}%
\pgfpathlineto{\pgfqpoint{2.605232in}{2.100184in}}%
\pgfpathlineto{\pgfqpoint{2.608004in}{2.109375in}}%
\pgfpathlineto{\pgfqpoint{2.610588in}{2.101521in}}%
\pgfpathlineto{\pgfqpoint{2.613393in}{2.099198in}}%
\pgfpathlineto{\pgfqpoint{2.615934in}{2.095091in}}%
\pgfpathlineto{\pgfqpoint{2.618773in}{2.098320in}}%
\pgfpathlineto{\pgfqpoint{2.621304in}{2.098643in}}%
\pgfpathlineto{\pgfqpoint{2.624077in}{2.101913in}}%
\pgfpathlineto{\pgfqpoint{2.626653in}{2.109838in}}%
\pgfpathlineto{\pgfqpoint{2.629340in}{2.109087in}}%
\pgfpathlineto{\pgfqpoint{2.632018in}{2.107316in}}%
\pgfpathlineto{\pgfqpoint{2.634700in}{2.105631in}}%
\pgfpathlineto{\pgfqpoint{2.637369in}{2.107504in}}%
\pgfpathlineto{\pgfqpoint{2.640053in}{2.105710in}}%
\pgfpathlineto{\pgfqpoint{2.642827in}{2.106720in}}%
\pgfpathlineto{\pgfqpoint{2.645408in}{2.103250in}}%
\pgfpathlineto{\pgfqpoint{2.648196in}{2.100021in}}%
\pgfpathlineto{\pgfqpoint{2.650767in}{2.098666in}}%
\pgfpathlineto{\pgfqpoint{2.653567in}{2.096333in}}%
\pgfpathlineto{\pgfqpoint{2.656124in}{2.099337in}}%
\pgfpathlineto{\pgfqpoint{2.658942in}{2.099875in}}%
\pgfpathlineto{\pgfqpoint{2.661481in}{2.091610in}}%
\pgfpathlineto{\pgfqpoint{2.664151in}{2.087112in}}%
\pgfpathlineto{\pgfqpoint{2.666836in}{2.089413in}}%
\pgfpathlineto{\pgfqpoint{2.669506in}{2.096742in}}%
\pgfpathlineto{\pgfqpoint{2.672301in}{2.099485in}}%
\pgfpathlineto{\pgfqpoint{2.674873in}{2.103447in}}%
\pgfpathlineto{\pgfqpoint{2.677650in}{2.106771in}}%
\pgfpathlineto{\pgfqpoint{2.680224in}{2.108442in}}%
\pgfpathlineto{\pgfqpoint{2.683009in}{2.109164in}}%
\pgfpathlineto{\pgfqpoint{2.685586in}{2.108110in}}%
\pgfpathlineto{\pgfqpoint{2.688328in}{2.113377in}}%
\pgfpathlineto{\pgfqpoint{2.690940in}{2.103424in}}%
\pgfpathlineto{\pgfqpoint{2.693611in}{2.110841in}}%
\pgfpathlineto{\pgfqpoint{2.696293in}{2.109482in}}%
\pgfpathlineto{\pgfqpoint{2.698968in}{2.109536in}}%
\pgfpathlineto{\pgfqpoint{2.701657in}{2.111944in}}%
\pgfpathlineto{\pgfqpoint{2.704326in}{2.107799in}}%
\pgfpathlineto{\pgfqpoint{2.707125in}{2.107735in}}%
\pgfpathlineto{\pgfqpoint{2.709683in}{2.113449in}}%
\pgfpathlineto{\pgfqpoint{2.712477in}{2.108861in}}%
\pgfpathlineto{\pgfqpoint{2.715036in}{2.111508in}}%
\pgfpathlineto{\pgfqpoint{2.717773in}{2.108003in}}%
\pgfpathlineto{\pgfqpoint{2.720404in}{2.107314in}}%
\pgfpathlineto{\pgfqpoint{2.723211in}{2.108378in}}%
\pgfpathlineto{\pgfqpoint{2.725760in}{2.111229in}}%
\pgfpathlineto{\pgfqpoint{2.728439in}{2.114481in}}%
\pgfpathlineto{\pgfqpoint{2.731119in}{2.111968in}}%
\pgfpathlineto{\pgfqpoint{2.733798in}{2.117182in}}%
\pgfpathlineto{\pgfqpoint{2.736476in}{2.105089in}}%
\pgfpathlineto{\pgfqpoint{2.739155in}{2.093635in}}%
\pgfpathlineto{\pgfqpoint{2.741928in}{2.096649in}}%
\pgfpathlineto{\pgfqpoint{2.744510in}{2.104791in}}%
\pgfpathlineto{\pgfqpoint{2.747260in}{2.104895in}}%
\pgfpathlineto{\pgfqpoint{2.749868in}{2.106314in}}%
\pgfpathlineto{\pgfqpoint{2.752614in}{2.102478in}}%
\pgfpathlineto{\pgfqpoint{2.755224in}{2.106483in}}%
\pgfpathlineto{\pgfqpoint{2.758028in}{2.099146in}}%
\pgfpathlineto{\pgfqpoint{2.760581in}{2.109393in}}%
\pgfpathlineto{\pgfqpoint{2.763253in}{2.103683in}}%
\pgfpathlineto{\pgfqpoint{2.765935in}{2.100119in}}%
\pgfpathlineto{\pgfqpoint{2.768617in}{2.100398in}}%
\pgfpathlineto{\pgfqpoint{2.771373in}{2.105780in}}%
\pgfpathlineto{\pgfqpoint{2.773972in}{2.107311in}}%
\pgfpathlineto{\pgfqpoint{2.776767in}{2.100221in}}%
\pgfpathlineto{\pgfqpoint{2.779330in}{2.105371in}}%
\pgfpathlineto{\pgfqpoint{2.782113in}{2.101648in}}%
\pgfpathlineto{\pgfqpoint{2.784687in}{2.107824in}}%
\pgfpathlineto{\pgfqpoint{2.787468in}{2.102494in}}%
\pgfpathlineto{\pgfqpoint{2.790044in}{2.108578in}}%
\pgfpathlineto{\pgfqpoint{2.792721in}{2.105678in}}%
\pgfpathlineto{\pgfqpoint{2.795398in}{2.099037in}}%
\pgfpathlineto{\pgfqpoint{2.798070in}{2.105137in}}%
\pgfpathlineto{\pgfqpoint{2.800756in}{2.112815in}}%
\pgfpathlineto{\pgfqpoint{2.803435in}{2.111619in}}%
\pgfpathlineto{\pgfqpoint{2.806175in}{2.105676in}}%
\pgfpathlineto{\pgfqpoint{2.808792in}{2.109258in}}%
\pgfpathlineto{\pgfqpoint{2.811597in}{2.106919in}}%
\pgfpathlineto{\pgfqpoint{2.814141in}{2.109192in}}%
\pgfpathlineto{\pgfqpoint{2.816867in}{2.105206in}}%
\pgfpathlineto{\pgfqpoint{2.819506in}{2.099473in}}%
\pgfpathlineto{\pgfqpoint{2.822303in}{2.105396in}}%
\pgfpathlineto{\pgfqpoint{2.824851in}{2.107078in}}%
\pgfpathlineto{\pgfqpoint{2.827567in}{2.106514in}}%
\pgfpathlineto{\pgfqpoint{2.830219in}{2.103318in}}%
\pgfpathlineto{\pgfqpoint{2.832894in}{2.099674in}}%
\pgfpathlineto{\pgfqpoint{2.835698in}{2.098541in}}%
\pgfpathlineto{\pgfqpoint{2.838254in}{2.103494in}}%
\pgfpathlineto{\pgfqpoint{2.841055in}{2.111741in}}%
\pgfpathlineto{\pgfqpoint{2.843611in}{2.108175in}}%
\pgfpathlineto{\pgfqpoint{2.846408in}{2.102908in}}%
\pgfpathlineto{\pgfqpoint{2.848960in}{2.107296in}}%
\pgfpathlineto{\pgfqpoint{2.851793in}{2.117676in}}%
\pgfpathlineto{\pgfqpoint{2.854325in}{2.108860in}}%
\pgfpathlineto{\pgfqpoint{2.857003in}{2.102697in}}%
\pgfpathlineto{\pgfqpoint{2.859668in}{2.105711in}}%
\pgfpathlineto{\pgfqpoint{2.862402in}{2.107120in}}%
\pgfpathlineto{\pgfqpoint{2.865031in}{2.105833in}}%
\pgfpathlineto{\pgfqpoint{2.867713in}{2.106738in}}%
\pgfpathlineto{\pgfqpoint{2.870475in}{2.107934in}}%
\pgfpathlineto{\pgfqpoint{2.873074in}{2.109297in}}%
\pgfpathlineto{\pgfqpoint{2.875882in}{2.110886in}}%
\pgfpathlineto{\pgfqpoint{2.878431in}{2.108582in}}%
\pgfpathlineto{\pgfqpoint{2.881254in}{2.107065in}}%
\pgfpathlineto{\pgfqpoint{2.883780in}{2.104842in}}%
\pgfpathlineto{\pgfqpoint{2.886578in}{2.100634in}}%
\pgfpathlineto{\pgfqpoint{2.889145in}{2.097415in}}%
\pgfpathlineto{\pgfqpoint{2.891809in}{2.095829in}}%
\pgfpathlineto{\pgfqpoint{2.894487in}{2.101280in}}%
\pgfpathlineto{\pgfqpoint{2.897179in}{2.105099in}}%
\pgfpathlineto{\pgfqpoint{2.899858in}{2.104730in}}%
\pgfpathlineto{\pgfqpoint{2.902535in}{2.106032in}}%
\pgfpathlineto{\pgfqpoint{2.905341in}{2.108608in}}%
\pgfpathlineto{\pgfqpoint{2.907882in}{2.102462in}}%
\pgfpathlineto{\pgfqpoint{2.910631in}{2.106661in}}%
\pgfpathlineto{\pgfqpoint{2.913243in}{2.099196in}}%
\pgfpathlineto{\pgfqpoint{2.916061in}{2.096545in}}%
\pgfpathlineto{\pgfqpoint{2.918606in}{2.094758in}}%
\pgfpathlineto{\pgfqpoint{2.921363in}{2.097765in}}%
\pgfpathlineto{\pgfqpoint{2.923963in}{2.099835in}}%
\pgfpathlineto{\pgfqpoint{2.926655in}{2.100255in}}%
\pgfpathlineto{\pgfqpoint{2.929321in}{2.103506in}}%
\pgfpathlineto{\pgfqpoint{2.932033in}{2.098982in}}%
\pgfpathlineto{\pgfqpoint{2.934759in}{2.100004in}}%
\pgfpathlineto{\pgfqpoint{2.937352in}{2.092805in}}%
\pgfpathlineto{\pgfqpoint{2.940120in}{2.093224in}}%
\pgfpathlineto{\pgfqpoint{2.942711in}{2.094648in}}%
\pgfpathlineto{\pgfqpoint{2.945461in}{2.093433in}}%
\pgfpathlineto{\pgfqpoint{2.948068in}{2.101432in}}%
\pgfpathlineto{\pgfqpoint{2.950884in}{2.104444in}}%
\pgfpathlineto{\pgfqpoint{2.953422in}{2.105945in}}%
\pgfpathlineto{\pgfqpoint{2.956103in}{2.104865in}}%
\pgfpathlineto{\pgfqpoint{2.958782in}{2.107558in}}%
\pgfpathlineto{\pgfqpoint{2.961460in}{2.107802in}}%
\pgfpathlineto{\pgfqpoint{2.964127in}{2.110073in}}%
\pgfpathlineto{\pgfqpoint{2.966812in}{2.113899in}}%
\pgfpathlineto{\pgfqpoint{2.969599in}{2.108883in}}%
\pgfpathlineto{\pgfqpoint{2.972177in}{2.110574in}}%
\pgfpathlineto{\pgfqpoint{2.974972in}{2.113193in}}%
\pgfpathlineto{\pgfqpoint{2.977517in}{2.108090in}}%
\pgfpathlineto{\pgfqpoint{2.980341in}{2.110968in}}%
\pgfpathlineto{\pgfqpoint{2.982885in}{2.113610in}}%
\pgfpathlineto{\pgfqpoint{2.985666in}{2.108373in}}%
\pgfpathlineto{\pgfqpoint{2.988238in}{2.107188in}}%
\pgfpathlineto{\pgfqpoint{2.990978in}{2.113543in}}%
\pgfpathlineto{\pgfqpoint{2.993595in}{2.115250in}}%
\pgfpathlineto{\pgfqpoint{2.996300in}{2.107171in}}%
\pgfpathlineto{\pgfqpoint{2.999103in}{2.098626in}}%
\pgfpathlineto{\pgfqpoint{3.001635in}{2.094685in}}%
\pgfpathlineto{\pgfqpoint{3.004419in}{2.102489in}}%
\pgfpathlineto{\pgfqpoint{3.006993in}{2.097981in}}%
\pgfpathlineto{\pgfqpoint{3.009784in}{2.099850in}}%
\pgfpathlineto{\pgfqpoint{3.012351in}{2.090413in}}%
\pgfpathlineto{\pgfqpoint{3.015097in}{2.098474in}}%
\pgfpathlineto{\pgfqpoint{3.017707in}{2.094012in}}%
\pgfpathlineto{\pgfqpoint{3.020382in}{2.086947in}}%
\pgfpathlineto{\pgfqpoint{3.023058in}{2.096883in}}%
\pgfpathlineto{\pgfqpoint{3.025803in}{2.108017in}}%
\pgfpathlineto{\pgfqpoint{3.028412in}{2.104648in}}%
\pgfpathlineto{\pgfqpoint{3.031091in}{2.107036in}}%
\pgfpathlineto{\pgfqpoint{3.033921in}{2.102474in}}%
\pgfpathlineto{\pgfqpoint{3.036456in}{2.100602in}}%
\pgfpathlineto{\pgfqpoint{3.039262in}{2.103640in}}%
\pgfpathlineto{\pgfqpoint{3.041813in}{2.098927in}}%
\pgfpathlineto{\pgfqpoint{3.044568in}{2.100407in}}%
\pgfpathlineto{\pgfqpoint{3.047157in}{2.116651in}}%
\pgfpathlineto{\pgfqpoint{3.049988in}{2.149643in}}%
\pgfpathlineto{\pgfqpoint{3.052526in}{2.130002in}}%
\pgfpathlineto{\pgfqpoint{3.055202in}{2.111896in}}%
\pgfpathlineto{\pgfqpoint{3.057884in}{2.104876in}}%
\pgfpathlineto{\pgfqpoint{3.060561in}{2.104991in}}%
\pgfpathlineto{\pgfqpoint{3.063230in}{2.102738in}}%
\pgfpathlineto{\pgfqpoint{3.065916in}{2.099854in}}%
\pgfpathlineto{\pgfqpoint{3.068709in}{2.102366in}}%
\pgfpathlineto{\pgfqpoint{3.071266in}{2.095676in}}%
\pgfpathlineto{\pgfqpoint{3.074056in}{2.101573in}}%
\pgfpathlineto{\pgfqpoint{3.076631in}{2.098601in}}%
\pgfpathlineto{\pgfqpoint{3.079381in}{2.104524in}}%
\pgfpathlineto{\pgfqpoint{3.081990in}{2.106274in}}%
\pgfpathlineto{\pgfqpoint{3.084671in}{2.103929in}}%
\pgfpathlineto{\pgfqpoint{3.087343in}{2.107964in}}%
\pgfpathlineto{\pgfqpoint{3.090023in}{2.101424in}}%
\pgfpathlineto{\pgfqpoint{3.092699in}{2.104921in}}%
\pgfpathlineto{\pgfqpoint{3.095388in}{2.101776in}}%
\pgfpathlineto{\pgfqpoint{3.098163in}{2.099699in}}%
\pgfpathlineto{\pgfqpoint{3.100737in}{2.095093in}}%
\pgfpathlineto{\pgfqpoint{3.103508in}{2.100704in}}%
\pgfpathlineto{\pgfqpoint{3.106094in}{2.105627in}}%
\pgfpathlineto{\pgfqpoint{3.108896in}{2.105437in}}%
\pgfpathlineto{\pgfqpoint{3.111451in}{2.105992in}}%
\pgfpathlineto{\pgfqpoint{3.114242in}{2.103427in}}%
\pgfpathlineto{\pgfqpoint{3.116807in}{2.103440in}}%
\pgfpathlineto{\pgfqpoint{3.119487in}{2.103299in}}%
\pgfpathlineto{\pgfqpoint{3.122163in}{2.102162in}}%
\pgfpathlineto{\pgfqpoint{3.124842in}{2.100584in}}%
\pgfpathlineto{\pgfqpoint{3.127512in}{2.104882in}}%
\pgfpathlineto{\pgfqpoint{3.130199in}{2.105345in}}%
\pgfpathlineto{\pgfqpoint{3.132946in}{2.110291in}}%
\pgfpathlineto{\pgfqpoint{3.135550in}{2.100807in}}%
\pgfpathlineto{\pgfqpoint{3.138375in}{2.101179in}}%
\pgfpathlineto{\pgfqpoint{3.140913in}{2.101645in}}%
\pgfpathlineto{\pgfqpoint{3.143740in}{2.110030in}}%
\pgfpathlineto{\pgfqpoint{3.146271in}{2.106918in}}%
\pgfpathlineto{\pgfqpoint{3.149057in}{2.099607in}}%
\pgfpathlineto{\pgfqpoint{3.151612in}{2.093558in}}%
\pgfpathlineto{\pgfqpoint{3.154327in}{2.099703in}}%
\pgfpathlineto{\pgfqpoint{3.156981in}{2.103585in}}%
\pgfpathlineto{\pgfqpoint{3.159675in}{2.102334in}}%
\pgfpathlineto{\pgfqpoint{3.162474in}{2.105573in}}%
\pgfpathlineto{\pgfqpoint{3.165019in}{2.106008in}}%
\pgfpathlineto{\pgfqpoint{3.167776in}{2.100330in}}%
\pgfpathlineto{\pgfqpoint{3.170375in}{2.099607in}}%
\pgfpathlineto{\pgfqpoint{3.173142in}{2.090547in}}%
\pgfpathlineto{\pgfqpoint{3.175724in}{2.095336in}}%
\pgfpathlineto{\pgfqpoint{3.178525in}{2.104014in}}%
\pgfpathlineto{\pgfqpoint{3.181089in}{2.101928in}}%
\pgfpathlineto{\pgfqpoint{3.183760in}{2.107811in}}%
\pgfpathlineto{\pgfqpoint{3.186440in}{2.100514in}}%
\pgfpathlineto{\pgfqpoint{3.189117in}{2.100792in}}%
\pgfpathlineto{\pgfqpoint{3.191796in}{2.104462in}}%
\pgfpathlineto{\pgfqpoint{3.194508in}{2.108564in}}%
\pgfpathlineto{\pgfqpoint{3.197226in}{2.106336in}}%
\pgfpathlineto{\pgfqpoint{3.199823in}{2.106969in}}%
\pgfpathlineto{\pgfqpoint{3.202562in}{2.105571in}}%
\pgfpathlineto{\pgfqpoint{3.205195in}{2.102808in}}%
\pgfpathlineto{\pgfqpoint{3.207984in}{2.102452in}}%
\pgfpathlineto{\pgfqpoint{3.210545in}{2.091912in}}%
\pgfpathlineto{\pgfqpoint{3.213342in}{2.099245in}}%
\pgfpathlineto{\pgfqpoint{3.215908in}{2.099756in}}%
\pgfpathlineto{\pgfqpoint{3.218586in}{2.105754in}}%
\pgfpathlineto{\pgfqpoint{3.221255in}{2.099100in}}%
\pgfpathlineto{\pgfqpoint{3.223942in}{2.105206in}}%
\pgfpathlineto{\pgfqpoint{3.226609in}{2.103739in}}%
\pgfpathlineto{\pgfqpoint{3.229310in}{2.102663in}}%
\pgfpathlineto{\pgfqpoint{3.232069in}{2.098207in}}%
\pgfpathlineto{\pgfqpoint{3.234658in}{2.106417in}}%
\pgfpathlineto{\pgfqpoint{3.237411in}{2.102156in}}%
\pgfpathlineto{\pgfqpoint{3.240010in}{2.105316in}}%
\pgfpathlineto{\pgfqpoint{3.242807in}{2.108229in}}%
\pgfpathlineto{\pgfqpoint{3.245363in}{2.109853in}}%
\pgfpathlineto{\pgfqpoint{3.248049in}{2.110474in}}%
\pgfpathlineto{\pgfqpoint{3.250716in}{2.107933in}}%
\pgfpathlineto{\pgfqpoint{3.253404in}{2.108683in}}%
\pgfpathlineto{\pgfqpoint{3.256083in}{2.100851in}}%
\pgfpathlineto{\pgfqpoint{3.258784in}{2.104428in}}%
\pgfpathlineto{\pgfqpoint{3.261594in}{2.098336in}}%
\pgfpathlineto{\pgfqpoint{3.264119in}{2.095808in}}%
\pgfpathlineto{\pgfqpoint{3.266849in}{2.087338in}}%
\pgfpathlineto{\pgfqpoint{3.269478in}{2.088688in}}%
\pgfpathlineto{\pgfqpoint{3.272254in}{2.102291in}}%
\pgfpathlineto{\pgfqpoint{3.274831in}{2.093418in}}%
\pgfpathlineto{\pgfqpoint{3.277603in}{2.107271in}}%
\pgfpathlineto{\pgfqpoint{3.280189in}{2.101427in}}%
\pgfpathlineto{\pgfqpoint{3.282870in}{2.099819in}}%
\pgfpathlineto{\pgfqpoint{3.285534in}{2.105659in}}%
\pgfpathlineto{\pgfqpoint{3.288225in}{2.110906in}}%
\pgfpathlineto{\pgfqpoint{3.290890in}{2.105017in}}%
\pgfpathlineto{\pgfqpoint{3.293574in}{2.097060in}}%
\pgfpathlineto{\pgfqpoint{3.296376in}{2.103141in}}%
\pgfpathlineto{\pgfqpoint{3.298937in}{2.100919in}}%
\pgfpathlineto{\pgfqpoint{3.301719in}{2.102760in}}%
\pgfpathlineto{\pgfqpoint{3.304295in}{2.103035in}}%
\pgfpathlineto{\pgfqpoint{3.307104in}{2.103613in}}%
\pgfpathlineto{\pgfqpoint{3.309652in}{2.102804in}}%
\pgfpathlineto{\pgfqpoint{3.312480in}{2.103747in}}%
\pgfpathlineto{\pgfqpoint{3.315008in}{2.100836in}}%
\pgfpathlineto{\pgfqpoint{3.317688in}{2.103790in}}%
\pgfpathlineto{\pgfqpoint{3.320366in}{2.104375in}}%
\pgfpathlineto{\pgfqpoint{3.323049in}{2.105678in}}%
\pgfpathlineto{\pgfqpoint{3.325860in}{2.106047in}}%
\pgfpathlineto{\pgfqpoint{3.328401in}{2.106899in}}%
\pgfpathlineto{\pgfqpoint{3.331183in}{2.109324in}}%
\pgfpathlineto{\pgfqpoint{3.333758in}{2.104542in}}%
\pgfpathlineto{\pgfqpoint{3.336541in}{2.103399in}}%
\pgfpathlineto{\pgfqpoint{3.339101in}{2.099954in}}%
\pgfpathlineto{\pgfqpoint{3.341893in}{2.103595in}}%
\pgfpathlineto{\pgfqpoint{3.344468in}{2.102512in}}%
\pgfpathlineto{\pgfqpoint{3.347139in}{2.099297in}}%
\pgfpathlineto{\pgfqpoint{3.349828in}{2.104817in}}%
\pgfpathlineto{\pgfqpoint{3.352505in}{2.105383in}}%
\pgfpathlineto{\pgfqpoint{3.355177in}{2.111871in}}%
\pgfpathlineto{\pgfqpoint{3.357862in}{2.107044in}}%
\pgfpathlineto{\pgfqpoint{3.360620in}{2.105310in}}%
\pgfpathlineto{\pgfqpoint{3.363221in}{2.104337in}}%
\pgfpathlineto{\pgfqpoint{3.365996in}{2.107863in}}%
\pgfpathlineto{\pgfqpoint{3.368577in}{2.102570in}}%
\pgfpathlineto{\pgfqpoint{3.371357in}{2.105843in}}%
\pgfpathlineto{\pgfqpoint{3.373921in}{2.102883in}}%
\pgfpathlineto{\pgfqpoint{3.376735in}{2.107351in}}%
\pgfpathlineto{\pgfqpoint{3.379290in}{2.112042in}}%
\pgfpathlineto{\pgfqpoint{3.381959in}{2.112982in}}%
\pgfpathlineto{\pgfqpoint{3.384647in}{2.108685in}}%
\pgfpathlineto{\pgfqpoint{3.387309in}{2.112126in}}%
\pgfpathlineto{\pgfqpoint{3.390102in}{2.111068in}}%
\pgfpathlineto{\pgfqpoint{3.392681in}{2.108614in}}%
\pgfpathlineto{\pgfqpoint{3.395461in}{2.103776in}}%
\pgfpathlineto{\pgfqpoint{3.398037in}{2.106777in}}%
\pgfpathlineto{\pgfqpoint{3.400783in}{2.111103in}}%
\pgfpathlineto{\pgfqpoint{3.403394in}{2.112255in}}%
\pgfpathlineto{\pgfqpoint{3.406202in}{2.110554in}}%
\pgfpathlineto{\pgfqpoint{3.408752in}{2.113307in}}%
\pgfpathlineto{\pgfqpoint{3.411431in}{2.107423in}}%
\pgfpathlineto{\pgfqpoint{3.414109in}{2.107892in}}%
\pgfpathlineto{\pgfqpoint{3.416780in}{2.115438in}}%
\pgfpathlineto{\pgfqpoint{3.419455in}{2.110657in}}%
\pgfpathlineto{\pgfqpoint{3.422142in}{2.109720in}}%
\pgfpathlineto{\pgfqpoint{3.424887in}{2.109380in}}%
\pgfpathlineto{\pgfqpoint{3.427501in}{2.113183in}}%
\pgfpathlineto{\pgfqpoint{3.430313in}{2.109865in}}%
\pgfpathlineto{\pgfqpoint{3.432851in}{2.111960in}}%
\pgfpathlineto{\pgfqpoint{3.435635in}{2.107130in}}%
\pgfpathlineto{\pgfqpoint{3.438210in}{2.103411in}}%
\pgfpathlineto{\pgfqpoint{3.440996in}{2.105061in}}%
\pgfpathlineto{\pgfqpoint{3.443574in}{2.112765in}}%
\pgfpathlineto{\pgfqpoint{3.446257in}{2.109931in}}%
\pgfpathlineto{\pgfqpoint{3.448926in}{2.105615in}}%
\pgfpathlineto{\pgfqpoint{3.451597in}{2.105301in}}%
\pgfpathlineto{\pgfqpoint{3.454285in}{2.103618in}}%
\pgfpathlineto{\pgfqpoint{3.456960in}{2.108864in}}%
\pgfpathlineto{\pgfqpoint{3.459695in}{2.103722in}}%
\pgfpathlineto{\pgfqpoint{3.462321in}{2.103231in}}%
\pgfpathlineto{\pgfqpoint{3.465072in}{2.102821in}}%
\pgfpathlineto{\pgfqpoint{3.467678in}{2.104476in}}%
\pgfpathlineto{\pgfqpoint{3.470466in}{2.096851in}}%
\pgfpathlineto{\pgfqpoint{3.473021in}{2.092570in}}%
\pgfpathlineto{\pgfqpoint{3.475821in}{2.093820in}}%
\pgfpathlineto{\pgfqpoint{3.478378in}{2.104647in}}%
\pgfpathlineto{\pgfqpoint{3.481072in}{2.100687in}}%
\pgfpathlineto{\pgfqpoint{3.483744in}{2.105918in}}%
\pgfpathlineto{\pgfqpoint{3.486442in}{2.104292in}}%
\pgfpathlineto{\pgfqpoint{3.489223in}{2.106538in}}%
\pgfpathlineto{\pgfqpoint{3.491783in}{2.108630in}}%
\pgfpathlineto{\pgfqpoint{3.494581in}{2.107196in}}%
\pgfpathlineto{\pgfqpoint{3.497139in}{2.104876in}}%
\pgfpathlineto{\pgfqpoint{3.499909in}{2.113413in}}%
\pgfpathlineto{\pgfqpoint{3.502488in}{2.110167in}}%
\pgfpathlineto{\pgfqpoint{3.505262in}{2.109912in}}%
\pgfpathlineto{\pgfqpoint{3.507840in}{2.119354in}}%
\pgfpathlineto{\pgfqpoint{3.510533in}{2.113835in}}%
\pgfpathlineto{\pgfqpoint{3.513209in}{2.113228in}}%
\pgfpathlineto{\pgfqpoint{3.515884in}{2.113116in}}%
\pgfpathlineto{\pgfqpoint{3.518565in}{2.110983in}}%
\pgfpathlineto{\pgfqpoint{3.521244in}{2.115430in}}%
\pgfpathlineto{\pgfqpoint{3.524041in}{2.114953in}}%
\pgfpathlineto{\pgfqpoint{3.526601in}{2.116672in}}%
\pgfpathlineto{\pgfqpoint{3.529327in}{2.111905in}}%
\pgfpathlineto{\pgfqpoint{3.531955in}{2.110021in}}%
\pgfpathlineto{\pgfqpoint{3.534783in}{2.108768in}}%
\pgfpathlineto{\pgfqpoint{3.537309in}{2.110336in}}%
\pgfpathlineto{\pgfqpoint{3.540093in}{2.109242in}}%
\pgfpathlineto{\pgfqpoint{3.542656in}{2.109328in}}%
\pgfpathlineto{\pgfqpoint{3.545349in}{2.107836in}}%
\pgfpathlineto{\pgfqpoint{3.548029in}{2.110610in}}%
\pgfpathlineto{\pgfqpoint{3.550713in}{2.106496in}}%
\pgfpathlineto{\pgfqpoint{3.553498in}{2.104467in}}%
\pgfpathlineto{\pgfqpoint{3.556061in}{2.108561in}}%
\pgfpathlineto{\pgfqpoint{3.558853in}{2.110692in}}%
\pgfpathlineto{\pgfqpoint{3.561420in}{2.113816in}}%
\pgfpathlineto{\pgfqpoint{3.564188in}{2.111695in}}%
\pgfpathlineto{\pgfqpoint{3.566774in}{2.116720in}}%
\pgfpathlineto{\pgfqpoint{3.569584in}{2.115522in}}%
\pgfpathlineto{\pgfqpoint{3.572126in}{2.111314in}}%
\pgfpathlineto{\pgfqpoint{3.574814in}{2.110978in}}%
\pgfpathlineto{\pgfqpoint{3.577487in}{2.105728in}}%
\pgfpathlineto{\pgfqpoint{3.580191in}{2.109409in}}%
\pgfpathlineto{\pgfqpoint{3.582851in}{2.108545in}}%
\pgfpathlineto{\pgfqpoint{3.585532in}{2.109718in}}%
\pgfpathlineto{\pgfqpoint{3.588258in}{2.110026in}}%
\pgfpathlineto{\pgfqpoint{3.590883in}{2.109816in}}%
\pgfpathlineto{\pgfqpoint{3.593620in}{2.107359in}}%
\pgfpathlineto{\pgfqpoint{3.596240in}{2.109324in}}%
\pgfpathlineto{\pgfqpoint{3.598998in}{2.108497in}}%
\pgfpathlineto{\pgfqpoint{3.601590in}{2.103469in}}%
\pgfpathlineto{\pgfqpoint{3.604387in}{2.112261in}}%
\pgfpathlineto{\pgfqpoint{3.606951in}{2.109138in}}%
\pgfpathlineto{\pgfqpoint{3.609632in}{2.109055in}}%
\pgfpathlineto{\pgfqpoint{3.612311in}{2.117316in}}%
\pgfpathlineto{\pgfqpoint{3.614982in}{2.110558in}}%
\pgfpathlineto{\pgfqpoint{3.617667in}{2.111225in}}%
\pgfpathlineto{\pgfqpoint{3.620345in}{2.112201in}}%
\pgfpathlineto{\pgfqpoint{3.623165in}{2.107378in}}%
\pgfpathlineto{\pgfqpoint{3.625689in}{2.104352in}}%
\pgfpathlineto{\pgfqpoint{3.628460in}{2.110272in}}%
\pgfpathlineto{\pgfqpoint{3.631058in}{2.107022in}}%
\pgfpathlineto{\pgfqpoint{3.633858in}{2.108730in}}%
\pgfpathlineto{\pgfqpoint{3.636413in}{2.106139in}}%
\pgfpathlineto{\pgfqpoint{3.639207in}{2.103006in}}%
\pgfpathlineto{\pgfqpoint{3.641773in}{2.109844in}}%
\pgfpathlineto{\pgfqpoint{3.644452in}{2.104494in}}%
\pgfpathlineto{\pgfqpoint{3.647130in}{2.089078in}}%
\pgfpathlineto{\pgfqpoint{3.649837in}{2.091992in}}%
\pgfpathlineto{\pgfqpoint{3.652628in}{2.092691in}}%
\pgfpathlineto{\pgfqpoint{3.655165in}{2.100049in}}%
\pgfpathlineto{\pgfqpoint{3.657917in}{2.109840in}}%
\pgfpathlineto{\pgfqpoint{3.660515in}{2.119868in}}%
\pgfpathlineto{\pgfqpoint{3.663276in}{2.128662in}}%
\pgfpathlineto{\pgfqpoint{3.665864in}{2.121328in}}%
\pgfpathlineto{\pgfqpoint{3.668665in}{2.108872in}}%
\pgfpathlineto{\pgfqpoint{3.671232in}{2.101728in}}%
\pgfpathlineto{\pgfqpoint{3.673911in}{2.107076in}}%
\pgfpathlineto{\pgfqpoint{3.676591in}{2.105859in}}%
\pgfpathlineto{\pgfqpoint{3.679273in}{2.113405in}}%
\pgfpathlineto{\pgfqpoint{3.681948in}{2.103620in}}%
\pgfpathlineto{\pgfqpoint{3.684620in}{2.094074in}}%
\pgfpathlineto{\pgfqpoint{3.687442in}{2.101203in}}%
\pgfpathlineto{\pgfqpoint{3.689983in}{2.103393in}}%
\pgfpathlineto{\pgfqpoint{3.692765in}{2.103475in}}%
\pgfpathlineto{\pgfqpoint{3.695331in}{2.099579in}}%
\pgfpathlineto{\pgfqpoint{3.698125in}{2.101204in}}%
\pgfpathlineto{\pgfqpoint{3.700684in}{2.100744in}}%
\pgfpathlineto{\pgfqpoint{3.703460in}{2.104726in}}%
\pgfpathlineto{\pgfqpoint{3.706053in}{2.102822in}}%
\pgfpathlineto{\pgfqpoint{3.708729in}{2.098881in}}%
\pgfpathlineto{\pgfqpoint{3.711410in}{2.103066in}}%
\pgfpathlineto{\pgfqpoint{3.714086in}{2.101116in}}%
\pgfpathlineto{\pgfqpoint{3.716875in}{2.095589in}}%
\pgfpathlineto{\pgfqpoint{3.719446in}{2.102809in}}%
\pgfpathlineto{\pgfqpoint{3.722228in}{2.104217in}}%
\pgfpathlineto{\pgfqpoint{3.724804in}{2.098704in}}%
\pgfpathlineto{\pgfqpoint{3.727581in}{2.101322in}}%
\pgfpathlineto{\pgfqpoint{3.730158in}{2.099682in}}%
\pgfpathlineto{\pgfqpoint{3.732950in}{2.099752in}}%
\pgfpathlineto{\pgfqpoint{3.735509in}{2.106129in}}%
\pgfpathlineto{\pgfqpoint{3.738194in}{2.108293in}}%
\pgfpathlineto{\pgfqpoint{3.740874in}{2.106641in}}%
\pgfpathlineto{\pgfqpoint{3.743548in}{2.108415in}}%
\pgfpathlineto{\pgfqpoint{3.746229in}{2.105286in}}%
\pgfpathlineto{\pgfqpoint{3.748903in}{2.106257in}}%
\pgfpathlineto{\pgfqpoint{3.751728in}{2.108406in}}%
\pgfpathlineto{\pgfqpoint{3.754265in}{2.101276in}}%
\pgfpathlineto{\pgfqpoint{3.757065in}{2.103396in}}%
\pgfpathlineto{\pgfqpoint{3.759622in}{2.107554in}}%
\pgfpathlineto{\pgfqpoint{3.762389in}{2.103291in}}%
\pgfpathlineto{\pgfqpoint{3.764966in}{2.098704in}}%
\pgfpathlineto{\pgfqpoint{3.767782in}{2.108771in}}%
\pgfpathlineto{\pgfqpoint{3.770323in}{2.110656in}}%
\pgfpathlineto{\pgfqpoint{3.773014in}{2.110258in}}%
\pgfpathlineto{\pgfqpoint{3.775691in}{2.109501in}}%
\pgfpathlineto{\pgfqpoint{3.778370in}{2.122671in}}%
\pgfpathlineto{\pgfqpoint{3.781046in}{2.139379in}}%
\pgfpathlineto{\pgfqpoint{3.783725in}{2.120608in}}%
\pgfpathlineto{\pgfqpoint{3.786504in}{2.115137in}}%
\pgfpathlineto{\pgfqpoint{3.789084in}{2.116163in}}%
\pgfpathlineto{\pgfqpoint{3.791897in}{2.110516in}}%
\pgfpathlineto{\pgfqpoint{3.794435in}{2.115702in}}%
\pgfpathlineto{\pgfqpoint{3.797265in}{2.117743in}}%
\pgfpathlineto{\pgfqpoint{3.799797in}{2.107068in}}%
\pgfpathlineto{\pgfqpoint{3.802569in}{2.114631in}}%
\pgfpathlineto{\pgfqpoint{3.805145in}{2.109223in}}%
\pgfpathlineto{\pgfqpoint{3.807832in}{2.116177in}}%
\pgfpathlineto{\pgfqpoint{3.810510in}{2.107589in}}%
\pgfpathlineto{\pgfqpoint{3.813172in}{2.112843in}}%
\pgfpathlineto{\pgfqpoint{3.815983in}{2.108790in}}%
\pgfpathlineto{\pgfqpoint{3.818546in}{2.110355in}}%
\pgfpathlineto{\pgfqpoint{3.821315in}{2.110363in}}%
\pgfpathlineto{\pgfqpoint{3.823903in}{2.100121in}}%
\pgfpathlineto{\pgfqpoint{3.826679in}{2.099515in}}%
\pgfpathlineto{\pgfqpoint{3.829252in}{2.100453in}}%
\pgfpathlineto{\pgfqpoint{3.832053in}{2.107822in}}%
\pgfpathlineto{\pgfqpoint{3.834616in}{2.106824in}}%
\pgfpathlineto{\pgfqpoint{3.837286in}{2.108630in}}%
\pgfpathlineto{\pgfqpoint{3.839960in}{2.103063in}}%
\pgfpathlineto{\pgfqpoint{3.842641in}{2.105487in}}%
\pgfpathlineto{\pgfqpoint{3.845329in}{2.102880in}}%
\pgfpathlineto{\pgfqpoint{3.848005in}{2.107382in}}%
\pgfpathlineto{\pgfqpoint{3.850814in}{2.101746in}}%
\pgfpathlineto{\pgfqpoint{3.853358in}{2.107415in}}%
\pgfpathlineto{\pgfqpoint{3.856100in}{2.102797in}}%
\pgfpathlineto{\pgfqpoint{3.858720in}{2.099847in}}%
\pgfpathlineto{\pgfqpoint{3.861561in}{2.104283in}}%
\pgfpathlineto{\pgfqpoint{3.864073in}{2.103210in}}%
\pgfpathlineto{\pgfqpoint{3.866815in}{2.107722in}}%
\pgfpathlineto{\pgfqpoint{3.869435in}{2.105553in}}%
\pgfpathlineto{\pgfqpoint{3.872114in}{2.107594in}}%
\pgfpathlineto{\pgfqpoint{3.874790in}{2.109788in}}%
\pgfpathlineto{\pgfqpoint{3.877466in}{2.108957in}}%
\pgfpathlineto{\pgfqpoint{3.880237in}{2.108737in}}%
\pgfpathlineto{\pgfqpoint{3.882850in}{2.110631in}}%
\pgfpathlineto{\pgfqpoint{3.885621in}{2.106606in}}%
\pgfpathlineto{\pgfqpoint{3.888188in}{2.106063in}}%
\pgfpathlineto{\pgfqpoint{3.890926in}{2.108679in}}%
\pgfpathlineto{\pgfqpoint{3.893541in}{2.107744in}}%
\pgfpathlineto{\pgfqpoint{3.896345in}{2.105728in}}%
\pgfpathlineto{\pgfqpoint{3.898891in}{2.107575in}}%
\pgfpathlineto{\pgfqpoint{3.901573in}{2.105516in}}%
\pgfpathlineto{\pgfqpoint{3.904252in}{2.107747in}}%
\pgfpathlineto{\pgfqpoint{3.906918in}{2.103829in}}%
\pgfpathlineto{\pgfqpoint{3.909602in}{2.112425in}}%
\pgfpathlineto{\pgfqpoint{3.912296in}{2.109800in}}%
\pgfpathlineto{\pgfqpoint{3.915107in}{2.104148in}}%
\pgfpathlineto{\pgfqpoint{3.917646in}{2.106334in}}%
\pgfpathlineto{\pgfqpoint{3.920412in}{2.096953in}}%
\pgfpathlineto{\pgfqpoint{3.923005in}{2.102667in}}%
\pgfpathlineto{\pgfqpoint{3.925778in}{2.102825in}}%
\pgfpathlineto{\pgfqpoint{3.928347in}{2.105098in}}%
\pgfpathlineto{\pgfqpoint{3.931202in}{2.107706in}}%
\pgfpathlineto{\pgfqpoint{3.933714in}{2.103067in}}%
\pgfpathlineto{\pgfqpoint{3.936395in}{2.091486in}}%
\pgfpathlineto{\pgfqpoint{3.939075in}{2.094142in}}%
\pgfpathlineto{\pgfqpoint{3.941778in}{2.102120in}}%
\pgfpathlineto{\pgfqpoint{3.944431in}{2.105513in}}%
\pgfpathlineto{\pgfqpoint{3.947101in}{2.103690in}}%
\pgfpathlineto{\pgfqpoint{3.949894in}{2.098940in}}%
\pgfpathlineto{\pgfqpoint{3.952464in}{2.097702in}}%
\pgfpathlineto{\pgfqpoint{3.955211in}{2.102282in}}%
\pgfpathlineto{\pgfqpoint{3.957823in}{2.098586in}}%
\pgfpathlineto{\pgfqpoint{3.960635in}{2.108306in}}%
\pgfpathlineto{\pgfqpoint{3.963176in}{2.104432in}}%
\pgfpathlineto{\pgfqpoint{3.966013in}{2.105364in}}%
\pgfpathlineto{\pgfqpoint{3.968523in}{2.111132in}}%
\pgfpathlineto{\pgfqpoint{3.971250in}{2.110153in}}%
\pgfpathlineto{\pgfqpoint{3.973885in}{2.102170in}}%
\pgfpathlineto{\pgfqpoint{3.976563in}{2.104781in}}%
\pgfpathlineto{\pgfqpoint{3.979389in}{2.106575in}}%
\pgfpathlineto{\pgfqpoint{3.981929in}{2.112558in}}%
\pgfpathlineto{\pgfqpoint{3.984714in}{2.111500in}}%
\pgfpathlineto{\pgfqpoint{3.987270in}{2.110647in}}%
\pgfpathlineto{\pgfqpoint{3.990055in}{2.109991in}}%
\pgfpathlineto{\pgfqpoint{3.992642in}{2.099618in}}%
\pgfpathlineto{\pgfqpoint{3.995417in}{2.096280in}}%
\pgfpathlineto{\pgfqpoint{3.997990in}{2.098607in}}%
\pgfpathlineto{\pgfqpoint{4.000674in}{2.107249in}}%
\pgfpathlineto{\pgfqpoint{4.003348in}{2.112537in}}%
\pgfpathlineto{\pgfqpoint{4.006034in}{2.105574in}}%
\pgfpathlineto{\pgfqpoint{4.008699in}{2.106995in}}%
\pgfpathlineto{\pgfqpoint{4.011394in}{2.111199in}}%
\pgfpathlineto{\pgfqpoint{4.014186in}{2.113774in}}%
\pgfpathlineto{\pgfqpoint{4.016744in}{2.110959in}}%
\pgfpathlineto{\pgfqpoint{4.019518in}{2.118189in}}%
\pgfpathlineto{\pgfqpoint{4.022097in}{2.118923in}}%
\pgfpathlineto{\pgfqpoint{4.024868in}{2.116136in}}%
\pgfpathlineto{\pgfqpoint{4.027447in}{2.120412in}}%
\pgfpathlineto{\pgfqpoint{4.030229in}{2.116243in}}%
\pgfpathlineto{\pgfqpoint{4.032817in}{2.116345in}}%
\pgfpathlineto{\pgfqpoint{4.035492in}{2.123218in}}%
\pgfpathlineto{\pgfqpoint{4.038174in}{2.114106in}}%
\pgfpathlineto{\pgfqpoint{4.040852in}{2.113761in}}%
\pgfpathlineto{\pgfqpoint{4.043667in}{2.112846in}}%
\pgfpathlineto{\pgfqpoint{4.046210in}{2.108022in}}%
\pgfpathlineto{\pgfqpoint{4.049006in}{2.111917in}}%
\pgfpathlineto{\pgfqpoint{4.051557in}{2.110389in}}%
\pgfpathlineto{\pgfqpoint{4.054326in}{2.114186in}}%
\pgfpathlineto{\pgfqpoint{4.056911in}{2.110265in}}%
\pgfpathlineto{\pgfqpoint{4.059702in}{2.112347in}}%
\pgfpathlineto{\pgfqpoint{4.062266in}{2.107441in}}%
\pgfpathlineto{\pgfqpoint{4.064957in}{2.105438in}}%
\pgfpathlineto{\pgfqpoint{4.067636in}{2.106608in}}%
\pgfpathlineto{\pgfqpoint{4.070313in}{2.103997in}}%
\pgfpathlineto{\pgfqpoint{4.072985in}{2.098715in}}%
\pgfpathlineto{\pgfqpoint{4.075705in}{2.105098in}}%
\pgfpathlineto{\pgfqpoint{4.078471in}{2.094862in}}%
\pgfpathlineto{\pgfqpoint{4.081018in}{2.095138in}}%
\pgfpathlineto{\pgfqpoint{4.083870in}{2.103061in}}%
\pgfpathlineto{\pgfqpoint{4.086385in}{2.099632in}}%
\pgfpathlineto{\pgfqpoint{4.089159in}{2.094413in}}%
\pgfpathlineto{\pgfqpoint{4.091729in}{2.100370in}}%
\pgfpathlineto{\pgfqpoint{4.094527in}{2.097264in}}%
\pgfpathlineto{\pgfqpoint{4.097092in}{2.094111in}}%
\pgfpathlineto{\pgfqpoint{4.099777in}{2.093194in}}%
\pgfpathlineto{\pgfqpoint{4.102456in}{2.099529in}}%
\pgfpathlineto{\pgfqpoint{4.105185in}{2.109070in}}%
\pgfpathlineto{\pgfqpoint{4.107814in}{2.105431in}}%
\pgfpathlineto{\pgfqpoint{4.110488in}{2.108153in}}%
\pgfpathlineto{\pgfqpoint{4.113252in}{2.107332in}}%
\pgfpathlineto{\pgfqpoint{4.115844in}{2.103748in}}%
\pgfpathlineto{\pgfqpoint{4.118554in}{2.108837in}}%
\pgfpathlineto{\pgfqpoint{4.121205in}{2.102773in}}%
\pgfpathlineto{\pgfqpoint{4.124019in}{2.108053in}}%
\pgfpathlineto{\pgfqpoint{4.126553in}{2.110601in}}%
\pgfpathlineto{\pgfqpoint{4.129349in}{2.112216in}}%
\pgfpathlineto{\pgfqpoint{4.131920in}{2.106361in}}%
\pgfpathlineto{\pgfqpoint{4.134615in}{2.106004in}}%
\pgfpathlineto{\pgfqpoint{4.137272in}{2.103762in}}%
\pgfpathlineto{\pgfqpoint{4.139963in}{2.105646in}}%
\pgfpathlineto{\pgfqpoint{4.142713in}{2.108512in}}%
\pgfpathlineto{\pgfqpoint{4.145310in}{2.111404in}}%
\pgfpathlineto{\pgfqpoint{4.148082in}{2.109733in}}%
\pgfpathlineto{\pgfqpoint{4.150665in}{2.109915in}}%
\pgfpathlineto{\pgfqpoint{4.153423in}{2.109105in}}%
\pgfpathlineto{\pgfqpoint{4.156016in}{2.108814in}}%
\pgfpathlineto{\pgfqpoint{4.158806in}{2.104885in}}%
\pgfpathlineto{\pgfqpoint{4.161380in}{2.107388in}}%
\pgfpathlineto{\pgfqpoint{4.164059in}{2.114305in}}%
\pgfpathlineto{\pgfqpoint{4.166737in}{2.111797in}}%
\pgfpathlineto{\pgfqpoint{4.169415in}{2.109279in}}%
\pgfpathlineto{\pgfqpoint{4.172093in}{2.111110in}}%
\pgfpathlineto{\pgfqpoint{4.174770in}{2.106676in}}%
\pgfpathlineto{\pgfqpoint{4.177593in}{2.107262in}}%
\pgfpathlineto{\pgfqpoint{4.180129in}{2.109787in}}%
\pgfpathlineto{\pgfqpoint{4.182899in}{2.110710in}}%
\pgfpathlineto{\pgfqpoint{4.185481in}{2.110684in}}%
\pgfpathlineto{\pgfqpoint{4.188318in}{2.110867in}}%
\pgfpathlineto{\pgfqpoint{4.190842in}{2.112838in}}%
\pgfpathlineto{\pgfqpoint{4.193638in}{2.110147in}}%
\pgfpathlineto{\pgfqpoint{4.196186in}{2.112324in}}%
\pgfpathlineto{\pgfqpoint{4.198878in}{2.105779in}}%
\pgfpathlineto{\pgfqpoint{4.201542in}{2.106500in}}%
\pgfpathlineto{\pgfqpoint{4.204240in}{2.104437in}}%
\pgfpathlineto{\pgfqpoint{4.207076in}{2.107669in}}%
\pgfpathlineto{\pgfqpoint{4.209597in}{2.106497in}}%
\pgfpathlineto{\pgfqpoint{4.212383in}{2.111851in}}%
\pgfpathlineto{\pgfqpoint{4.214948in}{2.108761in}}%
\pgfpathlineto{\pgfqpoint{4.217694in}{2.105198in}}%
\pgfpathlineto{\pgfqpoint{4.220304in}{2.099852in}}%
\pgfpathlineto{\pgfqpoint{4.223082in}{2.097846in}}%
\pgfpathlineto{\pgfqpoint{4.225654in}{2.099053in}}%
\pgfpathlineto{\pgfqpoint{4.228331in}{2.099384in}}%
\pgfpathlineto{\pgfqpoint{4.231013in}{2.094598in}}%
\pgfpathlineto{\pgfqpoint{4.233691in}{2.093057in}}%
\pgfpathlineto{\pgfqpoint{4.236375in}{2.092608in}}%
\pgfpathlineto{\pgfqpoint{4.239084in}{2.095979in}}%
\pgfpathlineto{\pgfqpoint{4.241900in}{2.100955in}}%
\pgfpathlineto{\pgfqpoint{4.244394in}{2.098229in}}%
\pgfpathlineto{\pgfqpoint{4.247225in}{2.094810in}}%
\pgfpathlineto{\pgfqpoint{4.249767in}{2.100223in}}%
\pgfpathlineto{\pgfqpoint{4.252581in}{2.094033in}}%
\pgfpathlineto{\pgfqpoint{4.255120in}{2.088924in}}%
\pgfpathlineto{\pgfqpoint{4.257958in}{2.088125in}}%
\pgfpathlineto{\pgfqpoint{4.260477in}{2.092642in}}%
\pgfpathlineto{\pgfqpoint{4.263157in}{2.086947in}}%
\pgfpathlineto{\pgfqpoint{4.265824in}{2.087209in}}%
\pgfpathlineto{\pgfqpoint{4.268590in}{2.086947in}}%
\pgfpathlineto{\pgfqpoint{4.271187in}{2.086947in}}%
\pgfpathlineto{\pgfqpoint{4.273874in}{2.086947in}}%
\pgfpathlineto{\pgfqpoint{4.276635in}{2.086947in}}%
\pgfpathlineto{\pgfqpoint{4.279212in}{2.095381in}}%
\pgfpathlineto{\pgfqpoint{4.282000in}{2.103892in}}%
\pgfpathlineto{\pgfqpoint{4.284586in}{2.109732in}}%
\pgfpathlineto{\pgfqpoint{4.287399in}{2.105208in}}%
\pgfpathlineto{\pgfqpoint{4.289936in}{2.094813in}}%
\pgfpathlineto{\pgfqpoint{4.292786in}{2.094325in}}%
\pgfpathlineto{\pgfqpoint{4.295299in}{2.091135in}}%
\pgfpathlineto{\pgfqpoint{4.297977in}{2.095582in}}%
\pgfpathlineto{\pgfqpoint{4.300656in}{2.101694in}}%
\pgfpathlineto{\pgfqpoint{4.303357in}{2.103746in}}%
\pgfpathlineto{\pgfqpoint{4.306118in}{2.106022in}}%
\pgfpathlineto{\pgfqpoint{4.308691in}{2.105131in}}%
\pgfpathlineto{\pgfqpoint{4.311494in}{2.107048in}}%
\pgfpathlineto{\pgfqpoint{4.314032in}{2.104177in}}%
\pgfpathlineto{\pgfqpoint{4.316856in}{2.105799in}}%
\pgfpathlineto{\pgfqpoint{4.319405in}{2.108773in}}%
\pgfpathlineto{\pgfqpoint{4.322181in}{2.107383in}}%
\pgfpathlineto{\pgfqpoint{4.324760in}{2.109907in}}%
\pgfpathlineto{\pgfqpoint{4.327440in}{2.103780in}}%
\pgfpathlineto{\pgfqpoint{4.330118in}{2.108205in}}%
\pgfpathlineto{\pgfqpoint{4.332796in}{2.104912in}}%
\pgfpathlineto{\pgfqpoint{4.335463in}{2.107339in}}%
\pgfpathlineto{\pgfqpoint{4.338154in}{2.110704in}}%
\pgfpathlineto{\pgfqpoint{4.340976in}{2.109577in}}%
\pgfpathlineto{\pgfqpoint{4.343510in}{2.101139in}}%
\pgfpathlineto{\pgfqpoint{4.346263in}{2.105235in}}%
\pgfpathlineto{\pgfqpoint{4.348868in}{2.101833in}}%
\pgfpathlineto{\pgfqpoint{4.351645in}{2.104737in}}%
\pgfpathlineto{\pgfqpoint{4.354224in}{2.102947in}}%
\pgfpathlineto{\pgfqpoint{4.357014in}{2.111151in}}%
\pgfpathlineto{\pgfqpoint{4.359582in}{2.108117in}}%
\pgfpathlineto{\pgfqpoint{4.362270in}{2.109425in}}%
\pgfpathlineto{\pgfqpoint{4.364936in}{2.107676in}}%
\pgfpathlineto{\pgfqpoint{4.367646in}{2.107057in}}%
\pgfpathlineto{\pgfqpoint{4.370437in}{2.108365in}}%
\pgfpathlineto{\pgfqpoint{4.372976in}{2.109644in}}%
\pgfpathlineto{\pgfqpoint{4.375761in}{2.109936in}}%
\pgfpathlineto{\pgfqpoint{4.378329in}{2.113779in}}%
\pgfpathlineto{\pgfqpoint{4.381097in}{2.111512in}}%
\pgfpathlineto{\pgfqpoint{4.383674in}{2.109610in}}%
\pgfpathlineto{\pgfqpoint{4.386431in}{2.115943in}}%
\pgfpathlineto{\pgfqpoint{4.389044in}{2.110564in}}%
\pgfpathlineto{\pgfqpoint{4.391721in}{2.111527in}}%
\pgfpathlineto{\pgfqpoint{4.394400in}{2.106015in}}%
\pgfpathlineto{\pgfqpoint{4.397076in}{2.101060in}}%
\pgfpathlineto{\pgfqpoint{4.399745in}{2.102047in}}%
\pgfpathlineto{\pgfqpoint{4.402468in}{2.100616in}}%
\pgfpathlineto{\pgfqpoint{4.405234in}{2.104085in}}%
\pgfpathlineto{\pgfqpoint{4.407788in}{2.102492in}}%
\pgfpathlineto{\pgfqpoint{4.410587in}{2.101984in}}%
\pgfpathlineto{\pgfqpoint{4.413149in}{2.108014in}}%
\pgfpathlineto{\pgfqpoint{4.415932in}{2.103734in}}%
\pgfpathlineto{\pgfqpoint{4.418506in}{2.106809in}}%
\pgfpathlineto{\pgfqpoint{4.421292in}{2.107036in}}%
\pgfpathlineto{\pgfqpoint{4.423863in}{2.105440in}}%
\pgfpathlineto{\pgfqpoint{4.426534in}{2.107591in}}%
\pgfpathlineto{\pgfqpoint{4.429220in}{2.107526in}}%
\pgfpathlineto{\pgfqpoint{4.431901in}{2.105317in}}%
\pgfpathlineto{\pgfqpoint{4.434569in}{2.105838in}}%
\pgfpathlineto{\pgfqpoint{4.437253in}{2.109566in}}%
\pgfpathlineto{\pgfqpoint{4.440041in}{2.109884in}}%
\pgfpathlineto{\pgfqpoint{4.442611in}{2.109707in}}%
\pgfpathlineto{\pgfqpoint{4.445423in}{2.095011in}}%
\pgfpathlineto{\pgfqpoint{4.447965in}{2.102754in}}%
\pgfpathlineto{\pgfqpoint{4.450767in}{2.100888in}}%
\pgfpathlineto{\pgfqpoint{4.453312in}{2.099059in}}%
\pgfpathlineto{\pgfqpoint{4.456138in}{2.095880in}}%
\pgfpathlineto{\pgfqpoint{4.458681in}{2.096476in}}%
\pgfpathlineto{\pgfqpoint{4.461367in}{2.095731in}}%
\pgfpathlineto{\pgfqpoint{4.464029in}{2.086947in}}%
\pgfpathlineto{\pgfqpoint{4.466717in}{2.087159in}}%
\pgfpathlineto{\pgfqpoint{4.469492in}{2.087567in}}%
\pgfpathlineto{\pgfqpoint{4.472059in}{2.086947in}}%
\pgfpathlineto{\pgfqpoint{4.474861in}{2.087434in}}%
\pgfpathlineto{\pgfqpoint{4.477430in}{2.086947in}}%
\pgfpathlineto{\pgfqpoint{4.480201in}{2.092773in}}%
\pgfpathlineto{\pgfqpoint{4.482778in}{2.098785in}}%
\pgfpathlineto{\pgfqpoint{4.485581in}{2.099159in}}%
\pgfpathlineto{\pgfqpoint{4.488130in}{2.096202in}}%
\pgfpathlineto{\pgfqpoint{4.490822in}{2.100383in}}%
\pgfpathlineto{\pgfqpoint{4.493492in}{2.106396in}}%
\pgfpathlineto{\pgfqpoint{4.496167in}{2.104220in}}%
\pgfpathlineto{\pgfqpoint{4.498850in}{2.102061in}}%
\pgfpathlineto{\pgfqpoint{4.501529in}{2.104901in}}%
\pgfpathlineto{\pgfqpoint{4.504305in}{2.105653in}}%
\pgfpathlineto{\pgfqpoint{4.506893in}{2.111304in}}%
\pgfpathlineto{\pgfqpoint{4.509643in}{2.111469in}}%
\pgfpathlineto{\pgfqpoint{4.512246in}{2.105537in}}%
\pgfpathlineto{\pgfqpoint{4.515080in}{2.101201in}}%
\pgfpathlineto{\pgfqpoint{4.517598in}{2.097697in}}%
\pgfpathlineto{\pgfqpoint{4.520345in}{2.102288in}}%
\pgfpathlineto{\pgfqpoint{4.522962in}{2.112963in}}%
\pgfpathlineto{\pgfqpoint{4.525640in}{2.105809in}}%
\pgfpathlineto{\pgfqpoint{4.528307in}{2.106467in}}%
\pgfpathlineto{\pgfqpoint{4.530990in}{2.103348in}}%
\pgfpathlineto{\pgfqpoint{4.533764in}{2.114123in}}%
\pgfpathlineto{\pgfqpoint{4.536400in}{2.117972in}}%
\pgfpathlineto{\pgfqpoint{4.539144in}{2.107158in}}%
\pgfpathlineto{\pgfqpoint{4.541711in}{2.108616in}}%
\pgfpathlineto{\pgfqpoint{4.544464in}{2.105836in}}%
\pgfpathlineto{\pgfqpoint{4.547064in}{2.101882in}}%
\pgfpathlineto{\pgfqpoint{4.549822in}{2.107238in}}%
\pgfpathlineto{\pgfqpoint{4.552425in}{2.106331in}}%
\pgfpathlineto{\pgfqpoint{4.555106in}{2.103544in}}%
\pgfpathlineto{\pgfqpoint{4.557777in}{2.105668in}}%
\pgfpathlineto{\pgfqpoint{4.560448in}{2.110207in}}%
\pgfpathlineto{\pgfqpoint{4.563125in}{2.107317in}}%
\pgfpathlineto{\pgfqpoint{4.565820in}{2.106387in}}%
\pgfpathlineto{\pgfqpoint{4.568612in}{2.114020in}}%
\pgfpathlineto{\pgfqpoint{4.571171in}{2.108991in}}%
\pgfpathlineto{\pgfqpoint{4.573947in}{2.107411in}}%
\pgfpathlineto{\pgfqpoint{4.576531in}{2.107006in}}%
\pgfpathlineto{\pgfqpoint{4.579305in}{2.107597in}}%
\pgfpathlineto{\pgfqpoint{4.581888in}{2.106536in}}%
\pgfpathlineto{\pgfqpoint{4.584672in}{2.104178in}}%
\pgfpathlineto{\pgfqpoint{4.587244in}{2.106761in}}%
\pgfpathlineto{\pgfqpoint{4.589920in}{2.107320in}}%
\pgfpathlineto{\pgfqpoint{4.592589in}{2.116211in}}%
\pgfpathlineto{\pgfqpoint{4.595281in}{2.106349in}}%
\pgfpathlineto{\pgfqpoint{4.597951in}{2.112849in}}%
\pgfpathlineto{\pgfqpoint{4.600633in}{2.105744in}}%
\pgfpathlineto{\pgfqpoint{4.603430in}{2.103683in}}%
\pgfpathlineto{\pgfqpoint{4.605990in}{2.107627in}}%
\pgfpathlineto{\pgfqpoint{4.608808in}{2.106726in}}%
\pgfpathlineto{\pgfqpoint{4.611350in}{2.109673in}}%
\pgfpathlineto{\pgfqpoint{4.614134in}{2.109618in}}%
\pgfpathlineto{\pgfqpoint{4.616702in}{2.109843in}}%
\pgfpathlineto{\pgfqpoint{4.619529in}{2.110986in}}%
\pgfpathlineto{\pgfqpoint{4.622056in}{2.110972in}}%
\pgfpathlineto{\pgfqpoint{4.624741in}{2.112482in}}%
\pgfpathlineto{\pgfqpoint{4.627411in}{2.110836in}}%
\pgfpathlineto{\pgfqpoint{4.630096in}{2.106790in}}%
\pgfpathlineto{\pgfqpoint{4.632902in}{2.109148in}}%
\pgfpathlineto{\pgfqpoint{4.635445in}{2.107640in}}%
\pgfpathlineto{\pgfqpoint{4.638204in}{2.108546in}}%
\pgfpathlineto{\pgfqpoint{4.640809in}{2.110490in}}%
\pgfpathlineto{\pgfqpoint{4.643628in}{2.107968in}}%
\pgfpathlineto{\pgfqpoint{4.646169in}{2.107879in}}%
\pgfpathlineto{\pgfqpoint{4.648922in}{2.108756in}}%
\pgfpathlineto{\pgfqpoint{4.651524in}{2.104210in}}%
\pgfpathlineto{\pgfqpoint{4.654203in}{2.104123in}}%
\pgfpathlineto{\pgfqpoint{4.656873in}{2.108227in}}%
\pgfpathlineto{\pgfqpoint{4.659590in}{2.102662in}}%
\pgfpathlineto{\pgfqpoint{4.662237in}{2.115869in}}%
\pgfpathlineto{\pgfqpoint{4.664923in}{2.106538in}}%
\pgfpathlineto{\pgfqpoint{4.667764in}{2.110465in}}%
\pgfpathlineto{\pgfqpoint{4.670261in}{2.111540in}}%
\pgfpathlineto{\pgfqpoint{4.673068in}{2.109891in}}%
\pgfpathlineto{\pgfqpoint{4.675619in}{2.111305in}}%
\pgfpathlineto{\pgfqpoint{4.678448in}{2.106616in}}%
\pgfpathlineto{\pgfqpoint{4.680988in}{2.108666in}}%
\pgfpathlineto{\pgfqpoint{4.683799in}{2.108375in}}%
\pgfpathlineto{\pgfqpoint{4.686337in}{2.107118in}}%
\pgfpathlineto{\pgfqpoint{4.689051in}{2.111668in}}%
\pgfpathlineto{\pgfqpoint{4.691694in}{2.113211in}}%
\pgfpathlineto{\pgfqpoint{4.694381in}{2.107059in}}%
\pgfpathlineto{\pgfqpoint{4.697170in}{2.108157in}}%
\pgfpathlineto{\pgfqpoint{4.699734in}{2.107832in}}%
\pgfpathlineto{\pgfqpoint{4.702517in}{2.110527in}}%
\pgfpathlineto{\pgfqpoint{4.705094in}{2.109028in}}%
\pgfpathlineto{\pgfqpoint{4.707824in}{2.111635in}}%
\pgfpathlineto{\pgfqpoint{4.710437in}{2.111377in}}%
\pgfpathlineto{\pgfqpoint{4.713275in}{2.107568in}}%
\pgfpathlineto{\pgfqpoint{4.715806in}{2.111006in}}%
\pgfpathlineto{\pgfqpoint{4.718486in}{2.107493in}}%
\pgfpathlineto{\pgfqpoint{4.721160in}{2.108347in}}%
\pgfpathlineto{\pgfqpoint{4.723873in}{2.111422in}}%
\pgfpathlineto{\pgfqpoint{4.726508in}{2.116725in}}%
\pgfpathlineto{\pgfqpoint{4.729233in}{2.110929in}}%
\pgfpathlineto{\pgfqpoint{4.731901in}{2.106078in}}%
\pgfpathlineto{\pgfqpoint{4.734552in}{2.100868in}}%
\pgfpathlineto{\pgfqpoint{4.737348in}{2.099177in}}%
\pgfpathlineto{\pgfqpoint{4.739912in}{2.103127in}}%
\pgfpathlineto{\pgfqpoint{4.742696in}{2.101448in}}%
\pgfpathlineto{\pgfqpoint{4.745256in}{2.092950in}}%
\pgfpathlineto{\pgfqpoint{4.748081in}{2.089750in}}%
\pgfpathlineto{\pgfqpoint{4.750627in}{2.095196in}}%
\pgfpathlineto{\pgfqpoint{4.753298in}{2.096501in}}%
\pgfpathlineto{\pgfqpoint{4.755983in}{2.096575in}}%
\pgfpathlineto{\pgfqpoint{4.758653in}{2.086947in}}%
\pgfpathlineto{\pgfqpoint{4.761337in}{2.086947in}}%
\pgfpathlineto{\pgfqpoint{4.764018in}{2.086947in}}%
\pgfpathlineto{\pgfqpoint{4.766783in}{2.086947in}}%
\pgfpathlineto{\pgfqpoint{4.769367in}{2.086947in}}%
\pgfpathlineto{\pgfqpoint{4.772198in}{2.093095in}}%
\pgfpathlineto{\pgfqpoint{4.774732in}{2.094392in}}%
\pgfpathlineto{\pgfqpoint{4.777535in}{2.091405in}}%
\pgfpathlineto{\pgfqpoint{4.780083in}{2.091530in}}%
\pgfpathlineto{\pgfqpoint{4.782872in}{2.086947in}}%
\pgfpathlineto{\pgfqpoint{4.785445in}{2.087762in}}%
\pgfpathlineto{\pgfqpoint{4.788116in}{2.088566in}}%
\pgfpathlineto{\pgfqpoint{4.790798in}{2.094988in}}%
\pgfpathlineto{\pgfqpoint{4.793512in}{2.111904in}}%
\pgfpathlineto{\pgfqpoint{4.796274in}{2.106158in}}%
\pgfpathlineto{\pgfqpoint{4.798830in}{2.112276in}}%
\pgfpathlineto{\pgfqpoint{4.801586in}{2.109219in}}%
\pgfpathlineto{\pgfqpoint{4.804193in}{2.107716in}}%
\pgfpathlineto{\pgfqpoint{4.807017in}{2.103325in}}%
\pgfpathlineto{\pgfqpoint{4.809538in}{2.105947in}}%
\pgfpathlineto{\pgfqpoint{4.812377in}{2.108904in}}%
\pgfpathlineto{\pgfqpoint{4.814907in}{2.096838in}}%
\pgfpathlineto{\pgfqpoint{4.817587in}{2.088714in}}%
\pgfpathlineto{\pgfqpoint{4.820265in}{2.086947in}}%
\pgfpathlineto{\pgfqpoint{4.822945in}{2.086947in}}%
\pgfpathlineto{\pgfqpoint{4.825619in}{2.086947in}}%
\pgfpathlineto{\pgfqpoint{4.828291in}{2.093929in}}%
\pgfpathlineto{\pgfqpoint{4.831045in}{2.102467in}}%
\pgfpathlineto{\pgfqpoint{4.833657in}{2.114097in}}%
\pgfpathlineto{\pgfqpoint{4.837992in}{2.118013in}}%
\pgfpathlineto{\pgfqpoint{4.839922in}{2.112538in}}%
\pgfpathlineto{\pgfqpoint{4.842380in}{2.109002in}}%
\pgfpathlineto{\pgfqpoint{4.844361in}{2.110602in}}%
\pgfpathlineto{\pgfqpoint{4.847127in}{2.107072in}}%
\pgfpathlineto{\pgfqpoint{4.849715in}{2.104686in}}%
\pgfpathlineto{\pgfqpoint{4.852404in}{2.104392in}}%
\pgfpathlineto{\pgfqpoint{4.855070in}{2.109514in}}%
\pgfpathlineto{\pgfqpoint{4.857807in}{2.110869in}}%
\pgfpathlineto{\pgfqpoint{4.860544in}{2.105229in}}%
\pgfpathlineto{\pgfqpoint{4.863116in}{2.106185in}}%
\pgfpathlineto{\pgfqpoint{4.865910in}{2.106534in}}%
\pgfpathlineto{\pgfqpoint{4.868474in}{2.103240in}}%
\pgfpathlineto{\pgfqpoint{4.871209in}{2.104046in}}%
\pgfpathlineto{\pgfqpoint{4.873832in}{2.107902in}}%
\pgfpathlineto{\pgfqpoint{4.876636in}{2.102994in}}%
\pgfpathlineto{\pgfqpoint{4.879180in}{2.104938in}}%
\pgfpathlineto{\pgfqpoint{4.881864in}{2.098787in}}%
\pgfpathlineto{\pgfqpoint{4.884540in}{2.098367in}}%
\pgfpathlineto{\pgfqpoint{4.887211in}{2.093765in}}%
\pgfpathlineto{\pgfqpoint{4.889902in}{2.093976in}}%
\pgfpathlineto{\pgfqpoint{4.892611in}{2.095675in}}%
\pgfpathlineto{\pgfqpoint{4.895399in}{2.098556in}}%
\pgfpathlineto{\pgfqpoint{4.897938in}{2.089856in}}%
\pgfpathlineto{\pgfqpoint{4.900712in}{2.089844in}}%
\pgfpathlineto{\pgfqpoint{4.903295in}{2.094145in}}%
\pgfpathlineto{\pgfqpoint{4.906096in}{2.091164in}}%
\pgfpathlineto{\pgfqpoint{4.908648in}{2.093603in}}%
\pgfpathlineto{\pgfqpoint{4.911435in}{2.104390in}}%
\pgfpathlineto{\pgfqpoint{4.914009in}{2.098178in}}%
\pgfpathlineto{\pgfqpoint{4.916681in}{2.091858in}}%
\pgfpathlineto{\pgfqpoint{4.919352in}{2.089963in}}%
\pgfpathlineto{\pgfqpoint{4.922041in}{2.086947in}}%
\pgfpathlineto{\pgfqpoint{4.924708in}{2.090760in}}%
\pgfpathlineto{\pgfqpoint{4.927400in}{2.086947in}}%
\pgfpathlineto{\pgfqpoint{4.930170in}{2.086947in}}%
\pgfpathlineto{\pgfqpoint{4.932742in}{2.086947in}}%
\pgfpathlineto{\pgfqpoint{4.935515in}{2.101574in}}%
\pgfpathlineto{\pgfqpoint{4.938112in}{2.110671in}}%
\pgfpathlineto{\pgfqpoint{4.940881in}{2.106802in}}%
\pgfpathlineto{\pgfqpoint{4.943466in}{2.105273in}}%
\pgfpathlineto{\pgfqpoint{4.946151in}{2.104983in}}%
\pgfpathlineto{\pgfqpoint{4.948827in}{2.102545in}}%
\pgfpathlineto{\pgfqpoint{4.951504in}{2.098150in}}%
\pgfpathlineto{\pgfqpoint{4.954182in}{2.098835in}}%
\pgfpathlineto{\pgfqpoint{4.956862in}{2.112266in}}%
\pgfpathlineto{\pgfqpoint{4.959689in}{2.106852in}}%
\pgfpathlineto{\pgfqpoint{4.962219in}{2.103482in}}%
\pgfpathlineto{\pgfqpoint{4.965002in}{2.105708in}}%
\pgfpathlineto{\pgfqpoint{4.967575in}{2.109528in}}%
\pgfpathlineto{\pgfqpoint{4.970314in}{2.108704in}}%
\pgfpathlineto{\pgfqpoint{4.972933in}{2.114459in}}%
\pgfpathlineto{\pgfqpoint{4.975703in}{2.108042in}}%
\pgfpathlineto{\pgfqpoint{4.978287in}{2.111209in}}%
\pgfpathlineto{\pgfqpoint{4.980967in}{2.110046in}}%
\pgfpathlineto{\pgfqpoint{4.983637in}{2.110151in}}%
\pgfpathlineto{\pgfqpoint{4.986325in}{2.108261in}}%
\pgfpathlineto{\pgfqpoint{4.989001in}{2.108485in}}%
\pgfpathlineto{\pgfqpoint{4.991683in}{2.106415in}}%
\pgfpathlineto{\pgfqpoint{4.994390in}{2.109645in}}%
\pgfpathlineto{\pgfqpoint{4.997028in}{2.105926in}}%
\pgfpathlineto{\pgfqpoint{4.999780in}{2.105925in}}%
\pgfpathlineto{\pgfqpoint{5.002384in}{2.107335in}}%
\pgfpathlineto{\pgfqpoint{5.005178in}{2.105723in}}%
\pgfpathlineto{\pgfqpoint{5.007751in}{2.102136in}}%
\pgfpathlineto{\pgfqpoint{5.010562in}{2.108046in}}%
\pgfpathlineto{\pgfqpoint{5.013104in}{2.109693in}}%
\pgfpathlineto{\pgfqpoint{5.015820in}{2.101433in}}%
\pgfpathlineto{\pgfqpoint{5.018466in}{2.108087in}}%
\pgfpathlineto{\pgfqpoint{5.021147in}{2.110309in}}%
\pgfpathlineto{\pgfqpoint{5.023927in}{2.113813in}}%
\pgfpathlineto{\pgfqpoint{5.026501in}{2.112615in}}%
\pgfpathlineto{\pgfqpoint{5.029275in}{2.110707in}}%
\pgfpathlineto{\pgfqpoint{5.031849in}{2.107768in}}%
\pgfpathlineto{\pgfqpoint{5.034649in}{2.103672in}}%
\pgfpathlineto{\pgfqpoint{5.037214in}{2.106596in}}%
\pgfpathlineto{\pgfqpoint{5.039962in}{2.109980in}}%
\pgfpathlineto{\pgfqpoint{5.042572in}{2.107961in}}%
\pgfpathlineto{\pgfqpoint{5.045249in}{2.112063in}}%
\pgfpathlineto{\pgfqpoint{5.047924in}{2.114273in}}%
\pgfpathlineto{\pgfqpoint{5.050606in}{2.107607in}}%
\pgfpathlineto{\pgfqpoint{5.053284in}{2.105915in}}%
\pgfpathlineto{\pgfqpoint{5.055952in}{2.106162in}}%
\pgfpathlineto{\pgfqpoint{5.058711in}{2.108195in}}%
\pgfpathlineto{\pgfqpoint{5.061315in}{2.104971in}}%
\pgfpathlineto{\pgfqpoint{5.064144in}{2.107285in}}%
\pgfpathlineto{\pgfqpoint{5.066677in}{2.103511in}}%
\pgfpathlineto{\pgfqpoint{5.069463in}{2.108752in}}%
\pgfpathlineto{\pgfqpoint{5.072030in}{2.104093in}}%
\pgfpathlineto{\pgfqpoint{5.074851in}{2.105248in}}%
\pgfpathlineto{\pgfqpoint{5.077390in}{2.110312in}}%
\pgfpathlineto{\pgfqpoint{5.080067in}{2.107304in}}%
\pgfpathlineto{\pgfqpoint{5.082746in}{2.112079in}}%
\pgfpathlineto{\pgfqpoint{5.085426in}{2.113090in}}%
\pgfpathlineto{\pgfqpoint{5.088103in}{2.112564in}}%
\pgfpathlineto{\pgfqpoint{5.090788in}{2.108155in}}%
\pgfpathlineto{\pgfqpoint{5.093579in}{2.104091in}}%
\pgfpathlineto{\pgfqpoint{5.096142in}{2.103147in}}%
\pgfpathlineto{\pgfqpoint{5.098948in}{2.104350in}}%
\pgfpathlineto{\pgfqpoint{5.101496in}{2.106764in}}%
\pgfpathlineto{\pgfqpoint{5.104312in}{2.107239in}}%
\pgfpathlineto{\pgfqpoint{5.106842in}{2.110326in}}%
\pgfpathlineto{\pgfqpoint{5.109530in}{2.111487in}}%
\pgfpathlineto{\pgfqpoint{5.112209in}{2.109354in}}%
\pgfpathlineto{\pgfqpoint{5.114887in}{2.111459in}}%
\pgfpathlineto{\pgfqpoint{5.117550in}{2.108111in}}%
\pgfpathlineto{\pgfqpoint{5.120243in}{2.113285in}}%
\pgfpathlineto{\pgfqpoint{5.123042in}{2.110141in}}%
\pgfpathlineto{\pgfqpoint{5.125599in}{2.106374in}}%
\pgfpathlineto{\pgfqpoint{5.128421in}{2.110295in}}%
\pgfpathlineto{\pgfqpoint{5.130953in}{2.111055in}}%
\pgfpathlineto{\pgfqpoint{5.133716in}{2.107583in}}%
\pgfpathlineto{\pgfqpoint{5.136311in}{2.108691in}}%
\pgfpathlineto{\pgfqpoint{5.139072in}{2.105377in}}%
\pgfpathlineto{\pgfqpoint{5.141660in}{2.106375in}}%
\pgfpathlineto{\pgfqpoint{5.144349in}{2.105993in}}%
\pgfpathlineto{\pgfqpoint{5.147029in}{2.112553in}}%
\pgfpathlineto{\pgfqpoint{5.149734in}{2.175031in}}%
\pgfpathlineto{\pgfqpoint{5.152382in}{2.281302in}}%
\pgfpathlineto{\pgfqpoint{5.155059in}{2.259366in}}%
\pgfpathlineto{\pgfqpoint{5.157815in}{2.219254in}}%
\pgfpathlineto{\pgfqpoint{5.160420in}{2.198283in}}%
\pgfpathlineto{\pgfqpoint{5.163243in}{2.186410in}}%
\pgfpathlineto{\pgfqpoint{5.165775in}{2.174018in}}%
\pgfpathlineto{\pgfqpoint{5.168591in}{2.170937in}}%
\pgfpathlineto{\pgfqpoint{5.171133in}{2.159889in}}%
\pgfpathlineto{\pgfqpoint{5.173925in}{2.158259in}}%
\pgfpathlineto{\pgfqpoint{5.176477in}{2.149730in}}%
\pgfpathlineto{\pgfqpoint{5.179188in}{2.140497in}}%
\pgfpathlineto{\pgfqpoint{5.181848in}{2.130935in}}%
\pgfpathlineto{\pgfqpoint{5.184522in}{2.127240in}}%
\pgfpathlineto{\pgfqpoint{5.187294in}{2.125379in}}%
\pgfpathlineto{\pgfqpoint{5.189880in}{2.119475in}}%
\pgfpathlineto{\pgfqpoint{5.192680in}{2.118486in}}%
\pgfpathlineto{\pgfqpoint{5.195239in}{2.114343in}}%
\pgfpathlineto{\pgfqpoint{5.198008in}{2.102584in}}%
\pgfpathlineto{\pgfqpoint{5.200594in}{2.098459in}}%
\pgfpathlineto{\pgfqpoint{5.203388in}{2.100084in}}%
\pgfpathlineto{\pgfqpoint{5.205952in}{2.092415in}}%
\pgfpathlineto{\pgfqpoint{5.208630in}{2.093314in}}%
\pgfpathlineto{\pgfqpoint{5.211299in}{2.103043in}}%
\pgfpathlineto{\pgfqpoint{5.214027in}{2.098007in}}%
\pgfpathlineto{\pgfqpoint{5.216667in}{2.099998in}}%
\pgfpathlineto{\pgfqpoint{5.219345in}{2.100635in}}%
\pgfpathlineto{\pgfqpoint{5.222151in}{2.098473in}}%
\pgfpathlineto{\pgfqpoint{5.224695in}{2.109227in}}%
\pgfpathlineto{\pgfqpoint{5.227470in}{2.112551in}}%
\pgfpathlineto{\pgfqpoint{5.230059in}{2.110267in}}%
\pgfpathlineto{\pgfqpoint{5.232855in}{2.112260in}}%
\pgfpathlineto{\pgfqpoint{5.235409in}{2.114684in}}%
\pgfpathlineto{\pgfqpoint{5.238173in}{2.110332in}}%
\pgfpathlineto{\pgfqpoint{5.240777in}{2.116049in}}%
\pgfpathlineto{\pgfqpoint{5.243445in}{2.113366in}}%
\pgfpathlineto{\pgfqpoint{5.246130in}{2.112004in}}%
\pgfpathlineto{\pgfqpoint{5.248816in}{2.108250in}}%
\pgfpathlineto{\pgfqpoint{5.251590in}{2.120010in}}%
\pgfpathlineto{\pgfqpoint{5.254236in}{2.113382in}}%
\pgfpathlineto{\pgfqpoint{5.256973in}{2.117916in}}%
\pgfpathlineto{\pgfqpoint{5.259511in}{2.108727in}}%
\pgfpathlineto{\pgfqpoint{5.262264in}{2.116260in}}%
\pgfpathlineto{\pgfqpoint{5.264876in}{2.109293in}}%
\pgfpathlineto{\pgfqpoint{5.267691in}{2.109931in}}%
\pgfpathlineto{\pgfqpoint{5.270238in}{2.108040in}}%
\pgfpathlineto{\pgfqpoint{5.272913in}{2.108680in}}%
\pgfpathlineto{\pgfqpoint{5.275589in}{2.103568in}}%
\pgfpathlineto{\pgfqpoint{5.278322in}{2.099395in}}%
\pgfpathlineto{\pgfqpoint{5.280947in}{2.100711in}}%
\pgfpathlineto{\pgfqpoint{5.283631in}{2.107742in}}%
\pgfpathlineto{\pgfqpoint{5.286436in}{2.102388in}}%
\pgfpathlineto{\pgfqpoint{5.288984in}{2.098375in}}%
\pgfpathlineto{\pgfqpoint{5.291794in}{2.102375in}}%
\pgfpathlineto{\pgfqpoint{5.294339in}{2.104202in}}%
\pgfpathlineto{\pgfqpoint{5.297140in}{2.103567in}}%
\pgfpathlineto{\pgfqpoint{5.299696in}{2.101827in}}%
\pgfpathlineto{\pgfqpoint{5.302443in}{2.103602in}}%
\pgfpathlineto{\pgfqpoint{5.305054in}{2.101161in}}%
\pgfpathlineto{\pgfqpoint{5.307731in}{2.099580in}}%
\pgfpathlineto{\pgfqpoint{5.310411in}{2.101422in}}%
\pgfpathlineto{\pgfqpoint{5.313089in}{2.101641in}}%
\pgfpathlineto{\pgfqpoint{5.315754in}{2.093745in}}%
\pgfpathlineto{\pgfqpoint{5.318430in}{2.092127in}}%
\pgfpathlineto{\pgfqpoint{5.321256in}{2.099143in}}%
\pgfpathlineto{\pgfqpoint{5.323802in}{2.097991in}}%
\pgfpathlineto{\pgfqpoint{5.326564in}{2.097390in}}%
\pgfpathlineto{\pgfqpoint{5.329159in}{2.099873in}}%
\pgfpathlineto{\pgfqpoint{5.331973in}{2.086947in}}%
\pgfpathlineto{\pgfqpoint{5.334510in}{2.086947in}}%
\pgfpathlineto{\pgfqpoint{5.337353in}{2.086947in}}%
\pgfpathlineto{\pgfqpoint{5.339872in}{2.086947in}}%
\pgfpathlineto{\pgfqpoint{5.342549in}{2.088654in}}%
\pgfpathlineto{\pgfqpoint{5.345224in}{2.087693in}}%
\pgfpathlineto{\pgfqpoint{5.347905in}{2.091250in}}%
\pgfpathlineto{\pgfqpoint{5.350723in}{2.095293in}}%
\pgfpathlineto{\pgfqpoint{5.353262in}{2.086947in}}%
\pgfpathlineto{\pgfqpoint{5.356056in}{2.086947in}}%
\pgfpathlineto{\pgfqpoint{5.358612in}{2.086947in}}%
\pgfpathlineto{\pgfqpoint{5.361370in}{2.093585in}}%
\pgfpathlineto{\pgfqpoint{5.363966in}{2.094444in}}%
\pgfpathlineto{\pgfqpoint{5.366727in}{2.098617in}}%
\pgfpathlineto{\pgfqpoint{5.369335in}{2.107428in}}%
\pgfpathlineto{\pgfqpoint{5.372013in}{2.109369in}}%
\pgfpathlineto{\pgfqpoint{5.374692in}{2.108934in}}%
\pgfpathlineto{\pgfqpoint{5.377370in}{2.110769in}}%
\pgfpathlineto{\pgfqpoint{5.380048in}{2.115965in}}%
\pgfpathlineto{\pgfqpoint{5.382725in}{2.114332in}}%
\pgfpathlineto{\pgfqpoint{5.385550in}{2.113155in}}%
\pgfpathlineto{\pgfqpoint{5.388083in}{2.112404in}}%
\pgfpathlineto{\pgfqpoint{5.390900in}{2.102730in}}%
\pgfpathlineto{\pgfqpoint{5.393441in}{2.106591in}}%
\pgfpathlineto{\pgfqpoint{5.396219in}{2.106413in}}%
\pgfpathlineto{\pgfqpoint{5.398784in}{2.107647in}}%
\pgfpathlineto{\pgfqpoint{5.401576in}{2.108311in}}%
\pgfpathlineto{\pgfqpoint{5.404154in}{2.105925in}}%
\pgfpathlineto{\pgfqpoint{5.406832in}{2.107778in}}%
\pgfpathlineto{\pgfqpoint{5.409507in}{2.103606in}}%
\pgfpathlineto{\pgfqpoint{5.412190in}{2.108191in}}%
\pgfpathlineto{\pgfqpoint{5.414954in}{2.106982in}}%
\pgfpathlineto{\pgfqpoint{5.417547in}{2.110606in}}%
\pgfpathlineto{\pgfqpoint{5.420304in}{2.106894in}}%
\pgfpathlineto{\pgfqpoint{5.422897in}{2.107222in}}%
\pgfpathlineto{\pgfqpoint{5.425661in}{2.106462in}}%
\pgfpathlineto{\pgfqpoint{5.428259in}{2.108354in}}%
\pgfpathlineto{\pgfqpoint{5.431015in}{2.110889in}}%
\pgfpathlineto{\pgfqpoint{5.433616in}{2.108026in}}%
\pgfpathlineto{\pgfqpoint{5.436295in}{2.103120in}}%
\pgfpathlineto{\pgfqpoint{5.438974in}{2.110968in}}%
\pgfpathlineto{\pgfqpoint{5.441698in}{2.115616in}}%
\pgfpathlineto{\pgfqpoint{5.444328in}{2.127345in}}%
\pgfpathlineto{\pgfqpoint{5.447021in}{2.110299in}}%
\pgfpathlineto{\pgfqpoint{5.449769in}{2.104791in}}%
\pgfpathlineto{\pgfqpoint{5.452365in}{2.108273in}}%
\pgfpathlineto{\pgfqpoint{5.455168in}{2.107278in}}%
\pgfpathlineto{\pgfqpoint{5.457721in}{2.104357in}}%
\pgfpathlineto{\pgfqpoint{5.460489in}{2.103565in}}%
\pgfpathlineto{\pgfqpoint{5.463079in}{2.100838in}}%
\pgfpathlineto{\pgfqpoint{5.465888in}{2.107195in}}%
\pgfpathlineto{\pgfqpoint{5.468425in}{2.108306in}}%
\pgfpathlineto{\pgfqpoint{5.471113in}{2.115822in}}%
\pgfpathlineto{\pgfqpoint{5.473792in}{2.114996in}}%
\pgfpathlineto{\pgfqpoint{5.476458in}{2.113944in}}%
\pgfpathlineto{\pgfqpoint{5.479152in}{2.107593in}}%
\pgfpathlineto{\pgfqpoint{5.481825in}{2.109557in}}%
\pgfpathlineto{\pgfqpoint{5.484641in}{2.117653in}}%
\pgfpathlineto{\pgfqpoint{5.487176in}{2.109121in}}%
\pgfpathlineto{\pgfqpoint{5.490000in}{2.108102in}}%
\pgfpathlineto{\pgfqpoint{5.492541in}{2.105297in}}%
\pgfpathlineto{\pgfqpoint{5.495346in}{2.105513in}}%
\pgfpathlineto{\pgfqpoint{5.497898in}{2.107406in}}%
\pgfpathlineto{\pgfqpoint{5.500687in}{2.104766in}}%
\pgfpathlineto{\pgfqpoint{5.503255in}{2.108263in}}%
\pgfpathlineto{\pgfqpoint{5.505933in}{2.104708in}}%
\pgfpathlineto{\pgfqpoint{5.508612in}{2.108205in}}%
\pgfpathlineto{\pgfqpoint{5.511290in}{2.107547in}}%
\pgfpathlineto{\pgfqpoint{5.514080in}{2.105224in}}%
\pgfpathlineto{\pgfqpoint{5.516646in}{2.095696in}}%
\pgfpathlineto{\pgfqpoint{5.519433in}{2.096933in}}%
\pgfpathlineto{\pgfqpoint{5.522003in}{2.096683in}}%
\pgfpathlineto{\pgfqpoint{5.524756in}{2.108578in}}%
\pgfpathlineto{\pgfqpoint{5.527360in}{2.106278in}}%
\pgfpathlineto{\pgfqpoint{5.530148in}{2.105917in}}%
\pgfpathlineto{\pgfqpoint{5.532717in}{2.104523in}}%
\pgfpathlineto{\pgfqpoint{5.535395in}{2.105892in}}%
\pgfpathlineto{\pgfqpoint{5.538074in}{2.108385in}}%
\pgfpathlineto{\pgfqpoint{5.540750in}{2.107464in}}%
\pgfpathlineto{\pgfqpoint{5.543421in}{2.110399in}}%
\pgfpathlineto{\pgfqpoint{5.546110in}{2.104617in}}%
\pgfpathlineto{\pgfqpoint{5.548921in}{2.098031in}}%
\pgfpathlineto{\pgfqpoint{5.551457in}{2.104639in}}%
\pgfpathlineto{\pgfqpoint{5.554198in}{2.109319in}}%
\pgfpathlineto{\pgfqpoint{5.556822in}{2.111303in}}%
\pgfpathlineto{\pgfqpoint{5.559612in}{2.105250in}}%
\pgfpathlineto{\pgfqpoint{5.562180in}{2.113383in}}%
\pgfpathlineto{\pgfqpoint{5.564940in}{2.112829in}}%
\pgfpathlineto{\pgfqpoint{5.567536in}{2.109056in}}%
\pgfpathlineto{\pgfqpoint{5.570215in}{2.117208in}}%
\pgfpathlineto{\pgfqpoint{5.572893in}{2.114997in}}%
\pgfpathlineto{\pgfqpoint{5.575596in}{2.107986in}}%
\pgfpathlineto{\pgfqpoint{5.578342in}{2.106020in}}%
\pgfpathlineto{\pgfqpoint{5.580914in}{2.107583in}}%
\pgfpathlineto{\pgfqpoint{5.583709in}{2.110318in}}%
\pgfpathlineto{\pgfqpoint{5.586269in}{2.114231in}}%
\pgfpathlineto{\pgfqpoint{5.589040in}{2.113738in}}%
\pgfpathlineto{\pgfqpoint{5.591641in}{2.110666in}}%
\pgfpathlineto{\pgfqpoint{5.594368in}{2.112167in}}%
\pgfpathlineto{\pgfqpoint{5.596999in}{2.114710in}}%
\pgfpathlineto{\pgfqpoint{5.599674in}{2.113719in}}%
\pgfpathlineto{\pgfqpoint{5.602352in}{2.107615in}}%
\pgfpathlineto{\pgfqpoint{5.605073in}{2.101993in}}%
\pgfpathlineto{\pgfqpoint{5.607698in}{2.092120in}}%
\pgfpathlineto{\pgfqpoint{5.610389in}{2.086947in}}%
\pgfpathlineto{\pgfqpoint{5.613235in}{2.090083in}}%
\pgfpathlineto{\pgfqpoint{5.615743in}{2.094771in}}%
\pgfpathlineto{\pgfqpoint{5.618526in}{2.088708in}}%
\pgfpathlineto{\pgfqpoint{5.621102in}{2.095501in}}%
\pgfpathlineto{\pgfqpoint{5.623868in}{2.105967in}}%
\pgfpathlineto{\pgfqpoint{5.626460in}{2.099233in}}%
\pgfpathlineto{\pgfqpoint{5.629232in}{2.098510in}}%
\pgfpathlineto{\pgfqpoint{5.631815in}{2.104105in}}%
\pgfpathlineto{\pgfqpoint{5.634496in}{2.107817in}}%
\pgfpathlineto{\pgfqpoint{5.637172in}{2.109319in}}%
\pgfpathlineto{\pgfqpoint{5.639852in}{2.108974in}}%
\pgfpathlineto{\pgfqpoint{5.642518in}{2.110429in}}%
\pgfpathlineto{\pgfqpoint{5.645243in}{2.114518in}}%
\pgfpathlineto{\pgfqpoint{5.648008in}{2.151322in}}%
\pgfpathlineto{\pgfqpoint{5.650563in}{2.134680in}}%
\pgfpathlineto{\pgfqpoint{5.653376in}{2.096553in}}%
\pgfpathlineto{\pgfqpoint{5.655919in}{2.094627in}}%
\pgfpathlineto{\pgfqpoint{5.658723in}{2.086947in}}%
\pgfpathlineto{\pgfqpoint{5.661273in}{2.086947in}}%
\pgfpathlineto{\pgfqpoint{5.664099in}{2.086947in}}%
\pgfpathlineto{\pgfqpoint{5.666632in}{2.094475in}}%
\pgfpathlineto{\pgfqpoint{5.669313in}{2.101472in}}%
\pgfpathlineto{\pgfqpoint{5.671991in}{2.089353in}}%
\pgfpathlineto{\pgfqpoint{5.674667in}{2.088179in}}%
\pgfpathlineto{\pgfqpoint{5.677486in}{2.086947in}}%
\pgfpathlineto{\pgfqpoint{5.680027in}{2.099520in}}%
\pgfpathlineto{\pgfqpoint{5.682836in}{2.094036in}}%
\pgfpathlineto{\pgfqpoint{5.685385in}{2.091914in}}%
\pgfpathlineto{\pgfqpoint{5.688159in}{2.105542in}}%
\pgfpathlineto{\pgfqpoint{5.690730in}{2.111115in}}%
\pgfpathlineto{\pgfqpoint{5.693473in}{2.111088in}}%
\pgfpathlineto{\pgfqpoint{5.696101in}{2.115864in}}%
\pgfpathlineto{\pgfqpoint{5.698775in}{2.112595in}}%
\pgfpathlineto{\pgfqpoint{5.701453in}{2.111735in}}%
\pgfpathlineto{\pgfqpoint{5.704130in}{2.130790in}}%
\pgfpathlineto{\pgfqpoint{5.706800in}{2.130553in}}%
\pgfpathlineto{\pgfqpoint{5.709490in}{2.140098in}}%
\pgfpathlineto{\pgfqpoint{5.712291in}{2.140573in}}%
\pgfpathlineto{\pgfqpoint{5.714834in}{2.134662in}}%
\pgfpathlineto{\pgfqpoint{5.717671in}{2.127971in}}%
\pgfpathlineto{\pgfqpoint{5.720201in}{2.120522in}}%
\pgfpathlineto{\pgfqpoint{5.722950in}{2.116567in}}%
\pgfpathlineto{\pgfqpoint{5.725548in}{2.117426in}}%
\pgfpathlineto{\pgfqpoint{5.728339in}{2.111045in}}%
\pgfpathlineto{\pgfqpoint{5.730919in}{2.111771in}}%
\pgfpathlineto{\pgfqpoint{5.733594in}{2.098152in}}%
\pgfpathlineto{\pgfqpoint{5.736276in}{2.099570in}}%
\pgfpathlineto{\pgfqpoint{5.738974in}{2.099911in}}%
\pgfpathlineto{\pgfqpoint{5.741745in}{2.100804in}}%
\pgfpathlineto{\pgfqpoint{5.744310in}{2.104623in}}%
\pgfpathlineto{\pgfqpoint{5.744310in}{0.413320in}}%
\pgfpathlineto{\pgfqpoint{5.744310in}{0.413320in}}%
\pgfpathlineto{\pgfqpoint{5.741745in}{0.413320in}}%
\pgfpathlineto{\pgfqpoint{5.738974in}{0.413320in}}%
\pgfpathlineto{\pgfqpoint{5.736276in}{0.413320in}}%
\pgfpathlineto{\pgfqpoint{5.733594in}{0.413320in}}%
\pgfpathlineto{\pgfqpoint{5.730919in}{0.413320in}}%
\pgfpathlineto{\pgfqpoint{5.728339in}{0.413320in}}%
\pgfpathlineto{\pgfqpoint{5.725548in}{0.413320in}}%
\pgfpathlineto{\pgfqpoint{5.722950in}{0.413320in}}%
\pgfpathlineto{\pgfqpoint{5.720201in}{0.413320in}}%
\pgfpathlineto{\pgfqpoint{5.717671in}{0.413320in}}%
\pgfpathlineto{\pgfqpoint{5.714834in}{0.413320in}}%
\pgfpathlineto{\pgfqpoint{5.712291in}{0.413320in}}%
\pgfpathlineto{\pgfqpoint{5.709490in}{0.413320in}}%
\pgfpathlineto{\pgfqpoint{5.706800in}{0.413320in}}%
\pgfpathlineto{\pgfqpoint{5.704130in}{0.413320in}}%
\pgfpathlineto{\pgfqpoint{5.701453in}{0.413320in}}%
\pgfpathlineto{\pgfqpoint{5.698775in}{0.413320in}}%
\pgfpathlineto{\pgfqpoint{5.696101in}{0.413320in}}%
\pgfpathlineto{\pgfqpoint{5.693473in}{0.413320in}}%
\pgfpathlineto{\pgfqpoint{5.690730in}{0.413320in}}%
\pgfpathlineto{\pgfqpoint{5.688159in}{0.413320in}}%
\pgfpathlineto{\pgfqpoint{5.685385in}{0.413320in}}%
\pgfpathlineto{\pgfqpoint{5.682836in}{0.413320in}}%
\pgfpathlineto{\pgfqpoint{5.680027in}{0.413320in}}%
\pgfpathlineto{\pgfqpoint{5.677486in}{0.413320in}}%
\pgfpathlineto{\pgfqpoint{5.674667in}{0.413320in}}%
\pgfpathlineto{\pgfqpoint{5.671991in}{0.413320in}}%
\pgfpathlineto{\pgfqpoint{5.669313in}{0.413320in}}%
\pgfpathlineto{\pgfqpoint{5.666632in}{0.413320in}}%
\pgfpathlineto{\pgfqpoint{5.664099in}{0.413320in}}%
\pgfpathlineto{\pgfqpoint{5.661273in}{0.413320in}}%
\pgfpathlineto{\pgfqpoint{5.658723in}{0.413320in}}%
\pgfpathlineto{\pgfqpoint{5.655919in}{0.413320in}}%
\pgfpathlineto{\pgfqpoint{5.653376in}{0.413320in}}%
\pgfpathlineto{\pgfqpoint{5.650563in}{0.413320in}}%
\pgfpathlineto{\pgfqpoint{5.648008in}{0.413320in}}%
\pgfpathlineto{\pgfqpoint{5.645243in}{0.413320in}}%
\pgfpathlineto{\pgfqpoint{5.642518in}{0.413320in}}%
\pgfpathlineto{\pgfqpoint{5.639852in}{0.413320in}}%
\pgfpathlineto{\pgfqpoint{5.637172in}{0.413320in}}%
\pgfpathlineto{\pgfqpoint{5.634496in}{0.413320in}}%
\pgfpathlineto{\pgfqpoint{5.631815in}{0.413320in}}%
\pgfpathlineto{\pgfqpoint{5.629232in}{0.413320in}}%
\pgfpathlineto{\pgfqpoint{5.626460in}{0.413320in}}%
\pgfpathlineto{\pgfqpoint{5.623868in}{0.413320in}}%
\pgfpathlineto{\pgfqpoint{5.621102in}{0.413320in}}%
\pgfpathlineto{\pgfqpoint{5.618526in}{0.413320in}}%
\pgfpathlineto{\pgfqpoint{5.615743in}{0.413320in}}%
\pgfpathlineto{\pgfqpoint{5.613235in}{0.413320in}}%
\pgfpathlineto{\pgfqpoint{5.610389in}{0.413320in}}%
\pgfpathlineto{\pgfqpoint{5.607698in}{0.413320in}}%
\pgfpathlineto{\pgfqpoint{5.605073in}{0.413320in}}%
\pgfpathlineto{\pgfqpoint{5.602352in}{0.413320in}}%
\pgfpathlineto{\pgfqpoint{5.599674in}{0.413320in}}%
\pgfpathlineto{\pgfqpoint{5.596999in}{0.413320in}}%
\pgfpathlineto{\pgfqpoint{5.594368in}{0.413320in}}%
\pgfpathlineto{\pgfqpoint{5.591641in}{0.413320in}}%
\pgfpathlineto{\pgfqpoint{5.589040in}{0.413320in}}%
\pgfpathlineto{\pgfqpoint{5.586269in}{0.413320in}}%
\pgfpathlineto{\pgfqpoint{5.583709in}{0.413320in}}%
\pgfpathlineto{\pgfqpoint{5.580914in}{0.413320in}}%
\pgfpathlineto{\pgfqpoint{5.578342in}{0.413320in}}%
\pgfpathlineto{\pgfqpoint{5.575596in}{0.413320in}}%
\pgfpathlineto{\pgfqpoint{5.572893in}{0.413320in}}%
\pgfpathlineto{\pgfqpoint{5.570215in}{0.413320in}}%
\pgfpathlineto{\pgfqpoint{5.567536in}{0.413320in}}%
\pgfpathlineto{\pgfqpoint{5.564940in}{0.413320in}}%
\pgfpathlineto{\pgfqpoint{5.562180in}{0.413320in}}%
\pgfpathlineto{\pgfqpoint{5.559612in}{0.413320in}}%
\pgfpathlineto{\pgfqpoint{5.556822in}{0.413320in}}%
\pgfpathlineto{\pgfqpoint{5.554198in}{0.413320in}}%
\pgfpathlineto{\pgfqpoint{5.551457in}{0.413320in}}%
\pgfpathlineto{\pgfqpoint{5.548921in}{0.413320in}}%
\pgfpathlineto{\pgfqpoint{5.546110in}{0.413320in}}%
\pgfpathlineto{\pgfqpoint{5.543421in}{0.413320in}}%
\pgfpathlineto{\pgfqpoint{5.540750in}{0.413320in}}%
\pgfpathlineto{\pgfqpoint{5.538074in}{0.413320in}}%
\pgfpathlineto{\pgfqpoint{5.535395in}{0.413320in}}%
\pgfpathlineto{\pgfqpoint{5.532717in}{0.413320in}}%
\pgfpathlineto{\pgfqpoint{5.530148in}{0.413320in}}%
\pgfpathlineto{\pgfqpoint{5.527360in}{0.413320in}}%
\pgfpathlineto{\pgfqpoint{5.524756in}{0.413320in}}%
\pgfpathlineto{\pgfqpoint{5.522003in}{0.413320in}}%
\pgfpathlineto{\pgfqpoint{5.519433in}{0.413320in}}%
\pgfpathlineto{\pgfqpoint{5.516646in}{0.413320in}}%
\pgfpathlineto{\pgfqpoint{5.514080in}{0.413320in}}%
\pgfpathlineto{\pgfqpoint{5.511290in}{0.413320in}}%
\pgfpathlineto{\pgfqpoint{5.508612in}{0.413320in}}%
\pgfpathlineto{\pgfqpoint{5.505933in}{0.413320in}}%
\pgfpathlineto{\pgfqpoint{5.503255in}{0.413320in}}%
\pgfpathlineto{\pgfqpoint{5.500687in}{0.413320in}}%
\pgfpathlineto{\pgfqpoint{5.497898in}{0.413320in}}%
\pgfpathlineto{\pgfqpoint{5.495346in}{0.413320in}}%
\pgfpathlineto{\pgfqpoint{5.492541in}{0.413320in}}%
\pgfpathlineto{\pgfqpoint{5.490000in}{0.413320in}}%
\pgfpathlineto{\pgfqpoint{5.487176in}{0.413320in}}%
\pgfpathlineto{\pgfqpoint{5.484641in}{0.413320in}}%
\pgfpathlineto{\pgfqpoint{5.481825in}{0.413320in}}%
\pgfpathlineto{\pgfqpoint{5.479152in}{0.413320in}}%
\pgfpathlineto{\pgfqpoint{5.476458in}{0.413320in}}%
\pgfpathlineto{\pgfqpoint{5.473792in}{0.413320in}}%
\pgfpathlineto{\pgfqpoint{5.471113in}{0.413320in}}%
\pgfpathlineto{\pgfqpoint{5.468425in}{0.413320in}}%
\pgfpathlineto{\pgfqpoint{5.465888in}{0.413320in}}%
\pgfpathlineto{\pgfqpoint{5.463079in}{0.413320in}}%
\pgfpathlineto{\pgfqpoint{5.460489in}{0.413320in}}%
\pgfpathlineto{\pgfqpoint{5.457721in}{0.413320in}}%
\pgfpathlineto{\pgfqpoint{5.455168in}{0.413320in}}%
\pgfpathlineto{\pgfqpoint{5.452365in}{0.413320in}}%
\pgfpathlineto{\pgfqpoint{5.449769in}{0.413320in}}%
\pgfpathlineto{\pgfqpoint{5.447021in}{0.413320in}}%
\pgfpathlineto{\pgfqpoint{5.444328in}{0.413320in}}%
\pgfpathlineto{\pgfqpoint{5.441698in}{0.413320in}}%
\pgfpathlineto{\pgfqpoint{5.438974in}{0.413320in}}%
\pgfpathlineto{\pgfqpoint{5.436295in}{0.413320in}}%
\pgfpathlineto{\pgfqpoint{5.433616in}{0.413320in}}%
\pgfpathlineto{\pgfqpoint{5.431015in}{0.413320in}}%
\pgfpathlineto{\pgfqpoint{5.428259in}{0.413320in}}%
\pgfpathlineto{\pgfqpoint{5.425661in}{0.413320in}}%
\pgfpathlineto{\pgfqpoint{5.422897in}{0.413320in}}%
\pgfpathlineto{\pgfqpoint{5.420304in}{0.413320in}}%
\pgfpathlineto{\pgfqpoint{5.417547in}{0.413320in}}%
\pgfpathlineto{\pgfqpoint{5.414954in}{0.413320in}}%
\pgfpathlineto{\pgfqpoint{5.412190in}{0.413320in}}%
\pgfpathlineto{\pgfqpoint{5.409507in}{0.413320in}}%
\pgfpathlineto{\pgfqpoint{5.406832in}{0.413320in}}%
\pgfpathlineto{\pgfqpoint{5.404154in}{0.413320in}}%
\pgfpathlineto{\pgfqpoint{5.401576in}{0.413320in}}%
\pgfpathlineto{\pgfqpoint{5.398784in}{0.413320in}}%
\pgfpathlineto{\pgfqpoint{5.396219in}{0.413320in}}%
\pgfpathlineto{\pgfqpoint{5.393441in}{0.413320in}}%
\pgfpathlineto{\pgfqpoint{5.390900in}{0.413320in}}%
\pgfpathlineto{\pgfqpoint{5.388083in}{0.413320in}}%
\pgfpathlineto{\pgfqpoint{5.385550in}{0.413320in}}%
\pgfpathlineto{\pgfqpoint{5.382725in}{0.413320in}}%
\pgfpathlineto{\pgfqpoint{5.380048in}{0.413320in}}%
\pgfpathlineto{\pgfqpoint{5.377370in}{0.413320in}}%
\pgfpathlineto{\pgfqpoint{5.374692in}{0.413320in}}%
\pgfpathlineto{\pgfqpoint{5.372013in}{0.413320in}}%
\pgfpathlineto{\pgfqpoint{5.369335in}{0.413320in}}%
\pgfpathlineto{\pgfqpoint{5.366727in}{0.413320in}}%
\pgfpathlineto{\pgfqpoint{5.363966in}{0.413320in}}%
\pgfpathlineto{\pgfqpoint{5.361370in}{0.413320in}}%
\pgfpathlineto{\pgfqpoint{5.358612in}{0.413320in}}%
\pgfpathlineto{\pgfqpoint{5.356056in}{0.413320in}}%
\pgfpathlineto{\pgfqpoint{5.353262in}{0.413320in}}%
\pgfpathlineto{\pgfqpoint{5.350723in}{0.413320in}}%
\pgfpathlineto{\pgfqpoint{5.347905in}{0.413320in}}%
\pgfpathlineto{\pgfqpoint{5.345224in}{0.413320in}}%
\pgfpathlineto{\pgfqpoint{5.342549in}{0.413320in}}%
\pgfpathlineto{\pgfqpoint{5.339872in}{0.413320in}}%
\pgfpathlineto{\pgfqpoint{5.337353in}{0.413320in}}%
\pgfpathlineto{\pgfqpoint{5.334510in}{0.413320in}}%
\pgfpathlineto{\pgfqpoint{5.331973in}{0.413320in}}%
\pgfpathlineto{\pgfqpoint{5.329159in}{0.413320in}}%
\pgfpathlineto{\pgfqpoint{5.326564in}{0.413320in}}%
\pgfpathlineto{\pgfqpoint{5.323802in}{0.413320in}}%
\pgfpathlineto{\pgfqpoint{5.321256in}{0.413320in}}%
\pgfpathlineto{\pgfqpoint{5.318430in}{0.413320in}}%
\pgfpathlineto{\pgfqpoint{5.315754in}{0.413320in}}%
\pgfpathlineto{\pgfqpoint{5.313089in}{0.413320in}}%
\pgfpathlineto{\pgfqpoint{5.310411in}{0.413320in}}%
\pgfpathlineto{\pgfqpoint{5.307731in}{0.413320in}}%
\pgfpathlineto{\pgfqpoint{5.305054in}{0.413320in}}%
\pgfpathlineto{\pgfqpoint{5.302443in}{0.413320in}}%
\pgfpathlineto{\pgfqpoint{5.299696in}{0.413320in}}%
\pgfpathlineto{\pgfqpoint{5.297140in}{0.413320in}}%
\pgfpathlineto{\pgfqpoint{5.294339in}{0.413320in}}%
\pgfpathlineto{\pgfqpoint{5.291794in}{0.413320in}}%
\pgfpathlineto{\pgfqpoint{5.288984in}{0.413320in}}%
\pgfpathlineto{\pgfqpoint{5.286436in}{0.413320in}}%
\pgfpathlineto{\pgfqpoint{5.283631in}{0.413320in}}%
\pgfpathlineto{\pgfqpoint{5.280947in}{0.413320in}}%
\pgfpathlineto{\pgfqpoint{5.278322in}{0.413320in}}%
\pgfpathlineto{\pgfqpoint{5.275589in}{0.413320in}}%
\pgfpathlineto{\pgfqpoint{5.272913in}{0.413320in}}%
\pgfpathlineto{\pgfqpoint{5.270238in}{0.413320in}}%
\pgfpathlineto{\pgfqpoint{5.267691in}{0.413320in}}%
\pgfpathlineto{\pgfqpoint{5.264876in}{0.413320in}}%
\pgfpathlineto{\pgfqpoint{5.262264in}{0.413320in}}%
\pgfpathlineto{\pgfqpoint{5.259511in}{0.413320in}}%
\pgfpathlineto{\pgfqpoint{5.256973in}{0.413320in}}%
\pgfpathlineto{\pgfqpoint{5.254236in}{0.413320in}}%
\pgfpathlineto{\pgfqpoint{5.251590in}{0.413320in}}%
\pgfpathlineto{\pgfqpoint{5.248816in}{0.413320in}}%
\pgfpathlineto{\pgfqpoint{5.246130in}{0.413320in}}%
\pgfpathlineto{\pgfqpoint{5.243445in}{0.413320in}}%
\pgfpathlineto{\pgfqpoint{5.240777in}{0.413320in}}%
\pgfpathlineto{\pgfqpoint{5.238173in}{0.413320in}}%
\pgfpathlineto{\pgfqpoint{5.235409in}{0.413320in}}%
\pgfpathlineto{\pgfqpoint{5.232855in}{0.413320in}}%
\pgfpathlineto{\pgfqpoint{5.230059in}{0.413320in}}%
\pgfpathlineto{\pgfqpoint{5.227470in}{0.413320in}}%
\pgfpathlineto{\pgfqpoint{5.224695in}{0.413320in}}%
\pgfpathlineto{\pgfqpoint{5.222151in}{0.413320in}}%
\pgfpathlineto{\pgfqpoint{5.219345in}{0.413320in}}%
\pgfpathlineto{\pgfqpoint{5.216667in}{0.413320in}}%
\pgfpathlineto{\pgfqpoint{5.214027in}{0.413320in}}%
\pgfpathlineto{\pgfqpoint{5.211299in}{0.413320in}}%
\pgfpathlineto{\pgfqpoint{5.208630in}{0.413320in}}%
\pgfpathlineto{\pgfqpoint{5.205952in}{0.413320in}}%
\pgfpathlineto{\pgfqpoint{5.203388in}{0.413320in}}%
\pgfpathlineto{\pgfqpoint{5.200594in}{0.413320in}}%
\pgfpathlineto{\pgfqpoint{5.198008in}{0.413320in}}%
\pgfpathlineto{\pgfqpoint{5.195239in}{0.413320in}}%
\pgfpathlineto{\pgfqpoint{5.192680in}{0.413320in}}%
\pgfpathlineto{\pgfqpoint{5.189880in}{0.413320in}}%
\pgfpathlineto{\pgfqpoint{5.187294in}{0.413320in}}%
\pgfpathlineto{\pgfqpoint{5.184522in}{0.413320in}}%
\pgfpathlineto{\pgfqpoint{5.181848in}{0.413320in}}%
\pgfpathlineto{\pgfqpoint{5.179188in}{0.413320in}}%
\pgfpathlineto{\pgfqpoint{5.176477in}{0.413320in}}%
\pgfpathlineto{\pgfqpoint{5.173925in}{0.413320in}}%
\pgfpathlineto{\pgfqpoint{5.171133in}{0.413320in}}%
\pgfpathlineto{\pgfqpoint{5.168591in}{0.413320in}}%
\pgfpathlineto{\pgfqpoint{5.165775in}{0.413320in}}%
\pgfpathlineto{\pgfqpoint{5.163243in}{0.413320in}}%
\pgfpathlineto{\pgfqpoint{5.160420in}{0.413320in}}%
\pgfpathlineto{\pgfqpoint{5.157815in}{0.413320in}}%
\pgfpathlineto{\pgfqpoint{5.155059in}{0.413320in}}%
\pgfpathlineto{\pgfqpoint{5.152382in}{0.413320in}}%
\pgfpathlineto{\pgfqpoint{5.149734in}{0.413320in}}%
\pgfpathlineto{\pgfqpoint{5.147029in}{0.413320in}}%
\pgfpathlineto{\pgfqpoint{5.144349in}{0.413320in}}%
\pgfpathlineto{\pgfqpoint{5.141660in}{0.413320in}}%
\pgfpathlineto{\pgfqpoint{5.139072in}{0.413320in}}%
\pgfpathlineto{\pgfqpoint{5.136311in}{0.413320in}}%
\pgfpathlineto{\pgfqpoint{5.133716in}{0.413320in}}%
\pgfpathlineto{\pgfqpoint{5.130953in}{0.413320in}}%
\pgfpathlineto{\pgfqpoint{5.128421in}{0.413320in}}%
\pgfpathlineto{\pgfqpoint{5.125599in}{0.413320in}}%
\pgfpathlineto{\pgfqpoint{5.123042in}{0.413320in}}%
\pgfpathlineto{\pgfqpoint{5.120243in}{0.413320in}}%
\pgfpathlineto{\pgfqpoint{5.117550in}{0.413320in}}%
\pgfpathlineto{\pgfqpoint{5.114887in}{0.413320in}}%
\pgfpathlineto{\pgfqpoint{5.112209in}{0.413320in}}%
\pgfpathlineto{\pgfqpoint{5.109530in}{0.413320in}}%
\pgfpathlineto{\pgfqpoint{5.106842in}{0.413320in}}%
\pgfpathlineto{\pgfqpoint{5.104312in}{0.413320in}}%
\pgfpathlineto{\pgfqpoint{5.101496in}{0.413320in}}%
\pgfpathlineto{\pgfqpoint{5.098948in}{0.413320in}}%
\pgfpathlineto{\pgfqpoint{5.096142in}{0.413320in}}%
\pgfpathlineto{\pgfqpoint{5.093579in}{0.413320in}}%
\pgfpathlineto{\pgfqpoint{5.090788in}{0.413320in}}%
\pgfpathlineto{\pgfqpoint{5.088103in}{0.413320in}}%
\pgfpathlineto{\pgfqpoint{5.085426in}{0.413320in}}%
\pgfpathlineto{\pgfqpoint{5.082746in}{0.413320in}}%
\pgfpathlineto{\pgfqpoint{5.080067in}{0.413320in}}%
\pgfpathlineto{\pgfqpoint{5.077390in}{0.413320in}}%
\pgfpathlineto{\pgfqpoint{5.074851in}{0.413320in}}%
\pgfpathlineto{\pgfqpoint{5.072030in}{0.413320in}}%
\pgfpathlineto{\pgfqpoint{5.069463in}{0.413320in}}%
\pgfpathlineto{\pgfqpoint{5.066677in}{0.413320in}}%
\pgfpathlineto{\pgfqpoint{5.064144in}{0.413320in}}%
\pgfpathlineto{\pgfqpoint{5.061315in}{0.413320in}}%
\pgfpathlineto{\pgfqpoint{5.058711in}{0.413320in}}%
\pgfpathlineto{\pgfqpoint{5.055952in}{0.413320in}}%
\pgfpathlineto{\pgfqpoint{5.053284in}{0.413320in}}%
\pgfpathlineto{\pgfqpoint{5.050606in}{0.413320in}}%
\pgfpathlineto{\pgfqpoint{5.047924in}{0.413320in}}%
\pgfpathlineto{\pgfqpoint{5.045249in}{0.413320in}}%
\pgfpathlineto{\pgfqpoint{5.042572in}{0.413320in}}%
\pgfpathlineto{\pgfqpoint{5.039962in}{0.413320in}}%
\pgfpathlineto{\pgfqpoint{5.037214in}{0.413320in}}%
\pgfpathlineto{\pgfqpoint{5.034649in}{0.413320in}}%
\pgfpathlineto{\pgfqpoint{5.031849in}{0.413320in}}%
\pgfpathlineto{\pgfqpoint{5.029275in}{0.413320in}}%
\pgfpathlineto{\pgfqpoint{5.026501in}{0.413320in}}%
\pgfpathlineto{\pgfqpoint{5.023927in}{0.413320in}}%
\pgfpathlineto{\pgfqpoint{5.021147in}{0.413320in}}%
\pgfpathlineto{\pgfqpoint{5.018466in}{0.413320in}}%
\pgfpathlineto{\pgfqpoint{5.015820in}{0.413320in}}%
\pgfpathlineto{\pgfqpoint{5.013104in}{0.413320in}}%
\pgfpathlineto{\pgfqpoint{5.010562in}{0.413320in}}%
\pgfpathlineto{\pgfqpoint{5.007751in}{0.413320in}}%
\pgfpathlineto{\pgfqpoint{5.005178in}{0.413320in}}%
\pgfpathlineto{\pgfqpoint{5.002384in}{0.413320in}}%
\pgfpathlineto{\pgfqpoint{4.999780in}{0.413320in}}%
\pgfpathlineto{\pgfqpoint{4.997028in}{0.413320in}}%
\pgfpathlineto{\pgfqpoint{4.994390in}{0.413320in}}%
\pgfpathlineto{\pgfqpoint{4.991683in}{0.413320in}}%
\pgfpathlineto{\pgfqpoint{4.989001in}{0.413320in}}%
\pgfpathlineto{\pgfqpoint{4.986325in}{0.413320in}}%
\pgfpathlineto{\pgfqpoint{4.983637in}{0.413320in}}%
\pgfpathlineto{\pgfqpoint{4.980967in}{0.413320in}}%
\pgfpathlineto{\pgfqpoint{4.978287in}{0.413320in}}%
\pgfpathlineto{\pgfqpoint{4.975703in}{0.413320in}}%
\pgfpathlineto{\pgfqpoint{4.972933in}{0.413320in}}%
\pgfpathlineto{\pgfqpoint{4.970314in}{0.413320in}}%
\pgfpathlineto{\pgfqpoint{4.967575in}{0.413320in}}%
\pgfpathlineto{\pgfqpoint{4.965002in}{0.413320in}}%
\pgfpathlineto{\pgfqpoint{4.962219in}{0.413320in}}%
\pgfpathlineto{\pgfqpoint{4.959689in}{0.413320in}}%
\pgfpathlineto{\pgfqpoint{4.956862in}{0.413320in}}%
\pgfpathlineto{\pgfqpoint{4.954182in}{0.413320in}}%
\pgfpathlineto{\pgfqpoint{4.951504in}{0.413320in}}%
\pgfpathlineto{\pgfqpoint{4.948827in}{0.413320in}}%
\pgfpathlineto{\pgfqpoint{4.946151in}{0.413320in}}%
\pgfpathlineto{\pgfqpoint{4.943466in}{0.413320in}}%
\pgfpathlineto{\pgfqpoint{4.940881in}{0.413320in}}%
\pgfpathlineto{\pgfqpoint{4.938112in}{0.413320in}}%
\pgfpathlineto{\pgfqpoint{4.935515in}{0.413320in}}%
\pgfpathlineto{\pgfqpoint{4.932742in}{0.413320in}}%
\pgfpathlineto{\pgfqpoint{4.930170in}{0.413320in}}%
\pgfpathlineto{\pgfqpoint{4.927400in}{0.413320in}}%
\pgfpathlineto{\pgfqpoint{4.924708in}{0.413320in}}%
\pgfpathlineto{\pgfqpoint{4.922041in}{0.413320in}}%
\pgfpathlineto{\pgfqpoint{4.919352in}{0.413320in}}%
\pgfpathlineto{\pgfqpoint{4.916681in}{0.413320in}}%
\pgfpathlineto{\pgfqpoint{4.914009in}{0.413320in}}%
\pgfpathlineto{\pgfqpoint{4.911435in}{0.413320in}}%
\pgfpathlineto{\pgfqpoint{4.908648in}{0.413320in}}%
\pgfpathlineto{\pgfqpoint{4.906096in}{0.413320in}}%
\pgfpathlineto{\pgfqpoint{4.903295in}{0.413320in}}%
\pgfpathlineto{\pgfqpoint{4.900712in}{0.413320in}}%
\pgfpathlineto{\pgfqpoint{4.897938in}{0.413320in}}%
\pgfpathlineto{\pgfqpoint{4.895399in}{0.413320in}}%
\pgfpathlineto{\pgfqpoint{4.892611in}{0.413320in}}%
\pgfpathlineto{\pgfqpoint{4.889902in}{0.413320in}}%
\pgfpathlineto{\pgfqpoint{4.887211in}{0.413320in}}%
\pgfpathlineto{\pgfqpoint{4.884540in}{0.413320in}}%
\pgfpathlineto{\pgfqpoint{4.881864in}{0.413320in}}%
\pgfpathlineto{\pgfqpoint{4.879180in}{0.413320in}}%
\pgfpathlineto{\pgfqpoint{4.876636in}{0.413320in}}%
\pgfpathlineto{\pgfqpoint{4.873832in}{0.413320in}}%
\pgfpathlineto{\pgfqpoint{4.871209in}{0.413320in}}%
\pgfpathlineto{\pgfqpoint{4.868474in}{0.413320in}}%
\pgfpathlineto{\pgfqpoint{4.865910in}{0.413320in}}%
\pgfpathlineto{\pgfqpoint{4.863116in}{0.413320in}}%
\pgfpathlineto{\pgfqpoint{4.860544in}{0.413320in}}%
\pgfpathlineto{\pgfqpoint{4.857807in}{0.413320in}}%
\pgfpathlineto{\pgfqpoint{4.855070in}{0.413320in}}%
\pgfpathlineto{\pgfqpoint{4.852404in}{0.413320in}}%
\pgfpathlineto{\pgfqpoint{4.849715in}{0.413320in}}%
\pgfpathlineto{\pgfqpoint{4.847127in}{0.413320in}}%
\pgfpathlineto{\pgfqpoint{4.844361in}{0.413320in}}%
\pgfpathlineto{\pgfqpoint{4.842380in}{0.413320in}}%
\pgfpathlineto{\pgfqpoint{4.839922in}{0.413320in}}%
\pgfpathlineto{\pgfqpoint{4.837992in}{0.413320in}}%
\pgfpathlineto{\pgfqpoint{4.833657in}{0.413320in}}%
\pgfpathlineto{\pgfqpoint{4.831045in}{0.413320in}}%
\pgfpathlineto{\pgfqpoint{4.828291in}{0.413320in}}%
\pgfpathlineto{\pgfqpoint{4.825619in}{0.413320in}}%
\pgfpathlineto{\pgfqpoint{4.822945in}{0.413320in}}%
\pgfpathlineto{\pgfqpoint{4.820265in}{0.413320in}}%
\pgfpathlineto{\pgfqpoint{4.817587in}{0.413320in}}%
\pgfpathlineto{\pgfqpoint{4.814907in}{0.413320in}}%
\pgfpathlineto{\pgfqpoint{4.812377in}{0.413320in}}%
\pgfpathlineto{\pgfqpoint{4.809538in}{0.413320in}}%
\pgfpathlineto{\pgfqpoint{4.807017in}{0.413320in}}%
\pgfpathlineto{\pgfqpoint{4.804193in}{0.413320in}}%
\pgfpathlineto{\pgfqpoint{4.801586in}{0.413320in}}%
\pgfpathlineto{\pgfqpoint{4.798830in}{0.413320in}}%
\pgfpathlineto{\pgfqpoint{4.796274in}{0.413320in}}%
\pgfpathlineto{\pgfqpoint{4.793512in}{0.413320in}}%
\pgfpathlineto{\pgfqpoint{4.790798in}{0.413320in}}%
\pgfpathlineto{\pgfqpoint{4.788116in}{0.413320in}}%
\pgfpathlineto{\pgfqpoint{4.785445in}{0.413320in}}%
\pgfpathlineto{\pgfqpoint{4.782872in}{0.413320in}}%
\pgfpathlineto{\pgfqpoint{4.780083in}{0.413320in}}%
\pgfpathlineto{\pgfqpoint{4.777535in}{0.413320in}}%
\pgfpathlineto{\pgfqpoint{4.774732in}{0.413320in}}%
\pgfpathlineto{\pgfqpoint{4.772198in}{0.413320in}}%
\pgfpathlineto{\pgfqpoint{4.769367in}{0.413320in}}%
\pgfpathlineto{\pgfqpoint{4.766783in}{0.413320in}}%
\pgfpathlineto{\pgfqpoint{4.764018in}{0.413320in}}%
\pgfpathlineto{\pgfqpoint{4.761337in}{0.413320in}}%
\pgfpathlineto{\pgfqpoint{4.758653in}{0.413320in}}%
\pgfpathlineto{\pgfqpoint{4.755983in}{0.413320in}}%
\pgfpathlineto{\pgfqpoint{4.753298in}{0.413320in}}%
\pgfpathlineto{\pgfqpoint{4.750627in}{0.413320in}}%
\pgfpathlineto{\pgfqpoint{4.748081in}{0.413320in}}%
\pgfpathlineto{\pgfqpoint{4.745256in}{0.413320in}}%
\pgfpathlineto{\pgfqpoint{4.742696in}{0.413320in}}%
\pgfpathlineto{\pgfqpoint{4.739912in}{0.413320in}}%
\pgfpathlineto{\pgfqpoint{4.737348in}{0.413320in}}%
\pgfpathlineto{\pgfqpoint{4.734552in}{0.413320in}}%
\pgfpathlineto{\pgfqpoint{4.731901in}{0.413320in}}%
\pgfpathlineto{\pgfqpoint{4.729233in}{0.413320in}}%
\pgfpathlineto{\pgfqpoint{4.726508in}{0.413320in}}%
\pgfpathlineto{\pgfqpoint{4.723873in}{0.413320in}}%
\pgfpathlineto{\pgfqpoint{4.721160in}{0.413320in}}%
\pgfpathlineto{\pgfqpoint{4.718486in}{0.413320in}}%
\pgfpathlineto{\pgfqpoint{4.715806in}{0.413320in}}%
\pgfpathlineto{\pgfqpoint{4.713275in}{0.413320in}}%
\pgfpathlineto{\pgfqpoint{4.710437in}{0.413320in}}%
\pgfpathlineto{\pgfqpoint{4.707824in}{0.413320in}}%
\pgfpathlineto{\pgfqpoint{4.705094in}{0.413320in}}%
\pgfpathlineto{\pgfqpoint{4.702517in}{0.413320in}}%
\pgfpathlineto{\pgfqpoint{4.699734in}{0.413320in}}%
\pgfpathlineto{\pgfqpoint{4.697170in}{0.413320in}}%
\pgfpathlineto{\pgfqpoint{4.694381in}{0.413320in}}%
\pgfpathlineto{\pgfqpoint{4.691694in}{0.413320in}}%
\pgfpathlineto{\pgfqpoint{4.689051in}{0.413320in}}%
\pgfpathlineto{\pgfqpoint{4.686337in}{0.413320in}}%
\pgfpathlineto{\pgfqpoint{4.683799in}{0.413320in}}%
\pgfpathlineto{\pgfqpoint{4.680988in}{0.413320in}}%
\pgfpathlineto{\pgfqpoint{4.678448in}{0.413320in}}%
\pgfpathlineto{\pgfqpoint{4.675619in}{0.413320in}}%
\pgfpathlineto{\pgfqpoint{4.673068in}{0.413320in}}%
\pgfpathlineto{\pgfqpoint{4.670261in}{0.413320in}}%
\pgfpathlineto{\pgfqpoint{4.667764in}{0.413320in}}%
\pgfpathlineto{\pgfqpoint{4.664923in}{0.413320in}}%
\pgfpathlineto{\pgfqpoint{4.662237in}{0.413320in}}%
\pgfpathlineto{\pgfqpoint{4.659590in}{0.413320in}}%
\pgfpathlineto{\pgfqpoint{4.656873in}{0.413320in}}%
\pgfpathlineto{\pgfqpoint{4.654203in}{0.413320in}}%
\pgfpathlineto{\pgfqpoint{4.651524in}{0.413320in}}%
\pgfpathlineto{\pgfqpoint{4.648922in}{0.413320in}}%
\pgfpathlineto{\pgfqpoint{4.646169in}{0.413320in}}%
\pgfpathlineto{\pgfqpoint{4.643628in}{0.413320in}}%
\pgfpathlineto{\pgfqpoint{4.640809in}{0.413320in}}%
\pgfpathlineto{\pgfqpoint{4.638204in}{0.413320in}}%
\pgfpathlineto{\pgfqpoint{4.635445in}{0.413320in}}%
\pgfpathlineto{\pgfqpoint{4.632902in}{0.413320in}}%
\pgfpathlineto{\pgfqpoint{4.630096in}{0.413320in}}%
\pgfpathlineto{\pgfqpoint{4.627411in}{0.413320in}}%
\pgfpathlineto{\pgfqpoint{4.624741in}{0.413320in}}%
\pgfpathlineto{\pgfqpoint{4.622056in}{0.413320in}}%
\pgfpathlineto{\pgfqpoint{4.619529in}{0.413320in}}%
\pgfpathlineto{\pgfqpoint{4.616702in}{0.413320in}}%
\pgfpathlineto{\pgfqpoint{4.614134in}{0.413320in}}%
\pgfpathlineto{\pgfqpoint{4.611350in}{0.413320in}}%
\pgfpathlineto{\pgfqpoint{4.608808in}{0.413320in}}%
\pgfpathlineto{\pgfqpoint{4.605990in}{0.413320in}}%
\pgfpathlineto{\pgfqpoint{4.603430in}{0.413320in}}%
\pgfpathlineto{\pgfqpoint{4.600633in}{0.413320in}}%
\pgfpathlineto{\pgfqpoint{4.597951in}{0.413320in}}%
\pgfpathlineto{\pgfqpoint{4.595281in}{0.413320in}}%
\pgfpathlineto{\pgfqpoint{4.592589in}{0.413320in}}%
\pgfpathlineto{\pgfqpoint{4.589920in}{0.413320in}}%
\pgfpathlineto{\pgfqpoint{4.587244in}{0.413320in}}%
\pgfpathlineto{\pgfqpoint{4.584672in}{0.413320in}}%
\pgfpathlineto{\pgfqpoint{4.581888in}{0.413320in}}%
\pgfpathlineto{\pgfqpoint{4.579305in}{0.413320in}}%
\pgfpathlineto{\pgfqpoint{4.576531in}{0.413320in}}%
\pgfpathlineto{\pgfqpoint{4.573947in}{0.413320in}}%
\pgfpathlineto{\pgfqpoint{4.571171in}{0.413320in}}%
\pgfpathlineto{\pgfqpoint{4.568612in}{0.413320in}}%
\pgfpathlineto{\pgfqpoint{4.565820in}{0.413320in}}%
\pgfpathlineto{\pgfqpoint{4.563125in}{0.413320in}}%
\pgfpathlineto{\pgfqpoint{4.560448in}{0.413320in}}%
\pgfpathlineto{\pgfqpoint{4.557777in}{0.413320in}}%
\pgfpathlineto{\pgfqpoint{4.555106in}{0.413320in}}%
\pgfpathlineto{\pgfqpoint{4.552425in}{0.413320in}}%
\pgfpathlineto{\pgfqpoint{4.549822in}{0.413320in}}%
\pgfpathlineto{\pgfqpoint{4.547064in}{0.413320in}}%
\pgfpathlineto{\pgfqpoint{4.544464in}{0.413320in}}%
\pgfpathlineto{\pgfqpoint{4.541711in}{0.413320in}}%
\pgfpathlineto{\pgfqpoint{4.539144in}{0.413320in}}%
\pgfpathlineto{\pgfqpoint{4.536400in}{0.413320in}}%
\pgfpathlineto{\pgfqpoint{4.533764in}{0.413320in}}%
\pgfpathlineto{\pgfqpoint{4.530990in}{0.413320in}}%
\pgfpathlineto{\pgfqpoint{4.528307in}{0.413320in}}%
\pgfpathlineto{\pgfqpoint{4.525640in}{0.413320in}}%
\pgfpathlineto{\pgfqpoint{4.522962in}{0.413320in}}%
\pgfpathlineto{\pgfqpoint{4.520345in}{0.413320in}}%
\pgfpathlineto{\pgfqpoint{4.517598in}{0.413320in}}%
\pgfpathlineto{\pgfqpoint{4.515080in}{0.413320in}}%
\pgfpathlineto{\pgfqpoint{4.512246in}{0.413320in}}%
\pgfpathlineto{\pgfqpoint{4.509643in}{0.413320in}}%
\pgfpathlineto{\pgfqpoint{4.506893in}{0.413320in}}%
\pgfpathlineto{\pgfqpoint{4.504305in}{0.413320in}}%
\pgfpathlineto{\pgfqpoint{4.501529in}{0.413320in}}%
\pgfpathlineto{\pgfqpoint{4.498850in}{0.413320in}}%
\pgfpathlineto{\pgfqpoint{4.496167in}{0.413320in}}%
\pgfpathlineto{\pgfqpoint{4.493492in}{0.413320in}}%
\pgfpathlineto{\pgfqpoint{4.490822in}{0.413320in}}%
\pgfpathlineto{\pgfqpoint{4.488130in}{0.413320in}}%
\pgfpathlineto{\pgfqpoint{4.485581in}{0.413320in}}%
\pgfpathlineto{\pgfqpoint{4.482778in}{0.413320in}}%
\pgfpathlineto{\pgfqpoint{4.480201in}{0.413320in}}%
\pgfpathlineto{\pgfqpoint{4.477430in}{0.413320in}}%
\pgfpathlineto{\pgfqpoint{4.474861in}{0.413320in}}%
\pgfpathlineto{\pgfqpoint{4.472059in}{0.413320in}}%
\pgfpathlineto{\pgfqpoint{4.469492in}{0.413320in}}%
\pgfpathlineto{\pgfqpoint{4.466717in}{0.413320in}}%
\pgfpathlineto{\pgfqpoint{4.464029in}{0.413320in}}%
\pgfpathlineto{\pgfqpoint{4.461367in}{0.413320in}}%
\pgfpathlineto{\pgfqpoint{4.458681in}{0.413320in}}%
\pgfpathlineto{\pgfqpoint{4.456138in}{0.413320in}}%
\pgfpathlineto{\pgfqpoint{4.453312in}{0.413320in}}%
\pgfpathlineto{\pgfqpoint{4.450767in}{0.413320in}}%
\pgfpathlineto{\pgfqpoint{4.447965in}{0.413320in}}%
\pgfpathlineto{\pgfqpoint{4.445423in}{0.413320in}}%
\pgfpathlineto{\pgfqpoint{4.442611in}{0.413320in}}%
\pgfpathlineto{\pgfqpoint{4.440041in}{0.413320in}}%
\pgfpathlineto{\pgfqpoint{4.437253in}{0.413320in}}%
\pgfpathlineto{\pgfqpoint{4.434569in}{0.413320in}}%
\pgfpathlineto{\pgfqpoint{4.431901in}{0.413320in}}%
\pgfpathlineto{\pgfqpoint{4.429220in}{0.413320in}}%
\pgfpathlineto{\pgfqpoint{4.426534in}{0.413320in}}%
\pgfpathlineto{\pgfqpoint{4.423863in}{0.413320in}}%
\pgfpathlineto{\pgfqpoint{4.421292in}{0.413320in}}%
\pgfpathlineto{\pgfqpoint{4.418506in}{0.413320in}}%
\pgfpathlineto{\pgfqpoint{4.415932in}{0.413320in}}%
\pgfpathlineto{\pgfqpoint{4.413149in}{0.413320in}}%
\pgfpathlineto{\pgfqpoint{4.410587in}{0.413320in}}%
\pgfpathlineto{\pgfqpoint{4.407788in}{0.413320in}}%
\pgfpathlineto{\pgfqpoint{4.405234in}{0.413320in}}%
\pgfpathlineto{\pgfqpoint{4.402468in}{0.413320in}}%
\pgfpathlineto{\pgfqpoint{4.399745in}{0.413320in}}%
\pgfpathlineto{\pgfqpoint{4.397076in}{0.413320in}}%
\pgfpathlineto{\pgfqpoint{4.394400in}{0.413320in}}%
\pgfpathlineto{\pgfqpoint{4.391721in}{0.413320in}}%
\pgfpathlineto{\pgfqpoint{4.389044in}{0.413320in}}%
\pgfpathlineto{\pgfqpoint{4.386431in}{0.413320in}}%
\pgfpathlineto{\pgfqpoint{4.383674in}{0.413320in}}%
\pgfpathlineto{\pgfqpoint{4.381097in}{0.413320in}}%
\pgfpathlineto{\pgfqpoint{4.378329in}{0.413320in}}%
\pgfpathlineto{\pgfqpoint{4.375761in}{0.413320in}}%
\pgfpathlineto{\pgfqpoint{4.372976in}{0.413320in}}%
\pgfpathlineto{\pgfqpoint{4.370437in}{0.413320in}}%
\pgfpathlineto{\pgfqpoint{4.367646in}{0.413320in}}%
\pgfpathlineto{\pgfqpoint{4.364936in}{0.413320in}}%
\pgfpathlineto{\pgfqpoint{4.362270in}{0.413320in}}%
\pgfpathlineto{\pgfqpoint{4.359582in}{0.413320in}}%
\pgfpathlineto{\pgfqpoint{4.357014in}{0.413320in}}%
\pgfpathlineto{\pgfqpoint{4.354224in}{0.413320in}}%
\pgfpathlineto{\pgfqpoint{4.351645in}{0.413320in}}%
\pgfpathlineto{\pgfqpoint{4.348868in}{0.413320in}}%
\pgfpathlineto{\pgfqpoint{4.346263in}{0.413320in}}%
\pgfpathlineto{\pgfqpoint{4.343510in}{0.413320in}}%
\pgfpathlineto{\pgfqpoint{4.340976in}{0.413320in}}%
\pgfpathlineto{\pgfqpoint{4.338154in}{0.413320in}}%
\pgfpathlineto{\pgfqpoint{4.335463in}{0.413320in}}%
\pgfpathlineto{\pgfqpoint{4.332796in}{0.413320in}}%
\pgfpathlineto{\pgfqpoint{4.330118in}{0.413320in}}%
\pgfpathlineto{\pgfqpoint{4.327440in}{0.413320in}}%
\pgfpathlineto{\pgfqpoint{4.324760in}{0.413320in}}%
\pgfpathlineto{\pgfqpoint{4.322181in}{0.413320in}}%
\pgfpathlineto{\pgfqpoint{4.319405in}{0.413320in}}%
\pgfpathlineto{\pgfqpoint{4.316856in}{0.413320in}}%
\pgfpathlineto{\pgfqpoint{4.314032in}{0.413320in}}%
\pgfpathlineto{\pgfqpoint{4.311494in}{0.413320in}}%
\pgfpathlineto{\pgfqpoint{4.308691in}{0.413320in}}%
\pgfpathlineto{\pgfqpoint{4.306118in}{0.413320in}}%
\pgfpathlineto{\pgfqpoint{4.303357in}{0.413320in}}%
\pgfpathlineto{\pgfqpoint{4.300656in}{0.413320in}}%
\pgfpathlineto{\pgfqpoint{4.297977in}{0.413320in}}%
\pgfpathlineto{\pgfqpoint{4.295299in}{0.413320in}}%
\pgfpathlineto{\pgfqpoint{4.292786in}{0.413320in}}%
\pgfpathlineto{\pgfqpoint{4.289936in}{0.413320in}}%
\pgfpathlineto{\pgfqpoint{4.287399in}{0.413320in}}%
\pgfpathlineto{\pgfqpoint{4.284586in}{0.413320in}}%
\pgfpathlineto{\pgfqpoint{4.282000in}{0.413320in}}%
\pgfpathlineto{\pgfqpoint{4.279212in}{0.413320in}}%
\pgfpathlineto{\pgfqpoint{4.276635in}{0.413320in}}%
\pgfpathlineto{\pgfqpoint{4.273874in}{0.413320in}}%
\pgfpathlineto{\pgfqpoint{4.271187in}{0.413320in}}%
\pgfpathlineto{\pgfqpoint{4.268590in}{0.413320in}}%
\pgfpathlineto{\pgfqpoint{4.265824in}{0.413320in}}%
\pgfpathlineto{\pgfqpoint{4.263157in}{0.413320in}}%
\pgfpathlineto{\pgfqpoint{4.260477in}{0.413320in}}%
\pgfpathlineto{\pgfqpoint{4.257958in}{0.413320in}}%
\pgfpathlineto{\pgfqpoint{4.255120in}{0.413320in}}%
\pgfpathlineto{\pgfqpoint{4.252581in}{0.413320in}}%
\pgfpathlineto{\pgfqpoint{4.249767in}{0.413320in}}%
\pgfpathlineto{\pgfqpoint{4.247225in}{0.413320in}}%
\pgfpathlineto{\pgfqpoint{4.244394in}{0.413320in}}%
\pgfpathlineto{\pgfqpoint{4.241900in}{0.413320in}}%
\pgfpathlineto{\pgfqpoint{4.239084in}{0.413320in}}%
\pgfpathlineto{\pgfqpoint{4.236375in}{0.413320in}}%
\pgfpathlineto{\pgfqpoint{4.233691in}{0.413320in}}%
\pgfpathlineto{\pgfqpoint{4.231013in}{0.413320in}}%
\pgfpathlineto{\pgfqpoint{4.228331in}{0.413320in}}%
\pgfpathlineto{\pgfqpoint{4.225654in}{0.413320in}}%
\pgfpathlineto{\pgfqpoint{4.223082in}{0.413320in}}%
\pgfpathlineto{\pgfqpoint{4.220304in}{0.413320in}}%
\pgfpathlineto{\pgfqpoint{4.217694in}{0.413320in}}%
\pgfpathlineto{\pgfqpoint{4.214948in}{0.413320in}}%
\pgfpathlineto{\pgfqpoint{4.212383in}{0.413320in}}%
\pgfpathlineto{\pgfqpoint{4.209597in}{0.413320in}}%
\pgfpathlineto{\pgfqpoint{4.207076in}{0.413320in}}%
\pgfpathlineto{\pgfqpoint{4.204240in}{0.413320in}}%
\pgfpathlineto{\pgfqpoint{4.201542in}{0.413320in}}%
\pgfpathlineto{\pgfqpoint{4.198878in}{0.413320in}}%
\pgfpathlineto{\pgfqpoint{4.196186in}{0.413320in}}%
\pgfpathlineto{\pgfqpoint{4.193638in}{0.413320in}}%
\pgfpathlineto{\pgfqpoint{4.190842in}{0.413320in}}%
\pgfpathlineto{\pgfqpoint{4.188318in}{0.413320in}}%
\pgfpathlineto{\pgfqpoint{4.185481in}{0.413320in}}%
\pgfpathlineto{\pgfqpoint{4.182899in}{0.413320in}}%
\pgfpathlineto{\pgfqpoint{4.180129in}{0.413320in}}%
\pgfpathlineto{\pgfqpoint{4.177593in}{0.413320in}}%
\pgfpathlineto{\pgfqpoint{4.174770in}{0.413320in}}%
\pgfpathlineto{\pgfqpoint{4.172093in}{0.413320in}}%
\pgfpathlineto{\pgfqpoint{4.169415in}{0.413320in}}%
\pgfpathlineto{\pgfqpoint{4.166737in}{0.413320in}}%
\pgfpathlineto{\pgfqpoint{4.164059in}{0.413320in}}%
\pgfpathlineto{\pgfqpoint{4.161380in}{0.413320in}}%
\pgfpathlineto{\pgfqpoint{4.158806in}{0.413320in}}%
\pgfpathlineto{\pgfqpoint{4.156016in}{0.413320in}}%
\pgfpathlineto{\pgfqpoint{4.153423in}{0.413320in}}%
\pgfpathlineto{\pgfqpoint{4.150665in}{0.413320in}}%
\pgfpathlineto{\pgfqpoint{4.148082in}{0.413320in}}%
\pgfpathlineto{\pgfqpoint{4.145310in}{0.413320in}}%
\pgfpathlineto{\pgfqpoint{4.142713in}{0.413320in}}%
\pgfpathlineto{\pgfqpoint{4.139963in}{0.413320in}}%
\pgfpathlineto{\pgfqpoint{4.137272in}{0.413320in}}%
\pgfpathlineto{\pgfqpoint{4.134615in}{0.413320in}}%
\pgfpathlineto{\pgfqpoint{4.131920in}{0.413320in}}%
\pgfpathlineto{\pgfqpoint{4.129349in}{0.413320in}}%
\pgfpathlineto{\pgfqpoint{4.126553in}{0.413320in}}%
\pgfpathlineto{\pgfqpoint{4.124019in}{0.413320in}}%
\pgfpathlineto{\pgfqpoint{4.121205in}{0.413320in}}%
\pgfpathlineto{\pgfqpoint{4.118554in}{0.413320in}}%
\pgfpathlineto{\pgfqpoint{4.115844in}{0.413320in}}%
\pgfpathlineto{\pgfqpoint{4.113252in}{0.413320in}}%
\pgfpathlineto{\pgfqpoint{4.110488in}{0.413320in}}%
\pgfpathlineto{\pgfqpoint{4.107814in}{0.413320in}}%
\pgfpathlineto{\pgfqpoint{4.105185in}{0.413320in}}%
\pgfpathlineto{\pgfqpoint{4.102456in}{0.413320in}}%
\pgfpathlineto{\pgfqpoint{4.099777in}{0.413320in}}%
\pgfpathlineto{\pgfqpoint{4.097092in}{0.413320in}}%
\pgfpathlineto{\pgfqpoint{4.094527in}{0.413320in}}%
\pgfpathlineto{\pgfqpoint{4.091729in}{0.413320in}}%
\pgfpathlineto{\pgfqpoint{4.089159in}{0.413320in}}%
\pgfpathlineto{\pgfqpoint{4.086385in}{0.413320in}}%
\pgfpathlineto{\pgfqpoint{4.083870in}{0.413320in}}%
\pgfpathlineto{\pgfqpoint{4.081018in}{0.413320in}}%
\pgfpathlineto{\pgfqpoint{4.078471in}{0.413320in}}%
\pgfpathlineto{\pgfqpoint{4.075705in}{0.413320in}}%
\pgfpathlineto{\pgfqpoint{4.072985in}{0.413320in}}%
\pgfpathlineto{\pgfqpoint{4.070313in}{0.413320in}}%
\pgfpathlineto{\pgfqpoint{4.067636in}{0.413320in}}%
\pgfpathlineto{\pgfqpoint{4.064957in}{0.413320in}}%
\pgfpathlineto{\pgfqpoint{4.062266in}{0.413320in}}%
\pgfpathlineto{\pgfqpoint{4.059702in}{0.413320in}}%
\pgfpathlineto{\pgfqpoint{4.056911in}{0.413320in}}%
\pgfpathlineto{\pgfqpoint{4.054326in}{0.413320in}}%
\pgfpathlineto{\pgfqpoint{4.051557in}{0.413320in}}%
\pgfpathlineto{\pgfqpoint{4.049006in}{0.413320in}}%
\pgfpathlineto{\pgfqpoint{4.046210in}{0.413320in}}%
\pgfpathlineto{\pgfqpoint{4.043667in}{0.413320in}}%
\pgfpathlineto{\pgfqpoint{4.040852in}{0.413320in}}%
\pgfpathlineto{\pgfqpoint{4.038174in}{0.413320in}}%
\pgfpathlineto{\pgfqpoint{4.035492in}{0.413320in}}%
\pgfpathlineto{\pgfqpoint{4.032817in}{0.413320in}}%
\pgfpathlineto{\pgfqpoint{4.030229in}{0.413320in}}%
\pgfpathlineto{\pgfqpoint{4.027447in}{0.413320in}}%
\pgfpathlineto{\pgfqpoint{4.024868in}{0.413320in}}%
\pgfpathlineto{\pgfqpoint{4.022097in}{0.413320in}}%
\pgfpathlineto{\pgfqpoint{4.019518in}{0.413320in}}%
\pgfpathlineto{\pgfqpoint{4.016744in}{0.413320in}}%
\pgfpathlineto{\pgfqpoint{4.014186in}{0.413320in}}%
\pgfpathlineto{\pgfqpoint{4.011394in}{0.413320in}}%
\pgfpathlineto{\pgfqpoint{4.008699in}{0.413320in}}%
\pgfpathlineto{\pgfqpoint{4.006034in}{0.413320in}}%
\pgfpathlineto{\pgfqpoint{4.003348in}{0.413320in}}%
\pgfpathlineto{\pgfqpoint{4.000674in}{0.413320in}}%
\pgfpathlineto{\pgfqpoint{3.997990in}{0.413320in}}%
\pgfpathlineto{\pgfqpoint{3.995417in}{0.413320in}}%
\pgfpathlineto{\pgfqpoint{3.992642in}{0.413320in}}%
\pgfpathlineto{\pgfqpoint{3.990055in}{0.413320in}}%
\pgfpathlineto{\pgfqpoint{3.987270in}{0.413320in}}%
\pgfpathlineto{\pgfqpoint{3.984714in}{0.413320in}}%
\pgfpathlineto{\pgfqpoint{3.981929in}{0.413320in}}%
\pgfpathlineto{\pgfqpoint{3.979389in}{0.413320in}}%
\pgfpathlineto{\pgfqpoint{3.976563in}{0.413320in}}%
\pgfpathlineto{\pgfqpoint{3.973885in}{0.413320in}}%
\pgfpathlineto{\pgfqpoint{3.971250in}{0.413320in}}%
\pgfpathlineto{\pgfqpoint{3.968523in}{0.413320in}}%
\pgfpathlineto{\pgfqpoint{3.966013in}{0.413320in}}%
\pgfpathlineto{\pgfqpoint{3.963176in}{0.413320in}}%
\pgfpathlineto{\pgfqpoint{3.960635in}{0.413320in}}%
\pgfpathlineto{\pgfqpoint{3.957823in}{0.413320in}}%
\pgfpathlineto{\pgfqpoint{3.955211in}{0.413320in}}%
\pgfpathlineto{\pgfqpoint{3.952464in}{0.413320in}}%
\pgfpathlineto{\pgfqpoint{3.949894in}{0.413320in}}%
\pgfpathlineto{\pgfqpoint{3.947101in}{0.413320in}}%
\pgfpathlineto{\pgfqpoint{3.944431in}{0.413320in}}%
\pgfpathlineto{\pgfqpoint{3.941778in}{0.413320in}}%
\pgfpathlineto{\pgfqpoint{3.939075in}{0.413320in}}%
\pgfpathlineto{\pgfqpoint{3.936395in}{0.413320in}}%
\pgfpathlineto{\pgfqpoint{3.933714in}{0.413320in}}%
\pgfpathlineto{\pgfqpoint{3.931202in}{0.413320in}}%
\pgfpathlineto{\pgfqpoint{3.928347in}{0.413320in}}%
\pgfpathlineto{\pgfqpoint{3.925778in}{0.413320in}}%
\pgfpathlineto{\pgfqpoint{3.923005in}{0.413320in}}%
\pgfpathlineto{\pgfqpoint{3.920412in}{0.413320in}}%
\pgfpathlineto{\pgfqpoint{3.917646in}{0.413320in}}%
\pgfpathlineto{\pgfqpoint{3.915107in}{0.413320in}}%
\pgfpathlineto{\pgfqpoint{3.912296in}{0.413320in}}%
\pgfpathlineto{\pgfqpoint{3.909602in}{0.413320in}}%
\pgfpathlineto{\pgfqpoint{3.906918in}{0.413320in}}%
\pgfpathlineto{\pgfqpoint{3.904252in}{0.413320in}}%
\pgfpathlineto{\pgfqpoint{3.901573in}{0.413320in}}%
\pgfpathlineto{\pgfqpoint{3.898891in}{0.413320in}}%
\pgfpathlineto{\pgfqpoint{3.896345in}{0.413320in}}%
\pgfpathlineto{\pgfqpoint{3.893541in}{0.413320in}}%
\pgfpathlineto{\pgfqpoint{3.890926in}{0.413320in}}%
\pgfpathlineto{\pgfqpoint{3.888188in}{0.413320in}}%
\pgfpathlineto{\pgfqpoint{3.885621in}{0.413320in}}%
\pgfpathlineto{\pgfqpoint{3.882850in}{0.413320in}}%
\pgfpathlineto{\pgfqpoint{3.880237in}{0.413320in}}%
\pgfpathlineto{\pgfqpoint{3.877466in}{0.413320in}}%
\pgfpathlineto{\pgfqpoint{3.874790in}{0.413320in}}%
\pgfpathlineto{\pgfqpoint{3.872114in}{0.413320in}}%
\pgfpathlineto{\pgfqpoint{3.869435in}{0.413320in}}%
\pgfpathlineto{\pgfqpoint{3.866815in}{0.413320in}}%
\pgfpathlineto{\pgfqpoint{3.864073in}{0.413320in}}%
\pgfpathlineto{\pgfqpoint{3.861561in}{0.413320in}}%
\pgfpathlineto{\pgfqpoint{3.858720in}{0.413320in}}%
\pgfpathlineto{\pgfqpoint{3.856100in}{0.413320in}}%
\pgfpathlineto{\pgfqpoint{3.853358in}{0.413320in}}%
\pgfpathlineto{\pgfqpoint{3.850814in}{0.413320in}}%
\pgfpathlineto{\pgfqpoint{3.848005in}{0.413320in}}%
\pgfpathlineto{\pgfqpoint{3.845329in}{0.413320in}}%
\pgfpathlineto{\pgfqpoint{3.842641in}{0.413320in}}%
\pgfpathlineto{\pgfqpoint{3.839960in}{0.413320in}}%
\pgfpathlineto{\pgfqpoint{3.837286in}{0.413320in}}%
\pgfpathlineto{\pgfqpoint{3.834616in}{0.413320in}}%
\pgfpathlineto{\pgfqpoint{3.832053in}{0.413320in}}%
\pgfpathlineto{\pgfqpoint{3.829252in}{0.413320in}}%
\pgfpathlineto{\pgfqpoint{3.826679in}{0.413320in}}%
\pgfpathlineto{\pgfqpoint{3.823903in}{0.413320in}}%
\pgfpathlineto{\pgfqpoint{3.821315in}{0.413320in}}%
\pgfpathlineto{\pgfqpoint{3.818546in}{0.413320in}}%
\pgfpathlineto{\pgfqpoint{3.815983in}{0.413320in}}%
\pgfpathlineto{\pgfqpoint{3.813172in}{0.413320in}}%
\pgfpathlineto{\pgfqpoint{3.810510in}{0.413320in}}%
\pgfpathlineto{\pgfqpoint{3.807832in}{0.413320in}}%
\pgfpathlineto{\pgfqpoint{3.805145in}{0.413320in}}%
\pgfpathlineto{\pgfqpoint{3.802569in}{0.413320in}}%
\pgfpathlineto{\pgfqpoint{3.799797in}{0.413320in}}%
\pgfpathlineto{\pgfqpoint{3.797265in}{0.413320in}}%
\pgfpathlineto{\pgfqpoint{3.794435in}{0.413320in}}%
\pgfpathlineto{\pgfqpoint{3.791897in}{0.413320in}}%
\pgfpathlineto{\pgfqpoint{3.789084in}{0.413320in}}%
\pgfpathlineto{\pgfqpoint{3.786504in}{0.413320in}}%
\pgfpathlineto{\pgfqpoint{3.783725in}{0.413320in}}%
\pgfpathlineto{\pgfqpoint{3.781046in}{0.413320in}}%
\pgfpathlineto{\pgfqpoint{3.778370in}{0.413320in}}%
\pgfpathlineto{\pgfqpoint{3.775691in}{0.413320in}}%
\pgfpathlineto{\pgfqpoint{3.773014in}{0.413320in}}%
\pgfpathlineto{\pgfqpoint{3.770323in}{0.413320in}}%
\pgfpathlineto{\pgfqpoint{3.767782in}{0.413320in}}%
\pgfpathlineto{\pgfqpoint{3.764966in}{0.413320in}}%
\pgfpathlineto{\pgfqpoint{3.762389in}{0.413320in}}%
\pgfpathlineto{\pgfqpoint{3.759622in}{0.413320in}}%
\pgfpathlineto{\pgfqpoint{3.757065in}{0.413320in}}%
\pgfpathlineto{\pgfqpoint{3.754265in}{0.413320in}}%
\pgfpathlineto{\pgfqpoint{3.751728in}{0.413320in}}%
\pgfpathlineto{\pgfqpoint{3.748903in}{0.413320in}}%
\pgfpathlineto{\pgfqpoint{3.746229in}{0.413320in}}%
\pgfpathlineto{\pgfqpoint{3.743548in}{0.413320in}}%
\pgfpathlineto{\pgfqpoint{3.740874in}{0.413320in}}%
\pgfpathlineto{\pgfqpoint{3.738194in}{0.413320in}}%
\pgfpathlineto{\pgfqpoint{3.735509in}{0.413320in}}%
\pgfpathlineto{\pgfqpoint{3.732950in}{0.413320in}}%
\pgfpathlineto{\pgfqpoint{3.730158in}{0.413320in}}%
\pgfpathlineto{\pgfqpoint{3.727581in}{0.413320in}}%
\pgfpathlineto{\pgfqpoint{3.724804in}{0.413320in}}%
\pgfpathlineto{\pgfqpoint{3.722228in}{0.413320in}}%
\pgfpathlineto{\pgfqpoint{3.719446in}{0.413320in}}%
\pgfpathlineto{\pgfqpoint{3.716875in}{0.413320in}}%
\pgfpathlineto{\pgfqpoint{3.714086in}{0.413320in}}%
\pgfpathlineto{\pgfqpoint{3.711410in}{0.413320in}}%
\pgfpathlineto{\pgfqpoint{3.708729in}{0.413320in}}%
\pgfpathlineto{\pgfqpoint{3.706053in}{0.413320in}}%
\pgfpathlineto{\pgfqpoint{3.703460in}{0.413320in}}%
\pgfpathlineto{\pgfqpoint{3.700684in}{0.413320in}}%
\pgfpathlineto{\pgfqpoint{3.698125in}{0.413320in}}%
\pgfpathlineto{\pgfqpoint{3.695331in}{0.413320in}}%
\pgfpathlineto{\pgfqpoint{3.692765in}{0.413320in}}%
\pgfpathlineto{\pgfqpoint{3.689983in}{0.413320in}}%
\pgfpathlineto{\pgfqpoint{3.687442in}{0.413320in}}%
\pgfpathlineto{\pgfqpoint{3.684620in}{0.413320in}}%
\pgfpathlineto{\pgfqpoint{3.681948in}{0.413320in}}%
\pgfpathlineto{\pgfqpoint{3.679273in}{0.413320in}}%
\pgfpathlineto{\pgfqpoint{3.676591in}{0.413320in}}%
\pgfpathlineto{\pgfqpoint{3.673911in}{0.413320in}}%
\pgfpathlineto{\pgfqpoint{3.671232in}{0.413320in}}%
\pgfpathlineto{\pgfqpoint{3.668665in}{0.413320in}}%
\pgfpathlineto{\pgfqpoint{3.665864in}{0.413320in}}%
\pgfpathlineto{\pgfqpoint{3.663276in}{0.413320in}}%
\pgfpathlineto{\pgfqpoint{3.660515in}{0.413320in}}%
\pgfpathlineto{\pgfqpoint{3.657917in}{0.413320in}}%
\pgfpathlineto{\pgfqpoint{3.655165in}{0.413320in}}%
\pgfpathlineto{\pgfqpoint{3.652628in}{0.413320in}}%
\pgfpathlineto{\pgfqpoint{3.649837in}{0.413320in}}%
\pgfpathlineto{\pgfqpoint{3.647130in}{0.413320in}}%
\pgfpathlineto{\pgfqpoint{3.644452in}{0.413320in}}%
\pgfpathlineto{\pgfqpoint{3.641773in}{0.413320in}}%
\pgfpathlineto{\pgfqpoint{3.639207in}{0.413320in}}%
\pgfpathlineto{\pgfqpoint{3.636413in}{0.413320in}}%
\pgfpathlineto{\pgfqpoint{3.633858in}{0.413320in}}%
\pgfpathlineto{\pgfqpoint{3.631058in}{0.413320in}}%
\pgfpathlineto{\pgfqpoint{3.628460in}{0.413320in}}%
\pgfpathlineto{\pgfqpoint{3.625689in}{0.413320in}}%
\pgfpathlineto{\pgfqpoint{3.623165in}{0.413320in}}%
\pgfpathlineto{\pgfqpoint{3.620345in}{0.413320in}}%
\pgfpathlineto{\pgfqpoint{3.617667in}{0.413320in}}%
\pgfpathlineto{\pgfqpoint{3.614982in}{0.413320in}}%
\pgfpathlineto{\pgfqpoint{3.612311in}{0.413320in}}%
\pgfpathlineto{\pgfqpoint{3.609632in}{0.413320in}}%
\pgfpathlineto{\pgfqpoint{3.606951in}{0.413320in}}%
\pgfpathlineto{\pgfqpoint{3.604387in}{0.413320in}}%
\pgfpathlineto{\pgfqpoint{3.601590in}{0.413320in}}%
\pgfpathlineto{\pgfqpoint{3.598998in}{0.413320in}}%
\pgfpathlineto{\pgfqpoint{3.596240in}{0.413320in}}%
\pgfpathlineto{\pgfqpoint{3.593620in}{0.413320in}}%
\pgfpathlineto{\pgfqpoint{3.590883in}{0.413320in}}%
\pgfpathlineto{\pgfqpoint{3.588258in}{0.413320in}}%
\pgfpathlineto{\pgfqpoint{3.585532in}{0.413320in}}%
\pgfpathlineto{\pgfqpoint{3.582851in}{0.413320in}}%
\pgfpathlineto{\pgfqpoint{3.580191in}{0.413320in}}%
\pgfpathlineto{\pgfqpoint{3.577487in}{0.413320in}}%
\pgfpathlineto{\pgfqpoint{3.574814in}{0.413320in}}%
\pgfpathlineto{\pgfqpoint{3.572126in}{0.413320in}}%
\pgfpathlineto{\pgfqpoint{3.569584in}{0.413320in}}%
\pgfpathlineto{\pgfqpoint{3.566774in}{0.413320in}}%
\pgfpathlineto{\pgfqpoint{3.564188in}{0.413320in}}%
\pgfpathlineto{\pgfqpoint{3.561420in}{0.413320in}}%
\pgfpathlineto{\pgfqpoint{3.558853in}{0.413320in}}%
\pgfpathlineto{\pgfqpoint{3.556061in}{0.413320in}}%
\pgfpathlineto{\pgfqpoint{3.553498in}{0.413320in}}%
\pgfpathlineto{\pgfqpoint{3.550713in}{0.413320in}}%
\pgfpathlineto{\pgfqpoint{3.548029in}{0.413320in}}%
\pgfpathlineto{\pgfqpoint{3.545349in}{0.413320in}}%
\pgfpathlineto{\pgfqpoint{3.542656in}{0.413320in}}%
\pgfpathlineto{\pgfqpoint{3.540093in}{0.413320in}}%
\pgfpathlineto{\pgfqpoint{3.537309in}{0.413320in}}%
\pgfpathlineto{\pgfqpoint{3.534783in}{0.413320in}}%
\pgfpathlineto{\pgfqpoint{3.531955in}{0.413320in}}%
\pgfpathlineto{\pgfqpoint{3.529327in}{0.413320in}}%
\pgfpathlineto{\pgfqpoint{3.526601in}{0.413320in}}%
\pgfpathlineto{\pgfqpoint{3.524041in}{0.413320in}}%
\pgfpathlineto{\pgfqpoint{3.521244in}{0.413320in}}%
\pgfpathlineto{\pgfqpoint{3.518565in}{0.413320in}}%
\pgfpathlineto{\pgfqpoint{3.515884in}{0.413320in}}%
\pgfpathlineto{\pgfqpoint{3.513209in}{0.413320in}}%
\pgfpathlineto{\pgfqpoint{3.510533in}{0.413320in}}%
\pgfpathlineto{\pgfqpoint{3.507840in}{0.413320in}}%
\pgfpathlineto{\pgfqpoint{3.505262in}{0.413320in}}%
\pgfpathlineto{\pgfqpoint{3.502488in}{0.413320in}}%
\pgfpathlineto{\pgfqpoint{3.499909in}{0.413320in}}%
\pgfpathlineto{\pgfqpoint{3.497139in}{0.413320in}}%
\pgfpathlineto{\pgfqpoint{3.494581in}{0.413320in}}%
\pgfpathlineto{\pgfqpoint{3.491783in}{0.413320in}}%
\pgfpathlineto{\pgfqpoint{3.489223in}{0.413320in}}%
\pgfpathlineto{\pgfqpoint{3.486442in}{0.413320in}}%
\pgfpathlineto{\pgfqpoint{3.483744in}{0.413320in}}%
\pgfpathlineto{\pgfqpoint{3.481072in}{0.413320in}}%
\pgfpathlineto{\pgfqpoint{3.478378in}{0.413320in}}%
\pgfpathlineto{\pgfqpoint{3.475821in}{0.413320in}}%
\pgfpathlineto{\pgfqpoint{3.473021in}{0.413320in}}%
\pgfpathlineto{\pgfqpoint{3.470466in}{0.413320in}}%
\pgfpathlineto{\pgfqpoint{3.467678in}{0.413320in}}%
\pgfpathlineto{\pgfqpoint{3.465072in}{0.413320in}}%
\pgfpathlineto{\pgfqpoint{3.462321in}{0.413320in}}%
\pgfpathlineto{\pgfqpoint{3.459695in}{0.413320in}}%
\pgfpathlineto{\pgfqpoint{3.456960in}{0.413320in}}%
\pgfpathlineto{\pgfqpoint{3.454285in}{0.413320in}}%
\pgfpathlineto{\pgfqpoint{3.451597in}{0.413320in}}%
\pgfpathlineto{\pgfqpoint{3.448926in}{0.413320in}}%
\pgfpathlineto{\pgfqpoint{3.446257in}{0.413320in}}%
\pgfpathlineto{\pgfqpoint{3.443574in}{0.413320in}}%
\pgfpathlineto{\pgfqpoint{3.440996in}{0.413320in}}%
\pgfpathlineto{\pgfqpoint{3.438210in}{0.413320in}}%
\pgfpathlineto{\pgfqpoint{3.435635in}{0.413320in}}%
\pgfpathlineto{\pgfqpoint{3.432851in}{0.413320in}}%
\pgfpathlineto{\pgfqpoint{3.430313in}{0.413320in}}%
\pgfpathlineto{\pgfqpoint{3.427501in}{0.413320in}}%
\pgfpathlineto{\pgfqpoint{3.424887in}{0.413320in}}%
\pgfpathlineto{\pgfqpoint{3.422142in}{0.413320in}}%
\pgfpathlineto{\pgfqpoint{3.419455in}{0.413320in}}%
\pgfpathlineto{\pgfqpoint{3.416780in}{0.413320in}}%
\pgfpathlineto{\pgfqpoint{3.414109in}{0.413320in}}%
\pgfpathlineto{\pgfqpoint{3.411431in}{0.413320in}}%
\pgfpathlineto{\pgfqpoint{3.408752in}{0.413320in}}%
\pgfpathlineto{\pgfqpoint{3.406202in}{0.413320in}}%
\pgfpathlineto{\pgfqpoint{3.403394in}{0.413320in}}%
\pgfpathlineto{\pgfqpoint{3.400783in}{0.413320in}}%
\pgfpathlineto{\pgfqpoint{3.398037in}{0.413320in}}%
\pgfpathlineto{\pgfqpoint{3.395461in}{0.413320in}}%
\pgfpathlineto{\pgfqpoint{3.392681in}{0.413320in}}%
\pgfpathlineto{\pgfqpoint{3.390102in}{0.413320in}}%
\pgfpathlineto{\pgfqpoint{3.387309in}{0.413320in}}%
\pgfpathlineto{\pgfqpoint{3.384647in}{0.413320in}}%
\pgfpathlineto{\pgfqpoint{3.381959in}{0.413320in}}%
\pgfpathlineto{\pgfqpoint{3.379290in}{0.413320in}}%
\pgfpathlineto{\pgfqpoint{3.376735in}{0.413320in}}%
\pgfpathlineto{\pgfqpoint{3.373921in}{0.413320in}}%
\pgfpathlineto{\pgfqpoint{3.371357in}{0.413320in}}%
\pgfpathlineto{\pgfqpoint{3.368577in}{0.413320in}}%
\pgfpathlineto{\pgfqpoint{3.365996in}{0.413320in}}%
\pgfpathlineto{\pgfqpoint{3.363221in}{0.413320in}}%
\pgfpathlineto{\pgfqpoint{3.360620in}{0.413320in}}%
\pgfpathlineto{\pgfqpoint{3.357862in}{0.413320in}}%
\pgfpathlineto{\pgfqpoint{3.355177in}{0.413320in}}%
\pgfpathlineto{\pgfqpoint{3.352505in}{0.413320in}}%
\pgfpathlineto{\pgfqpoint{3.349828in}{0.413320in}}%
\pgfpathlineto{\pgfqpoint{3.347139in}{0.413320in}}%
\pgfpathlineto{\pgfqpoint{3.344468in}{0.413320in}}%
\pgfpathlineto{\pgfqpoint{3.341893in}{0.413320in}}%
\pgfpathlineto{\pgfqpoint{3.339101in}{0.413320in}}%
\pgfpathlineto{\pgfqpoint{3.336541in}{0.413320in}}%
\pgfpathlineto{\pgfqpoint{3.333758in}{0.413320in}}%
\pgfpathlineto{\pgfqpoint{3.331183in}{0.413320in}}%
\pgfpathlineto{\pgfqpoint{3.328401in}{0.413320in}}%
\pgfpathlineto{\pgfqpoint{3.325860in}{0.413320in}}%
\pgfpathlineto{\pgfqpoint{3.323049in}{0.413320in}}%
\pgfpathlineto{\pgfqpoint{3.320366in}{0.413320in}}%
\pgfpathlineto{\pgfqpoint{3.317688in}{0.413320in}}%
\pgfpathlineto{\pgfqpoint{3.315008in}{0.413320in}}%
\pgfpathlineto{\pgfqpoint{3.312480in}{0.413320in}}%
\pgfpathlineto{\pgfqpoint{3.309652in}{0.413320in}}%
\pgfpathlineto{\pgfqpoint{3.307104in}{0.413320in}}%
\pgfpathlineto{\pgfqpoint{3.304295in}{0.413320in}}%
\pgfpathlineto{\pgfqpoint{3.301719in}{0.413320in}}%
\pgfpathlineto{\pgfqpoint{3.298937in}{0.413320in}}%
\pgfpathlineto{\pgfqpoint{3.296376in}{0.413320in}}%
\pgfpathlineto{\pgfqpoint{3.293574in}{0.413320in}}%
\pgfpathlineto{\pgfqpoint{3.290890in}{0.413320in}}%
\pgfpathlineto{\pgfqpoint{3.288225in}{0.413320in}}%
\pgfpathlineto{\pgfqpoint{3.285534in}{0.413320in}}%
\pgfpathlineto{\pgfqpoint{3.282870in}{0.413320in}}%
\pgfpathlineto{\pgfqpoint{3.280189in}{0.413320in}}%
\pgfpathlineto{\pgfqpoint{3.277603in}{0.413320in}}%
\pgfpathlineto{\pgfqpoint{3.274831in}{0.413320in}}%
\pgfpathlineto{\pgfqpoint{3.272254in}{0.413320in}}%
\pgfpathlineto{\pgfqpoint{3.269478in}{0.413320in}}%
\pgfpathlineto{\pgfqpoint{3.266849in}{0.413320in}}%
\pgfpathlineto{\pgfqpoint{3.264119in}{0.413320in}}%
\pgfpathlineto{\pgfqpoint{3.261594in}{0.413320in}}%
\pgfpathlineto{\pgfqpoint{3.258784in}{0.413320in}}%
\pgfpathlineto{\pgfqpoint{3.256083in}{0.413320in}}%
\pgfpathlineto{\pgfqpoint{3.253404in}{0.413320in}}%
\pgfpathlineto{\pgfqpoint{3.250716in}{0.413320in}}%
\pgfpathlineto{\pgfqpoint{3.248049in}{0.413320in}}%
\pgfpathlineto{\pgfqpoint{3.245363in}{0.413320in}}%
\pgfpathlineto{\pgfqpoint{3.242807in}{0.413320in}}%
\pgfpathlineto{\pgfqpoint{3.240010in}{0.413320in}}%
\pgfpathlineto{\pgfqpoint{3.237411in}{0.413320in}}%
\pgfpathlineto{\pgfqpoint{3.234658in}{0.413320in}}%
\pgfpathlineto{\pgfqpoint{3.232069in}{0.413320in}}%
\pgfpathlineto{\pgfqpoint{3.229310in}{0.413320in}}%
\pgfpathlineto{\pgfqpoint{3.226609in}{0.413320in}}%
\pgfpathlineto{\pgfqpoint{3.223942in}{0.413320in}}%
\pgfpathlineto{\pgfqpoint{3.221255in}{0.413320in}}%
\pgfpathlineto{\pgfqpoint{3.218586in}{0.413320in}}%
\pgfpathlineto{\pgfqpoint{3.215908in}{0.413320in}}%
\pgfpathlineto{\pgfqpoint{3.213342in}{0.413320in}}%
\pgfpathlineto{\pgfqpoint{3.210545in}{0.413320in}}%
\pgfpathlineto{\pgfqpoint{3.207984in}{0.413320in}}%
\pgfpathlineto{\pgfqpoint{3.205195in}{0.413320in}}%
\pgfpathlineto{\pgfqpoint{3.202562in}{0.413320in}}%
\pgfpathlineto{\pgfqpoint{3.199823in}{0.413320in}}%
\pgfpathlineto{\pgfqpoint{3.197226in}{0.413320in}}%
\pgfpathlineto{\pgfqpoint{3.194508in}{0.413320in}}%
\pgfpathlineto{\pgfqpoint{3.191796in}{0.413320in}}%
\pgfpathlineto{\pgfqpoint{3.189117in}{0.413320in}}%
\pgfpathlineto{\pgfqpoint{3.186440in}{0.413320in}}%
\pgfpathlineto{\pgfqpoint{3.183760in}{0.413320in}}%
\pgfpathlineto{\pgfqpoint{3.181089in}{0.413320in}}%
\pgfpathlineto{\pgfqpoint{3.178525in}{0.413320in}}%
\pgfpathlineto{\pgfqpoint{3.175724in}{0.413320in}}%
\pgfpathlineto{\pgfqpoint{3.173142in}{0.413320in}}%
\pgfpathlineto{\pgfqpoint{3.170375in}{0.413320in}}%
\pgfpathlineto{\pgfqpoint{3.167776in}{0.413320in}}%
\pgfpathlineto{\pgfqpoint{3.165019in}{0.413320in}}%
\pgfpathlineto{\pgfqpoint{3.162474in}{0.413320in}}%
\pgfpathlineto{\pgfqpoint{3.159675in}{0.413320in}}%
\pgfpathlineto{\pgfqpoint{3.156981in}{0.413320in}}%
\pgfpathlineto{\pgfqpoint{3.154327in}{0.413320in}}%
\pgfpathlineto{\pgfqpoint{3.151612in}{0.413320in}}%
\pgfpathlineto{\pgfqpoint{3.149057in}{0.413320in}}%
\pgfpathlineto{\pgfqpoint{3.146271in}{0.413320in}}%
\pgfpathlineto{\pgfqpoint{3.143740in}{0.413320in}}%
\pgfpathlineto{\pgfqpoint{3.140913in}{0.413320in}}%
\pgfpathlineto{\pgfqpoint{3.138375in}{0.413320in}}%
\pgfpathlineto{\pgfqpoint{3.135550in}{0.413320in}}%
\pgfpathlineto{\pgfqpoint{3.132946in}{0.413320in}}%
\pgfpathlineto{\pgfqpoint{3.130199in}{0.413320in}}%
\pgfpathlineto{\pgfqpoint{3.127512in}{0.413320in}}%
\pgfpathlineto{\pgfqpoint{3.124842in}{0.413320in}}%
\pgfpathlineto{\pgfqpoint{3.122163in}{0.413320in}}%
\pgfpathlineto{\pgfqpoint{3.119487in}{0.413320in}}%
\pgfpathlineto{\pgfqpoint{3.116807in}{0.413320in}}%
\pgfpathlineto{\pgfqpoint{3.114242in}{0.413320in}}%
\pgfpathlineto{\pgfqpoint{3.111451in}{0.413320in}}%
\pgfpathlineto{\pgfqpoint{3.108896in}{0.413320in}}%
\pgfpathlineto{\pgfqpoint{3.106094in}{0.413320in}}%
\pgfpathlineto{\pgfqpoint{3.103508in}{0.413320in}}%
\pgfpathlineto{\pgfqpoint{3.100737in}{0.413320in}}%
\pgfpathlineto{\pgfqpoint{3.098163in}{0.413320in}}%
\pgfpathlineto{\pgfqpoint{3.095388in}{0.413320in}}%
\pgfpathlineto{\pgfqpoint{3.092699in}{0.413320in}}%
\pgfpathlineto{\pgfqpoint{3.090023in}{0.413320in}}%
\pgfpathlineto{\pgfqpoint{3.087343in}{0.413320in}}%
\pgfpathlineto{\pgfqpoint{3.084671in}{0.413320in}}%
\pgfpathlineto{\pgfqpoint{3.081990in}{0.413320in}}%
\pgfpathlineto{\pgfqpoint{3.079381in}{0.413320in}}%
\pgfpathlineto{\pgfqpoint{3.076631in}{0.413320in}}%
\pgfpathlineto{\pgfqpoint{3.074056in}{0.413320in}}%
\pgfpathlineto{\pgfqpoint{3.071266in}{0.413320in}}%
\pgfpathlineto{\pgfqpoint{3.068709in}{0.413320in}}%
\pgfpathlineto{\pgfqpoint{3.065916in}{0.413320in}}%
\pgfpathlineto{\pgfqpoint{3.063230in}{0.413320in}}%
\pgfpathlineto{\pgfqpoint{3.060561in}{0.413320in}}%
\pgfpathlineto{\pgfqpoint{3.057884in}{0.413320in}}%
\pgfpathlineto{\pgfqpoint{3.055202in}{0.413320in}}%
\pgfpathlineto{\pgfqpoint{3.052526in}{0.413320in}}%
\pgfpathlineto{\pgfqpoint{3.049988in}{0.413320in}}%
\pgfpathlineto{\pgfqpoint{3.047157in}{0.413320in}}%
\pgfpathlineto{\pgfqpoint{3.044568in}{0.413320in}}%
\pgfpathlineto{\pgfqpoint{3.041813in}{0.413320in}}%
\pgfpathlineto{\pgfqpoint{3.039262in}{0.413320in}}%
\pgfpathlineto{\pgfqpoint{3.036456in}{0.413320in}}%
\pgfpathlineto{\pgfqpoint{3.033921in}{0.413320in}}%
\pgfpathlineto{\pgfqpoint{3.031091in}{0.413320in}}%
\pgfpathlineto{\pgfqpoint{3.028412in}{0.413320in}}%
\pgfpathlineto{\pgfqpoint{3.025803in}{0.413320in}}%
\pgfpathlineto{\pgfqpoint{3.023058in}{0.413320in}}%
\pgfpathlineto{\pgfqpoint{3.020382in}{0.413320in}}%
\pgfpathlineto{\pgfqpoint{3.017707in}{0.413320in}}%
\pgfpathlineto{\pgfqpoint{3.015097in}{0.413320in}}%
\pgfpathlineto{\pgfqpoint{3.012351in}{0.413320in}}%
\pgfpathlineto{\pgfqpoint{3.009784in}{0.413320in}}%
\pgfpathlineto{\pgfqpoint{3.006993in}{0.413320in}}%
\pgfpathlineto{\pgfqpoint{3.004419in}{0.413320in}}%
\pgfpathlineto{\pgfqpoint{3.001635in}{0.413320in}}%
\pgfpathlineto{\pgfqpoint{2.999103in}{0.413320in}}%
\pgfpathlineto{\pgfqpoint{2.996300in}{0.413320in}}%
\pgfpathlineto{\pgfqpoint{2.993595in}{0.413320in}}%
\pgfpathlineto{\pgfqpoint{2.990978in}{0.413320in}}%
\pgfpathlineto{\pgfqpoint{2.988238in}{0.413320in}}%
\pgfpathlineto{\pgfqpoint{2.985666in}{0.413320in}}%
\pgfpathlineto{\pgfqpoint{2.982885in}{0.413320in}}%
\pgfpathlineto{\pgfqpoint{2.980341in}{0.413320in}}%
\pgfpathlineto{\pgfqpoint{2.977517in}{0.413320in}}%
\pgfpathlineto{\pgfqpoint{2.974972in}{0.413320in}}%
\pgfpathlineto{\pgfqpoint{2.972177in}{0.413320in}}%
\pgfpathlineto{\pgfqpoint{2.969599in}{0.413320in}}%
\pgfpathlineto{\pgfqpoint{2.966812in}{0.413320in}}%
\pgfpathlineto{\pgfqpoint{2.964127in}{0.413320in}}%
\pgfpathlineto{\pgfqpoint{2.961460in}{0.413320in}}%
\pgfpathlineto{\pgfqpoint{2.958782in}{0.413320in}}%
\pgfpathlineto{\pgfqpoint{2.956103in}{0.413320in}}%
\pgfpathlineto{\pgfqpoint{2.953422in}{0.413320in}}%
\pgfpathlineto{\pgfqpoint{2.950884in}{0.413320in}}%
\pgfpathlineto{\pgfqpoint{2.948068in}{0.413320in}}%
\pgfpathlineto{\pgfqpoint{2.945461in}{0.413320in}}%
\pgfpathlineto{\pgfqpoint{2.942711in}{0.413320in}}%
\pgfpathlineto{\pgfqpoint{2.940120in}{0.413320in}}%
\pgfpathlineto{\pgfqpoint{2.937352in}{0.413320in}}%
\pgfpathlineto{\pgfqpoint{2.934759in}{0.413320in}}%
\pgfpathlineto{\pgfqpoint{2.932033in}{0.413320in}}%
\pgfpathlineto{\pgfqpoint{2.929321in}{0.413320in}}%
\pgfpathlineto{\pgfqpoint{2.926655in}{0.413320in}}%
\pgfpathlineto{\pgfqpoint{2.923963in}{0.413320in}}%
\pgfpathlineto{\pgfqpoint{2.921363in}{0.413320in}}%
\pgfpathlineto{\pgfqpoint{2.918606in}{0.413320in}}%
\pgfpathlineto{\pgfqpoint{2.916061in}{0.413320in}}%
\pgfpathlineto{\pgfqpoint{2.913243in}{0.413320in}}%
\pgfpathlineto{\pgfqpoint{2.910631in}{0.413320in}}%
\pgfpathlineto{\pgfqpoint{2.907882in}{0.413320in}}%
\pgfpathlineto{\pgfqpoint{2.905341in}{0.413320in}}%
\pgfpathlineto{\pgfqpoint{2.902535in}{0.413320in}}%
\pgfpathlineto{\pgfqpoint{2.899858in}{0.413320in}}%
\pgfpathlineto{\pgfqpoint{2.897179in}{0.413320in}}%
\pgfpathlineto{\pgfqpoint{2.894487in}{0.413320in}}%
\pgfpathlineto{\pgfqpoint{2.891809in}{0.413320in}}%
\pgfpathlineto{\pgfqpoint{2.889145in}{0.413320in}}%
\pgfpathlineto{\pgfqpoint{2.886578in}{0.413320in}}%
\pgfpathlineto{\pgfqpoint{2.883780in}{0.413320in}}%
\pgfpathlineto{\pgfqpoint{2.881254in}{0.413320in}}%
\pgfpathlineto{\pgfqpoint{2.878431in}{0.413320in}}%
\pgfpathlineto{\pgfqpoint{2.875882in}{0.413320in}}%
\pgfpathlineto{\pgfqpoint{2.873074in}{0.413320in}}%
\pgfpathlineto{\pgfqpoint{2.870475in}{0.413320in}}%
\pgfpathlineto{\pgfqpoint{2.867713in}{0.413320in}}%
\pgfpathlineto{\pgfqpoint{2.865031in}{0.413320in}}%
\pgfpathlineto{\pgfqpoint{2.862402in}{0.413320in}}%
\pgfpathlineto{\pgfqpoint{2.859668in}{0.413320in}}%
\pgfpathlineto{\pgfqpoint{2.857003in}{0.413320in}}%
\pgfpathlineto{\pgfqpoint{2.854325in}{0.413320in}}%
\pgfpathlineto{\pgfqpoint{2.851793in}{0.413320in}}%
\pgfpathlineto{\pgfqpoint{2.848960in}{0.413320in}}%
\pgfpathlineto{\pgfqpoint{2.846408in}{0.413320in}}%
\pgfpathlineto{\pgfqpoint{2.843611in}{0.413320in}}%
\pgfpathlineto{\pgfqpoint{2.841055in}{0.413320in}}%
\pgfpathlineto{\pgfqpoint{2.838254in}{0.413320in}}%
\pgfpathlineto{\pgfqpoint{2.835698in}{0.413320in}}%
\pgfpathlineto{\pgfqpoint{2.832894in}{0.413320in}}%
\pgfpathlineto{\pgfqpoint{2.830219in}{0.413320in}}%
\pgfpathlineto{\pgfqpoint{2.827567in}{0.413320in}}%
\pgfpathlineto{\pgfqpoint{2.824851in}{0.413320in}}%
\pgfpathlineto{\pgfqpoint{2.822303in}{0.413320in}}%
\pgfpathlineto{\pgfqpoint{2.819506in}{0.413320in}}%
\pgfpathlineto{\pgfqpoint{2.816867in}{0.413320in}}%
\pgfpathlineto{\pgfqpoint{2.814141in}{0.413320in}}%
\pgfpathlineto{\pgfqpoint{2.811597in}{0.413320in}}%
\pgfpathlineto{\pgfqpoint{2.808792in}{0.413320in}}%
\pgfpathlineto{\pgfqpoint{2.806175in}{0.413320in}}%
\pgfpathlineto{\pgfqpoint{2.803435in}{0.413320in}}%
\pgfpathlineto{\pgfqpoint{2.800756in}{0.413320in}}%
\pgfpathlineto{\pgfqpoint{2.798070in}{0.413320in}}%
\pgfpathlineto{\pgfqpoint{2.795398in}{0.413320in}}%
\pgfpathlineto{\pgfqpoint{2.792721in}{0.413320in}}%
\pgfpathlineto{\pgfqpoint{2.790044in}{0.413320in}}%
\pgfpathlineto{\pgfqpoint{2.787468in}{0.413320in}}%
\pgfpathlineto{\pgfqpoint{2.784687in}{0.413320in}}%
\pgfpathlineto{\pgfqpoint{2.782113in}{0.413320in}}%
\pgfpathlineto{\pgfqpoint{2.779330in}{0.413320in}}%
\pgfpathlineto{\pgfqpoint{2.776767in}{0.413320in}}%
\pgfpathlineto{\pgfqpoint{2.773972in}{0.413320in}}%
\pgfpathlineto{\pgfqpoint{2.771373in}{0.413320in}}%
\pgfpathlineto{\pgfqpoint{2.768617in}{0.413320in}}%
\pgfpathlineto{\pgfqpoint{2.765935in}{0.413320in}}%
\pgfpathlineto{\pgfqpoint{2.763253in}{0.413320in}}%
\pgfpathlineto{\pgfqpoint{2.760581in}{0.413320in}}%
\pgfpathlineto{\pgfqpoint{2.758028in}{0.413320in}}%
\pgfpathlineto{\pgfqpoint{2.755224in}{0.413320in}}%
\pgfpathlineto{\pgfqpoint{2.752614in}{0.413320in}}%
\pgfpathlineto{\pgfqpoint{2.749868in}{0.413320in}}%
\pgfpathlineto{\pgfqpoint{2.747260in}{0.413320in}}%
\pgfpathlineto{\pgfqpoint{2.744510in}{0.413320in}}%
\pgfpathlineto{\pgfqpoint{2.741928in}{0.413320in}}%
\pgfpathlineto{\pgfqpoint{2.739155in}{0.413320in}}%
\pgfpathlineto{\pgfqpoint{2.736476in}{0.413320in}}%
\pgfpathlineto{\pgfqpoint{2.733798in}{0.413320in}}%
\pgfpathlineto{\pgfqpoint{2.731119in}{0.413320in}}%
\pgfpathlineto{\pgfqpoint{2.728439in}{0.413320in}}%
\pgfpathlineto{\pgfqpoint{2.725760in}{0.413320in}}%
\pgfpathlineto{\pgfqpoint{2.723211in}{0.413320in}}%
\pgfpathlineto{\pgfqpoint{2.720404in}{0.413320in}}%
\pgfpathlineto{\pgfqpoint{2.717773in}{0.413320in}}%
\pgfpathlineto{\pgfqpoint{2.715036in}{0.413320in}}%
\pgfpathlineto{\pgfqpoint{2.712477in}{0.413320in}}%
\pgfpathlineto{\pgfqpoint{2.709683in}{0.413320in}}%
\pgfpathlineto{\pgfqpoint{2.707125in}{0.413320in}}%
\pgfpathlineto{\pgfqpoint{2.704326in}{0.413320in}}%
\pgfpathlineto{\pgfqpoint{2.701657in}{0.413320in}}%
\pgfpathlineto{\pgfqpoint{2.698968in}{0.413320in}}%
\pgfpathlineto{\pgfqpoint{2.696293in}{0.413320in}}%
\pgfpathlineto{\pgfqpoint{2.693611in}{0.413320in}}%
\pgfpathlineto{\pgfqpoint{2.690940in}{0.413320in}}%
\pgfpathlineto{\pgfqpoint{2.688328in}{0.413320in}}%
\pgfpathlineto{\pgfqpoint{2.685586in}{0.413320in}}%
\pgfpathlineto{\pgfqpoint{2.683009in}{0.413320in}}%
\pgfpathlineto{\pgfqpoint{2.680224in}{0.413320in}}%
\pgfpathlineto{\pgfqpoint{2.677650in}{0.413320in}}%
\pgfpathlineto{\pgfqpoint{2.674873in}{0.413320in}}%
\pgfpathlineto{\pgfqpoint{2.672301in}{0.413320in}}%
\pgfpathlineto{\pgfqpoint{2.669506in}{0.413320in}}%
\pgfpathlineto{\pgfqpoint{2.666836in}{0.413320in}}%
\pgfpathlineto{\pgfqpoint{2.664151in}{0.413320in}}%
\pgfpathlineto{\pgfqpoint{2.661481in}{0.413320in}}%
\pgfpathlineto{\pgfqpoint{2.658942in}{0.413320in}}%
\pgfpathlineto{\pgfqpoint{2.656124in}{0.413320in}}%
\pgfpathlineto{\pgfqpoint{2.653567in}{0.413320in}}%
\pgfpathlineto{\pgfqpoint{2.650767in}{0.413320in}}%
\pgfpathlineto{\pgfqpoint{2.648196in}{0.413320in}}%
\pgfpathlineto{\pgfqpoint{2.645408in}{0.413320in}}%
\pgfpathlineto{\pgfqpoint{2.642827in}{0.413320in}}%
\pgfpathlineto{\pgfqpoint{2.640053in}{0.413320in}}%
\pgfpathlineto{\pgfqpoint{2.637369in}{0.413320in}}%
\pgfpathlineto{\pgfqpoint{2.634700in}{0.413320in}}%
\pgfpathlineto{\pgfqpoint{2.632018in}{0.413320in}}%
\pgfpathlineto{\pgfqpoint{2.629340in}{0.413320in}}%
\pgfpathlineto{\pgfqpoint{2.626653in}{0.413320in}}%
\pgfpathlineto{\pgfqpoint{2.624077in}{0.413320in}}%
\pgfpathlineto{\pgfqpoint{2.621304in}{0.413320in}}%
\pgfpathlineto{\pgfqpoint{2.618773in}{0.413320in}}%
\pgfpathlineto{\pgfqpoint{2.615934in}{0.413320in}}%
\pgfpathlineto{\pgfqpoint{2.613393in}{0.413320in}}%
\pgfpathlineto{\pgfqpoint{2.610588in}{0.413320in}}%
\pgfpathlineto{\pgfqpoint{2.608004in}{0.413320in}}%
\pgfpathlineto{\pgfqpoint{2.605232in}{0.413320in}}%
\pgfpathlineto{\pgfqpoint{2.602557in}{0.413320in}}%
\pgfpathlineto{\pgfqpoint{2.599920in}{0.413320in}}%
\pgfpathlineto{\pgfqpoint{2.597196in}{0.413320in}}%
\pgfpathlineto{\pgfqpoint{2.594630in}{0.413320in}}%
\pgfpathlineto{\pgfqpoint{2.591842in}{0.413320in}}%
\pgfpathlineto{\pgfqpoint{2.589248in}{0.413320in}}%
\pgfpathlineto{\pgfqpoint{2.586484in}{0.413320in}}%
\pgfpathlineto{\pgfqpoint{2.583913in}{0.413320in}}%
\pgfpathlineto{\pgfqpoint{2.581129in}{0.413320in}}%
\pgfpathlineto{\pgfqpoint{2.578567in}{0.413320in}}%
\pgfpathlineto{\pgfqpoint{2.575779in}{0.413320in}}%
\pgfpathlineto{\pgfqpoint{2.573082in}{0.413320in}}%
\pgfpathlineto{\pgfqpoint{2.570411in}{0.413320in}}%
\pgfpathlineto{\pgfqpoint{2.567730in}{0.413320in}}%
\pgfpathlineto{\pgfqpoint{2.565045in}{0.413320in}}%
\pgfpathlineto{\pgfqpoint{2.562375in}{0.413320in}}%
\pgfpathlineto{\pgfqpoint{2.559790in}{0.413320in}}%
\pgfpathlineto{\pgfqpoint{2.557009in}{0.413320in}}%
\pgfpathlineto{\pgfqpoint{2.554493in}{0.413320in}}%
\pgfpathlineto{\pgfqpoint{2.551664in}{0.413320in}}%
\pgfpathlineto{\pgfqpoint{2.549114in}{0.413320in}}%
\pgfpathlineto{\pgfqpoint{2.546310in}{0.413320in}}%
\pgfpathlineto{\pgfqpoint{2.543765in}{0.413320in}}%
\pgfpathlineto{\pgfqpoint{2.540949in}{0.413320in}}%
\pgfpathlineto{\pgfqpoint{2.538274in}{0.413320in}}%
\pgfpathlineto{\pgfqpoint{2.535624in}{0.413320in}}%
\pgfpathlineto{\pgfqpoint{2.532917in}{0.413320in}}%
\pgfpathlineto{\pgfqpoint{2.530234in}{0.413320in}}%
\pgfpathlineto{\pgfqpoint{2.527560in}{0.413320in}}%
\pgfpathlineto{\pgfqpoint{2.524988in}{0.413320in}}%
\pgfpathlineto{\pgfqpoint{2.522197in}{0.413320in}}%
\pgfpathlineto{\pgfqpoint{2.519607in}{0.413320in}}%
\pgfpathlineto{\pgfqpoint{2.516845in}{0.413320in}}%
\pgfpathlineto{\pgfqpoint{2.514268in}{0.413320in}}%
\pgfpathlineto{\pgfqpoint{2.511478in}{0.413320in}}%
\pgfpathlineto{\pgfqpoint{2.508917in}{0.413320in}}%
\pgfpathlineto{\pgfqpoint{2.506163in}{0.413320in}}%
\pgfpathlineto{\pgfqpoint{2.503454in}{0.413320in}}%
\pgfpathlineto{\pgfqpoint{2.500801in}{0.413320in}}%
\pgfpathlineto{\pgfqpoint{2.498085in}{0.413320in}}%
\pgfpathlineto{\pgfqpoint{2.495542in}{0.413320in}}%
\pgfpathlineto{\pgfqpoint{2.492729in}{0.413320in}}%
\pgfpathlineto{\pgfqpoint{2.490183in}{0.413320in}}%
\pgfpathlineto{\pgfqpoint{2.487384in}{0.413320in}}%
\pgfpathlineto{\pgfqpoint{2.484870in}{0.413320in}}%
\pgfpathlineto{\pgfqpoint{2.482026in}{0.413320in}}%
\pgfpathlineto{\pgfqpoint{2.479420in}{0.413320in}}%
\pgfpathlineto{\pgfqpoint{2.476671in}{0.413320in}}%
\pgfpathlineto{\pgfqpoint{2.473989in}{0.413320in}}%
\pgfpathlineto{\pgfqpoint{2.471311in}{0.413320in}}%
\pgfpathlineto{\pgfqpoint{2.468635in}{0.413320in}}%
\pgfpathlineto{\pgfqpoint{2.465957in}{0.413320in}}%
\pgfpathlineto{\pgfqpoint{2.463280in}{0.413320in}}%
\pgfpathlineto{\pgfqpoint{2.460711in}{0.413320in}}%
\pgfpathlineto{\pgfqpoint{2.457917in}{0.413320in}}%
\pgfpathlineto{\pgfqpoint{2.455353in}{0.413320in}}%
\pgfpathlineto{\pgfqpoint{2.452562in}{0.413320in}}%
\pgfpathlineto{\pgfqpoint{2.450032in}{0.413320in}}%
\pgfpathlineto{\pgfqpoint{2.447209in}{0.413320in}}%
\pgfpathlineto{\pgfqpoint{2.444677in}{0.413320in}}%
\pgfpathlineto{\pgfqpoint{2.441876in}{0.413320in}}%
\pgfpathlineto{\pgfqpoint{2.439167in}{0.413320in}}%
\pgfpathlineto{\pgfqpoint{2.436518in}{0.413320in}}%
\pgfpathlineto{\pgfqpoint{2.433815in}{0.413320in}}%
\pgfpathlineto{\pgfqpoint{2.431251in}{0.413320in}}%
\pgfpathlineto{\pgfqpoint{2.428453in}{0.413320in}}%
\pgfpathlineto{\pgfqpoint{2.425878in}{0.413320in}}%
\pgfpathlineto{\pgfqpoint{2.423098in}{0.413320in}}%
\pgfpathlineto{\pgfqpoint{2.420528in}{0.413320in}}%
\pgfpathlineto{\pgfqpoint{2.417747in}{0.413320in}}%
\pgfpathlineto{\pgfqpoint{2.415184in}{0.413320in}}%
\pgfpathlineto{\pgfqpoint{2.412389in}{0.413320in}}%
\pgfpathlineto{\pgfqpoint{2.409699in}{0.413320in}}%
\pgfpathlineto{\pgfqpoint{2.407024in}{0.413320in}}%
\pgfpathlineto{\pgfqpoint{2.404352in}{0.413320in}}%
\pgfpathlineto{\pgfqpoint{2.401675in}{0.413320in}}%
\pgfpathlineto{\pgfqpoint{2.398995in}{0.413320in}}%
\pgfpathclose%
\pgfusepath{stroke,fill}%
\end{pgfscope}%
\begin{pgfscope}%
\pgfpathrectangle{\pgfqpoint{2.398995in}{0.319877in}}{\pgfqpoint{3.986877in}{1.993438in}} %
\pgfusepath{clip}%
\pgfsetbuttcap%
\pgfsetroundjoin%
\definecolor{currentfill}{rgb}{1.000000,1.000000,1.000000}%
\pgfsetfillcolor{currentfill}%
\pgfsetlinewidth{1.003750pt}%
\definecolor{currentstroke}{rgb}{0.204335,0.686386,0.540710}%
\pgfsetstrokecolor{currentstroke}%
\pgfsetdash{}{0pt}%
\pgfpathmoveto{\pgfqpoint{2.398995in}{0.413320in}}%
\pgfpathlineto{\pgfqpoint{2.398995in}{2.035554in}}%
\pgfpathlineto{\pgfqpoint{2.401675in}{2.043142in}}%
\pgfpathlineto{\pgfqpoint{2.404352in}{2.041256in}}%
\pgfpathlineto{\pgfqpoint{2.407024in}{2.038052in}}%
\pgfpathlineto{\pgfqpoint{2.409699in}{2.036363in}}%
\pgfpathlineto{\pgfqpoint{2.412389in}{2.040812in}}%
\pgfpathlineto{\pgfqpoint{2.415184in}{2.038377in}}%
\pgfpathlineto{\pgfqpoint{2.417747in}{2.037512in}}%
\pgfpathlineto{\pgfqpoint{2.420528in}{2.038173in}}%
\pgfpathlineto{\pgfqpoint{2.423098in}{2.037887in}}%
\pgfpathlineto{\pgfqpoint{2.425878in}{2.037986in}}%
\pgfpathlineto{\pgfqpoint{2.428453in}{2.036068in}}%
\pgfpathlineto{\pgfqpoint{2.431251in}{2.037826in}}%
\pgfpathlineto{\pgfqpoint{2.433815in}{2.041119in}}%
\pgfpathlineto{\pgfqpoint{2.436518in}{2.039273in}}%
\pgfpathlineto{\pgfqpoint{2.439167in}{2.041786in}}%
\pgfpathlineto{\pgfqpoint{2.441876in}{2.040632in}}%
\pgfpathlineto{\pgfqpoint{2.444677in}{2.040381in}}%
\pgfpathlineto{\pgfqpoint{2.447209in}{2.045292in}}%
\pgfpathlineto{\pgfqpoint{2.450032in}{2.038894in}}%
\pgfpathlineto{\pgfqpoint{2.452562in}{2.039045in}}%
\pgfpathlineto{\pgfqpoint{2.455353in}{2.039744in}}%
\pgfpathlineto{\pgfqpoint{2.457917in}{2.036344in}}%
\pgfpathlineto{\pgfqpoint{2.460711in}{2.041290in}}%
\pgfpathlineto{\pgfqpoint{2.463280in}{2.040319in}}%
\pgfpathlineto{\pgfqpoint{2.465957in}{2.036411in}}%
\pgfpathlineto{\pgfqpoint{2.468635in}{2.037975in}}%
\pgfpathlineto{\pgfqpoint{2.471311in}{2.033066in}}%
\pgfpathlineto{\pgfqpoint{2.473989in}{2.035429in}}%
\pgfpathlineto{\pgfqpoint{2.476671in}{2.031476in}}%
\pgfpathlineto{\pgfqpoint{2.479420in}{2.031251in}}%
\pgfpathlineto{\pgfqpoint{2.482026in}{2.033430in}}%
\pgfpathlineto{\pgfqpoint{2.484870in}{2.031198in}}%
\pgfpathlineto{\pgfqpoint{2.487384in}{2.031683in}}%
\pgfpathlineto{\pgfqpoint{2.490183in}{2.035208in}}%
\pgfpathlineto{\pgfqpoint{2.492729in}{2.036411in}}%
\pgfpathlineto{\pgfqpoint{2.495542in}{2.036147in}}%
\pgfpathlineto{\pgfqpoint{2.498085in}{2.032884in}}%
\pgfpathlineto{\pgfqpoint{2.500801in}{2.035455in}}%
\pgfpathlineto{\pgfqpoint{2.503454in}{2.037563in}}%
\pgfpathlineto{\pgfqpoint{2.506163in}{2.035288in}}%
\pgfpathlineto{\pgfqpoint{2.508917in}{2.036232in}}%
\pgfpathlineto{\pgfqpoint{2.511478in}{2.034001in}}%
\pgfpathlineto{\pgfqpoint{2.514268in}{2.033745in}}%
\pgfpathlineto{\pgfqpoint{2.516845in}{2.033558in}}%
\pgfpathlineto{\pgfqpoint{2.519607in}{2.034674in}}%
\pgfpathlineto{\pgfqpoint{2.522197in}{2.039142in}}%
\pgfpathlineto{\pgfqpoint{2.524988in}{2.035206in}}%
\pgfpathlineto{\pgfqpoint{2.527560in}{2.034239in}}%
\pgfpathlineto{\pgfqpoint{2.530234in}{2.034969in}}%
\pgfpathlineto{\pgfqpoint{2.532917in}{2.039264in}}%
\pgfpathlineto{\pgfqpoint{2.535624in}{2.033592in}}%
\pgfpathlineto{\pgfqpoint{2.538274in}{2.034518in}}%
\pgfpathlineto{\pgfqpoint{2.540949in}{2.036290in}}%
\pgfpathlineto{\pgfqpoint{2.543765in}{2.036570in}}%
\pgfpathlineto{\pgfqpoint{2.546310in}{2.038587in}}%
\pgfpathlineto{\pgfqpoint{2.549114in}{2.033390in}}%
\pgfpathlineto{\pgfqpoint{2.551664in}{2.030890in}}%
\pgfpathlineto{\pgfqpoint{2.554493in}{2.030890in}}%
\pgfpathlineto{\pgfqpoint{2.557009in}{2.032439in}}%
\pgfpathlineto{\pgfqpoint{2.559790in}{2.034162in}}%
\pgfpathlineto{\pgfqpoint{2.562375in}{2.032992in}}%
\pgfpathlineto{\pgfqpoint{2.565045in}{2.034376in}}%
\pgfpathlineto{\pgfqpoint{2.567730in}{2.031881in}}%
\pgfpathlineto{\pgfqpoint{2.570411in}{2.030890in}}%
\pgfpathlineto{\pgfqpoint{2.573082in}{2.030890in}}%
\pgfpathlineto{\pgfqpoint{2.575779in}{2.032998in}}%
\pgfpathlineto{\pgfqpoint{2.578567in}{2.035872in}}%
\pgfpathlineto{\pgfqpoint{2.581129in}{2.037660in}}%
\pgfpathlineto{\pgfqpoint{2.583913in}{2.036649in}}%
\pgfpathlineto{\pgfqpoint{2.586484in}{2.036240in}}%
\pgfpathlineto{\pgfqpoint{2.589248in}{2.039059in}}%
\pgfpathlineto{\pgfqpoint{2.591842in}{2.036056in}}%
\pgfpathlineto{\pgfqpoint{2.594630in}{2.038856in}}%
\pgfpathlineto{\pgfqpoint{2.597196in}{2.040199in}}%
\pgfpathlineto{\pgfqpoint{2.599920in}{2.040084in}}%
\pgfpathlineto{\pgfqpoint{2.602557in}{2.043384in}}%
\pgfpathlineto{\pgfqpoint{2.605232in}{2.038942in}}%
\pgfpathlineto{\pgfqpoint{2.608004in}{2.036383in}}%
\pgfpathlineto{\pgfqpoint{2.610588in}{2.040317in}}%
\pgfpathlineto{\pgfqpoint{2.613393in}{2.041111in}}%
\pgfpathlineto{\pgfqpoint{2.615934in}{2.043563in}}%
\pgfpathlineto{\pgfqpoint{2.618773in}{2.044201in}}%
\pgfpathlineto{\pgfqpoint{2.621304in}{2.043524in}}%
\pgfpathlineto{\pgfqpoint{2.624077in}{2.041602in}}%
\pgfpathlineto{\pgfqpoint{2.626653in}{2.036713in}}%
\pgfpathlineto{\pgfqpoint{2.629340in}{2.037243in}}%
\pgfpathlineto{\pgfqpoint{2.632018in}{2.040817in}}%
\pgfpathlineto{\pgfqpoint{2.634700in}{2.035086in}}%
\pgfpathlineto{\pgfqpoint{2.637369in}{2.037219in}}%
\pgfpathlineto{\pgfqpoint{2.640053in}{2.034259in}}%
\pgfpathlineto{\pgfqpoint{2.642827in}{2.037728in}}%
\pgfpathlineto{\pgfqpoint{2.645408in}{2.032497in}}%
\pgfpathlineto{\pgfqpoint{2.648196in}{2.038974in}}%
\pgfpathlineto{\pgfqpoint{2.650767in}{2.056822in}}%
\pgfpathlineto{\pgfqpoint{2.653567in}{2.064647in}}%
\pgfpathlineto{\pgfqpoint{2.656124in}{2.068690in}}%
\pgfpathlineto{\pgfqpoint{2.658942in}{2.060861in}}%
\pgfpathlineto{\pgfqpoint{2.661481in}{2.054468in}}%
\pgfpathlineto{\pgfqpoint{2.664151in}{2.051613in}}%
\pgfpathlineto{\pgfqpoint{2.666836in}{2.047908in}}%
\pgfpathlineto{\pgfqpoint{2.669506in}{2.044758in}}%
\pgfpathlineto{\pgfqpoint{2.672301in}{2.042371in}}%
\pgfpathlineto{\pgfqpoint{2.674873in}{2.036933in}}%
\pgfpathlineto{\pgfqpoint{2.677650in}{2.032040in}}%
\pgfpathlineto{\pgfqpoint{2.680224in}{2.031867in}}%
\pgfpathlineto{\pgfqpoint{2.683009in}{2.035649in}}%
\pgfpathlineto{\pgfqpoint{2.685586in}{2.036046in}}%
\pgfpathlineto{\pgfqpoint{2.688328in}{2.032445in}}%
\pgfpathlineto{\pgfqpoint{2.690940in}{2.036751in}}%
\pgfpathlineto{\pgfqpoint{2.693611in}{2.035086in}}%
\pgfpathlineto{\pgfqpoint{2.696293in}{2.035732in}}%
\pgfpathlineto{\pgfqpoint{2.698968in}{2.037347in}}%
\pgfpathlineto{\pgfqpoint{2.701657in}{2.035486in}}%
\pgfpathlineto{\pgfqpoint{2.704326in}{2.039311in}}%
\pgfpathlineto{\pgfqpoint{2.707125in}{2.035967in}}%
\pgfpathlineto{\pgfqpoint{2.709683in}{2.038426in}}%
\pgfpathlineto{\pgfqpoint{2.712477in}{2.042544in}}%
\pgfpathlineto{\pgfqpoint{2.715036in}{2.036459in}}%
\pgfpathlineto{\pgfqpoint{2.717773in}{2.030890in}}%
\pgfpathlineto{\pgfqpoint{2.720404in}{2.037878in}}%
\pgfpathlineto{\pgfqpoint{2.723211in}{2.037381in}}%
\pgfpathlineto{\pgfqpoint{2.725760in}{2.038842in}}%
\pgfpathlineto{\pgfqpoint{2.728439in}{2.032953in}}%
\pgfpathlineto{\pgfqpoint{2.731119in}{2.040537in}}%
\pgfpathlineto{\pgfqpoint{2.733798in}{2.034842in}}%
\pgfpathlineto{\pgfqpoint{2.736476in}{2.042620in}}%
\pgfpathlineto{\pgfqpoint{2.739155in}{2.040505in}}%
\pgfpathlineto{\pgfqpoint{2.741928in}{2.036834in}}%
\pgfpathlineto{\pgfqpoint{2.744510in}{2.039805in}}%
\pgfpathlineto{\pgfqpoint{2.747260in}{2.040447in}}%
\pgfpathlineto{\pgfqpoint{2.749868in}{2.034656in}}%
\pgfpathlineto{\pgfqpoint{2.752614in}{2.036233in}}%
\pgfpathlineto{\pgfqpoint{2.755224in}{2.041321in}}%
\pgfpathlineto{\pgfqpoint{2.758028in}{2.041490in}}%
\pgfpathlineto{\pgfqpoint{2.760581in}{2.042482in}}%
\pgfpathlineto{\pgfqpoint{2.763253in}{2.041892in}}%
\pgfpathlineto{\pgfqpoint{2.765935in}{2.041132in}}%
\pgfpathlineto{\pgfqpoint{2.768617in}{2.046552in}}%
\pgfpathlineto{\pgfqpoint{2.771373in}{2.041782in}}%
\pgfpathlineto{\pgfqpoint{2.773972in}{2.041574in}}%
\pgfpathlineto{\pgfqpoint{2.776767in}{2.035648in}}%
\pgfpathlineto{\pgfqpoint{2.779330in}{2.030890in}}%
\pgfpathlineto{\pgfqpoint{2.782113in}{2.030890in}}%
\pgfpathlineto{\pgfqpoint{2.784687in}{2.030890in}}%
\pgfpathlineto{\pgfqpoint{2.787468in}{2.033300in}}%
\pgfpathlineto{\pgfqpoint{2.790044in}{2.030890in}}%
\pgfpathlineto{\pgfqpoint{2.792721in}{2.033354in}}%
\pgfpathlineto{\pgfqpoint{2.795398in}{2.030890in}}%
\pgfpathlineto{\pgfqpoint{2.798070in}{2.032827in}}%
\pgfpathlineto{\pgfqpoint{2.800756in}{2.031132in}}%
\pgfpathlineto{\pgfqpoint{2.803435in}{2.032526in}}%
\pgfpathlineto{\pgfqpoint{2.806175in}{2.033585in}}%
\pgfpathlineto{\pgfqpoint{2.808792in}{2.032701in}}%
\pgfpathlineto{\pgfqpoint{2.811597in}{2.033056in}}%
\pgfpathlineto{\pgfqpoint{2.814141in}{2.034768in}}%
\pgfpathlineto{\pgfqpoint{2.816867in}{2.035087in}}%
\pgfpathlineto{\pgfqpoint{2.819506in}{2.035859in}}%
\pgfpathlineto{\pgfqpoint{2.822303in}{2.035240in}}%
\pgfpathlineto{\pgfqpoint{2.824851in}{2.035565in}}%
\pgfpathlineto{\pgfqpoint{2.827567in}{2.036052in}}%
\pgfpathlineto{\pgfqpoint{2.830219in}{2.032909in}}%
\pgfpathlineto{\pgfqpoint{2.832894in}{2.034548in}}%
\pgfpathlineto{\pgfqpoint{2.835698in}{2.034833in}}%
\pgfpathlineto{\pgfqpoint{2.838254in}{2.037247in}}%
\pgfpathlineto{\pgfqpoint{2.841055in}{2.040035in}}%
\pgfpathlineto{\pgfqpoint{2.843611in}{2.042939in}}%
\pgfpathlineto{\pgfqpoint{2.846408in}{2.039269in}}%
\pgfpathlineto{\pgfqpoint{2.848960in}{2.040473in}}%
\pgfpathlineto{\pgfqpoint{2.851793in}{2.039682in}}%
\pgfpathlineto{\pgfqpoint{2.854325in}{2.038553in}}%
\pgfpathlineto{\pgfqpoint{2.857003in}{2.039279in}}%
\pgfpathlineto{\pgfqpoint{2.859668in}{2.040032in}}%
\pgfpathlineto{\pgfqpoint{2.862402in}{2.039390in}}%
\pgfpathlineto{\pgfqpoint{2.865031in}{2.033241in}}%
\pgfpathlineto{\pgfqpoint{2.867713in}{2.034665in}}%
\pgfpathlineto{\pgfqpoint{2.870475in}{2.031930in}}%
\pgfpathlineto{\pgfqpoint{2.873074in}{2.034237in}}%
\pgfpathlineto{\pgfqpoint{2.875882in}{2.035835in}}%
\pgfpathlineto{\pgfqpoint{2.878431in}{2.036725in}}%
\pgfpathlineto{\pgfqpoint{2.881254in}{2.033419in}}%
\pgfpathlineto{\pgfqpoint{2.883780in}{2.033288in}}%
\pgfpathlineto{\pgfqpoint{2.886578in}{2.032559in}}%
\pgfpathlineto{\pgfqpoint{2.889145in}{2.034695in}}%
\pgfpathlineto{\pgfqpoint{2.891809in}{2.033876in}}%
\pgfpathlineto{\pgfqpoint{2.894487in}{2.036015in}}%
\pgfpathlineto{\pgfqpoint{2.897179in}{2.036276in}}%
\pgfpathlineto{\pgfqpoint{2.899858in}{2.036121in}}%
\pgfpathlineto{\pgfqpoint{2.902535in}{2.035873in}}%
\pgfpathlineto{\pgfqpoint{2.905341in}{2.038786in}}%
\pgfpathlineto{\pgfqpoint{2.907882in}{2.032930in}}%
\pgfpathlineto{\pgfqpoint{2.910631in}{2.037873in}}%
\pgfpathlineto{\pgfqpoint{2.913243in}{2.038006in}}%
\pgfpathlineto{\pgfqpoint{2.916061in}{2.040830in}}%
\pgfpathlineto{\pgfqpoint{2.918606in}{2.038320in}}%
\pgfpathlineto{\pgfqpoint{2.921363in}{2.040104in}}%
\pgfpathlineto{\pgfqpoint{2.923963in}{2.039557in}}%
\pgfpathlineto{\pgfqpoint{2.926655in}{2.040560in}}%
\pgfpathlineto{\pgfqpoint{2.929321in}{2.034581in}}%
\pgfpathlineto{\pgfqpoint{2.932033in}{2.035287in}}%
\pgfpathlineto{\pgfqpoint{2.934759in}{2.032718in}}%
\pgfpathlineto{\pgfqpoint{2.937352in}{2.034378in}}%
\pgfpathlineto{\pgfqpoint{2.940120in}{2.030890in}}%
\pgfpathlineto{\pgfqpoint{2.942711in}{2.034607in}}%
\pgfpathlineto{\pgfqpoint{2.945461in}{2.033844in}}%
\pgfpathlineto{\pgfqpoint{2.948068in}{2.031665in}}%
\pgfpathlineto{\pgfqpoint{2.950884in}{2.037037in}}%
\pgfpathlineto{\pgfqpoint{2.953422in}{2.057569in}}%
\pgfpathlineto{\pgfqpoint{2.956103in}{2.057499in}}%
\pgfpathlineto{\pgfqpoint{2.958782in}{2.053493in}}%
\pgfpathlineto{\pgfqpoint{2.961460in}{2.047086in}}%
\pgfpathlineto{\pgfqpoint{2.964127in}{2.046590in}}%
\pgfpathlineto{\pgfqpoint{2.966812in}{2.051413in}}%
\pgfpathlineto{\pgfqpoint{2.969599in}{2.046137in}}%
\pgfpathlineto{\pgfqpoint{2.972177in}{2.046140in}}%
\pgfpathlineto{\pgfqpoint{2.974972in}{2.044670in}}%
\pgfpathlineto{\pgfqpoint{2.977517in}{2.046699in}}%
\pgfpathlineto{\pgfqpoint{2.980341in}{2.047051in}}%
\pgfpathlineto{\pgfqpoint{2.982885in}{2.047624in}}%
\pgfpathlineto{\pgfqpoint{2.985666in}{2.043894in}}%
\pgfpathlineto{\pgfqpoint{2.988238in}{2.043465in}}%
\pgfpathlineto{\pgfqpoint{2.990978in}{2.041627in}}%
\pgfpathlineto{\pgfqpoint{2.993595in}{2.034632in}}%
\pgfpathlineto{\pgfqpoint{2.996300in}{2.037225in}}%
\pgfpathlineto{\pgfqpoint{2.999103in}{2.035089in}}%
\pgfpathlineto{\pgfqpoint{3.001635in}{2.038536in}}%
\pgfpathlineto{\pgfqpoint{3.004419in}{2.038042in}}%
\pgfpathlineto{\pgfqpoint{3.006993in}{2.040511in}}%
\pgfpathlineto{\pgfqpoint{3.009784in}{2.041829in}}%
\pgfpathlineto{\pgfqpoint{3.012351in}{2.038041in}}%
\pgfpathlineto{\pgfqpoint{3.015097in}{2.041247in}}%
\pgfpathlineto{\pgfqpoint{3.017707in}{2.044788in}}%
\pgfpathlineto{\pgfqpoint{3.020382in}{2.043920in}}%
\pgfpathlineto{\pgfqpoint{3.023058in}{2.043847in}}%
\pgfpathlineto{\pgfqpoint{3.025803in}{2.041248in}}%
\pgfpathlineto{\pgfqpoint{3.028412in}{2.043566in}}%
\pgfpathlineto{\pgfqpoint{3.031091in}{2.034127in}}%
\pgfpathlineto{\pgfqpoint{3.033921in}{2.033968in}}%
\pgfpathlineto{\pgfqpoint{3.036456in}{2.039499in}}%
\pgfpathlineto{\pgfqpoint{3.039262in}{2.036198in}}%
\pgfpathlineto{\pgfqpoint{3.041813in}{2.036479in}}%
\pgfpathlineto{\pgfqpoint{3.044568in}{2.042605in}}%
\pgfpathlineto{\pgfqpoint{3.047157in}{2.051706in}}%
\pgfpathlineto{\pgfqpoint{3.049988in}{2.068341in}}%
\pgfpathlineto{\pgfqpoint{3.052526in}{2.047458in}}%
\pgfpathlineto{\pgfqpoint{3.055202in}{2.042250in}}%
\pgfpathlineto{\pgfqpoint{3.057884in}{2.036069in}}%
\pgfpathlineto{\pgfqpoint{3.060561in}{2.035836in}}%
\pgfpathlineto{\pgfqpoint{3.063230in}{2.042652in}}%
\pgfpathlineto{\pgfqpoint{3.065916in}{2.037739in}}%
\pgfpathlineto{\pgfqpoint{3.068709in}{2.039178in}}%
\pgfpathlineto{\pgfqpoint{3.071266in}{2.036210in}}%
\pgfpathlineto{\pgfqpoint{3.074056in}{2.039725in}}%
\pgfpathlineto{\pgfqpoint{3.076631in}{2.039853in}}%
\pgfpathlineto{\pgfqpoint{3.079381in}{2.047562in}}%
\pgfpathlineto{\pgfqpoint{3.081990in}{2.056706in}}%
\pgfpathlineto{\pgfqpoint{3.084671in}{2.049744in}}%
\pgfpathlineto{\pgfqpoint{3.087343in}{2.047674in}}%
\pgfpathlineto{\pgfqpoint{3.090023in}{2.044080in}}%
\pgfpathlineto{\pgfqpoint{3.092699in}{2.051778in}}%
\pgfpathlineto{\pgfqpoint{3.095388in}{2.052913in}}%
\pgfpathlineto{\pgfqpoint{3.098163in}{2.045096in}}%
\pgfpathlineto{\pgfqpoint{3.100737in}{2.057893in}}%
\pgfpathlineto{\pgfqpoint{3.103508in}{2.064931in}}%
\pgfpathlineto{\pgfqpoint{3.106094in}{2.058647in}}%
\pgfpathlineto{\pgfqpoint{3.108896in}{2.062616in}}%
\pgfpathlineto{\pgfqpoint{3.111451in}{2.050557in}}%
\pgfpathlineto{\pgfqpoint{3.114242in}{2.050060in}}%
\pgfpathlineto{\pgfqpoint{3.116807in}{2.037112in}}%
\pgfpathlineto{\pgfqpoint{3.119487in}{2.044294in}}%
\pgfpathlineto{\pgfqpoint{3.122163in}{2.047539in}}%
\pgfpathlineto{\pgfqpoint{3.124842in}{2.031029in}}%
\pgfpathlineto{\pgfqpoint{3.127512in}{2.030890in}}%
\pgfpathlineto{\pgfqpoint{3.130199in}{2.030890in}}%
\pgfpathlineto{\pgfqpoint{3.132946in}{2.033135in}}%
\pgfpathlineto{\pgfqpoint{3.135550in}{2.047491in}}%
\pgfpathlineto{\pgfqpoint{3.138375in}{2.063254in}}%
\pgfpathlineto{\pgfqpoint{3.140913in}{2.069577in}}%
\pgfpathlineto{\pgfqpoint{3.143740in}{2.080525in}}%
\pgfpathlineto{\pgfqpoint{3.146271in}{2.063603in}}%
\pgfpathlineto{\pgfqpoint{3.149057in}{2.066974in}}%
\pgfpathlineto{\pgfqpoint{3.151612in}{2.075593in}}%
\pgfpathlineto{\pgfqpoint{3.154327in}{2.046801in}}%
\pgfpathlineto{\pgfqpoint{3.156981in}{2.030890in}}%
\pgfpathlineto{\pgfqpoint{3.159675in}{2.050130in}}%
\pgfpathlineto{\pgfqpoint{3.162474in}{2.056597in}}%
\pgfpathlineto{\pgfqpoint{3.165019in}{2.055279in}}%
\pgfpathlineto{\pgfqpoint{3.167776in}{2.075877in}}%
\pgfpathlineto{\pgfqpoint{3.170375in}{2.076394in}}%
\pgfpathlineto{\pgfqpoint{3.173142in}{2.076223in}}%
\pgfpathlineto{\pgfqpoint{3.175724in}{2.049362in}}%
\pgfpathlineto{\pgfqpoint{3.178525in}{2.035437in}}%
\pgfpathlineto{\pgfqpoint{3.181089in}{2.040773in}}%
\pgfpathlineto{\pgfqpoint{3.183760in}{2.044750in}}%
\pgfpathlineto{\pgfqpoint{3.186440in}{2.052151in}}%
\pgfpathlineto{\pgfqpoint{3.189117in}{2.056840in}}%
\pgfpathlineto{\pgfqpoint{3.191796in}{2.045703in}}%
\pgfpathlineto{\pgfqpoint{3.194508in}{2.034715in}}%
\pgfpathlineto{\pgfqpoint{3.197226in}{2.039778in}}%
\pgfpathlineto{\pgfqpoint{3.199823in}{2.055794in}}%
\pgfpathlineto{\pgfqpoint{3.202562in}{2.044314in}}%
\pgfpathlineto{\pgfqpoint{3.205195in}{2.030890in}}%
\pgfpathlineto{\pgfqpoint{3.207984in}{2.030890in}}%
\pgfpathlineto{\pgfqpoint{3.210545in}{2.034993in}}%
\pgfpathlineto{\pgfqpoint{3.213342in}{2.030890in}}%
\pgfpathlineto{\pgfqpoint{3.215908in}{2.030890in}}%
\pgfpathlineto{\pgfqpoint{3.218586in}{2.030946in}}%
\pgfpathlineto{\pgfqpoint{3.221255in}{2.030890in}}%
\pgfpathlineto{\pgfqpoint{3.223942in}{2.030890in}}%
\pgfpathlineto{\pgfqpoint{3.226609in}{2.030890in}}%
\pgfpathlineto{\pgfqpoint{3.229310in}{2.030890in}}%
\pgfpathlineto{\pgfqpoint{3.232069in}{2.030890in}}%
\pgfpathlineto{\pgfqpoint{3.234658in}{2.031978in}}%
\pgfpathlineto{\pgfqpoint{3.237411in}{2.033452in}}%
\pgfpathlineto{\pgfqpoint{3.240010in}{2.030890in}}%
\pgfpathlineto{\pgfqpoint{3.242807in}{2.030890in}}%
\pgfpathlineto{\pgfqpoint{3.245363in}{2.030890in}}%
\pgfpathlineto{\pgfqpoint{3.248049in}{2.030890in}}%
\pgfpathlineto{\pgfqpoint{3.250716in}{2.030890in}}%
\pgfpathlineto{\pgfqpoint{3.253404in}{2.034054in}}%
\pgfpathlineto{\pgfqpoint{3.256083in}{2.031094in}}%
\pgfpathlineto{\pgfqpoint{3.258784in}{2.033127in}}%
\pgfpathlineto{\pgfqpoint{3.261594in}{2.036540in}}%
\pgfpathlineto{\pgfqpoint{3.264119in}{2.033562in}}%
\pgfpathlineto{\pgfqpoint{3.266849in}{2.034084in}}%
\pgfpathlineto{\pgfqpoint{3.269478in}{2.036307in}}%
\pgfpathlineto{\pgfqpoint{3.272254in}{2.039141in}}%
\pgfpathlineto{\pgfqpoint{3.274831in}{2.038885in}}%
\pgfpathlineto{\pgfqpoint{3.277603in}{2.038975in}}%
\pgfpathlineto{\pgfqpoint{3.280189in}{2.038405in}}%
\pgfpathlineto{\pgfqpoint{3.282870in}{2.030890in}}%
\pgfpathlineto{\pgfqpoint{3.285534in}{2.037503in}}%
\pgfpathlineto{\pgfqpoint{3.288225in}{2.035005in}}%
\pgfpathlineto{\pgfqpoint{3.290890in}{2.038978in}}%
\pgfpathlineto{\pgfqpoint{3.293574in}{2.035247in}}%
\pgfpathlineto{\pgfqpoint{3.296376in}{2.031906in}}%
\pgfpathlineto{\pgfqpoint{3.298937in}{2.030926in}}%
\pgfpathlineto{\pgfqpoint{3.301719in}{2.035250in}}%
\pgfpathlineto{\pgfqpoint{3.304295in}{2.034561in}}%
\pgfpathlineto{\pgfqpoint{3.307104in}{2.034766in}}%
\pgfpathlineto{\pgfqpoint{3.309652in}{2.035691in}}%
\pgfpathlineto{\pgfqpoint{3.312480in}{2.034550in}}%
\pgfpathlineto{\pgfqpoint{3.315008in}{2.035807in}}%
\pgfpathlineto{\pgfqpoint{3.317688in}{2.036893in}}%
\pgfpathlineto{\pgfqpoint{3.320366in}{2.035557in}}%
\pgfpathlineto{\pgfqpoint{3.323049in}{2.035700in}}%
\pgfpathlineto{\pgfqpoint{3.325860in}{2.035454in}}%
\pgfpathlineto{\pgfqpoint{3.328401in}{2.033719in}}%
\pgfpathlineto{\pgfqpoint{3.331183in}{2.035645in}}%
\pgfpathlineto{\pgfqpoint{3.333758in}{2.034332in}}%
\pgfpathlineto{\pgfqpoint{3.336541in}{2.038836in}}%
\pgfpathlineto{\pgfqpoint{3.339101in}{2.034112in}}%
\pgfpathlineto{\pgfqpoint{3.341893in}{2.036572in}}%
\pgfpathlineto{\pgfqpoint{3.344468in}{2.032231in}}%
\pgfpathlineto{\pgfqpoint{3.347139in}{2.031931in}}%
\pgfpathlineto{\pgfqpoint{3.349828in}{2.032323in}}%
\pgfpathlineto{\pgfqpoint{3.352505in}{2.030890in}}%
\pgfpathlineto{\pgfqpoint{3.355177in}{2.030890in}}%
\pgfpathlineto{\pgfqpoint{3.357862in}{2.030890in}}%
\pgfpathlineto{\pgfqpoint{3.360620in}{2.032927in}}%
\pgfpathlineto{\pgfqpoint{3.363221in}{2.032834in}}%
\pgfpathlineto{\pgfqpoint{3.365996in}{2.033616in}}%
\pgfpathlineto{\pgfqpoint{3.368577in}{2.034495in}}%
\pgfpathlineto{\pgfqpoint{3.371357in}{2.032340in}}%
\pgfpathlineto{\pgfqpoint{3.373921in}{2.035139in}}%
\pgfpathlineto{\pgfqpoint{3.376735in}{2.036079in}}%
\pgfpathlineto{\pgfqpoint{3.379290in}{2.042138in}}%
\pgfpathlineto{\pgfqpoint{3.381959in}{2.036307in}}%
\pgfpathlineto{\pgfqpoint{3.384647in}{2.037626in}}%
\pgfpathlineto{\pgfqpoint{3.387309in}{2.034851in}}%
\pgfpathlineto{\pgfqpoint{3.390102in}{2.032919in}}%
\pgfpathlineto{\pgfqpoint{3.392681in}{2.037658in}}%
\pgfpathlineto{\pgfqpoint{3.395461in}{2.031381in}}%
\pgfpathlineto{\pgfqpoint{3.398037in}{2.033475in}}%
\pgfpathlineto{\pgfqpoint{3.400783in}{2.034918in}}%
\pgfpathlineto{\pgfqpoint{3.403394in}{2.035294in}}%
\pgfpathlineto{\pgfqpoint{3.406202in}{2.032543in}}%
\pgfpathlineto{\pgfqpoint{3.408752in}{2.038339in}}%
\pgfpathlineto{\pgfqpoint{3.411431in}{2.035329in}}%
\pgfpathlineto{\pgfqpoint{3.414109in}{2.036114in}}%
\pgfpathlineto{\pgfqpoint{3.416780in}{2.034062in}}%
\pgfpathlineto{\pgfqpoint{3.419455in}{2.039342in}}%
\pgfpathlineto{\pgfqpoint{3.422142in}{2.038820in}}%
\pgfpathlineto{\pgfqpoint{3.424887in}{2.034549in}}%
\pgfpathlineto{\pgfqpoint{3.427501in}{2.036523in}}%
\pgfpathlineto{\pgfqpoint{3.430313in}{2.039598in}}%
\pgfpathlineto{\pgfqpoint{3.432851in}{2.040168in}}%
\pgfpathlineto{\pgfqpoint{3.435635in}{2.042747in}}%
\pgfpathlineto{\pgfqpoint{3.438210in}{2.040876in}}%
\pgfpathlineto{\pgfqpoint{3.440996in}{2.040513in}}%
\pgfpathlineto{\pgfqpoint{3.443574in}{2.036647in}}%
\pgfpathlineto{\pgfqpoint{3.446257in}{2.040091in}}%
\pgfpathlineto{\pgfqpoint{3.448926in}{2.040013in}}%
\pgfpathlineto{\pgfqpoint{3.451597in}{2.042404in}}%
\pgfpathlineto{\pgfqpoint{3.454285in}{2.036633in}}%
\pgfpathlineto{\pgfqpoint{3.456960in}{2.036978in}}%
\pgfpathlineto{\pgfqpoint{3.459695in}{2.036271in}}%
\pgfpathlineto{\pgfqpoint{3.462321in}{2.041318in}}%
\pgfpathlineto{\pgfqpoint{3.465072in}{2.039568in}}%
\pgfpathlineto{\pgfqpoint{3.467678in}{2.038605in}}%
\pgfpathlineto{\pgfqpoint{3.470466in}{2.042644in}}%
\pgfpathlineto{\pgfqpoint{3.473021in}{2.041076in}}%
\pgfpathlineto{\pgfqpoint{3.475821in}{2.042071in}}%
\pgfpathlineto{\pgfqpoint{3.478378in}{2.045286in}}%
\pgfpathlineto{\pgfqpoint{3.481072in}{2.042462in}}%
\pgfpathlineto{\pgfqpoint{3.483744in}{2.041553in}}%
\pgfpathlineto{\pgfqpoint{3.486442in}{2.042179in}}%
\pgfpathlineto{\pgfqpoint{3.489223in}{2.039859in}}%
\pgfpathlineto{\pgfqpoint{3.491783in}{2.039569in}}%
\pgfpathlineto{\pgfqpoint{3.494581in}{2.042832in}}%
\pgfpathlineto{\pgfqpoint{3.497139in}{2.040185in}}%
\pgfpathlineto{\pgfqpoint{3.499909in}{2.038814in}}%
\pgfpathlineto{\pgfqpoint{3.502488in}{2.035505in}}%
\pgfpathlineto{\pgfqpoint{3.505262in}{2.035436in}}%
\pgfpathlineto{\pgfqpoint{3.507840in}{2.030890in}}%
\pgfpathlineto{\pgfqpoint{3.510533in}{2.032106in}}%
\pgfpathlineto{\pgfqpoint{3.513209in}{2.034670in}}%
\pgfpathlineto{\pgfqpoint{3.515884in}{2.031035in}}%
\pgfpathlineto{\pgfqpoint{3.518565in}{2.032398in}}%
\pgfpathlineto{\pgfqpoint{3.521244in}{2.036296in}}%
\pgfpathlineto{\pgfqpoint{3.524041in}{2.035983in}}%
\pgfpathlineto{\pgfqpoint{3.526601in}{2.036208in}}%
\pgfpathlineto{\pgfqpoint{3.529327in}{2.037869in}}%
\pgfpathlineto{\pgfqpoint{3.531955in}{2.048831in}}%
\pgfpathlineto{\pgfqpoint{3.534783in}{2.043181in}}%
\pgfpathlineto{\pgfqpoint{3.537309in}{2.038500in}}%
\pgfpathlineto{\pgfqpoint{3.540093in}{2.039448in}}%
\pgfpathlineto{\pgfqpoint{3.542656in}{2.039266in}}%
\pgfpathlineto{\pgfqpoint{3.545349in}{2.037220in}}%
\pgfpathlineto{\pgfqpoint{3.548029in}{2.038061in}}%
\pgfpathlineto{\pgfqpoint{3.550713in}{2.036150in}}%
\pgfpathlineto{\pgfqpoint{3.553498in}{2.036974in}}%
\pgfpathlineto{\pgfqpoint{3.556061in}{2.034832in}}%
\pgfpathlineto{\pgfqpoint{3.558853in}{2.034980in}}%
\pgfpathlineto{\pgfqpoint{3.561420in}{2.039239in}}%
\pgfpathlineto{\pgfqpoint{3.564188in}{2.041303in}}%
\pgfpathlineto{\pgfqpoint{3.566774in}{2.037250in}}%
\pgfpathlineto{\pgfqpoint{3.569584in}{2.037683in}}%
\pgfpathlineto{\pgfqpoint{3.572126in}{2.036668in}}%
\pgfpathlineto{\pgfqpoint{3.574814in}{2.039442in}}%
\pgfpathlineto{\pgfqpoint{3.577487in}{2.036330in}}%
\pgfpathlineto{\pgfqpoint{3.580191in}{2.038365in}}%
\pgfpathlineto{\pgfqpoint{3.582851in}{2.035725in}}%
\pgfpathlineto{\pgfqpoint{3.585532in}{2.039148in}}%
\pgfpathlineto{\pgfqpoint{3.588258in}{2.039728in}}%
\pgfpathlineto{\pgfqpoint{3.590883in}{2.039418in}}%
\pgfpathlineto{\pgfqpoint{3.593620in}{2.041906in}}%
\pgfpathlineto{\pgfqpoint{3.596240in}{2.039399in}}%
\pgfpathlineto{\pgfqpoint{3.598998in}{2.043170in}}%
\pgfpathlineto{\pgfqpoint{3.601590in}{2.045963in}}%
\pgfpathlineto{\pgfqpoint{3.604387in}{2.045051in}}%
\pgfpathlineto{\pgfqpoint{3.606951in}{2.041052in}}%
\pgfpathlineto{\pgfqpoint{3.609632in}{2.041824in}}%
\pgfpathlineto{\pgfqpoint{3.612311in}{2.039236in}}%
\pgfpathlineto{\pgfqpoint{3.614982in}{2.041922in}}%
\pgfpathlineto{\pgfqpoint{3.617667in}{2.044942in}}%
\pgfpathlineto{\pgfqpoint{3.620345in}{2.040033in}}%
\pgfpathlineto{\pgfqpoint{3.623165in}{2.041084in}}%
\pgfpathlineto{\pgfqpoint{3.625689in}{2.041423in}}%
\pgfpathlineto{\pgfqpoint{3.628460in}{2.038422in}}%
\pgfpathlineto{\pgfqpoint{3.631058in}{2.040409in}}%
\pgfpathlineto{\pgfqpoint{3.633858in}{2.038846in}}%
\pgfpathlineto{\pgfqpoint{3.636413in}{2.039695in}}%
\pgfpathlineto{\pgfqpoint{3.639207in}{2.034873in}}%
\pgfpathlineto{\pgfqpoint{3.641773in}{2.037495in}}%
\pgfpathlineto{\pgfqpoint{3.644452in}{2.040626in}}%
\pgfpathlineto{\pgfqpoint{3.647130in}{2.046008in}}%
\pgfpathlineto{\pgfqpoint{3.649837in}{2.041890in}}%
\pgfpathlineto{\pgfqpoint{3.652628in}{2.040965in}}%
\pgfpathlineto{\pgfqpoint{3.655165in}{2.040866in}}%
\pgfpathlineto{\pgfqpoint{3.657917in}{2.030890in}}%
\pgfpathlineto{\pgfqpoint{3.660515in}{2.030890in}}%
\pgfpathlineto{\pgfqpoint{3.663276in}{2.030890in}}%
\pgfpathlineto{\pgfqpoint{3.665864in}{2.030890in}}%
\pgfpathlineto{\pgfqpoint{3.668665in}{2.032876in}}%
\pgfpathlineto{\pgfqpoint{3.671232in}{2.036547in}}%
\pgfpathlineto{\pgfqpoint{3.673911in}{2.036039in}}%
\pgfpathlineto{\pgfqpoint{3.676591in}{2.030890in}}%
\pgfpathlineto{\pgfqpoint{3.679273in}{2.030890in}}%
\pgfpathlineto{\pgfqpoint{3.681948in}{2.030890in}}%
\pgfpathlineto{\pgfqpoint{3.684620in}{2.030890in}}%
\pgfpathlineto{\pgfqpoint{3.687442in}{2.030890in}}%
\pgfpathlineto{\pgfqpoint{3.689983in}{2.030890in}}%
\pgfpathlineto{\pgfqpoint{3.692765in}{2.031982in}}%
\pgfpathlineto{\pgfqpoint{3.695331in}{2.035686in}}%
\pgfpathlineto{\pgfqpoint{3.698125in}{2.032671in}}%
\pgfpathlineto{\pgfqpoint{3.700684in}{2.033794in}}%
\pgfpathlineto{\pgfqpoint{3.703460in}{2.039851in}}%
\pgfpathlineto{\pgfqpoint{3.706053in}{2.044975in}}%
\pgfpathlineto{\pgfqpoint{3.708729in}{2.040357in}}%
\pgfpathlineto{\pgfqpoint{3.711410in}{2.038426in}}%
\pgfpathlineto{\pgfqpoint{3.714086in}{2.034303in}}%
\pgfpathlineto{\pgfqpoint{3.716875in}{2.030890in}}%
\pgfpathlineto{\pgfqpoint{3.719446in}{2.036627in}}%
\pgfpathlineto{\pgfqpoint{3.722228in}{2.035059in}}%
\pgfpathlineto{\pgfqpoint{3.724804in}{2.041247in}}%
\pgfpathlineto{\pgfqpoint{3.727581in}{2.035759in}}%
\pgfpathlineto{\pgfqpoint{3.730158in}{2.034521in}}%
\pgfpathlineto{\pgfqpoint{3.732950in}{2.040232in}}%
\pgfpathlineto{\pgfqpoint{3.735509in}{2.041940in}}%
\pgfpathlineto{\pgfqpoint{3.738194in}{2.038245in}}%
\pgfpathlineto{\pgfqpoint{3.740874in}{2.037975in}}%
\pgfpathlineto{\pgfqpoint{3.743548in}{2.037718in}}%
\pgfpathlineto{\pgfqpoint{3.746229in}{2.038222in}}%
\pgfpathlineto{\pgfqpoint{3.748903in}{2.036025in}}%
\pgfpathlineto{\pgfqpoint{3.751728in}{2.038767in}}%
\pgfpathlineto{\pgfqpoint{3.754265in}{2.038102in}}%
\pgfpathlineto{\pgfqpoint{3.757065in}{2.032204in}}%
\pgfpathlineto{\pgfqpoint{3.759622in}{2.035040in}}%
\pgfpathlineto{\pgfqpoint{3.762389in}{2.037838in}}%
\pgfpathlineto{\pgfqpoint{3.764966in}{2.049350in}}%
\pgfpathlineto{\pgfqpoint{3.767782in}{2.053211in}}%
\pgfpathlineto{\pgfqpoint{3.770323in}{2.049880in}}%
\pgfpathlineto{\pgfqpoint{3.773014in}{2.042254in}}%
\pgfpathlineto{\pgfqpoint{3.775691in}{2.030890in}}%
\pgfpathlineto{\pgfqpoint{3.778370in}{2.030890in}}%
\pgfpathlineto{\pgfqpoint{3.781046in}{2.038152in}}%
\pgfpathlineto{\pgfqpoint{3.783725in}{2.043112in}}%
\pgfpathlineto{\pgfqpoint{3.786504in}{2.037782in}}%
\pgfpathlineto{\pgfqpoint{3.789084in}{2.034054in}}%
\pgfpathlineto{\pgfqpoint{3.791897in}{2.039521in}}%
\pgfpathlineto{\pgfqpoint{3.794435in}{2.036030in}}%
\pgfpathlineto{\pgfqpoint{3.797265in}{2.036572in}}%
\pgfpathlineto{\pgfqpoint{3.799797in}{2.039034in}}%
\pgfpathlineto{\pgfqpoint{3.802569in}{2.035450in}}%
\pgfpathlineto{\pgfqpoint{3.805145in}{2.035551in}}%
\pgfpathlineto{\pgfqpoint{3.807832in}{2.034506in}}%
\pgfpathlineto{\pgfqpoint{3.810510in}{2.034546in}}%
\pgfpathlineto{\pgfqpoint{3.813172in}{2.039431in}}%
\pgfpathlineto{\pgfqpoint{3.815983in}{2.033138in}}%
\pgfpathlineto{\pgfqpoint{3.818546in}{2.035773in}}%
\pgfpathlineto{\pgfqpoint{3.821315in}{2.039451in}}%
\pgfpathlineto{\pgfqpoint{3.823903in}{2.040240in}}%
\pgfpathlineto{\pgfqpoint{3.826679in}{2.037775in}}%
\pgfpathlineto{\pgfqpoint{3.829252in}{2.041702in}}%
\pgfpathlineto{\pgfqpoint{3.832053in}{2.038633in}}%
\pgfpathlineto{\pgfqpoint{3.834616in}{2.039035in}}%
\pgfpathlineto{\pgfqpoint{3.837286in}{2.039088in}}%
\pgfpathlineto{\pgfqpoint{3.839960in}{2.039497in}}%
\pgfpathlineto{\pgfqpoint{3.842641in}{2.040408in}}%
\pgfpathlineto{\pgfqpoint{3.845329in}{2.042676in}}%
\pgfpathlineto{\pgfqpoint{3.848005in}{2.054439in}}%
\pgfpathlineto{\pgfqpoint{3.850814in}{2.039583in}}%
\pgfpathlineto{\pgfqpoint{3.853358in}{2.042473in}}%
\pgfpathlineto{\pgfqpoint{3.856100in}{2.045695in}}%
\pgfpathlineto{\pgfqpoint{3.858720in}{2.042390in}}%
\pgfpathlineto{\pgfqpoint{3.861561in}{2.041521in}}%
\pgfpathlineto{\pgfqpoint{3.864073in}{2.039010in}}%
\pgfpathlineto{\pgfqpoint{3.866815in}{2.038136in}}%
\pgfpathlineto{\pgfqpoint{3.869435in}{2.033866in}}%
\pgfpathlineto{\pgfqpoint{3.872114in}{2.031604in}}%
\pgfpathlineto{\pgfqpoint{3.874790in}{2.034968in}}%
\pgfpathlineto{\pgfqpoint{3.877466in}{2.040149in}}%
\pgfpathlineto{\pgfqpoint{3.880237in}{2.035201in}}%
\pgfpathlineto{\pgfqpoint{3.882850in}{2.035847in}}%
\pgfpathlineto{\pgfqpoint{3.885621in}{2.034605in}}%
\pgfpathlineto{\pgfqpoint{3.888188in}{2.038684in}}%
\pgfpathlineto{\pgfqpoint{3.890926in}{2.044526in}}%
\pgfpathlineto{\pgfqpoint{3.893541in}{2.046195in}}%
\pgfpathlineto{\pgfqpoint{3.896345in}{2.037555in}}%
\pgfpathlineto{\pgfqpoint{3.898891in}{2.036059in}}%
\pgfpathlineto{\pgfqpoint{3.901573in}{2.039804in}}%
\pgfpathlineto{\pgfqpoint{3.904252in}{2.037179in}}%
\pgfpathlineto{\pgfqpoint{3.906918in}{2.039079in}}%
\pgfpathlineto{\pgfqpoint{3.909602in}{2.039436in}}%
\pgfpathlineto{\pgfqpoint{3.912296in}{2.037025in}}%
\pgfpathlineto{\pgfqpoint{3.915107in}{2.037047in}}%
\pgfpathlineto{\pgfqpoint{3.917646in}{2.042687in}}%
\pgfpathlineto{\pgfqpoint{3.920412in}{2.040927in}}%
\pgfpathlineto{\pgfqpoint{3.923005in}{2.044152in}}%
\pgfpathlineto{\pgfqpoint{3.925778in}{2.039231in}}%
\pgfpathlineto{\pgfqpoint{3.928347in}{2.041783in}}%
\pgfpathlineto{\pgfqpoint{3.931202in}{2.042781in}}%
\pgfpathlineto{\pgfqpoint{3.933714in}{2.040776in}}%
\pgfpathlineto{\pgfqpoint{3.936395in}{2.040547in}}%
\pgfpathlineto{\pgfqpoint{3.939075in}{2.032149in}}%
\pgfpathlineto{\pgfqpoint{3.941778in}{2.030890in}}%
\pgfpathlineto{\pgfqpoint{3.944431in}{2.030890in}}%
\pgfpathlineto{\pgfqpoint{3.947101in}{2.030890in}}%
\pgfpathlineto{\pgfqpoint{3.949894in}{2.034616in}}%
\pgfpathlineto{\pgfqpoint{3.952464in}{2.037503in}}%
\pgfpathlineto{\pgfqpoint{3.955211in}{2.035007in}}%
\pgfpathlineto{\pgfqpoint{3.957823in}{2.037840in}}%
\pgfpathlineto{\pgfqpoint{3.960635in}{2.039835in}}%
\pgfpathlineto{\pgfqpoint{3.963176in}{2.041620in}}%
\pgfpathlineto{\pgfqpoint{3.966013in}{2.042967in}}%
\pgfpathlineto{\pgfqpoint{3.968523in}{2.038206in}}%
\pgfpathlineto{\pgfqpoint{3.971250in}{2.036241in}}%
\pgfpathlineto{\pgfqpoint{3.973885in}{2.030890in}}%
\pgfpathlineto{\pgfqpoint{3.976563in}{2.030890in}}%
\pgfpathlineto{\pgfqpoint{3.979389in}{2.031509in}}%
\pgfpathlineto{\pgfqpoint{3.981929in}{2.037880in}}%
\pgfpathlineto{\pgfqpoint{3.984714in}{2.038460in}}%
\pgfpathlineto{\pgfqpoint{3.987270in}{2.033909in}}%
\pgfpathlineto{\pgfqpoint{3.990055in}{2.037874in}}%
\pgfpathlineto{\pgfqpoint{3.992642in}{2.036278in}}%
\pgfpathlineto{\pgfqpoint{3.995417in}{2.035411in}}%
\pgfpathlineto{\pgfqpoint{3.997990in}{2.034698in}}%
\pgfpathlineto{\pgfqpoint{4.000674in}{2.035669in}}%
\pgfpathlineto{\pgfqpoint{4.003348in}{2.039929in}}%
\pgfpathlineto{\pgfqpoint{4.006034in}{2.034422in}}%
\pgfpathlineto{\pgfqpoint{4.008699in}{2.032327in}}%
\pgfpathlineto{\pgfqpoint{4.011394in}{2.037522in}}%
\pgfpathlineto{\pgfqpoint{4.014186in}{2.036374in}}%
\pgfpathlineto{\pgfqpoint{4.016744in}{2.040586in}}%
\pgfpathlineto{\pgfqpoint{4.019518in}{2.033567in}}%
\pgfpathlineto{\pgfqpoint{4.022097in}{2.035123in}}%
\pgfpathlineto{\pgfqpoint{4.024868in}{2.036427in}}%
\pgfpathlineto{\pgfqpoint{4.027447in}{2.037408in}}%
\pgfpathlineto{\pgfqpoint{4.030229in}{2.040627in}}%
\pgfpathlineto{\pgfqpoint{4.032817in}{2.038023in}}%
\pgfpathlineto{\pgfqpoint{4.035492in}{2.037898in}}%
\pgfpathlineto{\pgfqpoint{4.038174in}{2.039170in}}%
\pgfpathlineto{\pgfqpoint{4.040852in}{2.040276in}}%
\pgfpathlineto{\pgfqpoint{4.043667in}{2.038908in}}%
\pgfpathlineto{\pgfqpoint{4.046210in}{2.034323in}}%
\pgfpathlineto{\pgfqpoint{4.049006in}{2.041353in}}%
\pgfpathlineto{\pgfqpoint{4.051557in}{2.040539in}}%
\pgfpathlineto{\pgfqpoint{4.054326in}{2.040249in}}%
\pgfpathlineto{\pgfqpoint{4.056911in}{2.037465in}}%
\pgfpathlineto{\pgfqpoint{4.059702in}{2.036544in}}%
\pgfpathlineto{\pgfqpoint{4.062266in}{2.038673in}}%
\pgfpathlineto{\pgfqpoint{4.064957in}{2.044586in}}%
\pgfpathlineto{\pgfqpoint{4.067636in}{2.042077in}}%
\pgfpathlineto{\pgfqpoint{4.070313in}{2.040994in}}%
\pgfpathlineto{\pgfqpoint{4.072985in}{2.040443in}}%
\pgfpathlineto{\pgfqpoint{4.075705in}{2.041529in}}%
\pgfpathlineto{\pgfqpoint{4.078471in}{2.034824in}}%
\pgfpathlineto{\pgfqpoint{4.081018in}{2.039320in}}%
\pgfpathlineto{\pgfqpoint{4.083870in}{2.036656in}}%
\pgfpathlineto{\pgfqpoint{4.086385in}{2.041368in}}%
\pgfpathlineto{\pgfqpoint{4.089159in}{2.038922in}}%
\pgfpathlineto{\pgfqpoint{4.091729in}{2.039252in}}%
\pgfpathlineto{\pgfqpoint{4.094527in}{2.043714in}}%
\pgfpathlineto{\pgfqpoint{4.097092in}{2.044097in}}%
\pgfpathlineto{\pgfqpoint{4.099777in}{2.045164in}}%
\pgfpathlineto{\pgfqpoint{4.102456in}{2.041120in}}%
\pgfpathlineto{\pgfqpoint{4.105185in}{2.040523in}}%
\pgfpathlineto{\pgfqpoint{4.107814in}{2.040906in}}%
\pgfpathlineto{\pgfqpoint{4.110488in}{2.042041in}}%
\pgfpathlineto{\pgfqpoint{4.113252in}{2.040895in}}%
\pgfpathlineto{\pgfqpoint{4.115844in}{2.038935in}}%
\pgfpathlineto{\pgfqpoint{4.118554in}{2.041982in}}%
\pgfpathlineto{\pgfqpoint{4.121205in}{2.040205in}}%
\pgfpathlineto{\pgfqpoint{4.124019in}{2.033784in}}%
\pgfpathlineto{\pgfqpoint{4.126553in}{2.032331in}}%
\pgfpathlineto{\pgfqpoint{4.129349in}{2.036926in}}%
\pgfpathlineto{\pgfqpoint{4.131920in}{2.031726in}}%
\pgfpathlineto{\pgfqpoint{4.134615in}{2.037101in}}%
\pgfpathlineto{\pgfqpoint{4.137272in}{2.038513in}}%
\pgfpathlineto{\pgfqpoint{4.139963in}{2.040970in}}%
\pgfpathlineto{\pgfqpoint{4.142713in}{2.033491in}}%
\pgfpathlineto{\pgfqpoint{4.145310in}{2.037857in}}%
\pgfpathlineto{\pgfqpoint{4.148082in}{2.039230in}}%
\pgfpathlineto{\pgfqpoint{4.150665in}{2.037689in}}%
\pgfpathlineto{\pgfqpoint{4.153423in}{2.037648in}}%
\pgfpathlineto{\pgfqpoint{4.156016in}{2.032661in}}%
\pgfpathlineto{\pgfqpoint{4.158806in}{2.030890in}}%
\pgfpathlineto{\pgfqpoint{4.161380in}{2.031034in}}%
\pgfpathlineto{\pgfqpoint{4.164059in}{2.036550in}}%
\pgfpathlineto{\pgfqpoint{4.166737in}{2.034066in}}%
\pgfpathlineto{\pgfqpoint{4.169415in}{2.035421in}}%
\pgfpathlineto{\pgfqpoint{4.172093in}{2.033619in}}%
\pgfpathlineto{\pgfqpoint{4.174770in}{2.036417in}}%
\pgfpathlineto{\pgfqpoint{4.177593in}{2.036548in}}%
\pgfpathlineto{\pgfqpoint{4.180129in}{2.036153in}}%
\pgfpathlineto{\pgfqpoint{4.182899in}{2.030890in}}%
\pgfpathlineto{\pgfqpoint{4.185481in}{2.030890in}}%
\pgfpathlineto{\pgfqpoint{4.188318in}{2.030890in}}%
\pgfpathlineto{\pgfqpoint{4.190842in}{2.036976in}}%
\pgfpathlineto{\pgfqpoint{4.193638in}{2.036920in}}%
\pgfpathlineto{\pgfqpoint{4.196186in}{2.034734in}}%
\pgfpathlineto{\pgfqpoint{4.198878in}{2.038364in}}%
\pgfpathlineto{\pgfqpoint{4.201542in}{2.040428in}}%
\pgfpathlineto{\pgfqpoint{4.204240in}{2.040671in}}%
\pgfpathlineto{\pgfqpoint{4.207076in}{2.033695in}}%
\pgfpathlineto{\pgfqpoint{4.209597in}{2.030890in}}%
\pgfpathlineto{\pgfqpoint{4.212383in}{2.031656in}}%
\pgfpathlineto{\pgfqpoint{4.214948in}{2.033645in}}%
\pgfpathlineto{\pgfqpoint{4.217694in}{2.036674in}}%
\pgfpathlineto{\pgfqpoint{4.220304in}{2.040549in}}%
\pgfpathlineto{\pgfqpoint{4.223082in}{2.042868in}}%
\pgfpathlineto{\pgfqpoint{4.225654in}{2.040475in}}%
\pgfpathlineto{\pgfqpoint{4.228331in}{2.048307in}}%
\pgfpathlineto{\pgfqpoint{4.231013in}{2.047300in}}%
\pgfpathlineto{\pgfqpoint{4.233691in}{2.043125in}}%
\pgfpathlineto{\pgfqpoint{4.236375in}{2.045794in}}%
\pgfpathlineto{\pgfqpoint{4.239084in}{2.041617in}}%
\pgfpathlineto{\pgfqpoint{4.241900in}{2.045562in}}%
\pgfpathlineto{\pgfqpoint{4.244394in}{2.043683in}}%
\pgfpathlineto{\pgfqpoint{4.247225in}{2.042149in}}%
\pgfpathlineto{\pgfqpoint{4.249767in}{2.042027in}}%
\pgfpathlineto{\pgfqpoint{4.252581in}{2.043732in}}%
\pgfpathlineto{\pgfqpoint{4.255120in}{2.045565in}}%
\pgfpathlineto{\pgfqpoint{4.257958in}{2.043994in}}%
\pgfpathlineto{\pgfqpoint{4.260477in}{2.042734in}}%
\pgfpathlineto{\pgfqpoint{4.263157in}{2.040064in}}%
\pgfpathlineto{\pgfqpoint{4.265824in}{2.038522in}}%
\pgfpathlineto{\pgfqpoint{4.268590in}{2.038180in}}%
\pgfpathlineto{\pgfqpoint{4.271187in}{2.042033in}}%
\pgfpathlineto{\pgfqpoint{4.273874in}{2.044015in}}%
\pgfpathlineto{\pgfqpoint{4.276635in}{2.042280in}}%
\pgfpathlineto{\pgfqpoint{4.279212in}{2.039854in}}%
\pgfpathlineto{\pgfqpoint{4.282000in}{2.038494in}}%
\pgfpathlineto{\pgfqpoint{4.284586in}{2.042826in}}%
\pgfpathlineto{\pgfqpoint{4.287399in}{2.039982in}}%
\pgfpathlineto{\pgfqpoint{4.289936in}{2.038252in}}%
\pgfpathlineto{\pgfqpoint{4.292786in}{2.039419in}}%
\pgfpathlineto{\pgfqpoint{4.295299in}{2.037522in}}%
\pgfpathlineto{\pgfqpoint{4.297977in}{2.039334in}}%
\pgfpathlineto{\pgfqpoint{4.300656in}{2.040542in}}%
\pgfpathlineto{\pgfqpoint{4.303357in}{2.036378in}}%
\pgfpathlineto{\pgfqpoint{4.306118in}{2.042553in}}%
\pgfpathlineto{\pgfqpoint{4.308691in}{2.041213in}}%
\pgfpathlineto{\pgfqpoint{4.311494in}{2.045551in}}%
\pgfpathlineto{\pgfqpoint{4.314032in}{2.050698in}}%
\pgfpathlineto{\pgfqpoint{4.316856in}{2.047769in}}%
\pgfpathlineto{\pgfqpoint{4.319405in}{2.040471in}}%
\pgfpathlineto{\pgfqpoint{4.322181in}{2.044356in}}%
\pgfpathlineto{\pgfqpoint{4.324760in}{2.043435in}}%
\pgfpathlineto{\pgfqpoint{4.327440in}{2.043943in}}%
\pgfpathlineto{\pgfqpoint{4.330118in}{2.044143in}}%
\pgfpathlineto{\pgfqpoint{4.332796in}{2.039517in}}%
\pgfpathlineto{\pgfqpoint{4.335463in}{2.036097in}}%
\pgfpathlineto{\pgfqpoint{4.338154in}{2.035974in}}%
\pgfpathlineto{\pgfqpoint{4.340976in}{2.039874in}}%
\pgfpathlineto{\pgfqpoint{4.343510in}{2.037990in}}%
\pgfpathlineto{\pgfqpoint{4.346263in}{2.036944in}}%
\pgfpathlineto{\pgfqpoint{4.348868in}{2.038652in}}%
\pgfpathlineto{\pgfqpoint{4.351645in}{2.035615in}}%
\pgfpathlineto{\pgfqpoint{4.354224in}{2.033640in}}%
\pgfpathlineto{\pgfqpoint{4.357014in}{2.034773in}}%
\pgfpathlineto{\pgfqpoint{4.359582in}{2.037235in}}%
\pgfpathlineto{\pgfqpoint{4.362270in}{2.033740in}}%
\pgfpathlineto{\pgfqpoint{4.364936in}{2.032773in}}%
\pgfpathlineto{\pgfqpoint{4.367646in}{2.033309in}}%
\pgfpathlineto{\pgfqpoint{4.370437in}{2.034330in}}%
\pgfpathlineto{\pgfqpoint{4.372976in}{2.035425in}}%
\pgfpathlineto{\pgfqpoint{4.375761in}{2.034316in}}%
\pgfpathlineto{\pgfqpoint{4.378329in}{2.038389in}}%
\pgfpathlineto{\pgfqpoint{4.381097in}{2.037689in}}%
\pgfpathlineto{\pgfqpoint{4.383674in}{2.037855in}}%
\pgfpathlineto{\pgfqpoint{4.386431in}{2.037259in}}%
\pgfpathlineto{\pgfqpoint{4.389044in}{2.037300in}}%
\pgfpathlineto{\pgfqpoint{4.391721in}{2.036943in}}%
\pgfpathlineto{\pgfqpoint{4.394400in}{2.031264in}}%
\pgfpathlineto{\pgfqpoint{4.397076in}{2.035893in}}%
\pgfpathlineto{\pgfqpoint{4.399745in}{2.034528in}}%
\pgfpathlineto{\pgfqpoint{4.402468in}{2.034678in}}%
\pgfpathlineto{\pgfqpoint{4.405234in}{2.035440in}}%
\pgfpathlineto{\pgfqpoint{4.407788in}{2.037649in}}%
\pgfpathlineto{\pgfqpoint{4.410587in}{2.034922in}}%
\pgfpathlineto{\pgfqpoint{4.413149in}{2.036455in}}%
\pgfpathlineto{\pgfqpoint{4.415932in}{2.041965in}}%
\pgfpathlineto{\pgfqpoint{4.418506in}{2.040055in}}%
\pgfpathlineto{\pgfqpoint{4.421292in}{2.039271in}}%
\pgfpathlineto{\pgfqpoint{4.423863in}{2.043356in}}%
\pgfpathlineto{\pgfqpoint{4.426534in}{2.042808in}}%
\pgfpathlineto{\pgfqpoint{4.429220in}{2.039343in}}%
\pgfpathlineto{\pgfqpoint{4.431901in}{2.042711in}}%
\pgfpathlineto{\pgfqpoint{4.434569in}{2.041413in}}%
\pgfpathlineto{\pgfqpoint{4.437253in}{2.037943in}}%
\pgfpathlineto{\pgfqpoint{4.440041in}{2.036734in}}%
\pgfpathlineto{\pgfqpoint{4.442611in}{2.041420in}}%
\pgfpathlineto{\pgfqpoint{4.445423in}{2.041999in}}%
\pgfpathlineto{\pgfqpoint{4.447965in}{2.037125in}}%
\pgfpathlineto{\pgfqpoint{4.450767in}{2.040885in}}%
\pgfpathlineto{\pgfqpoint{4.453312in}{2.044574in}}%
\pgfpathlineto{\pgfqpoint{4.456138in}{2.045913in}}%
\pgfpathlineto{\pgfqpoint{4.458681in}{2.053160in}}%
\pgfpathlineto{\pgfqpoint{4.461367in}{2.046757in}}%
\pgfpathlineto{\pgfqpoint{4.464029in}{2.047044in}}%
\pgfpathlineto{\pgfqpoint{4.466717in}{2.041820in}}%
\pgfpathlineto{\pgfqpoint{4.469492in}{2.044198in}}%
\pgfpathlineto{\pgfqpoint{4.472059in}{2.041436in}}%
\pgfpathlineto{\pgfqpoint{4.474861in}{2.045045in}}%
\pgfpathlineto{\pgfqpoint{4.477430in}{2.042378in}}%
\pgfpathlineto{\pgfqpoint{4.480201in}{2.046655in}}%
\pgfpathlineto{\pgfqpoint{4.482778in}{2.051326in}}%
\pgfpathlineto{\pgfqpoint{4.485581in}{2.049304in}}%
\pgfpathlineto{\pgfqpoint{4.488130in}{2.049999in}}%
\pgfpathlineto{\pgfqpoint{4.490822in}{2.047043in}}%
\pgfpathlineto{\pgfqpoint{4.493492in}{2.042686in}}%
\pgfpathlineto{\pgfqpoint{4.496167in}{2.046737in}}%
\pgfpathlineto{\pgfqpoint{4.498850in}{2.043842in}}%
\pgfpathlineto{\pgfqpoint{4.501529in}{2.042345in}}%
\pgfpathlineto{\pgfqpoint{4.504305in}{2.042344in}}%
\pgfpathlineto{\pgfqpoint{4.506893in}{2.039892in}}%
\pgfpathlineto{\pgfqpoint{4.509643in}{2.043204in}}%
\pgfpathlineto{\pgfqpoint{4.512246in}{2.041397in}}%
\pgfpathlineto{\pgfqpoint{4.515080in}{2.045475in}}%
\pgfpathlineto{\pgfqpoint{4.517598in}{2.044527in}}%
\pgfpathlineto{\pgfqpoint{4.520345in}{2.041953in}}%
\pgfpathlineto{\pgfqpoint{4.522962in}{2.039769in}}%
\pgfpathlineto{\pgfqpoint{4.525640in}{2.040307in}}%
\pgfpathlineto{\pgfqpoint{4.528307in}{2.041746in}}%
\pgfpathlineto{\pgfqpoint{4.530990in}{2.038136in}}%
\pgfpathlineto{\pgfqpoint{4.533764in}{2.036909in}}%
\pgfpathlineto{\pgfqpoint{4.536400in}{2.032664in}}%
\pgfpathlineto{\pgfqpoint{4.539144in}{2.030963in}}%
\pgfpathlineto{\pgfqpoint{4.541711in}{2.035181in}}%
\pgfpathlineto{\pgfqpoint{4.544464in}{2.037725in}}%
\pgfpathlineto{\pgfqpoint{4.547064in}{2.035354in}}%
\pgfpathlineto{\pgfqpoint{4.549822in}{2.041986in}}%
\pgfpathlineto{\pgfqpoint{4.552425in}{2.041141in}}%
\pgfpathlineto{\pgfqpoint{4.555106in}{2.039740in}}%
\pgfpathlineto{\pgfqpoint{4.557777in}{2.039034in}}%
\pgfpathlineto{\pgfqpoint{4.560448in}{2.038298in}}%
\pgfpathlineto{\pgfqpoint{4.563125in}{2.040035in}}%
\pgfpathlineto{\pgfqpoint{4.565820in}{2.043370in}}%
\pgfpathlineto{\pgfqpoint{4.568612in}{2.038408in}}%
\pgfpathlineto{\pgfqpoint{4.571171in}{2.037636in}}%
\pgfpathlineto{\pgfqpoint{4.573947in}{2.037696in}}%
\pgfpathlineto{\pgfqpoint{4.576531in}{2.041660in}}%
\pgfpathlineto{\pgfqpoint{4.579305in}{2.041045in}}%
\pgfpathlineto{\pgfqpoint{4.581888in}{2.039158in}}%
\pgfpathlineto{\pgfqpoint{4.584672in}{2.037418in}}%
\pgfpathlineto{\pgfqpoint{4.587244in}{2.035783in}}%
\pgfpathlineto{\pgfqpoint{4.589920in}{2.036391in}}%
\pgfpathlineto{\pgfqpoint{4.592589in}{2.034124in}}%
\pgfpathlineto{\pgfqpoint{4.595281in}{2.040042in}}%
\pgfpathlineto{\pgfqpoint{4.597951in}{2.033656in}}%
\pgfpathlineto{\pgfqpoint{4.600633in}{2.034040in}}%
\pgfpathlineto{\pgfqpoint{4.603430in}{2.039429in}}%
\pgfpathlineto{\pgfqpoint{4.605990in}{2.035814in}}%
\pgfpathlineto{\pgfqpoint{4.608808in}{2.036088in}}%
\pgfpathlineto{\pgfqpoint{4.611350in}{2.039545in}}%
\pgfpathlineto{\pgfqpoint{4.614134in}{2.039333in}}%
\pgfpathlineto{\pgfqpoint{4.616702in}{2.042615in}}%
\pgfpathlineto{\pgfqpoint{4.619529in}{2.041073in}}%
\pgfpathlineto{\pgfqpoint{4.622056in}{2.041414in}}%
\pgfpathlineto{\pgfqpoint{4.624741in}{2.041965in}}%
\pgfpathlineto{\pgfqpoint{4.627411in}{2.042249in}}%
\pgfpathlineto{\pgfqpoint{4.630096in}{2.043115in}}%
\pgfpathlineto{\pgfqpoint{4.632902in}{2.043108in}}%
\pgfpathlineto{\pgfqpoint{4.635445in}{2.042094in}}%
\pgfpathlineto{\pgfqpoint{4.638204in}{2.045973in}}%
\pgfpathlineto{\pgfqpoint{4.640809in}{2.042368in}}%
\pgfpathlineto{\pgfqpoint{4.643628in}{2.039578in}}%
\pgfpathlineto{\pgfqpoint{4.646169in}{2.042043in}}%
\pgfpathlineto{\pgfqpoint{4.648922in}{2.044114in}}%
\pgfpathlineto{\pgfqpoint{4.651524in}{2.043917in}}%
\pgfpathlineto{\pgfqpoint{4.654203in}{2.043171in}}%
\pgfpathlineto{\pgfqpoint{4.656873in}{2.046237in}}%
\pgfpathlineto{\pgfqpoint{4.659590in}{2.044205in}}%
\pgfpathlineto{\pgfqpoint{4.662237in}{2.045664in}}%
\pgfpathlineto{\pgfqpoint{4.664923in}{2.038660in}}%
\pgfpathlineto{\pgfqpoint{4.667764in}{2.042007in}}%
\pgfpathlineto{\pgfqpoint{4.670261in}{2.042593in}}%
\pgfpathlineto{\pgfqpoint{4.673068in}{2.042997in}}%
\pgfpathlineto{\pgfqpoint{4.675619in}{2.038879in}}%
\pgfpathlineto{\pgfqpoint{4.678448in}{2.037861in}}%
\pgfpathlineto{\pgfqpoint{4.680988in}{2.035618in}}%
\pgfpathlineto{\pgfqpoint{4.683799in}{2.039542in}}%
\pgfpathlineto{\pgfqpoint{4.686337in}{2.050341in}}%
\pgfpathlineto{\pgfqpoint{4.689051in}{2.040048in}}%
\pgfpathlineto{\pgfqpoint{4.691694in}{2.041108in}}%
\pgfpathlineto{\pgfqpoint{4.694381in}{2.041027in}}%
\pgfpathlineto{\pgfqpoint{4.697170in}{2.040130in}}%
\pgfpathlineto{\pgfqpoint{4.699734in}{2.039767in}}%
\pgfpathlineto{\pgfqpoint{4.702517in}{2.042586in}}%
\pgfpathlineto{\pgfqpoint{4.705094in}{2.041109in}}%
\pgfpathlineto{\pgfqpoint{4.707824in}{2.041700in}}%
\pgfpathlineto{\pgfqpoint{4.710437in}{2.046096in}}%
\pgfpathlineto{\pgfqpoint{4.713275in}{2.053380in}}%
\pgfpathlineto{\pgfqpoint{4.715806in}{2.044914in}}%
\pgfpathlineto{\pgfqpoint{4.718486in}{2.046088in}}%
\pgfpathlineto{\pgfqpoint{4.721160in}{2.039734in}}%
\pgfpathlineto{\pgfqpoint{4.723873in}{2.036695in}}%
\pgfpathlineto{\pgfqpoint{4.726508in}{2.038464in}}%
\pgfpathlineto{\pgfqpoint{4.729233in}{2.040527in}}%
\pgfpathlineto{\pgfqpoint{4.731901in}{2.042706in}}%
\pgfpathlineto{\pgfqpoint{4.734552in}{2.039128in}}%
\pgfpathlineto{\pgfqpoint{4.737348in}{2.045395in}}%
\pgfpathlineto{\pgfqpoint{4.739912in}{2.044311in}}%
\pgfpathlineto{\pgfqpoint{4.742696in}{2.043268in}}%
\pgfpathlineto{\pgfqpoint{4.745256in}{2.045686in}}%
\pgfpathlineto{\pgfqpoint{4.748081in}{2.046290in}}%
\pgfpathlineto{\pgfqpoint{4.750627in}{2.043292in}}%
\pgfpathlineto{\pgfqpoint{4.753298in}{2.044997in}}%
\pgfpathlineto{\pgfqpoint{4.755983in}{2.045603in}}%
\pgfpathlineto{\pgfqpoint{4.758653in}{2.046570in}}%
\pgfpathlineto{\pgfqpoint{4.761337in}{2.044593in}}%
\pgfpathlineto{\pgfqpoint{4.764018in}{2.044098in}}%
\pgfpathlineto{\pgfqpoint{4.766783in}{2.045875in}}%
\pgfpathlineto{\pgfqpoint{4.769367in}{2.046233in}}%
\pgfpathlineto{\pgfqpoint{4.772198in}{2.044879in}}%
\pgfpathlineto{\pgfqpoint{4.774732in}{2.044837in}}%
\pgfpathlineto{\pgfqpoint{4.777535in}{2.043620in}}%
\pgfpathlineto{\pgfqpoint{4.780083in}{2.045880in}}%
\pgfpathlineto{\pgfqpoint{4.782872in}{2.038017in}}%
\pgfpathlineto{\pgfqpoint{4.785445in}{2.042579in}}%
\pgfpathlineto{\pgfqpoint{4.788116in}{2.044127in}}%
\pgfpathlineto{\pgfqpoint{4.790798in}{2.042055in}}%
\pgfpathlineto{\pgfqpoint{4.793512in}{2.043143in}}%
\pgfpathlineto{\pgfqpoint{4.796274in}{2.042145in}}%
\pgfpathlineto{\pgfqpoint{4.798830in}{2.041059in}}%
\pgfpathlineto{\pgfqpoint{4.801586in}{2.039706in}}%
\pgfpathlineto{\pgfqpoint{4.804193in}{2.038686in}}%
\pgfpathlineto{\pgfqpoint{4.807017in}{2.036091in}}%
\pgfpathlineto{\pgfqpoint{4.809538in}{2.037367in}}%
\pgfpathlineto{\pgfqpoint{4.812377in}{2.034445in}}%
\pgfpathlineto{\pgfqpoint{4.814907in}{2.044204in}}%
\pgfpathlineto{\pgfqpoint{4.817587in}{2.084509in}}%
\pgfpathlineto{\pgfqpoint{4.820265in}{2.056821in}}%
\pgfpathlineto{\pgfqpoint{4.822945in}{2.039491in}}%
\pgfpathlineto{\pgfqpoint{4.825619in}{2.044306in}}%
\pgfpathlineto{\pgfqpoint{4.828291in}{2.042514in}}%
\pgfpathlineto{\pgfqpoint{4.831045in}{2.047479in}}%
\pgfpathlineto{\pgfqpoint{4.833657in}{2.061622in}}%
\pgfpathlineto{\pgfqpoint{4.837992in}{2.054240in}}%
\pgfpathlineto{\pgfqpoint{4.839922in}{2.051731in}}%
\pgfpathlineto{\pgfqpoint{4.842380in}{2.045436in}}%
\pgfpathlineto{\pgfqpoint{4.844361in}{2.038165in}}%
\pgfpathlineto{\pgfqpoint{4.847127in}{2.034549in}}%
\pgfpathlineto{\pgfqpoint{4.849715in}{2.040803in}}%
\pgfpathlineto{\pgfqpoint{4.852404in}{2.036160in}}%
\pgfpathlineto{\pgfqpoint{4.855070in}{2.035222in}}%
\pgfpathlineto{\pgfqpoint{4.857807in}{2.038769in}}%
\pgfpathlineto{\pgfqpoint{4.860544in}{2.034528in}}%
\pgfpathlineto{\pgfqpoint{4.863116in}{2.037723in}}%
\pgfpathlineto{\pgfqpoint{4.865910in}{2.041707in}}%
\pgfpathlineto{\pgfqpoint{4.868474in}{2.041738in}}%
\pgfpathlineto{\pgfqpoint{4.871209in}{2.043767in}}%
\pgfpathlineto{\pgfqpoint{4.873832in}{2.043183in}}%
\pgfpathlineto{\pgfqpoint{4.876636in}{2.039086in}}%
\pgfpathlineto{\pgfqpoint{4.879180in}{2.037225in}}%
\pgfpathlineto{\pgfqpoint{4.881864in}{2.038685in}}%
\pgfpathlineto{\pgfqpoint{4.884540in}{2.042365in}}%
\pgfpathlineto{\pgfqpoint{4.887211in}{2.039567in}}%
\pgfpathlineto{\pgfqpoint{4.889902in}{2.043120in}}%
\pgfpathlineto{\pgfqpoint{4.892611in}{2.045101in}}%
\pgfpathlineto{\pgfqpoint{4.895399in}{2.044800in}}%
\pgfpathlineto{\pgfqpoint{4.897938in}{2.051354in}}%
\pgfpathlineto{\pgfqpoint{4.900712in}{2.047015in}}%
\pgfpathlineto{\pgfqpoint{4.903295in}{2.049861in}}%
\pgfpathlineto{\pgfqpoint{4.906096in}{2.044042in}}%
\pgfpathlineto{\pgfqpoint{4.908648in}{2.040324in}}%
\pgfpathlineto{\pgfqpoint{4.911435in}{2.039352in}}%
\pgfpathlineto{\pgfqpoint{4.914009in}{2.036610in}}%
\pgfpathlineto{\pgfqpoint{4.916681in}{2.040055in}}%
\pgfpathlineto{\pgfqpoint{4.919352in}{2.041998in}}%
\pgfpathlineto{\pgfqpoint{4.922041in}{2.043363in}}%
\pgfpathlineto{\pgfqpoint{4.924708in}{2.042519in}}%
\pgfpathlineto{\pgfqpoint{4.927400in}{2.044716in}}%
\pgfpathlineto{\pgfqpoint{4.930170in}{2.045647in}}%
\pgfpathlineto{\pgfqpoint{4.932742in}{2.041698in}}%
\pgfpathlineto{\pgfqpoint{4.935515in}{2.039840in}}%
\pgfpathlineto{\pgfqpoint{4.938112in}{2.043139in}}%
\pgfpathlineto{\pgfqpoint{4.940881in}{2.039735in}}%
\pgfpathlineto{\pgfqpoint{4.943466in}{2.041933in}}%
\pgfpathlineto{\pgfqpoint{4.946151in}{2.046239in}}%
\pgfpathlineto{\pgfqpoint{4.948827in}{2.043923in}}%
\pgfpathlineto{\pgfqpoint{4.951504in}{2.046517in}}%
\pgfpathlineto{\pgfqpoint{4.954182in}{2.047831in}}%
\pgfpathlineto{\pgfqpoint{4.956862in}{2.049410in}}%
\pgfpathlineto{\pgfqpoint{4.959689in}{2.053054in}}%
\pgfpathlineto{\pgfqpoint{4.962219in}{2.048147in}}%
\pgfpathlineto{\pgfqpoint{4.965002in}{2.044356in}}%
\pgfpathlineto{\pgfqpoint{4.967575in}{2.039029in}}%
\pgfpathlineto{\pgfqpoint{4.970314in}{2.038489in}}%
\pgfpathlineto{\pgfqpoint{4.972933in}{2.042154in}}%
\pgfpathlineto{\pgfqpoint{4.975703in}{2.030890in}}%
\pgfpathlineto{\pgfqpoint{4.978287in}{2.039645in}}%
\pgfpathlineto{\pgfqpoint{4.980967in}{2.038647in}}%
\pgfpathlineto{\pgfqpoint{4.983637in}{2.036125in}}%
\pgfpathlineto{\pgfqpoint{4.986325in}{2.034655in}}%
\pgfpathlineto{\pgfqpoint{4.989001in}{2.037852in}}%
\pgfpathlineto{\pgfqpoint{4.991683in}{2.046561in}}%
\pgfpathlineto{\pgfqpoint{4.994390in}{2.042157in}}%
\pgfpathlineto{\pgfqpoint{4.997028in}{2.037984in}}%
\pgfpathlineto{\pgfqpoint{4.999780in}{2.039426in}}%
\pgfpathlineto{\pgfqpoint{5.002384in}{2.047413in}}%
\pgfpathlineto{\pgfqpoint{5.005178in}{2.041961in}}%
\pgfpathlineto{\pgfqpoint{5.007751in}{2.035771in}}%
\pgfpathlineto{\pgfqpoint{5.010562in}{2.038247in}}%
\pgfpathlineto{\pgfqpoint{5.013104in}{2.038647in}}%
\pgfpathlineto{\pgfqpoint{5.015820in}{2.041536in}}%
\pgfpathlineto{\pgfqpoint{5.018466in}{2.042068in}}%
\pgfpathlineto{\pgfqpoint{5.021147in}{2.040558in}}%
\pgfpathlineto{\pgfqpoint{5.023927in}{2.037480in}}%
\pgfpathlineto{\pgfqpoint{5.026501in}{2.037318in}}%
\pgfpathlineto{\pgfqpoint{5.029275in}{2.039599in}}%
\pgfpathlineto{\pgfqpoint{5.031849in}{2.039781in}}%
\pgfpathlineto{\pgfqpoint{5.034649in}{2.048189in}}%
\pgfpathlineto{\pgfqpoint{5.037214in}{2.045157in}}%
\pgfpathlineto{\pgfqpoint{5.039962in}{2.044807in}}%
\pgfpathlineto{\pgfqpoint{5.042572in}{2.040453in}}%
\pgfpathlineto{\pgfqpoint{5.045249in}{2.044005in}}%
\pgfpathlineto{\pgfqpoint{5.047924in}{2.039688in}}%
\pgfpathlineto{\pgfqpoint{5.050606in}{2.043321in}}%
\pgfpathlineto{\pgfqpoint{5.053284in}{2.049169in}}%
\pgfpathlineto{\pgfqpoint{5.055952in}{2.046806in}}%
\pgfpathlineto{\pgfqpoint{5.058711in}{2.047212in}}%
\pgfpathlineto{\pgfqpoint{5.061315in}{2.044985in}}%
\pgfpathlineto{\pgfqpoint{5.064144in}{2.047094in}}%
\pgfpathlineto{\pgfqpoint{5.066677in}{2.048085in}}%
\pgfpathlineto{\pgfqpoint{5.069463in}{2.047199in}}%
\pgfpathlineto{\pgfqpoint{5.072030in}{2.047406in}}%
\pgfpathlineto{\pgfqpoint{5.074851in}{2.054315in}}%
\pgfpathlineto{\pgfqpoint{5.077390in}{2.050494in}}%
\pgfpathlineto{\pgfqpoint{5.080067in}{2.048722in}}%
\pgfpathlineto{\pgfqpoint{5.082746in}{2.048387in}}%
\pgfpathlineto{\pgfqpoint{5.085426in}{2.044933in}}%
\pgfpathlineto{\pgfqpoint{5.088103in}{2.043572in}}%
\pgfpathlineto{\pgfqpoint{5.090788in}{2.045993in}}%
\pgfpathlineto{\pgfqpoint{5.093579in}{2.050298in}}%
\pgfpathlineto{\pgfqpoint{5.096142in}{2.051417in}}%
\pgfpathlineto{\pgfqpoint{5.098948in}{2.048886in}}%
\pgfpathlineto{\pgfqpoint{5.101496in}{2.044780in}}%
\pgfpathlineto{\pgfqpoint{5.104312in}{2.045981in}}%
\pgfpathlineto{\pgfqpoint{5.106842in}{2.042463in}}%
\pgfpathlineto{\pgfqpoint{5.109530in}{2.044135in}}%
\pgfpathlineto{\pgfqpoint{5.112209in}{2.043122in}}%
\pgfpathlineto{\pgfqpoint{5.114887in}{2.040273in}}%
\pgfpathlineto{\pgfqpoint{5.117550in}{2.041671in}}%
\pgfpathlineto{\pgfqpoint{5.120243in}{2.044015in}}%
\pgfpathlineto{\pgfqpoint{5.123042in}{2.040156in}}%
\pgfpathlineto{\pgfqpoint{5.125599in}{2.039231in}}%
\pgfpathlineto{\pgfqpoint{5.128421in}{2.035911in}}%
\pgfpathlineto{\pgfqpoint{5.130953in}{2.039933in}}%
\pgfpathlineto{\pgfqpoint{5.133716in}{2.042200in}}%
\pgfpathlineto{\pgfqpoint{5.136311in}{2.044587in}}%
\pgfpathlineto{\pgfqpoint{5.139072in}{2.045299in}}%
\pgfpathlineto{\pgfqpoint{5.141660in}{2.044906in}}%
\pgfpathlineto{\pgfqpoint{5.144349in}{2.047707in}}%
\pgfpathlineto{\pgfqpoint{5.147029in}{2.044613in}}%
\pgfpathlineto{\pgfqpoint{5.149734in}{2.043844in}}%
\pgfpathlineto{\pgfqpoint{5.152382in}{2.044210in}}%
\pgfpathlineto{\pgfqpoint{5.155059in}{2.049987in}}%
\pgfpathlineto{\pgfqpoint{5.157815in}{2.042107in}}%
\pgfpathlineto{\pgfqpoint{5.160420in}{2.045601in}}%
\pgfpathlineto{\pgfqpoint{5.163243in}{2.044509in}}%
\pgfpathlineto{\pgfqpoint{5.165775in}{2.040070in}}%
\pgfpathlineto{\pgfqpoint{5.168591in}{2.044111in}}%
\pgfpathlineto{\pgfqpoint{5.171133in}{2.042399in}}%
\pgfpathlineto{\pgfqpoint{5.173925in}{2.044152in}}%
\pgfpathlineto{\pgfqpoint{5.176477in}{2.044424in}}%
\pgfpathlineto{\pgfqpoint{5.179188in}{2.046303in}}%
\pgfpathlineto{\pgfqpoint{5.181848in}{2.038703in}}%
\pgfpathlineto{\pgfqpoint{5.184522in}{2.043143in}}%
\pgfpathlineto{\pgfqpoint{5.187294in}{2.041750in}}%
\pgfpathlineto{\pgfqpoint{5.189880in}{2.042228in}}%
\pgfpathlineto{\pgfqpoint{5.192680in}{2.041388in}}%
\pgfpathlineto{\pgfqpoint{5.195239in}{2.042782in}}%
\pgfpathlineto{\pgfqpoint{5.198008in}{2.045898in}}%
\pgfpathlineto{\pgfqpoint{5.200594in}{2.041586in}}%
\pgfpathlineto{\pgfqpoint{5.203388in}{2.038962in}}%
\pgfpathlineto{\pgfqpoint{5.205952in}{2.030890in}}%
\pgfpathlineto{\pgfqpoint{5.208630in}{2.030890in}}%
\pgfpathlineto{\pgfqpoint{5.211299in}{2.035197in}}%
\pgfpathlineto{\pgfqpoint{5.214027in}{2.038335in}}%
\pgfpathlineto{\pgfqpoint{5.216667in}{2.034757in}}%
\pgfpathlineto{\pgfqpoint{5.219345in}{2.038471in}}%
\pgfpathlineto{\pgfqpoint{5.222151in}{2.036957in}}%
\pgfpathlineto{\pgfqpoint{5.224695in}{2.035515in}}%
\pgfpathlineto{\pgfqpoint{5.227470in}{2.035585in}}%
\pgfpathlineto{\pgfqpoint{5.230059in}{2.037221in}}%
\pgfpathlineto{\pgfqpoint{5.232855in}{2.033534in}}%
\pgfpathlineto{\pgfqpoint{5.235409in}{2.030890in}}%
\pgfpathlineto{\pgfqpoint{5.238173in}{2.037541in}}%
\pgfpathlineto{\pgfqpoint{5.240777in}{2.038283in}}%
\pgfpathlineto{\pgfqpoint{5.243445in}{2.033377in}}%
\pgfpathlineto{\pgfqpoint{5.246130in}{2.034549in}}%
\pgfpathlineto{\pgfqpoint{5.248816in}{2.033768in}}%
\pgfpathlineto{\pgfqpoint{5.251590in}{2.037365in}}%
\pgfpathlineto{\pgfqpoint{5.254236in}{2.039313in}}%
\pgfpathlineto{\pgfqpoint{5.256973in}{2.039590in}}%
\pgfpathlineto{\pgfqpoint{5.259511in}{2.039694in}}%
\pgfpathlineto{\pgfqpoint{5.262264in}{2.041418in}}%
\pgfpathlineto{\pgfqpoint{5.264876in}{2.051499in}}%
\pgfpathlineto{\pgfqpoint{5.267691in}{2.045570in}}%
\pgfpathlineto{\pgfqpoint{5.270238in}{2.044142in}}%
\pgfpathlineto{\pgfqpoint{5.272913in}{2.032411in}}%
\pgfpathlineto{\pgfqpoint{5.275589in}{2.030890in}}%
\pgfpathlineto{\pgfqpoint{5.278322in}{2.041353in}}%
\pgfpathlineto{\pgfqpoint{5.280947in}{2.039123in}}%
\pgfpathlineto{\pgfqpoint{5.283631in}{2.038282in}}%
\pgfpathlineto{\pgfqpoint{5.286436in}{2.038416in}}%
\pgfpathlineto{\pgfqpoint{5.288984in}{2.043169in}}%
\pgfpathlineto{\pgfqpoint{5.291794in}{2.045678in}}%
\pgfpathlineto{\pgfqpoint{5.294339in}{2.048603in}}%
\pgfpathlineto{\pgfqpoint{5.297140in}{2.044740in}}%
\pgfpathlineto{\pgfqpoint{5.299696in}{2.043813in}}%
\pgfpathlineto{\pgfqpoint{5.302443in}{2.043205in}}%
\pgfpathlineto{\pgfqpoint{5.305054in}{2.044365in}}%
\pgfpathlineto{\pgfqpoint{5.307731in}{2.040879in}}%
\pgfpathlineto{\pgfqpoint{5.310411in}{2.045220in}}%
\pgfpathlineto{\pgfqpoint{5.313089in}{2.042751in}}%
\pgfpathlineto{\pgfqpoint{5.315754in}{2.046752in}}%
\pgfpathlineto{\pgfqpoint{5.318430in}{2.040934in}}%
\pgfpathlineto{\pgfqpoint{5.321256in}{2.033133in}}%
\pgfpathlineto{\pgfqpoint{5.323802in}{2.037886in}}%
\pgfpathlineto{\pgfqpoint{5.326564in}{2.032443in}}%
\pgfpathlineto{\pgfqpoint{5.329159in}{2.030890in}}%
\pgfpathlineto{\pgfqpoint{5.331973in}{2.030890in}}%
\pgfpathlineto{\pgfqpoint{5.334510in}{2.032925in}}%
\pgfpathlineto{\pgfqpoint{5.337353in}{2.039103in}}%
\pgfpathlineto{\pgfqpoint{5.339872in}{2.049856in}}%
\pgfpathlineto{\pgfqpoint{5.342549in}{2.062737in}}%
\pgfpathlineto{\pgfqpoint{5.345224in}{2.077565in}}%
\pgfpathlineto{\pgfqpoint{5.347905in}{2.068587in}}%
\pgfpathlineto{\pgfqpoint{5.350723in}{2.049265in}}%
\pgfpathlineto{\pgfqpoint{5.353262in}{2.042101in}}%
\pgfpathlineto{\pgfqpoint{5.356056in}{2.041966in}}%
\pgfpathlineto{\pgfqpoint{5.358612in}{2.042522in}}%
\pgfpathlineto{\pgfqpoint{5.361370in}{2.044664in}}%
\pgfpathlineto{\pgfqpoint{5.363966in}{2.053193in}}%
\pgfpathlineto{\pgfqpoint{5.366727in}{2.049784in}}%
\pgfpathlineto{\pgfqpoint{5.369335in}{2.044404in}}%
\pgfpathlineto{\pgfqpoint{5.372013in}{2.046691in}}%
\pgfpathlineto{\pgfqpoint{5.374692in}{2.045680in}}%
\pgfpathlineto{\pgfqpoint{5.377370in}{2.042385in}}%
\pgfpathlineto{\pgfqpoint{5.380048in}{2.042474in}}%
\pgfpathlineto{\pgfqpoint{5.382725in}{2.044158in}}%
\pgfpathlineto{\pgfqpoint{5.385550in}{2.044588in}}%
\pgfpathlineto{\pgfqpoint{5.388083in}{2.042286in}}%
\pgfpathlineto{\pgfqpoint{5.390900in}{2.048754in}}%
\pgfpathlineto{\pgfqpoint{5.393441in}{2.050538in}}%
\pgfpathlineto{\pgfqpoint{5.396219in}{2.041553in}}%
\pgfpathlineto{\pgfqpoint{5.398784in}{2.041980in}}%
\pgfpathlineto{\pgfqpoint{5.401576in}{2.039262in}}%
\pgfpathlineto{\pgfqpoint{5.404154in}{2.034662in}}%
\pgfpathlineto{\pgfqpoint{5.406832in}{2.038405in}}%
\pgfpathlineto{\pgfqpoint{5.409507in}{2.042336in}}%
\pgfpathlineto{\pgfqpoint{5.412190in}{2.037434in}}%
\pgfpathlineto{\pgfqpoint{5.414954in}{2.038758in}}%
\pgfpathlineto{\pgfqpoint{5.417547in}{2.035933in}}%
\pgfpathlineto{\pgfqpoint{5.420304in}{2.036738in}}%
\pgfpathlineto{\pgfqpoint{5.422897in}{2.041383in}}%
\pgfpathlineto{\pgfqpoint{5.425661in}{2.032846in}}%
\pgfpathlineto{\pgfqpoint{5.428259in}{2.036578in}}%
\pgfpathlineto{\pgfqpoint{5.431015in}{2.041843in}}%
\pgfpathlineto{\pgfqpoint{5.433616in}{2.036859in}}%
\pgfpathlineto{\pgfqpoint{5.436295in}{2.040887in}}%
\pgfpathlineto{\pgfqpoint{5.438974in}{2.037699in}}%
\pgfpathlineto{\pgfqpoint{5.441698in}{2.037981in}}%
\pgfpathlineto{\pgfqpoint{5.444328in}{2.039943in}}%
\pgfpathlineto{\pgfqpoint{5.447021in}{2.037687in}}%
\pgfpathlineto{\pgfqpoint{5.449769in}{2.036535in}}%
\pgfpathlineto{\pgfqpoint{5.452365in}{2.041219in}}%
\pgfpathlineto{\pgfqpoint{5.455168in}{2.042809in}}%
\pgfpathlineto{\pgfqpoint{5.457721in}{2.040391in}}%
\pgfpathlineto{\pgfqpoint{5.460489in}{2.043658in}}%
\pgfpathlineto{\pgfqpoint{5.463079in}{2.039063in}}%
\pgfpathlineto{\pgfqpoint{5.465888in}{2.034448in}}%
\pgfpathlineto{\pgfqpoint{5.468425in}{2.036002in}}%
\pgfpathlineto{\pgfqpoint{5.471113in}{2.036280in}}%
\pgfpathlineto{\pgfqpoint{5.473792in}{2.039688in}}%
\pgfpathlineto{\pgfqpoint{5.476458in}{2.038295in}}%
\pgfpathlineto{\pgfqpoint{5.479152in}{2.042125in}}%
\pgfpathlineto{\pgfqpoint{5.481825in}{2.038893in}}%
\pgfpathlineto{\pgfqpoint{5.484641in}{2.036792in}}%
\pgfpathlineto{\pgfqpoint{5.487176in}{2.037808in}}%
\pgfpathlineto{\pgfqpoint{5.490000in}{2.042122in}}%
\pgfpathlineto{\pgfqpoint{5.492541in}{2.041329in}}%
\pgfpathlineto{\pgfqpoint{5.495346in}{2.041327in}}%
\pgfpathlineto{\pgfqpoint{5.497898in}{2.044764in}}%
\pgfpathlineto{\pgfqpoint{5.500687in}{2.039756in}}%
\pgfpathlineto{\pgfqpoint{5.503255in}{2.044061in}}%
\pgfpathlineto{\pgfqpoint{5.505933in}{2.044352in}}%
\pgfpathlineto{\pgfqpoint{5.508612in}{2.041732in}}%
\pgfpathlineto{\pgfqpoint{5.511290in}{2.044161in}}%
\pgfpathlineto{\pgfqpoint{5.514080in}{2.043311in}}%
\pgfpathlineto{\pgfqpoint{5.516646in}{2.044369in}}%
\pgfpathlineto{\pgfqpoint{5.519433in}{2.045331in}}%
\pgfpathlineto{\pgfqpoint{5.522003in}{2.042923in}}%
\pgfpathlineto{\pgfqpoint{5.524756in}{2.043533in}}%
\pgfpathlineto{\pgfqpoint{5.527360in}{2.039757in}}%
\pgfpathlineto{\pgfqpoint{5.530148in}{2.044643in}}%
\pgfpathlineto{\pgfqpoint{5.532717in}{2.038952in}}%
\pgfpathlineto{\pgfqpoint{5.535395in}{2.039908in}}%
\pgfpathlineto{\pgfqpoint{5.538074in}{2.040468in}}%
\pgfpathlineto{\pgfqpoint{5.540750in}{2.041276in}}%
\pgfpathlineto{\pgfqpoint{5.543421in}{2.041969in}}%
\pgfpathlineto{\pgfqpoint{5.546110in}{2.040571in}}%
\pgfpathlineto{\pgfqpoint{5.548921in}{2.039984in}}%
\pgfpathlineto{\pgfqpoint{5.551457in}{2.041279in}}%
\pgfpathlineto{\pgfqpoint{5.554198in}{2.037785in}}%
\pgfpathlineto{\pgfqpoint{5.556822in}{2.042107in}}%
\pgfpathlineto{\pgfqpoint{5.559612in}{2.041651in}}%
\pgfpathlineto{\pgfqpoint{5.562180in}{2.039599in}}%
\pgfpathlineto{\pgfqpoint{5.564940in}{2.039384in}}%
\pgfpathlineto{\pgfqpoint{5.567536in}{2.042721in}}%
\pgfpathlineto{\pgfqpoint{5.570215in}{2.041973in}}%
\pgfpathlineto{\pgfqpoint{5.572893in}{2.040091in}}%
\pgfpathlineto{\pgfqpoint{5.575596in}{2.038288in}}%
\pgfpathlineto{\pgfqpoint{5.578342in}{2.041021in}}%
\pgfpathlineto{\pgfqpoint{5.580914in}{2.040007in}}%
\pgfpathlineto{\pgfqpoint{5.583709in}{2.033505in}}%
\pgfpathlineto{\pgfqpoint{5.586269in}{2.037817in}}%
\pgfpathlineto{\pgfqpoint{5.589040in}{2.041496in}}%
\pgfpathlineto{\pgfqpoint{5.591641in}{2.040901in}}%
\pgfpathlineto{\pgfqpoint{5.594368in}{2.038850in}}%
\pgfpathlineto{\pgfqpoint{5.596999in}{2.041282in}}%
\pgfpathlineto{\pgfqpoint{5.599674in}{2.042635in}}%
\pgfpathlineto{\pgfqpoint{5.602352in}{2.042656in}}%
\pgfpathlineto{\pgfqpoint{5.605073in}{2.041987in}}%
\pgfpathlineto{\pgfqpoint{5.607698in}{2.043151in}}%
\pgfpathlineto{\pgfqpoint{5.610389in}{2.042636in}}%
\pgfpathlineto{\pgfqpoint{5.613235in}{2.044977in}}%
\pgfpathlineto{\pgfqpoint{5.615743in}{2.047913in}}%
\pgfpathlineto{\pgfqpoint{5.618526in}{2.042957in}}%
\pgfpathlineto{\pgfqpoint{5.621102in}{2.042885in}}%
\pgfpathlineto{\pgfqpoint{5.623868in}{2.040917in}}%
\pgfpathlineto{\pgfqpoint{5.626460in}{2.040054in}}%
\pgfpathlineto{\pgfqpoint{5.629232in}{2.034345in}}%
\pgfpathlineto{\pgfqpoint{5.631815in}{2.038254in}}%
\pgfpathlineto{\pgfqpoint{5.634496in}{2.055101in}}%
\pgfpathlineto{\pgfqpoint{5.637172in}{2.077037in}}%
\pgfpathlineto{\pgfqpoint{5.639852in}{2.080461in}}%
\pgfpathlineto{\pgfqpoint{5.642518in}{2.065815in}}%
\pgfpathlineto{\pgfqpoint{5.645243in}{2.062598in}}%
\pgfpathlineto{\pgfqpoint{5.648008in}{2.047240in}}%
\pgfpathlineto{\pgfqpoint{5.650563in}{2.060589in}}%
\pgfpathlineto{\pgfqpoint{5.653376in}{2.071682in}}%
\pgfpathlineto{\pgfqpoint{5.655919in}{2.074446in}}%
\pgfpathlineto{\pgfqpoint{5.658723in}{2.071962in}}%
\pgfpathlineto{\pgfqpoint{5.661273in}{2.067120in}}%
\pgfpathlineto{\pgfqpoint{5.664099in}{2.056081in}}%
\pgfpathlineto{\pgfqpoint{5.666632in}{2.044666in}}%
\pgfpathlineto{\pgfqpoint{5.669313in}{2.038882in}}%
\pgfpathlineto{\pgfqpoint{5.671991in}{2.042666in}}%
\pgfpathlineto{\pgfqpoint{5.674667in}{2.044469in}}%
\pgfpathlineto{\pgfqpoint{5.677486in}{2.044032in}}%
\pgfpathlineto{\pgfqpoint{5.680027in}{2.040684in}}%
\pgfpathlineto{\pgfqpoint{5.682836in}{2.040050in}}%
\pgfpathlineto{\pgfqpoint{5.685385in}{2.039445in}}%
\pgfpathlineto{\pgfqpoint{5.688159in}{2.037435in}}%
\pgfpathlineto{\pgfqpoint{5.690730in}{2.043705in}}%
\pgfpathlineto{\pgfqpoint{5.693473in}{2.067660in}}%
\pgfpathlineto{\pgfqpoint{5.696101in}{2.062937in}}%
\pgfpathlineto{\pgfqpoint{5.698775in}{2.056245in}}%
\pgfpathlineto{\pgfqpoint{5.701453in}{2.050199in}}%
\pgfpathlineto{\pgfqpoint{5.704130in}{2.045093in}}%
\pgfpathlineto{\pgfqpoint{5.706800in}{2.041981in}}%
\pgfpathlineto{\pgfqpoint{5.709490in}{2.040902in}}%
\pgfpathlineto{\pgfqpoint{5.712291in}{2.035545in}}%
\pgfpathlineto{\pgfqpoint{5.714834in}{2.037789in}}%
\pgfpathlineto{\pgfqpoint{5.717671in}{2.037588in}}%
\pgfpathlineto{\pgfqpoint{5.720201in}{2.038484in}}%
\pgfpathlineto{\pgfqpoint{5.722950in}{2.036105in}}%
\pgfpathlineto{\pgfqpoint{5.725548in}{2.035566in}}%
\pgfpathlineto{\pgfqpoint{5.728339in}{2.031848in}}%
\pgfpathlineto{\pgfqpoint{5.730919in}{2.034272in}}%
\pgfpathlineto{\pgfqpoint{5.733594in}{2.033350in}}%
\pgfpathlineto{\pgfqpoint{5.736276in}{2.041022in}}%
\pgfpathlineto{\pgfqpoint{5.738974in}{2.044931in}}%
\pgfpathlineto{\pgfqpoint{5.741745in}{2.041578in}}%
\pgfpathlineto{\pgfqpoint{5.744310in}{2.044952in}}%
\pgfpathlineto{\pgfqpoint{5.744310in}{0.413320in}}%
\pgfpathlineto{\pgfqpoint{5.744310in}{0.413320in}}%
\pgfpathlineto{\pgfqpoint{5.741745in}{0.413320in}}%
\pgfpathlineto{\pgfqpoint{5.738974in}{0.413320in}}%
\pgfpathlineto{\pgfqpoint{5.736276in}{0.413320in}}%
\pgfpathlineto{\pgfqpoint{5.733594in}{0.413320in}}%
\pgfpathlineto{\pgfqpoint{5.730919in}{0.413320in}}%
\pgfpathlineto{\pgfqpoint{5.728339in}{0.413320in}}%
\pgfpathlineto{\pgfqpoint{5.725548in}{0.413320in}}%
\pgfpathlineto{\pgfqpoint{5.722950in}{0.413320in}}%
\pgfpathlineto{\pgfqpoint{5.720201in}{0.413320in}}%
\pgfpathlineto{\pgfqpoint{5.717671in}{0.413320in}}%
\pgfpathlineto{\pgfqpoint{5.714834in}{0.413320in}}%
\pgfpathlineto{\pgfqpoint{5.712291in}{0.413320in}}%
\pgfpathlineto{\pgfqpoint{5.709490in}{0.413320in}}%
\pgfpathlineto{\pgfqpoint{5.706800in}{0.413320in}}%
\pgfpathlineto{\pgfqpoint{5.704130in}{0.413320in}}%
\pgfpathlineto{\pgfqpoint{5.701453in}{0.413320in}}%
\pgfpathlineto{\pgfqpoint{5.698775in}{0.413320in}}%
\pgfpathlineto{\pgfqpoint{5.696101in}{0.413320in}}%
\pgfpathlineto{\pgfqpoint{5.693473in}{0.413320in}}%
\pgfpathlineto{\pgfqpoint{5.690730in}{0.413320in}}%
\pgfpathlineto{\pgfqpoint{5.688159in}{0.413320in}}%
\pgfpathlineto{\pgfqpoint{5.685385in}{0.413320in}}%
\pgfpathlineto{\pgfqpoint{5.682836in}{0.413320in}}%
\pgfpathlineto{\pgfqpoint{5.680027in}{0.413320in}}%
\pgfpathlineto{\pgfqpoint{5.677486in}{0.413320in}}%
\pgfpathlineto{\pgfqpoint{5.674667in}{0.413320in}}%
\pgfpathlineto{\pgfqpoint{5.671991in}{0.413320in}}%
\pgfpathlineto{\pgfqpoint{5.669313in}{0.413320in}}%
\pgfpathlineto{\pgfqpoint{5.666632in}{0.413320in}}%
\pgfpathlineto{\pgfqpoint{5.664099in}{0.413320in}}%
\pgfpathlineto{\pgfqpoint{5.661273in}{0.413320in}}%
\pgfpathlineto{\pgfqpoint{5.658723in}{0.413320in}}%
\pgfpathlineto{\pgfqpoint{5.655919in}{0.413320in}}%
\pgfpathlineto{\pgfqpoint{5.653376in}{0.413320in}}%
\pgfpathlineto{\pgfqpoint{5.650563in}{0.413320in}}%
\pgfpathlineto{\pgfqpoint{5.648008in}{0.413320in}}%
\pgfpathlineto{\pgfqpoint{5.645243in}{0.413320in}}%
\pgfpathlineto{\pgfqpoint{5.642518in}{0.413320in}}%
\pgfpathlineto{\pgfqpoint{5.639852in}{0.413320in}}%
\pgfpathlineto{\pgfqpoint{5.637172in}{0.413320in}}%
\pgfpathlineto{\pgfqpoint{5.634496in}{0.413320in}}%
\pgfpathlineto{\pgfqpoint{5.631815in}{0.413320in}}%
\pgfpathlineto{\pgfqpoint{5.629232in}{0.413320in}}%
\pgfpathlineto{\pgfqpoint{5.626460in}{0.413320in}}%
\pgfpathlineto{\pgfqpoint{5.623868in}{0.413320in}}%
\pgfpathlineto{\pgfqpoint{5.621102in}{0.413320in}}%
\pgfpathlineto{\pgfqpoint{5.618526in}{0.413320in}}%
\pgfpathlineto{\pgfqpoint{5.615743in}{0.413320in}}%
\pgfpathlineto{\pgfqpoint{5.613235in}{0.413320in}}%
\pgfpathlineto{\pgfqpoint{5.610389in}{0.413320in}}%
\pgfpathlineto{\pgfqpoint{5.607698in}{0.413320in}}%
\pgfpathlineto{\pgfqpoint{5.605073in}{0.413320in}}%
\pgfpathlineto{\pgfqpoint{5.602352in}{0.413320in}}%
\pgfpathlineto{\pgfqpoint{5.599674in}{0.413320in}}%
\pgfpathlineto{\pgfqpoint{5.596999in}{0.413320in}}%
\pgfpathlineto{\pgfqpoint{5.594368in}{0.413320in}}%
\pgfpathlineto{\pgfqpoint{5.591641in}{0.413320in}}%
\pgfpathlineto{\pgfqpoint{5.589040in}{0.413320in}}%
\pgfpathlineto{\pgfqpoint{5.586269in}{0.413320in}}%
\pgfpathlineto{\pgfqpoint{5.583709in}{0.413320in}}%
\pgfpathlineto{\pgfqpoint{5.580914in}{0.413320in}}%
\pgfpathlineto{\pgfqpoint{5.578342in}{0.413320in}}%
\pgfpathlineto{\pgfqpoint{5.575596in}{0.413320in}}%
\pgfpathlineto{\pgfqpoint{5.572893in}{0.413320in}}%
\pgfpathlineto{\pgfqpoint{5.570215in}{0.413320in}}%
\pgfpathlineto{\pgfqpoint{5.567536in}{0.413320in}}%
\pgfpathlineto{\pgfqpoint{5.564940in}{0.413320in}}%
\pgfpathlineto{\pgfqpoint{5.562180in}{0.413320in}}%
\pgfpathlineto{\pgfqpoint{5.559612in}{0.413320in}}%
\pgfpathlineto{\pgfqpoint{5.556822in}{0.413320in}}%
\pgfpathlineto{\pgfqpoint{5.554198in}{0.413320in}}%
\pgfpathlineto{\pgfqpoint{5.551457in}{0.413320in}}%
\pgfpathlineto{\pgfqpoint{5.548921in}{0.413320in}}%
\pgfpathlineto{\pgfqpoint{5.546110in}{0.413320in}}%
\pgfpathlineto{\pgfqpoint{5.543421in}{0.413320in}}%
\pgfpathlineto{\pgfqpoint{5.540750in}{0.413320in}}%
\pgfpathlineto{\pgfqpoint{5.538074in}{0.413320in}}%
\pgfpathlineto{\pgfqpoint{5.535395in}{0.413320in}}%
\pgfpathlineto{\pgfqpoint{5.532717in}{0.413320in}}%
\pgfpathlineto{\pgfqpoint{5.530148in}{0.413320in}}%
\pgfpathlineto{\pgfqpoint{5.527360in}{0.413320in}}%
\pgfpathlineto{\pgfqpoint{5.524756in}{0.413320in}}%
\pgfpathlineto{\pgfqpoint{5.522003in}{0.413320in}}%
\pgfpathlineto{\pgfqpoint{5.519433in}{0.413320in}}%
\pgfpathlineto{\pgfqpoint{5.516646in}{0.413320in}}%
\pgfpathlineto{\pgfqpoint{5.514080in}{0.413320in}}%
\pgfpathlineto{\pgfqpoint{5.511290in}{0.413320in}}%
\pgfpathlineto{\pgfqpoint{5.508612in}{0.413320in}}%
\pgfpathlineto{\pgfqpoint{5.505933in}{0.413320in}}%
\pgfpathlineto{\pgfqpoint{5.503255in}{0.413320in}}%
\pgfpathlineto{\pgfqpoint{5.500687in}{0.413320in}}%
\pgfpathlineto{\pgfqpoint{5.497898in}{0.413320in}}%
\pgfpathlineto{\pgfqpoint{5.495346in}{0.413320in}}%
\pgfpathlineto{\pgfqpoint{5.492541in}{0.413320in}}%
\pgfpathlineto{\pgfqpoint{5.490000in}{0.413320in}}%
\pgfpathlineto{\pgfqpoint{5.487176in}{0.413320in}}%
\pgfpathlineto{\pgfqpoint{5.484641in}{0.413320in}}%
\pgfpathlineto{\pgfqpoint{5.481825in}{0.413320in}}%
\pgfpathlineto{\pgfqpoint{5.479152in}{0.413320in}}%
\pgfpathlineto{\pgfqpoint{5.476458in}{0.413320in}}%
\pgfpathlineto{\pgfqpoint{5.473792in}{0.413320in}}%
\pgfpathlineto{\pgfqpoint{5.471113in}{0.413320in}}%
\pgfpathlineto{\pgfqpoint{5.468425in}{0.413320in}}%
\pgfpathlineto{\pgfqpoint{5.465888in}{0.413320in}}%
\pgfpathlineto{\pgfqpoint{5.463079in}{0.413320in}}%
\pgfpathlineto{\pgfqpoint{5.460489in}{0.413320in}}%
\pgfpathlineto{\pgfqpoint{5.457721in}{0.413320in}}%
\pgfpathlineto{\pgfqpoint{5.455168in}{0.413320in}}%
\pgfpathlineto{\pgfqpoint{5.452365in}{0.413320in}}%
\pgfpathlineto{\pgfqpoint{5.449769in}{0.413320in}}%
\pgfpathlineto{\pgfqpoint{5.447021in}{0.413320in}}%
\pgfpathlineto{\pgfqpoint{5.444328in}{0.413320in}}%
\pgfpathlineto{\pgfqpoint{5.441698in}{0.413320in}}%
\pgfpathlineto{\pgfqpoint{5.438974in}{0.413320in}}%
\pgfpathlineto{\pgfqpoint{5.436295in}{0.413320in}}%
\pgfpathlineto{\pgfqpoint{5.433616in}{0.413320in}}%
\pgfpathlineto{\pgfqpoint{5.431015in}{0.413320in}}%
\pgfpathlineto{\pgfqpoint{5.428259in}{0.413320in}}%
\pgfpathlineto{\pgfqpoint{5.425661in}{0.413320in}}%
\pgfpathlineto{\pgfqpoint{5.422897in}{0.413320in}}%
\pgfpathlineto{\pgfqpoint{5.420304in}{0.413320in}}%
\pgfpathlineto{\pgfqpoint{5.417547in}{0.413320in}}%
\pgfpathlineto{\pgfqpoint{5.414954in}{0.413320in}}%
\pgfpathlineto{\pgfqpoint{5.412190in}{0.413320in}}%
\pgfpathlineto{\pgfqpoint{5.409507in}{0.413320in}}%
\pgfpathlineto{\pgfqpoint{5.406832in}{0.413320in}}%
\pgfpathlineto{\pgfqpoint{5.404154in}{0.413320in}}%
\pgfpathlineto{\pgfqpoint{5.401576in}{0.413320in}}%
\pgfpathlineto{\pgfqpoint{5.398784in}{0.413320in}}%
\pgfpathlineto{\pgfqpoint{5.396219in}{0.413320in}}%
\pgfpathlineto{\pgfqpoint{5.393441in}{0.413320in}}%
\pgfpathlineto{\pgfqpoint{5.390900in}{0.413320in}}%
\pgfpathlineto{\pgfqpoint{5.388083in}{0.413320in}}%
\pgfpathlineto{\pgfqpoint{5.385550in}{0.413320in}}%
\pgfpathlineto{\pgfqpoint{5.382725in}{0.413320in}}%
\pgfpathlineto{\pgfqpoint{5.380048in}{0.413320in}}%
\pgfpathlineto{\pgfqpoint{5.377370in}{0.413320in}}%
\pgfpathlineto{\pgfqpoint{5.374692in}{0.413320in}}%
\pgfpathlineto{\pgfqpoint{5.372013in}{0.413320in}}%
\pgfpathlineto{\pgfqpoint{5.369335in}{0.413320in}}%
\pgfpathlineto{\pgfqpoint{5.366727in}{0.413320in}}%
\pgfpathlineto{\pgfqpoint{5.363966in}{0.413320in}}%
\pgfpathlineto{\pgfqpoint{5.361370in}{0.413320in}}%
\pgfpathlineto{\pgfqpoint{5.358612in}{0.413320in}}%
\pgfpathlineto{\pgfqpoint{5.356056in}{0.413320in}}%
\pgfpathlineto{\pgfqpoint{5.353262in}{0.413320in}}%
\pgfpathlineto{\pgfqpoint{5.350723in}{0.413320in}}%
\pgfpathlineto{\pgfqpoint{5.347905in}{0.413320in}}%
\pgfpathlineto{\pgfqpoint{5.345224in}{0.413320in}}%
\pgfpathlineto{\pgfqpoint{5.342549in}{0.413320in}}%
\pgfpathlineto{\pgfqpoint{5.339872in}{0.413320in}}%
\pgfpathlineto{\pgfqpoint{5.337353in}{0.413320in}}%
\pgfpathlineto{\pgfqpoint{5.334510in}{0.413320in}}%
\pgfpathlineto{\pgfqpoint{5.331973in}{0.413320in}}%
\pgfpathlineto{\pgfqpoint{5.329159in}{0.413320in}}%
\pgfpathlineto{\pgfqpoint{5.326564in}{0.413320in}}%
\pgfpathlineto{\pgfqpoint{5.323802in}{0.413320in}}%
\pgfpathlineto{\pgfqpoint{5.321256in}{0.413320in}}%
\pgfpathlineto{\pgfqpoint{5.318430in}{0.413320in}}%
\pgfpathlineto{\pgfqpoint{5.315754in}{0.413320in}}%
\pgfpathlineto{\pgfqpoint{5.313089in}{0.413320in}}%
\pgfpathlineto{\pgfqpoint{5.310411in}{0.413320in}}%
\pgfpathlineto{\pgfqpoint{5.307731in}{0.413320in}}%
\pgfpathlineto{\pgfqpoint{5.305054in}{0.413320in}}%
\pgfpathlineto{\pgfqpoint{5.302443in}{0.413320in}}%
\pgfpathlineto{\pgfqpoint{5.299696in}{0.413320in}}%
\pgfpathlineto{\pgfqpoint{5.297140in}{0.413320in}}%
\pgfpathlineto{\pgfqpoint{5.294339in}{0.413320in}}%
\pgfpathlineto{\pgfqpoint{5.291794in}{0.413320in}}%
\pgfpathlineto{\pgfqpoint{5.288984in}{0.413320in}}%
\pgfpathlineto{\pgfqpoint{5.286436in}{0.413320in}}%
\pgfpathlineto{\pgfqpoint{5.283631in}{0.413320in}}%
\pgfpathlineto{\pgfqpoint{5.280947in}{0.413320in}}%
\pgfpathlineto{\pgfqpoint{5.278322in}{0.413320in}}%
\pgfpathlineto{\pgfqpoint{5.275589in}{0.413320in}}%
\pgfpathlineto{\pgfqpoint{5.272913in}{0.413320in}}%
\pgfpathlineto{\pgfqpoint{5.270238in}{0.413320in}}%
\pgfpathlineto{\pgfqpoint{5.267691in}{0.413320in}}%
\pgfpathlineto{\pgfqpoint{5.264876in}{0.413320in}}%
\pgfpathlineto{\pgfqpoint{5.262264in}{0.413320in}}%
\pgfpathlineto{\pgfqpoint{5.259511in}{0.413320in}}%
\pgfpathlineto{\pgfqpoint{5.256973in}{0.413320in}}%
\pgfpathlineto{\pgfqpoint{5.254236in}{0.413320in}}%
\pgfpathlineto{\pgfqpoint{5.251590in}{0.413320in}}%
\pgfpathlineto{\pgfqpoint{5.248816in}{0.413320in}}%
\pgfpathlineto{\pgfqpoint{5.246130in}{0.413320in}}%
\pgfpathlineto{\pgfqpoint{5.243445in}{0.413320in}}%
\pgfpathlineto{\pgfqpoint{5.240777in}{0.413320in}}%
\pgfpathlineto{\pgfqpoint{5.238173in}{0.413320in}}%
\pgfpathlineto{\pgfqpoint{5.235409in}{0.413320in}}%
\pgfpathlineto{\pgfqpoint{5.232855in}{0.413320in}}%
\pgfpathlineto{\pgfqpoint{5.230059in}{0.413320in}}%
\pgfpathlineto{\pgfqpoint{5.227470in}{0.413320in}}%
\pgfpathlineto{\pgfqpoint{5.224695in}{0.413320in}}%
\pgfpathlineto{\pgfqpoint{5.222151in}{0.413320in}}%
\pgfpathlineto{\pgfqpoint{5.219345in}{0.413320in}}%
\pgfpathlineto{\pgfqpoint{5.216667in}{0.413320in}}%
\pgfpathlineto{\pgfqpoint{5.214027in}{0.413320in}}%
\pgfpathlineto{\pgfqpoint{5.211299in}{0.413320in}}%
\pgfpathlineto{\pgfqpoint{5.208630in}{0.413320in}}%
\pgfpathlineto{\pgfqpoint{5.205952in}{0.413320in}}%
\pgfpathlineto{\pgfqpoint{5.203388in}{0.413320in}}%
\pgfpathlineto{\pgfqpoint{5.200594in}{0.413320in}}%
\pgfpathlineto{\pgfqpoint{5.198008in}{0.413320in}}%
\pgfpathlineto{\pgfqpoint{5.195239in}{0.413320in}}%
\pgfpathlineto{\pgfqpoint{5.192680in}{0.413320in}}%
\pgfpathlineto{\pgfqpoint{5.189880in}{0.413320in}}%
\pgfpathlineto{\pgfqpoint{5.187294in}{0.413320in}}%
\pgfpathlineto{\pgfqpoint{5.184522in}{0.413320in}}%
\pgfpathlineto{\pgfqpoint{5.181848in}{0.413320in}}%
\pgfpathlineto{\pgfqpoint{5.179188in}{0.413320in}}%
\pgfpathlineto{\pgfqpoint{5.176477in}{0.413320in}}%
\pgfpathlineto{\pgfqpoint{5.173925in}{0.413320in}}%
\pgfpathlineto{\pgfqpoint{5.171133in}{0.413320in}}%
\pgfpathlineto{\pgfqpoint{5.168591in}{0.413320in}}%
\pgfpathlineto{\pgfqpoint{5.165775in}{0.413320in}}%
\pgfpathlineto{\pgfqpoint{5.163243in}{0.413320in}}%
\pgfpathlineto{\pgfqpoint{5.160420in}{0.413320in}}%
\pgfpathlineto{\pgfqpoint{5.157815in}{0.413320in}}%
\pgfpathlineto{\pgfqpoint{5.155059in}{0.413320in}}%
\pgfpathlineto{\pgfqpoint{5.152382in}{0.413320in}}%
\pgfpathlineto{\pgfqpoint{5.149734in}{0.413320in}}%
\pgfpathlineto{\pgfqpoint{5.147029in}{0.413320in}}%
\pgfpathlineto{\pgfqpoint{5.144349in}{0.413320in}}%
\pgfpathlineto{\pgfqpoint{5.141660in}{0.413320in}}%
\pgfpathlineto{\pgfqpoint{5.139072in}{0.413320in}}%
\pgfpathlineto{\pgfqpoint{5.136311in}{0.413320in}}%
\pgfpathlineto{\pgfqpoint{5.133716in}{0.413320in}}%
\pgfpathlineto{\pgfqpoint{5.130953in}{0.413320in}}%
\pgfpathlineto{\pgfqpoint{5.128421in}{0.413320in}}%
\pgfpathlineto{\pgfqpoint{5.125599in}{0.413320in}}%
\pgfpathlineto{\pgfqpoint{5.123042in}{0.413320in}}%
\pgfpathlineto{\pgfqpoint{5.120243in}{0.413320in}}%
\pgfpathlineto{\pgfqpoint{5.117550in}{0.413320in}}%
\pgfpathlineto{\pgfqpoint{5.114887in}{0.413320in}}%
\pgfpathlineto{\pgfqpoint{5.112209in}{0.413320in}}%
\pgfpathlineto{\pgfqpoint{5.109530in}{0.413320in}}%
\pgfpathlineto{\pgfqpoint{5.106842in}{0.413320in}}%
\pgfpathlineto{\pgfqpoint{5.104312in}{0.413320in}}%
\pgfpathlineto{\pgfqpoint{5.101496in}{0.413320in}}%
\pgfpathlineto{\pgfqpoint{5.098948in}{0.413320in}}%
\pgfpathlineto{\pgfqpoint{5.096142in}{0.413320in}}%
\pgfpathlineto{\pgfqpoint{5.093579in}{0.413320in}}%
\pgfpathlineto{\pgfqpoint{5.090788in}{0.413320in}}%
\pgfpathlineto{\pgfqpoint{5.088103in}{0.413320in}}%
\pgfpathlineto{\pgfqpoint{5.085426in}{0.413320in}}%
\pgfpathlineto{\pgfqpoint{5.082746in}{0.413320in}}%
\pgfpathlineto{\pgfqpoint{5.080067in}{0.413320in}}%
\pgfpathlineto{\pgfqpoint{5.077390in}{0.413320in}}%
\pgfpathlineto{\pgfqpoint{5.074851in}{0.413320in}}%
\pgfpathlineto{\pgfqpoint{5.072030in}{0.413320in}}%
\pgfpathlineto{\pgfqpoint{5.069463in}{0.413320in}}%
\pgfpathlineto{\pgfqpoint{5.066677in}{0.413320in}}%
\pgfpathlineto{\pgfqpoint{5.064144in}{0.413320in}}%
\pgfpathlineto{\pgfqpoint{5.061315in}{0.413320in}}%
\pgfpathlineto{\pgfqpoint{5.058711in}{0.413320in}}%
\pgfpathlineto{\pgfqpoint{5.055952in}{0.413320in}}%
\pgfpathlineto{\pgfqpoint{5.053284in}{0.413320in}}%
\pgfpathlineto{\pgfqpoint{5.050606in}{0.413320in}}%
\pgfpathlineto{\pgfqpoint{5.047924in}{0.413320in}}%
\pgfpathlineto{\pgfqpoint{5.045249in}{0.413320in}}%
\pgfpathlineto{\pgfqpoint{5.042572in}{0.413320in}}%
\pgfpathlineto{\pgfqpoint{5.039962in}{0.413320in}}%
\pgfpathlineto{\pgfqpoint{5.037214in}{0.413320in}}%
\pgfpathlineto{\pgfqpoint{5.034649in}{0.413320in}}%
\pgfpathlineto{\pgfqpoint{5.031849in}{0.413320in}}%
\pgfpathlineto{\pgfqpoint{5.029275in}{0.413320in}}%
\pgfpathlineto{\pgfqpoint{5.026501in}{0.413320in}}%
\pgfpathlineto{\pgfqpoint{5.023927in}{0.413320in}}%
\pgfpathlineto{\pgfqpoint{5.021147in}{0.413320in}}%
\pgfpathlineto{\pgfqpoint{5.018466in}{0.413320in}}%
\pgfpathlineto{\pgfqpoint{5.015820in}{0.413320in}}%
\pgfpathlineto{\pgfqpoint{5.013104in}{0.413320in}}%
\pgfpathlineto{\pgfqpoint{5.010562in}{0.413320in}}%
\pgfpathlineto{\pgfqpoint{5.007751in}{0.413320in}}%
\pgfpathlineto{\pgfqpoint{5.005178in}{0.413320in}}%
\pgfpathlineto{\pgfqpoint{5.002384in}{0.413320in}}%
\pgfpathlineto{\pgfqpoint{4.999780in}{0.413320in}}%
\pgfpathlineto{\pgfqpoint{4.997028in}{0.413320in}}%
\pgfpathlineto{\pgfqpoint{4.994390in}{0.413320in}}%
\pgfpathlineto{\pgfqpoint{4.991683in}{0.413320in}}%
\pgfpathlineto{\pgfqpoint{4.989001in}{0.413320in}}%
\pgfpathlineto{\pgfqpoint{4.986325in}{0.413320in}}%
\pgfpathlineto{\pgfqpoint{4.983637in}{0.413320in}}%
\pgfpathlineto{\pgfqpoint{4.980967in}{0.413320in}}%
\pgfpathlineto{\pgfqpoint{4.978287in}{0.413320in}}%
\pgfpathlineto{\pgfqpoint{4.975703in}{0.413320in}}%
\pgfpathlineto{\pgfqpoint{4.972933in}{0.413320in}}%
\pgfpathlineto{\pgfqpoint{4.970314in}{0.413320in}}%
\pgfpathlineto{\pgfqpoint{4.967575in}{0.413320in}}%
\pgfpathlineto{\pgfqpoint{4.965002in}{0.413320in}}%
\pgfpathlineto{\pgfqpoint{4.962219in}{0.413320in}}%
\pgfpathlineto{\pgfqpoint{4.959689in}{0.413320in}}%
\pgfpathlineto{\pgfqpoint{4.956862in}{0.413320in}}%
\pgfpathlineto{\pgfqpoint{4.954182in}{0.413320in}}%
\pgfpathlineto{\pgfqpoint{4.951504in}{0.413320in}}%
\pgfpathlineto{\pgfqpoint{4.948827in}{0.413320in}}%
\pgfpathlineto{\pgfqpoint{4.946151in}{0.413320in}}%
\pgfpathlineto{\pgfqpoint{4.943466in}{0.413320in}}%
\pgfpathlineto{\pgfqpoint{4.940881in}{0.413320in}}%
\pgfpathlineto{\pgfqpoint{4.938112in}{0.413320in}}%
\pgfpathlineto{\pgfqpoint{4.935515in}{0.413320in}}%
\pgfpathlineto{\pgfqpoint{4.932742in}{0.413320in}}%
\pgfpathlineto{\pgfqpoint{4.930170in}{0.413320in}}%
\pgfpathlineto{\pgfqpoint{4.927400in}{0.413320in}}%
\pgfpathlineto{\pgfqpoint{4.924708in}{0.413320in}}%
\pgfpathlineto{\pgfqpoint{4.922041in}{0.413320in}}%
\pgfpathlineto{\pgfqpoint{4.919352in}{0.413320in}}%
\pgfpathlineto{\pgfqpoint{4.916681in}{0.413320in}}%
\pgfpathlineto{\pgfqpoint{4.914009in}{0.413320in}}%
\pgfpathlineto{\pgfqpoint{4.911435in}{0.413320in}}%
\pgfpathlineto{\pgfqpoint{4.908648in}{0.413320in}}%
\pgfpathlineto{\pgfqpoint{4.906096in}{0.413320in}}%
\pgfpathlineto{\pgfqpoint{4.903295in}{0.413320in}}%
\pgfpathlineto{\pgfqpoint{4.900712in}{0.413320in}}%
\pgfpathlineto{\pgfqpoint{4.897938in}{0.413320in}}%
\pgfpathlineto{\pgfqpoint{4.895399in}{0.413320in}}%
\pgfpathlineto{\pgfqpoint{4.892611in}{0.413320in}}%
\pgfpathlineto{\pgfqpoint{4.889902in}{0.413320in}}%
\pgfpathlineto{\pgfqpoint{4.887211in}{0.413320in}}%
\pgfpathlineto{\pgfqpoint{4.884540in}{0.413320in}}%
\pgfpathlineto{\pgfqpoint{4.881864in}{0.413320in}}%
\pgfpathlineto{\pgfqpoint{4.879180in}{0.413320in}}%
\pgfpathlineto{\pgfqpoint{4.876636in}{0.413320in}}%
\pgfpathlineto{\pgfqpoint{4.873832in}{0.413320in}}%
\pgfpathlineto{\pgfqpoint{4.871209in}{0.413320in}}%
\pgfpathlineto{\pgfqpoint{4.868474in}{0.413320in}}%
\pgfpathlineto{\pgfqpoint{4.865910in}{0.413320in}}%
\pgfpathlineto{\pgfqpoint{4.863116in}{0.413320in}}%
\pgfpathlineto{\pgfqpoint{4.860544in}{0.413320in}}%
\pgfpathlineto{\pgfqpoint{4.857807in}{0.413320in}}%
\pgfpathlineto{\pgfqpoint{4.855070in}{0.413320in}}%
\pgfpathlineto{\pgfqpoint{4.852404in}{0.413320in}}%
\pgfpathlineto{\pgfqpoint{4.849715in}{0.413320in}}%
\pgfpathlineto{\pgfqpoint{4.847127in}{0.413320in}}%
\pgfpathlineto{\pgfqpoint{4.844361in}{0.413320in}}%
\pgfpathlineto{\pgfqpoint{4.842380in}{0.413320in}}%
\pgfpathlineto{\pgfqpoint{4.839922in}{0.413320in}}%
\pgfpathlineto{\pgfqpoint{4.837992in}{0.413320in}}%
\pgfpathlineto{\pgfqpoint{4.833657in}{0.413320in}}%
\pgfpathlineto{\pgfqpoint{4.831045in}{0.413320in}}%
\pgfpathlineto{\pgfqpoint{4.828291in}{0.413320in}}%
\pgfpathlineto{\pgfqpoint{4.825619in}{0.413320in}}%
\pgfpathlineto{\pgfqpoint{4.822945in}{0.413320in}}%
\pgfpathlineto{\pgfqpoint{4.820265in}{0.413320in}}%
\pgfpathlineto{\pgfqpoint{4.817587in}{0.413320in}}%
\pgfpathlineto{\pgfqpoint{4.814907in}{0.413320in}}%
\pgfpathlineto{\pgfqpoint{4.812377in}{0.413320in}}%
\pgfpathlineto{\pgfqpoint{4.809538in}{0.413320in}}%
\pgfpathlineto{\pgfqpoint{4.807017in}{0.413320in}}%
\pgfpathlineto{\pgfqpoint{4.804193in}{0.413320in}}%
\pgfpathlineto{\pgfqpoint{4.801586in}{0.413320in}}%
\pgfpathlineto{\pgfqpoint{4.798830in}{0.413320in}}%
\pgfpathlineto{\pgfqpoint{4.796274in}{0.413320in}}%
\pgfpathlineto{\pgfqpoint{4.793512in}{0.413320in}}%
\pgfpathlineto{\pgfqpoint{4.790798in}{0.413320in}}%
\pgfpathlineto{\pgfqpoint{4.788116in}{0.413320in}}%
\pgfpathlineto{\pgfqpoint{4.785445in}{0.413320in}}%
\pgfpathlineto{\pgfqpoint{4.782872in}{0.413320in}}%
\pgfpathlineto{\pgfqpoint{4.780083in}{0.413320in}}%
\pgfpathlineto{\pgfqpoint{4.777535in}{0.413320in}}%
\pgfpathlineto{\pgfqpoint{4.774732in}{0.413320in}}%
\pgfpathlineto{\pgfqpoint{4.772198in}{0.413320in}}%
\pgfpathlineto{\pgfqpoint{4.769367in}{0.413320in}}%
\pgfpathlineto{\pgfqpoint{4.766783in}{0.413320in}}%
\pgfpathlineto{\pgfqpoint{4.764018in}{0.413320in}}%
\pgfpathlineto{\pgfqpoint{4.761337in}{0.413320in}}%
\pgfpathlineto{\pgfqpoint{4.758653in}{0.413320in}}%
\pgfpathlineto{\pgfqpoint{4.755983in}{0.413320in}}%
\pgfpathlineto{\pgfqpoint{4.753298in}{0.413320in}}%
\pgfpathlineto{\pgfqpoint{4.750627in}{0.413320in}}%
\pgfpathlineto{\pgfqpoint{4.748081in}{0.413320in}}%
\pgfpathlineto{\pgfqpoint{4.745256in}{0.413320in}}%
\pgfpathlineto{\pgfqpoint{4.742696in}{0.413320in}}%
\pgfpathlineto{\pgfqpoint{4.739912in}{0.413320in}}%
\pgfpathlineto{\pgfqpoint{4.737348in}{0.413320in}}%
\pgfpathlineto{\pgfqpoint{4.734552in}{0.413320in}}%
\pgfpathlineto{\pgfqpoint{4.731901in}{0.413320in}}%
\pgfpathlineto{\pgfqpoint{4.729233in}{0.413320in}}%
\pgfpathlineto{\pgfqpoint{4.726508in}{0.413320in}}%
\pgfpathlineto{\pgfqpoint{4.723873in}{0.413320in}}%
\pgfpathlineto{\pgfqpoint{4.721160in}{0.413320in}}%
\pgfpathlineto{\pgfqpoint{4.718486in}{0.413320in}}%
\pgfpathlineto{\pgfqpoint{4.715806in}{0.413320in}}%
\pgfpathlineto{\pgfqpoint{4.713275in}{0.413320in}}%
\pgfpathlineto{\pgfqpoint{4.710437in}{0.413320in}}%
\pgfpathlineto{\pgfqpoint{4.707824in}{0.413320in}}%
\pgfpathlineto{\pgfqpoint{4.705094in}{0.413320in}}%
\pgfpathlineto{\pgfqpoint{4.702517in}{0.413320in}}%
\pgfpathlineto{\pgfqpoint{4.699734in}{0.413320in}}%
\pgfpathlineto{\pgfqpoint{4.697170in}{0.413320in}}%
\pgfpathlineto{\pgfqpoint{4.694381in}{0.413320in}}%
\pgfpathlineto{\pgfqpoint{4.691694in}{0.413320in}}%
\pgfpathlineto{\pgfqpoint{4.689051in}{0.413320in}}%
\pgfpathlineto{\pgfqpoint{4.686337in}{0.413320in}}%
\pgfpathlineto{\pgfqpoint{4.683799in}{0.413320in}}%
\pgfpathlineto{\pgfqpoint{4.680988in}{0.413320in}}%
\pgfpathlineto{\pgfqpoint{4.678448in}{0.413320in}}%
\pgfpathlineto{\pgfqpoint{4.675619in}{0.413320in}}%
\pgfpathlineto{\pgfqpoint{4.673068in}{0.413320in}}%
\pgfpathlineto{\pgfqpoint{4.670261in}{0.413320in}}%
\pgfpathlineto{\pgfqpoint{4.667764in}{0.413320in}}%
\pgfpathlineto{\pgfqpoint{4.664923in}{0.413320in}}%
\pgfpathlineto{\pgfqpoint{4.662237in}{0.413320in}}%
\pgfpathlineto{\pgfqpoint{4.659590in}{0.413320in}}%
\pgfpathlineto{\pgfqpoint{4.656873in}{0.413320in}}%
\pgfpathlineto{\pgfqpoint{4.654203in}{0.413320in}}%
\pgfpathlineto{\pgfqpoint{4.651524in}{0.413320in}}%
\pgfpathlineto{\pgfqpoint{4.648922in}{0.413320in}}%
\pgfpathlineto{\pgfqpoint{4.646169in}{0.413320in}}%
\pgfpathlineto{\pgfqpoint{4.643628in}{0.413320in}}%
\pgfpathlineto{\pgfqpoint{4.640809in}{0.413320in}}%
\pgfpathlineto{\pgfqpoint{4.638204in}{0.413320in}}%
\pgfpathlineto{\pgfqpoint{4.635445in}{0.413320in}}%
\pgfpathlineto{\pgfqpoint{4.632902in}{0.413320in}}%
\pgfpathlineto{\pgfqpoint{4.630096in}{0.413320in}}%
\pgfpathlineto{\pgfqpoint{4.627411in}{0.413320in}}%
\pgfpathlineto{\pgfqpoint{4.624741in}{0.413320in}}%
\pgfpathlineto{\pgfqpoint{4.622056in}{0.413320in}}%
\pgfpathlineto{\pgfqpoint{4.619529in}{0.413320in}}%
\pgfpathlineto{\pgfqpoint{4.616702in}{0.413320in}}%
\pgfpathlineto{\pgfqpoint{4.614134in}{0.413320in}}%
\pgfpathlineto{\pgfqpoint{4.611350in}{0.413320in}}%
\pgfpathlineto{\pgfqpoint{4.608808in}{0.413320in}}%
\pgfpathlineto{\pgfqpoint{4.605990in}{0.413320in}}%
\pgfpathlineto{\pgfqpoint{4.603430in}{0.413320in}}%
\pgfpathlineto{\pgfqpoint{4.600633in}{0.413320in}}%
\pgfpathlineto{\pgfqpoint{4.597951in}{0.413320in}}%
\pgfpathlineto{\pgfqpoint{4.595281in}{0.413320in}}%
\pgfpathlineto{\pgfqpoint{4.592589in}{0.413320in}}%
\pgfpathlineto{\pgfqpoint{4.589920in}{0.413320in}}%
\pgfpathlineto{\pgfqpoint{4.587244in}{0.413320in}}%
\pgfpathlineto{\pgfqpoint{4.584672in}{0.413320in}}%
\pgfpathlineto{\pgfqpoint{4.581888in}{0.413320in}}%
\pgfpathlineto{\pgfqpoint{4.579305in}{0.413320in}}%
\pgfpathlineto{\pgfqpoint{4.576531in}{0.413320in}}%
\pgfpathlineto{\pgfqpoint{4.573947in}{0.413320in}}%
\pgfpathlineto{\pgfqpoint{4.571171in}{0.413320in}}%
\pgfpathlineto{\pgfqpoint{4.568612in}{0.413320in}}%
\pgfpathlineto{\pgfqpoint{4.565820in}{0.413320in}}%
\pgfpathlineto{\pgfqpoint{4.563125in}{0.413320in}}%
\pgfpathlineto{\pgfqpoint{4.560448in}{0.413320in}}%
\pgfpathlineto{\pgfqpoint{4.557777in}{0.413320in}}%
\pgfpathlineto{\pgfqpoint{4.555106in}{0.413320in}}%
\pgfpathlineto{\pgfqpoint{4.552425in}{0.413320in}}%
\pgfpathlineto{\pgfqpoint{4.549822in}{0.413320in}}%
\pgfpathlineto{\pgfqpoint{4.547064in}{0.413320in}}%
\pgfpathlineto{\pgfqpoint{4.544464in}{0.413320in}}%
\pgfpathlineto{\pgfqpoint{4.541711in}{0.413320in}}%
\pgfpathlineto{\pgfqpoint{4.539144in}{0.413320in}}%
\pgfpathlineto{\pgfqpoint{4.536400in}{0.413320in}}%
\pgfpathlineto{\pgfqpoint{4.533764in}{0.413320in}}%
\pgfpathlineto{\pgfqpoint{4.530990in}{0.413320in}}%
\pgfpathlineto{\pgfqpoint{4.528307in}{0.413320in}}%
\pgfpathlineto{\pgfqpoint{4.525640in}{0.413320in}}%
\pgfpathlineto{\pgfqpoint{4.522962in}{0.413320in}}%
\pgfpathlineto{\pgfqpoint{4.520345in}{0.413320in}}%
\pgfpathlineto{\pgfqpoint{4.517598in}{0.413320in}}%
\pgfpathlineto{\pgfqpoint{4.515080in}{0.413320in}}%
\pgfpathlineto{\pgfqpoint{4.512246in}{0.413320in}}%
\pgfpathlineto{\pgfqpoint{4.509643in}{0.413320in}}%
\pgfpathlineto{\pgfqpoint{4.506893in}{0.413320in}}%
\pgfpathlineto{\pgfqpoint{4.504305in}{0.413320in}}%
\pgfpathlineto{\pgfqpoint{4.501529in}{0.413320in}}%
\pgfpathlineto{\pgfqpoint{4.498850in}{0.413320in}}%
\pgfpathlineto{\pgfqpoint{4.496167in}{0.413320in}}%
\pgfpathlineto{\pgfqpoint{4.493492in}{0.413320in}}%
\pgfpathlineto{\pgfqpoint{4.490822in}{0.413320in}}%
\pgfpathlineto{\pgfqpoint{4.488130in}{0.413320in}}%
\pgfpathlineto{\pgfqpoint{4.485581in}{0.413320in}}%
\pgfpathlineto{\pgfqpoint{4.482778in}{0.413320in}}%
\pgfpathlineto{\pgfqpoint{4.480201in}{0.413320in}}%
\pgfpathlineto{\pgfqpoint{4.477430in}{0.413320in}}%
\pgfpathlineto{\pgfqpoint{4.474861in}{0.413320in}}%
\pgfpathlineto{\pgfqpoint{4.472059in}{0.413320in}}%
\pgfpathlineto{\pgfqpoint{4.469492in}{0.413320in}}%
\pgfpathlineto{\pgfqpoint{4.466717in}{0.413320in}}%
\pgfpathlineto{\pgfqpoint{4.464029in}{0.413320in}}%
\pgfpathlineto{\pgfqpoint{4.461367in}{0.413320in}}%
\pgfpathlineto{\pgfqpoint{4.458681in}{0.413320in}}%
\pgfpathlineto{\pgfqpoint{4.456138in}{0.413320in}}%
\pgfpathlineto{\pgfqpoint{4.453312in}{0.413320in}}%
\pgfpathlineto{\pgfqpoint{4.450767in}{0.413320in}}%
\pgfpathlineto{\pgfqpoint{4.447965in}{0.413320in}}%
\pgfpathlineto{\pgfqpoint{4.445423in}{0.413320in}}%
\pgfpathlineto{\pgfqpoint{4.442611in}{0.413320in}}%
\pgfpathlineto{\pgfqpoint{4.440041in}{0.413320in}}%
\pgfpathlineto{\pgfqpoint{4.437253in}{0.413320in}}%
\pgfpathlineto{\pgfqpoint{4.434569in}{0.413320in}}%
\pgfpathlineto{\pgfqpoint{4.431901in}{0.413320in}}%
\pgfpathlineto{\pgfqpoint{4.429220in}{0.413320in}}%
\pgfpathlineto{\pgfqpoint{4.426534in}{0.413320in}}%
\pgfpathlineto{\pgfqpoint{4.423863in}{0.413320in}}%
\pgfpathlineto{\pgfqpoint{4.421292in}{0.413320in}}%
\pgfpathlineto{\pgfqpoint{4.418506in}{0.413320in}}%
\pgfpathlineto{\pgfqpoint{4.415932in}{0.413320in}}%
\pgfpathlineto{\pgfqpoint{4.413149in}{0.413320in}}%
\pgfpathlineto{\pgfqpoint{4.410587in}{0.413320in}}%
\pgfpathlineto{\pgfqpoint{4.407788in}{0.413320in}}%
\pgfpathlineto{\pgfqpoint{4.405234in}{0.413320in}}%
\pgfpathlineto{\pgfqpoint{4.402468in}{0.413320in}}%
\pgfpathlineto{\pgfqpoint{4.399745in}{0.413320in}}%
\pgfpathlineto{\pgfqpoint{4.397076in}{0.413320in}}%
\pgfpathlineto{\pgfqpoint{4.394400in}{0.413320in}}%
\pgfpathlineto{\pgfqpoint{4.391721in}{0.413320in}}%
\pgfpathlineto{\pgfqpoint{4.389044in}{0.413320in}}%
\pgfpathlineto{\pgfqpoint{4.386431in}{0.413320in}}%
\pgfpathlineto{\pgfqpoint{4.383674in}{0.413320in}}%
\pgfpathlineto{\pgfqpoint{4.381097in}{0.413320in}}%
\pgfpathlineto{\pgfqpoint{4.378329in}{0.413320in}}%
\pgfpathlineto{\pgfqpoint{4.375761in}{0.413320in}}%
\pgfpathlineto{\pgfqpoint{4.372976in}{0.413320in}}%
\pgfpathlineto{\pgfqpoint{4.370437in}{0.413320in}}%
\pgfpathlineto{\pgfqpoint{4.367646in}{0.413320in}}%
\pgfpathlineto{\pgfqpoint{4.364936in}{0.413320in}}%
\pgfpathlineto{\pgfqpoint{4.362270in}{0.413320in}}%
\pgfpathlineto{\pgfqpoint{4.359582in}{0.413320in}}%
\pgfpathlineto{\pgfqpoint{4.357014in}{0.413320in}}%
\pgfpathlineto{\pgfqpoint{4.354224in}{0.413320in}}%
\pgfpathlineto{\pgfqpoint{4.351645in}{0.413320in}}%
\pgfpathlineto{\pgfqpoint{4.348868in}{0.413320in}}%
\pgfpathlineto{\pgfqpoint{4.346263in}{0.413320in}}%
\pgfpathlineto{\pgfqpoint{4.343510in}{0.413320in}}%
\pgfpathlineto{\pgfqpoint{4.340976in}{0.413320in}}%
\pgfpathlineto{\pgfqpoint{4.338154in}{0.413320in}}%
\pgfpathlineto{\pgfqpoint{4.335463in}{0.413320in}}%
\pgfpathlineto{\pgfqpoint{4.332796in}{0.413320in}}%
\pgfpathlineto{\pgfqpoint{4.330118in}{0.413320in}}%
\pgfpathlineto{\pgfqpoint{4.327440in}{0.413320in}}%
\pgfpathlineto{\pgfqpoint{4.324760in}{0.413320in}}%
\pgfpathlineto{\pgfqpoint{4.322181in}{0.413320in}}%
\pgfpathlineto{\pgfqpoint{4.319405in}{0.413320in}}%
\pgfpathlineto{\pgfqpoint{4.316856in}{0.413320in}}%
\pgfpathlineto{\pgfqpoint{4.314032in}{0.413320in}}%
\pgfpathlineto{\pgfqpoint{4.311494in}{0.413320in}}%
\pgfpathlineto{\pgfqpoint{4.308691in}{0.413320in}}%
\pgfpathlineto{\pgfqpoint{4.306118in}{0.413320in}}%
\pgfpathlineto{\pgfqpoint{4.303357in}{0.413320in}}%
\pgfpathlineto{\pgfqpoint{4.300656in}{0.413320in}}%
\pgfpathlineto{\pgfqpoint{4.297977in}{0.413320in}}%
\pgfpathlineto{\pgfqpoint{4.295299in}{0.413320in}}%
\pgfpathlineto{\pgfqpoint{4.292786in}{0.413320in}}%
\pgfpathlineto{\pgfqpoint{4.289936in}{0.413320in}}%
\pgfpathlineto{\pgfqpoint{4.287399in}{0.413320in}}%
\pgfpathlineto{\pgfqpoint{4.284586in}{0.413320in}}%
\pgfpathlineto{\pgfqpoint{4.282000in}{0.413320in}}%
\pgfpathlineto{\pgfqpoint{4.279212in}{0.413320in}}%
\pgfpathlineto{\pgfqpoint{4.276635in}{0.413320in}}%
\pgfpathlineto{\pgfqpoint{4.273874in}{0.413320in}}%
\pgfpathlineto{\pgfqpoint{4.271187in}{0.413320in}}%
\pgfpathlineto{\pgfqpoint{4.268590in}{0.413320in}}%
\pgfpathlineto{\pgfqpoint{4.265824in}{0.413320in}}%
\pgfpathlineto{\pgfqpoint{4.263157in}{0.413320in}}%
\pgfpathlineto{\pgfqpoint{4.260477in}{0.413320in}}%
\pgfpathlineto{\pgfqpoint{4.257958in}{0.413320in}}%
\pgfpathlineto{\pgfqpoint{4.255120in}{0.413320in}}%
\pgfpathlineto{\pgfqpoint{4.252581in}{0.413320in}}%
\pgfpathlineto{\pgfqpoint{4.249767in}{0.413320in}}%
\pgfpathlineto{\pgfqpoint{4.247225in}{0.413320in}}%
\pgfpathlineto{\pgfqpoint{4.244394in}{0.413320in}}%
\pgfpathlineto{\pgfqpoint{4.241900in}{0.413320in}}%
\pgfpathlineto{\pgfqpoint{4.239084in}{0.413320in}}%
\pgfpathlineto{\pgfqpoint{4.236375in}{0.413320in}}%
\pgfpathlineto{\pgfqpoint{4.233691in}{0.413320in}}%
\pgfpathlineto{\pgfqpoint{4.231013in}{0.413320in}}%
\pgfpathlineto{\pgfqpoint{4.228331in}{0.413320in}}%
\pgfpathlineto{\pgfqpoint{4.225654in}{0.413320in}}%
\pgfpathlineto{\pgfqpoint{4.223082in}{0.413320in}}%
\pgfpathlineto{\pgfqpoint{4.220304in}{0.413320in}}%
\pgfpathlineto{\pgfqpoint{4.217694in}{0.413320in}}%
\pgfpathlineto{\pgfqpoint{4.214948in}{0.413320in}}%
\pgfpathlineto{\pgfqpoint{4.212383in}{0.413320in}}%
\pgfpathlineto{\pgfqpoint{4.209597in}{0.413320in}}%
\pgfpathlineto{\pgfqpoint{4.207076in}{0.413320in}}%
\pgfpathlineto{\pgfqpoint{4.204240in}{0.413320in}}%
\pgfpathlineto{\pgfqpoint{4.201542in}{0.413320in}}%
\pgfpathlineto{\pgfqpoint{4.198878in}{0.413320in}}%
\pgfpathlineto{\pgfqpoint{4.196186in}{0.413320in}}%
\pgfpathlineto{\pgfqpoint{4.193638in}{0.413320in}}%
\pgfpathlineto{\pgfqpoint{4.190842in}{0.413320in}}%
\pgfpathlineto{\pgfqpoint{4.188318in}{0.413320in}}%
\pgfpathlineto{\pgfqpoint{4.185481in}{0.413320in}}%
\pgfpathlineto{\pgfqpoint{4.182899in}{0.413320in}}%
\pgfpathlineto{\pgfqpoint{4.180129in}{0.413320in}}%
\pgfpathlineto{\pgfqpoint{4.177593in}{0.413320in}}%
\pgfpathlineto{\pgfqpoint{4.174770in}{0.413320in}}%
\pgfpathlineto{\pgfqpoint{4.172093in}{0.413320in}}%
\pgfpathlineto{\pgfqpoint{4.169415in}{0.413320in}}%
\pgfpathlineto{\pgfqpoint{4.166737in}{0.413320in}}%
\pgfpathlineto{\pgfqpoint{4.164059in}{0.413320in}}%
\pgfpathlineto{\pgfqpoint{4.161380in}{0.413320in}}%
\pgfpathlineto{\pgfqpoint{4.158806in}{0.413320in}}%
\pgfpathlineto{\pgfqpoint{4.156016in}{0.413320in}}%
\pgfpathlineto{\pgfqpoint{4.153423in}{0.413320in}}%
\pgfpathlineto{\pgfqpoint{4.150665in}{0.413320in}}%
\pgfpathlineto{\pgfqpoint{4.148082in}{0.413320in}}%
\pgfpathlineto{\pgfqpoint{4.145310in}{0.413320in}}%
\pgfpathlineto{\pgfqpoint{4.142713in}{0.413320in}}%
\pgfpathlineto{\pgfqpoint{4.139963in}{0.413320in}}%
\pgfpathlineto{\pgfqpoint{4.137272in}{0.413320in}}%
\pgfpathlineto{\pgfqpoint{4.134615in}{0.413320in}}%
\pgfpathlineto{\pgfqpoint{4.131920in}{0.413320in}}%
\pgfpathlineto{\pgfqpoint{4.129349in}{0.413320in}}%
\pgfpathlineto{\pgfqpoint{4.126553in}{0.413320in}}%
\pgfpathlineto{\pgfqpoint{4.124019in}{0.413320in}}%
\pgfpathlineto{\pgfqpoint{4.121205in}{0.413320in}}%
\pgfpathlineto{\pgfqpoint{4.118554in}{0.413320in}}%
\pgfpathlineto{\pgfqpoint{4.115844in}{0.413320in}}%
\pgfpathlineto{\pgfqpoint{4.113252in}{0.413320in}}%
\pgfpathlineto{\pgfqpoint{4.110488in}{0.413320in}}%
\pgfpathlineto{\pgfqpoint{4.107814in}{0.413320in}}%
\pgfpathlineto{\pgfqpoint{4.105185in}{0.413320in}}%
\pgfpathlineto{\pgfqpoint{4.102456in}{0.413320in}}%
\pgfpathlineto{\pgfqpoint{4.099777in}{0.413320in}}%
\pgfpathlineto{\pgfqpoint{4.097092in}{0.413320in}}%
\pgfpathlineto{\pgfqpoint{4.094527in}{0.413320in}}%
\pgfpathlineto{\pgfqpoint{4.091729in}{0.413320in}}%
\pgfpathlineto{\pgfqpoint{4.089159in}{0.413320in}}%
\pgfpathlineto{\pgfqpoint{4.086385in}{0.413320in}}%
\pgfpathlineto{\pgfqpoint{4.083870in}{0.413320in}}%
\pgfpathlineto{\pgfqpoint{4.081018in}{0.413320in}}%
\pgfpathlineto{\pgfqpoint{4.078471in}{0.413320in}}%
\pgfpathlineto{\pgfqpoint{4.075705in}{0.413320in}}%
\pgfpathlineto{\pgfqpoint{4.072985in}{0.413320in}}%
\pgfpathlineto{\pgfqpoint{4.070313in}{0.413320in}}%
\pgfpathlineto{\pgfqpoint{4.067636in}{0.413320in}}%
\pgfpathlineto{\pgfqpoint{4.064957in}{0.413320in}}%
\pgfpathlineto{\pgfqpoint{4.062266in}{0.413320in}}%
\pgfpathlineto{\pgfqpoint{4.059702in}{0.413320in}}%
\pgfpathlineto{\pgfqpoint{4.056911in}{0.413320in}}%
\pgfpathlineto{\pgfqpoint{4.054326in}{0.413320in}}%
\pgfpathlineto{\pgfqpoint{4.051557in}{0.413320in}}%
\pgfpathlineto{\pgfqpoint{4.049006in}{0.413320in}}%
\pgfpathlineto{\pgfqpoint{4.046210in}{0.413320in}}%
\pgfpathlineto{\pgfqpoint{4.043667in}{0.413320in}}%
\pgfpathlineto{\pgfqpoint{4.040852in}{0.413320in}}%
\pgfpathlineto{\pgfqpoint{4.038174in}{0.413320in}}%
\pgfpathlineto{\pgfqpoint{4.035492in}{0.413320in}}%
\pgfpathlineto{\pgfqpoint{4.032817in}{0.413320in}}%
\pgfpathlineto{\pgfqpoint{4.030229in}{0.413320in}}%
\pgfpathlineto{\pgfqpoint{4.027447in}{0.413320in}}%
\pgfpathlineto{\pgfqpoint{4.024868in}{0.413320in}}%
\pgfpathlineto{\pgfqpoint{4.022097in}{0.413320in}}%
\pgfpathlineto{\pgfqpoint{4.019518in}{0.413320in}}%
\pgfpathlineto{\pgfqpoint{4.016744in}{0.413320in}}%
\pgfpathlineto{\pgfqpoint{4.014186in}{0.413320in}}%
\pgfpathlineto{\pgfqpoint{4.011394in}{0.413320in}}%
\pgfpathlineto{\pgfqpoint{4.008699in}{0.413320in}}%
\pgfpathlineto{\pgfqpoint{4.006034in}{0.413320in}}%
\pgfpathlineto{\pgfqpoint{4.003348in}{0.413320in}}%
\pgfpathlineto{\pgfqpoint{4.000674in}{0.413320in}}%
\pgfpathlineto{\pgfqpoint{3.997990in}{0.413320in}}%
\pgfpathlineto{\pgfqpoint{3.995417in}{0.413320in}}%
\pgfpathlineto{\pgfqpoint{3.992642in}{0.413320in}}%
\pgfpathlineto{\pgfqpoint{3.990055in}{0.413320in}}%
\pgfpathlineto{\pgfqpoint{3.987270in}{0.413320in}}%
\pgfpathlineto{\pgfqpoint{3.984714in}{0.413320in}}%
\pgfpathlineto{\pgfqpoint{3.981929in}{0.413320in}}%
\pgfpathlineto{\pgfqpoint{3.979389in}{0.413320in}}%
\pgfpathlineto{\pgfqpoint{3.976563in}{0.413320in}}%
\pgfpathlineto{\pgfqpoint{3.973885in}{0.413320in}}%
\pgfpathlineto{\pgfqpoint{3.971250in}{0.413320in}}%
\pgfpathlineto{\pgfqpoint{3.968523in}{0.413320in}}%
\pgfpathlineto{\pgfqpoint{3.966013in}{0.413320in}}%
\pgfpathlineto{\pgfqpoint{3.963176in}{0.413320in}}%
\pgfpathlineto{\pgfqpoint{3.960635in}{0.413320in}}%
\pgfpathlineto{\pgfqpoint{3.957823in}{0.413320in}}%
\pgfpathlineto{\pgfqpoint{3.955211in}{0.413320in}}%
\pgfpathlineto{\pgfqpoint{3.952464in}{0.413320in}}%
\pgfpathlineto{\pgfqpoint{3.949894in}{0.413320in}}%
\pgfpathlineto{\pgfqpoint{3.947101in}{0.413320in}}%
\pgfpathlineto{\pgfqpoint{3.944431in}{0.413320in}}%
\pgfpathlineto{\pgfqpoint{3.941778in}{0.413320in}}%
\pgfpathlineto{\pgfqpoint{3.939075in}{0.413320in}}%
\pgfpathlineto{\pgfqpoint{3.936395in}{0.413320in}}%
\pgfpathlineto{\pgfqpoint{3.933714in}{0.413320in}}%
\pgfpathlineto{\pgfqpoint{3.931202in}{0.413320in}}%
\pgfpathlineto{\pgfqpoint{3.928347in}{0.413320in}}%
\pgfpathlineto{\pgfqpoint{3.925778in}{0.413320in}}%
\pgfpathlineto{\pgfqpoint{3.923005in}{0.413320in}}%
\pgfpathlineto{\pgfqpoint{3.920412in}{0.413320in}}%
\pgfpathlineto{\pgfqpoint{3.917646in}{0.413320in}}%
\pgfpathlineto{\pgfqpoint{3.915107in}{0.413320in}}%
\pgfpathlineto{\pgfqpoint{3.912296in}{0.413320in}}%
\pgfpathlineto{\pgfqpoint{3.909602in}{0.413320in}}%
\pgfpathlineto{\pgfqpoint{3.906918in}{0.413320in}}%
\pgfpathlineto{\pgfqpoint{3.904252in}{0.413320in}}%
\pgfpathlineto{\pgfqpoint{3.901573in}{0.413320in}}%
\pgfpathlineto{\pgfqpoint{3.898891in}{0.413320in}}%
\pgfpathlineto{\pgfqpoint{3.896345in}{0.413320in}}%
\pgfpathlineto{\pgfqpoint{3.893541in}{0.413320in}}%
\pgfpathlineto{\pgfqpoint{3.890926in}{0.413320in}}%
\pgfpathlineto{\pgfqpoint{3.888188in}{0.413320in}}%
\pgfpathlineto{\pgfqpoint{3.885621in}{0.413320in}}%
\pgfpathlineto{\pgfqpoint{3.882850in}{0.413320in}}%
\pgfpathlineto{\pgfqpoint{3.880237in}{0.413320in}}%
\pgfpathlineto{\pgfqpoint{3.877466in}{0.413320in}}%
\pgfpathlineto{\pgfqpoint{3.874790in}{0.413320in}}%
\pgfpathlineto{\pgfqpoint{3.872114in}{0.413320in}}%
\pgfpathlineto{\pgfqpoint{3.869435in}{0.413320in}}%
\pgfpathlineto{\pgfqpoint{3.866815in}{0.413320in}}%
\pgfpathlineto{\pgfqpoint{3.864073in}{0.413320in}}%
\pgfpathlineto{\pgfqpoint{3.861561in}{0.413320in}}%
\pgfpathlineto{\pgfqpoint{3.858720in}{0.413320in}}%
\pgfpathlineto{\pgfqpoint{3.856100in}{0.413320in}}%
\pgfpathlineto{\pgfqpoint{3.853358in}{0.413320in}}%
\pgfpathlineto{\pgfqpoint{3.850814in}{0.413320in}}%
\pgfpathlineto{\pgfqpoint{3.848005in}{0.413320in}}%
\pgfpathlineto{\pgfqpoint{3.845329in}{0.413320in}}%
\pgfpathlineto{\pgfqpoint{3.842641in}{0.413320in}}%
\pgfpathlineto{\pgfqpoint{3.839960in}{0.413320in}}%
\pgfpathlineto{\pgfqpoint{3.837286in}{0.413320in}}%
\pgfpathlineto{\pgfqpoint{3.834616in}{0.413320in}}%
\pgfpathlineto{\pgfqpoint{3.832053in}{0.413320in}}%
\pgfpathlineto{\pgfqpoint{3.829252in}{0.413320in}}%
\pgfpathlineto{\pgfqpoint{3.826679in}{0.413320in}}%
\pgfpathlineto{\pgfqpoint{3.823903in}{0.413320in}}%
\pgfpathlineto{\pgfqpoint{3.821315in}{0.413320in}}%
\pgfpathlineto{\pgfqpoint{3.818546in}{0.413320in}}%
\pgfpathlineto{\pgfqpoint{3.815983in}{0.413320in}}%
\pgfpathlineto{\pgfqpoint{3.813172in}{0.413320in}}%
\pgfpathlineto{\pgfqpoint{3.810510in}{0.413320in}}%
\pgfpathlineto{\pgfqpoint{3.807832in}{0.413320in}}%
\pgfpathlineto{\pgfqpoint{3.805145in}{0.413320in}}%
\pgfpathlineto{\pgfqpoint{3.802569in}{0.413320in}}%
\pgfpathlineto{\pgfqpoint{3.799797in}{0.413320in}}%
\pgfpathlineto{\pgfqpoint{3.797265in}{0.413320in}}%
\pgfpathlineto{\pgfqpoint{3.794435in}{0.413320in}}%
\pgfpathlineto{\pgfqpoint{3.791897in}{0.413320in}}%
\pgfpathlineto{\pgfqpoint{3.789084in}{0.413320in}}%
\pgfpathlineto{\pgfqpoint{3.786504in}{0.413320in}}%
\pgfpathlineto{\pgfqpoint{3.783725in}{0.413320in}}%
\pgfpathlineto{\pgfqpoint{3.781046in}{0.413320in}}%
\pgfpathlineto{\pgfqpoint{3.778370in}{0.413320in}}%
\pgfpathlineto{\pgfqpoint{3.775691in}{0.413320in}}%
\pgfpathlineto{\pgfqpoint{3.773014in}{0.413320in}}%
\pgfpathlineto{\pgfqpoint{3.770323in}{0.413320in}}%
\pgfpathlineto{\pgfqpoint{3.767782in}{0.413320in}}%
\pgfpathlineto{\pgfqpoint{3.764966in}{0.413320in}}%
\pgfpathlineto{\pgfqpoint{3.762389in}{0.413320in}}%
\pgfpathlineto{\pgfqpoint{3.759622in}{0.413320in}}%
\pgfpathlineto{\pgfqpoint{3.757065in}{0.413320in}}%
\pgfpathlineto{\pgfqpoint{3.754265in}{0.413320in}}%
\pgfpathlineto{\pgfqpoint{3.751728in}{0.413320in}}%
\pgfpathlineto{\pgfqpoint{3.748903in}{0.413320in}}%
\pgfpathlineto{\pgfqpoint{3.746229in}{0.413320in}}%
\pgfpathlineto{\pgfqpoint{3.743548in}{0.413320in}}%
\pgfpathlineto{\pgfqpoint{3.740874in}{0.413320in}}%
\pgfpathlineto{\pgfqpoint{3.738194in}{0.413320in}}%
\pgfpathlineto{\pgfqpoint{3.735509in}{0.413320in}}%
\pgfpathlineto{\pgfqpoint{3.732950in}{0.413320in}}%
\pgfpathlineto{\pgfqpoint{3.730158in}{0.413320in}}%
\pgfpathlineto{\pgfqpoint{3.727581in}{0.413320in}}%
\pgfpathlineto{\pgfqpoint{3.724804in}{0.413320in}}%
\pgfpathlineto{\pgfqpoint{3.722228in}{0.413320in}}%
\pgfpathlineto{\pgfqpoint{3.719446in}{0.413320in}}%
\pgfpathlineto{\pgfqpoint{3.716875in}{0.413320in}}%
\pgfpathlineto{\pgfqpoint{3.714086in}{0.413320in}}%
\pgfpathlineto{\pgfqpoint{3.711410in}{0.413320in}}%
\pgfpathlineto{\pgfqpoint{3.708729in}{0.413320in}}%
\pgfpathlineto{\pgfqpoint{3.706053in}{0.413320in}}%
\pgfpathlineto{\pgfqpoint{3.703460in}{0.413320in}}%
\pgfpathlineto{\pgfqpoint{3.700684in}{0.413320in}}%
\pgfpathlineto{\pgfqpoint{3.698125in}{0.413320in}}%
\pgfpathlineto{\pgfqpoint{3.695331in}{0.413320in}}%
\pgfpathlineto{\pgfqpoint{3.692765in}{0.413320in}}%
\pgfpathlineto{\pgfqpoint{3.689983in}{0.413320in}}%
\pgfpathlineto{\pgfqpoint{3.687442in}{0.413320in}}%
\pgfpathlineto{\pgfqpoint{3.684620in}{0.413320in}}%
\pgfpathlineto{\pgfqpoint{3.681948in}{0.413320in}}%
\pgfpathlineto{\pgfqpoint{3.679273in}{0.413320in}}%
\pgfpathlineto{\pgfqpoint{3.676591in}{0.413320in}}%
\pgfpathlineto{\pgfqpoint{3.673911in}{0.413320in}}%
\pgfpathlineto{\pgfqpoint{3.671232in}{0.413320in}}%
\pgfpathlineto{\pgfqpoint{3.668665in}{0.413320in}}%
\pgfpathlineto{\pgfqpoint{3.665864in}{0.413320in}}%
\pgfpathlineto{\pgfqpoint{3.663276in}{0.413320in}}%
\pgfpathlineto{\pgfqpoint{3.660515in}{0.413320in}}%
\pgfpathlineto{\pgfqpoint{3.657917in}{0.413320in}}%
\pgfpathlineto{\pgfqpoint{3.655165in}{0.413320in}}%
\pgfpathlineto{\pgfqpoint{3.652628in}{0.413320in}}%
\pgfpathlineto{\pgfqpoint{3.649837in}{0.413320in}}%
\pgfpathlineto{\pgfqpoint{3.647130in}{0.413320in}}%
\pgfpathlineto{\pgfqpoint{3.644452in}{0.413320in}}%
\pgfpathlineto{\pgfqpoint{3.641773in}{0.413320in}}%
\pgfpathlineto{\pgfqpoint{3.639207in}{0.413320in}}%
\pgfpathlineto{\pgfqpoint{3.636413in}{0.413320in}}%
\pgfpathlineto{\pgfqpoint{3.633858in}{0.413320in}}%
\pgfpathlineto{\pgfqpoint{3.631058in}{0.413320in}}%
\pgfpathlineto{\pgfqpoint{3.628460in}{0.413320in}}%
\pgfpathlineto{\pgfqpoint{3.625689in}{0.413320in}}%
\pgfpathlineto{\pgfqpoint{3.623165in}{0.413320in}}%
\pgfpathlineto{\pgfqpoint{3.620345in}{0.413320in}}%
\pgfpathlineto{\pgfqpoint{3.617667in}{0.413320in}}%
\pgfpathlineto{\pgfqpoint{3.614982in}{0.413320in}}%
\pgfpathlineto{\pgfqpoint{3.612311in}{0.413320in}}%
\pgfpathlineto{\pgfqpoint{3.609632in}{0.413320in}}%
\pgfpathlineto{\pgfqpoint{3.606951in}{0.413320in}}%
\pgfpathlineto{\pgfqpoint{3.604387in}{0.413320in}}%
\pgfpathlineto{\pgfqpoint{3.601590in}{0.413320in}}%
\pgfpathlineto{\pgfqpoint{3.598998in}{0.413320in}}%
\pgfpathlineto{\pgfqpoint{3.596240in}{0.413320in}}%
\pgfpathlineto{\pgfqpoint{3.593620in}{0.413320in}}%
\pgfpathlineto{\pgfqpoint{3.590883in}{0.413320in}}%
\pgfpathlineto{\pgfqpoint{3.588258in}{0.413320in}}%
\pgfpathlineto{\pgfqpoint{3.585532in}{0.413320in}}%
\pgfpathlineto{\pgfqpoint{3.582851in}{0.413320in}}%
\pgfpathlineto{\pgfqpoint{3.580191in}{0.413320in}}%
\pgfpathlineto{\pgfqpoint{3.577487in}{0.413320in}}%
\pgfpathlineto{\pgfqpoint{3.574814in}{0.413320in}}%
\pgfpathlineto{\pgfqpoint{3.572126in}{0.413320in}}%
\pgfpathlineto{\pgfqpoint{3.569584in}{0.413320in}}%
\pgfpathlineto{\pgfqpoint{3.566774in}{0.413320in}}%
\pgfpathlineto{\pgfqpoint{3.564188in}{0.413320in}}%
\pgfpathlineto{\pgfqpoint{3.561420in}{0.413320in}}%
\pgfpathlineto{\pgfqpoint{3.558853in}{0.413320in}}%
\pgfpathlineto{\pgfqpoint{3.556061in}{0.413320in}}%
\pgfpathlineto{\pgfqpoint{3.553498in}{0.413320in}}%
\pgfpathlineto{\pgfqpoint{3.550713in}{0.413320in}}%
\pgfpathlineto{\pgfqpoint{3.548029in}{0.413320in}}%
\pgfpathlineto{\pgfqpoint{3.545349in}{0.413320in}}%
\pgfpathlineto{\pgfqpoint{3.542656in}{0.413320in}}%
\pgfpathlineto{\pgfqpoint{3.540093in}{0.413320in}}%
\pgfpathlineto{\pgfqpoint{3.537309in}{0.413320in}}%
\pgfpathlineto{\pgfqpoint{3.534783in}{0.413320in}}%
\pgfpathlineto{\pgfqpoint{3.531955in}{0.413320in}}%
\pgfpathlineto{\pgfqpoint{3.529327in}{0.413320in}}%
\pgfpathlineto{\pgfqpoint{3.526601in}{0.413320in}}%
\pgfpathlineto{\pgfqpoint{3.524041in}{0.413320in}}%
\pgfpathlineto{\pgfqpoint{3.521244in}{0.413320in}}%
\pgfpathlineto{\pgfqpoint{3.518565in}{0.413320in}}%
\pgfpathlineto{\pgfqpoint{3.515884in}{0.413320in}}%
\pgfpathlineto{\pgfqpoint{3.513209in}{0.413320in}}%
\pgfpathlineto{\pgfqpoint{3.510533in}{0.413320in}}%
\pgfpathlineto{\pgfqpoint{3.507840in}{0.413320in}}%
\pgfpathlineto{\pgfqpoint{3.505262in}{0.413320in}}%
\pgfpathlineto{\pgfqpoint{3.502488in}{0.413320in}}%
\pgfpathlineto{\pgfqpoint{3.499909in}{0.413320in}}%
\pgfpathlineto{\pgfqpoint{3.497139in}{0.413320in}}%
\pgfpathlineto{\pgfqpoint{3.494581in}{0.413320in}}%
\pgfpathlineto{\pgfqpoint{3.491783in}{0.413320in}}%
\pgfpathlineto{\pgfqpoint{3.489223in}{0.413320in}}%
\pgfpathlineto{\pgfqpoint{3.486442in}{0.413320in}}%
\pgfpathlineto{\pgfqpoint{3.483744in}{0.413320in}}%
\pgfpathlineto{\pgfqpoint{3.481072in}{0.413320in}}%
\pgfpathlineto{\pgfqpoint{3.478378in}{0.413320in}}%
\pgfpathlineto{\pgfqpoint{3.475821in}{0.413320in}}%
\pgfpathlineto{\pgfqpoint{3.473021in}{0.413320in}}%
\pgfpathlineto{\pgfqpoint{3.470466in}{0.413320in}}%
\pgfpathlineto{\pgfqpoint{3.467678in}{0.413320in}}%
\pgfpathlineto{\pgfqpoint{3.465072in}{0.413320in}}%
\pgfpathlineto{\pgfqpoint{3.462321in}{0.413320in}}%
\pgfpathlineto{\pgfqpoint{3.459695in}{0.413320in}}%
\pgfpathlineto{\pgfqpoint{3.456960in}{0.413320in}}%
\pgfpathlineto{\pgfqpoint{3.454285in}{0.413320in}}%
\pgfpathlineto{\pgfqpoint{3.451597in}{0.413320in}}%
\pgfpathlineto{\pgfqpoint{3.448926in}{0.413320in}}%
\pgfpathlineto{\pgfqpoint{3.446257in}{0.413320in}}%
\pgfpathlineto{\pgfqpoint{3.443574in}{0.413320in}}%
\pgfpathlineto{\pgfqpoint{3.440996in}{0.413320in}}%
\pgfpathlineto{\pgfqpoint{3.438210in}{0.413320in}}%
\pgfpathlineto{\pgfqpoint{3.435635in}{0.413320in}}%
\pgfpathlineto{\pgfqpoint{3.432851in}{0.413320in}}%
\pgfpathlineto{\pgfqpoint{3.430313in}{0.413320in}}%
\pgfpathlineto{\pgfqpoint{3.427501in}{0.413320in}}%
\pgfpathlineto{\pgfqpoint{3.424887in}{0.413320in}}%
\pgfpathlineto{\pgfqpoint{3.422142in}{0.413320in}}%
\pgfpathlineto{\pgfqpoint{3.419455in}{0.413320in}}%
\pgfpathlineto{\pgfqpoint{3.416780in}{0.413320in}}%
\pgfpathlineto{\pgfqpoint{3.414109in}{0.413320in}}%
\pgfpathlineto{\pgfqpoint{3.411431in}{0.413320in}}%
\pgfpathlineto{\pgfqpoint{3.408752in}{0.413320in}}%
\pgfpathlineto{\pgfqpoint{3.406202in}{0.413320in}}%
\pgfpathlineto{\pgfqpoint{3.403394in}{0.413320in}}%
\pgfpathlineto{\pgfqpoint{3.400783in}{0.413320in}}%
\pgfpathlineto{\pgfqpoint{3.398037in}{0.413320in}}%
\pgfpathlineto{\pgfqpoint{3.395461in}{0.413320in}}%
\pgfpathlineto{\pgfqpoint{3.392681in}{0.413320in}}%
\pgfpathlineto{\pgfqpoint{3.390102in}{0.413320in}}%
\pgfpathlineto{\pgfqpoint{3.387309in}{0.413320in}}%
\pgfpathlineto{\pgfqpoint{3.384647in}{0.413320in}}%
\pgfpathlineto{\pgfqpoint{3.381959in}{0.413320in}}%
\pgfpathlineto{\pgfqpoint{3.379290in}{0.413320in}}%
\pgfpathlineto{\pgfqpoint{3.376735in}{0.413320in}}%
\pgfpathlineto{\pgfqpoint{3.373921in}{0.413320in}}%
\pgfpathlineto{\pgfqpoint{3.371357in}{0.413320in}}%
\pgfpathlineto{\pgfqpoint{3.368577in}{0.413320in}}%
\pgfpathlineto{\pgfqpoint{3.365996in}{0.413320in}}%
\pgfpathlineto{\pgfqpoint{3.363221in}{0.413320in}}%
\pgfpathlineto{\pgfqpoint{3.360620in}{0.413320in}}%
\pgfpathlineto{\pgfqpoint{3.357862in}{0.413320in}}%
\pgfpathlineto{\pgfqpoint{3.355177in}{0.413320in}}%
\pgfpathlineto{\pgfqpoint{3.352505in}{0.413320in}}%
\pgfpathlineto{\pgfqpoint{3.349828in}{0.413320in}}%
\pgfpathlineto{\pgfqpoint{3.347139in}{0.413320in}}%
\pgfpathlineto{\pgfqpoint{3.344468in}{0.413320in}}%
\pgfpathlineto{\pgfqpoint{3.341893in}{0.413320in}}%
\pgfpathlineto{\pgfqpoint{3.339101in}{0.413320in}}%
\pgfpathlineto{\pgfqpoint{3.336541in}{0.413320in}}%
\pgfpathlineto{\pgfqpoint{3.333758in}{0.413320in}}%
\pgfpathlineto{\pgfqpoint{3.331183in}{0.413320in}}%
\pgfpathlineto{\pgfqpoint{3.328401in}{0.413320in}}%
\pgfpathlineto{\pgfqpoint{3.325860in}{0.413320in}}%
\pgfpathlineto{\pgfqpoint{3.323049in}{0.413320in}}%
\pgfpathlineto{\pgfqpoint{3.320366in}{0.413320in}}%
\pgfpathlineto{\pgfqpoint{3.317688in}{0.413320in}}%
\pgfpathlineto{\pgfqpoint{3.315008in}{0.413320in}}%
\pgfpathlineto{\pgfqpoint{3.312480in}{0.413320in}}%
\pgfpathlineto{\pgfqpoint{3.309652in}{0.413320in}}%
\pgfpathlineto{\pgfqpoint{3.307104in}{0.413320in}}%
\pgfpathlineto{\pgfqpoint{3.304295in}{0.413320in}}%
\pgfpathlineto{\pgfqpoint{3.301719in}{0.413320in}}%
\pgfpathlineto{\pgfqpoint{3.298937in}{0.413320in}}%
\pgfpathlineto{\pgfqpoint{3.296376in}{0.413320in}}%
\pgfpathlineto{\pgfqpoint{3.293574in}{0.413320in}}%
\pgfpathlineto{\pgfqpoint{3.290890in}{0.413320in}}%
\pgfpathlineto{\pgfqpoint{3.288225in}{0.413320in}}%
\pgfpathlineto{\pgfqpoint{3.285534in}{0.413320in}}%
\pgfpathlineto{\pgfqpoint{3.282870in}{0.413320in}}%
\pgfpathlineto{\pgfqpoint{3.280189in}{0.413320in}}%
\pgfpathlineto{\pgfqpoint{3.277603in}{0.413320in}}%
\pgfpathlineto{\pgfqpoint{3.274831in}{0.413320in}}%
\pgfpathlineto{\pgfqpoint{3.272254in}{0.413320in}}%
\pgfpathlineto{\pgfqpoint{3.269478in}{0.413320in}}%
\pgfpathlineto{\pgfqpoint{3.266849in}{0.413320in}}%
\pgfpathlineto{\pgfqpoint{3.264119in}{0.413320in}}%
\pgfpathlineto{\pgfqpoint{3.261594in}{0.413320in}}%
\pgfpathlineto{\pgfqpoint{3.258784in}{0.413320in}}%
\pgfpathlineto{\pgfqpoint{3.256083in}{0.413320in}}%
\pgfpathlineto{\pgfqpoint{3.253404in}{0.413320in}}%
\pgfpathlineto{\pgfqpoint{3.250716in}{0.413320in}}%
\pgfpathlineto{\pgfqpoint{3.248049in}{0.413320in}}%
\pgfpathlineto{\pgfqpoint{3.245363in}{0.413320in}}%
\pgfpathlineto{\pgfqpoint{3.242807in}{0.413320in}}%
\pgfpathlineto{\pgfqpoint{3.240010in}{0.413320in}}%
\pgfpathlineto{\pgfqpoint{3.237411in}{0.413320in}}%
\pgfpathlineto{\pgfqpoint{3.234658in}{0.413320in}}%
\pgfpathlineto{\pgfqpoint{3.232069in}{0.413320in}}%
\pgfpathlineto{\pgfqpoint{3.229310in}{0.413320in}}%
\pgfpathlineto{\pgfqpoint{3.226609in}{0.413320in}}%
\pgfpathlineto{\pgfqpoint{3.223942in}{0.413320in}}%
\pgfpathlineto{\pgfqpoint{3.221255in}{0.413320in}}%
\pgfpathlineto{\pgfqpoint{3.218586in}{0.413320in}}%
\pgfpathlineto{\pgfqpoint{3.215908in}{0.413320in}}%
\pgfpathlineto{\pgfqpoint{3.213342in}{0.413320in}}%
\pgfpathlineto{\pgfqpoint{3.210545in}{0.413320in}}%
\pgfpathlineto{\pgfqpoint{3.207984in}{0.413320in}}%
\pgfpathlineto{\pgfqpoint{3.205195in}{0.413320in}}%
\pgfpathlineto{\pgfqpoint{3.202562in}{0.413320in}}%
\pgfpathlineto{\pgfqpoint{3.199823in}{0.413320in}}%
\pgfpathlineto{\pgfqpoint{3.197226in}{0.413320in}}%
\pgfpathlineto{\pgfqpoint{3.194508in}{0.413320in}}%
\pgfpathlineto{\pgfqpoint{3.191796in}{0.413320in}}%
\pgfpathlineto{\pgfqpoint{3.189117in}{0.413320in}}%
\pgfpathlineto{\pgfqpoint{3.186440in}{0.413320in}}%
\pgfpathlineto{\pgfqpoint{3.183760in}{0.413320in}}%
\pgfpathlineto{\pgfqpoint{3.181089in}{0.413320in}}%
\pgfpathlineto{\pgfqpoint{3.178525in}{0.413320in}}%
\pgfpathlineto{\pgfqpoint{3.175724in}{0.413320in}}%
\pgfpathlineto{\pgfqpoint{3.173142in}{0.413320in}}%
\pgfpathlineto{\pgfqpoint{3.170375in}{0.413320in}}%
\pgfpathlineto{\pgfqpoint{3.167776in}{0.413320in}}%
\pgfpathlineto{\pgfqpoint{3.165019in}{0.413320in}}%
\pgfpathlineto{\pgfqpoint{3.162474in}{0.413320in}}%
\pgfpathlineto{\pgfqpoint{3.159675in}{0.413320in}}%
\pgfpathlineto{\pgfqpoint{3.156981in}{0.413320in}}%
\pgfpathlineto{\pgfqpoint{3.154327in}{0.413320in}}%
\pgfpathlineto{\pgfqpoint{3.151612in}{0.413320in}}%
\pgfpathlineto{\pgfqpoint{3.149057in}{0.413320in}}%
\pgfpathlineto{\pgfqpoint{3.146271in}{0.413320in}}%
\pgfpathlineto{\pgfqpoint{3.143740in}{0.413320in}}%
\pgfpathlineto{\pgfqpoint{3.140913in}{0.413320in}}%
\pgfpathlineto{\pgfqpoint{3.138375in}{0.413320in}}%
\pgfpathlineto{\pgfqpoint{3.135550in}{0.413320in}}%
\pgfpathlineto{\pgfqpoint{3.132946in}{0.413320in}}%
\pgfpathlineto{\pgfqpoint{3.130199in}{0.413320in}}%
\pgfpathlineto{\pgfqpoint{3.127512in}{0.413320in}}%
\pgfpathlineto{\pgfqpoint{3.124842in}{0.413320in}}%
\pgfpathlineto{\pgfqpoint{3.122163in}{0.413320in}}%
\pgfpathlineto{\pgfqpoint{3.119487in}{0.413320in}}%
\pgfpathlineto{\pgfqpoint{3.116807in}{0.413320in}}%
\pgfpathlineto{\pgfqpoint{3.114242in}{0.413320in}}%
\pgfpathlineto{\pgfqpoint{3.111451in}{0.413320in}}%
\pgfpathlineto{\pgfqpoint{3.108896in}{0.413320in}}%
\pgfpathlineto{\pgfqpoint{3.106094in}{0.413320in}}%
\pgfpathlineto{\pgfqpoint{3.103508in}{0.413320in}}%
\pgfpathlineto{\pgfqpoint{3.100737in}{0.413320in}}%
\pgfpathlineto{\pgfqpoint{3.098163in}{0.413320in}}%
\pgfpathlineto{\pgfqpoint{3.095388in}{0.413320in}}%
\pgfpathlineto{\pgfqpoint{3.092699in}{0.413320in}}%
\pgfpathlineto{\pgfqpoint{3.090023in}{0.413320in}}%
\pgfpathlineto{\pgfqpoint{3.087343in}{0.413320in}}%
\pgfpathlineto{\pgfqpoint{3.084671in}{0.413320in}}%
\pgfpathlineto{\pgfqpoint{3.081990in}{0.413320in}}%
\pgfpathlineto{\pgfqpoint{3.079381in}{0.413320in}}%
\pgfpathlineto{\pgfqpoint{3.076631in}{0.413320in}}%
\pgfpathlineto{\pgfqpoint{3.074056in}{0.413320in}}%
\pgfpathlineto{\pgfqpoint{3.071266in}{0.413320in}}%
\pgfpathlineto{\pgfqpoint{3.068709in}{0.413320in}}%
\pgfpathlineto{\pgfqpoint{3.065916in}{0.413320in}}%
\pgfpathlineto{\pgfqpoint{3.063230in}{0.413320in}}%
\pgfpathlineto{\pgfqpoint{3.060561in}{0.413320in}}%
\pgfpathlineto{\pgfqpoint{3.057884in}{0.413320in}}%
\pgfpathlineto{\pgfqpoint{3.055202in}{0.413320in}}%
\pgfpathlineto{\pgfqpoint{3.052526in}{0.413320in}}%
\pgfpathlineto{\pgfqpoint{3.049988in}{0.413320in}}%
\pgfpathlineto{\pgfqpoint{3.047157in}{0.413320in}}%
\pgfpathlineto{\pgfqpoint{3.044568in}{0.413320in}}%
\pgfpathlineto{\pgfqpoint{3.041813in}{0.413320in}}%
\pgfpathlineto{\pgfqpoint{3.039262in}{0.413320in}}%
\pgfpathlineto{\pgfqpoint{3.036456in}{0.413320in}}%
\pgfpathlineto{\pgfqpoint{3.033921in}{0.413320in}}%
\pgfpathlineto{\pgfqpoint{3.031091in}{0.413320in}}%
\pgfpathlineto{\pgfqpoint{3.028412in}{0.413320in}}%
\pgfpathlineto{\pgfqpoint{3.025803in}{0.413320in}}%
\pgfpathlineto{\pgfqpoint{3.023058in}{0.413320in}}%
\pgfpathlineto{\pgfqpoint{3.020382in}{0.413320in}}%
\pgfpathlineto{\pgfqpoint{3.017707in}{0.413320in}}%
\pgfpathlineto{\pgfqpoint{3.015097in}{0.413320in}}%
\pgfpathlineto{\pgfqpoint{3.012351in}{0.413320in}}%
\pgfpathlineto{\pgfqpoint{3.009784in}{0.413320in}}%
\pgfpathlineto{\pgfqpoint{3.006993in}{0.413320in}}%
\pgfpathlineto{\pgfqpoint{3.004419in}{0.413320in}}%
\pgfpathlineto{\pgfqpoint{3.001635in}{0.413320in}}%
\pgfpathlineto{\pgfqpoint{2.999103in}{0.413320in}}%
\pgfpathlineto{\pgfqpoint{2.996300in}{0.413320in}}%
\pgfpathlineto{\pgfqpoint{2.993595in}{0.413320in}}%
\pgfpathlineto{\pgfqpoint{2.990978in}{0.413320in}}%
\pgfpathlineto{\pgfqpoint{2.988238in}{0.413320in}}%
\pgfpathlineto{\pgfqpoint{2.985666in}{0.413320in}}%
\pgfpathlineto{\pgfqpoint{2.982885in}{0.413320in}}%
\pgfpathlineto{\pgfqpoint{2.980341in}{0.413320in}}%
\pgfpathlineto{\pgfqpoint{2.977517in}{0.413320in}}%
\pgfpathlineto{\pgfqpoint{2.974972in}{0.413320in}}%
\pgfpathlineto{\pgfqpoint{2.972177in}{0.413320in}}%
\pgfpathlineto{\pgfqpoint{2.969599in}{0.413320in}}%
\pgfpathlineto{\pgfqpoint{2.966812in}{0.413320in}}%
\pgfpathlineto{\pgfqpoint{2.964127in}{0.413320in}}%
\pgfpathlineto{\pgfqpoint{2.961460in}{0.413320in}}%
\pgfpathlineto{\pgfqpoint{2.958782in}{0.413320in}}%
\pgfpathlineto{\pgfqpoint{2.956103in}{0.413320in}}%
\pgfpathlineto{\pgfqpoint{2.953422in}{0.413320in}}%
\pgfpathlineto{\pgfqpoint{2.950884in}{0.413320in}}%
\pgfpathlineto{\pgfqpoint{2.948068in}{0.413320in}}%
\pgfpathlineto{\pgfqpoint{2.945461in}{0.413320in}}%
\pgfpathlineto{\pgfqpoint{2.942711in}{0.413320in}}%
\pgfpathlineto{\pgfqpoint{2.940120in}{0.413320in}}%
\pgfpathlineto{\pgfqpoint{2.937352in}{0.413320in}}%
\pgfpathlineto{\pgfqpoint{2.934759in}{0.413320in}}%
\pgfpathlineto{\pgfqpoint{2.932033in}{0.413320in}}%
\pgfpathlineto{\pgfqpoint{2.929321in}{0.413320in}}%
\pgfpathlineto{\pgfqpoint{2.926655in}{0.413320in}}%
\pgfpathlineto{\pgfqpoint{2.923963in}{0.413320in}}%
\pgfpathlineto{\pgfqpoint{2.921363in}{0.413320in}}%
\pgfpathlineto{\pgfqpoint{2.918606in}{0.413320in}}%
\pgfpathlineto{\pgfqpoint{2.916061in}{0.413320in}}%
\pgfpathlineto{\pgfqpoint{2.913243in}{0.413320in}}%
\pgfpathlineto{\pgfqpoint{2.910631in}{0.413320in}}%
\pgfpathlineto{\pgfqpoint{2.907882in}{0.413320in}}%
\pgfpathlineto{\pgfqpoint{2.905341in}{0.413320in}}%
\pgfpathlineto{\pgfqpoint{2.902535in}{0.413320in}}%
\pgfpathlineto{\pgfqpoint{2.899858in}{0.413320in}}%
\pgfpathlineto{\pgfqpoint{2.897179in}{0.413320in}}%
\pgfpathlineto{\pgfqpoint{2.894487in}{0.413320in}}%
\pgfpathlineto{\pgfqpoint{2.891809in}{0.413320in}}%
\pgfpathlineto{\pgfqpoint{2.889145in}{0.413320in}}%
\pgfpathlineto{\pgfqpoint{2.886578in}{0.413320in}}%
\pgfpathlineto{\pgfqpoint{2.883780in}{0.413320in}}%
\pgfpathlineto{\pgfqpoint{2.881254in}{0.413320in}}%
\pgfpathlineto{\pgfqpoint{2.878431in}{0.413320in}}%
\pgfpathlineto{\pgfqpoint{2.875882in}{0.413320in}}%
\pgfpathlineto{\pgfqpoint{2.873074in}{0.413320in}}%
\pgfpathlineto{\pgfqpoint{2.870475in}{0.413320in}}%
\pgfpathlineto{\pgfqpoint{2.867713in}{0.413320in}}%
\pgfpathlineto{\pgfqpoint{2.865031in}{0.413320in}}%
\pgfpathlineto{\pgfqpoint{2.862402in}{0.413320in}}%
\pgfpathlineto{\pgfqpoint{2.859668in}{0.413320in}}%
\pgfpathlineto{\pgfqpoint{2.857003in}{0.413320in}}%
\pgfpathlineto{\pgfqpoint{2.854325in}{0.413320in}}%
\pgfpathlineto{\pgfqpoint{2.851793in}{0.413320in}}%
\pgfpathlineto{\pgfqpoint{2.848960in}{0.413320in}}%
\pgfpathlineto{\pgfqpoint{2.846408in}{0.413320in}}%
\pgfpathlineto{\pgfqpoint{2.843611in}{0.413320in}}%
\pgfpathlineto{\pgfqpoint{2.841055in}{0.413320in}}%
\pgfpathlineto{\pgfqpoint{2.838254in}{0.413320in}}%
\pgfpathlineto{\pgfqpoint{2.835698in}{0.413320in}}%
\pgfpathlineto{\pgfqpoint{2.832894in}{0.413320in}}%
\pgfpathlineto{\pgfqpoint{2.830219in}{0.413320in}}%
\pgfpathlineto{\pgfqpoint{2.827567in}{0.413320in}}%
\pgfpathlineto{\pgfqpoint{2.824851in}{0.413320in}}%
\pgfpathlineto{\pgfqpoint{2.822303in}{0.413320in}}%
\pgfpathlineto{\pgfqpoint{2.819506in}{0.413320in}}%
\pgfpathlineto{\pgfqpoint{2.816867in}{0.413320in}}%
\pgfpathlineto{\pgfqpoint{2.814141in}{0.413320in}}%
\pgfpathlineto{\pgfqpoint{2.811597in}{0.413320in}}%
\pgfpathlineto{\pgfqpoint{2.808792in}{0.413320in}}%
\pgfpathlineto{\pgfqpoint{2.806175in}{0.413320in}}%
\pgfpathlineto{\pgfqpoint{2.803435in}{0.413320in}}%
\pgfpathlineto{\pgfqpoint{2.800756in}{0.413320in}}%
\pgfpathlineto{\pgfqpoint{2.798070in}{0.413320in}}%
\pgfpathlineto{\pgfqpoint{2.795398in}{0.413320in}}%
\pgfpathlineto{\pgfqpoint{2.792721in}{0.413320in}}%
\pgfpathlineto{\pgfqpoint{2.790044in}{0.413320in}}%
\pgfpathlineto{\pgfqpoint{2.787468in}{0.413320in}}%
\pgfpathlineto{\pgfqpoint{2.784687in}{0.413320in}}%
\pgfpathlineto{\pgfqpoint{2.782113in}{0.413320in}}%
\pgfpathlineto{\pgfqpoint{2.779330in}{0.413320in}}%
\pgfpathlineto{\pgfqpoint{2.776767in}{0.413320in}}%
\pgfpathlineto{\pgfqpoint{2.773972in}{0.413320in}}%
\pgfpathlineto{\pgfqpoint{2.771373in}{0.413320in}}%
\pgfpathlineto{\pgfqpoint{2.768617in}{0.413320in}}%
\pgfpathlineto{\pgfqpoint{2.765935in}{0.413320in}}%
\pgfpathlineto{\pgfqpoint{2.763253in}{0.413320in}}%
\pgfpathlineto{\pgfqpoint{2.760581in}{0.413320in}}%
\pgfpathlineto{\pgfqpoint{2.758028in}{0.413320in}}%
\pgfpathlineto{\pgfqpoint{2.755224in}{0.413320in}}%
\pgfpathlineto{\pgfqpoint{2.752614in}{0.413320in}}%
\pgfpathlineto{\pgfqpoint{2.749868in}{0.413320in}}%
\pgfpathlineto{\pgfqpoint{2.747260in}{0.413320in}}%
\pgfpathlineto{\pgfqpoint{2.744510in}{0.413320in}}%
\pgfpathlineto{\pgfqpoint{2.741928in}{0.413320in}}%
\pgfpathlineto{\pgfqpoint{2.739155in}{0.413320in}}%
\pgfpathlineto{\pgfqpoint{2.736476in}{0.413320in}}%
\pgfpathlineto{\pgfqpoint{2.733798in}{0.413320in}}%
\pgfpathlineto{\pgfqpoint{2.731119in}{0.413320in}}%
\pgfpathlineto{\pgfqpoint{2.728439in}{0.413320in}}%
\pgfpathlineto{\pgfqpoint{2.725760in}{0.413320in}}%
\pgfpathlineto{\pgfqpoint{2.723211in}{0.413320in}}%
\pgfpathlineto{\pgfqpoint{2.720404in}{0.413320in}}%
\pgfpathlineto{\pgfqpoint{2.717773in}{0.413320in}}%
\pgfpathlineto{\pgfqpoint{2.715036in}{0.413320in}}%
\pgfpathlineto{\pgfqpoint{2.712477in}{0.413320in}}%
\pgfpathlineto{\pgfqpoint{2.709683in}{0.413320in}}%
\pgfpathlineto{\pgfqpoint{2.707125in}{0.413320in}}%
\pgfpathlineto{\pgfqpoint{2.704326in}{0.413320in}}%
\pgfpathlineto{\pgfqpoint{2.701657in}{0.413320in}}%
\pgfpathlineto{\pgfqpoint{2.698968in}{0.413320in}}%
\pgfpathlineto{\pgfqpoint{2.696293in}{0.413320in}}%
\pgfpathlineto{\pgfqpoint{2.693611in}{0.413320in}}%
\pgfpathlineto{\pgfqpoint{2.690940in}{0.413320in}}%
\pgfpathlineto{\pgfqpoint{2.688328in}{0.413320in}}%
\pgfpathlineto{\pgfqpoint{2.685586in}{0.413320in}}%
\pgfpathlineto{\pgfqpoint{2.683009in}{0.413320in}}%
\pgfpathlineto{\pgfqpoint{2.680224in}{0.413320in}}%
\pgfpathlineto{\pgfqpoint{2.677650in}{0.413320in}}%
\pgfpathlineto{\pgfqpoint{2.674873in}{0.413320in}}%
\pgfpathlineto{\pgfqpoint{2.672301in}{0.413320in}}%
\pgfpathlineto{\pgfqpoint{2.669506in}{0.413320in}}%
\pgfpathlineto{\pgfqpoint{2.666836in}{0.413320in}}%
\pgfpathlineto{\pgfqpoint{2.664151in}{0.413320in}}%
\pgfpathlineto{\pgfqpoint{2.661481in}{0.413320in}}%
\pgfpathlineto{\pgfqpoint{2.658942in}{0.413320in}}%
\pgfpathlineto{\pgfqpoint{2.656124in}{0.413320in}}%
\pgfpathlineto{\pgfqpoint{2.653567in}{0.413320in}}%
\pgfpathlineto{\pgfqpoint{2.650767in}{0.413320in}}%
\pgfpathlineto{\pgfqpoint{2.648196in}{0.413320in}}%
\pgfpathlineto{\pgfqpoint{2.645408in}{0.413320in}}%
\pgfpathlineto{\pgfqpoint{2.642827in}{0.413320in}}%
\pgfpathlineto{\pgfqpoint{2.640053in}{0.413320in}}%
\pgfpathlineto{\pgfqpoint{2.637369in}{0.413320in}}%
\pgfpathlineto{\pgfqpoint{2.634700in}{0.413320in}}%
\pgfpathlineto{\pgfqpoint{2.632018in}{0.413320in}}%
\pgfpathlineto{\pgfqpoint{2.629340in}{0.413320in}}%
\pgfpathlineto{\pgfqpoint{2.626653in}{0.413320in}}%
\pgfpathlineto{\pgfqpoint{2.624077in}{0.413320in}}%
\pgfpathlineto{\pgfqpoint{2.621304in}{0.413320in}}%
\pgfpathlineto{\pgfqpoint{2.618773in}{0.413320in}}%
\pgfpathlineto{\pgfqpoint{2.615934in}{0.413320in}}%
\pgfpathlineto{\pgfqpoint{2.613393in}{0.413320in}}%
\pgfpathlineto{\pgfqpoint{2.610588in}{0.413320in}}%
\pgfpathlineto{\pgfqpoint{2.608004in}{0.413320in}}%
\pgfpathlineto{\pgfqpoint{2.605232in}{0.413320in}}%
\pgfpathlineto{\pgfqpoint{2.602557in}{0.413320in}}%
\pgfpathlineto{\pgfqpoint{2.599920in}{0.413320in}}%
\pgfpathlineto{\pgfqpoint{2.597196in}{0.413320in}}%
\pgfpathlineto{\pgfqpoint{2.594630in}{0.413320in}}%
\pgfpathlineto{\pgfqpoint{2.591842in}{0.413320in}}%
\pgfpathlineto{\pgfqpoint{2.589248in}{0.413320in}}%
\pgfpathlineto{\pgfqpoint{2.586484in}{0.413320in}}%
\pgfpathlineto{\pgfqpoint{2.583913in}{0.413320in}}%
\pgfpathlineto{\pgfqpoint{2.581129in}{0.413320in}}%
\pgfpathlineto{\pgfqpoint{2.578567in}{0.413320in}}%
\pgfpathlineto{\pgfqpoint{2.575779in}{0.413320in}}%
\pgfpathlineto{\pgfqpoint{2.573082in}{0.413320in}}%
\pgfpathlineto{\pgfqpoint{2.570411in}{0.413320in}}%
\pgfpathlineto{\pgfqpoint{2.567730in}{0.413320in}}%
\pgfpathlineto{\pgfqpoint{2.565045in}{0.413320in}}%
\pgfpathlineto{\pgfqpoint{2.562375in}{0.413320in}}%
\pgfpathlineto{\pgfqpoint{2.559790in}{0.413320in}}%
\pgfpathlineto{\pgfqpoint{2.557009in}{0.413320in}}%
\pgfpathlineto{\pgfqpoint{2.554493in}{0.413320in}}%
\pgfpathlineto{\pgfqpoint{2.551664in}{0.413320in}}%
\pgfpathlineto{\pgfqpoint{2.549114in}{0.413320in}}%
\pgfpathlineto{\pgfqpoint{2.546310in}{0.413320in}}%
\pgfpathlineto{\pgfqpoint{2.543765in}{0.413320in}}%
\pgfpathlineto{\pgfqpoint{2.540949in}{0.413320in}}%
\pgfpathlineto{\pgfqpoint{2.538274in}{0.413320in}}%
\pgfpathlineto{\pgfqpoint{2.535624in}{0.413320in}}%
\pgfpathlineto{\pgfqpoint{2.532917in}{0.413320in}}%
\pgfpathlineto{\pgfqpoint{2.530234in}{0.413320in}}%
\pgfpathlineto{\pgfqpoint{2.527560in}{0.413320in}}%
\pgfpathlineto{\pgfqpoint{2.524988in}{0.413320in}}%
\pgfpathlineto{\pgfqpoint{2.522197in}{0.413320in}}%
\pgfpathlineto{\pgfqpoint{2.519607in}{0.413320in}}%
\pgfpathlineto{\pgfqpoint{2.516845in}{0.413320in}}%
\pgfpathlineto{\pgfqpoint{2.514268in}{0.413320in}}%
\pgfpathlineto{\pgfqpoint{2.511478in}{0.413320in}}%
\pgfpathlineto{\pgfqpoint{2.508917in}{0.413320in}}%
\pgfpathlineto{\pgfqpoint{2.506163in}{0.413320in}}%
\pgfpathlineto{\pgfqpoint{2.503454in}{0.413320in}}%
\pgfpathlineto{\pgfqpoint{2.500801in}{0.413320in}}%
\pgfpathlineto{\pgfqpoint{2.498085in}{0.413320in}}%
\pgfpathlineto{\pgfqpoint{2.495542in}{0.413320in}}%
\pgfpathlineto{\pgfqpoint{2.492729in}{0.413320in}}%
\pgfpathlineto{\pgfqpoint{2.490183in}{0.413320in}}%
\pgfpathlineto{\pgfqpoint{2.487384in}{0.413320in}}%
\pgfpathlineto{\pgfqpoint{2.484870in}{0.413320in}}%
\pgfpathlineto{\pgfqpoint{2.482026in}{0.413320in}}%
\pgfpathlineto{\pgfqpoint{2.479420in}{0.413320in}}%
\pgfpathlineto{\pgfqpoint{2.476671in}{0.413320in}}%
\pgfpathlineto{\pgfqpoint{2.473989in}{0.413320in}}%
\pgfpathlineto{\pgfqpoint{2.471311in}{0.413320in}}%
\pgfpathlineto{\pgfqpoint{2.468635in}{0.413320in}}%
\pgfpathlineto{\pgfqpoint{2.465957in}{0.413320in}}%
\pgfpathlineto{\pgfqpoint{2.463280in}{0.413320in}}%
\pgfpathlineto{\pgfqpoint{2.460711in}{0.413320in}}%
\pgfpathlineto{\pgfqpoint{2.457917in}{0.413320in}}%
\pgfpathlineto{\pgfqpoint{2.455353in}{0.413320in}}%
\pgfpathlineto{\pgfqpoint{2.452562in}{0.413320in}}%
\pgfpathlineto{\pgfqpoint{2.450032in}{0.413320in}}%
\pgfpathlineto{\pgfqpoint{2.447209in}{0.413320in}}%
\pgfpathlineto{\pgfqpoint{2.444677in}{0.413320in}}%
\pgfpathlineto{\pgfqpoint{2.441876in}{0.413320in}}%
\pgfpathlineto{\pgfqpoint{2.439167in}{0.413320in}}%
\pgfpathlineto{\pgfqpoint{2.436518in}{0.413320in}}%
\pgfpathlineto{\pgfqpoint{2.433815in}{0.413320in}}%
\pgfpathlineto{\pgfqpoint{2.431251in}{0.413320in}}%
\pgfpathlineto{\pgfqpoint{2.428453in}{0.413320in}}%
\pgfpathlineto{\pgfqpoint{2.425878in}{0.413320in}}%
\pgfpathlineto{\pgfqpoint{2.423098in}{0.413320in}}%
\pgfpathlineto{\pgfqpoint{2.420528in}{0.413320in}}%
\pgfpathlineto{\pgfqpoint{2.417747in}{0.413320in}}%
\pgfpathlineto{\pgfqpoint{2.415184in}{0.413320in}}%
\pgfpathlineto{\pgfqpoint{2.412389in}{0.413320in}}%
\pgfpathlineto{\pgfqpoint{2.409699in}{0.413320in}}%
\pgfpathlineto{\pgfqpoint{2.407024in}{0.413320in}}%
\pgfpathlineto{\pgfqpoint{2.404352in}{0.413320in}}%
\pgfpathlineto{\pgfqpoint{2.401675in}{0.413320in}}%
\pgfpathlineto{\pgfqpoint{2.398995in}{0.413320in}}%
\pgfpathclose%
\pgfusepath{stroke,fill}%
\end{pgfscope}%
\begin{pgfscope}%
\pgfpathrectangle{\pgfqpoint{2.398995in}{0.319877in}}{\pgfqpoint{3.986877in}{1.993438in}} %
\pgfusepath{clip}%
\pgfsetbuttcap%
\pgfsetroundjoin%
\definecolor{currentfill}{rgb}{1.000000,1.000000,1.000000}%
\pgfsetfillcolor{currentfill}%
\pgfsetlinewidth{1.003750pt}%
\definecolor{currentstroke}{rgb}{0.642304,0.549768,0.958265}%
\pgfsetstrokecolor{currentstroke}%
\pgfsetdash{}{0pt}%
\pgfpathmoveto{\pgfqpoint{2.398995in}{0.413320in}}%
\pgfpathlineto{\pgfqpoint{2.398995in}{1.973584in}}%
\pgfpathlineto{\pgfqpoint{2.401675in}{1.978770in}}%
\pgfpathlineto{\pgfqpoint{2.404352in}{1.977916in}}%
\pgfpathlineto{\pgfqpoint{2.407024in}{1.973781in}}%
\pgfpathlineto{\pgfqpoint{2.409699in}{1.983446in}}%
\pgfpathlineto{\pgfqpoint{2.412389in}{1.983493in}}%
\pgfpathlineto{\pgfqpoint{2.415184in}{1.976438in}}%
\pgfpathlineto{\pgfqpoint{2.417747in}{1.981551in}}%
\pgfpathlineto{\pgfqpoint{2.420528in}{1.983580in}}%
\pgfpathlineto{\pgfqpoint{2.423098in}{1.979881in}}%
\pgfpathlineto{\pgfqpoint{2.425878in}{1.980833in}}%
\pgfpathlineto{\pgfqpoint{2.428453in}{1.976126in}}%
\pgfpathlineto{\pgfqpoint{2.431251in}{1.978985in}}%
\pgfpathlineto{\pgfqpoint{2.433815in}{1.978451in}}%
\pgfpathlineto{\pgfqpoint{2.436518in}{1.983583in}}%
\pgfpathlineto{\pgfqpoint{2.439167in}{1.977445in}}%
\pgfpathlineto{\pgfqpoint{2.441876in}{1.976016in}}%
\pgfpathlineto{\pgfqpoint{2.444677in}{1.977806in}}%
\pgfpathlineto{\pgfqpoint{2.447209in}{1.977552in}}%
\pgfpathlineto{\pgfqpoint{2.450032in}{1.970305in}}%
\pgfpathlineto{\pgfqpoint{2.452562in}{1.975525in}}%
\pgfpathlineto{\pgfqpoint{2.455353in}{1.977418in}}%
\pgfpathlineto{\pgfqpoint{2.457917in}{1.977868in}}%
\pgfpathlineto{\pgfqpoint{2.460711in}{1.982094in}}%
\pgfpathlineto{\pgfqpoint{2.463280in}{1.980201in}}%
\pgfpathlineto{\pgfqpoint{2.465957in}{1.978234in}}%
\pgfpathlineto{\pgfqpoint{2.468635in}{1.982033in}}%
\pgfpathlineto{\pgfqpoint{2.471311in}{1.971879in}}%
\pgfpathlineto{\pgfqpoint{2.473989in}{1.976469in}}%
\pgfpathlineto{\pgfqpoint{2.476671in}{1.971971in}}%
\pgfpathlineto{\pgfqpoint{2.479420in}{1.965587in}}%
\pgfpathlineto{\pgfqpoint{2.482026in}{1.965791in}}%
\pgfpathlineto{\pgfqpoint{2.484870in}{1.963929in}}%
\pgfpathlineto{\pgfqpoint{2.487384in}{1.963929in}}%
\pgfpathlineto{\pgfqpoint{2.490183in}{1.969720in}}%
\pgfpathlineto{\pgfqpoint{2.492729in}{1.975504in}}%
\pgfpathlineto{\pgfqpoint{2.495542in}{1.972471in}}%
\pgfpathlineto{\pgfqpoint{2.498085in}{1.973144in}}%
\pgfpathlineto{\pgfqpoint{2.500801in}{1.973779in}}%
\pgfpathlineto{\pgfqpoint{2.503454in}{1.974492in}}%
\pgfpathlineto{\pgfqpoint{2.506163in}{1.975936in}}%
\pgfpathlineto{\pgfqpoint{2.508917in}{1.974321in}}%
\pgfpathlineto{\pgfqpoint{2.511478in}{1.972784in}}%
\pgfpathlineto{\pgfqpoint{2.514268in}{1.972938in}}%
\pgfpathlineto{\pgfqpoint{2.516845in}{1.980688in}}%
\pgfpathlineto{\pgfqpoint{2.519607in}{1.982510in}}%
\pgfpathlineto{\pgfqpoint{2.522197in}{1.983164in}}%
\pgfpathlineto{\pgfqpoint{2.524988in}{1.976977in}}%
\pgfpathlineto{\pgfqpoint{2.527560in}{1.966733in}}%
\pgfpathlineto{\pgfqpoint{2.530234in}{1.972016in}}%
\pgfpathlineto{\pgfqpoint{2.532917in}{1.969938in}}%
\pgfpathlineto{\pgfqpoint{2.535624in}{1.968863in}}%
\pgfpathlineto{\pgfqpoint{2.538274in}{1.974269in}}%
\pgfpathlineto{\pgfqpoint{2.540949in}{1.970158in}}%
\pgfpathlineto{\pgfqpoint{2.543765in}{1.971280in}}%
\pgfpathlineto{\pgfqpoint{2.546310in}{1.970245in}}%
\pgfpathlineto{\pgfqpoint{2.549114in}{1.977582in}}%
\pgfpathlineto{\pgfqpoint{2.551664in}{1.989563in}}%
\pgfpathlineto{\pgfqpoint{2.554493in}{2.008008in}}%
\pgfpathlineto{\pgfqpoint{2.557009in}{1.996615in}}%
\pgfpathlineto{\pgfqpoint{2.559790in}{1.985899in}}%
\pgfpathlineto{\pgfqpoint{2.562375in}{1.992243in}}%
\pgfpathlineto{\pgfqpoint{2.565045in}{1.994813in}}%
\pgfpathlineto{\pgfqpoint{2.567730in}{2.001060in}}%
\pgfpathlineto{\pgfqpoint{2.570411in}{1.989015in}}%
\pgfpathlineto{\pgfqpoint{2.573082in}{1.986346in}}%
\pgfpathlineto{\pgfqpoint{2.575779in}{1.974123in}}%
\pgfpathlineto{\pgfqpoint{2.578567in}{1.980357in}}%
\pgfpathlineto{\pgfqpoint{2.581129in}{1.976372in}}%
\pgfpathlineto{\pgfqpoint{2.583913in}{1.963929in}}%
\pgfpathlineto{\pgfqpoint{2.586484in}{1.974565in}}%
\pgfpathlineto{\pgfqpoint{2.589248in}{1.975189in}}%
\pgfpathlineto{\pgfqpoint{2.591842in}{1.983078in}}%
\pgfpathlineto{\pgfqpoint{2.594630in}{1.975519in}}%
\pgfpathlineto{\pgfqpoint{2.597196in}{1.973602in}}%
\pgfpathlineto{\pgfqpoint{2.599920in}{1.975076in}}%
\pgfpathlineto{\pgfqpoint{2.602557in}{1.970361in}}%
\pgfpathlineto{\pgfqpoint{2.605232in}{1.973788in}}%
\pgfpathlineto{\pgfqpoint{2.608004in}{1.979456in}}%
\pgfpathlineto{\pgfqpoint{2.610588in}{1.975890in}}%
\pgfpathlineto{\pgfqpoint{2.613393in}{1.974154in}}%
\pgfpathlineto{\pgfqpoint{2.615934in}{1.975087in}}%
\pgfpathlineto{\pgfqpoint{2.618773in}{1.978170in}}%
\pgfpathlineto{\pgfqpoint{2.621304in}{1.970666in}}%
\pgfpathlineto{\pgfqpoint{2.624077in}{1.974246in}}%
\pgfpathlineto{\pgfqpoint{2.626653in}{1.970592in}}%
\pgfpathlineto{\pgfqpoint{2.629340in}{1.975030in}}%
\pgfpathlineto{\pgfqpoint{2.632018in}{1.975605in}}%
\pgfpathlineto{\pgfqpoint{2.634700in}{1.972830in}}%
\pgfpathlineto{\pgfqpoint{2.637369in}{1.982884in}}%
\pgfpathlineto{\pgfqpoint{2.640053in}{1.980699in}}%
\pgfpathlineto{\pgfqpoint{2.642827in}{1.983043in}}%
\pgfpathlineto{\pgfqpoint{2.645408in}{1.990677in}}%
\pgfpathlineto{\pgfqpoint{2.648196in}{1.987186in}}%
\pgfpathlineto{\pgfqpoint{2.650767in}{1.998722in}}%
\pgfpathlineto{\pgfqpoint{2.653567in}{1.990796in}}%
\pgfpathlineto{\pgfqpoint{2.656124in}{1.993833in}}%
\pgfpathlineto{\pgfqpoint{2.658942in}{1.989933in}}%
\pgfpathlineto{\pgfqpoint{2.661481in}{1.990575in}}%
\pgfpathlineto{\pgfqpoint{2.664151in}{1.999029in}}%
\pgfpathlineto{\pgfqpoint{2.666836in}{1.982657in}}%
\pgfpathlineto{\pgfqpoint{2.669506in}{1.978229in}}%
\pgfpathlineto{\pgfqpoint{2.672301in}{1.977602in}}%
\pgfpathlineto{\pgfqpoint{2.674873in}{1.973240in}}%
\pgfpathlineto{\pgfqpoint{2.677650in}{1.976926in}}%
\pgfpathlineto{\pgfqpoint{2.680224in}{1.978458in}}%
\pgfpathlineto{\pgfqpoint{2.683009in}{1.977945in}}%
\pgfpathlineto{\pgfqpoint{2.685586in}{1.979716in}}%
\pgfpathlineto{\pgfqpoint{2.688328in}{1.978635in}}%
\pgfpathlineto{\pgfqpoint{2.690940in}{1.974222in}}%
\pgfpathlineto{\pgfqpoint{2.693611in}{1.974199in}}%
\pgfpathlineto{\pgfqpoint{2.696293in}{1.973211in}}%
\pgfpathlineto{\pgfqpoint{2.698968in}{1.978013in}}%
\pgfpathlineto{\pgfqpoint{2.701657in}{1.977669in}}%
\pgfpathlineto{\pgfqpoint{2.704326in}{1.977049in}}%
\pgfpathlineto{\pgfqpoint{2.707125in}{1.978275in}}%
\pgfpathlineto{\pgfqpoint{2.709683in}{1.974881in}}%
\pgfpathlineto{\pgfqpoint{2.712477in}{1.983855in}}%
\pgfpathlineto{\pgfqpoint{2.715036in}{1.983577in}}%
\pgfpathlineto{\pgfqpoint{2.717773in}{1.987134in}}%
\pgfpathlineto{\pgfqpoint{2.720404in}{1.985647in}}%
\pgfpathlineto{\pgfqpoint{2.723211in}{1.986397in}}%
\pgfpathlineto{\pgfqpoint{2.725760in}{1.978308in}}%
\pgfpathlineto{\pgfqpoint{2.728439in}{1.977119in}}%
\pgfpathlineto{\pgfqpoint{2.731119in}{1.975816in}}%
\pgfpathlineto{\pgfqpoint{2.733798in}{1.974030in}}%
\pgfpathlineto{\pgfqpoint{2.736476in}{1.970695in}}%
\pgfpathlineto{\pgfqpoint{2.739155in}{1.977548in}}%
\pgfpathlineto{\pgfqpoint{2.741928in}{1.972131in}}%
\pgfpathlineto{\pgfqpoint{2.744510in}{1.975882in}}%
\pgfpathlineto{\pgfqpoint{2.747260in}{1.974128in}}%
\pgfpathlineto{\pgfqpoint{2.749868in}{1.976027in}}%
\pgfpathlineto{\pgfqpoint{2.752614in}{1.978701in}}%
\pgfpathlineto{\pgfqpoint{2.755224in}{1.980240in}}%
\pgfpathlineto{\pgfqpoint{2.758028in}{1.979834in}}%
\pgfpathlineto{\pgfqpoint{2.760581in}{1.971850in}}%
\pgfpathlineto{\pgfqpoint{2.763253in}{1.972562in}}%
\pgfpathlineto{\pgfqpoint{2.765935in}{1.969047in}}%
\pgfpathlineto{\pgfqpoint{2.768617in}{1.971398in}}%
\pgfpathlineto{\pgfqpoint{2.771373in}{1.969449in}}%
\pgfpathlineto{\pgfqpoint{2.773972in}{1.972910in}}%
\pgfpathlineto{\pgfqpoint{2.776767in}{1.973149in}}%
\pgfpathlineto{\pgfqpoint{2.779330in}{1.968994in}}%
\pgfpathlineto{\pgfqpoint{2.782113in}{1.973108in}}%
\pgfpathlineto{\pgfqpoint{2.784687in}{1.966000in}}%
\pgfpathlineto{\pgfqpoint{2.787468in}{1.979193in}}%
\pgfpathlineto{\pgfqpoint{2.790044in}{1.966974in}}%
\pgfpathlineto{\pgfqpoint{2.792721in}{1.969632in}}%
\pgfpathlineto{\pgfqpoint{2.795398in}{1.970508in}}%
\pgfpathlineto{\pgfqpoint{2.798070in}{1.968883in}}%
\pgfpathlineto{\pgfqpoint{2.800756in}{1.977090in}}%
\pgfpathlineto{\pgfqpoint{2.803435in}{1.974555in}}%
\pgfpathlineto{\pgfqpoint{2.806175in}{1.972487in}}%
\pgfpathlineto{\pgfqpoint{2.808792in}{1.967494in}}%
\pgfpathlineto{\pgfqpoint{2.811597in}{1.975631in}}%
\pgfpathlineto{\pgfqpoint{2.814141in}{1.975468in}}%
\pgfpathlineto{\pgfqpoint{2.816867in}{1.970903in}}%
\pgfpathlineto{\pgfqpoint{2.819506in}{1.967979in}}%
\pgfpathlineto{\pgfqpoint{2.822303in}{1.974399in}}%
\pgfpathlineto{\pgfqpoint{2.824851in}{1.981364in}}%
\pgfpathlineto{\pgfqpoint{2.827567in}{1.974020in}}%
\pgfpathlineto{\pgfqpoint{2.830219in}{1.977304in}}%
\pgfpathlineto{\pgfqpoint{2.832894in}{1.978137in}}%
\pgfpathlineto{\pgfqpoint{2.835698in}{1.979065in}}%
\pgfpathlineto{\pgfqpoint{2.838254in}{1.980292in}}%
\pgfpathlineto{\pgfqpoint{2.841055in}{1.991064in}}%
\pgfpathlineto{\pgfqpoint{2.843611in}{1.986907in}}%
\pgfpathlineto{\pgfqpoint{2.846408in}{1.985397in}}%
\pgfpathlineto{\pgfqpoint{2.848960in}{1.986641in}}%
\pgfpathlineto{\pgfqpoint{2.851793in}{1.981992in}}%
\pgfpathlineto{\pgfqpoint{2.854325in}{1.981520in}}%
\pgfpathlineto{\pgfqpoint{2.857003in}{1.976404in}}%
\pgfpathlineto{\pgfqpoint{2.859668in}{1.978195in}}%
\pgfpathlineto{\pgfqpoint{2.862402in}{1.973296in}}%
\pgfpathlineto{\pgfqpoint{2.865031in}{1.972707in}}%
\pgfpathlineto{\pgfqpoint{2.867713in}{1.971821in}}%
\pgfpathlineto{\pgfqpoint{2.870475in}{1.971807in}}%
\pgfpathlineto{\pgfqpoint{2.873074in}{1.976812in}}%
\pgfpathlineto{\pgfqpoint{2.875882in}{1.970243in}}%
\pgfpathlineto{\pgfqpoint{2.878431in}{1.978363in}}%
\pgfpathlineto{\pgfqpoint{2.881254in}{1.976678in}}%
\pgfpathlineto{\pgfqpoint{2.883780in}{1.969721in}}%
\pgfpathlineto{\pgfqpoint{2.886578in}{1.968696in}}%
\pgfpathlineto{\pgfqpoint{2.889145in}{1.968537in}}%
\pgfpathlineto{\pgfqpoint{2.891809in}{1.974241in}}%
\pgfpathlineto{\pgfqpoint{2.894487in}{1.973455in}}%
\pgfpathlineto{\pgfqpoint{2.897179in}{1.971888in}}%
\pgfpathlineto{\pgfqpoint{2.899858in}{1.975647in}}%
\pgfpathlineto{\pgfqpoint{2.902535in}{1.975555in}}%
\pgfpathlineto{\pgfqpoint{2.905341in}{1.976647in}}%
\pgfpathlineto{\pgfqpoint{2.907882in}{1.976196in}}%
\pgfpathlineto{\pgfqpoint{2.910631in}{1.973944in}}%
\pgfpathlineto{\pgfqpoint{2.913243in}{1.973661in}}%
\pgfpathlineto{\pgfqpoint{2.916061in}{1.978013in}}%
\pgfpathlineto{\pgfqpoint{2.918606in}{1.974247in}}%
\pgfpathlineto{\pgfqpoint{2.921363in}{1.979829in}}%
\pgfpathlineto{\pgfqpoint{2.923963in}{1.977816in}}%
\pgfpathlineto{\pgfqpoint{2.926655in}{1.979422in}}%
\pgfpathlineto{\pgfqpoint{2.929321in}{1.978366in}}%
\pgfpathlineto{\pgfqpoint{2.932033in}{1.983978in}}%
\pgfpathlineto{\pgfqpoint{2.934759in}{1.986085in}}%
\pgfpathlineto{\pgfqpoint{2.937352in}{1.978062in}}%
\pgfpathlineto{\pgfqpoint{2.940120in}{1.972971in}}%
\pgfpathlineto{\pgfqpoint{2.942711in}{1.975681in}}%
\pgfpathlineto{\pgfqpoint{2.945461in}{1.973799in}}%
\pgfpathlineto{\pgfqpoint{2.948068in}{1.975921in}}%
\pgfpathlineto{\pgfqpoint{2.950884in}{1.975827in}}%
\pgfpathlineto{\pgfqpoint{2.953422in}{1.974348in}}%
\pgfpathlineto{\pgfqpoint{2.956103in}{1.974520in}}%
\pgfpathlineto{\pgfqpoint{2.958782in}{1.974675in}}%
\pgfpathlineto{\pgfqpoint{2.961460in}{1.970680in}}%
\pgfpathlineto{\pgfqpoint{2.964127in}{1.974951in}}%
\pgfpathlineto{\pgfqpoint{2.966812in}{1.973378in}}%
\pgfpathlineto{\pgfqpoint{2.969599in}{1.976340in}}%
\pgfpathlineto{\pgfqpoint{2.972177in}{1.977957in}}%
\pgfpathlineto{\pgfqpoint{2.974972in}{1.972748in}}%
\pgfpathlineto{\pgfqpoint{2.977517in}{1.976474in}}%
\pgfpathlineto{\pgfqpoint{2.980341in}{1.979736in}}%
\pgfpathlineto{\pgfqpoint{2.982885in}{1.981509in}}%
\pgfpathlineto{\pgfqpoint{2.985666in}{1.976767in}}%
\pgfpathlineto{\pgfqpoint{2.988238in}{1.985372in}}%
\pgfpathlineto{\pgfqpoint{2.990978in}{1.989801in}}%
\pgfpathlineto{\pgfqpoint{2.993595in}{1.983641in}}%
\pgfpathlineto{\pgfqpoint{2.996300in}{1.979177in}}%
\pgfpathlineto{\pgfqpoint{2.999103in}{1.975121in}}%
\pgfpathlineto{\pgfqpoint{3.001635in}{1.975133in}}%
\pgfpathlineto{\pgfqpoint{3.004419in}{1.973021in}}%
\pgfpathlineto{\pgfqpoint{3.006993in}{1.980986in}}%
\pgfpathlineto{\pgfqpoint{3.009784in}{1.975597in}}%
\pgfpathlineto{\pgfqpoint{3.012351in}{1.983168in}}%
\pgfpathlineto{\pgfqpoint{3.015097in}{1.982357in}}%
\pgfpathlineto{\pgfqpoint{3.017707in}{1.980709in}}%
\pgfpathlineto{\pgfqpoint{3.020382in}{1.984138in}}%
\pgfpathlineto{\pgfqpoint{3.023058in}{1.979283in}}%
\pgfpathlineto{\pgfqpoint{3.025803in}{1.983450in}}%
\pgfpathlineto{\pgfqpoint{3.028412in}{1.983797in}}%
\pgfpathlineto{\pgfqpoint{3.031091in}{1.981938in}}%
\pgfpathlineto{\pgfqpoint{3.033921in}{1.977976in}}%
\pgfpathlineto{\pgfqpoint{3.036456in}{1.977024in}}%
\pgfpathlineto{\pgfqpoint{3.039262in}{1.978311in}}%
\pgfpathlineto{\pgfqpoint{3.041813in}{1.975307in}}%
\pgfpathlineto{\pgfqpoint{3.044568in}{1.976285in}}%
\pgfpathlineto{\pgfqpoint{3.047157in}{1.974214in}}%
\pgfpathlineto{\pgfqpoint{3.049988in}{1.984587in}}%
\pgfpathlineto{\pgfqpoint{3.052526in}{1.993112in}}%
\pgfpathlineto{\pgfqpoint{3.055202in}{1.978651in}}%
\pgfpathlineto{\pgfqpoint{3.057884in}{1.978228in}}%
\pgfpathlineto{\pgfqpoint{3.060561in}{1.972674in}}%
\pgfpathlineto{\pgfqpoint{3.063230in}{1.971792in}}%
\pgfpathlineto{\pgfqpoint{3.065916in}{1.973852in}}%
\pgfpathlineto{\pgfqpoint{3.068709in}{1.968398in}}%
\pgfpathlineto{\pgfqpoint{3.071266in}{1.979958in}}%
\pgfpathlineto{\pgfqpoint{3.074056in}{1.975820in}}%
\pgfpathlineto{\pgfqpoint{3.076631in}{1.973964in}}%
\pgfpathlineto{\pgfqpoint{3.079381in}{1.976553in}}%
\pgfpathlineto{\pgfqpoint{3.081990in}{1.977435in}}%
\pgfpathlineto{\pgfqpoint{3.084671in}{1.970168in}}%
\pgfpathlineto{\pgfqpoint{3.087343in}{1.968025in}}%
\pgfpathlineto{\pgfqpoint{3.090023in}{1.973138in}}%
\pgfpathlineto{\pgfqpoint{3.092699in}{1.980587in}}%
\pgfpathlineto{\pgfqpoint{3.095388in}{1.978415in}}%
\pgfpathlineto{\pgfqpoint{3.098163in}{1.982672in}}%
\pgfpathlineto{\pgfqpoint{3.100737in}{1.981500in}}%
\pgfpathlineto{\pgfqpoint{3.103508in}{1.983823in}}%
\pgfpathlineto{\pgfqpoint{3.106094in}{1.967332in}}%
\pgfpathlineto{\pgfqpoint{3.108896in}{1.973945in}}%
\pgfpathlineto{\pgfqpoint{3.111451in}{1.965480in}}%
\pgfpathlineto{\pgfqpoint{3.114242in}{1.976545in}}%
\pgfpathlineto{\pgfqpoint{3.116807in}{1.976239in}}%
\pgfpathlineto{\pgfqpoint{3.119487in}{1.972829in}}%
\pgfpathlineto{\pgfqpoint{3.122163in}{1.963929in}}%
\pgfpathlineto{\pgfqpoint{3.124842in}{1.963929in}}%
\pgfpathlineto{\pgfqpoint{3.127512in}{1.968094in}}%
\pgfpathlineto{\pgfqpoint{3.130199in}{1.973614in}}%
\pgfpathlineto{\pgfqpoint{3.132946in}{1.974466in}}%
\pgfpathlineto{\pgfqpoint{3.135550in}{1.970743in}}%
\pgfpathlineto{\pgfqpoint{3.138375in}{1.968883in}}%
\pgfpathlineto{\pgfqpoint{3.140913in}{1.963929in}}%
\pgfpathlineto{\pgfqpoint{3.143740in}{1.963929in}}%
\pgfpathlineto{\pgfqpoint{3.146271in}{1.963929in}}%
\pgfpathlineto{\pgfqpoint{3.149057in}{1.963929in}}%
\pgfpathlineto{\pgfqpoint{3.151612in}{1.963929in}}%
\pgfpathlineto{\pgfqpoint{3.154327in}{1.968716in}}%
\pgfpathlineto{\pgfqpoint{3.156981in}{1.967285in}}%
\pgfpathlineto{\pgfqpoint{3.159675in}{1.972345in}}%
\pgfpathlineto{\pgfqpoint{3.162474in}{1.969347in}}%
\pgfpathlineto{\pgfqpoint{3.165019in}{1.964468in}}%
\pgfpathlineto{\pgfqpoint{3.167776in}{1.963929in}}%
\pgfpathlineto{\pgfqpoint{3.170375in}{1.963929in}}%
\pgfpathlineto{\pgfqpoint{3.173142in}{1.963929in}}%
\pgfpathlineto{\pgfqpoint{3.175724in}{1.963929in}}%
\pgfpathlineto{\pgfqpoint{3.178525in}{1.963929in}}%
\pgfpathlineto{\pgfqpoint{3.181089in}{1.967271in}}%
\pgfpathlineto{\pgfqpoint{3.183760in}{1.966666in}}%
\pgfpathlineto{\pgfqpoint{3.186440in}{1.963929in}}%
\pgfpathlineto{\pgfqpoint{3.189117in}{1.964193in}}%
\pgfpathlineto{\pgfqpoint{3.191796in}{1.963929in}}%
\pgfpathlineto{\pgfqpoint{3.194508in}{1.967634in}}%
\pgfpathlineto{\pgfqpoint{3.197226in}{1.970936in}}%
\pgfpathlineto{\pgfqpoint{3.199823in}{1.963929in}}%
\pgfpathlineto{\pgfqpoint{3.202562in}{1.963929in}}%
\pgfpathlineto{\pgfqpoint{3.205195in}{1.963929in}}%
\pgfpathlineto{\pgfqpoint{3.207984in}{1.967877in}}%
\pgfpathlineto{\pgfqpoint{3.210545in}{1.965671in}}%
\pgfpathlineto{\pgfqpoint{3.213342in}{1.963929in}}%
\pgfpathlineto{\pgfqpoint{3.215908in}{1.976483in}}%
\pgfpathlineto{\pgfqpoint{3.218586in}{1.975748in}}%
\pgfpathlineto{\pgfqpoint{3.221255in}{1.968825in}}%
\pgfpathlineto{\pgfqpoint{3.223942in}{1.963929in}}%
\pgfpathlineto{\pgfqpoint{3.226609in}{1.964812in}}%
\pgfpathlineto{\pgfqpoint{3.229310in}{1.970542in}}%
\pgfpathlineto{\pgfqpoint{3.232069in}{1.980618in}}%
\pgfpathlineto{\pgfqpoint{3.234658in}{1.977779in}}%
\pgfpathlineto{\pgfqpoint{3.237411in}{1.972612in}}%
\pgfpathlineto{\pgfqpoint{3.240010in}{1.972487in}}%
\pgfpathlineto{\pgfqpoint{3.242807in}{1.975110in}}%
\pgfpathlineto{\pgfqpoint{3.245363in}{1.976474in}}%
\pgfpathlineto{\pgfqpoint{3.248049in}{1.979821in}}%
\pgfpathlineto{\pgfqpoint{3.250716in}{1.975845in}}%
\pgfpathlineto{\pgfqpoint{3.253404in}{1.974357in}}%
\pgfpathlineto{\pgfqpoint{3.256083in}{1.980170in}}%
\pgfpathlineto{\pgfqpoint{3.258784in}{1.977986in}}%
\pgfpathlineto{\pgfqpoint{3.261594in}{1.985949in}}%
\pgfpathlineto{\pgfqpoint{3.264119in}{1.984775in}}%
\pgfpathlineto{\pgfqpoint{3.266849in}{1.989461in}}%
\pgfpathlineto{\pgfqpoint{3.269478in}{1.990981in}}%
\pgfpathlineto{\pgfqpoint{3.272254in}{1.992163in}}%
\pgfpathlineto{\pgfqpoint{3.274831in}{1.985627in}}%
\pgfpathlineto{\pgfqpoint{3.277603in}{1.987488in}}%
\pgfpathlineto{\pgfqpoint{3.280189in}{1.986494in}}%
\pgfpathlineto{\pgfqpoint{3.282870in}{1.980387in}}%
\pgfpathlineto{\pgfqpoint{3.285534in}{1.981055in}}%
\pgfpathlineto{\pgfqpoint{3.288225in}{1.974065in}}%
\pgfpathlineto{\pgfqpoint{3.290890in}{1.986635in}}%
\pgfpathlineto{\pgfqpoint{3.293574in}{1.980832in}}%
\pgfpathlineto{\pgfqpoint{3.296376in}{1.980769in}}%
\pgfpathlineto{\pgfqpoint{3.298937in}{1.987117in}}%
\pgfpathlineto{\pgfqpoint{3.301719in}{1.984792in}}%
\pgfpathlineto{\pgfqpoint{3.304295in}{1.982669in}}%
\pgfpathlineto{\pgfqpoint{3.307104in}{1.983118in}}%
\pgfpathlineto{\pgfqpoint{3.309652in}{1.981338in}}%
\pgfpathlineto{\pgfqpoint{3.312480in}{1.983699in}}%
\pgfpathlineto{\pgfqpoint{3.315008in}{1.988396in}}%
\pgfpathlineto{\pgfqpoint{3.317688in}{1.988929in}}%
\pgfpathlineto{\pgfqpoint{3.320366in}{1.984878in}}%
\pgfpathlineto{\pgfqpoint{3.323049in}{1.980920in}}%
\pgfpathlineto{\pgfqpoint{3.325860in}{1.982209in}}%
\pgfpathlineto{\pgfqpoint{3.328401in}{1.977972in}}%
\pgfpathlineto{\pgfqpoint{3.331183in}{1.980105in}}%
\pgfpathlineto{\pgfqpoint{3.333758in}{1.985095in}}%
\pgfpathlineto{\pgfqpoint{3.336541in}{1.982650in}}%
\pgfpathlineto{\pgfqpoint{3.339101in}{1.982751in}}%
\pgfpathlineto{\pgfqpoint{3.341893in}{1.980161in}}%
\pgfpathlineto{\pgfqpoint{3.344468in}{1.983202in}}%
\pgfpathlineto{\pgfqpoint{3.347139in}{1.982788in}}%
\pgfpathlineto{\pgfqpoint{3.349828in}{1.979425in}}%
\pgfpathlineto{\pgfqpoint{3.352505in}{1.978371in}}%
\pgfpathlineto{\pgfqpoint{3.355177in}{1.981396in}}%
\pgfpathlineto{\pgfqpoint{3.357862in}{1.985555in}}%
\pgfpathlineto{\pgfqpoint{3.360620in}{1.983024in}}%
\pgfpathlineto{\pgfqpoint{3.363221in}{1.986441in}}%
\pgfpathlineto{\pgfqpoint{3.365996in}{1.979655in}}%
\pgfpathlineto{\pgfqpoint{3.368577in}{1.975141in}}%
\pgfpathlineto{\pgfqpoint{3.371357in}{1.977808in}}%
\pgfpathlineto{\pgfqpoint{3.373921in}{1.978234in}}%
\pgfpathlineto{\pgfqpoint{3.376735in}{1.972278in}}%
\pgfpathlineto{\pgfqpoint{3.379290in}{1.978482in}}%
\pgfpathlineto{\pgfqpoint{3.381959in}{1.975261in}}%
\pgfpathlineto{\pgfqpoint{3.384647in}{1.973338in}}%
\pgfpathlineto{\pgfqpoint{3.387309in}{1.979945in}}%
\pgfpathlineto{\pgfqpoint{3.390102in}{1.975911in}}%
\pgfpathlineto{\pgfqpoint{3.392681in}{1.977349in}}%
\pgfpathlineto{\pgfqpoint{3.395461in}{1.980047in}}%
\pgfpathlineto{\pgfqpoint{3.398037in}{1.985110in}}%
\pgfpathlineto{\pgfqpoint{3.400783in}{1.978733in}}%
\pgfpathlineto{\pgfqpoint{3.403394in}{1.978177in}}%
\pgfpathlineto{\pgfqpoint{3.406202in}{1.978773in}}%
\pgfpathlineto{\pgfqpoint{3.408752in}{1.984201in}}%
\pgfpathlineto{\pgfqpoint{3.411431in}{1.986920in}}%
\pgfpathlineto{\pgfqpoint{3.414109in}{1.983658in}}%
\pgfpathlineto{\pgfqpoint{3.416780in}{1.978735in}}%
\pgfpathlineto{\pgfqpoint{3.419455in}{1.988106in}}%
\pgfpathlineto{\pgfqpoint{3.422142in}{1.988192in}}%
\pgfpathlineto{\pgfqpoint{3.424887in}{1.985563in}}%
\pgfpathlineto{\pgfqpoint{3.427501in}{1.980721in}}%
\pgfpathlineto{\pgfqpoint{3.430313in}{1.982582in}}%
\pgfpathlineto{\pgfqpoint{3.432851in}{1.984821in}}%
\pgfpathlineto{\pgfqpoint{3.435635in}{1.981223in}}%
\pgfpathlineto{\pgfqpoint{3.438210in}{1.984324in}}%
\pgfpathlineto{\pgfqpoint{3.440996in}{1.982634in}}%
\pgfpathlineto{\pgfqpoint{3.443574in}{1.986690in}}%
\pgfpathlineto{\pgfqpoint{3.446257in}{1.992531in}}%
\pgfpathlineto{\pgfqpoint{3.448926in}{1.985036in}}%
\pgfpathlineto{\pgfqpoint{3.451597in}{1.987504in}}%
\pgfpathlineto{\pgfqpoint{3.454285in}{1.981245in}}%
\pgfpathlineto{\pgfqpoint{3.456960in}{1.981786in}}%
\pgfpathlineto{\pgfqpoint{3.459695in}{1.980536in}}%
\pgfpathlineto{\pgfqpoint{3.462321in}{1.982299in}}%
\pgfpathlineto{\pgfqpoint{3.465072in}{1.980503in}}%
\pgfpathlineto{\pgfqpoint{3.467678in}{1.983907in}}%
\pgfpathlineto{\pgfqpoint{3.470466in}{1.987354in}}%
\pgfpathlineto{\pgfqpoint{3.473021in}{1.990262in}}%
\pgfpathlineto{\pgfqpoint{3.475821in}{1.990802in}}%
\pgfpathlineto{\pgfqpoint{3.478378in}{1.985090in}}%
\pgfpathlineto{\pgfqpoint{3.481072in}{1.986676in}}%
\pgfpathlineto{\pgfqpoint{3.483744in}{1.987726in}}%
\pgfpathlineto{\pgfqpoint{3.486442in}{1.985380in}}%
\pgfpathlineto{\pgfqpoint{3.489223in}{1.983594in}}%
\pgfpathlineto{\pgfqpoint{3.491783in}{1.987756in}}%
\pgfpathlineto{\pgfqpoint{3.494581in}{1.987737in}}%
\pgfpathlineto{\pgfqpoint{3.497139in}{1.985403in}}%
\pgfpathlineto{\pgfqpoint{3.499909in}{1.982865in}}%
\pgfpathlineto{\pgfqpoint{3.502488in}{1.983515in}}%
\pgfpathlineto{\pgfqpoint{3.505262in}{1.981199in}}%
\pgfpathlineto{\pgfqpoint{3.507840in}{1.982068in}}%
\pgfpathlineto{\pgfqpoint{3.510533in}{1.998777in}}%
\pgfpathlineto{\pgfqpoint{3.513209in}{1.991297in}}%
\pgfpathlineto{\pgfqpoint{3.515884in}{1.982022in}}%
\pgfpathlineto{\pgfqpoint{3.518565in}{1.983801in}}%
\pgfpathlineto{\pgfqpoint{3.521244in}{1.978542in}}%
\pgfpathlineto{\pgfqpoint{3.524041in}{1.976963in}}%
\pgfpathlineto{\pgfqpoint{3.526601in}{1.977088in}}%
\pgfpathlineto{\pgfqpoint{3.529327in}{1.975161in}}%
\pgfpathlineto{\pgfqpoint{3.531955in}{1.979266in}}%
\pgfpathlineto{\pgfqpoint{3.534783in}{1.976251in}}%
\pgfpathlineto{\pgfqpoint{3.537309in}{1.980198in}}%
\pgfpathlineto{\pgfqpoint{3.540093in}{1.973572in}}%
\pgfpathlineto{\pgfqpoint{3.542656in}{1.983144in}}%
\pgfpathlineto{\pgfqpoint{3.545349in}{1.988938in}}%
\pgfpathlineto{\pgfqpoint{3.548029in}{1.981226in}}%
\pgfpathlineto{\pgfqpoint{3.550713in}{1.977287in}}%
\pgfpathlineto{\pgfqpoint{3.553498in}{1.975359in}}%
\pgfpathlineto{\pgfqpoint{3.556061in}{1.979876in}}%
\pgfpathlineto{\pgfqpoint{3.558853in}{1.981995in}}%
\pgfpathlineto{\pgfqpoint{3.561420in}{1.980891in}}%
\pgfpathlineto{\pgfqpoint{3.564188in}{1.982467in}}%
\pgfpathlineto{\pgfqpoint{3.566774in}{1.979423in}}%
\pgfpathlineto{\pgfqpoint{3.569584in}{1.979253in}}%
\pgfpathlineto{\pgfqpoint{3.572126in}{1.970840in}}%
\pgfpathlineto{\pgfqpoint{3.574814in}{1.969915in}}%
\pgfpathlineto{\pgfqpoint{3.577487in}{1.978658in}}%
\pgfpathlineto{\pgfqpoint{3.580191in}{1.973031in}}%
\pgfpathlineto{\pgfqpoint{3.582851in}{1.977999in}}%
\pgfpathlineto{\pgfqpoint{3.585532in}{1.977881in}}%
\pgfpathlineto{\pgfqpoint{3.588258in}{1.975999in}}%
\pgfpathlineto{\pgfqpoint{3.590883in}{1.974407in}}%
\pgfpathlineto{\pgfqpoint{3.593620in}{1.977161in}}%
\pgfpathlineto{\pgfqpoint{3.596240in}{1.982901in}}%
\pgfpathlineto{\pgfqpoint{3.598998in}{1.974159in}}%
\pgfpathlineto{\pgfqpoint{3.601590in}{1.980045in}}%
\pgfpathlineto{\pgfqpoint{3.604387in}{1.981546in}}%
\pgfpathlineto{\pgfqpoint{3.606951in}{1.979628in}}%
\pgfpathlineto{\pgfqpoint{3.609632in}{1.985389in}}%
\pgfpathlineto{\pgfqpoint{3.612311in}{1.969758in}}%
\pgfpathlineto{\pgfqpoint{3.614982in}{1.975819in}}%
\pgfpathlineto{\pgfqpoint{3.617667in}{1.972817in}}%
\pgfpathlineto{\pgfqpoint{3.620345in}{1.970456in}}%
\pgfpathlineto{\pgfqpoint{3.623165in}{1.974978in}}%
\pgfpathlineto{\pgfqpoint{3.625689in}{1.976475in}}%
\pgfpathlineto{\pgfqpoint{3.628460in}{1.977734in}}%
\pgfpathlineto{\pgfqpoint{3.631058in}{1.972497in}}%
\pgfpathlineto{\pgfqpoint{3.633858in}{1.972098in}}%
\pgfpathlineto{\pgfqpoint{3.636413in}{1.974957in}}%
\pgfpathlineto{\pgfqpoint{3.639207in}{1.980590in}}%
\pgfpathlineto{\pgfqpoint{3.641773in}{1.989514in}}%
\pgfpathlineto{\pgfqpoint{3.644452in}{2.038130in}}%
\pgfpathlineto{\pgfqpoint{3.647130in}{2.048515in}}%
\pgfpathlineto{\pgfqpoint{3.649837in}{2.041610in}}%
\pgfpathlineto{\pgfqpoint{3.652628in}{2.025508in}}%
\pgfpathlineto{\pgfqpoint{3.655165in}{1.997777in}}%
\pgfpathlineto{\pgfqpoint{3.657917in}{1.964758in}}%
\pgfpathlineto{\pgfqpoint{3.660515in}{1.963929in}}%
\pgfpathlineto{\pgfqpoint{3.663276in}{1.963929in}}%
\pgfpathlineto{\pgfqpoint{3.665864in}{1.963929in}}%
\pgfpathlineto{\pgfqpoint{3.668665in}{1.965185in}}%
\pgfpathlineto{\pgfqpoint{3.671232in}{1.963929in}}%
\pgfpathlineto{\pgfqpoint{3.673911in}{1.967048in}}%
\pgfpathlineto{\pgfqpoint{3.676591in}{1.989823in}}%
\pgfpathlineto{\pgfqpoint{3.679273in}{2.050687in}}%
\pgfpathlineto{\pgfqpoint{3.681948in}{2.071911in}}%
\pgfpathlineto{\pgfqpoint{3.684620in}{2.079672in}}%
\pgfpathlineto{\pgfqpoint{3.687442in}{2.073409in}}%
\pgfpathlineto{\pgfqpoint{3.689983in}{2.053434in}}%
\pgfpathlineto{\pgfqpoint{3.692765in}{2.051098in}}%
\pgfpathlineto{\pgfqpoint{3.695331in}{2.026626in}}%
\pgfpathlineto{\pgfqpoint{3.698125in}{2.019375in}}%
\pgfpathlineto{\pgfqpoint{3.700684in}{2.013080in}}%
\pgfpathlineto{\pgfqpoint{3.703460in}{2.006814in}}%
\pgfpathlineto{\pgfqpoint{3.706053in}{2.009379in}}%
\pgfpathlineto{\pgfqpoint{3.708729in}{1.998475in}}%
\pgfpathlineto{\pgfqpoint{3.711410in}{1.992791in}}%
\pgfpathlineto{\pgfqpoint{3.714086in}{1.990900in}}%
\pgfpathlineto{\pgfqpoint{3.716875in}{1.988343in}}%
\pgfpathlineto{\pgfqpoint{3.719446in}{1.980406in}}%
\pgfpathlineto{\pgfqpoint{3.722228in}{1.976916in}}%
\pgfpathlineto{\pgfqpoint{3.724804in}{1.979845in}}%
\pgfpathlineto{\pgfqpoint{3.727581in}{1.981273in}}%
\pgfpathlineto{\pgfqpoint{3.730158in}{1.983058in}}%
\pgfpathlineto{\pgfqpoint{3.732950in}{1.985223in}}%
\pgfpathlineto{\pgfqpoint{3.735509in}{1.986439in}}%
\pgfpathlineto{\pgfqpoint{3.738194in}{1.986564in}}%
\pgfpathlineto{\pgfqpoint{3.740874in}{1.976433in}}%
\pgfpathlineto{\pgfqpoint{3.743548in}{1.984540in}}%
\pgfpathlineto{\pgfqpoint{3.746229in}{1.981705in}}%
\pgfpathlineto{\pgfqpoint{3.748903in}{1.978371in}}%
\pgfpathlineto{\pgfqpoint{3.751728in}{1.980310in}}%
\pgfpathlineto{\pgfqpoint{3.754265in}{1.981093in}}%
\pgfpathlineto{\pgfqpoint{3.757065in}{1.983610in}}%
\pgfpathlineto{\pgfqpoint{3.759622in}{1.979296in}}%
\pgfpathlineto{\pgfqpoint{3.762389in}{1.980166in}}%
\pgfpathlineto{\pgfqpoint{3.764966in}{1.987213in}}%
\pgfpathlineto{\pgfqpoint{3.767782in}{1.975545in}}%
\pgfpathlineto{\pgfqpoint{3.770323in}{1.981348in}}%
\pgfpathlineto{\pgfqpoint{3.773014in}{1.996778in}}%
\pgfpathlineto{\pgfqpoint{3.775691in}{2.020295in}}%
\pgfpathlineto{\pgfqpoint{3.778370in}{2.016931in}}%
\pgfpathlineto{\pgfqpoint{3.781046in}{2.012111in}}%
\pgfpathlineto{\pgfqpoint{3.783725in}{1.998712in}}%
\pgfpathlineto{\pgfqpoint{3.786504in}{1.983210in}}%
\pgfpathlineto{\pgfqpoint{3.789084in}{1.981049in}}%
\pgfpathlineto{\pgfqpoint{3.791897in}{1.984586in}}%
\pgfpathlineto{\pgfqpoint{3.794435in}{1.980883in}}%
\pgfpathlineto{\pgfqpoint{3.797265in}{1.977931in}}%
\pgfpathlineto{\pgfqpoint{3.799797in}{1.969805in}}%
\pgfpathlineto{\pgfqpoint{3.802569in}{1.977135in}}%
\pgfpathlineto{\pgfqpoint{3.805145in}{1.979625in}}%
\pgfpathlineto{\pgfqpoint{3.807832in}{1.981707in}}%
\pgfpathlineto{\pgfqpoint{3.810510in}{1.973994in}}%
\pgfpathlineto{\pgfqpoint{3.813172in}{1.968764in}}%
\pgfpathlineto{\pgfqpoint{3.815983in}{1.971943in}}%
\pgfpathlineto{\pgfqpoint{3.818546in}{1.973473in}}%
\pgfpathlineto{\pgfqpoint{3.821315in}{1.978499in}}%
\pgfpathlineto{\pgfqpoint{3.823903in}{1.981716in}}%
\pgfpathlineto{\pgfqpoint{3.826679in}{1.970671in}}%
\pgfpathlineto{\pgfqpoint{3.829252in}{1.975089in}}%
\pgfpathlineto{\pgfqpoint{3.832053in}{1.980293in}}%
\pgfpathlineto{\pgfqpoint{3.834616in}{1.983201in}}%
\pgfpathlineto{\pgfqpoint{3.837286in}{1.981006in}}%
\pgfpathlineto{\pgfqpoint{3.839960in}{1.985579in}}%
\pgfpathlineto{\pgfqpoint{3.842641in}{1.981292in}}%
\pgfpathlineto{\pgfqpoint{3.845329in}{1.982950in}}%
\pgfpathlineto{\pgfqpoint{3.848005in}{1.980827in}}%
\pgfpathlineto{\pgfqpoint{3.850814in}{1.976104in}}%
\pgfpathlineto{\pgfqpoint{3.853358in}{1.974918in}}%
\pgfpathlineto{\pgfqpoint{3.856100in}{1.972704in}}%
\pgfpathlineto{\pgfqpoint{3.858720in}{1.976299in}}%
\pgfpathlineto{\pgfqpoint{3.861561in}{1.974928in}}%
\pgfpathlineto{\pgfqpoint{3.864073in}{1.973277in}}%
\pgfpathlineto{\pgfqpoint{3.866815in}{1.975355in}}%
\pgfpathlineto{\pgfqpoint{3.869435in}{1.976382in}}%
\pgfpathlineto{\pgfqpoint{3.872114in}{1.979834in}}%
\pgfpathlineto{\pgfqpoint{3.874790in}{1.977673in}}%
\pgfpathlineto{\pgfqpoint{3.877466in}{1.984115in}}%
\pgfpathlineto{\pgfqpoint{3.880237in}{1.976712in}}%
\pgfpathlineto{\pgfqpoint{3.882850in}{1.978055in}}%
\pgfpathlineto{\pgfqpoint{3.885621in}{1.985746in}}%
\pgfpathlineto{\pgfqpoint{3.888188in}{1.980053in}}%
\pgfpathlineto{\pgfqpoint{3.890926in}{1.979864in}}%
\pgfpathlineto{\pgfqpoint{3.893541in}{1.975954in}}%
\pgfpathlineto{\pgfqpoint{3.896345in}{1.977089in}}%
\pgfpathlineto{\pgfqpoint{3.898891in}{1.976149in}}%
\pgfpathlineto{\pgfqpoint{3.901573in}{1.972765in}}%
\pgfpathlineto{\pgfqpoint{3.904252in}{1.973938in}}%
\pgfpathlineto{\pgfqpoint{3.906918in}{1.973390in}}%
\pgfpathlineto{\pgfqpoint{3.909602in}{1.978463in}}%
\pgfpathlineto{\pgfqpoint{3.912296in}{1.976494in}}%
\pgfpathlineto{\pgfqpoint{3.915107in}{1.981373in}}%
\pgfpathlineto{\pgfqpoint{3.917646in}{1.979058in}}%
\pgfpathlineto{\pgfqpoint{3.920412in}{1.979142in}}%
\pgfpathlineto{\pgfqpoint{3.923005in}{1.979452in}}%
\pgfpathlineto{\pgfqpoint{3.925778in}{1.981384in}}%
\pgfpathlineto{\pgfqpoint{3.928347in}{1.981381in}}%
\pgfpathlineto{\pgfqpoint{3.931202in}{1.985714in}}%
\pgfpathlineto{\pgfqpoint{3.933714in}{1.985775in}}%
\pgfpathlineto{\pgfqpoint{3.936395in}{1.984000in}}%
\pgfpathlineto{\pgfqpoint{3.939075in}{1.981941in}}%
\pgfpathlineto{\pgfqpoint{3.941778in}{1.975733in}}%
\pgfpathlineto{\pgfqpoint{3.944431in}{1.979374in}}%
\pgfpathlineto{\pgfqpoint{3.947101in}{1.976100in}}%
\pgfpathlineto{\pgfqpoint{3.949894in}{1.977155in}}%
\pgfpathlineto{\pgfqpoint{3.952464in}{1.970331in}}%
\pgfpathlineto{\pgfqpoint{3.955211in}{1.966024in}}%
\pgfpathlineto{\pgfqpoint{3.957823in}{1.970398in}}%
\pgfpathlineto{\pgfqpoint{3.960635in}{1.977542in}}%
\pgfpathlineto{\pgfqpoint{3.963176in}{1.975176in}}%
\pgfpathlineto{\pgfqpoint{3.966013in}{1.981000in}}%
\pgfpathlineto{\pgfqpoint{3.968523in}{1.976625in}}%
\pgfpathlineto{\pgfqpoint{3.971250in}{1.979438in}}%
\pgfpathlineto{\pgfqpoint{3.973885in}{1.975431in}}%
\pgfpathlineto{\pgfqpoint{3.976563in}{1.973438in}}%
\pgfpathlineto{\pgfqpoint{3.979389in}{1.974507in}}%
\pgfpathlineto{\pgfqpoint{3.981929in}{1.970949in}}%
\pgfpathlineto{\pgfqpoint{3.984714in}{1.971953in}}%
\pgfpathlineto{\pgfqpoint{3.987270in}{1.973601in}}%
\pgfpathlineto{\pgfqpoint{3.990055in}{1.971405in}}%
\pgfpathlineto{\pgfqpoint{3.992642in}{1.970273in}}%
\pgfpathlineto{\pgfqpoint{3.995417in}{1.975455in}}%
\pgfpathlineto{\pgfqpoint{3.997990in}{1.974484in}}%
\pgfpathlineto{\pgfqpoint{4.000674in}{1.975378in}}%
\pgfpathlineto{\pgfqpoint{4.003348in}{1.977486in}}%
\pgfpathlineto{\pgfqpoint{4.006034in}{1.979418in}}%
\pgfpathlineto{\pgfqpoint{4.008699in}{1.972115in}}%
\pgfpathlineto{\pgfqpoint{4.011394in}{1.980119in}}%
\pgfpathlineto{\pgfqpoint{4.014186in}{1.986694in}}%
\pgfpathlineto{\pgfqpoint{4.016744in}{1.983094in}}%
\pgfpathlineto{\pgfqpoint{4.019518in}{1.979725in}}%
\pgfpathlineto{\pgfqpoint{4.022097in}{1.980424in}}%
\pgfpathlineto{\pgfqpoint{4.024868in}{1.975175in}}%
\pgfpathlineto{\pgfqpoint{4.027447in}{1.977776in}}%
\pgfpathlineto{\pgfqpoint{4.030229in}{1.980315in}}%
\pgfpathlineto{\pgfqpoint{4.032817in}{1.979223in}}%
\pgfpathlineto{\pgfqpoint{4.035492in}{1.986345in}}%
\pgfpathlineto{\pgfqpoint{4.038174in}{1.982641in}}%
\pgfpathlineto{\pgfqpoint{4.040852in}{1.982884in}}%
\pgfpathlineto{\pgfqpoint{4.043667in}{1.983032in}}%
\pgfpathlineto{\pgfqpoint{4.046210in}{1.984582in}}%
\pgfpathlineto{\pgfqpoint{4.049006in}{1.985869in}}%
\pgfpathlineto{\pgfqpoint{4.051557in}{1.981594in}}%
\pgfpathlineto{\pgfqpoint{4.054326in}{1.980069in}}%
\pgfpathlineto{\pgfqpoint{4.056911in}{1.986049in}}%
\pgfpathlineto{\pgfqpoint{4.059702in}{1.985081in}}%
\pgfpathlineto{\pgfqpoint{4.062266in}{1.984600in}}%
\pgfpathlineto{\pgfqpoint{4.064957in}{1.987280in}}%
\pgfpathlineto{\pgfqpoint{4.067636in}{1.986734in}}%
\pgfpathlineto{\pgfqpoint{4.070313in}{1.983670in}}%
\pgfpathlineto{\pgfqpoint{4.072985in}{1.982763in}}%
\pgfpathlineto{\pgfqpoint{4.075705in}{1.982900in}}%
\pgfpathlineto{\pgfqpoint{4.078471in}{1.983682in}}%
\pgfpathlineto{\pgfqpoint{4.081018in}{1.982839in}}%
\pgfpathlineto{\pgfqpoint{4.083870in}{1.986482in}}%
\pgfpathlineto{\pgfqpoint{4.086385in}{1.985271in}}%
\pgfpathlineto{\pgfqpoint{4.089159in}{1.986044in}}%
\pgfpathlineto{\pgfqpoint{4.091729in}{1.988913in}}%
\pgfpathlineto{\pgfqpoint{4.094527in}{1.980865in}}%
\pgfpathlineto{\pgfqpoint{4.097092in}{1.983864in}}%
\pgfpathlineto{\pgfqpoint{4.099777in}{1.982753in}}%
\pgfpathlineto{\pgfqpoint{4.102456in}{1.980936in}}%
\pgfpathlineto{\pgfqpoint{4.105185in}{1.975807in}}%
\pgfpathlineto{\pgfqpoint{4.107814in}{1.974663in}}%
\pgfpathlineto{\pgfqpoint{4.110488in}{1.980854in}}%
\pgfpathlineto{\pgfqpoint{4.113252in}{1.978623in}}%
\pgfpathlineto{\pgfqpoint{4.115844in}{1.988111in}}%
\pgfpathlineto{\pgfqpoint{4.118554in}{1.985448in}}%
\pgfpathlineto{\pgfqpoint{4.121205in}{1.989550in}}%
\pgfpathlineto{\pgfqpoint{4.124019in}{1.981694in}}%
\pgfpathlineto{\pgfqpoint{4.126553in}{1.977728in}}%
\pgfpathlineto{\pgfqpoint{4.129349in}{1.978524in}}%
\pgfpathlineto{\pgfqpoint{4.131920in}{1.980534in}}%
\pgfpathlineto{\pgfqpoint{4.134615in}{1.981583in}}%
\pgfpathlineto{\pgfqpoint{4.137272in}{1.984615in}}%
\pgfpathlineto{\pgfqpoint{4.139963in}{1.988283in}}%
\pgfpathlineto{\pgfqpoint{4.142713in}{1.980573in}}%
\pgfpathlineto{\pgfqpoint{4.145310in}{1.982319in}}%
\pgfpathlineto{\pgfqpoint{4.148082in}{1.977905in}}%
\pgfpathlineto{\pgfqpoint{4.150665in}{1.985556in}}%
\pgfpathlineto{\pgfqpoint{4.153423in}{1.981852in}}%
\pgfpathlineto{\pgfqpoint{4.156016in}{1.981727in}}%
\pgfpathlineto{\pgfqpoint{4.158806in}{1.987336in}}%
\pgfpathlineto{\pgfqpoint{4.161380in}{1.975189in}}%
\pgfpathlineto{\pgfqpoint{4.164059in}{1.978254in}}%
\pgfpathlineto{\pgfqpoint{4.166737in}{1.971027in}}%
\pgfpathlineto{\pgfqpoint{4.169415in}{1.980936in}}%
\pgfpathlineto{\pgfqpoint{4.172093in}{1.986165in}}%
\pgfpathlineto{\pgfqpoint{4.174770in}{1.984763in}}%
\pgfpathlineto{\pgfqpoint{4.177593in}{1.985325in}}%
\pgfpathlineto{\pgfqpoint{4.180129in}{1.979405in}}%
\pgfpathlineto{\pgfqpoint{4.182899in}{1.974977in}}%
\pgfpathlineto{\pgfqpoint{4.185481in}{1.967116in}}%
\pgfpathlineto{\pgfqpoint{4.188318in}{1.971993in}}%
\pgfpathlineto{\pgfqpoint{4.190842in}{1.970578in}}%
\pgfpathlineto{\pgfqpoint{4.193638in}{1.970704in}}%
\pgfpathlineto{\pgfqpoint{4.196186in}{1.973720in}}%
\pgfpathlineto{\pgfqpoint{4.198878in}{1.972295in}}%
\pgfpathlineto{\pgfqpoint{4.201542in}{1.978379in}}%
\pgfpathlineto{\pgfqpoint{4.204240in}{1.980933in}}%
\pgfpathlineto{\pgfqpoint{4.207076in}{1.971473in}}%
\pgfpathlineto{\pgfqpoint{4.209597in}{1.970817in}}%
\pgfpathlineto{\pgfqpoint{4.212383in}{1.966210in}}%
\pgfpathlineto{\pgfqpoint{4.214948in}{1.975567in}}%
\pgfpathlineto{\pgfqpoint{4.217694in}{1.972190in}}%
\pgfpathlineto{\pgfqpoint{4.220304in}{1.978158in}}%
\pgfpathlineto{\pgfqpoint{4.223082in}{1.972445in}}%
\pgfpathlineto{\pgfqpoint{4.225654in}{1.972853in}}%
\pgfpathlineto{\pgfqpoint{4.228331in}{1.975586in}}%
\pgfpathlineto{\pgfqpoint{4.231013in}{1.981393in}}%
\pgfpathlineto{\pgfqpoint{4.233691in}{1.981961in}}%
\pgfpathlineto{\pgfqpoint{4.236375in}{1.981747in}}%
\pgfpathlineto{\pgfqpoint{4.239084in}{1.977893in}}%
\pgfpathlineto{\pgfqpoint{4.241900in}{1.976068in}}%
\pgfpathlineto{\pgfqpoint{4.244394in}{1.969753in}}%
\pgfpathlineto{\pgfqpoint{4.247225in}{1.976751in}}%
\pgfpathlineto{\pgfqpoint{4.249767in}{1.988569in}}%
\pgfpathlineto{\pgfqpoint{4.252581in}{1.976827in}}%
\pgfpathlineto{\pgfqpoint{4.255120in}{1.980320in}}%
\pgfpathlineto{\pgfqpoint{4.257958in}{1.983782in}}%
\pgfpathlineto{\pgfqpoint{4.260477in}{1.983438in}}%
\pgfpathlineto{\pgfqpoint{4.263157in}{1.985713in}}%
\pgfpathlineto{\pgfqpoint{4.265824in}{1.989326in}}%
\pgfpathlineto{\pgfqpoint{4.268590in}{1.988096in}}%
\pgfpathlineto{\pgfqpoint{4.271187in}{1.983869in}}%
\pgfpathlineto{\pgfqpoint{4.273874in}{1.988708in}}%
\pgfpathlineto{\pgfqpoint{4.276635in}{1.978809in}}%
\pgfpathlineto{\pgfqpoint{4.279212in}{1.989939in}}%
\pgfpathlineto{\pgfqpoint{4.282000in}{1.987899in}}%
\pgfpathlineto{\pgfqpoint{4.284586in}{1.995799in}}%
\pgfpathlineto{\pgfqpoint{4.287399in}{1.988395in}}%
\pgfpathlineto{\pgfqpoint{4.289936in}{1.986671in}}%
\pgfpathlineto{\pgfqpoint{4.292786in}{1.980313in}}%
\pgfpathlineto{\pgfqpoint{4.295299in}{1.979836in}}%
\pgfpathlineto{\pgfqpoint{4.297977in}{1.977337in}}%
\pgfpathlineto{\pgfqpoint{4.300656in}{1.982105in}}%
\pgfpathlineto{\pgfqpoint{4.303357in}{1.980566in}}%
\pgfpathlineto{\pgfqpoint{4.306118in}{1.975510in}}%
\pgfpathlineto{\pgfqpoint{4.308691in}{1.973710in}}%
\pgfpathlineto{\pgfqpoint{4.311494in}{1.977814in}}%
\pgfpathlineto{\pgfqpoint{4.314032in}{1.987198in}}%
\pgfpathlineto{\pgfqpoint{4.316856in}{1.976574in}}%
\pgfpathlineto{\pgfqpoint{4.319405in}{1.972448in}}%
\pgfpathlineto{\pgfqpoint{4.322181in}{1.972079in}}%
\pgfpathlineto{\pgfqpoint{4.324760in}{1.973694in}}%
\pgfpathlineto{\pgfqpoint{4.327440in}{1.981673in}}%
\pgfpathlineto{\pgfqpoint{4.330118in}{1.979125in}}%
\pgfpathlineto{\pgfqpoint{4.332796in}{1.970137in}}%
\pgfpathlineto{\pgfqpoint{4.335463in}{1.977981in}}%
\pgfpathlineto{\pgfqpoint{4.338154in}{1.984504in}}%
\pgfpathlineto{\pgfqpoint{4.340976in}{1.984348in}}%
\pgfpathlineto{\pgfqpoint{4.343510in}{1.977591in}}%
\pgfpathlineto{\pgfqpoint{4.346263in}{1.981469in}}%
\pgfpathlineto{\pgfqpoint{4.348868in}{1.974313in}}%
\pgfpathlineto{\pgfqpoint{4.351645in}{1.977060in}}%
\pgfpathlineto{\pgfqpoint{4.354224in}{1.977528in}}%
\pgfpathlineto{\pgfqpoint{4.357014in}{1.982718in}}%
\pgfpathlineto{\pgfqpoint{4.359582in}{1.983430in}}%
\pgfpathlineto{\pgfqpoint{4.362270in}{1.978614in}}%
\pgfpathlineto{\pgfqpoint{4.364936in}{1.979888in}}%
\pgfpathlineto{\pgfqpoint{4.367646in}{1.977951in}}%
\pgfpathlineto{\pgfqpoint{4.370437in}{1.972980in}}%
\pgfpathlineto{\pgfqpoint{4.372976in}{1.974128in}}%
\pgfpathlineto{\pgfqpoint{4.375761in}{1.982149in}}%
\pgfpathlineto{\pgfqpoint{4.378329in}{1.977713in}}%
\pgfpathlineto{\pgfqpoint{4.381097in}{1.974661in}}%
\pgfpathlineto{\pgfqpoint{4.383674in}{1.968760in}}%
\pgfpathlineto{\pgfqpoint{4.386431in}{1.973411in}}%
\pgfpathlineto{\pgfqpoint{4.389044in}{1.971681in}}%
\pgfpathlineto{\pgfqpoint{4.391721in}{1.971023in}}%
\pgfpathlineto{\pgfqpoint{4.394400in}{1.964377in}}%
\pgfpathlineto{\pgfqpoint{4.397076in}{1.966361in}}%
\pgfpathlineto{\pgfqpoint{4.399745in}{1.972049in}}%
\pgfpathlineto{\pgfqpoint{4.402468in}{1.982928in}}%
\pgfpathlineto{\pgfqpoint{4.405234in}{1.988056in}}%
\pgfpathlineto{\pgfqpoint{4.407788in}{1.989522in}}%
\pgfpathlineto{\pgfqpoint{4.410587in}{1.982846in}}%
\pgfpathlineto{\pgfqpoint{4.413149in}{1.978943in}}%
\pgfpathlineto{\pgfqpoint{4.415932in}{1.980970in}}%
\pgfpathlineto{\pgfqpoint{4.418506in}{1.989353in}}%
\pgfpathlineto{\pgfqpoint{4.421292in}{1.986137in}}%
\pgfpathlineto{\pgfqpoint{4.423863in}{1.986268in}}%
\pgfpathlineto{\pgfqpoint{4.426534in}{1.987792in}}%
\pgfpathlineto{\pgfqpoint{4.429220in}{1.979377in}}%
\pgfpathlineto{\pgfqpoint{4.431901in}{1.983344in}}%
\pgfpathlineto{\pgfqpoint{4.434569in}{1.984053in}}%
\pgfpathlineto{\pgfqpoint{4.437253in}{1.980205in}}%
\pgfpathlineto{\pgfqpoint{4.440041in}{1.984902in}}%
\pgfpathlineto{\pgfqpoint{4.442611in}{1.991479in}}%
\pgfpathlineto{\pgfqpoint{4.445423in}{1.986727in}}%
\pgfpathlineto{\pgfqpoint{4.447965in}{1.976705in}}%
\pgfpathlineto{\pgfqpoint{4.450767in}{1.973995in}}%
\pgfpathlineto{\pgfqpoint{4.453312in}{1.977303in}}%
\pgfpathlineto{\pgfqpoint{4.456138in}{1.976884in}}%
\pgfpathlineto{\pgfqpoint{4.458681in}{1.981200in}}%
\pgfpathlineto{\pgfqpoint{4.461367in}{1.976826in}}%
\pgfpathlineto{\pgfqpoint{4.464029in}{1.976726in}}%
\pgfpathlineto{\pgfqpoint{4.466717in}{1.978917in}}%
\pgfpathlineto{\pgfqpoint{4.469492in}{1.978521in}}%
\pgfpathlineto{\pgfqpoint{4.472059in}{1.978651in}}%
\pgfpathlineto{\pgfqpoint{4.474861in}{1.980915in}}%
\pgfpathlineto{\pgfqpoint{4.477430in}{1.987125in}}%
\pgfpathlineto{\pgfqpoint{4.480201in}{1.992527in}}%
\pgfpathlineto{\pgfqpoint{4.482778in}{1.990270in}}%
\pgfpathlineto{\pgfqpoint{4.485581in}{1.983101in}}%
\pgfpathlineto{\pgfqpoint{4.488130in}{1.978205in}}%
\pgfpathlineto{\pgfqpoint{4.490822in}{1.980479in}}%
\pgfpathlineto{\pgfqpoint{4.493492in}{1.975442in}}%
\pgfpathlineto{\pgfqpoint{4.496167in}{1.977411in}}%
\pgfpathlineto{\pgfqpoint{4.498850in}{1.979134in}}%
\pgfpathlineto{\pgfqpoint{4.501529in}{1.980085in}}%
\pgfpathlineto{\pgfqpoint{4.504305in}{1.974106in}}%
\pgfpathlineto{\pgfqpoint{4.506893in}{1.974584in}}%
\pgfpathlineto{\pgfqpoint{4.509643in}{1.972201in}}%
\pgfpathlineto{\pgfqpoint{4.512246in}{1.976712in}}%
\pgfpathlineto{\pgfqpoint{4.515080in}{1.977591in}}%
\pgfpathlineto{\pgfqpoint{4.517598in}{1.988173in}}%
\pgfpathlineto{\pgfqpoint{4.520345in}{1.979215in}}%
\pgfpathlineto{\pgfqpoint{4.522962in}{1.970492in}}%
\pgfpathlineto{\pgfqpoint{4.525640in}{1.975568in}}%
\pgfpathlineto{\pgfqpoint{4.528307in}{1.981254in}}%
\pgfpathlineto{\pgfqpoint{4.530990in}{1.982754in}}%
\pgfpathlineto{\pgfqpoint{4.533764in}{1.989600in}}%
\pgfpathlineto{\pgfqpoint{4.536400in}{1.997762in}}%
\pgfpathlineto{\pgfqpoint{4.539144in}{2.006957in}}%
\pgfpathlineto{\pgfqpoint{4.541711in}{1.995427in}}%
\pgfpathlineto{\pgfqpoint{4.544464in}{1.983547in}}%
\pgfpathlineto{\pgfqpoint{4.547064in}{1.981617in}}%
\pgfpathlineto{\pgfqpoint{4.549822in}{1.981761in}}%
\pgfpathlineto{\pgfqpoint{4.552425in}{1.979070in}}%
\pgfpathlineto{\pgfqpoint{4.555106in}{1.980209in}}%
\pgfpathlineto{\pgfqpoint{4.557777in}{1.984528in}}%
\pgfpathlineto{\pgfqpoint{4.560448in}{1.978756in}}%
\pgfpathlineto{\pgfqpoint{4.563125in}{1.984378in}}%
\pgfpathlineto{\pgfqpoint{4.565820in}{1.977481in}}%
\pgfpathlineto{\pgfqpoint{4.568612in}{1.977711in}}%
\pgfpathlineto{\pgfqpoint{4.571171in}{1.982593in}}%
\pgfpathlineto{\pgfqpoint{4.573947in}{1.985054in}}%
\pgfpathlineto{\pgfqpoint{4.576531in}{1.984837in}}%
\pgfpathlineto{\pgfqpoint{4.579305in}{1.981125in}}%
\pgfpathlineto{\pgfqpoint{4.581888in}{1.984478in}}%
\pgfpathlineto{\pgfqpoint{4.584672in}{1.979458in}}%
\pgfpathlineto{\pgfqpoint{4.587244in}{1.975208in}}%
\pgfpathlineto{\pgfqpoint{4.589920in}{1.985169in}}%
\pgfpathlineto{\pgfqpoint{4.592589in}{1.976332in}}%
\pgfpathlineto{\pgfqpoint{4.595281in}{1.980030in}}%
\pgfpathlineto{\pgfqpoint{4.597951in}{1.972930in}}%
\pgfpathlineto{\pgfqpoint{4.600633in}{1.976213in}}%
\pgfpathlineto{\pgfqpoint{4.603430in}{1.977909in}}%
\pgfpathlineto{\pgfqpoint{4.605990in}{1.976011in}}%
\pgfpathlineto{\pgfqpoint{4.608808in}{1.974313in}}%
\pgfpathlineto{\pgfqpoint{4.611350in}{1.981296in}}%
\pgfpathlineto{\pgfqpoint{4.614134in}{1.980937in}}%
\pgfpathlineto{\pgfqpoint{4.616702in}{1.976633in}}%
\pgfpathlineto{\pgfqpoint{4.619529in}{1.975539in}}%
\pgfpathlineto{\pgfqpoint{4.622056in}{1.979798in}}%
\pgfpathlineto{\pgfqpoint{4.624741in}{1.975159in}}%
\pgfpathlineto{\pgfqpoint{4.627411in}{1.971974in}}%
\pgfpathlineto{\pgfqpoint{4.630096in}{1.979269in}}%
\pgfpathlineto{\pgfqpoint{4.632902in}{1.976890in}}%
\pgfpathlineto{\pgfqpoint{4.635445in}{1.980494in}}%
\pgfpathlineto{\pgfqpoint{4.638204in}{1.977398in}}%
\pgfpathlineto{\pgfqpoint{4.640809in}{1.971960in}}%
\pgfpathlineto{\pgfqpoint{4.643628in}{1.976851in}}%
\pgfpathlineto{\pgfqpoint{4.646169in}{1.977992in}}%
\pgfpathlineto{\pgfqpoint{4.648922in}{1.979722in}}%
\pgfpathlineto{\pgfqpoint{4.651524in}{1.973964in}}%
\pgfpathlineto{\pgfqpoint{4.654203in}{1.980822in}}%
\pgfpathlineto{\pgfqpoint{4.656873in}{1.975486in}}%
\pgfpathlineto{\pgfqpoint{4.659590in}{1.974582in}}%
\pgfpathlineto{\pgfqpoint{4.662237in}{1.980243in}}%
\pgfpathlineto{\pgfqpoint{4.664923in}{1.979401in}}%
\pgfpathlineto{\pgfqpoint{4.667764in}{1.975585in}}%
\pgfpathlineto{\pgfqpoint{4.670261in}{1.983281in}}%
\pgfpathlineto{\pgfqpoint{4.673068in}{1.987975in}}%
\pgfpathlineto{\pgfqpoint{4.675619in}{1.982718in}}%
\pgfpathlineto{\pgfqpoint{4.678448in}{1.981171in}}%
\pgfpathlineto{\pgfqpoint{4.680988in}{1.980660in}}%
\pgfpathlineto{\pgfqpoint{4.683799in}{1.981179in}}%
\pgfpathlineto{\pgfqpoint{4.686337in}{1.972899in}}%
\pgfpathlineto{\pgfqpoint{4.689051in}{1.971705in}}%
\pgfpathlineto{\pgfqpoint{4.691694in}{1.974939in}}%
\pgfpathlineto{\pgfqpoint{4.694381in}{1.977489in}}%
\pgfpathlineto{\pgfqpoint{4.697170in}{1.974676in}}%
\pgfpathlineto{\pgfqpoint{4.699734in}{1.981588in}}%
\pgfpathlineto{\pgfqpoint{4.702517in}{1.979346in}}%
\pgfpathlineto{\pgfqpoint{4.705094in}{1.980476in}}%
\pgfpathlineto{\pgfqpoint{4.707824in}{1.974725in}}%
\pgfpathlineto{\pgfqpoint{4.710437in}{1.971877in}}%
\pgfpathlineto{\pgfqpoint{4.713275in}{1.975176in}}%
\pgfpathlineto{\pgfqpoint{4.715806in}{1.976265in}}%
\pgfpathlineto{\pgfqpoint{4.718486in}{1.976974in}}%
\pgfpathlineto{\pgfqpoint{4.721160in}{1.976111in}}%
\pgfpathlineto{\pgfqpoint{4.723873in}{1.977528in}}%
\pgfpathlineto{\pgfqpoint{4.726508in}{1.986041in}}%
\pgfpathlineto{\pgfqpoint{4.729233in}{1.983590in}}%
\pgfpathlineto{\pgfqpoint{4.731901in}{1.975030in}}%
\pgfpathlineto{\pgfqpoint{4.734552in}{1.986444in}}%
\pgfpathlineto{\pgfqpoint{4.737348in}{1.983193in}}%
\pgfpathlineto{\pgfqpoint{4.739912in}{1.980095in}}%
\pgfpathlineto{\pgfqpoint{4.742696in}{1.983768in}}%
\pgfpathlineto{\pgfqpoint{4.745256in}{1.988979in}}%
\pgfpathlineto{\pgfqpoint{4.748081in}{1.981509in}}%
\pgfpathlineto{\pgfqpoint{4.750627in}{1.983573in}}%
\pgfpathlineto{\pgfqpoint{4.753298in}{1.981895in}}%
\pgfpathlineto{\pgfqpoint{4.755983in}{1.978930in}}%
\pgfpathlineto{\pgfqpoint{4.758653in}{1.985712in}}%
\pgfpathlineto{\pgfqpoint{4.761337in}{1.994893in}}%
\pgfpathlineto{\pgfqpoint{4.764018in}{2.005602in}}%
\pgfpathlineto{\pgfqpoint{4.766783in}{1.997338in}}%
\pgfpathlineto{\pgfqpoint{4.769367in}{1.992152in}}%
\pgfpathlineto{\pgfqpoint{4.772198in}{1.986306in}}%
\pgfpathlineto{\pgfqpoint{4.774732in}{1.993888in}}%
\pgfpathlineto{\pgfqpoint{4.777535in}{1.986821in}}%
\pgfpathlineto{\pgfqpoint{4.780083in}{1.991287in}}%
\pgfpathlineto{\pgfqpoint{4.782872in}{1.988581in}}%
\pgfpathlineto{\pgfqpoint{4.785445in}{1.991605in}}%
\pgfpathlineto{\pgfqpoint{4.788116in}{1.975783in}}%
\pgfpathlineto{\pgfqpoint{4.790798in}{1.971647in}}%
\pgfpathlineto{\pgfqpoint{4.793512in}{1.973833in}}%
\pgfpathlineto{\pgfqpoint{4.796274in}{1.975824in}}%
\pgfpathlineto{\pgfqpoint{4.798830in}{1.976835in}}%
\pgfpathlineto{\pgfqpoint{4.801586in}{1.972965in}}%
\pgfpathlineto{\pgfqpoint{4.804193in}{1.973995in}}%
\pgfpathlineto{\pgfqpoint{4.807017in}{1.976595in}}%
\pgfpathlineto{\pgfqpoint{4.809538in}{1.976540in}}%
\pgfpathlineto{\pgfqpoint{4.812377in}{1.968595in}}%
\pgfpathlineto{\pgfqpoint{4.814907in}{1.971742in}}%
\pgfpathlineto{\pgfqpoint{4.817587in}{1.966074in}}%
\pgfpathlineto{\pgfqpoint{4.820265in}{1.972049in}}%
\pgfpathlineto{\pgfqpoint{4.822945in}{1.985200in}}%
\pgfpathlineto{\pgfqpoint{4.825619in}{1.985927in}}%
\pgfpathlineto{\pgfqpoint{4.828291in}{1.978819in}}%
\pgfpathlineto{\pgfqpoint{4.831045in}{1.967513in}}%
\pgfpathlineto{\pgfqpoint{4.833657in}{1.965989in}}%
\pgfpathlineto{\pgfqpoint{4.837992in}{1.966965in}}%
\pgfpathlineto{\pgfqpoint{4.839922in}{1.969793in}}%
\pgfpathlineto{\pgfqpoint{4.842380in}{1.969893in}}%
\pgfpathlineto{\pgfqpoint{4.844361in}{1.974053in}}%
\pgfpathlineto{\pgfqpoint{4.847127in}{1.976082in}}%
\pgfpathlineto{\pgfqpoint{4.849715in}{1.972655in}}%
\pgfpathlineto{\pgfqpoint{4.852404in}{1.971879in}}%
\pgfpathlineto{\pgfqpoint{4.855070in}{1.965832in}}%
\pgfpathlineto{\pgfqpoint{4.857807in}{1.969796in}}%
\pgfpathlineto{\pgfqpoint{4.860544in}{1.973166in}}%
\pgfpathlineto{\pgfqpoint{4.863116in}{1.973254in}}%
\pgfpathlineto{\pgfqpoint{4.865910in}{1.979830in}}%
\pgfpathlineto{\pgfqpoint{4.868474in}{1.980630in}}%
\pgfpathlineto{\pgfqpoint{4.871209in}{1.981213in}}%
\pgfpathlineto{\pgfqpoint{4.873832in}{1.982181in}}%
\pgfpathlineto{\pgfqpoint{4.876636in}{1.982276in}}%
\pgfpathlineto{\pgfqpoint{4.879180in}{1.983310in}}%
\pgfpathlineto{\pgfqpoint{4.881864in}{1.979930in}}%
\pgfpathlineto{\pgfqpoint{4.884540in}{1.983356in}}%
\pgfpathlineto{\pgfqpoint{4.887211in}{1.981786in}}%
\pgfpathlineto{\pgfqpoint{4.889902in}{1.980271in}}%
\pgfpathlineto{\pgfqpoint{4.892611in}{1.983432in}}%
\pgfpathlineto{\pgfqpoint{4.895399in}{1.983688in}}%
\pgfpathlineto{\pgfqpoint{4.897938in}{1.980220in}}%
\pgfpathlineto{\pgfqpoint{4.900712in}{1.983496in}}%
\pgfpathlineto{\pgfqpoint{4.903295in}{1.973254in}}%
\pgfpathlineto{\pgfqpoint{4.906096in}{1.976426in}}%
\pgfpathlineto{\pgfqpoint{4.908648in}{1.979082in}}%
\pgfpathlineto{\pgfqpoint{4.911435in}{1.994640in}}%
\pgfpathlineto{\pgfqpoint{4.914009in}{1.993678in}}%
\pgfpathlineto{\pgfqpoint{4.916681in}{1.990520in}}%
\pgfpathlineto{\pgfqpoint{4.919352in}{1.985591in}}%
\pgfpathlineto{\pgfqpoint{4.922041in}{1.981440in}}%
\pgfpathlineto{\pgfqpoint{4.924708in}{1.986086in}}%
\pgfpathlineto{\pgfqpoint{4.927400in}{1.981514in}}%
\pgfpathlineto{\pgfqpoint{4.930170in}{1.977233in}}%
\pgfpathlineto{\pgfqpoint{4.932742in}{1.981869in}}%
\pgfpathlineto{\pgfqpoint{4.935515in}{1.981874in}}%
\pgfpathlineto{\pgfqpoint{4.938112in}{1.978328in}}%
\pgfpathlineto{\pgfqpoint{4.940881in}{1.977922in}}%
\pgfpathlineto{\pgfqpoint{4.943466in}{1.981414in}}%
\pgfpathlineto{\pgfqpoint{4.946151in}{1.980029in}}%
\pgfpathlineto{\pgfqpoint{4.948827in}{1.977116in}}%
\pgfpathlineto{\pgfqpoint{4.951504in}{1.979840in}}%
\pgfpathlineto{\pgfqpoint{4.954182in}{1.984325in}}%
\pgfpathlineto{\pgfqpoint{4.956862in}{1.977743in}}%
\pgfpathlineto{\pgfqpoint{4.959689in}{1.974783in}}%
\pgfpathlineto{\pgfqpoint{4.962219in}{1.977997in}}%
\pgfpathlineto{\pgfqpoint{4.965002in}{1.973634in}}%
\pgfpathlineto{\pgfqpoint{4.967575in}{1.975443in}}%
\pgfpathlineto{\pgfqpoint{4.970314in}{1.979144in}}%
\pgfpathlineto{\pgfqpoint{4.972933in}{1.977936in}}%
\pgfpathlineto{\pgfqpoint{4.975703in}{1.979255in}}%
\pgfpathlineto{\pgfqpoint{4.978287in}{1.974278in}}%
\pgfpathlineto{\pgfqpoint{4.980967in}{1.976994in}}%
\pgfpathlineto{\pgfqpoint{4.983637in}{1.974434in}}%
\pgfpathlineto{\pgfqpoint{4.986325in}{1.973194in}}%
\pgfpathlineto{\pgfqpoint{4.989001in}{1.972474in}}%
\pgfpathlineto{\pgfqpoint{4.991683in}{1.971191in}}%
\pgfpathlineto{\pgfqpoint{4.994390in}{1.974087in}}%
\pgfpathlineto{\pgfqpoint{4.997028in}{1.972645in}}%
\pgfpathlineto{\pgfqpoint{4.999780in}{1.968349in}}%
\pgfpathlineto{\pgfqpoint{5.002384in}{1.964802in}}%
\pgfpathlineto{\pgfqpoint{5.005178in}{1.975749in}}%
\pgfpathlineto{\pgfqpoint{5.007751in}{1.972934in}}%
\pgfpathlineto{\pgfqpoint{5.010562in}{1.972810in}}%
\pgfpathlineto{\pgfqpoint{5.013104in}{1.971362in}}%
\pgfpathlineto{\pgfqpoint{5.015820in}{1.967314in}}%
\pgfpathlineto{\pgfqpoint{5.018466in}{1.968540in}}%
\pgfpathlineto{\pgfqpoint{5.021147in}{1.974596in}}%
\pgfpathlineto{\pgfqpoint{5.023927in}{1.975902in}}%
\pgfpathlineto{\pgfqpoint{5.026501in}{1.969158in}}%
\pgfpathlineto{\pgfqpoint{5.029275in}{1.972111in}}%
\pgfpathlineto{\pgfqpoint{5.031849in}{1.978695in}}%
\pgfpathlineto{\pgfqpoint{5.034649in}{1.974599in}}%
\pgfpathlineto{\pgfqpoint{5.037214in}{1.971524in}}%
\pgfpathlineto{\pgfqpoint{5.039962in}{1.974611in}}%
\pgfpathlineto{\pgfqpoint{5.042572in}{1.971531in}}%
\pgfpathlineto{\pgfqpoint{5.045249in}{1.973890in}}%
\pgfpathlineto{\pgfqpoint{5.047924in}{1.975862in}}%
\pgfpathlineto{\pgfqpoint{5.050606in}{1.973346in}}%
\pgfpathlineto{\pgfqpoint{5.053284in}{1.971064in}}%
\pgfpathlineto{\pgfqpoint{5.055952in}{1.973740in}}%
\pgfpathlineto{\pgfqpoint{5.058711in}{1.976217in}}%
\pgfpathlineto{\pgfqpoint{5.061315in}{1.970719in}}%
\pgfpathlineto{\pgfqpoint{5.064144in}{1.969836in}}%
\pgfpathlineto{\pgfqpoint{5.066677in}{1.973858in}}%
\pgfpathlineto{\pgfqpoint{5.069463in}{1.979470in}}%
\pgfpathlineto{\pgfqpoint{5.072030in}{1.976271in}}%
\pgfpathlineto{\pgfqpoint{5.074851in}{1.981930in}}%
\pgfpathlineto{\pgfqpoint{5.077390in}{1.974238in}}%
\pgfpathlineto{\pgfqpoint{5.080067in}{1.970802in}}%
\pgfpathlineto{\pgfqpoint{5.082746in}{1.978149in}}%
\pgfpathlineto{\pgfqpoint{5.085426in}{1.983011in}}%
\pgfpathlineto{\pgfqpoint{5.088103in}{1.980710in}}%
\pgfpathlineto{\pgfqpoint{5.090788in}{1.977763in}}%
\pgfpathlineto{\pgfqpoint{5.093579in}{1.974743in}}%
\pgfpathlineto{\pgfqpoint{5.096142in}{1.973824in}}%
\pgfpathlineto{\pgfqpoint{5.098948in}{1.972225in}}%
\pgfpathlineto{\pgfqpoint{5.101496in}{1.970364in}}%
\pgfpathlineto{\pgfqpoint{5.104312in}{1.976181in}}%
\pgfpathlineto{\pgfqpoint{5.106842in}{1.975082in}}%
\pgfpathlineto{\pgfqpoint{5.109530in}{1.975465in}}%
\pgfpathlineto{\pgfqpoint{5.112209in}{1.981195in}}%
\pgfpathlineto{\pgfqpoint{5.114887in}{1.979688in}}%
\pgfpathlineto{\pgfqpoint{5.117550in}{1.975777in}}%
\pgfpathlineto{\pgfqpoint{5.120243in}{1.977803in}}%
\pgfpathlineto{\pgfqpoint{5.123042in}{1.976685in}}%
\pgfpathlineto{\pgfqpoint{5.125599in}{1.972598in}}%
\pgfpathlineto{\pgfqpoint{5.128421in}{1.973811in}}%
\pgfpathlineto{\pgfqpoint{5.130953in}{1.972777in}}%
\pgfpathlineto{\pgfqpoint{5.133716in}{1.978493in}}%
\pgfpathlineto{\pgfqpoint{5.136311in}{1.971979in}}%
\pgfpathlineto{\pgfqpoint{5.139072in}{1.978032in}}%
\pgfpathlineto{\pgfqpoint{5.141660in}{1.977885in}}%
\pgfpathlineto{\pgfqpoint{5.144349in}{1.975689in}}%
\pgfpathlineto{\pgfqpoint{5.147029in}{1.975170in}}%
\pgfpathlineto{\pgfqpoint{5.149734in}{1.966568in}}%
\pgfpathlineto{\pgfqpoint{5.152382in}{1.966358in}}%
\pgfpathlineto{\pgfqpoint{5.155059in}{1.967916in}}%
\pgfpathlineto{\pgfqpoint{5.157815in}{1.972378in}}%
\pgfpathlineto{\pgfqpoint{5.160420in}{1.968052in}}%
\pgfpathlineto{\pgfqpoint{5.163243in}{1.973972in}}%
\pgfpathlineto{\pgfqpoint{5.165775in}{1.975774in}}%
\pgfpathlineto{\pgfqpoint{5.168591in}{1.975912in}}%
\pgfpathlineto{\pgfqpoint{5.171133in}{1.978242in}}%
\pgfpathlineto{\pgfqpoint{5.173925in}{1.981891in}}%
\pgfpathlineto{\pgfqpoint{5.176477in}{1.984740in}}%
\pgfpathlineto{\pgfqpoint{5.179188in}{1.977560in}}%
\pgfpathlineto{\pgfqpoint{5.181848in}{1.978099in}}%
\pgfpathlineto{\pgfqpoint{5.184522in}{1.977206in}}%
\pgfpathlineto{\pgfqpoint{5.187294in}{1.979314in}}%
\pgfpathlineto{\pgfqpoint{5.189880in}{1.980162in}}%
\pgfpathlineto{\pgfqpoint{5.192680in}{1.981532in}}%
\pgfpathlineto{\pgfqpoint{5.195239in}{1.981368in}}%
\pgfpathlineto{\pgfqpoint{5.198008in}{1.983038in}}%
\pgfpathlineto{\pgfqpoint{5.200594in}{1.975921in}}%
\pgfpathlineto{\pgfqpoint{5.203388in}{1.981839in}}%
\pgfpathlineto{\pgfqpoint{5.205952in}{1.976443in}}%
\pgfpathlineto{\pgfqpoint{5.208630in}{1.978100in}}%
\pgfpathlineto{\pgfqpoint{5.211299in}{1.978437in}}%
\pgfpathlineto{\pgfqpoint{5.214027in}{1.982597in}}%
\pgfpathlineto{\pgfqpoint{5.216667in}{1.987166in}}%
\pgfpathlineto{\pgfqpoint{5.219345in}{1.977502in}}%
\pgfpathlineto{\pgfqpoint{5.222151in}{1.977475in}}%
\pgfpathlineto{\pgfqpoint{5.224695in}{1.980862in}}%
\pgfpathlineto{\pgfqpoint{5.227470in}{1.976833in}}%
\pgfpathlineto{\pgfqpoint{5.230059in}{1.973015in}}%
\pgfpathlineto{\pgfqpoint{5.232855in}{1.975166in}}%
\pgfpathlineto{\pgfqpoint{5.235409in}{1.976431in}}%
\pgfpathlineto{\pgfqpoint{5.238173in}{1.977237in}}%
\pgfpathlineto{\pgfqpoint{5.240777in}{1.977415in}}%
\pgfpathlineto{\pgfqpoint{5.243445in}{1.976301in}}%
\pgfpathlineto{\pgfqpoint{5.246130in}{1.976278in}}%
\pgfpathlineto{\pgfqpoint{5.248816in}{1.979483in}}%
\pgfpathlineto{\pgfqpoint{5.251590in}{1.973348in}}%
\pgfpathlineto{\pgfqpoint{5.254236in}{1.978431in}}%
\pgfpathlineto{\pgfqpoint{5.256973in}{1.980497in}}%
\pgfpathlineto{\pgfqpoint{5.259511in}{1.978923in}}%
\pgfpathlineto{\pgfqpoint{5.262264in}{1.973309in}}%
\pgfpathlineto{\pgfqpoint{5.264876in}{1.973050in}}%
\pgfpathlineto{\pgfqpoint{5.267691in}{1.973235in}}%
\pgfpathlineto{\pgfqpoint{5.270238in}{1.972124in}}%
\pgfpathlineto{\pgfqpoint{5.272913in}{1.979398in}}%
\pgfpathlineto{\pgfqpoint{5.275589in}{1.970895in}}%
\pgfpathlineto{\pgfqpoint{5.278322in}{1.977781in}}%
\pgfpathlineto{\pgfqpoint{5.280947in}{1.976496in}}%
\pgfpathlineto{\pgfqpoint{5.283631in}{1.981579in}}%
\pgfpathlineto{\pgfqpoint{5.286436in}{1.989177in}}%
\pgfpathlineto{\pgfqpoint{5.288984in}{1.984168in}}%
\pgfpathlineto{\pgfqpoint{5.291794in}{1.974863in}}%
\pgfpathlineto{\pgfqpoint{5.294339in}{1.981077in}}%
\pgfpathlineto{\pgfqpoint{5.297140in}{1.978157in}}%
\pgfpathlineto{\pgfqpoint{5.299696in}{1.977742in}}%
\pgfpathlineto{\pgfqpoint{5.302443in}{1.979954in}}%
\pgfpathlineto{\pgfqpoint{5.305054in}{1.983721in}}%
\pgfpathlineto{\pgfqpoint{5.307731in}{1.980517in}}%
\pgfpathlineto{\pgfqpoint{5.310411in}{1.981602in}}%
\pgfpathlineto{\pgfqpoint{5.313089in}{1.981883in}}%
\pgfpathlineto{\pgfqpoint{5.315754in}{1.977819in}}%
\pgfpathlineto{\pgfqpoint{5.318430in}{1.983578in}}%
\pgfpathlineto{\pgfqpoint{5.321256in}{1.983505in}}%
\pgfpathlineto{\pgfqpoint{5.323802in}{1.982859in}}%
\pgfpathlineto{\pgfqpoint{5.326564in}{1.987456in}}%
\pgfpathlineto{\pgfqpoint{5.329159in}{1.988700in}}%
\pgfpathlineto{\pgfqpoint{5.331973in}{1.985965in}}%
\pgfpathlineto{\pgfqpoint{5.334510in}{1.982469in}}%
\pgfpathlineto{\pgfqpoint{5.337353in}{1.988251in}}%
\pgfpathlineto{\pgfqpoint{5.339872in}{1.978300in}}%
\pgfpathlineto{\pgfqpoint{5.342549in}{1.975360in}}%
\pgfpathlineto{\pgfqpoint{5.345224in}{1.963929in}}%
\pgfpathlineto{\pgfqpoint{5.347905in}{1.966499in}}%
\pgfpathlineto{\pgfqpoint{5.350723in}{1.978811in}}%
\pgfpathlineto{\pgfqpoint{5.353262in}{1.979490in}}%
\pgfpathlineto{\pgfqpoint{5.356056in}{1.979378in}}%
\pgfpathlineto{\pgfqpoint{5.358612in}{1.980807in}}%
\pgfpathlineto{\pgfqpoint{5.361370in}{1.980442in}}%
\pgfpathlineto{\pgfqpoint{5.363966in}{1.971806in}}%
\pgfpathlineto{\pgfqpoint{5.366727in}{1.978215in}}%
\pgfpathlineto{\pgfqpoint{5.369335in}{1.971155in}}%
\pgfpathlineto{\pgfqpoint{5.372013in}{1.974468in}}%
\pgfpathlineto{\pgfqpoint{5.374692in}{1.972297in}}%
\pgfpathlineto{\pgfqpoint{5.377370in}{1.975125in}}%
\pgfpathlineto{\pgfqpoint{5.380048in}{1.979477in}}%
\pgfpathlineto{\pgfqpoint{5.382725in}{1.983015in}}%
\pgfpathlineto{\pgfqpoint{5.385550in}{1.984028in}}%
\pgfpathlineto{\pgfqpoint{5.388083in}{1.982126in}}%
\pgfpathlineto{\pgfqpoint{5.390900in}{1.975847in}}%
\pgfpathlineto{\pgfqpoint{5.393441in}{1.965330in}}%
\pgfpathlineto{\pgfqpoint{5.396219in}{1.980705in}}%
\pgfpathlineto{\pgfqpoint{5.398784in}{2.014970in}}%
\pgfpathlineto{\pgfqpoint{5.401576in}{2.010723in}}%
\pgfpathlineto{\pgfqpoint{5.404154in}{1.998790in}}%
\pgfpathlineto{\pgfqpoint{5.406832in}{1.991729in}}%
\pgfpathlineto{\pgfqpoint{5.409507in}{1.986506in}}%
\pgfpathlineto{\pgfqpoint{5.412190in}{1.981432in}}%
\pgfpathlineto{\pgfqpoint{5.414954in}{1.981723in}}%
\pgfpathlineto{\pgfqpoint{5.417547in}{1.980544in}}%
\pgfpathlineto{\pgfqpoint{5.420304in}{1.978176in}}%
\pgfpathlineto{\pgfqpoint{5.422897in}{1.976308in}}%
\pgfpathlineto{\pgfqpoint{5.425661in}{1.984044in}}%
\pgfpathlineto{\pgfqpoint{5.428259in}{1.974339in}}%
\pgfpathlineto{\pgfqpoint{5.431015in}{1.983602in}}%
\pgfpathlineto{\pgfqpoint{5.433616in}{1.977504in}}%
\pgfpathlineto{\pgfqpoint{5.436295in}{1.979389in}}%
\pgfpathlineto{\pgfqpoint{5.438974in}{1.978979in}}%
\pgfpathlineto{\pgfqpoint{5.441698in}{1.977071in}}%
\pgfpathlineto{\pgfqpoint{5.444328in}{1.972786in}}%
\pgfpathlineto{\pgfqpoint{5.447021in}{1.983588in}}%
\pgfpathlineto{\pgfqpoint{5.449769in}{1.996469in}}%
\pgfpathlineto{\pgfqpoint{5.452365in}{1.991546in}}%
\pgfpathlineto{\pgfqpoint{5.455168in}{1.988522in}}%
\pgfpathlineto{\pgfqpoint{5.457721in}{1.987396in}}%
\pgfpathlineto{\pgfqpoint{5.460489in}{1.985582in}}%
\pgfpathlineto{\pgfqpoint{5.463079in}{1.980115in}}%
\pgfpathlineto{\pgfqpoint{5.465888in}{1.978236in}}%
\pgfpathlineto{\pgfqpoint{5.468425in}{1.981562in}}%
\pgfpathlineto{\pgfqpoint{5.471113in}{1.975214in}}%
\pgfpathlineto{\pgfqpoint{5.473792in}{1.980128in}}%
\pgfpathlineto{\pgfqpoint{5.476458in}{1.978989in}}%
\pgfpathlineto{\pgfqpoint{5.479152in}{1.976339in}}%
\pgfpathlineto{\pgfqpoint{5.481825in}{1.969246in}}%
\pgfpathlineto{\pgfqpoint{5.484641in}{1.972343in}}%
\pgfpathlineto{\pgfqpoint{5.487176in}{1.978566in}}%
\pgfpathlineto{\pgfqpoint{5.490000in}{1.980391in}}%
\pgfpathlineto{\pgfqpoint{5.492541in}{1.974476in}}%
\pgfpathlineto{\pgfqpoint{5.495346in}{1.977795in}}%
\pgfpathlineto{\pgfqpoint{5.497898in}{1.986176in}}%
\pgfpathlineto{\pgfqpoint{5.500687in}{1.973769in}}%
\pgfpathlineto{\pgfqpoint{5.503255in}{1.973512in}}%
\pgfpathlineto{\pgfqpoint{5.505933in}{1.978898in}}%
\pgfpathlineto{\pgfqpoint{5.508612in}{1.982021in}}%
\pgfpathlineto{\pgfqpoint{5.511290in}{1.975583in}}%
\pgfpathlineto{\pgfqpoint{5.514080in}{1.994475in}}%
\pgfpathlineto{\pgfqpoint{5.516646in}{1.986526in}}%
\pgfpathlineto{\pgfqpoint{5.519433in}{1.985452in}}%
\pgfpathlineto{\pgfqpoint{5.522003in}{1.980215in}}%
\pgfpathlineto{\pgfqpoint{5.524756in}{1.989886in}}%
\pgfpathlineto{\pgfqpoint{5.527360in}{1.980049in}}%
\pgfpathlineto{\pgfqpoint{5.530148in}{1.975383in}}%
\pgfpathlineto{\pgfqpoint{5.532717in}{1.976487in}}%
\pgfpathlineto{\pgfqpoint{5.535395in}{1.982120in}}%
\pgfpathlineto{\pgfqpoint{5.538074in}{1.981071in}}%
\pgfpathlineto{\pgfqpoint{5.540750in}{1.978493in}}%
\pgfpathlineto{\pgfqpoint{5.543421in}{1.976843in}}%
\pgfpathlineto{\pgfqpoint{5.546110in}{1.978028in}}%
\pgfpathlineto{\pgfqpoint{5.548921in}{1.972108in}}%
\pgfpathlineto{\pgfqpoint{5.551457in}{1.974939in}}%
\pgfpathlineto{\pgfqpoint{5.554198in}{1.975653in}}%
\pgfpathlineto{\pgfqpoint{5.556822in}{1.978793in}}%
\pgfpathlineto{\pgfqpoint{5.559612in}{1.969700in}}%
\pgfpathlineto{\pgfqpoint{5.562180in}{1.975502in}}%
\pgfpathlineto{\pgfqpoint{5.564940in}{1.980154in}}%
\pgfpathlineto{\pgfqpoint{5.567536in}{1.974033in}}%
\pgfpathlineto{\pgfqpoint{5.570215in}{1.975142in}}%
\pgfpathlineto{\pgfqpoint{5.572893in}{1.975367in}}%
\pgfpathlineto{\pgfqpoint{5.575596in}{1.979066in}}%
\pgfpathlineto{\pgfqpoint{5.578342in}{1.982107in}}%
\pgfpathlineto{\pgfqpoint{5.580914in}{1.974601in}}%
\pgfpathlineto{\pgfqpoint{5.583709in}{1.969854in}}%
\pgfpathlineto{\pgfqpoint{5.586269in}{1.978420in}}%
\pgfpathlineto{\pgfqpoint{5.589040in}{1.981719in}}%
\pgfpathlineto{\pgfqpoint{5.591641in}{1.977215in}}%
\pgfpathlineto{\pgfqpoint{5.594368in}{1.980590in}}%
\pgfpathlineto{\pgfqpoint{5.596999in}{1.977295in}}%
\pgfpathlineto{\pgfqpoint{5.599674in}{1.975960in}}%
\pgfpathlineto{\pgfqpoint{5.602352in}{1.985707in}}%
\pgfpathlineto{\pgfqpoint{5.605073in}{1.983759in}}%
\pgfpathlineto{\pgfqpoint{5.607698in}{1.980486in}}%
\pgfpathlineto{\pgfqpoint{5.610389in}{1.979183in}}%
\pgfpathlineto{\pgfqpoint{5.613235in}{1.982569in}}%
\pgfpathlineto{\pgfqpoint{5.615743in}{1.980938in}}%
\pgfpathlineto{\pgfqpoint{5.618526in}{1.980538in}}%
\pgfpathlineto{\pgfqpoint{5.621102in}{1.972263in}}%
\pgfpathlineto{\pgfqpoint{5.623868in}{1.979073in}}%
\pgfpathlineto{\pgfqpoint{5.626460in}{1.980519in}}%
\pgfpathlineto{\pgfqpoint{5.629232in}{1.978757in}}%
\pgfpathlineto{\pgfqpoint{5.631815in}{1.982302in}}%
\pgfpathlineto{\pgfqpoint{5.634496in}{1.988860in}}%
\pgfpathlineto{\pgfqpoint{5.637172in}{1.990853in}}%
\pgfpathlineto{\pgfqpoint{5.639852in}{2.006775in}}%
\pgfpathlineto{\pgfqpoint{5.642518in}{2.051306in}}%
\pgfpathlineto{\pgfqpoint{5.645243in}{2.055758in}}%
\pgfpathlineto{\pgfqpoint{5.648008in}{2.069531in}}%
\pgfpathlineto{\pgfqpoint{5.650563in}{2.074933in}}%
\pgfpathlineto{\pgfqpoint{5.653376in}{2.039645in}}%
\pgfpathlineto{\pgfqpoint{5.655919in}{2.024969in}}%
\pgfpathlineto{\pgfqpoint{5.658723in}{1.998645in}}%
\pgfpathlineto{\pgfqpoint{5.661273in}{1.987304in}}%
\pgfpathlineto{\pgfqpoint{5.664099in}{1.985989in}}%
\pgfpathlineto{\pgfqpoint{5.666632in}{2.007422in}}%
\pgfpathlineto{\pgfqpoint{5.669313in}{2.024058in}}%
\pgfpathlineto{\pgfqpoint{5.671991in}{2.016031in}}%
\pgfpathlineto{\pgfqpoint{5.674667in}{2.000219in}}%
\pgfpathlineto{\pgfqpoint{5.677486in}{1.994507in}}%
\pgfpathlineto{\pgfqpoint{5.680027in}{1.986295in}}%
\pgfpathlineto{\pgfqpoint{5.682836in}{1.990335in}}%
\pgfpathlineto{\pgfqpoint{5.685385in}{1.986609in}}%
\pgfpathlineto{\pgfqpoint{5.688159in}{1.986876in}}%
\pgfpathlineto{\pgfqpoint{5.690730in}{1.983898in}}%
\pgfpathlineto{\pgfqpoint{5.693473in}{2.003467in}}%
\pgfpathlineto{\pgfqpoint{5.696101in}{2.012940in}}%
\pgfpathlineto{\pgfqpoint{5.698775in}{2.004642in}}%
\pgfpathlineto{\pgfqpoint{5.701453in}{1.993072in}}%
\pgfpathlineto{\pgfqpoint{5.704130in}{1.993339in}}%
\pgfpathlineto{\pgfqpoint{5.706800in}{1.981356in}}%
\pgfpathlineto{\pgfqpoint{5.709490in}{1.972269in}}%
\pgfpathlineto{\pgfqpoint{5.712291in}{1.974415in}}%
\pgfpathlineto{\pgfqpoint{5.714834in}{1.969639in}}%
\pgfpathlineto{\pgfqpoint{5.717671in}{1.967695in}}%
\pgfpathlineto{\pgfqpoint{5.720201in}{1.971549in}}%
\pgfpathlineto{\pgfqpoint{5.722950in}{1.974463in}}%
\pgfpathlineto{\pgfqpoint{5.725548in}{1.967400in}}%
\pgfpathlineto{\pgfqpoint{5.728339in}{1.970267in}}%
\pgfpathlineto{\pgfqpoint{5.730919in}{1.971604in}}%
\pgfpathlineto{\pgfqpoint{5.733594in}{1.972576in}}%
\pgfpathlineto{\pgfqpoint{5.736276in}{1.975794in}}%
\pgfpathlineto{\pgfqpoint{5.738974in}{1.975374in}}%
\pgfpathlineto{\pgfqpoint{5.741745in}{1.972855in}}%
\pgfpathlineto{\pgfqpoint{5.744310in}{1.980761in}}%
\pgfpathlineto{\pgfqpoint{5.744310in}{0.413320in}}%
\pgfpathlineto{\pgfqpoint{5.744310in}{0.413320in}}%
\pgfpathlineto{\pgfqpoint{5.741745in}{0.413320in}}%
\pgfpathlineto{\pgfqpoint{5.738974in}{0.413320in}}%
\pgfpathlineto{\pgfqpoint{5.736276in}{0.413320in}}%
\pgfpathlineto{\pgfqpoint{5.733594in}{0.413320in}}%
\pgfpathlineto{\pgfqpoint{5.730919in}{0.413320in}}%
\pgfpathlineto{\pgfqpoint{5.728339in}{0.413320in}}%
\pgfpathlineto{\pgfqpoint{5.725548in}{0.413320in}}%
\pgfpathlineto{\pgfqpoint{5.722950in}{0.413320in}}%
\pgfpathlineto{\pgfqpoint{5.720201in}{0.413320in}}%
\pgfpathlineto{\pgfqpoint{5.717671in}{0.413320in}}%
\pgfpathlineto{\pgfqpoint{5.714834in}{0.413320in}}%
\pgfpathlineto{\pgfqpoint{5.712291in}{0.413320in}}%
\pgfpathlineto{\pgfqpoint{5.709490in}{0.413320in}}%
\pgfpathlineto{\pgfqpoint{5.706800in}{0.413320in}}%
\pgfpathlineto{\pgfqpoint{5.704130in}{0.413320in}}%
\pgfpathlineto{\pgfqpoint{5.701453in}{0.413320in}}%
\pgfpathlineto{\pgfqpoint{5.698775in}{0.413320in}}%
\pgfpathlineto{\pgfqpoint{5.696101in}{0.413320in}}%
\pgfpathlineto{\pgfqpoint{5.693473in}{0.413320in}}%
\pgfpathlineto{\pgfqpoint{5.690730in}{0.413320in}}%
\pgfpathlineto{\pgfqpoint{5.688159in}{0.413320in}}%
\pgfpathlineto{\pgfqpoint{5.685385in}{0.413320in}}%
\pgfpathlineto{\pgfqpoint{5.682836in}{0.413320in}}%
\pgfpathlineto{\pgfqpoint{5.680027in}{0.413320in}}%
\pgfpathlineto{\pgfqpoint{5.677486in}{0.413320in}}%
\pgfpathlineto{\pgfqpoint{5.674667in}{0.413320in}}%
\pgfpathlineto{\pgfqpoint{5.671991in}{0.413320in}}%
\pgfpathlineto{\pgfqpoint{5.669313in}{0.413320in}}%
\pgfpathlineto{\pgfqpoint{5.666632in}{0.413320in}}%
\pgfpathlineto{\pgfqpoint{5.664099in}{0.413320in}}%
\pgfpathlineto{\pgfqpoint{5.661273in}{0.413320in}}%
\pgfpathlineto{\pgfqpoint{5.658723in}{0.413320in}}%
\pgfpathlineto{\pgfqpoint{5.655919in}{0.413320in}}%
\pgfpathlineto{\pgfqpoint{5.653376in}{0.413320in}}%
\pgfpathlineto{\pgfqpoint{5.650563in}{0.413320in}}%
\pgfpathlineto{\pgfqpoint{5.648008in}{0.413320in}}%
\pgfpathlineto{\pgfqpoint{5.645243in}{0.413320in}}%
\pgfpathlineto{\pgfqpoint{5.642518in}{0.413320in}}%
\pgfpathlineto{\pgfqpoint{5.639852in}{0.413320in}}%
\pgfpathlineto{\pgfqpoint{5.637172in}{0.413320in}}%
\pgfpathlineto{\pgfqpoint{5.634496in}{0.413320in}}%
\pgfpathlineto{\pgfqpoint{5.631815in}{0.413320in}}%
\pgfpathlineto{\pgfqpoint{5.629232in}{0.413320in}}%
\pgfpathlineto{\pgfqpoint{5.626460in}{0.413320in}}%
\pgfpathlineto{\pgfqpoint{5.623868in}{0.413320in}}%
\pgfpathlineto{\pgfqpoint{5.621102in}{0.413320in}}%
\pgfpathlineto{\pgfqpoint{5.618526in}{0.413320in}}%
\pgfpathlineto{\pgfqpoint{5.615743in}{0.413320in}}%
\pgfpathlineto{\pgfqpoint{5.613235in}{0.413320in}}%
\pgfpathlineto{\pgfqpoint{5.610389in}{0.413320in}}%
\pgfpathlineto{\pgfqpoint{5.607698in}{0.413320in}}%
\pgfpathlineto{\pgfqpoint{5.605073in}{0.413320in}}%
\pgfpathlineto{\pgfqpoint{5.602352in}{0.413320in}}%
\pgfpathlineto{\pgfqpoint{5.599674in}{0.413320in}}%
\pgfpathlineto{\pgfqpoint{5.596999in}{0.413320in}}%
\pgfpathlineto{\pgfqpoint{5.594368in}{0.413320in}}%
\pgfpathlineto{\pgfqpoint{5.591641in}{0.413320in}}%
\pgfpathlineto{\pgfqpoint{5.589040in}{0.413320in}}%
\pgfpathlineto{\pgfqpoint{5.586269in}{0.413320in}}%
\pgfpathlineto{\pgfqpoint{5.583709in}{0.413320in}}%
\pgfpathlineto{\pgfqpoint{5.580914in}{0.413320in}}%
\pgfpathlineto{\pgfqpoint{5.578342in}{0.413320in}}%
\pgfpathlineto{\pgfqpoint{5.575596in}{0.413320in}}%
\pgfpathlineto{\pgfqpoint{5.572893in}{0.413320in}}%
\pgfpathlineto{\pgfqpoint{5.570215in}{0.413320in}}%
\pgfpathlineto{\pgfqpoint{5.567536in}{0.413320in}}%
\pgfpathlineto{\pgfqpoint{5.564940in}{0.413320in}}%
\pgfpathlineto{\pgfqpoint{5.562180in}{0.413320in}}%
\pgfpathlineto{\pgfqpoint{5.559612in}{0.413320in}}%
\pgfpathlineto{\pgfqpoint{5.556822in}{0.413320in}}%
\pgfpathlineto{\pgfqpoint{5.554198in}{0.413320in}}%
\pgfpathlineto{\pgfqpoint{5.551457in}{0.413320in}}%
\pgfpathlineto{\pgfqpoint{5.548921in}{0.413320in}}%
\pgfpathlineto{\pgfqpoint{5.546110in}{0.413320in}}%
\pgfpathlineto{\pgfqpoint{5.543421in}{0.413320in}}%
\pgfpathlineto{\pgfqpoint{5.540750in}{0.413320in}}%
\pgfpathlineto{\pgfqpoint{5.538074in}{0.413320in}}%
\pgfpathlineto{\pgfqpoint{5.535395in}{0.413320in}}%
\pgfpathlineto{\pgfqpoint{5.532717in}{0.413320in}}%
\pgfpathlineto{\pgfqpoint{5.530148in}{0.413320in}}%
\pgfpathlineto{\pgfqpoint{5.527360in}{0.413320in}}%
\pgfpathlineto{\pgfqpoint{5.524756in}{0.413320in}}%
\pgfpathlineto{\pgfqpoint{5.522003in}{0.413320in}}%
\pgfpathlineto{\pgfqpoint{5.519433in}{0.413320in}}%
\pgfpathlineto{\pgfqpoint{5.516646in}{0.413320in}}%
\pgfpathlineto{\pgfqpoint{5.514080in}{0.413320in}}%
\pgfpathlineto{\pgfqpoint{5.511290in}{0.413320in}}%
\pgfpathlineto{\pgfqpoint{5.508612in}{0.413320in}}%
\pgfpathlineto{\pgfqpoint{5.505933in}{0.413320in}}%
\pgfpathlineto{\pgfqpoint{5.503255in}{0.413320in}}%
\pgfpathlineto{\pgfqpoint{5.500687in}{0.413320in}}%
\pgfpathlineto{\pgfqpoint{5.497898in}{0.413320in}}%
\pgfpathlineto{\pgfqpoint{5.495346in}{0.413320in}}%
\pgfpathlineto{\pgfqpoint{5.492541in}{0.413320in}}%
\pgfpathlineto{\pgfqpoint{5.490000in}{0.413320in}}%
\pgfpathlineto{\pgfqpoint{5.487176in}{0.413320in}}%
\pgfpathlineto{\pgfqpoint{5.484641in}{0.413320in}}%
\pgfpathlineto{\pgfqpoint{5.481825in}{0.413320in}}%
\pgfpathlineto{\pgfqpoint{5.479152in}{0.413320in}}%
\pgfpathlineto{\pgfqpoint{5.476458in}{0.413320in}}%
\pgfpathlineto{\pgfqpoint{5.473792in}{0.413320in}}%
\pgfpathlineto{\pgfqpoint{5.471113in}{0.413320in}}%
\pgfpathlineto{\pgfqpoint{5.468425in}{0.413320in}}%
\pgfpathlineto{\pgfqpoint{5.465888in}{0.413320in}}%
\pgfpathlineto{\pgfqpoint{5.463079in}{0.413320in}}%
\pgfpathlineto{\pgfqpoint{5.460489in}{0.413320in}}%
\pgfpathlineto{\pgfqpoint{5.457721in}{0.413320in}}%
\pgfpathlineto{\pgfqpoint{5.455168in}{0.413320in}}%
\pgfpathlineto{\pgfqpoint{5.452365in}{0.413320in}}%
\pgfpathlineto{\pgfqpoint{5.449769in}{0.413320in}}%
\pgfpathlineto{\pgfqpoint{5.447021in}{0.413320in}}%
\pgfpathlineto{\pgfqpoint{5.444328in}{0.413320in}}%
\pgfpathlineto{\pgfqpoint{5.441698in}{0.413320in}}%
\pgfpathlineto{\pgfqpoint{5.438974in}{0.413320in}}%
\pgfpathlineto{\pgfqpoint{5.436295in}{0.413320in}}%
\pgfpathlineto{\pgfqpoint{5.433616in}{0.413320in}}%
\pgfpathlineto{\pgfqpoint{5.431015in}{0.413320in}}%
\pgfpathlineto{\pgfqpoint{5.428259in}{0.413320in}}%
\pgfpathlineto{\pgfqpoint{5.425661in}{0.413320in}}%
\pgfpathlineto{\pgfqpoint{5.422897in}{0.413320in}}%
\pgfpathlineto{\pgfqpoint{5.420304in}{0.413320in}}%
\pgfpathlineto{\pgfqpoint{5.417547in}{0.413320in}}%
\pgfpathlineto{\pgfqpoint{5.414954in}{0.413320in}}%
\pgfpathlineto{\pgfqpoint{5.412190in}{0.413320in}}%
\pgfpathlineto{\pgfqpoint{5.409507in}{0.413320in}}%
\pgfpathlineto{\pgfqpoint{5.406832in}{0.413320in}}%
\pgfpathlineto{\pgfqpoint{5.404154in}{0.413320in}}%
\pgfpathlineto{\pgfqpoint{5.401576in}{0.413320in}}%
\pgfpathlineto{\pgfqpoint{5.398784in}{0.413320in}}%
\pgfpathlineto{\pgfqpoint{5.396219in}{0.413320in}}%
\pgfpathlineto{\pgfqpoint{5.393441in}{0.413320in}}%
\pgfpathlineto{\pgfqpoint{5.390900in}{0.413320in}}%
\pgfpathlineto{\pgfqpoint{5.388083in}{0.413320in}}%
\pgfpathlineto{\pgfqpoint{5.385550in}{0.413320in}}%
\pgfpathlineto{\pgfqpoint{5.382725in}{0.413320in}}%
\pgfpathlineto{\pgfqpoint{5.380048in}{0.413320in}}%
\pgfpathlineto{\pgfqpoint{5.377370in}{0.413320in}}%
\pgfpathlineto{\pgfqpoint{5.374692in}{0.413320in}}%
\pgfpathlineto{\pgfqpoint{5.372013in}{0.413320in}}%
\pgfpathlineto{\pgfqpoint{5.369335in}{0.413320in}}%
\pgfpathlineto{\pgfqpoint{5.366727in}{0.413320in}}%
\pgfpathlineto{\pgfqpoint{5.363966in}{0.413320in}}%
\pgfpathlineto{\pgfqpoint{5.361370in}{0.413320in}}%
\pgfpathlineto{\pgfqpoint{5.358612in}{0.413320in}}%
\pgfpathlineto{\pgfqpoint{5.356056in}{0.413320in}}%
\pgfpathlineto{\pgfqpoint{5.353262in}{0.413320in}}%
\pgfpathlineto{\pgfqpoint{5.350723in}{0.413320in}}%
\pgfpathlineto{\pgfqpoint{5.347905in}{0.413320in}}%
\pgfpathlineto{\pgfqpoint{5.345224in}{0.413320in}}%
\pgfpathlineto{\pgfqpoint{5.342549in}{0.413320in}}%
\pgfpathlineto{\pgfqpoint{5.339872in}{0.413320in}}%
\pgfpathlineto{\pgfqpoint{5.337353in}{0.413320in}}%
\pgfpathlineto{\pgfqpoint{5.334510in}{0.413320in}}%
\pgfpathlineto{\pgfqpoint{5.331973in}{0.413320in}}%
\pgfpathlineto{\pgfqpoint{5.329159in}{0.413320in}}%
\pgfpathlineto{\pgfqpoint{5.326564in}{0.413320in}}%
\pgfpathlineto{\pgfqpoint{5.323802in}{0.413320in}}%
\pgfpathlineto{\pgfqpoint{5.321256in}{0.413320in}}%
\pgfpathlineto{\pgfqpoint{5.318430in}{0.413320in}}%
\pgfpathlineto{\pgfqpoint{5.315754in}{0.413320in}}%
\pgfpathlineto{\pgfqpoint{5.313089in}{0.413320in}}%
\pgfpathlineto{\pgfqpoint{5.310411in}{0.413320in}}%
\pgfpathlineto{\pgfqpoint{5.307731in}{0.413320in}}%
\pgfpathlineto{\pgfqpoint{5.305054in}{0.413320in}}%
\pgfpathlineto{\pgfqpoint{5.302443in}{0.413320in}}%
\pgfpathlineto{\pgfqpoint{5.299696in}{0.413320in}}%
\pgfpathlineto{\pgfqpoint{5.297140in}{0.413320in}}%
\pgfpathlineto{\pgfqpoint{5.294339in}{0.413320in}}%
\pgfpathlineto{\pgfqpoint{5.291794in}{0.413320in}}%
\pgfpathlineto{\pgfqpoint{5.288984in}{0.413320in}}%
\pgfpathlineto{\pgfqpoint{5.286436in}{0.413320in}}%
\pgfpathlineto{\pgfqpoint{5.283631in}{0.413320in}}%
\pgfpathlineto{\pgfqpoint{5.280947in}{0.413320in}}%
\pgfpathlineto{\pgfqpoint{5.278322in}{0.413320in}}%
\pgfpathlineto{\pgfqpoint{5.275589in}{0.413320in}}%
\pgfpathlineto{\pgfqpoint{5.272913in}{0.413320in}}%
\pgfpathlineto{\pgfqpoint{5.270238in}{0.413320in}}%
\pgfpathlineto{\pgfqpoint{5.267691in}{0.413320in}}%
\pgfpathlineto{\pgfqpoint{5.264876in}{0.413320in}}%
\pgfpathlineto{\pgfqpoint{5.262264in}{0.413320in}}%
\pgfpathlineto{\pgfqpoint{5.259511in}{0.413320in}}%
\pgfpathlineto{\pgfqpoint{5.256973in}{0.413320in}}%
\pgfpathlineto{\pgfqpoint{5.254236in}{0.413320in}}%
\pgfpathlineto{\pgfqpoint{5.251590in}{0.413320in}}%
\pgfpathlineto{\pgfqpoint{5.248816in}{0.413320in}}%
\pgfpathlineto{\pgfqpoint{5.246130in}{0.413320in}}%
\pgfpathlineto{\pgfqpoint{5.243445in}{0.413320in}}%
\pgfpathlineto{\pgfqpoint{5.240777in}{0.413320in}}%
\pgfpathlineto{\pgfqpoint{5.238173in}{0.413320in}}%
\pgfpathlineto{\pgfqpoint{5.235409in}{0.413320in}}%
\pgfpathlineto{\pgfqpoint{5.232855in}{0.413320in}}%
\pgfpathlineto{\pgfqpoint{5.230059in}{0.413320in}}%
\pgfpathlineto{\pgfqpoint{5.227470in}{0.413320in}}%
\pgfpathlineto{\pgfqpoint{5.224695in}{0.413320in}}%
\pgfpathlineto{\pgfqpoint{5.222151in}{0.413320in}}%
\pgfpathlineto{\pgfqpoint{5.219345in}{0.413320in}}%
\pgfpathlineto{\pgfqpoint{5.216667in}{0.413320in}}%
\pgfpathlineto{\pgfqpoint{5.214027in}{0.413320in}}%
\pgfpathlineto{\pgfqpoint{5.211299in}{0.413320in}}%
\pgfpathlineto{\pgfqpoint{5.208630in}{0.413320in}}%
\pgfpathlineto{\pgfqpoint{5.205952in}{0.413320in}}%
\pgfpathlineto{\pgfqpoint{5.203388in}{0.413320in}}%
\pgfpathlineto{\pgfqpoint{5.200594in}{0.413320in}}%
\pgfpathlineto{\pgfqpoint{5.198008in}{0.413320in}}%
\pgfpathlineto{\pgfqpoint{5.195239in}{0.413320in}}%
\pgfpathlineto{\pgfqpoint{5.192680in}{0.413320in}}%
\pgfpathlineto{\pgfqpoint{5.189880in}{0.413320in}}%
\pgfpathlineto{\pgfqpoint{5.187294in}{0.413320in}}%
\pgfpathlineto{\pgfqpoint{5.184522in}{0.413320in}}%
\pgfpathlineto{\pgfqpoint{5.181848in}{0.413320in}}%
\pgfpathlineto{\pgfqpoint{5.179188in}{0.413320in}}%
\pgfpathlineto{\pgfqpoint{5.176477in}{0.413320in}}%
\pgfpathlineto{\pgfqpoint{5.173925in}{0.413320in}}%
\pgfpathlineto{\pgfqpoint{5.171133in}{0.413320in}}%
\pgfpathlineto{\pgfqpoint{5.168591in}{0.413320in}}%
\pgfpathlineto{\pgfqpoint{5.165775in}{0.413320in}}%
\pgfpathlineto{\pgfqpoint{5.163243in}{0.413320in}}%
\pgfpathlineto{\pgfqpoint{5.160420in}{0.413320in}}%
\pgfpathlineto{\pgfqpoint{5.157815in}{0.413320in}}%
\pgfpathlineto{\pgfqpoint{5.155059in}{0.413320in}}%
\pgfpathlineto{\pgfqpoint{5.152382in}{0.413320in}}%
\pgfpathlineto{\pgfqpoint{5.149734in}{0.413320in}}%
\pgfpathlineto{\pgfqpoint{5.147029in}{0.413320in}}%
\pgfpathlineto{\pgfqpoint{5.144349in}{0.413320in}}%
\pgfpathlineto{\pgfqpoint{5.141660in}{0.413320in}}%
\pgfpathlineto{\pgfqpoint{5.139072in}{0.413320in}}%
\pgfpathlineto{\pgfqpoint{5.136311in}{0.413320in}}%
\pgfpathlineto{\pgfqpoint{5.133716in}{0.413320in}}%
\pgfpathlineto{\pgfqpoint{5.130953in}{0.413320in}}%
\pgfpathlineto{\pgfqpoint{5.128421in}{0.413320in}}%
\pgfpathlineto{\pgfqpoint{5.125599in}{0.413320in}}%
\pgfpathlineto{\pgfqpoint{5.123042in}{0.413320in}}%
\pgfpathlineto{\pgfqpoint{5.120243in}{0.413320in}}%
\pgfpathlineto{\pgfqpoint{5.117550in}{0.413320in}}%
\pgfpathlineto{\pgfqpoint{5.114887in}{0.413320in}}%
\pgfpathlineto{\pgfqpoint{5.112209in}{0.413320in}}%
\pgfpathlineto{\pgfqpoint{5.109530in}{0.413320in}}%
\pgfpathlineto{\pgfqpoint{5.106842in}{0.413320in}}%
\pgfpathlineto{\pgfqpoint{5.104312in}{0.413320in}}%
\pgfpathlineto{\pgfqpoint{5.101496in}{0.413320in}}%
\pgfpathlineto{\pgfqpoint{5.098948in}{0.413320in}}%
\pgfpathlineto{\pgfqpoint{5.096142in}{0.413320in}}%
\pgfpathlineto{\pgfqpoint{5.093579in}{0.413320in}}%
\pgfpathlineto{\pgfqpoint{5.090788in}{0.413320in}}%
\pgfpathlineto{\pgfqpoint{5.088103in}{0.413320in}}%
\pgfpathlineto{\pgfqpoint{5.085426in}{0.413320in}}%
\pgfpathlineto{\pgfqpoint{5.082746in}{0.413320in}}%
\pgfpathlineto{\pgfqpoint{5.080067in}{0.413320in}}%
\pgfpathlineto{\pgfqpoint{5.077390in}{0.413320in}}%
\pgfpathlineto{\pgfqpoint{5.074851in}{0.413320in}}%
\pgfpathlineto{\pgfqpoint{5.072030in}{0.413320in}}%
\pgfpathlineto{\pgfqpoint{5.069463in}{0.413320in}}%
\pgfpathlineto{\pgfqpoint{5.066677in}{0.413320in}}%
\pgfpathlineto{\pgfqpoint{5.064144in}{0.413320in}}%
\pgfpathlineto{\pgfqpoint{5.061315in}{0.413320in}}%
\pgfpathlineto{\pgfqpoint{5.058711in}{0.413320in}}%
\pgfpathlineto{\pgfqpoint{5.055952in}{0.413320in}}%
\pgfpathlineto{\pgfqpoint{5.053284in}{0.413320in}}%
\pgfpathlineto{\pgfqpoint{5.050606in}{0.413320in}}%
\pgfpathlineto{\pgfqpoint{5.047924in}{0.413320in}}%
\pgfpathlineto{\pgfqpoint{5.045249in}{0.413320in}}%
\pgfpathlineto{\pgfqpoint{5.042572in}{0.413320in}}%
\pgfpathlineto{\pgfqpoint{5.039962in}{0.413320in}}%
\pgfpathlineto{\pgfqpoint{5.037214in}{0.413320in}}%
\pgfpathlineto{\pgfqpoint{5.034649in}{0.413320in}}%
\pgfpathlineto{\pgfqpoint{5.031849in}{0.413320in}}%
\pgfpathlineto{\pgfqpoint{5.029275in}{0.413320in}}%
\pgfpathlineto{\pgfqpoint{5.026501in}{0.413320in}}%
\pgfpathlineto{\pgfqpoint{5.023927in}{0.413320in}}%
\pgfpathlineto{\pgfqpoint{5.021147in}{0.413320in}}%
\pgfpathlineto{\pgfqpoint{5.018466in}{0.413320in}}%
\pgfpathlineto{\pgfqpoint{5.015820in}{0.413320in}}%
\pgfpathlineto{\pgfqpoint{5.013104in}{0.413320in}}%
\pgfpathlineto{\pgfqpoint{5.010562in}{0.413320in}}%
\pgfpathlineto{\pgfqpoint{5.007751in}{0.413320in}}%
\pgfpathlineto{\pgfqpoint{5.005178in}{0.413320in}}%
\pgfpathlineto{\pgfqpoint{5.002384in}{0.413320in}}%
\pgfpathlineto{\pgfqpoint{4.999780in}{0.413320in}}%
\pgfpathlineto{\pgfqpoint{4.997028in}{0.413320in}}%
\pgfpathlineto{\pgfqpoint{4.994390in}{0.413320in}}%
\pgfpathlineto{\pgfqpoint{4.991683in}{0.413320in}}%
\pgfpathlineto{\pgfqpoint{4.989001in}{0.413320in}}%
\pgfpathlineto{\pgfqpoint{4.986325in}{0.413320in}}%
\pgfpathlineto{\pgfqpoint{4.983637in}{0.413320in}}%
\pgfpathlineto{\pgfqpoint{4.980967in}{0.413320in}}%
\pgfpathlineto{\pgfqpoint{4.978287in}{0.413320in}}%
\pgfpathlineto{\pgfqpoint{4.975703in}{0.413320in}}%
\pgfpathlineto{\pgfqpoint{4.972933in}{0.413320in}}%
\pgfpathlineto{\pgfqpoint{4.970314in}{0.413320in}}%
\pgfpathlineto{\pgfqpoint{4.967575in}{0.413320in}}%
\pgfpathlineto{\pgfqpoint{4.965002in}{0.413320in}}%
\pgfpathlineto{\pgfqpoint{4.962219in}{0.413320in}}%
\pgfpathlineto{\pgfqpoint{4.959689in}{0.413320in}}%
\pgfpathlineto{\pgfqpoint{4.956862in}{0.413320in}}%
\pgfpathlineto{\pgfqpoint{4.954182in}{0.413320in}}%
\pgfpathlineto{\pgfqpoint{4.951504in}{0.413320in}}%
\pgfpathlineto{\pgfqpoint{4.948827in}{0.413320in}}%
\pgfpathlineto{\pgfqpoint{4.946151in}{0.413320in}}%
\pgfpathlineto{\pgfqpoint{4.943466in}{0.413320in}}%
\pgfpathlineto{\pgfqpoint{4.940881in}{0.413320in}}%
\pgfpathlineto{\pgfqpoint{4.938112in}{0.413320in}}%
\pgfpathlineto{\pgfqpoint{4.935515in}{0.413320in}}%
\pgfpathlineto{\pgfqpoint{4.932742in}{0.413320in}}%
\pgfpathlineto{\pgfqpoint{4.930170in}{0.413320in}}%
\pgfpathlineto{\pgfqpoint{4.927400in}{0.413320in}}%
\pgfpathlineto{\pgfqpoint{4.924708in}{0.413320in}}%
\pgfpathlineto{\pgfqpoint{4.922041in}{0.413320in}}%
\pgfpathlineto{\pgfqpoint{4.919352in}{0.413320in}}%
\pgfpathlineto{\pgfqpoint{4.916681in}{0.413320in}}%
\pgfpathlineto{\pgfqpoint{4.914009in}{0.413320in}}%
\pgfpathlineto{\pgfqpoint{4.911435in}{0.413320in}}%
\pgfpathlineto{\pgfqpoint{4.908648in}{0.413320in}}%
\pgfpathlineto{\pgfqpoint{4.906096in}{0.413320in}}%
\pgfpathlineto{\pgfqpoint{4.903295in}{0.413320in}}%
\pgfpathlineto{\pgfqpoint{4.900712in}{0.413320in}}%
\pgfpathlineto{\pgfqpoint{4.897938in}{0.413320in}}%
\pgfpathlineto{\pgfqpoint{4.895399in}{0.413320in}}%
\pgfpathlineto{\pgfqpoint{4.892611in}{0.413320in}}%
\pgfpathlineto{\pgfqpoint{4.889902in}{0.413320in}}%
\pgfpathlineto{\pgfqpoint{4.887211in}{0.413320in}}%
\pgfpathlineto{\pgfqpoint{4.884540in}{0.413320in}}%
\pgfpathlineto{\pgfqpoint{4.881864in}{0.413320in}}%
\pgfpathlineto{\pgfqpoint{4.879180in}{0.413320in}}%
\pgfpathlineto{\pgfqpoint{4.876636in}{0.413320in}}%
\pgfpathlineto{\pgfqpoint{4.873832in}{0.413320in}}%
\pgfpathlineto{\pgfqpoint{4.871209in}{0.413320in}}%
\pgfpathlineto{\pgfqpoint{4.868474in}{0.413320in}}%
\pgfpathlineto{\pgfqpoint{4.865910in}{0.413320in}}%
\pgfpathlineto{\pgfqpoint{4.863116in}{0.413320in}}%
\pgfpathlineto{\pgfqpoint{4.860544in}{0.413320in}}%
\pgfpathlineto{\pgfqpoint{4.857807in}{0.413320in}}%
\pgfpathlineto{\pgfqpoint{4.855070in}{0.413320in}}%
\pgfpathlineto{\pgfqpoint{4.852404in}{0.413320in}}%
\pgfpathlineto{\pgfqpoint{4.849715in}{0.413320in}}%
\pgfpathlineto{\pgfqpoint{4.847127in}{0.413320in}}%
\pgfpathlineto{\pgfqpoint{4.844361in}{0.413320in}}%
\pgfpathlineto{\pgfqpoint{4.842380in}{0.413320in}}%
\pgfpathlineto{\pgfqpoint{4.839922in}{0.413320in}}%
\pgfpathlineto{\pgfqpoint{4.837992in}{0.413320in}}%
\pgfpathlineto{\pgfqpoint{4.833657in}{0.413320in}}%
\pgfpathlineto{\pgfqpoint{4.831045in}{0.413320in}}%
\pgfpathlineto{\pgfqpoint{4.828291in}{0.413320in}}%
\pgfpathlineto{\pgfqpoint{4.825619in}{0.413320in}}%
\pgfpathlineto{\pgfqpoint{4.822945in}{0.413320in}}%
\pgfpathlineto{\pgfqpoint{4.820265in}{0.413320in}}%
\pgfpathlineto{\pgfqpoint{4.817587in}{0.413320in}}%
\pgfpathlineto{\pgfqpoint{4.814907in}{0.413320in}}%
\pgfpathlineto{\pgfqpoint{4.812377in}{0.413320in}}%
\pgfpathlineto{\pgfqpoint{4.809538in}{0.413320in}}%
\pgfpathlineto{\pgfqpoint{4.807017in}{0.413320in}}%
\pgfpathlineto{\pgfqpoint{4.804193in}{0.413320in}}%
\pgfpathlineto{\pgfqpoint{4.801586in}{0.413320in}}%
\pgfpathlineto{\pgfqpoint{4.798830in}{0.413320in}}%
\pgfpathlineto{\pgfqpoint{4.796274in}{0.413320in}}%
\pgfpathlineto{\pgfqpoint{4.793512in}{0.413320in}}%
\pgfpathlineto{\pgfqpoint{4.790798in}{0.413320in}}%
\pgfpathlineto{\pgfqpoint{4.788116in}{0.413320in}}%
\pgfpathlineto{\pgfqpoint{4.785445in}{0.413320in}}%
\pgfpathlineto{\pgfqpoint{4.782872in}{0.413320in}}%
\pgfpathlineto{\pgfqpoint{4.780083in}{0.413320in}}%
\pgfpathlineto{\pgfqpoint{4.777535in}{0.413320in}}%
\pgfpathlineto{\pgfqpoint{4.774732in}{0.413320in}}%
\pgfpathlineto{\pgfqpoint{4.772198in}{0.413320in}}%
\pgfpathlineto{\pgfqpoint{4.769367in}{0.413320in}}%
\pgfpathlineto{\pgfqpoint{4.766783in}{0.413320in}}%
\pgfpathlineto{\pgfqpoint{4.764018in}{0.413320in}}%
\pgfpathlineto{\pgfqpoint{4.761337in}{0.413320in}}%
\pgfpathlineto{\pgfqpoint{4.758653in}{0.413320in}}%
\pgfpathlineto{\pgfqpoint{4.755983in}{0.413320in}}%
\pgfpathlineto{\pgfqpoint{4.753298in}{0.413320in}}%
\pgfpathlineto{\pgfqpoint{4.750627in}{0.413320in}}%
\pgfpathlineto{\pgfqpoint{4.748081in}{0.413320in}}%
\pgfpathlineto{\pgfqpoint{4.745256in}{0.413320in}}%
\pgfpathlineto{\pgfqpoint{4.742696in}{0.413320in}}%
\pgfpathlineto{\pgfqpoint{4.739912in}{0.413320in}}%
\pgfpathlineto{\pgfqpoint{4.737348in}{0.413320in}}%
\pgfpathlineto{\pgfqpoint{4.734552in}{0.413320in}}%
\pgfpathlineto{\pgfqpoint{4.731901in}{0.413320in}}%
\pgfpathlineto{\pgfqpoint{4.729233in}{0.413320in}}%
\pgfpathlineto{\pgfqpoint{4.726508in}{0.413320in}}%
\pgfpathlineto{\pgfqpoint{4.723873in}{0.413320in}}%
\pgfpathlineto{\pgfqpoint{4.721160in}{0.413320in}}%
\pgfpathlineto{\pgfqpoint{4.718486in}{0.413320in}}%
\pgfpathlineto{\pgfqpoint{4.715806in}{0.413320in}}%
\pgfpathlineto{\pgfqpoint{4.713275in}{0.413320in}}%
\pgfpathlineto{\pgfqpoint{4.710437in}{0.413320in}}%
\pgfpathlineto{\pgfqpoint{4.707824in}{0.413320in}}%
\pgfpathlineto{\pgfqpoint{4.705094in}{0.413320in}}%
\pgfpathlineto{\pgfqpoint{4.702517in}{0.413320in}}%
\pgfpathlineto{\pgfqpoint{4.699734in}{0.413320in}}%
\pgfpathlineto{\pgfqpoint{4.697170in}{0.413320in}}%
\pgfpathlineto{\pgfqpoint{4.694381in}{0.413320in}}%
\pgfpathlineto{\pgfqpoint{4.691694in}{0.413320in}}%
\pgfpathlineto{\pgfqpoint{4.689051in}{0.413320in}}%
\pgfpathlineto{\pgfqpoint{4.686337in}{0.413320in}}%
\pgfpathlineto{\pgfqpoint{4.683799in}{0.413320in}}%
\pgfpathlineto{\pgfqpoint{4.680988in}{0.413320in}}%
\pgfpathlineto{\pgfqpoint{4.678448in}{0.413320in}}%
\pgfpathlineto{\pgfqpoint{4.675619in}{0.413320in}}%
\pgfpathlineto{\pgfqpoint{4.673068in}{0.413320in}}%
\pgfpathlineto{\pgfqpoint{4.670261in}{0.413320in}}%
\pgfpathlineto{\pgfqpoint{4.667764in}{0.413320in}}%
\pgfpathlineto{\pgfqpoint{4.664923in}{0.413320in}}%
\pgfpathlineto{\pgfqpoint{4.662237in}{0.413320in}}%
\pgfpathlineto{\pgfqpoint{4.659590in}{0.413320in}}%
\pgfpathlineto{\pgfqpoint{4.656873in}{0.413320in}}%
\pgfpathlineto{\pgfqpoint{4.654203in}{0.413320in}}%
\pgfpathlineto{\pgfqpoint{4.651524in}{0.413320in}}%
\pgfpathlineto{\pgfqpoint{4.648922in}{0.413320in}}%
\pgfpathlineto{\pgfqpoint{4.646169in}{0.413320in}}%
\pgfpathlineto{\pgfqpoint{4.643628in}{0.413320in}}%
\pgfpathlineto{\pgfqpoint{4.640809in}{0.413320in}}%
\pgfpathlineto{\pgfqpoint{4.638204in}{0.413320in}}%
\pgfpathlineto{\pgfqpoint{4.635445in}{0.413320in}}%
\pgfpathlineto{\pgfqpoint{4.632902in}{0.413320in}}%
\pgfpathlineto{\pgfqpoint{4.630096in}{0.413320in}}%
\pgfpathlineto{\pgfqpoint{4.627411in}{0.413320in}}%
\pgfpathlineto{\pgfqpoint{4.624741in}{0.413320in}}%
\pgfpathlineto{\pgfqpoint{4.622056in}{0.413320in}}%
\pgfpathlineto{\pgfqpoint{4.619529in}{0.413320in}}%
\pgfpathlineto{\pgfqpoint{4.616702in}{0.413320in}}%
\pgfpathlineto{\pgfqpoint{4.614134in}{0.413320in}}%
\pgfpathlineto{\pgfqpoint{4.611350in}{0.413320in}}%
\pgfpathlineto{\pgfqpoint{4.608808in}{0.413320in}}%
\pgfpathlineto{\pgfqpoint{4.605990in}{0.413320in}}%
\pgfpathlineto{\pgfqpoint{4.603430in}{0.413320in}}%
\pgfpathlineto{\pgfqpoint{4.600633in}{0.413320in}}%
\pgfpathlineto{\pgfqpoint{4.597951in}{0.413320in}}%
\pgfpathlineto{\pgfqpoint{4.595281in}{0.413320in}}%
\pgfpathlineto{\pgfqpoint{4.592589in}{0.413320in}}%
\pgfpathlineto{\pgfqpoint{4.589920in}{0.413320in}}%
\pgfpathlineto{\pgfqpoint{4.587244in}{0.413320in}}%
\pgfpathlineto{\pgfqpoint{4.584672in}{0.413320in}}%
\pgfpathlineto{\pgfqpoint{4.581888in}{0.413320in}}%
\pgfpathlineto{\pgfqpoint{4.579305in}{0.413320in}}%
\pgfpathlineto{\pgfqpoint{4.576531in}{0.413320in}}%
\pgfpathlineto{\pgfqpoint{4.573947in}{0.413320in}}%
\pgfpathlineto{\pgfqpoint{4.571171in}{0.413320in}}%
\pgfpathlineto{\pgfqpoint{4.568612in}{0.413320in}}%
\pgfpathlineto{\pgfqpoint{4.565820in}{0.413320in}}%
\pgfpathlineto{\pgfqpoint{4.563125in}{0.413320in}}%
\pgfpathlineto{\pgfqpoint{4.560448in}{0.413320in}}%
\pgfpathlineto{\pgfqpoint{4.557777in}{0.413320in}}%
\pgfpathlineto{\pgfqpoint{4.555106in}{0.413320in}}%
\pgfpathlineto{\pgfqpoint{4.552425in}{0.413320in}}%
\pgfpathlineto{\pgfqpoint{4.549822in}{0.413320in}}%
\pgfpathlineto{\pgfqpoint{4.547064in}{0.413320in}}%
\pgfpathlineto{\pgfqpoint{4.544464in}{0.413320in}}%
\pgfpathlineto{\pgfqpoint{4.541711in}{0.413320in}}%
\pgfpathlineto{\pgfqpoint{4.539144in}{0.413320in}}%
\pgfpathlineto{\pgfqpoint{4.536400in}{0.413320in}}%
\pgfpathlineto{\pgfqpoint{4.533764in}{0.413320in}}%
\pgfpathlineto{\pgfqpoint{4.530990in}{0.413320in}}%
\pgfpathlineto{\pgfqpoint{4.528307in}{0.413320in}}%
\pgfpathlineto{\pgfqpoint{4.525640in}{0.413320in}}%
\pgfpathlineto{\pgfqpoint{4.522962in}{0.413320in}}%
\pgfpathlineto{\pgfqpoint{4.520345in}{0.413320in}}%
\pgfpathlineto{\pgfqpoint{4.517598in}{0.413320in}}%
\pgfpathlineto{\pgfqpoint{4.515080in}{0.413320in}}%
\pgfpathlineto{\pgfqpoint{4.512246in}{0.413320in}}%
\pgfpathlineto{\pgfqpoint{4.509643in}{0.413320in}}%
\pgfpathlineto{\pgfqpoint{4.506893in}{0.413320in}}%
\pgfpathlineto{\pgfqpoint{4.504305in}{0.413320in}}%
\pgfpathlineto{\pgfqpoint{4.501529in}{0.413320in}}%
\pgfpathlineto{\pgfqpoint{4.498850in}{0.413320in}}%
\pgfpathlineto{\pgfqpoint{4.496167in}{0.413320in}}%
\pgfpathlineto{\pgfqpoint{4.493492in}{0.413320in}}%
\pgfpathlineto{\pgfqpoint{4.490822in}{0.413320in}}%
\pgfpathlineto{\pgfqpoint{4.488130in}{0.413320in}}%
\pgfpathlineto{\pgfqpoint{4.485581in}{0.413320in}}%
\pgfpathlineto{\pgfqpoint{4.482778in}{0.413320in}}%
\pgfpathlineto{\pgfqpoint{4.480201in}{0.413320in}}%
\pgfpathlineto{\pgfqpoint{4.477430in}{0.413320in}}%
\pgfpathlineto{\pgfqpoint{4.474861in}{0.413320in}}%
\pgfpathlineto{\pgfqpoint{4.472059in}{0.413320in}}%
\pgfpathlineto{\pgfqpoint{4.469492in}{0.413320in}}%
\pgfpathlineto{\pgfqpoint{4.466717in}{0.413320in}}%
\pgfpathlineto{\pgfqpoint{4.464029in}{0.413320in}}%
\pgfpathlineto{\pgfqpoint{4.461367in}{0.413320in}}%
\pgfpathlineto{\pgfqpoint{4.458681in}{0.413320in}}%
\pgfpathlineto{\pgfqpoint{4.456138in}{0.413320in}}%
\pgfpathlineto{\pgfqpoint{4.453312in}{0.413320in}}%
\pgfpathlineto{\pgfqpoint{4.450767in}{0.413320in}}%
\pgfpathlineto{\pgfqpoint{4.447965in}{0.413320in}}%
\pgfpathlineto{\pgfqpoint{4.445423in}{0.413320in}}%
\pgfpathlineto{\pgfqpoint{4.442611in}{0.413320in}}%
\pgfpathlineto{\pgfqpoint{4.440041in}{0.413320in}}%
\pgfpathlineto{\pgfqpoint{4.437253in}{0.413320in}}%
\pgfpathlineto{\pgfqpoint{4.434569in}{0.413320in}}%
\pgfpathlineto{\pgfqpoint{4.431901in}{0.413320in}}%
\pgfpathlineto{\pgfqpoint{4.429220in}{0.413320in}}%
\pgfpathlineto{\pgfqpoint{4.426534in}{0.413320in}}%
\pgfpathlineto{\pgfqpoint{4.423863in}{0.413320in}}%
\pgfpathlineto{\pgfqpoint{4.421292in}{0.413320in}}%
\pgfpathlineto{\pgfqpoint{4.418506in}{0.413320in}}%
\pgfpathlineto{\pgfqpoint{4.415932in}{0.413320in}}%
\pgfpathlineto{\pgfqpoint{4.413149in}{0.413320in}}%
\pgfpathlineto{\pgfqpoint{4.410587in}{0.413320in}}%
\pgfpathlineto{\pgfqpoint{4.407788in}{0.413320in}}%
\pgfpathlineto{\pgfqpoint{4.405234in}{0.413320in}}%
\pgfpathlineto{\pgfqpoint{4.402468in}{0.413320in}}%
\pgfpathlineto{\pgfqpoint{4.399745in}{0.413320in}}%
\pgfpathlineto{\pgfqpoint{4.397076in}{0.413320in}}%
\pgfpathlineto{\pgfqpoint{4.394400in}{0.413320in}}%
\pgfpathlineto{\pgfqpoint{4.391721in}{0.413320in}}%
\pgfpathlineto{\pgfqpoint{4.389044in}{0.413320in}}%
\pgfpathlineto{\pgfqpoint{4.386431in}{0.413320in}}%
\pgfpathlineto{\pgfqpoint{4.383674in}{0.413320in}}%
\pgfpathlineto{\pgfqpoint{4.381097in}{0.413320in}}%
\pgfpathlineto{\pgfqpoint{4.378329in}{0.413320in}}%
\pgfpathlineto{\pgfqpoint{4.375761in}{0.413320in}}%
\pgfpathlineto{\pgfqpoint{4.372976in}{0.413320in}}%
\pgfpathlineto{\pgfqpoint{4.370437in}{0.413320in}}%
\pgfpathlineto{\pgfqpoint{4.367646in}{0.413320in}}%
\pgfpathlineto{\pgfqpoint{4.364936in}{0.413320in}}%
\pgfpathlineto{\pgfqpoint{4.362270in}{0.413320in}}%
\pgfpathlineto{\pgfqpoint{4.359582in}{0.413320in}}%
\pgfpathlineto{\pgfqpoint{4.357014in}{0.413320in}}%
\pgfpathlineto{\pgfqpoint{4.354224in}{0.413320in}}%
\pgfpathlineto{\pgfqpoint{4.351645in}{0.413320in}}%
\pgfpathlineto{\pgfqpoint{4.348868in}{0.413320in}}%
\pgfpathlineto{\pgfqpoint{4.346263in}{0.413320in}}%
\pgfpathlineto{\pgfqpoint{4.343510in}{0.413320in}}%
\pgfpathlineto{\pgfqpoint{4.340976in}{0.413320in}}%
\pgfpathlineto{\pgfqpoint{4.338154in}{0.413320in}}%
\pgfpathlineto{\pgfqpoint{4.335463in}{0.413320in}}%
\pgfpathlineto{\pgfqpoint{4.332796in}{0.413320in}}%
\pgfpathlineto{\pgfqpoint{4.330118in}{0.413320in}}%
\pgfpathlineto{\pgfqpoint{4.327440in}{0.413320in}}%
\pgfpathlineto{\pgfqpoint{4.324760in}{0.413320in}}%
\pgfpathlineto{\pgfqpoint{4.322181in}{0.413320in}}%
\pgfpathlineto{\pgfqpoint{4.319405in}{0.413320in}}%
\pgfpathlineto{\pgfqpoint{4.316856in}{0.413320in}}%
\pgfpathlineto{\pgfqpoint{4.314032in}{0.413320in}}%
\pgfpathlineto{\pgfqpoint{4.311494in}{0.413320in}}%
\pgfpathlineto{\pgfqpoint{4.308691in}{0.413320in}}%
\pgfpathlineto{\pgfqpoint{4.306118in}{0.413320in}}%
\pgfpathlineto{\pgfqpoint{4.303357in}{0.413320in}}%
\pgfpathlineto{\pgfqpoint{4.300656in}{0.413320in}}%
\pgfpathlineto{\pgfqpoint{4.297977in}{0.413320in}}%
\pgfpathlineto{\pgfqpoint{4.295299in}{0.413320in}}%
\pgfpathlineto{\pgfqpoint{4.292786in}{0.413320in}}%
\pgfpathlineto{\pgfqpoint{4.289936in}{0.413320in}}%
\pgfpathlineto{\pgfqpoint{4.287399in}{0.413320in}}%
\pgfpathlineto{\pgfqpoint{4.284586in}{0.413320in}}%
\pgfpathlineto{\pgfqpoint{4.282000in}{0.413320in}}%
\pgfpathlineto{\pgfqpoint{4.279212in}{0.413320in}}%
\pgfpathlineto{\pgfqpoint{4.276635in}{0.413320in}}%
\pgfpathlineto{\pgfqpoint{4.273874in}{0.413320in}}%
\pgfpathlineto{\pgfqpoint{4.271187in}{0.413320in}}%
\pgfpathlineto{\pgfqpoint{4.268590in}{0.413320in}}%
\pgfpathlineto{\pgfqpoint{4.265824in}{0.413320in}}%
\pgfpathlineto{\pgfqpoint{4.263157in}{0.413320in}}%
\pgfpathlineto{\pgfqpoint{4.260477in}{0.413320in}}%
\pgfpathlineto{\pgfqpoint{4.257958in}{0.413320in}}%
\pgfpathlineto{\pgfqpoint{4.255120in}{0.413320in}}%
\pgfpathlineto{\pgfqpoint{4.252581in}{0.413320in}}%
\pgfpathlineto{\pgfqpoint{4.249767in}{0.413320in}}%
\pgfpathlineto{\pgfqpoint{4.247225in}{0.413320in}}%
\pgfpathlineto{\pgfqpoint{4.244394in}{0.413320in}}%
\pgfpathlineto{\pgfqpoint{4.241900in}{0.413320in}}%
\pgfpathlineto{\pgfqpoint{4.239084in}{0.413320in}}%
\pgfpathlineto{\pgfqpoint{4.236375in}{0.413320in}}%
\pgfpathlineto{\pgfqpoint{4.233691in}{0.413320in}}%
\pgfpathlineto{\pgfqpoint{4.231013in}{0.413320in}}%
\pgfpathlineto{\pgfqpoint{4.228331in}{0.413320in}}%
\pgfpathlineto{\pgfqpoint{4.225654in}{0.413320in}}%
\pgfpathlineto{\pgfqpoint{4.223082in}{0.413320in}}%
\pgfpathlineto{\pgfqpoint{4.220304in}{0.413320in}}%
\pgfpathlineto{\pgfqpoint{4.217694in}{0.413320in}}%
\pgfpathlineto{\pgfqpoint{4.214948in}{0.413320in}}%
\pgfpathlineto{\pgfqpoint{4.212383in}{0.413320in}}%
\pgfpathlineto{\pgfqpoint{4.209597in}{0.413320in}}%
\pgfpathlineto{\pgfqpoint{4.207076in}{0.413320in}}%
\pgfpathlineto{\pgfqpoint{4.204240in}{0.413320in}}%
\pgfpathlineto{\pgfqpoint{4.201542in}{0.413320in}}%
\pgfpathlineto{\pgfqpoint{4.198878in}{0.413320in}}%
\pgfpathlineto{\pgfqpoint{4.196186in}{0.413320in}}%
\pgfpathlineto{\pgfqpoint{4.193638in}{0.413320in}}%
\pgfpathlineto{\pgfqpoint{4.190842in}{0.413320in}}%
\pgfpathlineto{\pgfqpoint{4.188318in}{0.413320in}}%
\pgfpathlineto{\pgfqpoint{4.185481in}{0.413320in}}%
\pgfpathlineto{\pgfqpoint{4.182899in}{0.413320in}}%
\pgfpathlineto{\pgfqpoint{4.180129in}{0.413320in}}%
\pgfpathlineto{\pgfqpoint{4.177593in}{0.413320in}}%
\pgfpathlineto{\pgfqpoint{4.174770in}{0.413320in}}%
\pgfpathlineto{\pgfqpoint{4.172093in}{0.413320in}}%
\pgfpathlineto{\pgfqpoint{4.169415in}{0.413320in}}%
\pgfpathlineto{\pgfqpoint{4.166737in}{0.413320in}}%
\pgfpathlineto{\pgfqpoint{4.164059in}{0.413320in}}%
\pgfpathlineto{\pgfqpoint{4.161380in}{0.413320in}}%
\pgfpathlineto{\pgfqpoint{4.158806in}{0.413320in}}%
\pgfpathlineto{\pgfqpoint{4.156016in}{0.413320in}}%
\pgfpathlineto{\pgfqpoint{4.153423in}{0.413320in}}%
\pgfpathlineto{\pgfqpoint{4.150665in}{0.413320in}}%
\pgfpathlineto{\pgfqpoint{4.148082in}{0.413320in}}%
\pgfpathlineto{\pgfqpoint{4.145310in}{0.413320in}}%
\pgfpathlineto{\pgfqpoint{4.142713in}{0.413320in}}%
\pgfpathlineto{\pgfqpoint{4.139963in}{0.413320in}}%
\pgfpathlineto{\pgfqpoint{4.137272in}{0.413320in}}%
\pgfpathlineto{\pgfqpoint{4.134615in}{0.413320in}}%
\pgfpathlineto{\pgfqpoint{4.131920in}{0.413320in}}%
\pgfpathlineto{\pgfqpoint{4.129349in}{0.413320in}}%
\pgfpathlineto{\pgfqpoint{4.126553in}{0.413320in}}%
\pgfpathlineto{\pgfqpoint{4.124019in}{0.413320in}}%
\pgfpathlineto{\pgfqpoint{4.121205in}{0.413320in}}%
\pgfpathlineto{\pgfqpoint{4.118554in}{0.413320in}}%
\pgfpathlineto{\pgfqpoint{4.115844in}{0.413320in}}%
\pgfpathlineto{\pgfqpoint{4.113252in}{0.413320in}}%
\pgfpathlineto{\pgfqpoint{4.110488in}{0.413320in}}%
\pgfpathlineto{\pgfqpoint{4.107814in}{0.413320in}}%
\pgfpathlineto{\pgfqpoint{4.105185in}{0.413320in}}%
\pgfpathlineto{\pgfqpoint{4.102456in}{0.413320in}}%
\pgfpathlineto{\pgfqpoint{4.099777in}{0.413320in}}%
\pgfpathlineto{\pgfqpoint{4.097092in}{0.413320in}}%
\pgfpathlineto{\pgfqpoint{4.094527in}{0.413320in}}%
\pgfpathlineto{\pgfqpoint{4.091729in}{0.413320in}}%
\pgfpathlineto{\pgfqpoint{4.089159in}{0.413320in}}%
\pgfpathlineto{\pgfqpoint{4.086385in}{0.413320in}}%
\pgfpathlineto{\pgfqpoint{4.083870in}{0.413320in}}%
\pgfpathlineto{\pgfqpoint{4.081018in}{0.413320in}}%
\pgfpathlineto{\pgfqpoint{4.078471in}{0.413320in}}%
\pgfpathlineto{\pgfqpoint{4.075705in}{0.413320in}}%
\pgfpathlineto{\pgfqpoint{4.072985in}{0.413320in}}%
\pgfpathlineto{\pgfqpoint{4.070313in}{0.413320in}}%
\pgfpathlineto{\pgfqpoint{4.067636in}{0.413320in}}%
\pgfpathlineto{\pgfqpoint{4.064957in}{0.413320in}}%
\pgfpathlineto{\pgfqpoint{4.062266in}{0.413320in}}%
\pgfpathlineto{\pgfqpoint{4.059702in}{0.413320in}}%
\pgfpathlineto{\pgfqpoint{4.056911in}{0.413320in}}%
\pgfpathlineto{\pgfqpoint{4.054326in}{0.413320in}}%
\pgfpathlineto{\pgfqpoint{4.051557in}{0.413320in}}%
\pgfpathlineto{\pgfqpoint{4.049006in}{0.413320in}}%
\pgfpathlineto{\pgfqpoint{4.046210in}{0.413320in}}%
\pgfpathlineto{\pgfqpoint{4.043667in}{0.413320in}}%
\pgfpathlineto{\pgfqpoint{4.040852in}{0.413320in}}%
\pgfpathlineto{\pgfqpoint{4.038174in}{0.413320in}}%
\pgfpathlineto{\pgfqpoint{4.035492in}{0.413320in}}%
\pgfpathlineto{\pgfqpoint{4.032817in}{0.413320in}}%
\pgfpathlineto{\pgfqpoint{4.030229in}{0.413320in}}%
\pgfpathlineto{\pgfqpoint{4.027447in}{0.413320in}}%
\pgfpathlineto{\pgfqpoint{4.024868in}{0.413320in}}%
\pgfpathlineto{\pgfqpoint{4.022097in}{0.413320in}}%
\pgfpathlineto{\pgfqpoint{4.019518in}{0.413320in}}%
\pgfpathlineto{\pgfqpoint{4.016744in}{0.413320in}}%
\pgfpathlineto{\pgfqpoint{4.014186in}{0.413320in}}%
\pgfpathlineto{\pgfqpoint{4.011394in}{0.413320in}}%
\pgfpathlineto{\pgfqpoint{4.008699in}{0.413320in}}%
\pgfpathlineto{\pgfqpoint{4.006034in}{0.413320in}}%
\pgfpathlineto{\pgfqpoint{4.003348in}{0.413320in}}%
\pgfpathlineto{\pgfqpoint{4.000674in}{0.413320in}}%
\pgfpathlineto{\pgfqpoint{3.997990in}{0.413320in}}%
\pgfpathlineto{\pgfqpoint{3.995417in}{0.413320in}}%
\pgfpathlineto{\pgfqpoint{3.992642in}{0.413320in}}%
\pgfpathlineto{\pgfqpoint{3.990055in}{0.413320in}}%
\pgfpathlineto{\pgfqpoint{3.987270in}{0.413320in}}%
\pgfpathlineto{\pgfqpoint{3.984714in}{0.413320in}}%
\pgfpathlineto{\pgfqpoint{3.981929in}{0.413320in}}%
\pgfpathlineto{\pgfqpoint{3.979389in}{0.413320in}}%
\pgfpathlineto{\pgfqpoint{3.976563in}{0.413320in}}%
\pgfpathlineto{\pgfqpoint{3.973885in}{0.413320in}}%
\pgfpathlineto{\pgfqpoint{3.971250in}{0.413320in}}%
\pgfpathlineto{\pgfqpoint{3.968523in}{0.413320in}}%
\pgfpathlineto{\pgfqpoint{3.966013in}{0.413320in}}%
\pgfpathlineto{\pgfqpoint{3.963176in}{0.413320in}}%
\pgfpathlineto{\pgfqpoint{3.960635in}{0.413320in}}%
\pgfpathlineto{\pgfqpoint{3.957823in}{0.413320in}}%
\pgfpathlineto{\pgfqpoint{3.955211in}{0.413320in}}%
\pgfpathlineto{\pgfqpoint{3.952464in}{0.413320in}}%
\pgfpathlineto{\pgfqpoint{3.949894in}{0.413320in}}%
\pgfpathlineto{\pgfqpoint{3.947101in}{0.413320in}}%
\pgfpathlineto{\pgfqpoint{3.944431in}{0.413320in}}%
\pgfpathlineto{\pgfqpoint{3.941778in}{0.413320in}}%
\pgfpathlineto{\pgfqpoint{3.939075in}{0.413320in}}%
\pgfpathlineto{\pgfqpoint{3.936395in}{0.413320in}}%
\pgfpathlineto{\pgfqpoint{3.933714in}{0.413320in}}%
\pgfpathlineto{\pgfqpoint{3.931202in}{0.413320in}}%
\pgfpathlineto{\pgfqpoint{3.928347in}{0.413320in}}%
\pgfpathlineto{\pgfqpoint{3.925778in}{0.413320in}}%
\pgfpathlineto{\pgfqpoint{3.923005in}{0.413320in}}%
\pgfpathlineto{\pgfqpoint{3.920412in}{0.413320in}}%
\pgfpathlineto{\pgfqpoint{3.917646in}{0.413320in}}%
\pgfpathlineto{\pgfqpoint{3.915107in}{0.413320in}}%
\pgfpathlineto{\pgfqpoint{3.912296in}{0.413320in}}%
\pgfpathlineto{\pgfqpoint{3.909602in}{0.413320in}}%
\pgfpathlineto{\pgfqpoint{3.906918in}{0.413320in}}%
\pgfpathlineto{\pgfqpoint{3.904252in}{0.413320in}}%
\pgfpathlineto{\pgfqpoint{3.901573in}{0.413320in}}%
\pgfpathlineto{\pgfqpoint{3.898891in}{0.413320in}}%
\pgfpathlineto{\pgfqpoint{3.896345in}{0.413320in}}%
\pgfpathlineto{\pgfqpoint{3.893541in}{0.413320in}}%
\pgfpathlineto{\pgfqpoint{3.890926in}{0.413320in}}%
\pgfpathlineto{\pgfqpoint{3.888188in}{0.413320in}}%
\pgfpathlineto{\pgfqpoint{3.885621in}{0.413320in}}%
\pgfpathlineto{\pgfqpoint{3.882850in}{0.413320in}}%
\pgfpathlineto{\pgfqpoint{3.880237in}{0.413320in}}%
\pgfpathlineto{\pgfqpoint{3.877466in}{0.413320in}}%
\pgfpathlineto{\pgfqpoint{3.874790in}{0.413320in}}%
\pgfpathlineto{\pgfqpoint{3.872114in}{0.413320in}}%
\pgfpathlineto{\pgfqpoint{3.869435in}{0.413320in}}%
\pgfpathlineto{\pgfqpoint{3.866815in}{0.413320in}}%
\pgfpathlineto{\pgfqpoint{3.864073in}{0.413320in}}%
\pgfpathlineto{\pgfqpoint{3.861561in}{0.413320in}}%
\pgfpathlineto{\pgfqpoint{3.858720in}{0.413320in}}%
\pgfpathlineto{\pgfqpoint{3.856100in}{0.413320in}}%
\pgfpathlineto{\pgfqpoint{3.853358in}{0.413320in}}%
\pgfpathlineto{\pgfqpoint{3.850814in}{0.413320in}}%
\pgfpathlineto{\pgfqpoint{3.848005in}{0.413320in}}%
\pgfpathlineto{\pgfqpoint{3.845329in}{0.413320in}}%
\pgfpathlineto{\pgfqpoint{3.842641in}{0.413320in}}%
\pgfpathlineto{\pgfqpoint{3.839960in}{0.413320in}}%
\pgfpathlineto{\pgfqpoint{3.837286in}{0.413320in}}%
\pgfpathlineto{\pgfqpoint{3.834616in}{0.413320in}}%
\pgfpathlineto{\pgfqpoint{3.832053in}{0.413320in}}%
\pgfpathlineto{\pgfqpoint{3.829252in}{0.413320in}}%
\pgfpathlineto{\pgfqpoint{3.826679in}{0.413320in}}%
\pgfpathlineto{\pgfqpoint{3.823903in}{0.413320in}}%
\pgfpathlineto{\pgfqpoint{3.821315in}{0.413320in}}%
\pgfpathlineto{\pgfqpoint{3.818546in}{0.413320in}}%
\pgfpathlineto{\pgfqpoint{3.815983in}{0.413320in}}%
\pgfpathlineto{\pgfqpoint{3.813172in}{0.413320in}}%
\pgfpathlineto{\pgfqpoint{3.810510in}{0.413320in}}%
\pgfpathlineto{\pgfqpoint{3.807832in}{0.413320in}}%
\pgfpathlineto{\pgfqpoint{3.805145in}{0.413320in}}%
\pgfpathlineto{\pgfqpoint{3.802569in}{0.413320in}}%
\pgfpathlineto{\pgfqpoint{3.799797in}{0.413320in}}%
\pgfpathlineto{\pgfqpoint{3.797265in}{0.413320in}}%
\pgfpathlineto{\pgfqpoint{3.794435in}{0.413320in}}%
\pgfpathlineto{\pgfqpoint{3.791897in}{0.413320in}}%
\pgfpathlineto{\pgfqpoint{3.789084in}{0.413320in}}%
\pgfpathlineto{\pgfqpoint{3.786504in}{0.413320in}}%
\pgfpathlineto{\pgfqpoint{3.783725in}{0.413320in}}%
\pgfpathlineto{\pgfqpoint{3.781046in}{0.413320in}}%
\pgfpathlineto{\pgfqpoint{3.778370in}{0.413320in}}%
\pgfpathlineto{\pgfqpoint{3.775691in}{0.413320in}}%
\pgfpathlineto{\pgfqpoint{3.773014in}{0.413320in}}%
\pgfpathlineto{\pgfqpoint{3.770323in}{0.413320in}}%
\pgfpathlineto{\pgfqpoint{3.767782in}{0.413320in}}%
\pgfpathlineto{\pgfqpoint{3.764966in}{0.413320in}}%
\pgfpathlineto{\pgfqpoint{3.762389in}{0.413320in}}%
\pgfpathlineto{\pgfqpoint{3.759622in}{0.413320in}}%
\pgfpathlineto{\pgfqpoint{3.757065in}{0.413320in}}%
\pgfpathlineto{\pgfqpoint{3.754265in}{0.413320in}}%
\pgfpathlineto{\pgfqpoint{3.751728in}{0.413320in}}%
\pgfpathlineto{\pgfqpoint{3.748903in}{0.413320in}}%
\pgfpathlineto{\pgfqpoint{3.746229in}{0.413320in}}%
\pgfpathlineto{\pgfqpoint{3.743548in}{0.413320in}}%
\pgfpathlineto{\pgfqpoint{3.740874in}{0.413320in}}%
\pgfpathlineto{\pgfqpoint{3.738194in}{0.413320in}}%
\pgfpathlineto{\pgfqpoint{3.735509in}{0.413320in}}%
\pgfpathlineto{\pgfqpoint{3.732950in}{0.413320in}}%
\pgfpathlineto{\pgfqpoint{3.730158in}{0.413320in}}%
\pgfpathlineto{\pgfqpoint{3.727581in}{0.413320in}}%
\pgfpathlineto{\pgfqpoint{3.724804in}{0.413320in}}%
\pgfpathlineto{\pgfqpoint{3.722228in}{0.413320in}}%
\pgfpathlineto{\pgfqpoint{3.719446in}{0.413320in}}%
\pgfpathlineto{\pgfqpoint{3.716875in}{0.413320in}}%
\pgfpathlineto{\pgfqpoint{3.714086in}{0.413320in}}%
\pgfpathlineto{\pgfqpoint{3.711410in}{0.413320in}}%
\pgfpathlineto{\pgfqpoint{3.708729in}{0.413320in}}%
\pgfpathlineto{\pgfqpoint{3.706053in}{0.413320in}}%
\pgfpathlineto{\pgfqpoint{3.703460in}{0.413320in}}%
\pgfpathlineto{\pgfqpoint{3.700684in}{0.413320in}}%
\pgfpathlineto{\pgfqpoint{3.698125in}{0.413320in}}%
\pgfpathlineto{\pgfqpoint{3.695331in}{0.413320in}}%
\pgfpathlineto{\pgfqpoint{3.692765in}{0.413320in}}%
\pgfpathlineto{\pgfqpoint{3.689983in}{0.413320in}}%
\pgfpathlineto{\pgfqpoint{3.687442in}{0.413320in}}%
\pgfpathlineto{\pgfqpoint{3.684620in}{0.413320in}}%
\pgfpathlineto{\pgfqpoint{3.681948in}{0.413320in}}%
\pgfpathlineto{\pgfqpoint{3.679273in}{0.413320in}}%
\pgfpathlineto{\pgfqpoint{3.676591in}{0.413320in}}%
\pgfpathlineto{\pgfqpoint{3.673911in}{0.413320in}}%
\pgfpathlineto{\pgfqpoint{3.671232in}{0.413320in}}%
\pgfpathlineto{\pgfqpoint{3.668665in}{0.413320in}}%
\pgfpathlineto{\pgfqpoint{3.665864in}{0.413320in}}%
\pgfpathlineto{\pgfqpoint{3.663276in}{0.413320in}}%
\pgfpathlineto{\pgfqpoint{3.660515in}{0.413320in}}%
\pgfpathlineto{\pgfqpoint{3.657917in}{0.413320in}}%
\pgfpathlineto{\pgfqpoint{3.655165in}{0.413320in}}%
\pgfpathlineto{\pgfqpoint{3.652628in}{0.413320in}}%
\pgfpathlineto{\pgfqpoint{3.649837in}{0.413320in}}%
\pgfpathlineto{\pgfqpoint{3.647130in}{0.413320in}}%
\pgfpathlineto{\pgfqpoint{3.644452in}{0.413320in}}%
\pgfpathlineto{\pgfqpoint{3.641773in}{0.413320in}}%
\pgfpathlineto{\pgfqpoint{3.639207in}{0.413320in}}%
\pgfpathlineto{\pgfqpoint{3.636413in}{0.413320in}}%
\pgfpathlineto{\pgfqpoint{3.633858in}{0.413320in}}%
\pgfpathlineto{\pgfqpoint{3.631058in}{0.413320in}}%
\pgfpathlineto{\pgfqpoint{3.628460in}{0.413320in}}%
\pgfpathlineto{\pgfqpoint{3.625689in}{0.413320in}}%
\pgfpathlineto{\pgfqpoint{3.623165in}{0.413320in}}%
\pgfpathlineto{\pgfqpoint{3.620345in}{0.413320in}}%
\pgfpathlineto{\pgfqpoint{3.617667in}{0.413320in}}%
\pgfpathlineto{\pgfqpoint{3.614982in}{0.413320in}}%
\pgfpathlineto{\pgfqpoint{3.612311in}{0.413320in}}%
\pgfpathlineto{\pgfqpoint{3.609632in}{0.413320in}}%
\pgfpathlineto{\pgfqpoint{3.606951in}{0.413320in}}%
\pgfpathlineto{\pgfqpoint{3.604387in}{0.413320in}}%
\pgfpathlineto{\pgfqpoint{3.601590in}{0.413320in}}%
\pgfpathlineto{\pgfqpoint{3.598998in}{0.413320in}}%
\pgfpathlineto{\pgfqpoint{3.596240in}{0.413320in}}%
\pgfpathlineto{\pgfqpoint{3.593620in}{0.413320in}}%
\pgfpathlineto{\pgfqpoint{3.590883in}{0.413320in}}%
\pgfpathlineto{\pgfqpoint{3.588258in}{0.413320in}}%
\pgfpathlineto{\pgfqpoint{3.585532in}{0.413320in}}%
\pgfpathlineto{\pgfqpoint{3.582851in}{0.413320in}}%
\pgfpathlineto{\pgfqpoint{3.580191in}{0.413320in}}%
\pgfpathlineto{\pgfqpoint{3.577487in}{0.413320in}}%
\pgfpathlineto{\pgfqpoint{3.574814in}{0.413320in}}%
\pgfpathlineto{\pgfqpoint{3.572126in}{0.413320in}}%
\pgfpathlineto{\pgfqpoint{3.569584in}{0.413320in}}%
\pgfpathlineto{\pgfqpoint{3.566774in}{0.413320in}}%
\pgfpathlineto{\pgfqpoint{3.564188in}{0.413320in}}%
\pgfpathlineto{\pgfqpoint{3.561420in}{0.413320in}}%
\pgfpathlineto{\pgfqpoint{3.558853in}{0.413320in}}%
\pgfpathlineto{\pgfqpoint{3.556061in}{0.413320in}}%
\pgfpathlineto{\pgfqpoint{3.553498in}{0.413320in}}%
\pgfpathlineto{\pgfqpoint{3.550713in}{0.413320in}}%
\pgfpathlineto{\pgfqpoint{3.548029in}{0.413320in}}%
\pgfpathlineto{\pgfqpoint{3.545349in}{0.413320in}}%
\pgfpathlineto{\pgfqpoint{3.542656in}{0.413320in}}%
\pgfpathlineto{\pgfqpoint{3.540093in}{0.413320in}}%
\pgfpathlineto{\pgfqpoint{3.537309in}{0.413320in}}%
\pgfpathlineto{\pgfqpoint{3.534783in}{0.413320in}}%
\pgfpathlineto{\pgfqpoint{3.531955in}{0.413320in}}%
\pgfpathlineto{\pgfqpoint{3.529327in}{0.413320in}}%
\pgfpathlineto{\pgfqpoint{3.526601in}{0.413320in}}%
\pgfpathlineto{\pgfqpoint{3.524041in}{0.413320in}}%
\pgfpathlineto{\pgfqpoint{3.521244in}{0.413320in}}%
\pgfpathlineto{\pgfqpoint{3.518565in}{0.413320in}}%
\pgfpathlineto{\pgfqpoint{3.515884in}{0.413320in}}%
\pgfpathlineto{\pgfqpoint{3.513209in}{0.413320in}}%
\pgfpathlineto{\pgfqpoint{3.510533in}{0.413320in}}%
\pgfpathlineto{\pgfqpoint{3.507840in}{0.413320in}}%
\pgfpathlineto{\pgfqpoint{3.505262in}{0.413320in}}%
\pgfpathlineto{\pgfqpoint{3.502488in}{0.413320in}}%
\pgfpathlineto{\pgfqpoint{3.499909in}{0.413320in}}%
\pgfpathlineto{\pgfqpoint{3.497139in}{0.413320in}}%
\pgfpathlineto{\pgfqpoint{3.494581in}{0.413320in}}%
\pgfpathlineto{\pgfqpoint{3.491783in}{0.413320in}}%
\pgfpathlineto{\pgfqpoint{3.489223in}{0.413320in}}%
\pgfpathlineto{\pgfqpoint{3.486442in}{0.413320in}}%
\pgfpathlineto{\pgfqpoint{3.483744in}{0.413320in}}%
\pgfpathlineto{\pgfqpoint{3.481072in}{0.413320in}}%
\pgfpathlineto{\pgfqpoint{3.478378in}{0.413320in}}%
\pgfpathlineto{\pgfqpoint{3.475821in}{0.413320in}}%
\pgfpathlineto{\pgfqpoint{3.473021in}{0.413320in}}%
\pgfpathlineto{\pgfqpoint{3.470466in}{0.413320in}}%
\pgfpathlineto{\pgfqpoint{3.467678in}{0.413320in}}%
\pgfpathlineto{\pgfqpoint{3.465072in}{0.413320in}}%
\pgfpathlineto{\pgfqpoint{3.462321in}{0.413320in}}%
\pgfpathlineto{\pgfqpoint{3.459695in}{0.413320in}}%
\pgfpathlineto{\pgfqpoint{3.456960in}{0.413320in}}%
\pgfpathlineto{\pgfqpoint{3.454285in}{0.413320in}}%
\pgfpathlineto{\pgfqpoint{3.451597in}{0.413320in}}%
\pgfpathlineto{\pgfqpoint{3.448926in}{0.413320in}}%
\pgfpathlineto{\pgfqpoint{3.446257in}{0.413320in}}%
\pgfpathlineto{\pgfqpoint{3.443574in}{0.413320in}}%
\pgfpathlineto{\pgfqpoint{3.440996in}{0.413320in}}%
\pgfpathlineto{\pgfqpoint{3.438210in}{0.413320in}}%
\pgfpathlineto{\pgfqpoint{3.435635in}{0.413320in}}%
\pgfpathlineto{\pgfqpoint{3.432851in}{0.413320in}}%
\pgfpathlineto{\pgfqpoint{3.430313in}{0.413320in}}%
\pgfpathlineto{\pgfqpoint{3.427501in}{0.413320in}}%
\pgfpathlineto{\pgfqpoint{3.424887in}{0.413320in}}%
\pgfpathlineto{\pgfqpoint{3.422142in}{0.413320in}}%
\pgfpathlineto{\pgfqpoint{3.419455in}{0.413320in}}%
\pgfpathlineto{\pgfqpoint{3.416780in}{0.413320in}}%
\pgfpathlineto{\pgfqpoint{3.414109in}{0.413320in}}%
\pgfpathlineto{\pgfqpoint{3.411431in}{0.413320in}}%
\pgfpathlineto{\pgfqpoint{3.408752in}{0.413320in}}%
\pgfpathlineto{\pgfqpoint{3.406202in}{0.413320in}}%
\pgfpathlineto{\pgfqpoint{3.403394in}{0.413320in}}%
\pgfpathlineto{\pgfqpoint{3.400783in}{0.413320in}}%
\pgfpathlineto{\pgfqpoint{3.398037in}{0.413320in}}%
\pgfpathlineto{\pgfqpoint{3.395461in}{0.413320in}}%
\pgfpathlineto{\pgfqpoint{3.392681in}{0.413320in}}%
\pgfpathlineto{\pgfqpoint{3.390102in}{0.413320in}}%
\pgfpathlineto{\pgfqpoint{3.387309in}{0.413320in}}%
\pgfpathlineto{\pgfqpoint{3.384647in}{0.413320in}}%
\pgfpathlineto{\pgfqpoint{3.381959in}{0.413320in}}%
\pgfpathlineto{\pgfqpoint{3.379290in}{0.413320in}}%
\pgfpathlineto{\pgfqpoint{3.376735in}{0.413320in}}%
\pgfpathlineto{\pgfqpoint{3.373921in}{0.413320in}}%
\pgfpathlineto{\pgfqpoint{3.371357in}{0.413320in}}%
\pgfpathlineto{\pgfqpoint{3.368577in}{0.413320in}}%
\pgfpathlineto{\pgfqpoint{3.365996in}{0.413320in}}%
\pgfpathlineto{\pgfqpoint{3.363221in}{0.413320in}}%
\pgfpathlineto{\pgfqpoint{3.360620in}{0.413320in}}%
\pgfpathlineto{\pgfqpoint{3.357862in}{0.413320in}}%
\pgfpathlineto{\pgfqpoint{3.355177in}{0.413320in}}%
\pgfpathlineto{\pgfqpoint{3.352505in}{0.413320in}}%
\pgfpathlineto{\pgfqpoint{3.349828in}{0.413320in}}%
\pgfpathlineto{\pgfqpoint{3.347139in}{0.413320in}}%
\pgfpathlineto{\pgfqpoint{3.344468in}{0.413320in}}%
\pgfpathlineto{\pgfqpoint{3.341893in}{0.413320in}}%
\pgfpathlineto{\pgfqpoint{3.339101in}{0.413320in}}%
\pgfpathlineto{\pgfqpoint{3.336541in}{0.413320in}}%
\pgfpathlineto{\pgfqpoint{3.333758in}{0.413320in}}%
\pgfpathlineto{\pgfqpoint{3.331183in}{0.413320in}}%
\pgfpathlineto{\pgfqpoint{3.328401in}{0.413320in}}%
\pgfpathlineto{\pgfqpoint{3.325860in}{0.413320in}}%
\pgfpathlineto{\pgfqpoint{3.323049in}{0.413320in}}%
\pgfpathlineto{\pgfqpoint{3.320366in}{0.413320in}}%
\pgfpathlineto{\pgfqpoint{3.317688in}{0.413320in}}%
\pgfpathlineto{\pgfqpoint{3.315008in}{0.413320in}}%
\pgfpathlineto{\pgfqpoint{3.312480in}{0.413320in}}%
\pgfpathlineto{\pgfqpoint{3.309652in}{0.413320in}}%
\pgfpathlineto{\pgfqpoint{3.307104in}{0.413320in}}%
\pgfpathlineto{\pgfqpoint{3.304295in}{0.413320in}}%
\pgfpathlineto{\pgfqpoint{3.301719in}{0.413320in}}%
\pgfpathlineto{\pgfqpoint{3.298937in}{0.413320in}}%
\pgfpathlineto{\pgfqpoint{3.296376in}{0.413320in}}%
\pgfpathlineto{\pgfqpoint{3.293574in}{0.413320in}}%
\pgfpathlineto{\pgfqpoint{3.290890in}{0.413320in}}%
\pgfpathlineto{\pgfqpoint{3.288225in}{0.413320in}}%
\pgfpathlineto{\pgfqpoint{3.285534in}{0.413320in}}%
\pgfpathlineto{\pgfqpoint{3.282870in}{0.413320in}}%
\pgfpathlineto{\pgfqpoint{3.280189in}{0.413320in}}%
\pgfpathlineto{\pgfqpoint{3.277603in}{0.413320in}}%
\pgfpathlineto{\pgfqpoint{3.274831in}{0.413320in}}%
\pgfpathlineto{\pgfqpoint{3.272254in}{0.413320in}}%
\pgfpathlineto{\pgfqpoint{3.269478in}{0.413320in}}%
\pgfpathlineto{\pgfqpoint{3.266849in}{0.413320in}}%
\pgfpathlineto{\pgfqpoint{3.264119in}{0.413320in}}%
\pgfpathlineto{\pgfqpoint{3.261594in}{0.413320in}}%
\pgfpathlineto{\pgfqpoint{3.258784in}{0.413320in}}%
\pgfpathlineto{\pgfqpoint{3.256083in}{0.413320in}}%
\pgfpathlineto{\pgfqpoint{3.253404in}{0.413320in}}%
\pgfpathlineto{\pgfqpoint{3.250716in}{0.413320in}}%
\pgfpathlineto{\pgfqpoint{3.248049in}{0.413320in}}%
\pgfpathlineto{\pgfqpoint{3.245363in}{0.413320in}}%
\pgfpathlineto{\pgfqpoint{3.242807in}{0.413320in}}%
\pgfpathlineto{\pgfqpoint{3.240010in}{0.413320in}}%
\pgfpathlineto{\pgfqpoint{3.237411in}{0.413320in}}%
\pgfpathlineto{\pgfqpoint{3.234658in}{0.413320in}}%
\pgfpathlineto{\pgfqpoint{3.232069in}{0.413320in}}%
\pgfpathlineto{\pgfqpoint{3.229310in}{0.413320in}}%
\pgfpathlineto{\pgfqpoint{3.226609in}{0.413320in}}%
\pgfpathlineto{\pgfqpoint{3.223942in}{0.413320in}}%
\pgfpathlineto{\pgfqpoint{3.221255in}{0.413320in}}%
\pgfpathlineto{\pgfqpoint{3.218586in}{0.413320in}}%
\pgfpathlineto{\pgfqpoint{3.215908in}{0.413320in}}%
\pgfpathlineto{\pgfqpoint{3.213342in}{0.413320in}}%
\pgfpathlineto{\pgfqpoint{3.210545in}{0.413320in}}%
\pgfpathlineto{\pgfqpoint{3.207984in}{0.413320in}}%
\pgfpathlineto{\pgfqpoint{3.205195in}{0.413320in}}%
\pgfpathlineto{\pgfqpoint{3.202562in}{0.413320in}}%
\pgfpathlineto{\pgfqpoint{3.199823in}{0.413320in}}%
\pgfpathlineto{\pgfqpoint{3.197226in}{0.413320in}}%
\pgfpathlineto{\pgfqpoint{3.194508in}{0.413320in}}%
\pgfpathlineto{\pgfqpoint{3.191796in}{0.413320in}}%
\pgfpathlineto{\pgfqpoint{3.189117in}{0.413320in}}%
\pgfpathlineto{\pgfqpoint{3.186440in}{0.413320in}}%
\pgfpathlineto{\pgfqpoint{3.183760in}{0.413320in}}%
\pgfpathlineto{\pgfqpoint{3.181089in}{0.413320in}}%
\pgfpathlineto{\pgfqpoint{3.178525in}{0.413320in}}%
\pgfpathlineto{\pgfqpoint{3.175724in}{0.413320in}}%
\pgfpathlineto{\pgfqpoint{3.173142in}{0.413320in}}%
\pgfpathlineto{\pgfqpoint{3.170375in}{0.413320in}}%
\pgfpathlineto{\pgfqpoint{3.167776in}{0.413320in}}%
\pgfpathlineto{\pgfqpoint{3.165019in}{0.413320in}}%
\pgfpathlineto{\pgfqpoint{3.162474in}{0.413320in}}%
\pgfpathlineto{\pgfqpoint{3.159675in}{0.413320in}}%
\pgfpathlineto{\pgfqpoint{3.156981in}{0.413320in}}%
\pgfpathlineto{\pgfqpoint{3.154327in}{0.413320in}}%
\pgfpathlineto{\pgfqpoint{3.151612in}{0.413320in}}%
\pgfpathlineto{\pgfqpoint{3.149057in}{0.413320in}}%
\pgfpathlineto{\pgfqpoint{3.146271in}{0.413320in}}%
\pgfpathlineto{\pgfqpoint{3.143740in}{0.413320in}}%
\pgfpathlineto{\pgfqpoint{3.140913in}{0.413320in}}%
\pgfpathlineto{\pgfqpoint{3.138375in}{0.413320in}}%
\pgfpathlineto{\pgfqpoint{3.135550in}{0.413320in}}%
\pgfpathlineto{\pgfqpoint{3.132946in}{0.413320in}}%
\pgfpathlineto{\pgfqpoint{3.130199in}{0.413320in}}%
\pgfpathlineto{\pgfqpoint{3.127512in}{0.413320in}}%
\pgfpathlineto{\pgfqpoint{3.124842in}{0.413320in}}%
\pgfpathlineto{\pgfqpoint{3.122163in}{0.413320in}}%
\pgfpathlineto{\pgfqpoint{3.119487in}{0.413320in}}%
\pgfpathlineto{\pgfqpoint{3.116807in}{0.413320in}}%
\pgfpathlineto{\pgfqpoint{3.114242in}{0.413320in}}%
\pgfpathlineto{\pgfqpoint{3.111451in}{0.413320in}}%
\pgfpathlineto{\pgfqpoint{3.108896in}{0.413320in}}%
\pgfpathlineto{\pgfqpoint{3.106094in}{0.413320in}}%
\pgfpathlineto{\pgfqpoint{3.103508in}{0.413320in}}%
\pgfpathlineto{\pgfqpoint{3.100737in}{0.413320in}}%
\pgfpathlineto{\pgfqpoint{3.098163in}{0.413320in}}%
\pgfpathlineto{\pgfqpoint{3.095388in}{0.413320in}}%
\pgfpathlineto{\pgfqpoint{3.092699in}{0.413320in}}%
\pgfpathlineto{\pgfqpoint{3.090023in}{0.413320in}}%
\pgfpathlineto{\pgfqpoint{3.087343in}{0.413320in}}%
\pgfpathlineto{\pgfqpoint{3.084671in}{0.413320in}}%
\pgfpathlineto{\pgfqpoint{3.081990in}{0.413320in}}%
\pgfpathlineto{\pgfqpoint{3.079381in}{0.413320in}}%
\pgfpathlineto{\pgfqpoint{3.076631in}{0.413320in}}%
\pgfpathlineto{\pgfqpoint{3.074056in}{0.413320in}}%
\pgfpathlineto{\pgfqpoint{3.071266in}{0.413320in}}%
\pgfpathlineto{\pgfqpoint{3.068709in}{0.413320in}}%
\pgfpathlineto{\pgfqpoint{3.065916in}{0.413320in}}%
\pgfpathlineto{\pgfqpoint{3.063230in}{0.413320in}}%
\pgfpathlineto{\pgfqpoint{3.060561in}{0.413320in}}%
\pgfpathlineto{\pgfqpoint{3.057884in}{0.413320in}}%
\pgfpathlineto{\pgfqpoint{3.055202in}{0.413320in}}%
\pgfpathlineto{\pgfqpoint{3.052526in}{0.413320in}}%
\pgfpathlineto{\pgfqpoint{3.049988in}{0.413320in}}%
\pgfpathlineto{\pgfqpoint{3.047157in}{0.413320in}}%
\pgfpathlineto{\pgfqpoint{3.044568in}{0.413320in}}%
\pgfpathlineto{\pgfqpoint{3.041813in}{0.413320in}}%
\pgfpathlineto{\pgfqpoint{3.039262in}{0.413320in}}%
\pgfpathlineto{\pgfqpoint{3.036456in}{0.413320in}}%
\pgfpathlineto{\pgfqpoint{3.033921in}{0.413320in}}%
\pgfpathlineto{\pgfqpoint{3.031091in}{0.413320in}}%
\pgfpathlineto{\pgfqpoint{3.028412in}{0.413320in}}%
\pgfpathlineto{\pgfqpoint{3.025803in}{0.413320in}}%
\pgfpathlineto{\pgfqpoint{3.023058in}{0.413320in}}%
\pgfpathlineto{\pgfqpoint{3.020382in}{0.413320in}}%
\pgfpathlineto{\pgfqpoint{3.017707in}{0.413320in}}%
\pgfpathlineto{\pgfqpoint{3.015097in}{0.413320in}}%
\pgfpathlineto{\pgfqpoint{3.012351in}{0.413320in}}%
\pgfpathlineto{\pgfqpoint{3.009784in}{0.413320in}}%
\pgfpathlineto{\pgfqpoint{3.006993in}{0.413320in}}%
\pgfpathlineto{\pgfqpoint{3.004419in}{0.413320in}}%
\pgfpathlineto{\pgfqpoint{3.001635in}{0.413320in}}%
\pgfpathlineto{\pgfqpoint{2.999103in}{0.413320in}}%
\pgfpathlineto{\pgfqpoint{2.996300in}{0.413320in}}%
\pgfpathlineto{\pgfqpoint{2.993595in}{0.413320in}}%
\pgfpathlineto{\pgfqpoint{2.990978in}{0.413320in}}%
\pgfpathlineto{\pgfqpoint{2.988238in}{0.413320in}}%
\pgfpathlineto{\pgfqpoint{2.985666in}{0.413320in}}%
\pgfpathlineto{\pgfqpoint{2.982885in}{0.413320in}}%
\pgfpathlineto{\pgfqpoint{2.980341in}{0.413320in}}%
\pgfpathlineto{\pgfqpoint{2.977517in}{0.413320in}}%
\pgfpathlineto{\pgfqpoint{2.974972in}{0.413320in}}%
\pgfpathlineto{\pgfqpoint{2.972177in}{0.413320in}}%
\pgfpathlineto{\pgfqpoint{2.969599in}{0.413320in}}%
\pgfpathlineto{\pgfqpoint{2.966812in}{0.413320in}}%
\pgfpathlineto{\pgfqpoint{2.964127in}{0.413320in}}%
\pgfpathlineto{\pgfqpoint{2.961460in}{0.413320in}}%
\pgfpathlineto{\pgfqpoint{2.958782in}{0.413320in}}%
\pgfpathlineto{\pgfqpoint{2.956103in}{0.413320in}}%
\pgfpathlineto{\pgfqpoint{2.953422in}{0.413320in}}%
\pgfpathlineto{\pgfqpoint{2.950884in}{0.413320in}}%
\pgfpathlineto{\pgfqpoint{2.948068in}{0.413320in}}%
\pgfpathlineto{\pgfqpoint{2.945461in}{0.413320in}}%
\pgfpathlineto{\pgfqpoint{2.942711in}{0.413320in}}%
\pgfpathlineto{\pgfqpoint{2.940120in}{0.413320in}}%
\pgfpathlineto{\pgfqpoint{2.937352in}{0.413320in}}%
\pgfpathlineto{\pgfqpoint{2.934759in}{0.413320in}}%
\pgfpathlineto{\pgfqpoint{2.932033in}{0.413320in}}%
\pgfpathlineto{\pgfqpoint{2.929321in}{0.413320in}}%
\pgfpathlineto{\pgfqpoint{2.926655in}{0.413320in}}%
\pgfpathlineto{\pgfqpoint{2.923963in}{0.413320in}}%
\pgfpathlineto{\pgfqpoint{2.921363in}{0.413320in}}%
\pgfpathlineto{\pgfqpoint{2.918606in}{0.413320in}}%
\pgfpathlineto{\pgfqpoint{2.916061in}{0.413320in}}%
\pgfpathlineto{\pgfqpoint{2.913243in}{0.413320in}}%
\pgfpathlineto{\pgfqpoint{2.910631in}{0.413320in}}%
\pgfpathlineto{\pgfqpoint{2.907882in}{0.413320in}}%
\pgfpathlineto{\pgfqpoint{2.905341in}{0.413320in}}%
\pgfpathlineto{\pgfqpoint{2.902535in}{0.413320in}}%
\pgfpathlineto{\pgfqpoint{2.899858in}{0.413320in}}%
\pgfpathlineto{\pgfqpoint{2.897179in}{0.413320in}}%
\pgfpathlineto{\pgfqpoint{2.894487in}{0.413320in}}%
\pgfpathlineto{\pgfqpoint{2.891809in}{0.413320in}}%
\pgfpathlineto{\pgfqpoint{2.889145in}{0.413320in}}%
\pgfpathlineto{\pgfqpoint{2.886578in}{0.413320in}}%
\pgfpathlineto{\pgfqpoint{2.883780in}{0.413320in}}%
\pgfpathlineto{\pgfqpoint{2.881254in}{0.413320in}}%
\pgfpathlineto{\pgfqpoint{2.878431in}{0.413320in}}%
\pgfpathlineto{\pgfqpoint{2.875882in}{0.413320in}}%
\pgfpathlineto{\pgfqpoint{2.873074in}{0.413320in}}%
\pgfpathlineto{\pgfqpoint{2.870475in}{0.413320in}}%
\pgfpathlineto{\pgfqpoint{2.867713in}{0.413320in}}%
\pgfpathlineto{\pgfqpoint{2.865031in}{0.413320in}}%
\pgfpathlineto{\pgfqpoint{2.862402in}{0.413320in}}%
\pgfpathlineto{\pgfqpoint{2.859668in}{0.413320in}}%
\pgfpathlineto{\pgfqpoint{2.857003in}{0.413320in}}%
\pgfpathlineto{\pgfqpoint{2.854325in}{0.413320in}}%
\pgfpathlineto{\pgfqpoint{2.851793in}{0.413320in}}%
\pgfpathlineto{\pgfqpoint{2.848960in}{0.413320in}}%
\pgfpathlineto{\pgfqpoint{2.846408in}{0.413320in}}%
\pgfpathlineto{\pgfqpoint{2.843611in}{0.413320in}}%
\pgfpathlineto{\pgfqpoint{2.841055in}{0.413320in}}%
\pgfpathlineto{\pgfqpoint{2.838254in}{0.413320in}}%
\pgfpathlineto{\pgfqpoint{2.835698in}{0.413320in}}%
\pgfpathlineto{\pgfqpoint{2.832894in}{0.413320in}}%
\pgfpathlineto{\pgfqpoint{2.830219in}{0.413320in}}%
\pgfpathlineto{\pgfqpoint{2.827567in}{0.413320in}}%
\pgfpathlineto{\pgfqpoint{2.824851in}{0.413320in}}%
\pgfpathlineto{\pgfqpoint{2.822303in}{0.413320in}}%
\pgfpathlineto{\pgfqpoint{2.819506in}{0.413320in}}%
\pgfpathlineto{\pgfqpoint{2.816867in}{0.413320in}}%
\pgfpathlineto{\pgfqpoint{2.814141in}{0.413320in}}%
\pgfpathlineto{\pgfqpoint{2.811597in}{0.413320in}}%
\pgfpathlineto{\pgfqpoint{2.808792in}{0.413320in}}%
\pgfpathlineto{\pgfqpoint{2.806175in}{0.413320in}}%
\pgfpathlineto{\pgfqpoint{2.803435in}{0.413320in}}%
\pgfpathlineto{\pgfqpoint{2.800756in}{0.413320in}}%
\pgfpathlineto{\pgfqpoint{2.798070in}{0.413320in}}%
\pgfpathlineto{\pgfqpoint{2.795398in}{0.413320in}}%
\pgfpathlineto{\pgfqpoint{2.792721in}{0.413320in}}%
\pgfpathlineto{\pgfqpoint{2.790044in}{0.413320in}}%
\pgfpathlineto{\pgfqpoint{2.787468in}{0.413320in}}%
\pgfpathlineto{\pgfqpoint{2.784687in}{0.413320in}}%
\pgfpathlineto{\pgfqpoint{2.782113in}{0.413320in}}%
\pgfpathlineto{\pgfqpoint{2.779330in}{0.413320in}}%
\pgfpathlineto{\pgfqpoint{2.776767in}{0.413320in}}%
\pgfpathlineto{\pgfqpoint{2.773972in}{0.413320in}}%
\pgfpathlineto{\pgfqpoint{2.771373in}{0.413320in}}%
\pgfpathlineto{\pgfqpoint{2.768617in}{0.413320in}}%
\pgfpathlineto{\pgfqpoint{2.765935in}{0.413320in}}%
\pgfpathlineto{\pgfqpoint{2.763253in}{0.413320in}}%
\pgfpathlineto{\pgfqpoint{2.760581in}{0.413320in}}%
\pgfpathlineto{\pgfqpoint{2.758028in}{0.413320in}}%
\pgfpathlineto{\pgfqpoint{2.755224in}{0.413320in}}%
\pgfpathlineto{\pgfqpoint{2.752614in}{0.413320in}}%
\pgfpathlineto{\pgfqpoint{2.749868in}{0.413320in}}%
\pgfpathlineto{\pgfqpoint{2.747260in}{0.413320in}}%
\pgfpathlineto{\pgfqpoint{2.744510in}{0.413320in}}%
\pgfpathlineto{\pgfqpoint{2.741928in}{0.413320in}}%
\pgfpathlineto{\pgfqpoint{2.739155in}{0.413320in}}%
\pgfpathlineto{\pgfqpoint{2.736476in}{0.413320in}}%
\pgfpathlineto{\pgfqpoint{2.733798in}{0.413320in}}%
\pgfpathlineto{\pgfqpoint{2.731119in}{0.413320in}}%
\pgfpathlineto{\pgfqpoint{2.728439in}{0.413320in}}%
\pgfpathlineto{\pgfqpoint{2.725760in}{0.413320in}}%
\pgfpathlineto{\pgfqpoint{2.723211in}{0.413320in}}%
\pgfpathlineto{\pgfqpoint{2.720404in}{0.413320in}}%
\pgfpathlineto{\pgfqpoint{2.717773in}{0.413320in}}%
\pgfpathlineto{\pgfqpoint{2.715036in}{0.413320in}}%
\pgfpathlineto{\pgfqpoint{2.712477in}{0.413320in}}%
\pgfpathlineto{\pgfqpoint{2.709683in}{0.413320in}}%
\pgfpathlineto{\pgfqpoint{2.707125in}{0.413320in}}%
\pgfpathlineto{\pgfqpoint{2.704326in}{0.413320in}}%
\pgfpathlineto{\pgfqpoint{2.701657in}{0.413320in}}%
\pgfpathlineto{\pgfqpoint{2.698968in}{0.413320in}}%
\pgfpathlineto{\pgfqpoint{2.696293in}{0.413320in}}%
\pgfpathlineto{\pgfqpoint{2.693611in}{0.413320in}}%
\pgfpathlineto{\pgfqpoint{2.690940in}{0.413320in}}%
\pgfpathlineto{\pgfqpoint{2.688328in}{0.413320in}}%
\pgfpathlineto{\pgfqpoint{2.685586in}{0.413320in}}%
\pgfpathlineto{\pgfqpoint{2.683009in}{0.413320in}}%
\pgfpathlineto{\pgfqpoint{2.680224in}{0.413320in}}%
\pgfpathlineto{\pgfqpoint{2.677650in}{0.413320in}}%
\pgfpathlineto{\pgfqpoint{2.674873in}{0.413320in}}%
\pgfpathlineto{\pgfqpoint{2.672301in}{0.413320in}}%
\pgfpathlineto{\pgfqpoint{2.669506in}{0.413320in}}%
\pgfpathlineto{\pgfqpoint{2.666836in}{0.413320in}}%
\pgfpathlineto{\pgfqpoint{2.664151in}{0.413320in}}%
\pgfpathlineto{\pgfqpoint{2.661481in}{0.413320in}}%
\pgfpathlineto{\pgfqpoint{2.658942in}{0.413320in}}%
\pgfpathlineto{\pgfqpoint{2.656124in}{0.413320in}}%
\pgfpathlineto{\pgfqpoint{2.653567in}{0.413320in}}%
\pgfpathlineto{\pgfqpoint{2.650767in}{0.413320in}}%
\pgfpathlineto{\pgfqpoint{2.648196in}{0.413320in}}%
\pgfpathlineto{\pgfqpoint{2.645408in}{0.413320in}}%
\pgfpathlineto{\pgfqpoint{2.642827in}{0.413320in}}%
\pgfpathlineto{\pgfqpoint{2.640053in}{0.413320in}}%
\pgfpathlineto{\pgfqpoint{2.637369in}{0.413320in}}%
\pgfpathlineto{\pgfqpoint{2.634700in}{0.413320in}}%
\pgfpathlineto{\pgfqpoint{2.632018in}{0.413320in}}%
\pgfpathlineto{\pgfqpoint{2.629340in}{0.413320in}}%
\pgfpathlineto{\pgfqpoint{2.626653in}{0.413320in}}%
\pgfpathlineto{\pgfqpoint{2.624077in}{0.413320in}}%
\pgfpathlineto{\pgfqpoint{2.621304in}{0.413320in}}%
\pgfpathlineto{\pgfqpoint{2.618773in}{0.413320in}}%
\pgfpathlineto{\pgfqpoint{2.615934in}{0.413320in}}%
\pgfpathlineto{\pgfqpoint{2.613393in}{0.413320in}}%
\pgfpathlineto{\pgfqpoint{2.610588in}{0.413320in}}%
\pgfpathlineto{\pgfqpoint{2.608004in}{0.413320in}}%
\pgfpathlineto{\pgfqpoint{2.605232in}{0.413320in}}%
\pgfpathlineto{\pgfqpoint{2.602557in}{0.413320in}}%
\pgfpathlineto{\pgfqpoint{2.599920in}{0.413320in}}%
\pgfpathlineto{\pgfqpoint{2.597196in}{0.413320in}}%
\pgfpathlineto{\pgfqpoint{2.594630in}{0.413320in}}%
\pgfpathlineto{\pgfqpoint{2.591842in}{0.413320in}}%
\pgfpathlineto{\pgfqpoint{2.589248in}{0.413320in}}%
\pgfpathlineto{\pgfqpoint{2.586484in}{0.413320in}}%
\pgfpathlineto{\pgfqpoint{2.583913in}{0.413320in}}%
\pgfpathlineto{\pgfqpoint{2.581129in}{0.413320in}}%
\pgfpathlineto{\pgfqpoint{2.578567in}{0.413320in}}%
\pgfpathlineto{\pgfqpoint{2.575779in}{0.413320in}}%
\pgfpathlineto{\pgfqpoint{2.573082in}{0.413320in}}%
\pgfpathlineto{\pgfqpoint{2.570411in}{0.413320in}}%
\pgfpathlineto{\pgfqpoint{2.567730in}{0.413320in}}%
\pgfpathlineto{\pgfqpoint{2.565045in}{0.413320in}}%
\pgfpathlineto{\pgfqpoint{2.562375in}{0.413320in}}%
\pgfpathlineto{\pgfqpoint{2.559790in}{0.413320in}}%
\pgfpathlineto{\pgfqpoint{2.557009in}{0.413320in}}%
\pgfpathlineto{\pgfqpoint{2.554493in}{0.413320in}}%
\pgfpathlineto{\pgfqpoint{2.551664in}{0.413320in}}%
\pgfpathlineto{\pgfqpoint{2.549114in}{0.413320in}}%
\pgfpathlineto{\pgfqpoint{2.546310in}{0.413320in}}%
\pgfpathlineto{\pgfqpoint{2.543765in}{0.413320in}}%
\pgfpathlineto{\pgfqpoint{2.540949in}{0.413320in}}%
\pgfpathlineto{\pgfqpoint{2.538274in}{0.413320in}}%
\pgfpathlineto{\pgfqpoint{2.535624in}{0.413320in}}%
\pgfpathlineto{\pgfqpoint{2.532917in}{0.413320in}}%
\pgfpathlineto{\pgfqpoint{2.530234in}{0.413320in}}%
\pgfpathlineto{\pgfqpoint{2.527560in}{0.413320in}}%
\pgfpathlineto{\pgfqpoint{2.524988in}{0.413320in}}%
\pgfpathlineto{\pgfqpoint{2.522197in}{0.413320in}}%
\pgfpathlineto{\pgfqpoint{2.519607in}{0.413320in}}%
\pgfpathlineto{\pgfqpoint{2.516845in}{0.413320in}}%
\pgfpathlineto{\pgfqpoint{2.514268in}{0.413320in}}%
\pgfpathlineto{\pgfqpoint{2.511478in}{0.413320in}}%
\pgfpathlineto{\pgfqpoint{2.508917in}{0.413320in}}%
\pgfpathlineto{\pgfqpoint{2.506163in}{0.413320in}}%
\pgfpathlineto{\pgfqpoint{2.503454in}{0.413320in}}%
\pgfpathlineto{\pgfqpoint{2.500801in}{0.413320in}}%
\pgfpathlineto{\pgfqpoint{2.498085in}{0.413320in}}%
\pgfpathlineto{\pgfqpoint{2.495542in}{0.413320in}}%
\pgfpathlineto{\pgfqpoint{2.492729in}{0.413320in}}%
\pgfpathlineto{\pgfqpoint{2.490183in}{0.413320in}}%
\pgfpathlineto{\pgfqpoint{2.487384in}{0.413320in}}%
\pgfpathlineto{\pgfqpoint{2.484870in}{0.413320in}}%
\pgfpathlineto{\pgfqpoint{2.482026in}{0.413320in}}%
\pgfpathlineto{\pgfqpoint{2.479420in}{0.413320in}}%
\pgfpathlineto{\pgfqpoint{2.476671in}{0.413320in}}%
\pgfpathlineto{\pgfqpoint{2.473989in}{0.413320in}}%
\pgfpathlineto{\pgfqpoint{2.471311in}{0.413320in}}%
\pgfpathlineto{\pgfqpoint{2.468635in}{0.413320in}}%
\pgfpathlineto{\pgfqpoint{2.465957in}{0.413320in}}%
\pgfpathlineto{\pgfqpoint{2.463280in}{0.413320in}}%
\pgfpathlineto{\pgfqpoint{2.460711in}{0.413320in}}%
\pgfpathlineto{\pgfqpoint{2.457917in}{0.413320in}}%
\pgfpathlineto{\pgfqpoint{2.455353in}{0.413320in}}%
\pgfpathlineto{\pgfqpoint{2.452562in}{0.413320in}}%
\pgfpathlineto{\pgfqpoint{2.450032in}{0.413320in}}%
\pgfpathlineto{\pgfqpoint{2.447209in}{0.413320in}}%
\pgfpathlineto{\pgfqpoint{2.444677in}{0.413320in}}%
\pgfpathlineto{\pgfqpoint{2.441876in}{0.413320in}}%
\pgfpathlineto{\pgfqpoint{2.439167in}{0.413320in}}%
\pgfpathlineto{\pgfqpoint{2.436518in}{0.413320in}}%
\pgfpathlineto{\pgfqpoint{2.433815in}{0.413320in}}%
\pgfpathlineto{\pgfqpoint{2.431251in}{0.413320in}}%
\pgfpathlineto{\pgfqpoint{2.428453in}{0.413320in}}%
\pgfpathlineto{\pgfqpoint{2.425878in}{0.413320in}}%
\pgfpathlineto{\pgfqpoint{2.423098in}{0.413320in}}%
\pgfpathlineto{\pgfqpoint{2.420528in}{0.413320in}}%
\pgfpathlineto{\pgfqpoint{2.417747in}{0.413320in}}%
\pgfpathlineto{\pgfqpoint{2.415184in}{0.413320in}}%
\pgfpathlineto{\pgfqpoint{2.412389in}{0.413320in}}%
\pgfpathlineto{\pgfqpoint{2.409699in}{0.413320in}}%
\pgfpathlineto{\pgfqpoint{2.407024in}{0.413320in}}%
\pgfpathlineto{\pgfqpoint{2.404352in}{0.413320in}}%
\pgfpathlineto{\pgfqpoint{2.401675in}{0.413320in}}%
\pgfpathlineto{\pgfqpoint{2.398995in}{0.413320in}}%
\pgfpathclose%
\pgfusepath{stroke,fill}%
\end{pgfscope}%
\begin{pgfscope}%
\pgfpathrectangle{\pgfqpoint{2.398995in}{0.319877in}}{\pgfqpoint{3.986877in}{1.993438in}} %
\pgfusepath{clip}%
\pgfsetbuttcap%
\pgfsetroundjoin%
\definecolor{currentfill}{rgb}{1.000000,1.000000,1.000000}%
\pgfsetfillcolor{currentfill}%
\pgfsetlinewidth{1.003750pt}%
\definecolor{currentstroke}{rgb}{0.756337,0.500864,0.958050}%
\pgfsetstrokecolor{currentstroke}%
\pgfsetdash{}{0pt}%
\pgfpathmoveto{\pgfqpoint{2.398995in}{0.413320in}}%
\pgfpathlineto{\pgfqpoint{2.398995in}{1.912193in}}%
\pgfpathlineto{\pgfqpoint{2.401675in}{1.915870in}}%
\pgfpathlineto{\pgfqpoint{2.404352in}{1.923612in}}%
\pgfpathlineto{\pgfqpoint{2.407024in}{1.918221in}}%
\pgfpathlineto{\pgfqpoint{2.409699in}{1.912346in}}%
\pgfpathlineto{\pgfqpoint{2.412389in}{1.911119in}}%
\pgfpathlineto{\pgfqpoint{2.415184in}{1.912919in}}%
\pgfpathlineto{\pgfqpoint{2.417747in}{1.917171in}}%
\pgfpathlineto{\pgfqpoint{2.420528in}{1.912364in}}%
\pgfpathlineto{\pgfqpoint{2.423098in}{1.914832in}}%
\pgfpathlineto{\pgfqpoint{2.425878in}{1.917284in}}%
\pgfpathlineto{\pgfqpoint{2.428453in}{1.919740in}}%
\pgfpathlineto{\pgfqpoint{2.431251in}{1.924876in}}%
\pgfpathlineto{\pgfqpoint{2.433815in}{1.921426in}}%
\pgfpathlineto{\pgfqpoint{2.436518in}{1.921964in}}%
\pgfpathlineto{\pgfqpoint{2.439167in}{1.915398in}}%
\pgfpathlineto{\pgfqpoint{2.441876in}{1.911902in}}%
\pgfpathlineto{\pgfqpoint{2.444677in}{1.911815in}}%
\pgfpathlineto{\pgfqpoint{2.447209in}{1.908455in}}%
\pgfpathlineto{\pgfqpoint{2.450032in}{1.918369in}}%
\pgfpathlineto{\pgfqpoint{2.452562in}{1.914696in}}%
\pgfpathlineto{\pgfqpoint{2.455353in}{1.910987in}}%
\pgfpathlineto{\pgfqpoint{2.457917in}{1.915952in}}%
\pgfpathlineto{\pgfqpoint{2.460711in}{1.911745in}}%
\pgfpathlineto{\pgfqpoint{2.463280in}{1.912962in}}%
\pgfpathlineto{\pgfqpoint{2.465957in}{1.907849in}}%
\pgfpathlineto{\pgfqpoint{2.468635in}{1.909899in}}%
\pgfpathlineto{\pgfqpoint{2.471311in}{1.907781in}}%
\pgfpathlineto{\pgfqpoint{2.473989in}{1.905160in}}%
\pgfpathlineto{\pgfqpoint{2.476671in}{1.910347in}}%
\pgfpathlineto{\pgfqpoint{2.479420in}{1.907333in}}%
\pgfpathlineto{\pgfqpoint{2.482026in}{1.914237in}}%
\pgfpathlineto{\pgfqpoint{2.484870in}{1.915857in}}%
\pgfpathlineto{\pgfqpoint{2.487384in}{1.909468in}}%
\pgfpathlineto{\pgfqpoint{2.490183in}{1.910393in}}%
\pgfpathlineto{\pgfqpoint{2.492729in}{1.907182in}}%
\pgfpathlineto{\pgfqpoint{2.495542in}{1.907186in}}%
\pgfpathlineto{\pgfqpoint{2.498085in}{1.906658in}}%
\pgfpathlineto{\pgfqpoint{2.500801in}{1.909631in}}%
\pgfpathlineto{\pgfqpoint{2.503454in}{1.911944in}}%
\pgfpathlineto{\pgfqpoint{2.506163in}{1.907613in}}%
\pgfpathlineto{\pgfqpoint{2.508917in}{1.912905in}}%
\pgfpathlineto{\pgfqpoint{2.511478in}{1.910902in}}%
\pgfpathlineto{\pgfqpoint{2.514268in}{1.911549in}}%
\pgfpathlineto{\pgfqpoint{2.516845in}{1.914787in}}%
\pgfpathlineto{\pgfqpoint{2.519607in}{1.917028in}}%
\pgfpathlineto{\pgfqpoint{2.522197in}{1.924963in}}%
\pgfpathlineto{\pgfqpoint{2.524988in}{1.925543in}}%
\pgfpathlineto{\pgfqpoint{2.527560in}{1.917403in}}%
\pgfpathlineto{\pgfqpoint{2.530234in}{1.913683in}}%
\pgfpathlineto{\pgfqpoint{2.532917in}{1.910189in}}%
\pgfpathlineto{\pgfqpoint{2.535624in}{1.908299in}}%
\pgfpathlineto{\pgfqpoint{2.538274in}{1.909906in}}%
\pgfpathlineto{\pgfqpoint{2.540949in}{1.912522in}}%
\pgfpathlineto{\pgfqpoint{2.543765in}{1.916064in}}%
\pgfpathlineto{\pgfqpoint{2.546310in}{1.908695in}}%
\pgfpathlineto{\pgfqpoint{2.549114in}{1.908554in}}%
\pgfpathlineto{\pgfqpoint{2.551664in}{1.907922in}}%
\pgfpathlineto{\pgfqpoint{2.554493in}{1.904702in}}%
\pgfpathlineto{\pgfqpoint{2.557009in}{1.903530in}}%
\pgfpathlineto{\pgfqpoint{2.559790in}{1.906322in}}%
\pgfpathlineto{\pgfqpoint{2.562375in}{1.901733in}}%
\pgfpathlineto{\pgfqpoint{2.565045in}{1.910891in}}%
\pgfpathlineto{\pgfqpoint{2.567730in}{1.910833in}}%
\pgfpathlineto{\pgfqpoint{2.570411in}{1.908291in}}%
\pgfpathlineto{\pgfqpoint{2.573082in}{1.916934in}}%
\pgfpathlineto{\pgfqpoint{2.575779in}{1.920449in}}%
\pgfpathlineto{\pgfqpoint{2.578567in}{1.912712in}}%
\pgfpathlineto{\pgfqpoint{2.581129in}{1.912014in}}%
\pgfpathlineto{\pgfqpoint{2.583913in}{1.910331in}}%
\pgfpathlineto{\pgfqpoint{2.586484in}{1.912365in}}%
\pgfpathlineto{\pgfqpoint{2.589248in}{1.913525in}}%
\pgfpathlineto{\pgfqpoint{2.591842in}{1.915755in}}%
\pgfpathlineto{\pgfqpoint{2.594630in}{1.916558in}}%
\pgfpathlineto{\pgfqpoint{2.597196in}{1.920436in}}%
\pgfpathlineto{\pgfqpoint{2.599920in}{1.911940in}}%
\pgfpathlineto{\pgfqpoint{2.602557in}{1.914673in}}%
\pgfpathlineto{\pgfqpoint{2.605232in}{1.917082in}}%
\pgfpathlineto{\pgfqpoint{2.608004in}{1.920065in}}%
\pgfpathlineto{\pgfqpoint{2.610588in}{1.911626in}}%
\pgfpathlineto{\pgfqpoint{2.613393in}{1.912158in}}%
\pgfpathlineto{\pgfqpoint{2.615934in}{1.912909in}}%
\pgfpathlineto{\pgfqpoint{2.618773in}{1.911380in}}%
\pgfpathlineto{\pgfqpoint{2.621304in}{1.911317in}}%
\pgfpathlineto{\pgfqpoint{2.624077in}{1.916705in}}%
\pgfpathlineto{\pgfqpoint{2.626653in}{1.912289in}}%
\pgfpathlineto{\pgfqpoint{2.629340in}{1.911145in}}%
\pgfpathlineto{\pgfqpoint{2.632018in}{1.913470in}}%
\pgfpathlineto{\pgfqpoint{2.634700in}{1.911428in}}%
\pgfpathlineto{\pgfqpoint{2.637369in}{1.910395in}}%
\pgfpathlineto{\pgfqpoint{2.640053in}{1.909767in}}%
\pgfpathlineto{\pgfqpoint{2.642827in}{1.911958in}}%
\pgfpathlineto{\pgfqpoint{2.645408in}{1.909164in}}%
\pgfpathlineto{\pgfqpoint{2.648196in}{1.914571in}}%
\pgfpathlineto{\pgfqpoint{2.650767in}{1.913896in}}%
\pgfpathlineto{\pgfqpoint{2.653567in}{1.911148in}}%
\pgfpathlineto{\pgfqpoint{2.656124in}{1.909508in}}%
\pgfpathlineto{\pgfqpoint{2.658942in}{1.906630in}}%
\pgfpathlineto{\pgfqpoint{2.661481in}{1.908094in}}%
\pgfpathlineto{\pgfqpoint{2.664151in}{1.901524in}}%
\pgfpathlineto{\pgfqpoint{2.666836in}{1.907904in}}%
\pgfpathlineto{\pgfqpoint{2.669506in}{1.911417in}}%
\pgfpathlineto{\pgfqpoint{2.672301in}{1.912986in}}%
\pgfpathlineto{\pgfqpoint{2.674873in}{1.911481in}}%
\pgfpathlineto{\pgfqpoint{2.677650in}{1.910857in}}%
\pgfpathlineto{\pgfqpoint{2.680224in}{1.911364in}}%
\pgfpathlineto{\pgfqpoint{2.683009in}{1.911184in}}%
\pgfpathlineto{\pgfqpoint{2.685586in}{1.914047in}}%
\pgfpathlineto{\pgfqpoint{2.688328in}{1.912576in}}%
\pgfpathlineto{\pgfqpoint{2.690940in}{1.912401in}}%
\pgfpathlineto{\pgfqpoint{2.693611in}{1.914600in}}%
\pgfpathlineto{\pgfqpoint{2.696293in}{1.915054in}}%
\pgfpathlineto{\pgfqpoint{2.698968in}{1.911370in}}%
\pgfpathlineto{\pgfqpoint{2.701657in}{1.915214in}}%
\pgfpathlineto{\pgfqpoint{2.704326in}{1.914689in}}%
\pgfpathlineto{\pgfqpoint{2.707125in}{1.913836in}}%
\pgfpathlineto{\pgfqpoint{2.709683in}{1.913421in}}%
\pgfpathlineto{\pgfqpoint{2.712477in}{1.920958in}}%
\pgfpathlineto{\pgfqpoint{2.715036in}{1.915885in}}%
\pgfpathlineto{\pgfqpoint{2.717773in}{1.910173in}}%
\pgfpathlineto{\pgfqpoint{2.720404in}{1.910886in}}%
\pgfpathlineto{\pgfqpoint{2.723211in}{1.912706in}}%
\pgfpathlineto{\pgfqpoint{2.725760in}{1.909661in}}%
\pgfpathlineto{\pgfqpoint{2.728439in}{1.909413in}}%
\pgfpathlineto{\pgfqpoint{2.731119in}{1.913796in}}%
\pgfpathlineto{\pgfqpoint{2.733798in}{1.915981in}}%
\pgfpathlineto{\pgfqpoint{2.736476in}{1.900796in}}%
\pgfpathlineto{\pgfqpoint{2.739155in}{1.900796in}}%
\pgfpathlineto{\pgfqpoint{2.741928in}{1.910345in}}%
\pgfpathlineto{\pgfqpoint{2.744510in}{1.909302in}}%
\pgfpathlineto{\pgfqpoint{2.747260in}{1.917693in}}%
\pgfpathlineto{\pgfqpoint{2.749868in}{1.914388in}}%
\pgfpathlineto{\pgfqpoint{2.752614in}{1.915166in}}%
\pgfpathlineto{\pgfqpoint{2.755224in}{1.916919in}}%
\pgfpathlineto{\pgfqpoint{2.758028in}{1.916151in}}%
\pgfpathlineto{\pgfqpoint{2.760581in}{1.916591in}}%
\pgfpathlineto{\pgfqpoint{2.763253in}{1.922931in}}%
\pgfpathlineto{\pgfqpoint{2.765935in}{1.919974in}}%
\pgfpathlineto{\pgfqpoint{2.768617in}{1.937697in}}%
\pgfpathlineto{\pgfqpoint{2.771373in}{1.924864in}}%
\pgfpathlineto{\pgfqpoint{2.773972in}{1.915998in}}%
\pgfpathlineto{\pgfqpoint{2.776767in}{1.914358in}}%
\pgfpathlineto{\pgfqpoint{2.779330in}{1.912509in}}%
\pgfpathlineto{\pgfqpoint{2.782113in}{1.913875in}}%
\pgfpathlineto{\pgfqpoint{2.784687in}{1.911321in}}%
\pgfpathlineto{\pgfqpoint{2.787468in}{1.915033in}}%
\pgfpathlineto{\pgfqpoint{2.790044in}{1.916585in}}%
\pgfpathlineto{\pgfqpoint{2.792721in}{1.912872in}}%
\pgfpathlineto{\pgfqpoint{2.795398in}{1.912457in}}%
\pgfpathlineto{\pgfqpoint{2.798070in}{1.912996in}}%
\pgfpathlineto{\pgfqpoint{2.800756in}{1.906505in}}%
\pgfpathlineto{\pgfqpoint{2.803435in}{1.906032in}}%
\pgfpathlineto{\pgfqpoint{2.806175in}{1.908864in}}%
\pgfpathlineto{\pgfqpoint{2.808792in}{1.914034in}}%
\pgfpathlineto{\pgfqpoint{2.811597in}{1.923844in}}%
\pgfpathlineto{\pgfqpoint{2.814141in}{1.914764in}}%
\pgfpathlineto{\pgfqpoint{2.816867in}{1.913753in}}%
\pgfpathlineto{\pgfqpoint{2.819506in}{1.916187in}}%
\pgfpathlineto{\pgfqpoint{2.822303in}{1.910153in}}%
\pgfpathlineto{\pgfqpoint{2.824851in}{1.919911in}}%
\pgfpathlineto{\pgfqpoint{2.827567in}{1.910977in}}%
\pgfpathlineto{\pgfqpoint{2.830219in}{1.907347in}}%
\pgfpathlineto{\pgfqpoint{2.832894in}{1.912166in}}%
\pgfpathlineto{\pgfqpoint{2.835698in}{1.906992in}}%
\pgfpathlineto{\pgfqpoint{2.838254in}{1.904649in}}%
\pgfpathlineto{\pgfqpoint{2.841055in}{1.912680in}}%
\pgfpathlineto{\pgfqpoint{2.843611in}{1.912631in}}%
\pgfpathlineto{\pgfqpoint{2.846408in}{1.907488in}}%
\pgfpathlineto{\pgfqpoint{2.848960in}{1.914011in}}%
\pgfpathlineto{\pgfqpoint{2.851793in}{1.908755in}}%
\pgfpathlineto{\pgfqpoint{2.854325in}{1.913901in}}%
\pgfpathlineto{\pgfqpoint{2.857003in}{1.908741in}}%
\pgfpathlineto{\pgfqpoint{2.859668in}{1.913694in}}%
\pgfpathlineto{\pgfqpoint{2.862402in}{1.915563in}}%
\pgfpathlineto{\pgfqpoint{2.865031in}{1.914566in}}%
\pgfpathlineto{\pgfqpoint{2.867713in}{1.913327in}}%
\pgfpathlineto{\pgfqpoint{2.870475in}{1.913177in}}%
\pgfpathlineto{\pgfqpoint{2.873074in}{1.916959in}}%
\pgfpathlineto{\pgfqpoint{2.875882in}{1.916633in}}%
\pgfpathlineto{\pgfqpoint{2.878431in}{1.912727in}}%
\pgfpathlineto{\pgfqpoint{2.881254in}{1.920163in}}%
\pgfpathlineto{\pgfqpoint{2.883780in}{1.929570in}}%
\pgfpathlineto{\pgfqpoint{2.886578in}{1.918414in}}%
\pgfpathlineto{\pgfqpoint{2.889145in}{1.915570in}}%
\pgfpathlineto{\pgfqpoint{2.891809in}{1.916570in}}%
\pgfpathlineto{\pgfqpoint{2.894487in}{1.914917in}}%
\pgfpathlineto{\pgfqpoint{2.897179in}{1.918729in}}%
\pgfpathlineto{\pgfqpoint{2.899858in}{1.913944in}}%
\pgfpathlineto{\pgfqpoint{2.902535in}{1.916745in}}%
\pgfpathlineto{\pgfqpoint{2.905341in}{1.915404in}}%
\pgfpathlineto{\pgfqpoint{2.907882in}{1.915074in}}%
\pgfpathlineto{\pgfqpoint{2.910631in}{1.916219in}}%
\pgfpathlineto{\pgfqpoint{2.913243in}{1.920942in}}%
\pgfpathlineto{\pgfqpoint{2.916061in}{1.923801in}}%
\pgfpathlineto{\pgfqpoint{2.918606in}{1.917289in}}%
\pgfpathlineto{\pgfqpoint{2.921363in}{1.911979in}}%
\pgfpathlineto{\pgfqpoint{2.923963in}{1.913372in}}%
\pgfpathlineto{\pgfqpoint{2.926655in}{1.910826in}}%
\pgfpathlineto{\pgfqpoint{2.929321in}{1.915479in}}%
\pgfpathlineto{\pgfqpoint{2.932033in}{1.911881in}}%
\pgfpathlineto{\pgfqpoint{2.934759in}{1.912514in}}%
\pgfpathlineto{\pgfqpoint{2.937352in}{1.921993in}}%
\pgfpathlineto{\pgfqpoint{2.940120in}{1.925059in}}%
\pgfpathlineto{\pgfqpoint{2.942711in}{1.922322in}}%
\pgfpathlineto{\pgfqpoint{2.945461in}{1.918241in}}%
\pgfpathlineto{\pgfqpoint{2.948068in}{1.915962in}}%
\pgfpathlineto{\pgfqpoint{2.950884in}{1.915949in}}%
\pgfpathlineto{\pgfqpoint{2.953422in}{1.912431in}}%
\pgfpathlineto{\pgfqpoint{2.956103in}{1.916778in}}%
\pgfpathlineto{\pgfqpoint{2.958782in}{1.918603in}}%
\pgfpathlineto{\pgfqpoint{2.961460in}{1.918087in}}%
\pgfpathlineto{\pgfqpoint{2.964127in}{1.914992in}}%
\pgfpathlineto{\pgfqpoint{2.966812in}{1.919253in}}%
\pgfpathlineto{\pgfqpoint{2.969599in}{1.918713in}}%
\pgfpathlineto{\pgfqpoint{2.972177in}{1.916167in}}%
\pgfpathlineto{\pgfqpoint{2.974972in}{1.913620in}}%
\pgfpathlineto{\pgfqpoint{2.977517in}{1.914876in}}%
\pgfpathlineto{\pgfqpoint{2.980341in}{1.916130in}}%
\pgfpathlineto{\pgfqpoint{2.982885in}{1.914662in}}%
\pgfpathlineto{\pgfqpoint{2.985666in}{1.919164in}}%
\pgfpathlineto{\pgfqpoint{2.988238in}{1.919494in}}%
\pgfpathlineto{\pgfqpoint{2.990978in}{1.913691in}}%
\pgfpathlineto{\pgfqpoint{2.993595in}{1.910240in}}%
\pgfpathlineto{\pgfqpoint{2.996300in}{1.915550in}}%
\pgfpathlineto{\pgfqpoint{2.999103in}{1.904339in}}%
\pgfpathlineto{\pgfqpoint{3.001635in}{1.904216in}}%
\pgfpathlineto{\pgfqpoint{3.004419in}{1.906054in}}%
\pgfpathlineto{\pgfqpoint{3.006993in}{1.911836in}}%
\pgfpathlineto{\pgfqpoint{3.009784in}{1.917083in}}%
\pgfpathlineto{\pgfqpoint{3.012351in}{1.912902in}}%
\pgfpathlineto{\pgfqpoint{3.015097in}{1.914143in}}%
\pgfpathlineto{\pgfqpoint{3.017707in}{1.917445in}}%
\pgfpathlineto{\pgfqpoint{3.020382in}{1.921903in}}%
\pgfpathlineto{\pgfqpoint{3.023058in}{1.914428in}}%
\pgfpathlineto{\pgfqpoint{3.025803in}{1.921162in}}%
\pgfpathlineto{\pgfqpoint{3.028412in}{1.914135in}}%
\pgfpathlineto{\pgfqpoint{3.031091in}{1.912276in}}%
\pgfpathlineto{\pgfqpoint{3.033921in}{1.910748in}}%
\pgfpathlineto{\pgfqpoint{3.036456in}{1.917515in}}%
\pgfpathlineto{\pgfqpoint{3.039262in}{1.912443in}}%
\pgfpathlineto{\pgfqpoint{3.041813in}{1.917862in}}%
\pgfpathlineto{\pgfqpoint{3.044568in}{1.927457in}}%
\pgfpathlineto{\pgfqpoint{3.047157in}{1.939581in}}%
\pgfpathlineto{\pgfqpoint{3.049988in}{1.944844in}}%
\pgfpathlineto{\pgfqpoint{3.052526in}{1.936736in}}%
\pgfpathlineto{\pgfqpoint{3.055202in}{1.924742in}}%
\pgfpathlineto{\pgfqpoint{3.057884in}{1.923969in}}%
\pgfpathlineto{\pgfqpoint{3.060561in}{1.922920in}}%
\pgfpathlineto{\pgfqpoint{3.063230in}{1.929658in}}%
\pgfpathlineto{\pgfqpoint{3.065916in}{1.920811in}}%
\pgfpathlineto{\pgfqpoint{3.068709in}{1.924486in}}%
\pgfpathlineto{\pgfqpoint{3.071266in}{1.922500in}}%
\pgfpathlineto{\pgfqpoint{3.074056in}{1.925118in}}%
\pgfpathlineto{\pgfqpoint{3.076631in}{1.920188in}}%
\pgfpathlineto{\pgfqpoint{3.079381in}{1.926028in}}%
\pgfpathlineto{\pgfqpoint{3.081990in}{1.931256in}}%
\pgfpathlineto{\pgfqpoint{3.084671in}{1.930667in}}%
\pgfpathlineto{\pgfqpoint{3.087343in}{1.925573in}}%
\pgfpathlineto{\pgfqpoint{3.090023in}{1.930175in}}%
\pgfpathlineto{\pgfqpoint{3.092699in}{1.927320in}}%
\pgfpathlineto{\pgfqpoint{3.095388in}{1.918220in}}%
\pgfpathlineto{\pgfqpoint{3.098163in}{1.927374in}}%
\pgfpathlineto{\pgfqpoint{3.100737in}{1.919409in}}%
\pgfpathlineto{\pgfqpoint{3.103508in}{1.905265in}}%
\pgfpathlineto{\pgfqpoint{3.106094in}{1.908417in}}%
\pgfpathlineto{\pgfqpoint{3.108896in}{1.907121in}}%
\pgfpathlineto{\pgfqpoint{3.111451in}{1.906330in}}%
\pgfpathlineto{\pgfqpoint{3.114242in}{1.900796in}}%
\pgfpathlineto{\pgfqpoint{3.116807in}{1.900796in}}%
\pgfpathlineto{\pgfqpoint{3.119487in}{1.900796in}}%
\pgfpathlineto{\pgfqpoint{3.122163in}{1.901486in}}%
\pgfpathlineto{\pgfqpoint{3.124842in}{1.906203in}}%
\pgfpathlineto{\pgfqpoint{3.127512in}{1.906078in}}%
\pgfpathlineto{\pgfqpoint{3.130199in}{1.901699in}}%
\pgfpathlineto{\pgfqpoint{3.132946in}{1.902564in}}%
\pgfpathlineto{\pgfqpoint{3.135550in}{1.900796in}}%
\pgfpathlineto{\pgfqpoint{3.138375in}{1.900796in}}%
\pgfpathlineto{\pgfqpoint{3.140913in}{1.929792in}}%
\pgfpathlineto{\pgfqpoint{3.143740in}{1.975217in}}%
\pgfpathlineto{\pgfqpoint{3.146271in}{1.994113in}}%
\pgfpathlineto{\pgfqpoint{3.149057in}{1.979924in}}%
\pgfpathlineto{\pgfqpoint{3.151612in}{1.977790in}}%
\pgfpathlineto{\pgfqpoint{3.154327in}{1.935411in}}%
\pgfpathlineto{\pgfqpoint{3.156981in}{1.919500in}}%
\pgfpathlineto{\pgfqpoint{3.159675in}{1.904297in}}%
\pgfpathlineto{\pgfqpoint{3.162474in}{1.900796in}}%
\pgfpathlineto{\pgfqpoint{3.165019in}{1.900796in}}%
\pgfpathlineto{\pgfqpoint{3.167776in}{1.923088in}}%
\pgfpathlineto{\pgfqpoint{3.170375in}{1.953029in}}%
\pgfpathlineto{\pgfqpoint{3.173142in}{1.988209in}}%
\pgfpathlineto{\pgfqpoint{3.175724in}{1.952628in}}%
\pgfpathlineto{\pgfqpoint{3.178525in}{1.924762in}}%
\pgfpathlineto{\pgfqpoint{3.181089in}{1.910999in}}%
\pgfpathlineto{\pgfqpoint{3.183760in}{1.900796in}}%
\pgfpathlineto{\pgfqpoint{3.186440in}{1.906505in}}%
\pgfpathlineto{\pgfqpoint{3.189117in}{1.900796in}}%
\pgfpathlineto{\pgfqpoint{3.191796in}{1.905529in}}%
\pgfpathlineto{\pgfqpoint{3.194508in}{1.908600in}}%
\pgfpathlineto{\pgfqpoint{3.197226in}{1.914497in}}%
\pgfpathlineto{\pgfqpoint{3.199823in}{1.951224in}}%
\pgfpathlineto{\pgfqpoint{3.202562in}{1.968073in}}%
\pgfpathlineto{\pgfqpoint{3.205195in}{1.938930in}}%
\pgfpathlineto{\pgfqpoint{3.207984in}{1.912200in}}%
\pgfpathlineto{\pgfqpoint{3.210545in}{1.920048in}}%
\pgfpathlineto{\pgfqpoint{3.213342in}{1.909123in}}%
\pgfpathlineto{\pgfqpoint{3.215908in}{1.903207in}}%
\pgfpathlineto{\pgfqpoint{3.218586in}{1.900796in}}%
\pgfpathlineto{\pgfqpoint{3.221255in}{1.900796in}}%
\pgfpathlineto{\pgfqpoint{3.223942in}{1.904155in}}%
\pgfpathlineto{\pgfqpoint{3.226609in}{1.906268in}}%
\pgfpathlineto{\pgfqpoint{3.229310in}{1.907296in}}%
\pgfpathlineto{\pgfqpoint{3.232069in}{1.910182in}}%
\pgfpathlineto{\pgfqpoint{3.234658in}{1.910427in}}%
\pgfpathlineto{\pgfqpoint{3.237411in}{1.901762in}}%
\pgfpathlineto{\pgfqpoint{3.240010in}{1.900985in}}%
\pgfpathlineto{\pgfqpoint{3.242807in}{1.900796in}}%
\pgfpathlineto{\pgfqpoint{3.245363in}{1.901067in}}%
\pgfpathlineto{\pgfqpoint{3.248049in}{1.904589in}}%
\pgfpathlineto{\pgfqpoint{3.250716in}{1.907943in}}%
\pgfpathlineto{\pgfqpoint{3.253404in}{1.904916in}}%
\pgfpathlineto{\pgfqpoint{3.256083in}{1.905471in}}%
\pgfpathlineto{\pgfqpoint{3.258784in}{1.906674in}}%
\pgfpathlineto{\pgfqpoint{3.261594in}{1.905218in}}%
\pgfpathlineto{\pgfqpoint{3.264119in}{1.908732in}}%
\pgfpathlineto{\pgfqpoint{3.266849in}{1.906618in}}%
\pgfpathlineto{\pgfqpoint{3.269478in}{1.905369in}}%
\pgfpathlineto{\pgfqpoint{3.272254in}{1.907788in}}%
\pgfpathlineto{\pgfqpoint{3.274831in}{1.913138in}}%
\pgfpathlineto{\pgfqpoint{3.277603in}{1.912980in}}%
\pgfpathlineto{\pgfqpoint{3.280189in}{1.912200in}}%
\pgfpathlineto{\pgfqpoint{3.282870in}{1.909675in}}%
\pgfpathlineto{\pgfqpoint{3.285534in}{1.913792in}}%
\pgfpathlineto{\pgfqpoint{3.288225in}{1.909270in}}%
\pgfpathlineto{\pgfqpoint{3.290890in}{1.906412in}}%
\pgfpathlineto{\pgfqpoint{3.293574in}{1.909964in}}%
\pgfpathlineto{\pgfqpoint{3.296376in}{1.908499in}}%
\pgfpathlineto{\pgfqpoint{3.298937in}{1.902716in}}%
\pgfpathlineto{\pgfqpoint{3.301719in}{1.907499in}}%
\pgfpathlineto{\pgfqpoint{3.304295in}{1.905851in}}%
\pgfpathlineto{\pgfqpoint{3.307104in}{1.908670in}}%
\pgfpathlineto{\pgfqpoint{3.309652in}{1.908337in}}%
\pgfpathlineto{\pgfqpoint{3.312480in}{1.909290in}}%
\pgfpathlineto{\pgfqpoint{3.315008in}{1.910740in}}%
\pgfpathlineto{\pgfqpoint{3.317688in}{1.910340in}}%
\pgfpathlineto{\pgfqpoint{3.320366in}{1.910965in}}%
\pgfpathlineto{\pgfqpoint{3.323049in}{1.910299in}}%
\pgfpathlineto{\pgfqpoint{3.325860in}{1.909098in}}%
\pgfpathlineto{\pgfqpoint{3.328401in}{1.914123in}}%
\pgfpathlineto{\pgfqpoint{3.331183in}{1.913589in}}%
\pgfpathlineto{\pgfqpoint{3.333758in}{1.912043in}}%
\pgfpathlineto{\pgfqpoint{3.336541in}{1.911229in}}%
\pgfpathlineto{\pgfqpoint{3.339101in}{1.908527in}}%
\pgfpathlineto{\pgfqpoint{3.341893in}{1.913261in}}%
\pgfpathlineto{\pgfqpoint{3.344468in}{1.911102in}}%
\pgfpathlineto{\pgfqpoint{3.347139in}{1.912872in}}%
\pgfpathlineto{\pgfqpoint{3.349828in}{1.908969in}}%
\pgfpathlineto{\pgfqpoint{3.352505in}{1.915194in}}%
\pgfpathlineto{\pgfqpoint{3.355177in}{1.914372in}}%
\pgfpathlineto{\pgfqpoint{3.357862in}{1.908468in}}%
\pgfpathlineto{\pgfqpoint{3.360620in}{1.910248in}}%
\pgfpathlineto{\pgfqpoint{3.363221in}{1.911792in}}%
\pgfpathlineto{\pgfqpoint{3.365996in}{1.916898in}}%
\pgfpathlineto{\pgfqpoint{3.368577in}{1.912263in}}%
\pgfpathlineto{\pgfqpoint{3.371357in}{1.910895in}}%
\pgfpathlineto{\pgfqpoint{3.373921in}{1.910236in}}%
\pgfpathlineto{\pgfqpoint{3.376735in}{1.909661in}}%
\pgfpathlineto{\pgfqpoint{3.379290in}{1.916879in}}%
\pgfpathlineto{\pgfqpoint{3.381959in}{1.912166in}}%
\pgfpathlineto{\pgfqpoint{3.384647in}{1.916815in}}%
\pgfpathlineto{\pgfqpoint{3.387309in}{1.918943in}}%
\pgfpathlineto{\pgfqpoint{3.390102in}{1.911648in}}%
\pgfpathlineto{\pgfqpoint{3.392681in}{1.911527in}}%
\pgfpathlineto{\pgfqpoint{3.395461in}{1.906696in}}%
\pgfpathlineto{\pgfqpoint{3.398037in}{1.912191in}}%
\pgfpathlineto{\pgfqpoint{3.400783in}{1.913483in}}%
\pgfpathlineto{\pgfqpoint{3.403394in}{1.910555in}}%
\pgfpathlineto{\pgfqpoint{3.406202in}{1.913154in}}%
\pgfpathlineto{\pgfqpoint{3.408752in}{1.915932in}}%
\pgfpathlineto{\pgfqpoint{3.411431in}{1.916595in}}%
\pgfpathlineto{\pgfqpoint{3.414109in}{1.913873in}}%
\pgfpathlineto{\pgfqpoint{3.416780in}{1.911372in}}%
\pgfpathlineto{\pgfqpoint{3.419455in}{1.910328in}}%
\pgfpathlineto{\pgfqpoint{3.422142in}{1.913631in}}%
\pgfpathlineto{\pgfqpoint{3.424887in}{1.915661in}}%
\pgfpathlineto{\pgfqpoint{3.427501in}{1.914567in}}%
\pgfpathlineto{\pgfqpoint{3.430313in}{1.914274in}}%
\pgfpathlineto{\pgfqpoint{3.432851in}{1.915268in}}%
\pgfpathlineto{\pgfqpoint{3.435635in}{1.916840in}}%
\pgfpathlineto{\pgfqpoint{3.438210in}{1.917046in}}%
\pgfpathlineto{\pgfqpoint{3.440996in}{1.919847in}}%
\pgfpathlineto{\pgfqpoint{3.443574in}{1.915459in}}%
\pgfpathlineto{\pgfqpoint{3.446257in}{1.913865in}}%
\pgfpathlineto{\pgfqpoint{3.448926in}{1.920180in}}%
\pgfpathlineto{\pgfqpoint{3.451597in}{1.919513in}}%
\pgfpathlineto{\pgfqpoint{3.454285in}{1.912631in}}%
\pgfpathlineto{\pgfqpoint{3.456960in}{1.912553in}}%
\pgfpathlineto{\pgfqpoint{3.459695in}{1.918143in}}%
\pgfpathlineto{\pgfqpoint{3.462321in}{1.915482in}}%
\pgfpathlineto{\pgfqpoint{3.465072in}{1.919016in}}%
\pgfpathlineto{\pgfqpoint{3.467678in}{1.920468in}}%
\pgfpathlineto{\pgfqpoint{3.470466in}{1.918021in}}%
\pgfpathlineto{\pgfqpoint{3.473021in}{1.914498in}}%
\pgfpathlineto{\pgfqpoint{3.475821in}{1.913464in}}%
\pgfpathlineto{\pgfqpoint{3.478378in}{1.913302in}}%
\pgfpathlineto{\pgfqpoint{3.481072in}{1.913112in}}%
\pgfpathlineto{\pgfqpoint{3.483744in}{1.910660in}}%
\pgfpathlineto{\pgfqpoint{3.486442in}{1.910054in}}%
\pgfpathlineto{\pgfqpoint{3.489223in}{1.911951in}}%
\pgfpathlineto{\pgfqpoint{3.491783in}{1.912570in}}%
\pgfpathlineto{\pgfqpoint{3.494581in}{1.913239in}}%
\pgfpathlineto{\pgfqpoint{3.497139in}{1.915937in}}%
\pgfpathlineto{\pgfqpoint{3.499909in}{1.920138in}}%
\pgfpathlineto{\pgfqpoint{3.502488in}{1.920050in}}%
\pgfpathlineto{\pgfqpoint{3.505262in}{1.926032in}}%
\pgfpathlineto{\pgfqpoint{3.507840in}{1.922249in}}%
\pgfpathlineto{\pgfqpoint{3.510533in}{1.921860in}}%
\pgfpathlineto{\pgfqpoint{3.513209in}{1.918571in}}%
\pgfpathlineto{\pgfqpoint{3.515884in}{1.916702in}}%
\pgfpathlineto{\pgfqpoint{3.518565in}{1.915664in}}%
\pgfpathlineto{\pgfqpoint{3.521244in}{1.915000in}}%
\pgfpathlineto{\pgfqpoint{3.524041in}{1.921954in}}%
\pgfpathlineto{\pgfqpoint{3.526601in}{1.935044in}}%
\pgfpathlineto{\pgfqpoint{3.529327in}{1.930250in}}%
\pgfpathlineto{\pgfqpoint{3.531955in}{1.919365in}}%
\pgfpathlineto{\pgfqpoint{3.534783in}{1.920453in}}%
\pgfpathlineto{\pgfqpoint{3.537309in}{1.918289in}}%
\pgfpathlineto{\pgfqpoint{3.540093in}{1.917430in}}%
\pgfpathlineto{\pgfqpoint{3.542656in}{1.917387in}}%
\pgfpathlineto{\pgfqpoint{3.545349in}{1.927765in}}%
\pgfpathlineto{\pgfqpoint{3.548029in}{1.926371in}}%
\pgfpathlineto{\pgfqpoint{3.550713in}{1.922136in}}%
\pgfpathlineto{\pgfqpoint{3.553498in}{1.915378in}}%
\pgfpathlineto{\pgfqpoint{3.556061in}{1.922609in}}%
\pgfpathlineto{\pgfqpoint{3.558853in}{1.917006in}}%
\pgfpathlineto{\pgfqpoint{3.561420in}{1.918487in}}%
\pgfpathlineto{\pgfqpoint{3.564188in}{1.912525in}}%
\pgfpathlineto{\pgfqpoint{3.566774in}{1.914029in}}%
\pgfpathlineto{\pgfqpoint{3.569584in}{1.912712in}}%
\pgfpathlineto{\pgfqpoint{3.572126in}{1.919303in}}%
\pgfpathlineto{\pgfqpoint{3.574814in}{1.917888in}}%
\pgfpathlineto{\pgfqpoint{3.577487in}{1.912395in}}%
\pgfpathlineto{\pgfqpoint{3.580191in}{1.911537in}}%
\pgfpathlineto{\pgfqpoint{3.582851in}{1.907213in}}%
\pgfpathlineto{\pgfqpoint{3.585532in}{1.911023in}}%
\pgfpathlineto{\pgfqpoint{3.588258in}{1.917002in}}%
\pgfpathlineto{\pgfqpoint{3.590883in}{1.920332in}}%
\pgfpathlineto{\pgfqpoint{3.593620in}{1.923744in}}%
\pgfpathlineto{\pgfqpoint{3.596240in}{1.919372in}}%
\pgfpathlineto{\pgfqpoint{3.598998in}{1.915868in}}%
\pgfpathlineto{\pgfqpoint{3.601590in}{1.923097in}}%
\pgfpathlineto{\pgfqpoint{3.604387in}{1.919812in}}%
\pgfpathlineto{\pgfqpoint{3.606951in}{1.917265in}}%
\pgfpathlineto{\pgfqpoint{3.609632in}{1.916418in}}%
\pgfpathlineto{\pgfqpoint{3.612311in}{1.920442in}}%
\pgfpathlineto{\pgfqpoint{3.614982in}{1.926075in}}%
\pgfpathlineto{\pgfqpoint{3.617667in}{1.920247in}}%
\pgfpathlineto{\pgfqpoint{3.620345in}{1.920564in}}%
\pgfpathlineto{\pgfqpoint{3.623165in}{1.922919in}}%
\pgfpathlineto{\pgfqpoint{3.625689in}{1.925890in}}%
\pgfpathlineto{\pgfqpoint{3.628460in}{1.921518in}}%
\pgfpathlineto{\pgfqpoint{3.631058in}{1.923984in}}%
\pgfpathlineto{\pgfqpoint{3.633858in}{1.923117in}}%
\pgfpathlineto{\pgfqpoint{3.636413in}{1.925689in}}%
\pgfpathlineto{\pgfqpoint{3.639207in}{1.919408in}}%
\pgfpathlineto{\pgfqpoint{3.641773in}{1.920035in}}%
\pgfpathlineto{\pgfqpoint{3.644452in}{1.913555in}}%
\pgfpathlineto{\pgfqpoint{3.647130in}{1.907138in}}%
\pgfpathlineto{\pgfqpoint{3.649837in}{1.912310in}}%
\pgfpathlineto{\pgfqpoint{3.652628in}{1.909965in}}%
\pgfpathlineto{\pgfqpoint{3.655165in}{1.910993in}}%
\pgfpathlineto{\pgfqpoint{3.657917in}{1.907223in}}%
\pgfpathlineto{\pgfqpoint{3.660515in}{1.903189in}}%
\pgfpathlineto{\pgfqpoint{3.663276in}{1.901094in}}%
\pgfpathlineto{\pgfqpoint{3.665864in}{1.901186in}}%
\pgfpathlineto{\pgfqpoint{3.668665in}{1.900796in}}%
\pgfpathlineto{\pgfqpoint{3.671232in}{1.906240in}}%
\pgfpathlineto{\pgfqpoint{3.673911in}{1.901068in}}%
\pgfpathlineto{\pgfqpoint{3.676591in}{1.900796in}}%
\pgfpathlineto{\pgfqpoint{3.679273in}{1.900796in}}%
\pgfpathlineto{\pgfqpoint{3.681948in}{1.900796in}}%
\pgfpathlineto{\pgfqpoint{3.684620in}{1.900796in}}%
\pgfpathlineto{\pgfqpoint{3.687442in}{1.900796in}}%
\pgfpathlineto{\pgfqpoint{3.689983in}{1.902481in}}%
\pgfpathlineto{\pgfqpoint{3.692765in}{1.907239in}}%
\pgfpathlineto{\pgfqpoint{3.695331in}{1.903900in}}%
\pgfpathlineto{\pgfqpoint{3.698125in}{1.906486in}}%
\pgfpathlineto{\pgfqpoint{3.700684in}{1.907241in}}%
\pgfpathlineto{\pgfqpoint{3.703460in}{1.904473in}}%
\pgfpathlineto{\pgfqpoint{3.706053in}{1.911136in}}%
\pgfpathlineto{\pgfqpoint{3.708729in}{1.902110in}}%
\pgfpathlineto{\pgfqpoint{3.711410in}{1.903750in}}%
\pgfpathlineto{\pgfqpoint{3.714086in}{1.902051in}}%
\pgfpathlineto{\pgfqpoint{3.716875in}{1.903411in}}%
\pgfpathlineto{\pgfqpoint{3.719446in}{1.907810in}}%
\pgfpathlineto{\pgfqpoint{3.722228in}{1.907158in}}%
\pgfpathlineto{\pgfqpoint{3.724804in}{1.910778in}}%
\pgfpathlineto{\pgfqpoint{3.727581in}{1.910866in}}%
\pgfpathlineto{\pgfqpoint{3.730158in}{1.908114in}}%
\pgfpathlineto{\pgfqpoint{3.732950in}{1.912791in}}%
\pgfpathlineto{\pgfqpoint{3.735509in}{1.917800in}}%
\pgfpathlineto{\pgfqpoint{3.738194in}{1.920300in}}%
\pgfpathlineto{\pgfqpoint{3.740874in}{1.916954in}}%
\pgfpathlineto{\pgfqpoint{3.743548in}{1.917092in}}%
\pgfpathlineto{\pgfqpoint{3.746229in}{1.914338in}}%
\pgfpathlineto{\pgfqpoint{3.748903in}{1.916081in}}%
\pgfpathlineto{\pgfqpoint{3.751728in}{1.920216in}}%
\pgfpathlineto{\pgfqpoint{3.754265in}{1.916878in}}%
\pgfpathlineto{\pgfqpoint{3.757065in}{1.914074in}}%
\pgfpathlineto{\pgfqpoint{3.759622in}{1.917439in}}%
\pgfpathlineto{\pgfqpoint{3.762389in}{1.918306in}}%
\pgfpathlineto{\pgfqpoint{3.764966in}{1.934058in}}%
\pgfpathlineto{\pgfqpoint{3.767782in}{1.931098in}}%
\pgfpathlineto{\pgfqpoint{3.770323in}{1.923519in}}%
\pgfpathlineto{\pgfqpoint{3.773014in}{1.910331in}}%
\pgfpathlineto{\pgfqpoint{3.775691in}{1.900796in}}%
\pgfpathlineto{\pgfqpoint{3.778370in}{1.903880in}}%
\pgfpathlineto{\pgfqpoint{3.781046in}{1.901899in}}%
\pgfpathlineto{\pgfqpoint{3.783725in}{1.901376in}}%
\pgfpathlineto{\pgfqpoint{3.786504in}{1.900796in}}%
\pgfpathlineto{\pgfqpoint{3.789084in}{1.907270in}}%
\pgfpathlineto{\pgfqpoint{3.791897in}{1.903936in}}%
\pgfpathlineto{\pgfqpoint{3.794435in}{1.910973in}}%
\pgfpathlineto{\pgfqpoint{3.797265in}{1.907516in}}%
\pgfpathlineto{\pgfqpoint{3.799797in}{1.911070in}}%
\pgfpathlineto{\pgfqpoint{3.802569in}{1.901264in}}%
\pgfpathlineto{\pgfqpoint{3.805145in}{1.906473in}}%
\pgfpathlineto{\pgfqpoint{3.807832in}{1.906795in}}%
\pgfpathlineto{\pgfqpoint{3.810510in}{1.910804in}}%
\pgfpathlineto{\pgfqpoint{3.813172in}{1.913989in}}%
\pgfpathlineto{\pgfqpoint{3.815983in}{1.912229in}}%
\pgfpathlineto{\pgfqpoint{3.818546in}{1.914323in}}%
\pgfpathlineto{\pgfqpoint{3.821315in}{1.909544in}}%
\pgfpathlineto{\pgfqpoint{3.823903in}{1.908405in}}%
\pgfpathlineto{\pgfqpoint{3.826679in}{1.917297in}}%
\pgfpathlineto{\pgfqpoint{3.829252in}{1.917710in}}%
\pgfpathlineto{\pgfqpoint{3.832053in}{1.914941in}}%
\pgfpathlineto{\pgfqpoint{3.834616in}{1.916955in}}%
\pgfpathlineto{\pgfqpoint{3.837286in}{1.916366in}}%
\pgfpathlineto{\pgfqpoint{3.839960in}{1.917906in}}%
\pgfpathlineto{\pgfqpoint{3.842641in}{1.919614in}}%
\pgfpathlineto{\pgfqpoint{3.845329in}{1.915519in}}%
\pgfpathlineto{\pgfqpoint{3.848005in}{1.918496in}}%
\pgfpathlineto{\pgfqpoint{3.850814in}{1.915009in}}%
\pgfpathlineto{\pgfqpoint{3.853358in}{1.915530in}}%
\pgfpathlineto{\pgfqpoint{3.856100in}{1.914602in}}%
\pgfpathlineto{\pgfqpoint{3.858720in}{1.916608in}}%
\pgfpathlineto{\pgfqpoint{3.861561in}{1.916765in}}%
\pgfpathlineto{\pgfqpoint{3.864073in}{1.915804in}}%
\pgfpathlineto{\pgfqpoint{3.866815in}{1.915144in}}%
\pgfpathlineto{\pgfqpoint{3.869435in}{1.910970in}}%
\pgfpathlineto{\pgfqpoint{3.872114in}{1.911388in}}%
\pgfpathlineto{\pgfqpoint{3.874790in}{1.914016in}}%
\pgfpathlineto{\pgfqpoint{3.877466in}{1.913610in}}%
\pgfpathlineto{\pgfqpoint{3.880237in}{1.911692in}}%
\pgfpathlineto{\pgfqpoint{3.882850in}{1.915642in}}%
\pgfpathlineto{\pgfqpoint{3.885621in}{1.916251in}}%
\pgfpathlineto{\pgfqpoint{3.888188in}{1.972717in}}%
\pgfpathlineto{\pgfqpoint{3.890926in}{2.037778in}}%
\pgfpathlineto{\pgfqpoint{3.893541in}{2.068203in}}%
\pgfpathlineto{\pgfqpoint{3.896345in}{2.043981in}}%
\pgfpathlineto{\pgfqpoint{3.898891in}{2.020638in}}%
\pgfpathlineto{\pgfqpoint{3.901573in}{2.007362in}}%
\pgfpathlineto{\pgfqpoint{3.904252in}{1.988394in}}%
\pgfpathlineto{\pgfqpoint{3.906918in}{1.974511in}}%
\pgfpathlineto{\pgfqpoint{3.909602in}{1.959286in}}%
\pgfpathlineto{\pgfqpoint{3.912296in}{1.946807in}}%
\pgfpathlineto{\pgfqpoint{3.915107in}{1.960555in}}%
\pgfpathlineto{\pgfqpoint{3.917646in}{1.954663in}}%
\pgfpathlineto{\pgfqpoint{3.920412in}{1.945388in}}%
\pgfpathlineto{\pgfqpoint{3.923005in}{1.947644in}}%
\pgfpathlineto{\pgfqpoint{3.925778in}{1.940137in}}%
\pgfpathlineto{\pgfqpoint{3.928347in}{1.934726in}}%
\pgfpathlineto{\pgfqpoint{3.931202in}{1.929910in}}%
\pgfpathlineto{\pgfqpoint{3.933714in}{1.922173in}}%
\pgfpathlineto{\pgfqpoint{3.936395in}{1.926623in}}%
\pgfpathlineto{\pgfqpoint{3.939075in}{1.924837in}}%
\pgfpathlineto{\pgfqpoint{3.941778in}{1.915995in}}%
\pgfpathlineto{\pgfqpoint{3.944431in}{1.914021in}}%
\pgfpathlineto{\pgfqpoint{3.947101in}{1.911258in}}%
\pgfpathlineto{\pgfqpoint{3.949894in}{1.913793in}}%
\pgfpathlineto{\pgfqpoint{3.952464in}{1.913808in}}%
\pgfpathlineto{\pgfqpoint{3.955211in}{1.915113in}}%
\pgfpathlineto{\pgfqpoint{3.957823in}{1.918373in}}%
\pgfpathlineto{\pgfqpoint{3.960635in}{1.920962in}}%
\pgfpathlineto{\pgfqpoint{3.963176in}{1.919326in}}%
\pgfpathlineto{\pgfqpoint{3.966013in}{1.919264in}}%
\pgfpathlineto{\pgfqpoint{3.968523in}{1.919848in}}%
\pgfpathlineto{\pgfqpoint{3.971250in}{1.920521in}}%
\pgfpathlineto{\pgfqpoint{3.973885in}{1.917332in}}%
\pgfpathlineto{\pgfqpoint{3.976563in}{1.921373in}}%
\pgfpathlineto{\pgfqpoint{3.979389in}{1.915893in}}%
\pgfpathlineto{\pgfqpoint{3.981929in}{1.919498in}}%
\pgfpathlineto{\pgfqpoint{3.984714in}{1.915678in}}%
\pgfpathlineto{\pgfqpoint{3.987270in}{1.913001in}}%
\pgfpathlineto{\pgfqpoint{3.990055in}{1.919321in}}%
\pgfpathlineto{\pgfqpoint{3.992642in}{1.912502in}}%
\pgfpathlineto{\pgfqpoint{3.995417in}{1.921037in}}%
\pgfpathlineto{\pgfqpoint{3.997990in}{1.920014in}}%
\pgfpathlineto{\pgfqpoint{4.000674in}{1.920311in}}%
\pgfpathlineto{\pgfqpoint{4.003348in}{1.920900in}}%
\pgfpathlineto{\pgfqpoint{4.006034in}{1.914168in}}%
\pgfpathlineto{\pgfqpoint{4.008699in}{1.919435in}}%
\pgfpathlineto{\pgfqpoint{4.011394in}{1.924561in}}%
\pgfpathlineto{\pgfqpoint{4.014186in}{1.916133in}}%
\pgfpathlineto{\pgfqpoint{4.016744in}{1.920076in}}%
\pgfpathlineto{\pgfqpoint{4.019518in}{1.919011in}}%
\pgfpathlineto{\pgfqpoint{4.022097in}{1.926067in}}%
\pgfpathlineto{\pgfqpoint{4.024868in}{1.920783in}}%
\pgfpathlineto{\pgfqpoint{4.027447in}{1.919469in}}%
\pgfpathlineto{\pgfqpoint{4.030229in}{1.917695in}}%
\pgfpathlineto{\pgfqpoint{4.032817in}{1.918215in}}%
\pgfpathlineto{\pgfqpoint{4.035492in}{1.918683in}}%
\pgfpathlineto{\pgfqpoint{4.038174in}{1.913812in}}%
\pgfpathlineto{\pgfqpoint{4.040852in}{1.912618in}}%
\pgfpathlineto{\pgfqpoint{4.043667in}{1.915832in}}%
\pgfpathlineto{\pgfqpoint{4.046210in}{1.915714in}}%
\pgfpathlineto{\pgfqpoint{4.049006in}{1.918525in}}%
\pgfpathlineto{\pgfqpoint{4.051557in}{1.918104in}}%
\pgfpathlineto{\pgfqpoint{4.054326in}{1.916168in}}%
\pgfpathlineto{\pgfqpoint{4.056911in}{1.911150in}}%
\pgfpathlineto{\pgfqpoint{4.059702in}{1.913400in}}%
\pgfpathlineto{\pgfqpoint{4.062266in}{1.914881in}}%
\pgfpathlineto{\pgfqpoint{4.064957in}{1.920312in}}%
\pgfpathlineto{\pgfqpoint{4.067636in}{1.915514in}}%
\pgfpathlineto{\pgfqpoint{4.070313in}{1.922987in}}%
\pgfpathlineto{\pgfqpoint{4.072985in}{1.916160in}}%
\pgfpathlineto{\pgfqpoint{4.075705in}{1.918650in}}%
\pgfpathlineto{\pgfqpoint{4.078471in}{1.915994in}}%
\pgfpathlineto{\pgfqpoint{4.081018in}{1.914940in}}%
\pgfpathlineto{\pgfqpoint{4.083870in}{1.914531in}}%
\pgfpathlineto{\pgfqpoint{4.086385in}{1.919703in}}%
\pgfpathlineto{\pgfqpoint{4.089159in}{1.914638in}}%
\pgfpathlineto{\pgfqpoint{4.091729in}{1.918667in}}%
\pgfpathlineto{\pgfqpoint{4.094527in}{1.913352in}}%
\pgfpathlineto{\pgfqpoint{4.097092in}{1.917437in}}%
\pgfpathlineto{\pgfqpoint{4.099777in}{1.914598in}}%
\pgfpathlineto{\pgfqpoint{4.102456in}{1.912221in}}%
\pgfpathlineto{\pgfqpoint{4.105185in}{1.913129in}}%
\pgfpathlineto{\pgfqpoint{4.107814in}{1.913231in}}%
\pgfpathlineto{\pgfqpoint{4.110488in}{1.915530in}}%
\pgfpathlineto{\pgfqpoint{4.113252in}{1.918976in}}%
\pgfpathlineto{\pgfqpoint{4.115844in}{1.916169in}}%
\pgfpathlineto{\pgfqpoint{4.118554in}{1.915606in}}%
\pgfpathlineto{\pgfqpoint{4.121205in}{1.918158in}}%
\pgfpathlineto{\pgfqpoint{4.124019in}{1.911478in}}%
\pgfpathlineto{\pgfqpoint{4.126553in}{1.911238in}}%
\pgfpathlineto{\pgfqpoint{4.129349in}{1.918385in}}%
\pgfpathlineto{\pgfqpoint{4.131920in}{1.917661in}}%
\pgfpathlineto{\pgfqpoint{4.134615in}{1.915716in}}%
\pgfpathlineto{\pgfqpoint{4.137272in}{1.920060in}}%
\pgfpathlineto{\pgfqpoint{4.139963in}{1.915869in}}%
\pgfpathlineto{\pgfqpoint{4.142713in}{1.911409in}}%
\pgfpathlineto{\pgfqpoint{4.145310in}{1.915525in}}%
\pgfpathlineto{\pgfqpoint{4.148082in}{1.913648in}}%
\pgfpathlineto{\pgfqpoint{4.150665in}{1.915666in}}%
\pgfpathlineto{\pgfqpoint{4.153423in}{1.918024in}}%
\pgfpathlineto{\pgfqpoint{4.156016in}{1.916737in}}%
\pgfpathlineto{\pgfqpoint{4.158806in}{1.911767in}}%
\pgfpathlineto{\pgfqpoint{4.161380in}{1.913012in}}%
\pgfpathlineto{\pgfqpoint{4.164059in}{1.919456in}}%
\pgfpathlineto{\pgfqpoint{4.166737in}{1.918225in}}%
\pgfpathlineto{\pgfqpoint{4.169415in}{1.912346in}}%
\pgfpathlineto{\pgfqpoint{4.172093in}{1.910670in}}%
\pgfpathlineto{\pgfqpoint{4.174770in}{1.911496in}}%
\pgfpathlineto{\pgfqpoint{4.177593in}{1.912243in}}%
\pgfpathlineto{\pgfqpoint{4.180129in}{1.909764in}}%
\pgfpathlineto{\pgfqpoint{4.182899in}{1.909433in}}%
\pgfpathlineto{\pgfqpoint{4.185481in}{1.906042in}}%
\pgfpathlineto{\pgfqpoint{4.188318in}{1.907343in}}%
\pgfpathlineto{\pgfqpoint{4.190842in}{1.906998in}}%
\pgfpathlineto{\pgfqpoint{4.193638in}{1.911170in}}%
\pgfpathlineto{\pgfqpoint{4.196186in}{1.915222in}}%
\pgfpathlineto{\pgfqpoint{4.198878in}{1.912129in}}%
\pgfpathlineto{\pgfqpoint{4.201542in}{1.910707in}}%
\pgfpathlineto{\pgfqpoint{4.204240in}{1.910824in}}%
\pgfpathlineto{\pgfqpoint{4.207076in}{1.908287in}}%
\pgfpathlineto{\pgfqpoint{4.209597in}{1.908894in}}%
\pgfpathlineto{\pgfqpoint{4.212383in}{1.913431in}}%
\pgfpathlineto{\pgfqpoint{4.214948in}{1.916124in}}%
\pgfpathlineto{\pgfqpoint{4.217694in}{1.919116in}}%
\pgfpathlineto{\pgfqpoint{4.220304in}{1.920985in}}%
\pgfpathlineto{\pgfqpoint{4.223082in}{1.919916in}}%
\pgfpathlineto{\pgfqpoint{4.225654in}{1.916191in}}%
\pgfpathlineto{\pgfqpoint{4.228331in}{1.910160in}}%
\pgfpathlineto{\pgfqpoint{4.231013in}{1.914018in}}%
\pgfpathlineto{\pgfqpoint{4.233691in}{1.913602in}}%
\pgfpathlineto{\pgfqpoint{4.236375in}{1.912354in}}%
\pgfpathlineto{\pgfqpoint{4.239084in}{1.914144in}}%
\pgfpathlineto{\pgfqpoint{4.241900in}{1.919096in}}%
\pgfpathlineto{\pgfqpoint{4.244394in}{1.918525in}}%
\pgfpathlineto{\pgfqpoint{4.247225in}{1.918569in}}%
\pgfpathlineto{\pgfqpoint{4.249767in}{1.913470in}}%
\pgfpathlineto{\pgfqpoint{4.252581in}{1.915152in}}%
\pgfpathlineto{\pgfqpoint{4.255120in}{1.917194in}}%
\pgfpathlineto{\pgfqpoint{4.257958in}{1.919026in}}%
\pgfpathlineto{\pgfqpoint{4.260477in}{1.913662in}}%
\pgfpathlineto{\pgfqpoint{4.263157in}{1.919768in}}%
\pgfpathlineto{\pgfqpoint{4.265824in}{1.916432in}}%
\pgfpathlineto{\pgfqpoint{4.268590in}{1.916386in}}%
\pgfpathlineto{\pgfqpoint{4.271187in}{1.922547in}}%
\pgfpathlineto{\pgfqpoint{4.273874in}{1.924836in}}%
\pgfpathlineto{\pgfqpoint{4.276635in}{1.922164in}}%
\pgfpathlineto{\pgfqpoint{4.279212in}{1.918136in}}%
\pgfpathlineto{\pgfqpoint{4.282000in}{1.914898in}}%
\pgfpathlineto{\pgfqpoint{4.284586in}{1.908489in}}%
\pgfpathlineto{\pgfqpoint{4.287399in}{1.910498in}}%
\pgfpathlineto{\pgfqpoint{4.289936in}{1.917784in}}%
\pgfpathlineto{\pgfqpoint{4.292786in}{1.917527in}}%
\pgfpathlineto{\pgfqpoint{4.295299in}{1.911151in}}%
\pgfpathlineto{\pgfqpoint{4.297977in}{1.914258in}}%
\pgfpathlineto{\pgfqpoint{4.300656in}{1.912619in}}%
\pgfpathlineto{\pgfqpoint{4.303357in}{1.915840in}}%
\pgfpathlineto{\pgfqpoint{4.306118in}{1.915078in}}%
\pgfpathlineto{\pgfqpoint{4.308691in}{1.918040in}}%
\pgfpathlineto{\pgfqpoint{4.311494in}{1.917091in}}%
\pgfpathlineto{\pgfqpoint{4.314032in}{1.919054in}}%
\pgfpathlineto{\pgfqpoint{4.316856in}{1.919550in}}%
\pgfpathlineto{\pgfqpoint{4.319405in}{1.920085in}}%
\pgfpathlineto{\pgfqpoint{4.322181in}{1.922907in}}%
\pgfpathlineto{\pgfqpoint{4.324760in}{1.925263in}}%
\pgfpathlineto{\pgfqpoint{4.327440in}{1.919753in}}%
\pgfpathlineto{\pgfqpoint{4.330118in}{1.924995in}}%
\pgfpathlineto{\pgfqpoint{4.332796in}{1.918790in}}%
\pgfpathlineto{\pgfqpoint{4.335463in}{1.917424in}}%
\pgfpathlineto{\pgfqpoint{4.338154in}{1.918244in}}%
\pgfpathlineto{\pgfqpoint{4.340976in}{1.917936in}}%
\pgfpathlineto{\pgfqpoint{4.343510in}{1.915387in}}%
\pgfpathlineto{\pgfqpoint{4.346263in}{1.916953in}}%
\pgfpathlineto{\pgfqpoint{4.348868in}{1.914738in}}%
\pgfpathlineto{\pgfqpoint{4.351645in}{1.915876in}}%
\pgfpathlineto{\pgfqpoint{4.354224in}{1.917341in}}%
\pgfpathlineto{\pgfqpoint{4.357014in}{1.917581in}}%
\pgfpathlineto{\pgfqpoint{4.359582in}{1.919911in}}%
\pgfpathlineto{\pgfqpoint{4.362270in}{1.915471in}}%
\pgfpathlineto{\pgfqpoint{4.364936in}{1.915100in}}%
\pgfpathlineto{\pgfqpoint{4.367646in}{1.918033in}}%
\pgfpathlineto{\pgfqpoint{4.370437in}{1.914527in}}%
\pgfpathlineto{\pgfqpoint{4.372976in}{1.919765in}}%
\pgfpathlineto{\pgfqpoint{4.375761in}{1.917032in}}%
\pgfpathlineto{\pgfqpoint{4.378329in}{1.910370in}}%
\pgfpathlineto{\pgfqpoint{4.381097in}{1.912241in}}%
\pgfpathlineto{\pgfqpoint{4.383674in}{1.916895in}}%
\pgfpathlineto{\pgfqpoint{4.386431in}{1.908605in}}%
\pgfpathlineto{\pgfqpoint{4.389044in}{1.911756in}}%
\pgfpathlineto{\pgfqpoint{4.391721in}{1.913545in}}%
\pgfpathlineto{\pgfqpoint{4.394400in}{1.929000in}}%
\pgfpathlineto{\pgfqpoint{4.397076in}{1.961172in}}%
\pgfpathlineto{\pgfqpoint{4.399745in}{1.947020in}}%
\pgfpathlineto{\pgfqpoint{4.402468in}{1.926329in}}%
\pgfpathlineto{\pgfqpoint{4.405234in}{1.919493in}}%
\pgfpathlineto{\pgfqpoint{4.407788in}{1.910592in}}%
\pgfpathlineto{\pgfqpoint{4.410587in}{1.918034in}}%
\pgfpathlineto{\pgfqpoint{4.413149in}{1.913655in}}%
\pgfpathlineto{\pgfqpoint{4.415932in}{1.910060in}}%
\pgfpathlineto{\pgfqpoint{4.418506in}{1.913731in}}%
\pgfpathlineto{\pgfqpoint{4.421292in}{1.915392in}}%
\pgfpathlineto{\pgfqpoint{4.423863in}{1.914923in}}%
\pgfpathlineto{\pgfqpoint{4.426534in}{1.917244in}}%
\pgfpathlineto{\pgfqpoint{4.429220in}{1.915264in}}%
\pgfpathlineto{\pgfqpoint{4.431901in}{1.916140in}}%
\pgfpathlineto{\pgfqpoint{4.434569in}{1.918670in}}%
\pgfpathlineto{\pgfqpoint{4.437253in}{1.918702in}}%
\pgfpathlineto{\pgfqpoint{4.440041in}{1.918950in}}%
\pgfpathlineto{\pgfqpoint{4.442611in}{1.918899in}}%
\pgfpathlineto{\pgfqpoint{4.445423in}{1.921889in}}%
\pgfpathlineto{\pgfqpoint{4.447965in}{1.918601in}}%
\pgfpathlineto{\pgfqpoint{4.450767in}{1.919366in}}%
\pgfpathlineto{\pgfqpoint{4.453312in}{1.920597in}}%
\pgfpathlineto{\pgfqpoint{4.456138in}{1.918899in}}%
\pgfpathlineto{\pgfqpoint{4.458681in}{1.916811in}}%
\pgfpathlineto{\pgfqpoint{4.461367in}{1.917154in}}%
\pgfpathlineto{\pgfqpoint{4.464029in}{1.918533in}}%
\pgfpathlineto{\pgfqpoint{4.466717in}{1.916707in}}%
\pgfpathlineto{\pgfqpoint{4.469492in}{1.917031in}}%
\pgfpathlineto{\pgfqpoint{4.472059in}{1.914683in}}%
\pgfpathlineto{\pgfqpoint{4.474861in}{1.919703in}}%
\pgfpathlineto{\pgfqpoint{4.477430in}{1.917549in}}%
\pgfpathlineto{\pgfqpoint{4.480201in}{1.921476in}}%
\pgfpathlineto{\pgfqpoint{4.482778in}{1.916527in}}%
\pgfpathlineto{\pgfqpoint{4.485581in}{1.920875in}}%
\pgfpathlineto{\pgfqpoint{4.488130in}{1.917178in}}%
\pgfpathlineto{\pgfqpoint{4.490822in}{1.916576in}}%
\pgfpathlineto{\pgfqpoint{4.493492in}{1.915621in}}%
\pgfpathlineto{\pgfqpoint{4.496167in}{1.917877in}}%
\pgfpathlineto{\pgfqpoint{4.498850in}{1.914725in}}%
\pgfpathlineto{\pgfqpoint{4.501529in}{1.917235in}}%
\pgfpathlineto{\pgfqpoint{4.504305in}{1.916472in}}%
\pgfpathlineto{\pgfqpoint{4.506893in}{1.916177in}}%
\pgfpathlineto{\pgfqpoint{4.509643in}{1.916943in}}%
\pgfpathlineto{\pgfqpoint{4.512246in}{1.915195in}}%
\pgfpathlineto{\pgfqpoint{4.515080in}{1.917495in}}%
\pgfpathlineto{\pgfqpoint{4.517598in}{1.916523in}}%
\pgfpathlineto{\pgfqpoint{4.520345in}{1.922706in}}%
\pgfpathlineto{\pgfqpoint{4.522962in}{1.925823in}}%
\pgfpathlineto{\pgfqpoint{4.525640in}{1.917845in}}%
\pgfpathlineto{\pgfqpoint{4.528307in}{1.923401in}}%
\pgfpathlineto{\pgfqpoint{4.530990in}{1.915239in}}%
\pgfpathlineto{\pgfqpoint{4.533764in}{1.922419in}}%
\pgfpathlineto{\pgfqpoint{4.536400in}{1.944507in}}%
\pgfpathlineto{\pgfqpoint{4.539144in}{1.930726in}}%
\pgfpathlineto{\pgfqpoint{4.541711in}{1.914392in}}%
\pgfpathlineto{\pgfqpoint{4.544464in}{1.919924in}}%
\pgfpathlineto{\pgfqpoint{4.547064in}{1.914572in}}%
\pgfpathlineto{\pgfqpoint{4.549822in}{1.917148in}}%
\pgfpathlineto{\pgfqpoint{4.552425in}{1.913848in}}%
\pgfpathlineto{\pgfqpoint{4.555106in}{1.915929in}}%
\pgfpathlineto{\pgfqpoint{4.557777in}{1.921874in}}%
\pgfpathlineto{\pgfqpoint{4.560448in}{1.917701in}}%
\pgfpathlineto{\pgfqpoint{4.563125in}{1.915708in}}%
\pgfpathlineto{\pgfqpoint{4.565820in}{1.915981in}}%
\pgfpathlineto{\pgfqpoint{4.568612in}{1.919081in}}%
\pgfpathlineto{\pgfqpoint{4.571171in}{1.915231in}}%
\pgfpathlineto{\pgfqpoint{4.573947in}{1.915907in}}%
\pgfpathlineto{\pgfqpoint{4.576531in}{1.916240in}}%
\pgfpathlineto{\pgfqpoint{4.579305in}{1.920490in}}%
\pgfpathlineto{\pgfqpoint{4.581888in}{1.919180in}}%
\pgfpathlineto{\pgfqpoint{4.584672in}{1.922178in}}%
\pgfpathlineto{\pgfqpoint{4.587244in}{1.915515in}}%
\pgfpathlineto{\pgfqpoint{4.589920in}{1.911535in}}%
\pgfpathlineto{\pgfqpoint{4.592589in}{1.904642in}}%
\pgfpathlineto{\pgfqpoint{4.595281in}{1.912955in}}%
\pgfpathlineto{\pgfqpoint{4.597951in}{1.907138in}}%
\pgfpathlineto{\pgfqpoint{4.600633in}{1.915507in}}%
\pgfpathlineto{\pgfqpoint{4.603430in}{1.914522in}}%
\pgfpathlineto{\pgfqpoint{4.605990in}{1.919011in}}%
\pgfpathlineto{\pgfqpoint{4.608808in}{1.918575in}}%
\pgfpathlineto{\pgfqpoint{4.611350in}{1.918298in}}%
\pgfpathlineto{\pgfqpoint{4.614134in}{1.919807in}}%
\pgfpathlineto{\pgfqpoint{4.616702in}{1.915332in}}%
\pgfpathlineto{\pgfqpoint{4.619529in}{1.915406in}}%
\pgfpathlineto{\pgfqpoint{4.622056in}{1.918999in}}%
\pgfpathlineto{\pgfqpoint{4.624741in}{1.914004in}}%
\pgfpathlineto{\pgfqpoint{4.627411in}{1.916117in}}%
\pgfpathlineto{\pgfqpoint{4.630096in}{1.923634in}}%
\pgfpathlineto{\pgfqpoint{4.632902in}{1.924736in}}%
\pgfpathlineto{\pgfqpoint{4.635445in}{1.920865in}}%
\pgfpathlineto{\pgfqpoint{4.638204in}{1.916707in}}%
\pgfpathlineto{\pgfqpoint{4.640809in}{1.919035in}}%
\pgfpathlineto{\pgfqpoint{4.643628in}{1.916603in}}%
\pgfpathlineto{\pgfqpoint{4.646169in}{1.913622in}}%
\pgfpathlineto{\pgfqpoint{4.648922in}{1.914189in}}%
\pgfpathlineto{\pgfqpoint{4.651524in}{1.917625in}}%
\pgfpathlineto{\pgfqpoint{4.654203in}{1.921352in}}%
\pgfpathlineto{\pgfqpoint{4.656873in}{1.920735in}}%
\pgfpathlineto{\pgfqpoint{4.659590in}{1.923509in}}%
\pgfpathlineto{\pgfqpoint{4.662237in}{1.921152in}}%
\pgfpathlineto{\pgfqpoint{4.664923in}{1.914675in}}%
\pgfpathlineto{\pgfqpoint{4.667764in}{1.916834in}}%
\pgfpathlineto{\pgfqpoint{4.670261in}{1.913292in}}%
\pgfpathlineto{\pgfqpoint{4.673068in}{1.915779in}}%
\pgfpathlineto{\pgfqpoint{4.675619in}{1.921787in}}%
\pgfpathlineto{\pgfqpoint{4.678448in}{1.918449in}}%
\pgfpathlineto{\pgfqpoint{4.680988in}{1.914817in}}%
\pgfpathlineto{\pgfqpoint{4.683799in}{1.916003in}}%
\pgfpathlineto{\pgfqpoint{4.686337in}{1.926867in}}%
\pgfpathlineto{\pgfqpoint{4.689051in}{1.925838in}}%
\pgfpathlineto{\pgfqpoint{4.691694in}{1.923627in}}%
\pgfpathlineto{\pgfqpoint{4.694381in}{1.926216in}}%
\pgfpathlineto{\pgfqpoint{4.697170in}{1.924445in}}%
\pgfpathlineto{\pgfqpoint{4.699734in}{1.925340in}}%
\pgfpathlineto{\pgfqpoint{4.702517in}{1.918236in}}%
\pgfpathlineto{\pgfqpoint{4.705094in}{1.926430in}}%
\pgfpathlineto{\pgfqpoint{4.707824in}{1.918541in}}%
\pgfpathlineto{\pgfqpoint{4.710437in}{1.931618in}}%
\pgfpathlineto{\pgfqpoint{4.713275in}{1.939759in}}%
\pgfpathlineto{\pgfqpoint{4.715806in}{1.927459in}}%
\pgfpathlineto{\pgfqpoint{4.718486in}{1.924487in}}%
\pgfpathlineto{\pgfqpoint{4.721160in}{1.925343in}}%
\pgfpathlineto{\pgfqpoint{4.723873in}{1.924576in}}%
\pgfpathlineto{\pgfqpoint{4.726508in}{1.917581in}}%
\pgfpathlineto{\pgfqpoint{4.729233in}{1.916733in}}%
\pgfpathlineto{\pgfqpoint{4.731901in}{1.921814in}}%
\pgfpathlineto{\pgfqpoint{4.734552in}{1.916120in}}%
\pgfpathlineto{\pgfqpoint{4.737348in}{1.920510in}}%
\pgfpathlineto{\pgfqpoint{4.739912in}{1.914847in}}%
\pgfpathlineto{\pgfqpoint{4.742696in}{1.921157in}}%
\pgfpathlineto{\pgfqpoint{4.745256in}{1.925831in}}%
\pgfpathlineto{\pgfqpoint{4.748081in}{1.921422in}}%
\pgfpathlineto{\pgfqpoint{4.750627in}{1.921510in}}%
\pgfpathlineto{\pgfqpoint{4.753298in}{1.928675in}}%
\pgfpathlineto{\pgfqpoint{4.755983in}{1.922305in}}%
\pgfpathlineto{\pgfqpoint{4.758653in}{1.932545in}}%
\pgfpathlineto{\pgfqpoint{4.761337in}{1.943130in}}%
\pgfpathlineto{\pgfqpoint{4.764018in}{1.940874in}}%
\pgfpathlineto{\pgfqpoint{4.766783in}{1.950790in}}%
\pgfpathlineto{\pgfqpoint{4.769367in}{1.948633in}}%
\pgfpathlineto{\pgfqpoint{4.772198in}{1.936963in}}%
\pgfpathlineto{\pgfqpoint{4.774732in}{1.932664in}}%
\pgfpathlineto{\pgfqpoint{4.777535in}{1.927439in}}%
\pgfpathlineto{\pgfqpoint{4.780083in}{1.930954in}}%
\pgfpathlineto{\pgfqpoint{4.782872in}{1.925414in}}%
\pgfpathlineto{\pgfqpoint{4.785445in}{1.924536in}}%
\pgfpathlineto{\pgfqpoint{4.788116in}{1.923200in}}%
\pgfpathlineto{\pgfqpoint{4.790798in}{1.919463in}}%
\pgfpathlineto{\pgfqpoint{4.793512in}{1.923176in}}%
\pgfpathlineto{\pgfqpoint{4.796274in}{1.919247in}}%
\pgfpathlineto{\pgfqpoint{4.798830in}{1.920267in}}%
\pgfpathlineto{\pgfqpoint{4.801586in}{1.929285in}}%
\pgfpathlineto{\pgfqpoint{4.804193in}{1.944718in}}%
\pgfpathlineto{\pgfqpoint{4.807017in}{1.953455in}}%
\pgfpathlineto{\pgfqpoint{4.809538in}{1.939174in}}%
\pgfpathlineto{\pgfqpoint{4.812377in}{1.925011in}}%
\pgfpathlineto{\pgfqpoint{4.814907in}{1.927494in}}%
\pgfpathlineto{\pgfqpoint{4.817587in}{1.924666in}}%
\pgfpathlineto{\pgfqpoint{4.820265in}{1.914655in}}%
\pgfpathlineto{\pgfqpoint{4.822945in}{1.912169in}}%
\pgfpathlineto{\pgfqpoint{4.825619in}{1.912840in}}%
\pgfpathlineto{\pgfqpoint{4.828291in}{1.912588in}}%
\pgfpathlineto{\pgfqpoint{4.831045in}{1.916517in}}%
\pgfpathlineto{\pgfqpoint{4.833657in}{1.916855in}}%
\pgfpathlineto{\pgfqpoint{4.837992in}{1.917583in}}%
\pgfpathlineto{\pgfqpoint{4.839922in}{1.922629in}}%
\pgfpathlineto{\pgfqpoint{4.842380in}{1.912746in}}%
\pgfpathlineto{\pgfqpoint{4.844361in}{1.916013in}}%
\pgfpathlineto{\pgfqpoint{4.847127in}{1.912060in}}%
\pgfpathlineto{\pgfqpoint{4.849715in}{1.909445in}}%
\pgfpathlineto{\pgfqpoint{4.852404in}{1.909164in}}%
\pgfpathlineto{\pgfqpoint{4.855070in}{1.912663in}}%
\pgfpathlineto{\pgfqpoint{4.857807in}{1.911880in}}%
\pgfpathlineto{\pgfqpoint{4.860544in}{1.910749in}}%
\pgfpathlineto{\pgfqpoint{4.863116in}{1.908192in}}%
\pgfpathlineto{\pgfqpoint{4.865910in}{1.917015in}}%
\pgfpathlineto{\pgfqpoint{4.868474in}{1.915347in}}%
\pgfpathlineto{\pgfqpoint{4.871209in}{1.916513in}}%
\pgfpathlineto{\pgfqpoint{4.873832in}{1.918451in}}%
\pgfpathlineto{\pgfqpoint{4.876636in}{1.914584in}}%
\pgfpathlineto{\pgfqpoint{4.879180in}{1.916452in}}%
\pgfpathlineto{\pgfqpoint{4.881864in}{1.913292in}}%
\pgfpathlineto{\pgfqpoint{4.884540in}{1.909755in}}%
\pgfpathlineto{\pgfqpoint{4.887211in}{1.909821in}}%
\pgfpathlineto{\pgfqpoint{4.889902in}{1.918309in}}%
\pgfpathlineto{\pgfqpoint{4.892611in}{1.921141in}}%
\pgfpathlineto{\pgfqpoint{4.895399in}{1.921915in}}%
\pgfpathlineto{\pgfqpoint{4.897938in}{1.924056in}}%
\pgfpathlineto{\pgfqpoint{4.900712in}{1.918506in}}%
\pgfpathlineto{\pgfqpoint{4.903295in}{1.926648in}}%
\pgfpathlineto{\pgfqpoint{4.906096in}{1.930019in}}%
\pgfpathlineto{\pgfqpoint{4.908648in}{1.928497in}}%
\pgfpathlineto{\pgfqpoint{4.911435in}{1.919679in}}%
\pgfpathlineto{\pgfqpoint{4.914009in}{1.917683in}}%
\pgfpathlineto{\pgfqpoint{4.916681in}{1.922252in}}%
\pgfpathlineto{\pgfqpoint{4.919352in}{1.917518in}}%
\pgfpathlineto{\pgfqpoint{4.922041in}{1.918894in}}%
\pgfpathlineto{\pgfqpoint{4.924708in}{1.923575in}}%
\pgfpathlineto{\pgfqpoint{4.927400in}{1.923183in}}%
\pgfpathlineto{\pgfqpoint{4.930170in}{1.922532in}}%
\pgfpathlineto{\pgfqpoint{4.932742in}{1.926028in}}%
\pgfpathlineto{\pgfqpoint{4.935515in}{1.922087in}}%
\pgfpathlineto{\pgfqpoint{4.938112in}{1.920426in}}%
\pgfpathlineto{\pgfqpoint{4.940881in}{1.917620in}}%
\pgfpathlineto{\pgfqpoint{4.943466in}{1.920084in}}%
\pgfpathlineto{\pgfqpoint{4.946151in}{1.916656in}}%
\pgfpathlineto{\pgfqpoint{4.948827in}{1.916020in}}%
\pgfpathlineto{\pgfqpoint{4.951504in}{1.912168in}}%
\pgfpathlineto{\pgfqpoint{4.954182in}{1.911073in}}%
\pgfpathlineto{\pgfqpoint{4.956862in}{1.914490in}}%
\pgfpathlineto{\pgfqpoint{4.959689in}{1.915253in}}%
\pgfpathlineto{\pgfqpoint{4.962219in}{1.910594in}}%
\pgfpathlineto{\pgfqpoint{4.965002in}{1.911628in}}%
\pgfpathlineto{\pgfqpoint{4.967575in}{1.907843in}}%
\pgfpathlineto{\pgfqpoint{4.970314in}{1.911857in}}%
\pgfpathlineto{\pgfqpoint{4.972933in}{1.912072in}}%
\pgfpathlineto{\pgfqpoint{4.975703in}{1.908543in}}%
\pgfpathlineto{\pgfqpoint{4.978287in}{1.911162in}}%
\pgfpathlineto{\pgfqpoint{4.980967in}{1.915915in}}%
\pgfpathlineto{\pgfqpoint{4.983637in}{1.920021in}}%
\pgfpathlineto{\pgfqpoint{4.986325in}{1.920667in}}%
\pgfpathlineto{\pgfqpoint{4.989001in}{1.920943in}}%
\pgfpathlineto{\pgfqpoint{4.991683in}{1.915506in}}%
\pgfpathlineto{\pgfqpoint{4.994390in}{1.912408in}}%
\pgfpathlineto{\pgfqpoint{4.997028in}{1.915368in}}%
\pgfpathlineto{\pgfqpoint{4.999780in}{1.913206in}}%
\pgfpathlineto{\pgfqpoint{5.002384in}{1.919015in}}%
\pgfpathlineto{\pgfqpoint{5.005178in}{1.914167in}}%
\pgfpathlineto{\pgfqpoint{5.007751in}{1.920407in}}%
\pgfpathlineto{\pgfqpoint{5.010562in}{1.922378in}}%
\pgfpathlineto{\pgfqpoint{5.013104in}{1.922445in}}%
\pgfpathlineto{\pgfqpoint{5.015820in}{1.920567in}}%
\pgfpathlineto{\pgfqpoint{5.018466in}{1.921039in}}%
\pgfpathlineto{\pgfqpoint{5.021147in}{1.914636in}}%
\pgfpathlineto{\pgfqpoint{5.023927in}{1.914528in}}%
\pgfpathlineto{\pgfqpoint{5.026501in}{1.910252in}}%
\pgfpathlineto{\pgfqpoint{5.029275in}{1.915892in}}%
\pgfpathlineto{\pgfqpoint{5.031849in}{1.918483in}}%
\pgfpathlineto{\pgfqpoint{5.034649in}{1.915920in}}%
\pgfpathlineto{\pgfqpoint{5.037214in}{1.909206in}}%
\pgfpathlineto{\pgfqpoint{5.039962in}{1.905342in}}%
\pgfpathlineto{\pgfqpoint{5.042572in}{1.906463in}}%
\pgfpathlineto{\pgfqpoint{5.045249in}{1.906001in}}%
\pgfpathlineto{\pgfqpoint{5.047924in}{1.909362in}}%
\pgfpathlineto{\pgfqpoint{5.050606in}{1.912869in}}%
\pgfpathlineto{\pgfqpoint{5.053284in}{1.916704in}}%
\pgfpathlineto{\pgfqpoint{5.055952in}{1.924247in}}%
\pgfpathlineto{\pgfqpoint{5.058711in}{1.914391in}}%
\pgfpathlineto{\pgfqpoint{5.061315in}{1.913339in}}%
\pgfpathlineto{\pgfqpoint{5.064144in}{1.913684in}}%
\pgfpathlineto{\pgfqpoint{5.066677in}{1.915121in}}%
\pgfpathlineto{\pgfqpoint{5.069463in}{1.917621in}}%
\pgfpathlineto{\pgfqpoint{5.072030in}{1.913808in}}%
\pgfpathlineto{\pgfqpoint{5.074851in}{1.912950in}}%
\pgfpathlineto{\pgfqpoint{5.077390in}{1.917347in}}%
\pgfpathlineto{\pgfqpoint{5.080067in}{1.916642in}}%
\pgfpathlineto{\pgfqpoint{5.082746in}{1.913588in}}%
\pgfpathlineto{\pgfqpoint{5.085426in}{1.914893in}}%
\pgfpathlineto{\pgfqpoint{5.088103in}{1.919262in}}%
\pgfpathlineto{\pgfqpoint{5.090788in}{1.921376in}}%
\pgfpathlineto{\pgfqpoint{5.093579in}{1.926610in}}%
\pgfpathlineto{\pgfqpoint{5.096142in}{1.929128in}}%
\pgfpathlineto{\pgfqpoint{5.098948in}{1.928501in}}%
\pgfpathlineto{\pgfqpoint{5.101496in}{1.922684in}}%
\pgfpathlineto{\pgfqpoint{5.104312in}{1.923718in}}%
\pgfpathlineto{\pgfqpoint{5.106842in}{1.919611in}}%
\pgfpathlineto{\pgfqpoint{5.109530in}{1.920260in}}%
\pgfpathlineto{\pgfqpoint{5.112209in}{1.921637in}}%
\pgfpathlineto{\pgfqpoint{5.114887in}{1.916896in}}%
\pgfpathlineto{\pgfqpoint{5.117550in}{1.919945in}}%
\pgfpathlineto{\pgfqpoint{5.120243in}{1.919536in}}%
\pgfpathlineto{\pgfqpoint{5.123042in}{1.918761in}}%
\pgfpathlineto{\pgfqpoint{5.125599in}{1.918610in}}%
\pgfpathlineto{\pgfqpoint{5.128421in}{1.924808in}}%
\pgfpathlineto{\pgfqpoint{5.130953in}{1.927098in}}%
\pgfpathlineto{\pgfqpoint{5.133716in}{1.924961in}}%
\pgfpathlineto{\pgfqpoint{5.136311in}{1.924089in}}%
\pgfpathlineto{\pgfqpoint{5.139072in}{1.919560in}}%
\pgfpathlineto{\pgfqpoint{5.141660in}{1.919230in}}%
\pgfpathlineto{\pgfqpoint{5.144349in}{1.917057in}}%
\pgfpathlineto{\pgfqpoint{5.147029in}{1.919920in}}%
\pgfpathlineto{\pgfqpoint{5.149734in}{1.927739in}}%
\pgfpathlineto{\pgfqpoint{5.152382in}{1.930464in}}%
\pgfpathlineto{\pgfqpoint{5.155059in}{1.919975in}}%
\pgfpathlineto{\pgfqpoint{5.157815in}{1.918072in}}%
\pgfpathlineto{\pgfqpoint{5.160420in}{1.911145in}}%
\pgfpathlineto{\pgfqpoint{5.163243in}{1.917188in}}%
\pgfpathlineto{\pgfqpoint{5.165775in}{1.913374in}}%
\pgfpathlineto{\pgfqpoint{5.168591in}{1.918429in}}%
\pgfpathlineto{\pgfqpoint{5.171133in}{1.914641in}}%
\pgfpathlineto{\pgfqpoint{5.173925in}{1.914034in}}%
\pgfpathlineto{\pgfqpoint{5.176477in}{1.919296in}}%
\pgfpathlineto{\pgfqpoint{5.179188in}{1.918463in}}%
\pgfpathlineto{\pgfqpoint{5.181848in}{1.926496in}}%
\pgfpathlineto{\pgfqpoint{5.184522in}{1.928070in}}%
\pgfpathlineto{\pgfqpoint{5.187294in}{1.919070in}}%
\pgfpathlineto{\pgfqpoint{5.189880in}{1.922902in}}%
\pgfpathlineto{\pgfqpoint{5.192680in}{1.922972in}}%
\pgfpathlineto{\pgfqpoint{5.195239in}{1.920197in}}%
\pgfpathlineto{\pgfqpoint{5.198008in}{1.919392in}}%
\pgfpathlineto{\pgfqpoint{5.200594in}{1.927225in}}%
\pgfpathlineto{\pgfqpoint{5.203388in}{1.928335in}}%
\pgfpathlineto{\pgfqpoint{5.205952in}{1.923572in}}%
\pgfpathlineto{\pgfqpoint{5.208630in}{1.922313in}}%
\pgfpathlineto{\pgfqpoint{5.211299in}{1.919983in}}%
\pgfpathlineto{\pgfqpoint{5.214027in}{1.922564in}}%
\pgfpathlineto{\pgfqpoint{5.216667in}{1.921190in}}%
\pgfpathlineto{\pgfqpoint{5.219345in}{1.916518in}}%
\pgfpathlineto{\pgfqpoint{5.222151in}{1.915856in}}%
\pgfpathlineto{\pgfqpoint{5.224695in}{1.917219in}}%
\pgfpathlineto{\pgfqpoint{5.227470in}{1.919525in}}%
\pgfpathlineto{\pgfqpoint{5.230059in}{1.920985in}}%
\pgfpathlineto{\pgfqpoint{5.232855in}{1.919473in}}%
\pgfpathlineto{\pgfqpoint{5.235409in}{1.920871in}}%
\pgfpathlineto{\pgfqpoint{5.238173in}{1.919871in}}%
\pgfpathlineto{\pgfqpoint{5.240777in}{1.916553in}}%
\pgfpathlineto{\pgfqpoint{5.243445in}{1.920380in}}%
\pgfpathlineto{\pgfqpoint{5.246130in}{1.919128in}}%
\pgfpathlineto{\pgfqpoint{5.248816in}{1.915736in}}%
\pgfpathlineto{\pgfqpoint{5.251590in}{1.918946in}}%
\pgfpathlineto{\pgfqpoint{5.254236in}{1.922063in}}%
\pgfpathlineto{\pgfqpoint{5.256973in}{1.922186in}}%
\pgfpathlineto{\pgfqpoint{5.259511in}{1.918177in}}%
\pgfpathlineto{\pgfqpoint{5.262264in}{1.913517in}}%
\pgfpathlineto{\pgfqpoint{5.264876in}{1.907267in}}%
\pgfpathlineto{\pgfqpoint{5.267691in}{1.913570in}}%
\pgfpathlineto{\pgfqpoint{5.270238in}{1.925915in}}%
\pgfpathlineto{\pgfqpoint{5.272913in}{1.946668in}}%
\pgfpathlineto{\pgfqpoint{5.275589in}{1.971095in}}%
\pgfpathlineto{\pgfqpoint{5.278322in}{1.948995in}}%
\pgfpathlineto{\pgfqpoint{5.280947in}{1.933974in}}%
\pgfpathlineto{\pgfqpoint{5.283631in}{1.921762in}}%
\pgfpathlineto{\pgfqpoint{5.286436in}{1.913652in}}%
\pgfpathlineto{\pgfqpoint{5.288984in}{1.915739in}}%
\pgfpathlineto{\pgfqpoint{5.291794in}{1.916096in}}%
\pgfpathlineto{\pgfqpoint{5.294339in}{1.915626in}}%
\pgfpathlineto{\pgfqpoint{5.297140in}{1.919279in}}%
\pgfpathlineto{\pgfqpoint{5.299696in}{1.916016in}}%
\pgfpathlineto{\pgfqpoint{5.302443in}{1.924836in}}%
\pgfpathlineto{\pgfqpoint{5.305054in}{1.926836in}}%
\pgfpathlineto{\pgfqpoint{5.307731in}{1.924822in}}%
\pgfpathlineto{\pgfqpoint{5.310411in}{1.922804in}}%
\pgfpathlineto{\pgfqpoint{5.313089in}{1.926837in}}%
\pgfpathlineto{\pgfqpoint{5.315754in}{1.924245in}}%
\pgfpathlineto{\pgfqpoint{5.318430in}{1.935675in}}%
\pgfpathlineto{\pgfqpoint{5.321256in}{1.932126in}}%
\pgfpathlineto{\pgfqpoint{5.323802in}{1.924566in}}%
\pgfpathlineto{\pgfqpoint{5.326564in}{1.921880in}}%
\pgfpathlineto{\pgfqpoint{5.329159in}{1.913438in}}%
\pgfpathlineto{\pgfqpoint{5.331973in}{1.911291in}}%
\pgfpathlineto{\pgfqpoint{5.334510in}{1.912376in}}%
\pgfpathlineto{\pgfqpoint{5.337353in}{1.920603in}}%
\pgfpathlineto{\pgfqpoint{5.339872in}{1.919623in}}%
\pgfpathlineto{\pgfqpoint{5.342549in}{1.915523in}}%
\pgfpathlineto{\pgfqpoint{5.345224in}{1.914562in}}%
\pgfpathlineto{\pgfqpoint{5.347905in}{1.912544in}}%
\pgfpathlineto{\pgfqpoint{5.350723in}{1.911638in}}%
\pgfpathlineto{\pgfqpoint{5.353262in}{1.911613in}}%
\pgfpathlineto{\pgfqpoint{5.356056in}{1.916303in}}%
\pgfpathlineto{\pgfqpoint{5.358612in}{1.914920in}}%
\pgfpathlineto{\pgfqpoint{5.361370in}{1.916124in}}%
\pgfpathlineto{\pgfqpoint{5.363966in}{1.917076in}}%
\pgfpathlineto{\pgfqpoint{5.366727in}{1.914117in}}%
\pgfpathlineto{\pgfqpoint{5.369335in}{1.915626in}}%
\pgfpathlineto{\pgfqpoint{5.372013in}{1.915963in}}%
\pgfpathlineto{\pgfqpoint{5.374692in}{1.914997in}}%
\pgfpathlineto{\pgfqpoint{5.377370in}{1.916929in}}%
\pgfpathlineto{\pgfqpoint{5.380048in}{1.912468in}}%
\pgfpathlineto{\pgfqpoint{5.382725in}{1.919176in}}%
\pgfpathlineto{\pgfqpoint{5.385550in}{1.916868in}}%
\pgfpathlineto{\pgfqpoint{5.388083in}{1.916426in}}%
\pgfpathlineto{\pgfqpoint{5.390900in}{1.911878in}}%
\pgfpathlineto{\pgfqpoint{5.393441in}{1.911172in}}%
\pgfpathlineto{\pgfqpoint{5.396219in}{1.912260in}}%
\pgfpathlineto{\pgfqpoint{5.398784in}{1.933715in}}%
\pgfpathlineto{\pgfqpoint{5.401576in}{1.939381in}}%
\pgfpathlineto{\pgfqpoint{5.404154in}{1.936816in}}%
\pgfpathlineto{\pgfqpoint{5.406832in}{1.932719in}}%
\pgfpathlineto{\pgfqpoint{5.409507in}{1.923328in}}%
\pgfpathlineto{\pgfqpoint{5.412190in}{1.917658in}}%
\pgfpathlineto{\pgfqpoint{5.414954in}{1.918388in}}%
\pgfpathlineto{\pgfqpoint{5.417547in}{1.921757in}}%
\pgfpathlineto{\pgfqpoint{5.420304in}{1.951545in}}%
\pgfpathlineto{\pgfqpoint{5.422897in}{1.944851in}}%
\pgfpathlineto{\pgfqpoint{5.425661in}{1.926825in}}%
\pgfpathlineto{\pgfqpoint{5.428259in}{1.924090in}}%
\pgfpathlineto{\pgfqpoint{5.431015in}{1.922283in}}%
\pgfpathlineto{\pgfqpoint{5.433616in}{1.921133in}}%
\pgfpathlineto{\pgfqpoint{5.436295in}{1.926185in}}%
\pgfpathlineto{\pgfqpoint{5.438974in}{1.922861in}}%
\pgfpathlineto{\pgfqpoint{5.441698in}{1.920414in}}%
\pgfpathlineto{\pgfqpoint{5.444328in}{1.923986in}}%
\pgfpathlineto{\pgfqpoint{5.447021in}{1.919261in}}%
\pgfpathlineto{\pgfqpoint{5.449769in}{1.919615in}}%
\pgfpathlineto{\pgfqpoint{5.452365in}{1.921507in}}%
\pgfpathlineto{\pgfqpoint{5.455168in}{1.920153in}}%
\pgfpathlineto{\pgfqpoint{5.457721in}{1.924697in}}%
\pgfpathlineto{\pgfqpoint{5.460489in}{1.926704in}}%
\pgfpathlineto{\pgfqpoint{5.463079in}{1.927322in}}%
\pgfpathlineto{\pgfqpoint{5.465888in}{1.922995in}}%
\pgfpathlineto{\pgfqpoint{5.468425in}{1.917502in}}%
\pgfpathlineto{\pgfqpoint{5.471113in}{1.917057in}}%
\pgfpathlineto{\pgfqpoint{5.473792in}{1.922379in}}%
\pgfpathlineto{\pgfqpoint{5.476458in}{1.922660in}}%
\pgfpathlineto{\pgfqpoint{5.479152in}{1.918338in}}%
\pgfpathlineto{\pgfqpoint{5.481825in}{1.917252in}}%
\pgfpathlineto{\pgfqpoint{5.484641in}{1.920295in}}%
\pgfpathlineto{\pgfqpoint{5.487176in}{1.922746in}}%
\pgfpathlineto{\pgfqpoint{5.490000in}{1.917971in}}%
\pgfpathlineto{\pgfqpoint{5.492541in}{1.921761in}}%
\pgfpathlineto{\pgfqpoint{5.495346in}{1.920599in}}%
\pgfpathlineto{\pgfqpoint{5.497898in}{1.918699in}}%
\pgfpathlineto{\pgfqpoint{5.500687in}{1.925116in}}%
\pgfpathlineto{\pgfqpoint{5.503255in}{1.925031in}}%
\pgfpathlineto{\pgfqpoint{5.505933in}{1.926795in}}%
\pgfpathlineto{\pgfqpoint{5.508612in}{1.926264in}}%
\pgfpathlineto{\pgfqpoint{5.511290in}{1.926119in}}%
\pgfpathlineto{\pgfqpoint{5.514080in}{1.930447in}}%
\pgfpathlineto{\pgfqpoint{5.516646in}{1.926421in}}%
\pgfpathlineto{\pgfqpoint{5.519433in}{1.922754in}}%
\pgfpathlineto{\pgfqpoint{5.522003in}{1.919430in}}%
\pgfpathlineto{\pgfqpoint{5.524756in}{1.923629in}}%
\pgfpathlineto{\pgfqpoint{5.527360in}{1.921714in}}%
\pgfpathlineto{\pgfqpoint{5.530148in}{1.924500in}}%
\pgfpathlineto{\pgfqpoint{5.532717in}{1.921294in}}%
\pgfpathlineto{\pgfqpoint{5.535395in}{1.921293in}}%
\pgfpathlineto{\pgfqpoint{5.538074in}{1.920412in}}%
\pgfpathlineto{\pgfqpoint{5.540750in}{1.922228in}}%
\pgfpathlineto{\pgfqpoint{5.543421in}{1.922040in}}%
\pgfpathlineto{\pgfqpoint{5.546110in}{1.922405in}}%
\pgfpathlineto{\pgfqpoint{5.548921in}{1.917637in}}%
\pgfpathlineto{\pgfqpoint{5.551457in}{1.920812in}}%
\pgfpathlineto{\pgfqpoint{5.554198in}{1.919349in}}%
\pgfpathlineto{\pgfqpoint{5.556822in}{1.922128in}}%
\pgfpathlineto{\pgfqpoint{5.559612in}{1.923019in}}%
\pgfpathlineto{\pgfqpoint{5.562180in}{1.919641in}}%
\pgfpathlineto{\pgfqpoint{5.564940in}{1.912829in}}%
\pgfpathlineto{\pgfqpoint{5.567536in}{1.914304in}}%
\pgfpathlineto{\pgfqpoint{5.570215in}{1.916973in}}%
\pgfpathlineto{\pgfqpoint{5.572893in}{1.914850in}}%
\pgfpathlineto{\pgfqpoint{5.575596in}{1.919161in}}%
\pgfpathlineto{\pgfqpoint{5.578342in}{1.916657in}}%
\pgfpathlineto{\pgfqpoint{5.580914in}{1.918575in}}%
\pgfpathlineto{\pgfqpoint{5.583709in}{1.919068in}}%
\pgfpathlineto{\pgfqpoint{5.586269in}{1.913559in}}%
\pgfpathlineto{\pgfqpoint{5.589040in}{1.917398in}}%
\pgfpathlineto{\pgfqpoint{5.591641in}{1.915141in}}%
\pgfpathlineto{\pgfqpoint{5.594368in}{1.912929in}}%
\pgfpathlineto{\pgfqpoint{5.596999in}{1.917916in}}%
\pgfpathlineto{\pgfqpoint{5.599674in}{1.923966in}}%
\pgfpathlineto{\pgfqpoint{5.602352in}{1.919303in}}%
\pgfpathlineto{\pgfqpoint{5.605073in}{1.917659in}}%
\pgfpathlineto{\pgfqpoint{5.607698in}{1.924579in}}%
\pgfpathlineto{\pgfqpoint{5.610389in}{1.926018in}}%
\pgfpathlineto{\pgfqpoint{5.613235in}{1.930156in}}%
\pgfpathlineto{\pgfqpoint{5.615743in}{1.932232in}}%
\pgfpathlineto{\pgfqpoint{5.618526in}{1.927775in}}%
\pgfpathlineto{\pgfqpoint{5.621102in}{1.921479in}}%
\pgfpathlineto{\pgfqpoint{5.623868in}{1.912084in}}%
\pgfpathlineto{\pgfqpoint{5.626460in}{1.922069in}}%
\pgfpathlineto{\pgfqpoint{5.629232in}{1.926552in}}%
\pgfpathlineto{\pgfqpoint{5.631815in}{1.921566in}}%
\pgfpathlineto{\pgfqpoint{5.634496in}{1.913446in}}%
\pgfpathlineto{\pgfqpoint{5.637172in}{1.912549in}}%
\pgfpathlineto{\pgfqpoint{5.639852in}{1.917942in}}%
\pgfpathlineto{\pgfqpoint{5.642518in}{1.923900in}}%
\pgfpathlineto{\pgfqpoint{5.645243in}{1.920135in}}%
\pgfpathlineto{\pgfqpoint{5.648008in}{1.911522in}}%
\pgfpathlineto{\pgfqpoint{5.650563in}{1.917918in}}%
\pgfpathlineto{\pgfqpoint{5.653376in}{1.914168in}}%
\pgfpathlineto{\pgfqpoint{5.655919in}{1.918937in}}%
\pgfpathlineto{\pgfqpoint{5.658723in}{1.918047in}}%
\pgfpathlineto{\pgfqpoint{5.661273in}{1.923513in}}%
\pgfpathlineto{\pgfqpoint{5.664099in}{1.924615in}}%
\pgfpathlineto{\pgfqpoint{5.666632in}{1.917157in}}%
\pgfpathlineto{\pgfqpoint{5.669313in}{1.916599in}}%
\pgfpathlineto{\pgfqpoint{5.671991in}{1.924656in}}%
\pgfpathlineto{\pgfqpoint{5.674667in}{1.928600in}}%
\pgfpathlineto{\pgfqpoint{5.677486in}{1.924092in}}%
\pgfpathlineto{\pgfqpoint{5.680027in}{1.921503in}}%
\pgfpathlineto{\pgfqpoint{5.682836in}{1.921777in}}%
\pgfpathlineto{\pgfqpoint{5.685385in}{1.925688in}}%
\pgfpathlineto{\pgfqpoint{5.688159in}{1.924901in}}%
\pgfpathlineto{\pgfqpoint{5.690730in}{1.921619in}}%
\pgfpathlineto{\pgfqpoint{5.693473in}{1.917392in}}%
\pgfpathlineto{\pgfqpoint{5.696101in}{1.915089in}}%
\pgfpathlineto{\pgfqpoint{5.698775in}{1.916206in}}%
\pgfpathlineto{\pgfqpoint{5.701453in}{1.917284in}}%
\pgfpathlineto{\pgfqpoint{5.704130in}{1.921102in}}%
\pgfpathlineto{\pgfqpoint{5.706800in}{1.920286in}}%
\pgfpathlineto{\pgfqpoint{5.709490in}{1.919816in}}%
\pgfpathlineto{\pgfqpoint{5.712291in}{1.919516in}}%
\pgfpathlineto{\pgfqpoint{5.714834in}{1.922460in}}%
\pgfpathlineto{\pgfqpoint{5.717671in}{1.918659in}}%
\pgfpathlineto{\pgfqpoint{5.720201in}{1.921830in}}%
\pgfpathlineto{\pgfqpoint{5.722950in}{1.922848in}}%
\pgfpathlineto{\pgfqpoint{5.725548in}{1.921864in}}%
\pgfpathlineto{\pgfqpoint{5.728339in}{1.923385in}}%
\pgfpathlineto{\pgfqpoint{5.730919in}{1.923632in}}%
\pgfpathlineto{\pgfqpoint{5.733594in}{1.917982in}}%
\pgfpathlineto{\pgfqpoint{5.736276in}{1.920903in}}%
\pgfpathlineto{\pgfqpoint{5.738974in}{1.927241in}}%
\pgfpathlineto{\pgfqpoint{5.741745in}{1.927320in}}%
\pgfpathlineto{\pgfqpoint{5.744310in}{1.923048in}}%
\pgfpathlineto{\pgfqpoint{5.744310in}{0.413320in}}%
\pgfpathlineto{\pgfqpoint{5.744310in}{0.413320in}}%
\pgfpathlineto{\pgfqpoint{5.741745in}{0.413320in}}%
\pgfpathlineto{\pgfqpoint{5.738974in}{0.413320in}}%
\pgfpathlineto{\pgfqpoint{5.736276in}{0.413320in}}%
\pgfpathlineto{\pgfqpoint{5.733594in}{0.413320in}}%
\pgfpathlineto{\pgfqpoint{5.730919in}{0.413320in}}%
\pgfpathlineto{\pgfqpoint{5.728339in}{0.413320in}}%
\pgfpathlineto{\pgfqpoint{5.725548in}{0.413320in}}%
\pgfpathlineto{\pgfqpoint{5.722950in}{0.413320in}}%
\pgfpathlineto{\pgfqpoint{5.720201in}{0.413320in}}%
\pgfpathlineto{\pgfqpoint{5.717671in}{0.413320in}}%
\pgfpathlineto{\pgfqpoint{5.714834in}{0.413320in}}%
\pgfpathlineto{\pgfqpoint{5.712291in}{0.413320in}}%
\pgfpathlineto{\pgfqpoint{5.709490in}{0.413320in}}%
\pgfpathlineto{\pgfqpoint{5.706800in}{0.413320in}}%
\pgfpathlineto{\pgfqpoint{5.704130in}{0.413320in}}%
\pgfpathlineto{\pgfqpoint{5.701453in}{0.413320in}}%
\pgfpathlineto{\pgfqpoint{5.698775in}{0.413320in}}%
\pgfpathlineto{\pgfqpoint{5.696101in}{0.413320in}}%
\pgfpathlineto{\pgfqpoint{5.693473in}{0.413320in}}%
\pgfpathlineto{\pgfqpoint{5.690730in}{0.413320in}}%
\pgfpathlineto{\pgfqpoint{5.688159in}{0.413320in}}%
\pgfpathlineto{\pgfqpoint{5.685385in}{0.413320in}}%
\pgfpathlineto{\pgfqpoint{5.682836in}{0.413320in}}%
\pgfpathlineto{\pgfqpoint{5.680027in}{0.413320in}}%
\pgfpathlineto{\pgfqpoint{5.677486in}{0.413320in}}%
\pgfpathlineto{\pgfqpoint{5.674667in}{0.413320in}}%
\pgfpathlineto{\pgfqpoint{5.671991in}{0.413320in}}%
\pgfpathlineto{\pgfqpoint{5.669313in}{0.413320in}}%
\pgfpathlineto{\pgfqpoint{5.666632in}{0.413320in}}%
\pgfpathlineto{\pgfqpoint{5.664099in}{0.413320in}}%
\pgfpathlineto{\pgfqpoint{5.661273in}{0.413320in}}%
\pgfpathlineto{\pgfqpoint{5.658723in}{0.413320in}}%
\pgfpathlineto{\pgfqpoint{5.655919in}{0.413320in}}%
\pgfpathlineto{\pgfqpoint{5.653376in}{0.413320in}}%
\pgfpathlineto{\pgfqpoint{5.650563in}{0.413320in}}%
\pgfpathlineto{\pgfqpoint{5.648008in}{0.413320in}}%
\pgfpathlineto{\pgfqpoint{5.645243in}{0.413320in}}%
\pgfpathlineto{\pgfqpoint{5.642518in}{0.413320in}}%
\pgfpathlineto{\pgfqpoint{5.639852in}{0.413320in}}%
\pgfpathlineto{\pgfqpoint{5.637172in}{0.413320in}}%
\pgfpathlineto{\pgfqpoint{5.634496in}{0.413320in}}%
\pgfpathlineto{\pgfqpoint{5.631815in}{0.413320in}}%
\pgfpathlineto{\pgfqpoint{5.629232in}{0.413320in}}%
\pgfpathlineto{\pgfqpoint{5.626460in}{0.413320in}}%
\pgfpathlineto{\pgfqpoint{5.623868in}{0.413320in}}%
\pgfpathlineto{\pgfqpoint{5.621102in}{0.413320in}}%
\pgfpathlineto{\pgfqpoint{5.618526in}{0.413320in}}%
\pgfpathlineto{\pgfqpoint{5.615743in}{0.413320in}}%
\pgfpathlineto{\pgfqpoint{5.613235in}{0.413320in}}%
\pgfpathlineto{\pgfqpoint{5.610389in}{0.413320in}}%
\pgfpathlineto{\pgfqpoint{5.607698in}{0.413320in}}%
\pgfpathlineto{\pgfqpoint{5.605073in}{0.413320in}}%
\pgfpathlineto{\pgfqpoint{5.602352in}{0.413320in}}%
\pgfpathlineto{\pgfqpoint{5.599674in}{0.413320in}}%
\pgfpathlineto{\pgfqpoint{5.596999in}{0.413320in}}%
\pgfpathlineto{\pgfqpoint{5.594368in}{0.413320in}}%
\pgfpathlineto{\pgfqpoint{5.591641in}{0.413320in}}%
\pgfpathlineto{\pgfqpoint{5.589040in}{0.413320in}}%
\pgfpathlineto{\pgfqpoint{5.586269in}{0.413320in}}%
\pgfpathlineto{\pgfqpoint{5.583709in}{0.413320in}}%
\pgfpathlineto{\pgfqpoint{5.580914in}{0.413320in}}%
\pgfpathlineto{\pgfqpoint{5.578342in}{0.413320in}}%
\pgfpathlineto{\pgfqpoint{5.575596in}{0.413320in}}%
\pgfpathlineto{\pgfqpoint{5.572893in}{0.413320in}}%
\pgfpathlineto{\pgfqpoint{5.570215in}{0.413320in}}%
\pgfpathlineto{\pgfqpoint{5.567536in}{0.413320in}}%
\pgfpathlineto{\pgfqpoint{5.564940in}{0.413320in}}%
\pgfpathlineto{\pgfqpoint{5.562180in}{0.413320in}}%
\pgfpathlineto{\pgfqpoint{5.559612in}{0.413320in}}%
\pgfpathlineto{\pgfqpoint{5.556822in}{0.413320in}}%
\pgfpathlineto{\pgfqpoint{5.554198in}{0.413320in}}%
\pgfpathlineto{\pgfqpoint{5.551457in}{0.413320in}}%
\pgfpathlineto{\pgfqpoint{5.548921in}{0.413320in}}%
\pgfpathlineto{\pgfqpoint{5.546110in}{0.413320in}}%
\pgfpathlineto{\pgfqpoint{5.543421in}{0.413320in}}%
\pgfpathlineto{\pgfqpoint{5.540750in}{0.413320in}}%
\pgfpathlineto{\pgfqpoint{5.538074in}{0.413320in}}%
\pgfpathlineto{\pgfqpoint{5.535395in}{0.413320in}}%
\pgfpathlineto{\pgfqpoint{5.532717in}{0.413320in}}%
\pgfpathlineto{\pgfqpoint{5.530148in}{0.413320in}}%
\pgfpathlineto{\pgfqpoint{5.527360in}{0.413320in}}%
\pgfpathlineto{\pgfqpoint{5.524756in}{0.413320in}}%
\pgfpathlineto{\pgfqpoint{5.522003in}{0.413320in}}%
\pgfpathlineto{\pgfqpoint{5.519433in}{0.413320in}}%
\pgfpathlineto{\pgfqpoint{5.516646in}{0.413320in}}%
\pgfpathlineto{\pgfqpoint{5.514080in}{0.413320in}}%
\pgfpathlineto{\pgfqpoint{5.511290in}{0.413320in}}%
\pgfpathlineto{\pgfqpoint{5.508612in}{0.413320in}}%
\pgfpathlineto{\pgfqpoint{5.505933in}{0.413320in}}%
\pgfpathlineto{\pgfqpoint{5.503255in}{0.413320in}}%
\pgfpathlineto{\pgfqpoint{5.500687in}{0.413320in}}%
\pgfpathlineto{\pgfqpoint{5.497898in}{0.413320in}}%
\pgfpathlineto{\pgfqpoint{5.495346in}{0.413320in}}%
\pgfpathlineto{\pgfqpoint{5.492541in}{0.413320in}}%
\pgfpathlineto{\pgfqpoint{5.490000in}{0.413320in}}%
\pgfpathlineto{\pgfqpoint{5.487176in}{0.413320in}}%
\pgfpathlineto{\pgfqpoint{5.484641in}{0.413320in}}%
\pgfpathlineto{\pgfqpoint{5.481825in}{0.413320in}}%
\pgfpathlineto{\pgfqpoint{5.479152in}{0.413320in}}%
\pgfpathlineto{\pgfqpoint{5.476458in}{0.413320in}}%
\pgfpathlineto{\pgfqpoint{5.473792in}{0.413320in}}%
\pgfpathlineto{\pgfqpoint{5.471113in}{0.413320in}}%
\pgfpathlineto{\pgfqpoint{5.468425in}{0.413320in}}%
\pgfpathlineto{\pgfqpoint{5.465888in}{0.413320in}}%
\pgfpathlineto{\pgfqpoint{5.463079in}{0.413320in}}%
\pgfpathlineto{\pgfqpoint{5.460489in}{0.413320in}}%
\pgfpathlineto{\pgfqpoint{5.457721in}{0.413320in}}%
\pgfpathlineto{\pgfqpoint{5.455168in}{0.413320in}}%
\pgfpathlineto{\pgfqpoint{5.452365in}{0.413320in}}%
\pgfpathlineto{\pgfqpoint{5.449769in}{0.413320in}}%
\pgfpathlineto{\pgfqpoint{5.447021in}{0.413320in}}%
\pgfpathlineto{\pgfqpoint{5.444328in}{0.413320in}}%
\pgfpathlineto{\pgfqpoint{5.441698in}{0.413320in}}%
\pgfpathlineto{\pgfqpoint{5.438974in}{0.413320in}}%
\pgfpathlineto{\pgfqpoint{5.436295in}{0.413320in}}%
\pgfpathlineto{\pgfqpoint{5.433616in}{0.413320in}}%
\pgfpathlineto{\pgfqpoint{5.431015in}{0.413320in}}%
\pgfpathlineto{\pgfqpoint{5.428259in}{0.413320in}}%
\pgfpathlineto{\pgfqpoint{5.425661in}{0.413320in}}%
\pgfpathlineto{\pgfqpoint{5.422897in}{0.413320in}}%
\pgfpathlineto{\pgfqpoint{5.420304in}{0.413320in}}%
\pgfpathlineto{\pgfqpoint{5.417547in}{0.413320in}}%
\pgfpathlineto{\pgfqpoint{5.414954in}{0.413320in}}%
\pgfpathlineto{\pgfqpoint{5.412190in}{0.413320in}}%
\pgfpathlineto{\pgfqpoint{5.409507in}{0.413320in}}%
\pgfpathlineto{\pgfqpoint{5.406832in}{0.413320in}}%
\pgfpathlineto{\pgfqpoint{5.404154in}{0.413320in}}%
\pgfpathlineto{\pgfqpoint{5.401576in}{0.413320in}}%
\pgfpathlineto{\pgfqpoint{5.398784in}{0.413320in}}%
\pgfpathlineto{\pgfqpoint{5.396219in}{0.413320in}}%
\pgfpathlineto{\pgfqpoint{5.393441in}{0.413320in}}%
\pgfpathlineto{\pgfqpoint{5.390900in}{0.413320in}}%
\pgfpathlineto{\pgfqpoint{5.388083in}{0.413320in}}%
\pgfpathlineto{\pgfqpoint{5.385550in}{0.413320in}}%
\pgfpathlineto{\pgfqpoint{5.382725in}{0.413320in}}%
\pgfpathlineto{\pgfqpoint{5.380048in}{0.413320in}}%
\pgfpathlineto{\pgfqpoint{5.377370in}{0.413320in}}%
\pgfpathlineto{\pgfqpoint{5.374692in}{0.413320in}}%
\pgfpathlineto{\pgfqpoint{5.372013in}{0.413320in}}%
\pgfpathlineto{\pgfqpoint{5.369335in}{0.413320in}}%
\pgfpathlineto{\pgfqpoint{5.366727in}{0.413320in}}%
\pgfpathlineto{\pgfqpoint{5.363966in}{0.413320in}}%
\pgfpathlineto{\pgfqpoint{5.361370in}{0.413320in}}%
\pgfpathlineto{\pgfqpoint{5.358612in}{0.413320in}}%
\pgfpathlineto{\pgfqpoint{5.356056in}{0.413320in}}%
\pgfpathlineto{\pgfqpoint{5.353262in}{0.413320in}}%
\pgfpathlineto{\pgfqpoint{5.350723in}{0.413320in}}%
\pgfpathlineto{\pgfqpoint{5.347905in}{0.413320in}}%
\pgfpathlineto{\pgfqpoint{5.345224in}{0.413320in}}%
\pgfpathlineto{\pgfqpoint{5.342549in}{0.413320in}}%
\pgfpathlineto{\pgfqpoint{5.339872in}{0.413320in}}%
\pgfpathlineto{\pgfqpoint{5.337353in}{0.413320in}}%
\pgfpathlineto{\pgfqpoint{5.334510in}{0.413320in}}%
\pgfpathlineto{\pgfqpoint{5.331973in}{0.413320in}}%
\pgfpathlineto{\pgfqpoint{5.329159in}{0.413320in}}%
\pgfpathlineto{\pgfqpoint{5.326564in}{0.413320in}}%
\pgfpathlineto{\pgfqpoint{5.323802in}{0.413320in}}%
\pgfpathlineto{\pgfqpoint{5.321256in}{0.413320in}}%
\pgfpathlineto{\pgfqpoint{5.318430in}{0.413320in}}%
\pgfpathlineto{\pgfqpoint{5.315754in}{0.413320in}}%
\pgfpathlineto{\pgfqpoint{5.313089in}{0.413320in}}%
\pgfpathlineto{\pgfqpoint{5.310411in}{0.413320in}}%
\pgfpathlineto{\pgfqpoint{5.307731in}{0.413320in}}%
\pgfpathlineto{\pgfqpoint{5.305054in}{0.413320in}}%
\pgfpathlineto{\pgfqpoint{5.302443in}{0.413320in}}%
\pgfpathlineto{\pgfqpoint{5.299696in}{0.413320in}}%
\pgfpathlineto{\pgfqpoint{5.297140in}{0.413320in}}%
\pgfpathlineto{\pgfqpoint{5.294339in}{0.413320in}}%
\pgfpathlineto{\pgfqpoint{5.291794in}{0.413320in}}%
\pgfpathlineto{\pgfqpoint{5.288984in}{0.413320in}}%
\pgfpathlineto{\pgfqpoint{5.286436in}{0.413320in}}%
\pgfpathlineto{\pgfqpoint{5.283631in}{0.413320in}}%
\pgfpathlineto{\pgfqpoint{5.280947in}{0.413320in}}%
\pgfpathlineto{\pgfqpoint{5.278322in}{0.413320in}}%
\pgfpathlineto{\pgfqpoint{5.275589in}{0.413320in}}%
\pgfpathlineto{\pgfqpoint{5.272913in}{0.413320in}}%
\pgfpathlineto{\pgfqpoint{5.270238in}{0.413320in}}%
\pgfpathlineto{\pgfqpoint{5.267691in}{0.413320in}}%
\pgfpathlineto{\pgfqpoint{5.264876in}{0.413320in}}%
\pgfpathlineto{\pgfqpoint{5.262264in}{0.413320in}}%
\pgfpathlineto{\pgfqpoint{5.259511in}{0.413320in}}%
\pgfpathlineto{\pgfqpoint{5.256973in}{0.413320in}}%
\pgfpathlineto{\pgfqpoint{5.254236in}{0.413320in}}%
\pgfpathlineto{\pgfqpoint{5.251590in}{0.413320in}}%
\pgfpathlineto{\pgfqpoint{5.248816in}{0.413320in}}%
\pgfpathlineto{\pgfqpoint{5.246130in}{0.413320in}}%
\pgfpathlineto{\pgfqpoint{5.243445in}{0.413320in}}%
\pgfpathlineto{\pgfqpoint{5.240777in}{0.413320in}}%
\pgfpathlineto{\pgfqpoint{5.238173in}{0.413320in}}%
\pgfpathlineto{\pgfqpoint{5.235409in}{0.413320in}}%
\pgfpathlineto{\pgfqpoint{5.232855in}{0.413320in}}%
\pgfpathlineto{\pgfqpoint{5.230059in}{0.413320in}}%
\pgfpathlineto{\pgfqpoint{5.227470in}{0.413320in}}%
\pgfpathlineto{\pgfqpoint{5.224695in}{0.413320in}}%
\pgfpathlineto{\pgfqpoint{5.222151in}{0.413320in}}%
\pgfpathlineto{\pgfqpoint{5.219345in}{0.413320in}}%
\pgfpathlineto{\pgfqpoint{5.216667in}{0.413320in}}%
\pgfpathlineto{\pgfqpoint{5.214027in}{0.413320in}}%
\pgfpathlineto{\pgfqpoint{5.211299in}{0.413320in}}%
\pgfpathlineto{\pgfqpoint{5.208630in}{0.413320in}}%
\pgfpathlineto{\pgfqpoint{5.205952in}{0.413320in}}%
\pgfpathlineto{\pgfqpoint{5.203388in}{0.413320in}}%
\pgfpathlineto{\pgfqpoint{5.200594in}{0.413320in}}%
\pgfpathlineto{\pgfqpoint{5.198008in}{0.413320in}}%
\pgfpathlineto{\pgfqpoint{5.195239in}{0.413320in}}%
\pgfpathlineto{\pgfqpoint{5.192680in}{0.413320in}}%
\pgfpathlineto{\pgfqpoint{5.189880in}{0.413320in}}%
\pgfpathlineto{\pgfqpoint{5.187294in}{0.413320in}}%
\pgfpathlineto{\pgfqpoint{5.184522in}{0.413320in}}%
\pgfpathlineto{\pgfqpoint{5.181848in}{0.413320in}}%
\pgfpathlineto{\pgfqpoint{5.179188in}{0.413320in}}%
\pgfpathlineto{\pgfqpoint{5.176477in}{0.413320in}}%
\pgfpathlineto{\pgfqpoint{5.173925in}{0.413320in}}%
\pgfpathlineto{\pgfqpoint{5.171133in}{0.413320in}}%
\pgfpathlineto{\pgfqpoint{5.168591in}{0.413320in}}%
\pgfpathlineto{\pgfqpoint{5.165775in}{0.413320in}}%
\pgfpathlineto{\pgfqpoint{5.163243in}{0.413320in}}%
\pgfpathlineto{\pgfqpoint{5.160420in}{0.413320in}}%
\pgfpathlineto{\pgfqpoint{5.157815in}{0.413320in}}%
\pgfpathlineto{\pgfqpoint{5.155059in}{0.413320in}}%
\pgfpathlineto{\pgfqpoint{5.152382in}{0.413320in}}%
\pgfpathlineto{\pgfqpoint{5.149734in}{0.413320in}}%
\pgfpathlineto{\pgfqpoint{5.147029in}{0.413320in}}%
\pgfpathlineto{\pgfqpoint{5.144349in}{0.413320in}}%
\pgfpathlineto{\pgfqpoint{5.141660in}{0.413320in}}%
\pgfpathlineto{\pgfqpoint{5.139072in}{0.413320in}}%
\pgfpathlineto{\pgfqpoint{5.136311in}{0.413320in}}%
\pgfpathlineto{\pgfqpoint{5.133716in}{0.413320in}}%
\pgfpathlineto{\pgfqpoint{5.130953in}{0.413320in}}%
\pgfpathlineto{\pgfqpoint{5.128421in}{0.413320in}}%
\pgfpathlineto{\pgfqpoint{5.125599in}{0.413320in}}%
\pgfpathlineto{\pgfqpoint{5.123042in}{0.413320in}}%
\pgfpathlineto{\pgfqpoint{5.120243in}{0.413320in}}%
\pgfpathlineto{\pgfqpoint{5.117550in}{0.413320in}}%
\pgfpathlineto{\pgfqpoint{5.114887in}{0.413320in}}%
\pgfpathlineto{\pgfqpoint{5.112209in}{0.413320in}}%
\pgfpathlineto{\pgfqpoint{5.109530in}{0.413320in}}%
\pgfpathlineto{\pgfqpoint{5.106842in}{0.413320in}}%
\pgfpathlineto{\pgfqpoint{5.104312in}{0.413320in}}%
\pgfpathlineto{\pgfqpoint{5.101496in}{0.413320in}}%
\pgfpathlineto{\pgfqpoint{5.098948in}{0.413320in}}%
\pgfpathlineto{\pgfqpoint{5.096142in}{0.413320in}}%
\pgfpathlineto{\pgfqpoint{5.093579in}{0.413320in}}%
\pgfpathlineto{\pgfqpoint{5.090788in}{0.413320in}}%
\pgfpathlineto{\pgfqpoint{5.088103in}{0.413320in}}%
\pgfpathlineto{\pgfqpoint{5.085426in}{0.413320in}}%
\pgfpathlineto{\pgfqpoint{5.082746in}{0.413320in}}%
\pgfpathlineto{\pgfqpoint{5.080067in}{0.413320in}}%
\pgfpathlineto{\pgfqpoint{5.077390in}{0.413320in}}%
\pgfpathlineto{\pgfqpoint{5.074851in}{0.413320in}}%
\pgfpathlineto{\pgfqpoint{5.072030in}{0.413320in}}%
\pgfpathlineto{\pgfqpoint{5.069463in}{0.413320in}}%
\pgfpathlineto{\pgfqpoint{5.066677in}{0.413320in}}%
\pgfpathlineto{\pgfqpoint{5.064144in}{0.413320in}}%
\pgfpathlineto{\pgfqpoint{5.061315in}{0.413320in}}%
\pgfpathlineto{\pgfqpoint{5.058711in}{0.413320in}}%
\pgfpathlineto{\pgfqpoint{5.055952in}{0.413320in}}%
\pgfpathlineto{\pgfqpoint{5.053284in}{0.413320in}}%
\pgfpathlineto{\pgfqpoint{5.050606in}{0.413320in}}%
\pgfpathlineto{\pgfqpoint{5.047924in}{0.413320in}}%
\pgfpathlineto{\pgfqpoint{5.045249in}{0.413320in}}%
\pgfpathlineto{\pgfqpoint{5.042572in}{0.413320in}}%
\pgfpathlineto{\pgfqpoint{5.039962in}{0.413320in}}%
\pgfpathlineto{\pgfqpoint{5.037214in}{0.413320in}}%
\pgfpathlineto{\pgfqpoint{5.034649in}{0.413320in}}%
\pgfpathlineto{\pgfqpoint{5.031849in}{0.413320in}}%
\pgfpathlineto{\pgfqpoint{5.029275in}{0.413320in}}%
\pgfpathlineto{\pgfqpoint{5.026501in}{0.413320in}}%
\pgfpathlineto{\pgfqpoint{5.023927in}{0.413320in}}%
\pgfpathlineto{\pgfqpoint{5.021147in}{0.413320in}}%
\pgfpathlineto{\pgfqpoint{5.018466in}{0.413320in}}%
\pgfpathlineto{\pgfqpoint{5.015820in}{0.413320in}}%
\pgfpathlineto{\pgfqpoint{5.013104in}{0.413320in}}%
\pgfpathlineto{\pgfqpoint{5.010562in}{0.413320in}}%
\pgfpathlineto{\pgfqpoint{5.007751in}{0.413320in}}%
\pgfpathlineto{\pgfqpoint{5.005178in}{0.413320in}}%
\pgfpathlineto{\pgfqpoint{5.002384in}{0.413320in}}%
\pgfpathlineto{\pgfqpoint{4.999780in}{0.413320in}}%
\pgfpathlineto{\pgfqpoint{4.997028in}{0.413320in}}%
\pgfpathlineto{\pgfqpoint{4.994390in}{0.413320in}}%
\pgfpathlineto{\pgfqpoint{4.991683in}{0.413320in}}%
\pgfpathlineto{\pgfqpoint{4.989001in}{0.413320in}}%
\pgfpathlineto{\pgfqpoint{4.986325in}{0.413320in}}%
\pgfpathlineto{\pgfqpoint{4.983637in}{0.413320in}}%
\pgfpathlineto{\pgfqpoint{4.980967in}{0.413320in}}%
\pgfpathlineto{\pgfqpoint{4.978287in}{0.413320in}}%
\pgfpathlineto{\pgfqpoint{4.975703in}{0.413320in}}%
\pgfpathlineto{\pgfqpoint{4.972933in}{0.413320in}}%
\pgfpathlineto{\pgfqpoint{4.970314in}{0.413320in}}%
\pgfpathlineto{\pgfqpoint{4.967575in}{0.413320in}}%
\pgfpathlineto{\pgfqpoint{4.965002in}{0.413320in}}%
\pgfpathlineto{\pgfqpoint{4.962219in}{0.413320in}}%
\pgfpathlineto{\pgfqpoint{4.959689in}{0.413320in}}%
\pgfpathlineto{\pgfqpoint{4.956862in}{0.413320in}}%
\pgfpathlineto{\pgfqpoint{4.954182in}{0.413320in}}%
\pgfpathlineto{\pgfqpoint{4.951504in}{0.413320in}}%
\pgfpathlineto{\pgfqpoint{4.948827in}{0.413320in}}%
\pgfpathlineto{\pgfqpoint{4.946151in}{0.413320in}}%
\pgfpathlineto{\pgfqpoint{4.943466in}{0.413320in}}%
\pgfpathlineto{\pgfqpoint{4.940881in}{0.413320in}}%
\pgfpathlineto{\pgfqpoint{4.938112in}{0.413320in}}%
\pgfpathlineto{\pgfqpoint{4.935515in}{0.413320in}}%
\pgfpathlineto{\pgfqpoint{4.932742in}{0.413320in}}%
\pgfpathlineto{\pgfqpoint{4.930170in}{0.413320in}}%
\pgfpathlineto{\pgfqpoint{4.927400in}{0.413320in}}%
\pgfpathlineto{\pgfqpoint{4.924708in}{0.413320in}}%
\pgfpathlineto{\pgfqpoint{4.922041in}{0.413320in}}%
\pgfpathlineto{\pgfqpoint{4.919352in}{0.413320in}}%
\pgfpathlineto{\pgfqpoint{4.916681in}{0.413320in}}%
\pgfpathlineto{\pgfqpoint{4.914009in}{0.413320in}}%
\pgfpathlineto{\pgfqpoint{4.911435in}{0.413320in}}%
\pgfpathlineto{\pgfqpoint{4.908648in}{0.413320in}}%
\pgfpathlineto{\pgfqpoint{4.906096in}{0.413320in}}%
\pgfpathlineto{\pgfqpoint{4.903295in}{0.413320in}}%
\pgfpathlineto{\pgfqpoint{4.900712in}{0.413320in}}%
\pgfpathlineto{\pgfqpoint{4.897938in}{0.413320in}}%
\pgfpathlineto{\pgfqpoint{4.895399in}{0.413320in}}%
\pgfpathlineto{\pgfqpoint{4.892611in}{0.413320in}}%
\pgfpathlineto{\pgfqpoint{4.889902in}{0.413320in}}%
\pgfpathlineto{\pgfqpoint{4.887211in}{0.413320in}}%
\pgfpathlineto{\pgfqpoint{4.884540in}{0.413320in}}%
\pgfpathlineto{\pgfqpoint{4.881864in}{0.413320in}}%
\pgfpathlineto{\pgfqpoint{4.879180in}{0.413320in}}%
\pgfpathlineto{\pgfqpoint{4.876636in}{0.413320in}}%
\pgfpathlineto{\pgfqpoint{4.873832in}{0.413320in}}%
\pgfpathlineto{\pgfqpoint{4.871209in}{0.413320in}}%
\pgfpathlineto{\pgfqpoint{4.868474in}{0.413320in}}%
\pgfpathlineto{\pgfqpoint{4.865910in}{0.413320in}}%
\pgfpathlineto{\pgfqpoint{4.863116in}{0.413320in}}%
\pgfpathlineto{\pgfqpoint{4.860544in}{0.413320in}}%
\pgfpathlineto{\pgfqpoint{4.857807in}{0.413320in}}%
\pgfpathlineto{\pgfqpoint{4.855070in}{0.413320in}}%
\pgfpathlineto{\pgfqpoint{4.852404in}{0.413320in}}%
\pgfpathlineto{\pgfqpoint{4.849715in}{0.413320in}}%
\pgfpathlineto{\pgfqpoint{4.847127in}{0.413320in}}%
\pgfpathlineto{\pgfqpoint{4.844361in}{0.413320in}}%
\pgfpathlineto{\pgfqpoint{4.842380in}{0.413320in}}%
\pgfpathlineto{\pgfqpoint{4.839922in}{0.413320in}}%
\pgfpathlineto{\pgfqpoint{4.837992in}{0.413320in}}%
\pgfpathlineto{\pgfqpoint{4.833657in}{0.413320in}}%
\pgfpathlineto{\pgfqpoint{4.831045in}{0.413320in}}%
\pgfpathlineto{\pgfqpoint{4.828291in}{0.413320in}}%
\pgfpathlineto{\pgfqpoint{4.825619in}{0.413320in}}%
\pgfpathlineto{\pgfqpoint{4.822945in}{0.413320in}}%
\pgfpathlineto{\pgfqpoint{4.820265in}{0.413320in}}%
\pgfpathlineto{\pgfqpoint{4.817587in}{0.413320in}}%
\pgfpathlineto{\pgfqpoint{4.814907in}{0.413320in}}%
\pgfpathlineto{\pgfqpoint{4.812377in}{0.413320in}}%
\pgfpathlineto{\pgfqpoint{4.809538in}{0.413320in}}%
\pgfpathlineto{\pgfqpoint{4.807017in}{0.413320in}}%
\pgfpathlineto{\pgfqpoint{4.804193in}{0.413320in}}%
\pgfpathlineto{\pgfqpoint{4.801586in}{0.413320in}}%
\pgfpathlineto{\pgfqpoint{4.798830in}{0.413320in}}%
\pgfpathlineto{\pgfqpoint{4.796274in}{0.413320in}}%
\pgfpathlineto{\pgfqpoint{4.793512in}{0.413320in}}%
\pgfpathlineto{\pgfqpoint{4.790798in}{0.413320in}}%
\pgfpathlineto{\pgfqpoint{4.788116in}{0.413320in}}%
\pgfpathlineto{\pgfqpoint{4.785445in}{0.413320in}}%
\pgfpathlineto{\pgfqpoint{4.782872in}{0.413320in}}%
\pgfpathlineto{\pgfqpoint{4.780083in}{0.413320in}}%
\pgfpathlineto{\pgfqpoint{4.777535in}{0.413320in}}%
\pgfpathlineto{\pgfqpoint{4.774732in}{0.413320in}}%
\pgfpathlineto{\pgfqpoint{4.772198in}{0.413320in}}%
\pgfpathlineto{\pgfqpoint{4.769367in}{0.413320in}}%
\pgfpathlineto{\pgfqpoint{4.766783in}{0.413320in}}%
\pgfpathlineto{\pgfqpoint{4.764018in}{0.413320in}}%
\pgfpathlineto{\pgfqpoint{4.761337in}{0.413320in}}%
\pgfpathlineto{\pgfqpoint{4.758653in}{0.413320in}}%
\pgfpathlineto{\pgfqpoint{4.755983in}{0.413320in}}%
\pgfpathlineto{\pgfqpoint{4.753298in}{0.413320in}}%
\pgfpathlineto{\pgfqpoint{4.750627in}{0.413320in}}%
\pgfpathlineto{\pgfqpoint{4.748081in}{0.413320in}}%
\pgfpathlineto{\pgfqpoint{4.745256in}{0.413320in}}%
\pgfpathlineto{\pgfqpoint{4.742696in}{0.413320in}}%
\pgfpathlineto{\pgfqpoint{4.739912in}{0.413320in}}%
\pgfpathlineto{\pgfqpoint{4.737348in}{0.413320in}}%
\pgfpathlineto{\pgfqpoint{4.734552in}{0.413320in}}%
\pgfpathlineto{\pgfqpoint{4.731901in}{0.413320in}}%
\pgfpathlineto{\pgfqpoint{4.729233in}{0.413320in}}%
\pgfpathlineto{\pgfqpoint{4.726508in}{0.413320in}}%
\pgfpathlineto{\pgfqpoint{4.723873in}{0.413320in}}%
\pgfpathlineto{\pgfqpoint{4.721160in}{0.413320in}}%
\pgfpathlineto{\pgfqpoint{4.718486in}{0.413320in}}%
\pgfpathlineto{\pgfqpoint{4.715806in}{0.413320in}}%
\pgfpathlineto{\pgfqpoint{4.713275in}{0.413320in}}%
\pgfpathlineto{\pgfqpoint{4.710437in}{0.413320in}}%
\pgfpathlineto{\pgfqpoint{4.707824in}{0.413320in}}%
\pgfpathlineto{\pgfqpoint{4.705094in}{0.413320in}}%
\pgfpathlineto{\pgfqpoint{4.702517in}{0.413320in}}%
\pgfpathlineto{\pgfqpoint{4.699734in}{0.413320in}}%
\pgfpathlineto{\pgfqpoint{4.697170in}{0.413320in}}%
\pgfpathlineto{\pgfqpoint{4.694381in}{0.413320in}}%
\pgfpathlineto{\pgfqpoint{4.691694in}{0.413320in}}%
\pgfpathlineto{\pgfqpoint{4.689051in}{0.413320in}}%
\pgfpathlineto{\pgfqpoint{4.686337in}{0.413320in}}%
\pgfpathlineto{\pgfqpoint{4.683799in}{0.413320in}}%
\pgfpathlineto{\pgfqpoint{4.680988in}{0.413320in}}%
\pgfpathlineto{\pgfqpoint{4.678448in}{0.413320in}}%
\pgfpathlineto{\pgfqpoint{4.675619in}{0.413320in}}%
\pgfpathlineto{\pgfqpoint{4.673068in}{0.413320in}}%
\pgfpathlineto{\pgfqpoint{4.670261in}{0.413320in}}%
\pgfpathlineto{\pgfqpoint{4.667764in}{0.413320in}}%
\pgfpathlineto{\pgfqpoint{4.664923in}{0.413320in}}%
\pgfpathlineto{\pgfqpoint{4.662237in}{0.413320in}}%
\pgfpathlineto{\pgfqpoint{4.659590in}{0.413320in}}%
\pgfpathlineto{\pgfqpoint{4.656873in}{0.413320in}}%
\pgfpathlineto{\pgfqpoint{4.654203in}{0.413320in}}%
\pgfpathlineto{\pgfqpoint{4.651524in}{0.413320in}}%
\pgfpathlineto{\pgfqpoint{4.648922in}{0.413320in}}%
\pgfpathlineto{\pgfqpoint{4.646169in}{0.413320in}}%
\pgfpathlineto{\pgfqpoint{4.643628in}{0.413320in}}%
\pgfpathlineto{\pgfqpoint{4.640809in}{0.413320in}}%
\pgfpathlineto{\pgfqpoint{4.638204in}{0.413320in}}%
\pgfpathlineto{\pgfqpoint{4.635445in}{0.413320in}}%
\pgfpathlineto{\pgfqpoint{4.632902in}{0.413320in}}%
\pgfpathlineto{\pgfqpoint{4.630096in}{0.413320in}}%
\pgfpathlineto{\pgfqpoint{4.627411in}{0.413320in}}%
\pgfpathlineto{\pgfqpoint{4.624741in}{0.413320in}}%
\pgfpathlineto{\pgfqpoint{4.622056in}{0.413320in}}%
\pgfpathlineto{\pgfqpoint{4.619529in}{0.413320in}}%
\pgfpathlineto{\pgfqpoint{4.616702in}{0.413320in}}%
\pgfpathlineto{\pgfqpoint{4.614134in}{0.413320in}}%
\pgfpathlineto{\pgfqpoint{4.611350in}{0.413320in}}%
\pgfpathlineto{\pgfqpoint{4.608808in}{0.413320in}}%
\pgfpathlineto{\pgfqpoint{4.605990in}{0.413320in}}%
\pgfpathlineto{\pgfqpoint{4.603430in}{0.413320in}}%
\pgfpathlineto{\pgfqpoint{4.600633in}{0.413320in}}%
\pgfpathlineto{\pgfqpoint{4.597951in}{0.413320in}}%
\pgfpathlineto{\pgfqpoint{4.595281in}{0.413320in}}%
\pgfpathlineto{\pgfqpoint{4.592589in}{0.413320in}}%
\pgfpathlineto{\pgfqpoint{4.589920in}{0.413320in}}%
\pgfpathlineto{\pgfqpoint{4.587244in}{0.413320in}}%
\pgfpathlineto{\pgfqpoint{4.584672in}{0.413320in}}%
\pgfpathlineto{\pgfqpoint{4.581888in}{0.413320in}}%
\pgfpathlineto{\pgfqpoint{4.579305in}{0.413320in}}%
\pgfpathlineto{\pgfqpoint{4.576531in}{0.413320in}}%
\pgfpathlineto{\pgfqpoint{4.573947in}{0.413320in}}%
\pgfpathlineto{\pgfqpoint{4.571171in}{0.413320in}}%
\pgfpathlineto{\pgfqpoint{4.568612in}{0.413320in}}%
\pgfpathlineto{\pgfqpoint{4.565820in}{0.413320in}}%
\pgfpathlineto{\pgfqpoint{4.563125in}{0.413320in}}%
\pgfpathlineto{\pgfqpoint{4.560448in}{0.413320in}}%
\pgfpathlineto{\pgfqpoint{4.557777in}{0.413320in}}%
\pgfpathlineto{\pgfqpoint{4.555106in}{0.413320in}}%
\pgfpathlineto{\pgfqpoint{4.552425in}{0.413320in}}%
\pgfpathlineto{\pgfqpoint{4.549822in}{0.413320in}}%
\pgfpathlineto{\pgfqpoint{4.547064in}{0.413320in}}%
\pgfpathlineto{\pgfqpoint{4.544464in}{0.413320in}}%
\pgfpathlineto{\pgfqpoint{4.541711in}{0.413320in}}%
\pgfpathlineto{\pgfqpoint{4.539144in}{0.413320in}}%
\pgfpathlineto{\pgfqpoint{4.536400in}{0.413320in}}%
\pgfpathlineto{\pgfqpoint{4.533764in}{0.413320in}}%
\pgfpathlineto{\pgfqpoint{4.530990in}{0.413320in}}%
\pgfpathlineto{\pgfqpoint{4.528307in}{0.413320in}}%
\pgfpathlineto{\pgfqpoint{4.525640in}{0.413320in}}%
\pgfpathlineto{\pgfqpoint{4.522962in}{0.413320in}}%
\pgfpathlineto{\pgfqpoint{4.520345in}{0.413320in}}%
\pgfpathlineto{\pgfqpoint{4.517598in}{0.413320in}}%
\pgfpathlineto{\pgfqpoint{4.515080in}{0.413320in}}%
\pgfpathlineto{\pgfqpoint{4.512246in}{0.413320in}}%
\pgfpathlineto{\pgfqpoint{4.509643in}{0.413320in}}%
\pgfpathlineto{\pgfqpoint{4.506893in}{0.413320in}}%
\pgfpathlineto{\pgfqpoint{4.504305in}{0.413320in}}%
\pgfpathlineto{\pgfqpoint{4.501529in}{0.413320in}}%
\pgfpathlineto{\pgfqpoint{4.498850in}{0.413320in}}%
\pgfpathlineto{\pgfqpoint{4.496167in}{0.413320in}}%
\pgfpathlineto{\pgfqpoint{4.493492in}{0.413320in}}%
\pgfpathlineto{\pgfqpoint{4.490822in}{0.413320in}}%
\pgfpathlineto{\pgfqpoint{4.488130in}{0.413320in}}%
\pgfpathlineto{\pgfqpoint{4.485581in}{0.413320in}}%
\pgfpathlineto{\pgfqpoint{4.482778in}{0.413320in}}%
\pgfpathlineto{\pgfqpoint{4.480201in}{0.413320in}}%
\pgfpathlineto{\pgfqpoint{4.477430in}{0.413320in}}%
\pgfpathlineto{\pgfqpoint{4.474861in}{0.413320in}}%
\pgfpathlineto{\pgfqpoint{4.472059in}{0.413320in}}%
\pgfpathlineto{\pgfqpoint{4.469492in}{0.413320in}}%
\pgfpathlineto{\pgfqpoint{4.466717in}{0.413320in}}%
\pgfpathlineto{\pgfqpoint{4.464029in}{0.413320in}}%
\pgfpathlineto{\pgfqpoint{4.461367in}{0.413320in}}%
\pgfpathlineto{\pgfqpoint{4.458681in}{0.413320in}}%
\pgfpathlineto{\pgfqpoint{4.456138in}{0.413320in}}%
\pgfpathlineto{\pgfqpoint{4.453312in}{0.413320in}}%
\pgfpathlineto{\pgfqpoint{4.450767in}{0.413320in}}%
\pgfpathlineto{\pgfqpoint{4.447965in}{0.413320in}}%
\pgfpathlineto{\pgfqpoint{4.445423in}{0.413320in}}%
\pgfpathlineto{\pgfqpoint{4.442611in}{0.413320in}}%
\pgfpathlineto{\pgfqpoint{4.440041in}{0.413320in}}%
\pgfpathlineto{\pgfqpoint{4.437253in}{0.413320in}}%
\pgfpathlineto{\pgfqpoint{4.434569in}{0.413320in}}%
\pgfpathlineto{\pgfqpoint{4.431901in}{0.413320in}}%
\pgfpathlineto{\pgfqpoint{4.429220in}{0.413320in}}%
\pgfpathlineto{\pgfqpoint{4.426534in}{0.413320in}}%
\pgfpathlineto{\pgfqpoint{4.423863in}{0.413320in}}%
\pgfpathlineto{\pgfqpoint{4.421292in}{0.413320in}}%
\pgfpathlineto{\pgfqpoint{4.418506in}{0.413320in}}%
\pgfpathlineto{\pgfqpoint{4.415932in}{0.413320in}}%
\pgfpathlineto{\pgfqpoint{4.413149in}{0.413320in}}%
\pgfpathlineto{\pgfqpoint{4.410587in}{0.413320in}}%
\pgfpathlineto{\pgfqpoint{4.407788in}{0.413320in}}%
\pgfpathlineto{\pgfqpoint{4.405234in}{0.413320in}}%
\pgfpathlineto{\pgfqpoint{4.402468in}{0.413320in}}%
\pgfpathlineto{\pgfqpoint{4.399745in}{0.413320in}}%
\pgfpathlineto{\pgfqpoint{4.397076in}{0.413320in}}%
\pgfpathlineto{\pgfqpoint{4.394400in}{0.413320in}}%
\pgfpathlineto{\pgfqpoint{4.391721in}{0.413320in}}%
\pgfpathlineto{\pgfqpoint{4.389044in}{0.413320in}}%
\pgfpathlineto{\pgfqpoint{4.386431in}{0.413320in}}%
\pgfpathlineto{\pgfqpoint{4.383674in}{0.413320in}}%
\pgfpathlineto{\pgfqpoint{4.381097in}{0.413320in}}%
\pgfpathlineto{\pgfqpoint{4.378329in}{0.413320in}}%
\pgfpathlineto{\pgfqpoint{4.375761in}{0.413320in}}%
\pgfpathlineto{\pgfqpoint{4.372976in}{0.413320in}}%
\pgfpathlineto{\pgfqpoint{4.370437in}{0.413320in}}%
\pgfpathlineto{\pgfqpoint{4.367646in}{0.413320in}}%
\pgfpathlineto{\pgfqpoint{4.364936in}{0.413320in}}%
\pgfpathlineto{\pgfqpoint{4.362270in}{0.413320in}}%
\pgfpathlineto{\pgfqpoint{4.359582in}{0.413320in}}%
\pgfpathlineto{\pgfqpoint{4.357014in}{0.413320in}}%
\pgfpathlineto{\pgfqpoint{4.354224in}{0.413320in}}%
\pgfpathlineto{\pgfqpoint{4.351645in}{0.413320in}}%
\pgfpathlineto{\pgfqpoint{4.348868in}{0.413320in}}%
\pgfpathlineto{\pgfqpoint{4.346263in}{0.413320in}}%
\pgfpathlineto{\pgfqpoint{4.343510in}{0.413320in}}%
\pgfpathlineto{\pgfqpoint{4.340976in}{0.413320in}}%
\pgfpathlineto{\pgfqpoint{4.338154in}{0.413320in}}%
\pgfpathlineto{\pgfqpoint{4.335463in}{0.413320in}}%
\pgfpathlineto{\pgfqpoint{4.332796in}{0.413320in}}%
\pgfpathlineto{\pgfqpoint{4.330118in}{0.413320in}}%
\pgfpathlineto{\pgfqpoint{4.327440in}{0.413320in}}%
\pgfpathlineto{\pgfqpoint{4.324760in}{0.413320in}}%
\pgfpathlineto{\pgfqpoint{4.322181in}{0.413320in}}%
\pgfpathlineto{\pgfqpoint{4.319405in}{0.413320in}}%
\pgfpathlineto{\pgfqpoint{4.316856in}{0.413320in}}%
\pgfpathlineto{\pgfqpoint{4.314032in}{0.413320in}}%
\pgfpathlineto{\pgfqpoint{4.311494in}{0.413320in}}%
\pgfpathlineto{\pgfqpoint{4.308691in}{0.413320in}}%
\pgfpathlineto{\pgfqpoint{4.306118in}{0.413320in}}%
\pgfpathlineto{\pgfqpoint{4.303357in}{0.413320in}}%
\pgfpathlineto{\pgfqpoint{4.300656in}{0.413320in}}%
\pgfpathlineto{\pgfqpoint{4.297977in}{0.413320in}}%
\pgfpathlineto{\pgfqpoint{4.295299in}{0.413320in}}%
\pgfpathlineto{\pgfqpoint{4.292786in}{0.413320in}}%
\pgfpathlineto{\pgfqpoint{4.289936in}{0.413320in}}%
\pgfpathlineto{\pgfqpoint{4.287399in}{0.413320in}}%
\pgfpathlineto{\pgfqpoint{4.284586in}{0.413320in}}%
\pgfpathlineto{\pgfqpoint{4.282000in}{0.413320in}}%
\pgfpathlineto{\pgfqpoint{4.279212in}{0.413320in}}%
\pgfpathlineto{\pgfqpoint{4.276635in}{0.413320in}}%
\pgfpathlineto{\pgfqpoint{4.273874in}{0.413320in}}%
\pgfpathlineto{\pgfqpoint{4.271187in}{0.413320in}}%
\pgfpathlineto{\pgfqpoint{4.268590in}{0.413320in}}%
\pgfpathlineto{\pgfqpoint{4.265824in}{0.413320in}}%
\pgfpathlineto{\pgfqpoint{4.263157in}{0.413320in}}%
\pgfpathlineto{\pgfqpoint{4.260477in}{0.413320in}}%
\pgfpathlineto{\pgfqpoint{4.257958in}{0.413320in}}%
\pgfpathlineto{\pgfqpoint{4.255120in}{0.413320in}}%
\pgfpathlineto{\pgfqpoint{4.252581in}{0.413320in}}%
\pgfpathlineto{\pgfqpoint{4.249767in}{0.413320in}}%
\pgfpathlineto{\pgfqpoint{4.247225in}{0.413320in}}%
\pgfpathlineto{\pgfqpoint{4.244394in}{0.413320in}}%
\pgfpathlineto{\pgfqpoint{4.241900in}{0.413320in}}%
\pgfpathlineto{\pgfqpoint{4.239084in}{0.413320in}}%
\pgfpathlineto{\pgfqpoint{4.236375in}{0.413320in}}%
\pgfpathlineto{\pgfqpoint{4.233691in}{0.413320in}}%
\pgfpathlineto{\pgfqpoint{4.231013in}{0.413320in}}%
\pgfpathlineto{\pgfqpoint{4.228331in}{0.413320in}}%
\pgfpathlineto{\pgfqpoint{4.225654in}{0.413320in}}%
\pgfpathlineto{\pgfqpoint{4.223082in}{0.413320in}}%
\pgfpathlineto{\pgfqpoint{4.220304in}{0.413320in}}%
\pgfpathlineto{\pgfqpoint{4.217694in}{0.413320in}}%
\pgfpathlineto{\pgfqpoint{4.214948in}{0.413320in}}%
\pgfpathlineto{\pgfqpoint{4.212383in}{0.413320in}}%
\pgfpathlineto{\pgfqpoint{4.209597in}{0.413320in}}%
\pgfpathlineto{\pgfqpoint{4.207076in}{0.413320in}}%
\pgfpathlineto{\pgfqpoint{4.204240in}{0.413320in}}%
\pgfpathlineto{\pgfqpoint{4.201542in}{0.413320in}}%
\pgfpathlineto{\pgfqpoint{4.198878in}{0.413320in}}%
\pgfpathlineto{\pgfqpoint{4.196186in}{0.413320in}}%
\pgfpathlineto{\pgfqpoint{4.193638in}{0.413320in}}%
\pgfpathlineto{\pgfqpoint{4.190842in}{0.413320in}}%
\pgfpathlineto{\pgfqpoint{4.188318in}{0.413320in}}%
\pgfpathlineto{\pgfqpoint{4.185481in}{0.413320in}}%
\pgfpathlineto{\pgfqpoint{4.182899in}{0.413320in}}%
\pgfpathlineto{\pgfqpoint{4.180129in}{0.413320in}}%
\pgfpathlineto{\pgfqpoint{4.177593in}{0.413320in}}%
\pgfpathlineto{\pgfqpoint{4.174770in}{0.413320in}}%
\pgfpathlineto{\pgfqpoint{4.172093in}{0.413320in}}%
\pgfpathlineto{\pgfqpoint{4.169415in}{0.413320in}}%
\pgfpathlineto{\pgfqpoint{4.166737in}{0.413320in}}%
\pgfpathlineto{\pgfqpoint{4.164059in}{0.413320in}}%
\pgfpathlineto{\pgfqpoint{4.161380in}{0.413320in}}%
\pgfpathlineto{\pgfqpoint{4.158806in}{0.413320in}}%
\pgfpathlineto{\pgfqpoint{4.156016in}{0.413320in}}%
\pgfpathlineto{\pgfqpoint{4.153423in}{0.413320in}}%
\pgfpathlineto{\pgfqpoint{4.150665in}{0.413320in}}%
\pgfpathlineto{\pgfqpoint{4.148082in}{0.413320in}}%
\pgfpathlineto{\pgfqpoint{4.145310in}{0.413320in}}%
\pgfpathlineto{\pgfqpoint{4.142713in}{0.413320in}}%
\pgfpathlineto{\pgfqpoint{4.139963in}{0.413320in}}%
\pgfpathlineto{\pgfqpoint{4.137272in}{0.413320in}}%
\pgfpathlineto{\pgfqpoint{4.134615in}{0.413320in}}%
\pgfpathlineto{\pgfqpoint{4.131920in}{0.413320in}}%
\pgfpathlineto{\pgfqpoint{4.129349in}{0.413320in}}%
\pgfpathlineto{\pgfqpoint{4.126553in}{0.413320in}}%
\pgfpathlineto{\pgfqpoint{4.124019in}{0.413320in}}%
\pgfpathlineto{\pgfqpoint{4.121205in}{0.413320in}}%
\pgfpathlineto{\pgfqpoint{4.118554in}{0.413320in}}%
\pgfpathlineto{\pgfqpoint{4.115844in}{0.413320in}}%
\pgfpathlineto{\pgfqpoint{4.113252in}{0.413320in}}%
\pgfpathlineto{\pgfqpoint{4.110488in}{0.413320in}}%
\pgfpathlineto{\pgfqpoint{4.107814in}{0.413320in}}%
\pgfpathlineto{\pgfqpoint{4.105185in}{0.413320in}}%
\pgfpathlineto{\pgfqpoint{4.102456in}{0.413320in}}%
\pgfpathlineto{\pgfqpoint{4.099777in}{0.413320in}}%
\pgfpathlineto{\pgfqpoint{4.097092in}{0.413320in}}%
\pgfpathlineto{\pgfqpoint{4.094527in}{0.413320in}}%
\pgfpathlineto{\pgfqpoint{4.091729in}{0.413320in}}%
\pgfpathlineto{\pgfqpoint{4.089159in}{0.413320in}}%
\pgfpathlineto{\pgfqpoint{4.086385in}{0.413320in}}%
\pgfpathlineto{\pgfqpoint{4.083870in}{0.413320in}}%
\pgfpathlineto{\pgfqpoint{4.081018in}{0.413320in}}%
\pgfpathlineto{\pgfqpoint{4.078471in}{0.413320in}}%
\pgfpathlineto{\pgfqpoint{4.075705in}{0.413320in}}%
\pgfpathlineto{\pgfqpoint{4.072985in}{0.413320in}}%
\pgfpathlineto{\pgfqpoint{4.070313in}{0.413320in}}%
\pgfpathlineto{\pgfqpoint{4.067636in}{0.413320in}}%
\pgfpathlineto{\pgfqpoint{4.064957in}{0.413320in}}%
\pgfpathlineto{\pgfqpoint{4.062266in}{0.413320in}}%
\pgfpathlineto{\pgfqpoint{4.059702in}{0.413320in}}%
\pgfpathlineto{\pgfqpoint{4.056911in}{0.413320in}}%
\pgfpathlineto{\pgfqpoint{4.054326in}{0.413320in}}%
\pgfpathlineto{\pgfqpoint{4.051557in}{0.413320in}}%
\pgfpathlineto{\pgfqpoint{4.049006in}{0.413320in}}%
\pgfpathlineto{\pgfqpoint{4.046210in}{0.413320in}}%
\pgfpathlineto{\pgfqpoint{4.043667in}{0.413320in}}%
\pgfpathlineto{\pgfqpoint{4.040852in}{0.413320in}}%
\pgfpathlineto{\pgfqpoint{4.038174in}{0.413320in}}%
\pgfpathlineto{\pgfqpoint{4.035492in}{0.413320in}}%
\pgfpathlineto{\pgfqpoint{4.032817in}{0.413320in}}%
\pgfpathlineto{\pgfqpoint{4.030229in}{0.413320in}}%
\pgfpathlineto{\pgfqpoint{4.027447in}{0.413320in}}%
\pgfpathlineto{\pgfqpoint{4.024868in}{0.413320in}}%
\pgfpathlineto{\pgfqpoint{4.022097in}{0.413320in}}%
\pgfpathlineto{\pgfqpoint{4.019518in}{0.413320in}}%
\pgfpathlineto{\pgfqpoint{4.016744in}{0.413320in}}%
\pgfpathlineto{\pgfqpoint{4.014186in}{0.413320in}}%
\pgfpathlineto{\pgfqpoint{4.011394in}{0.413320in}}%
\pgfpathlineto{\pgfqpoint{4.008699in}{0.413320in}}%
\pgfpathlineto{\pgfqpoint{4.006034in}{0.413320in}}%
\pgfpathlineto{\pgfqpoint{4.003348in}{0.413320in}}%
\pgfpathlineto{\pgfqpoint{4.000674in}{0.413320in}}%
\pgfpathlineto{\pgfqpoint{3.997990in}{0.413320in}}%
\pgfpathlineto{\pgfqpoint{3.995417in}{0.413320in}}%
\pgfpathlineto{\pgfqpoint{3.992642in}{0.413320in}}%
\pgfpathlineto{\pgfqpoint{3.990055in}{0.413320in}}%
\pgfpathlineto{\pgfqpoint{3.987270in}{0.413320in}}%
\pgfpathlineto{\pgfqpoint{3.984714in}{0.413320in}}%
\pgfpathlineto{\pgfqpoint{3.981929in}{0.413320in}}%
\pgfpathlineto{\pgfqpoint{3.979389in}{0.413320in}}%
\pgfpathlineto{\pgfqpoint{3.976563in}{0.413320in}}%
\pgfpathlineto{\pgfqpoint{3.973885in}{0.413320in}}%
\pgfpathlineto{\pgfqpoint{3.971250in}{0.413320in}}%
\pgfpathlineto{\pgfqpoint{3.968523in}{0.413320in}}%
\pgfpathlineto{\pgfqpoint{3.966013in}{0.413320in}}%
\pgfpathlineto{\pgfqpoint{3.963176in}{0.413320in}}%
\pgfpathlineto{\pgfqpoint{3.960635in}{0.413320in}}%
\pgfpathlineto{\pgfqpoint{3.957823in}{0.413320in}}%
\pgfpathlineto{\pgfqpoint{3.955211in}{0.413320in}}%
\pgfpathlineto{\pgfqpoint{3.952464in}{0.413320in}}%
\pgfpathlineto{\pgfqpoint{3.949894in}{0.413320in}}%
\pgfpathlineto{\pgfqpoint{3.947101in}{0.413320in}}%
\pgfpathlineto{\pgfqpoint{3.944431in}{0.413320in}}%
\pgfpathlineto{\pgfqpoint{3.941778in}{0.413320in}}%
\pgfpathlineto{\pgfqpoint{3.939075in}{0.413320in}}%
\pgfpathlineto{\pgfqpoint{3.936395in}{0.413320in}}%
\pgfpathlineto{\pgfqpoint{3.933714in}{0.413320in}}%
\pgfpathlineto{\pgfqpoint{3.931202in}{0.413320in}}%
\pgfpathlineto{\pgfqpoint{3.928347in}{0.413320in}}%
\pgfpathlineto{\pgfqpoint{3.925778in}{0.413320in}}%
\pgfpathlineto{\pgfqpoint{3.923005in}{0.413320in}}%
\pgfpathlineto{\pgfqpoint{3.920412in}{0.413320in}}%
\pgfpathlineto{\pgfqpoint{3.917646in}{0.413320in}}%
\pgfpathlineto{\pgfqpoint{3.915107in}{0.413320in}}%
\pgfpathlineto{\pgfqpoint{3.912296in}{0.413320in}}%
\pgfpathlineto{\pgfqpoint{3.909602in}{0.413320in}}%
\pgfpathlineto{\pgfqpoint{3.906918in}{0.413320in}}%
\pgfpathlineto{\pgfqpoint{3.904252in}{0.413320in}}%
\pgfpathlineto{\pgfqpoint{3.901573in}{0.413320in}}%
\pgfpathlineto{\pgfqpoint{3.898891in}{0.413320in}}%
\pgfpathlineto{\pgfqpoint{3.896345in}{0.413320in}}%
\pgfpathlineto{\pgfqpoint{3.893541in}{0.413320in}}%
\pgfpathlineto{\pgfqpoint{3.890926in}{0.413320in}}%
\pgfpathlineto{\pgfqpoint{3.888188in}{0.413320in}}%
\pgfpathlineto{\pgfqpoint{3.885621in}{0.413320in}}%
\pgfpathlineto{\pgfqpoint{3.882850in}{0.413320in}}%
\pgfpathlineto{\pgfqpoint{3.880237in}{0.413320in}}%
\pgfpathlineto{\pgfqpoint{3.877466in}{0.413320in}}%
\pgfpathlineto{\pgfqpoint{3.874790in}{0.413320in}}%
\pgfpathlineto{\pgfqpoint{3.872114in}{0.413320in}}%
\pgfpathlineto{\pgfqpoint{3.869435in}{0.413320in}}%
\pgfpathlineto{\pgfqpoint{3.866815in}{0.413320in}}%
\pgfpathlineto{\pgfqpoint{3.864073in}{0.413320in}}%
\pgfpathlineto{\pgfqpoint{3.861561in}{0.413320in}}%
\pgfpathlineto{\pgfqpoint{3.858720in}{0.413320in}}%
\pgfpathlineto{\pgfqpoint{3.856100in}{0.413320in}}%
\pgfpathlineto{\pgfqpoint{3.853358in}{0.413320in}}%
\pgfpathlineto{\pgfqpoint{3.850814in}{0.413320in}}%
\pgfpathlineto{\pgfqpoint{3.848005in}{0.413320in}}%
\pgfpathlineto{\pgfqpoint{3.845329in}{0.413320in}}%
\pgfpathlineto{\pgfqpoint{3.842641in}{0.413320in}}%
\pgfpathlineto{\pgfqpoint{3.839960in}{0.413320in}}%
\pgfpathlineto{\pgfqpoint{3.837286in}{0.413320in}}%
\pgfpathlineto{\pgfqpoint{3.834616in}{0.413320in}}%
\pgfpathlineto{\pgfqpoint{3.832053in}{0.413320in}}%
\pgfpathlineto{\pgfqpoint{3.829252in}{0.413320in}}%
\pgfpathlineto{\pgfqpoint{3.826679in}{0.413320in}}%
\pgfpathlineto{\pgfqpoint{3.823903in}{0.413320in}}%
\pgfpathlineto{\pgfqpoint{3.821315in}{0.413320in}}%
\pgfpathlineto{\pgfqpoint{3.818546in}{0.413320in}}%
\pgfpathlineto{\pgfqpoint{3.815983in}{0.413320in}}%
\pgfpathlineto{\pgfqpoint{3.813172in}{0.413320in}}%
\pgfpathlineto{\pgfqpoint{3.810510in}{0.413320in}}%
\pgfpathlineto{\pgfqpoint{3.807832in}{0.413320in}}%
\pgfpathlineto{\pgfqpoint{3.805145in}{0.413320in}}%
\pgfpathlineto{\pgfqpoint{3.802569in}{0.413320in}}%
\pgfpathlineto{\pgfqpoint{3.799797in}{0.413320in}}%
\pgfpathlineto{\pgfqpoint{3.797265in}{0.413320in}}%
\pgfpathlineto{\pgfqpoint{3.794435in}{0.413320in}}%
\pgfpathlineto{\pgfqpoint{3.791897in}{0.413320in}}%
\pgfpathlineto{\pgfqpoint{3.789084in}{0.413320in}}%
\pgfpathlineto{\pgfqpoint{3.786504in}{0.413320in}}%
\pgfpathlineto{\pgfqpoint{3.783725in}{0.413320in}}%
\pgfpathlineto{\pgfqpoint{3.781046in}{0.413320in}}%
\pgfpathlineto{\pgfqpoint{3.778370in}{0.413320in}}%
\pgfpathlineto{\pgfqpoint{3.775691in}{0.413320in}}%
\pgfpathlineto{\pgfqpoint{3.773014in}{0.413320in}}%
\pgfpathlineto{\pgfqpoint{3.770323in}{0.413320in}}%
\pgfpathlineto{\pgfqpoint{3.767782in}{0.413320in}}%
\pgfpathlineto{\pgfqpoint{3.764966in}{0.413320in}}%
\pgfpathlineto{\pgfqpoint{3.762389in}{0.413320in}}%
\pgfpathlineto{\pgfqpoint{3.759622in}{0.413320in}}%
\pgfpathlineto{\pgfqpoint{3.757065in}{0.413320in}}%
\pgfpathlineto{\pgfqpoint{3.754265in}{0.413320in}}%
\pgfpathlineto{\pgfqpoint{3.751728in}{0.413320in}}%
\pgfpathlineto{\pgfqpoint{3.748903in}{0.413320in}}%
\pgfpathlineto{\pgfqpoint{3.746229in}{0.413320in}}%
\pgfpathlineto{\pgfqpoint{3.743548in}{0.413320in}}%
\pgfpathlineto{\pgfqpoint{3.740874in}{0.413320in}}%
\pgfpathlineto{\pgfqpoint{3.738194in}{0.413320in}}%
\pgfpathlineto{\pgfqpoint{3.735509in}{0.413320in}}%
\pgfpathlineto{\pgfqpoint{3.732950in}{0.413320in}}%
\pgfpathlineto{\pgfqpoint{3.730158in}{0.413320in}}%
\pgfpathlineto{\pgfqpoint{3.727581in}{0.413320in}}%
\pgfpathlineto{\pgfqpoint{3.724804in}{0.413320in}}%
\pgfpathlineto{\pgfqpoint{3.722228in}{0.413320in}}%
\pgfpathlineto{\pgfqpoint{3.719446in}{0.413320in}}%
\pgfpathlineto{\pgfqpoint{3.716875in}{0.413320in}}%
\pgfpathlineto{\pgfqpoint{3.714086in}{0.413320in}}%
\pgfpathlineto{\pgfqpoint{3.711410in}{0.413320in}}%
\pgfpathlineto{\pgfqpoint{3.708729in}{0.413320in}}%
\pgfpathlineto{\pgfqpoint{3.706053in}{0.413320in}}%
\pgfpathlineto{\pgfqpoint{3.703460in}{0.413320in}}%
\pgfpathlineto{\pgfqpoint{3.700684in}{0.413320in}}%
\pgfpathlineto{\pgfqpoint{3.698125in}{0.413320in}}%
\pgfpathlineto{\pgfqpoint{3.695331in}{0.413320in}}%
\pgfpathlineto{\pgfqpoint{3.692765in}{0.413320in}}%
\pgfpathlineto{\pgfqpoint{3.689983in}{0.413320in}}%
\pgfpathlineto{\pgfqpoint{3.687442in}{0.413320in}}%
\pgfpathlineto{\pgfqpoint{3.684620in}{0.413320in}}%
\pgfpathlineto{\pgfqpoint{3.681948in}{0.413320in}}%
\pgfpathlineto{\pgfqpoint{3.679273in}{0.413320in}}%
\pgfpathlineto{\pgfqpoint{3.676591in}{0.413320in}}%
\pgfpathlineto{\pgfqpoint{3.673911in}{0.413320in}}%
\pgfpathlineto{\pgfqpoint{3.671232in}{0.413320in}}%
\pgfpathlineto{\pgfqpoint{3.668665in}{0.413320in}}%
\pgfpathlineto{\pgfqpoint{3.665864in}{0.413320in}}%
\pgfpathlineto{\pgfqpoint{3.663276in}{0.413320in}}%
\pgfpathlineto{\pgfqpoint{3.660515in}{0.413320in}}%
\pgfpathlineto{\pgfqpoint{3.657917in}{0.413320in}}%
\pgfpathlineto{\pgfqpoint{3.655165in}{0.413320in}}%
\pgfpathlineto{\pgfqpoint{3.652628in}{0.413320in}}%
\pgfpathlineto{\pgfqpoint{3.649837in}{0.413320in}}%
\pgfpathlineto{\pgfqpoint{3.647130in}{0.413320in}}%
\pgfpathlineto{\pgfqpoint{3.644452in}{0.413320in}}%
\pgfpathlineto{\pgfqpoint{3.641773in}{0.413320in}}%
\pgfpathlineto{\pgfqpoint{3.639207in}{0.413320in}}%
\pgfpathlineto{\pgfqpoint{3.636413in}{0.413320in}}%
\pgfpathlineto{\pgfqpoint{3.633858in}{0.413320in}}%
\pgfpathlineto{\pgfqpoint{3.631058in}{0.413320in}}%
\pgfpathlineto{\pgfqpoint{3.628460in}{0.413320in}}%
\pgfpathlineto{\pgfqpoint{3.625689in}{0.413320in}}%
\pgfpathlineto{\pgfqpoint{3.623165in}{0.413320in}}%
\pgfpathlineto{\pgfqpoint{3.620345in}{0.413320in}}%
\pgfpathlineto{\pgfqpoint{3.617667in}{0.413320in}}%
\pgfpathlineto{\pgfqpoint{3.614982in}{0.413320in}}%
\pgfpathlineto{\pgfqpoint{3.612311in}{0.413320in}}%
\pgfpathlineto{\pgfqpoint{3.609632in}{0.413320in}}%
\pgfpathlineto{\pgfqpoint{3.606951in}{0.413320in}}%
\pgfpathlineto{\pgfqpoint{3.604387in}{0.413320in}}%
\pgfpathlineto{\pgfqpoint{3.601590in}{0.413320in}}%
\pgfpathlineto{\pgfqpoint{3.598998in}{0.413320in}}%
\pgfpathlineto{\pgfqpoint{3.596240in}{0.413320in}}%
\pgfpathlineto{\pgfqpoint{3.593620in}{0.413320in}}%
\pgfpathlineto{\pgfqpoint{3.590883in}{0.413320in}}%
\pgfpathlineto{\pgfqpoint{3.588258in}{0.413320in}}%
\pgfpathlineto{\pgfqpoint{3.585532in}{0.413320in}}%
\pgfpathlineto{\pgfqpoint{3.582851in}{0.413320in}}%
\pgfpathlineto{\pgfqpoint{3.580191in}{0.413320in}}%
\pgfpathlineto{\pgfqpoint{3.577487in}{0.413320in}}%
\pgfpathlineto{\pgfqpoint{3.574814in}{0.413320in}}%
\pgfpathlineto{\pgfqpoint{3.572126in}{0.413320in}}%
\pgfpathlineto{\pgfqpoint{3.569584in}{0.413320in}}%
\pgfpathlineto{\pgfqpoint{3.566774in}{0.413320in}}%
\pgfpathlineto{\pgfqpoint{3.564188in}{0.413320in}}%
\pgfpathlineto{\pgfqpoint{3.561420in}{0.413320in}}%
\pgfpathlineto{\pgfqpoint{3.558853in}{0.413320in}}%
\pgfpathlineto{\pgfqpoint{3.556061in}{0.413320in}}%
\pgfpathlineto{\pgfqpoint{3.553498in}{0.413320in}}%
\pgfpathlineto{\pgfqpoint{3.550713in}{0.413320in}}%
\pgfpathlineto{\pgfqpoint{3.548029in}{0.413320in}}%
\pgfpathlineto{\pgfqpoint{3.545349in}{0.413320in}}%
\pgfpathlineto{\pgfqpoint{3.542656in}{0.413320in}}%
\pgfpathlineto{\pgfqpoint{3.540093in}{0.413320in}}%
\pgfpathlineto{\pgfqpoint{3.537309in}{0.413320in}}%
\pgfpathlineto{\pgfqpoint{3.534783in}{0.413320in}}%
\pgfpathlineto{\pgfqpoint{3.531955in}{0.413320in}}%
\pgfpathlineto{\pgfqpoint{3.529327in}{0.413320in}}%
\pgfpathlineto{\pgfqpoint{3.526601in}{0.413320in}}%
\pgfpathlineto{\pgfqpoint{3.524041in}{0.413320in}}%
\pgfpathlineto{\pgfqpoint{3.521244in}{0.413320in}}%
\pgfpathlineto{\pgfqpoint{3.518565in}{0.413320in}}%
\pgfpathlineto{\pgfqpoint{3.515884in}{0.413320in}}%
\pgfpathlineto{\pgfqpoint{3.513209in}{0.413320in}}%
\pgfpathlineto{\pgfqpoint{3.510533in}{0.413320in}}%
\pgfpathlineto{\pgfqpoint{3.507840in}{0.413320in}}%
\pgfpathlineto{\pgfqpoint{3.505262in}{0.413320in}}%
\pgfpathlineto{\pgfqpoint{3.502488in}{0.413320in}}%
\pgfpathlineto{\pgfqpoint{3.499909in}{0.413320in}}%
\pgfpathlineto{\pgfqpoint{3.497139in}{0.413320in}}%
\pgfpathlineto{\pgfqpoint{3.494581in}{0.413320in}}%
\pgfpathlineto{\pgfqpoint{3.491783in}{0.413320in}}%
\pgfpathlineto{\pgfqpoint{3.489223in}{0.413320in}}%
\pgfpathlineto{\pgfqpoint{3.486442in}{0.413320in}}%
\pgfpathlineto{\pgfqpoint{3.483744in}{0.413320in}}%
\pgfpathlineto{\pgfqpoint{3.481072in}{0.413320in}}%
\pgfpathlineto{\pgfqpoint{3.478378in}{0.413320in}}%
\pgfpathlineto{\pgfqpoint{3.475821in}{0.413320in}}%
\pgfpathlineto{\pgfqpoint{3.473021in}{0.413320in}}%
\pgfpathlineto{\pgfqpoint{3.470466in}{0.413320in}}%
\pgfpathlineto{\pgfqpoint{3.467678in}{0.413320in}}%
\pgfpathlineto{\pgfqpoint{3.465072in}{0.413320in}}%
\pgfpathlineto{\pgfqpoint{3.462321in}{0.413320in}}%
\pgfpathlineto{\pgfqpoint{3.459695in}{0.413320in}}%
\pgfpathlineto{\pgfqpoint{3.456960in}{0.413320in}}%
\pgfpathlineto{\pgfqpoint{3.454285in}{0.413320in}}%
\pgfpathlineto{\pgfqpoint{3.451597in}{0.413320in}}%
\pgfpathlineto{\pgfqpoint{3.448926in}{0.413320in}}%
\pgfpathlineto{\pgfqpoint{3.446257in}{0.413320in}}%
\pgfpathlineto{\pgfqpoint{3.443574in}{0.413320in}}%
\pgfpathlineto{\pgfqpoint{3.440996in}{0.413320in}}%
\pgfpathlineto{\pgfqpoint{3.438210in}{0.413320in}}%
\pgfpathlineto{\pgfqpoint{3.435635in}{0.413320in}}%
\pgfpathlineto{\pgfqpoint{3.432851in}{0.413320in}}%
\pgfpathlineto{\pgfqpoint{3.430313in}{0.413320in}}%
\pgfpathlineto{\pgfqpoint{3.427501in}{0.413320in}}%
\pgfpathlineto{\pgfqpoint{3.424887in}{0.413320in}}%
\pgfpathlineto{\pgfqpoint{3.422142in}{0.413320in}}%
\pgfpathlineto{\pgfqpoint{3.419455in}{0.413320in}}%
\pgfpathlineto{\pgfqpoint{3.416780in}{0.413320in}}%
\pgfpathlineto{\pgfqpoint{3.414109in}{0.413320in}}%
\pgfpathlineto{\pgfqpoint{3.411431in}{0.413320in}}%
\pgfpathlineto{\pgfqpoint{3.408752in}{0.413320in}}%
\pgfpathlineto{\pgfqpoint{3.406202in}{0.413320in}}%
\pgfpathlineto{\pgfqpoint{3.403394in}{0.413320in}}%
\pgfpathlineto{\pgfqpoint{3.400783in}{0.413320in}}%
\pgfpathlineto{\pgfqpoint{3.398037in}{0.413320in}}%
\pgfpathlineto{\pgfqpoint{3.395461in}{0.413320in}}%
\pgfpathlineto{\pgfqpoint{3.392681in}{0.413320in}}%
\pgfpathlineto{\pgfqpoint{3.390102in}{0.413320in}}%
\pgfpathlineto{\pgfqpoint{3.387309in}{0.413320in}}%
\pgfpathlineto{\pgfqpoint{3.384647in}{0.413320in}}%
\pgfpathlineto{\pgfqpoint{3.381959in}{0.413320in}}%
\pgfpathlineto{\pgfqpoint{3.379290in}{0.413320in}}%
\pgfpathlineto{\pgfqpoint{3.376735in}{0.413320in}}%
\pgfpathlineto{\pgfqpoint{3.373921in}{0.413320in}}%
\pgfpathlineto{\pgfqpoint{3.371357in}{0.413320in}}%
\pgfpathlineto{\pgfqpoint{3.368577in}{0.413320in}}%
\pgfpathlineto{\pgfqpoint{3.365996in}{0.413320in}}%
\pgfpathlineto{\pgfqpoint{3.363221in}{0.413320in}}%
\pgfpathlineto{\pgfqpoint{3.360620in}{0.413320in}}%
\pgfpathlineto{\pgfqpoint{3.357862in}{0.413320in}}%
\pgfpathlineto{\pgfqpoint{3.355177in}{0.413320in}}%
\pgfpathlineto{\pgfqpoint{3.352505in}{0.413320in}}%
\pgfpathlineto{\pgfqpoint{3.349828in}{0.413320in}}%
\pgfpathlineto{\pgfqpoint{3.347139in}{0.413320in}}%
\pgfpathlineto{\pgfqpoint{3.344468in}{0.413320in}}%
\pgfpathlineto{\pgfqpoint{3.341893in}{0.413320in}}%
\pgfpathlineto{\pgfqpoint{3.339101in}{0.413320in}}%
\pgfpathlineto{\pgfqpoint{3.336541in}{0.413320in}}%
\pgfpathlineto{\pgfqpoint{3.333758in}{0.413320in}}%
\pgfpathlineto{\pgfqpoint{3.331183in}{0.413320in}}%
\pgfpathlineto{\pgfqpoint{3.328401in}{0.413320in}}%
\pgfpathlineto{\pgfqpoint{3.325860in}{0.413320in}}%
\pgfpathlineto{\pgfqpoint{3.323049in}{0.413320in}}%
\pgfpathlineto{\pgfqpoint{3.320366in}{0.413320in}}%
\pgfpathlineto{\pgfqpoint{3.317688in}{0.413320in}}%
\pgfpathlineto{\pgfqpoint{3.315008in}{0.413320in}}%
\pgfpathlineto{\pgfqpoint{3.312480in}{0.413320in}}%
\pgfpathlineto{\pgfqpoint{3.309652in}{0.413320in}}%
\pgfpathlineto{\pgfqpoint{3.307104in}{0.413320in}}%
\pgfpathlineto{\pgfqpoint{3.304295in}{0.413320in}}%
\pgfpathlineto{\pgfqpoint{3.301719in}{0.413320in}}%
\pgfpathlineto{\pgfqpoint{3.298937in}{0.413320in}}%
\pgfpathlineto{\pgfqpoint{3.296376in}{0.413320in}}%
\pgfpathlineto{\pgfqpoint{3.293574in}{0.413320in}}%
\pgfpathlineto{\pgfqpoint{3.290890in}{0.413320in}}%
\pgfpathlineto{\pgfqpoint{3.288225in}{0.413320in}}%
\pgfpathlineto{\pgfqpoint{3.285534in}{0.413320in}}%
\pgfpathlineto{\pgfqpoint{3.282870in}{0.413320in}}%
\pgfpathlineto{\pgfqpoint{3.280189in}{0.413320in}}%
\pgfpathlineto{\pgfqpoint{3.277603in}{0.413320in}}%
\pgfpathlineto{\pgfqpoint{3.274831in}{0.413320in}}%
\pgfpathlineto{\pgfqpoint{3.272254in}{0.413320in}}%
\pgfpathlineto{\pgfqpoint{3.269478in}{0.413320in}}%
\pgfpathlineto{\pgfqpoint{3.266849in}{0.413320in}}%
\pgfpathlineto{\pgfqpoint{3.264119in}{0.413320in}}%
\pgfpathlineto{\pgfqpoint{3.261594in}{0.413320in}}%
\pgfpathlineto{\pgfqpoint{3.258784in}{0.413320in}}%
\pgfpathlineto{\pgfqpoint{3.256083in}{0.413320in}}%
\pgfpathlineto{\pgfqpoint{3.253404in}{0.413320in}}%
\pgfpathlineto{\pgfqpoint{3.250716in}{0.413320in}}%
\pgfpathlineto{\pgfqpoint{3.248049in}{0.413320in}}%
\pgfpathlineto{\pgfqpoint{3.245363in}{0.413320in}}%
\pgfpathlineto{\pgfqpoint{3.242807in}{0.413320in}}%
\pgfpathlineto{\pgfqpoint{3.240010in}{0.413320in}}%
\pgfpathlineto{\pgfqpoint{3.237411in}{0.413320in}}%
\pgfpathlineto{\pgfqpoint{3.234658in}{0.413320in}}%
\pgfpathlineto{\pgfqpoint{3.232069in}{0.413320in}}%
\pgfpathlineto{\pgfqpoint{3.229310in}{0.413320in}}%
\pgfpathlineto{\pgfqpoint{3.226609in}{0.413320in}}%
\pgfpathlineto{\pgfqpoint{3.223942in}{0.413320in}}%
\pgfpathlineto{\pgfqpoint{3.221255in}{0.413320in}}%
\pgfpathlineto{\pgfqpoint{3.218586in}{0.413320in}}%
\pgfpathlineto{\pgfqpoint{3.215908in}{0.413320in}}%
\pgfpathlineto{\pgfqpoint{3.213342in}{0.413320in}}%
\pgfpathlineto{\pgfqpoint{3.210545in}{0.413320in}}%
\pgfpathlineto{\pgfqpoint{3.207984in}{0.413320in}}%
\pgfpathlineto{\pgfqpoint{3.205195in}{0.413320in}}%
\pgfpathlineto{\pgfqpoint{3.202562in}{0.413320in}}%
\pgfpathlineto{\pgfqpoint{3.199823in}{0.413320in}}%
\pgfpathlineto{\pgfqpoint{3.197226in}{0.413320in}}%
\pgfpathlineto{\pgfqpoint{3.194508in}{0.413320in}}%
\pgfpathlineto{\pgfqpoint{3.191796in}{0.413320in}}%
\pgfpathlineto{\pgfqpoint{3.189117in}{0.413320in}}%
\pgfpathlineto{\pgfqpoint{3.186440in}{0.413320in}}%
\pgfpathlineto{\pgfqpoint{3.183760in}{0.413320in}}%
\pgfpathlineto{\pgfqpoint{3.181089in}{0.413320in}}%
\pgfpathlineto{\pgfqpoint{3.178525in}{0.413320in}}%
\pgfpathlineto{\pgfqpoint{3.175724in}{0.413320in}}%
\pgfpathlineto{\pgfqpoint{3.173142in}{0.413320in}}%
\pgfpathlineto{\pgfqpoint{3.170375in}{0.413320in}}%
\pgfpathlineto{\pgfqpoint{3.167776in}{0.413320in}}%
\pgfpathlineto{\pgfqpoint{3.165019in}{0.413320in}}%
\pgfpathlineto{\pgfqpoint{3.162474in}{0.413320in}}%
\pgfpathlineto{\pgfqpoint{3.159675in}{0.413320in}}%
\pgfpathlineto{\pgfqpoint{3.156981in}{0.413320in}}%
\pgfpathlineto{\pgfqpoint{3.154327in}{0.413320in}}%
\pgfpathlineto{\pgfqpoint{3.151612in}{0.413320in}}%
\pgfpathlineto{\pgfqpoint{3.149057in}{0.413320in}}%
\pgfpathlineto{\pgfqpoint{3.146271in}{0.413320in}}%
\pgfpathlineto{\pgfqpoint{3.143740in}{0.413320in}}%
\pgfpathlineto{\pgfqpoint{3.140913in}{0.413320in}}%
\pgfpathlineto{\pgfqpoint{3.138375in}{0.413320in}}%
\pgfpathlineto{\pgfqpoint{3.135550in}{0.413320in}}%
\pgfpathlineto{\pgfqpoint{3.132946in}{0.413320in}}%
\pgfpathlineto{\pgfqpoint{3.130199in}{0.413320in}}%
\pgfpathlineto{\pgfqpoint{3.127512in}{0.413320in}}%
\pgfpathlineto{\pgfqpoint{3.124842in}{0.413320in}}%
\pgfpathlineto{\pgfqpoint{3.122163in}{0.413320in}}%
\pgfpathlineto{\pgfqpoint{3.119487in}{0.413320in}}%
\pgfpathlineto{\pgfqpoint{3.116807in}{0.413320in}}%
\pgfpathlineto{\pgfqpoint{3.114242in}{0.413320in}}%
\pgfpathlineto{\pgfqpoint{3.111451in}{0.413320in}}%
\pgfpathlineto{\pgfqpoint{3.108896in}{0.413320in}}%
\pgfpathlineto{\pgfqpoint{3.106094in}{0.413320in}}%
\pgfpathlineto{\pgfqpoint{3.103508in}{0.413320in}}%
\pgfpathlineto{\pgfqpoint{3.100737in}{0.413320in}}%
\pgfpathlineto{\pgfqpoint{3.098163in}{0.413320in}}%
\pgfpathlineto{\pgfqpoint{3.095388in}{0.413320in}}%
\pgfpathlineto{\pgfqpoint{3.092699in}{0.413320in}}%
\pgfpathlineto{\pgfqpoint{3.090023in}{0.413320in}}%
\pgfpathlineto{\pgfqpoint{3.087343in}{0.413320in}}%
\pgfpathlineto{\pgfqpoint{3.084671in}{0.413320in}}%
\pgfpathlineto{\pgfqpoint{3.081990in}{0.413320in}}%
\pgfpathlineto{\pgfqpoint{3.079381in}{0.413320in}}%
\pgfpathlineto{\pgfqpoint{3.076631in}{0.413320in}}%
\pgfpathlineto{\pgfqpoint{3.074056in}{0.413320in}}%
\pgfpathlineto{\pgfqpoint{3.071266in}{0.413320in}}%
\pgfpathlineto{\pgfqpoint{3.068709in}{0.413320in}}%
\pgfpathlineto{\pgfqpoint{3.065916in}{0.413320in}}%
\pgfpathlineto{\pgfqpoint{3.063230in}{0.413320in}}%
\pgfpathlineto{\pgfqpoint{3.060561in}{0.413320in}}%
\pgfpathlineto{\pgfqpoint{3.057884in}{0.413320in}}%
\pgfpathlineto{\pgfqpoint{3.055202in}{0.413320in}}%
\pgfpathlineto{\pgfqpoint{3.052526in}{0.413320in}}%
\pgfpathlineto{\pgfqpoint{3.049988in}{0.413320in}}%
\pgfpathlineto{\pgfqpoint{3.047157in}{0.413320in}}%
\pgfpathlineto{\pgfqpoint{3.044568in}{0.413320in}}%
\pgfpathlineto{\pgfqpoint{3.041813in}{0.413320in}}%
\pgfpathlineto{\pgfqpoint{3.039262in}{0.413320in}}%
\pgfpathlineto{\pgfqpoint{3.036456in}{0.413320in}}%
\pgfpathlineto{\pgfqpoint{3.033921in}{0.413320in}}%
\pgfpathlineto{\pgfqpoint{3.031091in}{0.413320in}}%
\pgfpathlineto{\pgfqpoint{3.028412in}{0.413320in}}%
\pgfpathlineto{\pgfqpoint{3.025803in}{0.413320in}}%
\pgfpathlineto{\pgfqpoint{3.023058in}{0.413320in}}%
\pgfpathlineto{\pgfqpoint{3.020382in}{0.413320in}}%
\pgfpathlineto{\pgfqpoint{3.017707in}{0.413320in}}%
\pgfpathlineto{\pgfqpoint{3.015097in}{0.413320in}}%
\pgfpathlineto{\pgfqpoint{3.012351in}{0.413320in}}%
\pgfpathlineto{\pgfqpoint{3.009784in}{0.413320in}}%
\pgfpathlineto{\pgfqpoint{3.006993in}{0.413320in}}%
\pgfpathlineto{\pgfqpoint{3.004419in}{0.413320in}}%
\pgfpathlineto{\pgfqpoint{3.001635in}{0.413320in}}%
\pgfpathlineto{\pgfqpoint{2.999103in}{0.413320in}}%
\pgfpathlineto{\pgfqpoint{2.996300in}{0.413320in}}%
\pgfpathlineto{\pgfqpoint{2.993595in}{0.413320in}}%
\pgfpathlineto{\pgfqpoint{2.990978in}{0.413320in}}%
\pgfpathlineto{\pgfqpoint{2.988238in}{0.413320in}}%
\pgfpathlineto{\pgfqpoint{2.985666in}{0.413320in}}%
\pgfpathlineto{\pgfqpoint{2.982885in}{0.413320in}}%
\pgfpathlineto{\pgfqpoint{2.980341in}{0.413320in}}%
\pgfpathlineto{\pgfqpoint{2.977517in}{0.413320in}}%
\pgfpathlineto{\pgfqpoint{2.974972in}{0.413320in}}%
\pgfpathlineto{\pgfqpoint{2.972177in}{0.413320in}}%
\pgfpathlineto{\pgfqpoint{2.969599in}{0.413320in}}%
\pgfpathlineto{\pgfqpoint{2.966812in}{0.413320in}}%
\pgfpathlineto{\pgfqpoint{2.964127in}{0.413320in}}%
\pgfpathlineto{\pgfqpoint{2.961460in}{0.413320in}}%
\pgfpathlineto{\pgfqpoint{2.958782in}{0.413320in}}%
\pgfpathlineto{\pgfqpoint{2.956103in}{0.413320in}}%
\pgfpathlineto{\pgfqpoint{2.953422in}{0.413320in}}%
\pgfpathlineto{\pgfqpoint{2.950884in}{0.413320in}}%
\pgfpathlineto{\pgfqpoint{2.948068in}{0.413320in}}%
\pgfpathlineto{\pgfqpoint{2.945461in}{0.413320in}}%
\pgfpathlineto{\pgfqpoint{2.942711in}{0.413320in}}%
\pgfpathlineto{\pgfqpoint{2.940120in}{0.413320in}}%
\pgfpathlineto{\pgfqpoint{2.937352in}{0.413320in}}%
\pgfpathlineto{\pgfqpoint{2.934759in}{0.413320in}}%
\pgfpathlineto{\pgfqpoint{2.932033in}{0.413320in}}%
\pgfpathlineto{\pgfqpoint{2.929321in}{0.413320in}}%
\pgfpathlineto{\pgfqpoint{2.926655in}{0.413320in}}%
\pgfpathlineto{\pgfqpoint{2.923963in}{0.413320in}}%
\pgfpathlineto{\pgfqpoint{2.921363in}{0.413320in}}%
\pgfpathlineto{\pgfqpoint{2.918606in}{0.413320in}}%
\pgfpathlineto{\pgfqpoint{2.916061in}{0.413320in}}%
\pgfpathlineto{\pgfqpoint{2.913243in}{0.413320in}}%
\pgfpathlineto{\pgfqpoint{2.910631in}{0.413320in}}%
\pgfpathlineto{\pgfqpoint{2.907882in}{0.413320in}}%
\pgfpathlineto{\pgfqpoint{2.905341in}{0.413320in}}%
\pgfpathlineto{\pgfqpoint{2.902535in}{0.413320in}}%
\pgfpathlineto{\pgfqpoint{2.899858in}{0.413320in}}%
\pgfpathlineto{\pgfqpoint{2.897179in}{0.413320in}}%
\pgfpathlineto{\pgfqpoint{2.894487in}{0.413320in}}%
\pgfpathlineto{\pgfqpoint{2.891809in}{0.413320in}}%
\pgfpathlineto{\pgfqpoint{2.889145in}{0.413320in}}%
\pgfpathlineto{\pgfqpoint{2.886578in}{0.413320in}}%
\pgfpathlineto{\pgfqpoint{2.883780in}{0.413320in}}%
\pgfpathlineto{\pgfqpoint{2.881254in}{0.413320in}}%
\pgfpathlineto{\pgfqpoint{2.878431in}{0.413320in}}%
\pgfpathlineto{\pgfqpoint{2.875882in}{0.413320in}}%
\pgfpathlineto{\pgfqpoint{2.873074in}{0.413320in}}%
\pgfpathlineto{\pgfqpoint{2.870475in}{0.413320in}}%
\pgfpathlineto{\pgfqpoint{2.867713in}{0.413320in}}%
\pgfpathlineto{\pgfqpoint{2.865031in}{0.413320in}}%
\pgfpathlineto{\pgfqpoint{2.862402in}{0.413320in}}%
\pgfpathlineto{\pgfqpoint{2.859668in}{0.413320in}}%
\pgfpathlineto{\pgfqpoint{2.857003in}{0.413320in}}%
\pgfpathlineto{\pgfqpoint{2.854325in}{0.413320in}}%
\pgfpathlineto{\pgfqpoint{2.851793in}{0.413320in}}%
\pgfpathlineto{\pgfqpoint{2.848960in}{0.413320in}}%
\pgfpathlineto{\pgfqpoint{2.846408in}{0.413320in}}%
\pgfpathlineto{\pgfqpoint{2.843611in}{0.413320in}}%
\pgfpathlineto{\pgfqpoint{2.841055in}{0.413320in}}%
\pgfpathlineto{\pgfqpoint{2.838254in}{0.413320in}}%
\pgfpathlineto{\pgfqpoint{2.835698in}{0.413320in}}%
\pgfpathlineto{\pgfqpoint{2.832894in}{0.413320in}}%
\pgfpathlineto{\pgfqpoint{2.830219in}{0.413320in}}%
\pgfpathlineto{\pgfqpoint{2.827567in}{0.413320in}}%
\pgfpathlineto{\pgfqpoint{2.824851in}{0.413320in}}%
\pgfpathlineto{\pgfqpoint{2.822303in}{0.413320in}}%
\pgfpathlineto{\pgfqpoint{2.819506in}{0.413320in}}%
\pgfpathlineto{\pgfqpoint{2.816867in}{0.413320in}}%
\pgfpathlineto{\pgfqpoint{2.814141in}{0.413320in}}%
\pgfpathlineto{\pgfqpoint{2.811597in}{0.413320in}}%
\pgfpathlineto{\pgfqpoint{2.808792in}{0.413320in}}%
\pgfpathlineto{\pgfqpoint{2.806175in}{0.413320in}}%
\pgfpathlineto{\pgfqpoint{2.803435in}{0.413320in}}%
\pgfpathlineto{\pgfqpoint{2.800756in}{0.413320in}}%
\pgfpathlineto{\pgfqpoint{2.798070in}{0.413320in}}%
\pgfpathlineto{\pgfqpoint{2.795398in}{0.413320in}}%
\pgfpathlineto{\pgfqpoint{2.792721in}{0.413320in}}%
\pgfpathlineto{\pgfqpoint{2.790044in}{0.413320in}}%
\pgfpathlineto{\pgfqpoint{2.787468in}{0.413320in}}%
\pgfpathlineto{\pgfqpoint{2.784687in}{0.413320in}}%
\pgfpathlineto{\pgfqpoint{2.782113in}{0.413320in}}%
\pgfpathlineto{\pgfqpoint{2.779330in}{0.413320in}}%
\pgfpathlineto{\pgfqpoint{2.776767in}{0.413320in}}%
\pgfpathlineto{\pgfqpoint{2.773972in}{0.413320in}}%
\pgfpathlineto{\pgfqpoint{2.771373in}{0.413320in}}%
\pgfpathlineto{\pgfqpoint{2.768617in}{0.413320in}}%
\pgfpathlineto{\pgfqpoint{2.765935in}{0.413320in}}%
\pgfpathlineto{\pgfqpoint{2.763253in}{0.413320in}}%
\pgfpathlineto{\pgfqpoint{2.760581in}{0.413320in}}%
\pgfpathlineto{\pgfqpoint{2.758028in}{0.413320in}}%
\pgfpathlineto{\pgfqpoint{2.755224in}{0.413320in}}%
\pgfpathlineto{\pgfqpoint{2.752614in}{0.413320in}}%
\pgfpathlineto{\pgfqpoint{2.749868in}{0.413320in}}%
\pgfpathlineto{\pgfqpoint{2.747260in}{0.413320in}}%
\pgfpathlineto{\pgfqpoint{2.744510in}{0.413320in}}%
\pgfpathlineto{\pgfqpoint{2.741928in}{0.413320in}}%
\pgfpathlineto{\pgfqpoint{2.739155in}{0.413320in}}%
\pgfpathlineto{\pgfqpoint{2.736476in}{0.413320in}}%
\pgfpathlineto{\pgfqpoint{2.733798in}{0.413320in}}%
\pgfpathlineto{\pgfqpoint{2.731119in}{0.413320in}}%
\pgfpathlineto{\pgfqpoint{2.728439in}{0.413320in}}%
\pgfpathlineto{\pgfqpoint{2.725760in}{0.413320in}}%
\pgfpathlineto{\pgfqpoint{2.723211in}{0.413320in}}%
\pgfpathlineto{\pgfqpoint{2.720404in}{0.413320in}}%
\pgfpathlineto{\pgfqpoint{2.717773in}{0.413320in}}%
\pgfpathlineto{\pgfqpoint{2.715036in}{0.413320in}}%
\pgfpathlineto{\pgfqpoint{2.712477in}{0.413320in}}%
\pgfpathlineto{\pgfqpoint{2.709683in}{0.413320in}}%
\pgfpathlineto{\pgfqpoint{2.707125in}{0.413320in}}%
\pgfpathlineto{\pgfqpoint{2.704326in}{0.413320in}}%
\pgfpathlineto{\pgfqpoint{2.701657in}{0.413320in}}%
\pgfpathlineto{\pgfqpoint{2.698968in}{0.413320in}}%
\pgfpathlineto{\pgfqpoint{2.696293in}{0.413320in}}%
\pgfpathlineto{\pgfqpoint{2.693611in}{0.413320in}}%
\pgfpathlineto{\pgfqpoint{2.690940in}{0.413320in}}%
\pgfpathlineto{\pgfqpoint{2.688328in}{0.413320in}}%
\pgfpathlineto{\pgfqpoint{2.685586in}{0.413320in}}%
\pgfpathlineto{\pgfqpoint{2.683009in}{0.413320in}}%
\pgfpathlineto{\pgfqpoint{2.680224in}{0.413320in}}%
\pgfpathlineto{\pgfqpoint{2.677650in}{0.413320in}}%
\pgfpathlineto{\pgfqpoint{2.674873in}{0.413320in}}%
\pgfpathlineto{\pgfqpoint{2.672301in}{0.413320in}}%
\pgfpathlineto{\pgfqpoint{2.669506in}{0.413320in}}%
\pgfpathlineto{\pgfqpoint{2.666836in}{0.413320in}}%
\pgfpathlineto{\pgfqpoint{2.664151in}{0.413320in}}%
\pgfpathlineto{\pgfqpoint{2.661481in}{0.413320in}}%
\pgfpathlineto{\pgfqpoint{2.658942in}{0.413320in}}%
\pgfpathlineto{\pgfqpoint{2.656124in}{0.413320in}}%
\pgfpathlineto{\pgfqpoint{2.653567in}{0.413320in}}%
\pgfpathlineto{\pgfqpoint{2.650767in}{0.413320in}}%
\pgfpathlineto{\pgfqpoint{2.648196in}{0.413320in}}%
\pgfpathlineto{\pgfqpoint{2.645408in}{0.413320in}}%
\pgfpathlineto{\pgfqpoint{2.642827in}{0.413320in}}%
\pgfpathlineto{\pgfqpoint{2.640053in}{0.413320in}}%
\pgfpathlineto{\pgfqpoint{2.637369in}{0.413320in}}%
\pgfpathlineto{\pgfqpoint{2.634700in}{0.413320in}}%
\pgfpathlineto{\pgfqpoint{2.632018in}{0.413320in}}%
\pgfpathlineto{\pgfqpoint{2.629340in}{0.413320in}}%
\pgfpathlineto{\pgfqpoint{2.626653in}{0.413320in}}%
\pgfpathlineto{\pgfqpoint{2.624077in}{0.413320in}}%
\pgfpathlineto{\pgfqpoint{2.621304in}{0.413320in}}%
\pgfpathlineto{\pgfqpoint{2.618773in}{0.413320in}}%
\pgfpathlineto{\pgfqpoint{2.615934in}{0.413320in}}%
\pgfpathlineto{\pgfqpoint{2.613393in}{0.413320in}}%
\pgfpathlineto{\pgfqpoint{2.610588in}{0.413320in}}%
\pgfpathlineto{\pgfqpoint{2.608004in}{0.413320in}}%
\pgfpathlineto{\pgfqpoint{2.605232in}{0.413320in}}%
\pgfpathlineto{\pgfqpoint{2.602557in}{0.413320in}}%
\pgfpathlineto{\pgfqpoint{2.599920in}{0.413320in}}%
\pgfpathlineto{\pgfqpoint{2.597196in}{0.413320in}}%
\pgfpathlineto{\pgfqpoint{2.594630in}{0.413320in}}%
\pgfpathlineto{\pgfqpoint{2.591842in}{0.413320in}}%
\pgfpathlineto{\pgfqpoint{2.589248in}{0.413320in}}%
\pgfpathlineto{\pgfqpoint{2.586484in}{0.413320in}}%
\pgfpathlineto{\pgfqpoint{2.583913in}{0.413320in}}%
\pgfpathlineto{\pgfqpoint{2.581129in}{0.413320in}}%
\pgfpathlineto{\pgfqpoint{2.578567in}{0.413320in}}%
\pgfpathlineto{\pgfqpoint{2.575779in}{0.413320in}}%
\pgfpathlineto{\pgfqpoint{2.573082in}{0.413320in}}%
\pgfpathlineto{\pgfqpoint{2.570411in}{0.413320in}}%
\pgfpathlineto{\pgfqpoint{2.567730in}{0.413320in}}%
\pgfpathlineto{\pgfqpoint{2.565045in}{0.413320in}}%
\pgfpathlineto{\pgfqpoint{2.562375in}{0.413320in}}%
\pgfpathlineto{\pgfqpoint{2.559790in}{0.413320in}}%
\pgfpathlineto{\pgfqpoint{2.557009in}{0.413320in}}%
\pgfpathlineto{\pgfqpoint{2.554493in}{0.413320in}}%
\pgfpathlineto{\pgfqpoint{2.551664in}{0.413320in}}%
\pgfpathlineto{\pgfqpoint{2.549114in}{0.413320in}}%
\pgfpathlineto{\pgfqpoint{2.546310in}{0.413320in}}%
\pgfpathlineto{\pgfqpoint{2.543765in}{0.413320in}}%
\pgfpathlineto{\pgfqpoint{2.540949in}{0.413320in}}%
\pgfpathlineto{\pgfqpoint{2.538274in}{0.413320in}}%
\pgfpathlineto{\pgfqpoint{2.535624in}{0.413320in}}%
\pgfpathlineto{\pgfqpoint{2.532917in}{0.413320in}}%
\pgfpathlineto{\pgfqpoint{2.530234in}{0.413320in}}%
\pgfpathlineto{\pgfqpoint{2.527560in}{0.413320in}}%
\pgfpathlineto{\pgfqpoint{2.524988in}{0.413320in}}%
\pgfpathlineto{\pgfqpoint{2.522197in}{0.413320in}}%
\pgfpathlineto{\pgfqpoint{2.519607in}{0.413320in}}%
\pgfpathlineto{\pgfqpoint{2.516845in}{0.413320in}}%
\pgfpathlineto{\pgfqpoint{2.514268in}{0.413320in}}%
\pgfpathlineto{\pgfqpoint{2.511478in}{0.413320in}}%
\pgfpathlineto{\pgfqpoint{2.508917in}{0.413320in}}%
\pgfpathlineto{\pgfqpoint{2.506163in}{0.413320in}}%
\pgfpathlineto{\pgfqpoint{2.503454in}{0.413320in}}%
\pgfpathlineto{\pgfqpoint{2.500801in}{0.413320in}}%
\pgfpathlineto{\pgfqpoint{2.498085in}{0.413320in}}%
\pgfpathlineto{\pgfqpoint{2.495542in}{0.413320in}}%
\pgfpathlineto{\pgfqpoint{2.492729in}{0.413320in}}%
\pgfpathlineto{\pgfqpoint{2.490183in}{0.413320in}}%
\pgfpathlineto{\pgfqpoint{2.487384in}{0.413320in}}%
\pgfpathlineto{\pgfqpoint{2.484870in}{0.413320in}}%
\pgfpathlineto{\pgfqpoint{2.482026in}{0.413320in}}%
\pgfpathlineto{\pgfqpoint{2.479420in}{0.413320in}}%
\pgfpathlineto{\pgfqpoint{2.476671in}{0.413320in}}%
\pgfpathlineto{\pgfqpoint{2.473989in}{0.413320in}}%
\pgfpathlineto{\pgfqpoint{2.471311in}{0.413320in}}%
\pgfpathlineto{\pgfqpoint{2.468635in}{0.413320in}}%
\pgfpathlineto{\pgfqpoint{2.465957in}{0.413320in}}%
\pgfpathlineto{\pgfqpoint{2.463280in}{0.413320in}}%
\pgfpathlineto{\pgfqpoint{2.460711in}{0.413320in}}%
\pgfpathlineto{\pgfqpoint{2.457917in}{0.413320in}}%
\pgfpathlineto{\pgfqpoint{2.455353in}{0.413320in}}%
\pgfpathlineto{\pgfqpoint{2.452562in}{0.413320in}}%
\pgfpathlineto{\pgfqpoint{2.450032in}{0.413320in}}%
\pgfpathlineto{\pgfqpoint{2.447209in}{0.413320in}}%
\pgfpathlineto{\pgfqpoint{2.444677in}{0.413320in}}%
\pgfpathlineto{\pgfqpoint{2.441876in}{0.413320in}}%
\pgfpathlineto{\pgfqpoint{2.439167in}{0.413320in}}%
\pgfpathlineto{\pgfqpoint{2.436518in}{0.413320in}}%
\pgfpathlineto{\pgfqpoint{2.433815in}{0.413320in}}%
\pgfpathlineto{\pgfqpoint{2.431251in}{0.413320in}}%
\pgfpathlineto{\pgfqpoint{2.428453in}{0.413320in}}%
\pgfpathlineto{\pgfqpoint{2.425878in}{0.413320in}}%
\pgfpathlineto{\pgfqpoint{2.423098in}{0.413320in}}%
\pgfpathlineto{\pgfqpoint{2.420528in}{0.413320in}}%
\pgfpathlineto{\pgfqpoint{2.417747in}{0.413320in}}%
\pgfpathlineto{\pgfqpoint{2.415184in}{0.413320in}}%
\pgfpathlineto{\pgfqpoint{2.412389in}{0.413320in}}%
\pgfpathlineto{\pgfqpoint{2.409699in}{0.413320in}}%
\pgfpathlineto{\pgfqpoint{2.407024in}{0.413320in}}%
\pgfpathlineto{\pgfqpoint{2.404352in}{0.413320in}}%
\pgfpathlineto{\pgfqpoint{2.401675in}{0.413320in}}%
\pgfpathlineto{\pgfqpoint{2.398995in}{0.413320in}}%
\pgfpathclose%
\pgfusepath{stroke,fill}%
\end{pgfscope}%
\begin{pgfscope}%
\pgfpathrectangle{\pgfqpoint{2.398995in}{0.319877in}}{\pgfqpoint{3.986877in}{1.993438in}} %
\pgfusepath{clip}%
\pgfsetbuttcap%
\pgfsetroundjoin%
\definecolor{currentfill}{rgb}{1.000000,1.000000,1.000000}%
\pgfsetfillcolor{currentfill}%
\pgfsetlinewidth{1.003750pt}%
\definecolor{currentstroke}{rgb}{0.938681,0.489016,0.196231}%
\pgfsetstrokecolor{currentstroke}%
\pgfsetdash{}{0pt}%
\pgfpathmoveto{\pgfqpoint{2.398995in}{0.413320in}}%
\pgfpathlineto{\pgfqpoint{2.398995in}{1.851764in}}%
\pgfpathlineto{\pgfqpoint{2.401675in}{1.847448in}}%
\pgfpathlineto{\pgfqpoint{2.404352in}{1.849869in}}%
\pgfpathlineto{\pgfqpoint{2.407024in}{1.850846in}}%
\pgfpathlineto{\pgfqpoint{2.409699in}{1.847389in}}%
\pgfpathlineto{\pgfqpoint{2.412389in}{1.853172in}}%
\pgfpathlineto{\pgfqpoint{2.415184in}{1.845413in}}%
\pgfpathlineto{\pgfqpoint{2.417747in}{1.846708in}}%
\pgfpathlineto{\pgfqpoint{2.420528in}{1.847814in}}%
\pgfpathlineto{\pgfqpoint{2.423098in}{1.850975in}}%
\pgfpathlineto{\pgfqpoint{2.425878in}{1.848276in}}%
\pgfpathlineto{\pgfqpoint{2.428453in}{1.842710in}}%
\pgfpathlineto{\pgfqpoint{2.431251in}{1.851091in}}%
\pgfpathlineto{\pgfqpoint{2.433815in}{1.846913in}}%
\pgfpathlineto{\pgfqpoint{2.436518in}{1.842731in}}%
\pgfpathlineto{\pgfqpoint{2.439167in}{1.841086in}}%
\pgfpathlineto{\pgfqpoint{2.441876in}{1.842669in}}%
\pgfpathlineto{\pgfqpoint{2.444677in}{1.843869in}}%
\pgfpathlineto{\pgfqpoint{2.447209in}{1.843321in}}%
\pgfpathlineto{\pgfqpoint{2.450032in}{1.849441in}}%
\pgfpathlineto{\pgfqpoint{2.452562in}{1.849704in}}%
\pgfpathlineto{\pgfqpoint{2.455353in}{1.845272in}}%
\pgfpathlineto{\pgfqpoint{2.457917in}{1.846565in}}%
\pgfpathlineto{\pgfqpoint{2.460711in}{1.845577in}}%
\pgfpathlineto{\pgfqpoint{2.463280in}{1.853066in}}%
\pgfpathlineto{\pgfqpoint{2.465957in}{1.850504in}}%
\pgfpathlineto{\pgfqpoint{2.468635in}{1.853797in}}%
\pgfpathlineto{\pgfqpoint{2.471311in}{1.849099in}}%
\pgfpathlineto{\pgfqpoint{2.473989in}{1.852677in}}%
\pgfpathlineto{\pgfqpoint{2.476671in}{1.848420in}}%
\pgfpathlineto{\pgfqpoint{2.479420in}{1.852488in}}%
\pgfpathlineto{\pgfqpoint{2.482026in}{1.839645in}}%
\pgfpathlineto{\pgfqpoint{2.484870in}{1.838072in}}%
\pgfpathlineto{\pgfqpoint{2.487384in}{1.842498in}}%
\pgfpathlineto{\pgfqpoint{2.490183in}{1.846607in}}%
\pgfpathlineto{\pgfqpoint{2.492729in}{1.851039in}}%
\pgfpathlineto{\pgfqpoint{2.495542in}{1.853707in}}%
\pgfpathlineto{\pgfqpoint{2.498085in}{1.856966in}}%
\pgfpathlineto{\pgfqpoint{2.500801in}{1.860733in}}%
\pgfpathlineto{\pgfqpoint{2.503454in}{1.853410in}}%
\pgfpathlineto{\pgfqpoint{2.506163in}{1.853348in}}%
\pgfpathlineto{\pgfqpoint{2.508917in}{1.856436in}}%
\pgfpathlineto{\pgfqpoint{2.511478in}{1.858163in}}%
\pgfpathlineto{\pgfqpoint{2.514268in}{1.854435in}}%
\pgfpathlineto{\pgfqpoint{2.516845in}{1.855643in}}%
\pgfpathlineto{\pgfqpoint{2.519607in}{1.849840in}}%
\pgfpathlineto{\pgfqpoint{2.522197in}{1.854107in}}%
\pgfpathlineto{\pgfqpoint{2.524988in}{1.855766in}}%
\pgfpathlineto{\pgfqpoint{2.527560in}{1.852912in}}%
\pgfpathlineto{\pgfqpoint{2.530234in}{1.860493in}}%
\pgfpathlineto{\pgfqpoint{2.532917in}{1.857190in}}%
\pgfpathlineto{\pgfqpoint{2.535624in}{1.855234in}}%
\pgfpathlineto{\pgfqpoint{2.538274in}{1.852858in}}%
\pgfpathlineto{\pgfqpoint{2.540949in}{1.856876in}}%
\pgfpathlineto{\pgfqpoint{2.543765in}{1.855102in}}%
\pgfpathlineto{\pgfqpoint{2.546310in}{1.855964in}}%
\pgfpathlineto{\pgfqpoint{2.549114in}{1.853159in}}%
\pgfpathlineto{\pgfqpoint{2.551664in}{1.848988in}}%
\pgfpathlineto{\pgfqpoint{2.554493in}{1.849674in}}%
\pgfpathlineto{\pgfqpoint{2.557009in}{1.852621in}}%
\pgfpathlineto{\pgfqpoint{2.559790in}{1.854354in}}%
\pgfpathlineto{\pgfqpoint{2.562375in}{1.850514in}}%
\pgfpathlineto{\pgfqpoint{2.565045in}{1.849395in}}%
\pgfpathlineto{\pgfqpoint{2.567730in}{1.846293in}}%
\pgfpathlineto{\pgfqpoint{2.570411in}{1.850806in}}%
\pgfpathlineto{\pgfqpoint{2.573082in}{1.880567in}}%
\pgfpathlineto{\pgfqpoint{2.575779in}{1.879092in}}%
\pgfpathlineto{\pgfqpoint{2.578567in}{1.851180in}}%
\pgfpathlineto{\pgfqpoint{2.581129in}{1.844445in}}%
\pgfpathlineto{\pgfqpoint{2.583913in}{1.846648in}}%
\pgfpathlineto{\pgfqpoint{2.586484in}{1.847682in}}%
\pgfpathlineto{\pgfqpoint{2.589248in}{1.850993in}}%
\pgfpathlineto{\pgfqpoint{2.591842in}{1.848935in}}%
\pgfpathlineto{\pgfqpoint{2.594630in}{1.843429in}}%
\pgfpathlineto{\pgfqpoint{2.597196in}{1.847258in}}%
\pgfpathlineto{\pgfqpoint{2.599920in}{1.839009in}}%
\pgfpathlineto{\pgfqpoint{2.602557in}{1.843787in}}%
\pgfpathlineto{\pgfqpoint{2.605232in}{1.842560in}}%
\pgfpathlineto{\pgfqpoint{2.608004in}{1.843704in}}%
\pgfpathlineto{\pgfqpoint{2.610588in}{1.844952in}}%
\pgfpathlineto{\pgfqpoint{2.613393in}{1.864141in}}%
\pgfpathlineto{\pgfqpoint{2.615934in}{1.951826in}}%
\pgfpathlineto{\pgfqpoint{2.618773in}{1.962357in}}%
\pgfpathlineto{\pgfqpoint{2.621304in}{1.929653in}}%
\pgfpathlineto{\pgfqpoint{2.624077in}{1.898241in}}%
\pgfpathlineto{\pgfqpoint{2.626653in}{1.889430in}}%
\pgfpathlineto{\pgfqpoint{2.629340in}{1.872026in}}%
\pgfpathlineto{\pgfqpoint{2.632018in}{1.871548in}}%
\pgfpathlineto{\pgfqpoint{2.634700in}{1.861446in}}%
\pgfpathlineto{\pgfqpoint{2.637369in}{1.857466in}}%
\pgfpathlineto{\pgfqpoint{2.640053in}{1.849237in}}%
\pgfpathlineto{\pgfqpoint{2.642827in}{1.848698in}}%
\pgfpathlineto{\pgfqpoint{2.645408in}{1.849172in}}%
\pgfpathlineto{\pgfqpoint{2.648196in}{1.853902in}}%
\pgfpathlineto{\pgfqpoint{2.650767in}{1.849673in}}%
\pgfpathlineto{\pgfqpoint{2.653567in}{1.852374in}}%
\pgfpathlineto{\pgfqpoint{2.656124in}{1.850602in}}%
\pgfpathlineto{\pgfqpoint{2.658942in}{1.852024in}}%
\pgfpathlineto{\pgfqpoint{2.661481in}{1.851275in}}%
\pgfpathlineto{\pgfqpoint{2.664151in}{1.845821in}}%
\pgfpathlineto{\pgfqpoint{2.666836in}{1.846507in}}%
\pgfpathlineto{\pgfqpoint{2.669506in}{1.845469in}}%
\pgfpathlineto{\pgfqpoint{2.672301in}{1.842926in}}%
\pgfpathlineto{\pgfqpoint{2.674873in}{1.850077in}}%
\pgfpathlineto{\pgfqpoint{2.677650in}{1.845320in}}%
\pgfpathlineto{\pgfqpoint{2.680224in}{1.853851in}}%
\pgfpathlineto{\pgfqpoint{2.683009in}{1.850391in}}%
\pgfpathlineto{\pgfqpoint{2.685586in}{1.848810in}}%
\pgfpathlineto{\pgfqpoint{2.688328in}{1.848221in}}%
\pgfpathlineto{\pgfqpoint{2.690940in}{1.846372in}}%
\pgfpathlineto{\pgfqpoint{2.693611in}{1.856147in}}%
\pgfpathlineto{\pgfqpoint{2.696293in}{1.848653in}}%
\pgfpathlineto{\pgfqpoint{2.698968in}{1.845373in}}%
\pgfpathlineto{\pgfqpoint{2.701657in}{1.846525in}}%
\pgfpathlineto{\pgfqpoint{2.704326in}{1.845145in}}%
\pgfpathlineto{\pgfqpoint{2.707125in}{1.843265in}}%
\pgfpathlineto{\pgfqpoint{2.709683in}{1.850610in}}%
\pgfpathlineto{\pgfqpoint{2.712477in}{1.841027in}}%
\pgfpathlineto{\pgfqpoint{2.715036in}{1.847162in}}%
\pgfpathlineto{\pgfqpoint{2.717773in}{1.844050in}}%
\pgfpathlineto{\pgfqpoint{2.720404in}{1.838072in}}%
\pgfpathlineto{\pgfqpoint{2.723211in}{1.838549in}}%
\pgfpathlineto{\pgfqpoint{2.725760in}{1.842294in}}%
\pgfpathlineto{\pgfqpoint{2.728439in}{1.855031in}}%
\pgfpathlineto{\pgfqpoint{2.731119in}{1.869622in}}%
\pgfpathlineto{\pgfqpoint{2.733798in}{1.860523in}}%
\pgfpathlineto{\pgfqpoint{2.736476in}{1.851033in}}%
\pgfpathlineto{\pgfqpoint{2.739155in}{1.842993in}}%
\pgfpathlineto{\pgfqpoint{2.741928in}{1.838072in}}%
\pgfpathlineto{\pgfqpoint{2.744510in}{1.844762in}}%
\pgfpathlineto{\pgfqpoint{2.747260in}{1.840159in}}%
\pgfpathlineto{\pgfqpoint{2.749868in}{1.838471in}}%
\pgfpathlineto{\pgfqpoint{2.752614in}{1.845365in}}%
\pgfpathlineto{\pgfqpoint{2.755224in}{1.842626in}}%
\pgfpathlineto{\pgfqpoint{2.758028in}{1.839544in}}%
\pgfpathlineto{\pgfqpoint{2.760581in}{1.848373in}}%
\pgfpathlineto{\pgfqpoint{2.763253in}{1.851812in}}%
\pgfpathlineto{\pgfqpoint{2.765935in}{1.849747in}}%
\pgfpathlineto{\pgfqpoint{2.768617in}{1.853152in}}%
\pgfpathlineto{\pgfqpoint{2.771373in}{1.843315in}}%
\pgfpathlineto{\pgfqpoint{2.773972in}{1.845648in}}%
\pgfpathlineto{\pgfqpoint{2.776767in}{1.843228in}}%
\pgfpathlineto{\pgfqpoint{2.779330in}{1.840907in}}%
\pgfpathlineto{\pgfqpoint{2.782113in}{1.838984in}}%
\pgfpathlineto{\pgfqpoint{2.784687in}{1.845160in}}%
\pgfpathlineto{\pgfqpoint{2.787468in}{1.845753in}}%
\pgfpathlineto{\pgfqpoint{2.790044in}{1.840769in}}%
\pgfpathlineto{\pgfqpoint{2.792721in}{1.847224in}}%
\pgfpathlineto{\pgfqpoint{2.795398in}{1.841739in}}%
\pgfpathlineto{\pgfqpoint{2.798070in}{1.841320in}}%
\pgfpathlineto{\pgfqpoint{2.800756in}{1.844034in}}%
\pgfpathlineto{\pgfqpoint{2.803435in}{1.838554in}}%
\pgfpathlineto{\pgfqpoint{2.806175in}{1.848803in}}%
\pgfpathlineto{\pgfqpoint{2.808792in}{1.846037in}}%
\pgfpathlineto{\pgfqpoint{2.811597in}{1.846987in}}%
\pgfpathlineto{\pgfqpoint{2.814141in}{1.846942in}}%
\pgfpathlineto{\pgfqpoint{2.816867in}{1.853306in}}%
\pgfpathlineto{\pgfqpoint{2.819506in}{1.846645in}}%
\pgfpathlineto{\pgfqpoint{2.822303in}{1.849129in}}%
\pgfpathlineto{\pgfqpoint{2.824851in}{1.850682in}}%
\pgfpathlineto{\pgfqpoint{2.827567in}{1.849980in}}%
\pgfpathlineto{\pgfqpoint{2.830219in}{1.848076in}}%
\pgfpathlineto{\pgfqpoint{2.832894in}{1.844686in}}%
\pgfpathlineto{\pgfqpoint{2.835698in}{1.848526in}}%
\pgfpathlineto{\pgfqpoint{2.838254in}{1.850923in}}%
\pgfpathlineto{\pgfqpoint{2.841055in}{1.842008in}}%
\pgfpathlineto{\pgfqpoint{2.843611in}{1.841571in}}%
\pgfpathlineto{\pgfqpoint{2.846408in}{1.843418in}}%
\pgfpathlineto{\pgfqpoint{2.848960in}{1.846110in}}%
\pgfpathlineto{\pgfqpoint{2.851793in}{1.854668in}}%
\pgfpathlineto{\pgfqpoint{2.854325in}{1.849991in}}%
\pgfpathlineto{\pgfqpoint{2.857003in}{1.857070in}}%
\pgfpathlineto{\pgfqpoint{2.859668in}{1.853605in}}%
\pgfpathlineto{\pgfqpoint{2.862402in}{1.852388in}}%
\pgfpathlineto{\pgfqpoint{2.865031in}{1.851358in}}%
\pgfpathlineto{\pgfqpoint{2.867713in}{1.850593in}}%
\pgfpathlineto{\pgfqpoint{2.870475in}{1.858838in}}%
\pgfpathlineto{\pgfqpoint{2.873074in}{1.859618in}}%
\pgfpathlineto{\pgfqpoint{2.875882in}{1.856341in}}%
\pgfpathlineto{\pgfqpoint{2.878431in}{1.853835in}}%
\pgfpathlineto{\pgfqpoint{2.881254in}{1.846583in}}%
\pgfpathlineto{\pgfqpoint{2.883780in}{1.849531in}}%
\pgfpathlineto{\pgfqpoint{2.886578in}{1.850769in}}%
\pgfpathlineto{\pgfqpoint{2.889145in}{1.860150in}}%
\pgfpathlineto{\pgfqpoint{2.891809in}{1.860429in}}%
\pgfpathlineto{\pgfqpoint{2.894487in}{1.852933in}}%
\pgfpathlineto{\pgfqpoint{2.897179in}{1.848363in}}%
\pgfpathlineto{\pgfqpoint{2.899858in}{1.849284in}}%
\pgfpathlineto{\pgfqpoint{2.902535in}{1.850971in}}%
\pgfpathlineto{\pgfqpoint{2.905341in}{1.851070in}}%
\pgfpathlineto{\pgfqpoint{2.907882in}{1.851917in}}%
\pgfpathlineto{\pgfqpoint{2.910631in}{1.849368in}}%
\pgfpathlineto{\pgfqpoint{2.913243in}{1.848652in}}%
\pgfpathlineto{\pgfqpoint{2.916061in}{1.851954in}}%
\pgfpathlineto{\pgfqpoint{2.918606in}{1.855647in}}%
\pgfpathlineto{\pgfqpoint{2.921363in}{1.852255in}}%
\pgfpathlineto{\pgfqpoint{2.923963in}{1.854945in}}%
\pgfpathlineto{\pgfqpoint{2.926655in}{1.850136in}}%
\pgfpathlineto{\pgfqpoint{2.929321in}{1.853609in}}%
\pgfpathlineto{\pgfqpoint{2.932033in}{1.855435in}}%
\pgfpathlineto{\pgfqpoint{2.934759in}{1.851304in}}%
\pgfpathlineto{\pgfqpoint{2.937352in}{1.858072in}}%
\pgfpathlineto{\pgfqpoint{2.940120in}{1.861051in}}%
\pgfpathlineto{\pgfqpoint{2.942711in}{1.862788in}}%
\pgfpathlineto{\pgfqpoint{2.945461in}{1.865035in}}%
\pgfpathlineto{\pgfqpoint{2.948068in}{1.855752in}}%
\pgfpathlineto{\pgfqpoint{2.950884in}{1.858267in}}%
\pgfpathlineto{\pgfqpoint{2.953422in}{1.854909in}}%
\pgfpathlineto{\pgfqpoint{2.956103in}{1.858263in}}%
\pgfpathlineto{\pgfqpoint{2.958782in}{1.854815in}}%
\pgfpathlineto{\pgfqpoint{2.961460in}{1.857890in}}%
\pgfpathlineto{\pgfqpoint{2.964127in}{1.845538in}}%
\pgfpathlineto{\pgfqpoint{2.966812in}{1.848791in}}%
\pgfpathlineto{\pgfqpoint{2.969599in}{1.850446in}}%
\pgfpathlineto{\pgfqpoint{2.972177in}{1.853968in}}%
\pgfpathlineto{\pgfqpoint{2.974972in}{1.849728in}}%
\pgfpathlineto{\pgfqpoint{2.977517in}{1.852000in}}%
\pgfpathlineto{\pgfqpoint{2.980341in}{1.854640in}}%
\pgfpathlineto{\pgfqpoint{2.982885in}{1.852445in}}%
\pgfpathlineto{\pgfqpoint{2.985666in}{1.849522in}}%
\pgfpathlineto{\pgfqpoint{2.988238in}{1.849744in}}%
\pgfpathlineto{\pgfqpoint{2.990978in}{1.847893in}}%
\pgfpathlineto{\pgfqpoint{2.993595in}{1.858780in}}%
\pgfpathlineto{\pgfqpoint{2.996300in}{1.860023in}}%
\pgfpathlineto{\pgfqpoint{2.999103in}{1.843833in}}%
\pgfpathlineto{\pgfqpoint{3.001635in}{1.844336in}}%
\pgfpathlineto{\pgfqpoint{3.004419in}{1.853075in}}%
\pgfpathlineto{\pgfqpoint{3.006993in}{1.854080in}}%
\pgfpathlineto{\pgfqpoint{3.009784in}{1.854564in}}%
\pgfpathlineto{\pgfqpoint{3.012351in}{1.857210in}}%
\pgfpathlineto{\pgfqpoint{3.015097in}{1.857314in}}%
\pgfpathlineto{\pgfqpoint{3.017707in}{1.857175in}}%
\pgfpathlineto{\pgfqpoint{3.020382in}{1.856267in}}%
\pgfpathlineto{\pgfqpoint{3.023058in}{1.859040in}}%
\pgfpathlineto{\pgfqpoint{3.025803in}{1.861555in}}%
\pgfpathlineto{\pgfqpoint{3.028412in}{1.854820in}}%
\pgfpathlineto{\pgfqpoint{3.031091in}{1.853231in}}%
\pgfpathlineto{\pgfqpoint{3.033921in}{1.853065in}}%
\pgfpathlineto{\pgfqpoint{3.036456in}{1.854214in}}%
\pgfpathlineto{\pgfqpoint{3.039262in}{1.850413in}}%
\pgfpathlineto{\pgfqpoint{3.041813in}{1.852716in}}%
\pgfpathlineto{\pgfqpoint{3.044568in}{1.847541in}}%
\pgfpathlineto{\pgfqpoint{3.047157in}{1.846776in}}%
\pgfpathlineto{\pgfqpoint{3.049988in}{1.844554in}}%
\pgfpathlineto{\pgfqpoint{3.052526in}{1.849979in}}%
\pgfpathlineto{\pgfqpoint{3.055202in}{1.849027in}}%
\pgfpathlineto{\pgfqpoint{3.057884in}{1.859271in}}%
\pgfpathlineto{\pgfqpoint{3.060561in}{1.864044in}}%
\pgfpathlineto{\pgfqpoint{3.063230in}{1.856444in}}%
\pgfpathlineto{\pgfqpoint{3.065916in}{1.855280in}}%
\pgfpathlineto{\pgfqpoint{3.068709in}{1.855641in}}%
\pgfpathlineto{\pgfqpoint{3.071266in}{1.850153in}}%
\pgfpathlineto{\pgfqpoint{3.074056in}{1.849259in}}%
\pgfpathlineto{\pgfqpoint{3.076631in}{1.851454in}}%
\pgfpathlineto{\pgfqpoint{3.079381in}{1.858453in}}%
\pgfpathlineto{\pgfqpoint{3.081990in}{1.861090in}}%
\pgfpathlineto{\pgfqpoint{3.084671in}{1.856640in}}%
\pgfpathlineto{\pgfqpoint{3.087343in}{1.844679in}}%
\pgfpathlineto{\pgfqpoint{3.090023in}{1.847280in}}%
\pgfpathlineto{\pgfqpoint{3.092699in}{1.843876in}}%
\pgfpathlineto{\pgfqpoint{3.095388in}{1.844747in}}%
\pgfpathlineto{\pgfqpoint{3.098163in}{1.845665in}}%
\pgfpathlineto{\pgfqpoint{3.100737in}{1.845705in}}%
\pgfpathlineto{\pgfqpoint{3.103508in}{1.842139in}}%
\pgfpathlineto{\pgfqpoint{3.106094in}{1.856787in}}%
\pgfpathlineto{\pgfqpoint{3.108896in}{1.862712in}}%
\pgfpathlineto{\pgfqpoint{3.111451in}{1.866581in}}%
\pgfpathlineto{\pgfqpoint{3.114242in}{1.853018in}}%
\pgfpathlineto{\pgfqpoint{3.116807in}{1.850980in}}%
\pgfpathlineto{\pgfqpoint{3.119487in}{1.851334in}}%
\pgfpathlineto{\pgfqpoint{3.122163in}{1.865623in}}%
\pgfpathlineto{\pgfqpoint{3.124842in}{1.861602in}}%
\pgfpathlineto{\pgfqpoint{3.127512in}{1.860029in}}%
\pgfpathlineto{\pgfqpoint{3.130199in}{1.857370in}}%
\pgfpathlineto{\pgfqpoint{3.132946in}{1.859171in}}%
\pgfpathlineto{\pgfqpoint{3.135550in}{1.864628in}}%
\pgfpathlineto{\pgfqpoint{3.138375in}{1.873272in}}%
\pgfpathlineto{\pgfqpoint{3.140913in}{1.862301in}}%
\pgfpathlineto{\pgfqpoint{3.143740in}{1.854662in}}%
\pgfpathlineto{\pgfqpoint{3.146271in}{1.857498in}}%
\pgfpathlineto{\pgfqpoint{3.149057in}{1.850422in}}%
\pgfpathlineto{\pgfqpoint{3.151612in}{1.851033in}}%
\pgfpathlineto{\pgfqpoint{3.154327in}{1.842693in}}%
\pgfpathlineto{\pgfqpoint{3.156981in}{1.860251in}}%
\pgfpathlineto{\pgfqpoint{3.159675in}{1.856110in}}%
\pgfpathlineto{\pgfqpoint{3.162474in}{1.855607in}}%
\pgfpathlineto{\pgfqpoint{3.165019in}{1.859170in}}%
\pgfpathlineto{\pgfqpoint{3.167776in}{1.858547in}}%
\pgfpathlineto{\pgfqpoint{3.170375in}{1.860891in}}%
\pgfpathlineto{\pgfqpoint{3.173142in}{1.855267in}}%
\pgfpathlineto{\pgfqpoint{3.175724in}{1.849967in}}%
\pgfpathlineto{\pgfqpoint{3.178525in}{1.849694in}}%
\pgfpathlineto{\pgfqpoint{3.181089in}{1.853515in}}%
\pgfpathlineto{\pgfqpoint{3.183760in}{1.838864in}}%
\pgfpathlineto{\pgfqpoint{3.186440in}{1.846653in}}%
\pgfpathlineto{\pgfqpoint{3.189117in}{1.846963in}}%
\pgfpathlineto{\pgfqpoint{3.191796in}{1.851398in}}%
\pgfpathlineto{\pgfqpoint{3.194508in}{1.851234in}}%
\pgfpathlineto{\pgfqpoint{3.197226in}{1.843881in}}%
\pgfpathlineto{\pgfqpoint{3.199823in}{1.854405in}}%
\pgfpathlineto{\pgfqpoint{3.202562in}{1.848162in}}%
\pgfpathlineto{\pgfqpoint{3.205195in}{1.859585in}}%
\pgfpathlineto{\pgfqpoint{3.207984in}{1.863841in}}%
\pgfpathlineto{\pgfqpoint{3.210545in}{1.873820in}}%
\pgfpathlineto{\pgfqpoint{3.213342in}{1.861170in}}%
\pgfpathlineto{\pgfqpoint{3.215908in}{1.858338in}}%
\pgfpathlineto{\pgfqpoint{3.218586in}{1.854869in}}%
\pgfpathlineto{\pgfqpoint{3.221255in}{1.856955in}}%
\pgfpathlineto{\pgfqpoint{3.223942in}{1.847671in}}%
\pgfpathlineto{\pgfqpoint{3.226609in}{1.849956in}}%
\pgfpathlineto{\pgfqpoint{3.229310in}{1.852924in}}%
\pgfpathlineto{\pgfqpoint{3.232069in}{1.864010in}}%
\pgfpathlineto{\pgfqpoint{3.234658in}{1.857867in}}%
\pgfpathlineto{\pgfqpoint{3.237411in}{1.856245in}}%
\pgfpathlineto{\pgfqpoint{3.240010in}{1.849049in}}%
\pgfpathlineto{\pgfqpoint{3.242807in}{1.852157in}}%
\pgfpathlineto{\pgfqpoint{3.245363in}{1.855453in}}%
\pgfpathlineto{\pgfqpoint{3.248049in}{1.854997in}}%
\pgfpathlineto{\pgfqpoint{3.250716in}{1.857134in}}%
\pgfpathlineto{\pgfqpoint{3.253404in}{1.852639in}}%
\pgfpathlineto{\pgfqpoint{3.256083in}{1.855138in}}%
\pgfpathlineto{\pgfqpoint{3.258784in}{1.853471in}}%
\pgfpathlineto{\pgfqpoint{3.261594in}{1.855102in}}%
\pgfpathlineto{\pgfqpoint{3.264119in}{1.858007in}}%
\pgfpathlineto{\pgfqpoint{3.266849in}{1.851568in}}%
\pgfpathlineto{\pgfqpoint{3.269478in}{1.852936in}}%
\pgfpathlineto{\pgfqpoint{3.272254in}{1.846984in}}%
\pgfpathlineto{\pgfqpoint{3.274831in}{1.843493in}}%
\pgfpathlineto{\pgfqpoint{3.277603in}{1.849936in}}%
\pgfpathlineto{\pgfqpoint{3.280189in}{1.848919in}}%
\pgfpathlineto{\pgfqpoint{3.282870in}{1.854688in}}%
\pgfpathlineto{\pgfqpoint{3.285534in}{1.850834in}}%
\pgfpathlineto{\pgfqpoint{3.288225in}{1.856589in}}%
\pgfpathlineto{\pgfqpoint{3.290890in}{1.851704in}}%
\pgfpathlineto{\pgfqpoint{3.293574in}{1.856754in}}%
\pgfpathlineto{\pgfqpoint{3.296376in}{1.854612in}}%
\pgfpathlineto{\pgfqpoint{3.298937in}{1.853431in}}%
\pgfpathlineto{\pgfqpoint{3.301719in}{1.848913in}}%
\pgfpathlineto{\pgfqpoint{3.304295in}{1.856690in}}%
\pgfpathlineto{\pgfqpoint{3.307104in}{1.851677in}}%
\pgfpathlineto{\pgfqpoint{3.309652in}{1.852858in}}%
\pgfpathlineto{\pgfqpoint{3.312480in}{1.852283in}}%
\pgfpathlineto{\pgfqpoint{3.315008in}{1.859312in}}%
\pgfpathlineto{\pgfqpoint{3.317688in}{1.848264in}}%
\pgfpathlineto{\pgfqpoint{3.320366in}{1.854461in}}%
\pgfpathlineto{\pgfqpoint{3.323049in}{1.854330in}}%
\pgfpathlineto{\pgfqpoint{3.325860in}{1.855457in}}%
\pgfpathlineto{\pgfqpoint{3.328401in}{1.850956in}}%
\pgfpathlineto{\pgfqpoint{3.331183in}{1.849386in}}%
\pgfpathlineto{\pgfqpoint{3.333758in}{1.851442in}}%
\pgfpathlineto{\pgfqpoint{3.336541in}{1.851206in}}%
\pgfpathlineto{\pgfqpoint{3.339101in}{1.853768in}}%
\pgfpathlineto{\pgfqpoint{3.341893in}{1.848007in}}%
\pgfpathlineto{\pgfqpoint{3.344468in}{1.847119in}}%
\pgfpathlineto{\pgfqpoint{3.347139in}{1.853533in}}%
\pgfpathlineto{\pgfqpoint{3.349828in}{1.858130in}}%
\pgfpathlineto{\pgfqpoint{3.352505in}{1.861271in}}%
\pgfpathlineto{\pgfqpoint{3.355177in}{1.853499in}}%
\pgfpathlineto{\pgfqpoint{3.357862in}{1.854892in}}%
\pgfpathlineto{\pgfqpoint{3.360620in}{1.849975in}}%
\pgfpathlineto{\pgfqpoint{3.363221in}{1.853435in}}%
\pgfpathlineto{\pgfqpoint{3.365996in}{1.858005in}}%
\pgfpathlineto{\pgfqpoint{3.368577in}{1.856792in}}%
\pgfpathlineto{\pgfqpoint{3.371357in}{1.847962in}}%
\pgfpathlineto{\pgfqpoint{3.373921in}{1.853526in}}%
\pgfpathlineto{\pgfqpoint{3.376735in}{1.854643in}}%
\pgfpathlineto{\pgfqpoint{3.379290in}{1.849795in}}%
\pgfpathlineto{\pgfqpoint{3.381959in}{1.856442in}}%
\pgfpathlineto{\pgfqpoint{3.384647in}{1.858050in}}%
\pgfpathlineto{\pgfqpoint{3.387309in}{1.852396in}}%
\pgfpathlineto{\pgfqpoint{3.390102in}{1.857195in}}%
\pgfpathlineto{\pgfqpoint{3.392681in}{1.857763in}}%
\pgfpathlineto{\pgfqpoint{3.395461in}{1.853031in}}%
\pgfpathlineto{\pgfqpoint{3.398037in}{1.860069in}}%
\pgfpathlineto{\pgfqpoint{3.400783in}{1.852114in}}%
\pgfpathlineto{\pgfqpoint{3.403394in}{1.858848in}}%
\pgfpathlineto{\pgfqpoint{3.406202in}{1.854162in}}%
\pgfpathlineto{\pgfqpoint{3.408752in}{1.852130in}}%
\pgfpathlineto{\pgfqpoint{3.411431in}{1.854339in}}%
\pgfpathlineto{\pgfqpoint{3.414109in}{1.852811in}}%
\pgfpathlineto{\pgfqpoint{3.416780in}{1.857199in}}%
\pgfpathlineto{\pgfqpoint{3.419455in}{1.848985in}}%
\pgfpathlineto{\pgfqpoint{3.422142in}{1.847260in}}%
\pgfpathlineto{\pgfqpoint{3.424887in}{1.855775in}}%
\pgfpathlineto{\pgfqpoint{3.427501in}{1.852126in}}%
\pgfpathlineto{\pgfqpoint{3.430313in}{1.853677in}}%
\pgfpathlineto{\pgfqpoint{3.432851in}{1.849286in}}%
\pgfpathlineto{\pgfqpoint{3.435635in}{1.858391in}}%
\pgfpathlineto{\pgfqpoint{3.438210in}{1.851216in}}%
\pgfpathlineto{\pgfqpoint{3.440996in}{1.850630in}}%
\pgfpathlineto{\pgfqpoint{3.443574in}{1.851995in}}%
\pgfpathlineto{\pgfqpoint{3.446257in}{1.853917in}}%
\pgfpathlineto{\pgfqpoint{3.448926in}{1.850848in}}%
\pgfpathlineto{\pgfqpoint{3.451597in}{1.852647in}}%
\pgfpathlineto{\pgfqpoint{3.454285in}{1.851736in}}%
\pgfpathlineto{\pgfqpoint{3.456960in}{1.850283in}}%
\pgfpathlineto{\pgfqpoint{3.459695in}{1.860054in}}%
\pgfpathlineto{\pgfqpoint{3.462321in}{1.856474in}}%
\pgfpathlineto{\pgfqpoint{3.465072in}{1.854010in}}%
\pgfpathlineto{\pgfqpoint{3.467678in}{1.850202in}}%
\pgfpathlineto{\pgfqpoint{3.470466in}{1.849745in}}%
\pgfpathlineto{\pgfqpoint{3.473021in}{1.858486in}}%
\pgfpathlineto{\pgfqpoint{3.475821in}{1.856235in}}%
\pgfpathlineto{\pgfqpoint{3.478378in}{1.850405in}}%
\pgfpathlineto{\pgfqpoint{3.481072in}{1.852545in}}%
\pgfpathlineto{\pgfqpoint{3.483744in}{1.855886in}}%
\pgfpathlineto{\pgfqpoint{3.486442in}{1.847347in}}%
\pgfpathlineto{\pgfqpoint{3.489223in}{1.848156in}}%
\pgfpathlineto{\pgfqpoint{3.491783in}{1.852832in}}%
\pgfpathlineto{\pgfqpoint{3.494581in}{1.855960in}}%
\pgfpathlineto{\pgfqpoint{3.497139in}{1.854556in}}%
\pgfpathlineto{\pgfqpoint{3.499909in}{1.849414in}}%
\pgfpathlineto{\pgfqpoint{3.502488in}{1.851922in}}%
\pgfpathlineto{\pgfqpoint{3.505262in}{1.848113in}}%
\pgfpathlineto{\pgfqpoint{3.507840in}{1.853702in}}%
\pgfpathlineto{\pgfqpoint{3.510533in}{1.848791in}}%
\pgfpathlineto{\pgfqpoint{3.513209in}{1.856297in}}%
\pgfpathlineto{\pgfqpoint{3.515884in}{1.855173in}}%
\pgfpathlineto{\pgfqpoint{3.518565in}{1.855006in}}%
\pgfpathlineto{\pgfqpoint{3.521244in}{1.859155in}}%
\pgfpathlineto{\pgfqpoint{3.524041in}{1.855069in}}%
\pgfpathlineto{\pgfqpoint{3.526601in}{1.857961in}}%
\pgfpathlineto{\pgfqpoint{3.529327in}{1.857810in}}%
\pgfpathlineto{\pgfqpoint{3.531955in}{1.857485in}}%
\pgfpathlineto{\pgfqpoint{3.534783in}{1.849442in}}%
\pgfpathlineto{\pgfqpoint{3.537309in}{1.849276in}}%
\pgfpathlineto{\pgfqpoint{3.540093in}{1.849082in}}%
\pgfpathlineto{\pgfqpoint{3.542656in}{1.850515in}}%
\pgfpathlineto{\pgfqpoint{3.545349in}{1.855477in}}%
\pgfpathlineto{\pgfqpoint{3.548029in}{1.855485in}}%
\pgfpathlineto{\pgfqpoint{3.550713in}{1.864995in}}%
\pgfpathlineto{\pgfqpoint{3.553498in}{1.858839in}}%
\pgfpathlineto{\pgfqpoint{3.556061in}{1.859952in}}%
\pgfpathlineto{\pgfqpoint{3.558853in}{1.853249in}}%
\pgfpathlineto{\pgfqpoint{3.561420in}{1.855997in}}%
\pgfpathlineto{\pgfqpoint{3.564188in}{1.860537in}}%
\pgfpathlineto{\pgfqpoint{3.566774in}{1.860899in}}%
\pgfpathlineto{\pgfqpoint{3.569584in}{1.851220in}}%
\pgfpathlineto{\pgfqpoint{3.572126in}{1.853611in}}%
\pgfpathlineto{\pgfqpoint{3.574814in}{1.850962in}}%
\pgfpathlineto{\pgfqpoint{3.577487in}{1.848175in}}%
\pgfpathlineto{\pgfqpoint{3.580191in}{1.853401in}}%
\pgfpathlineto{\pgfqpoint{3.582851in}{1.852579in}}%
\pgfpathlineto{\pgfqpoint{3.585532in}{1.853306in}}%
\pgfpathlineto{\pgfqpoint{3.588258in}{1.851730in}}%
\pgfpathlineto{\pgfqpoint{3.590883in}{1.856505in}}%
\pgfpathlineto{\pgfqpoint{3.593620in}{1.857701in}}%
\pgfpathlineto{\pgfqpoint{3.596240in}{1.848642in}}%
\pgfpathlineto{\pgfqpoint{3.598998in}{1.852508in}}%
\pgfpathlineto{\pgfqpoint{3.601590in}{1.853319in}}%
\pgfpathlineto{\pgfqpoint{3.604387in}{1.854880in}}%
\pgfpathlineto{\pgfqpoint{3.606951in}{1.860243in}}%
\pgfpathlineto{\pgfqpoint{3.609632in}{1.854929in}}%
\pgfpathlineto{\pgfqpoint{3.612311in}{1.848640in}}%
\pgfpathlineto{\pgfqpoint{3.614982in}{1.850323in}}%
\pgfpathlineto{\pgfqpoint{3.617667in}{1.853490in}}%
\pgfpathlineto{\pgfqpoint{3.620345in}{1.852462in}}%
\pgfpathlineto{\pgfqpoint{3.623165in}{1.850917in}}%
\pgfpathlineto{\pgfqpoint{3.625689in}{1.854081in}}%
\pgfpathlineto{\pgfqpoint{3.628460in}{1.850466in}}%
\pgfpathlineto{\pgfqpoint{3.631058in}{1.851638in}}%
\pgfpathlineto{\pgfqpoint{3.633858in}{1.854656in}}%
\pgfpathlineto{\pgfqpoint{3.636413in}{1.850637in}}%
\pgfpathlineto{\pgfqpoint{3.639207in}{1.851693in}}%
\pgfpathlineto{\pgfqpoint{3.641773in}{1.856170in}}%
\pgfpathlineto{\pgfqpoint{3.644452in}{1.866154in}}%
\pgfpathlineto{\pgfqpoint{3.647130in}{1.856360in}}%
\pgfpathlineto{\pgfqpoint{3.649837in}{1.850931in}}%
\pgfpathlineto{\pgfqpoint{3.652628in}{1.851696in}}%
\pgfpathlineto{\pgfqpoint{3.655165in}{1.850356in}}%
\pgfpathlineto{\pgfqpoint{3.657917in}{1.851326in}}%
\pgfpathlineto{\pgfqpoint{3.660515in}{1.851453in}}%
\pgfpathlineto{\pgfqpoint{3.663276in}{1.870663in}}%
\pgfpathlineto{\pgfqpoint{3.665864in}{1.854754in}}%
\pgfpathlineto{\pgfqpoint{3.668665in}{1.850722in}}%
\pgfpathlineto{\pgfqpoint{3.671232in}{1.855799in}}%
\pgfpathlineto{\pgfqpoint{3.673911in}{1.856171in}}%
\pgfpathlineto{\pgfqpoint{3.676591in}{1.862642in}}%
\pgfpathlineto{\pgfqpoint{3.679273in}{1.867181in}}%
\pgfpathlineto{\pgfqpoint{3.681948in}{1.864742in}}%
\pgfpathlineto{\pgfqpoint{3.684620in}{1.852378in}}%
\pgfpathlineto{\pgfqpoint{3.687442in}{1.845107in}}%
\pgfpathlineto{\pgfqpoint{3.689983in}{1.846406in}}%
\pgfpathlineto{\pgfqpoint{3.692765in}{1.845552in}}%
\pgfpathlineto{\pgfqpoint{3.695331in}{1.851922in}}%
\pgfpathlineto{\pgfqpoint{3.698125in}{1.848061in}}%
\pgfpathlineto{\pgfqpoint{3.700684in}{1.855265in}}%
\pgfpathlineto{\pgfqpoint{3.703460in}{1.853995in}}%
\pgfpathlineto{\pgfqpoint{3.706053in}{1.858383in}}%
\pgfpathlineto{\pgfqpoint{3.708729in}{1.867729in}}%
\pgfpathlineto{\pgfqpoint{3.711410in}{1.857154in}}%
\pgfpathlineto{\pgfqpoint{3.714086in}{1.853662in}}%
\pgfpathlineto{\pgfqpoint{3.716875in}{1.855455in}}%
\pgfpathlineto{\pgfqpoint{3.719446in}{1.858227in}}%
\pgfpathlineto{\pgfqpoint{3.722228in}{1.863447in}}%
\pgfpathlineto{\pgfqpoint{3.724804in}{1.845850in}}%
\pgfpathlineto{\pgfqpoint{3.727581in}{1.848039in}}%
\pgfpathlineto{\pgfqpoint{3.730158in}{1.846721in}}%
\pgfpathlineto{\pgfqpoint{3.732950in}{1.849949in}}%
\pgfpathlineto{\pgfqpoint{3.735509in}{1.849906in}}%
\pgfpathlineto{\pgfqpoint{3.738194in}{1.850993in}}%
\pgfpathlineto{\pgfqpoint{3.740874in}{1.856452in}}%
\pgfpathlineto{\pgfqpoint{3.743548in}{1.850858in}}%
\pgfpathlineto{\pgfqpoint{3.746229in}{1.853682in}}%
\pgfpathlineto{\pgfqpoint{3.748903in}{1.856842in}}%
\pgfpathlineto{\pgfqpoint{3.751728in}{1.857163in}}%
\pgfpathlineto{\pgfqpoint{3.754265in}{1.852531in}}%
\pgfpathlineto{\pgfqpoint{3.757065in}{1.856541in}}%
\pgfpathlineto{\pgfqpoint{3.759622in}{1.857935in}}%
\pgfpathlineto{\pgfqpoint{3.762389in}{1.857753in}}%
\pgfpathlineto{\pgfqpoint{3.764966in}{1.852124in}}%
\pgfpathlineto{\pgfqpoint{3.767782in}{1.860453in}}%
\pgfpathlineto{\pgfqpoint{3.770323in}{1.843679in}}%
\pgfpathlineto{\pgfqpoint{3.773014in}{1.847158in}}%
\pgfpathlineto{\pgfqpoint{3.775691in}{1.842623in}}%
\pgfpathlineto{\pgfqpoint{3.778370in}{1.845135in}}%
\pgfpathlineto{\pgfqpoint{3.781046in}{1.854206in}}%
\pgfpathlineto{\pgfqpoint{3.783725in}{1.858547in}}%
\pgfpathlineto{\pgfqpoint{3.786504in}{1.854274in}}%
\pgfpathlineto{\pgfqpoint{3.789084in}{1.844953in}}%
\pgfpathlineto{\pgfqpoint{3.791897in}{1.847476in}}%
\pgfpathlineto{\pgfqpoint{3.794435in}{1.844258in}}%
\pgfpathlineto{\pgfqpoint{3.797265in}{1.847848in}}%
\pgfpathlineto{\pgfqpoint{3.799797in}{1.845101in}}%
\pgfpathlineto{\pgfqpoint{3.802569in}{1.851066in}}%
\pgfpathlineto{\pgfqpoint{3.805145in}{1.853080in}}%
\pgfpathlineto{\pgfqpoint{3.807832in}{1.854682in}}%
\pgfpathlineto{\pgfqpoint{3.810510in}{1.853062in}}%
\pgfpathlineto{\pgfqpoint{3.813172in}{1.856414in}}%
\pgfpathlineto{\pgfqpoint{3.815983in}{1.863856in}}%
\pgfpathlineto{\pgfqpoint{3.818546in}{1.860266in}}%
\pgfpathlineto{\pgfqpoint{3.821315in}{1.855323in}}%
\pgfpathlineto{\pgfqpoint{3.823903in}{1.853769in}}%
\pgfpathlineto{\pgfqpoint{3.826679in}{1.853389in}}%
\pgfpathlineto{\pgfqpoint{3.829252in}{1.849341in}}%
\pgfpathlineto{\pgfqpoint{3.832053in}{1.851235in}}%
\pgfpathlineto{\pgfqpoint{3.834616in}{1.850570in}}%
\pgfpathlineto{\pgfqpoint{3.837286in}{1.850159in}}%
\pgfpathlineto{\pgfqpoint{3.839960in}{1.846360in}}%
\pgfpathlineto{\pgfqpoint{3.842641in}{1.854889in}}%
\pgfpathlineto{\pgfqpoint{3.845329in}{1.856155in}}%
\pgfpathlineto{\pgfqpoint{3.848005in}{1.857095in}}%
\pgfpathlineto{\pgfqpoint{3.850814in}{1.862017in}}%
\pgfpathlineto{\pgfqpoint{3.853358in}{1.860415in}}%
\pgfpathlineto{\pgfqpoint{3.856100in}{1.860472in}}%
\pgfpathlineto{\pgfqpoint{3.858720in}{1.868781in}}%
\pgfpathlineto{\pgfqpoint{3.861561in}{1.862719in}}%
\pgfpathlineto{\pgfqpoint{3.864073in}{1.857107in}}%
\pgfpathlineto{\pgfqpoint{3.866815in}{1.855858in}}%
\pgfpathlineto{\pgfqpoint{3.869435in}{1.857989in}}%
\pgfpathlineto{\pgfqpoint{3.872114in}{1.862819in}}%
\pgfpathlineto{\pgfqpoint{3.874790in}{1.863445in}}%
\pgfpathlineto{\pgfqpoint{3.877466in}{1.852386in}}%
\pgfpathlineto{\pgfqpoint{3.880237in}{1.851042in}}%
\pgfpathlineto{\pgfqpoint{3.882850in}{1.845041in}}%
\pgfpathlineto{\pgfqpoint{3.885621in}{1.847377in}}%
\pgfpathlineto{\pgfqpoint{3.888188in}{1.862023in}}%
\pgfpathlineto{\pgfqpoint{3.890926in}{1.868582in}}%
\pgfpathlineto{\pgfqpoint{3.893541in}{1.876999in}}%
\pgfpathlineto{\pgfqpoint{3.896345in}{1.872309in}}%
\pgfpathlineto{\pgfqpoint{3.898891in}{1.860514in}}%
\pgfpathlineto{\pgfqpoint{3.901573in}{1.856406in}}%
\pgfpathlineto{\pgfqpoint{3.904252in}{1.851855in}}%
\pgfpathlineto{\pgfqpoint{3.906918in}{1.854230in}}%
\pgfpathlineto{\pgfqpoint{3.909602in}{1.853300in}}%
\pgfpathlineto{\pgfqpoint{3.912296in}{1.853419in}}%
\pgfpathlineto{\pgfqpoint{3.915107in}{1.852759in}}%
\pgfpathlineto{\pgfqpoint{3.917646in}{1.855399in}}%
\pgfpathlineto{\pgfqpoint{3.920412in}{1.865412in}}%
\pgfpathlineto{\pgfqpoint{3.923005in}{1.852783in}}%
\pgfpathlineto{\pgfqpoint{3.925778in}{1.860009in}}%
\pgfpathlineto{\pgfqpoint{3.928347in}{1.854834in}}%
\pgfpathlineto{\pgfqpoint{3.931202in}{1.854747in}}%
\pgfpathlineto{\pgfqpoint{3.933714in}{1.856237in}}%
\pgfpathlineto{\pgfqpoint{3.936395in}{1.858499in}}%
\pgfpathlineto{\pgfqpoint{3.939075in}{1.844614in}}%
\pgfpathlineto{\pgfqpoint{3.941778in}{1.838121in}}%
\pgfpathlineto{\pgfqpoint{3.944431in}{1.838072in}}%
\pgfpathlineto{\pgfqpoint{3.947101in}{1.842766in}}%
\pgfpathlineto{\pgfqpoint{3.949894in}{1.848095in}}%
\pgfpathlineto{\pgfqpoint{3.952464in}{1.846463in}}%
\pgfpathlineto{\pgfqpoint{3.955211in}{1.853397in}}%
\pgfpathlineto{\pgfqpoint{3.957823in}{1.859579in}}%
\pgfpathlineto{\pgfqpoint{3.960635in}{1.857855in}}%
\pgfpathlineto{\pgfqpoint{3.963176in}{1.855308in}}%
\pgfpathlineto{\pgfqpoint{3.966013in}{1.859221in}}%
\pgfpathlineto{\pgfqpoint{3.968523in}{1.852974in}}%
\pgfpathlineto{\pgfqpoint{3.971250in}{1.850186in}}%
\pgfpathlineto{\pgfqpoint{3.973885in}{1.849369in}}%
\pgfpathlineto{\pgfqpoint{3.976563in}{1.852898in}}%
\pgfpathlineto{\pgfqpoint{3.979389in}{1.853708in}}%
\pgfpathlineto{\pgfqpoint{3.981929in}{1.853110in}}%
\pgfpathlineto{\pgfqpoint{3.984714in}{1.852130in}}%
\pgfpathlineto{\pgfqpoint{3.987270in}{1.854411in}}%
\pgfpathlineto{\pgfqpoint{3.990055in}{1.852738in}}%
\pgfpathlineto{\pgfqpoint{3.992642in}{1.848717in}}%
\pgfpathlineto{\pgfqpoint{3.995417in}{1.855051in}}%
\pgfpathlineto{\pgfqpoint{3.997990in}{1.852211in}}%
\pgfpathlineto{\pgfqpoint{4.000674in}{1.853762in}}%
\pgfpathlineto{\pgfqpoint{4.003348in}{1.850378in}}%
\pgfpathlineto{\pgfqpoint{4.006034in}{1.852000in}}%
\pgfpathlineto{\pgfqpoint{4.008699in}{1.853562in}}%
\pgfpathlineto{\pgfqpoint{4.011394in}{1.851046in}}%
\pgfpathlineto{\pgfqpoint{4.014186in}{1.851866in}}%
\pgfpathlineto{\pgfqpoint{4.016744in}{1.854235in}}%
\pgfpathlineto{\pgfqpoint{4.019518in}{1.861738in}}%
\pgfpathlineto{\pgfqpoint{4.022097in}{1.855464in}}%
\pgfpathlineto{\pgfqpoint{4.024868in}{1.862125in}}%
\pgfpathlineto{\pgfqpoint{4.027447in}{1.857793in}}%
\pgfpathlineto{\pgfqpoint{4.030229in}{1.855853in}}%
\pgfpathlineto{\pgfqpoint{4.032817in}{1.851300in}}%
\pgfpathlineto{\pgfqpoint{4.035492in}{1.855867in}}%
\pgfpathlineto{\pgfqpoint{4.038174in}{1.856429in}}%
\pgfpathlineto{\pgfqpoint{4.040852in}{1.855642in}}%
\pgfpathlineto{\pgfqpoint{4.043667in}{1.857894in}}%
\pgfpathlineto{\pgfqpoint{4.046210in}{1.852120in}}%
\pgfpathlineto{\pgfqpoint{4.049006in}{1.854799in}}%
\pgfpathlineto{\pgfqpoint{4.051557in}{1.857994in}}%
\pgfpathlineto{\pgfqpoint{4.054326in}{1.860013in}}%
\pgfpathlineto{\pgfqpoint{4.056911in}{1.849612in}}%
\pgfpathlineto{\pgfqpoint{4.059702in}{1.854971in}}%
\pgfpathlineto{\pgfqpoint{4.062266in}{1.856187in}}%
\pgfpathlineto{\pgfqpoint{4.064957in}{1.850454in}}%
\pgfpathlineto{\pgfqpoint{4.067636in}{1.854172in}}%
\pgfpathlineto{\pgfqpoint{4.070313in}{1.853795in}}%
\pgfpathlineto{\pgfqpoint{4.072985in}{1.861687in}}%
\pgfpathlineto{\pgfqpoint{4.075705in}{1.859992in}}%
\pgfpathlineto{\pgfqpoint{4.078471in}{1.951316in}}%
\pgfpathlineto{\pgfqpoint{4.081018in}{1.994754in}}%
\pgfpathlineto{\pgfqpoint{4.083870in}{1.987111in}}%
\pgfpathlineto{\pgfqpoint{4.086385in}{1.990732in}}%
\pgfpathlineto{\pgfqpoint{4.089159in}{1.977774in}}%
\pgfpathlineto{\pgfqpoint{4.091729in}{1.956899in}}%
\pgfpathlineto{\pgfqpoint{4.094527in}{1.941335in}}%
\pgfpathlineto{\pgfqpoint{4.097092in}{1.936557in}}%
\pgfpathlineto{\pgfqpoint{4.099777in}{1.914742in}}%
\pgfpathlineto{\pgfqpoint{4.102456in}{1.897592in}}%
\pgfpathlineto{\pgfqpoint{4.105185in}{1.879075in}}%
\pgfpathlineto{\pgfqpoint{4.107814in}{1.877344in}}%
\pgfpathlineto{\pgfqpoint{4.110488in}{1.880865in}}%
\pgfpathlineto{\pgfqpoint{4.113252in}{1.870804in}}%
\pgfpathlineto{\pgfqpoint{4.115844in}{1.869403in}}%
\pgfpathlineto{\pgfqpoint{4.118554in}{1.864358in}}%
\pgfpathlineto{\pgfqpoint{4.121205in}{1.858692in}}%
\pgfpathlineto{\pgfqpoint{4.124019in}{1.840248in}}%
\pgfpathlineto{\pgfqpoint{4.126553in}{1.838497in}}%
\pgfpathlineto{\pgfqpoint{4.129349in}{1.850135in}}%
\pgfpathlineto{\pgfqpoint{4.131920in}{1.848169in}}%
\pgfpathlineto{\pgfqpoint{4.134615in}{1.849169in}}%
\pgfpathlineto{\pgfqpoint{4.137272in}{1.846409in}}%
\pgfpathlineto{\pgfqpoint{4.139963in}{1.842470in}}%
\pgfpathlineto{\pgfqpoint{4.142713in}{1.839163in}}%
\pgfpathlineto{\pgfqpoint{4.145310in}{1.846257in}}%
\pgfpathlineto{\pgfqpoint{4.148082in}{1.843863in}}%
\pgfpathlineto{\pgfqpoint{4.150665in}{1.840889in}}%
\pgfpathlineto{\pgfqpoint{4.153423in}{1.849368in}}%
\pgfpathlineto{\pgfqpoint{4.156016in}{1.843581in}}%
\pgfpathlineto{\pgfqpoint{4.158806in}{1.845840in}}%
\pgfpathlineto{\pgfqpoint{4.161380in}{1.847635in}}%
\pgfpathlineto{\pgfqpoint{4.164059in}{1.846964in}}%
\pgfpathlineto{\pgfqpoint{4.166737in}{1.838072in}}%
\pgfpathlineto{\pgfqpoint{4.169415in}{1.838072in}}%
\pgfpathlineto{\pgfqpoint{4.172093in}{1.838072in}}%
\pgfpathlineto{\pgfqpoint{4.174770in}{1.838072in}}%
\pgfpathlineto{\pgfqpoint{4.177593in}{1.849087in}}%
\pgfpathlineto{\pgfqpoint{4.180129in}{1.852159in}}%
\pgfpathlineto{\pgfqpoint{4.182899in}{1.852024in}}%
\pgfpathlineto{\pgfqpoint{4.185481in}{1.848973in}}%
\pgfpathlineto{\pgfqpoint{4.188318in}{1.848357in}}%
\pgfpathlineto{\pgfqpoint{4.190842in}{1.843984in}}%
\pgfpathlineto{\pgfqpoint{4.193638in}{1.844681in}}%
\pgfpathlineto{\pgfqpoint{4.196186in}{1.850762in}}%
\pgfpathlineto{\pgfqpoint{4.198878in}{1.852377in}}%
\pgfpathlineto{\pgfqpoint{4.201542in}{1.849624in}}%
\pgfpathlineto{\pgfqpoint{4.204240in}{1.855185in}}%
\pgfpathlineto{\pgfqpoint{4.207076in}{1.850877in}}%
\pgfpathlineto{\pgfqpoint{4.209597in}{1.848384in}}%
\pgfpathlineto{\pgfqpoint{4.212383in}{1.844922in}}%
\pgfpathlineto{\pgfqpoint{4.214948in}{1.849408in}}%
\pgfpathlineto{\pgfqpoint{4.217694in}{1.851192in}}%
\pgfpathlineto{\pgfqpoint{4.220304in}{1.857375in}}%
\pgfpathlineto{\pgfqpoint{4.223082in}{1.853261in}}%
\pgfpathlineto{\pgfqpoint{4.225654in}{1.854952in}}%
\pgfpathlineto{\pgfqpoint{4.228331in}{1.853282in}}%
\pgfpathlineto{\pgfqpoint{4.231013in}{1.852666in}}%
\pgfpathlineto{\pgfqpoint{4.233691in}{1.852321in}}%
\pgfpathlineto{\pgfqpoint{4.236375in}{1.851932in}}%
\pgfpathlineto{\pgfqpoint{4.239084in}{1.855669in}}%
\pgfpathlineto{\pgfqpoint{4.241900in}{1.857427in}}%
\pgfpathlineto{\pgfqpoint{4.244394in}{1.858228in}}%
\pgfpathlineto{\pgfqpoint{4.247225in}{1.857438in}}%
\pgfpathlineto{\pgfqpoint{4.249767in}{1.851490in}}%
\pgfpathlineto{\pgfqpoint{4.252581in}{1.856498in}}%
\pgfpathlineto{\pgfqpoint{4.255120in}{1.856796in}}%
\pgfpathlineto{\pgfqpoint{4.257958in}{1.859889in}}%
\pgfpathlineto{\pgfqpoint{4.260477in}{1.858651in}}%
\pgfpathlineto{\pgfqpoint{4.263157in}{1.854310in}}%
\pgfpathlineto{\pgfqpoint{4.265824in}{1.852919in}}%
\pgfpathlineto{\pgfqpoint{4.268590in}{1.850235in}}%
\pgfpathlineto{\pgfqpoint{4.271187in}{1.852448in}}%
\pgfpathlineto{\pgfqpoint{4.273874in}{1.854289in}}%
\pgfpathlineto{\pgfqpoint{4.276635in}{1.859053in}}%
\pgfpathlineto{\pgfqpoint{4.279212in}{1.867002in}}%
\pgfpathlineto{\pgfqpoint{4.282000in}{1.873113in}}%
\pgfpathlineto{\pgfqpoint{4.284586in}{1.855855in}}%
\pgfpathlineto{\pgfqpoint{4.287399in}{1.856706in}}%
\pgfpathlineto{\pgfqpoint{4.289936in}{1.856742in}}%
\pgfpathlineto{\pgfqpoint{4.292786in}{1.855476in}}%
\pgfpathlineto{\pgfqpoint{4.295299in}{1.857465in}}%
\pgfpathlineto{\pgfqpoint{4.297977in}{1.857191in}}%
\pgfpathlineto{\pgfqpoint{4.300656in}{1.850984in}}%
\pgfpathlineto{\pgfqpoint{4.303357in}{1.855087in}}%
\pgfpathlineto{\pgfqpoint{4.306118in}{1.854323in}}%
\pgfpathlineto{\pgfqpoint{4.308691in}{1.860459in}}%
\pgfpathlineto{\pgfqpoint{4.311494in}{1.855541in}}%
\pgfpathlineto{\pgfqpoint{4.314032in}{1.854104in}}%
\pgfpathlineto{\pgfqpoint{4.316856in}{1.852366in}}%
\pgfpathlineto{\pgfqpoint{4.319405in}{1.861871in}}%
\pgfpathlineto{\pgfqpoint{4.322181in}{1.856001in}}%
\pgfpathlineto{\pgfqpoint{4.324760in}{1.854363in}}%
\pgfpathlineto{\pgfqpoint{4.327440in}{1.854390in}}%
\pgfpathlineto{\pgfqpoint{4.330118in}{1.855129in}}%
\pgfpathlineto{\pgfqpoint{4.332796in}{1.862370in}}%
\pgfpathlineto{\pgfqpoint{4.335463in}{1.859619in}}%
\pgfpathlineto{\pgfqpoint{4.338154in}{1.853392in}}%
\pgfpathlineto{\pgfqpoint{4.340976in}{1.851438in}}%
\pgfpathlineto{\pgfqpoint{4.343510in}{1.849773in}}%
\pgfpathlineto{\pgfqpoint{4.346263in}{1.859729in}}%
\pgfpathlineto{\pgfqpoint{4.348868in}{1.856701in}}%
\pgfpathlineto{\pgfqpoint{4.351645in}{1.851785in}}%
\pgfpathlineto{\pgfqpoint{4.354224in}{1.857799in}}%
\pgfpathlineto{\pgfqpoint{4.357014in}{1.854791in}}%
\pgfpathlineto{\pgfqpoint{4.359582in}{1.855735in}}%
\pgfpathlineto{\pgfqpoint{4.362270in}{1.856619in}}%
\pgfpathlineto{\pgfqpoint{4.364936in}{1.856512in}}%
\pgfpathlineto{\pgfqpoint{4.367646in}{1.859295in}}%
\pgfpathlineto{\pgfqpoint{4.370437in}{1.861021in}}%
\pgfpathlineto{\pgfqpoint{4.372976in}{1.858422in}}%
\pgfpathlineto{\pgfqpoint{4.375761in}{1.852602in}}%
\pgfpathlineto{\pgfqpoint{4.378329in}{1.849435in}}%
\pgfpathlineto{\pgfqpoint{4.381097in}{1.851043in}}%
\pgfpathlineto{\pgfqpoint{4.383674in}{1.854020in}}%
\pgfpathlineto{\pgfqpoint{4.386431in}{1.860111in}}%
\pgfpathlineto{\pgfqpoint{4.389044in}{1.862054in}}%
\pgfpathlineto{\pgfqpoint{4.391721in}{1.856280in}}%
\pgfpathlineto{\pgfqpoint{4.394400in}{1.856421in}}%
\pgfpathlineto{\pgfqpoint{4.397076in}{1.847295in}}%
\pgfpathlineto{\pgfqpoint{4.399745in}{1.848398in}}%
\pgfpathlineto{\pgfqpoint{4.402468in}{1.856663in}}%
\pgfpathlineto{\pgfqpoint{4.405234in}{1.858943in}}%
\pgfpathlineto{\pgfqpoint{4.407788in}{1.862472in}}%
\pgfpathlineto{\pgfqpoint{4.410587in}{1.853184in}}%
\pgfpathlineto{\pgfqpoint{4.413149in}{1.858640in}}%
\pgfpathlineto{\pgfqpoint{4.415932in}{1.853123in}}%
\pgfpathlineto{\pgfqpoint{4.418506in}{1.853315in}}%
\pgfpathlineto{\pgfqpoint{4.421292in}{1.850003in}}%
\pgfpathlineto{\pgfqpoint{4.423863in}{1.852654in}}%
\pgfpathlineto{\pgfqpoint{4.426534in}{1.850779in}}%
\pgfpathlineto{\pgfqpoint{4.429220in}{1.851519in}}%
\pgfpathlineto{\pgfqpoint{4.431901in}{1.849837in}}%
\pgfpathlineto{\pgfqpoint{4.434569in}{1.854268in}}%
\pgfpathlineto{\pgfqpoint{4.437253in}{1.847831in}}%
\pgfpathlineto{\pgfqpoint{4.440041in}{1.850546in}}%
\pgfpathlineto{\pgfqpoint{4.442611in}{1.846050in}}%
\pgfpathlineto{\pgfqpoint{4.445423in}{1.847823in}}%
\pgfpathlineto{\pgfqpoint{4.447965in}{1.847716in}}%
\pgfpathlineto{\pgfqpoint{4.450767in}{1.856193in}}%
\pgfpathlineto{\pgfqpoint{4.453312in}{1.852297in}}%
\pgfpathlineto{\pgfqpoint{4.456138in}{1.844053in}}%
\pgfpathlineto{\pgfqpoint{4.458681in}{1.844652in}}%
\pgfpathlineto{\pgfqpoint{4.461367in}{1.851416in}}%
\pgfpathlineto{\pgfqpoint{4.464029in}{1.855947in}}%
\pgfpathlineto{\pgfqpoint{4.466717in}{1.852099in}}%
\pgfpathlineto{\pgfqpoint{4.469492in}{1.856737in}}%
\pgfpathlineto{\pgfqpoint{4.472059in}{1.862852in}}%
\pgfpathlineto{\pgfqpoint{4.474861in}{1.867542in}}%
\pgfpathlineto{\pgfqpoint{4.477430in}{1.858125in}}%
\pgfpathlineto{\pgfqpoint{4.480201in}{1.849554in}}%
\pgfpathlineto{\pgfqpoint{4.482778in}{1.851451in}}%
\pgfpathlineto{\pgfqpoint{4.485581in}{1.858870in}}%
\pgfpathlineto{\pgfqpoint{4.488130in}{1.860603in}}%
\pgfpathlineto{\pgfqpoint{4.490822in}{1.860397in}}%
\pgfpathlineto{\pgfqpoint{4.493492in}{1.865946in}}%
\pgfpathlineto{\pgfqpoint{4.496167in}{1.860869in}}%
\pgfpathlineto{\pgfqpoint{4.498850in}{1.867887in}}%
\pgfpathlineto{\pgfqpoint{4.501529in}{1.861260in}}%
\pgfpathlineto{\pgfqpoint{4.504305in}{1.854080in}}%
\pgfpathlineto{\pgfqpoint{4.506893in}{1.862723in}}%
\pgfpathlineto{\pgfqpoint{4.509643in}{1.861949in}}%
\pgfpathlineto{\pgfqpoint{4.512246in}{1.863464in}}%
\pgfpathlineto{\pgfqpoint{4.515080in}{1.863131in}}%
\pgfpathlineto{\pgfqpoint{4.517598in}{1.871623in}}%
\pgfpathlineto{\pgfqpoint{4.520345in}{1.864631in}}%
\pgfpathlineto{\pgfqpoint{4.522962in}{1.863751in}}%
\pgfpathlineto{\pgfqpoint{4.525640in}{1.856783in}}%
\pgfpathlineto{\pgfqpoint{4.528307in}{1.859498in}}%
\pgfpathlineto{\pgfqpoint{4.530990in}{1.859003in}}%
\pgfpathlineto{\pgfqpoint{4.533764in}{1.855315in}}%
\pgfpathlineto{\pgfqpoint{4.536400in}{1.855335in}}%
\pgfpathlineto{\pgfqpoint{4.539144in}{1.858257in}}%
\pgfpathlineto{\pgfqpoint{4.541711in}{1.857495in}}%
\pgfpathlineto{\pgfqpoint{4.544464in}{1.857191in}}%
\pgfpathlineto{\pgfqpoint{4.547064in}{1.857469in}}%
\pgfpathlineto{\pgfqpoint{4.549822in}{1.859836in}}%
\pgfpathlineto{\pgfqpoint{4.552425in}{1.856292in}}%
\pgfpathlineto{\pgfqpoint{4.555106in}{1.853522in}}%
\pgfpathlineto{\pgfqpoint{4.557777in}{1.856565in}}%
\pgfpathlineto{\pgfqpoint{4.560448in}{1.858997in}}%
\pgfpathlineto{\pgfqpoint{4.563125in}{1.855505in}}%
\pgfpathlineto{\pgfqpoint{4.565820in}{1.855599in}}%
\pgfpathlineto{\pgfqpoint{4.568612in}{1.848737in}}%
\pgfpathlineto{\pgfqpoint{4.571171in}{1.848406in}}%
\pgfpathlineto{\pgfqpoint{4.573947in}{1.852178in}}%
\pgfpathlineto{\pgfqpoint{4.576531in}{1.852768in}}%
\pgfpathlineto{\pgfqpoint{4.579305in}{1.858161in}}%
\pgfpathlineto{\pgfqpoint{4.581888in}{1.855483in}}%
\pgfpathlineto{\pgfqpoint{4.584672in}{1.852443in}}%
\pgfpathlineto{\pgfqpoint{4.587244in}{1.848903in}}%
\pgfpathlineto{\pgfqpoint{4.589920in}{1.845747in}}%
\pgfpathlineto{\pgfqpoint{4.592589in}{1.843059in}}%
\pgfpathlineto{\pgfqpoint{4.595281in}{1.850544in}}%
\pgfpathlineto{\pgfqpoint{4.597951in}{1.849338in}}%
\pgfpathlineto{\pgfqpoint{4.600633in}{1.858733in}}%
\pgfpathlineto{\pgfqpoint{4.603430in}{1.862697in}}%
\pgfpathlineto{\pgfqpoint{4.605990in}{1.860004in}}%
\pgfpathlineto{\pgfqpoint{4.608808in}{1.860256in}}%
\pgfpathlineto{\pgfqpoint{4.611350in}{1.858215in}}%
\pgfpathlineto{\pgfqpoint{4.614134in}{1.854401in}}%
\pgfpathlineto{\pgfqpoint{4.616702in}{1.864230in}}%
\pgfpathlineto{\pgfqpoint{4.619529in}{1.860679in}}%
\pgfpathlineto{\pgfqpoint{4.622056in}{1.856729in}}%
\pgfpathlineto{\pgfqpoint{4.624741in}{1.862089in}}%
\pgfpathlineto{\pgfqpoint{4.627411in}{1.859456in}}%
\pgfpathlineto{\pgfqpoint{4.630096in}{1.865071in}}%
\pgfpathlineto{\pgfqpoint{4.632902in}{1.863411in}}%
\pgfpathlineto{\pgfqpoint{4.635445in}{1.860265in}}%
\pgfpathlineto{\pgfqpoint{4.638204in}{1.863910in}}%
\pgfpathlineto{\pgfqpoint{4.640809in}{1.862999in}}%
\pgfpathlineto{\pgfqpoint{4.643628in}{1.864619in}}%
\pgfpathlineto{\pgfqpoint{4.646169in}{1.862202in}}%
\pgfpathlineto{\pgfqpoint{4.648922in}{1.863698in}}%
\pgfpathlineto{\pgfqpoint{4.651524in}{1.858335in}}%
\pgfpathlineto{\pgfqpoint{4.654203in}{1.862137in}}%
\pgfpathlineto{\pgfqpoint{4.656873in}{1.860287in}}%
\pgfpathlineto{\pgfqpoint{4.659590in}{1.857543in}}%
\pgfpathlineto{\pgfqpoint{4.662237in}{1.860103in}}%
\pgfpathlineto{\pgfqpoint{4.664923in}{1.856724in}}%
\pgfpathlineto{\pgfqpoint{4.667764in}{1.860857in}}%
\pgfpathlineto{\pgfqpoint{4.670261in}{1.861141in}}%
\pgfpathlineto{\pgfqpoint{4.673068in}{1.858156in}}%
\pgfpathlineto{\pgfqpoint{4.675619in}{1.854930in}}%
\pgfpathlineto{\pgfqpoint{4.678448in}{1.856692in}}%
\pgfpathlineto{\pgfqpoint{4.680988in}{1.857517in}}%
\pgfpathlineto{\pgfqpoint{4.683799in}{1.853042in}}%
\pgfpathlineto{\pgfqpoint{4.686337in}{1.848728in}}%
\pgfpathlineto{\pgfqpoint{4.689051in}{1.853779in}}%
\pgfpathlineto{\pgfqpoint{4.691694in}{1.854907in}}%
\pgfpathlineto{\pgfqpoint{4.694381in}{1.858905in}}%
\pgfpathlineto{\pgfqpoint{4.697170in}{1.852786in}}%
\pgfpathlineto{\pgfqpoint{4.699734in}{1.852567in}}%
\pgfpathlineto{\pgfqpoint{4.702517in}{1.854012in}}%
\pgfpathlineto{\pgfqpoint{4.705094in}{1.856198in}}%
\pgfpathlineto{\pgfqpoint{4.707824in}{1.859224in}}%
\pgfpathlineto{\pgfqpoint{4.710437in}{1.858345in}}%
\pgfpathlineto{\pgfqpoint{4.713275in}{1.855310in}}%
\pgfpathlineto{\pgfqpoint{4.715806in}{1.856222in}}%
\pgfpathlineto{\pgfqpoint{4.718486in}{1.856860in}}%
\pgfpathlineto{\pgfqpoint{4.721160in}{1.848771in}}%
\pgfpathlineto{\pgfqpoint{4.723873in}{1.857622in}}%
\pgfpathlineto{\pgfqpoint{4.726508in}{1.856515in}}%
\pgfpathlineto{\pgfqpoint{4.729233in}{1.857123in}}%
\pgfpathlineto{\pgfqpoint{4.731901in}{1.862242in}}%
\pgfpathlineto{\pgfqpoint{4.734552in}{1.860572in}}%
\pgfpathlineto{\pgfqpoint{4.737348in}{1.861429in}}%
\pgfpathlineto{\pgfqpoint{4.739912in}{1.865089in}}%
\pgfpathlineto{\pgfqpoint{4.742696in}{1.859511in}}%
\pgfpathlineto{\pgfqpoint{4.745256in}{1.860333in}}%
\pgfpathlineto{\pgfqpoint{4.748081in}{1.863738in}}%
\pgfpathlineto{\pgfqpoint{4.750627in}{1.866226in}}%
\pgfpathlineto{\pgfqpoint{4.753298in}{1.859410in}}%
\pgfpathlineto{\pgfqpoint{4.755983in}{1.858656in}}%
\pgfpathlineto{\pgfqpoint{4.758653in}{1.863877in}}%
\pgfpathlineto{\pgfqpoint{4.761337in}{1.879688in}}%
\pgfpathlineto{\pgfqpoint{4.764018in}{1.869224in}}%
\pgfpathlineto{\pgfqpoint{4.766783in}{1.863197in}}%
\pgfpathlineto{\pgfqpoint{4.769367in}{1.855814in}}%
\pgfpathlineto{\pgfqpoint{4.772198in}{1.862852in}}%
\pgfpathlineto{\pgfqpoint{4.774732in}{1.858557in}}%
\pgfpathlineto{\pgfqpoint{4.777535in}{1.861515in}}%
\pgfpathlineto{\pgfqpoint{4.780083in}{1.857785in}}%
\pgfpathlineto{\pgfqpoint{4.782872in}{1.854185in}}%
\pgfpathlineto{\pgfqpoint{4.785445in}{1.859025in}}%
\pgfpathlineto{\pgfqpoint{4.788116in}{1.856792in}}%
\pgfpathlineto{\pgfqpoint{4.790798in}{1.858753in}}%
\pgfpathlineto{\pgfqpoint{4.793512in}{1.858754in}}%
\pgfpathlineto{\pgfqpoint{4.796274in}{1.855031in}}%
\pgfpathlineto{\pgfqpoint{4.798830in}{1.857185in}}%
\pgfpathlineto{\pgfqpoint{4.801586in}{1.861490in}}%
\pgfpathlineto{\pgfqpoint{4.804193in}{1.863618in}}%
\pgfpathlineto{\pgfqpoint{4.807017in}{1.856521in}}%
\pgfpathlineto{\pgfqpoint{4.809538in}{1.876472in}}%
\pgfpathlineto{\pgfqpoint{4.812377in}{1.883629in}}%
\pgfpathlineto{\pgfqpoint{4.814907in}{1.889716in}}%
\pgfpathlineto{\pgfqpoint{4.817587in}{1.897021in}}%
\pgfpathlineto{\pgfqpoint{4.820265in}{1.894584in}}%
\pgfpathlineto{\pgfqpoint{4.822945in}{1.898594in}}%
\pgfpathlineto{\pgfqpoint{4.825619in}{1.883834in}}%
\pgfpathlineto{\pgfqpoint{4.828291in}{1.869579in}}%
\pgfpathlineto{\pgfqpoint{4.831045in}{1.859188in}}%
\pgfpathlineto{\pgfqpoint{4.833657in}{1.860943in}}%
\pgfpathlineto{\pgfqpoint{4.837992in}{1.866471in}}%
\pgfpathlineto{\pgfqpoint{4.839922in}{1.880803in}}%
\pgfpathlineto{\pgfqpoint{4.842380in}{1.886383in}}%
\pgfpathlineto{\pgfqpoint{4.844361in}{1.885565in}}%
\pgfpathlineto{\pgfqpoint{4.847127in}{1.890690in}}%
\pgfpathlineto{\pgfqpoint{4.849715in}{1.892960in}}%
\pgfpathlineto{\pgfqpoint{4.852404in}{1.912072in}}%
\pgfpathlineto{\pgfqpoint{4.855070in}{1.892224in}}%
\pgfpathlineto{\pgfqpoint{4.857807in}{1.884440in}}%
\pgfpathlineto{\pgfqpoint{4.860544in}{1.877596in}}%
\pgfpathlineto{\pgfqpoint{4.863116in}{1.866452in}}%
\pgfpathlineto{\pgfqpoint{4.865910in}{1.861512in}}%
\pgfpathlineto{\pgfqpoint{4.868474in}{1.856307in}}%
\pgfpathlineto{\pgfqpoint{4.871209in}{1.853914in}}%
\pgfpathlineto{\pgfqpoint{4.873832in}{1.847420in}}%
\pgfpathlineto{\pgfqpoint{4.876636in}{1.848223in}}%
\pgfpathlineto{\pgfqpoint{4.879180in}{1.845007in}}%
\pgfpathlineto{\pgfqpoint{4.881864in}{1.855718in}}%
\pgfpathlineto{\pgfqpoint{4.884540in}{1.944059in}}%
\pgfpathlineto{\pgfqpoint{4.887211in}{1.964262in}}%
\pgfpathlineto{\pgfqpoint{4.889902in}{1.935395in}}%
\pgfpathlineto{\pgfqpoint{4.892611in}{1.906700in}}%
\pgfpathlineto{\pgfqpoint{4.895399in}{1.899047in}}%
\pgfpathlineto{\pgfqpoint{4.897938in}{1.904376in}}%
\pgfpathlineto{\pgfqpoint{4.900712in}{1.926544in}}%
\pgfpathlineto{\pgfqpoint{4.903295in}{1.905130in}}%
\pgfpathlineto{\pgfqpoint{4.906096in}{1.912565in}}%
\pgfpathlineto{\pgfqpoint{4.908648in}{1.913087in}}%
\pgfpathlineto{\pgfqpoint{4.911435in}{1.911370in}}%
\pgfpathlineto{\pgfqpoint{4.914009in}{1.914674in}}%
\pgfpathlineto{\pgfqpoint{4.916681in}{1.891736in}}%
\pgfpathlineto{\pgfqpoint{4.919352in}{1.884981in}}%
\pgfpathlineto{\pgfqpoint{4.922041in}{1.864834in}}%
\pgfpathlineto{\pgfqpoint{4.924708in}{1.858268in}}%
\pgfpathlineto{\pgfqpoint{4.927400in}{1.857316in}}%
\pgfpathlineto{\pgfqpoint{4.930170in}{1.856341in}}%
\pgfpathlineto{\pgfqpoint{4.932742in}{1.848257in}}%
\pgfpathlineto{\pgfqpoint{4.935515in}{1.852951in}}%
\pgfpathlineto{\pgfqpoint{4.938112in}{1.856772in}}%
\pgfpathlineto{\pgfqpoint{4.940881in}{1.866117in}}%
\pgfpathlineto{\pgfqpoint{4.943466in}{1.870037in}}%
\pgfpathlineto{\pgfqpoint{4.946151in}{1.877949in}}%
\pgfpathlineto{\pgfqpoint{4.948827in}{1.904156in}}%
\pgfpathlineto{\pgfqpoint{4.951504in}{1.913339in}}%
\pgfpathlineto{\pgfqpoint{4.954182in}{1.889264in}}%
\pgfpathlineto{\pgfqpoint{4.956862in}{1.892392in}}%
\pgfpathlineto{\pgfqpoint{4.959689in}{1.877400in}}%
\pgfpathlineto{\pgfqpoint{4.962219in}{1.865515in}}%
\pgfpathlineto{\pgfqpoint{4.965002in}{1.858375in}}%
\pgfpathlineto{\pgfqpoint{4.967575in}{1.857469in}}%
\pgfpathlineto{\pgfqpoint{4.970314in}{1.851574in}}%
\pgfpathlineto{\pgfqpoint{4.972933in}{1.851776in}}%
\pgfpathlineto{\pgfqpoint{4.975703in}{1.853180in}}%
\pgfpathlineto{\pgfqpoint{4.978287in}{1.852383in}}%
\pgfpathlineto{\pgfqpoint{4.980967in}{1.854542in}}%
\pgfpathlineto{\pgfqpoint{4.983637in}{1.852795in}}%
\pgfpathlineto{\pgfqpoint{4.986325in}{1.853307in}}%
\pgfpathlineto{\pgfqpoint{4.989001in}{1.855392in}}%
\pgfpathlineto{\pgfqpoint{4.991683in}{1.858986in}}%
\pgfpathlineto{\pgfqpoint{4.994390in}{1.856313in}}%
\pgfpathlineto{\pgfqpoint{4.997028in}{1.857039in}}%
\pgfpathlineto{\pgfqpoint{4.999780in}{1.852128in}}%
\pgfpathlineto{\pgfqpoint{5.002384in}{1.865594in}}%
\pgfpathlineto{\pgfqpoint{5.005178in}{1.854364in}}%
\pgfpathlineto{\pgfqpoint{5.007751in}{1.855762in}}%
\pgfpathlineto{\pgfqpoint{5.010562in}{1.852539in}}%
\pgfpathlineto{\pgfqpoint{5.013104in}{1.859795in}}%
\pgfpathlineto{\pgfqpoint{5.015820in}{1.872937in}}%
\pgfpathlineto{\pgfqpoint{5.018466in}{1.868809in}}%
\pgfpathlineto{\pgfqpoint{5.021147in}{1.859036in}}%
\pgfpathlineto{\pgfqpoint{5.023927in}{1.857874in}}%
\pgfpathlineto{\pgfqpoint{5.026501in}{1.858856in}}%
\pgfpathlineto{\pgfqpoint{5.029275in}{1.857180in}}%
\pgfpathlineto{\pgfqpoint{5.031849in}{1.865808in}}%
\pgfpathlineto{\pgfqpoint{5.034649in}{1.879565in}}%
\pgfpathlineto{\pgfqpoint{5.037214in}{1.870022in}}%
\pgfpathlineto{\pgfqpoint{5.039962in}{1.862798in}}%
\pgfpathlineto{\pgfqpoint{5.042572in}{1.858769in}}%
\pgfpathlineto{\pgfqpoint{5.045249in}{1.857869in}}%
\pgfpathlineto{\pgfqpoint{5.047924in}{1.852494in}}%
\pgfpathlineto{\pgfqpoint{5.050606in}{1.859817in}}%
\pgfpathlineto{\pgfqpoint{5.053284in}{1.859765in}}%
\pgfpathlineto{\pgfqpoint{5.055952in}{1.854808in}}%
\pgfpathlineto{\pgfqpoint{5.058711in}{1.857882in}}%
\pgfpathlineto{\pgfqpoint{5.061315in}{1.863404in}}%
\pgfpathlineto{\pgfqpoint{5.064144in}{1.862405in}}%
\pgfpathlineto{\pgfqpoint{5.066677in}{1.856534in}}%
\pgfpathlineto{\pgfqpoint{5.069463in}{1.858394in}}%
\pgfpathlineto{\pgfqpoint{5.072030in}{1.860088in}}%
\pgfpathlineto{\pgfqpoint{5.074851in}{1.851665in}}%
\pgfpathlineto{\pgfqpoint{5.077390in}{1.858917in}}%
\pgfpathlineto{\pgfqpoint{5.080067in}{1.859751in}}%
\pgfpathlineto{\pgfqpoint{5.082746in}{1.852899in}}%
\pgfpathlineto{\pgfqpoint{5.085426in}{1.857370in}}%
\pgfpathlineto{\pgfqpoint{5.088103in}{1.860870in}}%
\pgfpathlineto{\pgfqpoint{5.090788in}{1.856558in}}%
\pgfpathlineto{\pgfqpoint{5.093579in}{1.858305in}}%
\pgfpathlineto{\pgfqpoint{5.096142in}{1.851575in}}%
\pgfpathlineto{\pgfqpoint{5.098948in}{1.859577in}}%
\pgfpathlineto{\pgfqpoint{5.101496in}{1.856176in}}%
\pgfpathlineto{\pgfqpoint{5.104312in}{1.851430in}}%
\pgfpathlineto{\pgfqpoint{5.106842in}{1.852754in}}%
\pgfpathlineto{\pgfqpoint{5.109530in}{1.857906in}}%
\pgfpathlineto{\pgfqpoint{5.112209in}{1.860651in}}%
\pgfpathlineto{\pgfqpoint{5.114887in}{1.859008in}}%
\pgfpathlineto{\pgfqpoint{5.117550in}{1.856384in}}%
\pgfpathlineto{\pgfqpoint{5.120243in}{1.855407in}}%
\pgfpathlineto{\pgfqpoint{5.123042in}{1.853771in}}%
\pgfpathlineto{\pgfqpoint{5.125599in}{1.854051in}}%
\pgfpathlineto{\pgfqpoint{5.128421in}{1.855973in}}%
\pgfpathlineto{\pgfqpoint{5.130953in}{1.852867in}}%
\pgfpathlineto{\pgfqpoint{5.133716in}{1.854855in}}%
\pgfpathlineto{\pgfqpoint{5.136311in}{1.860278in}}%
\pgfpathlineto{\pgfqpoint{5.139072in}{1.866473in}}%
\pgfpathlineto{\pgfqpoint{5.141660in}{1.855940in}}%
\pgfpathlineto{\pgfqpoint{5.144349in}{1.865282in}}%
\pgfpathlineto{\pgfqpoint{5.147029in}{1.865213in}}%
\pgfpathlineto{\pgfqpoint{5.149734in}{1.868245in}}%
\pgfpathlineto{\pgfqpoint{5.152382in}{1.871483in}}%
\pgfpathlineto{\pgfqpoint{5.155059in}{1.869577in}}%
\pgfpathlineto{\pgfqpoint{5.157815in}{1.861257in}}%
\pgfpathlineto{\pgfqpoint{5.160420in}{1.866917in}}%
\pgfpathlineto{\pgfqpoint{5.163243in}{1.861323in}}%
\pgfpathlineto{\pgfqpoint{5.165775in}{1.861355in}}%
\pgfpathlineto{\pgfqpoint{5.168591in}{1.860578in}}%
\pgfpathlineto{\pgfqpoint{5.171133in}{1.858139in}}%
\pgfpathlineto{\pgfqpoint{5.173925in}{1.862249in}}%
\pgfpathlineto{\pgfqpoint{5.176477in}{1.858794in}}%
\pgfpathlineto{\pgfqpoint{5.179188in}{1.859193in}}%
\pgfpathlineto{\pgfqpoint{5.181848in}{1.856638in}}%
\pgfpathlineto{\pgfqpoint{5.184522in}{1.855781in}}%
\pgfpathlineto{\pgfqpoint{5.187294in}{1.862805in}}%
\pgfpathlineto{\pgfqpoint{5.189880in}{1.862818in}}%
\pgfpathlineto{\pgfqpoint{5.192680in}{1.857395in}}%
\pgfpathlineto{\pgfqpoint{5.195239in}{1.852858in}}%
\pgfpathlineto{\pgfqpoint{5.198008in}{1.840692in}}%
\pgfpathlineto{\pgfqpoint{5.200594in}{1.840948in}}%
\pgfpathlineto{\pgfqpoint{5.203388in}{1.840540in}}%
\pgfpathlineto{\pgfqpoint{5.205952in}{1.838072in}}%
\pgfpathlineto{\pgfqpoint{5.208630in}{1.842007in}}%
\pgfpathlineto{\pgfqpoint{5.211299in}{1.843577in}}%
\pgfpathlineto{\pgfqpoint{5.214027in}{1.839612in}}%
\pgfpathlineto{\pgfqpoint{5.216667in}{1.839153in}}%
\pgfpathlineto{\pgfqpoint{5.219345in}{1.857846in}}%
\pgfpathlineto{\pgfqpoint{5.222151in}{1.851859in}}%
\pgfpathlineto{\pgfqpoint{5.224695in}{1.854910in}}%
\pgfpathlineto{\pgfqpoint{5.227470in}{1.856367in}}%
\pgfpathlineto{\pgfqpoint{5.230059in}{1.859529in}}%
\pgfpathlineto{\pgfqpoint{5.232855in}{1.864873in}}%
\pgfpathlineto{\pgfqpoint{5.235409in}{1.859408in}}%
\pgfpathlineto{\pgfqpoint{5.238173in}{1.851539in}}%
\pgfpathlineto{\pgfqpoint{5.240777in}{1.858194in}}%
\pgfpathlineto{\pgfqpoint{5.243445in}{1.864136in}}%
\pgfpathlineto{\pgfqpoint{5.246130in}{1.857030in}}%
\pgfpathlineto{\pgfqpoint{5.248816in}{1.858455in}}%
\pgfpathlineto{\pgfqpoint{5.251590in}{1.851247in}}%
\pgfpathlineto{\pgfqpoint{5.254236in}{1.855437in}}%
\pgfpathlineto{\pgfqpoint{5.256973in}{1.857507in}}%
\pgfpathlineto{\pgfqpoint{5.259511in}{1.859999in}}%
\pgfpathlineto{\pgfqpoint{5.262264in}{1.855481in}}%
\pgfpathlineto{\pgfqpoint{5.264876in}{1.841950in}}%
\pgfpathlineto{\pgfqpoint{5.267691in}{1.849873in}}%
\pgfpathlineto{\pgfqpoint{5.270238in}{1.854756in}}%
\pgfpathlineto{\pgfqpoint{5.272913in}{1.848144in}}%
\pgfpathlineto{\pgfqpoint{5.275589in}{1.846295in}}%
\pgfpathlineto{\pgfqpoint{5.278322in}{1.841086in}}%
\pgfpathlineto{\pgfqpoint{5.280947in}{1.838072in}}%
\pgfpathlineto{\pgfqpoint{5.283631in}{1.848502in}}%
\pgfpathlineto{\pgfqpoint{5.286436in}{1.845650in}}%
\pgfpathlineto{\pgfqpoint{5.288984in}{1.844102in}}%
\pgfpathlineto{\pgfqpoint{5.291794in}{1.845420in}}%
\pgfpathlineto{\pgfqpoint{5.294339in}{1.849601in}}%
\pgfpathlineto{\pgfqpoint{5.297140in}{1.850965in}}%
\pgfpathlineto{\pgfqpoint{5.299696in}{1.850679in}}%
\pgfpathlineto{\pgfqpoint{5.302443in}{1.848444in}}%
\pgfpathlineto{\pgfqpoint{5.305054in}{1.845042in}}%
\pgfpathlineto{\pgfqpoint{5.307731in}{1.845537in}}%
\pgfpathlineto{\pgfqpoint{5.310411in}{1.856613in}}%
\pgfpathlineto{\pgfqpoint{5.313089in}{1.853500in}}%
\pgfpathlineto{\pgfqpoint{5.315754in}{1.854376in}}%
\pgfpathlineto{\pgfqpoint{5.318430in}{1.864056in}}%
\pgfpathlineto{\pgfqpoint{5.321256in}{1.897452in}}%
\pgfpathlineto{\pgfqpoint{5.323802in}{1.988131in}}%
\pgfpathlineto{\pgfqpoint{5.326564in}{1.993631in}}%
\pgfpathlineto{\pgfqpoint{5.329159in}{1.970881in}}%
\pgfpathlineto{\pgfqpoint{5.331973in}{1.935099in}}%
\pgfpathlineto{\pgfqpoint{5.334510in}{1.911242in}}%
\pgfpathlineto{\pgfqpoint{5.337353in}{1.887782in}}%
\pgfpathlineto{\pgfqpoint{5.339872in}{1.880324in}}%
\pgfpathlineto{\pgfqpoint{5.342549in}{1.885324in}}%
\pgfpathlineto{\pgfqpoint{5.345224in}{1.877841in}}%
\pgfpathlineto{\pgfqpoint{5.347905in}{1.859851in}}%
\pgfpathlineto{\pgfqpoint{5.350723in}{1.855153in}}%
\pgfpathlineto{\pgfqpoint{5.353262in}{1.862971in}}%
\pgfpathlineto{\pgfqpoint{5.356056in}{1.857164in}}%
\pgfpathlineto{\pgfqpoint{5.358612in}{1.855430in}}%
\pgfpathlineto{\pgfqpoint{5.361370in}{1.858531in}}%
\pgfpathlineto{\pgfqpoint{5.363966in}{1.857841in}}%
\pgfpathlineto{\pgfqpoint{5.366727in}{1.864968in}}%
\pgfpathlineto{\pgfqpoint{5.369335in}{1.871826in}}%
\pgfpathlineto{\pgfqpoint{5.372013in}{1.862870in}}%
\pgfpathlineto{\pgfqpoint{5.374692in}{1.857712in}}%
\pgfpathlineto{\pgfqpoint{5.377370in}{1.855634in}}%
\pgfpathlineto{\pgfqpoint{5.380048in}{1.848428in}}%
\pgfpathlineto{\pgfqpoint{5.382725in}{1.854262in}}%
\pgfpathlineto{\pgfqpoint{5.385550in}{1.859921in}}%
\pgfpathlineto{\pgfqpoint{5.388083in}{1.856579in}}%
\pgfpathlineto{\pgfqpoint{5.390900in}{1.865921in}}%
\pgfpathlineto{\pgfqpoint{5.393441in}{1.869982in}}%
\pgfpathlineto{\pgfqpoint{5.396219in}{1.859486in}}%
\pgfpathlineto{\pgfqpoint{5.398784in}{1.869081in}}%
\pgfpathlineto{\pgfqpoint{5.401576in}{1.870349in}}%
\pgfpathlineto{\pgfqpoint{5.404154in}{1.865630in}}%
\pgfpathlineto{\pgfqpoint{5.406832in}{1.866514in}}%
\pgfpathlineto{\pgfqpoint{5.409507in}{1.865702in}}%
\pgfpathlineto{\pgfqpoint{5.412190in}{1.860059in}}%
\pgfpathlineto{\pgfqpoint{5.414954in}{1.860365in}}%
\pgfpathlineto{\pgfqpoint{5.417547in}{1.851904in}}%
\pgfpathlineto{\pgfqpoint{5.420304in}{1.850070in}}%
\pgfpathlineto{\pgfqpoint{5.422897in}{1.855938in}}%
\pgfpathlineto{\pgfqpoint{5.425661in}{1.859709in}}%
\pgfpathlineto{\pgfqpoint{5.428259in}{1.857060in}}%
\pgfpathlineto{\pgfqpoint{5.431015in}{1.859464in}}%
\pgfpathlineto{\pgfqpoint{5.433616in}{1.858584in}}%
\pgfpathlineto{\pgfqpoint{5.436295in}{1.861182in}}%
\pgfpathlineto{\pgfqpoint{5.438974in}{1.857928in}}%
\pgfpathlineto{\pgfqpoint{5.441698in}{1.855025in}}%
\pgfpathlineto{\pgfqpoint{5.444328in}{1.858173in}}%
\pgfpathlineto{\pgfqpoint{5.447021in}{1.858444in}}%
\pgfpathlineto{\pgfqpoint{5.449769in}{1.855195in}}%
\pgfpathlineto{\pgfqpoint{5.452365in}{1.862800in}}%
\pgfpathlineto{\pgfqpoint{5.455168in}{1.855941in}}%
\pgfpathlineto{\pgfqpoint{5.457721in}{1.849938in}}%
\pgfpathlineto{\pgfqpoint{5.460489in}{1.853663in}}%
\pgfpathlineto{\pgfqpoint{5.463079in}{1.855882in}}%
\pgfpathlineto{\pgfqpoint{5.465888in}{1.860971in}}%
\pgfpathlineto{\pgfqpoint{5.468425in}{1.864426in}}%
\pgfpathlineto{\pgfqpoint{5.471113in}{1.865378in}}%
\pgfpathlineto{\pgfqpoint{5.473792in}{1.860669in}}%
\pgfpathlineto{\pgfqpoint{5.476458in}{1.865159in}}%
\pgfpathlineto{\pgfqpoint{5.479152in}{1.861519in}}%
\pgfpathlineto{\pgfqpoint{5.481825in}{1.860173in}}%
\pgfpathlineto{\pgfqpoint{5.484641in}{1.857933in}}%
\pgfpathlineto{\pgfqpoint{5.487176in}{1.866193in}}%
\pgfpathlineto{\pgfqpoint{5.490000in}{1.855469in}}%
\pgfpathlineto{\pgfqpoint{5.492541in}{1.860153in}}%
\pgfpathlineto{\pgfqpoint{5.495346in}{1.862191in}}%
\pgfpathlineto{\pgfqpoint{5.497898in}{1.859136in}}%
\pgfpathlineto{\pgfqpoint{5.500687in}{1.866210in}}%
\pgfpathlineto{\pgfqpoint{5.503255in}{1.867947in}}%
\pgfpathlineto{\pgfqpoint{5.505933in}{1.868530in}}%
\pgfpathlineto{\pgfqpoint{5.508612in}{1.860439in}}%
\pgfpathlineto{\pgfqpoint{5.511290in}{1.862535in}}%
\pgfpathlineto{\pgfqpoint{5.514080in}{1.863390in}}%
\pgfpathlineto{\pgfqpoint{5.516646in}{1.862992in}}%
\pgfpathlineto{\pgfqpoint{5.519433in}{1.859920in}}%
\pgfpathlineto{\pgfqpoint{5.522003in}{1.864546in}}%
\pgfpathlineto{\pgfqpoint{5.524756in}{1.863867in}}%
\pgfpathlineto{\pgfqpoint{5.527360in}{1.863370in}}%
\pgfpathlineto{\pgfqpoint{5.530148in}{1.856753in}}%
\pgfpathlineto{\pgfqpoint{5.532717in}{1.861435in}}%
\pgfpathlineto{\pgfqpoint{5.535395in}{1.861360in}}%
\pgfpathlineto{\pgfqpoint{5.538074in}{1.859206in}}%
\pgfpathlineto{\pgfqpoint{5.540750in}{1.864213in}}%
\pgfpathlineto{\pgfqpoint{5.543421in}{1.864170in}}%
\pgfpathlineto{\pgfqpoint{5.546110in}{1.861713in}}%
\pgfpathlineto{\pgfqpoint{5.548921in}{1.867827in}}%
\pgfpathlineto{\pgfqpoint{5.551457in}{1.858051in}}%
\pgfpathlineto{\pgfqpoint{5.554198in}{1.865573in}}%
\pgfpathlineto{\pgfqpoint{5.556822in}{1.862095in}}%
\pgfpathlineto{\pgfqpoint{5.559612in}{1.866821in}}%
\pgfpathlineto{\pgfqpoint{5.562180in}{1.867572in}}%
\pgfpathlineto{\pgfqpoint{5.564940in}{1.863479in}}%
\pgfpathlineto{\pgfqpoint{5.567536in}{1.865377in}}%
\pgfpathlineto{\pgfqpoint{5.570215in}{1.866145in}}%
\pgfpathlineto{\pgfqpoint{5.572893in}{1.858075in}}%
\pgfpathlineto{\pgfqpoint{5.575596in}{1.856852in}}%
\pgfpathlineto{\pgfqpoint{5.578342in}{1.853986in}}%
\pgfpathlineto{\pgfqpoint{5.580914in}{1.862802in}}%
\pgfpathlineto{\pgfqpoint{5.583709in}{1.872289in}}%
\pgfpathlineto{\pgfqpoint{5.586269in}{1.860899in}}%
\pgfpathlineto{\pgfqpoint{5.589040in}{1.854933in}}%
\pgfpathlineto{\pgfqpoint{5.591641in}{1.857434in}}%
\pgfpathlineto{\pgfqpoint{5.594368in}{1.862080in}}%
\pgfpathlineto{\pgfqpoint{5.596999in}{1.853078in}}%
\pgfpathlineto{\pgfqpoint{5.599674in}{1.852155in}}%
\pgfpathlineto{\pgfqpoint{5.602352in}{1.856702in}}%
\pgfpathlineto{\pgfqpoint{5.605073in}{1.855776in}}%
\pgfpathlineto{\pgfqpoint{5.607698in}{1.856314in}}%
\pgfpathlineto{\pgfqpoint{5.610389in}{1.855710in}}%
\pgfpathlineto{\pgfqpoint{5.613235in}{1.872505in}}%
\pgfpathlineto{\pgfqpoint{5.615743in}{1.873843in}}%
\pgfpathlineto{\pgfqpoint{5.618526in}{1.872901in}}%
\pgfpathlineto{\pgfqpoint{5.621102in}{1.863805in}}%
\pgfpathlineto{\pgfqpoint{5.623868in}{1.848385in}}%
\pgfpathlineto{\pgfqpoint{5.626460in}{1.842341in}}%
\pgfpathlineto{\pgfqpoint{5.629232in}{1.852737in}}%
\pgfpathlineto{\pgfqpoint{5.631815in}{1.851068in}}%
\pgfpathlineto{\pgfqpoint{5.634496in}{1.846734in}}%
\pgfpathlineto{\pgfqpoint{5.637172in}{1.849220in}}%
\pgfpathlineto{\pgfqpoint{5.639852in}{1.848647in}}%
\pgfpathlineto{\pgfqpoint{5.642518in}{1.847128in}}%
\pgfpathlineto{\pgfqpoint{5.645243in}{1.844381in}}%
\pgfpathlineto{\pgfqpoint{5.648008in}{1.841525in}}%
\pgfpathlineto{\pgfqpoint{5.650563in}{1.843419in}}%
\pgfpathlineto{\pgfqpoint{5.653376in}{1.853184in}}%
\pgfpathlineto{\pgfqpoint{5.655919in}{1.850380in}}%
\pgfpathlineto{\pgfqpoint{5.658723in}{1.846582in}}%
\pgfpathlineto{\pgfqpoint{5.661273in}{1.848685in}}%
\pgfpathlineto{\pgfqpoint{5.664099in}{1.848104in}}%
\pgfpathlineto{\pgfqpoint{5.666632in}{1.848017in}}%
\pgfpathlineto{\pgfqpoint{5.669313in}{1.854442in}}%
\pgfpathlineto{\pgfqpoint{5.671991in}{1.849025in}}%
\pgfpathlineto{\pgfqpoint{5.674667in}{1.854219in}}%
\pgfpathlineto{\pgfqpoint{5.677486in}{1.852994in}}%
\pgfpathlineto{\pgfqpoint{5.680027in}{1.856232in}}%
\pgfpathlineto{\pgfqpoint{5.682836in}{1.856211in}}%
\pgfpathlineto{\pgfqpoint{5.685385in}{1.852045in}}%
\pgfpathlineto{\pgfqpoint{5.688159in}{1.857352in}}%
\pgfpathlineto{\pgfqpoint{5.690730in}{1.865613in}}%
\pgfpathlineto{\pgfqpoint{5.693473in}{1.850032in}}%
\pgfpathlineto{\pgfqpoint{5.696101in}{1.859449in}}%
\pgfpathlineto{\pgfqpoint{5.698775in}{1.862931in}}%
\pgfpathlineto{\pgfqpoint{5.701453in}{1.860703in}}%
\pgfpathlineto{\pgfqpoint{5.704130in}{1.863071in}}%
\pgfpathlineto{\pgfqpoint{5.706800in}{1.864302in}}%
\pgfpathlineto{\pgfqpoint{5.709490in}{1.858105in}}%
\pgfpathlineto{\pgfqpoint{5.712291in}{1.855769in}}%
\pgfpathlineto{\pgfqpoint{5.714834in}{1.855022in}}%
\pgfpathlineto{\pgfqpoint{5.717671in}{1.858216in}}%
\pgfpathlineto{\pgfqpoint{5.720201in}{1.859836in}}%
\pgfpathlineto{\pgfqpoint{5.722950in}{1.862522in}}%
\pgfpathlineto{\pgfqpoint{5.725548in}{1.861500in}}%
\pgfpathlineto{\pgfqpoint{5.728339in}{1.858535in}}%
\pgfpathlineto{\pgfqpoint{5.730919in}{1.859037in}}%
\pgfpathlineto{\pgfqpoint{5.733594in}{1.859034in}}%
\pgfpathlineto{\pgfqpoint{5.736276in}{1.854192in}}%
\pgfpathlineto{\pgfqpoint{5.738974in}{1.864038in}}%
\pgfpathlineto{\pgfqpoint{5.741745in}{1.861713in}}%
\pgfpathlineto{\pgfqpoint{5.744310in}{1.867083in}}%
\pgfpathlineto{\pgfqpoint{5.744310in}{0.413320in}}%
\pgfpathlineto{\pgfqpoint{5.744310in}{0.413320in}}%
\pgfpathlineto{\pgfqpoint{5.741745in}{0.413320in}}%
\pgfpathlineto{\pgfqpoint{5.738974in}{0.413320in}}%
\pgfpathlineto{\pgfqpoint{5.736276in}{0.413320in}}%
\pgfpathlineto{\pgfqpoint{5.733594in}{0.413320in}}%
\pgfpathlineto{\pgfqpoint{5.730919in}{0.413320in}}%
\pgfpathlineto{\pgfqpoint{5.728339in}{0.413320in}}%
\pgfpathlineto{\pgfqpoint{5.725548in}{0.413320in}}%
\pgfpathlineto{\pgfqpoint{5.722950in}{0.413320in}}%
\pgfpathlineto{\pgfqpoint{5.720201in}{0.413320in}}%
\pgfpathlineto{\pgfqpoint{5.717671in}{0.413320in}}%
\pgfpathlineto{\pgfqpoint{5.714834in}{0.413320in}}%
\pgfpathlineto{\pgfqpoint{5.712291in}{0.413320in}}%
\pgfpathlineto{\pgfqpoint{5.709490in}{0.413320in}}%
\pgfpathlineto{\pgfqpoint{5.706800in}{0.413320in}}%
\pgfpathlineto{\pgfqpoint{5.704130in}{0.413320in}}%
\pgfpathlineto{\pgfqpoint{5.701453in}{0.413320in}}%
\pgfpathlineto{\pgfqpoint{5.698775in}{0.413320in}}%
\pgfpathlineto{\pgfqpoint{5.696101in}{0.413320in}}%
\pgfpathlineto{\pgfqpoint{5.693473in}{0.413320in}}%
\pgfpathlineto{\pgfqpoint{5.690730in}{0.413320in}}%
\pgfpathlineto{\pgfqpoint{5.688159in}{0.413320in}}%
\pgfpathlineto{\pgfqpoint{5.685385in}{0.413320in}}%
\pgfpathlineto{\pgfqpoint{5.682836in}{0.413320in}}%
\pgfpathlineto{\pgfqpoint{5.680027in}{0.413320in}}%
\pgfpathlineto{\pgfqpoint{5.677486in}{0.413320in}}%
\pgfpathlineto{\pgfqpoint{5.674667in}{0.413320in}}%
\pgfpathlineto{\pgfqpoint{5.671991in}{0.413320in}}%
\pgfpathlineto{\pgfqpoint{5.669313in}{0.413320in}}%
\pgfpathlineto{\pgfqpoint{5.666632in}{0.413320in}}%
\pgfpathlineto{\pgfqpoint{5.664099in}{0.413320in}}%
\pgfpathlineto{\pgfqpoint{5.661273in}{0.413320in}}%
\pgfpathlineto{\pgfqpoint{5.658723in}{0.413320in}}%
\pgfpathlineto{\pgfqpoint{5.655919in}{0.413320in}}%
\pgfpathlineto{\pgfqpoint{5.653376in}{0.413320in}}%
\pgfpathlineto{\pgfqpoint{5.650563in}{0.413320in}}%
\pgfpathlineto{\pgfqpoint{5.648008in}{0.413320in}}%
\pgfpathlineto{\pgfqpoint{5.645243in}{0.413320in}}%
\pgfpathlineto{\pgfqpoint{5.642518in}{0.413320in}}%
\pgfpathlineto{\pgfqpoint{5.639852in}{0.413320in}}%
\pgfpathlineto{\pgfqpoint{5.637172in}{0.413320in}}%
\pgfpathlineto{\pgfqpoint{5.634496in}{0.413320in}}%
\pgfpathlineto{\pgfqpoint{5.631815in}{0.413320in}}%
\pgfpathlineto{\pgfqpoint{5.629232in}{0.413320in}}%
\pgfpathlineto{\pgfqpoint{5.626460in}{0.413320in}}%
\pgfpathlineto{\pgfqpoint{5.623868in}{0.413320in}}%
\pgfpathlineto{\pgfqpoint{5.621102in}{0.413320in}}%
\pgfpathlineto{\pgfqpoint{5.618526in}{0.413320in}}%
\pgfpathlineto{\pgfqpoint{5.615743in}{0.413320in}}%
\pgfpathlineto{\pgfqpoint{5.613235in}{0.413320in}}%
\pgfpathlineto{\pgfqpoint{5.610389in}{0.413320in}}%
\pgfpathlineto{\pgfqpoint{5.607698in}{0.413320in}}%
\pgfpathlineto{\pgfqpoint{5.605073in}{0.413320in}}%
\pgfpathlineto{\pgfqpoint{5.602352in}{0.413320in}}%
\pgfpathlineto{\pgfqpoint{5.599674in}{0.413320in}}%
\pgfpathlineto{\pgfqpoint{5.596999in}{0.413320in}}%
\pgfpathlineto{\pgfqpoint{5.594368in}{0.413320in}}%
\pgfpathlineto{\pgfqpoint{5.591641in}{0.413320in}}%
\pgfpathlineto{\pgfqpoint{5.589040in}{0.413320in}}%
\pgfpathlineto{\pgfqpoint{5.586269in}{0.413320in}}%
\pgfpathlineto{\pgfqpoint{5.583709in}{0.413320in}}%
\pgfpathlineto{\pgfqpoint{5.580914in}{0.413320in}}%
\pgfpathlineto{\pgfqpoint{5.578342in}{0.413320in}}%
\pgfpathlineto{\pgfqpoint{5.575596in}{0.413320in}}%
\pgfpathlineto{\pgfqpoint{5.572893in}{0.413320in}}%
\pgfpathlineto{\pgfqpoint{5.570215in}{0.413320in}}%
\pgfpathlineto{\pgfqpoint{5.567536in}{0.413320in}}%
\pgfpathlineto{\pgfqpoint{5.564940in}{0.413320in}}%
\pgfpathlineto{\pgfqpoint{5.562180in}{0.413320in}}%
\pgfpathlineto{\pgfqpoint{5.559612in}{0.413320in}}%
\pgfpathlineto{\pgfqpoint{5.556822in}{0.413320in}}%
\pgfpathlineto{\pgfqpoint{5.554198in}{0.413320in}}%
\pgfpathlineto{\pgfqpoint{5.551457in}{0.413320in}}%
\pgfpathlineto{\pgfqpoint{5.548921in}{0.413320in}}%
\pgfpathlineto{\pgfqpoint{5.546110in}{0.413320in}}%
\pgfpathlineto{\pgfqpoint{5.543421in}{0.413320in}}%
\pgfpathlineto{\pgfqpoint{5.540750in}{0.413320in}}%
\pgfpathlineto{\pgfqpoint{5.538074in}{0.413320in}}%
\pgfpathlineto{\pgfqpoint{5.535395in}{0.413320in}}%
\pgfpathlineto{\pgfqpoint{5.532717in}{0.413320in}}%
\pgfpathlineto{\pgfqpoint{5.530148in}{0.413320in}}%
\pgfpathlineto{\pgfqpoint{5.527360in}{0.413320in}}%
\pgfpathlineto{\pgfqpoint{5.524756in}{0.413320in}}%
\pgfpathlineto{\pgfqpoint{5.522003in}{0.413320in}}%
\pgfpathlineto{\pgfqpoint{5.519433in}{0.413320in}}%
\pgfpathlineto{\pgfqpoint{5.516646in}{0.413320in}}%
\pgfpathlineto{\pgfqpoint{5.514080in}{0.413320in}}%
\pgfpathlineto{\pgfqpoint{5.511290in}{0.413320in}}%
\pgfpathlineto{\pgfqpoint{5.508612in}{0.413320in}}%
\pgfpathlineto{\pgfqpoint{5.505933in}{0.413320in}}%
\pgfpathlineto{\pgfqpoint{5.503255in}{0.413320in}}%
\pgfpathlineto{\pgfqpoint{5.500687in}{0.413320in}}%
\pgfpathlineto{\pgfqpoint{5.497898in}{0.413320in}}%
\pgfpathlineto{\pgfqpoint{5.495346in}{0.413320in}}%
\pgfpathlineto{\pgfqpoint{5.492541in}{0.413320in}}%
\pgfpathlineto{\pgfqpoint{5.490000in}{0.413320in}}%
\pgfpathlineto{\pgfqpoint{5.487176in}{0.413320in}}%
\pgfpathlineto{\pgfqpoint{5.484641in}{0.413320in}}%
\pgfpathlineto{\pgfqpoint{5.481825in}{0.413320in}}%
\pgfpathlineto{\pgfqpoint{5.479152in}{0.413320in}}%
\pgfpathlineto{\pgfqpoint{5.476458in}{0.413320in}}%
\pgfpathlineto{\pgfqpoint{5.473792in}{0.413320in}}%
\pgfpathlineto{\pgfqpoint{5.471113in}{0.413320in}}%
\pgfpathlineto{\pgfqpoint{5.468425in}{0.413320in}}%
\pgfpathlineto{\pgfqpoint{5.465888in}{0.413320in}}%
\pgfpathlineto{\pgfqpoint{5.463079in}{0.413320in}}%
\pgfpathlineto{\pgfqpoint{5.460489in}{0.413320in}}%
\pgfpathlineto{\pgfqpoint{5.457721in}{0.413320in}}%
\pgfpathlineto{\pgfqpoint{5.455168in}{0.413320in}}%
\pgfpathlineto{\pgfqpoint{5.452365in}{0.413320in}}%
\pgfpathlineto{\pgfqpoint{5.449769in}{0.413320in}}%
\pgfpathlineto{\pgfqpoint{5.447021in}{0.413320in}}%
\pgfpathlineto{\pgfqpoint{5.444328in}{0.413320in}}%
\pgfpathlineto{\pgfqpoint{5.441698in}{0.413320in}}%
\pgfpathlineto{\pgfqpoint{5.438974in}{0.413320in}}%
\pgfpathlineto{\pgfqpoint{5.436295in}{0.413320in}}%
\pgfpathlineto{\pgfqpoint{5.433616in}{0.413320in}}%
\pgfpathlineto{\pgfqpoint{5.431015in}{0.413320in}}%
\pgfpathlineto{\pgfqpoint{5.428259in}{0.413320in}}%
\pgfpathlineto{\pgfqpoint{5.425661in}{0.413320in}}%
\pgfpathlineto{\pgfqpoint{5.422897in}{0.413320in}}%
\pgfpathlineto{\pgfqpoint{5.420304in}{0.413320in}}%
\pgfpathlineto{\pgfqpoint{5.417547in}{0.413320in}}%
\pgfpathlineto{\pgfqpoint{5.414954in}{0.413320in}}%
\pgfpathlineto{\pgfqpoint{5.412190in}{0.413320in}}%
\pgfpathlineto{\pgfqpoint{5.409507in}{0.413320in}}%
\pgfpathlineto{\pgfqpoint{5.406832in}{0.413320in}}%
\pgfpathlineto{\pgfqpoint{5.404154in}{0.413320in}}%
\pgfpathlineto{\pgfqpoint{5.401576in}{0.413320in}}%
\pgfpathlineto{\pgfqpoint{5.398784in}{0.413320in}}%
\pgfpathlineto{\pgfqpoint{5.396219in}{0.413320in}}%
\pgfpathlineto{\pgfqpoint{5.393441in}{0.413320in}}%
\pgfpathlineto{\pgfqpoint{5.390900in}{0.413320in}}%
\pgfpathlineto{\pgfqpoint{5.388083in}{0.413320in}}%
\pgfpathlineto{\pgfqpoint{5.385550in}{0.413320in}}%
\pgfpathlineto{\pgfqpoint{5.382725in}{0.413320in}}%
\pgfpathlineto{\pgfqpoint{5.380048in}{0.413320in}}%
\pgfpathlineto{\pgfqpoint{5.377370in}{0.413320in}}%
\pgfpathlineto{\pgfqpoint{5.374692in}{0.413320in}}%
\pgfpathlineto{\pgfqpoint{5.372013in}{0.413320in}}%
\pgfpathlineto{\pgfqpoint{5.369335in}{0.413320in}}%
\pgfpathlineto{\pgfqpoint{5.366727in}{0.413320in}}%
\pgfpathlineto{\pgfqpoint{5.363966in}{0.413320in}}%
\pgfpathlineto{\pgfqpoint{5.361370in}{0.413320in}}%
\pgfpathlineto{\pgfqpoint{5.358612in}{0.413320in}}%
\pgfpathlineto{\pgfqpoint{5.356056in}{0.413320in}}%
\pgfpathlineto{\pgfqpoint{5.353262in}{0.413320in}}%
\pgfpathlineto{\pgfqpoint{5.350723in}{0.413320in}}%
\pgfpathlineto{\pgfqpoint{5.347905in}{0.413320in}}%
\pgfpathlineto{\pgfqpoint{5.345224in}{0.413320in}}%
\pgfpathlineto{\pgfqpoint{5.342549in}{0.413320in}}%
\pgfpathlineto{\pgfqpoint{5.339872in}{0.413320in}}%
\pgfpathlineto{\pgfqpoint{5.337353in}{0.413320in}}%
\pgfpathlineto{\pgfqpoint{5.334510in}{0.413320in}}%
\pgfpathlineto{\pgfqpoint{5.331973in}{0.413320in}}%
\pgfpathlineto{\pgfqpoint{5.329159in}{0.413320in}}%
\pgfpathlineto{\pgfqpoint{5.326564in}{0.413320in}}%
\pgfpathlineto{\pgfqpoint{5.323802in}{0.413320in}}%
\pgfpathlineto{\pgfqpoint{5.321256in}{0.413320in}}%
\pgfpathlineto{\pgfqpoint{5.318430in}{0.413320in}}%
\pgfpathlineto{\pgfqpoint{5.315754in}{0.413320in}}%
\pgfpathlineto{\pgfqpoint{5.313089in}{0.413320in}}%
\pgfpathlineto{\pgfqpoint{5.310411in}{0.413320in}}%
\pgfpathlineto{\pgfqpoint{5.307731in}{0.413320in}}%
\pgfpathlineto{\pgfqpoint{5.305054in}{0.413320in}}%
\pgfpathlineto{\pgfqpoint{5.302443in}{0.413320in}}%
\pgfpathlineto{\pgfqpoint{5.299696in}{0.413320in}}%
\pgfpathlineto{\pgfqpoint{5.297140in}{0.413320in}}%
\pgfpathlineto{\pgfqpoint{5.294339in}{0.413320in}}%
\pgfpathlineto{\pgfqpoint{5.291794in}{0.413320in}}%
\pgfpathlineto{\pgfqpoint{5.288984in}{0.413320in}}%
\pgfpathlineto{\pgfqpoint{5.286436in}{0.413320in}}%
\pgfpathlineto{\pgfqpoint{5.283631in}{0.413320in}}%
\pgfpathlineto{\pgfqpoint{5.280947in}{0.413320in}}%
\pgfpathlineto{\pgfqpoint{5.278322in}{0.413320in}}%
\pgfpathlineto{\pgfqpoint{5.275589in}{0.413320in}}%
\pgfpathlineto{\pgfqpoint{5.272913in}{0.413320in}}%
\pgfpathlineto{\pgfqpoint{5.270238in}{0.413320in}}%
\pgfpathlineto{\pgfqpoint{5.267691in}{0.413320in}}%
\pgfpathlineto{\pgfqpoint{5.264876in}{0.413320in}}%
\pgfpathlineto{\pgfqpoint{5.262264in}{0.413320in}}%
\pgfpathlineto{\pgfqpoint{5.259511in}{0.413320in}}%
\pgfpathlineto{\pgfqpoint{5.256973in}{0.413320in}}%
\pgfpathlineto{\pgfqpoint{5.254236in}{0.413320in}}%
\pgfpathlineto{\pgfqpoint{5.251590in}{0.413320in}}%
\pgfpathlineto{\pgfqpoint{5.248816in}{0.413320in}}%
\pgfpathlineto{\pgfqpoint{5.246130in}{0.413320in}}%
\pgfpathlineto{\pgfqpoint{5.243445in}{0.413320in}}%
\pgfpathlineto{\pgfqpoint{5.240777in}{0.413320in}}%
\pgfpathlineto{\pgfqpoint{5.238173in}{0.413320in}}%
\pgfpathlineto{\pgfqpoint{5.235409in}{0.413320in}}%
\pgfpathlineto{\pgfqpoint{5.232855in}{0.413320in}}%
\pgfpathlineto{\pgfqpoint{5.230059in}{0.413320in}}%
\pgfpathlineto{\pgfqpoint{5.227470in}{0.413320in}}%
\pgfpathlineto{\pgfqpoint{5.224695in}{0.413320in}}%
\pgfpathlineto{\pgfqpoint{5.222151in}{0.413320in}}%
\pgfpathlineto{\pgfqpoint{5.219345in}{0.413320in}}%
\pgfpathlineto{\pgfqpoint{5.216667in}{0.413320in}}%
\pgfpathlineto{\pgfqpoint{5.214027in}{0.413320in}}%
\pgfpathlineto{\pgfqpoint{5.211299in}{0.413320in}}%
\pgfpathlineto{\pgfqpoint{5.208630in}{0.413320in}}%
\pgfpathlineto{\pgfqpoint{5.205952in}{0.413320in}}%
\pgfpathlineto{\pgfqpoint{5.203388in}{0.413320in}}%
\pgfpathlineto{\pgfqpoint{5.200594in}{0.413320in}}%
\pgfpathlineto{\pgfqpoint{5.198008in}{0.413320in}}%
\pgfpathlineto{\pgfqpoint{5.195239in}{0.413320in}}%
\pgfpathlineto{\pgfqpoint{5.192680in}{0.413320in}}%
\pgfpathlineto{\pgfqpoint{5.189880in}{0.413320in}}%
\pgfpathlineto{\pgfqpoint{5.187294in}{0.413320in}}%
\pgfpathlineto{\pgfqpoint{5.184522in}{0.413320in}}%
\pgfpathlineto{\pgfqpoint{5.181848in}{0.413320in}}%
\pgfpathlineto{\pgfqpoint{5.179188in}{0.413320in}}%
\pgfpathlineto{\pgfqpoint{5.176477in}{0.413320in}}%
\pgfpathlineto{\pgfqpoint{5.173925in}{0.413320in}}%
\pgfpathlineto{\pgfqpoint{5.171133in}{0.413320in}}%
\pgfpathlineto{\pgfqpoint{5.168591in}{0.413320in}}%
\pgfpathlineto{\pgfqpoint{5.165775in}{0.413320in}}%
\pgfpathlineto{\pgfqpoint{5.163243in}{0.413320in}}%
\pgfpathlineto{\pgfqpoint{5.160420in}{0.413320in}}%
\pgfpathlineto{\pgfqpoint{5.157815in}{0.413320in}}%
\pgfpathlineto{\pgfqpoint{5.155059in}{0.413320in}}%
\pgfpathlineto{\pgfqpoint{5.152382in}{0.413320in}}%
\pgfpathlineto{\pgfqpoint{5.149734in}{0.413320in}}%
\pgfpathlineto{\pgfqpoint{5.147029in}{0.413320in}}%
\pgfpathlineto{\pgfqpoint{5.144349in}{0.413320in}}%
\pgfpathlineto{\pgfqpoint{5.141660in}{0.413320in}}%
\pgfpathlineto{\pgfqpoint{5.139072in}{0.413320in}}%
\pgfpathlineto{\pgfqpoint{5.136311in}{0.413320in}}%
\pgfpathlineto{\pgfqpoint{5.133716in}{0.413320in}}%
\pgfpathlineto{\pgfqpoint{5.130953in}{0.413320in}}%
\pgfpathlineto{\pgfqpoint{5.128421in}{0.413320in}}%
\pgfpathlineto{\pgfqpoint{5.125599in}{0.413320in}}%
\pgfpathlineto{\pgfqpoint{5.123042in}{0.413320in}}%
\pgfpathlineto{\pgfqpoint{5.120243in}{0.413320in}}%
\pgfpathlineto{\pgfqpoint{5.117550in}{0.413320in}}%
\pgfpathlineto{\pgfqpoint{5.114887in}{0.413320in}}%
\pgfpathlineto{\pgfqpoint{5.112209in}{0.413320in}}%
\pgfpathlineto{\pgfqpoint{5.109530in}{0.413320in}}%
\pgfpathlineto{\pgfqpoint{5.106842in}{0.413320in}}%
\pgfpathlineto{\pgfqpoint{5.104312in}{0.413320in}}%
\pgfpathlineto{\pgfqpoint{5.101496in}{0.413320in}}%
\pgfpathlineto{\pgfqpoint{5.098948in}{0.413320in}}%
\pgfpathlineto{\pgfqpoint{5.096142in}{0.413320in}}%
\pgfpathlineto{\pgfqpoint{5.093579in}{0.413320in}}%
\pgfpathlineto{\pgfqpoint{5.090788in}{0.413320in}}%
\pgfpathlineto{\pgfqpoint{5.088103in}{0.413320in}}%
\pgfpathlineto{\pgfqpoint{5.085426in}{0.413320in}}%
\pgfpathlineto{\pgfqpoint{5.082746in}{0.413320in}}%
\pgfpathlineto{\pgfqpoint{5.080067in}{0.413320in}}%
\pgfpathlineto{\pgfqpoint{5.077390in}{0.413320in}}%
\pgfpathlineto{\pgfqpoint{5.074851in}{0.413320in}}%
\pgfpathlineto{\pgfqpoint{5.072030in}{0.413320in}}%
\pgfpathlineto{\pgfqpoint{5.069463in}{0.413320in}}%
\pgfpathlineto{\pgfqpoint{5.066677in}{0.413320in}}%
\pgfpathlineto{\pgfqpoint{5.064144in}{0.413320in}}%
\pgfpathlineto{\pgfqpoint{5.061315in}{0.413320in}}%
\pgfpathlineto{\pgfqpoint{5.058711in}{0.413320in}}%
\pgfpathlineto{\pgfqpoint{5.055952in}{0.413320in}}%
\pgfpathlineto{\pgfqpoint{5.053284in}{0.413320in}}%
\pgfpathlineto{\pgfqpoint{5.050606in}{0.413320in}}%
\pgfpathlineto{\pgfqpoint{5.047924in}{0.413320in}}%
\pgfpathlineto{\pgfqpoint{5.045249in}{0.413320in}}%
\pgfpathlineto{\pgfqpoint{5.042572in}{0.413320in}}%
\pgfpathlineto{\pgfqpoint{5.039962in}{0.413320in}}%
\pgfpathlineto{\pgfqpoint{5.037214in}{0.413320in}}%
\pgfpathlineto{\pgfqpoint{5.034649in}{0.413320in}}%
\pgfpathlineto{\pgfqpoint{5.031849in}{0.413320in}}%
\pgfpathlineto{\pgfqpoint{5.029275in}{0.413320in}}%
\pgfpathlineto{\pgfqpoint{5.026501in}{0.413320in}}%
\pgfpathlineto{\pgfqpoint{5.023927in}{0.413320in}}%
\pgfpathlineto{\pgfqpoint{5.021147in}{0.413320in}}%
\pgfpathlineto{\pgfqpoint{5.018466in}{0.413320in}}%
\pgfpathlineto{\pgfqpoint{5.015820in}{0.413320in}}%
\pgfpathlineto{\pgfqpoint{5.013104in}{0.413320in}}%
\pgfpathlineto{\pgfqpoint{5.010562in}{0.413320in}}%
\pgfpathlineto{\pgfqpoint{5.007751in}{0.413320in}}%
\pgfpathlineto{\pgfqpoint{5.005178in}{0.413320in}}%
\pgfpathlineto{\pgfqpoint{5.002384in}{0.413320in}}%
\pgfpathlineto{\pgfqpoint{4.999780in}{0.413320in}}%
\pgfpathlineto{\pgfqpoint{4.997028in}{0.413320in}}%
\pgfpathlineto{\pgfqpoint{4.994390in}{0.413320in}}%
\pgfpathlineto{\pgfqpoint{4.991683in}{0.413320in}}%
\pgfpathlineto{\pgfqpoint{4.989001in}{0.413320in}}%
\pgfpathlineto{\pgfqpoint{4.986325in}{0.413320in}}%
\pgfpathlineto{\pgfqpoint{4.983637in}{0.413320in}}%
\pgfpathlineto{\pgfqpoint{4.980967in}{0.413320in}}%
\pgfpathlineto{\pgfqpoint{4.978287in}{0.413320in}}%
\pgfpathlineto{\pgfqpoint{4.975703in}{0.413320in}}%
\pgfpathlineto{\pgfqpoint{4.972933in}{0.413320in}}%
\pgfpathlineto{\pgfqpoint{4.970314in}{0.413320in}}%
\pgfpathlineto{\pgfqpoint{4.967575in}{0.413320in}}%
\pgfpathlineto{\pgfqpoint{4.965002in}{0.413320in}}%
\pgfpathlineto{\pgfqpoint{4.962219in}{0.413320in}}%
\pgfpathlineto{\pgfqpoint{4.959689in}{0.413320in}}%
\pgfpathlineto{\pgfqpoint{4.956862in}{0.413320in}}%
\pgfpathlineto{\pgfqpoint{4.954182in}{0.413320in}}%
\pgfpathlineto{\pgfqpoint{4.951504in}{0.413320in}}%
\pgfpathlineto{\pgfqpoint{4.948827in}{0.413320in}}%
\pgfpathlineto{\pgfqpoint{4.946151in}{0.413320in}}%
\pgfpathlineto{\pgfqpoint{4.943466in}{0.413320in}}%
\pgfpathlineto{\pgfqpoint{4.940881in}{0.413320in}}%
\pgfpathlineto{\pgfqpoint{4.938112in}{0.413320in}}%
\pgfpathlineto{\pgfqpoint{4.935515in}{0.413320in}}%
\pgfpathlineto{\pgfqpoint{4.932742in}{0.413320in}}%
\pgfpathlineto{\pgfqpoint{4.930170in}{0.413320in}}%
\pgfpathlineto{\pgfqpoint{4.927400in}{0.413320in}}%
\pgfpathlineto{\pgfqpoint{4.924708in}{0.413320in}}%
\pgfpathlineto{\pgfqpoint{4.922041in}{0.413320in}}%
\pgfpathlineto{\pgfqpoint{4.919352in}{0.413320in}}%
\pgfpathlineto{\pgfqpoint{4.916681in}{0.413320in}}%
\pgfpathlineto{\pgfqpoint{4.914009in}{0.413320in}}%
\pgfpathlineto{\pgfqpoint{4.911435in}{0.413320in}}%
\pgfpathlineto{\pgfqpoint{4.908648in}{0.413320in}}%
\pgfpathlineto{\pgfqpoint{4.906096in}{0.413320in}}%
\pgfpathlineto{\pgfqpoint{4.903295in}{0.413320in}}%
\pgfpathlineto{\pgfqpoint{4.900712in}{0.413320in}}%
\pgfpathlineto{\pgfqpoint{4.897938in}{0.413320in}}%
\pgfpathlineto{\pgfqpoint{4.895399in}{0.413320in}}%
\pgfpathlineto{\pgfqpoint{4.892611in}{0.413320in}}%
\pgfpathlineto{\pgfqpoint{4.889902in}{0.413320in}}%
\pgfpathlineto{\pgfqpoint{4.887211in}{0.413320in}}%
\pgfpathlineto{\pgfqpoint{4.884540in}{0.413320in}}%
\pgfpathlineto{\pgfqpoint{4.881864in}{0.413320in}}%
\pgfpathlineto{\pgfqpoint{4.879180in}{0.413320in}}%
\pgfpathlineto{\pgfqpoint{4.876636in}{0.413320in}}%
\pgfpathlineto{\pgfqpoint{4.873832in}{0.413320in}}%
\pgfpathlineto{\pgfqpoint{4.871209in}{0.413320in}}%
\pgfpathlineto{\pgfqpoint{4.868474in}{0.413320in}}%
\pgfpathlineto{\pgfqpoint{4.865910in}{0.413320in}}%
\pgfpathlineto{\pgfqpoint{4.863116in}{0.413320in}}%
\pgfpathlineto{\pgfqpoint{4.860544in}{0.413320in}}%
\pgfpathlineto{\pgfqpoint{4.857807in}{0.413320in}}%
\pgfpathlineto{\pgfqpoint{4.855070in}{0.413320in}}%
\pgfpathlineto{\pgfqpoint{4.852404in}{0.413320in}}%
\pgfpathlineto{\pgfqpoint{4.849715in}{0.413320in}}%
\pgfpathlineto{\pgfqpoint{4.847127in}{0.413320in}}%
\pgfpathlineto{\pgfqpoint{4.844361in}{0.413320in}}%
\pgfpathlineto{\pgfqpoint{4.842380in}{0.413320in}}%
\pgfpathlineto{\pgfqpoint{4.839922in}{0.413320in}}%
\pgfpathlineto{\pgfqpoint{4.837992in}{0.413320in}}%
\pgfpathlineto{\pgfqpoint{4.833657in}{0.413320in}}%
\pgfpathlineto{\pgfqpoint{4.831045in}{0.413320in}}%
\pgfpathlineto{\pgfqpoint{4.828291in}{0.413320in}}%
\pgfpathlineto{\pgfqpoint{4.825619in}{0.413320in}}%
\pgfpathlineto{\pgfqpoint{4.822945in}{0.413320in}}%
\pgfpathlineto{\pgfqpoint{4.820265in}{0.413320in}}%
\pgfpathlineto{\pgfqpoint{4.817587in}{0.413320in}}%
\pgfpathlineto{\pgfqpoint{4.814907in}{0.413320in}}%
\pgfpathlineto{\pgfqpoint{4.812377in}{0.413320in}}%
\pgfpathlineto{\pgfqpoint{4.809538in}{0.413320in}}%
\pgfpathlineto{\pgfqpoint{4.807017in}{0.413320in}}%
\pgfpathlineto{\pgfqpoint{4.804193in}{0.413320in}}%
\pgfpathlineto{\pgfqpoint{4.801586in}{0.413320in}}%
\pgfpathlineto{\pgfqpoint{4.798830in}{0.413320in}}%
\pgfpathlineto{\pgfqpoint{4.796274in}{0.413320in}}%
\pgfpathlineto{\pgfqpoint{4.793512in}{0.413320in}}%
\pgfpathlineto{\pgfqpoint{4.790798in}{0.413320in}}%
\pgfpathlineto{\pgfqpoint{4.788116in}{0.413320in}}%
\pgfpathlineto{\pgfqpoint{4.785445in}{0.413320in}}%
\pgfpathlineto{\pgfqpoint{4.782872in}{0.413320in}}%
\pgfpathlineto{\pgfqpoint{4.780083in}{0.413320in}}%
\pgfpathlineto{\pgfqpoint{4.777535in}{0.413320in}}%
\pgfpathlineto{\pgfqpoint{4.774732in}{0.413320in}}%
\pgfpathlineto{\pgfqpoint{4.772198in}{0.413320in}}%
\pgfpathlineto{\pgfqpoint{4.769367in}{0.413320in}}%
\pgfpathlineto{\pgfqpoint{4.766783in}{0.413320in}}%
\pgfpathlineto{\pgfqpoint{4.764018in}{0.413320in}}%
\pgfpathlineto{\pgfqpoint{4.761337in}{0.413320in}}%
\pgfpathlineto{\pgfqpoint{4.758653in}{0.413320in}}%
\pgfpathlineto{\pgfqpoint{4.755983in}{0.413320in}}%
\pgfpathlineto{\pgfqpoint{4.753298in}{0.413320in}}%
\pgfpathlineto{\pgfqpoint{4.750627in}{0.413320in}}%
\pgfpathlineto{\pgfqpoint{4.748081in}{0.413320in}}%
\pgfpathlineto{\pgfqpoint{4.745256in}{0.413320in}}%
\pgfpathlineto{\pgfqpoint{4.742696in}{0.413320in}}%
\pgfpathlineto{\pgfqpoint{4.739912in}{0.413320in}}%
\pgfpathlineto{\pgfqpoint{4.737348in}{0.413320in}}%
\pgfpathlineto{\pgfqpoint{4.734552in}{0.413320in}}%
\pgfpathlineto{\pgfqpoint{4.731901in}{0.413320in}}%
\pgfpathlineto{\pgfqpoint{4.729233in}{0.413320in}}%
\pgfpathlineto{\pgfqpoint{4.726508in}{0.413320in}}%
\pgfpathlineto{\pgfqpoint{4.723873in}{0.413320in}}%
\pgfpathlineto{\pgfqpoint{4.721160in}{0.413320in}}%
\pgfpathlineto{\pgfqpoint{4.718486in}{0.413320in}}%
\pgfpathlineto{\pgfqpoint{4.715806in}{0.413320in}}%
\pgfpathlineto{\pgfqpoint{4.713275in}{0.413320in}}%
\pgfpathlineto{\pgfqpoint{4.710437in}{0.413320in}}%
\pgfpathlineto{\pgfqpoint{4.707824in}{0.413320in}}%
\pgfpathlineto{\pgfqpoint{4.705094in}{0.413320in}}%
\pgfpathlineto{\pgfqpoint{4.702517in}{0.413320in}}%
\pgfpathlineto{\pgfqpoint{4.699734in}{0.413320in}}%
\pgfpathlineto{\pgfqpoint{4.697170in}{0.413320in}}%
\pgfpathlineto{\pgfqpoint{4.694381in}{0.413320in}}%
\pgfpathlineto{\pgfqpoint{4.691694in}{0.413320in}}%
\pgfpathlineto{\pgfqpoint{4.689051in}{0.413320in}}%
\pgfpathlineto{\pgfqpoint{4.686337in}{0.413320in}}%
\pgfpathlineto{\pgfqpoint{4.683799in}{0.413320in}}%
\pgfpathlineto{\pgfqpoint{4.680988in}{0.413320in}}%
\pgfpathlineto{\pgfqpoint{4.678448in}{0.413320in}}%
\pgfpathlineto{\pgfqpoint{4.675619in}{0.413320in}}%
\pgfpathlineto{\pgfqpoint{4.673068in}{0.413320in}}%
\pgfpathlineto{\pgfqpoint{4.670261in}{0.413320in}}%
\pgfpathlineto{\pgfqpoint{4.667764in}{0.413320in}}%
\pgfpathlineto{\pgfqpoint{4.664923in}{0.413320in}}%
\pgfpathlineto{\pgfqpoint{4.662237in}{0.413320in}}%
\pgfpathlineto{\pgfqpoint{4.659590in}{0.413320in}}%
\pgfpathlineto{\pgfqpoint{4.656873in}{0.413320in}}%
\pgfpathlineto{\pgfqpoint{4.654203in}{0.413320in}}%
\pgfpathlineto{\pgfqpoint{4.651524in}{0.413320in}}%
\pgfpathlineto{\pgfqpoint{4.648922in}{0.413320in}}%
\pgfpathlineto{\pgfqpoint{4.646169in}{0.413320in}}%
\pgfpathlineto{\pgfqpoint{4.643628in}{0.413320in}}%
\pgfpathlineto{\pgfqpoint{4.640809in}{0.413320in}}%
\pgfpathlineto{\pgfqpoint{4.638204in}{0.413320in}}%
\pgfpathlineto{\pgfqpoint{4.635445in}{0.413320in}}%
\pgfpathlineto{\pgfqpoint{4.632902in}{0.413320in}}%
\pgfpathlineto{\pgfqpoint{4.630096in}{0.413320in}}%
\pgfpathlineto{\pgfqpoint{4.627411in}{0.413320in}}%
\pgfpathlineto{\pgfqpoint{4.624741in}{0.413320in}}%
\pgfpathlineto{\pgfqpoint{4.622056in}{0.413320in}}%
\pgfpathlineto{\pgfqpoint{4.619529in}{0.413320in}}%
\pgfpathlineto{\pgfqpoint{4.616702in}{0.413320in}}%
\pgfpathlineto{\pgfqpoint{4.614134in}{0.413320in}}%
\pgfpathlineto{\pgfqpoint{4.611350in}{0.413320in}}%
\pgfpathlineto{\pgfqpoint{4.608808in}{0.413320in}}%
\pgfpathlineto{\pgfqpoint{4.605990in}{0.413320in}}%
\pgfpathlineto{\pgfqpoint{4.603430in}{0.413320in}}%
\pgfpathlineto{\pgfqpoint{4.600633in}{0.413320in}}%
\pgfpathlineto{\pgfqpoint{4.597951in}{0.413320in}}%
\pgfpathlineto{\pgfqpoint{4.595281in}{0.413320in}}%
\pgfpathlineto{\pgfqpoint{4.592589in}{0.413320in}}%
\pgfpathlineto{\pgfqpoint{4.589920in}{0.413320in}}%
\pgfpathlineto{\pgfqpoint{4.587244in}{0.413320in}}%
\pgfpathlineto{\pgfqpoint{4.584672in}{0.413320in}}%
\pgfpathlineto{\pgfqpoint{4.581888in}{0.413320in}}%
\pgfpathlineto{\pgfqpoint{4.579305in}{0.413320in}}%
\pgfpathlineto{\pgfqpoint{4.576531in}{0.413320in}}%
\pgfpathlineto{\pgfqpoint{4.573947in}{0.413320in}}%
\pgfpathlineto{\pgfqpoint{4.571171in}{0.413320in}}%
\pgfpathlineto{\pgfqpoint{4.568612in}{0.413320in}}%
\pgfpathlineto{\pgfqpoint{4.565820in}{0.413320in}}%
\pgfpathlineto{\pgfqpoint{4.563125in}{0.413320in}}%
\pgfpathlineto{\pgfqpoint{4.560448in}{0.413320in}}%
\pgfpathlineto{\pgfqpoint{4.557777in}{0.413320in}}%
\pgfpathlineto{\pgfqpoint{4.555106in}{0.413320in}}%
\pgfpathlineto{\pgfqpoint{4.552425in}{0.413320in}}%
\pgfpathlineto{\pgfqpoint{4.549822in}{0.413320in}}%
\pgfpathlineto{\pgfqpoint{4.547064in}{0.413320in}}%
\pgfpathlineto{\pgfqpoint{4.544464in}{0.413320in}}%
\pgfpathlineto{\pgfqpoint{4.541711in}{0.413320in}}%
\pgfpathlineto{\pgfqpoint{4.539144in}{0.413320in}}%
\pgfpathlineto{\pgfqpoint{4.536400in}{0.413320in}}%
\pgfpathlineto{\pgfqpoint{4.533764in}{0.413320in}}%
\pgfpathlineto{\pgfqpoint{4.530990in}{0.413320in}}%
\pgfpathlineto{\pgfqpoint{4.528307in}{0.413320in}}%
\pgfpathlineto{\pgfqpoint{4.525640in}{0.413320in}}%
\pgfpathlineto{\pgfqpoint{4.522962in}{0.413320in}}%
\pgfpathlineto{\pgfqpoint{4.520345in}{0.413320in}}%
\pgfpathlineto{\pgfqpoint{4.517598in}{0.413320in}}%
\pgfpathlineto{\pgfqpoint{4.515080in}{0.413320in}}%
\pgfpathlineto{\pgfqpoint{4.512246in}{0.413320in}}%
\pgfpathlineto{\pgfqpoint{4.509643in}{0.413320in}}%
\pgfpathlineto{\pgfqpoint{4.506893in}{0.413320in}}%
\pgfpathlineto{\pgfqpoint{4.504305in}{0.413320in}}%
\pgfpathlineto{\pgfqpoint{4.501529in}{0.413320in}}%
\pgfpathlineto{\pgfqpoint{4.498850in}{0.413320in}}%
\pgfpathlineto{\pgfqpoint{4.496167in}{0.413320in}}%
\pgfpathlineto{\pgfqpoint{4.493492in}{0.413320in}}%
\pgfpathlineto{\pgfqpoint{4.490822in}{0.413320in}}%
\pgfpathlineto{\pgfqpoint{4.488130in}{0.413320in}}%
\pgfpathlineto{\pgfqpoint{4.485581in}{0.413320in}}%
\pgfpathlineto{\pgfqpoint{4.482778in}{0.413320in}}%
\pgfpathlineto{\pgfqpoint{4.480201in}{0.413320in}}%
\pgfpathlineto{\pgfqpoint{4.477430in}{0.413320in}}%
\pgfpathlineto{\pgfqpoint{4.474861in}{0.413320in}}%
\pgfpathlineto{\pgfqpoint{4.472059in}{0.413320in}}%
\pgfpathlineto{\pgfqpoint{4.469492in}{0.413320in}}%
\pgfpathlineto{\pgfqpoint{4.466717in}{0.413320in}}%
\pgfpathlineto{\pgfqpoint{4.464029in}{0.413320in}}%
\pgfpathlineto{\pgfqpoint{4.461367in}{0.413320in}}%
\pgfpathlineto{\pgfqpoint{4.458681in}{0.413320in}}%
\pgfpathlineto{\pgfqpoint{4.456138in}{0.413320in}}%
\pgfpathlineto{\pgfqpoint{4.453312in}{0.413320in}}%
\pgfpathlineto{\pgfqpoint{4.450767in}{0.413320in}}%
\pgfpathlineto{\pgfqpoint{4.447965in}{0.413320in}}%
\pgfpathlineto{\pgfqpoint{4.445423in}{0.413320in}}%
\pgfpathlineto{\pgfqpoint{4.442611in}{0.413320in}}%
\pgfpathlineto{\pgfqpoint{4.440041in}{0.413320in}}%
\pgfpathlineto{\pgfqpoint{4.437253in}{0.413320in}}%
\pgfpathlineto{\pgfqpoint{4.434569in}{0.413320in}}%
\pgfpathlineto{\pgfqpoint{4.431901in}{0.413320in}}%
\pgfpathlineto{\pgfqpoint{4.429220in}{0.413320in}}%
\pgfpathlineto{\pgfqpoint{4.426534in}{0.413320in}}%
\pgfpathlineto{\pgfqpoint{4.423863in}{0.413320in}}%
\pgfpathlineto{\pgfqpoint{4.421292in}{0.413320in}}%
\pgfpathlineto{\pgfqpoint{4.418506in}{0.413320in}}%
\pgfpathlineto{\pgfqpoint{4.415932in}{0.413320in}}%
\pgfpathlineto{\pgfqpoint{4.413149in}{0.413320in}}%
\pgfpathlineto{\pgfqpoint{4.410587in}{0.413320in}}%
\pgfpathlineto{\pgfqpoint{4.407788in}{0.413320in}}%
\pgfpathlineto{\pgfqpoint{4.405234in}{0.413320in}}%
\pgfpathlineto{\pgfqpoint{4.402468in}{0.413320in}}%
\pgfpathlineto{\pgfqpoint{4.399745in}{0.413320in}}%
\pgfpathlineto{\pgfqpoint{4.397076in}{0.413320in}}%
\pgfpathlineto{\pgfqpoint{4.394400in}{0.413320in}}%
\pgfpathlineto{\pgfqpoint{4.391721in}{0.413320in}}%
\pgfpathlineto{\pgfqpoint{4.389044in}{0.413320in}}%
\pgfpathlineto{\pgfqpoint{4.386431in}{0.413320in}}%
\pgfpathlineto{\pgfqpoint{4.383674in}{0.413320in}}%
\pgfpathlineto{\pgfqpoint{4.381097in}{0.413320in}}%
\pgfpathlineto{\pgfqpoint{4.378329in}{0.413320in}}%
\pgfpathlineto{\pgfqpoint{4.375761in}{0.413320in}}%
\pgfpathlineto{\pgfqpoint{4.372976in}{0.413320in}}%
\pgfpathlineto{\pgfqpoint{4.370437in}{0.413320in}}%
\pgfpathlineto{\pgfqpoint{4.367646in}{0.413320in}}%
\pgfpathlineto{\pgfqpoint{4.364936in}{0.413320in}}%
\pgfpathlineto{\pgfqpoint{4.362270in}{0.413320in}}%
\pgfpathlineto{\pgfqpoint{4.359582in}{0.413320in}}%
\pgfpathlineto{\pgfqpoint{4.357014in}{0.413320in}}%
\pgfpathlineto{\pgfqpoint{4.354224in}{0.413320in}}%
\pgfpathlineto{\pgfqpoint{4.351645in}{0.413320in}}%
\pgfpathlineto{\pgfqpoint{4.348868in}{0.413320in}}%
\pgfpathlineto{\pgfqpoint{4.346263in}{0.413320in}}%
\pgfpathlineto{\pgfqpoint{4.343510in}{0.413320in}}%
\pgfpathlineto{\pgfqpoint{4.340976in}{0.413320in}}%
\pgfpathlineto{\pgfqpoint{4.338154in}{0.413320in}}%
\pgfpathlineto{\pgfqpoint{4.335463in}{0.413320in}}%
\pgfpathlineto{\pgfqpoint{4.332796in}{0.413320in}}%
\pgfpathlineto{\pgfqpoint{4.330118in}{0.413320in}}%
\pgfpathlineto{\pgfqpoint{4.327440in}{0.413320in}}%
\pgfpathlineto{\pgfqpoint{4.324760in}{0.413320in}}%
\pgfpathlineto{\pgfqpoint{4.322181in}{0.413320in}}%
\pgfpathlineto{\pgfqpoint{4.319405in}{0.413320in}}%
\pgfpathlineto{\pgfqpoint{4.316856in}{0.413320in}}%
\pgfpathlineto{\pgfqpoint{4.314032in}{0.413320in}}%
\pgfpathlineto{\pgfqpoint{4.311494in}{0.413320in}}%
\pgfpathlineto{\pgfqpoint{4.308691in}{0.413320in}}%
\pgfpathlineto{\pgfqpoint{4.306118in}{0.413320in}}%
\pgfpathlineto{\pgfqpoint{4.303357in}{0.413320in}}%
\pgfpathlineto{\pgfqpoint{4.300656in}{0.413320in}}%
\pgfpathlineto{\pgfqpoint{4.297977in}{0.413320in}}%
\pgfpathlineto{\pgfqpoint{4.295299in}{0.413320in}}%
\pgfpathlineto{\pgfqpoint{4.292786in}{0.413320in}}%
\pgfpathlineto{\pgfqpoint{4.289936in}{0.413320in}}%
\pgfpathlineto{\pgfqpoint{4.287399in}{0.413320in}}%
\pgfpathlineto{\pgfqpoint{4.284586in}{0.413320in}}%
\pgfpathlineto{\pgfqpoint{4.282000in}{0.413320in}}%
\pgfpathlineto{\pgfqpoint{4.279212in}{0.413320in}}%
\pgfpathlineto{\pgfqpoint{4.276635in}{0.413320in}}%
\pgfpathlineto{\pgfqpoint{4.273874in}{0.413320in}}%
\pgfpathlineto{\pgfqpoint{4.271187in}{0.413320in}}%
\pgfpathlineto{\pgfqpoint{4.268590in}{0.413320in}}%
\pgfpathlineto{\pgfqpoint{4.265824in}{0.413320in}}%
\pgfpathlineto{\pgfqpoint{4.263157in}{0.413320in}}%
\pgfpathlineto{\pgfqpoint{4.260477in}{0.413320in}}%
\pgfpathlineto{\pgfqpoint{4.257958in}{0.413320in}}%
\pgfpathlineto{\pgfqpoint{4.255120in}{0.413320in}}%
\pgfpathlineto{\pgfqpoint{4.252581in}{0.413320in}}%
\pgfpathlineto{\pgfqpoint{4.249767in}{0.413320in}}%
\pgfpathlineto{\pgfqpoint{4.247225in}{0.413320in}}%
\pgfpathlineto{\pgfqpoint{4.244394in}{0.413320in}}%
\pgfpathlineto{\pgfqpoint{4.241900in}{0.413320in}}%
\pgfpathlineto{\pgfqpoint{4.239084in}{0.413320in}}%
\pgfpathlineto{\pgfqpoint{4.236375in}{0.413320in}}%
\pgfpathlineto{\pgfqpoint{4.233691in}{0.413320in}}%
\pgfpathlineto{\pgfqpoint{4.231013in}{0.413320in}}%
\pgfpathlineto{\pgfqpoint{4.228331in}{0.413320in}}%
\pgfpathlineto{\pgfqpoint{4.225654in}{0.413320in}}%
\pgfpathlineto{\pgfqpoint{4.223082in}{0.413320in}}%
\pgfpathlineto{\pgfqpoint{4.220304in}{0.413320in}}%
\pgfpathlineto{\pgfqpoint{4.217694in}{0.413320in}}%
\pgfpathlineto{\pgfqpoint{4.214948in}{0.413320in}}%
\pgfpathlineto{\pgfqpoint{4.212383in}{0.413320in}}%
\pgfpathlineto{\pgfqpoint{4.209597in}{0.413320in}}%
\pgfpathlineto{\pgfqpoint{4.207076in}{0.413320in}}%
\pgfpathlineto{\pgfqpoint{4.204240in}{0.413320in}}%
\pgfpathlineto{\pgfqpoint{4.201542in}{0.413320in}}%
\pgfpathlineto{\pgfqpoint{4.198878in}{0.413320in}}%
\pgfpathlineto{\pgfqpoint{4.196186in}{0.413320in}}%
\pgfpathlineto{\pgfqpoint{4.193638in}{0.413320in}}%
\pgfpathlineto{\pgfqpoint{4.190842in}{0.413320in}}%
\pgfpathlineto{\pgfqpoint{4.188318in}{0.413320in}}%
\pgfpathlineto{\pgfqpoint{4.185481in}{0.413320in}}%
\pgfpathlineto{\pgfqpoint{4.182899in}{0.413320in}}%
\pgfpathlineto{\pgfqpoint{4.180129in}{0.413320in}}%
\pgfpathlineto{\pgfqpoint{4.177593in}{0.413320in}}%
\pgfpathlineto{\pgfqpoint{4.174770in}{0.413320in}}%
\pgfpathlineto{\pgfqpoint{4.172093in}{0.413320in}}%
\pgfpathlineto{\pgfqpoint{4.169415in}{0.413320in}}%
\pgfpathlineto{\pgfqpoint{4.166737in}{0.413320in}}%
\pgfpathlineto{\pgfqpoint{4.164059in}{0.413320in}}%
\pgfpathlineto{\pgfqpoint{4.161380in}{0.413320in}}%
\pgfpathlineto{\pgfqpoint{4.158806in}{0.413320in}}%
\pgfpathlineto{\pgfqpoint{4.156016in}{0.413320in}}%
\pgfpathlineto{\pgfqpoint{4.153423in}{0.413320in}}%
\pgfpathlineto{\pgfqpoint{4.150665in}{0.413320in}}%
\pgfpathlineto{\pgfqpoint{4.148082in}{0.413320in}}%
\pgfpathlineto{\pgfqpoint{4.145310in}{0.413320in}}%
\pgfpathlineto{\pgfqpoint{4.142713in}{0.413320in}}%
\pgfpathlineto{\pgfqpoint{4.139963in}{0.413320in}}%
\pgfpathlineto{\pgfqpoint{4.137272in}{0.413320in}}%
\pgfpathlineto{\pgfqpoint{4.134615in}{0.413320in}}%
\pgfpathlineto{\pgfqpoint{4.131920in}{0.413320in}}%
\pgfpathlineto{\pgfqpoint{4.129349in}{0.413320in}}%
\pgfpathlineto{\pgfqpoint{4.126553in}{0.413320in}}%
\pgfpathlineto{\pgfqpoint{4.124019in}{0.413320in}}%
\pgfpathlineto{\pgfqpoint{4.121205in}{0.413320in}}%
\pgfpathlineto{\pgfqpoint{4.118554in}{0.413320in}}%
\pgfpathlineto{\pgfqpoint{4.115844in}{0.413320in}}%
\pgfpathlineto{\pgfqpoint{4.113252in}{0.413320in}}%
\pgfpathlineto{\pgfqpoint{4.110488in}{0.413320in}}%
\pgfpathlineto{\pgfqpoint{4.107814in}{0.413320in}}%
\pgfpathlineto{\pgfqpoint{4.105185in}{0.413320in}}%
\pgfpathlineto{\pgfqpoint{4.102456in}{0.413320in}}%
\pgfpathlineto{\pgfqpoint{4.099777in}{0.413320in}}%
\pgfpathlineto{\pgfqpoint{4.097092in}{0.413320in}}%
\pgfpathlineto{\pgfqpoint{4.094527in}{0.413320in}}%
\pgfpathlineto{\pgfqpoint{4.091729in}{0.413320in}}%
\pgfpathlineto{\pgfqpoint{4.089159in}{0.413320in}}%
\pgfpathlineto{\pgfqpoint{4.086385in}{0.413320in}}%
\pgfpathlineto{\pgfqpoint{4.083870in}{0.413320in}}%
\pgfpathlineto{\pgfqpoint{4.081018in}{0.413320in}}%
\pgfpathlineto{\pgfqpoint{4.078471in}{0.413320in}}%
\pgfpathlineto{\pgfqpoint{4.075705in}{0.413320in}}%
\pgfpathlineto{\pgfqpoint{4.072985in}{0.413320in}}%
\pgfpathlineto{\pgfqpoint{4.070313in}{0.413320in}}%
\pgfpathlineto{\pgfqpoint{4.067636in}{0.413320in}}%
\pgfpathlineto{\pgfqpoint{4.064957in}{0.413320in}}%
\pgfpathlineto{\pgfqpoint{4.062266in}{0.413320in}}%
\pgfpathlineto{\pgfqpoint{4.059702in}{0.413320in}}%
\pgfpathlineto{\pgfqpoint{4.056911in}{0.413320in}}%
\pgfpathlineto{\pgfqpoint{4.054326in}{0.413320in}}%
\pgfpathlineto{\pgfqpoint{4.051557in}{0.413320in}}%
\pgfpathlineto{\pgfqpoint{4.049006in}{0.413320in}}%
\pgfpathlineto{\pgfqpoint{4.046210in}{0.413320in}}%
\pgfpathlineto{\pgfqpoint{4.043667in}{0.413320in}}%
\pgfpathlineto{\pgfqpoint{4.040852in}{0.413320in}}%
\pgfpathlineto{\pgfqpoint{4.038174in}{0.413320in}}%
\pgfpathlineto{\pgfqpoint{4.035492in}{0.413320in}}%
\pgfpathlineto{\pgfqpoint{4.032817in}{0.413320in}}%
\pgfpathlineto{\pgfqpoint{4.030229in}{0.413320in}}%
\pgfpathlineto{\pgfqpoint{4.027447in}{0.413320in}}%
\pgfpathlineto{\pgfqpoint{4.024868in}{0.413320in}}%
\pgfpathlineto{\pgfqpoint{4.022097in}{0.413320in}}%
\pgfpathlineto{\pgfqpoint{4.019518in}{0.413320in}}%
\pgfpathlineto{\pgfqpoint{4.016744in}{0.413320in}}%
\pgfpathlineto{\pgfqpoint{4.014186in}{0.413320in}}%
\pgfpathlineto{\pgfqpoint{4.011394in}{0.413320in}}%
\pgfpathlineto{\pgfqpoint{4.008699in}{0.413320in}}%
\pgfpathlineto{\pgfqpoint{4.006034in}{0.413320in}}%
\pgfpathlineto{\pgfqpoint{4.003348in}{0.413320in}}%
\pgfpathlineto{\pgfqpoint{4.000674in}{0.413320in}}%
\pgfpathlineto{\pgfqpoint{3.997990in}{0.413320in}}%
\pgfpathlineto{\pgfqpoint{3.995417in}{0.413320in}}%
\pgfpathlineto{\pgfqpoint{3.992642in}{0.413320in}}%
\pgfpathlineto{\pgfqpoint{3.990055in}{0.413320in}}%
\pgfpathlineto{\pgfqpoint{3.987270in}{0.413320in}}%
\pgfpathlineto{\pgfqpoint{3.984714in}{0.413320in}}%
\pgfpathlineto{\pgfqpoint{3.981929in}{0.413320in}}%
\pgfpathlineto{\pgfqpoint{3.979389in}{0.413320in}}%
\pgfpathlineto{\pgfqpoint{3.976563in}{0.413320in}}%
\pgfpathlineto{\pgfqpoint{3.973885in}{0.413320in}}%
\pgfpathlineto{\pgfqpoint{3.971250in}{0.413320in}}%
\pgfpathlineto{\pgfqpoint{3.968523in}{0.413320in}}%
\pgfpathlineto{\pgfqpoint{3.966013in}{0.413320in}}%
\pgfpathlineto{\pgfqpoint{3.963176in}{0.413320in}}%
\pgfpathlineto{\pgfqpoint{3.960635in}{0.413320in}}%
\pgfpathlineto{\pgfqpoint{3.957823in}{0.413320in}}%
\pgfpathlineto{\pgfqpoint{3.955211in}{0.413320in}}%
\pgfpathlineto{\pgfqpoint{3.952464in}{0.413320in}}%
\pgfpathlineto{\pgfqpoint{3.949894in}{0.413320in}}%
\pgfpathlineto{\pgfqpoint{3.947101in}{0.413320in}}%
\pgfpathlineto{\pgfqpoint{3.944431in}{0.413320in}}%
\pgfpathlineto{\pgfqpoint{3.941778in}{0.413320in}}%
\pgfpathlineto{\pgfqpoint{3.939075in}{0.413320in}}%
\pgfpathlineto{\pgfqpoint{3.936395in}{0.413320in}}%
\pgfpathlineto{\pgfqpoint{3.933714in}{0.413320in}}%
\pgfpathlineto{\pgfqpoint{3.931202in}{0.413320in}}%
\pgfpathlineto{\pgfqpoint{3.928347in}{0.413320in}}%
\pgfpathlineto{\pgfqpoint{3.925778in}{0.413320in}}%
\pgfpathlineto{\pgfqpoint{3.923005in}{0.413320in}}%
\pgfpathlineto{\pgfqpoint{3.920412in}{0.413320in}}%
\pgfpathlineto{\pgfqpoint{3.917646in}{0.413320in}}%
\pgfpathlineto{\pgfqpoint{3.915107in}{0.413320in}}%
\pgfpathlineto{\pgfqpoint{3.912296in}{0.413320in}}%
\pgfpathlineto{\pgfqpoint{3.909602in}{0.413320in}}%
\pgfpathlineto{\pgfqpoint{3.906918in}{0.413320in}}%
\pgfpathlineto{\pgfqpoint{3.904252in}{0.413320in}}%
\pgfpathlineto{\pgfqpoint{3.901573in}{0.413320in}}%
\pgfpathlineto{\pgfqpoint{3.898891in}{0.413320in}}%
\pgfpathlineto{\pgfqpoint{3.896345in}{0.413320in}}%
\pgfpathlineto{\pgfqpoint{3.893541in}{0.413320in}}%
\pgfpathlineto{\pgfqpoint{3.890926in}{0.413320in}}%
\pgfpathlineto{\pgfqpoint{3.888188in}{0.413320in}}%
\pgfpathlineto{\pgfqpoint{3.885621in}{0.413320in}}%
\pgfpathlineto{\pgfqpoint{3.882850in}{0.413320in}}%
\pgfpathlineto{\pgfqpoint{3.880237in}{0.413320in}}%
\pgfpathlineto{\pgfqpoint{3.877466in}{0.413320in}}%
\pgfpathlineto{\pgfqpoint{3.874790in}{0.413320in}}%
\pgfpathlineto{\pgfqpoint{3.872114in}{0.413320in}}%
\pgfpathlineto{\pgfqpoint{3.869435in}{0.413320in}}%
\pgfpathlineto{\pgfqpoint{3.866815in}{0.413320in}}%
\pgfpathlineto{\pgfqpoint{3.864073in}{0.413320in}}%
\pgfpathlineto{\pgfqpoint{3.861561in}{0.413320in}}%
\pgfpathlineto{\pgfqpoint{3.858720in}{0.413320in}}%
\pgfpathlineto{\pgfqpoint{3.856100in}{0.413320in}}%
\pgfpathlineto{\pgfqpoint{3.853358in}{0.413320in}}%
\pgfpathlineto{\pgfqpoint{3.850814in}{0.413320in}}%
\pgfpathlineto{\pgfqpoint{3.848005in}{0.413320in}}%
\pgfpathlineto{\pgfqpoint{3.845329in}{0.413320in}}%
\pgfpathlineto{\pgfqpoint{3.842641in}{0.413320in}}%
\pgfpathlineto{\pgfqpoint{3.839960in}{0.413320in}}%
\pgfpathlineto{\pgfqpoint{3.837286in}{0.413320in}}%
\pgfpathlineto{\pgfqpoint{3.834616in}{0.413320in}}%
\pgfpathlineto{\pgfqpoint{3.832053in}{0.413320in}}%
\pgfpathlineto{\pgfqpoint{3.829252in}{0.413320in}}%
\pgfpathlineto{\pgfqpoint{3.826679in}{0.413320in}}%
\pgfpathlineto{\pgfqpoint{3.823903in}{0.413320in}}%
\pgfpathlineto{\pgfqpoint{3.821315in}{0.413320in}}%
\pgfpathlineto{\pgfqpoint{3.818546in}{0.413320in}}%
\pgfpathlineto{\pgfqpoint{3.815983in}{0.413320in}}%
\pgfpathlineto{\pgfqpoint{3.813172in}{0.413320in}}%
\pgfpathlineto{\pgfqpoint{3.810510in}{0.413320in}}%
\pgfpathlineto{\pgfqpoint{3.807832in}{0.413320in}}%
\pgfpathlineto{\pgfqpoint{3.805145in}{0.413320in}}%
\pgfpathlineto{\pgfqpoint{3.802569in}{0.413320in}}%
\pgfpathlineto{\pgfqpoint{3.799797in}{0.413320in}}%
\pgfpathlineto{\pgfqpoint{3.797265in}{0.413320in}}%
\pgfpathlineto{\pgfqpoint{3.794435in}{0.413320in}}%
\pgfpathlineto{\pgfqpoint{3.791897in}{0.413320in}}%
\pgfpathlineto{\pgfqpoint{3.789084in}{0.413320in}}%
\pgfpathlineto{\pgfqpoint{3.786504in}{0.413320in}}%
\pgfpathlineto{\pgfqpoint{3.783725in}{0.413320in}}%
\pgfpathlineto{\pgfqpoint{3.781046in}{0.413320in}}%
\pgfpathlineto{\pgfqpoint{3.778370in}{0.413320in}}%
\pgfpathlineto{\pgfqpoint{3.775691in}{0.413320in}}%
\pgfpathlineto{\pgfqpoint{3.773014in}{0.413320in}}%
\pgfpathlineto{\pgfqpoint{3.770323in}{0.413320in}}%
\pgfpathlineto{\pgfqpoint{3.767782in}{0.413320in}}%
\pgfpathlineto{\pgfqpoint{3.764966in}{0.413320in}}%
\pgfpathlineto{\pgfqpoint{3.762389in}{0.413320in}}%
\pgfpathlineto{\pgfqpoint{3.759622in}{0.413320in}}%
\pgfpathlineto{\pgfqpoint{3.757065in}{0.413320in}}%
\pgfpathlineto{\pgfqpoint{3.754265in}{0.413320in}}%
\pgfpathlineto{\pgfqpoint{3.751728in}{0.413320in}}%
\pgfpathlineto{\pgfqpoint{3.748903in}{0.413320in}}%
\pgfpathlineto{\pgfqpoint{3.746229in}{0.413320in}}%
\pgfpathlineto{\pgfqpoint{3.743548in}{0.413320in}}%
\pgfpathlineto{\pgfqpoint{3.740874in}{0.413320in}}%
\pgfpathlineto{\pgfqpoint{3.738194in}{0.413320in}}%
\pgfpathlineto{\pgfqpoint{3.735509in}{0.413320in}}%
\pgfpathlineto{\pgfqpoint{3.732950in}{0.413320in}}%
\pgfpathlineto{\pgfqpoint{3.730158in}{0.413320in}}%
\pgfpathlineto{\pgfqpoint{3.727581in}{0.413320in}}%
\pgfpathlineto{\pgfqpoint{3.724804in}{0.413320in}}%
\pgfpathlineto{\pgfqpoint{3.722228in}{0.413320in}}%
\pgfpathlineto{\pgfqpoint{3.719446in}{0.413320in}}%
\pgfpathlineto{\pgfqpoint{3.716875in}{0.413320in}}%
\pgfpathlineto{\pgfqpoint{3.714086in}{0.413320in}}%
\pgfpathlineto{\pgfqpoint{3.711410in}{0.413320in}}%
\pgfpathlineto{\pgfqpoint{3.708729in}{0.413320in}}%
\pgfpathlineto{\pgfqpoint{3.706053in}{0.413320in}}%
\pgfpathlineto{\pgfqpoint{3.703460in}{0.413320in}}%
\pgfpathlineto{\pgfqpoint{3.700684in}{0.413320in}}%
\pgfpathlineto{\pgfqpoint{3.698125in}{0.413320in}}%
\pgfpathlineto{\pgfqpoint{3.695331in}{0.413320in}}%
\pgfpathlineto{\pgfqpoint{3.692765in}{0.413320in}}%
\pgfpathlineto{\pgfqpoint{3.689983in}{0.413320in}}%
\pgfpathlineto{\pgfqpoint{3.687442in}{0.413320in}}%
\pgfpathlineto{\pgfqpoint{3.684620in}{0.413320in}}%
\pgfpathlineto{\pgfqpoint{3.681948in}{0.413320in}}%
\pgfpathlineto{\pgfqpoint{3.679273in}{0.413320in}}%
\pgfpathlineto{\pgfqpoint{3.676591in}{0.413320in}}%
\pgfpathlineto{\pgfqpoint{3.673911in}{0.413320in}}%
\pgfpathlineto{\pgfqpoint{3.671232in}{0.413320in}}%
\pgfpathlineto{\pgfqpoint{3.668665in}{0.413320in}}%
\pgfpathlineto{\pgfqpoint{3.665864in}{0.413320in}}%
\pgfpathlineto{\pgfqpoint{3.663276in}{0.413320in}}%
\pgfpathlineto{\pgfqpoint{3.660515in}{0.413320in}}%
\pgfpathlineto{\pgfqpoint{3.657917in}{0.413320in}}%
\pgfpathlineto{\pgfqpoint{3.655165in}{0.413320in}}%
\pgfpathlineto{\pgfqpoint{3.652628in}{0.413320in}}%
\pgfpathlineto{\pgfqpoint{3.649837in}{0.413320in}}%
\pgfpathlineto{\pgfqpoint{3.647130in}{0.413320in}}%
\pgfpathlineto{\pgfqpoint{3.644452in}{0.413320in}}%
\pgfpathlineto{\pgfqpoint{3.641773in}{0.413320in}}%
\pgfpathlineto{\pgfqpoint{3.639207in}{0.413320in}}%
\pgfpathlineto{\pgfqpoint{3.636413in}{0.413320in}}%
\pgfpathlineto{\pgfqpoint{3.633858in}{0.413320in}}%
\pgfpathlineto{\pgfqpoint{3.631058in}{0.413320in}}%
\pgfpathlineto{\pgfqpoint{3.628460in}{0.413320in}}%
\pgfpathlineto{\pgfqpoint{3.625689in}{0.413320in}}%
\pgfpathlineto{\pgfqpoint{3.623165in}{0.413320in}}%
\pgfpathlineto{\pgfqpoint{3.620345in}{0.413320in}}%
\pgfpathlineto{\pgfqpoint{3.617667in}{0.413320in}}%
\pgfpathlineto{\pgfqpoint{3.614982in}{0.413320in}}%
\pgfpathlineto{\pgfqpoint{3.612311in}{0.413320in}}%
\pgfpathlineto{\pgfqpoint{3.609632in}{0.413320in}}%
\pgfpathlineto{\pgfqpoint{3.606951in}{0.413320in}}%
\pgfpathlineto{\pgfqpoint{3.604387in}{0.413320in}}%
\pgfpathlineto{\pgfqpoint{3.601590in}{0.413320in}}%
\pgfpathlineto{\pgfqpoint{3.598998in}{0.413320in}}%
\pgfpathlineto{\pgfqpoint{3.596240in}{0.413320in}}%
\pgfpathlineto{\pgfqpoint{3.593620in}{0.413320in}}%
\pgfpathlineto{\pgfqpoint{3.590883in}{0.413320in}}%
\pgfpathlineto{\pgfqpoint{3.588258in}{0.413320in}}%
\pgfpathlineto{\pgfqpoint{3.585532in}{0.413320in}}%
\pgfpathlineto{\pgfqpoint{3.582851in}{0.413320in}}%
\pgfpathlineto{\pgfqpoint{3.580191in}{0.413320in}}%
\pgfpathlineto{\pgfqpoint{3.577487in}{0.413320in}}%
\pgfpathlineto{\pgfqpoint{3.574814in}{0.413320in}}%
\pgfpathlineto{\pgfqpoint{3.572126in}{0.413320in}}%
\pgfpathlineto{\pgfqpoint{3.569584in}{0.413320in}}%
\pgfpathlineto{\pgfqpoint{3.566774in}{0.413320in}}%
\pgfpathlineto{\pgfqpoint{3.564188in}{0.413320in}}%
\pgfpathlineto{\pgfqpoint{3.561420in}{0.413320in}}%
\pgfpathlineto{\pgfqpoint{3.558853in}{0.413320in}}%
\pgfpathlineto{\pgfqpoint{3.556061in}{0.413320in}}%
\pgfpathlineto{\pgfqpoint{3.553498in}{0.413320in}}%
\pgfpathlineto{\pgfqpoint{3.550713in}{0.413320in}}%
\pgfpathlineto{\pgfqpoint{3.548029in}{0.413320in}}%
\pgfpathlineto{\pgfqpoint{3.545349in}{0.413320in}}%
\pgfpathlineto{\pgfqpoint{3.542656in}{0.413320in}}%
\pgfpathlineto{\pgfqpoint{3.540093in}{0.413320in}}%
\pgfpathlineto{\pgfqpoint{3.537309in}{0.413320in}}%
\pgfpathlineto{\pgfqpoint{3.534783in}{0.413320in}}%
\pgfpathlineto{\pgfqpoint{3.531955in}{0.413320in}}%
\pgfpathlineto{\pgfqpoint{3.529327in}{0.413320in}}%
\pgfpathlineto{\pgfqpoint{3.526601in}{0.413320in}}%
\pgfpathlineto{\pgfqpoint{3.524041in}{0.413320in}}%
\pgfpathlineto{\pgfqpoint{3.521244in}{0.413320in}}%
\pgfpathlineto{\pgfqpoint{3.518565in}{0.413320in}}%
\pgfpathlineto{\pgfqpoint{3.515884in}{0.413320in}}%
\pgfpathlineto{\pgfqpoint{3.513209in}{0.413320in}}%
\pgfpathlineto{\pgfqpoint{3.510533in}{0.413320in}}%
\pgfpathlineto{\pgfqpoint{3.507840in}{0.413320in}}%
\pgfpathlineto{\pgfqpoint{3.505262in}{0.413320in}}%
\pgfpathlineto{\pgfqpoint{3.502488in}{0.413320in}}%
\pgfpathlineto{\pgfqpoint{3.499909in}{0.413320in}}%
\pgfpathlineto{\pgfqpoint{3.497139in}{0.413320in}}%
\pgfpathlineto{\pgfqpoint{3.494581in}{0.413320in}}%
\pgfpathlineto{\pgfqpoint{3.491783in}{0.413320in}}%
\pgfpathlineto{\pgfqpoint{3.489223in}{0.413320in}}%
\pgfpathlineto{\pgfqpoint{3.486442in}{0.413320in}}%
\pgfpathlineto{\pgfqpoint{3.483744in}{0.413320in}}%
\pgfpathlineto{\pgfqpoint{3.481072in}{0.413320in}}%
\pgfpathlineto{\pgfqpoint{3.478378in}{0.413320in}}%
\pgfpathlineto{\pgfqpoint{3.475821in}{0.413320in}}%
\pgfpathlineto{\pgfqpoint{3.473021in}{0.413320in}}%
\pgfpathlineto{\pgfqpoint{3.470466in}{0.413320in}}%
\pgfpathlineto{\pgfqpoint{3.467678in}{0.413320in}}%
\pgfpathlineto{\pgfqpoint{3.465072in}{0.413320in}}%
\pgfpathlineto{\pgfqpoint{3.462321in}{0.413320in}}%
\pgfpathlineto{\pgfqpoint{3.459695in}{0.413320in}}%
\pgfpathlineto{\pgfqpoint{3.456960in}{0.413320in}}%
\pgfpathlineto{\pgfqpoint{3.454285in}{0.413320in}}%
\pgfpathlineto{\pgfqpoint{3.451597in}{0.413320in}}%
\pgfpathlineto{\pgfqpoint{3.448926in}{0.413320in}}%
\pgfpathlineto{\pgfqpoint{3.446257in}{0.413320in}}%
\pgfpathlineto{\pgfqpoint{3.443574in}{0.413320in}}%
\pgfpathlineto{\pgfqpoint{3.440996in}{0.413320in}}%
\pgfpathlineto{\pgfqpoint{3.438210in}{0.413320in}}%
\pgfpathlineto{\pgfqpoint{3.435635in}{0.413320in}}%
\pgfpathlineto{\pgfqpoint{3.432851in}{0.413320in}}%
\pgfpathlineto{\pgfqpoint{3.430313in}{0.413320in}}%
\pgfpathlineto{\pgfqpoint{3.427501in}{0.413320in}}%
\pgfpathlineto{\pgfqpoint{3.424887in}{0.413320in}}%
\pgfpathlineto{\pgfqpoint{3.422142in}{0.413320in}}%
\pgfpathlineto{\pgfqpoint{3.419455in}{0.413320in}}%
\pgfpathlineto{\pgfqpoint{3.416780in}{0.413320in}}%
\pgfpathlineto{\pgfqpoint{3.414109in}{0.413320in}}%
\pgfpathlineto{\pgfqpoint{3.411431in}{0.413320in}}%
\pgfpathlineto{\pgfqpoint{3.408752in}{0.413320in}}%
\pgfpathlineto{\pgfqpoint{3.406202in}{0.413320in}}%
\pgfpathlineto{\pgfqpoint{3.403394in}{0.413320in}}%
\pgfpathlineto{\pgfqpoint{3.400783in}{0.413320in}}%
\pgfpathlineto{\pgfqpoint{3.398037in}{0.413320in}}%
\pgfpathlineto{\pgfqpoint{3.395461in}{0.413320in}}%
\pgfpathlineto{\pgfqpoint{3.392681in}{0.413320in}}%
\pgfpathlineto{\pgfqpoint{3.390102in}{0.413320in}}%
\pgfpathlineto{\pgfqpoint{3.387309in}{0.413320in}}%
\pgfpathlineto{\pgfqpoint{3.384647in}{0.413320in}}%
\pgfpathlineto{\pgfqpoint{3.381959in}{0.413320in}}%
\pgfpathlineto{\pgfqpoint{3.379290in}{0.413320in}}%
\pgfpathlineto{\pgfqpoint{3.376735in}{0.413320in}}%
\pgfpathlineto{\pgfqpoint{3.373921in}{0.413320in}}%
\pgfpathlineto{\pgfqpoint{3.371357in}{0.413320in}}%
\pgfpathlineto{\pgfqpoint{3.368577in}{0.413320in}}%
\pgfpathlineto{\pgfqpoint{3.365996in}{0.413320in}}%
\pgfpathlineto{\pgfqpoint{3.363221in}{0.413320in}}%
\pgfpathlineto{\pgfqpoint{3.360620in}{0.413320in}}%
\pgfpathlineto{\pgfqpoint{3.357862in}{0.413320in}}%
\pgfpathlineto{\pgfqpoint{3.355177in}{0.413320in}}%
\pgfpathlineto{\pgfqpoint{3.352505in}{0.413320in}}%
\pgfpathlineto{\pgfqpoint{3.349828in}{0.413320in}}%
\pgfpathlineto{\pgfqpoint{3.347139in}{0.413320in}}%
\pgfpathlineto{\pgfqpoint{3.344468in}{0.413320in}}%
\pgfpathlineto{\pgfqpoint{3.341893in}{0.413320in}}%
\pgfpathlineto{\pgfqpoint{3.339101in}{0.413320in}}%
\pgfpathlineto{\pgfqpoint{3.336541in}{0.413320in}}%
\pgfpathlineto{\pgfqpoint{3.333758in}{0.413320in}}%
\pgfpathlineto{\pgfqpoint{3.331183in}{0.413320in}}%
\pgfpathlineto{\pgfqpoint{3.328401in}{0.413320in}}%
\pgfpathlineto{\pgfqpoint{3.325860in}{0.413320in}}%
\pgfpathlineto{\pgfqpoint{3.323049in}{0.413320in}}%
\pgfpathlineto{\pgfqpoint{3.320366in}{0.413320in}}%
\pgfpathlineto{\pgfqpoint{3.317688in}{0.413320in}}%
\pgfpathlineto{\pgfqpoint{3.315008in}{0.413320in}}%
\pgfpathlineto{\pgfqpoint{3.312480in}{0.413320in}}%
\pgfpathlineto{\pgfqpoint{3.309652in}{0.413320in}}%
\pgfpathlineto{\pgfqpoint{3.307104in}{0.413320in}}%
\pgfpathlineto{\pgfqpoint{3.304295in}{0.413320in}}%
\pgfpathlineto{\pgfqpoint{3.301719in}{0.413320in}}%
\pgfpathlineto{\pgfqpoint{3.298937in}{0.413320in}}%
\pgfpathlineto{\pgfqpoint{3.296376in}{0.413320in}}%
\pgfpathlineto{\pgfqpoint{3.293574in}{0.413320in}}%
\pgfpathlineto{\pgfqpoint{3.290890in}{0.413320in}}%
\pgfpathlineto{\pgfqpoint{3.288225in}{0.413320in}}%
\pgfpathlineto{\pgfqpoint{3.285534in}{0.413320in}}%
\pgfpathlineto{\pgfqpoint{3.282870in}{0.413320in}}%
\pgfpathlineto{\pgfqpoint{3.280189in}{0.413320in}}%
\pgfpathlineto{\pgfqpoint{3.277603in}{0.413320in}}%
\pgfpathlineto{\pgfqpoint{3.274831in}{0.413320in}}%
\pgfpathlineto{\pgfqpoint{3.272254in}{0.413320in}}%
\pgfpathlineto{\pgfqpoint{3.269478in}{0.413320in}}%
\pgfpathlineto{\pgfqpoint{3.266849in}{0.413320in}}%
\pgfpathlineto{\pgfqpoint{3.264119in}{0.413320in}}%
\pgfpathlineto{\pgfqpoint{3.261594in}{0.413320in}}%
\pgfpathlineto{\pgfqpoint{3.258784in}{0.413320in}}%
\pgfpathlineto{\pgfqpoint{3.256083in}{0.413320in}}%
\pgfpathlineto{\pgfqpoint{3.253404in}{0.413320in}}%
\pgfpathlineto{\pgfqpoint{3.250716in}{0.413320in}}%
\pgfpathlineto{\pgfqpoint{3.248049in}{0.413320in}}%
\pgfpathlineto{\pgfqpoint{3.245363in}{0.413320in}}%
\pgfpathlineto{\pgfqpoint{3.242807in}{0.413320in}}%
\pgfpathlineto{\pgfqpoint{3.240010in}{0.413320in}}%
\pgfpathlineto{\pgfqpoint{3.237411in}{0.413320in}}%
\pgfpathlineto{\pgfqpoint{3.234658in}{0.413320in}}%
\pgfpathlineto{\pgfqpoint{3.232069in}{0.413320in}}%
\pgfpathlineto{\pgfqpoint{3.229310in}{0.413320in}}%
\pgfpathlineto{\pgfqpoint{3.226609in}{0.413320in}}%
\pgfpathlineto{\pgfqpoint{3.223942in}{0.413320in}}%
\pgfpathlineto{\pgfqpoint{3.221255in}{0.413320in}}%
\pgfpathlineto{\pgfqpoint{3.218586in}{0.413320in}}%
\pgfpathlineto{\pgfqpoint{3.215908in}{0.413320in}}%
\pgfpathlineto{\pgfqpoint{3.213342in}{0.413320in}}%
\pgfpathlineto{\pgfqpoint{3.210545in}{0.413320in}}%
\pgfpathlineto{\pgfqpoint{3.207984in}{0.413320in}}%
\pgfpathlineto{\pgfqpoint{3.205195in}{0.413320in}}%
\pgfpathlineto{\pgfqpoint{3.202562in}{0.413320in}}%
\pgfpathlineto{\pgfqpoint{3.199823in}{0.413320in}}%
\pgfpathlineto{\pgfqpoint{3.197226in}{0.413320in}}%
\pgfpathlineto{\pgfqpoint{3.194508in}{0.413320in}}%
\pgfpathlineto{\pgfqpoint{3.191796in}{0.413320in}}%
\pgfpathlineto{\pgfqpoint{3.189117in}{0.413320in}}%
\pgfpathlineto{\pgfqpoint{3.186440in}{0.413320in}}%
\pgfpathlineto{\pgfqpoint{3.183760in}{0.413320in}}%
\pgfpathlineto{\pgfqpoint{3.181089in}{0.413320in}}%
\pgfpathlineto{\pgfqpoint{3.178525in}{0.413320in}}%
\pgfpathlineto{\pgfqpoint{3.175724in}{0.413320in}}%
\pgfpathlineto{\pgfqpoint{3.173142in}{0.413320in}}%
\pgfpathlineto{\pgfqpoint{3.170375in}{0.413320in}}%
\pgfpathlineto{\pgfqpoint{3.167776in}{0.413320in}}%
\pgfpathlineto{\pgfqpoint{3.165019in}{0.413320in}}%
\pgfpathlineto{\pgfqpoint{3.162474in}{0.413320in}}%
\pgfpathlineto{\pgfqpoint{3.159675in}{0.413320in}}%
\pgfpathlineto{\pgfqpoint{3.156981in}{0.413320in}}%
\pgfpathlineto{\pgfqpoint{3.154327in}{0.413320in}}%
\pgfpathlineto{\pgfqpoint{3.151612in}{0.413320in}}%
\pgfpathlineto{\pgfqpoint{3.149057in}{0.413320in}}%
\pgfpathlineto{\pgfqpoint{3.146271in}{0.413320in}}%
\pgfpathlineto{\pgfqpoint{3.143740in}{0.413320in}}%
\pgfpathlineto{\pgfqpoint{3.140913in}{0.413320in}}%
\pgfpathlineto{\pgfqpoint{3.138375in}{0.413320in}}%
\pgfpathlineto{\pgfqpoint{3.135550in}{0.413320in}}%
\pgfpathlineto{\pgfqpoint{3.132946in}{0.413320in}}%
\pgfpathlineto{\pgfqpoint{3.130199in}{0.413320in}}%
\pgfpathlineto{\pgfqpoint{3.127512in}{0.413320in}}%
\pgfpathlineto{\pgfqpoint{3.124842in}{0.413320in}}%
\pgfpathlineto{\pgfqpoint{3.122163in}{0.413320in}}%
\pgfpathlineto{\pgfqpoint{3.119487in}{0.413320in}}%
\pgfpathlineto{\pgfqpoint{3.116807in}{0.413320in}}%
\pgfpathlineto{\pgfqpoint{3.114242in}{0.413320in}}%
\pgfpathlineto{\pgfqpoint{3.111451in}{0.413320in}}%
\pgfpathlineto{\pgfqpoint{3.108896in}{0.413320in}}%
\pgfpathlineto{\pgfqpoint{3.106094in}{0.413320in}}%
\pgfpathlineto{\pgfqpoint{3.103508in}{0.413320in}}%
\pgfpathlineto{\pgfqpoint{3.100737in}{0.413320in}}%
\pgfpathlineto{\pgfqpoint{3.098163in}{0.413320in}}%
\pgfpathlineto{\pgfqpoint{3.095388in}{0.413320in}}%
\pgfpathlineto{\pgfqpoint{3.092699in}{0.413320in}}%
\pgfpathlineto{\pgfqpoint{3.090023in}{0.413320in}}%
\pgfpathlineto{\pgfqpoint{3.087343in}{0.413320in}}%
\pgfpathlineto{\pgfqpoint{3.084671in}{0.413320in}}%
\pgfpathlineto{\pgfqpoint{3.081990in}{0.413320in}}%
\pgfpathlineto{\pgfqpoint{3.079381in}{0.413320in}}%
\pgfpathlineto{\pgfqpoint{3.076631in}{0.413320in}}%
\pgfpathlineto{\pgfqpoint{3.074056in}{0.413320in}}%
\pgfpathlineto{\pgfqpoint{3.071266in}{0.413320in}}%
\pgfpathlineto{\pgfqpoint{3.068709in}{0.413320in}}%
\pgfpathlineto{\pgfqpoint{3.065916in}{0.413320in}}%
\pgfpathlineto{\pgfqpoint{3.063230in}{0.413320in}}%
\pgfpathlineto{\pgfqpoint{3.060561in}{0.413320in}}%
\pgfpathlineto{\pgfqpoint{3.057884in}{0.413320in}}%
\pgfpathlineto{\pgfqpoint{3.055202in}{0.413320in}}%
\pgfpathlineto{\pgfqpoint{3.052526in}{0.413320in}}%
\pgfpathlineto{\pgfqpoint{3.049988in}{0.413320in}}%
\pgfpathlineto{\pgfqpoint{3.047157in}{0.413320in}}%
\pgfpathlineto{\pgfqpoint{3.044568in}{0.413320in}}%
\pgfpathlineto{\pgfqpoint{3.041813in}{0.413320in}}%
\pgfpathlineto{\pgfqpoint{3.039262in}{0.413320in}}%
\pgfpathlineto{\pgfqpoint{3.036456in}{0.413320in}}%
\pgfpathlineto{\pgfqpoint{3.033921in}{0.413320in}}%
\pgfpathlineto{\pgfqpoint{3.031091in}{0.413320in}}%
\pgfpathlineto{\pgfqpoint{3.028412in}{0.413320in}}%
\pgfpathlineto{\pgfqpoint{3.025803in}{0.413320in}}%
\pgfpathlineto{\pgfqpoint{3.023058in}{0.413320in}}%
\pgfpathlineto{\pgfqpoint{3.020382in}{0.413320in}}%
\pgfpathlineto{\pgfqpoint{3.017707in}{0.413320in}}%
\pgfpathlineto{\pgfqpoint{3.015097in}{0.413320in}}%
\pgfpathlineto{\pgfqpoint{3.012351in}{0.413320in}}%
\pgfpathlineto{\pgfqpoint{3.009784in}{0.413320in}}%
\pgfpathlineto{\pgfqpoint{3.006993in}{0.413320in}}%
\pgfpathlineto{\pgfqpoint{3.004419in}{0.413320in}}%
\pgfpathlineto{\pgfqpoint{3.001635in}{0.413320in}}%
\pgfpathlineto{\pgfqpoint{2.999103in}{0.413320in}}%
\pgfpathlineto{\pgfqpoint{2.996300in}{0.413320in}}%
\pgfpathlineto{\pgfqpoint{2.993595in}{0.413320in}}%
\pgfpathlineto{\pgfqpoint{2.990978in}{0.413320in}}%
\pgfpathlineto{\pgfqpoint{2.988238in}{0.413320in}}%
\pgfpathlineto{\pgfqpoint{2.985666in}{0.413320in}}%
\pgfpathlineto{\pgfqpoint{2.982885in}{0.413320in}}%
\pgfpathlineto{\pgfqpoint{2.980341in}{0.413320in}}%
\pgfpathlineto{\pgfqpoint{2.977517in}{0.413320in}}%
\pgfpathlineto{\pgfqpoint{2.974972in}{0.413320in}}%
\pgfpathlineto{\pgfqpoint{2.972177in}{0.413320in}}%
\pgfpathlineto{\pgfqpoint{2.969599in}{0.413320in}}%
\pgfpathlineto{\pgfqpoint{2.966812in}{0.413320in}}%
\pgfpathlineto{\pgfqpoint{2.964127in}{0.413320in}}%
\pgfpathlineto{\pgfqpoint{2.961460in}{0.413320in}}%
\pgfpathlineto{\pgfqpoint{2.958782in}{0.413320in}}%
\pgfpathlineto{\pgfqpoint{2.956103in}{0.413320in}}%
\pgfpathlineto{\pgfqpoint{2.953422in}{0.413320in}}%
\pgfpathlineto{\pgfqpoint{2.950884in}{0.413320in}}%
\pgfpathlineto{\pgfqpoint{2.948068in}{0.413320in}}%
\pgfpathlineto{\pgfqpoint{2.945461in}{0.413320in}}%
\pgfpathlineto{\pgfqpoint{2.942711in}{0.413320in}}%
\pgfpathlineto{\pgfqpoint{2.940120in}{0.413320in}}%
\pgfpathlineto{\pgfqpoint{2.937352in}{0.413320in}}%
\pgfpathlineto{\pgfqpoint{2.934759in}{0.413320in}}%
\pgfpathlineto{\pgfqpoint{2.932033in}{0.413320in}}%
\pgfpathlineto{\pgfqpoint{2.929321in}{0.413320in}}%
\pgfpathlineto{\pgfqpoint{2.926655in}{0.413320in}}%
\pgfpathlineto{\pgfqpoint{2.923963in}{0.413320in}}%
\pgfpathlineto{\pgfqpoint{2.921363in}{0.413320in}}%
\pgfpathlineto{\pgfqpoint{2.918606in}{0.413320in}}%
\pgfpathlineto{\pgfqpoint{2.916061in}{0.413320in}}%
\pgfpathlineto{\pgfqpoint{2.913243in}{0.413320in}}%
\pgfpathlineto{\pgfqpoint{2.910631in}{0.413320in}}%
\pgfpathlineto{\pgfqpoint{2.907882in}{0.413320in}}%
\pgfpathlineto{\pgfqpoint{2.905341in}{0.413320in}}%
\pgfpathlineto{\pgfqpoint{2.902535in}{0.413320in}}%
\pgfpathlineto{\pgfqpoint{2.899858in}{0.413320in}}%
\pgfpathlineto{\pgfqpoint{2.897179in}{0.413320in}}%
\pgfpathlineto{\pgfqpoint{2.894487in}{0.413320in}}%
\pgfpathlineto{\pgfqpoint{2.891809in}{0.413320in}}%
\pgfpathlineto{\pgfqpoint{2.889145in}{0.413320in}}%
\pgfpathlineto{\pgfqpoint{2.886578in}{0.413320in}}%
\pgfpathlineto{\pgfqpoint{2.883780in}{0.413320in}}%
\pgfpathlineto{\pgfqpoint{2.881254in}{0.413320in}}%
\pgfpathlineto{\pgfqpoint{2.878431in}{0.413320in}}%
\pgfpathlineto{\pgfqpoint{2.875882in}{0.413320in}}%
\pgfpathlineto{\pgfqpoint{2.873074in}{0.413320in}}%
\pgfpathlineto{\pgfqpoint{2.870475in}{0.413320in}}%
\pgfpathlineto{\pgfqpoint{2.867713in}{0.413320in}}%
\pgfpathlineto{\pgfqpoint{2.865031in}{0.413320in}}%
\pgfpathlineto{\pgfqpoint{2.862402in}{0.413320in}}%
\pgfpathlineto{\pgfqpoint{2.859668in}{0.413320in}}%
\pgfpathlineto{\pgfqpoint{2.857003in}{0.413320in}}%
\pgfpathlineto{\pgfqpoint{2.854325in}{0.413320in}}%
\pgfpathlineto{\pgfqpoint{2.851793in}{0.413320in}}%
\pgfpathlineto{\pgfqpoint{2.848960in}{0.413320in}}%
\pgfpathlineto{\pgfqpoint{2.846408in}{0.413320in}}%
\pgfpathlineto{\pgfqpoint{2.843611in}{0.413320in}}%
\pgfpathlineto{\pgfqpoint{2.841055in}{0.413320in}}%
\pgfpathlineto{\pgfqpoint{2.838254in}{0.413320in}}%
\pgfpathlineto{\pgfqpoint{2.835698in}{0.413320in}}%
\pgfpathlineto{\pgfqpoint{2.832894in}{0.413320in}}%
\pgfpathlineto{\pgfqpoint{2.830219in}{0.413320in}}%
\pgfpathlineto{\pgfqpoint{2.827567in}{0.413320in}}%
\pgfpathlineto{\pgfqpoint{2.824851in}{0.413320in}}%
\pgfpathlineto{\pgfqpoint{2.822303in}{0.413320in}}%
\pgfpathlineto{\pgfqpoint{2.819506in}{0.413320in}}%
\pgfpathlineto{\pgfqpoint{2.816867in}{0.413320in}}%
\pgfpathlineto{\pgfqpoint{2.814141in}{0.413320in}}%
\pgfpathlineto{\pgfqpoint{2.811597in}{0.413320in}}%
\pgfpathlineto{\pgfqpoint{2.808792in}{0.413320in}}%
\pgfpathlineto{\pgfqpoint{2.806175in}{0.413320in}}%
\pgfpathlineto{\pgfqpoint{2.803435in}{0.413320in}}%
\pgfpathlineto{\pgfqpoint{2.800756in}{0.413320in}}%
\pgfpathlineto{\pgfqpoint{2.798070in}{0.413320in}}%
\pgfpathlineto{\pgfqpoint{2.795398in}{0.413320in}}%
\pgfpathlineto{\pgfqpoint{2.792721in}{0.413320in}}%
\pgfpathlineto{\pgfqpoint{2.790044in}{0.413320in}}%
\pgfpathlineto{\pgfqpoint{2.787468in}{0.413320in}}%
\pgfpathlineto{\pgfqpoint{2.784687in}{0.413320in}}%
\pgfpathlineto{\pgfqpoint{2.782113in}{0.413320in}}%
\pgfpathlineto{\pgfqpoint{2.779330in}{0.413320in}}%
\pgfpathlineto{\pgfqpoint{2.776767in}{0.413320in}}%
\pgfpathlineto{\pgfqpoint{2.773972in}{0.413320in}}%
\pgfpathlineto{\pgfqpoint{2.771373in}{0.413320in}}%
\pgfpathlineto{\pgfqpoint{2.768617in}{0.413320in}}%
\pgfpathlineto{\pgfqpoint{2.765935in}{0.413320in}}%
\pgfpathlineto{\pgfqpoint{2.763253in}{0.413320in}}%
\pgfpathlineto{\pgfqpoint{2.760581in}{0.413320in}}%
\pgfpathlineto{\pgfqpoint{2.758028in}{0.413320in}}%
\pgfpathlineto{\pgfqpoint{2.755224in}{0.413320in}}%
\pgfpathlineto{\pgfqpoint{2.752614in}{0.413320in}}%
\pgfpathlineto{\pgfqpoint{2.749868in}{0.413320in}}%
\pgfpathlineto{\pgfqpoint{2.747260in}{0.413320in}}%
\pgfpathlineto{\pgfqpoint{2.744510in}{0.413320in}}%
\pgfpathlineto{\pgfqpoint{2.741928in}{0.413320in}}%
\pgfpathlineto{\pgfqpoint{2.739155in}{0.413320in}}%
\pgfpathlineto{\pgfqpoint{2.736476in}{0.413320in}}%
\pgfpathlineto{\pgfqpoint{2.733798in}{0.413320in}}%
\pgfpathlineto{\pgfqpoint{2.731119in}{0.413320in}}%
\pgfpathlineto{\pgfqpoint{2.728439in}{0.413320in}}%
\pgfpathlineto{\pgfqpoint{2.725760in}{0.413320in}}%
\pgfpathlineto{\pgfqpoint{2.723211in}{0.413320in}}%
\pgfpathlineto{\pgfqpoint{2.720404in}{0.413320in}}%
\pgfpathlineto{\pgfqpoint{2.717773in}{0.413320in}}%
\pgfpathlineto{\pgfqpoint{2.715036in}{0.413320in}}%
\pgfpathlineto{\pgfqpoint{2.712477in}{0.413320in}}%
\pgfpathlineto{\pgfqpoint{2.709683in}{0.413320in}}%
\pgfpathlineto{\pgfqpoint{2.707125in}{0.413320in}}%
\pgfpathlineto{\pgfqpoint{2.704326in}{0.413320in}}%
\pgfpathlineto{\pgfqpoint{2.701657in}{0.413320in}}%
\pgfpathlineto{\pgfqpoint{2.698968in}{0.413320in}}%
\pgfpathlineto{\pgfqpoint{2.696293in}{0.413320in}}%
\pgfpathlineto{\pgfqpoint{2.693611in}{0.413320in}}%
\pgfpathlineto{\pgfqpoint{2.690940in}{0.413320in}}%
\pgfpathlineto{\pgfqpoint{2.688328in}{0.413320in}}%
\pgfpathlineto{\pgfqpoint{2.685586in}{0.413320in}}%
\pgfpathlineto{\pgfqpoint{2.683009in}{0.413320in}}%
\pgfpathlineto{\pgfqpoint{2.680224in}{0.413320in}}%
\pgfpathlineto{\pgfqpoint{2.677650in}{0.413320in}}%
\pgfpathlineto{\pgfqpoint{2.674873in}{0.413320in}}%
\pgfpathlineto{\pgfqpoint{2.672301in}{0.413320in}}%
\pgfpathlineto{\pgfqpoint{2.669506in}{0.413320in}}%
\pgfpathlineto{\pgfqpoint{2.666836in}{0.413320in}}%
\pgfpathlineto{\pgfqpoint{2.664151in}{0.413320in}}%
\pgfpathlineto{\pgfqpoint{2.661481in}{0.413320in}}%
\pgfpathlineto{\pgfqpoint{2.658942in}{0.413320in}}%
\pgfpathlineto{\pgfqpoint{2.656124in}{0.413320in}}%
\pgfpathlineto{\pgfqpoint{2.653567in}{0.413320in}}%
\pgfpathlineto{\pgfqpoint{2.650767in}{0.413320in}}%
\pgfpathlineto{\pgfqpoint{2.648196in}{0.413320in}}%
\pgfpathlineto{\pgfqpoint{2.645408in}{0.413320in}}%
\pgfpathlineto{\pgfqpoint{2.642827in}{0.413320in}}%
\pgfpathlineto{\pgfqpoint{2.640053in}{0.413320in}}%
\pgfpathlineto{\pgfqpoint{2.637369in}{0.413320in}}%
\pgfpathlineto{\pgfqpoint{2.634700in}{0.413320in}}%
\pgfpathlineto{\pgfqpoint{2.632018in}{0.413320in}}%
\pgfpathlineto{\pgfqpoint{2.629340in}{0.413320in}}%
\pgfpathlineto{\pgfqpoint{2.626653in}{0.413320in}}%
\pgfpathlineto{\pgfqpoint{2.624077in}{0.413320in}}%
\pgfpathlineto{\pgfqpoint{2.621304in}{0.413320in}}%
\pgfpathlineto{\pgfqpoint{2.618773in}{0.413320in}}%
\pgfpathlineto{\pgfqpoint{2.615934in}{0.413320in}}%
\pgfpathlineto{\pgfqpoint{2.613393in}{0.413320in}}%
\pgfpathlineto{\pgfqpoint{2.610588in}{0.413320in}}%
\pgfpathlineto{\pgfqpoint{2.608004in}{0.413320in}}%
\pgfpathlineto{\pgfqpoint{2.605232in}{0.413320in}}%
\pgfpathlineto{\pgfqpoint{2.602557in}{0.413320in}}%
\pgfpathlineto{\pgfqpoint{2.599920in}{0.413320in}}%
\pgfpathlineto{\pgfqpoint{2.597196in}{0.413320in}}%
\pgfpathlineto{\pgfqpoint{2.594630in}{0.413320in}}%
\pgfpathlineto{\pgfqpoint{2.591842in}{0.413320in}}%
\pgfpathlineto{\pgfqpoint{2.589248in}{0.413320in}}%
\pgfpathlineto{\pgfqpoint{2.586484in}{0.413320in}}%
\pgfpathlineto{\pgfqpoint{2.583913in}{0.413320in}}%
\pgfpathlineto{\pgfqpoint{2.581129in}{0.413320in}}%
\pgfpathlineto{\pgfqpoint{2.578567in}{0.413320in}}%
\pgfpathlineto{\pgfqpoint{2.575779in}{0.413320in}}%
\pgfpathlineto{\pgfqpoint{2.573082in}{0.413320in}}%
\pgfpathlineto{\pgfqpoint{2.570411in}{0.413320in}}%
\pgfpathlineto{\pgfqpoint{2.567730in}{0.413320in}}%
\pgfpathlineto{\pgfqpoint{2.565045in}{0.413320in}}%
\pgfpathlineto{\pgfqpoint{2.562375in}{0.413320in}}%
\pgfpathlineto{\pgfqpoint{2.559790in}{0.413320in}}%
\pgfpathlineto{\pgfqpoint{2.557009in}{0.413320in}}%
\pgfpathlineto{\pgfqpoint{2.554493in}{0.413320in}}%
\pgfpathlineto{\pgfqpoint{2.551664in}{0.413320in}}%
\pgfpathlineto{\pgfqpoint{2.549114in}{0.413320in}}%
\pgfpathlineto{\pgfqpoint{2.546310in}{0.413320in}}%
\pgfpathlineto{\pgfqpoint{2.543765in}{0.413320in}}%
\pgfpathlineto{\pgfqpoint{2.540949in}{0.413320in}}%
\pgfpathlineto{\pgfqpoint{2.538274in}{0.413320in}}%
\pgfpathlineto{\pgfqpoint{2.535624in}{0.413320in}}%
\pgfpathlineto{\pgfqpoint{2.532917in}{0.413320in}}%
\pgfpathlineto{\pgfqpoint{2.530234in}{0.413320in}}%
\pgfpathlineto{\pgfqpoint{2.527560in}{0.413320in}}%
\pgfpathlineto{\pgfqpoint{2.524988in}{0.413320in}}%
\pgfpathlineto{\pgfqpoint{2.522197in}{0.413320in}}%
\pgfpathlineto{\pgfqpoint{2.519607in}{0.413320in}}%
\pgfpathlineto{\pgfqpoint{2.516845in}{0.413320in}}%
\pgfpathlineto{\pgfqpoint{2.514268in}{0.413320in}}%
\pgfpathlineto{\pgfqpoint{2.511478in}{0.413320in}}%
\pgfpathlineto{\pgfqpoint{2.508917in}{0.413320in}}%
\pgfpathlineto{\pgfqpoint{2.506163in}{0.413320in}}%
\pgfpathlineto{\pgfqpoint{2.503454in}{0.413320in}}%
\pgfpathlineto{\pgfqpoint{2.500801in}{0.413320in}}%
\pgfpathlineto{\pgfqpoint{2.498085in}{0.413320in}}%
\pgfpathlineto{\pgfqpoint{2.495542in}{0.413320in}}%
\pgfpathlineto{\pgfqpoint{2.492729in}{0.413320in}}%
\pgfpathlineto{\pgfqpoint{2.490183in}{0.413320in}}%
\pgfpathlineto{\pgfqpoint{2.487384in}{0.413320in}}%
\pgfpathlineto{\pgfqpoint{2.484870in}{0.413320in}}%
\pgfpathlineto{\pgfqpoint{2.482026in}{0.413320in}}%
\pgfpathlineto{\pgfqpoint{2.479420in}{0.413320in}}%
\pgfpathlineto{\pgfqpoint{2.476671in}{0.413320in}}%
\pgfpathlineto{\pgfqpoint{2.473989in}{0.413320in}}%
\pgfpathlineto{\pgfqpoint{2.471311in}{0.413320in}}%
\pgfpathlineto{\pgfqpoint{2.468635in}{0.413320in}}%
\pgfpathlineto{\pgfqpoint{2.465957in}{0.413320in}}%
\pgfpathlineto{\pgfqpoint{2.463280in}{0.413320in}}%
\pgfpathlineto{\pgfqpoint{2.460711in}{0.413320in}}%
\pgfpathlineto{\pgfqpoint{2.457917in}{0.413320in}}%
\pgfpathlineto{\pgfqpoint{2.455353in}{0.413320in}}%
\pgfpathlineto{\pgfqpoint{2.452562in}{0.413320in}}%
\pgfpathlineto{\pgfqpoint{2.450032in}{0.413320in}}%
\pgfpathlineto{\pgfqpoint{2.447209in}{0.413320in}}%
\pgfpathlineto{\pgfqpoint{2.444677in}{0.413320in}}%
\pgfpathlineto{\pgfqpoint{2.441876in}{0.413320in}}%
\pgfpathlineto{\pgfqpoint{2.439167in}{0.413320in}}%
\pgfpathlineto{\pgfqpoint{2.436518in}{0.413320in}}%
\pgfpathlineto{\pgfqpoint{2.433815in}{0.413320in}}%
\pgfpathlineto{\pgfqpoint{2.431251in}{0.413320in}}%
\pgfpathlineto{\pgfqpoint{2.428453in}{0.413320in}}%
\pgfpathlineto{\pgfqpoint{2.425878in}{0.413320in}}%
\pgfpathlineto{\pgfqpoint{2.423098in}{0.413320in}}%
\pgfpathlineto{\pgfqpoint{2.420528in}{0.413320in}}%
\pgfpathlineto{\pgfqpoint{2.417747in}{0.413320in}}%
\pgfpathlineto{\pgfqpoint{2.415184in}{0.413320in}}%
\pgfpathlineto{\pgfqpoint{2.412389in}{0.413320in}}%
\pgfpathlineto{\pgfqpoint{2.409699in}{0.413320in}}%
\pgfpathlineto{\pgfqpoint{2.407024in}{0.413320in}}%
\pgfpathlineto{\pgfqpoint{2.404352in}{0.413320in}}%
\pgfpathlineto{\pgfqpoint{2.401675in}{0.413320in}}%
\pgfpathlineto{\pgfqpoint{2.398995in}{0.413320in}}%
\pgfpathclose%
\pgfusepath{stroke,fill}%
\end{pgfscope}%
\begin{pgfscope}%
\pgfpathrectangle{\pgfqpoint{2.398995in}{0.319877in}}{\pgfqpoint{3.986877in}{1.993438in}} %
\pgfusepath{clip}%
\pgfsetbuttcap%
\pgfsetroundjoin%
\definecolor{currentfill}{rgb}{1.000000,1.000000,1.000000}%
\pgfsetfillcolor{currentfill}%
\pgfsetlinewidth{1.003750pt}%
\definecolor{currentstroke}{rgb}{0.964025,0.412068,0.730354}%
\pgfsetstrokecolor{currentstroke}%
\pgfsetdash{}{0pt}%
\pgfpathmoveto{\pgfqpoint{2.398995in}{0.413320in}}%
\pgfpathlineto{\pgfqpoint{2.398995in}{1.785093in}}%
\pgfpathlineto{\pgfqpoint{2.401675in}{1.792354in}}%
\pgfpathlineto{\pgfqpoint{2.404352in}{1.785795in}}%
\pgfpathlineto{\pgfqpoint{2.407024in}{1.791359in}}%
\pgfpathlineto{\pgfqpoint{2.409699in}{1.791977in}}%
\pgfpathlineto{\pgfqpoint{2.412389in}{1.796006in}}%
\pgfpathlineto{\pgfqpoint{2.415184in}{1.790483in}}%
\pgfpathlineto{\pgfqpoint{2.417747in}{1.798492in}}%
\pgfpathlineto{\pgfqpoint{2.420528in}{1.801075in}}%
\pgfpathlineto{\pgfqpoint{2.423098in}{1.800611in}}%
\pgfpathlineto{\pgfqpoint{2.425878in}{1.789597in}}%
\pgfpathlineto{\pgfqpoint{2.428453in}{1.793821in}}%
\pgfpathlineto{\pgfqpoint{2.431251in}{1.805363in}}%
\pgfpathlineto{\pgfqpoint{2.433815in}{1.806915in}}%
\pgfpathlineto{\pgfqpoint{2.436518in}{1.808756in}}%
\pgfpathlineto{\pgfqpoint{2.439167in}{1.799936in}}%
\pgfpathlineto{\pgfqpoint{2.441876in}{1.807049in}}%
\pgfpathlineto{\pgfqpoint{2.444677in}{1.808124in}}%
\pgfpathlineto{\pgfqpoint{2.447209in}{1.798351in}}%
\pgfpathlineto{\pgfqpoint{2.450032in}{1.804429in}}%
\pgfpathlineto{\pgfqpoint{2.452562in}{1.804008in}}%
\pgfpathlineto{\pgfqpoint{2.455353in}{1.798026in}}%
\pgfpathlineto{\pgfqpoint{2.457917in}{1.797407in}}%
\pgfpathlineto{\pgfqpoint{2.460711in}{1.789223in}}%
\pgfpathlineto{\pgfqpoint{2.463280in}{1.796456in}}%
\pgfpathlineto{\pgfqpoint{2.465957in}{1.788683in}}%
\pgfpathlineto{\pgfqpoint{2.468635in}{1.794035in}}%
\pgfpathlineto{\pgfqpoint{2.471311in}{1.788559in}}%
\pgfpathlineto{\pgfqpoint{2.473989in}{1.791207in}}%
\pgfpathlineto{\pgfqpoint{2.476671in}{1.785176in}}%
\pgfpathlineto{\pgfqpoint{2.479420in}{1.791624in}}%
\pgfpathlineto{\pgfqpoint{2.482026in}{1.813994in}}%
\pgfpathlineto{\pgfqpoint{2.484870in}{1.851666in}}%
\pgfpathlineto{\pgfqpoint{2.487384in}{1.837417in}}%
\pgfpathlineto{\pgfqpoint{2.490183in}{1.816740in}}%
\pgfpathlineto{\pgfqpoint{2.492729in}{1.805544in}}%
\pgfpathlineto{\pgfqpoint{2.495542in}{1.807859in}}%
\pgfpathlineto{\pgfqpoint{2.498085in}{1.798370in}}%
\pgfpathlineto{\pgfqpoint{2.500801in}{1.796055in}}%
\pgfpathlineto{\pgfqpoint{2.503454in}{1.805011in}}%
\pgfpathlineto{\pgfqpoint{2.506163in}{1.803561in}}%
\pgfpathlineto{\pgfqpoint{2.508917in}{1.791525in}}%
\pgfpathlineto{\pgfqpoint{2.511478in}{1.793450in}}%
\pgfpathlineto{\pgfqpoint{2.514268in}{1.795392in}}%
\pgfpathlineto{\pgfqpoint{2.516845in}{1.794554in}}%
\pgfpathlineto{\pgfqpoint{2.519607in}{1.795415in}}%
\pgfpathlineto{\pgfqpoint{2.522197in}{1.808127in}}%
\pgfpathlineto{\pgfqpoint{2.524988in}{1.826190in}}%
\pgfpathlineto{\pgfqpoint{2.527560in}{1.833535in}}%
\pgfpathlineto{\pgfqpoint{2.530234in}{1.818555in}}%
\pgfpathlineto{\pgfqpoint{2.532917in}{1.813608in}}%
\pgfpathlineto{\pgfqpoint{2.535624in}{1.811267in}}%
\pgfpathlineto{\pgfqpoint{2.538274in}{1.812616in}}%
\pgfpathlineto{\pgfqpoint{2.540949in}{1.815382in}}%
\pgfpathlineto{\pgfqpoint{2.543765in}{1.811522in}}%
\pgfpathlineto{\pgfqpoint{2.546310in}{1.804845in}}%
\pgfpathlineto{\pgfqpoint{2.549114in}{1.800099in}}%
\pgfpathlineto{\pgfqpoint{2.551664in}{1.794945in}}%
\pgfpathlineto{\pgfqpoint{2.554493in}{1.806152in}}%
\pgfpathlineto{\pgfqpoint{2.557009in}{1.807562in}}%
\pgfpathlineto{\pgfqpoint{2.559790in}{1.802230in}}%
\pgfpathlineto{\pgfqpoint{2.562375in}{1.799249in}}%
\pgfpathlineto{\pgfqpoint{2.565045in}{1.796221in}}%
\pgfpathlineto{\pgfqpoint{2.567730in}{1.796291in}}%
\pgfpathlineto{\pgfqpoint{2.570411in}{1.791666in}}%
\pgfpathlineto{\pgfqpoint{2.573082in}{1.791930in}}%
\pgfpathlineto{\pgfqpoint{2.575779in}{1.796504in}}%
\pgfpathlineto{\pgfqpoint{2.578567in}{1.805102in}}%
\pgfpathlineto{\pgfqpoint{2.581129in}{1.802250in}}%
\pgfpathlineto{\pgfqpoint{2.583913in}{1.797253in}}%
\pgfpathlineto{\pgfqpoint{2.586484in}{1.793632in}}%
\pgfpathlineto{\pgfqpoint{2.589248in}{1.793942in}}%
\pgfpathlineto{\pgfqpoint{2.591842in}{1.794970in}}%
\pgfpathlineto{\pgfqpoint{2.594630in}{1.793543in}}%
\pgfpathlineto{\pgfqpoint{2.597196in}{1.788463in}}%
\pgfpathlineto{\pgfqpoint{2.599920in}{1.792150in}}%
\pgfpathlineto{\pgfqpoint{2.602557in}{1.796252in}}%
\pgfpathlineto{\pgfqpoint{2.605232in}{1.795501in}}%
\pgfpathlineto{\pgfqpoint{2.608004in}{1.791237in}}%
\pgfpathlineto{\pgfqpoint{2.610588in}{1.792119in}}%
\pgfpathlineto{\pgfqpoint{2.613393in}{1.790121in}}%
\pgfpathlineto{\pgfqpoint{2.615934in}{1.792217in}}%
\pgfpathlineto{\pgfqpoint{2.618773in}{1.791444in}}%
\pgfpathlineto{\pgfqpoint{2.621304in}{1.787827in}}%
\pgfpathlineto{\pgfqpoint{2.624077in}{1.785088in}}%
\pgfpathlineto{\pgfqpoint{2.626653in}{1.785979in}}%
\pgfpathlineto{\pgfqpoint{2.629340in}{1.785645in}}%
\pgfpathlineto{\pgfqpoint{2.632018in}{1.786448in}}%
\pgfpathlineto{\pgfqpoint{2.634700in}{1.789700in}}%
\pgfpathlineto{\pgfqpoint{2.637369in}{1.790986in}}%
\pgfpathlineto{\pgfqpoint{2.640053in}{1.795015in}}%
\pgfpathlineto{\pgfqpoint{2.642827in}{1.788463in}}%
\pgfpathlineto{\pgfqpoint{2.645408in}{1.791984in}}%
\pgfpathlineto{\pgfqpoint{2.648196in}{1.785948in}}%
\pgfpathlineto{\pgfqpoint{2.650767in}{1.788745in}}%
\pgfpathlineto{\pgfqpoint{2.653567in}{1.791263in}}%
\pgfpathlineto{\pgfqpoint{2.656124in}{1.785451in}}%
\pgfpathlineto{\pgfqpoint{2.658942in}{1.789946in}}%
\pgfpathlineto{\pgfqpoint{2.661481in}{1.784318in}}%
\pgfpathlineto{\pgfqpoint{2.664151in}{1.780977in}}%
\pgfpathlineto{\pgfqpoint{2.666836in}{1.786305in}}%
\pgfpathlineto{\pgfqpoint{2.669506in}{1.781391in}}%
\pgfpathlineto{\pgfqpoint{2.672301in}{1.788059in}}%
\pgfpathlineto{\pgfqpoint{2.674873in}{1.778518in}}%
\pgfpathlineto{\pgfqpoint{2.677650in}{1.784912in}}%
\pgfpathlineto{\pgfqpoint{2.680224in}{1.780648in}}%
\pgfpathlineto{\pgfqpoint{2.683009in}{1.779555in}}%
\pgfpathlineto{\pgfqpoint{2.685586in}{1.785897in}}%
\pgfpathlineto{\pgfqpoint{2.688328in}{1.787827in}}%
\pgfpathlineto{\pgfqpoint{2.690940in}{1.792804in}}%
\pgfpathlineto{\pgfqpoint{2.693611in}{1.786916in}}%
\pgfpathlineto{\pgfqpoint{2.696293in}{1.790956in}}%
\pgfpathlineto{\pgfqpoint{2.698968in}{1.789326in}}%
\pgfpathlineto{\pgfqpoint{2.701657in}{1.786067in}}%
\pgfpathlineto{\pgfqpoint{2.704326in}{1.792343in}}%
\pgfpathlineto{\pgfqpoint{2.707125in}{1.796001in}}%
\pgfpathlineto{\pgfqpoint{2.709683in}{1.785734in}}%
\pgfpathlineto{\pgfqpoint{2.712477in}{1.793693in}}%
\pgfpathlineto{\pgfqpoint{2.715036in}{1.784153in}}%
\pgfpathlineto{\pgfqpoint{2.717773in}{1.790036in}}%
\pgfpathlineto{\pgfqpoint{2.720404in}{1.791173in}}%
\pgfpathlineto{\pgfqpoint{2.723211in}{1.793588in}}%
\pgfpathlineto{\pgfqpoint{2.725760in}{1.789417in}}%
\pgfpathlineto{\pgfqpoint{2.728439in}{1.788849in}}%
\pgfpathlineto{\pgfqpoint{2.731119in}{1.791557in}}%
\pgfpathlineto{\pgfqpoint{2.733798in}{1.794367in}}%
\pgfpathlineto{\pgfqpoint{2.736476in}{1.803619in}}%
\pgfpathlineto{\pgfqpoint{2.739155in}{1.806346in}}%
\pgfpathlineto{\pgfqpoint{2.741928in}{1.797682in}}%
\pgfpathlineto{\pgfqpoint{2.744510in}{1.790334in}}%
\pgfpathlineto{\pgfqpoint{2.747260in}{1.793907in}}%
\pgfpathlineto{\pgfqpoint{2.749868in}{1.797768in}}%
\pgfpathlineto{\pgfqpoint{2.752614in}{1.796772in}}%
\pgfpathlineto{\pgfqpoint{2.755224in}{1.796241in}}%
\pgfpathlineto{\pgfqpoint{2.758028in}{1.801764in}}%
\pgfpathlineto{\pgfqpoint{2.760581in}{1.795139in}}%
\pgfpathlineto{\pgfqpoint{2.763253in}{1.792290in}}%
\pgfpathlineto{\pgfqpoint{2.765935in}{1.790947in}}%
\pgfpathlineto{\pgfqpoint{2.768617in}{1.793040in}}%
\pgfpathlineto{\pgfqpoint{2.771373in}{1.790738in}}%
\pgfpathlineto{\pgfqpoint{2.773972in}{1.789246in}}%
\pgfpathlineto{\pgfqpoint{2.776767in}{1.789749in}}%
\pgfpathlineto{\pgfqpoint{2.779330in}{1.791110in}}%
\pgfpathlineto{\pgfqpoint{2.782113in}{1.791431in}}%
\pgfpathlineto{\pgfqpoint{2.784687in}{1.786425in}}%
\pgfpathlineto{\pgfqpoint{2.787468in}{1.794977in}}%
\pgfpathlineto{\pgfqpoint{2.790044in}{1.794291in}}%
\pgfpathlineto{\pgfqpoint{2.792721in}{1.792730in}}%
\pgfpathlineto{\pgfqpoint{2.795398in}{1.794817in}}%
\pgfpathlineto{\pgfqpoint{2.798070in}{1.787576in}}%
\pgfpathlineto{\pgfqpoint{2.800756in}{1.782434in}}%
\pgfpathlineto{\pgfqpoint{2.803435in}{1.787515in}}%
\pgfpathlineto{\pgfqpoint{2.806175in}{1.789236in}}%
\pgfpathlineto{\pgfqpoint{2.808792in}{1.785899in}}%
\pgfpathlineto{\pgfqpoint{2.811597in}{1.780729in}}%
\pgfpathlineto{\pgfqpoint{2.814141in}{1.780376in}}%
\pgfpathlineto{\pgfqpoint{2.816867in}{1.782942in}}%
\pgfpathlineto{\pgfqpoint{2.819506in}{1.786804in}}%
\pgfpathlineto{\pgfqpoint{2.822303in}{1.787861in}}%
\pgfpathlineto{\pgfqpoint{2.824851in}{1.785676in}}%
\pgfpathlineto{\pgfqpoint{2.827567in}{1.791344in}}%
\pgfpathlineto{\pgfqpoint{2.830219in}{1.787023in}}%
\pgfpathlineto{\pgfqpoint{2.832894in}{1.786965in}}%
\pgfpathlineto{\pgfqpoint{2.835698in}{1.787888in}}%
\pgfpathlineto{\pgfqpoint{2.838254in}{1.792736in}}%
\pgfpathlineto{\pgfqpoint{2.841055in}{1.801276in}}%
\pgfpathlineto{\pgfqpoint{2.843611in}{1.802499in}}%
\pgfpathlineto{\pgfqpoint{2.846408in}{1.797435in}}%
\pgfpathlineto{\pgfqpoint{2.848960in}{1.794117in}}%
\pgfpathlineto{\pgfqpoint{2.851793in}{1.789542in}}%
\pgfpathlineto{\pgfqpoint{2.854325in}{1.784348in}}%
\pgfpathlineto{\pgfqpoint{2.857003in}{1.786375in}}%
\pgfpathlineto{\pgfqpoint{2.859668in}{1.785346in}}%
\pgfpathlineto{\pgfqpoint{2.862402in}{1.788603in}}%
\pgfpathlineto{\pgfqpoint{2.865031in}{1.784810in}}%
\pgfpathlineto{\pgfqpoint{2.867713in}{1.784088in}}%
\pgfpathlineto{\pgfqpoint{2.870475in}{1.785471in}}%
\pgfpathlineto{\pgfqpoint{2.873074in}{1.781113in}}%
\pgfpathlineto{\pgfqpoint{2.875882in}{1.786807in}}%
\pgfpathlineto{\pgfqpoint{2.878431in}{1.793552in}}%
\pgfpathlineto{\pgfqpoint{2.881254in}{1.789438in}}%
\pgfpathlineto{\pgfqpoint{2.883780in}{1.786934in}}%
\pgfpathlineto{\pgfqpoint{2.886578in}{1.789089in}}%
\pgfpathlineto{\pgfqpoint{2.889145in}{1.783816in}}%
\pgfpathlineto{\pgfqpoint{2.891809in}{1.785804in}}%
\pgfpathlineto{\pgfqpoint{2.894487in}{1.787677in}}%
\pgfpathlineto{\pgfqpoint{2.897179in}{1.790320in}}%
\pgfpathlineto{\pgfqpoint{2.899858in}{1.791986in}}%
\pgfpathlineto{\pgfqpoint{2.902535in}{1.786172in}}%
\pgfpathlineto{\pgfqpoint{2.905341in}{1.791063in}}%
\pgfpathlineto{\pgfqpoint{2.907882in}{1.788057in}}%
\pgfpathlineto{\pgfqpoint{2.910631in}{1.792410in}}%
\pgfpathlineto{\pgfqpoint{2.913243in}{1.791794in}}%
\pgfpathlineto{\pgfqpoint{2.916061in}{1.779419in}}%
\pgfpathlineto{\pgfqpoint{2.918606in}{1.780731in}}%
\pgfpathlineto{\pgfqpoint{2.921363in}{1.784345in}}%
\pgfpathlineto{\pgfqpoint{2.923963in}{1.783348in}}%
\pgfpathlineto{\pgfqpoint{2.926655in}{1.788222in}}%
\pgfpathlineto{\pgfqpoint{2.929321in}{1.785689in}}%
\pgfpathlineto{\pgfqpoint{2.932033in}{1.788107in}}%
\pgfpathlineto{\pgfqpoint{2.934759in}{1.778935in}}%
\pgfpathlineto{\pgfqpoint{2.937352in}{1.781901in}}%
\pgfpathlineto{\pgfqpoint{2.940120in}{1.774885in}}%
\pgfpathlineto{\pgfqpoint{2.942711in}{1.774885in}}%
\pgfpathlineto{\pgfqpoint{2.945461in}{1.774885in}}%
\pgfpathlineto{\pgfqpoint{2.948068in}{1.785363in}}%
\pgfpathlineto{\pgfqpoint{2.950884in}{1.786404in}}%
\pgfpathlineto{\pgfqpoint{2.953422in}{1.787939in}}%
\pgfpathlineto{\pgfqpoint{2.956103in}{1.792925in}}%
\pgfpathlineto{\pgfqpoint{2.958782in}{1.788674in}}%
\pgfpathlineto{\pgfqpoint{2.961460in}{1.791418in}}%
\pgfpathlineto{\pgfqpoint{2.964127in}{1.792261in}}%
\pgfpathlineto{\pgfqpoint{2.966812in}{1.786153in}}%
\pgfpathlineto{\pgfqpoint{2.969599in}{1.789973in}}%
\pgfpathlineto{\pgfqpoint{2.972177in}{1.792331in}}%
\pgfpathlineto{\pgfqpoint{2.974972in}{1.788431in}}%
\pgfpathlineto{\pgfqpoint{2.977517in}{1.787690in}}%
\pgfpathlineto{\pgfqpoint{2.980341in}{1.791909in}}%
\pgfpathlineto{\pgfqpoint{2.982885in}{1.789047in}}%
\pgfpathlineto{\pgfqpoint{2.985666in}{1.789453in}}%
\pgfpathlineto{\pgfqpoint{2.988238in}{1.789454in}}%
\pgfpathlineto{\pgfqpoint{2.990978in}{1.792389in}}%
\pgfpathlineto{\pgfqpoint{2.993595in}{1.790028in}}%
\pgfpathlineto{\pgfqpoint{2.996300in}{1.786652in}}%
\pgfpathlineto{\pgfqpoint{2.999103in}{1.794300in}}%
\pgfpathlineto{\pgfqpoint{3.001635in}{1.792093in}}%
\pgfpathlineto{\pgfqpoint{3.004419in}{1.784213in}}%
\pgfpathlineto{\pgfqpoint{3.006993in}{1.787721in}}%
\pgfpathlineto{\pgfqpoint{3.009784in}{1.788362in}}%
\pgfpathlineto{\pgfqpoint{3.012351in}{1.783393in}}%
\pgfpathlineto{\pgfqpoint{3.015097in}{1.785065in}}%
\pgfpathlineto{\pgfqpoint{3.017707in}{1.789058in}}%
\pgfpathlineto{\pgfqpoint{3.020382in}{1.788009in}}%
\pgfpathlineto{\pgfqpoint{3.023058in}{1.801023in}}%
\pgfpathlineto{\pgfqpoint{3.025803in}{1.803711in}}%
\pgfpathlineto{\pgfqpoint{3.028412in}{1.790299in}}%
\pgfpathlineto{\pgfqpoint{3.031091in}{1.787071in}}%
\pgfpathlineto{\pgfqpoint{3.033921in}{1.788988in}}%
\pgfpathlineto{\pgfqpoint{3.036456in}{1.788283in}}%
\pgfpathlineto{\pgfqpoint{3.039262in}{1.782993in}}%
\pgfpathlineto{\pgfqpoint{3.041813in}{1.795585in}}%
\pgfpathlineto{\pgfqpoint{3.044568in}{1.792372in}}%
\pgfpathlineto{\pgfqpoint{3.047157in}{1.783747in}}%
\pgfpathlineto{\pgfqpoint{3.049988in}{1.774885in}}%
\pgfpathlineto{\pgfqpoint{3.052526in}{1.774885in}}%
\pgfpathlineto{\pgfqpoint{3.055202in}{1.778317in}}%
\pgfpathlineto{\pgfqpoint{3.057884in}{1.780694in}}%
\pgfpathlineto{\pgfqpoint{3.060561in}{1.780131in}}%
\pgfpathlineto{\pgfqpoint{3.063230in}{1.787777in}}%
\pgfpathlineto{\pgfqpoint{3.065916in}{1.788807in}}%
\pgfpathlineto{\pgfqpoint{3.068709in}{1.786116in}}%
\pgfpathlineto{\pgfqpoint{3.071266in}{1.794012in}}%
\pgfpathlineto{\pgfqpoint{3.074056in}{1.791963in}}%
\pgfpathlineto{\pgfqpoint{3.076631in}{1.789340in}}%
\pgfpathlineto{\pgfqpoint{3.079381in}{1.787055in}}%
\pgfpathlineto{\pgfqpoint{3.081990in}{1.789909in}}%
\pgfpathlineto{\pgfqpoint{3.084671in}{1.788604in}}%
\pgfpathlineto{\pgfqpoint{3.087343in}{1.786942in}}%
\pgfpathlineto{\pgfqpoint{3.090023in}{1.793023in}}%
\pgfpathlineto{\pgfqpoint{3.092699in}{1.788781in}}%
\pgfpathlineto{\pgfqpoint{3.095388in}{1.790209in}}%
\pgfpathlineto{\pgfqpoint{3.098163in}{1.788976in}}%
\pgfpathlineto{\pgfqpoint{3.100737in}{1.792124in}}%
\pgfpathlineto{\pgfqpoint{3.103508in}{1.791873in}}%
\pgfpathlineto{\pgfqpoint{3.106094in}{1.789659in}}%
\pgfpathlineto{\pgfqpoint{3.108896in}{1.786844in}}%
\pgfpathlineto{\pgfqpoint{3.111451in}{1.782859in}}%
\pgfpathlineto{\pgfqpoint{3.114242in}{1.783014in}}%
\pgfpathlineto{\pgfqpoint{3.116807in}{1.789235in}}%
\pgfpathlineto{\pgfqpoint{3.119487in}{1.787546in}}%
\pgfpathlineto{\pgfqpoint{3.122163in}{1.783641in}}%
\pgfpathlineto{\pgfqpoint{3.124842in}{1.783463in}}%
\pgfpathlineto{\pgfqpoint{3.127512in}{1.786740in}}%
\pgfpathlineto{\pgfqpoint{3.130199in}{1.787805in}}%
\pgfpathlineto{\pgfqpoint{3.132946in}{1.786421in}}%
\pgfpathlineto{\pgfqpoint{3.135550in}{1.784712in}}%
\pgfpathlineto{\pgfqpoint{3.138375in}{1.782341in}}%
\pgfpathlineto{\pgfqpoint{3.140913in}{1.775454in}}%
\pgfpathlineto{\pgfqpoint{3.143740in}{1.775758in}}%
\pgfpathlineto{\pgfqpoint{3.146271in}{1.780015in}}%
\pgfpathlineto{\pgfqpoint{3.149057in}{1.779247in}}%
\pgfpathlineto{\pgfqpoint{3.151612in}{1.781528in}}%
\pgfpathlineto{\pgfqpoint{3.154327in}{1.791289in}}%
\pgfpathlineto{\pgfqpoint{3.156981in}{1.783001in}}%
\pgfpathlineto{\pgfqpoint{3.159675in}{1.788353in}}%
\pgfpathlineto{\pgfqpoint{3.162474in}{1.779398in}}%
\pgfpathlineto{\pgfqpoint{3.165019in}{1.783276in}}%
\pgfpathlineto{\pgfqpoint{3.167776in}{1.782889in}}%
\pgfpathlineto{\pgfqpoint{3.170375in}{1.784991in}}%
\pgfpathlineto{\pgfqpoint{3.173142in}{1.790384in}}%
\pgfpathlineto{\pgfqpoint{3.175724in}{1.791366in}}%
\pgfpathlineto{\pgfqpoint{3.178525in}{1.789768in}}%
\pgfpathlineto{\pgfqpoint{3.181089in}{1.786248in}}%
\pgfpathlineto{\pgfqpoint{3.183760in}{1.778306in}}%
\pgfpathlineto{\pgfqpoint{3.186440in}{1.784302in}}%
\pgfpathlineto{\pgfqpoint{3.189117in}{1.781460in}}%
\pgfpathlineto{\pgfqpoint{3.191796in}{1.784351in}}%
\pgfpathlineto{\pgfqpoint{3.194508in}{1.787567in}}%
\pgfpathlineto{\pgfqpoint{3.197226in}{1.786233in}}%
\pgfpathlineto{\pgfqpoint{3.199823in}{1.783031in}}%
\pgfpathlineto{\pgfqpoint{3.202562in}{1.790409in}}%
\pgfpathlineto{\pgfqpoint{3.205195in}{1.784760in}}%
\pgfpathlineto{\pgfqpoint{3.207984in}{1.784204in}}%
\pgfpathlineto{\pgfqpoint{3.210545in}{1.780514in}}%
\pgfpathlineto{\pgfqpoint{3.213342in}{1.783870in}}%
\pgfpathlineto{\pgfqpoint{3.215908in}{1.783740in}}%
\pgfpathlineto{\pgfqpoint{3.218586in}{1.780978in}}%
\pgfpathlineto{\pgfqpoint{3.221255in}{1.780581in}}%
\pgfpathlineto{\pgfqpoint{3.223942in}{1.782311in}}%
\pgfpathlineto{\pgfqpoint{3.226609in}{1.786879in}}%
\pgfpathlineto{\pgfqpoint{3.229310in}{1.790966in}}%
\pgfpathlineto{\pgfqpoint{3.232069in}{1.782679in}}%
\pgfpathlineto{\pgfqpoint{3.234658in}{1.783927in}}%
\pgfpathlineto{\pgfqpoint{3.237411in}{1.791814in}}%
\pgfpathlineto{\pgfqpoint{3.240010in}{1.791847in}}%
\pgfpathlineto{\pgfqpoint{3.242807in}{1.787166in}}%
\pgfpathlineto{\pgfqpoint{3.245363in}{1.794010in}}%
\pgfpathlineto{\pgfqpoint{3.248049in}{1.784785in}}%
\pgfpathlineto{\pgfqpoint{3.250716in}{1.787505in}}%
\pgfpathlineto{\pgfqpoint{3.253404in}{1.787128in}}%
\pgfpathlineto{\pgfqpoint{3.256083in}{1.787368in}}%
\pgfpathlineto{\pgfqpoint{3.258784in}{1.788160in}}%
\pgfpathlineto{\pgfqpoint{3.261594in}{1.785093in}}%
\pgfpathlineto{\pgfqpoint{3.264119in}{1.781812in}}%
\pgfpathlineto{\pgfqpoint{3.266849in}{1.783160in}}%
\pgfpathlineto{\pgfqpoint{3.269478in}{1.781667in}}%
\pgfpathlineto{\pgfqpoint{3.272254in}{1.780778in}}%
\pgfpathlineto{\pgfqpoint{3.274831in}{1.788281in}}%
\pgfpathlineto{\pgfqpoint{3.277603in}{1.792944in}}%
\pgfpathlineto{\pgfqpoint{3.280189in}{1.793132in}}%
\pgfpathlineto{\pgfqpoint{3.282870in}{1.792180in}}%
\pgfpathlineto{\pgfqpoint{3.285534in}{1.794545in}}%
\pgfpathlineto{\pgfqpoint{3.288225in}{1.790643in}}%
\pgfpathlineto{\pgfqpoint{3.290890in}{1.794821in}}%
\pgfpathlineto{\pgfqpoint{3.293574in}{1.785439in}}%
\pgfpathlineto{\pgfqpoint{3.296376in}{1.792829in}}%
\pgfpathlineto{\pgfqpoint{3.298937in}{1.790143in}}%
\pgfpathlineto{\pgfqpoint{3.301719in}{1.788313in}}%
\pgfpathlineto{\pgfqpoint{3.304295in}{1.788569in}}%
\pgfpathlineto{\pgfqpoint{3.307104in}{1.788029in}}%
\pgfpathlineto{\pgfqpoint{3.309652in}{1.786800in}}%
\pgfpathlineto{\pgfqpoint{3.312480in}{1.791955in}}%
\pgfpathlineto{\pgfqpoint{3.315008in}{1.781756in}}%
\pgfpathlineto{\pgfqpoint{3.317688in}{1.789293in}}%
\pgfpathlineto{\pgfqpoint{3.320366in}{1.786190in}}%
\pgfpathlineto{\pgfqpoint{3.323049in}{1.788641in}}%
\pgfpathlineto{\pgfqpoint{3.325860in}{1.781057in}}%
\pgfpathlineto{\pgfqpoint{3.328401in}{1.791405in}}%
\pgfpathlineto{\pgfqpoint{3.331183in}{1.789768in}}%
\pgfpathlineto{\pgfqpoint{3.333758in}{1.791837in}}%
\pgfpathlineto{\pgfqpoint{3.336541in}{1.795760in}}%
\pgfpathlineto{\pgfqpoint{3.339101in}{1.788229in}}%
\pgfpathlineto{\pgfqpoint{3.341893in}{1.795802in}}%
\pgfpathlineto{\pgfqpoint{3.344468in}{1.797076in}}%
\pgfpathlineto{\pgfqpoint{3.347139in}{1.791014in}}%
\pgfpathlineto{\pgfqpoint{3.349828in}{1.793495in}}%
\pgfpathlineto{\pgfqpoint{3.352505in}{1.784084in}}%
\pgfpathlineto{\pgfqpoint{3.355177in}{1.790957in}}%
\pgfpathlineto{\pgfqpoint{3.357862in}{1.796108in}}%
\pgfpathlineto{\pgfqpoint{3.360620in}{1.790360in}}%
\pgfpathlineto{\pgfqpoint{3.363221in}{1.787756in}}%
\pgfpathlineto{\pgfqpoint{3.365996in}{1.787505in}}%
\pgfpathlineto{\pgfqpoint{3.368577in}{1.789290in}}%
\pgfpathlineto{\pgfqpoint{3.371357in}{1.786103in}}%
\pgfpathlineto{\pgfqpoint{3.373921in}{1.787051in}}%
\pgfpathlineto{\pgfqpoint{3.376735in}{1.791315in}}%
\pgfpathlineto{\pgfqpoint{3.379290in}{1.790152in}}%
\pgfpathlineto{\pgfqpoint{3.381959in}{1.786851in}}%
\pgfpathlineto{\pgfqpoint{3.384647in}{1.791457in}}%
\pgfpathlineto{\pgfqpoint{3.387309in}{1.788727in}}%
\pgfpathlineto{\pgfqpoint{3.390102in}{1.788428in}}%
\pgfpathlineto{\pgfqpoint{3.392681in}{1.782662in}}%
\pgfpathlineto{\pgfqpoint{3.395461in}{1.787219in}}%
\pgfpathlineto{\pgfqpoint{3.398037in}{1.785475in}}%
\pgfpathlineto{\pgfqpoint{3.400783in}{1.793000in}}%
\pgfpathlineto{\pgfqpoint{3.403394in}{1.788473in}}%
\pgfpathlineto{\pgfqpoint{3.406202in}{1.788954in}}%
\pgfpathlineto{\pgfqpoint{3.408752in}{1.797682in}}%
\pgfpathlineto{\pgfqpoint{3.411431in}{1.794857in}}%
\pgfpathlineto{\pgfqpoint{3.414109in}{1.791081in}}%
\pgfpathlineto{\pgfqpoint{3.416780in}{1.795805in}}%
\pgfpathlineto{\pgfqpoint{3.419455in}{1.795456in}}%
\pgfpathlineto{\pgfqpoint{3.422142in}{1.794479in}}%
\pgfpathlineto{\pgfqpoint{3.424887in}{1.785874in}}%
\pgfpathlineto{\pgfqpoint{3.427501in}{1.793865in}}%
\pgfpathlineto{\pgfqpoint{3.430313in}{1.791921in}}%
\pgfpathlineto{\pgfqpoint{3.432851in}{1.786422in}}%
\pgfpathlineto{\pgfqpoint{3.435635in}{1.792516in}}%
\pgfpathlineto{\pgfqpoint{3.438210in}{1.792088in}}%
\pgfpathlineto{\pgfqpoint{3.440996in}{1.792002in}}%
\pgfpathlineto{\pgfqpoint{3.443574in}{1.789745in}}%
\pgfpathlineto{\pgfqpoint{3.446257in}{1.796764in}}%
\pgfpathlineto{\pgfqpoint{3.448926in}{1.797647in}}%
\pgfpathlineto{\pgfqpoint{3.451597in}{1.793169in}}%
\pgfpathlineto{\pgfqpoint{3.454285in}{1.787695in}}%
\pgfpathlineto{\pgfqpoint{3.456960in}{1.788141in}}%
\pgfpathlineto{\pgfqpoint{3.459695in}{1.790951in}}%
\pgfpathlineto{\pgfqpoint{3.462321in}{1.818527in}}%
\pgfpathlineto{\pgfqpoint{3.465072in}{1.845987in}}%
\pgfpathlineto{\pgfqpoint{3.467678in}{1.840103in}}%
\pgfpathlineto{\pgfqpoint{3.470466in}{1.831140in}}%
\pgfpathlineto{\pgfqpoint{3.473021in}{1.814144in}}%
\pgfpathlineto{\pgfqpoint{3.475821in}{1.802184in}}%
\pgfpathlineto{\pgfqpoint{3.478378in}{1.791511in}}%
\pgfpathlineto{\pgfqpoint{3.481072in}{1.791431in}}%
\pgfpathlineto{\pgfqpoint{3.483744in}{1.788074in}}%
\pgfpathlineto{\pgfqpoint{3.486442in}{1.794907in}}%
\pgfpathlineto{\pgfqpoint{3.489223in}{1.796820in}}%
\pgfpathlineto{\pgfqpoint{3.491783in}{1.788911in}}%
\pgfpathlineto{\pgfqpoint{3.494581in}{1.787221in}}%
\pgfpathlineto{\pgfqpoint{3.497139in}{1.793371in}}%
\pgfpathlineto{\pgfqpoint{3.499909in}{1.796122in}}%
\pgfpathlineto{\pgfqpoint{3.502488in}{1.796363in}}%
\pgfpathlineto{\pgfqpoint{3.505262in}{1.790195in}}%
\pgfpathlineto{\pgfqpoint{3.507840in}{1.784449in}}%
\pgfpathlineto{\pgfqpoint{3.510533in}{1.788236in}}%
\pgfpathlineto{\pgfqpoint{3.513209in}{1.789875in}}%
\pgfpathlineto{\pgfqpoint{3.515884in}{1.789388in}}%
\pgfpathlineto{\pgfqpoint{3.518565in}{1.785354in}}%
\pgfpathlineto{\pgfqpoint{3.521244in}{1.790697in}}%
\pgfpathlineto{\pgfqpoint{3.524041in}{1.791250in}}%
\pgfpathlineto{\pgfqpoint{3.526601in}{1.782893in}}%
\pgfpathlineto{\pgfqpoint{3.529327in}{1.789540in}}%
\pgfpathlineto{\pgfqpoint{3.531955in}{1.785862in}}%
\pgfpathlineto{\pgfqpoint{3.534783in}{1.785401in}}%
\pgfpathlineto{\pgfqpoint{3.537309in}{1.790785in}}%
\pgfpathlineto{\pgfqpoint{3.540093in}{1.787607in}}%
\pgfpathlineto{\pgfqpoint{3.542656in}{1.791170in}}%
\pgfpathlineto{\pgfqpoint{3.545349in}{1.791404in}}%
\pgfpathlineto{\pgfqpoint{3.548029in}{1.786325in}}%
\pgfpathlineto{\pgfqpoint{3.550713in}{1.786966in}}%
\pgfpathlineto{\pgfqpoint{3.553498in}{1.788190in}}%
\pgfpathlineto{\pgfqpoint{3.556061in}{1.791794in}}%
\pgfpathlineto{\pgfqpoint{3.558853in}{1.789917in}}%
\pgfpathlineto{\pgfqpoint{3.561420in}{1.788506in}}%
\pgfpathlineto{\pgfqpoint{3.564188in}{1.790026in}}%
\pgfpathlineto{\pgfqpoint{3.566774in}{1.789988in}}%
\pgfpathlineto{\pgfqpoint{3.569584in}{1.793716in}}%
\pgfpathlineto{\pgfqpoint{3.572126in}{1.797110in}}%
\pgfpathlineto{\pgfqpoint{3.574814in}{1.788365in}}%
\pgfpathlineto{\pgfqpoint{3.577487in}{1.794149in}}%
\pgfpathlineto{\pgfqpoint{3.580191in}{1.790916in}}%
\pgfpathlineto{\pgfqpoint{3.582851in}{1.791353in}}%
\pgfpathlineto{\pgfqpoint{3.585532in}{1.791072in}}%
\pgfpathlineto{\pgfqpoint{3.588258in}{1.793324in}}%
\pgfpathlineto{\pgfqpoint{3.590883in}{1.793424in}}%
\pgfpathlineto{\pgfqpoint{3.593620in}{1.787175in}}%
\pgfpathlineto{\pgfqpoint{3.596240in}{1.791451in}}%
\pgfpathlineto{\pgfqpoint{3.598998in}{1.793985in}}%
\pgfpathlineto{\pgfqpoint{3.601590in}{1.795330in}}%
\pgfpathlineto{\pgfqpoint{3.604387in}{1.795232in}}%
\pgfpathlineto{\pgfqpoint{3.606951in}{1.786276in}}%
\pgfpathlineto{\pgfqpoint{3.609632in}{1.790625in}}%
\pgfpathlineto{\pgfqpoint{3.612311in}{1.785836in}}%
\pgfpathlineto{\pgfqpoint{3.614982in}{1.783846in}}%
\pgfpathlineto{\pgfqpoint{3.617667in}{1.794503in}}%
\pgfpathlineto{\pgfqpoint{3.620345in}{1.797787in}}%
\pgfpathlineto{\pgfqpoint{3.623165in}{1.797812in}}%
\pgfpathlineto{\pgfqpoint{3.625689in}{1.803566in}}%
\pgfpathlineto{\pgfqpoint{3.628460in}{1.798571in}}%
\pgfpathlineto{\pgfqpoint{3.631058in}{1.790517in}}%
\pgfpathlineto{\pgfqpoint{3.633858in}{1.789770in}}%
\pgfpathlineto{\pgfqpoint{3.636413in}{1.790082in}}%
\pgfpathlineto{\pgfqpoint{3.639207in}{1.795990in}}%
\pgfpathlineto{\pgfqpoint{3.641773in}{1.799756in}}%
\pgfpathlineto{\pgfqpoint{3.644452in}{1.794157in}}%
\pgfpathlineto{\pgfqpoint{3.647130in}{1.791401in}}%
\pgfpathlineto{\pgfqpoint{3.649837in}{1.790422in}}%
\pgfpathlineto{\pgfqpoint{3.652628in}{1.795427in}}%
\pgfpathlineto{\pgfqpoint{3.655165in}{1.793481in}}%
\pgfpathlineto{\pgfqpoint{3.657917in}{1.794873in}}%
\pgfpathlineto{\pgfqpoint{3.660515in}{1.784345in}}%
\pgfpathlineto{\pgfqpoint{3.663276in}{1.778553in}}%
\pgfpathlineto{\pgfqpoint{3.665864in}{1.781552in}}%
\pgfpathlineto{\pgfqpoint{3.668665in}{1.787600in}}%
\pgfpathlineto{\pgfqpoint{3.671232in}{1.790479in}}%
\pgfpathlineto{\pgfqpoint{3.673911in}{1.792975in}}%
\pgfpathlineto{\pgfqpoint{3.676591in}{1.785209in}}%
\pgfpathlineto{\pgfqpoint{3.679273in}{1.789731in}}%
\pgfpathlineto{\pgfqpoint{3.681948in}{1.790307in}}%
\pgfpathlineto{\pgfqpoint{3.684620in}{1.794414in}}%
\pgfpathlineto{\pgfqpoint{3.687442in}{1.795548in}}%
\pgfpathlineto{\pgfqpoint{3.689983in}{1.792229in}}%
\pgfpathlineto{\pgfqpoint{3.692765in}{1.787265in}}%
\pgfpathlineto{\pgfqpoint{3.695331in}{1.787672in}}%
\pgfpathlineto{\pgfqpoint{3.698125in}{1.792244in}}%
\pgfpathlineto{\pgfqpoint{3.700684in}{1.789505in}}%
\pgfpathlineto{\pgfqpoint{3.703460in}{1.786105in}}%
\pgfpathlineto{\pgfqpoint{3.706053in}{1.782066in}}%
\pgfpathlineto{\pgfqpoint{3.708729in}{1.780624in}}%
\pgfpathlineto{\pgfqpoint{3.711410in}{1.785981in}}%
\pgfpathlineto{\pgfqpoint{3.714086in}{1.788226in}}%
\pgfpathlineto{\pgfqpoint{3.716875in}{1.789230in}}%
\pgfpathlineto{\pgfqpoint{3.719446in}{1.794274in}}%
\pgfpathlineto{\pgfqpoint{3.722228in}{1.789270in}}%
\pgfpathlineto{\pgfqpoint{3.724804in}{1.788168in}}%
\pgfpathlineto{\pgfqpoint{3.727581in}{1.788100in}}%
\pgfpathlineto{\pgfqpoint{3.730158in}{1.784080in}}%
\pgfpathlineto{\pgfqpoint{3.732950in}{1.790587in}}%
\pgfpathlineto{\pgfqpoint{3.735509in}{1.791706in}}%
\pgfpathlineto{\pgfqpoint{3.738194in}{1.786390in}}%
\pgfpathlineto{\pgfqpoint{3.740874in}{1.787071in}}%
\pgfpathlineto{\pgfqpoint{3.743548in}{1.789306in}}%
\pgfpathlineto{\pgfqpoint{3.746229in}{1.784436in}}%
\pgfpathlineto{\pgfqpoint{3.748903in}{1.786668in}}%
\pgfpathlineto{\pgfqpoint{3.751728in}{1.790851in}}%
\pgfpathlineto{\pgfqpoint{3.754265in}{1.793745in}}%
\pgfpathlineto{\pgfqpoint{3.757065in}{1.788822in}}%
\pgfpathlineto{\pgfqpoint{3.759622in}{1.791775in}}%
\pgfpathlineto{\pgfqpoint{3.762389in}{1.787729in}}%
\pgfpathlineto{\pgfqpoint{3.764966in}{1.795928in}}%
\pgfpathlineto{\pgfqpoint{3.767782in}{1.797635in}}%
\pgfpathlineto{\pgfqpoint{3.770323in}{1.795396in}}%
\pgfpathlineto{\pgfqpoint{3.773014in}{1.794652in}}%
\pgfpathlineto{\pgfqpoint{3.775691in}{1.807349in}}%
\pgfpathlineto{\pgfqpoint{3.778370in}{1.830668in}}%
\pgfpathlineto{\pgfqpoint{3.781046in}{1.852292in}}%
\pgfpathlineto{\pgfqpoint{3.783725in}{1.822923in}}%
\pgfpathlineto{\pgfqpoint{3.786504in}{1.816591in}}%
\pgfpathlineto{\pgfqpoint{3.789084in}{1.811946in}}%
\pgfpathlineto{\pgfqpoint{3.791897in}{1.807663in}}%
\pgfpathlineto{\pgfqpoint{3.794435in}{1.797735in}}%
\pgfpathlineto{\pgfqpoint{3.797265in}{1.803011in}}%
\pgfpathlineto{\pgfqpoint{3.799797in}{1.797168in}}%
\pgfpathlineto{\pgfqpoint{3.802569in}{1.792691in}}%
\pgfpathlineto{\pgfqpoint{3.805145in}{1.790943in}}%
\pgfpathlineto{\pgfqpoint{3.807832in}{1.784793in}}%
\pgfpathlineto{\pgfqpoint{3.810510in}{1.784252in}}%
\pgfpathlineto{\pgfqpoint{3.813172in}{1.775705in}}%
\pgfpathlineto{\pgfqpoint{3.815983in}{1.775039in}}%
\pgfpathlineto{\pgfqpoint{3.818546in}{1.774885in}}%
\pgfpathlineto{\pgfqpoint{3.821315in}{1.796376in}}%
\pgfpathlineto{\pgfqpoint{3.823903in}{1.799819in}}%
\pgfpathlineto{\pgfqpoint{3.826679in}{1.798005in}}%
\pgfpathlineto{\pgfqpoint{3.829252in}{1.793926in}}%
\pgfpathlineto{\pgfqpoint{3.832053in}{1.792601in}}%
\pgfpathlineto{\pgfqpoint{3.834616in}{1.793552in}}%
\pgfpathlineto{\pgfqpoint{3.837286in}{1.795286in}}%
\pgfpathlineto{\pgfqpoint{3.839960in}{1.793077in}}%
\pgfpathlineto{\pgfqpoint{3.842641in}{1.790977in}}%
\pgfpathlineto{\pgfqpoint{3.845329in}{1.796413in}}%
\pgfpathlineto{\pgfqpoint{3.848005in}{1.791873in}}%
\pgfpathlineto{\pgfqpoint{3.850814in}{1.782985in}}%
\pgfpathlineto{\pgfqpoint{3.853358in}{1.779501in}}%
\pgfpathlineto{\pgfqpoint{3.856100in}{1.781752in}}%
\pgfpathlineto{\pgfqpoint{3.858720in}{1.777193in}}%
\pgfpathlineto{\pgfqpoint{3.861561in}{1.781659in}}%
\pgfpathlineto{\pgfqpoint{3.864073in}{1.779758in}}%
\pgfpathlineto{\pgfqpoint{3.866815in}{1.779663in}}%
\pgfpathlineto{\pgfqpoint{3.869435in}{1.787451in}}%
\pgfpathlineto{\pgfqpoint{3.872114in}{1.780454in}}%
\pgfpathlineto{\pgfqpoint{3.874790in}{1.787009in}}%
\pgfpathlineto{\pgfqpoint{3.877466in}{1.794823in}}%
\pgfpathlineto{\pgfqpoint{3.880237in}{1.790147in}}%
\pgfpathlineto{\pgfqpoint{3.882850in}{1.791271in}}%
\pgfpathlineto{\pgfqpoint{3.885621in}{1.787307in}}%
\pgfpathlineto{\pgfqpoint{3.888188in}{1.786125in}}%
\pgfpathlineto{\pgfqpoint{3.890926in}{1.781969in}}%
\pgfpathlineto{\pgfqpoint{3.893541in}{1.780427in}}%
\pgfpathlineto{\pgfqpoint{3.896345in}{1.785510in}}%
\pgfpathlineto{\pgfqpoint{3.898891in}{1.784971in}}%
\pgfpathlineto{\pgfqpoint{3.901573in}{1.784307in}}%
\pgfpathlineto{\pgfqpoint{3.904252in}{1.792117in}}%
\pgfpathlineto{\pgfqpoint{3.906918in}{1.789228in}}%
\pgfpathlineto{\pgfqpoint{3.909602in}{1.793996in}}%
\pgfpathlineto{\pgfqpoint{3.912296in}{1.794069in}}%
\pgfpathlineto{\pgfqpoint{3.915107in}{1.790468in}}%
\pgfpathlineto{\pgfqpoint{3.917646in}{1.786673in}}%
\pgfpathlineto{\pgfqpoint{3.920412in}{1.783508in}}%
\pgfpathlineto{\pgfqpoint{3.923005in}{1.788084in}}%
\pgfpathlineto{\pgfqpoint{3.925778in}{1.789150in}}%
\pgfpathlineto{\pgfqpoint{3.928347in}{1.788706in}}%
\pgfpathlineto{\pgfqpoint{3.931202in}{1.792686in}}%
\pgfpathlineto{\pgfqpoint{3.933714in}{1.792450in}}%
\pgfpathlineto{\pgfqpoint{3.936395in}{1.793700in}}%
\pgfpathlineto{\pgfqpoint{3.939075in}{1.788708in}}%
\pgfpathlineto{\pgfqpoint{3.941778in}{1.781845in}}%
\pgfpathlineto{\pgfqpoint{3.944431in}{1.786139in}}%
\pgfpathlineto{\pgfqpoint{3.947101in}{1.782591in}}%
\pgfpathlineto{\pgfqpoint{3.949894in}{1.781490in}}%
\pgfpathlineto{\pgfqpoint{3.952464in}{1.788557in}}%
\pgfpathlineto{\pgfqpoint{3.955211in}{1.793413in}}%
\pgfpathlineto{\pgfqpoint{3.957823in}{1.783844in}}%
\pgfpathlineto{\pgfqpoint{3.960635in}{1.787057in}}%
\pgfpathlineto{\pgfqpoint{3.963176in}{1.790630in}}%
\pgfpathlineto{\pgfqpoint{3.966013in}{1.790479in}}%
\pgfpathlineto{\pgfqpoint{3.968523in}{1.787128in}}%
\pgfpathlineto{\pgfqpoint{3.971250in}{1.791105in}}%
\pgfpathlineto{\pgfqpoint{3.973885in}{1.788961in}}%
\pgfpathlineto{\pgfqpoint{3.976563in}{1.784482in}}%
\pgfpathlineto{\pgfqpoint{3.979389in}{1.783385in}}%
\pgfpathlineto{\pgfqpoint{3.981929in}{1.785703in}}%
\pgfpathlineto{\pgfqpoint{3.984714in}{1.790950in}}%
\pgfpathlineto{\pgfqpoint{3.987270in}{1.784397in}}%
\pgfpathlineto{\pgfqpoint{3.990055in}{1.787540in}}%
\pgfpathlineto{\pgfqpoint{3.992642in}{1.786643in}}%
\pgfpathlineto{\pgfqpoint{3.995417in}{1.785519in}}%
\pgfpathlineto{\pgfqpoint{3.997990in}{1.787177in}}%
\pgfpathlineto{\pgfqpoint{4.000674in}{1.786265in}}%
\pgfpathlineto{\pgfqpoint{4.003348in}{1.783335in}}%
\pgfpathlineto{\pgfqpoint{4.006034in}{1.785766in}}%
\pgfpathlineto{\pgfqpoint{4.008699in}{1.785089in}}%
\pgfpathlineto{\pgfqpoint{4.011394in}{1.793823in}}%
\pgfpathlineto{\pgfqpoint{4.014186in}{1.788303in}}%
\pgfpathlineto{\pgfqpoint{4.016744in}{1.788879in}}%
\pgfpathlineto{\pgfqpoint{4.019518in}{1.795386in}}%
\pgfpathlineto{\pgfqpoint{4.022097in}{1.805762in}}%
\pgfpathlineto{\pgfqpoint{4.024868in}{1.810578in}}%
\pgfpathlineto{\pgfqpoint{4.027447in}{1.818813in}}%
\pgfpathlineto{\pgfqpoint{4.030229in}{1.808854in}}%
\pgfpathlineto{\pgfqpoint{4.032817in}{1.807551in}}%
\pgfpathlineto{\pgfqpoint{4.035492in}{1.802959in}}%
\pgfpathlineto{\pgfqpoint{4.038174in}{1.804209in}}%
\pgfpathlineto{\pgfqpoint{4.040852in}{1.806602in}}%
\pgfpathlineto{\pgfqpoint{4.043667in}{1.796455in}}%
\pgfpathlineto{\pgfqpoint{4.046210in}{1.795863in}}%
\pgfpathlineto{\pgfqpoint{4.049006in}{1.800495in}}%
\pgfpathlineto{\pgfqpoint{4.051557in}{1.792171in}}%
\pgfpathlineto{\pgfqpoint{4.054326in}{1.792535in}}%
\pgfpathlineto{\pgfqpoint{4.056911in}{1.798112in}}%
\pgfpathlineto{\pgfqpoint{4.059702in}{1.800271in}}%
\pgfpathlineto{\pgfqpoint{4.062266in}{1.798452in}}%
\pgfpathlineto{\pgfqpoint{4.064957in}{1.797613in}}%
\pgfpathlineto{\pgfqpoint{4.067636in}{1.793727in}}%
\pgfpathlineto{\pgfqpoint{4.070313in}{1.795620in}}%
\pgfpathlineto{\pgfqpoint{4.072985in}{1.796888in}}%
\pgfpathlineto{\pgfqpoint{4.075705in}{1.798088in}}%
\pgfpathlineto{\pgfqpoint{4.078471in}{1.787341in}}%
\pgfpathlineto{\pgfqpoint{4.081018in}{1.791212in}}%
\pgfpathlineto{\pgfqpoint{4.083870in}{1.791862in}}%
\pgfpathlineto{\pgfqpoint{4.086385in}{1.791666in}}%
\pgfpathlineto{\pgfqpoint{4.089159in}{1.792335in}}%
\pgfpathlineto{\pgfqpoint{4.091729in}{1.791328in}}%
\pgfpathlineto{\pgfqpoint{4.094527in}{1.789988in}}%
\pgfpathlineto{\pgfqpoint{4.097092in}{1.787155in}}%
\pgfpathlineto{\pgfqpoint{4.099777in}{1.790891in}}%
\pgfpathlineto{\pgfqpoint{4.102456in}{1.794806in}}%
\pgfpathlineto{\pgfqpoint{4.105185in}{1.793489in}}%
\pgfpathlineto{\pgfqpoint{4.107814in}{1.787385in}}%
\pgfpathlineto{\pgfqpoint{4.110488in}{1.789665in}}%
\pgfpathlineto{\pgfqpoint{4.113252in}{1.788441in}}%
\pgfpathlineto{\pgfqpoint{4.115844in}{1.793211in}}%
\pgfpathlineto{\pgfqpoint{4.118554in}{1.800675in}}%
\pgfpathlineto{\pgfqpoint{4.121205in}{1.797943in}}%
\pgfpathlineto{\pgfqpoint{4.124019in}{1.800538in}}%
\pgfpathlineto{\pgfqpoint{4.126553in}{1.792753in}}%
\pgfpathlineto{\pgfqpoint{4.129349in}{1.795961in}}%
\pgfpathlineto{\pgfqpoint{4.131920in}{1.791184in}}%
\pgfpathlineto{\pgfqpoint{4.134615in}{1.798816in}}%
\pgfpathlineto{\pgfqpoint{4.137272in}{1.797557in}}%
\pgfpathlineto{\pgfqpoint{4.139963in}{1.799368in}}%
\pgfpathlineto{\pgfqpoint{4.142713in}{1.789297in}}%
\pgfpathlineto{\pgfqpoint{4.145310in}{1.790664in}}%
\pgfpathlineto{\pgfqpoint{4.148082in}{1.786227in}}%
\pgfpathlineto{\pgfqpoint{4.150665in}{1.797924in}}%
\pgfpathlineto{\pgfqpoint{4.153423in}{1.794677in}}%
\pgfpathlineto{\pgfqpoint{4.156016in}{1.790435in}}%
\pgfpathlineto{\pgfqpoint{4.158806in}{1.829238in}}%
\pgfpathlineto{\pgfqpoint{4.161380in}{1.853526in}}%
\pgfpathlineto{\pgfqpoint{4.164059in}{1.826942in}}%
\pgfpathlineto{\pgfqpoint{4.166737in}{1.816645in}}%
\pgfpathlineto{\pgfqpoint{4.169415in}{1.802061in}}%
\pgfpathlineto{\pgfqpoint{4.172093in}{1.797842in}}%
\pgfpathlineto{\pgfqpoint{4.174770in}{1.795771in}}%
\pgfpathlineto{\pgfqpoint{4.177593in}{1.792125in}}%
\pgfpathlineto{\pgfqpoint{4.180129in}{1.791210in}}%
\pgfpathlineto{\pgfqpoint{4.182899in}{1.788441in}}%
\pgfpathlineto{\pgfqpoint{4.185481in}{1.781598in}}%
\pgfpathlineto{\pgfqpoint{4.188318in}{1.779418in}}%
\pgfpathlineto{\pgfqpoint{4.190842in}{1.789010in}}%
\pgfpathlineto{\pgfqpoint{4.193638in}{1.788030in}}%
\pgfpathlineto{\pgfqpoint{4.196186in}{1.785073in}}%
\pgfpathlineto{\pgfqpoint{4.198878in}{1.785690in}}%
\pgfpathlineto{\pgfqpoint{4.201542in}{1.787296in}}%
\pgfpathlineto{\pgfqpoint{4.204240in}{1.784066in}}%
\pgfpathlineto{\pgfqpoint{4.207076in}{1.780848in}}%
\pgfpathlineto{\pgfqpoint{4.209597in}{1.780095in}}%
\pgfpathlineto{\pgfqpoint{4.212383in}{1.786286in}}%
\pgfpathlineto{\pgfqpoint{4.214948in}{1.790638in}}%
\pgfpathlineto{\pgfqpoint{4.217694in}{1.792148in}}%
\pgfpathlineto{\pgfqpoint{4.220304in}{1.793590in}}%
\pgfpathlineto{\pgfqpoint{4.223082in}{1.798858in}}%
\pgfpathlineto{\pgfqpoint{4.225654in}{1.790099in}}%
\pgfpathlineto{\pgfqpoint{4.228331in}{1.788913in}}%
\pgfpathlineto{\pgfqpoint{4.231013in}{1.782897in}}%
\pgfpathlineto{\pgfqpoint{4.233691in}{1.790185in}}%
\pgfpathlineto{\pgfqpoint{4.236375in}{1.789048in}}%
\pgfpathlineto{\pgfqpoint{4.239084in}{1.788095in}}%
\pgfpathlineto{\pgfqpoint{4.241900in}{1.785971in}}%
\pgfpathlineto{\pgfqpoint{4.244394in}{1.785499in}}%
\pgfpathlineto{\pgfqpoint{4.247225in}{1.789262in}}%
\pgfpathlineto{\pgfqpoint{4.249767in}{1.782745in}}%
\pgfpathlineto{\pgfqpoint{4.252581in}{1.785978in}}%
\pgfpathlineto{\pgfqpoint{4.255120in}{1.785162in}}%
\pgfpathlineto{\pgfqpoint{4.257958in}{1.787800in}}%
\pgfpathlineto{\pgfqpoint{4.260477in}{1.786587in}}%
\pgfpathlineto{\pgfqpoint{4.263157in}{1.784758in}}%
\pgfpathlineto{\pgfqpoint{4.265824in}{1.786442in}}%
\pgfpathlineto{\pgfqpoint{4.268590in}{1.792270in}}%
\pgfpathlineto{\pgfqpoint{4.271187in}{1.779569in}}%
\pgfpathlineto{\pgfqpoint{4.273874in}{1.780420in}}%
\pgfpathlineto{\pgfqpoint{4.276635in}{1.778560in}}%
\pgfpathlineto{\pgfqpoint{4.279212in}{1.779948in}}%
\pgfpathlineto{\pgfqpoint{4.282000in}{1.791614in}}%
\pgfpathlineto{\pgfqpoint{4.284586in}{1.774885in}}%
\pgfpathlineto{\pgfqpoint{4.287399in}{1.774885in}}%
\pgfpathlineto{\pgfqpoint{4.289936in}{1.776981in}}%
\pgfpathlineto{\pgfqpoint{4.292786in}{1.776431in}}%
\pgfpathlineto{\pgfqpoint{4.295299in}{1.786135in}}%
\pgfpathlineto{\pgfqpoint{4.297977in}{1.790958in}}%
\pgfpathlineto{\pgfqpoint{4.300656in}{1.791685in}}%
\pgfpathlineto{\pgfqpoint{4.303357in}{1.790230in}}%
\pgfpathlineto{\pgfqpoint{4.306118in}{1.790632in}}%
\pgfpathlineto{\pgfqpoint{4.308691in}{1.789079in}}%
\pgfpathlineto{\pgfqpoint{4.311494in}{1.790360in}}%
\pgfpathlineto{\pgfqpoint{4.314032in}{1.787852in}}%
\pgfpathlineto{\pgfqpoint{4.316856in}{1.791893in}}%
\pgfpathlineto{\pgfqpoint{4.319405in}{1.786446in}}%
\pgfpathlineto{\pgfqpoint{4.322181in}{1.789144in}}%
\pgfpathlineto{\pgfqpoint{4.324760in}{1.787426in}}%
\pgfpathlineto{\pgfqpoint{4.327440in}{1.792538in}}%
\pgfpathlineto{\pgfqpoint{4.330118in}{1.788750in}}%
\pgfpathlineto{\pgfqpoint{4.332796in}{1.795482in}}%
\pgfpathlineto{\pgfqpoint{4.335463in}{1.802805in}}%
\pgfpathlineto{\pgfqpoint{4.338154in}{1.795302in}}%
\pgfpathlineto{\pgfqpoint{4.340976in}{1.792527in}}%
\pgfpathlineto{\pgfqpoint{4.343510in}{1.787180in}}%
\pgfpathlineto{\pgfqpoint{4.346263in}{1.791674in}}%
\pgfpathlineto{\pgfqpoint{4.348868in}{1.782692in}}%
\pgfpathlineto{\pgfqpoint{4.351645in}{1.791856in}}%
\pgfpathlineto{\pgfqpoint{4.354224in}{1.789839in}}%
\pgfpathlineto{\pgfqpoint{4.357014in}{1.790219in}}%
\pgfpathlineto{\pgfqpoint{4.359582in}{1.787472in}}%
\pgfpathlineto{\pgfqpoint{4.362270in}{1.784419in}}%
\pgfpathlineto{\pgfqpoint{4.364936in}{1.782176in}}%
\pgfpathlineto{\pgfqpoint{4.367646in}{1.791388in}}%
\pgfpathlineto{\pgfqpoint{4.370437in}{1.794506in}}%
\pgfpathlineto{\pgfqpoint{4.372976in}{1.795448in}}%
\pgfpathlineto{\pgfqpoint{4.375761in}{1.789278in}}%
\pgfpathlineto{\pgfqpoint{4.378329in}{1.797397in}}%
\pgfpathlineto{\pgfqpoint{4.381097in}{1.790306in}}%
\pgfpathlineto{\pgfqpoint{4.383674in}{1.786349in}}%
\pgfpathlineto{\pgfqpoint{4.386431in}{1.787562in}}%
\pgfpathlineto{\pgfqpoint{4.389044in}{1.787740in}}%
\pgfpathlineto{\pgfqpoint{4.391721in}{1.786423in}}%
\pgfpathlineto{\pgfqpoint{4.394400in}{1.784907in}}%
\pgfpathlineto{\pgfqpoint{4.397076in}{1.791321in}}%
\pgfpathlineto{\pgfqpoint{4.399745in}{1.792114in}}%
\pgfpathlineto{\pgfqpoint{4.402468in}{1.798063in}}%
\pgfpathlineto{\pgfqpoint{4.405234in}{1.789672in}}%
\pgfpathlineto{\pgfqpoint{4.407788in}{1.791553in}}%
\pgfpathlineto{\pgfqpoint{4.410587in}{1.788033in}}%
\pgfpathlineto{\pgfqpoint{4.413149in}{1.789123in}}%
\pgfpathlineto{\pgfqpoint{4.415932in}{1.790946in}}%
\pgfpathlineto{\pgfqpoint{4.418506in}{1.794307in}}%
\pgfpathlineto{\pgfqpoint{4.421292in}{1.800578in}}%
\pgfpathlineto{\pgfqpoint{4.423863in}{1.799301in}}%
\pgfpathlineto{\pgfqpoint{4.426534in}{1.794747in}}%
\pgfpathlineto{\pgfqpoint{4.429220in}{1.795436in}}%
\pgfpathlineto{\pgfqpoint{4.431901in}{1.799189in}}%
\pgfpathlineto{\pgfqpoint{4.434569in}{1.793831in}}%
\pgfpathlineto{\pgfqpoint{4.437253in}{1.795289in}}%
\pgfpathlineto{\pgfqpoint{4.440041in}{1.795890in}}%
\pgfpathlineto{\pgfqpoint{4.442611in}{1.789200in}}%
\pgfpathlineto{\pgfqpoint{4.445423in}{1.794860in}}%
\pgfpathlineto{\pgfqpoint{4.447965in}{1.790531in}}%
\pgfpathlineto{\pgfqpoint{4.450767in}{1.793651in}}%
\pgfpathlineto{\pgfqpoint{4.453312in}{1.789536in}}%
\pgfpathlineto{\pgfqpoint{4.456138in}{1.793782in}}%
\pgfpathlineto{\pgfqpoint{4.458681in}{1.796515in}}%
\pgfpathlineto{\pgfqpoint{4.461367in}{1.796188in}}%
\pgfpathlineto{\pgfqpoint{4.464029in}{1.795724in}}%
\pgfpathlineto{\pgfqpoint{4.466717in}{1.792964in}}%
\pgfpathlineto{\pgfqpoint{4.469492in}{1.792292in}}%
\pgfpathlineto{\pgfqpoint{4.472059in}{1.788253in}}%
\pgfpathlineto{\pgfqpoint{4.474861in}{1.790430in}}%
\pgfpathlineto{\pgfqpoint{4.477430in}{1.793009in}}%
\pgfpathlineto{\pgfqpoint{4.480201in}{1.788655in}}%
\pgfpathlineto{\pgfqpoint{4.482778in}{1.795931in}}%
\pgfpathlineto{\pgfqpoint{4.485581in}{1.786268in}}%
\pgfpathlineto{\pgfqpoint{4.488130in}{1.790759in}}%
\pgfpathlineto{\pgfqpoint{4.490822in}{1.787908in}}%
\pgfpathlineto{\pgfqpoint{4.493492in}{1.785909in}}%
\pgfpathlineto{\pgfqpoint{4.496167in}{1.790295in}}%
\pgfpathlineto{\pgfqpoint{4.498850in}{1.786567in}}%
\pgfpathlineto{\pgfqpoint{4.501529in}{1.796007in}}%
\pgfpathlineto{\pgfqpoint{4.504305in}{1.788515in}}%
\pgfpathlineto{\pgfqpoint{4.506893in}{1.780806in}}%
\pgfpathlineto{\pgfqpoint{4.509643in}{1.786443in}}%
\pgfpathlineto{\pgfqpoint{4.512246in}{1.783496in}}%
\pgfpathlineto{\pgfqpoint{4.515080in}{1.788553in}}%
\pgfpathlineto{\pgfqpoint{4.517598in}{1.790195in}}%
\pgfpathlineto{\pgfqpoint{4.520345in}{1.795700in}}%
\pgfpathlineto{\pgfqpoint{4.522962in}{1.787353in}}%
\pgfpathlineto{\pgfqpoint{4.525640in}{1.794156in}}%
\pgfpathlineto{\pgfqpoint{4.528307in}{1.786074in}}%
\pgfpathlineto{\pgfqpoint{4.530990in}{1.792296in}}%
\pgfpathlineto{\pgfqpoint{4.533764in}{1.791379in}}%
\pgfpathlineto{\pgfqpoint{4.536400in}{1.797628in}}%
\pgfpathlineto{\pgfqpoint{4.539144in}{1.793619in}}%
\pgfpathlineto{\pgfqpoint{4.541711in}{1.797515in}}%
\pgfpathlineto{\pgfqpoint{4.544464in}{1.799400in}}%
\pgfpathlineto{\pgfqpoint{4.547064in}{1.797358in}}%
\pgfpathlineto{\pgfqpoint{4.549822in}{1.797982in}}%
\pgfpathlineto{\pgfqpoint{4.552425in}{1.797913in}}%
\pgfpathlineto{\pgfqpoint{4.555106in}{1.791354in}}%
\pgfpathlineto{\pgfqpoint{4.557777in}{1.795453in}}%
\pgfpathlineto{\pgfqpoint{4.560448in}{1.795623in}}%
\pgfpathlineto{\pgfqpoint{4.563125in}{1.794985in}}%
\pgfpathlineto{\pgfqpoint{4.565820in}{1.793928in}}%
\pgfpathlineto{\pgfqpoint{4.568612in}{1.790473in}}%
\pgfpathlineto{\pgfqpoint{4.571171in}{1.799082in}}%
\pgfpathlineto{\pgfqpoint{4.573947in}{1.795236in}}%
\pgfpathlineto{\pgfqpoint{4.576531in}{1.793464in}}%
\pgfpathlineto{\pgfqpoint{4.579305in}{1.792648in}}%
\pgfpathlineto{\pgfqpoint{4.581888in}{1.796640in}}%
\pgfpathlineto{\pgfqpoint{4.584672in}{1.794756in}}%
\pgfpathlineto{\pgfqpoint{4.587244in}{1.791515in}}%
\pgfpathlineto{\pgfqpoint{4.589920in}{1.791411in}}%
\pgfpathlineto{\pgfqpoint{4.592589in}{1.806041in}}%
\pgfpathlineto{\pgfqpoint{4.595281in}{1.794137in}}%
\pgfpathlineto{\pgfqpoint{4.597951in}{1.785473in}}%
\pgfpathlineto{\pgfqpoint{4.600633in}{1.780390in}}%
\pgfpathlineto{\pgfqpoint{4.603430in}{1.778542in}}%
\pgfpathlineto{\pgfqpoint{4.605990in}{1.785626in}}%
\pgfpathlineto{\pgfqpoint{4.608808in}{1.781279in}}%
\pgfpathlineto{\pgfqpoint{4.611350in}{1.789774in}}%
\pgfpathlineto{\pgfqpoint{4.614134in}{1.788659in}}%
\pgfpathlineto{\pgfqpoint{4.616702in}{1.784610in}}%
\pgfpathlineto{\pgfqpoint{4.619529in}{1.783721in}}%
\pgfpathlineto{\pgfqpoint{4.622056in}{1.792356in}}%
\pgfpathlineto{\pgfqpoint{4.624741in}{1.788249in}}%
\pgfpathlineto{\pgfqpoint{4.627411in}{1.785029in}}%
\pgfpathlineto{\pgfqpoint{4.630096in}{1.791716in}}%
\pgfpathlineto{\pgfqpoint{4.632902in}{1.793078in}}%
\pgfpathlineto{\pgfqpoint{4.635445in}{1.790843in}}%
\pgfpathlineto{\pgfqpoint{4.638204in}{1.789767in}}%
\pgfpathlineto{\pgfqpoint{4.640809in}{1.791600in}}%
\pgfpathlineto{\pgfqpoint{4.643628in}{1.789942in}}%
\pgfpathlineto{\pgfqpoint{4.646169in}{1.789097in}}%
\pgfpathlineto{\pgfqpoint{4.648922in}{1.792086in}}%
\pgfpathlineto{\pgfqpoint{4.651524in}{1.786985in}}%
\pgfpathlineto{\pgfqpoint{4.654203in}{1.791142in}}%
\pgfpathlineto{\pgfqpoint{4.656873in}{1.788115in}}%
\pgfpathlineto{\pgfqpoint{4.659590in}{1.785514in}}%
\pgfpathlineto{\pgfqpoint{4.662237in}{1.797224in}}%
\pgfpathlineto{\pgfqpoint{4.664923in}{1.798389in}}%
\pgfpathlineto{\pgfqpoint{4.667764in}{1.791666in}}%
\pgfpathlineto{\pgfqpoint{4.670261in}{1.794137in}}%
\pgfpathlineto{\pgfqpoint{4.673068in}{1.791608in}}%
\pgfpathlineto{\pgfqpoint{4.675619in}{1.804458in}}%
\pgfpathlineto{\pgfqpoint{4.678448in}{1.796890in}}%
\pgfpathlineto{\pgfqpoint{4.680988in}{1.795939in}}%
\pgfpathlineto{\pgfqpoint{4.683799in}{1.797458in}}%
\pgfpathlineto{\pgfqpoint{4.686337in}{1.803064in}}%
\pgfpathlineto{\pgfqpoint{4.689051in}{1.800611in}}%
\pgfpathlineto{\pgfqpoint{4.691694in}{1.796075in}}%
\pgfpathlineto{\pgfqpoint{4.694381in}{1.795914in}}%
\pgfpathlineto{\pgfqpoint{4.697170in}{1.796265in}}%
\pgfpathlineto{\pgfqpoint{4.699734in}{1.793977in}}%
\pgfpathlineto{\pgfqpoint{4.702517in}{1.795371in}}%
\pgfpathlineto{\pgfqpoint{4.705094in}{1.796757in}}%
\pgfpathlineto{\pgfqpoint{4.707824in}{1.792403in}}%
\pgfpathlineto{\pgfqpoint{4.710437in}{1.793128in}}%
\pgfpathlineto{\pgfqpoint{4.713275in}{1.795494in}}%
\pgfpathlineto{\pgfqpoint{4.715806in}{1.790206in}}%
\pgfpathlineto{\pgfqpoint{4.718486in}{1.791314in}}%
\pgfpathlineto{\pgfqpoint{4.721160in}{1.795427in}}%
\pgfpathlineto{\pgfqpoint{4.723873in}{1.789379in}}%
\pgfpathlineto{\pgfqpoint{4.726508in}{1.795854in}}%
\pgfpathlineto{\pgfqpoint{4.729233in}{1.794474in}}%
\pgfpathlineto{\pgfqpoint{4.731901in}{1.792373in}}%
\pgfpathlineto{\pgfqpoint{4.734552in}{1.788341in}}%
\pgfpathlineto{\pgfqpoint{4.737348in}{1.791313in}}%
\pgfpathlineto{\pgfqpoint{4.739912in}{1.789168in}}%
\pgfpathlineto{\pgfqpoint{4.742696in}{1.793759in}}%
\pgfpathlineto{\pgfqpoint{4.745256in}{1.797255in}}%
\pgfpathlineto{\pgfqpoint{4.748081in}{1.794923in}}%
\pgfpathlineto{\pgfqpoint{4.750627in}{1.790245in}}%
\pgfpathlineto{\pgfqpoint{4.753298in}{1.796309in}}%
\pgfpathlineto{\pgfqpoint{4.755983in}{1.808834in}}%
\pgfpathlineto{\pgfqpoint{4.758653in}{1.806027in}}%
\pgfpathlineto{\pgfqpoint{4.761337in}{1.806954in}}%
\pgfpathlineto{\pgfqpoint{4.764018in}{1.819680in}}%
\pgfpathlineto{\pgfqpoint{4.766783in}{1.809413in}}%
\pgfpathlineto{\pgfqpoint{4.769367in}{1.803454in}}%
\pgfpathlineto{\pgfqpoint{4.772198in}{1.801068in}}%
\pgfpathlineto{\pgfqpoint{4.774732in}{1.801133in}}%
\pgfpathlineto{\pgfqpoint{4.777535in}{1.800538in}}%
\pgfpathlineto{\pgfqpoint{4.780083in}{1.800551in}}%
\pgfpathlineto{\pgfqpoint{4.782872in}{1.799385in}}%
\pgfpathlineto{\pgfqpoint{4.785445in}{1.798513in}}%
\pgfpathlineto{\pgfqpoint{4.788116in}{1.797082in}}%
\pgfpathlineto{\pgfqpoint{4.790798in}{1.797396in}}%
\pgfpathlineto{\pgfqpoint{4.793512in}{1.819103in}}%
\pgfpathlineto{\pgfqpoint{4.796274in}{1.817125in}}%
\pgfpathlineto{\pgfqpoint{4.798830in}{1.803054in}}%
\pgfpathlineto{\pgfqpoint{4.801586in}{1.800248in}}%
\pgfpathlineto{\pgfqpoint{4.804193in}{1.791404in}}%
\pgfpathlineto{\pgfqpoint{4.807017in}{1.783699in}}%
\pgfpathlineto{\pgfqpoint{4.809538in}{1.781395in}}%
\pgfpathlineto{\pgfqpoint{4.812377in}{1.777058in}}%
\pgfpathlineto{\pgfqpoint{4.814907in}{1.781625in}}%
\pgfpathlineto{\pgfqpoint{4.817587in}{1.780902in}}%
\pgfpathlineto{\pgfqpoint{4.820265in}{1.782994in}}%
\pgfpathlineto{\pgfqpoint{4.822945in}{1.783205in}}%
\pgfpathlineto{\pgfqpoint{4.825619in}{1.792535in}}%
\pgfpathlineto{\pgfqpoint{4.828291in}{1.791830in}}%
\pgfpathlineto{\pgfqpoint{4.831045in}{1.786684in}}%
\pgfpathlineto{\pgfqpoint{4.833657in}{1.778083in}}%
\pgfpathlineto{\pgfqpoint{4.837992in}{1.778618in}}%
\pgfpathlineto{\pgfqpoint{4.839922in}{1.780394in}}%
\pgfpathlineto{\pgfqpoint{4.842380in}{1.789030in}}%
\pgfpathlineto{\pgfqpoint{4.844361in}{1.786677in}}%
\pgfpathlineto{\pgfqpoint{4.847127in}{1.781764in}}%
\pgfpathlineto{\pgfqpoint{4.849715in}{1.786917in}}%
\pgfpathlineto{\pgfqpoint{4.852404in}{1.779937in}}%
\pgfpathlineto{\pgfqpoint{4.855070in}{1.781324in}}%
\pgfpathlineto{\pgfqpoint{4.857807in}{1.785769in}}%
\pgfpathlineto{\pgfqpoint{4.860544in}{1.788011in}}%
\pgfpathlineto{\pgfqpoint{4.863116in}{1.789002in}}%
\pgfpathlineto{\pgfqpoint{4.865910in}{1.794567in}}%
\pgfpathlineto{\pgfqpoint{4.868474in}{1.800036in}}%
\pgfpathlineto{\pgfqpoint{4.871209in}{1.796522in}}%
\pgfpathlineto{\pgfqpoint{4.873832in}{1.794695in}}%
\pgfpathlineto{\pgfqpoint{4.876636in}{1.799277in}}%
\pgfpathlineto{\pgfqpoint{4.879180in}{1.790456in}}%
\pgfpathlineto{\pgfqpoint{4.881864in}{1.783366in}}%
\pgfpathlineto{\pgfqpoint{4.884540in}{1.783947in}}%
\pgfpathlineto{\pgfqpoint{4.887211in}{1.785547in}}%
\pgfpathlineto{\pgfqpoint{4.889902in}{1.786489in}}%
\pgfpathlineto{\pgfqpoint{4.892611in}{1.795200in}}%
\pgfpathlineto{\pgfqpoint{4.895399in}{1.794072in}}%
\pgfpathlineto{\pgfqpoint{4.897938in}{1.794547in}}%
\pgfpathlineto{\pgfqpoint{4.900712in}{1.786470in}}%
\pgfpathlineto{\pgfqpoint{4.903295in}{1.786779in}}%
\pgfpathlineto{\pgfqpoint{4.906096in}{1.782612in}}%
\pgfpathlineto{\pgfqpoint{4.908648in}{1.787095in}}%
\pgfpathlineto{\pgfqpoint{4.911435in}{1.789534in}}%
\pgfpathlineto{\pgfqpoint{4.914009in}{1.788014in}}%
\pgfpathlineto{\pgfqpoint{4.916681in}{1.788800in}}%
\pgfpathlineto{\pgfqpoint{4.919352in}{1.787994in}}%
\pgfpathlineto{\pgfqpoint{4.922041in}{1.792942in}}%
\pgfpathlineto{\pgfqpoint{4.924708in}{1.781325in}}%
\pgfpathlineto{\pgfqpoint{4.927400in}{1.785552in}}%
\pgfpathlineto{\pgfqpoint{4.930170in}{1.796206in}}%
\pgfpathlineto{\pgfqpoint{4.932742in}{1.796823in}}%
\pgfpathlineto{\pgfqpoint{4.935515in}{1.792777in}}%
\pgfpathlineto{\pgfqpoint{4.938112in}{1.791671in}}%
\pgfpathlineto{\pgfqpoint{4.940881in}{1.791558in}}%
\pgfpathlineto{\pgfqpoint{4.943466in}{1.794277in}}%
\pgfpathlineto{\pgfqpoint{4.946151in}{1.797399in}}%
\pgfpathlineto{\pgfqpoint{4.948827in}{1.794377in}}%
\pgfpathlineto{\pgfqpoint{4.951504in}{1.788037in}}%
\pgfpathlineto{\pgfqpoint{4.954182in}{1.799227in}}%
\pgfpathlineto{\pgfqpoint{4.956862in}{1.812828in}}%
\pgfpathlineto{\pgfqpoint{4.959689in}{1.809379in}}%
\pgfpathlineto{\pgfqpoint{4.962219in}{1.797841in}}%
\pgfpathlineto{\pgfqpoint{4.965002in}{1.797181in}}%
\pgfpathlineto{\pgfqpoint{4.967575in}{1.794065in}}%
\pgfpathlineto{\pgfqpoint{4.970314in}{1.794111in}}%
\pgfpathlineto{\pgfqpoint{4.972933in}{1.790491in}}%
\pgfpathlineto{\pgfqpoint{4.975703in}{1.794012in}}%
\pgfpathlineto{\pgfqpoint{4.978287in}{1.790232in}}%
\pgfpathlineto{\pgfqpoint{4.980967in}{1.797191in}}%
\pgfpathlineto{\pgfqpoint{4.983637in}{1.798280in}}%
\pgfpathlineto{\pgfqpoint{4.986325in}{1.792783in}}%
\pgfpathlineto{\pgfqpoint{4.989001in}{1.796517in}}%
\pgfpathlineto{\pgfqpoint{4.991683in}{1.794128in}}%
\pgfpathlineto{\pgfqpoint{4.994390in}{1.789450in}}%
\pgfpathlineto{\pgfqpoint{4.997028in}{1.788661in}}%
\pgfpathlineto{\pgfqpoint{4.999780in}{1.792785in}}%
\pgfpathlineto{\pgfqpoint{5.002384in}{1.789086in}}%
\pgfpathlineto{\pgfqpoint{5.005178in}{1.793746in}}%
\pgfpathlineto{\pgfqpoint{5.007751in}{1.795910in}}%
\pgfpathlineto{\pgfqpoint{5.010562in}{1.793874in}}%
\pgfpathlineto{\pgfqpoint{5.013104in}{1.792028in}}%
\pgfpathlineto{\pgfqpoint{5.015820in}{1.788967in}}%
\pgfpathlineto{\pgfqpoint{5.018466in}{1.793793in}}%
\pgfpathlineto{\pgfqpoint{5.021147in}{1.795848in}}%
\pgfpathlineto{\pgfqpoint{5.023927in}{1.792164in}}%
\pgfpathlineto{\pgfqpoint{5.026501in}{1.789180in}}%
\pgfpathlineto{\pgfqpoint{5.029275in}{1.790171in}}%
\pgfpathlineto{\pgfqpoint{5.031849in}{1.793659in}}%
\pgfpathlineto{\pgfqpoint{5.034649in}{1.788729in}}%
\pgfpathlineto{\pgfqpoint{5.037214in}{1.787389in}}%
\pgfpathlineto{\pgfqpoint{5.039962in}{1.787676in}}%
\pgfpathlineto{\pgfqpoint{5.042572in}{1.795062in}}%
\pgfpathlineto{\pgfqpoint{5.045249in}{1.791829in}}%
\pgfpathlineto{\pgfqpoint{5.047924in}{1.794964in}}%
\pgfpathlineto{\pgfqpoint{5.050606in}{1.792254in}}%
\pgfpathlineto{\pgfqpoint{5.053284in}{1.789602in}}%
\pgfpathlineto{\pgfqpoint{5.055952in}{1.788626in}}%
\pgfpathlineto{\pgfqpoint{5.058711in}{1.789454in}}%
\pgfpathlineto{\pgfqpoint{5.061315in}{1.787418in}}%
\pgfpathlineto{\pgfqpoint{5.064144in}{1.786755in}}%
\pgfpathlineto{\pgfqpoint{5.066677in}{1.792357in}}%
\pgfpathlineto{\pgfqpoint{5.069463in}{1.788859in}}%
\pgfpathlineto{\pgfqpoint{5.072030in}{1.789630in}}%
\pgfpathlineto{\pgfqpoint{5.074851in}{1.790714in}}%
\pgfpathlineto{\pgfqpoint{5.077390in}{1.784955in}}%
\pgfpathlineto{\pgfqpoint{5.080067in}{1.783273in}}%
\pgfpathlineto{\pgfqpoint{5.082746in}{1.792713in}}%
\pgfpathlineto{\pgfqpoint{5.085426in}{1.792640in}}%
\pgfpathlineto{\pgfqpoint{5.088103in}{1.793719in}}%
\pgfpathlineto{\pgfqpoint{5.090788in}{1.791765in}}%
\pgfpathlineto{\pgfqpoint{5.093579in}{1.793095in}}%
\pgfpathlineto{\pgfqpoint{5.096142in}{1.796975in}}%
\pgfpathlineto{\pgfqpoint{5.098948in}{1.794700in}}%
\pgfpathlineto{\pgfqpoint{5.101496in}{1.795508in}}%
\pgfpathlineto{\pgfqpoint{5.104312in}{1.792349in}}%
\pgfpathlineto{\pgfqpoint{5.106842in}{1.795178in}}%
\pgfpathlineto{\pgfqpoint{5.109530in}{1.791895in}}%
\pgfpathlineto{\pgfqpoint{5.112209in}{1.796806in}}%
\pgfpathlineto{\pgfqpoint{5.114887in}{1.799187in}}%
\pgfpathlineto{\pgfqpoint{5.117550in}{1.789774in}}%
\pgfpathlineto{\pgfqpoint{5.120243in}{1.795958in}}%
\pgfpathlineto{\pgfqpoint{5.123042in}{1.792081in}}%
\pgfpathlineto{\pgfqpoint{5.125599in}{1.798324in}}%
\pgfpathlineto{\pgfqpoint{5.128421in}{1.795008in}}%
\pgfpathlineto{\pgfqpoint{5.130953in}{1.792498in}}%
\pgfpathlineto{\pgfqpoint{5.133716in}{1.795719in}}%
\pgfpathlineto{\pgfqpoint{5.136311in}{1.793682in}}%
\pgfpathlineto{\pgfqpoint{5.139072in}{1.787832in}}%
\pgfpathlineto{\pgfqpoint{5.141660in}{1.794608in}}%
\pgfpathlineto{\pgfqpoint{5.144349in}{1.792535in}}%
\pgfpathlineto{\pgfqpoint{5.147029in}{1.796438in}}%
\pgfpathlineto{\pgfqpoint{5.149734in}{1.816881in}}%
\pgfpathlineto{\pgfqpoint{5.152382in}{1.847258in}}%
\pgfpathlineto{\pgfqpoint{5.155059in}{1.854932in}}%
\pgfpathlineto{\pgfqpoint{5.157815in}{1.850216in}}%
\pgfpathlineto{\pgfqpoint{5.160420in}{1.842768in}}%
\pgfpathlineto{\pgfqpoint{5.163243in}{1.841732in}}%
\pgfpathlineto{\pgfqpoint{5.165775in}{1.835642in}}%
\pgfpathlineto{\pgfqpoint{5.168591in}{1.840939in}}%
\pgfpathlineto{\pgfqpoint{5.171133in}{1.836218in}}%
\pgfpathlineto{\pgfqpoint{5.173925in}{1.840218in}}%
\pgfpathlineto{\pgfqpoint{5.176477in}{1.840660in}}%
\pgfpathlineto{\pgfqpoint{5.179188in}{1.835732in}}%
\pgfpathlineto{\pgfqpoint{5.181848in}{1.817501in}}%
\pgfpathlineto{\pgfqpoint{5.184522in}{1.827846in}}%
\pgfpathlineto{\pgfqpoint{5.187294in}{1.823591in}}%
\pgfpathlineto{\pgfqpoint{5.189880in}{1.826396in}}%
\pgfpathlineto{\pgfqpoint{5.192680in}{1.838626in}}%
\pgfpathlineto{\pgfqpoint{5.195239in}{1.831937in}}%
\pgfpathlineto{\pgfqpoint{5.198008in}{1.833394in}}%
\pgfpathlineto{\pgfqpoint{5.200594in}{1.820372in}}%
\pgfpathlineto{\pgfqpoint{5.203388in}{1.812220in}}%
\pgfpathlineto{\pgfqpoint{5.205952in}{1.804887in}}%
\pgfpathlineto{\pgfqpoint{5.208630in}{1.802966in}}%
\pgfpathlineto{\pgfqpoint{5.211299in}{1.802472in}}%
\pgfpathlineto{\pgfqpoint{5.214027in}{1.804422in}}%
\pgfpathlineto{\pgfqpoint{5.216667in}{1.803364in}}%
\pgfpathlineto{\pgfqpoint{5.219345in}{1.794534in}}%
\pgfpathlineto{\pgfqpoint{5.222151in}{1.796044in}}%
\pgfpathlineto{\pgfqpoint{5.224695in}{1.792288in}}%
\pgfpathlineto{\pgfqpoint{5.227470in}{1.793174in}}%
\pgfpathlineto{\pgfqpoint{5.230059in}{1.804821in}}%
\pgfpathlineto{\pgfqpoint{5.232855in}{1.795444in}}%
\pgfpathlineto{\pgfqpoint{5.235409in}{1.803859in}}%
\pgfpathlineto{\pgfqpoint{5.238173in}{1.802687in}}%
\pgfpathlineto{\pgfqpoint{5.240777in}{1.800528in}}%
\pgfpathlineto{\pgfqpoint{5.243445in}{1.798038in}}%
\pgfpathlineto{\pgfqpoint{5.246130in}{1.801006in}}%
\pgfpathlineto{\pgfqpoint{5.248816in}{1.798888in}}%
\pgfpathlineto{\pgfqpoint{5.251590in}{1.802298in}}%
\pgfpathlineto{\pgfqpoint{5.254236in}{1.796120in}}%
\pgfpathlineto{\pgfqpoint{5.256973in}{1.785976in}}%
\pgfpathlineto{\pgfqpoint{5.259511in}{1.783199in}}%
\pgfpathlineto{\pgfqpoint{5.262264in}{1.781861in}}%
\pgfpathlineto{\pgfqpoint{5.264876in}{1.788788in}}%
\pgfpathlineto{\pgfqpoint{5.267691in}{1.788364in}}%
\pgfpathlineto{\pgfqpoint{5.270238in}{1.792402in}}%
\pgfpathlineto{\pgfqpoint{5.272913in}{1.790576in}}%
\pgfpathlineto{\pgfqpoint{5.275589in}{1.789642in}}%
\pgfpathlineto{\pgfqpoint{5.278322in}{1.799693in}}%
\pgfpathlineto{\pgfqpoint{5.280947in}{1.800184in}}%
\pgfpathlineto{\pgfqpoint{5.283631in}{1.799365in}}%
\pgfpathlineto{\pgfqpoint{5.286436in}{1.803731in}}%
\pgfpathlineto{\pgfqpoint{5.288984in}{1.801788in}}%
\pgfpathlineto{\pgfqpoint{5.291794in}{1.800139in}}%
\pgfpathlineto{\pgfqpoint{5.294339in}{1.803895in}}%
\pgfpathlineto{\pgfqpoint{5.297140in}{1.799935in}}%
\pgfpathlineto{\pgfqpoint{5.299696in}{1.799456in}}%
\pgfpathlineto{\pgfqpoint{5.302443in}{1.799763in}}%
\pgfpathlineto{\pgfqpoint{5.305054in}{1.799090in}}%
\pgfpathlineto{\pgfqpoint{5.307731in}{1.802171in}}%
\pgfpathlineto{\pgfqpoint{5.310411in}{1.801181in}}%
\pgfpathlineto{\pgfqpoint{5.313089in}{1.795474in}}%
\pgfpathlineto{\pgfqpoint{5.315754in}{1.802126in}}%
\pgfpathlineto{\pgfqpoint{5.318430in}{1.795312in}}%
\pgfpathlineto{\pgfqpoint{5.321256in}{1.797184in}}%
\pgfpathlineto{\pgfqpoint{5.323802in}{1.795071in}}%
\pgfpathlineto{\pgfqpoint{5.326564in}{1.793517in}}%
\pgfpathlineto{\pgfqpoint{5.329159in}{1.790623in}}%
\pgfpathlineto{\pgfqpoint{5.331973in}{1.787307in}}%
\pgfpathlineto{\pgfqpoint{5.334510in}{1.792706in}}%
\pgfpathlineto{\pgfqpoint{5.337353in}{1.794350in}}%
\pgfpathlineto{\pgfqpoint{5.339872in}{1.793230in}}%
\pgfpathlineto{\pgfqpoint{5.342549in}{1.789403in}}%
\pgfpathlineto{\pgfqpoint{5.345224in}{1.782283in}}%
\pgfpathlineto{\pgfqpoint{5.347905in}{1.788465in}}%
\pgfpathlineto{\pgfqpoint{5.350723in}{1.787170in}}%
\pgfpathlineto{\pgfqpoint{5.353262in}{1.791822in}}%
\pgfpathlineto{\pgfqpoint{5.356056in}{1.792022in}}%
\pgfpathlineto{\pgfqpoint{5.358612in}{1.796760in}}%
\pgfpathlineto{\pgfqpoint{5.361370in}{1.789310in}}%
\pgfpathlineto{\pgfqpoint{5.363966in}{1.787428in}}%
\pgfpathlineto{\pgfqpoint{5.366727in}{1.791652in}}%
\pgfpathlineto{\pgfqpoint{5.369335in}{1.793941in}}%
\pgfpathlineto{\pgfqpoint{5.372013in}{1.796323in}}%
\pgfpathlineto{\pgfqpoint{5.374692in}{1.794906in}}%
\pgfpathlineto{\pgfqpoint{5.377370in}{1.788183in}}%
\pgfpathlineto{\pgfqpoint{5.380048in}{1.793781in}}%
\pgfpathlineto{\pgfqpoint{5.382725in}{1.797461in}}%
\pgfpathlineto{\pgfqpoint{5.385550in}{1.795850in}}%
\pgfpathlineto{\pgfqpoint{5.388083in}{1.788683in}}%
\pgfpathlineto{\pgfqpoint{5.390900in}{1.793743in}}%
\pgfpathlineto{\pgfqpoint{5.393441in}{1.781902in}}%
\pgfpathlineto{\pgfqpoint{5.396219in}{1.794839in}}%
\pgfpathlineto{\pgfqpoint{5.398784in}{1.812034in}}%
\pgfpathlineto{\pgfqpoint{5.401576in}{1.805242in}}%
\pgfpathlineto{\pgfqpoint{5.404154in}{1.794872in}}%
\pgfpathlineto{\pgfqpoint{5.406832in}{1.793760in}}%
\pgfpathlineto{\pgfqpoint{5.409507in}{1.792786in}}%
\pgfpathlineto{\pgfqpoint{5.412190in}{1.790615in}}%
\pgfpathlineto{\pgfqpoint{5.414954in}{1.793732in}}%
\pgfpathlineto{\pgfqpoint{5.417547in}{1.798621in}}%
\pgfpathlineto{\pgfqpoint{5.420304in}{1.797814in}}%
\pgfpathlineto{\pgfqpoint{5.422897in}{1.799220in}}%
\pgfpathlineto{\pgfqpoint{5.425661in}{1.794948in}}%
\pgfpathlineto{\pgfqpoint{5.428259in}{1.797253in}}%
\pgfpathlineto{\pgfqpoint{5.431015in}{1.796971in}}%
\pgfpathlineto{\pgfqpoint{5.433616in}{1.795867in}}%
\pgfpathlineto{\pgfqpoint{5.436295in}{1.799451in}}%
\pgfpathlineto{\pgfqpoint{5.438974in}{1.793939in}}%
\pgfpathlineto{\pgfqpoint{5.441698in}{1.794001in}}%
\pgfpathlineto{\pgfqpoint{5.444328in}{1.794347in}}%
\pgfpathlineto{\pgfqpoint{5.447021in}{1.795299in}}%
\pgfpathlineto{\pgfqpoint{5.449769in}{1.812846in}}%
\pgfpathlineto{\pgfqpoint{5.452365in}{1.795923in}}%
\pgfpathlineto{\pgfqpoint{5.455168in}{1.802258in}}%
\pgfpathlineto{\pgfqpoint{5.457721in}{1.800608in}}%
\pgfpathlineto{\pgfqpoint{5.460489in}{1.796841in}}%
\pgfpathlineto{\pgfqpoint{5.463079in}{1.798589in}}%
\pgfpathlineto{\pgfqpoint{5.465888in}{1.795230in}}%
\pgfpathlineto{\pgfqpoint{5.468425in}{1.798343in}}%
\pgfpathlineto{\pgfqpoint{5.471113in}{1.794503in}}%
\pgfpathlineto{\pgfqpoint{5.473792in}{1.793247in}}%
\pgfpathlineto{\pgfqpoint{5.476458in}{1.791578in}}%
\pgfpathlineto{\pgfqpoint{5.479152in}{1.796855in}}%
\pgfpathlineto{\pgfqpoint{5.481825in}{1.794091in}}%
\pgfpathlineto{\pgfqpoint{5.484641in}{1.795844in}}%
\pgfpathlineto{\pgfqpoint{5.487176in}{1.795974in}}%
\pgfpathlineto{\pgfqpoint{5.490000in}{1.798673in}}%
\pgfpathlineto{\pgfqpoint{5.492541in}{1.794231in}}%
\pgfpathlineto{\pgfqpoint{5.495346in}{1.788006in}}%
\pgfpathlineto{\pgfqpoint{5.497898in}{1.793355in}}%
\pgfpathlineto{\pgfqpoint{5.500687in}{1.786303in}}%
\pgfpathlineto{\pgfqpoint{5.503255in}{1.790013in}}%
\pgfpathlineto{\pgfqpoint{5.505933in}{1.792589in}}%
\pgfpathlineto{\pgfqpoint{5.508612in}{1.794303in}}%
\pgfpathlineto{\pgfqpoint{5.511290in}{1.789945in}}%
\pgfpathlineto{\pgfqpoint{5.514080in}{1.794929in}}%
\pgfpathlineto{\pgfqpoint{5.516646in}{1.796424in}}%
\pgfpathlineto{\pgfqpoint{5.519433in}{1.794817in}}%
\pgfpathlineto{\pgfqpoint{5.522003in}{1.791393in}}%
\pgfpathlineto{\pgfqpoint{5.524756in}{1.790744in}}%
\pgfpathlineto{\pgfqpoint{5.527360in}{1.787868in}}%
\pgfpathlineto{\pgfqpoint{5.530148in}{1.791724in}}%
\pgfpathlineto{\pgfqpoint{5.532717in}{1.788041in}}%
\pgfpathlineto{\pgfqpoint{5.535395in}{1.792925in}}%
\pgfpathlineto{\pgfqpoint{5.538074in}{1.787071in}}%
\pgfpathlineto{\pgfqpoint{5.540750in}{1.792519in}}%
\pgfpathlineto{\pgfqpoint{5.543421in}{1.793290in}}%
\pgfpathlineto{\pgfqpoint{5.546110in}{1.796159in}}%
\pgfpathlineto{\pgfqpoint{5.548921in}{1.792557in}}%
\pgfpathlineto{\pgfqpoint{5.551457in}{1.797028in}}%
\pgfpathlineto{\pgfqpoint{5.554198in}{1.791684in}}%
\pgfpathlineto{\pgfqpoint{5.556822in}{1.790779in}}%
\pgfpathlineto{\pgfqpoint{5.559612in}{1.785344in}}%
\pgfpathlineto{\pgfqpoint{5.562180in}{1.790585in}}%
\pgfpathlineto{\pgfqpoint{5.564940in}{1.793551in}}%
\pgfpathlineto{\pgfqpoint{5.567536in}{1.787058in}}%
\pgfpathlineto{\pgfqpoint{5.570215in}{1.785856in}}%
\pgfpathlineto{\pgfqpoint{5.572893in}{1.797786in}}%
\pgfpathlineto{\pgfqpoint{5.575596in}{1.793351in}}%
\pgfpathlineto{\pgfqpoint{5.578342in}{1.795642in}}%
\pgfpathlineto{\pgfqpoint{5.580914in}{1.796644in}}%
\pgfpathlineto{\pgfqpoint{5.583709in}{1.784756in}}%
\pgfpathlineto{\pgfqpoint{5.586269in}{1.788691in}}%
\pgfpathlineto{\pgfqpoint{5.589040in}{1.794048in}}%
\pgfpathlineto{\pgfqpoint{5.591641in}{1.789417in}}%
\pgfpathlineto{\pgfqpoint{5.594368in}{1.792706in}}%
\pgfpathlineto{\pgfqpoint{5.596999in}{1.788116in}}%
\pgfpathlineto{\pgfqpoint{5.599674in}{1.792931in}}%
\pgfpathlineto{\pgfqpoint{5.602352in}{1.797853in}}%
\pgfpathlineto{\pgfqpoint{5.605073in}{1.799594in}}%
\pgfpathlineto{\pgfqpoint{5.607698in}{1.799339in}}%
\pgfpathlineto{\pgfqpoint{5.610389in}{1.801040in}}%
\pgfpathlineto{\pgfqpoint{5.613235in}{1.801176in}}%
\pgfpathlineto{\pgfqpoint{5.615743in}{1.802390in}}%
\pgfpathlineto{\pgfqpoint{5.618526in}{1.796294in}}%
\pgfpathlineto{\pgfqpoint{5.621102in}{1.794429in}}%
\pgfpathlineto{\pgfqpoint{5.623868in}{1.797936in}}%
\pgfpathlineto{\pgfqpoint{5.626460in}{1.800593in}}%
\pgfpathlineto{\pgfqpoint{5.629232in}{1.802899in}}%
\pgfpathlineto{\pgfqpoint{5.631815in}{1.802222in}}%
\pgfpathlineto{\pgfqpoint{5.634496in}{1.795872in}}%
\pgfpathlineto{\pgfqpoint{5.637172in}{1.797417in}}%
\pgfpathlineto{\pgfqpoint{5.639852in}{1.799782in}}%
\pgfpathlineto{\pgfqpoint{5.642518in}{1.805170in}}%
\pgfpathlineto{\pgfqpoint{5.645243in}{1.793965in}}%
\pgfpathlineto{\pgfqpoint{5.648008in}{1.789704in}}%
\pgfpathlineto{\pgfqpoint{5.650563in}{1.792271in}}%
\pgfpathlineto{\pgfqpoint{5.653376in}{1.781194in}}%
\pgfpathlineto{\pgfqpoint{5.655919in}{1.779059in}}%
\pgfpathlineto{\pgfqpoint{5.658723in}{1.791880in}}%
\pgfpathlineto{\pgfqpoint{5.661273in}{1.789941in}}%
\pgfpathlineto{\pgfqpoint{5.664099in}{1.798451in}}%
\pgfpathlineto{\pgfqpoint{5.666632in}{1.803160in}}%
\pgfpathlineto{\pgfqpoint{5.669313in}{1.804573in}}%
\pgfpathlineto{\pgfqpoint{5.671991in}{1.801757in}}%
\pgfpathlineto{\pgfqpoint{5.674667in}{1.799801in}}%
\pgfpathlineto{\pgfqpoint{5.677486in}{1.799888in}}%
\pgfpathlineto{\pgfqpoint{5.680027in}{1.800407in}}%
\pgfpathlineto{\pgfqpoint{5.682836in}{1.792778in}}%
\pgfpathlineto{\pgfqpoint{5.685385in}{1.792856in}}%
\pgfpathlineto{\pgfqpoint{5.688159in}{1.794717in}}%
\pgfpathlineto{\pgfqpoint{5.690730in}{1.787783in}}%
\pgfpathlineto{\pgfqpoint{5.693473in}{1.798730in}}%
\pgfpathlineto{\pgfqpoint{5.696101in}{1.798546in}}%
\pgfpathlineto{\pgfqpoint{5.698775in}{1.793069in}}%
\pgfpathlineto{\pgfqpoint{5.701453in}{1.789020in}}%
\pgfpathlineto{\pgfqpoint{5.704130in}{1.794409in}}%
\pgfpathlineto{\pgfqpoint{5.706800in}{1.786812in}}%
\pgfpathlineto{\pgfqpoint{5.709490in}{1.794418in}}%
\pgfpathlineto{\pgfqpoint{5.712291in}{1.798847in}}%
\pgfpathlineto{\pgfqpoint{5.714834in}{1.792456in}}%
\pgfpathlineto{\pgfqpoint{5.717671in}{1.795479in}}%
\pgfpathlineto{\pgfqpoint{5.720201in}{1.796390in}}%
\pgfpathlineto{\pgfqpoint{5.722950in}{1.792496in}}%
\pgfpathlineto{\pgfqpoint{5.725548in}{1.791975in}}%
\pgfpathlineto{\pgfqpoint{5.728339in}{1.795910in}}%
\pgfpathlineto{\pgfqpoint{5.730919in}{1.783853in}}%
\pgfpathlineto{\pgfqpoint{5.733594in}{1.795520in}}%
\pgfpathlineto{\pgfqpoint{5.736276in}{1.797325in}}%
\pgfpathlineto{\pgfqpoint{5.738974in}{1.793811in}}%
\pgfpathlineto{\pgfqpoint{5.741745in}{1.788915in}}%
\pgfpathlineto{\pgfqpoint{5.744310in}{1.789118in}}%
\pgfpathlineto{\pgfqpoint{5.744310in}{0.413320in}}%
\pgfpathlineto{\pgfqpoint{5.744310in}{0.413320in}}%
\pgfpathlineto{\pgfqpoint{5.741745in}{0.413320in}}%
\pgfpathlineto{\pgfqpoint{5.738974in}{0.413320in}}%
\pgfpathlineto{\pgfqpoint{5.736276in}{0.413320in}}%
\pgfpathlineto{\pgfqpoint{5.733594in}{0.413320in}}%
\pgfpathlineto{\pgfqpoint{5.730919in}{0.413320in}}%
\pgfpathlineto{\pgfqpoint{5.728339in}{0.413320in}}%
\pgfpathlineto{\pgfqpoint{5.725548in}{0.413320in}}%
\pgfpathlineto{\pgfqpoint{5.722950in}{0.413320in}}%
\pgfpathlineto{\pgfqpoint{5.720201in}{0.413320in}}%
\pgfpathlineto{\pgfqpoint{5.717671in}{0.413320in}}%
\pgfpathlineto{\pgfqpoint{5.714834in}{0.413320in}}%
\pgfpathlineto{\pgfqpoint{5.712291in}{0.413320in}}%
\pgfpathlineto{\pgfqpoint{5.709490in}{0.413320in}}%
\pgfpathlineto{\pgfqpoint{5.706800in}{0.413320in}}%
\pgfpathlineto{\pgfqpoint{5.704130in}{0.413320in}}%
\pgfpathlineto{\pgfqpoint{5.701453in}{0.413320in}}%
\pgfpathlineto{\pgfqpoint{5.698775in}{0.413320in}}%
\pgfpathlineto{\pgfqpoint{5.696101in}{0.413320in}}%
\pgfpathlineto{\pgfqpoint{5.693473in}{0.413320in}}%
\pgfpathlineto{\pgfqpoint{5.690730in}{0.413320in}}%
\pgfpathlineto{\pgfqpoint{5.688159in}{0.413320in}}%
\pgfpathlineto{\pgfqpoint{5.685385in}{0.413320in}}%
\pgfpathlineto{\pgfqpoint{5.682836in}{0.413320in}}%
\pgfpathlineto{\pgfqpoint{5.680027in}{0.413320in}}%
\pgfpathlineto{\pgfqpoint{5.677486in}{0.413320in}}%
\pgfpathlineto{\pgfqpoint{5.674667in}{0.413320in}}%
\pgfpathlineto{\pgfqpoint{5.671991in}{0.413320in}}%
\pgfpathlineto{\pgfqpoint{5.669313in}{0.413320in}}%
\pgfpathlineto{\pgfqpoint{5.666632in}{0.413320in}}%
\pgfpathlineto{\pgfqpoint{5.664099in}{0.413320in}}%
\pgfpathlineto{\pgfqpoint{5.661273in}{0.413320in}}%
\pgfpathlineto{\pgfqpoint{5.658723in}{0.413320in}}%
\pgfpathlineto{\pgfqpoint{5.655919in}{0.413320in}}%
\pgfpathlineto{\pgfqpoint{5.653376in}{0.413320in}}%
\pgfpathlineto{\pgfqpoint{5.650563in}{0.413320in}}%
\pgfpathlineto{\pgfqpoint{5.648008in}{0.413320in}}%
\pgfpathlineto{\pgfqpoint{5.645243in}{0.413320in}}%
\pgfpathlineto{\pgfqpoint{5.642518in}{0.413320in}}%
\pgfpathlineto{\pgfqpoint{5.639852in}{0.413320in}}%
\pgfpathlineto{\pgfqpoint{5.637172in}{0.413320in}}%
\pgfpathlineto{\pgfqpoint{5.634496in}{0.413320in}}%
\pgfpathlineto{\pgfqpoint{5.631815in}{0.413320in}}%
\pgfpathlineto{\pgfqpoint{5.629232in}{0.413320in}}%
\pgfpathlineto{\pgfqpoint{5.626460in}{0.413320in}}%
\pgfpathlineto{\pgfqpoint{5.623868in}{0.413320in}}%
\pgfpathlineto{\pgfqpoint{5.621102in}{0.413320in}}%
\pgfpathlineto{\pgfqpoint{5.618526in}{0.413320in}}%
\pgfpathlineto{\pgfqpoint{5.615743in}{0.413320in}}%
\pgfpathlineto{\pgfqpoint{5.613235in}{0.413320in}}%
\pgfpathlineto{\pgfqpoint{5.610389in}{0.413320in}}%
\pgfpathlineto{\pgfqpoint{5.607698in}{0.413320in}}%
\pgfpathlineto{\pgfqpoint{5.605073in}{0.413320in}}%
\pgfpathlineto{\pgfqpoint{5.602352in}{0.413320in}}%
\pgfpathlineto{\pgfqpoint{5.599674in}{0.413320in}}%
\pgfpathlineto{\pgfqpoint{5.596999in}{0.413320in}}%
\pgfpathlineto{\pgfqpoint{5.594368in}{0.413320in}}%
\pgfpathlineto{\pgfqpoint{5.591641in}{0.413320in}}%
\pgfpathlineto{\pgfqpoint{5.589040in}{0.413320in}}%
\pgfpathlineto{\pgfqpoint{5.586269in}{0.413320in}}%
\pgfpathlineto{\pgfqpoint{5.583709in}{0.413320in}}%
\pgfpathlineto{\pgfqpoint{5.580914in}{0.413320in}}%
\pgfpathlineto{\pgfqpoint{5.578342in}{0.413320in}}%
\pgfpathlineto{\pgfqpoint{5.575596in}{0.413320in}}%
\pgfpathlineto{\pgfqpoint{5.572893in}{0.413320in}}%
\pgfpathlineto{\pgfqpoint{5.570215in}{0.413320in}}%
\pgfpathlineto{\pgfqpoint{5.567536in}{0.413320in}}%
\pgfpathlineto{\pgfqpoint{5.564940in}{0.413320in}}%
\pgfpathlineto{\pgfqpoint{5.562180in}{0.413320in}}%
\pgfpathlineto{\pgfqpoint{5.559612in}{0.413320in}}%
\pgfpathlineto{\pgfqpoint{5.556822in}{0.413320in}}%
\pgfpathlineto{\pgfqpoint{5.554198in}{0.413320in}}%
\pgfpathlineto{\pgfqpoint{5.551457in}{0.413320in}}%
\pgfpathlineto{\pgfqpoint{5.548921in}{0.413320in}}%
\pgfpathlineto{\pgfqpoint{5.546110in}{0.413320in}}%
\pgfpathlineto{\pgfqpoint{5.543421in}{0.413320in}}%
\pgfpathlineto{\pgfqpoint{5.540750in}{0.413320in}}%
\pgfpathlineto{\pgfqpoint{5.538074in}{0.413320in}}%
\pgfpathlineto{\pgfqpoint{5.535395in}{0.413320in}}%
\pgfpathlineto{\pgfqpoint{5.532717in}{0.413320in}}%
\pgfpathlineto{\pgfqpoint{5.530148in}{0.413320in}}%
\pgfpathlineto{\pgfqpoint{5.527360in}{0.413320in}}%
\pgfpathlineto{\pgfqpoint{5.524756in}{0.413320in}}%
\pgfpathlineto{\pgfqpoint{5.522003in}{0.413320in}}%
\pgfpathlineto{\pgfqpoint{5.519433in}{0.413320in}}%
\pgfpathlineto{\pgfqpoint{5.516646in}{0.413320in}}%
\pgfpathlineto{\pgfqpoint{5.514080in}{0.413320in}}%
\pgfpathlineto{\pgfqpoint{5.511290in}{0.413320in}}%
\pgfpathlineto{\pgfqpoint{5.508612in}{0.413320in}}%
\pgfpathlineto{\pgfqpoint{5.505933in}{0.413320in}}%
\pgfpathlineto{\pgfqpoint{5.503255in}{0.413320in}}%
\pgfpathlineto{\pgfqpoint{5.500687in}{0.413320in}}%
\pgfpathlineto{\pgfqpoint{5.497898in}{0.413320in}}%
\pgfpathlineto{\pgfqpoint{5.495346in}{0.413320in}}%
\pgfpathlineto{\pgfqpoint{5.492541in}{0.413320in}}%
\pgfpathlineto{\pgfqpoint{5.490000in}{0.413320in}}%
\pgfpathlineto{\pgfqpoint{5.487176in}{0.413320in}}%
\pgfpathlineto{\pgfqpoint{5.484641in}{0.413320in}}%
\pgfpathlineto{\pgfqpoint{5.481825in}{0.413320in}}%
\pgfpathlineto{\pgfqpoint{5.479152in}{0.413320in}}%
\pgfpathlineto{\pgfqpoint{5.476458in}{0.413320in}}%
\pgfpathlineto{\pgfqpoint{5.473792in}{0.413320in}}%
\pgfpathlineto{\pgfqpoint{5.471113in}{0.413320in}}%
\pgfpathlineto{\pgfqpoint{5.468425in}{0.413320in}}%
\pgfpathlineto{\pgfqpoint{5.465888in}{0.413320in}}%
\pgfpathlineto{\pgfqpoint{5.463079in}{0.413320in}}%
\pgfpathlineto{\pgfqpoint{5.460489in}{0.413320in}}%
\pgfpathlineto{\pgfqpoint{5.457721in}{0.413320in}}%
\pgfpathlineto{\pgfqpoint{5.455168in}{0.413320in}}%
\pgfpathlineto{\pgfqpoint{5.452365in}{0.413320in}}%
\pgfpathlineto{\pgfqpoint{5.449769in}{0.413320in}}%
\pgfpathlineto{\pgfqpoint{5.447021in}{0.413320in}}%
\pgfpathlineto{\pgfqpoint{5.444328in}{0.413320in}}%
\pgfpathlineto{\pgfqpoint{5.441698in}{0.413320in}}%
\pgfpathlineto{\pgfqpoint{5.438974in}{0.413320in}}%
\pgfpathlineto{\pgfqpoint{5.436295in}{0.413320in}}%
\pgfpathlineto{\pgfqpoint{5.433616in}{0.413320in}}%
\pgfpathlineto{\pgfqpoint{5.431015in}{0.413320in}}%
\pgfpathlineto{\pgfqpoint{5.428259in}{0.413320in}}%
\pgfpathlineto{\pgfqpoint{5.425661in}{0.413320in}}%
\pgfpathlineto{\pgfqpoint{5.422897in}{0.413320in}}%
\pgfpathlineto{\pgfqpoint{5.420304in}{0.413320in}}%
\pgfpathlineto{\pgfqpoint{5.417547in}{0.413320in}}%
\pgfpathlineto{\pgfqpoint{5.414954in}{0.413320in}}%
\pgfpathlineto{\pgfqpoint{5.412190in}{0.413320in}}%
\pgfpathlineto{\pgfqpoint{5.409507in}{0.413320in}}%
\pgfpathlineto{\pgfqpoint{5.406832in}{0.413320in}}%
\pgfpathlineto{\pgfqpoint{5.404154in}{0.413320in}}%
\pgfpathlineto{\pgfqpoint{5.401576in}{0.413320in}}%
\pgfpathlineto{\pgfqpoint{5.398784in}{0.413320in}}%
\pgfpathlineto{\pgfqpoint{5.396219in}{0.413320in}}%
\pgfpathlineto{\pgfqpoint{5.393441in}{0.413320in}}%
\pgfpathlineto{\pgfqpoint{5.390900in}{0.413320in}}%
\pgfpathlineto{\pgfqpoint{5.388083in}{0.413320in}}%
\pgfpathlineto{\pgfqpoint{5.385550in}{0.413320in}}%
\pgfpathlineto{\pgfqpoint{5.382725in}{0.413320in}}%
\pgfpathlineto{\pgfqpoint{5.380048in}{0.413320in}}%
\pgfpathlineto{\pgfqpoint{5.377370in}{0.413320in}}%
\pgfpathlineto{\pgfqpoint{5.374692in}{0.413320in}}%
\pgfpathlineto{\pgfqpoint{5.372013in}{0.413320in}}%
\pgfpathlineto{\pgfqpoint{5.369335in}{0.413320in}}%
\pgfpathlineto{\pgfqpoint{5.366727in}{0.413320in}}%
\pgfpathlineto{\pgfqpoint{5.363966in}{0.413320in}}%
\pgfpathlineto{\pgfqpoint{5.361370in}{0.413320in}}%
\pgfpathlineto{\pgfqpoint{5.358612in}{0.413320in}}%
\pgfpathlineto{\pgfqpoint{5.356056in}{0.413320in}}%
\pgfpathlineto{\pgfqpoint{5.353262in}{0.413320in}}%
\pgfpathlineto{\pgfqpoint{5.350723in}{0.413320in}}%
\pgfpathlineto{\pgfqpoint{5.347905in}{0.413320in}}%
\pgfpathlineto{\pgfqpoint{5.345224in}{0.413320in}}%
\pgfpathlineto{\pgfqpoint{5.342549in}{0.413320in}}%
\pgfpathlineto{\pgfqpoint{5.339872in}{0.413320in}}%
\pgfpathlineto{\pgfqpoint{5.337353in}{0.413320in}}%
\pgfpathlineto{\pgfqpoint{5.334510in}{0.413320in}}%
\pgfpathlineto{\pgfqpoint{5.331973in}{0.413320in}}%
\pgfpathlineto{\pgfqpoint{5.329159in}{0.413320in}}%
\pgfpathlineto{\pgfqpoint{5.326564in}{0.413320in}}%
\pgfpathlineto{\pgfqpoint{5.323802in}{0.413320in}}%
\pgfpathlineto{\pgfqpoint{5.321256in}{0.413320in}}%
\pgfpathlineto{\pgfqpoint{5.318430in}{0.413320in}}%
\pgfpathlineto{\pgfqpoint{5.315754in}{0.413320in}}%
\pgfpathlineto{\pgfqpoint{5.313089in}{0.413320in}}%
\pgfpathlineto{\pgfqpoint{5.310411in}{0.413320in}}%
\pgfpathlineto{\pgfqpoint{5.307731in}{0.413320in}}%
\pgfpathlineto{\pgfqpoint{5.305054in}{0.413320in}}%
\pgfpathlineto{\pgfqpoint{5.302443in}{0.413320in}}%
\pgfpathlineto{\pgfqpoint{5.299696in}{0.413320in}}%
\pgfpathlineto{\pgfqpoint{5.297140in}{0.413320in}}%
\pgfpathlineto{\pgfqpoint{5.294339in}{0.413320in}}%
\pgfpathlineto{\pgfqpoint{5.291794in}{0.413320in}}%
\pgfpathlineto{\pgfqpoint{5.288984in}{0.413320in}}%
\pgfpathlineto{\pgfqpoint{5.286436in}{0.413320in}}%
\pgfpathlineto{\pgfqpoint{5.283631in}{0.413320in}}%
\pgfpathlineto{\pgfqpoint{5.280947in}{0.413320in}}%
\pgfpathlineto{\pgfqpoint{5.278322in}{0.413320in}}%
\pgfpathlineto{\pgfqpoint{5.275589in}{0.413320in}}%
\pgfpathlineto{\pgfqpoint{5.272913in}{0.413320in}}%
\pgfpathlineto{\pgfqpoint{5.270238in}{0.413320in}}%
\pgfpathlineto{\pgfqpoint{5.267691in}{0.413320in}}%
\pgfpathlineto{\pgfqpoint{5.264876in}{0.413320in}}%
\pgfpathlineto{\pgfqpoint{5.262264in}{0.413320in}}%
\pgfpathlineto{\pgfqpoint{5.259511in}{0.413320in}}%
\pgfpathlineto{\pgfqpoint{5.256973in}{0.413320in}}%
\pgfpathlineto{\pgfqpoint{5.254236in}{0.413320in}}%
\pgfpathlineto{\pgfqpoint{5.251590in}{0.413320in}}%
\pgfpathlineto{\pgfqpoint{5.248816in}{0.413320in}}%
\pgfpathlineto{\pgfqpoint{5.246130in}{0.413320in}}%
\pgfpathlineto{\pgfqpoint{5.243445in}{0.413320in}}%
\pgfpathlineto{\pgfqpoint{5.240777in}{0.413320in}}%
\pgfpathlineto{\pgfqpoint{5.238173in}{0.413320in}}%
\pgfpathlineto{\pgfqpoint{5.235409in}{0.413320in}}%
\pgfpathlineto{\pgfqpoint{5.232855in}{0.413320in}}%
\pgfpathlineto{\pgfqpoint{5.230059in}{0.413320in}}%
\pgfpathlineto{\pgfqpoint{5.227470in}{0.413320in}}%
\pgfpathlineto{\pgfqpoint{5.224695in}{0.413320in}}%
\pgfpathlineto{\pgfqpoint{5.222151in}{0.413320in}}%
\pgfpathlineto{\pgfqpoint{5.219345in}{0.413320in}}%
\pgfpathlineto{\pgfqpoint{5.216667in}{0.413320in}}%
\pgfpathlineto{\pgfqpoint{5.214027in}{0.413320in}}%
\pgfpathlineto{\pgfqpoint{5.211299in}{0.413320in}}%
\pgfpathlineto{\pgfqpoint{5.208630in}{0.413320in}}%
\pgfpathlineto{\pgfqpoint{5.205952in}{0.413320in}}%
\pgfpathlineto{\pgfqpoint{5.203388in}{0.413320in}}%
\pgfpathlineto{\pgfqpoint{5.200594in}{0.413320in}}%
\pgfpathlineto{\pgfqpoint{5.198008in}{0.413320in}}%
\pgfpathlineto{\pgfqpoint{5.195239in}{0.413320in}}%
\pgfpathlineto{\pgfqpoint{5.192680in}{0.413320in}}%
\pgfpathlineto{\pgfqpoint{5.189880in}{0.413320in}}%
\pgfpathlineto{\pgfqpoint{5.187294in}{0.413320in}}%
\pgfpathlineto{\pgfqpoint{5.184522in}{0.413320in}}%
\pgfpathlineto{\pgfqpoint{5.181848in}{0.413320in}}%
\pgfpathlineto{\pgfqpoint{5.179188in}{0.413320in}}%
\pgfpathlineto{\pgfqpoint{5.176477in}{0.413320in}}%
\pgfpathlineto{\pgfqpoint{5.173925in}{0.413320in}}%
\pgfpathlineto{\pgfqpoint{5.171133in}{0.413320in}}%
\pgfpathlineto{\pgfqpoint{5.168591in}{0.413320in}}%
\pgfpathlineto{\pgfqpoint{5.165775in}{0.413320in}}%
\pgfpathlineto{\pgfqpoint{5.163243in}{0.413320in}}%
\pgfpathlineto{\pgfqpoint{5.160420in}{0.413320in}}%
\pgfpathlineto{\pgfqpoint{5.157815in}{0.413320in}}%
\pgfpathlineto{\pgfqpoint{5.155059in}{0.413320in}}%
\pgfpathlineto{\pgfqpoint{5.152382in}{0.413320in}}%
\pgfpathlineto{\pgfqpoint{5.149734in}{0.413320in}}%
\pgfpathlineto{\pgfqpoint{5.147029in}{0.413320in}}%
\pgfpathlineto{\pgfqpoint{5.144349in}{0.413320in}}%
\pgfpathlineto{\pgfqpoint{5.141660in}{0.413320in}}%
\pgfpathlineto{\pgfqpoint{5.139072in}{0.413320in}}%
\pgfpathlineto{\pgfqpoint{5.136311in}{0.413320in}}%
\pgfpathlineto{\pgfqpoint{5.133716in}{0.413320in}}%
\pgfpathlineto{\pgfqpoint{5.130953in}{0.413320in}}%
\pgfpathlineto{\pgfqpoint{5.128421in}{0.413320in}}%
\pgfpathlineto{\pgfqpoint{5.125599in}{0.413320in}}%
\pgfpathlineto{\pgfqpoint{5.123042in}{0.413320in}}%
\pgfpathlineto{\pgfqpoint{5.120243in}{0.413320in}}%
\pgfpathlineto{\pgfqpoint{5.117550in}{0.413320in}}%
\pgfpathlineto{\pgfqpoint{5.114887in}{0.413320in}}%
\pgfpathlineto{\pgfqpoint{5.112209in}{0.413320in}}%
\pgfpathlineto{\pgfqpoint{5.109530in}{0.413320in}}%
\pgfpathlineto{\pgfqpoint{5.106842in}{0.413320in}}%
\pgfpathlineto{\pgfqpoint{5.104312in}{0.413320in}}%
\pgfpathlineto{\pgfqpoint{5.101496in}{0.413320in}}%
\pgfpathlineto{\pgfqpoint{5.098948in}{0.413320in}}%
\pgfpathlineto{\pgfqpoint{5.096142in}{0.413320in}}%
\pgfpathlineto{\pgfqpoint{5.093579in}{0.413320in}}%
\pgfpathlineto{\pgfqpoint{5.090788in}{0.413320in}}%
\pgfpathlineto{\pgfqpoint{5.088103in}{0.413320in}}%
\pgfpathlineto{\pgfqpoint{5.085426in}{0.413320in}}%
\pgfpathlineto{\pgfqpoint{5.082746in}{0.413320in}}%
\pgfpathlineto{\pgfqpoint{5.080067in}{0.413320in}}%
\pgfpathlineto{\pgfqpoint{5.077390in}{0.413320in}}%
\pgfpathlineto{\pgfqpoint{5.074851in}{0.413320in}}%
\pgfpathlineto{\pgfqpoint{5.072030in}{0.413320in}}%
\pgfpathlineto{\pgfqpoint{5.069463in}{0.413320in}}%
\pgfpathlineto{\pgfqpoint{5.066677in}{0.413320in}}%
\pgfpathlineto{\pgfqpoint{5.064144in}{0.413320in}}%
\pgfpathlineto{\pgfqpoint{5.061315in}{0.413320in}}%
\pgfpathlineto{\pgfqpoint{5.058711in}{0.413320in}}%
\pgfpathlineto{\pgfqpoint{5.055952in}{0.413320in}}%
\pgfpathlineto{\pgfqpoint{5.053284in}{0.413320in}}%
\pgfpathlineto{\pgfqpoint{5.050606in}{0.413320in}}%
\pgfpathlineto{\pgfqpoint{5.047924in}{0.413320in}}%
\pgfpathlineto{\pgfqpoint{5.045249in}{0.413320in}}%
\pgfpathlineto{\pgfqpoint{5.042572in}{0.413320in}}%
\pgfpathlineto{\pgfqpoint{5.039962in}{0.413320in}}%
\pgfpathlineto{\pgfqpoint{5.037214in}{0.413320in}}%
\pgfpathlineto{\pgfqpoint{5.034649in}{0.413320in}}%
\pgfpathlineto{\pgfqpoint{5.031849in}{0.413320in}}%
\pgfpathlineto{\pgfqpoint{5.029275in}{0.413320in}}%
\pgfpathlineto{\pgfqpoint{5.026501in}{0.413320in}}%
\pgfpathlineto{\pgfqpoint{5.023927in}{0.413320in}}%
\pgfpathlineto{\pgfqpoint{5.021147in}{0.413320in}}%
\pgfpathlineto{\pgfqpoint{5.018466in}{0.413320in}}%
\pgfpathlineto{\pgfqpoint{5.015820in}{0.413320in}}%
\pgfpathlineto{\pgfqpoint{5.013104in}{0.413320in}}%
\pgfpathlineto{\pgfqpoint{5.010562in}{0.413320in}}%
\pgfpathlineto{\pgfqpoint{5.007751in}{0.413320in}}%
\pgfpathlineto{\pgfqpoint{5.005178in}{0.413320in}}%
\pgfpathlineto{\pgfqpoint{5.002384in}{0.413320in}}%
\pgfpathlineto{\pgfqpoint{4.999780in}{0.413320in}}%
\pgfpathlineto{\pgfqpoint{4.997028in}{0.413320in}}%
\pgfpathlineto{\pgfqpoint{4.994390in}{0.413320in}}%
\pgfpathlineto{\pgfqpoint{4.991683in}{0.413320in}}%
\pgfpathlineto{\pgfqpoint{4.989001in}{0.413320in}}%
\pgfpathlineto{\pgfqpoint{4.986325in}{0.413320in}}%
\pgfpathlineto{\pgfqpoint{4.983637in}{0.413320in}}%
\pgfpathlineto{\pgfqpoint{4.980967in}{0.413320in}}%
\pgfpathlineto{\pgfqpoint{4.978287in}{0.413320in}}%
\pgfpathlineto{\pgfqpoint{4.975703in}{0.413320in}}%
\pgfpathlineto{\pgfqpoint{4.972933in}{0.413320in}}%
\pgfpathlineto{\pgfqpoint{4.970314in}{0.413320in}}%
\pgfpathlineto{\pgfqpoint{4.967575in}{0.413320in}}%
\pgfpathlineto{\pgfqpoint{4.965002in}{0.413320in}}%
\pgfpathlineto{\pgfqpoint{4.962219in}{0.413320in}}%
\pgfpathlineto{\pgfqpoint{4.959689in}{0.413320in}}%
\pgfpathlineto{\pgfqpoint{4.956862in}{0.413320in}}%
\pgfpathlineto{\pgfqpoint{4.954182in}{0.413320in}}%
\pgfpathlineto{\pgfqpoint{4.951504in}{0.413320in}}%
\pgfpathlineto{\pgfqpoint{4.948827in}{0.413320in}}%
\pgfpathlineto{\pgfqpoint{4.946151in}{0.413320in}}%
\pgfpathlineto{\pgfqpoint{4.943466in}{0.413320in}}%
\pgfpathlineto{\pgfqpoint{4.940881in}{0.413320in}}%
\pgfpathlineto{\pgfqpoint{4.938112in}{0.413320in}}%
\pgfpathlineto{\pgfqpoint{4.935515in}{0.413320in}}%
\pgfpathlineto{\pgfqpoint{4.932742in}{0.413320in}}%
\pgfpathlineto{\pgfqpoint{4.930170in}{0.413320in}}%
\pgfpathlineto{\pgfqpoint{4.927400in}{0.413320in}}%
\pgfpathlineto{\pgfqpoint{4.924708in}{0.413320in}}%
\pgfpathlineto{\pgfqpoint{4.922041in}{0.413320in}}%
\pgfpathlineto{\pgfqpoint{4.919352in}{0.413320in}}%
\pgfpathlineto{\pgfqpoint{4.916681in}{0.413320in}}%
\pgfpathlineto{\pgfqpoint{4.914009in}{0.413320in}}%
\pgfpathlineto{\pgfqpoint{4.911435in}{0.413320in}}%
\pgfpathlineto{\pgfqpoint{4.908648in}{0.413320in}}%
\pgfpathlineto{\pgfqpoint{4.906096in}{0.413320in}}%
\pgfpathlineto{\pgfqpoint{4.903295in}{0.413320in}}%
\pgfpathlineto{\pgfqpoint{4.900712in}{0.413320in}}%
\pgfpathlineto{\pgfqpoint{4.897938in}{0.413320in}}%
\pgfpathlineto{\pgfqpoint{4.895399in}{0.413320in}}%
\pgfpathlineto{\pgfqpoint{4.892611in}{0.413320in}}%
\pgfpathlineto{\pgfqpoint{4.889902in}{0.413320in}}%
\pgfpathlineto{\pgfqpoint{4.887211in}{0.413320in}}%
\pgfpathlineto{\pgfqpoint{4.884540in}{0.413320in}}%
\pgfpathlineto{\pgfqpoint{4.881864in}{0.413320in}}%
\pgfpathlineto{\pgfqpoint{4.879180in}{0.413320in}}%
\pgfpathlineto{\pgfqpoint{4.876636in}{0.413320in}}%
\pgfpathlineto{\pgfqpoint{4.873832in}{0.413320in}}%
\pgfpathlineto{\pgfqpoint{4.871209in}{0.413320in}}%
\pgfpathlineto{\pgfqpoint{4.868474in}{0.413320in}}%
\pgfpathlineto{\pgfqpoint{4.865910in}{0.413320in}}%
\pgfpathlineto{\pgfqpoint{4.863116in}{0.413320in}}%
\pgfpathlineto{\pgfqpoint{4.860544in}{0.413320in}}%
\pgfpathlineto{\pgfqpoint{4.857807in}{0.413320in}}%
\pgfpathlineto{\pgfqpoint{4.855070in}{0.413320in}}%
\pgfpathlineto{\pgfqpoint{4.852404in}{0.413320in}}%
\pgfpathlineto{\pgfqpoint{4.849715in}{0.413320in}}%
\pgfpathlineto{\pgfqpoint{4.847127in}{0.413320in}}%
\pgfpathlineto{\pgfqpoint{4.844361in}{0.413320in}}%
\pgfpathlineto{\pgfqpoint{4.842380in}{0.413320in}}%
\pgfpathlineto{\pgfqpoint{4.839922in}{0.413320in}}%
\pgfpathlineto{\pgfqpoint{4.837992in}{0.413320in}}%
\pgfpathlineto{\pgfqpoint{4.833657in}{0.413320in}}%
\pgfpathlineto{\pgfqpoint{4.831045in}{0.413320in}}%
\pgfpathlineto{\pgfqpoint{4.828291in}{0.413320in}}%
\pgfpathlineto{\pgfqpoint{4.825619in}{0.413320in}}%
\pgfpathlineto{\pgfqpoint{4.822945in}{0.413320in}}%
\pgfpathlineto{\pgfqpoint{4.820265in}{0.413320in}}%
\pgfpathlineto{\pgfqpoint{4.817587in}{0.413320in}}%
\pgfpathlineto{\pgfqpoint{4.814907in}{0.413320in}}%
\pgfpathlineto{\pgfqpoint{4.812377in}{0.413320in}}%
\pgfpathlineto{\pgfqpoint{4.809538in}{0.413320in}}%
\pgfpathlineto{\pgfqpoint{4.807017in}{0.413320in}}%
\pgfpathlineto{\pgfqpoint{4.804193in}{0.413320in}}%
\pgfpathlineto{\pgfqpoint{4.801586in}{0.413320in}}%
\pgfpathlineto{\pgfqpoint{4.798830in}{0.413320in}}%
\pgfpathlineto{\pgfqpoint{4.796274in}{0.413320in}}%
\pgfpathlineto{\pgfqpoint{4.793512in}{0.413320in}}%
\pgfpathlineto{\pgfqpoint{4.790798in}{0.413320in}}%
\pgfpathlineto{\pgfqpoint{4.788116in}{0.413320in}}%
\pgfpathlineto{\pgfqpoint{4.785445in}{0.413320in}}%
\pgfpathlineto{\pgfqpoint{4.782872in}{0.413320in}}%
\pgfpathlineto{\pgfqpoint{4.780083in}{0.413320in}}%
\pgfpathlineto{\pgfqpoint{4.777535in}{0.413320in}}%
\pgfpathlineto{\pgfqpoint{4.774732in}{0.413320in}}%
\pgfpathlineto{\pgfqpoint{4.772198in}{0.413320in}}%
\pgfpathlineto{\pgfqpoint{4.769367in}{0.413320in}}%
\pgfpathlineto{\pgfqpoint{4.766783in}{0.413320in}}%
\pgfpathlineto{\pgfqpoint{4.764018in}{0.413320in}}%
\pgfpathlineto{\pgfqpoint{4.761337in}{0.413320in}}%
\pgfpathlineto{\pgfqpoint{4.758653in}{0.413320in}}%
\pgfpathlineto{\pgfqpoint{4.755983in}{0.413320in}}%
\pgfpathlineto{\pgfqpoint{4.753298in}{0.413320in}}%
\pgfpathlineto{\pgfqpoint{4.750627in}{0.413320in}}%
\pgfpathlineto{\pgfqpoint{4.748081in}{0.413320in}}%
\pgfpathlineto{\pgfqpoint{4.745256in}{0.413320in}}%
\pgfpathlineto{\pgfqpoint{4.742696in}{0.413320in}}%
\pgfpathlineto{\pgfqpoint{4.739912in}{0.413320in}}%
\pgfpathlineto{\pgfqpoint{4.737348in}{0.413320in}}%
\pgfpathlineto{\pgfqpoint{4.734552in}{0.413320in}}%
\pgfpathlineto{\pgfqpoint{4.731901in}{0.413320in}}%
\pgfpathlineto{\pgfqpoint{4.729233in}{0.413320in}}%
\pgfpathlineto{\pgfqpoint{4.726508in}{0.413320in}}%
\pgfpathlineto{\pgfqpoint{4.723873in}{0.413320in}}%
\pgfpathlineto{\pgfqpoint{4.721160in}{0.413320in}}%
\pgfpathlineto{\pgfqpoint{4.718486in}{0.413320in}}%
\pgfpathlineto{\pgfqpoint{4.715806in}{0.413320in}}%
\pgfpathlineto{\pgfqpoint{4.713275in}{0.413320in}}%
\pgfpathlineto{\pgfqpoint{4.710437in}{0.413320in}}%
\pgfpathlineto{\pgfqpoint{4.707824in}{0.413320in}}%
\pgfpathlineto{\pgfqpoint{4.705094in}{0.413320in}}%
\pgfpathlineto{\pgfqpoint{4.702517in}{0.413320in}}%
\pgfpathlineto{\pgfqpoint{4.699734in}{0.413320in}}%
\pgfpathlineto{\pgfqpoint{4.697170in}{0.413320in}}%
\pgfpathlineto{\pgfqpoint{4.694381in}{0.413320in}}%
\pgfpathlineto{\pgfqpoint{4.691694in}{0.413320in}}%
\pgfpathlineto{\pgfqpoint{4.689051in}{0.413320in}}%
\pgfpathlineto{\pgfqpoint{4.686337in}{0.413320in}}%
\pgfpathlineto{\pgfqpoint{4.683799in}{0.413320in}}%
\pgfpathlineto{\pgfqpoint{4.680988in}{0.413320in}}%
\pgfpathlineto{\pgfqpoint{4.678448in}{0.413320in}}%
\pgfpathlineto{\pgfqpoint{4.675619in}{0.413320in}}%
\pgfpathlineto{\pgfqpoint{4.673068in}{0.413320in}}%
\pgfpathlineto{\pgfqpoint{4.670261in}{0.413320in}}%
\pgfpathlineto{\pgfqpoint{4.667764in}{0.413320in}}%
\pgfpathlineto{\pgfqpoint{4.664923in}{0.413320in}}%
\pgfpathlineto{\pgfqpoint{4.662237in}{0.413320in}}%
\pgfpathlineto{\pgfqpoint{4.659590in}{0.413320in}}%
\pgfpathlineto{\pgfqpoint{4.656873in}{0.413320in}}%
\pgfpathlineto{\pgfqpoint{4.654203in}{0.413320in}}%
\pgfpathlineto{\pgfqpoint{4.651524in}{0.413320in}}%
\pgfpathlineto{\pgfqpoint{4.648922in}{0.413320in}}%
\pgfpathlineto{\pgfqpoint{4.646169in}{0.413320in}}%
\pgfpathlineto{\pgfqpoint{4.643628in}{0.413320in}}%
\pgfpathlineto{\pgfqpoint{4.640809in}{0.413320in}}%
\pgfpathlineto{\pgfqpoint{4.638204in}{0.413320in}}%
\pgfpathlineto{\pgfqpoint{4.635445in}{0.413320in}}%
\pgfpathlineto{\pgfqpoint{4.632902in}{0.413320in}}%
\pgfpathlineto{\pgfqpoint{4.630096in}{0.413320in}}%
\pgfpathlineto{\pgfqpoint{4.627411in}{0.413320in}}%
\pgfpathlineto{\pgfqpoint{4.624741in}{0.413320in}}%
\pgfpathlineto{\pgfqpoint{4.622056in}{0.413320in}}%
\pgfpathlineto{\pgfqpoint{4.619529in}{0.413320in}}%
\pgfpathlineto{\pgfqpoint{4.616702in}{0.413320in}}%
\pgfpathlineto{\pgfqpoint{4.614134in}{0.413320in}}%
\pgfpathlineto{\pgfqpoint{4.611350in}{0.413320in}}%
\pgfpathlineto{\pgfqpoint{4.608808in}{0.413320in}}%
\pgfpathlineto{\pgfqpoint{4.605990in}{0.413320in}}%
\pgfpathlineto{\pgfqpoint{4.603430in}{0.413320in}}%
\pgfpathlineto{\pgfqpoint{4.600633in}{0.413320in}}%
\pgfpathlineto{\pgfqpoint{4.597951in}{0.413320in}}%
\pgfpathlineto{\pgfqpoint{4.595281in}{0.413320in}}%
\pgfpathlineto{\pgfqpoint{4.592589in}{0.413320in}}%
\pgfpathlineto{\pgfqpoint{4.589920in}{0.413320in}}%
\pgfpathlineto{\pgfqpoint{4.587244in}{0.413320in}}%
\pgfpathlineto{\pgfqpoint{4.584672in}{0.413320in}}%
\pgfpathlineto{\pgfqpoint{4.581888in}{0.413320in}}%
\pgfpathlineto{\pgfqpoint{4.579305in}{0.413320in}}%
\pgfpathlineto{\pgfqpoint{4.576531in}{0.413320in}}%
\pgfpathlineto{\pgfqpoint{4.573947in}{0.413320in}}%
\pgfpathlineto{\pgfqpoint{4.571171in}{0.413320in}}%
\pgfpathlineto{\pgfqpoint{4.568612in}{0.413320in}}%
\pgfpathlineto{\pgfqpoint{4.565820in}{0.413320in}}%
\pgfpathlineto{\pgfqpoint{4.563125in}{0.413320in}}%
\pgfpathlineto{\pgfqpoint{4.560448in}{0.413320in}}%
\pgfpathlineto{\pgfqpoint{4.557777in}{0.413320in}}%
\pgfpathlineto{\pgfqpoint{4.555106in}{0.413320in}}%
\pgfpathlineto{\pgfqpoint{4.552425in}{0.413320in}}%
\pgfpathlineto{\pgfqpoint{4.549822in}{0.413320in}}%
\pgfpathlineto{\pgfqpoint{4.547064in}{0.413320in}}%
\pgfpathlineto{\pgfqpoint{4.544464in}{0.413320in}}%
\pgfpathlineto{\pgfqpoint{4.541711in}{0.413320in}}%
\pgfpathlineto{\pgfqpoint{4.539144in}{0.413320in}}%
\pgfpathlineto{\pgfqpoint{4.536400in}{0.413320in}}%
\pgfpathlineto{\pgfqpoint{4.533764in}{0.413320in}}%
\pgfpathlineto{\pgfqpoint{4.530990in}{0.413320in}}%
\pgfpathlineto{\pgfqpoint{4.528307in}{0.413320in}}%
\pgfpathlineto{\pgfqpoint{4.525640in}{0.413320in}}%
\pgfpathlineto{\pgfqpoint{4.522962in}{0.413320in}}%
\pgfpathlineto{\pgfqpoint{4.520345in}{0.413320in}}%
\pgfpathlineto{\pgfqpoint{4.517598in}{0.413320in}}%
\pgfpathlineto{\pgfqpoint{4.515080in}{0.413320in}}%
\pgfpathlineto{\pgfqpoint{4.512246in}{0.413320in}}%
\pgfpathlineto{\pgfqpoint{4.509643in}{0.413320in}}%
\pgfpathlineto{\pgfqpoint{4.506893in}{0.413320in}}%
\pgfpathlineto{\pgfqpoint{4.504305in}{0.413320in}}%
\pgfpathlineto{\pgfqpoint{4.501529in}{0.413320in}}%
\pgfpathlineto{\pgfqpoint{4.498850in}{0.413320in}}%
\pgfpathlineto{\pgfqpoint{4.496167in}{0.413320in}}%
\pgfpathlineto{\pgfqpoint{4.493492in}{0.413320in}}%
\pgfpathlineto{\pgfqpoint{4.490822in}{0.413320in}}%
\pgfpathlineto{\pgfqpoint{4.488130in}{0.413320in}}%
\pgfpathlineto{\pgfqpoint{4.485581in}{0.413320in}}%
\pgfpathlineto{\pgfqpoint{4.482778in}{0.413320in}}%
\pgfpathlineto{\pgfqpoint{4.480201in}{0.413320in}}%
\pgfpathlineto{\pgfqpoint{4.477430in}{0.413320in}}%
\pgfpathlineto{\pgfqpoint{4.474861in}{0.413320in}}%
\pgfpathlineto{\pgfqpoint{4.472059in}{0.413320in}}%
\pgfpathlineto{\pgfqpoint{4.469492in}{0.413320in}}%
\pgfpathlineto{\pgfqpoint{4.466717in}{0.413320in}}%
\pgfpathlineto{\pgfqpoint{4.464029in}{0.413320in}}%
\pgfpathlineto{\pgfqpoint{4.461367in}{0.413320in}}%
\pgfpathlineto{\pgfqpoint{4.458681in}{0.413320in}}%
\pgfpathlineto{\pgfqpoint{4.456138in}{0.413320in}}%
\pgfpathlineto{\pgfqpoint{4.453312in}{0.413320in}}%
\pgfpathlineto{\pgfqpoint{4.450767in}{0.413320in}}%
\pgfpathlineto{\pgfqpoint{4.447965in}{0.413320in}}%
\pgfpathlineto{\pgfqpoint{4.445423in}{0.413320in}}%
\pgfpathlineto{\pgfqpoint{4.442611in}{0.413320in}}%
\pgfpathlineto{\pgfqpoint{4.440041in}{0.413320in}}%
\pgfpathlineto{\pgfqpoint{4.437253in}{0.413320in}}%
\pgfpathlineto{\pgfqpoint{4.434569in}{0.413320in}}%
\pgfpathlineto{\pgfqpoint{4.431901in}{0.413320in}}%
\pgfpathlineto{\pgfqpoint{4.429220in}{0.413320in}}%
\pgfpathlineto{\pgfqpoint{4.426534in}{0.413320in}}%
\pgfpathlineto{\pgfqpoint{4.423863in}{0.413320in}}%
\pgfpathlineto{\pgfqpoint{4.421292in}{0.413320in}}%
\pgfpathlineto{\pgfqpoint{4.418506in}{0.413320in}}%
\pgfpathlineto{\pgfqpoint{4.415932in}{0.413320in}}%
\pgfpathlineto{\pgfqpoint{4.413149in}{0.413320in}}%
\pgfpathlineto{\pgfqpoint{4.410587in}{0.413320in}}%
\pgfpathlineto{\pgfqpoint{4.407788in}{0.413320in}}%
\pgfpathlineto{\pgfqpoint{4.405234in}{0.413320in}}%
\pgfpathlineto{\pgfqpoint{4.402468in}{0.413320in}}%
\pgfpathlineto{\pgfqpoint{4.399745in}{0.413320in}}%
\pgfpathlineto{\pgfqpoint{4.397076in}{0.413320in}}%
\pgfpathlineto{\pgfqpoint{4.394400in}{0.413320in}}%
\pgfpathlineto{\pgfqpoint{4.391721in}{0.413320in}}%
\pgfpathlineto{\pgfqpoint{4.389044in}{0.413320in}}%
\pgfpathlineto{\pgfqpoint{4.386431in}{0.413320in}}%
\pgfpathlineto{\pgfqpoint{4.383674in}{0.413320in}}%
\pgfpathlineto{\pgfqpoint{4.381097in}{0.413320in}}%
\pgfpathlineto{\pgfqpoint{4.378329in}{0.413320in}}%
\pgfpathlineto{\pgfqpoint{4.375761in}{0.413320in}}%
\pgfpathlineto{\pgfqpoint{4.372976in}{0.413320in}}%
\pgfpathlineto{\pgfqpoint{4.370437in}{0.413320in}}%
\pgfpathlineto{\pgfqpoint{4.367646in}{0.413320in}}%
\pgfpathlineto{\pgfqpoint{4.364936in}{0.413320in}}%
\pgfpathlineto{\pgfqpoint{4.362270in}{0.413320in}}%
\pgfpathlineto{\pgfqpoint{4.359582in}{0.413320in}}%
\pgfpathlineto{\pgfqpoint{4.357014in}{0.413320in}}%
\pgfpathlineto{\pgfqpoint{4.354224in}{0.413320in}}%
\pgfpathlineto{\pgfqpoint{4.351645in}{0.413320in}}%
\pgfpathlineto{\pgfqpoint{4.348868in}{0.413320in}}%
\pgfpathlineto{\pgfqpoint{4.346263in}{0.413320in}}%
\pgfpathlineto{\pgfqpoint{4.343510in}{0.413320in}}%
\pgfpathlineto{\pgfqpoint{4.340976in}{0.413320in}}%
\pgfpathlineto{\pgfqpoint{4.338154in}{0.413320in}}%
\pgfpathlineto{\pgfqpoint{4.335463in}{0.413320in}}%
\pgfpathlineto{\pgfqpoint{4.332796in}{0.413320in}}%
\pgfpathlineto{\pgfqpoint{4.330118in}{0.413320in}}%
\pgfpathlineto{\pgfqpoint{4.327440in}{0.413320in}}%
\pgfpathlineto{\pgfqpoint{4.324760in}{0.413320in}}%
\pgfpathlineto{\pgfqpoint{4.322181in}{0.413320in}}%
\pgfpathlineto{\pgfqpoint{4.319405in}{0.413320in}}%
\pgfpathlineto{\pgfqpoint{4.316856in}{0.413320in}}%
\pgfpathlineto{\pgfqpoint{4.314032in}{0.413320in}}%
\pgfpathlineto{\pgfqpoint{4.311494in}{0.413320in}}%
\pgfpathlineto{\pgfqpoint{4.308691in}{0.413320in}}%
\pgfpathlineto{\pgfqpoint{4.306118in}{0.413320in}}%
\pgfpathlineto{\pgfqpoint{4.303357in}{0.413320in}}%
\pgfpathlineto{\pgfqpoint{4.300656in}{0.413320in}}%
\pgfpathlineto{\pgfqpoint{4.297977in}{0.413320in}}%
\pgfpathlineto{\pgfqpoint{4.295299in}{0.413320in}}%
\pgfpathlineto{\pgfqpoint{4.292786in}{0.413320in}}%
\pgfpathlineto{\pgfqpoint{4.289936in}{0.413320in}}%
\pgfpathlineto{\pgfqpoint{4.287399in}{0.413320in}}%
\pgfpathlineto{\pgfqpoint{4.284586in}{0.413320in}}%
\pgfpathlineto{\pgfqpoint{4.282000in}{0.413320in}}%
\pgfpathlineto{\pgfqpoint{4.279212in}{0.413320in}}%
\pgfpathlineto{\pgfqpoint{4.276635in}{0.413320in}}%
\pgfpathlineto{\pgfqpoint{4.273874in}{0.413320in}}%
\pgfpathlineto{\pgfqpoint{4.271187in}{0.413320in}}%
\pgfpathlineto{\pgfqpoint{4.268590in}{0.413320in}}%
\pgfpathlineto{\pgfqpoint{4.265824in}{0.413320in}}%
\pgfpathlineto{\pgfqpoint{4.263157in}{0.413320in}}%
\pgfpathlineto{\pgfqpoint{4.260477in}{0.413320in}}%
\pgfpathlineto{\pgfqpoint{4.257958in}{0.413320in}}%
\pgfpathlineto{\pgfqpoint{4.255120in}{0.413320in}}%
\pgfpathlineto{\pgfqpoint{4.252581in}{0.413320in}}%
\pgfpathlineto{\pgfqpoint{4.249767in}{0.413320in}}%
\pgfpathlineto{\pgfqpoint{4.247225in}{0.413320in}}%
\pgfpathlineto{\pgfqpoint{4.244394in}{0.413320in}}%
\pgfpathlineto{\pgfqpoint{4.241900in}{0.413320in}}%
\pgfpathlineto{\pgfqpoint{4.239084in}{0.413320in}}%
\pgfpathlineto{\pgfqpoint{4.236375in}{0.413320in}}%
\pgfpathlineto{\pgfqpoint{4.233691in}{0.413320in}}%
\pgfpathlineto{\pgfqpoint{4.231013in}{0.413320in}}%
\pgfpathlineto{\pgfqpoint{4.228331in}{0.413320in}}%
\pgfpathlineto{\pgfqpoint{4.225654in}{0.413320in}}%
\pgfpathlineto{\pgfqpoint{4.223082in}{0.413320in}}%
\pgfpathlineto{\pgfqpoint{4.220304in}{0.413320in}}%
\pgfpathlineto{\pgfqpoint{4.217694in}{0.413320in}}%
\pgfpathlineto{\pgfqpoint{4.214948in}{0.413320in}}%
\pgfpathlineto{\pgfqpoint{4.212383in}{0.413320in}}%
\pgfpathlineto{\pgfqpoint{4.209597in}{0.413320in}}%
\pgfpathlineto{\pgfqpoint{4.207076in}{0.413320in}}%
\pgfpathlineto{\pgfqpoint{4.204240in}{0.413320in}}%
\pgfpathlineto{\pgfqpoint{4.201542in}{0.413320in}}%
\pgfpathlineto{\pgfqpoint{4.198878in}{0.413320in}}%
\pgfpathlineto{\pgfqpoint{4.196186in}{0.413320in}}%
\pgfpathlineto{\pgfqpoint{4.193638in}{0.413320in}}%
\pgfpathlineto{\pgfqpoint{4.190842in}{0.413320in}}%
\pgfpathlineto{\pgfqpoint{4.188318in}{0.413320in}}%
\pgfpathlineto{\pgfqpoint{4.185481in}{0.413320in}}%
\pgfpathlineto{\pgfqpoint{4.182899in}{0.413320in}}%
\pgfpathlineto{\pgfqpoint{4.180129in}{0.413320in}}%
\pgfpathlineto{\pgfqpoint{4.177593in}{0.413320in}}%
\pgfpathlineto{\pgfqpoint{4.174770in}{0.413320in}}%
\pgfpathlineto{\pgfqpoint{4.172093in}{0.413320in}}%
\pgfpathlineto{\pgfqpoint{4.169415in}{0.413320in}}%
\pgfpathlineto{\pgfqpoint{4.166737in}{0.413320in}}%
\pgfpathlineto{\pgfqpoint{4.164059in}{0.413320in}}%
\pgfpathlineto{\pgfqpoint{4.161380in}{0.413320in}}%
\pgfpathlineto{\pgfqpoint{4.158806in}{0.413320in}}%
\pgfpathlineto{\pgfqpoint{4.156016in}{0.413320in}}%
\pgfpathlineto{\pgfqpoint{4.153423in}{0.413320in}}%
\pgfpathlineto{\pgfqpoint{4.150665in}{0.413320in}}%
\pgfpathlineto{\pgfqpoint{4.148082in}{0.413320in}}%
\pgfpathlineto{\pgfqpoint{4.145310in}{0.413320in}}%
\pgfpathlineto{\pgfqpoint{4.142713in}{0.413320in}}%
\pgfpathlineto{\pgfqpoint{4.139963in}{0.413320in}}%
\pgfpathlineto{\pgfqpoint{4.137272in}{0.413320in}}%
\pgfpathlineto{\pgfqpoint{4.134615in}{0.413320in}}%
\pgfpathlineto{\pgfqpoint{4.131920in}{0.413320in}}%
\pgfpathlineto{\pgfqpoint{4.129349in}{0.413320in}}%
\pgfpathlineto{\pgfqpoint{4.126553in}{0.413320in}}%
\pgfpathlineto{\pgfqpoint{4.124019in}{0.413320in}}%
\pgfpathlineto{\pgfqpoint{4.121205in}{0.413320in}}%
\pgfpathlineto{\pgfqpoint{4.118554in}{0.413320in}}%
\pgfpathlineto{\pgfqpoint{4.115844in}{0.413320in}}%
\pgfpathlineto{\pgfqpoint{4.113252in}{0.413320in}}%
\pgfpathlineto{\pgfqpoint{4.110488in}{0.413320in}}%
\pgfpathlineto{\pgfqpoint{4.107814in}{0.413320in}}%
\pgfpathlineto{\pgfqpoint{4.105185in}{0.413320in}}%
\pgfpathlineto{\pgfqpoint{4.102456in}{0.413320in}}%
\pgfpathlineto{\pgfqpoint{4.099777in}{0.413320in}}%
\pgfpathlineto{\pgfqpoint{4.097092in}{0.413320in}}%
\pgfpathlineto{\pgfqpoint{4.094527in}{0.413320in}}%
\pgfpathlineto{\pgfqpoint{4.091729in}{0.413320in}}%
\pgfpathlineto{\pgfqpoint{4.089159in}{0.413320in}}%
\pgfpathlineto{\pgfqpoint{4.086385in}{0.413320in}}%
\pgfpathlineto{\pgfqpoint{4.083870in}{0.413320in}}%
\pgfpathlineto{\pgfqpoint{4.081018in}{0.413320in}}%
\pgfpathlineto{\pgfqpoint{4.078471in}{0.413320in}}%
\pgfpathlineto{\pgfqpoint{4.075705in}{0.413320in}}%
\pgfpathlineto{\pgfqpoint{4.072985in}{0.413320in}}%
\pgfpathlineto{\pgfqpoint{4.070313in}{0.413320in}}%
\pgfpathlineto{\pgfqpoint{4.067636in}{0.413320in}}%
\pgfpathlineto{\pgfqpoint{4.064957in}{0.413320in}}%
\pgfpathlineto{\pgfqpoint{4.062266in}{0.413320in}}%
\pgfpathlineto{\pgfqpoint{4.059702in}{0.413320in}}%
\pgfpathlineto{\pgfqpoint{4.056911in}{0.413320in}}%
\pgfpathlineto{\pgfqpoint{4.054326in}{0.413320in}}%
\pgfpathlineto{\pgfqpoint{4.051557in}{0.413320in}}%
\pgfpathlineto{\pgfqpoint{4.049006in}{0.413320in}}%
\pgfpathlineto{\pgfqpoint{4.046210in}{0.413320in}}%
\pgfpathlineto{\pgfqpoint{4.043667in}{0.413320in}}%
\pgfpathlineto{\pgfqpoint{4.040852in}{0.413320in}}%
\pgfpathlineto{\pgfqpoint{4.038174in}{0.413320in}}%
\pgfpathlineto{\pgfqpoint{4.035492in}{0.413320in}}%
\pgfpathlineto{\pgfqpoint{4.032817in}{0.413320in}}%
\pgfpathlineto{\pgfqpoint{4.030229in}{0.413320in}}%
\pgfpathlineto{\pgfqpoint{4.027447in}{0.413320in}}%
\pgfpathlineto{\pgfqpoint{4.024868in}{0.413320in}}%
\pgfpathlineto{\pgfqpoint{4.022097in}{0.413320in}}%
\pgfpathlineto{\pgfqpoint{4.019518in}{0.413320in}}%
\pgfpathlineto{\pgfqpoint{4.016744in}{0.413320in}}%
\pgfpathlineto{\pgfqpoint{4.014186in}{0.413320in}}%
\pgfpathlineto{\pgfqpoint{4.011394in}{0.413320in}}%
\pgfpathlineto{\pgfqpoint{4.008699in}{0.413320in}}%
\pgfpathlineto{\pgfqpoint{4.006034in}{0.413320in}}%
\pgfpathlineto{\pgfqpoint{4.003348in}{0.413320in}}%
\pgfpathlineto{\pgfqpoint{4.000674in}{0.413320in}}%
\pgfpathlineto{\pgfqpoint{3.997990in}{0.413320in}}%
\pgfpathlineto{\pgfqpoint{3.995417in}{0.413320in}}%
\pgfpathlineto{\pgfqpoint{3.992642in}{0.413320in}}%
\pgfpathlineto{\pgfqpoint{3.990055in}{0.413320in}}%
\pgfpathlineto{\pgfqpoint{3.987270in}{0.413320in}}%
\pgfpathlineto{\pgfqpoint{3.984714in}{0.413320in}}%
\pgfpathlineto{\pgfqpoint{3.981929in}{0.413320in}}%
\pgfpathlineto{\pgfqpoint{3.979389in}{0.413320in}}%
\pgfpathlineto{\pgfqpoint{3.976563in}{0.413320in}}%
\pgfpathlineto{\pgfqpoint{3.973885in}{0.413320in}}%
\pgfpathlineto{\pgfqpoint{3.971250in}{0.413320in}}%
\pgfpathlineto{\pgfqpoint{3.968523in}{0.413320in}}%
\pgfpathlineto{\pgfqpoint{3.966013in}{0.413320in}}%
\pgfpathlineto{\pgfqpoint{3.963176in}{0.413320in}}%
\pgfpathlineto{\pgfqpoint{3.960635in}{0.413320in}}%
\pgfpathlineto{\pgfqpoint{3.957823in}{0.413320in}}%
\pgfpathlineto{\pgfqpoint{3.955211in}{0.413320in}}%
\pgfpathlineto{\pgfqpoint{3.952464in}{0.413320in}}%
\pgfpathlineto{\pgfqpoint{3.949894in}{0.413320in}}%
\pgfpathlineto{\pgfqpoint{3.947101in}{0.413320in}}%
\pgfpathlineto{\pgfqpoint{3.944431in}{0.413320in}}%
\pgfpathlineto{\pgfqpoint{3.941778in}{0.413320in}}%
\pgfpathlineto{\pgfqpoint{3.939075in}{0.413320in}}%
\pgfpathlineto{\pgfqpoint{3.936395in}{0.413320in}}%
\pgfpathlineto{\pgfqpoint{3.933714in}{0.413320in}}%
\pgfpathlineto{\pgfqpoint{3.931202in}{0.413320in}}%
\pgfpathlineto{\pgfqpoint{3.928347in}{0.413320in}}%
\pgfpathlineto{\pgfqpoint{3.925778in}{0.413320in}}%
\pgfpathlineto{\pgfqpoint{3.923005in}{0.413320in}}%
\pgfpathlineto{\pgfqpoint{3.920412in}{0.413320in}}%
\pgfpathlineto{\pgfqpoint{3.917646in}{0.413320in}}%
\pgfpathlineto{\pgfqpoint{3.915107in}{0.413320in}}%
\pgfpathlineto{\pgfqpoint{3.912296in}{0.413320in}}%
\pgfpathlineto{\pgfqpoint{3.909602in}{0.413320in}}%
\pgfpathlineto{\pgfqpoint{3.906918in}{0.413320in}}%
\pgfpathlineto{\pgfqpoint{3.904252in}{0.413320in}}%
\pgfpathlineto{\pgfqpoint{3.901573in}{0.413320in}}%
\pgfpathlineto{\pgfqpoint{3.898891in}{0.413320in}}%
\pgfpathlineto{\pgfqpoint{3.896345in}{0.413320in}}%
\pgfpathlineto{\pgfqpoint{3.893541in}{0.413320in}}%
\pgfpathlineto{\pgfqpoint{3.890926in}{0.413320in}}%
\pgfpathlineto{\pgfqpoint{3.888188in}{0.413320in}}%
\pgfpathlineto{\pgfqpoint{3.885621in}{0.413320in}}%
\pgfpathlineto{\pgfqpoint{3.882850in}{0.413320in}}%
\pgfpathlineto{\pgfqpoint{3.880237in}{0.413320in}}%
\pgfpathlineto{\pgfqpoint{3.877466in}{0.413320in}}%
\pgfpathlineto{\pgfqpoint{3.874790in}{0.413320in}}%
\pgfpathlineto{\pgfqpoint{3.872114in}{0.413320in}}%
\pgfpathlineto{\pgfqpoint{3.869435in}{0.413320in}}%
\pgfpathlineto{\pgfqpoint{3.866815in}{0.413320in}}%
\pgfpathlineto{\pgfqpoint{3.864073in}{0.413320in}}%
\pgfpathlineto{\pgfqpoint{3.861561in}{0.413320in}}%
\pgfpathlineto{\pgfqpoint{3.858720in}{0.413320in}}%
\pgfpathlineto{\pgfqpoint{3.856100in}{0.413320in}}%
\pgfpathlineto{\pgfqpoint{3.853358in}{0.413320in}}%
\pgfpathlineto{\pgfqpoint{3.850814in}{0.413320in}}%
\pgfpathlineto{\pgfqpoint{3.848005in}{0.413320in}}%
\pgfpathlineto{\pgfqpoint{3.845329in}{0.413320in}}%
\pgfpathlineto{\pgfqpoint{3.842641in}{0.413320in}}%
\pgfpathlineto{\pgfqpoint{3.839960in}{0.413320in}}%
\pgfpathlineto{\pgfqpoint{3.837286in}{0.413320in}}%
\pgfpathlineto{\pgfqpoint{3.834616in}{0.413320in}}%
\pgfpathlineto{\pgfqpoint{3.832053in}{0.413320in}}%
\pgfpathlineto{\pgfqpoint{3.829252in}{0.413320in}}%
\pgfpathlineto{\pgfqpoint{3.826679in}{0.413320in}}%
\pgfpathlineto{\pgfqpoint{3.823903in}{0.413320in}}%
\pgfpathlineto{\pgfqpoint{3.821315in}{0.413320in}}%
\pgfpathlineto{\pgfqpoint{3.818546in}{0.413320in}}%
\pgfpathlineto{\pgfqpoint{3.815983in}{0.413320in}}%
\pgfpathlineto{\pgfqpoint{3.813172in}{0.413320in}}%
\pgfpathlineto{\pgfqpoint{3.810510in}{0.413320in}}%
\pgfpathlineto{\pgfqpoint{3.807832in}{0.413320in}}%
\pgfpathlineto{\pgfqpoint{3.805145in}{0.413320in}}%
\pgfpathlineto{\pgfqpoint{3.802569in}{0.413320in}}%
\pgfpathlineto{\pgfqpoint{3.799797in}{0.413320in}}%
\pgfpathlineto{\pgfqpoint{3.797265in}{0.413320in}}%
\pgfpathlineto{\pgfqpoint{3.794435in}{0.413320in}}%
\pgfpathlineto{\pgfqpoint{3.791897in}{0.413320in}}%
\pgfpathlineto{\pgfqpoint{3.789084in}{0.413320in}}%
\pgfpathlineto{\pgfqpoint{3.786504in}{0.413320in}}%
\pgfpathlineto{\pgfqpoint{3.783725in}{0.413320in}}%
\pgfpathlineto{\pgfqpoint{3.781046in}{0.413320in}}%
\pgfpathlineto{\pgfqpoint{3.778370in}{0.413320in}}%
\pgfpathlineto{\pgfqpoint{3.775691in}{0.413320in}}%
\pgfpathlineto{\pgfqpoint{3.773014in}{0.413320in}}%
\pgfpathlineto{\pgfqpoint{3.770323in}{0.413320in}}%
\pgfpathlineto{\pgfqpoint{3.767782in}{0.413320in}}%
\pgfpathlineto{\pgfqpoint{3.764966in}{0.413320in}}%
\pgfpathlineto{\pgfqpoint{3.762389in}{0.413320in}}%
\pgfpathlineto{\pgfqpoint{3.759622in}{0.413320in}}%
\pgfpathlineto{\pgfqpoint{3.757065in}{0.413320in}}%
\pgfpathlineto{\pgfqpoint{3.754265in}{0.413320in}}%
\pgfpathlineto{\pgfqpoint{3.751728in}{0.413320in}}%
\pgfpathlineto{\pgfqpoint{3.748903in}{0.413320in}}%
\pgfpathlineto{\pgfqpoint{3.746229in}{0.413320in}}%
\pgfpathlineto{\pgfqpoint{3.743548in}{0.413320in}}%
\pgfpathlineto{\pgfqpoint{3.740874in}{0.413320in}}%
\pgfpathlineto{\pgfqpoint{3.738194in}{0.413320in}}%
\pgfpathlineto{\pgfqpoint{3.735509in}{0.413320in}}%
\pgfpathlineto{\pgfqpoint{3.732950in}{0.413320in}}%
\pgfpathlineto{\pgfqpoint{3.730158in}{0.413320in}}%
\pgfpathlineto{\pgfqpoint{3.727581in}{0.413320in}}%
\pgfpathlineto{\pgfqpoint{3.724804in}{0.413320in}}%
\pgfpathlineto{\pgfqpoint{3.722228in}{0.413320in}}%
\pgfpathlineto{\pgfqpoint{3.719446in}{0.413320in}}%
\pgfpathlineto{\pgfqpoint{3.716875in}{0.413320in}}%
\pgfpathlineto{\pgfqpoint{3.714086in}{0.413320in}}%
\pgfpathlineto{\pgfqpoint{3.711410in}{0.413320in}}%
\pgfpathlineto{\pgfqpoint{3.708729in}{0.413320in}}%
\pgfpathlineto{\pgfqpoint{3.706053in}{0.413320in}}%
\pgfpathlineto{\pgfqpoint{3.703460in}{0.413320in}}%
\pgfpathlineto{\pgfqpoint{3.700684in}{0.413320in}}%
\pgfpathlineto{\pgfqpoint{3.698125in}{0.413320in}}%
\pgfpathlineto{\pgfqpoint{3.695331in}{0.413320in}}%
\pgfpathlineto{\pgfqpoint{3.692765in}{0.413320in}}%
\pgfpathlineto{\pgfqpoint{3.689983in}{0.413320in}}%
\pgfpathlineto{\pgfqpoint{3.687442in}{0.413320in}}%
\pgfpathlineto{\pgfqpoint{3.684620in}{0.413320in}}%
\pgfpathlineto{\pgfqpoint{3.681948in}{0.413320in}}%
\pgfpathlineto{\pgfqpoint{3.679273in}{0.413320in}}%
\pgfpathlineto{\pgfqpoint{3.676591in}{0.413320in}}%
\pgfpathlineto{\pgfqpoint{3.673911in}{0.413320in}}%
\pgfpathlineto{\pgfqpoint{3.671232in}{0.413320in}}%
\pgfpathlineto{\pgfqpoint{3.668665in}{0.413320in}}%
\pgfpathlineto{\pgfqpoint{3.665864in}{0.413320in}}%
\pgfpathlineto{\pgfqpoint{3.663276in}{0.413320in}}%
\pgfpathlineto{\pgfqpoint{3.660515in}{0.413320in}}%
\pgfpathlineto{\pgfqpoint{3.657917in}{0.413320in}}%
\pgfpathlineto{\pgfqpoint{3.655165in}{0.413320in}}%
\pgfpathlineto{\pgfqpoint{3.652628in}{0.413320in}}%
\pgfpathlineto{\pgfqpoint{3.649837in}{0.413320in}}%
\pgfpathlineto{\pgfqpoint{3.647130in}{0.413320in}}%
\pgfpathlineto{\pgfqpoint{3.644452in}{0.413320in}}%
\pgfpathlineto{\pgfqpoint{3.641773in}{0.413320in}}%
\pgfpathlineto{\pgfqpoint{3.639207in}{0.413320in}}%
\pgfpathlineto{\pgfqpoint{3.636413in}{0.413320in}}%
\pgfpathlineto{\pgfqpoint{3.633858in}{0.413320in}}%
\pgfpathlineto{\pgfqpoint{3.631058in}{0.413320in}}%
\pgfpathlineto{\pgfqpoint{3.628460in}{0.413320in}}%
\pgfpathlineto{\pgfqpoint{3.625689in}{0.413320in}}%
\pgfpathlineto{\pgfqpoint{3.623165in}{0.413320in}}%
\pgfpathlineto{\pgfqpoint{3.620345in}{0.413320in}}%
\pgfpathlineto{\pgfqpoint{3.617667in}{0.413320in}}%
\pgfpathlineto{\pgfqpoint{3.614982in}{0.413320in}}%
\pgfpathlineto{\pgfqpoint{3.612311in}{0.413320in}}%
\pgfpathlineto{\pgfqpoint{3.609632in}{0.413320in}}%
\pgfpathlineto{\pgfqpoint{3.606951in}{0.413320in}}%
\pgfpathlineto{\pgfqpoint{3.604387in}{0.413320in}}%
\pgfpathlineto{\pgfqpoint{3.601590in}{0.413320in}}%
\pgfpathlineto{\pgfqpoint{3.598998in}{0.413320in}}%
\pgfpathlineto{\pgfqpoint{3.596240in}{0.413320in}}%
\pgfpathlineto{\pgfqpoint{3.593620in}{0.413320in}}%
\pgfpathlineto{\pgfqpoint{3.590883in}{0.413320in}}%
\pgfpathlineto{\pgfqpoint{3.588258in}{0.413320in}}%
\pgfpathlineto{\pgfqpoint{3.585532in}{0.413320in}}%
\pgfpathlineto{\pgfqpoint{3.582851in}{0.413320in}}%
\pgfpathlineto{\pgfqpoint{3.580191in}{0.413320in}}%
\pgfpathlineto{\pgfqpoint{3.577487in}{0.413320in}}%
\pgfpathlineto{\pgfqpoint{3.574814in}{0.413320in}}%
\pgfpathlineto{\pgfqpoint{3.572126in}{0.413320in}}%
\pgfpathlineto{\pgfqpoint{3.569584in}{0.413320in}}%
\pgfpathlineto{\pgfqpoint{3.566774in}{0.413320in}}%
\pgfpathlineto{\pgfqpoint{3.564188in}{0.413320in}}%
\pgfpathlineto{\pgfqpoint{3.561420in}{0.413320in}}%
\pgfpathlineto{\pgfqpoint{3.558853in}{0.413320in}}%
\pgfpathlineto{\pgfqpoint{3.556061in}{0.413320in}}%
\pgfpathlineto{\pgfqpoint{3.553498in}{0.413320in}}%
\pgfpathlineto{\pgfqpoint{3.550713in}{0.413320in}}%
\pgfpathlineto{\pgfqpoint{3.548029in}{0.413320in}}%
\pgfpathlineto{\pgfqpoint{3.545349in}{0.413320in}}%
\pgfpathlineto{\pgfqpoint{3.542656in}{0.413320in}}%
\pgfpathlineto{\pgfqpoint{3.540093in}{0.413320in}}%
\pgfpathlineto{\pgfqpoint{3.537309in}{0.413320in}}%
\pgfpathlineto{\pgfqpoint{3.534783in}{0.413320in}}%
\pgfpathlineto{\pgfqpoint{3.531955in}{0.413320in}}%
\pgfpathlineto{\pgfqpoint{3.529327in}{0.413320in}}%
\pgfpathlineto{\pgfqpoint{3.526601in}{0.413320in}}%
\pgfpathlineto{\pgfqpoint{3.524041in}{0.413320in}}%
\pgfpathlineto{\pgfqpoint{3.521244in}{0.413320in}}%
\pgfpathlineto{\pgfqpoint{3.518565in}{0.413320in}}%
\pgfpathlineto{\pgfqpoint{3.515884in}{0.413320in}}%
\pgfpathlineto{\pgfqpoint{3.513209in}{0.413320in}}%
\pgfpathlineto{\pgfqpoint{3.510533in}{0.413320in}}%
\pgfpathlineto{\pgfqpoint{3.507840in}{0.413320in}}%
\pgfpathlineto{\pgfqpoint{3.505262in}{0.413320in}}%
\pgfpathlineto{\pgfqpoint{3.502488in}{0.413320in}}%
\pgfpathlineto{\pgfqpoint{3.499909in}{0.413320in}}%
\pgfpathlineto{\pgfqpoint{3.497139in}{0.413320in}}%
\pgfpathlineto{\pgfqpoint{3.494581in}{0.413320in}}%
\pgfpathlineto{\pgfqpoint{3.491783in}{0.413320in}}%
\pgfpathlineto{\pgfqpoint{3.489223in}{0.413320in}}%
\pgfpathlineto{\pgfqpoint{3.486442in}{0.413320in}}%
\pgfpathlineto{\pgfqpoint{3.483744in}{0.413320in}}%
\pgfpathlineto{\pgfqpoint{3.481072in}{0.413320in}}%
\pgfpathlineto{\pgfqpoint{3.478378in}{0.413320in}}%
\pgfpathlineto{\pgfqpoint{3.475821in}{0.413320in}}%
\pgfpathlineto{\pgfqpoint{3.473021in}{0.413320in}}%
\pgfpathlineto{\pgfqpoint{3.470466in}{0.413320in}}%
\pgfpathlineto{\pgfqpoint{3.467678in}{0.413320in}}%
\pgfpathlineto{\pgfqpoint{3.465072in}{0.413320in}}%
\pgfpathlineto{\pgfqpoint{3.462321in}{0.413320in}}%
\pgfpathlineto{\pgfqpoint{3.459695in}{0.413320in}}%
\pgfpathlineto{\pgfqpoint{3.456960in}{0.413320in}}%
\pgfpathlineto{\pgfqpoint{3.454285in}{0.413320in}}%
\pgfpathlineto{\pgfqpoint{3.451597in}{0.413320in}}%
\pgfpathlineto{\pgfqpoint{3.448926in}{0.413320in}}%
\pgfpathlineto{\pgfqpoint{3.446257in}{0.413320in}}%
\pgfpathlineto{\pgfqpoint{3.443574in}{0.413320in}}%
\pgfpathlineto{\pgfqpoint{3.440996in}{0.413320in}}%
\pgfpathlineto{\pgfqpoint{3.438210in}{0.413320in}}%
\pgfpathlineto{\pgfqpoint{3.435635in}{0.413320in}}%
\pgfpathlineto{\pgfqpoint{3.432851in}{0.413320in}}%
\pgfpathlineto{\pgfqpoint{3.430313in}{0.413320in}}%
\pgfpathlineto{\pgfqpoint{3.427501in}{0.413320in}}%
\pgfpathlineto{\pgfqpoint{3.424887in}{0.413320in}}%
\pgfpathlineto{\pgfqpoint{3.422142in}{0.413320in}}%
\pgfpathlineto{\pgfqpoint{3.419455in}{0.413320in}}%
\pgfpathlineto{\pgfqpoint{3.416780in}{0.413320in}}%
\pgfpathlineto{\pgfqpoint{3.414109in}{0.413320in}}%
\pgfpathlineto{\pgfqpoint{3.411431in}{0.413320in}}%
\pgfpathlineto{\pgfqpoint{3.408752in}{0.413320in}}%
\pgfpathlineto{\pgfqpoint{3.406202in}{0.413320in}}%
\pgfpathlineto{\pgfqpoint{3.403394in}{0.413320in}}%
\pgfpathlineto{\pgfqpoint{3.400783in}{0.413320in}}%
\pgfpathlineto{\pgfqpoint{3.398037in}{0.413320in}}%
\pgfpathlineto{\pgfqpoint{3.395461in}{0.413320in}}%
\pgfpathlineto{\pgfqpoint{3.392681in}{0.413320in}}%
\pgfpathlineto{\pgfqpoint{3.390102in}{0.413320in}}%
\pgfpathlineto{\pgfqpoint{3.387309in}{0.413320in}}%
\pgfpathlineto{\pgfqpoint{3.384647in}{0.413320in}}%
\pgfpathlineto{\pgfqpoint{3.381959in}{0.413320in}}%
\pgfpathlineto{\pgfqpoint{3.379290in}{0.413320in}}%
\pgfpathlineto{\pgfqpoint{3.376735in}{0.413320in}}%
\pgfpathlineto{\pgfqpoint{3.373921in}{0.413320in}}%
\pgfpathlineto{\pgfqpoint{3.371357in}{0.413320in}}%
\pgfpathlineto{\pgfqpoint{3.368577in}{0.413320in}}%
\pgfpathlineto{\pgfqpoint{3.365996in}{0.413320in}}%
\pgfpathlineto{\pgfqpoint{3.363221in}{0.413320in}}%
\pgfpathlineto{\pgfqpoint{3.360620in}{0.413320in}}%
\pgfpathlineto{\pgfqpoint{3.357862in}{0.413320in}}%
\pgfpathlineto{\pgfqpoint{3.355177in}{0.413320in}}%
\pgfpathlineto{\pgfqpoint{3.352505in}{0.413320in}}%
\pgfpathlineto{\pgfqpoint{3.349828in}{0.413320in}}%
\pgfpathlineto{\pgfqpoint{3.347139in}{0.413320in}}%
\pgfpathlineto{\pgfqpoint{3.344468in}{0.413320in}}%
\pgfpathlineto{\pgfqpoint{3.341893in}{0.413320in}}%
\pgfpathlineto{\pgfqpoint{3.339101in}{0.413320in}}%
\pgfpathlineto{\pgfqpoint{3.336541in}{0.413320in}}%
\pgfpathlineto{\pgfqpoint{3.333758in}{0.413320in}}%
\pgfpathlineto{\pgfqpoint{3.331183in}{0.413320in}}%
\pgfpathlineto{\pgfqpoint{3.328401in}{0.413320in}}%
\pgfpathlineto{\pgfqpoint{3.325860in}{0.413320in}}%
\pgfpathlineto{\pgfqpoint{3.323049in}{0.413320in}}%
\pgfpathlineto{\pgfqpoint{3.320366in}{0.413320in}}%
\pgfpathlineto{\pgfqpoint{3.317688in}{0.413320in}}%
\pgfpathlineto{\pgfqpoint{3.315008in}{0.413320in}}%
\pgfpathlineto{\pgfqpoint{3.312480in}{0.413320in}}%
\pgfpathlineto{\pgfqpoint{3.309652in}{0.413320in}}%
\pgfpathlineto{\pgfqpoint{3.307104in}{0.413320in}}%
\pgfpathlineto{\pgfqpoint{3.304295in}{0.413320in}}%
\pgfpathlineto{\pgfqpoint{3.301719in}{0.413320in}}%
\pgfpathlineto{\pgfqpoint{3.298937in}{0.413320in}}%
\pgfpathlineto{\pgfqpoint{3.296376in}{0.413320in}}%
\pgfpathlineto{\pgfqpoint{3.293574in}{0.413320in}}%
\pgfpathlineto{\pgfqpoint{3.290890in}{0.413320in}}%
\pgfpathlineto{\pgfqpoint{3.288225in}{0.413320in}}%
\pgfpathlineto{\pgfqpoint{3.285534in}{0.413320in}}%
\pgfpathlineto{\pgfqpoint{3.282870in}{0.413320in}}%
\pgfpathlineto{\pgfqpoint{3.280189in}{0.413320in}}%
\pgfpathlineto{\pgfqpoint{3.277603in}{0.413320in}}%
\pgfpathlineto{\pgfqpoint{3.274831in}{0.413320in}}%
\pgfpathlineto{\pgfqpoint{3.272254in}{0.413320in}}%
\pgfpathlineto{\pgfqpoint{3.269478in}{0.413320in}}%
\pgfpathlineto{\pgfqpoint{3.266849in}{0.413320in}}%
\pgfpathlineto{\pgfqpoint{3.264119in}{0.413320in}}%
\pgfpathlineto{\pgfqpoint{3.261594in}{0.413320in}}%
\pgfpathlineto{\pgfqpoint{3.258784in}{0.413320in}}%
\pgfpathlineto{\pgfqpoint{3.256083in}{0.413320in}}%
\pgfpathlineto{\pgfqpoint{3.253404in}{0.413320in}}%
\pgfpathlineto{\pgfqpoint{3.250716in}{0.413320in}}%
\pgfpathlineto{\pgfqpoint{3.248049in}{0.413320in}}%
\pgfpathlineto{\pgfqpoint{3.245363in}{0.413320in}}%
\pgfpathlineto{\pgfqpoint{3.242807in}{0.413320in}}%
\pgfpathlineto{\pgfqpoint{3.240010in}{0.413320in}}%
\pgfpathlineto{\pgfqpoint{3.237411in}{0.413320in}}%
\pgfpathlineto{\pgfqpoint{3.234658in}{0.413320in}}%
\pgfpathlineto{\pgfqpoint{3.232069in}{0.413320in}}%
\pgfpathlineto{\pgfqpoint{3.229310in}{0.413320in}}%
\pgfpathlineto{\pgfqpoint{3.226609in}{0.413320in}}%
\pgfpathlineto{\pgfqpoint{3.223942in}{0.413320in}}%
\pgfpathlineto{\pgfqpoint{3.221255in}{0.413320in}}%
\pgfpathlineto{\pgfqpoint{3.218586in}{0.413320in}}%
\pgfpathlineto{\pgfqpoint{3.215908in}{0.413320in}}%
\pgfpathlineto{\pgfqpoint{3.213342in}{0.413320in}}%
\pgfpathlineto{\pgfqpoint{3.210545in}{0.413320in}}%
\pgfpathlineto{\pgfqpoint{3.207984in}{0.413320in}}%
\pgfpathlineto{\pgfqpoint{3.205195in}{0.413320in}}%
\pgfpathlineto{\pgfqpoint{3.202562in}{0.413320in}}%
\pgfpathlineto{\pgfqpoint{3.199823in}{0.413320in}}%
\pgfpathlineto{\pgfqpoint{3.197226in}{0.413320in}}%
\pgfpathlineto{\pgfqpoint{3.194508in}{0.413320in}}%
\pgfpathlineto{\pgfqpoint{3.191796in}{0.413320in}}%
\pgfpathlineto{\pgfqpoint{3.189117in}{0.413320in}}%
\pgfpathlineto{\pgfqpoint{3.186440in}{0.413320in}}%
\pgfpathlineto{\pgfqpoint{3.183760in}{0.413320in}}%
\pgfpathlineto{\pgfqpoint{3.181089in}{0.413320in}}%
\pgfpathlineto{\pgfqpoint{3.178525in}{0.413320in}}%
\pgfpathlineto{\pgfqpoint{3.175724in}{0.413320in}}%
\pgfpathlineto{\pgfqpoint{3.173142in}{0.413320in}}%
\pgfpathlineto{\pgfqpoint{3.170375in}{0.413320in}}%
\pgfpathlineto{\pgfqpoint{3.167776in}{0.413320in}}%
\pgfpathlineto{\pgfqpoint{3.165019in}{0.413320in}}%
\pgfpathlineto{\pgfqpoint{3.162474in}{0.413320in}}%
\pgfpathlineto{\pgfqpoint{3.159675in}{0.413320in}}%
\pgfpathlineto{\pgfqpoint{3.156981in}{0.413320in}}%
\pgfpathlineto{\pgfqpoint{3.154327in}{0.413320in}}%
\pgfpathlineto{\pgfqpoint{3.151612in}{0.413320in}}%
\pgfpathlineto{\pgfqpoint{3.149057in}{0.413320in}}%
\pgfpathlineto{\pgfqpoint{3.146271in}{0.413320in}}%
\pgfpathlineto{\pgfqpoint{3.143740in}{0.413320in}}%
\pgfpathlineto{\pgfqpoint{3.140913in}{0.413320in}}%
\pgfpathlineto{\pgfqpoint{3.138375in}{0.413320in}}%
\pgfpathlineto{\pgfqpoint{3.135550in}{0.413320in}}%
\pgfpathlineto{\pgfqpoint{3.132946in}{0.413320in}}%
\pgfpathlineto{\pgfqpoint{3.130199in}{0.413320in}}%
\pgfpathlineto{\pgfqpoint{3.127512in}{0.413320in}}%
\pgfpathlineto{\pgfqpoint{3.124842in}{0.413320in}}%
\pgfpathlineto{\pgfqpoint{3.122163in}{0.413320in}}%
\pgfpathlineto{\pgfqpoint{3.119487in}{0.413320in}}%
\pgfpathlineto{\pgfqpoint{3.116807in}{0.413320in}}%
\pgfpathlineto{\pgfqpoint{3.114242in}{0.413320in}}%
\pgfpathlineto{\pgfqpoint{3.111451in}{0.413320in}}%
\pgfpathlineto{\pgfqpoint{3.108896in}{0.413320in}}%
\pgfpathlineto{\pgfqpoint{3.106094in}{0.413320in}}%
\pgfpathlineto{\pgfqpoint{3.103508in}{0.413320in}}%
\pgfpathlineto{\pgfqpoint{3.100737in}{0.413320in}}%
\pgfpathlineto{\pgfqpoint{3.098163in}{0.413320in}}%
\pgfpathlineto{\pgfqpoint{3.095388in}{0.413320in}}%
\pgfpathlineto{\pgfqpoint{3.092699in}{0.413320in}}%
\pgfpathlineto{\pgfqpoint{3.090023in}{0.413320in}}%
\pgfpathlineto{\pgfqpoint{3.087343in}{0.413320in}}%
\pgfpathlineto{\pgfqpoint{3.084671in}{0.413320in}}%
\pgfpathlineto{\pgfqpoint{3.081990in}{0.413320in}}%
\pgfpathlineto{\pgfqpoint{3.079381in}{0.413320in}}%
\pgfpathlineto{\pgfqpoint{3.076631in}{0.413320in}}%
\pgfpathlineto{\pgfqpoint{3.074056in}{0.413320in}}%
\pgfpathlineto{\pgfqpoint{3.071266in}{0.413320in}}%
\pgfpathlineto{\pgfqpoint{3.068709in}{0.413320in}}%
\pgfpathlineto{\pgfqpoint{3.065916in}{0.413320in}}%
\pgfpathlineto{\pgfqpoint{3.063230in}{0.413320in}}%
\pgfpathlineto{\pgfqpoint{3.060561in}{0.413320in}}%
\pgfpathlineto{\pgfqpoint{3.057884in}{0.413320in}}%
\pgfpathlineto{\pgfqpoint{3.055202in}{0.413320in}}%
\pgfpathlineto{\pgfqpoint{3.052526in}{0.413320in}}%
\pgfpathlineto{\pgfqpoint{3.049988in}{0.413320in}}%
\pgfpathlineto{\pgfqpoint{3.047157in}{0.413320in}}%
\pgfpathlineto{\pgfqpoint{3.044568in}{0.413320in}}%
\pgfpathlineto{\pgfqpoint{3.041813in}{0.413320in}}%
\pgfpathlineto{\pgfqpoint{3.039262in}{0.413320in}}%
\pgfpathlineto{\pgfqpoint{3.036456in}{0.413320in}}%
\pgfpathlineto{\pgfqpoint{3.033921in}{0.413320in}}%
\pgfpathlineto{\pgfqpoint{3.031091in}{0.413320in}}%
\pgfpathlineto{\pgfqpoint{3.028412in}{0.413320in}}%
\pgfpathlineto{\pgfqpoint{3.025803in}{0.413320in}}%
\pgfpathlineto{\pgfqpoint{3.023058in}{0.413320in}}%
\pgfpathlineto{\pgfqpoint{3.020382in}{0.413320in}}%
\pgfpathlineto{\pgfqpoint{3.017707in}{0.413320in}}%
\pgfpathlineto{\pgfqpoint{3.015097in}{0.413320in}}%
\pgfpathlineto{\pgfqpoint{3.012351in}{0.413320in}}%
\pgfpathlineto{\pgfqpoint{3.009784in}{0.413320in}}%
\pgfpathlineto{\pgfqpoint{3.006993in}{0.413320in}}%
\pgfpathlineto{\pgfqpoint{3.004419in}{0.413320in}}%
\pgfpathlineto{\pgfqpoint{3.001635in}{0.413320in}}%
\pgfpathlineto{\pgfqpoint{2.999103in}{0.413320in}}%
\pgfpathlineto{\pgfqpoint{2.996300in}{0.413320in}}%
\pgfpathlineto{\pgfqpoint{2.993595in}{0.413320in}}%
\pgfpathlineto{\pgfqpoint{2.990978in}{0.413320in}}%
\pgfpathlineto{\pgfqpoint{2.988238in}{0.413320in}}%
\pgfpathlineto{\pgfqpoint{2.985666in}{0.413320in}}%
\pgfpathlineto{\pgfqpoint{2.982885in}{0.413320in}}%
\pgfpathlineto{\pgfqpoint{2.980341in}{0.413320in}}%
\pgfpathlineto{\pgfqpoint{2.977517in}{0.413320in}}%
\pgfpathlineto{\pgfqpoint{2.974972in}{0.413320in}}%
\pgfpathlineto{\pgfqpoint{2.972177in}{0.413320in}}%
\pgfpathlineto{\pgfqpoint{2.969599in}{0.413320in}}%
\pgfpathlineto{\pgfqpoint{2.966812in}{0.413320in}}%
\pgfpathlineto{\pgfqpoint{2.964127in}{0.413320in}}%
\pgfpathlineto{\pgfqpoint{2.961460in}{0.413320in}}%
\pgfpathlineto{\pgfqpoint{2.958782in}{0.413320in}}%
\pgfpathlineto{\pgfqpoint{2.956103in}{0.413320in}}%
\pgfpathlineto{\pgfqpoint{2.953422in}{0.413320in}}%
\pgfpathlineto{\pgfqpoint{2.950884in}{0.413320in}}%
\pgfpathlineto{\pgfqpoint{2.948068in}{0.413320in}}%
\pgfpathlineto{\pgfqpoint{2.945461in}{0.413320in}}%
\pgfpathlineto{\pgfqpoint{2.942711in}{0.413320in}}%
\pgfpathlineto{\pgfqpoint{2.940120in}{0.413320in}}%
\pgfpathlineto{\pgfqpoint{2.937352in}{0.413320in}}%
\pgfpathlineto{\pgfqpoint{2.934759in}{0.413320in}}%
\pgfpathlineto{\pgfqpoint{2.932033in}{0.413320in}}%
\pgfpathlineto{\pgfqpoint{2.929321in}{0.413320in}}%
\pgfpathlineto{\pgfqpoint{2.926655in}{0.413320in}}%
\pgfpathlineto{\pgfqpoint{2.923963in}{0.413320in}}%
\pgfpathlineto{\pgfqpoint{2.921363in}{0.413320in}}%
\pgfpathlineto{\pgfqpoint{2.918606in}{0.413320in}}%
\pgfpathlineto{\pgfqpoint{2.916061in}{0.413320in}}%
\pgfpathlineto{\pgfqpoint{2.913243in}{0.413320in}}%
\pgfpathlineto{\pgfqpoint{2.910631in}{0.413320in}}%
\pgfpathlineto{\pgfqpoint{2.907882in}{0.413320in}}%
\pgfpathlineto{\pgfqpoint{2.905341in}{0.413320in}}%
\pgfpathlineto{\pgfqpoint{2.902535in}{0.413320in}}%
\pgfpathlineto{\pgfqpoint{2.899858in}{0.413320in}}%
\pgfpathlineto{\pgfqpoint{2.897179in}{0.413320in}}%
\pgfpathlineto{\pgfqpoint{2.894487in}{0.413320in}}%
\pgfpathlineto{\pgfqpoint{2.891809in}{0.413320in}}%
\pgfpathlineto{\pgfqpoint{2.889145in}{0.413320in}}%
\pgfpathlineto{\pgfqpoint{2.886578in}{0.413320in}}%
\pgfpathlineto{\pgfqpoint{2.883780in}{0.413320in}}%
\pgfpathlineto{\pgfqpoint{2.881254in}{0.413320in}}%
\pgfpathlineto{\pgfqpoint{2.878431in}{0.413320in}}%
\pgfpathlineto{\pgfqpoint{2.875882in}{0.413320in}}%
\pgfpathlineto{\pgfqpoint{2.873074in}{0.413320in}}%
\pgfpathlineto{\pgfqpoint{2.870475in}{0.413320in}}%
\pgfpathlineto{\pgfqpoint{2.867713in}{0.413320in}}%
\pgfpathlineto{\pgfqpoint{2.865031in}{0.413320in}}%
\pgfpathlineto{\pgfqpoint{2.862402in}{0.413320in}}%
\pgfpathlineto{\pgfqpoint{2.859668in}{0.413320in}}%
\pgfpathlineto{\pgfqpoint{2.857003in}{0.413320in}}%
\pgfpathlineto{\pgfqpoint{2.854325in}{0.413320in}}%
\pgfpathlineto{\pgfqpoint{2.851793in}{0.413320in}}%
\pgfpathlineto{\pgfqpoint{2.848960in}{0.413320in}}%
\pgfpathlineto{\pgfqpoint{2.846408in}{0.413320in}}%
\pgfpathlineto{\pgfqpoint{2.843611in}{0.413320in}}%
\pgfpathlineto{\pgfqpoint{2.841055in}{0.413320in}}%
\pgfpathlineto{\pgfqpoint{2.838254in}{0.413320in}}%
\pgfpathlineto{\pgfqpoint{2.835698in}{0.413320in}}%
\pgfpathlineto{\pgfqpoint{2.832894in}{0.413320in}}%
\pgfpathlineto{\pgfqpoint{2.830219in}{0.413320in}}%
\pgfpathlineto{\pgfqpoint{2.827567in}{0.413320in}}%
\pgfpathlineto{\pgfqpoint{2.824851in}{0.413320in}}%
\pgfpathlineto{\pgfqpoint{2.822303in}{0.413320in}}%
\pgfpathlineto{\pgfqpoint{2.819506in}{0.413320in}}%
\pgfpathlineto{\pgfqpoint{2.816867in}{0.413320in}}%
\pgfpathlineto{\pgfqpoint{2.814141in}{0.413320in}}%
\pgfpathlineto{\pgfqpoint{2.811597in}{0.413320in}}%
\pgfpathlineto{\pgfqpoint{2.808792in}{0.413320in}}%
\pgfpathlineto{\pgfqpoint{2.806175in}{0.413320in}}%
\pgfpathlineto{\pgfqpoint{2.803435in}{0.413320in}}%
\pgfpathlineto{\pgfqpoint{2.800756in}{0.413320in}}%
\pgfpathlineto{\pgfqpoint{2.798070in}{0.413320in}}%
\pgfpathlineto{\pgfqpoint{2.795398in}{0.413320in}}%
\pgfpathlineto{\pgfqpoint{2.792721in}{0.413320in}}%
\pgfpathlineto{\pgfqpoint{2.790044in}{0.413320in}}%
\pgfpathlineto{\pgfqpoint{2.787468in}{0.413320in}}%
\pgfpathlineto{\pgfqpoint{2.784687in}{0.413320in}}%
\pgfpathlineto{\pgfqpoint{2.782113in}{0.413320in}}%
\pgfpathlineto{\pgfqpoint{2.779330in}{0.413320in}}%
\pgfpathlineto{\pgfqpoint{2.776767in}{0.413320in}}%
\pgfpathlineto{\pgfqpoint{2.773972in}{0.413320in}}%
\pgfpathlineto{\pgfqpoint{2.771373in}{0.413320in}}%
\pgfpathlineto{\pgfqpoint{2.768617in}{0.413320in}}%
\pgfpathlineto{\pgfqpoint{2.765935in}{0.413320in}}%
\pgfpathlineto{\pgfqpoint{2.763253in}{0.413320in}}%
\pgfpathlineto{\pgfqpoint{2.760581in}{0.413320in}}%
\pgfpathlineto{\pgfqpoint{2.758028in}{0.413320in}}%
\pgfpathlineto{\pgfqpoint{2.755224in}{0.413320in}}%
\pgfpathlineto{\pgfqpoint{2.752614in}{0.413320in}}%
\pgfpathlineto{\pgfqpoint{2.749868in}{0.413320in}}%
\pgfpathlineto{\pgfqpoint{2.747260in}{0.413320in}}%
\pgfpathlineto{\pgfqpoint{2.744510in}{0.413320in}}%
\pgfpathlineto{\pgfqpoint{2.741928in}{0.413320in}}%
\pgfpathlineto{\pgfqpoint{2.739155in}{0.413320in}}%
\pgfpathlineto{\pgfqpoint{2.736476in}{0.413320in}}%
\pgfpathlineto{\pgfqpoint{2.733798in}{0.413320in}}%
\pgfpathlineto{\pgfqpoint{2.731119in}{0.413320in}}%
\pgfpathlineto{\pgfqpoint{2.728439in}{0.413320in}}%
\pgfpathlineto{\pgfqpoint{2.725760in}{0.413320in}}%
\pgfpathlineto{\pgfqpoint{2.723211in}{0.413320in}}%
\pgfpathlineto{\pgfqpoint{2.720404in}{0.413320in}}%
\pgfpathlineto{\pgfqpoint{2.717773in}{0.413320in}}%
\pgfpathlineto{\pgfqpoint{2.715036in}{0.413320in}}%
\pgfpathlineto{\pgfqpoint{2.712477in}{0.413320in}}%
\pgfpathlineto{\pgfqpoint{2.709683in}{0.413320in}}%
\pgfpathlineto{\pgfqpoint{2.707125in}{0.413320in}}%
\pgfpathlineto{\pgfqpoint{2.704326in}{0.413320in}}%
\pgfpathlineto{\pgfqpoint{2.701657in}{0.413320in}}%
\pgfpathlineto{\pgfqpoint{2.698968in}{0.413320in}}%
\pgfpathlineto{\pgfqpoint{2.696293in}{0.413320in}}%
\pgfpathlineto{\pgfqpoint{2.693611in}{0.413320in}}%
\pgfpathlineto{\pgfqpoint{2.690940in}{0.413320in}}%
\pgfpathlineto{\pgfqpoint{2.688328in}{0.413320in}}%
\pgfpathlineto{\pgfqpoint{2.685586in}{0.413320in}}%
\pgfpathlineto{\pgfqpoint{2.683009in}{0.413320in}}%
\pgfpathlineto{\pgfqpoint{2.680224in}{0.413320in}}%
\pgfpathlineto{\pgfqpoint{2.677650in}{0.413320in}}%
\pgfpathlineto{\pgfqpoint{2.674873in}{0.413320in}}%
\pgfpathlineto{\pgfqpoint{2.672301in}{0.413320in}}%
\pgfpathlineto{\pgfqpoint{2.669506in}{0.413320in}}%
\pgfpathlineto{\pgfqpoint{2.666836in}{0.413320in}}%
\pgfpathlineto{\pgfqpoint{2.664151in}{0.413320in}}%
\pgfpathlineto{\pgfqpoint{2.661481in}{0.413320in}}%
\pgfpathlineto{\pgfqpoint{2.658942in}{0.413320in}}%
\pgfpathlineto{\pgfqpoint{2.656124in}{0.413320in}}%
\pgfpathlineto{\pgfqpoint{2.653567in}{0.413320in}}%
\pgfpathlineto{\pgfqpoint{2.650767in}{0.413320in}}%
\pgfpathlineto{\pgfqpoint{2.648196in}{0.413320in}}%
\pgfpathlineto{\pgfqpoint{2.645408in}{0.413320in}}%
\pgfpathlineto{\pgfqpoint{2.642827in}{0.413320in}}%
\pgfpathlineto{\pgfqpoint{2.640053in}{0.413320in}}%
\pgfpathlineto{\pgfqpoint{2.637369in}{0.413320in}}%
\pgfpathlineto{\pgfqpoint{2.634700in}{0.413320in}}%
\pgfpathlineto{\pgfqpoint{2.632018in}{0.413320in}}%
\pgfpathlineto{\pgfqpoint{2.629340in}{0.413320in}}%
\pgfpathlineto{\pgfqpoint{2.626653in}{0.413320in}}%
\pgfpathlineto{\pgfqpoint{2.624077in}{0.413320in}}%
\pgfpathlineto{\pgfqpoint{2.621304in}{0.413320in}}%
\pgfpathlineto{\pgfqpoint{2.618773in}{0.413320in}}%
\pgfpathlineto{\pgfqpoint{2.615934in}{0.413320in}}%
\pgfpathlineto{\pgfqpoint{2.613393in}{0.413320in}}%
\pgfpathlineto{\pgfqpoint{2.610588in}{0.413320in}}%
\pgfpathlineto{\pgfqpoint{2.608004in}{0.413320in}}%
\pgfpathlineto{\pgfqpoint{2.605232in}{0.413320in}}%
\pgfpathlineto{\pgfqpoint{2.602557in}{0.413320in}}%
\pgfpathlineto{\pgfqpoint{2.599920in}{0.413320in}}%
\pgfpathlineto{\pgfqpoint{2.597196in}{0.413320in}}%
\pgfpathlineto{\pgfqpoint{2.594630in}{0.413320in}}%
\pgfpathlineto{\pgfqpoint{2.591842in}{0.413320in}}%
\pgfpathlineto{\pgfqpoint{2.589248in}{0.413320in}}%
\pgfpathlineto{\pgfqpoint{2.586484in}{0.413320in}}%
\pgfpathlineto{\pgfqpoint{2.583913in}{0.413320in}}%
\pgfpathlineto{\pgfqpoint{2.581129in}{0.413320in}}%
\pgfpathlineto{\pgfqpoint{2.578567in}{0.413320in}}%
\pgfpathlineto{\pgfqpoint{2.575779in}{0.413320in}}%
\pgfpathlineto{\pgfqpoint{2.573082in}{0.413320in}}%
\pgfpathlineto{\pgfqpoint{2.570411in}{0.413320in}}%
\pgfpathlineto{\pgfqpoint{2.567730in}{0.413320in}}%
\pgfpathlineto{\pgfqpoint{2.565045in}{0.413320in}}%
\pgfpathlineto{\pgfqpoint{2.562375in}{0.413320in}}%
\pgfpathlineto{\pgfqpoint{2.559790in}{0.413320in}}%
\pgfpathlineto{\pgfqpoint{2.557009in}{0.413320in}}%
\pgfpathlineto{\pgfqpoint{2.554493in}{0.413320in}}%
\pgfpathlineto{\pgfqpoint{2.551664in}{0.413320in}}%
\pgfpathlineto{\pgfqpoint{2.549114in}{0.413320in}}%
\pgfpathlineto{\pgfqpoint{2.546310in}{0.413320in}}%
\pgfpathlineto{\pgfqpoint{2.543765in}{0.413320in}}%
\pgfpathlineto{\pgfqpoint{2.540949in}{0.413320in}}%
\pgfpathlineto{\pgfqpoint{2.538274in}{0.413320in}}%
\pgfpathlineto{\pgfqpoint{2.535624in}{0.413320in}}%
\pgfpathlineto{\pgfqpoint{2.532917in}{0.413320in}}%
\pgfpathlineto{\pgfqpoint{2.530234in}{0.413320in}}%
\pgfpathlineto{\pgfqpoint{2.527560in}{0.413320in}}%
\pgfpathlineto{\pgfqpoint{2.524988in}{0.413320in}}%
\pgfpathlineto{\pgfqpoint{2.522197in}{0.413320in}}%
\pgfpathlineto{\pgfqpoint{2.519607in}{0.413320in}}%
\pgfpathlineto{\pgfqpoint{2.516845in}{0.413320in}}%
\pgfpathlineto{\pgfqpoint{2.514268in}{0.413320in}}%
\pgfpathlineto{\pgfqpoint{2.511478in}{0.413320in}}%
\pgfpathlineto{\pgfqpoint{2.508917in}{0.413320in}}%
\pgfpathlineto{\pgfqpoint{2.506163in}{0.413320in}}%
\pgfpathlineto{\pgfqpoint{2.503454in}{0.413320in}}%
\pgfpathlineto{\pgfqpoint{2.500801in}{0.413320in}}%
\pgfpathlineto{\pgfqpoint{2.498085in}{0.413320in}}%
\pgfpathlineto{\pgfqpoint{2.495542in}{0.413320in}}%
\pgfpathlineto{\pgfqpoint{2.492729in}{0.413320in}}%
\pgfpathlineto{\pgfqpoint{2.490183in}{0.413320in}}%
\pgfpathlineto{\pgfqpoint{2.487384in}{0.413320in}}%
\pgfpathlineto{\pgfqpoint{2.484870in}{0.413320in}}%
\pgfpathlineto{\pgfqpoint{2.482026in}{0.413320in}}%
\pgfpathlineto{\pgfqpoint{2.479420in}{0.413320in}}%
\pgfpathlineto{\pgfqpoint{2.476671in}{0.413320in}}%
\pgfpathlineto{\pgfqpoint{2.473989in}{0.413320in}}%
\pgfpathlineto{\pgfqpoint{2.471311in}{0.413320in}}%
\pgfpathlineto{\pgfqpoint{2.468635in}{0.413320in}}%
\pgfpathlineto{\pgfqpoint{2.465957in}{0.413320in}}%
\pgfpathlineto{\pgfqpoint{2.463280in}{0.413320in}}%
\pgfpathlineto{\pgfqpoint{2.460711in}{0.413320in}}%
\pgfpathlineto{\pgfqpoint{2.457917in}{0.413320in}}%
\pgfpathlineto{\pgfqpoint{2.455353in}{0.413320in}}%
\pgfpathlineto{\pgfqpoint{2.452562in}{0.413320in}}%
\pgfpathlineto{\pgfqpoint{2.450032in}{0.413320in}}%
\pgfpathlineto{\pgfqpoint{2.447209in}{0.413320in}}%
\pgfpathlineto{\pgfqpoint{2.444677in}{0.413320in}}%
\pgfpathlineto{\pgfqpoint{2.441876in}{0.413320in}}%
\pgfpathlineto{\pgfqpoint{2.439167in}{0.413320in}}%
\pgfpathlineto{\pgfqpoint{2.436518in}{0.413320in}}%
\pgfpathlineto{\pgfqpoint{2.433815in}{0.413320in}}%
\pgfpathlineto{\pgfqpoint{2.431251in}{0.413320in}}%
\pgfpathlineto{\pgfqpoint{2.428453in}{0.413320in}}%
\pgfpathlineto{\pgfqpoint{2.425878in}{0.413320in}}%
\pgfpathlineto{\pgfqpoint{2.423098in}{0.413320in}}%
\pgfpathlineto{\pgfqpoint{2.420528in}{0.413320in}}%
\pgfpathlineto{\pgfqpoint{2.417747in}{0.413320in}}%
\pgfpathlineto{\pgfqpoint{2.415184in}{0.413320in}}%
\pgfpathlineto{\pgfqpoint{2.412389in}{0.413320in}}%
\pgfpathlineto{\pgfqpoint{2.409699in}{0.413320in}}%
\pgfpathlineto{\pgfqpoint{2.407024in}{0.413320in}}%
\pgfpathlineto{\pgfqpoint{2.404352in}{0.413320in}}%
\pgfpathlineto{\pgfqpoint{2.401675in}{0.413320in}}%
\pgfpathlineto{\pgfqpoint{2.398995in}{0.413320in}}%
\pgfpathclose%
\pgfusepath{stroke,fill}%
\end{pgfscope}%
\begin{pgfscope}%
\pgfpathrectangle{\pgfqpoint{2.398995in}{0.319877in}}{\pgfqpoint{3.986877in}{1.993438in}} %
\pgfusepath{clip}%
\pgfsetbuttcap%
\pgfsetroundjoin%
\definecolor{currentfill}{rgb}{1.000000,1.000000,1.000000}%
\pgfsetfillcolor{currentfill}%
\pgfsetlinewidth{1.003750pt}%
\definecolor{currentstroke}{rgb}{0.200364,0.692043,0.460354}%
\pgfsetstrokecolor{currentstroke}%
\pgfsetdash{}{0pt}%
\pgfpathmoveto{\pgfqpoint{2.398995in}{0.413320in}}%
\pgfpathlineto{\pgfqpoint{2.398995in}{1.723725in}}%
\pgfpathlineto{\pgfqpoint{2.401675in}{1.727744in}}%
\pgfpathlineto{\pgfqpoint{2.404352in}{1.725649in}}%
\pgfpathlineto{\pgfqpoint{2.407024in}{1.721798in}}%
\pgfpathlineto{\pgfqpoint{2.409699in}{1.718127in}}%
\pgfpathlineto{\pgfqpoint{2.412389in}{1.719860in}}%
\pgfpathlineto{\pgfqpoint{2.415184in}{1.726126in}}%
\pgfpathlineto{\pgfqpoint{2.417747in}{1.725631in}}%
\pgfpathlineto{\pgfqpoint{2.420528in}{1.721192in}}%
\pgfpathlineto{\pgfqpoint{2.423098in}{1.724406in}}%
\pgfpathlineto{\pgfqpoint{2.425878in}{1.722708in}}%
\pgfpathlineto{\pgfqpoint{2.428453in}{1.724818in}}%
\pgfpathlineto{\pgfqpoint{2.431251in}{1.748006in}}%
\pgfpathlineto{\pgfqpoint{2.433815in}{1.796070in}}%
\pgfpathlineto{\pgfqpoint{2.436518in}{1.805610in}}%
\pgfpathlineto{\pgfqpoint{2.439167in}{1.785010in}}%
\pgfpathlineto{\pgfqpoint{2.441876in}{1.787167in}}%
\pgfpathlineto{\pgfqpoint{2.444677in}{1.787721in}}%
\pgfpathlineto{\pgfqpoint{2.447209in}{1.778059in}}%
\pgfpathlineto{\pgfqpoint{2.450032in}{1.769878in}}%
\pgfpathlineto{\pgfqpoint{2.452562in}{1.768243in}}%
\pgfpathlineto{\pgfqpoint{2.455353in}{1.765614in}}%
\pgfpathlineto{\pgfqpoint{2.457917in}{1.763502in}}%
\pgfpathlineto{\pgfqpoint{2.460711in}{1.755348in}}%
\pgfpathlineto{\pgfqpoint{2.463280in}{1.757121in}}%
\pgfpathlineto{\pgfqpoint{2.465957in}{1.753238in}}%
\pgfpathlineto{\pgfqpoint{2.468635in}{1.741276in}}%
\pgfpathlineto{\pgfqpoint{2.471311in}{1.744553in}}%
\pgfpathlineto{\pgfqpoint{2.473989in}{1.745866in}}%
\pgfpathlineto{\pgfqpoint{2.476671in}{1.735916in}}%
\pgfpathlineto{\pgfqpoint{2.479420in}{1.729765in}}%
\pgfpathlineto{\pgfqpoint{2.482026in}{1.727858in}}%
\pgfpathlineto{\pgfqpoint{2.484870in}{1.723919in}}%
\pgfpathlineto{\pgfqpoint{2.487384in}{1.728494in}}%
\pgfpathlineto{\pgfqpoint{2.490183in}{1.731862in}}%
\pgfpathlineto{\pgfqpoint{2.492729in}{1.727068in}}%
\pgfpathlineto{\pgfqpoint{2.495542in}{1.724689in}}%
\pgfpathlineto{\pgfqpoint{2.498085in}{1.722766in}}%
\pgfpathlineto{\pgfqpoint{2.500801in}{1.726405in}}%
\pgfpathlineto{\pgfqpoint{2.503454in}{1.725815in}}%
\pgfpathlineto{\pgfqpoint{2.506163in}{1.732597in}}%
\pgfpathlineto{\pgfqpoint{2.508917in}{1.728898in}}%
\pgfpathlineto{\pgfqpoint{2.511478in}{1.727417in}}%
\pgfpathlineto{\pgfqpoint{2.514268in}{1.722888in}}%
\pgfpathlineto{\pgfqpoint{2.516845in}{1.726441in}}%
\pgfpathlineto{\pgfqpoint{2.519607in}{1.725996in}}%
\pgfpathlineto{\pgfqpoint{2.522197in}{1.728448in}}%
\pgfpathlineto{\pgfqpoint{2.524988in}{1.729354in}}%
\pgfpathlineto{\pgfqpoint{2.527560in}{1.720365in}}%
\pgfpathlineto{\pgfqpoint{2.530234in}{1.726222in}}%
\pgfpathlineto{\pgfqpoint{2.532917in}{1.735159in}}%
\pgfpathlineto{\pgfqpoint{2.535624in}{1.729549in}}%
\pgfpathlineto{\pgfqpoint{2.538274in}{1.727970in}}%
\pgfpathlineto{\pgfqpoint{2.540949in}{1.730799in}}%
\pgfpathlineto{\pgfqpoint{2.543765in}{1.729776in}}%
\pgfpathlineto{\pgfqpoint{2.546310in}{1.723520in}}%
\pgfpathlineto{\pgfqpoint{2.549114in}{1.721579in}}%
\pgfpathlineto{\pgfqpoint{2.551664in}{1.719059in}}%
\pgfpathlineto{\pgfqpoint{2.554493in}{1.720365in}}%
\pgfpathlineto{\pgfqpoint{2.557009in}{1.715546in}}%
\pgfpathlineto{\pgfqpoint{2.559790in}{1.716380in}}%
\pgfpathlineto{\pgfqpoint{2.562375in}{1.715546in}}%
\pgfpathlineto{\pgfqpoint{2.565045in}{1.717249in}}%
\pgfpathlineto{\pgfqpoint{2.567730in}{1.718413in}}%
\pgfpathlineto{\pgfqpoint{2.570411in}{1.716090in}}%
\pgfpathlineto{\pgfqpoint{2.573082in}{1.731429in}}%
\pgfpathlineto{\pgfqpoint{2.575779in}{1.728930in}}%
\pgfpathlineto{\pgfqpoint{2.578567in}{1.734977in}}%
\pgfpathlineto{\pgfqpoint{2.581129in}{1.728538in}}%
\pgfpathlineto{\pgfqpoint{2.583913in}{1.731144in}}%
\pgfpathlineto{\pgfqpoint{2.586484in}{1.729098in}}%
\pgfpathlineto{\pgfqpoint{2.589248in}{1.727342in}}%
\pgfpathlineto{\pgfqpoint{2.591842in}{1.730817in}}%
\pgfpathlineto{\pgfqpoint{2.594630in}{1.731711in}}%
\pgfpathlineto{\pgfqpoint{2.597196in}{1.728855in}}%
\pgfpathlineto{\pgfqpoint{2.599920in}{1.727525in}}%
\pgfpathlineto{\pgfqpoint{2.602557in}{1.726739in}}%
\pgfpathlineto{\pgfqpoint{2.605232in}{1.727847in}}%
\pgfpathlineto{\pgfqpoint{2.608004in}{1.732444in}}%
\pgfpathlineto{\pgfqpoint{2.610588in}{1.734660in}}%
\pgfpathlineto{\pgfqpoint{2.613393in}{1.727114in}}%
\pgfpathlineto{\pgfqpoint{2.615934in}{1.719218in}}%
\pgfpathlineto{\pgfqpoint{2.618773in}{1.722635in}}%
\pgfpathlineto{\pgfqpoint{2.621304in}{1.726910in}}%
\pgfpathlineto{\pgfqpoint{2.624077in}{1.726143in}}%
\pgfpathlineto{\pgfqpoint{2.626653in}{1.727453in}}%
\pgfpathlineto{\pgfqpoint{2.629340in}{1.720551in}}%
\pgfpathlineto{\pgfqpoint{2.632018in}{1.728860in}}%
\pgfpathlineto{\pgfqpoint{2.634700in}{1.730189in}}%
\pgfpathlineto{\pgfqpoint{2.637369in}{1.728565in}}%
\pgfpathlineto{\pgfqpoint{2.640053in}{1.728791in}}%
\pgfpathlineto{\pgfqpoint{2.642827in}{1.730971in}}%
\pgfpathlineto{\pgfqpoint{2.645408in}{1.728837in}}%
\pgfpathlineto{\pgfqpoint{2.648196in}{1.729868in}}%
\pgfpathlineto{\pgfqpoint{2.650767in}{1.725765in}}%
\pgfpathlineto{\pgfqpoint{2.653567in}{1.723508in}}%
\pgfpathlineto{\pgfqpoint{2.656124in}{1.718630in}}%
\pgfpathlineto{\pgfqpoint{2.658942in}{1.727084in}}%
\pgfpathlineto{\pgfqpoint{2.661481in}{1.727489in}}%
\pgfpathlineto{\pgfqpoint{2.664151in}{1.729353in}}%
\pgfpathlineto{\pgfqpoint{2.666836in}{1.731174in}}%
\pgfpathlineto{\pgfqpoint{2.669506in}{1.727636in}}%
\pgfpathlineto{\pgfqpoint{2.672301in}{1.728248in}}%
\pgfpathlineto{\pgfqpoint{2.674873in}{1.722129in}}%
\pgfpathlineto{\pgfqpoint{2.677650in}{1.724801in}}%
\pgfpathlineto{\pgfqpoint{2.680224in}{1.723413in}}%
\pgfpathlineto{\pgfqpoint{2.683009in}{1.725422in}}%
\pgfpathlineto{\pgfqpoint{2.685586in}{1.728302in}}%
\pgfpathlineto{\pgfqpoint{2.688328in}{1.729113in}}%
\pgfpathlineto{\pgfqpoint{2.690940in}{1.732434in}}%
\pgfpathlineto{\pgfqpoint{2.693611in}{1.726027in}}%
\pgfpathlineto{\pgfqpoint{2.696293in}{1.725945in}}%
\pgfpathlineto{\pgfqpoint{2.698968in}{1.722206in}}%
\pgfpathlineto{\pgfqpoint{2.701657in}{1.726059in}}%
\pgfpathlineto{\pgfqpoint{2.704326in}{1.728607in}}%
\pgfpathlineto{\pgfqpoint{2.707125in}{1.723937in}}%
\pgfpathlineto{\pgfqpoint{2.709683in}{1.725579in}}%
\pgfpathlineto{\pgfqpoint{2.712477in}{1.733335in}}%
\pgfpathlineto{\pgfqpoint{2.715036in}{1.724932in}}%
\pgfpathlineto{\pgfqpoint{2.717773in}{1.728732in}}%
\pgfpathlineto{\pgfqpoint{2.720404in}{1.731877in}}%
\pgfpathlineto{\pgfqpoint{2.723211in}{1.728876in}}%
\pgfpathlineto{\pgfqpoint{2.725760in}{1.725944in}}%
\pgfpathlineto{\pgfqpoint{2.728439in}{1.717695in}}%
\pgfpathlineto{\pgfqpoint{2.731119in}{1.724777in}}%
\pgfpathlineto{\pgfqpoint{2.733798in}{1.715546in}}%
\pgfpathlineto{\pgfqpoint{2.736476in}{1.722585in}}%
\pgfpathlineto{\pgfqpoint{2.739155in}{1.727621in}}%
\pgfpathlineto{\pgfqpoint{2.741928in}{1.732812in}}%
\pgfpathlineto{\pgfqpoint{2.744510in}{1.730592in}}%
\pgfpathlineto{\pgfqpoint{2.747260in}{1.734340in}}%
\pgfpathlineto{\pgfqpoint{2.749868in}{1.731479in}}%
\pgfpathlineto{\pgfqpoint{2.752614in}{1.739200in}}%
\pgfpathlineto{\pgfqpoint{2.755224in}{1.731062in}}%
\pgfpathlineto{\pgfqpoint{2.758028in}{1.733881in}}%
\pgfpathlineto{\pgfqpoint{2.760581in}{1.733798in}}%
\pgfpathlineto{\pgfqpoint{2.763253in}{1.729065in}}%
\pgfpathlineto{\pgfqpoint{2.765935in}{1.734136in}}%
\pgfpathlineto{\pgfqpoint{2.768617in}{1.727824in}}%
\pgfpathlineto{\pgfqpoint{2.771373in}{1.726746in}}%
\pgfpathlineto{\pgfqpoint{2.773972in}{1.729061in}}%
\pgfpathlineto{\pgfqpoint{2.776767in}{1.733334in}}%
\pgfpathlineto{\pgfqpoint{2.779330in}{1.722745in}}%
\pgfpathlineto{\pgfqpoint{2.782113in}{1.727299in}}%
\pgfpathlineto{\pgfqpoint{2.784687in}{1.717462in}}%
\pgfpathlineto{\pgfqpoint{2.787468in}{1.722144in}}%
\pgfpathlineto{\pgfqpoint{2.790044in}{1.717440in}}%
\pgfpathlineto{\pgfqpoint{2.792721in}{1.720076in}}%
\pgfpathlineto{\pgfqpoint{2.795398in}{1.721674in}}%
\pgfpathlineto{\pgfqpoint{2.798070in}{1.724380in}}%
\pgfpathlineto{\pgfqpoint{2.800756in}{1.726437in}}%
\pgfpathlineto{\pgfqpoint{2.803435in}{1.725285in}}%
\pgfpathlineto{\pgfqpoint{2.806175in}{1.722499in}}%
\pgfpathlineto{\pgfqpoint{2.808792in}{1.720871in}}%
\pgfpathlineto{\pgfqpoint{2.811597in}{1.729783in}}%
\pgfpathlineto{\pgfqpoint{2.814141in}{1.721515in}}%
\pgfpathlineto{\pgfqpoint{2.816867in}{1.726406in}}%
\pgfpathlineto{\pgfqpoint{2.819506in}{1.731564in}}%
\pgfpathlineto{\pgfqpoint{2.822303in}{1.732374in}}%
\pgfpathlineto{\pgfqpoint{2.824851in}{1.728633in}}%
\pgfpathlineto{\pgfqpoint{2.827567in}{1.730582in}}%
\pgfpathlineto{\pgfqpoint{2.830219in}{1.731662in}}%
\pgfpathlineto{\pgfqpoint{2.832894in}{1.728915in}}%
\pgfpathlineto{\pgfqpoint{2.835698in}{1.728370in}}%
\pgfpathlineto{\pgfqpoint{2.838254in}{1.720918in}}%
\pgfpathlineto{\pgfqpoint{2.841055in}{1.716806in}}%
\pgfpathlineto{\pgfqpoint{2.843611in}{1.719267in}}%
\pgfpathlineto{\pgfqpoint{2.846408in}{1.725615in}}%
\pgfpathlineto{\pgfqpoint{2.848960in}{1.728959in}}%
\pgfpathlineto{\pgfqpoint{2.851793in}{1.723809in}}%
\pgfpathlineto{\pgfqpoint{2.854325in}{1.721703in}}%
\pgfpathlineto{\pgfqpoint{2.857003in}{1.722739in}}%
\pgfpathlineto{\pgfqpoint{2.859668in}{1.724987in}}%
\pgfpathlineto{\pgfqpoint{2.862402in}{1.729768in}}%
\pgfpathlineto{\pgfqpoint{2.865031in}{1.735597in}}%
\pgfpathlineto{\pgfqpoint{2.867713in}{1.729880in}}%
\pgfpathlineto{\pgfqpoint{2.870475in}{1.724648in}}%
\pgfpathlineto{\pgfqpoint{2.873074in}{1.731375in}}%
\pgfpathlineto{\pgfqpoint{2.875882in}{1.729514in}}%
\pgfpathlineto{\pgfqpoint{2.878431in}{1.731377in}}%
\pgfpathlineto{\pgfqpoint{2.881254in}{1.734648in}}%
\pgfpathlineto{\pgfqpoint{2.883780in}{1.735243in}}%
\pgfpathlineto{\pgfqpoint{2.886578in}{1.732208in}}%
\pgfpathlineto{\pgfqpoint{2.889145in}{1.722901in}}%
\pgfpathlineto{\pgfqpoint{2.891809in}{1.722406in}}%
\pgfpathlineto{\pgfqpoint{2.894487in}{1.730800in}}%
\pgfpathlineto{\pgfqpoint{2.897179in}{1.724446in}}%
\pgfpathlineto{\pgfqpoint{2.899858in}{1.732128in}}%
\pgfpathlineto{\pgfqpoint{2.902535in}{1.729859in}}%
\pgfpathlineto{\pgfqpoint{2.905341in}{1.730467in}}%
\pgfpathlineto{\pgfqpoint{2.907882in}{1.729678in}}%
\pgfpathlineto{\pgfqpoint{2.910631in}{1.730886in}}%
\pgfpathlineto{\pgfqpoint{2.913243in}{1.737762in}}%
\pgfpathlineto{\pgfqpoint{2.916061in}{1.733288in}}%
\pgfpathlineto{\pgfqpoint{2.918606in}{1.728718in}}%
\pgfpathlineto{\pgfqpoint{2.921363in}{1.729111in}}%
\pgfpathlineto{\pgfqpoint{2.923963in}{1.731487in}}%
\pgfpathlineto{\pgfqpoint{2.926655in}{1.730321in}}%
\pgfpathlineto{\pgfqpoint{2.929321in}{1.732364in}}%
\pgfpathlineto{\pgfqpoint{2.932033in}{1.730644in}}%
\pgfpathlineto{\pgfqpoint{2.934759in}{1.732169in}}%
\pgfpathlineto{\pgfqpoint{2.937352in}{1.724704in}}%
\pgfpathlineto{\pgfqpoint{2.940120in}{1.720718in}}%
\pgfpathlineto{\pgfqpoint{2.942711in}{1.719231in}}%
\pgfpathlineto{\pgfqpoint{2.945461in}{1.726416in}}%
\pgfpathlineto{\pgfqpoint{2.948068in}{1.728554in}}%
\pgfpathlineto{\pgfqpoint{2.950884in}{1.730706in}}%
\pgfpathlineto{\pgfqpoint{2.953422in}{1.725959in}}%
\pgfpathlineto{\pgfqpoint{2.956103in}{1.724842in}}%
\pgfpathlineto{\pgfqpoint{2.958782in}{1.722871in}}%
\pgfpathlineto{\pgfqpoint{2.961460in}{1.724552in}}%
\pgfpathlineto{\pgfqpoint{2.964127in}{1.723893in}}%
\pgfpathlineto{\pgfqpoint{2.966812in}{1.725388in}}%
\pgfpathlineto{\pgfqpoint{2.969599in}{1.729445in}}%
\pgfpathlineto{\pgfqpoint{2.972177in}{1.728047in}}%
\pgfpathlineto{\pgfqpoint{2.974972in}{1.728162in}}%
\pgfpathlineto{\pgfqpoint{2.977517in}{1.731824in}}%
\pgfpathlineto{\pgfqpoint{2.980341in}{1.729406in}}%
\pgfpathlineto{\pgfqpoint{2.982885in}{1.732173in}}%
\pgfpathlineto{\pgfqpoint{2.985666in}{1.729932in}}%
\pgfpathlineto{\pgfqpoint{2.988238in}{1.727470in}}%
\pgfpathlineto{\pgfqpoint{2.990978in}{1.729575in}}%
\pgfpathlineto{\pgfqpoint{2.993595in}{1.731068in}}%
\pgfpathlineto{\pgfqpoint{2.996300in}{1.740408in}}%
\pgfpathlineto{\pgfqpoint{2.999103in}{1.788075in}}%
\pgfpathlineto{\pgfqpoint{3.001635in}{1.769292in}}%
\pgfpathlineto{\pgfqpoint{3.004419in}{1.758338in}}%
\pgfpathlineto{\pgfqpoint{3.006993in}{1.753984in}}%
\pgfpathlineto{\pgfqpoint{3.009784in}{1.745931in}}%
\pgfpathlineto{\pgfqpoint{3.012351in}{1.736790in}}%
\pgfpathlineto{\pgfqpoint{3.015097in}{1.735343in}}%
\pgfpathlineto{\pgfqpoint{3.017707in}{1.735602in}}%
\pgfpathlineto{\pgfqpoint{3.020382in}{1.725749in}}%
\pgfpathlineto{\pgfqpoint{3.023058in}{1.727056in}}%
\pgfpathlineto{\pgfqpoint{3.025803in}{1.740192in}}%
\pgfpathlineto{\pgfqpoint{3.028412in}{1.741736in}}%
\pgfpathlineto{\pgfqpoint{3.031091in}{1.731383in}}%
\pgfpathlineto{\pgfqpoint{3.033921in}{1.738787in}}%
\pgfpathlineto{\pgfqpoint{3.036456in}{1.737518in}}%
\pgfpathlineto{\pgfqpoint{3.039262in}{1.732013in}}%
\pgfpathlineto{\pgfqpoint{3.041813in}{1.735246in}}%
\pgfpathlineto{\pgfqpoint{3.044568in}{1.731158in}}%
\pgfpathlineto{\pgfqpoint{3.047157in}{1.733421in}}%
\pgfpathlineto{\pgfqpoint{3.049988in}{1.729623in}}%
\pgfpathlineto{\pgfqpoint{3.052526in}{1.729969in}}%
\pgfpathlineto{\pgfqpoint{3.055202in}{1.727785in}}%
\pgfpathlineto{\pgfqpoint{3.057884in}{1.733342in}}%
\pgfpathlineto{\pgfqpoint{3.060561in}{1.725633in}}%
\pgfpathlineto{\pgfqpoint{3.063230in}{1.735217in}}%
\pgfpathlineto{\pgfqpoint{3.065916in}{1.730848in}}%
\pgfpathlineto{\pgfqpoint{3.068709in}{1.728719in}}%
\pgfpathlineto{\pgfqpoint{3.071266in}{1.732777in}}%
\pgfpathlineto{\pgfqpoint{3.074056in}{1.730768in}}%
\pgfpathlineto{\pgfqpoint{3.076631in}{1.727697in}}%
\pgfpathlineto{\pgfqpoint{3.079381in}{1.723107in}}%
\pgfpathlineto{\pgfqpoint{3.081990in}{1.717919in}}%
\pgfpathlineto{\pgfqpoint{3.084671in}{1.716480in}}%
\pgfpathlineto{\pgfqpoint{3.087343in}{1.722801in}}%
\pgfpathlineto{\pgfqpoint{3.090023in}{1.725046in}}%
\pgfpathlineto{\pgfqpoint{3.092699in}{1.725890in}}%
\pgfpathlineto{\pgfqpoint{3.095388in}{1.725750in}}%
\pgfpathlineto{\pgfqpoint{3.098163in}{1.727717in}}%
\pgfpathlineto{\pgfqpoint{3.100737in}{1.721612in}}%
\pgfpathlineto{\pgfqpoint{3.103508in}{1.723684in}}%
\pgfpathlineto{\pgfqpoint{3.106094in}{1.715546in}}%
\pgfpathlineto{\pgfqpoint{3.108896in}{1.715546in}}%
\pgfpathlineto{\pgfqpoint{3.111451in}{1.721038in}}%
\pgfpathlineto{\pgfqpoint{3.114242in}{1.715546in}}%
\pgfpathlineto{\pgfqpoint{3.116807in}{1.715988in}}%
\pgfpathlineto{\pgfqpoint{3.119487in}{1.719115in}}%
\pgfpathlineto{\pgfqpoint{3.122163in}{1.723088in}}%
\pgfpathlineto{\pgfqpoint{3.124842in}{1.720468in}}%
\pgfpathlineto{\pgfqpoint{3.127512in}{1.720099in}}%
\pgfpathlineto{\pgfqpoint{3.130199in}{1.721657in}}%
\pgfpathlineto{\pgfqpoint{3.132946in}{1.716976in}}%
\pgfpathlineto{\pgfqpoint{3.135550in}{1.726655in}}%
\pgfpathlineto{\pgfqpoint{3.138375in}{1.717949in}}%
\pgfpathlineto{\pgfqpoint{3.140913in}{1.719132in}}%
\pgfpathlineto{\pgfqpoint{3.143740in}{1.723936in}}%
\pgfpathlineto{\pgfqpoint{3.146271in}{1.725056in}}%
\pgfpathlineto{\pgfqpoint{3.149057in}{1.723557in}}%
\pgfpathlineto{\pgfqpoint{3.151612in}{1.721817in}}%
\pgfpathlineto{\pgfqpoint{3.154327in}{1.715812in}}%
\pgfpathlineto{\pgfqpoint{3.156981in}{1.715546in}}%
\pgfpathlineto{\pgfqpoint{3.159675in}{1.724166in}}%
\pgfpathlineto{\pgfqpoint{3.162474in}{1.720104in}}%
\pgfpathlineto{\pgfqpoint{3.165019in}{1.720507in}}%
\pgfpathlineto{\pgfqpoint{3.167776in}{1.719136in}}%
\pgfpathlineto{\pgfqpoint{3.170375in}{1.725089in}}%
\pgfpathlineto{\pgfqpoint{3.173142in}{1.733347in}}%
\pgfpathlineto{\pgfqpoint{3.175724in}{1.726105in}}%
\pgfpathlineto{\pgfqpoint{3.178525in}{1.722595in}}%
\pgfpathlineto{\pgfqpoint{3.181089in}{1.720509in}}%
\pgfpathlineto{\pgfqpoint{3.183760in}{1.722074in}}%
\pgfpathlineto{\pgfqpoint{3.186440in}{1.724799in}}%
\pgfpathlineto{\pgfqpoint{3.189117in}{1.718232in}}%
\pgfpathlineto{\pgfqpoint{3.191796in}{1.722585in}}%
\pgfpathlineto{\pgfqpoint{3.194508in}{1.720589in}}%
\pgfpathlineto{\pgfqpoint{3.197226in}{1.721068in}}%
\pgfpathlineto{\pgfqpoint{3.199823in}{1.724685in}}%
\pgfpathlineto{\pgfqpoint{3.202562in}{1.726301in}}%
\pgfpathlineto{\pgfqpoint{3.205195in}{1.723541in}}%
\pgfpathlineto{\pgfqpoint{3.207984in}{1.716055in}}%
\pgfpathlineto{\pgfqpoint{3.210545in}{1.720934in}}%
\pgfpathlineto{\pgfqpoint{3.213342in}{1.726499in}}%
\pgfpathlineto{\pgfqpoint{3.215908in}{1.718022in}}%
\pgfpathlineto{\pgfqpoint{3.218586in}{1.717001in}}%
\pgfpathlineto{\pgfqpoint{3.221255in}{1.718970in}}%
\pgfpathlineto{\pgfqpoint{3.223942in}{1.715546in}}%
\pgfpathlineto{\pgfqpoint{3.226609in}{1.715546in}}%
\pgfpathlineto{\pgfqpoint{3.229310in}{1.715546in}}%
\pgfpathlineto{\pgfqpoint{3.232069in}{1.716318in}}%
\pgfpathlineto{\pgfqpoint{3.234658in}{1.721768in}}%
\pgfpathlineto{\pgfqpoint{3.237411in}{1.724587in}}%
\pgfpathlineto{\pgfqpoint{3.240010in}{1.721689in}}%
\pgfpathlineto{\pgfqpoint{3.242807in}{1.717567in}}%
\pgfpathlineto{\pgfqpoint{3.245363in}{1.719077in}}%
\pgfpathlineto{\pgfqpoint{3.248049in}{1.724528in}}%
\pgfpathlineto{\pgfqpoint{3.250716in}{1.720923in}}%
\pgfpathlineto{\pgfqpoint{3.253404in}{1.721776in}}%
\pgfpathlineto{\pgfqpoint{3.256083in}{1.729571in}}%
\pgfpathlineto{\pgfqpoint{3.258784in}{1.724391in}}%
\pgfpathlineto{\pgfqpoint{3.261594in}{1.729285in}}%
\pgfpathlineto{\pgfqpoint{3.264119in}{1.728582in}}%
\pgfpathlineto{\pgfqpoint{3.266849in}{1.727085in}}%
\pgfpathlineto{\pgfqpoint{3.269478in}{1.731034in}}%
\pgfpathlineto{\pgfqpoint{3.272254in}{1.732641in}}%
\pgfpathlineto{\pgfqpoint{3.274831in}{1.732479in}}%
\pgfpathlineto{\pgfqpoint{3.277603in}{1.726762in}}%
\pgfpathlineto{\pgfqpoint{3.280189in}{1.724868in}}%
\pgfpathlineto{\pgfqpoint{3.282870in}{1.728472in}}%
\pgfpathlineto{\pgfqpoint{3.285534in}{1.726656in}}%
\pgfpathlineto{\pgfqpoint{3.288225in}{1.727016in}}%
\pgfpathlineto{\pgfqpoint{3.290890in}{1.731989in}}%
\pgfpathlineto{\pgfqpoint{3.293574in}{1.724890in}}%
\pgfpathlineto{\pgfqpoint{3.296376in}{1.727604in}}%
\pgfpathlineto{\pgfqpoint{3.298937in}{1.726432in}}%
\pgfpathlineto{\pgfqpoint{3.301719in}{1.729939in}}%
\pgfpathlineto{\pgfqpoint{3.304295in}{1.732596in}}%
\pgfpathlineto{\pgfqpoint{3.307104in}{1.731958in}}%
\pgfpathlineto{\pgfqpoint{3.309652in}{1.734064in}}%
\pgfpathlineto{\pgfqpoint{3.312480in}{1.733534in}}%
\pgfpathlineto{\pgfqpoint{3.315008in}{1.730724in}}%
\pgfpathlineto{\pgfqpoint{3.317688in}{1.738058in}}%
\pgfpathlineto{\pgfqpoint{3.320366in}{1.728515in}}%
\pgfpathlineto{\pgfqpoint{3.323049in}{1.732892in}}%
\pgfpathlineto{\pgfqpoint{3.325860in}{1.731229in}}%
\pgfpathlineto{\pgfqpoint{3.328401in}{1.733093in}}%
\pgfpathlineto{\pgfqpoint{3.331183in}{1.731583in}}%
\pgfpathlineto{\pgfqpoint{3.333758in}{1.731416in}}%
\pgfpathlineto{\pgfqpoint{3.336541in}{1.739856in}}%
\pgfpathlineto{\pgfqpoint{3.339101in}{1.733858in}}%
\pgfpathlineto{\pgfqpoint{3.341893in}{1.734795in}}%
\pgfpathlineto{\pgfqpoint{3.344468in}{1.728464in}}%
\pgfpathlineto{\pgfqpoint{3.347139in}{1.723616in}}%
\pgfpathlineto{\pgfqpoint{3.349828in}{1.721712in}}%
\pgfpathlineto{\pgfqpoint{3.352505in}{1.730336in}}%
\pgfpathlineto{\pgfqpoint{3.355177in}{1.725457in}}%
\pgfpathlineto{\pgfqpoint{3.357862in}{1.725622in}}%
\pgfpathlineto{\pgfqpoint{3.360620in}{1.730579in}}%
\pgfpathlineto{\pgfqpoint{3.363221in}{1.730345in}}%
\pgfpathlineto{\pgfqpoint{3.365996in}{1.733018in}}%
\pgfpathlineto{\pgfqpoint{3.368577in}{1.732290in}}%
\pgfpathlineto{\pgfqpoint{3.371357in}{1.729393in}}%
\pgfpathlineto{\pgfqpoint{3.373921in}{1.724115in}}%
\pgfpathlineto{\pgfqpoint{3.376735in}{1.725076in}}%
\pgfpathlineto{\pgfqpoint{3.379290in}{1.727774in}}%
\pgfpathlineto{\pgfqpoint{3.381959in}{1.723901in}}%
\pgfpathlineto{\pgfqpoint{3.384647in}{1.726813in}}%
\pgfpathlineto{\pgfqpoint{3.387309in}{1.726052in}}%
\pgfpathlineto{\pgfqpoint{3.390102in}{1.727078in}}%
\pgfpathlineto{\pgfqpoint{3.392681in}{1.727262in}}%
\pgfpathlineto{\pgfqpoint{3.395461in}{1.724387in}}%
\pgfpathlineto{\pgfqpoint{3.398037in}{1.729073in}}%
\pgfpathlineto{\pgfqpoint{3.400783in}{1.729430in}}%
\pgfpathlineto{\pgfqpoint{3.403394in}{1.731191in}}%
\pgfpathlineto{\pgfqpoint{3.406202in}{1.725518in}}%
\pgfpathlineto{\pgfqpoint{3.408752in}{1.727819in}}%
\pgfpathlineto{\pgfqpoint{3.411431in}{1.726162in}}%
\pgfpathlineto{\pgfqpoint{3.414109in}{1.727562in}}%
\pgfpathlineto{\pgfqpoint{3.416780in}{1.727518in}}%
\pgfpathlineto{\pgfqpoint{3.419455in}{1.728803in}}%
\pgfpathlineto{\pgfqpoint{3.422142in}{1.727354in}}%
\pgfpathlineto{\pgfqpoint{3.424887in}{1.724592in}}%
\pgfpathlineto{\pgfqpoint{3.427501in}{1.727541in}}%
\pgfpathlineto{\pgfqpoint{3.430313in}{1.726308in}}%
\pgfpathlineto{\pgfqpoint{3.432851in}{1.727311in}}%
\pgfpathlineto{\pgfqpoint{3.435635in}{1.730042in}}%
\pgfpathlineto{\pgfqpoint{3.438210in}{1.728578in}}%
\pgfpathlineto{\pgfqpoint{3.440996in}{1.731330in}}%
\pgfpathlineto{\pgfqpoint{3.443574in}{1.736177in}}%
\pgfpathlineto{\pgfqpoint{3.446257in}{1.729333in}}%
\pgfpathlineto{\pgfqpoint{3.448926in}{1.735008in}}%
\pgfpathlineto{\pgfqpoint{3.451597in}{1.732186in}}%
\pgfpathlineto{\pgfqpoint{3.454285in}{1.736909in}}%
\pgfpathlineto{\pgfqpoint{3.456960in}{1.737880in}}%
\pgfpathlineto{\pgfqpoint{3.459695in}{1.735389in}}%
\pgfpathlineto{\pgfqpoint{3.462321in}{1.739702in}}%
\pgfpathlineto{\pgfqpoint{3.465072in}{1.732103in}}%
\pgfpathlineto{\pgfqpoint{3.467678in}{1.739521in}}%
\pgfpathlineto{\pgfqpoint{3.470466in}{1.738318in}}%
\pgfpathlineto{\pgfqpoint{3.473021in}{1.739889in}}%
\pgfpathlineto{\pgfqpoint{3.475821in}{1.739491in}}%
\pgfpathlineto{\pgfqpoint{3.478378in}{1.740787in}}%
\pgfpathlineto{\pgfqpoint{3.481072in}{1.742759in}}%
\pgfpathlineto{\pgfqpoint{3.483744in}{1.745942in}}%
\pgfpathlineto{\pgfqpoint{3.486442in}{1.740335in}}%
\pgfpathlineto{\pgfqpoint{3.489223in}{1.740395in}}%
\pgfpathlineto{\pgfqpoint{3.491783in}{1.742394in}}%
\pgfpathlineto{\pgfqpoint{3.494581in}{1.736646in}}%
\pgfpathlineto{\pgfqpoint{3.497139in}{1.735693in}}%
\pgfpathlineto{\pgfqpoint{3.499909in}{1.740460in}}%
\pgfpathlineto{\pgfqpoint{3.502488in}{1.739951in}}%
\pgfpathlineto{\pgfqpoint{3.505262in}{1.730317in}}%
\pgfpathlineto{\pgfqpoint{3.507840in}{1.731090in}}%
\pgfpathlineto{\pgfqpoint{3.510533in}{1.730072in}}%
\pgfpathlineto{\pgfqpoint{3.513209in}{1.735246in}}%
\pgfpathlineto{\pgfqpoint{3.515884in}{1.729199in}}%
\pgfpathlineto{\pgfqpoint{3.518565in}{1.732355in}}%
\pgfpathlineto{\pgfqpoint{3.521244in}{1.729215in}}%
\pgfpathlineto{\pgfqpoint{3.524041in}{1.735034in}}%
\pgfpathlineto{\pgfqpoint{3.526601in}{1.731295in}}%
\pgfpathlineto{\pgfqpoint{3.529327in}{1.729723in}}%
\pgfpathlineto{\pgfqpoint{3.531955in}{1.727682in}}%
\pgfpathlineto{\pgfqpoint{3.534783in}{1.726853in}}%
\pgfpathlineto{\pgfqpoint{3.537309in}{1.724414in}}%
\pgfpathlineto{\pgfqpoint{3.540093in}{1.726778in}}%
\pgfpathlineto{\pgfqpoint{3.542656in}{1.728391in}}%
\pgfpathlineto{\pgfqpoint{3.545349in}{1.726031in}}%
\pgfpathlineto{\pgfqpoint{3.548029in}{1.725085in}}%
\pgfpathlineto{\pgfqpoint{3.550713in}{1.727159in}}%
\pgfpathlineto{\pgfqpoint{3.553498in}{1.723641in}}%
\pgfpathlineto{\pgfqpoint{3.556061in}{1.726897in}}%
\pgfpathlineto{\pgfqpoint{3.558853in}{1.729538in}}%
\pgfpathlineto{\pgfqpoint{3.561420in}{1.723499in}}%
\pgfpathlineto{\pgfqpoint{3.564188in}{1.723969in}}%
\pgfpathlineto{\pgfqpoint{3.566774in}{1.725456in}}%
\pgfpathlineto{\pgfqpoint{3.569584in}{1.728042in}}%
\pgfpathlineto{\pgfqpoint{3.572126in}{1.727765in}}%
\pgfpathlineto{\pgfqpoint{3.574814in}{1.724259in}}%
\pgfpathlineto{\pgfqpoint{3.577487in}{1.720146in}}%
\pgfpathlineto{\pgfqpoint{3.580191in}{1.715546in}}%
\pgfpathlineto{\pgfqpoint{3.582851in}{1.718171in}}%
\pgfpathlineto{\pgfqpoint{3.585532in}{1.715546in}}%
\pgfpathlineto{\pgfqpoint{3.588258in}{1.728162in}}%
\pgfpathlineto{\pgfqpoint{3.590883in}{1.730046in}}%
\pgfpathlineto{\pgfqpoint{3.593620in}{1.725199in}}%
\pgfpathlineto{\pgfqpoint{3.596240in}{1.731176in}}%
\pgfpathlineto{\pgfqpoint{3.598998in}{1.722631in}}%
\pgfpathlineto{\pgfqpoint{3.601590in}{1.732333in}}%
\pgfpathlineto{\pgfqpoint{3.604387in}{1.728360in}}%
\pgfpathlineto{\pgfqpoint{3.606951in}{1.721227in}}%
\pgfpathlineto{\pgfqpoint{3.609632in}{1.726169in}}%
\pgfpathlineto{\pgfqpoint{3.612311in}{1.731405in}}%
\pgfpathlineto{\pgfqpoint{3.614982in}{1.721174in}}%
\pgfpathlineto{\pgfqpoint{3.617667in}{1.724361in}}%
\pgfpathlineto{\pgfqpoint{3.620345in}{1.728688in}}%
\pgfpathlineto{\pgfqpoint{3.623165in}{1.719355in}}%
\pgfpathlineto{\pgfqpoint{3.625689in}{1.722714in}}%
\pgfpathlineto{\pgfqpoint{3.628460in}{1.729203in}}%
\pgfpathlineto{\pgfqpoint{3.631058in}{1.734041in}}%
\pgfpathlineto{\pgfqpoint{3.633858in}{1.727352in}}%
\pgfpathlineto{\pgfqpoint{3.636413in}{1.723529in}}%
\pgfpathlineto{\pgfqpoint{3.639207in}{1.726876in}}%
\pgfpathlineto{\pgfqpoint{3.641773in}{1.726543in}}%
\pgfpathlineto{\pgfqpoint{3.644452in}{1.715546in}}%
\pgfpathlineto{\pgfqpoint{3.647130in}{1.715546in}}%
\pgfpathlineto{\pgfqpoint{3.649837in}{1.715546in}}%
\pgfpathlineto{\pgfqpoint{3.652628in}{1.719446in}}%
\pgfpathlineto{\pgfqpoint{3.655165in}{1.733752in}}%
\pgfpathlineto{\pgfqpoint{3.657917in}{1.818671in}}%
\pgfpathlineto{\pgfqpoint{3.660515in}{1.874216in}}%
\pgfpathlineto{\pgfqpoint{3.663276in}{1.833845in}}%
\pgfpathlineto{\pgfqpoint{3.665864in}{1.796602in}}%
\pgfpathlineto{\pgfqpoint{3.668665in}{1.772626in}}%
\pgfpathlineto{\pgfqpoint{3.671232in}{1.770813in}}%
\pgfpathlineto{\pgfqpoint{3.673911in}{1.774197in}}%
\pgfpathlineto{\pgfqpoint{3.676591in}{1.782894in}}%
\pgfpathlineto{\pgfqpoint{3.679273in}{1.785338in}}%
\pgfpathlineto{\pgfqpoint{3.681948in}{1.777017in}}%
\pgfpathlineto{\pgfqpoint{3.684620in}{1.769797in}}%
\pgfpathlineto{\pgfqpoint{3.687442in}{1.756472in}}%
\pgfpathlineto{\pgfqpoint{3.689983in}{1.751059in}}%
\pgfpathlineto{\pgfqpoint{3.692765in}{1.739676in}}%
\pgfpathlineto{\pgfqpoint{3.695331in}{1.732400in}}%
\pgfpathlineto{\pgfqpoint{3.698125in}{1.731812in}}%
\pgfpathlineto{\pgfqpoint{3.700684in}{1.730624in}}%
\pgfpathlineto{\pgfqpoint{3.703460in}{1.731710in}}%
\pgfpathlineto{\pgfqpoint{3.706053in}{1.729893in}}%
\pgfpathlineto{\pgfqpoint{3.708729in}{1.742522in}}%
\pgfpathlineto{\pgfqpoint{3.711410in}{1.736318in}}%
\pgfpathlineto{\pgfqpoint{3.714086in}{1.736678in}}%
\pgfpathlineto{\pgfqpoint{3.716875in}{1.736570in}}%
\pgfpathlineto{\pgfqpoint{3.719446in}{1.778805in}}%
\pgfpathlineto{\pgfqpoint{3.722228in}{1.832433in}}%
\pgfpathlineto{\pgfqpoint{3.724804in}{1.829905in}}%
\pgfpathlineto{\pgfqpoint{3.727581in}{1.823055in}}%
\pgfpathlineto{\pgfqpoint{3.730158in}{1.802515in}}%
\pgfpathlineto{\pgfqpoint{3.732950in}{1.783506in}}%
\pgfpathlineto{\pgfqpoint{3.735509in}{1.776376in}}%
\pgfpathlineto{\pgfqpoint{3.738194in}{1.777580in}}%
\pgfpathlineto{\pgfqpoint{3.740874in}{1.765329in}}%
\pgfpathlineto{\pgfqpoint{3.743548in}{1.759585in}}%
\pgfpathlineto{\pgfqpoint{3.746229in}{1.757948in}}%
\pgfpathlineto{\pgfqpoint{3.748903in}{1.765007in}}%
\pgfpathlineto{\pgfqpoint{3.751728in}{1.762365in}}%
\pgfpathlineto{\pgfqpoint{3.754265in}{1.757893in}}%
\pgfpathlineto{\pgfqpoint{3.757065in}{1.755420in}}%
\pgfpathlineto{\pgfqpoint{3.759622in}{1.754636in}}%
\pgfpathlineto{\pgfqpoint{3.762389in}{1.745488in}}%
\pgfpathlineto{\pgfqpoint{3.764966in}{1.738225in}}%
\pgfpathlineto{\pgfqpoint{3.767782in}{1.739328in}}%
\pgfpathlineto{\pgfqpoint{3.770323in}{1.741341in}}%
\pgfpathlineto{\pgfqpoint{3.773014in}{1.740316in}}%
\pgfpathlineto{\pgfqpoint{3.775691in}{1.742476in}}%
\pgfpathlineto{\pgfqpoint{3.778370in}{1.719641in}}%
\pgfpathlineto{\pgfqpoint{3.781046in}{1.715546in}}%
\pgfpathlineto{\pgfqpoint{3.783725in}{1.715546in}}%
\pgfpathlineto{\pgfqpoint{3.786504in}{1.721153in}}%
\pgfpathlineto{\pgfqpoint{3.789084in}{1.723879in}}%
\pgfpathlineto{\pgfqpoint{3.791897in}{1.726266in}}%
\pgfpathlineto{\pgfqpoint{3.794435in}{1.726427in}}%
\pgfpathlineto{\pgfqpoint{3.797265in}{1.722895in}}%
\pgfpathlineto{\pgfqpoint{3.799797in}{1.723728in}}%
\pgfpathlineto{\pgfqpoint{3.802569in}{1.722388in}}%
\pgfpathlineto{\pgfqpoint{3.805145in}{1.721708in}}%
\pgfpathlineto{\pgfqpoint{3.807832in}{1.722493in}}%
\pgfpathlineto{\pgfqpoint{3.810510in}{1.721503in}}%
\pgfpathlineto{\pgfqpoint{3.813172in}{1.725559in}}%
\pgfpathlineto{\pgfqpoint{3.815983in}{1.724523in}}%
\pgfpathlineto{\pgfqpoint{3.818546in}{1.725031in}}%
\pgfpathlineto{\pgfqpoint{3.821315in}{1.722687in}}%
\pgfpathlineto{\pgfqpoint{3.823903in}{1.723902in}}%
\pgfpathlineto{\pgfqpoint{3.826679in}{1.734809in}}%
\pgfpathlineto{\pgfqpoint{3.829252in}{1.729377in}}%
\pgfpathlineto{\pgfqpoint{3.832053in}{1.727295in}}%
\pgfpathlineto{\pgfqpoint{3.834616in}{1.727305in}}%
\pgfpathlineto{\pgfqpoint{3.837286in}{1.724454in}}%
\pgfpathlineto{\pgfqpoint{3.839960in}{1.732297in}}%
\pgfpathlineto{\pgfqpoint{3.842641in}{1.731043in}}%
\pgfpathlineto{\pgfqpoint{3.845329in}{1.728184in}}%
\pgfpathlineto{\pgfqpoint{3.848005in}{1.725918in}}%
\pgfpathlineto{\pgfqpoint{3.850814in}{1.726305in}}%
\pgfpathlineto{\pgfqpoint{3.853358in}{1.725538in}}%
\pgfpathlineto{\pgfqpoint{3.856100in}{1.725313in}}%
\pgfpathlineto{\pgfqpoint{3.858720in}{1.729187in}}%
\pgfpathlineto{\pgfqpoint{3.861561in}{1.734782in}}%
\pgfpathlineto{\pgfqpoint{3.864073in}{1.726267in}}%
\pgfpathlineto{\pgfqpoint{3.866815in}{1.723550in}}%
\pgfpathlineto{\pgfqpoint{3.869435in}{1.727095in}}%
\pgfpathlineto{\pgfqpoint{3.872114in}{1.727422in}}%
\pgfpathlineto{\pgfqpoint{3.874790in}{1.729654in}}%
\pgfpathlineto{\pgfqpoint{3.877466in}{1.722831in}}%
\pgfpathlineto{\pgfqpoint{3.880237in}{1.722440in}}%
\pgfpathlineto{\pgfqpoint{3.882850in}{1.724622in}}%
\pgfpathlineto{\pgfqpoint{3.885621in}{1.728927in}}%
\pgfpathlineto{\pgfqpoint{3.888188in}{1.719669in}}%
\pgfpathlineto{\pgfqpoint{3.890926in}{1.718395in}}%
\pgfpathlineto{\pgfqpoint{3.893541in}{1.718133in}}%
\pgfpathlineto{\pgfqpoint{3.896345in}{1.718556in}}%
\pgfpathlineto{\pgfqpoint{3.898891in}{1.722570in}}%
\pgfpathlineto{\pgfqpoint{3.901573in}{1.717574in}}%
\pgfpathlineto{\pgfqpoint{3.904252in}{1.718523in}}%
\pgfpathlineto{\pgfqpoint{3.906918in}{1.718690in}}%
\pgfpathlineto{\pgfqpoint{3.909602in}{1.721438in}}%
\pgfpathlineto{\pgfqpoint{3.912296in}{1.723730in}}%
\pgfpathlineto{\pgfqpoint{3.915107in}{1.723535in}}%
\pgfpathlineto{\pgfqpoint{3.917646in}{1.720447in}}%
\pgfpathlineto{\pgfqpoint{3.920412in}{1.722560in}}%
\pgfpathlineto{\pgfqpoint{3.923005in}{1.729300in}}%
\pgfpathlineto{\pgfqpoint{3.925778in}{1.725466in}}%
\pgfpathlineto{\pgfqpoint{3.928347in}{1.732319in}}%
\pgfpathlineto{\pgfqpoint{3.931202in}{1.723223in}}%
\pgfpathlineto{\pgfqpoint{3.933714in}{1.727711in}}%
\pgfpathlineto{\pgfqpoint{3.936395in}{1.733135in}}%
\pgfpathlineto{\pgfqpoint{3.939075in}{1.728264in}}%
\pgfpathlineto{\pgfqpoint{3.941778in}{1.722604in}}%
\pgfpathlineto{\pgfqpoint{3.944431in}{1.726173in}}%
\pgfpathlineto{\pgfqpoint{3.947101in}{1.726022in}}%
\pgfpathlineto{\pgfqpoint{3.949894in}{1.720794in}}%
\pgfpathlineto{\pgfqpoint{3.952464in}{1.724735in}}%
\pgfpathlineto{\pgfqpoint{3.955211in}{1.730887in}}%
\pgfpathlineto{\pgfqpoint{3.957823in}{1.725613in}}%
\pgfpathlineto{\pgfqpoint{3.960635in}{1.731552in}}%
\pgfpathlineto{\pgfqpoint{3.963176in}{1.728505in}}%
\pgfpathlineto{\pgfqpoint{3.966013in}{1.733493in}}%
\pgfpathlineto{\pgfqpoint{3.968523in}{1.732063in}}%
\pgfpathlineto{\pgfqpoint{3.971250in}{1.731245in}}%
\pgfpathlineto{\pgfqpoint{3.973885in}{1.732081in}}%
\pgfpathlineto{\pgfqpoint{3.976563in}{1.731286in}}%
\pgfpathlineto{\pgfqpoint{3.979389in}{1.728776in}}%
\pgfpathlineto{\pgfqpoint{3.981929in}{1.727843in}}%
\pgfpathlineto{\pgfqpoint{3.984714in}{1.723777in}}%
\pgfpathlineto{\pgfqpoint{3.987270in}{1.728641in}}%
\pgfpathlineto{\pgfqpoint{3.990055in}{1.731101in}}%
\pgfpathlineto{\pgfqpoint{3.992642in}{1.730431in}}%
\pgfpathlineto{\pgfqpoint{3.995417in}{1.728455in}}%
\pgfpathlineto{\pgfqpoint{3.997990in}{1.730637in}}%
\pgfpathlineto{\pgfqpoint{4.000674in}{1.730225in}}%
\pgfpathlineto{\pgfqpoint{4.003348in}{1.722372in}}%
\pgfpathlineto{\pgfqpoint{4.006034in}{1.730955in}}%
\pgfpathlineto{\pgfqpoint{4.008699in}{1.732063in}}%
\pgfpathlineto{\pgfqpoint{4.011394in}{1.727992in}}%
\pgfpathlineto{\pgfqpoint{4.014186in}{1.730779in}}%
\pgfpathlineto{\pgfqpoint{4.016744in}{1.729776in}}%
\pgfpathlineto{\pgfqpoint{4.019518in}{1.730724in}}%
\pgfpathlineto{\pgfqpoint{4.022097in}{1.726776in}}%
\pgfpathlineto{\pgfqpoint{4.024868in}{1.726766in}}%
\pgfpathlineto{\pgfqpoint{4.027447in}{1.726090in}}%
\pgfpathlineto{\pgfqpoint{4.030229in}{1.730311in}}%
\pgfpathlineto{\pgfqpoint{4.032817in}{1.728364in}}%
\pgfpathlineto{\pgfqpoint{4.035492in}{1.729310in}}%
\pgfpathlineto{\pgfqpoint{4.038174in}{1.723891in}}%
\pgfpathlineto{\pgfqpoint{4.040852in}{1.728973in}}%
\pgfpathlineto{\pgfqpoint{4.043667in}{1.724730in}}%
\pgfpathlineto{\pgfqpoint{4.046210in}{1.730137in}}%
\pgfpathlineto{\pgfqpoint{4.049006in}{1.726891in}}%
\pgfpathlineto{\pgfqpoint{4.051557in}{1.729961in}}%
\pgfpathlineto{\pgfqpoint{4.054326in}{1.729315in}}%
\pgfpathlineto{\pgfqpoint{4.056911in}{1.730384in}}%
\pgfpathlineto{\pgfqpoint{4.059702in}{1.725061in}}%
\pgfpathlineto{\pgfqpoint{4.062266in}{1.730127in}}%
\pgfpathlineto{\pgfqpoint{4.064957in}{1.727476in}}%
\pgfpathlineto{\pgfqpoint{4.067636in}{1.724566in}}%
\pgfpathlineto{\pgfqpoint{4.070313in}{1.728767in}}%
\pgfpathlineto{\pgfqpoint{4.072985in}{1.727212in}}%
\pgfpathlineto{\pgfqpoint{4.075705in}{1.726335in}}%
\pgfpathlineto{\pgfqpoint{4.078471in}{1.722366in}}%
\pgfpathlineto{\pgfqpoint{4.081018in}{1.721703in}}%
\pgfpathlineto{\pgfqpoint{4.083870in}{1.727762in}}%
\pgfpathlineto{\pgfqpoint{4.086385in}{1.721731in}}%
\pgfpathlineto{\pgfqpoint{4.089159in}{1.721285in}}%
\pgfpathlineto{\pgfqpoint{4.091729in}{1.726329in}}%
\pgfpathlineto{\pgfqpoint{4.094527in}{1.723307in}}%
\pgfpathlineto{\pgfqpoint{4.097092in}{1.731022in}}%
\pgfpathlineto{\pgfqpoint{4.099777in}{1.724924in}}%
\pgfpathlineto{\pgfqpoint{4.102456in}{1.727845in}}%
\pgfpathlineto{\pgfqpoint{4.105185in}{1.729488in}}%
\pgfpathlineto{\pgfqpoint{4.107814in}{1.724784in}}%
\pgfpathlineto{\pgfqpoint{4.110488in}{1.725917in}}%
\pgfpathlineto{\pgfqpoint{4.113252in}{1.726421in}}%
\pgfpathlineto{\pgfqpoint{4.115844in}{1.728190in}}%
\pgfpathlineto{\pgfqpoint{4.118554in}{1.731579in}}%
\pgfpathlineto{\pgfqpoint{4.121205in}{1.732678in}}%
\pgfpathlineto{\pgfqpoint{4.124019in}{1.723740in}}%
\pgfpathlineto{\pgfqpoint{4.126553in}{1.727296in}}%
\pgfpathlineto{\pgfqpoint{4.129349in}{1.724740in}}%
\pgfpathlineto{\pgfqpoint{4.131920in}{1.723980in}}%
\pgfpathlineto{\pgfqpoint{4.134615in}{1.727294in}}%
\pgfpathlineto{\pgfqpoint{4.137272in}{1.727036in}}%
\pgfpathlineto{\pgfqpoint{4.139963in}{1.735482in}}%
\pgfpathlineto{\pgfqpoint{4.142713in}{1.727440in}}%
\pgfpathlineto{\pgfqpoint{4.145310in}{1.727794in}}%
\pgfpathlineto{\pgfqpoint{4.148082in}{1.727581in}}%
\pgfpathlineto{\pgfqpoint{4.150665in}{1.724397in}}%
\pgfpathlineto{\pgfqpoint{4.153423in}{1.732980in}}%
\pgfpathlineto{\pgfqpoint{4.156016in}{1.729083in}}%
\pgfpathlineto{\pgfqpoint{4.158806in}{1.725387in}}%
\pgfpathlineto{\pgfqpoint{4.161380in}{1.724683in}}%
\pgfpathlineto{\pgfqpoint{4.164059in}{1.724168in}}%
\pgfpathlineto{\pgfqpoint{4.166737in}{1.716574in}}%
\pgfpathlineto{\pgfqpoint{4.169415in}{1.716957in}}%
\pgfpathlineto{\pgfqpoint{4.172093in}{1.723221in}}%
\pgfpathlineto{\pgfqpoint{4.174770in}{1.721502in}}%
\pgfpathlineto{\pgfqpoint{4.177593in}{1.724800in}}%
\pgfpathlineto{\pgfqpoint{4.180129in}{1.717173in}}%
\pgfpathlineto{\pgfqpoint{4.182899in}{1.715546in}}%
\pgfpathlineto{\pgfqpoint{4.185481in}{1.715546in}}%
\pgfpathlineto{\pgfqpoint{4.188318in}{1.715546in}}%
\pgfpathlineto{\pgfqpoint{4.190842in}{1.727875in}}%
\pgfpathlineto{\pgfqpoint{4.193638in}{1.724603in}}%
\pgfpathlineto{\pgfqpoint{4.196186in}{1.724668in}}%
\pgfpathlineto{\pgfqpoint{4.198878in}{1.727196in}}%
\pgfpathlineto{\pgfqpoint{4.201542in}{1.725635in}}%
\pgfpathlineto{\pgfqpoint{4.204240in}{1.726545in}}%
\pgfpathlineto{\pgfqpoint{4.207076in}{1.725198in}}%
\pgfpathlineto{\pgfqpoint{4.209597in}{1.740761in}}%
\pgfpathlineto{\pgfqpoint{4.212383in}{1.732455in}}%
\pgfpathlineto{\pgfqpoint{4.214948in}{1.724863in}}%
\pgfpathlineto{\pgfqpoint{4.217694in}{1.727235in}}%
\pgfpathlineto{\pgfqpoint{4.220304in}{1.720498in}}%
\pgfpathlineto{\pgfqpoint{4.223082in}{1.720858in}}%
\pgfpathlineto{\pgfqpoint{4.225654in}{1.721890in}}%
\pgfpathlineto{\pgfqpoint{4.228331in}{1.725368in}}%
\pgfpathlineto{\pgfqpoint{4.231013in}{1.727466in}}%
\pgfpathlineto{\pgfqpoint{4.233691in}{1.724805in}}%
\pgfpathlineto{\pgfqpoint{4.236375in}{1.725495in}}%
\pgfpathlineto{\pgfqpoint{4.239084in}{1.725090in}}%
\pgfpathlineto{\pgfqpoint{4.241900in}{1.726046in}}%
\pgfpathlineto{\pgfqpoint{4.244394in}{1.728629in}}%
\pgfpathlineto{\pgfqpoint{4.247225in}{1.731300in}}%
\pgfpathlineto{\pgfqpoint{4.249767in}{1.725375in}}%
\pgfpathlineto{\pgfqpoint{4.252581in}{1.729399in}}%
\pgfpathlineto{\pgfqpoint{4.255120in}{1.727310in}}%
\pgfpathlineto{\pgfqpoint{4.257958in}{1.726416in}}%
\pgfpathlineto{\pgfqpoint{4.260477in}{1.729139in}}%
\pgfpathlineto{\pgfqpoint{4.263157in}{1.726598in}}%
\pgfpathlineto{\pgfqpoint{4.265824in}{1.732387in}}%
\pgfpathlineto{\pgfqpoint{4.268590in}{1.730055in}}%
\pgfpathlineto{\pgfqpoint{4.271187in}{1.728090in}}%
\pgfpathlineto{\pgfqpoint{4.273874in}{1.730131in}}%
\pgfpathlineto{\pgfqpoint{4.276635in}{1.733671in}}%
\pgfpathlineto{\pgfqpoint{4.279212in}{1.729766in}}%
\pgfpathlineto{\pgfqpoint{4.282000in}{1.729669in}}%
\pgfpathlineto{\pgfqpoint{4.284586in}{1.730455in}}%
\pgfpathlineto{\pgfqpoint{4.287399in}{1.726954in}}%
\pgfpathlineto{\pgfqpoint{4.289936in}{1.728223in}}%
\pgfpathlineto{\pgfqpoint{4.292786in}{1.724740in}}%
\pgfpathlineto{\pgfqpoint{4.295299in}{1.725061in}}%
\pgfpathlineto{\pgfqpoint{4.297977in}{1.725509in}}%
\pgfpathlineto{\pgfqpoint{4.300656in}{1.727065in}}%
\pgfpathlineto{\pgfqpoint{4.303357in}{1.731387in}}%
\pgfpathlineto{\pgfqpoint{4.306118in}{1.729197in}}%
\pgfpathlineto{\pgfqpoint{4.308691in}{1.725417in}}%
\pgfpathlineto{\pgfqpoint{4.311494in}{1.730246in}}%
\pgfpathlineto{\pgfqpoint{4.314032in}{1.734370in}}%
\pgfpathlineto{\pgfqpoint{4.316856in}{1.731250in}}%
\pgfpathlineto{\pgfqpoint{4.319405in}{1.730989in}}%
\pgfpathlineto{\pgfqpoint{4.322181in}{1.737423in}}%
\pgfpathlineto{\pgfqpoint{4.324760in}{1.734534in}}%
\pgfpathlineto{\pgfqpoint{4.327440in}{1.738042in}}%
\pgfpathlineto{\pgfqpoint{4.330118in}{1.736181in}}%
\pgfpathlineto{\pgfqpoint{4.332796in}{1.735979in}}%
\pgfpathlineto{\pgfqpoint{4.335463in}{1.735764in}}%
\pgfpathlineto{\pgfqpoint{4.338154in}{1.729565in}}%
\pgfpathlineto{\pgfqpoint{4.340976in}{1.730216in}}%
\pgfpathlineto{\pgfqpoint{4.343510in}{1.728873in}}%
\pgfpathlineto{\pgfqpoint{4.346263in}{1.728922in}}%
\pgfpathlineto{\pgfqpoint{4.348868in}{1.727661in}}%
\pgfpathlineto{\pgfqpoint{4.351645in}{1.727940in}}%
\pgfpathlineto{\pgfqpoint{4.354224in}{1.728084in}}%
\pgfpathlineto{\pgfqpoint{4.357014in}{1.725696in}}%
\pgfpathlineto{\pgfqpoint{4.359582in}{1.727670in}}%
\pgfpathlineto{\pgfqpoint{4.362270in}{1.729335in}}%
\pgfpathlineto{\pgfqpoint{4.364936in}{1.723742in}}%
\pgfpathlineto{\pgfqpoint{4.367646in}{1.724681in}}%
\pgfpathlineto{\pgfqpoint{4.370437in}{1.724421in}}%
\pgfpathlineto{\pgfqpoint{4.372976in}{1.723337in}}%
\pgfpathlineto{\pgfqpoint{4.375761in}{1.723492in}}%
\pgfpathlineto{\pgfqpoint{4.378329in}{1.723193in}}%
\pgfpathlineto{\pgfqpoint{4.381097in}{1.719653in}}%
\pgfpathlineto{\pgfqpoint{4.383674in}{1.720753in}}%
\pgfpathlineto{\pgfqpoint{4.386431in}{1.721220in}}%
\pgfpathlineto{\pgfqpoint{4.389044in}{1.725661in}}%
\pgfpathlineto{\pgfqpoint{4.391721in}{1.727364in}}%
\pgfpathlineto{\pgfqpoint{4.394400in}{1.731173in}}%
\pgfpathlineto{\pgfqpoint{4.397076in}{1.716189in}}%
\pgfpathlineto{\pgfqpoint{4.399745in}{1.721711in}}%
\pgfpathlineto{\pgfqpoint{4.402468in}{1.734825in}}%
\pgfpathlineto{\pgfqpoint{4.405234in}{1.733270in}}%
\pgfpathlineto{\pgfqpoint{4.407788in}{1.728965in}}%
\pgfpathlineto{\pgfqpoint{4.410587in}{1.726078in}}%
\pgfpathlineto{\pgfqpoint{4.413149in}{1.729272in}}%
\pgfpathlineto{\pgfqpoint{4.415932in}{1.726216in}}%
\pgfpathlineto{\pgfqpoint{4.418506in}{1.729874in}}%
\pgfpathlineto{\pgfqpoint{4.421292in}{1.728784in}}%
\pgfpathlineto{\pgfqpoint{4.423863in}{1.733482in}}%
\pgfpathlineto{\pgfqpoint{4.426534in}{1.728915in}}%
\pgfpathlineto{\pgfqpoint{4.429220in}{1.732455in}}%
\pgfpathlineto{\pgfqpoint{4.431901in}{1.728789in}}%
\pgfpathlineto{\pgfqpoint{4.434569in}{1.728997in}}%
\pgfpathlineto{\pgfqpoint{4.437253in}{1.730679in}}%
\pgfpathlineto{\pgfqpoint{4.440041in}{1.728138in}}%
\pgfpathlineto{\pgfqpoint{4.442611in}{1.729626in}}%
\pgfpathlineto{\pgfqpoint{4.445423in}{1.724997in}}%
\pgfpathlineto{\pgfqpoint{4.447965in}{1.731393in}}%
\pgfpathlineto{\pgfqpoint{4.450767in}{1.725753in}}%
\pgfpathlineto{\pgfqpoint{4.453312in}{1.724148in}}%
\pgfpathlineto{\pgfqpoint{4.456138in}{1.756910in}}%
\pgfpathlineto{\pgfqpoint{4.458681in}{1.797406in}}%
\pgfpathlineto{\pgfqpoint{4.461367in}{1.767950in}}%
\pgfpathlineto{\pgfqpoint{4.464029in}{1.751672in}}%
\pgfpathlineto{\pgfqpoint{4.466717in}{1.742422in}}%
\pgfpathlineto{\pgfqpoint{4.469492in}{1.734438in}}%
\pgfpathlineto{\pgfqpoint{4.472059in}{1.732512in}}%
\pgfpathlineto{\pgfqpoint{4.474861in}{1.735610in}}%
\pgfpathlineto{\pgfqpoint{4.477430in}{1.737332in}}%
\pgfpathlineto{\pgfqpoint{4.480201in}{1.758380in}}%
\pgfpathlineto{\pgfqpoint{4.482778in}{1.758431in}}%
\pgfpathlineto{\pgfqpoint{4.485581in}{1.750204in}}%
\pgfpathlineto{\pgfqpoint{4.488130in}{1.748593in}}%
\pgfpathlineto{\pgfqpoint{4.490822in}{1.742867in}}%
\pgfpathlineto{\pgfqpoint{4.493492in}{1.744468in}}%
\pgfpathlineto{\pgfqpoint{4.496167in}{1.746768in}}%
\pgfpathlineto{\pgfqpoint{4.498850in}{1.738080in}}%
\pgfpathlineto{\pgfqpoint{4.501529in}{1.739778in}}%
\pgfpathlineto{\pgfqpoint{4.504305in}{1.736258in}}%
\pgfpathlineto{\pgfqpoint{4.506893in}{1.734771in}}%
\pgfpathlineto{\pgfqpoint{4.509643in}{1.737648in}}%
\pgfpathlineto{\pgfqpoint{4.512246in}{1.734355in}}%
\pgfpathlineto{\pgfqpoint{4.515080in}{1.735115in}}%
\pgfpathlineto{\pgfqpoint{4.517598in}{1.737135in}}%
\pgfpathlineto{\pgfqpoint{4.520345in}{1.732540in}}%
\pgfpathlineto{\pgfqpoint{4.522962in}{1.728283in}}%
\pgfpathlineto{\pgfqpoint{4.525640in}{1.737426in}}%
\pgfpathlineto{\pgfqpoint{4.528307in}{1.736407in}}%
\pgfpathlineto{\pgfqpoint{4.530990in}{1.729266in}}%
\pgfpathlineto{\pgfqpoint{4.533764in}{1.730376in}}%
\pgfpathlineto{\pgfqpoint{4.536400in}{1.737216in}}%
\pgfpathlineto{\pgfqpoint{4.539144in}{1.731832in}}%
\pgfpathlineto{\pgfqpoint{4.541711in}{1.729143in}}%
\pgfpathlineto{\pgfqpoint{4.544464in}{1.724276in}}%
\pgfpathlineto{\pgfqpoint{4.547064in}{1.729590in}}%
\pgfpathlineto{\pgfqpoint{4.549822in}{1.726915in}}%
\pgfpathlineto{\pgfqpoint{4.552425in}{1.727028in}}%
\pgfpathlineto{\pgfqpoint{4.555106in}{1.726255in}}%
\pgfpathlineto{\pgfqpoint{4.557777in}{1.728375in}}%
\pgfpathlineto{\pgfqpoint{4.560448in}{1.726021in}}%
\pgfpathlineto{\pgfqpoint{4.563125in}{1.727970in}}%
\pgfpathlineto{\pgfqpoint{4.565820in}{1.731789in}}%
\pgfpathlineto{\pgfqpoint{4.568612in}{1.728932in}}%
\pgfpathlineto{\pgfqpoint{4.571171in}{1.722168in}}%
\pgfpathlineto{\pgfqpoint{4.573947in}{1.725132in}}%
\pgfpathlineto{\pgfqpoint{4.576531in}{1.725856in}}%
\pgfpathlineto{\pgfqpoint{4.579305in}{1.722856in}}%
\pgfpathlineto{\pgfqpoint{4.581888in}{1.727883in}}%
\pgfpathlineto{\pgfqpoint{4.584672in}{1.729153in}}%
\pgfpathlineto{\pgfqpoint{4.587244in}{1.726249in}}%
\pgfpathlineto{\pgfqpoint{4.589920in}{1.725976in}}%
\pgfpathlineto{\pgfqpoint{4.592589in}{1.715546in}}%
\pgfpathlineto{\pgfqpoint{4.595281in}{1.720165in}}%
\pgfpathlineto{\pgfqpoint{4.597951in}{1.727544in}}%
\pgfpathlineto{\pgfqpoint{4.600633in}{1.722779in}}%
\pgfpathlineto{\pgfqpoint{4.603430in}{1.721795in}}%
\pgfpathlineto{\pgfqpoint{4.605990in}{1.727444in}}%
\pgfpathlineto{\pgfqpoint{4.608808in}{1.729611in}}%
\pgfpathlineto{\pgfqpoint{4.611350in}{1.728111in}}%
\pgfpathlineto{\pgfqpoint{4.614134in}{1.725314in}}%
\pgfpathlineto{\pgfqpoint{4.616702in}{1.727631in}}%
\pgfpathlineto{\pgfqpoint{4.619529in}{1.727982in}}%
\pgfpathlineto{\pgfqpoint{4.622056in}{1.730996in}}%
\pgfpathlineto{\pgfqpoint{4.624741in}{1.723292in}}%
\pgfpathlineto{\pgfqpoint{4.627411in}{1.730690in}}%
\pgfpathlineto{\pgfqpoint{4.630096in}{1.726571in}}%
\pgfpathlineto{\pgfqpoint{4.632902in}{1.731440in}}%
\pgfpathlineto{\pgfqpoint{4.635445in}{1.727559in}}%
\pgfpathlineto{\pgfqpoint{4.638204in}{1.734851in}}%
\pgfpathlineto{\pgfqpoint{4.640809in}{1.733769in}}%
\pgfpathlineto{\pgfqpoint{4.643628in}{1.731319in}}%
\pgfpathlineto{\pgfqpoint{4.646169in}{1.731880in}}%
\pgfpathlineto{\pgfqpoint{4.648922in}{1.730642in}}%
\pgfpathlineto{\pgfqpoint{4.651524in}{1.732989in}}%
\pgfpathlineto{\pgfqpoint{4.654203in}{1.728404in}}%
\pgfpathlineto{\pgfqpoint{4.656873in}{1.724072in}}%
\pgfpathlineto{\pgfqpoint{4.659590in}{1.728362in}}%
\pgfpathlineto{\pgfqpoint{4.662237in}{1.730301in}}%
\pgfpathlineto{\pgfqpoint{4.664923in}{1.727709in}}%
\pgfpathlineto{\pgfqpoint{4.667764in}{1.727439in}}%
\pgfpathlineto{\pgfqpoint{4.670261in}{1.724698in}}%
\pgfpathlineto{\pgfqpoint{4.673068in}{1.725016in}}%
\pgfpathlineto{\pgfqpoint{4.675619in}{1.726357in}}%
\pgfpathlineto{\pgfqpoint{4.678448in}{1.727172in}}%
\pgfpathlineto{\pgfqpoint{4.680988in}{1.725763in}}%
\pgfpathlineto{\pgfqpoint{4.683799in}{1.727003in}}%
\pgfpathlineto{\pgfqpoint{4.686337in}{1.725264in}}%
\pgfpathlineto{\pgfqpoint{4.689051in}{1.720572in}}%
\pgfpathlineto{\pgfqpoint{4.691694in}{1.720057in}}%
\pgfpathlineto{\pgfqpoint{4.694381in}{1.727746in}}%
\pgfpathlineto{\pgfqpoint{4.697170in}{1.730689in}}%
\pgfpathlineto{\pgfqpoint{4.699734in}{1.726048in}}%
\pgfpathlineto{\pgfqpoint{4.702517in}{1.726961in}}%
\pgfpathlineto{\pgfqpoint{4.705094in}{1.727530in}}%
\pgfpathlineto{\pgfqpoint{4.707824in}{1.724338in}}%
\pgfpathlineto{\pgfqpoint{4.710437in}{1.726937in}}%
\pgfpathlineto{\pgfqpoint{4.713275in}{1.731597in}}%
\pgfpathlineto{\pgfqpoint{4.715806in}{1.728179in}}%
\pgfpathlineto{\pgfqpoint{4.718486in}{1.728859in}}%
\pgfpathlineto{\pgfqpoint{4.721160in}{1.729449in}}%
\pgfpathlineto{\pgfqpoint{4.723873in}{1.726363in}}%
\pgfpathlineto{\pgfqpoint{4.726508in}{1.724982in}}%
\pgfpathlineto{\pgfqpoint{4.729233in}{1.730999in}}%
\pgfpathlineto{\pgfqpoint{4.731901in}{1.731771in}}%
\pgfpathlineto{\pgfqpoint{4.734552in}{1.730572in}}%
\pgfpathlineto{\pgfqpoint{4.737348in}{1.733606in}}%
\pgfpathlineto{\pgfqpoint{4.739912in}{1.729784in}}%
\pgfpathlineto{\pgfqpoint{4.742696in}{1.731972in}}%
\pgfpathlineto{\pgfqpoint{4.745256in}{1.733042in}}%
\pgfpathlineto{\pgfqpoint{4.748081in}{1.729666in}}%
\pgfpathlineto{\pgfqpoint{4.750627in}{1.734812in}}%
\pgfpathlineto{\pgfqpoint{4.753298in}{1.732960in}}%
\pgfpathlineto{\pgfqpoint{4.755983in}{1.727262in}}%
\pgfpathlineto{\pgfqpoint{4.758653in}{1.737775in}}%
\pgfpathlineto{\pgfqpoint{4.761337in}{1.736625in}}%
\pgfpathlineto{\pgfqpoint{4.764018in}{1.739815in}}%
\pgfpathlineto{\pgfqpoint{4.766783in}{1.729874in}}%
\pgfpathlineto{\pgfqpoint{4.769367in}{1.730474in}}%
\pgfpathlineto{\pgfqpoint{4.772198in}{1.731726in}}%
\pgfpathlineto{\pgfqpoint{4.774732in}{1.730024in}}%
\pgfpathlineto{\pgfqpoint{4.777535in}{1.732878in}}%
\pgfpathlineto{\pgfqpoint{4.780083in}{1.727370in}}%
\pgfpathlineto{\pgfqpoint{4.782872in}{1.727144in}}%
\pgfpathlineto{\pgfqpoint{4.785445in}{1.724919in}}%
\pgfpathlineto{\pgfqpoint{4.788116in}{1.733515in}}%
\pgfpathlineto{\pgfqpoint{4.790798in}{1.733071in}}%
\pgfpathlineto{\pgfqpoint{4.793512in}{1.731460in}}%
\pgfpathlineto{\pgfqpoint{4.796274in}{1.734709in}}%
\pgfpathlineto{\pgfqpoint{4.798830in}{1.739720in}}%
\pgfpathlineto{\pgfqpoint{4.801586in}{1.737667in}}%
\pgfpathlineto{\pgfqpoint{4.804193in}{1.738109in}}%
\pgfpathlineto{\pgfqpoint{4.807017in}{1.734117in}}%
\pgfpathlineto{\pgfqpoint{4.809538in}{1.741253in}}%
\pgfpathlineto{\pgfqpoint{4.812377in}{1.746060in}}%
\pgfpathlineto{\pgfqpoint{4.814907in}{1.751624in}}%
\pgfpathlineto{\pgfqpoint{4.817587in}{1.748284in}}%
\pgfpathlineto{\pgfqpoint{4.820265in}{1.746276in}}%
\pgfpathlineto{\pgfqpoint{4.822945in}{1.744174in}}%
\pgfpathlineto{\pgfqpoint{4.825619in}{1.748975in}}%
\pgfpathlineto{\pgfqpoint{4.828291in}{1.748867in}}%
\pgfpathlineto{\pgfqpoint{4.831045in}{1.748443in}}%
\pgfpathlineto{\pgfqpoint{4.833657in}{1.750632in}}%
\pgfpathlineto{\pgfqpoint{4.837992in}{1.753856in}}%
\pgfpathlineto{\pgfqpoint{4.839922in}{1.748442in}}%
\pgfpathlineto{\pgfqpoint{4.842380in}{1.757992in}}%
\pgfpathlineto{\pgfqpoint{4.844361in}{1.746331in}}%
\pgfpathlineto{\pgfqpoint{4.847127in}{1.752547in}}%
\pgfpathlineto{\pgfqpoint{4.849715in}{1.737590in}}%
\pgfpathlineto{\pgfqpoint{4.852404in}{1.740362in}}%
\pgfpathlineto{\pgfqpoint{4.855070in}{1.724372in}}%
\pgfpathlineto{\pgfqpoint{4.857807in}{1.743857in}}%
\pgfpathlineto{\pgfqpoint{4.860544in}{1.742737in}}%
\pgfpathlineto{\pgfqpoint{4.863116in}{1.734063in}}%
\pgfpathlineto{\pgfqpoint{4.865910in}{1.738272in}}%
\pgfpathlineto{\pgfqpoint{4.868474in}{1.730907in}}%
\pgfpathlineto{\pgfqpoint{4.871209in}{1.740130in}}%
\pgfpathlineto{\pgfqpoint{4.873832in}{1.735742in}}%
\pgfpathlineto{\pgfqpoint{4.876636in}{1.734574in}}%
\pgfpathlineto{\pgfqpoint{4.879180in}{1.733113in}}%
\pgfpathlineto{\pgfqpoint{4.881864in}{1.730222in}}%
\pgfpathlineto{\pgfqpoint{4.884540in}{1.725645in}}%
\pgfpathlineto{\pgfqpoint{4.887211in}{1.726905in}}%
\pgfpathlineto{\pgfqpoint{4.889902in}{1.734379in}}%
\pgfpathlineto{\pgfqpoint{4.892611in}{1.734884in}}%
\pgfpathlineto{\pgfqpoint{4.895399in}{1.740205in}}%
\pgfpathlineto{\pgfqpoint{4.897938in}{1.733401in}}%
\pgfpathlineto{\pgfqpoint{4.900712in}{1.731048in}}%
\pgfpathlineto{\pgfqpoint{4.903295in}{1.762128in}}%
\pgfpathlineto{\pgfqpoint{4.906096in}{1.828439in}}%
\pgfpathlineto{\pgfqpoint{4.908648in}{1.877703in}}%
\pgfpathlineto{\pgfqpoint{4.911435in}{1.895219in}}%
\pgfpathlineto{\pgfqpoint{4.914009in}{1.871105in}}%
\pgfpathlineto{\pgfqpoint{4.916681in}{1.843300in}}%
\pgfpathlineto{\pgfqpoint{4.919352in}{1.822430in}}%
\pgfpathlineto{\pgfqpoint{4.922041in}{1.797279in}}%
\pgfpathlineto{\pgfqpoint{4.924708in}{1.780219in}}%
\pgfpathlineto{\pgfqpoint{4.927400in}{1.769785in}}%
\pgfpathlineto{\pgfqpoint{4.930170in}{1.751006in}}%
\pgfpathlineto{\pgfqpoint{4.932742in}{1.742442in}}%
\pgfpathlineto{\pgfqpoint{4.935515in}{1.800461in}}%
\pgfpathlineto{\pgfqpoint{4.938112in}{1.870221in}}%
\pgfpathlineto{\pgfqpoint{4.940881in}{1.868810in}}%
\pgfpathlineto{\pgfqpoint{4.943466in}{1.866989in}}%
\pgfpathlineto{\pgfqpoint{4.946151in}{1.849580in}}%
\pgfpathlineto{\pgfqpoint{4.948827in}{1.843199in}}%
\pgfpathlineto{\pgfqpoint{4.951504in}{1.825539in}}%
\pgfpathlineto{\pgfqpoint{4.954182in}{1.805849in}}%
\pgfpathlineto{\pgfqpoint{4.956862in}{1.785765in}}%
\pgfpathlineto{\pgfqpoint{4.959689in}{1.778813in}}%
\pgfpathlineto{\pgfqpoint{4.962219in}{1.783565in}}%
\pgfpathlineto{\pgfqpoint{4.965002in}{1.785110in}}%
\pgfpathlineto{\pgfqpoint{4.967575in}{1.779851in}}%
\pgfpathlineto{\pgfqpoint{4.970314in}{1.766486in}}%
\pgfpathlineto{\pgfqpoint{4.972933in}{1.761957in}}%
\pgfpathlineto{\pgfqpoint{4.975703in}{1.755453in}}%
\pgfpathlineto{\pgfqpoint{4.978287in}{1.747716in}}%
\pgfpathlineto{\pgfqpoint{4.980967in}{1.752038in}}%
\pgfpathlineto{\pgfqpoint{4.983637in}{1.745940in}}%
\pgfpathlineto{\pgfqpoint{4.986325in}{1.741641in}}%
\pgfpathlineto{\pgfqpoint{4.989001in}{1.739215in}}%
\pgfpathlineto{\pgfqpoint{4.991683in}{1.738758in}}%
\pgfpathlineto{\pgfqpoint{4.994390in}{1.734567in}}%
\pgfpathlineto{\pgfqpoint{4.997028in}{1.735651in}}%
\pgfpathlineto{\pgfqpoint{4.999780in}{1.733793in}}%
\pgfpathlineto{\pgfqpoint{5.002384in}{1.725272in}}%
\pgfpathlineto{\pgfqpoint{5.005178in}{1.733474in}}%
\pgfpathlineto{\pgfqpoint{5.007751in}{1.731444in}}%
\pgfpathlineto{\pgfqpoint{5.010562in}{1.734043in}}%
\pgfpathlineto{\pgfqpoint{5.013104in}{1.739472in}}%
\pgfpathlineto{\pgfqpoint{5.015820in}{1.742705in}}%
\pgfpathlineto{\pgfqpoint{5.018466in}{1.742541in}}%
\pgfpathlineto{\pgfqpoint{5.021147in}{1.735772in}}%
\pgfpathlineto{\pgfqpoint{5.023927in}{1.722296in}}%
\pgfpathlineto{\pgfqpoint{5.026501in}{1.726380in}}%
\pgfpathlineto{\pgfqpoint{5.029275in}{1.730381in}}%
\pgfpathlineto{\pgfqpoint{5.031849in}{1.721182in}}%
\pgfpathlineto{\pgfqpoint{5.034649in}{1.719158in}}%
\pgfpathlineto{\pgfqpoint{5.037214in}{1.720784in}}%
\pgfpathlineto{\pgfqpoint{5.039962in}{1.725415in}}%
\pgfpathlineto{\pgfqpoint{5.042572in}{1.722660in}}%
\pgfpathlineto{\pgfqpoint{5.045249in}{1.715671in}}%
\pgfpathlineto{\pgfqpoint{5.047924in}{1.722517in}}%
\pgfpathlineto{\pgfqpoint{5.050606in}{1.718702in}}%
\pgfpathlineto{\pgfqpoint{5.053284in}{1.720091in}}%
\pgfpathlineto{\pgfqpoint{5.055952in}{1.728451in}}%
\pgfpathlineto{\pgfqpoint{5.058711in}{1.731560in}}%
\pgfpathlineto{\pgfqpoint{5.061315in}{1.724820in}}%
\pgfpathlineto{\pgfqpoint{5.064144in}{1.726254in}}%
\pgfpathlineto{\pgfqpoint{5.066677in}{1.730734in}}%
\pgfpathlineto{\pgfqpoint{5.069463in}{1.721994in}}%
\pgfpathlineto{\pgfqpoint{5.072030in}{1.720321in}}%
\pgfpathlineto{\pgfqpoint{5.074851in}{1.726174in}}%
\pgfpathlineto{\pgfqpoint{5.077390in}{1.724437in}}%
\pgfpathlineto{\pgfqpoint{5.080067in}{1.732245in}}%
\pgfpathlineto{\pgfqpoint{5.082746in}{1.726378in}}%
\pgfpathlineto{\pgfqpoint{5.085426in}{1.731875in}}%
\pgfpathlineto{\pgfqpoint{5.088103in}{1.730097in}}%
\pgfpathlineto{\pgfqpoint{5.090788in}{1.732002in}}%
\pgfpathlineto{\pgfqpoint{5.093579in}{1.735578in}}%
\pgfpathlineto{\pgfqpoint{5.096142in}{1.729979in}}%
\pgfpathlineto{\pgfqpoint{5.098948in}{1.728469in}}%
\pgfpathlineto{\pgfqpoint{5.101496in}{1.727338in}}%
\pgfpathlineto{\pgfqpoint{5.104312in}{1.729275in}}%
\pgfpathlineto{\pgfqpoint{5.106842in}{1.731837in}}%
\pgfpathlineto{\pgfqpoint{5.109530in}{1.727518in}}%
\pgfpathlineto{\pgfqpoint{5.112209in}{1.730309in}}%
\pgfpathlineto{\pgfqpoint{5.114887in}{1.724181in}}%
\pgfpathlineto{\pgfqpoint{5.117550in}{1.729893in}}%
\pgfpathlineto{\pgfqpoint{5.120243in}{1.730248in}}%
\pgfpathlineto{\pgfqpoint{5.123042in}{1.731433in}}%
\pgfpathlineto{\pgfqpoint{5.125599in}{1.735686in}}%
\pgfpathlineto{\pgfqpoint{5.128421in}{1.731177in}}%
\pgfpathlineto{\pgfqpoint{5.130953in}{1.724801in}}%
\pgfpathlineto{\pgfqpoint{5.133716in}{1.729502in}}%
\pgfpathlineto{\pgfqpoint{5.136311in}{1.734099in}}%
\pgfpathlineto{\pgfqpoint{5.139072in}{1.726177in}}%
\pgfpathlineto{\pgfqpoint{5.141660in}{1.728872in}}%
\pgfpathlineto{\pgfqpoint{5.144349in}{1.727636in}}%
\pgfpathlineto{\pgfqpoint{5.147029in}{1.727341in}}%
\pgfpathlineto{\pgfqpoint{5.149734in}{1.721731in}}%
\pgfpathlineto{\pgfqpoint{5.152382in}{1.715546in}}%
\pgfpathlineto{\pgfqpoint{5.155059in}{1.721119in}}%
\pgfpathlineto{\pgfqpoint{5.157815in}{1.726930in}}%
\pgfpathlineto{\pgfqpoint{5.160420in}{1.731894in}}%
\pgfpathlineto{\pgfqpoint{5.163243in}{1.727753in}}%
\pgfpathlineto{\pgfqpoint{5.165775in}{1.734368in}}%
\pgfpathlineto{\pgfqpoint{5.168591in}{1.732166in}}%
\pgfpathlineto{\pgfqpoint{5.171133in}{1.733314in}}%
\pgfpathlineto{\pgfqpoint{5.173925in}{1.733301in}}%
\pgfpathlineto{\pgfqpoint{5.176477in}{1.732200in}}%
\pgfpathlineto{\pgfqpoint{5.179188in}{1.730991in}}%
\pgfpathlineto{\pgfqpoint{5.181848in}{1.726636in}}%
\pgfpathlineto{\pgfqpoint{5.184522in}{1.731243in}}%
\pgfpathlineto{\pgfqpoint{5.187294in}{1.732056in}}%
\pgfpathlineto{\pgfqpoint{5.189880in}{1.730145in}}%
\pgfpathlineto{\pgfqpoint{5.192680in}{1.741102in}}%
\pgfpathlineto{\pgfqpoint{5.195239in}{1.734673in}}%
\pgfpathlineto{\pgfqpoint{5.198008in}{1.735808in}}%
\pgfpathlineto{\pgfqpoint{5.200594in}{1.734933in}}%
\pgfpathlineto{\pgfqpoint{5.203388in}{1.731468in}}%
\pgfpathlineto{\pgfqpoint{5.205952in}{1.731183in}}%
\pgfpathlineto{\pgfqpoint{5.208630in}{1.730876in}}%
\pgfpathlineto{\pgfqpoint{5.211299in}{1.732545in}}%
\pgfpathlineto{\pgfqpoint{5.214027in}{1.732127in}}%
\pgfpathlineto{\pgfqpoint{5.216667in}{1.734256in}}%
\pgfpathlineto{\pgfqpoint{5.219345in}{1.728269in}}%
\pgfpathlineto{\pgfqpoint{5.222151in}{1.725734in}}%
\pgfpathlineto{\pgfqpoint{5.224695in}{1.723796in}}%
\pgfpathlineto{\pgfqpoint{5.227470in}{1.724637in}}%
\pgfpathlineto{\pgfqpoint{5.230059in}{1.726543in}}%
\pgfpathlineto{\pgfqpoint{5.232855in}{1.731210in}}%
\pgfpathlineto{\pgfqpoint{5.235409in}{1.729381in}}%
\pgfpathlineto{\pgfqpoint{5.238173in}{1.728215in}}%
\pgfpathlineto{\pgfqpoint{5.240777in}{1.725897in}}%
\pgfpathlineto{\pgfqpoint{5.243445in}{1.731246in}}%
\pgfpathlineto{\pgfqpoint{5.246130in}{1.731142in}}%
\pgfpathlineto{\pgfqpoint{5.248816in}{1.730406in}}%
\pgfpathlineto{\pgfqpoint{5.251590in}{1.730760in}}%
\pgfpathlineto{\pgfqpoint{5.254236in}{1.727510in}}%
\pgfpathlineto{\pgfqpoint{5.256973in}{1.725097in}}%
\pgfpathlineto{\pgfqpoint{5.259511in}{1.726822in}}%
\pgfpathlineto{\pgfqpoint{5.262264in}{1.724853in}}%
\pgfpathlineto{\pgfqpoint{5.264876in}{1.721520in}}%
\pgfpathlineto{\pgfqpoint{5.267691in}{1.724770in}}%
\pgfpathlineto{\pgfqpoint{5.270238in}{1.733933in}}%
\pgfpathlineto{\pgfqpoint{5.272913in}{1.726591in}}%
\pgfpathlineto{\pgfqpoint{5.275589in}{1.718450in}}%
\pgfpathlineto{\pgfqpoint{5.278322in}{1.716556in}}%
\pgfpathlineto{\pgfqpoint{5.280947in}{1.716389in}}%
\pgfpathlineto{\pgfqpoint{5.283631in}{1.720428in}}%
\pgfpathlineto{\pgfqpoint{5.286436in}{1.724997in}}%
\pgfpathlineto{\pgfqpoint{5.288984in}{1.728126in}}%
\pgfpathlineto{\pgfqpoint{5.291794in}{1.730138in}}%
\pgfpathlineto{\pgfqpoint{5.294339in}{1.727735in}}%
\pgfpathlineto{\pgfqpoint{5.297140in}{1.730884in}}%
\pgfpathlineto{\pgfqpoint{5.299696in}{1.730822in}}%
\pgfpathlineto{\pgfqpoint{5.302443in}{1.732275in}}%
\pgfpathlineto{\pgfqpoint{5.305054in}{1.736150in}}%
\pgfpathlineto{\pgfqpoint{5.307731in}{1.734829in}}%
\pgfpathlineto{\pgfqpoint{5.310411in}{1.732253in}}%
\pgfpathlineto{\pgfqpoint{5.313089in}{1.733504in}}%
\pgfpathlineto{\pgfqpoint{5.315754in}{1.728689in}}%
\pgfpathlineto{\pgfqpoint{5.318430in}{1.732166in}}%
\pgfpathlineto{\pgfqpoint{5.321256in}{1.734603in}}%
\pgfpathlineto{\pgfqpoint{5.323802in}{1.726665in}}%
\pgfpathlineto{\pgfqpoint{5.326564in}{1.724705in}}%
\pgfpathlineto{\pgfqpoint{5.329159in}{1.725933in}}%
\pgfpathlineto{\pgfqpoint{5.331973in}{1.720980in}}%
\pgfpathlineto{\pgfqpoint{5.334510in}{1.718936in}}%
\pgfpathlineto{\pgfqpoint{5.337353in}{1.717770in}}%
\pgfpathlineto{\pgfqpoint{5.339872in}{1.719781in}}%
\pgfpathlineto{\pgfqpoint{5.342549in}{1.715546in}}%
\pgfpathlineto{\pgfqpoint{5.345224in}{1.718964in}}%
\pgfpathlineto{\pgfqpoint{5.347905in}{1.725233in}}%
\pgfpathlineto{\pgfqpoint{5.350723in}{1.720733in}}%
\pgfpathlineto{\pgfqpoint{5.353262in}{1.721674in}}%
\pgfpathlineto{\pgfqpoint{5.356056in}{1.724155in}}%
\pgfpathlineto{\pgfqpoint{5.358612in}{1.720219in}}%
\pgfpathlineto{\pgfqpoint{5.361370in}{1.722597in}}%
\pgfpathlineto{\pgfqpoint{5.363966in}{1.723167in}}%
\pgfpathlineto{\pgfqpoint{5.366727in}{1.726324in}}%
\pgfpathlineto{\pgfqpoint{5.369335in}{1.724653in}}%
\pgfpathlineto{\pgfqpoint{5.372013in}{1.726520in}}%
\pgfpathlineto{\pgfqpoint{5.374692in}{1.723658in}}%
\pgfpathlineto{\pgfqpoint{5.377370in}{1.728457in}}%
\pgfpathlineto{\pgfqpoint{5.380048in}{1.726502in}}%
\pgfpathlineto{\pgfqpoint{5.382725in}{1.728071in}}%
\pgfpathlineto{\pgfqpoint{5.385550in}{1.728944in}}%
\pgfpathlineto{\pgfqpoint{5.388083in}{1.730091in}}%
\pgfpathlineto{\pgfqpoint{5.390900in}{1.740830in}}%
\pgfpathlineto{\pgfqpoint{5.393441in}{1.733376in}}%
\pgfpathlineto{\pgfqpoint{5.396219in}{1.731603in}}%
\pgfpathlineto{\pgfqpoint{5.398784in}{1.736786in}}%
\pgfpathlineto{\pgfqpoint{5.401576in}{1.738301in}}%
\pgfpathlineto{\pgfqpoint{5.404154in}{1.731562in}}%
\pgfpathlineto{\pgfqpoint{5.406832in}{1.732279in}}%
\pgfpathlineto{\pgfqpoint{5.409507in}{1.731779in}}%
\pgfpathlineto{\pgfqpoint{5.412190in}{1.725980in}}%
\pgfpathlineto{\pgfqpoint{5.414954in}{1.722642in}}%
\pgfpathlineto{\pgfqpoint{5.417547in}{1.727002in}}%
\pgfpathlineto{\pgfqpoint{5.420304in}{1.722123in}}%
\pgfpathlineto{\pgfqpoint{5.422897in}{1.731912in}}%
\pgfpathlineto{\pgfqpoint{5.425661in}{1.724650in}}%
\pgfpathlineto{\pgfqpoint{5.428259in}{1.730425in}}%
\pgfpathlineto{\pgfqpoint{5.431015in}{1.729736in}}%
\pgfpathlineto{\pgfqpoint{5.433616in}{1.729316in}}%
\pgfpathlineto{\pgfqpoint{5.436295in}{1.733654in}}%
\pgfpathlineto{\pgfqpoint{5.438974in}{1.726668in}}%
\pgfpathlineto{\pgfqpoint{5.441698in}{1.727077in}}%
\pgfpathlineto{\pgfqpoint{5.444328in}{1.727379in}}%
\pgfpathlineto{\pgfqpoint{5.447021in}{1.724896in}}%
\pgfpathlineto{\pgfqpoint{5.449769in}{1.734642in}}%
\pgfpathlineto{\pgfqpoint{5.452365in}{1.730086in}}%
\pgfpathlineto{\pgfqpoint{5.455168in}{1.729683in}}%
\pgfpathlineto{\pgfqpoint{5.457721in}{1.724629in}}%
\pgfpathlineto{\pgfqpoint{5.460489in}{1.730016in}}%
\pgfpathlineto{\pgfqpoint{5.463079in}{1.727734in}}%
\pgfpathlineto{\pgfqpoint{5.465888in}{1.732181in}}%
\pgfpathlineto{\pgfqpoint{5.468425in}{1.723172in}}%
\pgfpathlineto{\pgfqpoint{5.471113in}{1.723005in}}%
\pgfpathlineto{\pgfqpoint{5.473792in}{1.727468in}}%
\pgfpathlineto{\pgfqpoint{5.476458in}{1.725967in}}%
\pgfpathlineto{\pgfqpoint{5.479152in}{1.727375in}}%
\pgfpathlineto{\pgfqpoint{5.481825in}{1.727042in}}%
\pgfpathlineto{\pgfqpoint{5.484641in}{1.726668in}}%
\pgfpathlineto{\pgfqpoint{5.487176in}{1.729665in}}%
\pgfpathlineto{\pgfqpoint{5.490000in}{1.730238in}}%
\pgfpathlineto{\pgfqpoint{5.492541in}{1.726899in}}%
\pgfpathlineto{\pgfqpoint{5.495346in}{1.730449in}}%
\pgfpathlineto{\pgfqpoint{5.497898in}{1.734567in}}%
\pgfpathlineto{\pgfqpoint{5.500687in}{1.722251in}}%
\pgfpathlineto{\pgfqpoint{5.503255in}{1.733449in}}%
\pgfpathlineto{\pgfqpoint{5.505933in}{1.728511in}}%
\pgfpathlineto{\pgfqpoint{5.508612in}{1.725051in}}%
\pgfpathlineto{\pgfqpoint{5.511290in}{1.731754in}}%
\pgfpathlineto{\pgfqpoint{5.514080in}{1.729913in}}%
\pgfpathlineto{\pgfqpoint{5.516646in}{1.731813in}}%
\pgfpathlineto{\pgfqpoint{5.519433in}{1.738261in}}%
\pgfpathlineto{\pgfqpoint{5.522003in}{1.734101in}}%
\pgfpathlineto{\pgfqpoint{5.524756in}{1.732515in}}%
\pgfpathlineto{\pgfqpoint{5.527360in}{1.729783in}}%
\pgfpathlineto{\pgfqpoint{5.530148in}{1.734683in}}%
\pgfpathlineto{\pgfqpoint{5.532717in}{1.727243in}}%
\pgfpathlineto{\pgfqpoint{5.535395in}{1.734978in}}%
\pgfpathlineto{\pgfqpoint{5.538074in}{1.728584in}}%
\pgfpathlineto{\pgfqpoint{5.540750in}{1.734624in}}%
\pgfpathlineto{\pgfqpoint{5.543421in}{1.732957in}}%
\pgfpathlineto{\pgfqpoint{5.546110in}{1.732011in}}%
\pgfpathlineto{\pgfqpoint{5.548921in}{1.730875in}}%
\pgfpathlineto{\pgfqpoint{5.551457in}{1.730435in}}%
\pgfpathlineto{\pgfqpoint{5.554198in}{1.725819in}}%
\pgfpathlineto{\pgfqpoint{5.556822in}{1.729737in}}%
\pgfpathlineto{\pgfqpoint{5.559612in}{1.726538in}}%
\pgfpathlineto{\pgfqpoint{5.562180in}{1.723619in}}%
\pgfpathlineto{\pgfqpoint{5.564940in}{1.724176in}}%
\pgfpathlineto{\pgfqpoint{5.567536in}{1.727096in}}%
\pgfpathlineto{\pgfqpoint{5.570215in}{1.723307in}}%
\pgfpathlineto{\pgfqpoint{5.572893in}{1.726152in}}%
\pgfpathlineto{\pgfqpoint{5.575596in}{1.732129in}}%
\pgfpathlineto{\pgfqpoint{5.578342in}{1.730026in}}%
\pgfpathlineto{\pgfqpoint{5.580914in}{1.726424in}}%
\pgfpathlineto{\pgfqpoint{5.583709in}{1.724813in}}%
\pgfpathlineto{\pgfqpoint{5.586269in}{1.729609in}}%
\pgfpathlineto{\pgfqpoint{5.589040in}{1.726883in}}%
\pgfpathlineto{\pgfqpoint{5.591641in}{1.728605in}}%
\pgfpathlineto{\pgfqpoint{5.594368in}{1.721670in}}%
\pgfpathlineto{\pgfqpoint{5.596999in}{1.725502in}}%
\pgfpathlineto{\pgfqpoint{5.599674in}{1.726149in}}%
\pgfpathlineto{\pgfqpoint{5.602352in}{1.727536in}}%
\pgfpathlineto{\pgfqpoint{5.605073in}{1.726423in}}%
\pgfpathlineto{\pgfqpoint{5.607698in}{1.727361in}}%
\pgfpathlineto{\pgfqpoint{5.610389in}{1.725447in}}%
\pgfpathlineto{\pgfqpoint{5.613235in}{1.726996in}}%
\pgfpathlineto{\pgfqpoint{5.615743in}{1.728140in}}%
\pgfpathlineto{\pgfqpoint{5.618526in}{1.729710in}}%
\pgfpathlineto{\pgfqpoint{5.621102in}{1.724043in}}%
\pgfpathlineto{\pgfqpoint{5.623868in}{1.724651in}}%
\pgfpathlineto{\pgfqpoint{5.626460in}{1.723790in}}%
\pgfpathlineto{\pgfqpoint{5.629232in}{1.726011in}}%
\pgfpathlineto{\pgfqpoint{5.631815in}{1.728234in}}%
\pgfpathlineto{\pgfqpoint{5.634496in}{1.728500in}}%
\pgfpathlineto{\pgfqpoint{5.637172in}{1.729452in}}%
\pgfpathlineto{\pgfqpoint{5.639852in}{1.726584in}}%
\pgfpathlineto{\pgfqpoint{5.642518in}{1.729916in}}%
\pgfpathlineto{\pgfqpoint{5.645243in}{1.738549in}}%
\pgfpathlineto{\pgfqpoint{5.648008in}{1.747028in}}%
\pgfpathlineto{\pgfqpoint{5.650563in}{1.740423in}}%
\pgfpathlineto{\pgfqpoint{5.653376in}{1.723652in}}%
\pgfpathlineto{\pgfqpoint{5.655919in}{1.727370in}}%
\pgfpathlineto{\pgfqpoint{5.658723in}{1.729096in}}%
\pgfpathlineto{\pgfqpoint{5.661273in}{1.726063in}}%
\pgfpathlineto{\pgfqpoint{5.664099in}{1.728584in}}%
\pgfpathlineto{\pgfqpoint{5.666632in}{1.733033in}}%
\pgfpathlineto{\pgfqpoint{5.669313in}{1.727011in}}%
\pgfpathlineto{\pgfqpoint{5.671991in}{1.727728in}}%
\pgfpathlineto{\pgfqpoint{5.674667in}{1.732491in}}%
\pgfpathlineto{\pgfqpoint{5.677486in}{1.728651in}}%
\pgfpathlineto{\pgfqpoint{5.680027in}{1.733938in}}%
\pgfpathlineto{\pgfqpoint{5.682836in}{1.732355in}}%
\pgfpathlineto{\pgfqpoint{5.685385in}{1.738566in}}%
\pgfpathlineto{\pgfqpoint{5.688159in}{1.740742in}}%
\pgfpathlineto{\pgfqpoint{5.690730in}{1.732313in}}%
\pgfpathlineto{\pgfqpoint{5.693473in}{1.735711in}}%
\pgfpathlineto{\pgfqpoint{5.696101in}{1.738049in}}%
\pgfpathlineto{\pgfqpoint{5.698775in}{1.731931in}}%
\pgfpathlineto{\pgfqpoint{5.701453in}{1.732249in}}%
\pgfpathlineto{\pgfqpoint{5.704130in}{1.733376in}}%
\pgfpathlineto{\pgfqpoint{5.706800in}{1.734434in}}%
\pgfpathlineto{\pgfqpoint{5.709490in}{1.738108in}}%
\pgfpathlineto{\pgfqpoint{5.712291in}{1.737139in}}%
\pgfpathlineto{\pgfqpoint{5.714834in}{1.735096in}}%
\pgfpathlineto{\pgfqpoint{5.717671in}{1.734849in}}%
\pgfpathlineto{\pgfqpoint{5.720201in}{1.736261in}}%
\pgfpathlineto{\pgfqpoint{5.722950in}{1.734236in}}%
\pgfpathlineto{\pgfqpoint{5.725548in}{1.735958in}}%
\pgfpathlineto{\pgfqpoint{5.728339in}{1.728379in}}%
\pgfpathlineto{\pgfqpoint{5.730919in}{1.731256in}}%
\pgfpathlineto{\pgfqpoint{5.733594in}{1.735497in}}%
\pgfpathlineto{\pgfqpoint{5.736276in}{1.736654in}}%
\pgfpathlineto{\pgfqpoint{5.738974in}{1.741612in}}%
\pgfpathlineto{\pgfqpoint{5.741745in}{1.731242in}}%
\pgfpathlineto{\pgfqpoint{5.744310in}{1.734827in}}%
\pgfpathlineto{\pgfqpoint{5.744310in}{0.413320in}}%
\pgfpathlineto{\pgfqpoint{5.744310in}{0.413320in}}%
\pgfpathlineto{\pgfqpoint{5.741745in}{0.413320in}}%
\pgfpathlineto{\pgfqpoint{5.738974in}{0.413320in}}%
\pgfpathlineto{\pgfqpoint{5.736276in}{0.413320in}}%
\pgfpathlineto{\pgfqpoint{5.733594in}{0.413320in}}%
\pgfpathlineto{\pgfqpoint{5.730919in}{0.413320in}}%
\pgfpathlineto{\pgfqpoint{5.728339in}{0.413320in}}%
\pgfpathlineto{\pgfqpoint{5.725548in}{0.413320in}}%
\pgfpathlineto{\pgfqpoint{5.722950in}{0.413320in}}%
\pgfpathlineto{\pgfqpoint{5.720201in}{0.413320in}}%
\pgfpathlineto{\pgfqpoint{5.717671in}{0.413320in}}%
\pgfpathlineto{\pgfqpoint{5.714834in}{0.413320in}}%
\pgfpathlineto{\pgfqpoint{5.712291in}{0.413320in}}%
\pgfpathlineto{\pgfqpoint{5.709490in}{0.413320in}}%
\pgfpathlineto{\pgfqpoint{5.706800in}{0.413320in}}%
\pgfpathlineto{\pgfqpoint{5.704130in}{0.413320in}}%
\pgfpathlineto{\pgfqpoint{5.701453in}{0.413320in}}%
\pgfpathlineto{\pgfqpoint{5.698775in}{0.413320in}}%
\pgfpathlineto{\pgfqpoint{5.696101in}{0.413320in}}%
\pgfpathlineto{\pgfqpoint{5.693473in}{0.413320in}}%
\pgfpathlineto{\pgfqpoint{5.690730in}{0.413320in}}%
\pgfpathlineto{\pgfqpoint{5.688159in}{0.413320in}}%
\pgfpathlineto{\pgfqpoint{5.685385in}{0.413320in}}%
\pgfpathlineto{\pgfqpoint{5.682836in}{0.413320in}}%
\pgfpathlineto{\pgfqpoint{5.680027in}{0.413320in}}%
\pgfpathlineto{\pgfqpoint{5.677486in}{0.413320in}}%
\pgfpathlineto{\pgfqpoint{5.674667in}{0.413320in}}%
\pgfpathlineto{\pgfqpoint{5.671991in}{0.413320in}}%
\pgfpathlineto{\pgfqpoint{5.669313in}{0.413320in}}%
\pgfpathlineto{\pgfqpoint{5.666632in}{0.413320in}}%
\pgfpathlineto{\pgfqpoint{5.664099in}{0.413320in}}%
\pgfpathlineto{\pgfqpoint{5.661273in}{0.413320in}}%
\pgfpathlineto{\pgfqpoint{5.658723in}{0.413320in}}%
\pgfpathlineto{\pgfqpoint{5.655919in}{0.413320in}}%
\pgfpathlineto{\pgfqpoint{5.653376in}{0.413320in}}%
\pgfpathlineto{\pgfqpoint{5.650563in}{0.413320in}}%
\pgfpathlineto{\pgfqpoint{5.648008in}{0.413320in}}%
\pgfpathlineto{\pgfqpoint{5.645243in}{0.413320in}}%
\pgfpathlineto{\pgfqpoint{5.642518in}{0.413320in}}%
\pgfpathlineto{\pgfqpoint{5.639852in}{0.413320in}}%
\pgfpathlineto{\pgfqpoint{5.637172in}{0.413320in}}%
\pgfpathlineto{\pgfqpoint{5.634496in}{0.413320in}}%
\pgfpathlineto{\pgfqpoint{5.631815in}{0.413320in}}%
\pgfpathlineto{\pgfqpoint{5.629232in}{0.413320in}}%
\pgfpathlineto{\pgfqpoint{5.626460in}{0.413320in}}%
\pgfpathlineto{\pgfqpoint{5.623868in}{0.413320in}}%
\pgfpathlineto{\pgfqpoint{5.621102in}{0.413320in}}%
\pgfpathlineto{\pgfqpoint{5.618526in}{0.413320in}}%
\pgfpathlineto{\pgfqpoint{5.615743in}{0.413320in}}%
\pgfpathlineto{\pgfqpoint{5.613235in}{0.413320in}}%
\pgfpathlineto{\pgfqpoint{5.610389in}{0.413320in}}%
\pgfpathlineto{\pgfqpoint{5.607698in}{0.413320in}}%
\pgfpathlineto{\pgfqpoint{5.605073in}{0.413320in}}%
\pgfpathlineto{\pgfqpoint{5.602352in}{0.413320in}}%
\pgfpathlineto{\pgfqpoint{5.599674in}{0.413320in}}%
\pgfpathlineto{\pgfqpoint{5.596999in}{0.413320in}}%
\pgfpathlineto{\pgfqpoint{5.594368in}{0.413320in}}%
\pgfpathlineto{\pgfqpoint{5.591641in}{0.413320in}}%
\pgfpathlineto{\pgfqpoint{5.589040in}{0.413320in}}%
\pgfpathlineto{\pgfqpoint{5.586269in}{0.413320in}}%
\pgfpathlineto{\pgfqpoint{5.583709in}{0.413320in}}%
\pgfpathlineto{\pgfqpoint{5.580914in}{0.413320in}}%
\pgfpathlineto{\pgfqpoint{5.578342in}{0.413320in}}%
\pgfpathlineto{\pgfqpoint{5.575596in}{0.413320in}}%
\pgfpathlineto{\pgfqpoint{5.572893in}{0.413320in}}%
\pgfpathlineto{\pgfqpoint{5.570215in}{0.413320in}}%
\pgfpathlineto{\pgfqpoint{5.567536in}{0.413320in}}%
\pgfpathlineto{\pgfqpoint{5.564940in}{0.413320in}}%
\pgfpathlineto{\pgfqpoint{5.562180in}{0.413320in}}%
\pgfpathlineto{\pgfqpoint{5.559612in}{0.413320in}}%
\pgfpathlineto{\pgfqpoint{5.556822in}{0.413320in}}%
\pgfpathlineto{\pgfqpoint{5.554198in}{0.413320in}}%
\pgfpathlineto{\pgfqpoint{5.551457in}{0.413320in}}%
\pgfpathlineto{\pgfqpoint{5.548921in}{0.413320in}}%
\pgfpathlineto{\pgfqpoint{5.546110in}{0.413320in}}%
\pgfpathlineto{\pgfqpoint{5.543421in}{0.413320in}}%
\pgfpathlineto{\pgfqpoint{5.540750in}{0.413320in}}%
\pgfpathlineto{\pgfqpoint{5.538074in}{0.413320in}}%
\pgfpathlineto{\pgfqpoint{5.535395in}{0.413320in}}%
\pgfpathlineto{\pgfqpoint{5.532717in}{0.413320in}}%
\pgfpathlineto{\pgfqpoint{5.530148in}{0.413320in}}%
\pgfpathlineto{\pgfqpoint{5.527360in}{0.413320in}}%
\pgfpathlineto{\pgfqpoint{5.524756in}{0.413320in}}%
\pgfpathlineto{\pgfqpoint{5.522003in}{0.413320in}}%
\pgfpathlineto{\pgfqpoint{5.519433in}{0.413320in}}%
\pgfpathlineto{\pgfqpoint{5.516646in}{0.413320in}}%
\pgfpathlineto{\pgfqpoint{5.514080in}{0.413320in}}%
\pgfpathlineto{\pgfqpoint{5.511290in}{0.413320in}}%
\pgfpathlineto{\pgfqpoint{5.508612in}{0.413320in}}%
\pgfpathlineto{\pgfqpoint{5.505933in}{0.413320in}}%
\pgfpathlineto{\pgfqpoint{5.503255in}{0.413320in}}%
\pgfpathlineto{\pgfqpoint{5.500687in}{0.413320in}}%
\pgfpathlineto{\pgfqpoint{5.497898in}{0.413320in}}%
\pgfpathlineto{\pgfqpoint{5.495346in}{0.413320in}}%
\pgfpathlineto{\pgfqpoint{5.492541in}{0.413320in}}%
\pgfpathlineto{\pgfqpoint{5.490000in}{0.413320in}}%
\pgfpathlineto{\pgfqpoint{5.487176in}{0.413320in}}%
\pgfpathlineto{\pgfqpoint{5.484641in}{0.413320in}}%
\pgfpathlineto{\pgfqpoint{5.481825in}{0.413320in}}%
\pgfpathlineto{\pgfqpoint{5.479152in}{0.413320in}}%
\pgfpathlineto{\pgfqpoint{5.476458in}{0.413320in}}%
\pgfpathlineto{\pgfqpoint{5.473792in}{0.413320in}}%
\pgfpathlineto{\pgfqpoint{5.471113in}{0.413320in}}%
\pgfpathlineto{\pgfqpoint{5.468425in}{0.413320in}}%
\pgfpathlineto{\pgfqpoint{5.465888in}{0.413320in}}%
\pgfpathlineto{\pgfqpoint{5.463079in}{0.413320in}}%
\pgfpathlineto{\pgfqpoint{5.460489in}{0.413320in}}%
\pgfpathlineto{\pgfqpoint{5.457721in}{0.413320in}}%
\pgfpathlineto{\pgfqpoint{5.455168in}{0.413320in}}%
\pgfpathlineto{\pgfqpoint{5.452365in}{0.413320in}}%
\pgfpathlineto{\pgfqpoint{5.449769in}{0.413320in}}%
\pgfpathlineto{\pgfqpoint{5.447021in}{0.413320in}}%
\pgfpathlineto{\pgfqpoint{5.444328in}{0.413320in}}%
\pgfpathlineto{\pgfqpoint{5.441698in}{0.413320in}}%
\pgfpathlineto{\pgfqpoint{5.438974in}{0.413320in}}%
\pgfpathlineto{\pgfqpoint{5.436295in}{0.413320in}}%
\pgfpathlineto{\pgfqpoint{5.433616in}{0.413320in}}%
\pgfpathlineto{\pgfqpoint{5.431015in}{0.413320in}}%
\pgfpathlineto{\pgfqpoint{5.428259in}{0.413320in}}%
\pgfpathlineto{\pgfqpoint{5.425661in}{0.413320in}}%
\pgfpathlineto{\pgfqpoint{5.422897in}{0.413320in}}%
\pgfpathlineto{\pgfqpoint{5.420304in}{0.413320in}}%
\pgfpathlineto{\pgfqpoint{5.417547in}{0.413320in}}%
\pgfpathlineto{\pgfqpoint{5.414954in}{0.413320in}}%
\pgfpathlineto{\pgfqpoint{5.412190in}{0.413320in}}%
\pgfpathlineto{\pgfqpoint{5.409507in}{0.413320in}}%
\pgfpathlineto{\pgfqpoint{5.406832in}{0.413320in}}%
\pgfpathlineto{\pgfqpoint{5.404154in}{0.413320in}}%
\pgfpathlineto{\pgfqpoint{5.401576in}{0.413320in}}%
\pgfpathlineto{\pgfqpoint{5.398784in}{0.413320in}}%
\pgfpathlineto{\pgfqpoint{5.396219in}{0.413320in}}%
\pgfpathlineto{\pgfqpoint{5.393441in}{0.413320in}}%
\pgfpathlineto{\pgfqpoint{5.390900in}{0.413320in}}%
\pgfpathlineto{\pgfqpoint{5.388083in}{0.413320in}}%
\pgfpathlineto{\pgfqpoint{5.385550in}{0.413320in}}%
\pgfpathlineto{\pgfqpoint{5.382725in}{0.413320in}}%
\pgfpathlineto{\pgfqpoint{5.380048in}{0.413320in}}%
\pgfpathlineto{\pgfqpoint{5.377370in}{0.413320in}}%
\pgfpathlineto{\pgfqpoint{5.374692in}{0.413320in}}%
\pgfpathlineto{\pgfqpoint{5.372013in}{0.413320in}}%
\pgfpathlineto{\pgfqpoint{5.369335in}{0.413320in}}%
\pgfpathlineto{\pgfqpoint{5.366727in}{0.413320in}}%
\pgfpathlineto{\pgfqpoint{5.363966in}{0.413320in}}%
\pgfpathlineto{\pgfqpoint{5.361370in}{0.413320in}}%
\pgfpathlineto{\pgfqpoint{5.358612in}{0.413320in}}%
\pgfpathlineto{\pgfqpoint{5.356056in}{0.413320in}}%
\pgfpathlineto{\pgfqpoint{5.353262in}{0.413320in}}%
\pgfpathlineto{\pgfqpoint{5.350723in}{0.413320in}}%
\pgfpathlineto{\pgfqpoint{5.347905in}{0.413320in}}%
\pgfpathlineto{\pgfqpoint{5.345224in}{0.413320in}}%
\pgfpathlineto{\pgfqpoint{5.342549in}{0.413320in}}%
\pgfpathlineto{\pgfqpoint{5.339872in}{0.413320in}}%
\pgfpathlineto{\pgfqpoint{5.337353in}{0.413320in}}%
\pgfpathlineto{\pgfqpoint{5.334510in}{0.413320in}}%
\pgfpathlineto{\pgfqpoint{5.331973in}{0.413320in}}%
\pgfpathlineto{\pgfqpoint{5.329159in}{0.413320in}}%
\pgfpathlineto{\pgfqpoint{5.326564in}{0.413320in}}%
\pgfpathlineto{\pgfqpoint{5.323802in}{0.413320in}}%
\pgfpathlineto{\pgfqpoint{5.321256in}{0.413320in}}%
\pgfpathlineto{\pgfqpoint{5.318430in}{0.413320in}}%
\pgfpathlineto{\pgfqpoint{5.315754in}{0.413320in}}%
\pgfpathlineto{\pgfqpoint{5.313089in}{0.413320in}}%
\pgfpathlineto{\pgfqpoint{5.310411in}{0.413320in}}%
\pgfpathlineto{\pgfqpoint{5.307731in}{0.413320in}}%
\pgfpathlineto{\pgfqpoint{5.305054in}{0.413320in}}%
\pgfpathlineto{\pgfqpoint{5.302443in}{0.413320in}}%
\pgfpathlineto{\pgfqpoint{5.299696in}{0.413320in}}%
\pgfpathlineto{\pgfqpoint{5.297140in}{0.413320in}}%
\pgfpathlineto{\pgfqpoint{5.294339in}{0.413320in}}%
\pgfpathlineto{\pgfqpoint{5.291794in}{0.413320in}}%
\pgfpathlineto{\pgfqpoint{5.288984in}{0.413320in}}%
\pgfpathlineto{\pgfqpoint{5.286436in}{0.413320in}}%
\pgfpathlineto{\pgfqpoint{5.283631in}{0.413320in}}%
\pgfpathlineto{\pgfqpoint{5.280947in}{0.413320in}}%
\pgfpathlineto{\pgfqpoint{5.278322in}{0.413320in}}%
\pgfpathlineto{\pgfqpoint{5.275589in}{0.413320in}}%
\pgfpathlineto{\pgfqpoint{5.272913in}{0.413320in}}%
\pgfpathlineto{\pgfqpoint{5.270238in}{0.413320in}}%
\pgfpathlineto{\pgfqpoint{5.267691in}{0.413320in}}%
\pgfpathlineto{\pgfqpoint{5.264876in}{0.413320in}}%
\pgfpathlineto{\pgfqpoint{5.262264in}{0.413320in}}%
\pgfpathlineto{\pgfqpoint{5.259511in}{0.413320in}}%
\pgfpathlineto{\pgfqpoint{5.256973in}{0.413320in}}%
\pgfpathlineto{\pgfqpoint{5.254236in}{0.413320in}}%
\pgfpathlineto{\pgfqpoint{5.251590in}{0.413320in}}%
\pgfpathlineto{\pgfqpoint{5.248816in}{0.413320in}}%
\pgfpathlineto{\pgfqpoint{5.246130in}{0.413320in}}%
\pgfpathlineto{\pgfqpoint{5.243445in}{0.413320in}}%
\pgfpathlineto{\pgfqpoint{5.240777in}{0.413320in}}%
\pgfpathlineto{\pgfqpoint{5.238173in}{0.413320in}}%
\pgfpathlineto{\pgfqpoint{5.235409in}{0.413320in}}%
\pgfpathlineto{\pgfqpoint{5.232855in}{0.413320in}}%
\pgfpathlineto{\pgfqpoint{5.230059in}{0.413320in}}%
\pgfpathlineto{\pgfqpoint{5.227470in}{0.413320in}}%
\pgfpathlineto{\pgfqpoint{5.224695in}{0.413320in}}%
\pgfpathlineto{\pgfqpoint{5.222151in}{0.413320in}}%
\pgfpathlineto{\pgfqpoint{5.219345in}{0.413320in}}%
\pgfpathlineto{\pgfqpoint{5.216667in}{0.413320in}}%
\pgfpathlineto{\pgfqpoint{5.214027in}{0.413320in}}%
\pgfpathlineto{\pgfqpoint{5.211299in}{0.413320in}}%
\pgfpathlineto{\pgfqpoint{5.208630in}{0.413320in}}%
\pgfpathlineto{\pgfqpoint{5.205952in}{0.413320in}}%
\pgfpathlineto{\pgfqpoint{5.203388in}{0.413320in}}%
\pgfpathlineto{\pgfqpoint{5.200594in}{0.413320in}}%
\pgfpathlineto{\pgfqpoint{5.198008in}{0.413320in}}%
\pgfpathlineto{\pgfqpoint{5.195239in}{0.413320in}}%
\pgfpathlineto{\pgfqpoint{5.192680in}{0.413320in}}%
\pgfpathlineto{\pgfqpoint{5.189880in}{0.413320in}}%
\pgfpathlineto{\pgfqpoint{5.187294in}{0.413320in}}%
\pgfpathlineto{\pgfqpoint{5.184522in}{0.413320in}}%
\pgfpathlineto{\pgfqpoint{5.181848in}{0.413320in}}%
\pgfpathlineto{\pgfqpoint{5.179188in}{0.413320in}}%
\pgfpathlineto{\pgfqpoint{5.176477in}{0.413320in}}%
\pgfpathlineto{\pgfqpoint{5.173925in}{0.413320in}}%
\pgfpathlineto{\pgfqpoint{5.171133in}{0.413320in}}%
\pgfpathlineto{\pgfqpoint{5.168591in}{0.413320in}}%
\pgfpathlineto{\pgfqpoint{5.165775in}{0.413320in}}%
\pgfpathlineto{\pgfqpoint{5.163243in}{0.413320in}}%
\pgfpathlineto{\pgfqpoint{5.160420in}{0.413320in}}%
\pgfpathlineto{\pgfqpoint{5.157815in}{0.413320in}}%
\pgfpathlineto{\pgfqpoint{5.155059in}{0.413320in}}%
\pgfpathlineto{\pgfqpoint{5.152382in}{0.413320in}}%
\pgfpathlineto{\pgfqpoint{5.149734in}{0.413320in}}%
\pgfpathlineto{\pgfqpoint{5.147029in}{0.413320in}}%
\pgfpathlineto{\pgfqpoint{5.144349in}{0.413320in}}%
\pgfpathlineto{\pgfqpoint{5.141660in}{0.413320in}}%
\pgfpathlineto{\pgfqpoint{5.139072in}{0.413320in}}%
\pgfpathlineto{\pgfqpoint{5.136311in}{0.413320in}}%
\pgfpathlineto{\pgfqpoint{5.133716in}{0.413320in}}%
\pgfpathlineto{\pgfqpoint{5.130953in}{0.413320in}}%
\pgfpathlineto{\pgfqpoint{5.128421in}{0.413320in}}%
\pgfpathlineto{\pgfqpoint{5.125599in}{0.413320in}}%
\pgfpathlineto{\pgfqpoint{5.123042in}{0.413320in}}%
\pgfpathlineto{\pgfqpoint{5.120243in}{0.413320in}}%
\pgfpathlineto{\pgfqpoint{5.117550in}{0.413320in}}%
\pgfpathlineto{\pgfqpoint{5.114887in}{0.413320in}}%
\pgfpathlineto{\pgfqpoint{5.112209in}{0.413320in}}%
\pgfpathlineto{\pgfqpoint{5.109530in}{0.413320in}}%
\pgfpathlineto{\pgfqpoint{5.106842in}{0.413320in}}%
\pgfpathlineto{\pgfqpoint{5.104312in}{0.413320in}}%
\pgfpathlineto{\pgfqpoint{5.101496in}{0.413320in}}%
\pgfpathlineto{\pgfqpoint{5.098948in}{0.413320in}}%
\pgfpathlineto{\pgfqpoint{5.096142in}{0.413320in}}%
\pgfpathlineto{\pgfqpoint{5.093579in}{0.413320in}}%
\pgfpathlineto{\pgfqpoint{5.090788in}{0.413320in}}%
\pgfpathlineto{\pgfqpoint{5.088103in}{0.413320in}}%
\pgfpathlineto{\pgfqpoint{5.085426in}{0.413320in}}%
\pgfpathlineto{\pgfqpoint{5.082746in}{0.413320in}}%
\pgfpathlineto{\pgfqpoint{5.080067in}{0.413320in}}%
\pgfpathlineto{\pgfqpoint{5.077390in}{0.413320in}}%
\pgfpathlineto{\pgfqpoint{5.074851in}{0.413320in}}%
\pgfpathlineto{\pgfqpoint{5.072030in}{0.413320in}}%
\pgfpathlineto{\pgfqpoint{5.069463in}{0.413320in}}%
\pgfpathlineto{\pgfqpoint{5.066677in}{0.413320in}}%
\pgfpathlineto{\pgfqpoint{5.064144in}{0.413320in}}%
\pgfpathlineto{\pgfqpoint{5.061315in}{0.413320in}}%
\pgfpathlineto{\pgfqpoint{5.058711in}{0.413320in}}%
\pgfpathlineto{\pgfqpoint{5.055952in}{0.413320in}}%
\pgfpathlineto{\pgfqpoint{5.053284in}{0.413320in}}%
\pgfpathlineto{\pgfqpoint{5.050606in}{0.413320in}}%
\pgfpathlineto{\pgfqpoint{5.047924in}{0.413320in}}%
\pgfpathlineto{\pgfqpoint{5.045249in}{0.413320in}}%
\pgfpathlineto{\pgfqpoint{5.042572in}{0.413320in}}%
\pgfpathlineto{\pgfqpoint{5.039962in}{0.413320in}}%
\pgfpathlineto{\pgfqpoint{5.037214in}{0.413320in}}%
\pgfpathlineto{\pgfqpoint{5.034649in}{0.413320in}}%
\pgfpathlineto{\pgfqpoint{5.031849in}{0.413320in}}%
\pgfpathlineto{\pgfqpoint{5.029275in}{0.413320in}}%
\pgfpathlineto{\pgfqpoint{5.026501in}{0.413320in}}%
\pgfpathlineto{\pgfqpoint{5.023927in}{0.413320in}}%
\pgfpathlineto{\pgfqpoint{5.021147in}{0.413320in}}%
\pgfpathlineto{\pgfqpoint{5.018466in}{0.413320in}}%
\pgfpathlineto{\pgfqpoint{5.015820in}{0.413320in}}%
\pgfpathlineto{\pgfqpoint{5.013104in}{0.413320in}}%
\pgfpathlineto{\pgfqpoint{5.010562in}{0.413320in}}%
\pgfpathlineto{\pgfqpoint{5.007751in}{0.413320in}}%
\pgfpathlineto{\pgfqpoint{5.005178in}{0.413320in}}%
\pgfpathlineto{\pgfqpoint{5.002384in}{0.413320in}}%
\pgfpathlineto{\pgfqpoint{4.999780in}{0.413320in}}%
\pgfpathlineto{\pgfqpoint{4.997028in}{0.413320in}}%
\pgfpathlineto{\pgfqpoint{4.994390in}{0.413320in}}%
\pgfpathlineto{\pgfqpoint{4.991683in}{0.413320in}}%
\pgfpathlineto{\pgfqpoint{4.989001in}{0.413320in}}%
\pgfpathlineto{\pgfqpoint{4.986325in}{0.413320in}}%
\pgfpathlineto{\pgfqpoint{4.983637in}{0.413320in}}%
\pgfpathlineto{\pgfqpoint{4.980967in}{0.413320in}}%
\pgfpathlineto{\pgfqpoint{4.978287in}{0.413320in}}%
\pgfpathlineto{\pgfqpoint{4.975703in}{0.413320in}}%
\pgfpathlineto{\pgfqpoint{4.972933in}{0.413320in}}%
\pgfpathlineto{\pgfqpoint{4.970314in}{0.413320in}}%
\pgfpathlineto{\pgfqpoint{4.967575in}{0.413320in}}%
\pgfpathlineto{\pgfqpoint{4.965002in}{0.413320in}}%
\pgfpathlineto{\pgfqpoint{4.962219in}{0.413320in}}%
\pgfpathlineto{\pgfqpoint{4.959689in}{0.413320in}}%
\pgfpathlineto{\pgfqpoint{4.956862in}{0.413320in}}%
\pgfpathlineto{\pgfqpoint{4.954182in}{0.413320in}}%
\pgfpathlineto{\pgfqpoint{4.951504in}{0.413320in}}%
\pgfpathlineto{\pgfqpoint{4.948827in}{0.413320in}}%
\pgfpathlineto{\pgfqpoint{4.946151in}{0.413320in}}%
\pgfpathlineto{\pgfqpoint{4.943466in}{0.413320in}}%
\pgfpathlineto{\pgfqpoint{4.940881in}{0.413320in}}%
\pgfpathlineto{\pgfqpoint{4.938112in}{0.413320in}}%
\pgfpathlineto{\pgfqpoint{4.935515in}{0.413320in}}%
\pgfpathlineto{\pgfqpoint{4.932742in}{0.413320in}}%
\pgfpathlineto{\pgfqpoint{4.930170in}{0.413320in}}%
\pgfpathlineto{\pgfqpoint{4.927400in}{0.413320in}}%
\pgfpathlineto{\pgfqpoint{4.924708in}{0.413320in}}%
\pgfpathlineto{\pgfqpoint{4.922041in}{0.413320in}}%
\pgfpathlineto{\pgfqpoint{4.919352in}{0.413320in}}%
\pgfpathlineto{\pgfqpoint{4.916681in}{0.413320in}}%
\pgfpathlineto{\pgfqpoint{4.914009in}{0.413320in}}%
\pgfpathlineto{\pgfqpoint{4.911435in}{0.413320in}}%
\pgfpathlineto{\pgfqpoint{4.908648in}{0.413320in}}%
\pgfpathlineto{\pgfqpoint{4.906096in}{0.413320in}}%
\pgfpathlineto{\pgfqpoint{4.903295in}{0.413320in}}%
\pgfpathlineto{\pgfqpoint{4.900712in}{0.413320in}}%
\pgfpathlineto{\pgfqpoint{4.897938in}{0.413320in}}%
\pgfpathlineto{\pgfqpoint{4.895399in}{0.413320in}}%
\pgfpathlineto{\pgfqpoint{4.892611in}{0.413320in}}%
\pgfpathlineto{\pgfqpoint{4.889902in}{0.413320in}}%
\pgfpathlineto{\pgfqpoint{4.887211in}{0.413320in}}%
\pgfpathlineto{\pgfqpoint{4.884540in}{0.413320in}}%
\pgfpathlineto{\pgfqpoint{4.881864in}{0.413320in}}%
\pgfpathlineto{\pgfqpoint{4.879180in}{0.413320in}}%
\pgfpathlineto{\pgfqpoint{4.876636in}{0.413320in}}%
\pgfpathlineto{\pgfqpoint{4.873832in}{0.413320in}}%
\pgfpathlineto{\pgfqpoint{4.871209in}{0.413320in}}%
\pgfpathlineto{\pgfqpoint{4.868474in}{0.413320in}}%
\pgfpathlineto{\pgfqpoint{4.865910in}{0.413320in}}%
\pgfpathlineto{\pgfqpoint{4.863116in}{0.413320in}}%
\pgfpathlineto{\pgfqpoint{4.860544in}{0.413320in}}%
\pgfpathlineto{\pgfqpoint{4.857807in}{0.413320in}}%
\pgfpathlineto{\pgfqpoint{4.855070in}{0.413320in}}%
\pgfpathlineto{\pgfqpoint{4.852404in}{0.413320in}}%
\pgfpathlineto{\pgfqpoint{4.849715in}{0.413320in}}%
\pgfpathlineto{\pgfqpoint{4.847127in}{0.413320in}}%
\pgfpathlineto{\pgfqpoint{4.844361in}{0.413320in}}%
\pgfpathlineto{\pgfqpoint{4.842380in}{0.413320in}}%
\pgfpathlineto{\pgfqpoint{4.839922in}{0.413320in}}%
\pgfpathlineto{\pgfqpoint{4.837992in}{0.413320in}}%
\pgfpathlineto{\pgfqpoint{4.833657in}{0.413320in}}%
\pgfpathlineto{\pgfqpoint{4.831045in}{0.413320in}}%
\pgfpathlineto{\pgfqpoint{4.828291in}{0.413320in}}%
\pgfpathlineto{\pgfqpoint{4.825619in}{0.413320in}}%
\pgfpathlineto{\pgfqpoint{4.822945in}{0.413320in}}%
\pgfpathlineto{\pgfqpoint{4.820265in}{0.413320in}}%
\pgfpathlineto{\pgfqpoint{4.817587in}{0.413320in}}%
\pgfpathlineto{\pgfqpoint{4.814907in}{0.413320in}}%
\pgfpathlineto{\pgfqpoint{4.812377in}{0.413320in}}%
\pgfpathlineto{\pgfqpoint{4.809538in}{0.413320in}}%
\pgfpathlineto{\pgfqpoint{4.807017in}{0.413320in}}%
\pgfpathlineto{\pgfqpoint{4.804193in}{0.413320in}}%
\pgfpathlineto{\pgfqpoint{4.801586in}{0.413320in}}%
\pgfpathlineto{\pgfqpoint{4.798830in}{0.413320in}}%
\pgfpathlineto{\pgfqpoint{4.796274in}{0.413320in}}%
\pgfpathlineto{\pgfqpoint{4.793512in}{0.413320in}}%
\pgfpathlineto{\pgfqpoint{4.790798in}{0.413320in}}%
\pgfpathlineto{\pgfqpoint{4.788116in}{0.413320in}}%
\pgfpathlineto{\pgfqpoint{4.785445in}{0.413320in}}%
\pgfpathlineto{\pgfqpoint{4.782872in}{0.413320in}}%
\pgfpathlineto{\pgfqpoint{4.780083in}{0.413320in}}%
\pgfpathlineto{\pgfqpoint{4.777535in}{0.413320in}}%
\pgfpathlineto{\pgfqpoint{4.774732in}{0.413320in}}%
\pgfpathlineto{\pgfqpoint{4.772198in}{0.413320in}}%
\pgfpathlineto{\pgfqpoint{4.769367in}{0.413320in}}%
\pgfpathlineto{\pgfqpoint{4.766783in}{0.413320in}}%
\pgfpathlineto{\pgfqpoint{4.764018in}{0.413320in}}%
\pgfpathlineto{\pgfqpoint{4.761337in}{0.413320in}}%
\pgfpathlineto{\pgfqpoint{4.758653in}{0.413320in}}%
\pgfpathlineto{\pgfqpoint{4.755983in}{0.413320in}}%
\pgfpathlineto{\pgfqpoint{4.753298in}{0.413320in}}%
\pgfpathlineto{\pgfqpoint{4.750627in}{0.413320in}}%
\pgfpathlineto{\pgfqpoint{4.748081in}{0.413320in}}%
\pgfpathlineto{\pgfqpoint{4.745256in}{0.413320in}}%
\pgfpathlineto{\pgfqpoint{4.742696in}{0.413320in}}%
\pgfpathlineto{\pgfqpoint{4.739912in}{0.413320in}}%
\pgfpathlineto{\pgfqpoint{4.737348in}{0.413320in}}%
\pgfpathlineto{\pgfqpoint{4.734552in}{0.413320in}}%
\pgfpathlineto{\pgfqpoint{4.731901in}{0.413320in}}%
\pgfpathlineto{\pgfqpoint{4.729233in}{0.413320in}}%
\pgfpathlineto{\pgfqpoint{4.726508in}{0.413320in}}%
\pgfpathlineto{\pgfqpoint{4.723873in}{0.413320in}}%
\pgfpathlineto{\pgfqpoint{4.721160in}{0.413320in}}%
\pgfpathlineto{\pgfqpoint{4.718486in}{0.413320in}}%
\pgfpathlineto{\pgfqpoint{4.715806in}{0.413320in}}%
\pgfpathlineto{\pgfqpoint{4.713275in}{0.413320in}}%
\pgfpathlineto{\pgfqpoint{4.710437in}{0.413320in}}%
\pgfpathlineto{\pgfqpoint{4.707824in}{0.413320in}}%
\pgfpathlineto{\pgfqpoint{4.705094in}{0.413320in}}%
\pgfpathlineto{\pgfqpoint{4.702517in}{0.413320in}}%
\pgfpathlineto{\pgfqpoint{4.699734in}{0.413320in}}%
\pgfpathlineto{\pgfqpoint{4.697170in}{0.413320in}}%
\pgfpathlineto{\pgfqpoint{4.694381in}{0.413320in}}%
\pgfpathlineto{\pgfqpoint{4.691694in}{0.413320in}}%
\pgfpathlineto{\pgfqpoint{4.689051in}{0.413320in}}%
\pgfpathlineto{\pgfqpoint{4.686337in}{0.413320in}}%
\pgfpathlineto{\pgfqpoint{4.683799in}{0.413320in}}%
\pgfpathlineto{\pgfqpoint{4.680988in}{0.413320in}}%
\pgfpathlineto{\pgfqpoint{4.678448in}{0.413320in}}%
\pgfpathlineto{\pgfqpoint{4.675619in}{0.413320in}}%
\pgfpathlineto{\pgfqpoint{4.673068in}{0.413320in}}%
\pgfpathlineto{\pgfqpoint{4.670261in}{0.413320in}}%
\pgfpathlineto{\pgfqpoint{4.667764in}{0.413320in}}%
\pgfpathlineto{\pgfqpoint{4.664923in}{0.413320in}}%
\pgfpathlineto{\pgfqpoint{4.662237in}{0.413320in}}%
\pgfpathlineto{\pgfqpoint{4.659590in}{0.413320in}}%
\pgfpathlineto{\pgfqpoint{4.656873in}{0.413320in}}%
\pgfpathlineto{\pgfqpoint{4.654203in}{0.413320in}}%
\pgfpathlineto{\pgfqpoint{4.651524in}{0.413320in}}%
\pgfpathlineto{\pgfqpoint{4.648922in}{0.413320in}}%
\pgfpathlineto{\pgfqpoint{4.646169in}{0.413320in}}%
\pgfpathlineto{\pgfqpoint{4.643628in}{0.413320in}}%
\pgfpathlineto{\pgfqpoint{4.640809in}{0.413320in}}%
\pgfpathlineto{\pgfqpoint{4.638204in}{0.413320in}}%
\pgfpathlineto{\pgfqpoint{4.635445in}{0.413320in}}%
\pgfpathlineto{\pgfqpoint{4.632902in}{0.413320in}}%
\pgfpathlineto{\pgfqpoint{4.630096in}{0.413320in}}%
\pgfpathlineto{\pgfqpoint{4.627411in}{0.413320in}}%
\pgfpathlineto{\pgfqpoint{4.624741in}{0.413320in}}%
\pgfpathlineto{\pgfqpoint{4.622056in}{0.413320in}}%
\pgfpathlineto{\pgfqpoint{4.619529in}{0.413320in}}%
\pgfpathlineto{\pgfqpoint{4.616702in}{0.413320in}}%
\pgfpathlineto{\pgfqpoint{4.614134in}{0.413320in}}%
\pgfpathlineto{\pgfqpoint{4.611350in}{0.413320in}}%
\pgfpathlineto{\pgfqpoint{4.608808in}{0.413320in}}%
\pgfpathlineto{\pgfqpoint{4.605990in}{0.413320in}}%
\pgfpathlineto{\pgfqpoint{4.603430in}{0.413320in}}%
\pgfpathlineto{\pgfqpoint{4.600633in}{0.413320in}}%
\pgfpathlineto{\pgfqpoint{4.597951in}{0.413320in}}%
\pgfpathlineto{\pgfqpoint{4.595281in}{0.413320in}}%
\pgfpathlineto{\pgfqpoint{4.592589in}{0.413320in}}%
\pgfpathlineto{\pgfqpoint{4.589920in}{0.413320in}}%
\pgfpathlineto{\pgfqpoint{4.587244in}{0.413320in}}%
\pgfpathlineto{\pgfqpoint{4.584672in}{0.413320in}}%
\pgfpathlineto{\pgfqpoint{4.581888in}{0.413320in}}%
\pgfpathlineto{\pgfqpoint{4.579305in}{0.413320in}}%
\pgfpathlineto{\pgfqpoint{4.576531in}{0.413320in}}%
\pgfpathlineto{\pgfqpoint{4.573947in}{0.413320in}}%
\pgfpathlineto{\pgfqpoint{4.571171in}{0.413320in}}%
\pgfpathlineto{\pgfqpoint{4.568612in}{0.413320in}}%
\pgfpathlineto{\pgfqpoint{4.565820in}{0.413320in}}%
\pgfpathlineto{\pgfqpoint{4.563125in}{0.413320in}}%
\pgfpathlineto{\pgfqpoint{4.560448in}{0.413320in}}%
\pgfpathlineto{\pgfqpoint{4.557777in}{0.413320in}}%
\pgfpathlineto{\pgfqpoint{4.555106in}{0.413320in}}%
\pgfpathlineto{\pgfqpoint{4.552425in}{0.413320in}}%
\pgfpathlineto{\pgfqpoint{4.549822in}{0.413320in}}%
\pgfpathlineto{\pgfqpoint{4.547064in}{0.413320in}}%
\pgfpathlineto{\pgfqpoint{4.544464in}{0.413320in}}%
\pgfpathlineto{\pgfqpoint{4.541711in}{0.413320in}}%
\pgfpathlineto{\pgfqpoint{4.539144in}{0.413320in}}%
\pgfpathlineto{\pgfqpoint{4.536400in}{0.413320in}}%
\pgfpathlineto{\pgfqpoint{4.533764in}{0.413320in}}%
\pgfpathlineto{\pgfqpoint{4.530990in}{0.413320in}}%
\pgfpathlineto{\pgfqpoint{4.528307in}{0.413320in}}%
\pgfpathlineto{\pgfqpoint{4.525640in}{0.413320in}}%
\pgfpathlineto{\pgfqpoint{4.522962in}{0.413320in}}%
\pgfpathlineto{\pgfqpoint{4.520345in}{0.413320in}}%
\pgfpathlineto{\pgfqpoint{4.517598in}{0.413320in}}%
\pgfpathlineto{\pgfqpoint{4.515080in}{0.413320in}}%
\pgfpathlineto{\pgfqpoint{4.512246in}{0.413320in}}%
\pgfpathlineto{\pgfqpoint{4.509643in}{0.413320in}}%
\pgfpathlineto{\pgfqpoint{4.506893in}{0.413320in}}%
\pgfpathlineto{\pgfqpoint{4.504305in}{0.413320in}}%
\pgfpathlineto{\pgfqpoint{4.501529in}{0.413320in}}%
\pgfpathlineto{\pgfqpoint{4.498850in}{0.413320in}}%
\pgfpathlineto{\pgfqpoint{4.496167in}{0.413320in}}%
\pgfpathlineto{\pgfqpoint{4.493492in}{0.413320in}}%
\pgfpathlineto{\pgfqpoint{4.490822in}{0.413320in}}%
\pgfpathlineto{\pgfqpoint{4.488130in}{0.413320in}}%
\pgfpathlineto{\pgfqpoint{4.485581in}{0.413320in}}%
\pgfpathlineto{\pgfqpoint{4.482778in}{0.413320in}}%
\pgfpathlineto{\pgfqpoint{4.480201in}{0.413320in}}%
\pgfpathlineto{\pgfqpoint{4.477430in}{0.413320in}}%
\pgfpathlineto{\pgfqpoint{4.474861in}{0.413320in}}%
\pgfpathlineto{\pgfqpoint{4.472059in}{0.413320in}}%
\pgfpathlineto{\pgfqpoint{4.469492in}{0.413320in}}%
\pgfpathlineto{\pgfqpoint{4.466717in}{0.413320in}}%
\pgfpathlineto{\pgfqpoint{4.464029in}{0.413320in}}%
\pgfpathlineto{\pgfqpoint{4.461367in}{0.413320in}}%
\pgfpathlineto{\pgfqpoint{4.458681in}{0.413320in}}%
\pgfpathlineto{\pgfqpoint{4.456138in}{0.413320in}}%
\pgfpathlineto{\pgfqpoint{4.453312in}{0.413320in}}%
\pgfpathlineto{\pgfqpoint{4.450767in}{0.413320in}}%
\pgfpathlineto{\pgfqpoint{4.447965in}{0.413320in}}%
\pgfpathlineto{\pgfqpoint{4.445423in}{0.413320in}}%
\pgfpathlineto{\pgfqpoint{4.442611in}{0.413320in}}%
\pgfpathlineto{\pgfqpoint{4.440041in}{0.413320in}}%
\pgfpathlineto{\pgfqpoint{4.437253in}{0.413320in}}%
\pgfpathlineto{\pgfqpoint{4.434569in}{0.413320in}}%
\pgfpathlineto{\pgfqpoint{4.431901in}{0.413320in}}%
\pgfpathlineto{\pgfqpoint{4.429220in}{0.413320in}}%
\pgfpathlineto{\pgfqpoint{4.426534in}{0.413320in}}%
\pgfpathlineto{\pgfqpoint{4.423863in}{0.413320in}}%
\pgfpathlineto{\pgfqpoint{4.421292in}{0.413320in}}%
\pgfpathlineto{\pgfqpoint{4.418506in}{0.413320in}}%
\pgfpathlineto{\pgfqpoint{4.415932in}{0.413320in}}%
\pgfpathlineto{\pgfqpoint{4.413149in}{0.413320in}}%
\pgfpathlineto{\pgfqpoint{4.410587in}{0.413320in}}%
\pgfpathlineto{\pgfqpoint{4.407788in}{0.413320in}}%
\pgfpathlineto{\pgfqpoint{4.405234in}{0.413320in}}%
\pgfpathlineto{\pgfqpoint{4.402468in}{0.413320in}}%
\pgfpathlineto{\pgfqpoint{4.399745in}{0.413320in}}%
\pgfpathlineto{\pgfqpoint{4.397076in}{0.413320in}}%
\pgfpathlineto{\pgfqpoint{4.394400in}{0.413320in}}%
\pgfpathlineto{\pgfqpoint{4.391721in}{0.413320in}}%
\pgfpathlineto{\pgfqpoint{4.389044in}{0.413320in}}%
\pgfpathlineto{\pgfqpoint{4.386431in}{0.413320in}}%
\pgfpathlineto{\pgfqpoint{4.383674in}{0.413320in}}%
\pgfpathlineto{\pgfqpoint{4.381097in}{0.413320in}}%
\pgfpathlineto{\pgfqpoint{4.378329in}{0.413320in}}%
\pgfpathlineto{\pgfqpoint{4.375761in}{0.413320in}}%
\pgfpathlineto{\pgfqpoint{4.372976in}{0.413320in}}%
\pgfpathlineto{\pgfqpoint{4.370437in}{0.413320in}}%
\pgfpathlineto{\pgfqpoint{4.367646in}{0.413320in}}%
\pgfpathlineto{\pgfqpoint{4.364936in}{0.413320in}}%
\pgfpathlineto{\pgfqpoint{4.362270in}{0.413320in}}%
\pgfpathlineto{\pgfqpoint{4.359582in}{0.413320in}}%
\pgfpathlineto{\pgfqpoint{4.357014in}{0.413320in}}%
\pgfpathlineto{\pgfqpoint{4.354224in}{0.413320in}}%
\pgfpathlineto{\pgfqpoint{4.351645in}{0.413320in}}%
\pgfpathlineto{\pgfqpoint{4.348868in}{0.413320in}}%
\pgfpathlineto{\pgfqpoint{4.346263in}{0.413320in}}%
\pgfpathlineto{\pgfqpoint{4.343510in}{0.413320in}}%
\pgfpathlineto{\pgfqpoint{4.340976in}{0.413320in}}%
\pgfpathlineto{\pgfqpoint{4.338154in}{0.413320in}}%
\pgfpathlineto{\pgfqpoint{4.335463in}{0.413320in}}%
\pgfpathlineto{\pgfqpoint{4.332796in}{0.413320in}}%
\pgfpathlineto{\pgfqpoint{4.330118in}{0.413320in}}%
\pgfpathlineto{\pgfqpoint{4.327440in}{0.413320in}}%
\pgfpathlineto{\pgfqpoint{4.324760in}{0.413320in}}%
\pgfpathlineto{\pgfqpoint{4.322181in}{0.413320in}}%
\pgfpathlineto{\pgfqpoint{4.319405in}{0.413320in}}%
\pgfpathlineto{\pgfqpoint{4.316856in}{0.413320in}}%
\pgfpathlineto{\pgfqpoint{4.314032in}{0.413320in}}%
\pgfpathlineto{\pgfqpoint{4.311494in}{0.413320in}}%
\pgfpathlineto{\pgfqpoint{4.308691in}{0.413320in}}%
\pgfpathlineto{\pgfqpoint{4.306118in}{0.413320in}}%
\pgfpathlineto{\pgfqpoint{4.303357in}{0.413320in}}%
\pgfpathlineto{\pgfqpoint{4.300656in}{0.413320in}}%
\pgfpathlineto{\pgfqpoint{4.297977in}{0.413320in}}%
\pgfpathlineto{\pgfqpoint{4.295299in}{0.413320in}}%
\pgfpathlineto{\pgfqpoint{4.292786in}{0.413320in}}%
\pgfpathlineto{\pgfqpoint{4.289936in}{0.413320in}}%
\pgfpathlineto{\pgfqpoint{4.287399in}{0.413320in}}%
\pgfpathlineto{\pgfqpoint{4.284586in}{0.413320in}}%
\pgfpathlineto{\pgfqpoint{4.282000in}{0.413320in}}%
\pgfpathlineto{\pgfqpoint{4.279212in}{0.413320in}}%
\pgfpathlineto{\pgfqpoint{4.276635in}{0.413320in}}%
\pgfpathlineto{\pgfqpoint{4.273874in}{0.413320in}}%
\pgfpathlineto{\pgfqpoint{4.271187in}{0.413320in}}%
\pgfpathlineto{\pgfqpoint{4.268590in}{0.413320in}}%
\pgfpathlineto{\pgfqpoint{4.265824in}{0.413320in}}%
\pgfpathlineto{\pgfqpoint{4.263157in}{0.413320in}}%
\pgfpathlineto{\pgfqpoint{4.260477in}{0.413320in}}%
\pgfpathlineto{\pgfqpoint{4.257958in}{0.413320in}}%
\pgfpathlineto{\pgfqpoint{4.255120in}{0.413320in}}%
\pgfpathlineto{\pgfqpoint{4.252581in}{0.413320in}}%
\pgfpathlineto{\pgfqpoint{4.249767in}{0.413320in}}%
\pgfpathlineto{\pgfqpoint{4.247225in}{0.413320in}}%
\pgfpathlineto{\pgfqpoint{4.244394in}{0.413320in}}%
\pgfpathlineto{\pgfqpoint{4.241900in}{0.413320in}}%
\pgfpathlineto{\pgfqpoint{4.239084in}{0.413320in}}%
\pgfpathlineto{\pgfqpoint{4.236375in}{0.413320in}}%
\pgfpathlineto{\pgfqpoint{4.233691in}{0.413320in}}%
\pgfpathlineto{\pgfqpoint{4.231013in}{0.413320in}}%
\pgfpathlineto{\pgfqpoint{4.228331in}{0.413320in}}%
\pgfpathlineto{\pgfqpoint{4.225654in}{0.413320in}}%
\pgfpathlineto{\pgfqpoint{4.223082in}{0.413320in}}%
\pgfpathlineto{\pgfqpoint{4.220304in}{0.413320in}}%
\pgfpathlineto{\pgfqpoint{4.217694in}{0.413320in}}%
\pgfpathlineto{\pgfqpoint{4.214948in}{0.413320in}}%
\pgfpathlineto{\pgfqpoint{4.212383in}{0.413320in}}%
\pgfpathlineto{\pgfqpoint{4.209597in}{0.413320in}}%
\pgfpathlineto{\pgfqpoint{4.207076in}{0.413320in}}%
\pgfpathlineto{\pgfqpoint{4.204240in}{0.413320in}}%
\pgfpathlineto{\pgfqpoint{4.201542in}{0.413320in}}%
\pgfpathlineto{\pgfqpoint{4.198878in}{0.413320in}}%
\pgfpathlineto{\pgfqpoint{4.196186in}{0.413320in}}%
\pgfpathlineto{\pgfqpoint{4.193638in}{0.413320in}}%
\pgfpathlineto{\pgfqpoint{4.190842in}{0.413320in}}%
\pgfpathlineto{\pgfqpoint{4.188318in}{0.413320in}}%
\pgfpathlineto{\pgfqpoint{4.185481in}{0.413320in}}%
\pgfpathlineto{\pgfqpoint{4.182899in}{0.413320in}}%
\pgfpathlineto{\pgfqpoint{4.180129in}{0.413320in}}%
\pgfpathlineto{\pgfqpoint{4.177593in}{0.413320in}}%
\pgfpathlineto{\pgfqpoint{4.174770in}{0.413320in}}%
\pgfpathlineto{\pgfqpoint{4.172093in}{0.413320in}}%
\pgfpathlineto{\pgfqpoint{4.169415in}{0.413320in}}%
\pgfpathlineto{\pgfqpoint{4.166737in}{0.413320in}}%
\pgfpathlineto{\pgfqpoint{4.164059in}{0.413320in}}%
\pgfpathlineto{\pgfqpoint{4.161380in}{0.413320in}}%
\pgfpathlineto{\pgfqpoint{4.158806in}{0.413320in}}%
\pgfpathlineto{\pgfqpoint{4.156016in}{0.413320in}}%
\pgfpathlineto{\pgfqpoint{4.153423in}{0.413320in}}%
\pgfpathlineto{\pgfqpoint{4.150665in}{0.413320in}}%
\pgfpathlineto{\pgfqpoint{4.148082in}{0.413320in}}%
\pgfpathlineto{\pgfqpoint{4.145310in}{0.413320in}}%
\pgfpathlineto{\pgfqpoint{4.142713in}{0.413320in}}%
\pgfpathlineto{\pgfqpoint{4.139963in}{0.413320in}}%
\pgfpathlineto{\pgfqpoint{4.137272in}{0.413320in}}%
\pgfpathlineto{\pgfqpoint{4.134615in}{0.413320in}}%
\pgfpathlineto{\pgfqpoint{4.131920in}{0.413320in}}%
\pgfpathlineto{\pgfqpoint{4.129349in}{0.413320in}}%
\pgfpathlineto{\pgfqpoint{4.126553in}{0.413320in}}%
\pgfpathlineto{\pgfqpoint{4.124019in}{0.413320in}}%
\pgfpathlineto{\pgfqpoint{4.121205in}{0.413320in}}%
\pgfpathlineto{\pgfqpoint{4.118554in}{0.413320in}}%
\pgfpathlineto{\pgfqpoint{4.115844in}{0.413320in}}%
\pgfpathlineto{\pgfqpoint{4.113252in}{0.413320in}}%
\pgfpathlineto{\pgfqpoint{4.110488in}{0.413320in}}%
\pgfpathlineto{\pgfqpoint{4.107814in}{0.413320in}}%
\pgfpathlineto{\pgfqpoint{4.105185in}{0.413320in}}%
\pgfpathlineto{\pgfqpoint{4.102456in}{0.413320in}}%
\pgfpathlineto{\pgfqpoint{4.099777in}{0.413320in}}%
\pgfpathlineto{\pgfqpoint{4.097092in}{0.413320in}}%
\pgfpathlineto{\pgfqpoint{4.094527in}{0.413320in}}%
\pgfpathlineto{\pgfqpoint{4.091729in}{0.413320in}}%
\pgfpathlineto{\pgfqpoint{4.089159in}{0.413320in}}%
\pgfpathlineto{\pgfqpoint{4.086385in}{0.413320in}}%
\pgfpathlineto{\pgfqpoint{4.083870in}{0.413320in}}%
\pgfpathlineto{\pgfqpoint{4.081018in}{0.413320in}}%
\pgfpathlineto{\pgfqpoint{4.078471in}{0.413320in}}%
\pgfpathlineto{\pgfqpoint{4.075705in}{0.413320in}}%
\pgfpathlineto{\pgfqpoint{4.072985in}{0.413320in}}%
\pgfpathlineto{\pgfqpoint{4.070313in}{0.413320in}}%
\pgfpathlineto{\pgfqpoint{4.067636in}{0.413320in}}%
\pgfpathlineto{\pgfqpoint{4.064957in}{0.413320in}}%
\pgfpathlineto{\pgfqpoint{4.062266in}{0.413320in}}%
\pgfpathlineto{\pgfqpoint{4.059702in}{0.413320in}}%
\pgfpathlineto{\pgfqpoint{4.056911in}{0.413320in}}%
\pgfpathlineto{\pgfqpoint{4.054326in}{0.413320in}}%
\pgfpathlineto{\pgfqpoint{4.051557in}{0.413320in}}%
\pgfpathlineto{\pgfqpoint{4.049006in}{0.413320in}}%
\pgfpathlineto{\pgfqpoint{4.046210in}{0.413320in}}%
\pgfpathlineto{\pgfqpoint{4.043667in}{0.413320in}}%
\pgfpathlineto{\pgfqpoint{4.040852in}{0.413320in}}%
\pgfpathlineto{\pgfqpoint{4.038174in}{0.413320in}}%
\pgfpathlineto{\pgfqpoint{4.035492in}{0.413320in}}%
\pgfpathlineto{\pgfqpoint{4.032817in}{0.413320in}}%
\pgfpathlineto{\pgfqpoint{4.030229in}{0.413320in}}%
\pgfpathlineto{\pgfqpoint{4.027447in}{0.413320in}}%
\pgfpathlineto{\pgfqpoint{4.024868in}{0.413320in}}%
\pgfpathlineto{\pgfqpoint{4.022097in}{0.413320in}}%
\pgfpathlineto{\pgfqpoint{4.019518in}{0.413320in}}%
\pgfpathlineto{\pgfqpoint{4.016744in}{0.413320in}}%
\pgfpathlineto{\pgfqpoint{4.014186in}{0.413320in}}%
\pgfpathlineto{\pgfqpoint{4.011394in}{0.413320in}}%
\pgfpathlineto{\pgfqpoint{4.008699in}{0.413320in}}%
\pgfpathlineto{\pgfqpoint{4.006034in}{0.413320in}}%
\pgfpathlineto{\pgfqpoint{4.003348in}{0.413320in}}%
\pgfpathlineto{\pgfqpoint{4.000674in}{0.413320in}}%
\pgfpathlineto{\pgfqpoint{3.997990in}{0.413320in}}%
\pgfpathlineto{\pgfqpoint{3.995417in}{0.413320in}}%
\pgfpathlineto{\pgfqpoint{3.992642in}{0.413320in}}%
\pgfpathlineto{\pgfqpoint{3.990055in}{0.413320in}}%
\pgfpathlineto{\pgfqpoint{3.987270in}{0.413320in}}%
\pgfpathlineto{\pgfqpoint{3.984714in}{0.413320in}}%
\pgfpathlineto{\pgfqpoint{3.981929in}{0.413320in}}%
\pgfpathlineto{\pgfqpoint{3.979389in}{0.413320in}}%
\pgfpathlineto{\pgfqpoint{3.976563in}{0.413320in}}%
\pgfpathlineto{\pgfqpoint{3.973885in}{0.413320in}}%
\pgfpathlineto{\pgfqpoint{3.971250in}{0.413320in}}%
\pgfpathlineto{\pgfqpoint{3.968523in}{0.413320in}}%
\pgfpathlineto{\pgfqpoint{3.966013in}{0.413320in}}%
\pgfpathlineto{\pgfqpoint{3.963176in}{0.413320in}}%
\pgfpathlineto{\pgfqpoint{3.960635in}{0.413320in}}%
\pgfpathlineto{\pgfqpoint{3.957823in}{0.413320in}}%
\pgfpathlineto{\pgfqpoint{3.955211in}{0.413320in}}%
\pgfpathlineto{\pgfqpoint{3.952464in}{0.413320in}}%
\pgfpathlineto{\pgfqpoint{3.949894in}{0.413320in}}%
\pgfpathlineto{\pgfqpoint{3.947101in}{0.413320in}}%
\pgfpathlineto{\pgfqpoint{3.944431in}{0.413320in}}%
\pgfpathlineto{\pgfqpoint{3.941778in}{0.413320in}}%
\pgfpathlineto{\pgfqpoint{3.939075in}{0.413320in}}%
\pgfpathlineto{\pgfqpoint{3.936395in}{0.413320in}}%
\pgfpathlineto{\pgfqpoint{3.933714in}{0.413320in}}%
\pgfpathlineto{\pgfqpoint{3.931202in}{0.413320in}}%
\pgfpathlineto{\pgfqpoint{3.928347in}{0.413320in}}%
\pgfpathlineto{\pgfqpoint{3.925778in}{0.413320in}}%
\pgfpathlineto{\pgfqpoint{3.923005in}{0.413320in}}%
\pgfpathlineto{\pgfqpoint{3.920412in}{0.413320in}}%
\pgfpathlineto{\pgfqpoint{3.917646in}{0.413320in}}%
\pgfpathlineto{\pgfqpoint{3.915107in}{0.413320in}}%
\pgfpathlineto{\pgfqpoint{3.912296in}{0.413320in}}%
\pgfpathlineto{\pgfqpoint{3.909602in}{0.413320in}}%
\pgfpathlineto{\pgfqpoint{3.906918in}{0.413320in}}%
\pgfpathlineto{\pgfqpoint{3.904252in}{0.413320in}}%
\pgfpathlineto{\pgfqpoint{3.901573in}{0.413320in}}%
\pgfpathlineto{\pgfqpoint{3.898891in}{0.413320in}}%
\pgfpathlineto{\pgfqpoint{3.896345in}{0.413320in}}%
\pgfpathlineto{\pgfqpoint{3.893541in}{0.413320in}}%
\pgfpathlineto{\pgfqpoint{3.890926in}{0.413320in}}%
\pgfpathlineto{\pgfqpoint{3.888188in}{0.413320in}}%
\pgfpathlineto{\pgfqpoint{3.885621in}{0.413320in}}%
\pgfpathlineto{\pgfqpoint{3.882850in}{0.413320in}}%
\pgfpathlineto{\pgfqpoint{3.880237in}{0.413320in}}%
\pgfpathlineto{\pgfqpoint{3.877466in}{0.413320in}}%
\pgfpathlineto{\pgfqpoint{3.874790in}{0.413320in}}%
\pgfpathlineto{\pgfqpoint{3.872114in}{0.413320in}}%
\pgfpathlineto{\pgfqpoint{3.869435in}{0.413320in}}%
\pgfpathlineto{\pgfqpoint{3.866815in}{0.413320in}}%
\pgfpathlineto{\pgfqpoint{3.864073in}{0.413320in}}%
\pgfpathlineto{\pgfqpoint{3.861561in}{0.413320in}}%
\pgfpathlineto{\pgfqpoint{3.858720in}{0.413320in}}%
\pgfpathlineto{\pgfqpoint{3.856100in}{0.413320in}}%
\pgfpathlineto{\pgfqpoint{3.853358in}{0.413320in}}%
\pgfpathlineto{\pgfqpoint{3.850814in}{0.413320in}}%
\pgfpathlineto{\pgfqpoint{3.848005in}{0.413320in}}%
\pgfpathlineto{\pgfqpoint{3.845329in}{0.413320in}}%
\pgfpathlineto{\pgfqpoint{3.842641in}{0.413320in}}%
\pgfpathlineto{\pgfqpoint{3.839960in}{0.413320in}}%
\pgfpathlineto{\pgfqpoint{3.837286in}{0.413320in}}%
\pgfpathlineto{\pgfqpoint{3.834616in}{0.413320in}}%
\pgfpathlineto{\pgfqpoint{3.832053in}{0.413320in}}%
\pgfpathlineto{\pgfqpoint{3.829252in}{0.413320in}}%
\pgfpathlineto{\pgfqpoint{3.826679in}{0.413320in}}%
\pgfpathlineto{\pgfqpoint{3.823903in}{0.413320in}}%
\pgfpathlineto{\pgfqpoint{3.821315in}{0.413320in}}%
\pgfpathlineto{\pgfqpoint{3.818546in}{0.413320in}}%
\pgfpathlineto{\pgfqpoint{3.815983in}{0.413320in}}%
\pgfpathlineto{\pgfqpoint{3.813172in}{0.413320in}}%
\pgfpathlineto{\pgfqpoint{3.810510in}{0.413320in}}%
\pgfpathlineto{\pgfqpoint{3.807832in}{0.413320in}}%
\pgfpathlineto{\pgfqpoint{3.805145in}{0.413320in}}%
\pgfpathlineto{\pgfqpoint{3.802569in}{0.413320in}}%
\pgfpathlineto{\pgfqpoint{3.799797in}{0.413320in}}%
\pgfpathlineto{\pgfqpoint{3.797265in}{0.413320in}}%
\pgfpathlineto{\pgfqpoint{3.794435in}{0.413320in}}%
\pgfpathlineto{\pgfqpoint{3.791897in}{0.413320in}}%
\pgfpathlineto{\pgfqpoint{3.789084in}{0.413320in}}%
\pgfpathlineto{\pgfqpoint{3.786504in}{0.413320in}}%
\pgfpathlineto{\pgfqpoint{3.783725in}{0.413320in}}%
\pgfpathlineto{\pgfqpoint{3.781046in}{0.413320in}}%
\pgfpathlineto{\pgfqpoint{3.778370in}{0.413320in}}%
\pgfpathlineto{\pgfqpoint{3.775691in}{0.413320in}}%
\pgfpathlineto{\pgfqpoint{3.773014in}{0.413320in}}%
\pgfpathlineto{\pgfqpoint{3.770323in}{0.413320in}}%
\pgfpathlineto{\pgfqpoint{3.767782in}{0.413320in}}%
\pgfpathlineto{\pgfqpoint{3.764966in}{0.413320in}}%
\pgfpathlineto{\pgfqpoint{3.762389in}{0.413320in}}%
\pgfpathlineto{\pgfqpoint{3.759622in}{0.413320in}}%
\pgfpathlineto{\pgfqpoint{3.757065in}{0.413320in}}%
\pgfpathlineto{\pgfqpoint{3.754265in}{0.413320in}}%
\pgfpathlineto{\pgfqpoint{3.751728in}{0.413320in}}%
\pgfpathlineto{\pgfqpoint{3.748903in}{0.413320in}}%
\pgfpathlineto{\pgfqpoint{3.746229in}{0.413320in}}%
\pgfpathlineto{\pgfqpoint{3.743548in}{0.413320in}}%
\pgfpathlineto{\pgfqpoint{3.740874in}{0.413320in}}%
\pgfpathlineto{\pgfqpoint{3.738194in}{0.413320in}}%
\pgfpathlineto{\pgfqpoint{3.735509in}{0.413320in}}%
\pgfpathlineto{\pgfqpoint{3.732950in}{0.413320in}}%
\pgfpathlineto{\pgfqpoint{3.730158in}{0.413320in}}%
\pgfpathlineto{\pgfqpoint{3.727581in}{0.413320in}}%
\pgfpathlineto{\pgfqpoint{3.724804in}{0.413320in}}%
\pgfpathlineto{\pgfqpoint{3.722228in}{0.413320in}}%
\pgfpathlineto{\pgfqpoint{3.719446in}{0.413320in}}%
\pgfpathlineto{\pgfqpoint{3.716875in}{0.413320in}}%
\pgfpathlineto{\pgfqpoint{3.714086in}{0.413320in}}%
\pgfpathlineto{\pgfqpoint{3.711410in}{0.413320in}}%
\pgfpathlineto{\pgfqpoint{3.708729in}{0.413320in}}%
\pgfpathlineto{\pgfqpoint{3.706053in}{0.413320in}}%
\pgfpathlineto{\pgfqpoint{3.703460in}{0.413320in}}%
\pgfpathlineto{\pgfqpoint{3.700684in}{0.413320in}}%
\pgfpathlineto{\pgfqpoint{3.698125in}{0.413320in}}%
\pgfpathlineto{\pgfqpoint{3.695331in}{0.413320in}}%
\pgfpathlineto{\pgfqpoint{3.692765in}{0.413320in}}%
\pgfpathlineto{\pgfqpoint{3.689983in}{0.413320in}}%
\pgfpathlineto{\pgfqpoint{3.687442in}{0.413320in}}%
\pgfpathlineto{\pgfqpoint{3.684620in}{0.413320in}}%
\pgfpathlineto{\pgfqpoint{3.681948in}{0.413320in}}%
\pgfpathlineto{\pgfqpoint{3.679273in}{0.413320in}}%
\pgfpathlineto{\pgfqpoint{3.676591in}{0.413320in}}%
\pgfpathlineto{\pgfqpoint{3.673911in}{0.413320in}}%
\pgfpathlineto{\pgfqpoint{3.671232in}{0.413320in}}%
\pgfpathlineto{\pgfqpoint{3.668665in}{0.413320in}}%
\pgfpathlineto{\pgfqpoint{3.665864in}{0.413320in}}%
\pgfpathlineto{\pgfqpoint{3.663276in}{0.413320in}}%
\pgfpathlineto{\pgfqpoint{3.660515in}{0.413320in}}%
\pgfpathlineto{\pgfqpoint{3.657917in}{0.413320in}}%
\pgfpathlineto{\pgfqpoint{3.655165in}{0.413320in}}%
\pgfpathlineto{\pgfqpoint{3.652628in}{0.413320in}}%
\pgfpathlineto{\pgfqpoint{3.649837in}{0.413320in}}%
\pgfpathlineto{\pgfqpoint{3.647130in}{0.413320in}}%
\pgfpathlineto{\pgfqpoint{3.644452in}{0.413320in}}%
\pgfpathlineto{\pgfqpoint{3.641773in}{0.413320in}}%
\pgfpathlineto{\pgfqpoint{3.639207in}{0.413320in}}%
\pgfpathlineto{\pgfqpoint{3.636413in}{0.413320in}}%
\pgfpathlineto{\pgfqpoint{3.633858in}{0.413320in}}%
\pgfpathlineto{\pgfqpoint{3.631058in}{0.413320in}}%
\pgfpathlineto{\pgfqpoint{3.628460in}{0.413320in}}%
\pgfpathlineto{\pgfqpoint{3.625689in}{0.413320in}}%
\pgfpathlineto{\pgfqpoint{3.623165in}{0.413320in}}%
\pgfpathlineto{\pgfqpoint{3.620345in}{0.413320in}}%
\pgfpathlineto{\pgfqpoint{3.617667in}{0.413320in}}%
\pgfpathlineto{\pgfqpoint{3.614982in}{0.413320in}}%
\pgfpathlineto{\pgfqpoint{3.612311in}{0.413320in}}%
\pgfpathlineto{\pgfqpoint{3.609632in}{0.413320in}}%
\pgfpathlineto{\pgfqpoint{3.606951in}{0.413320in}}%
\pgfpathlineto{\pgfqpoint{3.604387in}{0.413320in}}%
\pgfpathlineto{\pgfqpoint{3.601590in}{0.413320in}}%
\pgfpathlineto{\pgfqpoint{3.598998in}{0.413320in}}%
\pgfpathlineto{\pgfqpoint{3.596240in}{0.413320in}}%
\pgfpathlineto{\pgfqpoint{3.593620in}{0.413320in}}%
\pgfpathlineto{\pgfqpoint{3.590883in}{0.413320in}}%
\pgfpathlineto{\pgfqpoint{3.588258in}{0.413320in}}%
\pgfpathlineto{\pgfqpoint{3.585532in}{0.413320in}}%
\pgfpathlineto{\pgfqpoint{3.582851in}{0.413320in}}%
\pgfpathlineto{\pgfqpoint{3.580191in}{0.413320in}}%
\pgfpathlineto{\pgfqpoint{3.577487in}{0.413320in}}%
\pgfpathlineto{\pgfqpoint{3.574814in}{0.413320in}}%
\pgfpathlineto{\pgfqpoint{3.572126in}{0.413320in}}%
\pgfpathlineto{\pgfqpoint{3.569584in}{0.413320in}}%
\pgfpathlineto{\pgfqpoint{3.566774in}{0.413320in}}%
\pgfpathlineto{\pgfqpoint{3.564188in}{0.413320in}}%
\pgfpathlineto{\pgfqpoint{3.561420in}{0.413320in}}%
\pgfpathlineto{\pgfqpoint{3.558853in}{0.413320in}}%
\pgfpathlineto{\pgfqpoint{3.556061in}{0.413320in}}%
\pgfpathlineto{\pgfqpoint{3.553498in}{0.413320in}}%
\pgfpathlineto{\pgfqpoint{3.550713in}{0.413320in}}%
\pgfpathlineto{\pgfqpoint{3.548029in}{0.413320in}}%
\pgfpathlineto{\pgfqpoint{3.545349in}{0.413320in}}%
\pgfpathlineto{\pgfqpoint{3.542656in}{0.413320in}}%
\pgfpathlineto{\pgfqpoint{3.540093in}{0.413320in}}%
\pgfpathlineto{\pgfqpoint{3.537309in}{0.413320in}}%
\pgfpathlineto{\pgfqpoint{3.534783in}{0.413320in}}%
\pgfpathlineto{\pgfqpoint{3.531955in}{0.413320in}}%
\pgfpathlineto{\pgfqpoint{3.529327in}{0.413320in}}%
\pgfpathlineto{\pgfqpoint{3.526601in}{0.413320in}}%
\pgfpathlineto{\pgfqpoint{3.524041in}{0.413320in}}%
\pgfpathlineto{\pgfqpoint{3.521244in}{0.413320in}}%
\pgfpathlineto{\pgfqpoint{3.518565in}{0.413320in}}%
\pgfpathlineto{\pgfqpoint{3.515884in}{0.413320in}}%
\pgfpathlineto{\pgfqpoint{3.513209in}{0.413320in}}%
\pgfpathlineto{\pgfqpoint{3.510533in}{0.413320in}}%
\pgfpathlineto{\pgfqpoint{3.507840in}{0.413320in}}%
\pgfpathlineto{\pgfqpoint{3.505262in}{0.413320in}}%
\pgfpathlineto{\pgfqpoint{3.502488in}{0.413320in}}%
\pgfpathlineto{\pgfqpoint{3.499909in}{0.413320in}}%
\pgfpathlineto{\pgfqpoint{3.497139in}{0.413320in}}%
\pgfpathlineto{\pgfqpoint{3.494581in}{0.413320in}}%
\pgfpathlineto{\pgfqpoint{3.491783in}{0.413320in}}%
\pgfpathlineto{\pgfqpoint{3.489223in}{0.413320in}}%
\pgfpathlineto{\pgfqpoint{3.486442in}{0.413320in}}%
\pgfpathlineto{\pgfqpoint{3.483744in}{0.413320in}}%
\pgfpathlineto{\pgfqpoint{3.481072in}{0.413320in}}%
\pgfpathlineto{\pgfqpoint{3.478378in}{0.413320in}}%
\pgfpathlineto{\pgfqpoint{3.475821in}{0.413320in}}%
\pgfpathlineto{\pgfqpoint{3.473021in}{0.413320in}}%
\pgfpathlineto{\pgfqpoint{3.470466in}{0.413320in}}%
\pgfpathlineto{\pgfqpoint{3.467678in}{0.413320in}}%
\pgfpathlineto{\pgfqpoint{3.465072in}{0.413320in}}%
\pgfpathlineto{\pgfqpoint{3.462321in}{0.413320in}}%
\pgfpathlineto{\pgfqpoint{3.459695in}{0.413320in}}%
\pgfpathlineto{\pgfqpoint{3.456960in}{0.413320in}}%
\pgfpathlineto{\pgfqpoint{3.454285in}{0.413320in}}%
\pgfpathlineto{\pgfqpoint{3.451597in}{0.413320in}}%
\pgfpathlineto{\pgfqpoint{3.448926in}{0.413320in}}%
\pgfpathlineto{\pgfqpoint{3.446257in}{0.413320in}}%
\pgfpathlineto{\pgfqpoint{3.443574in}{0.413320in}}%
\pgfpathlineto{\pgfqpoint{3.440996in}{0.413320in}}%
\pgfpathlineto{\pgfqpoint{3.438210in}{0.413320in}}%
\pgfpathlineto{\pgfqpoint{3.435635in}{0.413320in}}%
\pgfpathlineto{\pgfqpoint{3.432851in}{0.413320in}}%
\pgfpathlineto{\pgfqpoint{3.430313in}{0.413320in}}%
\pgfpathlineto{\pgfqpoint{3.427501in}{0.413320in}}%
\pgfpathlineto{\pgfqpoint{3.424887in}{0.413320in}}%
\pgfpathlineto{\pgfqpoint{3.422142in}{0.413320in}}%
\pgfpathlineto{\pgfqpoint{3.419455in}{0.413320in}}%
\pgfpathlineto{\pgfqpoint{3.416780in}{0.413320in}}%
\pgfpathlineto{\pgfqpoint{3.414109in}{0.413320in}}%
\pgfpathlineto{\pgfqpoint{3.411431in}{0.413320in}}%
\pgfpathlineto{\pgfqpoint{3.408752in}{0.413320in}}%
\pgfpathlineto{\pgfqpoint{3.406202in}{0.413320in}}%
\pgfpathlineto{\pgfqpoint{3.403394in}{0.413320in}}%
\pgfpathlineto{\pgfqpoint{3.400783in}{0.413320in}}%
\pgfpathlineto{\pgfqpoint{3.398037in}{0.413320in}}%
\pgfpathlineto{\pgfqpoint{3.395461in}{0.413320in}}%
\pgfpathlineto{\pgfqpoint{3.392681in}{0.413320in}}%
\pgfpathlineto{\pgfqpoint{3.390102in}{0.413320in}}%
\pgfpathlineto{\pgfqpoint{3.387309in}{0.413320in}}%
\pgfpathlineto{\pgfqpoint{3.384647in}{0.413320in}}%
\pgfpathlineto{\pgfqpoint{3.381959in}{0.413320in}}%
\pgfpathlineto{\pgfqpoint{3.379290in}{0.413320in}}%
\pgfpathlineto{\pgfqpoint{3.376735in}{0.413320in}}%
\pgfpathlineto{\pgfqpoint{3.373921in}{0.413320in}}%
\pgfpathlineto{\pgfqpoint{3.371357in}{0.413320in}}%
\pgfpathlineto{\pgfqpoint{3.368577in}{0.413320in}}%
\pgfpathlineto{\pgfqpoint{3.365996in}{0.413320in}}%
\pgfpathlineto{\pgfqpoint{3.363221in}{0.413320in}}%
\pgfpathlineto{\pgfqpoint{3.360620in}{0.413320in}}%
\pgfpathlineto{\pgfqpoint{3.357862in}{0.413320in}}%
\pgfpathlineto{\pgfqpoint{3.355177in}{0.413320in}}%
\pgfpathlineto{\pgfqpoint{3.352505in}{0.413320in}}%
\pgfpathlineto{\pgfqpoint{3.349828in}{0.413320in}}%
\pgfpathlineto{\pgfqpoint{3.347139in}{0.413320in}}%
\pgfpathlineto{\pgfqpoint{3.344468in}{0.413320in}}%
\pgfpathlineto{\pgfqpoint{3.341893in}{0.413320in}}%
\pgfpathlineto{\pgfqpoint{3.339101in}{0.413320in}}%
\pgfpathlineto{\pgfqpoint{3.336541in}{0.413320in}}%
\pgfpathlineto{\pgfqpoint{3.333758in}{0.413320in}}%
\pgfpathlineto{\pgfqpoint{3.331183in}{0.413320in}}%
\pgfpathlineto{\pgfqpoint{3.328401in}{0.413320in}}%
\pgfpathlineto{\pgfqpoint{3.325860in}{0.413320in}}%
\pgfpathlineto{\pgfqpoint{3.323049in}{0.413320in}}%
\pgfpathlineto{\pgfqpoint{3.320366in}{0.413320in}}%
\pgfpathlineto{\pgfqpoint{3.317688in}{0.413320in}}%
\pgfpathlineto{\pgfqpoint{3.315008in}{0.413320in}}%
\pgfpathlineto{\pgfqpoint{3.312480in}{0.413320in}}%
\pgfpathlineto{\pgfqpoint{3.309652in}{0.413320in}}%
\pgfpathlineto{\pgfqpoint{3.307104in}{0.413320in}}%
\pgfpathlineto{\pgfqpoint{3.304295in}{0.413320in}}%
\pgfpathlineto{\pgfqpoint{3.301719in}{0.413320in}}%
\pgfpathlineto{\pgfqpoint{3.298937in}{0.413320in}}%
\pgfpathlineto{\pgfqpoint{3.296376in}{0.413320in}}%
\pgfpathlineto{\pgfqpoint{3.293574in}{0.413320in}}%
\pgfpathlineto{\pgfqpoint{3.290890in}{0.413320in}}%
\pgfpathlineto{\pgfqpoint{3.288225in}{0.413320in}}%
\pgfpathlineto{\pgfqpoint{3.285534in}{0.413320in}}%
\pgfpathlineto{\pgfqpoint{3.282870in}{0.413320in}}%
\pgfpathlineto{\pgfqpoint{3.280189in}{0.413320in}}%
\pgfpathlineto{\pgfqpoint{3.277603in}{0.413320in}}%
\pgfpathlineto{\pgfqpoint{3.274831in}{0.413320in}}%
\pgfpathlineto{\pgfqpoint{3.272254in}{0.413320in}}%
\pgfpathlineto{\pgfqpoint{3.269478in}{0.413320in}}%
\pgfpathlineto{\pgfqpoint{3.266849in}{0.413320in}}%
\pgfpathlineto{\pgfqpoint{3.264119in}{0.413320in}}%
\pgfpathlineto{\pgfqpoint{3.261594in}{0.413320in}}%
\pgfpathlineto{\pgfqpoint{3.258784in}{0.413320in}}%
\pgfpathlineto{\pgfqpoint{3.256083in}{0.413320in}}%
\pgfpathlineto{\pgfqpoint{3.253404in}{0.413320in}}%
\pgfpathlineto{\pgfqpoint{3.250716in}{0.413320in}}%
\pgfpathlineto{\pgfqpoint{3.248049in}{0.413320in}}%
\pgfpathlineto{\pgfqpoint{3.245363in}{0.413320in}}%
\pgfpathlineto{\pgfqpoint{3.242807in}{0.413320in}}%
\pgfpathlineto{\pgfqpoint{3.240010in}{0.413320in}}%
\pgfpathlineto{\pgfqpoint{3.237411in}{0.413320in}}%
\pgfpathlineto{\pgfqpoint{3.234658in}{0.413320in}}%
\pgfpathlineto{\pgfqpoint{3.232069in}{0.413320in}}%
\pgfpathlineto{\pgfqpoint{3.229310in}{0.413320in}}%
\pgfpathlineto{\pgfqpoint{3.226609in}{0.413320in}}%
\pgfpathlineto{\pgfqpoint{3.223942in}{0.413320in}}%
\pgfpathlineto{\pgfqpoint{3.221255in}{0.413320in}}%
\pgfpathlineto{\pgfqpoint{3.218586in}{0.413320in}}%
\pgfpathlineto{\pgfqpoint{3.215908in}{0.413320in}}%
\pgfpathlineto{\pgfqpoint{3.213342in}{0.413320in}}%
\pgfpathlineto{\pgfqpoint{3.210545in}{0.413320in}}%
\pgfpathlineto{\pgfqpoint{3.207984in}{0.413320in}}%
\pgfpathlineto{\pgfqpoint{3.205195in}{0.413320in}}%
\pgfpathlineto{\pgfqpoint{3.202562in}{0.413320in}}%
\pgfpathlineto{\pgfqpoint{3.199823in}{0.413320in}}%
\pgfpathlineto{\pgfqpoint{3.197226in}{0.413320in}}%
\pgfpathlineto{\pgfqpoint{3.194508in}{0.413320in}}%
\pgfpathlineto{\pgfqpoint{3.191796in}{0.413320in}}%
\pgfpathlineto{\pgfqpoint{3.189117in}{0.413320in}}%
\pgfpathlineto{\pgfqpoint{3.186440in}{0.413320in}}%
\pgfpathlineto{\pgfqpoint{3.183760in}{0.413320in}}%
\pgfpathlineto{\pgfqpoint{3.181089in}{0.413320in}}%
\pgfpathlineto{\pgfqpoint{3.178525in}{0.413320in}}%
\pgfpathlineto{\pgfqpoint{3.175724in}{0.413320in}}%
\pgfpathlineto{\pgfqpoint{3.173142in}{0.413320in}}%
\pgfpathlineto{\pgfqpoint{3.170375in}{0.413320in}}%
\pgfpathlineto{\pgfqpoint{3.167776in}{0.413320in}}%
\pgfpathlineto{\pgfqpoint{3.165019in}{0.413320in}}%
\pgfpathlineto{\pgfqpoint{3.162474in}{0.413320in}}%
\pgfpathlineto{\pgfqpoint{3.159675in}{0.413320in}}%
\pgfpathlineto{\pgfqpoint{3.156981in}{0.413320in}}%
\pgfpathlineto{\pgfqpoint{3.154327in}{0.413320in}}%
\pgfpathlineto{\pgfqpoint{3.151612in}{0.413320in}}%
\pgfpathlineto{\pgfqpoint{3.149057in}{0.413320in}}%
\pgfpathlineto{\pgfqpoint{3.146271in}{0.413320in}}%
\pgfpathlineto{\pgfqpoint{3.143740in}{0.413320in}}%
\pgfpathlineto{\pgfqpoint{3.140913in}{0.413320in}}%
\pgfpathlineto{\pgfqpoint{3.138375in}{0.413320in}}%
\pgfpathlineto{\pgfqpoint{3.135550in}{0.413320in}}%
\pgfpathlineto{\pgfqpoint{3.132946in}{0.413320in}}%
\pgfpathlineto{\pgfqpoint{3.130199in}{0.413320in}}%
\pgfpathlineto{\pgfqpoint{3.127512in}{0.413320in}}%
\pgfpathlineto{\pgfqpoint{3.124842in}{0.413320in}}%
\pgfpathlineto{\pgfqpoint{3.122163in}{0.413320in}}%
\pgfpathlineto{\pgfqpoint{3.119487in}{0.413320in}}%
\pgfpathlineto{\pgfqpoint{3.116807in}{0.413320in}}%
\pgfpathlineto{\pgfqpoint{3.114242in}{0.413320in}}%
\pgfpathlineto{\pgfqpoint{3.111451in}{0.413320in}}%
\pgfpathlineto{\pgfqpoint{3.108896in}{0.413320in}}%
\pgfpathlineto{\pgfqpoint{3.106094in}{0.413320in}}%
\pgfpathlineto{\pgfqpoint{3.103508in}{0.413320in}}%
\pgfpathlineto{\pgfqpoint{3.100737in}{0.413320in}}%
\pgfpathlineto{\pgfqpoint{3.098163in}{0.413320in}}%
\pgfpathlineto{\pgfqpoint{3.095388in}{0.413320in}}%
\pgfpathlineto{\pgfqpoint{3.092699in}{0.413320in}}%
\pgfpathlineto{\pgfqpoint{3.090023in}{0.413320in}}%
\pgfpathlineto{\pgfqpoint{3.087343in}{0.413320in}}%
\pgfpathlineto{\pgfqpoint{3.084671in}{0.413320in}}%
\pgfpathlineto{\pgfqpoint{3.081990in}{0.413320in}}%
\pgfpathlineto{\pgfqpoint{3.079381in}{0.413320in}}%
\pgfpathlineto{\pgfqpoint{3.076631in}{0.413320in}}%
\pgfpathlineto{\pgfqpoint{3.074056in}{0.413320in}}%
\pgfpathlineto{\pgfqpoint{3.071266in}{0.413320in}}%
\pgfpathlineto{\pgfqpoint{3.068709in}{0.413320in}}%
\pgfpathlineto{\pgfqpoint{3.065916in}{0.413320in}}%
\pgfpathlineto{\pgfqpoint{3.063230in}{0.413320in}}%
\pgfpathlineto{\pgfqpoint{3.060561in}{0.413320in}}%
\pgfpathlineto{\pgfqpoint{3.057884in}{0.413320in}}%
\pgfpathlineto{\pgfqpoint{3.055202in}{0.413320in}}%
\pgfpathlineto{\pgfqpoint{3.052526in}{0.413320in}}%
\pgfpathlineto{\pgfqpoint{3.049988in}{0.413320in}}%
\pgfpathlineto{\pgfqpoint{3.047157in}{0.413320in}}%
\pgfpathlineto{\pgfqpoint{3.044568in}{0.413320in}}%
\pgfpathlineto{\pgfqpoint{3.041813in}{0.413320in}}%
\pgfpathlineto{\pgfqpoint{3.039262in}{0.413320in}}%
\pgfpathlineto{\pgfqpoint{3.036456in}{0.413320in}}%
\pgfpathlineto{\pgfqpoint{3.033921in}{0.413320in}}%
\pgfpathlineto{\pgfqpoint{3.031091in}{0.413320in}}%
\pgfpathlineto{\pgfqpoint{3.028412in}{0.413320in}}%
\pgfpathlineto{\pgfqpoint{3.025803in}{0.413320in}}%
\pgfpathlineto{\pgfqpoint{3.023058in}{0.413320in}}%
\pgfpathlineto{\pgfqpoint{3.020382in}{0.413320in}}%
\pgfpathlineto{\pgfqpoint{3.017707in}{0.413320in}}%
\pgfpathlineto{\pgfqpoint{3.015097in}{0.413320in}}%
\pgfpathlineto{\pgfqpoint{3.012351in}{0.413320in}}%
\pgfpathlineto{\pgfqpoint{3.009784in}{0.413320in}}%
\pgfpathlineto{\pgfqpoint{3.006993in}{0.413320in}}%
\pgfpathlineto{\pgfqpoint{3.004419in}{0.413320in}}%
\pgfpathlineto{\pgfqpoint{3.001635in}{0.413320in}}%
\pgfpathlineto{\pgfqpoint{2.999103in}{0.413320in}}%
\pgfpathlineto{\pgfqpoint{2.996300in}{0.413320in}}%
\pgfpathlineto{\pgfqpoint{2.993595in}{0.413320in}}%
\pgfpathlineto{\pgfqpoint{2.990978in}{0.413320in}}%
\pgfpathlineto{\pgfqpoint{2.988238in}{0.413320in}}%
\pgfpathlineto{\pgfqpoint{2.985666in}{0.413320in}}%
\pgfpathlineto{\pgfqpoint{2.982885in}{0.413320in}}%
\pgfpathlineto{\pgfqpoint{2.980341in}{0.413320in}}%
\pgfpathlineto{\pgfqpoint{2.977517in}{0.413320in}}%
\pgfpathlineto{\pgfqpoint{2.974972in}{0.413320in}}%
\pgfpathlineto{\pgfqpoint{2.972177in}{0.413320in}}%
\pgfpathlineto{\pgfqpoint{2.969599in}{0.413320in}}%
\pgfpathlineto{\pgfqpoint{2.966812in}{0.413320in}}%
\pgfpathlineto{\pgfqpoint{2.964127in}{0.413320in}}%
\pgfpathlineto{\pgfqpoint{2.961460in}{0.413320in}}%
\pgfpathlineto{\pgfqpoint{2.958782in}{0.413320in}}%
\pgfpathlineto{\pgfqpoint{2.956103in}{0.413320in}}%
\pgfpathlineto{\pgfqpoint{2.953422in}{0.413320in}}%
\pgfpathlineto{\pgfqpoint{2.950884in}{0.413320in}}%
\pgfpathlineto{\pgfqpoint{2.948068in}{0.413320in}}%
\pgfpathlineto{\pgfqpoint{2.945461in}{0.413320in}}%
\pgfpathlineto{\pgfqpoint{2.942711in}{0.413320in}}%
\pgfpathlineto{\pgfqpoint{2.940120in}{0.413320in}}%
\pgfpathlineto{\pgfqpoint{2.937352in}{0.413320in}}%
\pgfpathlineto{\pgfqpoint{2.934759in}{0.413320in}}%
\pgfpathlineto{\pgfqpoint{2.932033in}{0.413320in}}%
\pgfpathlineto{\pgfqpoint{2.929321in}{0.413320in}}%
\pgfpathlineto{\pgfqpoint{2.926655in}{0.413320in}}%
\pgfpathlineto{\pgfqpoint{2.923963in}{0.413320in}}%
\pgfpathlineto{\pgfqpoint{2.921363in}{0.413320in}}%
\pgfpathlineto{\pgfqpoint{2.918606in}{0.413320in}}%
\pgfpathlineto{\pgfqpoint{2.916061in}{0.413320in}}%
\pgfpathlineto{\pgfqpoint{2.913243in}{0.413320in}}%
\pgfpathlineto{\pgfqpoint{2.910631in}{0.413320in}}%
\pgfpathlineto{\pgfqpoint{2.907882in}{0.413320in}}%
\pgfpathlineto{\pgfqpoint{2.905341in}{0.413320in}}%
\pgfpathlineto{\pgfqpoint{2.902535in}{0.413320in}}%
\pgfpathlineto{\pgfqpoint{2.899858in}{0.413320in}}%
\pgfpathlineto{\pgfqpoint{2.897179in}{0.413320in}}%
\pgfpathlineto{\pgfqpoint{2.894487in}{0.413320in}}%
\pgfpathlineto{\pgfqpoint{2.891809in}{0.413320in}}%
\pgfpathlineto{\pgfqpoint{2.889145in}{0.413320in}}%
\pgfpathlineto{\pgfqpoint{2.886578in}{0.413320in}}%
\pgfpathlineto{\pgfqpoint{2.883780in}{0.413320in}}%
\pgfpathlineto{\pgfqpoint{2.881254in}{0.413320in}}%
\pgfpathlineto{\pgfqpoint{2.878431in}{0.413320in}}%
\pgfpathlineto{\pgfqpoint{2.875882in}{0.413320in}}%
\pgfpathlineto{\pgfqpoint{2.873074in}{0.413320in}}%
\pgfpathlineto{\pgfqpoint{2.870475in}{0.413320in}}%
\pgfpathlineto{\pgfqpoint{2.867713in}{0.413320in}}%
\pgfpathlineto{\pgfqpoint{2.865031in}{0.413320in}}%
\pgfpathlineto{\pgfqpoint{2.862402in}{0.413320in}}%
\pgfpathlineto{\pgfqpoint{2.859668in}{0.413320in}}%
\pgfpathlineto{\pgfqpoint{2.857003in}{0.413320in}}%
\pgfpathlineto{\pgfqpoint{2.854325in}{0.413320in}}%
\pgfpathlineto{\pgfqpoint{2.851793in}{0.413320in}}%
\pgfpathlineto{\pgfqpoint{2.848960in}{0.413320in}}%
\pgfpathlineto{\pgfqpoint{2.846408in}{0.413320in}}%
\pgfpathlineto{\pgfqpoint{2.843611in}{0.413320in}}%
\pgfpathlineto{\pgfqpoint{2.841055in}{0.413320in}}%
\pgfpathlineto{\pgfqpoint{2.838254in}{0.413320in}}%
\pgfpathlineto{\pgfqpoint{2.835698in}{0.413320in}}%
\pgfpathlineto{\pgfqpoint{2.832894in}{0.413320in}}%
\pgfpathlineto{\pgfqpoint{2.830219in}{0.413320in}}%
\pgfpathlineto{\pgfqpoint{2.827567in}{0.413320in}}%
\pgfpathlineto{\pgfqpoint{2.824851in}{0.413320in}}%
\pgfpathlineto{\pgfqpoint{2.822303in}{0.413320in}}%
\pgfpathlineto{\pgfqpoint{2.819506in}{0.413320in}}%
\pgfpathlineto{\pgfqpoint{2.816867in}{0.413320in}}%
\pgfpathlineto{\pgfqpoint{2.814141in}{0.413320in}}%
\pgfpathlineto{\pgfqpoint{2.811597in}{0.413320in}}%
\pgfpathlineto{\pgfqpoint{2.808792in}{0.413320in}}%
\pgfpathlineto{\pgfqpoint{2.806175in}{0.413320in}}%
\pgfpathlineto{\pgfqpoint{2.803435in}{0.413320in}}%
\pgfpathlineto{\pgfqpoint{2.800756in}{0.413320in}}%
\pgfpathlineto{\pgfqpoint{2.798070in}{0.413320in}}%
\pgfpathlineto{\pgfqpoint{2.795398in}{0.413320in}}%
\pgfpathlineto{\pgfqpoint{2.792721in}{0.413320in}}%
\pgfpathlineto{\pgfqpoint{2.790044in}{0.413320in}}%
\pgfpathlineto{\pgfqpoint{2.787468in}{0.413320in}}%
\pgfpathlineto{\pgfqpoint{2.784687in}{0.413320in}}%
\pgfpathlineto{\pgfqpoint{2.782113in}{0.413320in}}%
\pgfpathlineto{\pgfqpoint{2.779330in}{0.413320in}}%
\pgfpathlineto{\pgfqpoint{2.776767in}{0.413320in}}%
\pgfpathlineto{\pgfqpoint{2.773972in}{0.413320in}}%
\pgfpathlineto{\pgfqpoint{2.771373in}{0.413320in}}%
\pgfpathlineto{\pgfqpoint{2.768617in}{0.413320in}}%
\pgfpathlineto{\pgfqpoint{2.765935in}{0.413320in}}%
\pgfpathlineto{\pgfqpoint{2.763253in}{0.413320in}}%
\pgfpathlineto{\pgfqpoint{2.760581in}{0.413320in}}%
\pgfpathlineto{\pgfqpoint{2.758028in}{0.413320in}}%
\pgfpathlineto{\pgfqpoint{2.755224in}{0.413320in}}%
\pgfpathlineto{\pgfqpoint{2.752614in}{0.413320in}}%
\pgfpathlineto{\pgfqpoint{2.749868in}{0.413320in}}%
\pgfpathlineto{\pgfqpoint{2.747260in}{0.413320in}}%
\pgfpathlineto{\pgfqpoint{2.744510in}{0.413320in}}%
\pgfpathlineto{\pgfqpoint{2.741928in}{0.413320in}}%
\pgfpathlineto{\pgfqpoint{2.739155in}{0.413320in}}%
\pgfpathlineto{\pgfqpoint{2.736476in}{0.413320in}}%
\pgfpathlineto{\pgfqpoint{2.733798in}{0.413320in}}%
\pgfpathlineto{\pgfqpoint{2.731119in}{0.413320in}}%
\pgfpathlineto{\pgfqpoint{2.728439in}{0.413320in}}%
\pgfpathlineto{\pgfqpoint{2.725760in}{0.413320in}}%
\pgfpathlineto{\pgfqpoint{2.723211in}{0.413320in}}%
\pgfpathlineto{\pgfqpoint{2.720404in}{0.413320in}}%
\pgfpathlineto{\pgfqpoint{2.717773in}{0.413320in}}%
\pgfpathlineto{\pgfqpoint{2.715036in}{0.413320in}}%
\pgfpathlineto{\pgfqpoint{2.712477in}{0.413320in}}%
\pgfpathlineto{\pgfqpoint{2.709683in}{0.413320in}}%
\pgfpathlineto{\pgfqpoint{2.707125in}{0.413320in}}%
\pgfpathlineto{\pgfqpoint{2.704326in}{0.413320in}}%
\pgfpathlineto{\pgfqpoint{2.701657in}{0.413320in}}%
\pgfpathlineto{\pgfqpoint{2.698968in}{0.413320in}}%
\pgfpathlineto{\pgfqpoint{2.696293in}{0.413320in}}%
\pgfpathlineto{\pgfqpoint{2.693611in}{0.413320in}}%
\pgfpathlineto{\pgfqpoint{2.690940in}{0.413320in}}%
\pgfpathlineto{\pgfqpoint{2.688328in}{0.413320in}}%
\pgfpathlineto{\pgfqpoint{2.685586in}{0.413320in}}%
\pgfpathlineto{\pgfqpoint{2.683009in}{0.413320in}}%
\pgfpathlineto{\pgfqpoint{2.680224in}{0.413320in}}%
\pgfpathlineto{\pgfqpoint{2.677650in}{0.413320in}}%
\pgfpathlineto{\pgfqpoint{2.674873in}{0.413320in}}%
\pgfpathlineto{\pgfqpoint{2.672301in}{0.413320in}}%
\pgfpathlineto{\pgfqpoint{2.669506in}{0.413320in}}%
\pgfpathlineto{\pgfqpoint{2.666836in}{0.413320in}}%
\pgfpathlineto{\pgfqpoint{2.664151in}{0.413320in}}%
\pgfpathlineto{\pgfqpoint{2.661481in}{0.413320in}}%
\pgfpathlineto{\pgfqpoint{2.658942in}{0.413320in}}%
\pgfpathlineto{\pgfqpoint{2.656124in}{0.413320in}}%
\pgfpathlineto{\pgfqpoint{2.653567in}{0.413320in}}%
\pgfpathlineto{\pgfqpoint{2.650767in}{0.413320in}}%
\pgfpathlineto{\pgfqpoint{2.648196in}{0.413320in}}%
\pgfpathlineto{\pgfqpoint{2.645408in}{0.413320in}}%
\pgfpathlineto{\pgfqpoint{2.642827in}{0.413320in}}%
\pgfpathlineto{\pgfqpoint{2.640053in}{0.413320in}}%
\pgfpathlineto{\pgfqpoint{2.637369in}{0.413320in}}%
\pgfpathlineto{\pgfqpoint{2.634700in}{0.413320in}}%
\pgfpathlineto{\pgfqpoint{2.632018in}{0.413320in}}%
\pgfpathlineto{\pgfqpoint{2.629340in}{0.413320in}}%
\pgfpathlineto{\pgfqpoint{2.626653in}{0.413320in}}%
\pgfpathlineto{\pgfqpoint{2.624077in}{0.413320in}}%
\pgfpathlineto{\pgfqpoint{2.621304in}{0.413320in}}%
\pgfpathlineto{\pgfqpoint{2.618773in}{0.413320in}}%
\pgfpathlineto{\pgfqpoint{2.615934in}{0.413320in}}%
\pgfpathlineto{\pgfqpoint{2.613393in}{0.413320in}}%
\pgfpathlineto{\pgfqpoint{2.610588in}{0.413320in}}%
\pgfpathlineto{\pgfqpoint{2.608004in}{0.413320in}}%
\pgfpathlineto{\pgfqpoint{2.605232in}{0.413320in}}%
\pgfpathlineto{\pgfqpoint{2.602557in}{0.413320in}}%
\pgfpathlineto{\pgfqpoint{2.599920in}{0.413320in}}%
\pgfpathlineto{\pgfqpoint{2.597196in}{0.413320in}}%
\pgfpathlineto{\pgfqpoint{2.594630in}{0.413320in}}%
\pgfpathlineto{\pgfqpoint{2.591842in}{0.413320in}}%
\pgfpathlineto{\pgfqpoint{2.589248in}{0.413320in}}%
\pgfpathlineto{\pgfqpoint{2.586484in}{0.413320in}}%
\pgfpathlineto{\pgfqpoint{2.583913in}{0.413320in}}%
\pgfpathlineto{\pgfqpoint{2.581129in}{0.413320in}}%
\pgfpathlineto{\pgfqpoint{2.578567in}{0.413320in}}%
\pgfpathlineto{\pgfqpoint{2.575779in}{0.413320in}}%
\pgfpathlineto{\pgfqpoint{2.573082in}{0.413320in}}%
\pgfpathlineto{\pgfqpoint{2.570411in}{0.413320in}}%
\pgfpathlineto{\pgfqpoint{2.567730in}{0.413320in}}%
\pgfpathlineto{\pgfqpoint{2.565045in}{0.413320in}}%
\pgfpathlineto{\pgfqpoint{2.562375in}{0.413320in}}%
\pgfpathlineto{\pgfqpoint{2.559790in}{0.413320in}}%
\pgfpathlineto{\pgfqpoint{2.557009in}{0.413320in}}%
\pgfpathlineto{\pgfqpoint{2.554493in}{0.413320in}}%
\pgfpathlineto{\pgfqpoint{2.551664in}{0.413320in}}%
\pgfpathlineto{\pgfqpoint{2.549114in}{0.413320in}}%
\pgfpathlineto{\pgfqpoint{2.546310in}{0.413320in}}%
\pgfpathlineto{\pgfqpoint{2.543765in}{0.413320in}}%
\pgfpathlineto{\pgfqpoint{2.540949in}{0.413320in}}%
\pgfpathlineto{\pgfqpoint{2.538274in}{0.413320in}}%
\pgfpathlineto{\pgfqpoint{2.535624in}{0.413320in}}%
\pgfpathlineto{\pgfqpoint{2.532917in}{0.413320in}}%
\pgfpathlineto{\pgfqpoint{2.530234in}{0.413320in}}%
\pgfpathlineto{\pgfqpoint{2.527560in}{0.413320in}}%
\pgfpathlineto{\pgfqpoint{2.524988in}{0.413320in}}%
\pgfpathlineto{\pgfqpoint{2.522197in}{0.413320in}}%
\pgfpathlineto{\pgfqpoint{2.519607in}{0.413320in}}%
\pgfpathlineto{\pgfqpoint{2.516845in}{0.413320in}}%
\pgfpathlineto{\pgfqpoint{2.514268in}{0.413320in}}%
\pgfpathlineto{\pgfqpoint{2.511478in}{0.413320in}}%
\pgfpathlineto{\pgfqpoint{2.508917in}{0.413320in}}%
\pgfpathlineto{\pgfqpoint{2.506163in}{0.413320in}}%
\pgfpathlineto{\pgfqpoint{2.503454in}{0.413320in}}%
\pgfpathlineto{\pgfqpoint{2.500801in}{0.413320in}}%
\pgfpathlineto{\pgfqpoint{2.498085in}{0.413320in}}%
\pgfpathlineto{\pgfqpoint{2.495542in}{0.413320in}}%
\pgfpathlineto{\pgfqpoint{2.492729in}{0.413320in}}%
\pgfpathlineto{\pgfqpoint{2.490183in}{0.413320in}}%
\pgfpathlineto{\pgfqpoint{2.487384in}{0.413320in}}%
\pgfpathlineto{\pgfqpoint{2.484870in}{0.413320in}}%
\pgfpathlineto{\pgfqpoint{2.482026in}{0.413320in}}%
\pgfpathlineto{\pgfqpoint{2.479420in}{0.413320in}}%
\pgfpathlineto{\pgfqpoint{2.476671in}{0.413320in}}%
\pgfpathlineto{\pgfqpoint{2.473989in}{0.413320in}}%
\pgfpathlineto{\pgfqpoint{2.471311in}{0.413320in}}%
\pgfpathlineto{\pgfqpoint{2.468635in}{0.413320in}}%
\pgfpathlineto{\pgfqpoint{2.465957in}{0.413320in}}%
\pgfpathlineto{\pgfqpoint{2.463280in}{0.413320in}}%
\pgfpathlineto{\pgfqpoint{2.460711in}{0.413320in}}%
\pgfpathlineto{\pgfqpoint{2.457917in}{0.413320in}}%
\pgfpathlineto{\pgfqpoint{2.455353in}{0.413320in}}%
\pgfpathlineto{\pgfqpoint{2.452562in}{0.413320in}}%
\pgfpathlineto{\pgfqpoint{2.450032in}{0.413320in}}%
\pgfpathlineto{\pgfqpoint{2.447209in}{0.413320in}}%
\pgfpathlineto{\pgfqpoint{2.444677in}{0.413320in}}%
\pgfpathlineto{\pgfqpoint{2.441876in}{0.413320in}}%
\pgfpathlineto{\pgfqpoint{2.439167in}{0.413320in}}%
\pgfpathlineto{\pgfqpoint{2.436518in}{0.413320in}}%
\pgfpathlineto{\pgfqpoint{2.433815in}{0.413320in}}%
\pgfpathlineto{\pgfqpoint{2.431251in}{0.413320in}}%
\pgfpathlineto{\pgfqpoint{2.428453in}{0.413320in}}%
\pgfpathlineto{\pgfqpoint{2.425878in}{0.413320in}}%
\pgfpathlineto{\pgfqpoint{2.423098in}{0.413320in}}%
\pgfpathlineto{\pgfqpoint{2.420528in}{0.413320in}}%
\pgfpathlineto{\pgfqpoint{2.417747in}{0.413320in}}%
\pgfpathlineto{\pgfqpoint{2.415184in}{0.413320in}}%
\pgfpathlineto{\pgfqpoint{2.412389in}{0.413320in}}%
\pgfpathlineto{\pgfqpoint{2.409699in}{0.413320in}}%
\pgfpathlineto{\pgfqpoint{2.407024in}{0.413320in}}%
\pgfpathlineto{\pgfqpoint{2.404352in}{0.413320in}}%
\pgfpathlineto{\pgfqpoint{2.401675in}{0.413320in}}%
\pgfpathlineto{\pgfqpoint{2.398995in}{0.413320in}}%
\pgfpathclose%
\pgfusepath{stroke,fill}%
\end{pgfscope}%
\begin{pgfscope}%
\pgfpathrectangle{\pgfqpoint{2.398995in}{0.319877in}}{\pgfqpoint{3.986877in}{1.993438in}} %
\pgfusepath{clip}%
\pgfsetbuttcap%
\pgfsetroundjoin%
\definecolor{currentfill}{rgb}{1.000000,1.000000,1.000000}%
\pgfsetfillcolor{currentfill}%
\pgfsetlinewidth{1.003750pt}%
\definecolor{currentstroke}{rgb}{0.715434,0.602731,0.194299}%
\pgfsetstrokecolor{currentstroke}%
\pgfsetdash{}{0pt}%
\pgfpathmoveto{\pgfqpoint{2.398995in}{0.413320in}}%
\pgfpathlineto{\pgfqpoint{2.398995in}{1.666892in}}%
\pgfpathlineto{\pgfqpoint{2.401675in}{1.667459in}}%
\pgfpathlineto{\pgfqpoint{2.404352in}{1.668290in}}%
\pgfpathlineto{\pgfqpoint{2.407024in}{1.668943in}}%
\pgfpathlineto{\pgfqpoint{2.409699in}{1.667518in}}%
\pgfpathlineto{\pgfqpoint{2.412389in}{1.665278in}}%
\pgfpathlineto{\pgfqpoint{2.415184in}{1.666762in}}%
\pgfpathlineto{\pgfqpoint{2.417747in}{1.665379in}}%
\pgfpathlineto{\pgfqpoint{2.420528in}{1.667668in}}%
\pgfpathlineto{\pgfqpoint{2.423098in}{1.664324in}}%
\pgfpathlineto{\pgfqpoint{2.425878in}{1.660777in}}%
\pgfpathlineto{\pgfqpoint{2.428453in}{1.661174in}}%
\pgfpathlineto{\pgfqpoint{2.431251in}{1.661120in}}%
\pgfpathlineto{\pgfqpoint{2.433815in}{1.657302in}}%
\pgfpathlineto{\pgfqpoint{2.436518in}{1.657302in}}%
\pgfpathlineto{\pgfqpoint{2.439167in}{1.660153in}}%
\pgfpathlineto{\pgfqpoint{2.441876in}{1.657998in}}%
\pgfpathlineto{\pgfqpoint{2.444677in}{1.660157in}}%
\pgfpathlineto{\pgfqpoint{2.447209in}{1.666384in}}%
\pgfpathlineto{\pgfqpoint{2.450032in}{1.661423in}}%
\pgfpathlineto{\pgfqpoint{2.452562in}{1.665552in}}%
\pgfpathlineto{\pgfqpoint{2.455353in}{1.668916in}}%
\pgfpathlineto{\pgfqpoint{2.457917in}{1.668140in}}%
\pgfpathlineto{\pgfqpoint{2.460711in}{1.667883in}}%
\pgfpathlineto{\pgfqpoint{2.463280in}{1.669282in}}%
\pgfpathlineto{\pgfqpoint{2.465957in}{1.669952in}}%
\pgfpathlineto{\pgfqpoint{2.468635in}{1.668337in}}%
\pgfpathlineto{\pgfqpoint{2.471311in}{1.664995in}}%
\pgfpathlineto{\pgfqpoint{2.473989in}{1.664490in}}%
\pgfpathlineto{\pgfqpoint{2.476671in}{1.660964in}}%
\pgfpathlineto{\pgfqpoint{2.479420in}{1.661871in}}%
\pgfpathlineto{\pgfqpoint{2.482026in}{1.657302in}}%
\pgfpathlineto{\pgfqpoint{2.484870in}{1.657302in}}%
\pgfpathlineto{\pgfqpoint{2.487384in}{1.657907in}}%
\pgfpathlineto{\pgfqpoint{2.490183in}{1.659530in}}%
\pgfpathlineto{\pgfqpoint{2.492729in}{1.659099in}}%
\pgfpathlineto{\pgfqpoint{2.495542in}{1.664296in}}%
\pgfpathlineto{\pgfqpoint{2.498085in}{1.659301in}}%
\pgfpathlineto{\pgfqpoint{2.500801in}{1.662996in}}%
\pgfpathlineto{\pgfqpoint{2.503454in}{1.661528in}}%
\pgfpathlineto{\pgfqpoint{2.506163in}{1.664282in}}%
\pgfpathlineto{\pgfqpoint{2.508917in}{1.667962in}}%
\pgfpathlineto{\pgfqpoint{2.511478in}{1.662577in}}%
\pgfpathlineto{\pgfqpoint{2.514268in}{1.664379in}}%
\pgfpathlineto{\pgfqpoint{2.516845in}{1.669872in}}%
\pgfpathlineto{\pgfqpoint{2.519607in}{1.669873in}}%
\pgfpathlineto{\pgfqpoint{2.522197in}{1.670558in}}%
\pgfpathlineto{\pgfqpoint{2.524988in}{1.662962in}}%
\pgfpathlineto{\pgfqpoint{2.527560in}{1.660690in}}%
\pgfpathlineto{\pgfqpoint{2.530234in}{1.661797in}}%
\pgfpathlineto{\pgfqpoint{2.532917in}{1.664554in}}%
\pgfpathlineto{\pgfqpoint{2.535624in}{1.658816in}}%
\pgfpathlineto{\pgfqpoint{2.538274in}{1.657804in}}%
\pgfpathlineto{\pgfqpoint{2.540949in}{1.662835in}}%
\pgfpathlineto{\pgfqpoint{2.543765in}{1.660082in}}%
\pgfpathlineto{\pgfqpoint{2.546310in}{1.664483in}}%
\pgfpathlineto{\pgfqpoint{2.549114in}{1.664723in}}%
\pgfpathlineto{\pgfqpoint{2.551664in}{1.659739in}}%
\pgfpathlineto{\pgfqpoint{2.554493in}{1.664113in}}%
\pgfpathlineto{\pgfqpoint{2.557009in}{1.665516in}}%
\pgfpathlineto{\pgfqpoint{2.559790in}{1.663833in}}%
\pgfpathlineto{\pgfqpoint{2.562375in}{1.664055in}}%
\pgfpathlineto{\pgfqpoint{2.565045in}{1.659734in}}%
\pgfpathlineto{\pgfqpoint{2.567730in}{1.664075in}}%
\pgfpathlineto{\pgfqpoint{2.570411in}{1.660259in}}%
\pgfpathlineto{\pgfqpoint{2.573082in}{1.669856in}}%
\pgfpathlineto{\pgfqpoint{2.575779in}{1.669875in}}%
\pgfpathlineto{\pgfqpoint{2.578567in}{1.662063in}}%
\pgfpathlineto{\pgfqpoint{2.581129in}{1.662250in}}%
\pgfpathlineto{\pgfqpoint{2.583913in}{1.663507in}}%
\pgfpathlineto{\pgfqpoint{2.586484in}{1.660363in}}%
\pgfpathlineto{\pgfqpoint{2.589248in}{1.662722in}}%
\pgfpathlineto{\pgfqpoint{2.591842in}{1.663604in}}%
\pgfpathlineto{\pgfqpoint{2.594630in}{1.667230in}}%
\pgfpathlineto{\pgfqpoint{2.597196in}{1.672276in}}%
\pgfpathlineto{\pgfqpoint{2.599920in}{1.668441in}}%
\pgfpathlineto{\pgfqpoint{2.602557in}{1.683234in}}%
\pgfpathlineto{\pgfqpoint{2.605232in}{1.691450in}}%
\pgfpathlineto{\pgfqpoint{2.608004in}{1.679258in}}%
\pgfpathlineto{\pgfqpoint{2.610588in}{1.679324in}}%
\pgfpathlineto{\pgfqpoint{2.613393in}{1.676392in}}%
\pgfpathlineto{\pgfqpoint{2.615934in}{1.695659in}}%
\pgfpathlineto{\pgfqpoint{2.618773in}{1.697500in}}%
\pgfpathlineto{\pgfqpoint{2.621304in}{1.698677in}}%
\pgfpathlineto{\pgfqpoint{2.624077in}{1.697291in}}%
\pgfpathlineto{\pgfqpoint{2.626653in}{1.698238in}}%
\pgfpathlineto{\pgfqpoint{2.629340in}{1.694106in}}%
\pgfpathlineto{\pgfqpoint{2.632018in}{1.691889in}}%
\pgfpathlineto{\pgfqpoint{2.634700in}{1.688714in}}%
\pgfpathlineto{\pgfqpoint{2.637369in}{1.686371in}}%
\pgfpathlineto{\pgfqpoint{2.640053in}{1.675899in}}%
\pgfpathlineto{\pgfqpoint{2.642827in}{1.682104in}}%
\pgfpathlineto{\pgfqpoint{2.645408in}{1.676180in}}%
\pgfpathlineto{\pgfqpoint{2.648196in}{1.677638in}}%
\pgfpathlineto{\pgfqpoint{2.650767in}{1.670660in}}%
\pgfpathlineto{\pgfqpoint{2.653567in}{1.666974in}}%
\pgfpathlineto{\pgfqpoint{2.656124in}{1.668554in}}%
\pgfpathlineto{\pgfqpoint{2.658942in}{1.663831in}}%
\pgfpathlineto{\pgfqpoint{2.661481in}{1.670919in}}%
\pgfpathlineto{\pgfqpoint{2.664151in}{1.679219in}}%
\pgfpathlineto{\pgfqpoint{2.666836in}{1.685406in}}%
\pgfpathlineto{\pgfqpoint{2.669506in}{1.685415in}}%
\pgfpathlineto{\pgfqpoint{2.672301in}{1.684830in}}%
\pgfpathlineto{\pgfqpoint{2.674873in}{1.682119in}}%
\pgfpathlineto{\pgfqpoint{2.677650in}{1.676608in}}%
\pgfpathlineto{\pgfqpoint{2.680224in}{1.674793in}}%
\pgfpathlineto{\pgfqpoint{2.683009in}{1.674952in}}%
\pgfpathlineto{\pgfqpoint{2.685586in}{1.673246in}}%
\pgfpathlineto{\pgfqpoint{2.688328in}{1.665562in}}%
\pgfpathlineto{\pgfqpoint{2.690940in}{1.667217in}}%
\pgfpathlineto{\pgfqpoint{2.693611in}{1.680326in}}%
\pgfpathlineto{\pgfqpoint{2.696293in}{1.675423in}}%
\pgfpathlineto{\pgfqpoint{2.698968in}{1.671610in}}%
\pgfpathlineto{\pgfqpoint{2.701657in}{1.667265in}}%
\pgfpathlineto{\pgfqpoint{2.704326in}{1.666621in}}%
\pgfpathlineto{\pgfqpoint{2.707125in}{1.667341in}}%
\pgfpathlineto{\pgfqpoint{2.709683in}{1.668166in}}%
\pgfpathlineto{\pgfqpoint{2.712477in}{1.671991in}}%
\pgfpathlineto{\pgfqpoint{2.715036in}{1.669185in}}%
\pgfpathlineto{\pgfqpoint{2.717773in}{1.661440in}}%
\pgfpathlineto{\pgfqpoint{2.720404in}{1.658400in}}%
\pgfpathlineto{\pgfqpoint{2.723211in}{1.666588in}}%
\pgfpathlineto{\pgfqpoint{2.725760in}{1.711735in}}%
\pgfpathlineto{\pgfqpoint{2.728439in}{1.743511in}}%
\pgfpathlineto{\pgfqpoint{2.731119in}{1.823125in}}%
\pgfpathlineto{\pgfqpoint{2.733798in}{1.794820in}}%
\pgfpathlineto{\pgfqpoint{2.736476in}{1.776216in}}%
\pgfpathlineto{\pgfqpoint{2.739155in}{1.752709in}}%
\pgfpathlineto{\pgfqpoint{2.741928in}{1.751590in}}%
\pgfpathlineto{\pgfqpoint{2.744510in}{1.745088in}}%
\pgfpathlineto{\pgfqpoint{2.747260in}{1.736969in}}%
\pgfpathlineto{\pgfqpoint{2.749868in}{1.737377in}}%
\pgfpathlineto{\pgfqpoint{2.752614in}{1.716781in}}%
\pgfpathlineto{\pgfqpoint{2.755224in}{1.711458in}}%
\pgfpathlineto{\pgfqpoint{2.758028in}{1.720580in}}%
\pgfpathlineto{\pgfqpoint{2.760581in}{1.724831in}}%
\pgfpathlineto{\pgfqpoint{2.763253in}{1.746871in}}%
\pgfpathlineto{\pgfqpoint{2.765935in}{1.767055in}}%
\pgfpathlineto{\pgfqpoint{2.768617in}{1.767377in}}%
\pgfpathlineto{\pgfqpoint{2.771373in}{1.755399in}}%
\pgfpathlineto{\pgfqpoint{2.773972in}{1.718894in}}%
\pgfpathlineto{\pgfqpoint{2.776767in}{1.696357in}}%
\pgfpathlineto{\pgfqpoint{2.779330in}{1.662384in}}%
\pgfpathlineto{\pgfqpoint{2.782113in}{1.672158in}}%
\pgfpathlineto{\pgfqpoint{2.784687in}{1.661195in}}%
\pgfpathlineto{\pgfqpoint{2.787468in}{1.657302in}}%
\pgfpathlineto{\pgfqpoint{2.790044in}{1.657302in}}%
\pgfpathlineto{\pgfqpoint{2.792721in}{1.657302in}}%
\pgfpathlineto{\pgfqpoint{2.795398in}{1.657302in}}%
\pgfpathlineto{\pgfqpoint{2.798070in}{1.659506in}}%
\pgfpathlineto{\pgfqpoint{2.800756in}{1.657302in}}%
\pgfpathlineto{\pgfqpoint{2.803435in}{1.657302in}}%
\pgfpathlineto{\pgfqpoint{2.806175in}{1.657302in}}%
\pgfpathlineto{\pgfqpoint{2.808792in}{1.657302in}}%
\pgfpathlineto{\pgfqpoint{2.811597in}{1.657302in}}%
\pgfpathlineto{\pgfqpoint{2.814141in}{1.659841in}}%
\pgfpathlineto{\pgfqpoint{2.816867in}{1.662936in}}%
\pgfpathlineto{\pgfqpoint{2.819506in}{1.663926in}}%
\pgfpathlineto{\pgfqpoint{2.822303in}{1.662586in}}%
\pgfpathlineto{\pgfqpoint{2.824851in}{1.660316in}}%
\pgfpathlineto{\pgfqpoint{2.827567in}{1.657584in}}%
\pgfpathlineto{\pgfqpoint{2.830219in}{1.657302in}}%
\pgfpathlineto{\pgfqpoint{2.832894in}{1.662842in}}%
\pgfpathlineto{\pgfqpoint{2.835698in}{1.664375in}}%
\pgfpathlineto{\pgfqpoint{2.838254in}{1.659235in}}%
\pgfpathlineto{\pgfqpoint{2.841055in}{1.661212in}}%
\pgfpathlineto{\pgfqpoint{2.843611in}{1.662108in}}%
\pgfpathlineto{\pgfqpoint{2.846408in}{1.664902in}}%
\pgfpathlineto{\pgfqpoint{2.848960in}{1.672331in}}%
\pgfpathlineto{\pgfqpoint{2.851793in}{1.664713in}}%
\pgfpathlineto{\pgfqpoint{2.854325in}{1.666470in}}%
\pgfpathlineto{\pgfqpoint{2.857003in}{1.665015in}}%
\pgfpathlineto{\pgfqpoint{2.859668in}{1.662926in}}%
\pgfpathlineto{\pgfqpoint{2.862402in}{1.661887in}}%
\pgfpathlineto{\pgfqpoint{2.865031in}{1.657302in}}%
\pgfpathlineto{\pgfqpoint{2.867713in}{1.664056in}}%
\pgfpathlineto{\pgfqpoint{2.870475in}{1.661259in}}%
\pgfpathlineto{\pgfqpoint{2.873074in}{1.665561in}}%
\pgfpathlineto{\pgfqpoint{2.875882in}{1.664331in}}%
\pgfpathlineto{\pgfqpoint{2.878431in}{1.663750in}}%
\pgfpathlineto{\pgfqpoint{2.881254in}{1.664294in}}%
\pgfpathlineto{\pgfqpoint{2.883780in}{1.658586in}}%
\pgfpathlineto{\pgfqpoint{2.886578in}{1.665137in}}%
\pgfpathlineto{\pgfqpoint{2.889145in}{1.669357in}}%
\pgfpathlineto{\pgfqpoint{2.891809in}{1.671719in}}%
\pgfpathlineto{\pgfqpoint{2.894487in}{1.663615in}}%
\pgfpathlineto{\pgfqpoint{2.897179in}{1.661836in}}%
\pgfpathlineto{\pgfqpoint{2.899858in}{1.658211in}}%
\pgfpathlineto{\pgfqpoint{2.902535in}{1.660769in}}%
\pgfpathlineto{\pgfqpoint{2.905341in}{1.662303in}}%
\pgfpathlineto{\pgfqpoint{2.907882in}{1.670475in}}%
\pgfpathlineto{\pgfqpoint{2.910631in}{1.663537in}}%
\pgfpathlineto{\pgfqpoint{2.913243in}{1.666512in}}%
\pgfpathlineto{\pgfqpoint{2.916061in}{1.663672in}}%
\pgfpathlineto{\pgfqpoint{2.918606in}{1.660965in}}%
\pgfpathlineto{\pgfqpoint{2.921363in}{1.666372in}}%
\pgfpathlineto{\pgfqpoint{2.923963in}{1.664980in}}%
\pgfpathlineto{\pgfqpoint{2.926655in}{1.668715in}}%
\pgfpathlineto{\pgfqpoint{2.929321in}{1.666880in}}%
\pgfpathlineto{\pgfqpoint{2.932033in}{1.666112in}}%
\pgfpathlineto{\pgfqpoint{2.934759in}{1.667376in}}%
\pgfpathlineto{\pgfqpoint{2.937352in}{1.663126in}}%
\pgfpathlineto{\pgfqpoint{2.940120in}{1.660560in}}%
\pgfpathlineto{\pgfqpoint{2.942711in}{1.657302in}}%
\pgfpathlineto{\pgfqpoint{2.945461in}{1.662149in}}%
\pgfpathlineto{\pgfqpoint{2.948068in}{1.663553in}}%
\pgfpathlineto{\pgfqpoint{2.950884in}{1.664845in}}%
\pgfpathlineto{\pgfqpoint{2.953422in}{1.660330in}}%
\pgfpathlineto{\pgfqpoint{2.956103in}{1.664686in}}%
\pgfpathlineto{\pgfqpoint{2.958782in}{1.660688in}}%
\pgfpathlineto{\pgfqpoint{2.961460in}{1.664164in}}%
\pgfpathlineto{\pgfqpoint{2.964127in}{1.664151in}}%
\pgfpathlineto{\pgfqpoint{2.966812in}{1.666649in}}%
\pgfpathlineto{\pgfqpoint{2.969599in}{1.666620in}}%
\pgfpathlineto{\pgfqpoint{2.972177in}{1.663647in}}%
\pgfpathlineto{\pgfqpoint{2.974972in}{1.663838in}}%
\pgfpathlineto{\pgfqpoint{2.977517in}{1.665383in}}%
\pgfpathlineto{\pgfqpoint{2.980341in}{1.666546in}}%
\pgfpathlineto{\pgfqpoint{2.982885in}{1.667480in}}%
\pgfpathlineto{\pgfqpoint{2.985666in}{1.667790in}}%
\pgfpathlineto{\pgfqpoint{2.988238in}{1.664517in}}%
\pgfpathlineto{\pgfqpoint{2.990978in}{1.666400in}}%
\pgfpathlineto{\pgfqpoint{2.993595in}{1.668439in}}%
\pgfpathlineto{\pgfqpoint{2.996300in}{1.665484in}}%
\pgfpathlineto{\pgfqpoint{2.999103in}{1.660297in}}%
\pgfpathlineto{\pgfqpoint{3.001635in}{1.662974in}}%
\pgfpathlineto{\pgfqpoint{3.004419in}{1.665145in}}%
\pgfpathlineto{\pgfqpoint{3.006993in}{1.664934in}}%
\pgfpathlineto{\pgfqpoint{3.009784in}{1.664228in}}%
\pgfpathlineto{\pgfqpoint{3.012351in}{1.661091in}}%
\pgfpathlineto{\pgfqpoint{3.015097in}{1.665061in}}%
\pgfpathlineto{\pgfqpoint{3.017707in}{1.671361in}}%
\pgfpathlineto{\pgfqpoint{3.020382in}{1.669332in}}%
\pgfpathlineto{\pgfqpoint{3.023058in}{1.670067in}}%
\pgfpathlineto{\pgfqpoint{3.025803in}{1.668088in}}%
\pgfpathlineto{\pgfqpoint{3.028412in}{1.668222in}}%
\pgfpathlineto{\pgfqpoint{3.031091in}{1.657327in}}%
\pgfpathlineto{\pgfqpoint{3.033921in}{1.658651in}}%
\pgfpathlineto{\pgfqpoint{3.036456in}{1.665352in}}%
\pgfpathlineto{\pgfqpoint{3.039262in}{1.661463in}}%
\pgfpathlineto{\pgfqpoint{3.041813in}{1.660744in}}%
\pgfpathlineto{\pgfqpoint{3.044568in}{1.658456in}}%
\pgfpathlineto{\pgfqpoint{3.047157in}{1.664915in}}%
\pgfpathlineto{\pgfqpoint{3.049988in}{1.664834in}}%
\pgfpathlineto{\pgfqpoint{3.052526in}{1.661524in}}%
\pgfpathlineto{\pgfqpoint{3.055202in}{1.659318in}}%
\pgfpathlineto{\pgfqpoint{3.057884in}{1.657302in}}%
\pgfpathlineto{\pgfqpoint{3.060561in}{1.657302in}}%
\pgfpathlineto{\pgfqpoint{3.063230in}{1.660349in}}%
\pgfpathlineto{\pgfqpoint{3.065916in}{1.658297in}}%
\pgfpathlineto{\pgfqpoint{3.068709in}{1.660159in}}%
\pgfpathlineto{\pgfqpoint{3.071266in}{1.658155in}}%
\pgfpathlineto{\pgfqpoint{3.074056in}{1.660319in}}%
\pgfpathlineto{\pgfqpoint{3.076631in}{1.659500in}}%
\pgfpathlineto{\pgfqpoint{3.079381in}{1.668542in}}%
\pgfpathlineto{\pgfqpoint{3.081990in}{1.675455in}}%
\pgfpathlineto{\pgfqpoint{3.084671in}{1.672443in}}%
\pgfpathlineto{\pgfqpoint{3.087343in}{1.662729in}}%
\pgfpathlineto{\pgfqpoint{3.090023in}{1.665358in}}%
\pgfpathlineto{\pgfqpoint{3.092699in}{1.662492in}}%
\pgfpathlineto{\pgfqpoint{3.095388in}{1.660335in}}%
\pgfpathlineto{\pgfqpoint{3.098163in}{1.659222in}}%
\pgfpathlineto{\pgfqpoint{3.100737in}{1.657302in}}%
\pgfpathlineto{\pgfqpoint{3.103508in}{1.657302in}}%
\pgfpathlineto{\pgfqpoint{3.106094in}{1.658688in}}%
\pgfpathlineto{\pgfqpoint{3.108896in}{1.675704in}}%
\pgfpathlineto{\pgfqpoint{3.111451in}{1.686568in}}%
\pgfpathlineto{\pgfqpoint{3.114242in}{1.676465in}}%
\pgfpathlineto{\pgfqpoint{3.116807in}{1.671792in}}%
\pgfpathlineto{\pgfqpoint{3.119487in}{1.659349in}}%
\pgfpathlineto{\pgfqpoint{3.122163in}{1.664680in}}%
\pgfpathlineto{\pgfqpoint{3.124842in}{1.657989in}}%
\pgfpathlineto{\pgfqpoint{3.127512in}{1.659810in}}%
\pgfpathlineto{\pgfqpoint{3.130199in}{1.663603in}}%
\pgfpathlineto{\pgfqpoint{3.132946in}{1.663321in}}%
\pgfpathlineto{\pgfqpoint{3.135550in}{1.675436in}}%
\pgfpathlineto{\pgfqpoint{3.138375in}{1.683127in}}%
\pgfpathlineto{\pgfqpoint{3.140913in}{1.671540in}}%
\pgfpathlineto{\pgfqpoint{3.143740in}{1.667666in}}%
\pgfpathlineto{\pgfqpoint{3.146271in}{1.663516in}}%
\pgfpathlineto{\pgfqpoint{3.149057in}{1.670727in}}%
\pgfpathlineto{\pgfqpoint{3.151612in}{1.665358in}}%
\pgfpathlineto{\pgfqpoint{3.154327in}{1.657302in}}%
\pgfpathlineto{\pgfqpoint{3.156981in}{1.661643in}}%
\pgfpathlineto{\pgfqpoint{3.159675in}{1.664730in}}%
\pgfpathlineto{\pgfqpoint{3.162474in}{1.664503in}}%
\pgfpathlineto{\pgfqpoint{3.165019in}{1.657302in}}%
\pgfpathlineto{\pgfqpoint{3.167776in}{1.657302in}}%
\pgfpathlineto{\pgfqpoint{3.170375in}{1.657302in}}%
\pgfpathlineto{\pgfqpoint{3.173142in}{1.657302in}}%
\pgfpathlineto{\pgfqpoint{3.175724in}{1.657302in}}%
\pgfpathlineto{\pgfqpoint{3.178525in}{1.659298in}}%
\pgfpathlineto{\pgfqpoint{3.181089in}{1.657302in}}%
\pgfpathlineto{\pgfqpoint{3.183760in}{1.657302in}}%
\pgfpathlineto{\pgfqpoint{3.186440in}{1.666628in}}%
\pgfpathlineto{\pgfqpoint{3.189117in}{1.660659in}}%
\pgfpathlineto{\pgfqpoint{3.191796in}{1.664306in}}%
\pgfpathlineto{\pgfqpoint{3.194508in}{1.657302in}}%
\pgfpathlineto{\pgfqpoint{3.197226in}{1.657302in}}%
\pgfpathlineto{\pgfqpoint{3.199823in}{1.657302in}}%
\pgfpathlineto{\pgfqpoint{3.202562in}{1.657302in}}%
\pgfpathlineto{\pgfqpoint{3.205195in}{1.657302in}}%
\pgfpathlineto{\pgfqpoint{3.207984in}{1.657718in}}%
\pgfpathlineto{\pgfqpoint{3.210545in}{1.663842in}}%
\pgfpathlineto{\pgfqpoint{3.213342in}{1.661450in}}%
\pgfpathlineto{\pgfqpoint{3.215908in}{1.664167in}}%
\pgfpathlineto{\pgfqpoint{3.218586in}{1.666757in}}%
\pgfpathlineto{\pgfqpoint{3.221255in}{1.661489in}}%
\pgfpathlineto{\pgfqpoint{3.223942in}{1.657302in}}%
\pgfpathlineto{\pgfqpoint{3.226609in}{1.657302in}}%
\pgfpathlineto{\pgfqpoint{3.229310in}{1.660127in}}%
\pgfpathlineto{\pgfqpoint{3.232069in}{1.664799in}}%
\pgfpathlineto{\pgfqpoint{3.234658in}{1.665762in}}%
\pgfpathlineto{\pgfqpoint{3.237411in}{1.663956in}}%
\pgfpathlineto{\pgfqpoint{3.240010in}{1.657302in}}%
\pgfpathlineto{\pgfqpoint{3.242807in}{1.663049in}}%
\pgfpathlineto{\pgfqpoint{3.245363in}{1.659840in}}%
\pgfpathlineto{\pgfqpoint{3.248049in}{1.661250in}}%
\pgfpathlineto{\pgfqpoint{3.250716in}{1.669969in}}%
\pgfpathlineto{\pgfqpoint{3.253404in}{1.666191in}}%
\pgfpathlineto{\pgfqpoint{3.256083in}{1.669996in}}%
\pgfpathlineto{\pgfqpoint{3.258784in}{1.667530in}}%
\pgfpathlineto{\pgfqpoint{3.261594in}{1.669552in}}%
\pgfpathlineto{\pgfqpoint{3.264119in}{1.671533in}}%
\pgfpathlineto{\pgfqpoint{3.266849in}{1.668022in}}%
\pgfpathlineto{\pgfqpoint{3.269478in}{1.662144in}}%
\pgfpathlineto{\pgfqpoint{3.272254in}{1.661792in}}%
\pgfpathlineto{\pgfqpoint{3.274831in}{1.665472in}}%
\pgfpathlineto{\pgfqpoint{3.277603in}{1.667009in}}%
\pgfpathlineto{\pgfqpoint{3.280189in}{1.662051in}}%
\pgfpathlineto{\pgfqpoint{3.282870in}{1.660520in}}%
\pgfpathlineto{\pgfqpoint{3.285534in}{1.662586in}}%
\pgfpathlineto{\pgfqpoint{3.288225in}{1.662748in}}%
\pgfpathlineto{\pgfqpoint{3.290890in}{1.664089in}}%
\pgfpathlineto{\pgfqpoint{3.293574in}{1.668592in}}%
\pgfpathlineto{\pgfqpoint{3.296376in}{1.660144in}}%
\pgfpathlineto{\pgfqpoint{3.298937in}{1.664613in}}%
\pgfpathlineto{\pgfqpoint{3.301719in}{1.669549in}}%
\pgfpathlineto{\pgfqpoint{3.304295in}{1.665214in}}%
\pgfpathlineto{\pgfqpoint{3.307104in}{1.668423in}}%
\pgfpathlineto{\pgfqpoint{3.309652in}{1.671497in}}%
\pgfpathlineto{\pgfqpoint{3.312480in}{1.665662in}}%
\pgfpathlineto{\pgfqpoint{3.315008in}{1.670900in}}%
\pgfpathlineto{\pgfqpoint{3.317688in}{1.668642in}}%
\pgfpathlineto{\pgfqpoint{3.320366in}{1.666587in}}%
\pgfpathlineto{\pgfqpoint{3.323049in}{1.664024in}}%
\pgfpathlineto{\pgfqpoint{3.325860in}{1.664044in}}%
\pgfpathlineto{\pgfqpoint{3.328401in}{1.660792in}}%
\pgfpathlineto{\pgfqpoint{3.331183in}{1.670357in}}%
\pgfpathlineto{\pgfqpoint{3.333758in}{1.666528in}}%
\pgfpathlineto{\pgfqpoint{3.336541in}{1.672367in}}%
\pgfpathlineto{\pgfqpoint{3.339101in}{1.671105in}}%
\pgfpathlineto{\pgfqpoint{3.341893in}{1.671139in}}%
\pgfpathlineto{\pgfqpoint{3.344468in}{1.668599in}}%
\pgfpathlineto{\pgfqpoint{3.347139in}{1.663773in}}%
\pgfpathlineto{\pgfqpoint{3.349828in}{1.665386in}}%
\pgfpathlineto{\pgfqpoint{3.352505in}{1.657796in}}%
\pgfpathlineto{\pgfqpoint{3.355177in}{1.663467in}}%
\pgfpathlineto{\pgfqpoint{3.357862in}{1.663550in}}%
\pgfpathlineto{\pgfqpoint{3.360620in}{1.658506in}}%
\pgfpathlineto{\pgfqpoint{3.363221in}{1.666763in}}%
\pgfpathlineto{\pgfqpoint{3.365996in}{1.664559in}}%
\pgfpathlineto{\pgfqpoint{3.368577in}{1.663948in}}%
\pgfpathlineto{\pgfqpoint{3.371357in}{1.664390in}}%
\pgfpathlineto{\pgfqpoint{3.373921in}{1.659384in}}%
\pgfpathlineto{\pgfqpoint{3.376735in}{1.667540in}}%
\pgfpathlineto{\pgfqpoint{3.379290in}{1.664177in}}%
\pgfpathlineto{\pgfqpoint{3.381959in}{1.663525in}}%
\pgfpathlineto{\pgfqpoint{3.384647in}{1.666128in}}%
\pgfpathlineto{\pgfqpoint{3.387309in}{1.662338in}}%
\pgfpathlineto{\pgfqpoint{3.390102in}{1.666095in}}%
\pgfpathlineto{\pgfqpoint{3.392681in}{1.662776in}}%
\pgfpathlineto{\pgfqpoint{3.395461in}{1.663161in}}%
\pgfpathlineto{\pgfqpoint{3.398037in}{1.660834in}}%
\pgfpathlineto{\pgfqpoint{3.400783in}{1.663853in}}%
\pgfpathlineto{\pgfqpoint{3.403394in}{1.660397in}}%
\pgfpathlineto{\pgfqpoint{3.406202in}{1.666568in}}%
\pgfpathlineto{\pgfqpoint{3.408752in}{1.664137in}}%
\pgfpathlineto{\pgfqpoint{3.411431in}{1.668954in}}%
\pgfpathlineto{\pgfqpoint{3.414109in}{1.664335in}}%
\pgfpathlineto{\pgfqpoint{3.416780in}{1.662626in}}%
\pgfpathlineto{\pgfqpoint{3.419455in}{1.666446in}}%
\pgfpathlineto{\pgfqpoint{3.422142in}{1.666574in}}%
\pgfpathlineto{\pgfqpoint{3.424887in}{1.665071in}}%
\pgfpathlineto{\pgfqpoint{3.427501in}{1.670448in}}%
\pgfpathlineto{\pgfqpoint{3.430313in}{1.664307in}}%
\pgfpathlineto{\pgfqpoint{3.432851in}{1.666350in}}%
\pgfpathlineto{\pgfqpoint{3.435635in}{1.664848in}}%
\pgfpathlineto{\pgfqpoint{3.438210in}{1.668185in}}%
\pgfpathlineto{\pgfqpoint{3.440996in}{1.668107in}}%
\pgfpathlineto{\pgfqpoint{3.443574in}{1.664502in}}%
\pgfpathlineto{\pgfqpoint{3.446257in}{1.670377in}}%
\pgfpathlineto{\pgfqpoint{3.448926in}{1.667412in}}%
\pgfpathlineto{\pgfqpoint{3.451597in}{1.669890in}}%
\pgfpathlineto{\pgfqpoint{3.454285in}{1.666398in}}%
\pgfpathlineto{\pgfqpoint{3.456960in}{1.669035in}}%
\pgfpathlineto{\pgfqpoint{3.459695in}{1.675061in}}%
\pgfpathlineto{\pgfqpoint{3.462321in}{1.673346in}}%
\pgfpathlineto{\pgfqpoint{3.465072in}{1.672391in}}%
\pgfpathlineto{\pgfqpoint{3.467678in}{1.674298in}}%
\pgfpathlineto{\pgfqpoint{3.470466in}{1.669000in}}%
\pgfpathlineto{\pgfqpoint{3.473021in}{1.673127in}}%
\pgfpathlineto{\pgfqpoint{3.475821in}{1.670554in}}%
\pgfpathlineto{\pgfqpoint{3.478378in}{1.681100in}}%
\pgfpathlineto{\pgfqpoint{3.481072in}{1.680298in}}%
\pgfpathlineto{\pgfqpoint{3.483744in}{1.674966in}}%
\pgfpathlineto{\pgfqpoint{3.486442in}{1.680059in}}%
\pgfpathlineto{\pgfqpoint{3.489223in}{1.676075in}}%
\pgfpathlineto{\pgfqpoint{3.491783in}{1.671711in}}%
\pgfpathlineto{\pgfqpoint{3.494581in}{1.668866in}}%
\pgfpathlineto{\pgfqpoint{3.497139in}{1.668476in}}%
\pgfpathlineto{\pgfqpoint{3.499909in}{1.670018in}}%
\pgfpathlineto{\pgfqpoint{3.502488in}{1.664558in}}%
\pgfpathlineto{\pgfqpoint{3.505262in}{1.663624in}}%
\pgfpathlineto{\pgfqpoint{3.507840in}{1.662775in}}%
\pgfpathlineto{\pgfqpoint{3.510533in}{1.663607in}}%
\pgfpathlineto{\pgfqpoint{3.513209in}{1.664282in}}%
\pgfpathlineto{\pgfqpoint{3.515884in}{1.661018in}}%
\pgfpathlineto{\pgfqpoint{3.518565in}{1.665928in}}%
\pgfpathlineto{\pgfqpoint{3.521244in}{1.665707in}}%
\pgfpathlineto{\pgfqpoint{3.524041in}{1.666913in}}%
\pgfpathlineto{\pgfqpoint{3.526601in}{1.666485in}}%
\pgfpathlineto{\pgfqpoint{3.529327in}{1.660736in}}%
\pgfpathlineto{\pgfqpoint{3.531955in}{1.662779in}}%
\pgfpathlineto{\pgfqpoint{3.534783in}{1.662484in}}%
\pgfpathlineto{\pgfqpoint{3.537309in}{1.662978in}}%
\pgfpathlineto{\pgfqpoint{3.540093in}{1.664559in}}%
\pgfpathlineto{\pgfqpoint{3.542656in}{1.664117in}}%
\pgfpathlineto{\pgfqpoint{3.545349in}{1.666675in}}%
\pgfpathlineto{\pgfqpoint{3.548029in}{1.663684in}}%
\pgfpathlineto{\pgfqpoint{3.550713in}{1.665191in}}%
\pgfpathlineto{\pgfqpoint{3.553498in}{1.665889in}}%
\pgfpathlineto{\pgfqpoint{3.556061in}{1.668053in}}%
\pgfpathlineto{\pgfqpoint{3.558853in}{1.664996in}}%
\pgfpathlineto{\pgfqpoint{3.561420in}{1.676154in}}%
\pgfpathlineto{\pgfqpoint{3.564188in}{1.666618in}}%
\pgfpathlineto{\pgfqpoint{3.566774in}{1.666058in}}%
\pgfpathlineto{\pgfqpoint{3.569584in}{1.662709in}}%
\pgfpathlineto{\pgfqpoint{3.572126in}{1.664304in}}%
\pgfpathlineto{\pgfqpoint{3.574814in}{1.664315in}}%
\pgfpathlineto{\pgfqpoint{3.577487in}{1.661030in}}%
\pgfpathlineto{\pgfqpoint{3.580191in}{1.661855in}}%
\pgfpathlineto{\pgfqpoint{3.582851in}{1.667582in}}%
\pgfpathlineto{\pgfqpoint{3.585532in}{1.664492in}}%
\pgfpathlineto{\pgfqpoint{3.588258in}{1.665494in}}%
\pgfpathlineto{\pgfqpoint{3.590883in}{1.664791in}}%
\pgfpathlineto{\pgfqpoint{3.593620in}{1.664046in}}%
\pgfpathlineto{\pgfqpoint{3.596240in}{1.665788in}}%
\pgfpathlineto{\pgfqpoint{3.598998in}{1.666315in}}%
\pgfpathlineto{\pgfqpoint{3.601590in}{1.663636in}}%
\pgfpathlineto{\pgfqpoint{3.604387in}{1.668753in}}%
\pgfpathlineto{\pgfqpoint{3.606951in}{1.666537in}}%
\pgfpathlineto{\pgfqpoint{3.609632in}{1.670855in}}%
\pgfpathlineto{\pgfqpoint{3.612311in}{1.666514in}}%
\pgfpathlineto{\pgfqpoint{3.614982in}{1.664347in}}%
\pgfpathlineto{\pgfqpoint{3.617667in}{1.671432in}}%
\pgfpathlineto{\pgfqpoint{3.620345in}{1.667496in}}%
\pgfpathlineto{\pgfqpoint{3.623165in}{1.670470in}}%
\pgfpathlineto{\pgfqpoint{3.625689in}{1.663660in}}%
\pgfpathlineto{\pgfqpoint{3.628460in}{1.666274in}}%
\pgfpathlineto{\pgfqpoint{3.631058in}{1.669718in}}%
\pgfpathlineto{\pgfqpoint{3.633858in}{1.661013in}}%
\pgfpathlineto{\pgfqpoint{3.636413in}{1.667418in}}%
\pgfpathlineto{\pgfqpoint{3.639207in}{1.670847in}}%
\pgfpathlineto{\pgfqpoint{3.641773in}{1.677034in}}%
\pgfpathlineto{\pgfqpoint{3.644452in}{1.701070in}}%
\pgfpathlineto{\pgfqpoint{3.647130in}{1.727226in}}%
\pgfpathlineto{\pgfqpoint{3.649837in}{1.732901in}}%
\pgfpathlineto{\pgfqpoint{3.652628in}{1.726094in}}%
\pgfpathlineto{\pgfqpoint{3.655165in}{1.712453in}}%
\pgfpathlineto{\pgfqpoint{3.657917in}{1.689386in}}%
\pgfpathlineto{\pgfqpoint{3.660515in}{1.679101in}}%
\pgfpathlineto{\pgfqpoint{3.663276in}{1.726271in}}%
\pgfpathlineto{\pgfqpoint{3.665864in}{1.772607in}}%
\pgfpathlineto{\pgfqpoint{3.668665in}{1.760068in}}%
\pgfpathlineto{\pgfqpoint{3.671232in}{1.757512in}}%
\pgfpathlineto{\pgfqpoint{3.673911in}{1.736314in}}%
\pgfpathlineto{\pgfqpoint{3.676591in}{1.734791in}}%
\pgfpathlineto{\pgfqpoint{3.679273in}{1.711593in}}%
\pgfpathlineto{\pgfqpoint{3.681948in}{1.704864in}}%
\pgfpathlineto{\pgfqpoint{3.684620in}{1.689874in}}%
\pgfpathlineto{\pgfqpoint{3.687442in}{1.671212in}}%
\pgfpathlineto{\pgfqpoint{3.689983in}{1.671090in}}%
\pgfpathlineto{\pgfqpoint{3.692765in}{1.670899in}}%
\pgfpathlineto{\pgfqpoint{3.695331in}{1.664482in}}%
\pgfpathlineto{\pgfqpoint{3.698125in}{1.667194in}}%
\pgfpathlineto{\pgfqpoint{3.700684in}{1.665125in}}%
\pgfpathlineto{\pgfqpoint{3.703460in}{1.667324in}}%
\pgfpathlineto{\pgfqpoint{3.706053in}{1.668949in}}%
\pgfpathlineto{\pgfqpoint{3.708729in}{1.672898in}}%
\pgfpathlineto{\pgfqpoint{3.711410in}{1.666046in}}%
\pgfpathlineto{\pgfqpoint{3.714086in}{1.665353in}}%
\pgfpathlineto{\pgfqpoint{3.716875in}{1.660706in}}%
\pgfpathlineto{\pgfqpoint{3.719446in}{1.657302in}}%
\pgfpathlineto{\pgfqpoint{3.722228in}{1.657302in}}%
\pgfpathlineto{\pgfqpoint{3.724804in}{1.657302in}}%
\pgfpathlineto{\pgfqpoint{3.727581in}{1.660582in}}%
\pgfpathlineto{\pgfqpoint{3.730158in}{1.659528in}}%
\pgfpathlineto{\pgfqpoint{3.732950in}{1.658107in}}%
\pgfpathlineto{\pgfqpoint{3.735509in}{1.661043in}}%
\pgfpathlineto{\pgfqpoint{3.738194in}{1.661159in}}%
\pgfpathlineto{\pgfqpoint{3.740874in}{1.663336in}}%
\pgfpathlineto{\pgfqpoint{3.743548in}{1.662253in}}%
\pgfpathlineto{\pgfqpoint{3.746229in}{1.660906in}}%
\pgfpathlineto{\pgfqpoint{3.748903in}{1.662826in}}%
\pgfpathlineto{\pgfqpoint{3.751728in}{1.664254in}}%
\pgfpathlineto{\pgfqpoint{3.754265in}{1.665729in}}%
\pgfpathlineto{\pgfqpoint{3.757065in}{1.659417in}}%
\pgfpathlineto{\pgfqpoint{3.759622in}{1.658202in}}%
\pgfpathlineto{\pgfqpoint{3.762389in}{1.657985in}}%
\pgfpathlineto{\pgfqpoint{3.764966in}{1.668353in}}%
\pgfpathlineto{\pgfqpoint{3.767782in}{1.674708in}}%
\pgfpathlineto{\pgfqpoint{3.770323in}{1.666829in}}%
\pgfpathlineto{\pgfqpoint{3.773014in}{1.665776in}}%
\pgfpathlineto{\pgfqpoint{3.775691in}{1.657302in}}%
\pgfpathlineto{\pgfqpoint{3.778370in}{1.657917in}}%
\pgfpathlineto{\pgfqpoint{3.781046in}{1.675257in}}%
\pgfpathlineto{\pgfqpoint{3.783725in}{1.675240in}}%
\pgfpathlineto{\pgfqpoint{3.786504in}{1.667989in}}%
\pgfpathlineto{\pgfqpoint{3.789084in}{1.659947in}}%
\pgfpathlineto{\pgfqpoint{3.791897in}{1.658081in}}%
\pgfpathlineto{\pgfqpoint{3.794435in}{1.657302in}}%
\pgfpathlineto{\pgfqpoint{3.797265in}{1.657302in}}%
\pgfpathlineto{\pgfqpoint{3.799797in}{1.657302in}}%
\pgfpathlineto{\pgfqpoint{3.802569in}{1.659202in}}%
\pgfpathlineto{\pgfqpoint{3.805145in}{1.661405in}}%
\pgfpathlineto{\pgfqpoint{3.807832in}{1.661789in}}%
\pgfpathlineto{\pgfqpoint{3.810510in}{1.659896in}}%
\pgfpathlineto{\pgfqpoint{3.813172in}{1.657302in}}%
\pgfpathlineto{\pgfqpoint{3.815983in}{1.663177in}}%
\pgfpathlineto{\pgfqpoint{3.818546in}{1.666894in}}%
\pgfpathlineto{\pgfqpoint{3.821315in}{1.662007in}}%
\pgfpathlineto{\pgfqpoint{3.823903in}{1.665336in}}%
\pgfpathlineto{\pgfqpoint{3.826679in}{1.666487in}}%
\pgfpathlineto{\pgfqpoint{3.829252in}{1.663935in}}%
\pgfpathlineto{\pgfqpoint{3.832053in}{1.664772in}}%
\pgfpathlineto{\pgfqpoint{3.834616in}{1.666081in}}%
\pgfpathlineto{\pgfqpoint{3.837286in}{1.663664in}}%
\pgfpathlineto{\pgfqpoint{3.839960in}{1.669957in}}%
\pgfpathlineto{\pgfqpoint{3.842641in}{1.665660in}}%
\pgfpathlineto{\pgfqpoint{3.845329in}{1.665975in}}%
\pgfpathlineto{\pgfqpoint{3.848005in}{1.667940in}}%
\pgfpathlineto{\pgfqpoint{3.850814in}{1.663978in}}%
\pgfpathlineto{\pgfqpoint{3.853358in}{1.664006in}}%
\pgfpathlineto{\pgfqpoint{3.856100in}{1.671996in}}%
\pgfpathlineto{\pgfqpoint{3.858720in}{1.668209in}}%
\pgfpathlineto{\pgfqpoint{3.861561in}{1.670451in}}%
\pgfpathlineto{\pgfqpoint{3.864073in}{1.672365in}}%
\pgfpathlineto{\pgfqpoint{3.866815in}{1.666672in}}%
\pgfpathlineto{\pgfqpoint{3.869435in}{1.663565in}}%
\pgfpathlineto{\pgfqpoint{3.872114in}{1.662117in}}%
\pgfpathlineto{\pgfqpoint{3.874790in}{1.663972in}}%
\pgfpathlineto{\pgfqpoint{3.877466in}{1.661468in}}%
\pgfpathlineto{\pgfqpoint{3.880237in}{1.663456in}}%
\pgfpathlineto{\pgfqpoint{3.882850in}{1.663496in}}%
\pgfpathlineto{\pgfqpoint{3.885621in}{1.659595in}}%
\pgfpathlineto{\pgfqpoint{3.888188in}{1.674354in}}%
\pgfpathlineto{\pgfqpoint{3.890926in}{1.689180in}}%
\pgfpathlineto{\pgfqpoint{3.893541in}{1.691879in}}%
\pgfpathlineto{\pgfqpoint{3.896345in}{1.680313in}}%
\pgfpathlineto{\pgfqpoint{3.898891in}{1.668487in}}%
\pgfpathlineto{\pgfqpoint{3.901573in}{1.664430in}}%
\pgfpathlineto{\pgfqpoint{3.904252in}{1.658854in}}%
\pgfpathlineto{\pgfqpoint{3.906918in}{1.665172in}}%
\pgfpathlineto{\pgfqpoint{3.909602in}{1.662402in}}%
\pgfpathlineto{\pgfqpoint{3.912296in}{1.658357in}}%
\pgfpathlineto{\pgfqpoint{3.915107in}{1.664314in}}%
\pgfpathlineto{\pgfqpoint{3.917646in}{1.664992in}}%
\pgfpathlineto{\pgfqpoint{3.920412in}{1.662757in}}%
\pgfpathlineto{\pgfqpoint{3.923005in}{1.663336in}}%
\pgfpathlineto{\pgfqpoint{3.925778in}{1.657302in}}%
\pgfpathlineto{\pgfqpoint{3.928347in}{1.667812in}}%
\pgfpathlineto{\pgfqpoint{3.931202in}{1.667884in}}%
\pgfpathlineto{\pgfqpoint{3.933714in}{1.667069in}}%
\pgfpathlineto{\pgfqpoint{3.936395in}{1.671671in}}%
\pgfpathlineto{\pgfqpoint{3.939075in}{1.670836in}}%
\pgfpathlineto{\pgfqpoint{3.941778in}{1.661695in}}%
\pgfpathlineto{\pgfqpoint{3.944431in}{1.667728in}}%
\pgfpathlineto{\pgfqpoint{3.947101in}{1.664977in}}%
\pgfpathlineto{\pgfqpoint{3.949894in}{1.661577in}}%
\pgfpathlineto{\pgfqpoint{3.952464in}{1.669431in}}%
\pgfpathlineto{\pgfqpoint{3.955211in}{1.665652in}}%
\pgfpathlineto{\pgfqpoint{3.957823in}{1.665284in}}%
\pgfpathlineto{\pgfqpoint{3.960635in}{1.661482in}}%
\pgfpathlineto{\pgfqpoint{3.963176in}{1.665696in}}%
\pgfpathlineto{\pgfqpoint{3.966013in}{1.665637in}}%
\pgfpathlineto{\pgfqpoint{3.968523in}{1.661564in}}%
\pgfpathlineto{\pgfqpoint{3.971250in}{1.659115in}}%
\pgfpathlineto{\pgfqpoint{3.973885in}{1.657302in}}%
\pgfpathlineto{\pgfqpoint{3.976563in}{1.657302in}}%
\pgfpathlineto{\pgfqpoint{3.979389in}{1.657302in}}%
\pgfpathlineto{\pgfqpoint{3.981929in}{1.657302in}}%
\pgfpathlineto{\pgfqpoint{3.984714in}{1.657535in}}%
\pgfpathlineto{\pgfqpoint{3.987270in}{1.659000in}}%
\pgfpathlineto{\pgfqpoint{3.990055in}{1.658012in}}%
\pgfpathlineto{\pgfqpoint{3.992642in}{1.661807in}}%
\pgfpathlineto{\pgfqpoint{3.995417in}{1.659852in}}%
\pgfpathlineto{\pgfqpoint{3.997990in}{1.659498in}}%
\pgfpathlineto{\pgfqpoint{4.000674in}{1.658572in}}%
\pgfpathlineto{\pgfqpoint{4.003348in}{1.660557in}}%
\pgfpathlineto{\pgfqpoint{4.006034in}{1.657302in}}%
\pgfpathlineto{\pgfqpoint{4.008699in}{1.657418in}}%
\pgfpathlineto{\pgfqpoint{4.011394in}{1.660351in}}%
\pgfpathlineto{\pgfqpoint{4.014186in}{1.662778in}}%
\pgfpathlineto{\pgfqpoint{4.016744in}{1.660076in}}%
\pgfpathlineto{\pgfqpoint{4.019518in}{1.660108in}}%
\pgfpathlineto{\pgfqpoint{4.022097in}{1.657432in}}%
\pgfpathlineto{\pgfqpoint{4.024868in}{1.662241in}}%
\pgfpathlineto{\pgfqpoint{4.027447in}{1.663064in}}%
\pgfpathlineto{\pgfqpoint{4.030229in}{1.660278in}}%
\pgfpathlineto{\pgfqpoint{4.032817in}{1.664347in}}%
\pgfpathlineto{\pgfqpoint{4.035492in}{1.662578in}}%
\pgfpathlineto{\pgfqpoint{4.038174in}{1.657302in}}%
\pgfpathlineto{\pgfqpoint{4.040852in}{1.659468in}}%
\pgfpathlineto{\pgfqpoint{4.043667in}{1.664206in}}%
\pgfpathlineto{\pgfqpoint{4.046210in}{1.666897in}}%
\pgfpathlineto{\pgfqpoint{4.049006in}{1.660977in}}%
\pgfpathlineto{\pgfqpoint{4.051557in}{1.664901in}}%
\pgfpathlineto{\pgfqpoint{4.054326in}{1.660484in}}%
\pgfpathlineto{\pgfqpoint{4.056911in}{1.661002in}}%
\pgfpathlineto{\pgfqpoint{4.059702in}{1.657418in}}%
\pgfpathlineto{\pgfqpoint{4.062266in}{1.665968in}}%
\pgfpathlineto{\pgfqpoint{4.064957in}{1.668213in}}%
\pgfpathlineto{\pgfqpoint{4.067636in}{1.667297in}}%
\pgfpathlineto{\pgfqpoint{4.070313in}{1.666904in}}%
\pgfpathlineto{\pgfqpoint{4.072985in}{1.668040in}}%
\pgfpathlineto{\pgfqpoint{4.075705in}{1.669280in}}%
\pgfpathlineto{\pgfqpoint{4.078471in}{1.668262in}}%
\pgfpathlineto{\pgfqpoint{4.081018in}{1.669871in}}%
\pgfpathlineto{\pgfqpoint{4.083870in}{1.674357in}}%
\pgfpathlineto{\pgfqpoint{4.086385in}{1.670467in}}%
\pgfpathlineto{\pgfqpoint{4.089159in}{1.671830in}}%
\pgfpathlineto{\pgfqpoint{4.091729in}{1.669815in}}%
\pgfpathlineto{\pgfqpoint{4.094527in}{1.665527in}}%
\pgfpathlineto{\pgfqpoint{4.097092in}{1.665346in}}%
\pgfpathlineto{\pgfqpoint{4.099777in}{1.671403in}}%
\pgfpathlineto{\pgfqpoint{4.102456in}{1.672533in}}%
\pgfpathlineto{\pgfqpoint{4.105185in}{1.674000in}}%
\pgfpathlineto{\pgfqpoint{4.107814in}{1.671359in}}%
\pgfpathlineto{\pgfqpoint{4.110488in}{1.675327in}}%
\pgfpathlineto{\pgfqpoint{4.113252in}{1.669486in}}%
\pgfpathlineto{\pgfqpoint{4.115844in}{1.664954in}}%
\pgfpathlineto{\pgfqpoint{4.118554in}{1.665458in}}%
\pgfpathlineto{\pgfqpoint{4.121205in}{1.664323in}}%
\pgfpathlineto{\pgfqpoint{4.124019in}{1.659954in}}%
\pgfpathlineto{\pgfqpoint{4.126553in}{1.657302in}}%
\pgfpathlineto{\pgfqpoint{4.129349in}{1.659458in}}%
\pgfpathlineto{\pgfqpoint{4.131920in}{1.660383in}}%
\pgfpathlineto{\pgfqpoint{4.134615in}{1.661432in}}%
\pgfpathlineto{\pgfqpoint{4.137272in}{1.657302in}}%
\pgfpathlineto{\pgfqpoint{4.139963in}{1.657302in}}%
\pgfpathlineto{\pgfqpoint{4.142713in}{1.657302in}}%
\pgfpathlineto{\pgfqpoint{4.145310in}{1.657302in}}%
\pgfpathlineto{\pgfqpoint{4.148082in}{1.657521in}}%
\pgfpathlineto{\pgfqpoint{4.150665in}{1.657302in}}%
\pgfpathlineto{\pgfqpoint{4.153423in}{1.660396in}}%
\pgfpathlineto{\pgfqpoint{4.156016in}{1.657302in}}%
\pgfpathlineto{\pgfqpoint{4.158806in}{1.657405in}}%
\pgfpathlineto{\pgfqpoint{4.161380in}{1.658781in}}%
\pgfpathlineto{\pgfqpoint{4.164059in}{1.657789in}}%
\pgfpathlineto{\pgfqpoint{4.166737in}{1.657302in}}%
\pgfpathlineto{\pgfqpoint{4.169415in}{1.657302in}}%
\pgfpathlineto{\pgfqpoint{4.172093in}{1.657302in}}%
\pgfpathlineto{\pgfqpoint{4.174770in}{1.657302in}}%
\pgfpathlineto{\pgfqpoint{4.177593in}{1.657302in}}%
\pgfpathlineto{\pgfqpoint{4.180129in}{1.657302in}}%
\pgfpathlineto{\pgfqpoint{4.182899in}{1.657302in}}%
\pgfpathlineto{\pgfqpoint{4.185481in}{1.657302in}}%
\pgfpathlineto{\pgfqpoint{4.188318in}{1.657302in}}%
\pgfpathlineto{\pgfqpoint{4.190842in}{1.657302in}}%
\pgfpathlineto{\pgfqpoint{4.193638in}{1.660150in}}%
\pgfpathlineto{\pgfqpoint{4.196186in}{1.662426in}}%
\pgfpathlineto{\pgfqpoint{4.198878in}{1.661842in}}%
\pgfpathlineto{\pgfqpoint{4.201542in}{1.662718in}}%
\pgfpathlineto{\pgfqpoint{4.204240in}{1.664482in}}%
\pgfpathlineto{\pgfqpoint{4.207076in}{1.661198in}}%
\pgfpathlineto{\pgfqpoint{4.209597in}{1.663078in}}%
\pgfpathlineto{\pgfqpoint{4.212383in}{1.663051in}}%
\pgfpathlineto{\pgfqpoint{4.214948in}{1.663629in}}%
\pgfpathlineto{\pgfqpoint{4.217694in}{1.664134in}}%
\pgfpathlineto{\pgfqpoint{4.220304in}{1.660935in}}%
\pgfpathlineto{\pgfqpoint{4.223082in}{1.670174in}}%
\pgfpathlineto{\pgfqpoint{4.225654in}{1.663929in}}%
\pgfpathlineto{\pgfqpoint{4.228331in}{1.668313in}}%
\pgfpathlineto{\pgfqpoint{4.231013in}{1.665389in}}%
\pgfpathlineto{\pgfqpoint{4.233691in}{1.666828in}}%
\pgfpathlineto{\pgfqpoint{4.236375in}{1.669530in}}%
\pgfpathlineto{\pgfqpoint{4.239084in}{1.670989in}}%
\pgfpathlineto{\pgfqpoint{4.241900in}{1.669000in}}%
\pgfpathlineto{\pgfqpoint{4.244394in}{1.667140in}}%
\pgfpathlineto{\pgfqpoint{4.247225in}{1.663722in}}%
\pgfpathlineto{\pgfqpoint{4.249767in}{1.666818in}}%
\pgfpathlineto{\pgfqpoint{4.252581in}{1.667947in}}%
\pgfpathlineto{\pgfqpoint{4.255120in}{1.666757in}}%
\pgfpathlineto{\pgfqpoint{4.257958in}{1.665355in}}%
\pgfpathlineto{\pgfqpoint{4.260477in}{1.671308in}}%
\pgfpathlineto{\pgfqpoint{4.263157in}{1.667506in}}%
\pgfpathlineto{\pgfqpoint{4.265824in}{1.671355in}}%
\pgfpathlineto{\pgfqpoint{4.268590in}{1.671655in}}%
\pgfpathlineto{\pgfqpoint{4.271187in}{1.670826in}}%
\pgfpathlineto{\pgfqpoint{4.273874in}{1.666633in}}%
\pgfpathlineto{\pgfqpoint{4.276635in}{1.675093in}}%
\pgfpathlineto{\pgfqpoint{4.279212in}{1.670146in}}%
\pgfpathlineto{\pgfqpoint{4.282000in}{1.671685in}}%
\pgfpathlineto{\pgfqpoint{4.284586in}{1.670233in}}%
\pgfpathlineto{\pgfqpoint{4.287399in}{1.665913in}}%
\pgfpathlineto{\pgfqpoint{4.289936in}{1.669454in}}%
\pgfpathlineto{\pgfqpoint{4.292786in}{1.666732in}}%
\pgfpathlineto{\pgfqpoint{4.295299in}{1.662617in}}%
\pgfpathlineto{\pgfqpoint{4.297977in}{1.671185in}}%
\pgfpathlineto{\pgfqpoint{4.300656in}{1.669356in}}%
\pgfpathlineto{\pgfqpoint{4.303357in}{1.666109in}}%
\pgfpathlineto{\pgfqpoint{4.306118in}{1.669159in}}%
\pgfpathlineto{\pgfqpoint{4.308691in}{1.664713in}}%
\pgfpathlineto{\pgfqpoint{4.311494in}{1.670812in}}%
\pgfpathlineto{\pgfqpoint{4.314032in}{1.674068in}}%
\pgfpathlineto{\pgfqpoint{4.316856in}{1.668813in}}%
\pgfpathlineto{\pgfqpoint{4.319405in}{1.669951in}}%
\pgfpathlineto{\pgfqpoint{4.322181in}{1.671972in}}%
\pgfpathlineto{\pgfqpoint{4.324760in}{1.668145in}}%
\pgfpathlineto{\pgfqpoint{4.327440in}{1.667735in}}%
\pgfpathlineto{\pgfqpoint{4.330118in}{1.668074in}}%
\pgfpathlineto{\pgfqpoint{4.332796in}{1.662241in}}%
\pgfpathlineto{\pgfqpoint{4.335463in}{1.659850in}}%
\pgfpathlineto{\pgfqpoint{4.338154in}{1.659221in}}%
\pgfpathlineto{\pgfqpoint{4.340976in}{1.662709in}}%
\pgfpathlineto{\pgfqpoint{4.343510in}{1.664763in}}%
\pgfpathlineto{\pgfqpoint{4.346263in}{1.664696in}}%
\pgfpathlineto{\pgfqpoint{4.348868in}{1.662933in}}%
\pgfpathlineto{\pgfqpoint{4.351645in}{1.663618in}}%
\pgfpathlineto{\pgfqpoint{4.354224in}{1.662752in}}%
\pgfpathlineto{\pgfqpoint{4.357014in}{1.657929in}}%
\pgfpathlineto{\pgfqpoint{4.359582in}{1.661929in}}%
\pgfpathlineto{\pgfqpoint{4.362270in}{1.667444in}}%
\pgfpathlineto{\pgfqpoint{4.364936in}{1.662802in}}%
\pgfpathlineto{\pgfqpoint{4.367646in}{1.662351in}}%
\pgfpathlineto{\pgfqpoint{4.370437in}{1.665201in}}%
\pgfpathlineto{\pgfqpoint{4.372976in}{1.664535in}}%
\pgfpathlineto{\pgfqpoint{4.375761in}{1.663889in}}%
\pgfpathlineto{\pgfqpoint{4.378329in}{1.664347in}}%
\pgfpathlineto{\pgfqpoint{4.381097in}{1.661779in}}%
\pgfpathlineto{\pgfqpoint{4.383674in}{1.660269in}}%
\pgfpathlineto{\pgfqpoint{4.386431in}{1.665797in}}%
\pgfpathlineto{\pgfqpoint{4.389044in}{1.667194in}}%
\pgfpathlineto{\pgfqpoint{4.391721in}{1.665416in}}%
\pgfpathlineto{\pgfqpoint{4.394400in}{1.657302in}}%
\pgfpathlineto{\pgfqpoint{4.397076in}{1.657630in}}%
\pgfpathlineto{\pgfqpoint{4.399745in}{1.659151in}}%
\pgfpathlineto{\pgfqpoint{4.402468in}{1.658727in}}%
\pgfpathlineto{\pgfqpoint{4.405234in}{1.666142in}}%
\pgfpathlineto{\pgfqpoint{4.407788in}{1.664439in}}%
\pgfpathlineto{\pgfqpoint{4.410587in}{1.664149in}}%
\pgfpathlineto{\pgfqpoint{4.413149in}{1.664531in}}%
\pgfpathlineto{\pgfqpoint{4.415932in}{1.663829in}}%
\pgfpathlineto{\pgfqpoint{4.418506in}{1.669459in}}%
\pgfpathlineto{\pgfqpoint{4.421292in}{1.671655in}}%
\pgfpathlineto{\pgfqpoint{4.423863in}{1.671470in}}%
\pgfpathlineto{\pgfqpoint{4.426534in}{1.671498in}}%
\pgfpathlineto{\pgfqpoint{4.429220in}{1.671697in}}%
\pgfpathlineto{\pgfqpoint{4.431901in}{1.670246in}}%
\pgfpathlineto{\pgfqpoint{4.434569in}{1.670774in}}%
\pgfpathlineto{\pgfqpoint{4.437253in}{1.669838in}}%
\pgfpathlineto{\pgfqpoint{4.440041in}{1.675171in}}%
\pgfpathlineto{\pgfqpoint{4.442611in}{1.685671in}}%
\pgfpathlineto{\pgfqpoint{4.445423in}{1.680540in}}%
\pgfpathlineto{\pgfqpoint{4.447965in}{1.679609in}}%
\pgfpathlineto{\pgfqpoint{4.450767in}{1.690267in}}%
\pgfpathlineto{\pgfqpoint{4.453312in}{1.686307in}}%
\pgfpathlineto{\pgfqpoint{4.456138in}{1.675520in}}%
\pgfpathlineto{\pgfqpoint{4.458681in}{1.678089in}}%
\pgfpathlineto{\pgfqpoint{4.461367in}{1.679045in}}%
\pgfpathlineto{\pgfqpoint{4.464029in}{1.677810in}}%
\pgfpathlineto{\pgfqpoint{4.466717in}{1.677093in}}%
\pgfpathlineto{\pgfqpoint{4.469492in}{1.672931in}}%
\pgfpathlineto{\pgfqpoint{4.472059in}{1.674268in}}%
\pgfpathlineto{\pgfqpoint{4.474861in}{1.671848in}}%
\pgfpathlineto{\pgfqpoint{4.477430in}{1.671858in}}%
\pgfpathlineto{\pgfqpoint{4.480201in}{1.673008in}}%
\pgfpathlineto{\pgfqpoint{4.482778in}{1.670445in}}%
\pgfpathlineto{\pgfqpoint{4.485581in}{1.667409in}}%
\pgfpathlineto{\pgfqpoint{4.488130in}{1.666787in}}%
\pgfpathlineto{\pgfqpoint{4.490822in}{1.668002in}}%
\pgfpathlineto{\pgfqpoint{4.493492in}{1.667166in}}%
\pgfpathlineto{\pgfqpoint{4.496167in}{1.667967in}}%
\pgfpathlineto{\pgfqpoint{4.498850in}{1.667838in}}%
\pgfpathlineto{\pgfqpoint{4.501529in}{1.672136in}}%
\pgfpathlineto{\pgfqpoint{4.504305in}{1.667979in}}%
\pgfpathlineto{\pgfqpoint{4.506893in}{1.663156in}}%
\pgfpathlineto{\pgfqpoint{4.509643in}{1.664936in}}%
\pgfpathlineto{\pgfqpoint{4.512246in}{1.665279in}}%
\pgfpathlineto{\pgfqpoint{4.515080in}{1.667752in}}%
\pgfpathlineto{\pgfqpoint{4.517598in}{1.663948in}}%
\pgfpathlineto{\pgfqpoint{4.520345in}{1.665410in}}%
\pgfpathlineto{\pgfqpoint{4.522962in}{1.667859in}}%
\pgfpathlineto{\pgfqpoint{4.525640in}{1.670070in}}%
\pgfpathlineto{\pgfqpoint{4.528307in}{1.667375in}}%
\pgfpathlineto{\pgfqpoint{4.530990in}{1.670204in}}%
\pgfpathlineto{\pgfqpoint{4.533764in}{1.666596in}}%
\pgfpathlineto{\pgfqpoint{4.536400in}{1.666646in}}%
\pgfpathlineto{\pgfqpoint{4.539144in}{1.666820in}}%
\pgfpathlineto{\pgfqpoint{4.541711in}{1.670662in}}%
\pgfpathlineto{\pgfqpoint{4.544464in}{1.667135in}}%
\pgfpathlineto{\pgfqpoint{4.547064in}{1.669519in}}%
\pgfpathlineto{\pgfqpoint{4.549822in}{1.663183in}}%
\pgfpathlineto{\pgfqpoint{4.552425in}{1.671349in}}%
\pgfpathlineto{\pgfqpoint{4.555106in}{1.668942in}}%
\pgfpathlineto{\pgfqpoint{4.557777in}{1.674175in}}%
\pgfpathlineto{\pgfqpoint{4.560448in}{1.670804in}}%
\pgfpathlineto{\pgfqpoint{4.563125in}{1.670271in}}%
\pgfpathlineto{\pgfqpoint{4.565820in}{1.666877in}}%
\pgfpathlineto{\pgfqpoint{4.568612in}{1.667941in}}%
\pgfpathlineto{\pgfqpoint{4.571171in}{1.669375in}}%
\pgfpathlineto{\pgfqpoint{4.573947in}{1.665014in}}%
\pgfpathlineto{\pgfqpoint{4.576531in}{1.667407in}}%
\pgfpathlineto{\pgfqpoint{4.579305in}{1.665124in}}%
\pgfpathlineto{\pgfqpoint{4.581888in}{1.669876in}}%
\pgfpathlineto{\pgfqpoint{4.584672in}{1.665263in}}%
\pgfpathlineto{\pgfqpoint{4.587244in}{1.669224in}}%
\pgfpathlineto{\pgfqpoint{4.589920in}{1.665197in}}%
\pgfpathlineto{\pgfqpoint{4.592589in}{1.657532in}}%
\pgfpathlineto{\pgfqpoint{4.595281in}{1.660370in}}%
\pgfpathlineto{\pgfqpoint{4.597951in}{1.665297in}}%
\pgfpathlineto{\pgfqpoint{4.600633in}{1.661589in}}%
\pgfpathlineto{\pgfqpoint{4.603430in}{1.664064in}}%
\pgfpathlineto{\pgfqpoint{4.605990in}{1.668420in}}%
\pgfpathlineto{\pgfqpoint{4.608808in}{1.664182in}}%
\pgfpathlineto{\pgfqpoint{4.611350in}{1.664971in}}%
\pgfpathlineto{\pgfqpoint{4.614134in}{1.669659in}}%
\pgfpathlineto{\pgfqpoint{4.616702in}{1.664934in}}%
\pgfpathlineto{\pgfqpoint{4.619529in}{1.668906in}}%
\pgfpathlineto{\pgfqpoint{4.622056in}{1.669331in}}%
\pgfpathlineto{\pgfqpoint{4.624741in}{1.665325in}}%
\pgfpathlineto{\pgfqpoint{4.627411in}{1.665666in}}%
\pgfpathlineto{\pgfqpoint{4.630096in}{1.664106in}}%
\pgfpathlineto{\pgfqpoint{4.632902in}{1.662881in}}%
\pgfpathlineto{\pgfqpoint{4.635445in}{1.665660in}}%
\pgfpathlineto{\pgfqpoint{4.638204in}{1.667062in}}%
\pgfpathlineto{\pgfqpoint{4.640809in}{1.670584in}}%
\pgfpathlineto{\pgfqpoint{4.643628in}{1.671590in}}%
\pgfpathlineto{\pgfqpoint{4.646169in}{1.669495in}}%
\pgfpathlineto{\pgfqpoint{4.648922in}{1.670949in}}%
\pgfpathlineto{\pgfqpoint{4.651524in}{1.669377in}}%
\pgfpathlineto{\pgfqpoint{4.654203in}{1.670410in}}%
\pgfpathlineto{\pgfqpoint{4.656873in}{1.666548in}}%
\pgfpathlineto{\pgfqpoint{4.659590in}{1.665677in}}%
\pgfpathlineto{\pgfqpoint{4.662237in}{1.664914in}}%
\pgfpathlineto{\pgfqpoint{4.664923in}{1.670828in}}%
\pgfpathlineto{\pgfqpoint{4.667764in}{1.666844in}}%
\pgfpathlineto{\pgfqpoint{4.670261in}{1.669444in}}%
\pgfpathlineto{\pgfqpoint{4.673068in}{1.666832in}}%
\pgfpathlineto{\pgfqpoint{4.675619in}{1.664954in}}%
\pgfpathlineto{\pgfqpoint{4.678448in}{1.665857in}}%
\pgfpathlineto{\pgfqpoint{4.680988in}{1.660254in}}%
\pgfpathlineto{\pgfqpoint{4.683799in}{1.665146in}}%
\pgfpathlineto{\pgfqpoint{4.686337in}{1.662128in}}%
\pgfpathlineto{\pgfqpoint{4.689051in}{1.658685in}}%
\pgfpathlineto{\pgfqpoint{4.691694in}{1.664369in}}%
\pgfpathlineto{\pgfqpoint{4.694381in}{1.665093in}}%
\pgfpathlineto{\pgfqpoint{4.697170in}{1.666839in}}%
\pgfpathlineto{\pgfqpoint{4.699734in}{1.663465in}}%
\pgfpathlineto{\pgfqpoint{4.702517in}{1.667114in}}%
\pgfpathlineto{\pgfqpoint{4.705094in}{1.667523in}}%
\pgfpathlineto{\pgfqpoint{4.707824in}{1.663780in}}%
\pgfpathlineto{\pgfqpoint{4.710437in}{1.660127in}}%
\pgfpathlineto{\pgfqpoint{4.713275in}{1.657302in}}%
\pgfpathlineto{\pgfqpoint{4.715806in}{1.662589in}}%
\pgfpathlineto{\pgfqpoint{4.718486in}{1.661713in}}%
\pgfpathlineto{\pgfqpoint{4.721160in}{1.663230in}}%
\pgfpathlineto{\pgfqpoint{4.723873in}{1.666355in}}%
\pgfpathlineto{\pgfqpoint{4.726508in}{1.666063in}}%
\pgfpathlineto{\pgfqpoint{4.729233in}{1.666778in}}%
\pgfpathlineto{\pgfqpoint{4.731901in}{1.669980in}}%
\pgfpathlineto{\pgfqpoint{4.734552in}{1.668622in}}%
\pgfpathlineto{\pgfqpoint{4.737348in}{1.671928in}}%
\pgfpathlineto{\pgfqpoint{4.739912in}{1.665449in}}%
\pgfpathlineto{\pgfqpoint{4.742696in}{1.669938in}}%
\pgfpathlineto{\pgfqpoint{4.745256in}{1.671028in}}%
\pgfpathlineto{\pgfqpoint{4.748081in}{1.671952in}}%
\pgfpathlineto{\pgfqpoint{4.750627in}{1.671550in}}%
\pgfpathlineto{\pgfqpoint{4.753298in}{1.673411in}}%
\pgfpathlineto{\pgfqpoint{4.755983in}{1.668575in}}%
\pgfpathlineto{\pgfqpoint{4.758653in}{1.675522in}}%
\pgfpathlineto{\pgfqpoint{4.761337in}{1.681752in}}%
\pgfpathlineto{\pgfqpoint{4.764018in}{1.669734in}}%
\pgfpathlineto{\pgfqpoint{4.766783in}{1.678657in}}%
\pgfpathlineto{\pgfqpoint{4.769367in}{1.672860in}}%
\pgfpathlineto{\pgfqpoint{4.772198in}{1.673202in}}%
\pgfpathlineto{\pgfqpoint{4.774732in}{1.672502in}}%
\pgfpathlineto{\pgfqpoint{4.777535in}{1.669650in}}%
\pgfpathlineto{\pgfqpoint{4.780083in}{1.674898in}}%
\pgfpathlineto{\pgfqpoint{4.782872in}{1.670666in}}%
\pgfpathlineto{\pgfqpoint{4.785445in}{1.675974in}}%
\pgfpathlineto{\pgfqpoint{4.788116in}{1.673631in}}%
\pgfpathlineto{\pgfqpoint{4.790798in}{1.670216in}}%
\pgfpathlineto{\pgfqpoint{4.793512in}{1.674344in}}%
\pgfpathlineto{\pgfqpoint{4.796274in}{1.672457in}}%
\pgfpathlineto{\pgfqpoint{4.798830in}{1.661815in}}%
\pgfpathlineto{\pgfqpoint{4.801586in}{1.666044in}}%
\pgfpathlineto{\pgfqpoint{4.804193in}{1.668556in}}%
\pgfpathlineto{\pgfqpoint{4.807017in}{1.669052in}}%
\pgfpathlineto{\pgfqpoint{4.809538in}{1.671368in}}%
\pgfpathlineto{\pgfqpoint{4.812377in}{1.677569in}}%
\pgfpathlineto{\pgfqpoint{4.814907in}{1.671905in}}%
\pgfpathlineto{\pgfqpoint{4.817587in}{1.677505in}}%
\pgfpathlineto{\pgfqpoint{4.820265in}{1.674385in}}%
\pgfpathlineto{\pgfqpoint{4.822945in}{1.671917in}}%
\pgfpathlineto{\pgfqpoint{4.825619in}{1.668445in}}%
\pgfpathlineto{\pgfqpoint{4.828291in}{1.668416in}}%
\pgfpathlineto{\pgfqpoint{4.831045in}{1.670823in}}%
\pgfpathlineto{\pgfqpoint{4.833657in}{1.669760in}}%
\pgfpathlineto{\pgfqpoint{4.837992in}{1.664287in}}%
\pgfpathlineto{\pgfqpoint{4.839922in}{1.668835in}}%
\pgfpathlineto{\pgfqpoint{4.842380in}{1.676155in}}%
\pgfpathlineto{\pgfqpoint{4.844361in}{1.667734in}}%
\pgfpathlineto{\pgfqpoint{4.847127in}{1.676097in}}%
\pgfpathlineto{\pgfqpoint{4.849715in}{1.673247in}}%
\pgfpathlineto{\pgfqpoint{4.852404in}{1.675705in}}%
\pgfpathlineto{\pgfqpoint{4.855070in}{1.669717in}}%
\pgfpathlineto{\pgfqpoint{4.857807in}{1.670800in}}%
\pgfpathlineto{\pgfqpoint{4.860544in}{1.663702in}}%
\pgfpathlineto{\pgfqpoint{4.863116in}{1.665017in}}%
\pgfpathlineto{\pgfqpoint{4.865910in}{1.661481in}}%
\pgfpathlineto{\pgfqpoint{4.868474in}{1.663534in}}%
\pgfpathlineto{\pgfqpoint{4.871209in}{1.667938in}}%
\pgfpathlineto{\pgfqpoint{4.873832in}{1.670260in}}%
\pgfpathlineto{\pgfqpoint{4.876636in}{1.664158in}}%
\pgfpathlineto{\pgfqpoint{4.879180in}{1.667917in}}%
\pgfpathlineto{\pgfqpoint{4.881864in}{1.664947in}}%
\pgfpathlineto{\pgfqpoint{4.884540in}{1.673382in}}%
\pgfpathlineto{\pgfqpoint{4.887211in}{1.676000in}}%
\pgfpathlineto{\pgfqpoint{4.889902in}{1.678488in}}%
\pgfpathlineto{\pgfqpoint{4.892611in}{1.677152in}}%
\pgfpathlineto{\pgfqpoint{4.895399in}{1.692143in}}%
\pgfpathlineto{\pgfqpoint{4.897938in}{1.711998in}}%
\pgfpathlineto{\pgfqpoint{4.900712in}{1.716902in}}%
\pgfpathlineto{\pgfqpoint{4.903295in}{1.694649in}}%
\pgfpathlineto{\pgfqpoint{4.906096in}{1.683752in}}%
\pgfpathlineto{\pgfqpoint{4.908648in}{1.677925in}}%
\pgfpathlineto{\pgfqpoint{4.911435in}{1.671793in}}%
\pgfpathlineto{\pgfqpoint{4.914009in}{1.679357in}}%
\pgfpathlineto{\pgfqpoint{4.916681in}{1.673292in}}%
\pgfpathlineto{\pgfqpoint{4.919352in}{1.671804in}}%
\pgfpathlineto{\pgfqpoint{4.922041in}{1.667812in}}%
\pgfpathlineto{\pgfqpoint{4.924708in}{1.661594in}}%
\pgfpathlineto{\pgfqpoint{4.927400in}{1.670933in}}%
\pgfpathlineto{\pgfqpoint{4.930170in}{1.672737in}}%
\pgfpathlineto{\pgfqpoint{4.932742in}{1.673746in}}%
\pgfpathlineto{\pgfqpoint{4.935515in}{1.665495in}}%
\pgfpathlineto{\pgfqpoint{4.938112in}{1.665828in}}%
\pgfpathlineto{\pgfqpoint{4.940881in}{1.669425in}}%
\pgfpathlineto{\pgfqpoint{4.943466in}{1.688416in}}%
\pgfpathlineto{\pgfqpoint{4.946151in}{1.706696in}}%
\pgfpathlineto{\pgfqpoint{4.948827in}{1.785052in}}%
\pgfpathlineto{\pgfqpoint{4.951504in}{1.834576in}}%
\pgfpathlineto{\pgfqpoint{4.954182in}{1.834882in}}%
\pgfpathlineto{\pgfqpoint{4.956862in}{1.841978in}}%
\pgfpathlineto{\pgfqpoint{4.959689in}{1.857161in}}%
\pgfpathlineto{\pgfqpoint{4.962219in}{1.834340in}}%
\pgfpathlineto{\pgfqpoint{4.965002in}{1.797254in}}%
\pgfpathlineto{\pgfqpoint{4.967575in}{1.740930in}}%
\pgfpathlineto{\pgfqpoint{4.970314in}{1.710280in}}%
\pgfpathlineto{\pgfqpoint{4.972933in}{1.707033in}}%
\pgfpathlineto{\pgfqpoint{4.975703in}{1.679116in}}%
\pgfpathlineto{\pgfqpoint{4.978287in}{1.681073in}}%
\pgfpathlineto{\pgfqpoint{4.980967in}{1.681001in}}%
\pgfpathlineto{\pgfqpoint{4.983637in}{1.679303in}}%
\pgfpathlineto{\pgfqpoint{4.986325in}{1.669295in}}%
\pgfpathlineto{\pgfqpoint{4.989001in}{1.667455in}}%
\pgfpathlineto{\pgfqpoint{4.991683in}{1.685690in}}%
\pgfpathlineto{\pgfqpoint{4.994390in}{1.674225in}}%
\pgfpathlineto{\pgfqpoint{4.997028in}{1.672200in}}%
\pgfpathlineto{\pgfqpoint{4.999780in}{1.670398in}}%
\pgfpathlineto{\pgfqpoint{5.002384in}{1.681831in}}%
\pgfpathlineto{\pgfqpoint{5.005178in}{1.668303in}}%
\pgfpathlineto{\pgfqpoint{5.007751in}{1.669464in}}%
\pgfpathlineto{\pgfqpoint{5.010562in}{1.668928in}}%
\pgfpathlineto{\pgfqpoint{5.013104in}{1.674138in}}%
\pgfpathlineto{\pgfqpoint{5.015820in}{1.680517in}}%
\pgfpathlineto{\pgfqpoint{5.018466in}{1.676377in}}%
\pgfpathlineto{\pgfqpoint{5.021147in}{1.665763in}}%
\pgfpathlineto{\pgfqpoint{5.023927in}{1.661085in}}%
\pgfpathlineto{\pgfqpoint{5.026501in}{1.666604in}}%
\pgfpathlineto{\pgfqpoint{5.029275in}{1.664881in}}%
\pgfpathlineto{\pgfqpoint{5.031849in}{1.667591in}}%
\pgfpathlineto{\pgfqpoint{5.034649in}{1.688286in}}%
\pgfpathlineto{\pgfqpoint{5.037214in}{1.678641in}}%
\pgfpathlineto{\pgfqpoint{5.039962in}{1.672389in}}%
\pgfpathlineto{\pgfqpoint{5.042572in}{1.665067in}}%
\pgfpathlineto{\pgfqpoint{5.045249in}{1.664765in}}%
\pgfpathlineto{\pgfqpoint{5.047924in}{1.661140in}}%
\pgfpathlineto{\pgfqpoint{5.050606in}{1.668935in}}%
\pgfpathlineto{\pgfqpoint{5.053284in}{1.673619in}}%
\pgfpathlineto{\pgfqpoint{5.055952in}{1.669425in}}%
\pgfpathlineto{\pgfqpoint{5.058711in}{1.671611in}}%
\pgfpathlineto{\pgfqpoint{5.061315in}{1.664089in}}%
\pgfpathlineto{\pgfqpoint{5.064144in}{1.670811in}}%
\pgfpathlineto{\pgfqpoint{5.066677in}{1.673667in}}%
\pgfpathlineto{\pgfqpoint{5.069463in}{1.667900in}}%
\pgfpathlineto{\pgfqpoint{5.072030in}{1.665471in}}%
\pgfpathlineto{\pgfqpoint{5.074851in}{1.664787in}}%
\pgfpathlineto{\pgfqpoint{5.077390in}{1.657302in}}%
\pgfpathlineto{\pgfqpoint{5.080067in}{1.658184in}}%
\pgfpathlineto{\pgfqpoint{5.082746in}{1.661433in}}%
\pgfpathlineto{\pgfqpoint{5.085426in}{1.663941in}}%
\pgfpathlineto{\pgfqpoint{5.088103in}{1.662549in}}%
\pgfpathlineto{\pgfqpoint{5.090788in}{1.660344in}}%
\pgfpathlineto{\pgfqpoint{5.093579in}{1.668258in}}%
\pgfpathlineto{\pgfqpoint{5.096142in}{1.669686in}}%
\pgfpathlineto{\pgfqpoint{5.098948in}{1.665939in}}%
\pgfpathlineto{\pgfqpoint{5.101496in}{1.666058in}}%
\pgfpathlineto{\pgfqpoint{5.104312in}{1.666398in}}%
\pgfpathlineto{\pgfqpoint{5.106842in}{1.668155in}}%
\pgfpathlineto{\pgfqpoint{5.109530in}{1.661248in}}%
\pgfpathlineto{\pgfqpoint{5.112209in}{1.662448in}}%
\pgfpathlineto{\pgfqpoint{5.114887in}{1.661265in}}%
\pgfpathlineto{\pgfqpoint{5.117550in}{1.665957in}}%
\pgfpathlineto{\pgfqpoint{5.120243in}{1.660458in}}%
\pgfpathlineto{\pgfqpoint{5.123042in}{1.664388in}}%
\pgfpathlineto{\pgfqpoint{5.125599in}{1.659461in}}%
\pgfpathlineto{\pgfqpoint{5.128421in}{1.663370in}}%
\pgfpathlineto{\pgfqpoint{5.130953in}{1.662960in}}%
\pgfpathlineto{\pgfqpoint{5.133716in}{1.664116in}}%
\pgfpathlineto{\pgfqpoint{5.136311in}{1.667100in}}%
\pgfpathlineto{\pgfqpoint{5.139072in}{1.668707in}}%
\pgfpathlineto{\pgfqpoint{5.141660in}{1.669285in}}%
\pgfpathlineto{\pgfqpoint{5.144349in}{1.664433in}}%
\pgfpathlineto{\pgfqpoint{5.147029in}{1.666698in}}%
\pgfpathlineto{\pgfqpoint{5.149734in}{1.669483in}}%
\pgfpathlineto{\pgfqpoint{5.152382in}{1.657347in}}%
\pgfpathlineto{\pgfqpoint{5.155059in}{1.657302in}}%
\pgfpathlineto{\pgfqpoint{5.157815in}{1.661586in}}%
\pgfpathlineto{\pgfqpoint{5.160420in}{1.660951in}}%
\pgfpathlineto{\pgfqpoint{5.163243in}{1.666677in}}%
\pgfpathlineto{\pgfqpoint{5.165775in}{1.668024in}}%
\pgfpathlineto{\pgfqpoint{5.168591in}{1.661961in}}%
\pgfpathlineto{\pgfqpoint{5.171133in}{1.665854in}}%
\pgfpathlineto{\pgfqpoint{5.173925in}{1.668759in}}%
\pgfpathlineto{\pgfqpoint{5.176477in}{1.672610in}}%
\pgfpathlineto{\pgfqpoint{5.179188in}{1.664980in}}%
\pgfpathlineto{\pgfqpoint{5.181848in}{1.668770in}}%
\pgfpathlineto{\pgfqpoint{5.184522in}{1.660996in}}%
\pgfpathlineto{\pgfqpoint{5.187294in}{1.667087in}}%
\pgfpathlineto{\pgfqpoint{5.189880in}{1.665822in}}%
\pgfpathlineto{\pgfqpoint{5.192680in}{1.668742in}}%
\pgfpathlineto{\pgfqpoint{5.195239in}{1.673278in}}%
\pgfpathlineto{\pgfqpoint{5.198008in}{1.669946in}}%
\pgfpathlineto{\pgfqpoint{5.200594in}{1.673868in}}%
\pgfpathlineto{\pgfqpoint{5.203388in}{1.659191in}}%
\pgfpathlineto{\pgfqpoint{5.205952in}{1.657867in}}%
\pgfpathlineto{\pgfqpoint{5.208630in}{1.657302in}}%
\pgfpathlineto{\pgfqpoint{5.211299in}{1.657442in}}%
\pgfpathlineto{\pgfqpoint{5.214027in}{1.668965in}}%
\pgfpathlineto{\pgfqpoint{5.216667in}{1.664890in}}%
\pgfpathlineto{\pgfqpoint{5.219345in}{1.661711in}}%
\pgfpathlineto{\pgfqpoint{5.222151in}{1.665280in}}%
\pgfpathlineto{\pgfqpoint{5.224695in}{1.664138in}}%
\pgfpathlineto{\pgfqpoint{5.227470in}{1.660578in}}%
\pgfpathlineto{\pgfqpoint{5.230059in}{1.698202in}}%
\pgfpathlineto{\pgfqpoint{5.232855in}{1.678438in}}%
\pgfpathlineto{\pgfqpoint{5.235409in}{1.663335in}}%
\pgfpathlineto{\pgfqpoint{5.238173in}{1.670287in}}%
\pgfpathlineto{\pgfqpoint{5.240777in}{1.669453in}}%
\pgfpathlineto{\pgfqpoint{5.243445in}{1.665717in}}%
\pgfpathlineto{\pgfqpoint{5.246130in}{1.665336in}}%
\pgfpathlineto{\pgfqpoint{5.248816in}{1.671335in}}%
\pgfpathlineto{\pgfqpoint{5.251590in}{1.665827in}}%
\pgfpathlineto{\pgfqpoint{5.254236in}{1.663408in}}%
\pgfpathlineto{\pgfqpoint{5.256973in}{1.666386in}}%
\pgfpathlineto{\pgfqpoint{5.259511in}{1.665981in}}%
\pgfpathlineto{\pgfqpoint{5.262264in}{1.660385in}}%
\pgfpathlineto{\pgfqpoint{5.264876in}{1.668426in}}%
\pgfpathlineto{\pgfqpoint{5.267691in}{1.663798in}}%
\pgfpathlineto{\pgfqpoint{5.270238in}{1.662880in}}%
\pgfpathlineto{\pgfqpoint{5.272913in}{1.662257in}}%
\pgfpathlineto{\pgfqpoint{5.275589in}{1.660796in}}%
\pgfpathlineto{\pgfqpoint{5.278322in}{1.663413in}}%
\pgfpathlineto{\pgfqpoint{5.280947in}{1.661835in}}%
\pgfpathlineto{\pgfqpoint{5.283631in}{1.665258in}}%
\pgfpathlineto{\pgfqpoint{5.286436in}{1.665450in}}%
\pgfpathlineto{\pgfqpoint{5.288984in}{1.672861in}}%
\pgfpathlineto{\pgfqpoint{5.291794in}{1.663123in}}%
\pgfpathlineto{\pgfqpoint{5.294339in}{1.670675in}}%
\pgfpathlineto{\pgfqpoint{5.297140in}{1.666303in}}%
\pgfpathlineto{\pgfqpoint{5.299696in}{1.670550in}}%
\pgfpathlineto{\pgfqpoint{5.302443in}{1.671383in}}%
\pgfpathlineto{\pgfqpoint{5.305054in}{1.672838in}}%
\pgfpathlineto{\pgfqpoint{5.307731in}{1.674260in}}%
\pgfpathlineto{\pgfqpoint{5.310411in}{1.669929in}}%
\pgfpathlineto{\pgfqpoint{5.313089in}{1.672986in}}%
\pgfpathlineto{\pgfqpoint{5.315754in}{1.672161in}}%
\pgfpathlineto{\pgfqpoint{5.318430in}{1.673759in}}%
\pgfpathlineto{\pgfqpoint{5.321256in}{1.663938in}}%
\pgfpathlineto{\pgfqpoint{5.323802in}{1.667365in}}%
\pgfpathlineto{\pgfqpoint{5.326564in}{1.668335in}}%
\pgfpathlineto{\pgfqpoint{5.329159in}{1.663761in}}%
\pgfpathlineto{\pgfqpoint{5.331973in}{1.673844in}}%
\pgfpathlineto{\pgfqpoint{5.334510in}{1.688580in}}%
\pgfpathlineto{\pgfqpoint{5.337353in}{1.700137in}}%
\pgfpathlineto{\pgfqpoint{5.339872in}{1.707362in}}%
\pgfpathlineto{\pgfqpoint{5.342549in}{1.731893in}}%
\pgfpathlineto{\pgfqpoint{5.345224in}{1.747293in}}%
\pgfpathlineto{\pgfqpoint{5.347905in}{1.715081in}}%
\pgfpathlineto{\pgfqpoint{5.350723in}{1.703293in}}%
\pgfpathlineto{\pgfqpoint{5.353262in}{1.694423in}}%
\pgfpathlineto{\pgfqpoint{5.356056in}{1.697157in}}%
\pgfpathlineto{\pgfqpoint{5.358612in}{1.688633in}}%
\pgfpathlineto{\pgfqpoint{5.361370in}{1.684872in}}%
\pgfpathlineto{\pgfqpoint{5.363966in}{1.693816in}}%
\pgfpathlineto{\pgfqpoint{5.366727in}{1.696265in}}%
\pgfpathlineto{\pgfqpoint{5.369335in}{1.695820in}}%
\pgfpathlineto{\pgfqpoint{5.372013in}{1.684456in}}%
\pgfpathlineto{\pgfqpoint{5.374692in}{1.689596in}}%
\pgfpathlineto{\pgfqpoint{5.377370in}{1.681321in}}%
\pgfpathlineto{\pgfqpoint{5.380048in}{1.674773in}}%
\pgfpathlineto{\pgfqpoint{5.382725in}{1.673075in}}%
\pgfpathlineto{\pgfqpoint{5.385550in}{1.672838in}}%
\pgfpathlineto{\pgfqpoint{5.388083in}{1.671508in}}%
\pgfpathlineto{\pgfqpoint{5.390900in}{1.677826in}}%
\pgfpathlineto{\pgfqpoint{5.393441in}{1.676620in}}%
\pgfpathlineto{\pgfqpoint{5.396219in}{1.680701in}}%
\pgfpathlineto{\pgfqpoint{5.398784in}{1.797681in}}%
\pgfpathlineto{\pgfqpoint{5.401576in}{1.788667in}}%
\pgfpathlineto{\pgfqpoint{5.404154in}{1.755929in}}%
\pgfpathlineto{\pgfqpoint{5.406832in}{1.726211in}}%
\pgfpathlineto{\pgfqpoint{5.409507in}{1.719810in}}%
\pgfpathlineto{\pgfqpoint{5.412190in}{1.699931in}}%
\pgfpathlineto{\pgfqpoint{5.414954in}{1.689192in}}%
\pgfpathlineto{\pgfqpoint{5.417547in}{1.677733in}}%
\pgfpathlineto{\pgfqpoint{5.420304in}{1.681265in}}%
\pgfpathlineto{\pgfqpoint{5.422897in}{1.677487in}}%
\pgfpathlineto{\pgfqpoint{5.425661in}{1.674118in}}%
\pgfpathlineto{\pgfqpoint{5.428259in}{1.671111in}}%
\pgfpathlineto{\pgfqpoint{5.431015in}{1.665077in}}%
\pgfpathlineto{\pgfqpoint{5.433616in}{1.672906in}}%
\pgfpathlineto{\pgfqpoint{5.436295in}{1.672239in}}%
\pgfpathlineto{\pgfqpoint{5.438974in}{1.666362in}}%
\pgfpathlineto{\pgfqpoint{5.441698in}{1.669443in}}%
\pgfpathlineto{\pgfqpoint{5.444328in}{1.665318in}}%
\pgfpathlineto{\pgfqpoint{5.447021in}{1.661951in}}%
\pgfpathlineto{\pgfqpoint{5.449769in}{1.666891in}}%
\pgfpathlineto{\pgfqpoint{5.452365in}{1.670904in}}%
\pgfpathlineto{\pgfqpoint{5.455168in}{1.670989in}}%
\pgfpathlineto{\pgfqpoint{5.457721in}{1.669898in}}%
\pgfpathlineto{\pgfqpoint{5.460489in}{1.673294in}}%
\pgfpathlineto{\pgfqpoint{5.463079in}{1.669729in}}%
\pgfpathlineto{\pgfqpoint{5.465888in}{1.668870in}}%
\pgfpathlineto{\pgfqpoint{5.468425in}{1.662658in}}%
\pgfpathlineto{\pgfqpoint{5.471113in}{1.662669in}}%
\pgfpathlineto{\pgfqpoint{5.473792in}{1.666301in}}%
\pgfpathlineto{\pgfqpoint{5.476458in}{1.664732in}}%
\pgfpathlineto{\pgfqpoint{5.479152in}{1.662516in}}%
\pgfpathlineto{\pgfqpoint{5.481825in}{1.665871in}}%
\pgfpathlineto{\pgfqpoint{5.484641in}{1.670596in}}%
\pgfpathlineto{\pgfqpoint{5.487176in}{1.666318in}}%
\pgfpathlineto{\pgfqpoint{5.490000in}{1.671140in}}%
\pgfpathlineto{\pgfqpoint{5.492541in}{1.667993in}}%
\pgfpathlineto{\pgfqpoint{5.495346in}{1.670658in}}%
\pgfpathlineto{\pgfqpoint{5.497898in}{1.671786in}}%
\pgfpathlineto{\pgfqpoint{5.500687in}{1.663949in}}%
\pgfpathlineto{\pgfqpoint{5.503255in}{1.668297in}}%
\pgfpathlineto{\pgfqpoint{5.505933in}{1.669519in}}%
\pgfpathlineto{\pgfqpoint{5.508612in}{1.664189in}}%
\pgfpathlineto{\pgfqpoint{5.511290in}{1.664362in}}%
\pgfpathlineto{\pgfqpoint{5.514080in}{1.662339in}}%
\pgfpathlineto{\pgfqpoint{5.516646in}{1.664284in}}%
\pgfpathlineto{\pgfqpoint{5.519433in}{1.666408in}}%
\pgfpathlineto{\pgfqpoint{5.522003in}{1.661712in}}%
\pgfpathlineto{\pgfqpoint{5.524756in}{1.663763in}}%
\pgfpathlineto{\pgfqpoint{5.527360in}{1.666921in}}%
\pgfpathlineto{\pgfqpoint{5.530148in}{1.665058in}}%
\pgfpathlineto{\pgfqpoint{5.532717in}{1.664971in}}%
\pgfpathlineto{\pgfqpoint{5.535395in}{1.663565in}}%
\pgfpathlineto{\pgfqpoint{5.538074in}{1.663148in}}%
\pgfpathlineto{\pgfqpoint{5.540750in}{1.665450in}}%
\pgfpathlineto{\pgfqpoint{5.543421in}{1.664184in}}%
\pgfpathlineto{\pgfqpoint{5.546110in}{1.666616in}}%
\pgfpathlineto{\pgfqpoint{5.548921in}{1.660322in}}%
\pgfpathlineto{\pgfqpoint{5.551457in}{1.664686in}}%
\pgfpathlineto{\pgfqpoint{5.554198in}{1.663737in}}%
\pgfpathlineto{\pgfqpoint{5.556822in}{1.668717in}}%
\pgfpathlineto{\pgfqpoint{5.559612in}{1.667952in}}%
\pgfpathlineto{\pgfqpoint{5.562180in}{1.669445in}}%
\pgfpathlineto{\pgfqpoint{5.564940in}{1.670561in}}%
\pgfpathlineto{\pgfqpoint{5.567536in}{1.667161in}}%
\pgfpathlineto{\pgfqpoint{5.570215in}{1.666269in}}%
\pgfpathlineto{\pgfqpoint{5.572893in}{1.664179in}}%
\pgfpathlineto{\pgfqpoint{5.575596in}{1.664757in}}%
\pgfpathlineto{\pgfqpoint{5.578342in}{1.659572in}}%
\pgfpathlineto{\pgfqpoint{5.580914in}{1.663967in}}%
\pgfpathlineto{\pgfqpoint{5.583709in}{1.659048in}}%
\pgfpathlineto{\pgfqpoint{5.586269in}{1.665108in}}%
\pgfpathlineto{\pgfqpoint{5.589040in}{1.668617in}}%
\pgfpathlineto{\pgfqpoint{5.591641in}{1.664505in}}%
\pgfpathlineto{\pgfqpoint{5.594368in}{1.659122in}}%
\pgfpathlineto{\pgfqpoint{5.596999in}{1.662585in}}%
\pgfpathlineto{\pgfqpoint{5.599674in}{1.668498in}}%
\pgfpathlineto{\pgfqpoint{5.602352in}{1.665163in}}%
\pgfpathlineto{\pgfqpoint{5.605073in}{1.674606in}}%
\pgfpathlineto{\pgfqpoint{5.607698in}{1.672073in}}%
\pgfpathlineto{\pgfqpoint{5.610389in}{1.679670in}}%
\pgfpathlineto{\pgfqpoint{5.613235in}{1.682740in}}%
\pgfpathlineto{\pgfqpoint{5.615743in}{1.690728in}}%
\pgfpathlineto{\pgfqpoint{5.618526in}{1.691485in}}%
\pgfpathlineto{\pgfqpoint{5.621102in}{1.693495in}}%
\pgfpathlineto{\pgfqpoint{5.623868in}{1.683395in}}%
\pgfpathlineto{\pgfqpoint{5.626460in}{1.688598in}}%
\pgfpathlineto{\pgfqpoint{5.629232in}{1.684786in}}%
\pgfpathlineto{\pgfqpoint{5.631815in}{1.683740in}}%
\pgfpathlineto{\pgfqpoint{5.634496in}{1.681724in}}%
\pgfpathlineto{\pgfqpoint{5.637172in}{1.683112in}}%
\pgfpathlineto{\pgfqpoint{5.639852in}{1.691835in}}%
\pgfpathlineto{\pgfqpoint{5.642518in}{1.692953in}}%
\pgfpathlineto{\pgfqpoint{5.645243in}{1.686726in}}%
\pgfpathlineto{\pgfqpoint{5.648008in}{1.674333in}}%
\pgfpathlineto{\pgfqpoint{5.650563in}{1.686965in}}%
\pgfpathlineto{\pgfqpoint{5.653376in}{1.691807in}}%
\pgfpathlineto{\pgfqpoint{5.655919in}{1.696685in}}%
\pgfpathlineto{\pgfqpoint{5.658723in}{1.689484in}}%
\pgfpathlineto{\pgfqpoint{5.661273in}{1.683461in}}%
\pgfpathlineto{\pgfqpoint{5.664099in}{1.681117in}}%
\pgfpathlineto{\pgfqpoint{5.666632in}{1.677603in}}%
\pgfpathlineto{\pgfqpoint{5.669313in}{1.669033in}}%
\pgfpathlineto{\pgfqpoint{5.671991in}{1.668534in}}%
\pgfpathlineto{\pgfqpoint{5.674667in}{1.668412in}}%
\pgfpathlineto{\pgfqpoint{5.677486in}{1.673292in}}%
\pgfpathlineto{\pgfqpoint{5.680027in}{1.669679in}}%
\pgfpathlineto{\pgfqpoint{5.682836in}{1.667086in}}%
\pgfpathlineto{\pgfqpoint{5.685385in}{1.670275in}}%
\pgfpathlineto{\pgfqpoint{5.688159in}{1.671476in}}%
\pgfpathlineto{\pgfqpoint{5.690730in}{1.664084in}}%
\pgfpathlineto{\pgfqpoint{5.693473in}{1.679685in}}%
\pgfpathlineto{\pgfqpoint{5.696101in}{1.671187in}}%
\pgfpathlineto{\pgfqpoint{5.698775in}{1.670773in}}%
\pgfpathlineto{\pgfqpoint{5.701453in}{1.661402in}}%
\pgfpathlineto{\pgfqpoint{5.704130in}{1.669134in}}%
\pgfpathlineto{\pgfqpoint{5.706800in}{1.662884in}}%
\pgfpathlineto{\pgfqpoint{5.709490in}{1.667430in}}%
\pgfpathlineto{\pgfqpoint{5.712291in}{1.664351in}}%
\pgfpathlineto{\pgfqpoint{5.714834in}{1.663777in}}%
\pgfpathlineto{\pgfqpoint{5.717671in}{1.668232in}}%
\pgfpathlineto{\pgfqpoint{5.720201in}{1.666528in}}%
\pgfpathlineto{\pgfqpoint{5.722950in}{1.667310in}}%
\pgfpathlineto{\pgfqpoint{5.725548in}{1.668721in}}%
\pgfpathlineto{\pgfqpoint{5.728339in}{1.669577in}}%
\pgfpathlineto{\pgfqpoint{5.730919in}{1.664434in}}%
\pgfpathlineto{\pgfqpoint{5.733594in}{1.663296in}}%
\pgfpathlineto{\pgfqpoint{5.736276in}{1.666191in}}%
\pgfpathlineto{\pgfqpoint{5.738974in}{1.670959in}}%
\pgfpathlineto{\pgfqpoint{5.741745in}{1.666935in}}%
\pgfpathlineto{\pgfqpoint{5.744310in}{1.663752in}}%
\pgfpathlineto{\pgfqpoint{5.744310in}{0.413320in}}%
\pgfpathlineto{\pgfqpoint{5.744310in}{0.413320in}}%
\pgfpathlineto{\pgfqpoint{5.741745in}{0.413320in}}%
\pgfpathlineto{\pgfqpoint{5.738974in}{0.413320in}}%
\pgfpathlineto{\pgfqpoint{5.736276in}{0.413320in}}%
\pgfpathlineto{\pgfqpoint{5.733594in}{0.413320in}}%
\pgfpathlineto{\pgfqpoint{5.730919in}{0.413320in}}%
\pgfpathlineto{\pgfqpoint{5.728339in}{0.413320in}}%
\pgfpathlineto{\pgfqpoint{5.725548in}{0.413320in}}%
\pgfpathlineto{\pgfqpoint{5.722950in}{0.413320in}}%
\pgfpathlineto{\pgfqpoint{5.720201in}{0.413320in}}%
\pgfpathlineto{\pgfqpoint{5.717671in}{0.413320in}}%
\pgfpathlineto{\pgfqpoint{5.714834in}{0.413320in}}%
\pgfpathlineto{\pgfqpoint{5.712291in}{0.413320in}}%
\pgfpathlineto{\pgfqpoint{5.709490in}{0.413320in}}%
\pgfpathlineto{\pgfqpoint{5.706800in}{0.413320in}}%
\pgfpathlineto{\pgfqpoint{5.704130in}{0.413320in}}%
\pgfpathlineto{\pgfqpoint{5.701453in}{0.413320in}}%
\pgfpathlineto{\pgfqpoint{5.698775in}{0.413320in}}%
\pgfpathlineto{\pgfqpoint{5.696101in}{0.413320in}}%
\pgfpathlineto{\pgfqpoint{5.693473in}{0.413320in}}%
\pgfpathlineto{\pgfqpoint{5.690730in}{0.413320in}}%
\pgfpathlineto{\pgfqpoint{5.688159in}{0.413320in}}%
\pgfpathlineto{\pgfqpoint{5.685385in}{0.413320in}}%
\pgfpathlineto{\pgfqpoint{5.682836in}{0.413320in}}%
\pgfpathlineto{\pgfqpoint{5.680027in}{0.413320in}}%
\pgfpathlineto{\pgfqpoint{5.677486in}{0.413320in}}%
\pgfpathlineto{\pgfqpoint{5.674667in}{0.413320in}}%
\pgfpathlineto{\pgfqpoint{5.671991in}{0.413320in}}%
\pgfpathlineto{\pgfqpoint{5.669313in}{0.413320in}}%
\pgfpathlineto{\pgfqpoint{5.666632in}{0.413320in}}%
\pgfpathlineto{\pgfqpoint{5.664099in}{0.413320in}}%
\pgfpathlineto{\pgfqpoint{5.661273in}{0.413320in}}%
\pgfpathlineto{\pgfqpoint{5.658723in}{0.413320in}}%
\pgfpathlineto{\pgfqpoint{5.655919in}{0.413320in}}%
\pgfpathlineto{\pgfqpoint{5.653376in}{0.413320in}}%
\pgfpathlineto{\pgfqpoint{5.650563in}{0.413320in}}%
\pgfpathlineto{\pgfqpoint{5.648008in}{0.413320in}}%
\pgfpathlineto{\pgfqpoint{5.645243in}{0.413320in}}%
\pgfpathlineto{\pgfqpoint{5.642518in}{0.413320in}}%
\pgfpathlineto{\pgfqpoint{5.639852in}{0.413320in}}%
\pgfpathlineto{\pgfqpoint{5.637172in}{0.413320in}}%
\pgfpathlineto{\pgfqpoint{5.634496in}{0.413320in}}%
\pgfpathlineto{\pgfqpoint{5.631815in}{0.413320in}}%
\pgfpathlineto{\pgfqpoint{5.629232in}{0.413320in}}%
\pgfpathlineto{\pgfqpoint{5.626460in}{0.413320in}}%
\pgfpathlineto{\pgfqpoint{5.623868in}{0.413320in}}%
\pgfpathlineto{\pgfqpoint{5.621102in}{0.413320in}}%
\pgfpathlineto{\pgfqpoint{5.618526in}{0.413320in}}%
\pgfpathlineto{\pgfqpoint{5.615743in}{0.413320in}}%
\pgfpathlineto{\pgfqpoint{5.613235in}{0.413320in}}%
\pgfpathlineto{\pgfqpoint{5.610389in}{0.413320in}}%
\pgfpathlineto{\pgfqpoint{5.607698in}{0.413320in}}%
\pgfpathlineto{\pgfqpoint{5.605073in}{0.413320in}}%
\pgfpathlineto{\pgfqpoint{5.602352in}{0.413320in}}%
\pgfpathlineto{\pgfqpoint{5.599674in}{0.413320in}}%
\pgfpathlineto{\pgfqpoint{5.596999in}{0.413320in}}%
\pgfpathlineto{\pgfqpoint{5.594368in}{0.413320in}}%
\pgfpathlineto{\pgfqpoint{5.591641in}{0.413320in}}%
\pgfpathlineto{\pgfqpoint{5.589040in}{0.413320in}}%
\pgfpathlineto{\pgfqpoint{5.586269in}{0.413320in}}%
\pgfpathlineto{\pgfqpoint{5.583709in}{0.413320in}}%
\pgfpathlineto{\pgfqpoint{5.580914in}{0.413320in}}%
\pgfpathlineto{\pgfqpoint{5.578342in}{0.413320in}}%
\pgfpathlineto{\pgfqpoint{5.575596in}{0.413320in}}%
\pgfpathlineto{\pgfqpoint{5.572893in}{0.413320in}}%
\pgfpathlineto{\pgfqpoint{5.570215in}{0.413320in}}%
\pgfpathlineto{\pgfqpoint{5.567536in}{0.413320in}}%
\pgfpathlineto{\pgfqpoint{5.564940in}{0.413320in}}%
\pgfpathlineto{\pgfqpoint{5.562180in}{0.413320in}}%
\pgfpathlineto{\pgfqpoint{5.559612in}{0.413320in}}%
\pgfpathlineto{\pgfqpoint{5.556822in}{0.413320in}}%
\pgfpathlineto{\pgfqpoint{5.554198in}{0.413320in}}%
\pgfpathlineto{\pgfqpoint{5.551457in}{0.413320in}}%
\pgfpathlineto{\pgfqpoint{5.548921in}{0.413320in}}%
\pgfpathlineto{\pgfqpoint{5.546110in}{0.413320in}}%
\pgfpathlineto{\pgfqpoint{5.543421in}{0.413320in}}%
\pgfpathlineto{\pgfqpoint{5.540750in}{0.413320in}}%
\pgfpathlineto{\pgfqpoint{5.538074in}{0.413320in}}%
\pgfpathlineto{\pgfqpoint{5.535395in}{0.413320in}}%
\pgfpathlineto{\pgfqpoint{5.532717in}{0.413320in}}%
\pgfpathlineto{\pgfqpoint{5.530148in}{0.413320in}}%
\pgfpathlineto{\pgfqpoint{5.527360in}{0.413320in}}%
\pgfpathlineto{\pgfqpoint{5.524756in}{0.413320in}}%
\pgfpathlineto{\pgfqpoint{5.522003in}{0.413320in}}%
\pgfpathlineto{\pgfqpoint{5.519433in}{0.413320in}}%
\pgfpathlineto{\pgfqpoint{5.516646in}{0.413320in}}%
\pgfpathlineto{\pgfqpoint{5.514080in}{0.413320in}}%
\pgfpathlineto{\pgfqpoint{5.511290in}{0.413320in}}%
\pgfpathlineto{\pgfqpoint{5.508612in}{0.413320in}}%
\pgfpathlineto{\pgfqpoint{5.505933in}{0.413320in}}%
\pgfpathlineto{\pgfqpoint{5.503255in}{0.413320in}}%
\pgfpathlineto{\pgfqpoint{5.500687in}{0.413320in}}%
\pgfpathlineto{\pgfqpoint{5.497898in}{0.413320in}}%
\pgfpathlineto{\pgfqpoint{5.495346in}{0.413320in}}%
\pgfpathlineto{\pgfqpoint{5.492541in}{0.413320in}}%
\pgfpathlineto{\pgfqpoint{5.490000in}{0.413320in}}%
\pgfpathlineto{\pgfqpoint{5.487176in}{0.413320in}}%
\pgfpathlineto{\pgfqpoint{5.484641in}{0.413320in}}%
\pgfpathlineto{\pgfqpoint{5.481825in}{0.413320in}}%
\pgfpathlineto{\pgfqpoint{5.479152in}{0.413320in}}%
\pgfpathlineto{\pgfqpoint{5.476458in}{0.413320in}}%
\pgfpathlineto{\pgfqpoint{5.473792in}{0.413320in}}%
\pgfpathlineto{\pgfqpoint{5.471113in}{0.413320in}}%
\pgfpathlineto{\pgfqpoint{5.468425in}{0.413320in}}%
\pgfpathlineto{\pgfqpoint{5.465888in}{0.413320in}}%
\pgfpathlineto{\pgfqpoint{5.463079in}{0.413320in}}%
\pgfpathlineto{\pgfqpoint{5.460489in}{0.413320in}}%
\pgfpathlineto{\pgfqpoint{5.457721in}{0.413320in}}%
\pgfpathlineto{\pgfqpoint{5.455168in}{0.413320in}}%
\pgfpathlineto{\pgfqpoint{5.452365in}{0.413320in}}%
\pgfpathlineto{\pgfqpoint{5.449769in}{0.413320in}}%
\pgfpathlineto{\pgfqpoint{5.447021in}{0.413320in}}%
\pgfpathlineto{\pgfqpoint{5.444328in}{0.413320in}}%
\pgfpathlineto{\pgfqpoint{5.441698in}{0.413320in}}%
\pgfpathlineto{\pgfqpoint{5.438974in}{0.413320in}}%
\pgfpathlineto{\pgfqpoint{5.436295in}{0.413320in}}%
\pgfpathlineto{\pgfqpoint{5.433616in}{0.413320in}}%
\pgfpathlineto{\pgfqpoint{5.431015in}{0.413320in}}%
\pgfpathlineto{\pgfqpoint{5.428259in}{0.413320in}}%
\pgfpathlineto{\pgfqpoint{5.425661in}{0.413320in}}%
\pgfpathlineto{\pgfqpoint{5.422897in}{0.413320in}}%
\pgfpathlineto{\pgfqpoint{5.420304in}{0.413320in}}%
\pgfpathlineto{\pgfqpoint{5.417547in}{0.413320in}}%
\pgfpathlineto{\pgfqpoint{5.414954in}{0.413320in}}%
\pgfpathlineto{\pgfqpoint{5.412190in}{0.413320in}}%
\pgfpathlineto{\pgfqpoint{5.409507in}{0.413320in}}%
\pgfpathlineto{\pgfqpoint{5.406832in}{0.413320in}}%
\pgfpathlineto{\pgfqpoint{5.404154in}{0.413320in}}%
\pgfpathlineto{\pgfqpoint{5.401576in}{0.413320in}}%
\pgfpathlineto{\pgfqpoint{5.398784in}{0.413320in}}%
\pgfpathlineto{\pgfqpoint{5.396219in}{0.413320in}}%
\pgfpathlineto{\pgfqpoint{5.393441in}{0.413320in}}%
\pgfpathlineto{\pgfqpoint{5.390900in}{0.413320in}}%
\pgfpathlineto{\pgfqpoint{5.388083in}{0.413320in}}%
\pgfpathlineto{\pgfqpoint{5.385550in}{0.413320in}}%
\pgfpathlineto{\pgfqpoint{5.382725in}{0.413320in}}%
\pgfpathlineto{\pgfqpoint{5.380048in}{0.413320in}}%
\pgfpathlineto{\pgfqpoint{5.377370in}{0.413320in}}%
\pgfpathlineto{\pgfqpoint{5.374692in}{0.413320in}}%
\pgfpathlineto{\pgfqpoint{5.372013in}{0.413320in}}%
\pgfpathlineto{\pgfqpoint{5.369335in}{0.413320in}}%
\pgfpathlineto{\pgfqpoint{5.366727in}{0.413320in}}%
\pgfpathlineto{\pgfqpoint{5.363966in}{0.413320in}}%
\pgfpathlineto{\pgfqpoint{5.361370in}{0.413320in}}%
\pgfpathlineto{\pgfqpoint{5.358612in}{0.413320in}}%
\pgfpathlineto{\pgfqpoint{5.356056in}{0.413320in}}%
\pgfpathlineto{\pgfqpoint{5.353262in}{0.413320in}}%
\pgfpathlineto{\pgfqpoint{5.350723in}{0.413320in}}%
\pgfpathlineto{\pgfqpoint{5.347905in}{0.413320in}}%
\pgfpathlineto{\pgfqpoint{5.345224in}{0.413320in}}%
\pgfpathlineto{\pgfqpoint{5.342549in}{0.413320in}}%
\pgfpathlineto{\pgfqpoint{5.339872in}{0.413320in}}%
\pgfpathlineto{\pgfqpoint{5.337353in}{0.413320in}}%
\pgfpathlineto{\pgfqpoint{5.334510in}{0.413320in}}%
\pgfpathlineto{\pgfqpoint{5.331973in}{0.413320in}}%
\pgfpathlineto{\pgfqpoint{5.329159in}{0.413320in}}%
\pgfpathlineto{\pgfqpoint{5.326564in}{0.413320in}}%
\pgfpathlineto{\pgfqpoint{5.323802in}{0.413320in}}%
\pgfpathlineto{\pgfqpoint{5.321256in}{0.413320in}}%
\pgfpathlineto{\pgfqpoint{5.318430in}{0.413320in}}%
\pgfpathlineto{\pgfqpoint{5.315754in}{0.413320in}}%
\pgfpathlineto{\pgfqpoint{5.313089in}{0.413320in}}%
\pgfpathlineto{\pgfqpoint{5.310411in}{0.413320in}}%
\pgfpathlineto{\pgfqpoint{5.307731in}{0.413320in}}%
\pgfpathlineto{\pgfqpoint{5.305054in}{0.413320in}}%
\pgfpathlineto{\pgfqpoint{5.302443in}{0.413320in}}%
\pgfpathlineto{\pgfqpoint{5.299696in}{0.413320in}}%
\pgfpathlineto{\pgfqpoint{5.297140in}{0.413320in}}%
\pgfpathlineto{\pgfqpoint{5.294339in}{0.413320in}}%
\pgfpathlineto{\pgfqpoint{5.291794in}{0.413320in}}%
\pgfpathlineto{\pgfqpoint{5.288984in}{0.413320in}}%
\pgfpathlineto{\pgfqpoint{5.286436in}{0.413320in}}%
\pgfpathlineto{\pgfqpoint{5.283631in}{0.413320in}}%
\pgfpathlineto{\pgfqpoint{5.280947in}{0.413320in}}%
\pgfpathlineto{\pgfqpoint{5.278322in}{0.413320in}}%
\pgfpathlineto{\pgfqpoint{5.275589in}{0.413320in}}%
\pgfpathlineto{\pgfqpoint{5.272913in}{0.413320in}}%
\pgfpathlineto{\pgfqpoint{5.270238in}{0.413320in}}%
\pgfpathlineto{\pgfqpoint{5.267691in}{0.413320in}}%
\pgfpathlineto{\pgfqpoint{5.264876in}{0.413320in}}%
\pgfpathlineto{\pgfqpoint{5.262264in}{0.413320in}}%
\pgfpathlineto{\pgfqpoint{5.259511in}{0.413320in}}%
\pgfpathlineto{\pgfqpoint{5.256973in}{0.413320in}}%
\pgfpathlineto{\pgfqpoint{5.254236in}{0.413320in}}%
\pgfpathlineto{\pgfqpoint{5.251590in}{0.413320in}}%
\pgfpathlineto{\pgfqpoint{5.248816in}{0.413320in}}%
\pgfpathlineto{\pgfqpoint{5.246130in}{0.413320in}}%
\pgfpathlineto{\pgfqpoint{5.243445in}{0.413320in}}%
\pgfpathlineto{\pgfqpoint{5.240777in}{0.413320in}}%
\pgfpathlineto{\pgfqpoint{5.238173in}{0.413320in}}%
\pgfpathlineto{\pgfqpoint{5.235409in}{0.413320in}}%
\pgfpathlineto{\pgfqpoint{5.232855in}{0.413320in}}%
\pgfpathlineto{\pgfqpoint{5.230059in}{0.413320in}}%
\pgfpathlineto{\pgfqpoint{5.227470in}{0.413320in}}%
\pgfpathlineto{\pgfqpoint{5.224695in}{0.413320in}}%
\pgfpathlineto{\pgfqpoint{5.222151in}{0.413320in}}%
\pgfpathlineto{\pgfqpoint{5.219345in}{0.413320in}}%
\pgfpathlineto{\pgfqpoint{5.216667in}{0.413320in}}%
\pgfpathlineto{\pgfqpoint{5.214027in}{0.413320in}}%
\pgfpathlineto{\pgfqpoint{5.211299in}{0.413320in}}%
\pgfpathlineto{\pgfqpoint{5.208630in}{0.413320in}}%
\pgfpathlineto{\pgfqpoint{5.205952in}{0.413320in}}%
\pgfpathlineto{\pgfqpoint{5.203388in}{0.413320in}}%
\pgfpathlineto{\pgfqpoint{5.200594in}{0.413320in}}%
\pgfpathlineto{\pgfqpoint{5.198008in}{0.413320in}}%
\pgfpathlineto{\pgfqpoint{5.195239in}{0.413320in}}%
\pgfpathlineto{\pgfqpoint{5.192680in}{0.413320in}}%
\pgfpathlineto{\pgfqpoint{5.189880in}{0.413320in}}%
\pgfpathlineto{\pgfqpoint{5.187294in}{0.413320in}}%
\pgfpathlineto{\pgfqpoint{5.184522in}{0.413320in}}%
\pgfpathlineto{\pgfqpoint{5.181848in}{0.413320in}}%
\pgfpathlineto{\pgfqpoint{5.179188in}{0.413320in}}%
\pgfpathlineto{\pgfqpoint{5.176477in}{0.413320in}}%
\pgfpathlineto{\pgfqpoint{5.173925in}{0.413320in}}%
\pgfpathlineto{\pgfqpoint{5.171133in}{0.413320in}}%
\pgfpathlineto{\pgfqpoint{5.168591in}{0.413320in}}%
\pgfpathlineto{\pgfqpoint{5.165775in}{0.413320in}}%
\pgfpathlineto{\pgfqpoint{5.163243in}{0.413320in}}%
\pgfpathlineto{\pgfqpoint{5.160420in}{0.413320in}}%
\pgfpathlineto{\pgfqpoint{5.157815in}{0.413320in}}%
\pgfpathlineto{\pgfqpoint{5.155059in}{0.413320in}}%
\pgfpathlineto{\pgfqpoint{5.152382in}{0.413320in}}%
\pgfpathlineto{\pgfqpoint{5.149734in}{0.413320in}}%
\pgfpathlineto{\pgfqpoint{5.147029in}{0.413320in}}%
\pgfpathlineto{\pgfqpoint{5.144349in}{0.413320in}}%
\pgfpathlineto{\pgfqpoint{5.141660in}{0.413320in}}%
\pgfpathlineto{\pgfqpoint{5.139072in}{0.413320in}}%
\pgfpathlineto{\pgfqpoint{5.136311in}{0.413320in}}%
\pgfpathlineto{\pgfqpoint{5.133716in}{0.413320in}}%
\pgfpathlineto{\pgfqpoint{5.130953in}{0.413320in}}%
\pgfpathlineto{\pgfqpoint{5.128421in}{0.413320in}}%
\pgfpathlineto{\pgfqpoint{5.125599in}{0.413320in}}%
\pgfpathlineto{\pgfqpoint{5.123042in}{0.413320in}}%
\pgfpathlineto{\pgfqpoint{5.120243in}{0.413320in}}%
\pgfpathlineto{\pgfqpoint{5.117550in}{0.413320in}}%
\pgfpathlineto{\pgfqpoint{5.114887in}{0.413320in}}%
\pgfpathlineto{\pgfqpoint{5.112209in}{0.413320in}}%
\pgfpathlineto{\pgfqpoint{5.109530in}{0.413320in}}%
\pgfpathlineto{\pgfqpoint{5.106842in}{0.413320in}}%
\pgfpathlineto{\pgfqpoint{5.104312in}{0.413320in}}%
\pgfpathlineto{\pgfqpoint{5.101496in}{0.413320in}}%
\pgfpathlineto{\pgfqpoint{5.098948in}{0.413320in}}%
\pgfpathlineto{\pgfqpoint{5.096142in}{0.413320in}}%
\pgfpathlineto{\pgfqpoint{5.093579in}{0.413320in}}%
\pgfpathlineto{\pgfqpoint{5.090788in}{0.413320in}}%
\pgfpathlineto{\pgfqpoint{5.088103in}{0.413320in}}%
\pgfpathlineto{\pgfqpoint{5.085426in}{0.413320in}}%
\pgfpathlineto{\pgfqpoint{5.082746in}{0.413320in}}%
\pgfpathlineto{\pgfqpoint{5.080067in}{0.413320in}}%
\pgfpathlineto{\pgfqpoint{5.077390in}{0.413320in}}%
\pgfpathlineto{\pgfqpoint{5.074851in}{0.413320in}}%
\pgfpathlineto{\pgfqpoint{5.072030in}{0.413320in}}%
\pgfpathlineto{\pgfqpoint{5.069463in}{0.413320in}}%
\pgfpathlineto{\pgfqpoint{5.066677in}{0.413320in}}%
\pgfpathlineto{\pgfqpoint{5.064144in}{0.413320in}}%
\pgfpathlineto{\pgfqpoint{5.061315in}{0.413320in}}%
\pgfpathlineto{\pgfqpoint{5.058711in}{0.413320in}}%
\pgfpathlineto{\pgfqpoint{5.055952in}{0.413320in}}%
\pgfpathlineto{\pgfqpoint{5.053284in}{0.413320in}}%
\pgfpathlineto{\pgfqpoint{5.050606in}{0.413320in}}%
\pgfpathlineto{\pgfqpoint{5.047924in}{0.413320in}}%
\pgfpathlineto{\pgfqpoint{5.045249in}{0.413320in}}%
\pgfpathlineto{\pgfqpoint{5.042572in}{0.413320in}}%
\pgfpathlineto{\pgfqpoint{5.039962in}{0.413320in}}%
\pgfpathlineto{\pgfqpoint{5.037214in}{0.413320in}}%
\pgfpathlineto{\pgfqpoint{5.034649in}{0.413320in}}%
\pgfpathlineto{\pgfqpoint{5.031849in}{0.413320in}}%
\pgfpathlineto{\pgfqpoint{5.029275in}{0.413320in}}%
\pgfpathlineto{\pgfqpoint{5.026501in}{0.413320in}}%
\pgfpathlineto{\pgfqpoint{5.023927in}{0.413320in}}%
\pgfpathlineto{\pgfqpoint{5.021147in}{0.413320in}}%
\pgfpathlineto{\pgfqpoint{5.018466in}{0.413320in}}%
\pgfpathlineto{\pgfqpoint{5.015820in}{0.413320in}}%
\pgfpathlineto{\pgfqpoint{5.013104in}{0.413320in}}%
\pgfpathlineto{\pgfqpoint{5.010562in}{0.413320in}}%
\pgfpathlineto{\pgfqpoint{5.007751in}{0.413320in}}%
\pgfpathlineto{\pgfqpoint{5.005178in}{0.413320in}}%
\pgfpathlineto{\pgfqpoint{5.002384in}{0.413320in}}%
\pgfpathlineto{\pgfqpoint{4.999780in}{0.413320in}}%
\pgfpathlineto{\pgfqpoint{4.997028in}{0.413320in}}%
\pgfpathlineto{\pgfqpoint{4.994390in}{0.413320in}}%
\pgfpathlineto{\pgfqpoint{4.991683in}{0.413320in}}%
\pgfpathlineto{\pgfqpoint{4.989001in}{0.413320in}}%
\pgfpathlineto{\pgfqpoint{4.986325in}{0.413320in}}%
\pgfpathlineto{\pgfqpoint{4.983637in}{0.413320in}}%
\pgfpathlineto{\pgfqpoint{4.980967in}{0.413320in}}%
\pgfpathlineto{\pgfqpoint{4.978287in}{0.413320in}}%
\pgfpathlineto{\pgfqpoint{4.975703in}{0.413320in}}%
\pgfpathlineto{\pgfqpoint{4.972933in}{0.413320in}}%
\pgfpathlineto{\pgfqpoint{4.970314in}{0.413320in}}%
\pgfpathlineto{\pgfqpoint{4.967575in}{0.413320in}}%
\pgfpathlineto{\pgfqpoint{4.965002in}{0.413320in}}%
\pgfpathlineto{\pgfqpoint{4.962219in}{0.413320in}}%
\pgfpathlineto{\pgfqpoint{4.959689in}{0.413320in}}%
\pgfpathlineto{\pgfqpoint{4.956862in}{0.413320in}}%
\pgfpathlineto{\pgfqpoint{4.954182in}{0.413320in}}%
\pgfpathlineto{\pgfqpoint{4.951504in}{0.413320in}}%
\pgfpathlineto{\pgfqpoint{4.948827in}{0.413320in}}%
\pgfpathlineto{\pgfqpoint{4.946151in}{0.413320in}}%
\pgfpathlineto{\pgfqpoint{4.943466in}{0.413320in}}%
\pgfpathlineto{\pgfqpoint{4.940881in}{0.413320in}}%
\pgfpathlineto{\pgfqpoint{4.938112in}{0.413320in}}%
\pgfpathlineto{\pgfqpoint{4.935515in}{0.413320in}}%
\pgfpathlineto{\pgfqpoint{4.932742in}{0.413320in}}%
\pgfpathlineto{\pgfqpoint{4.930170in}{0.413320in}}%
\pgfpathlineto{\pgfqpoint{4.927400in}{0.413320in}}%
\pgfpathlineto{\pgfqpoint{4.924708in}{0.413320in}}%
\pgfpathlineto{\pgfqpoint{4.922041in}{0.413320in}}%
\pgfpathlineto{\pgfqpoint{4.919352in}{0.413320in}}%
\pgfpathlineto{\pgfqpoint{4.916681in}{0.413320in}}%
\pgfpathlineto{\pgfqpoint{4.914009in}{0.413320in}}%
\pgfpathlineto{\pgfqpoint{4.911435in}{0.413320in}}%
\pgfpathlineto{\pgfqpoint{4.908648in}{0.413320in}}%
\pgfpathlineto{\pgfqpoint{4.906096in}{0.413320in}}%
\pgfpathlineto{\pgfqpoint{4.903295in}{0.413320in}}%
\pgfpathlineto{\pgfqpoint{4.900712in}{0.413320in}}%
\pgfpathlineto{\pgfqpoint{4.897938in}{0.413320in}}%
\pgfpathlineto{\pgfqpoint{4.895399in}{0.413320in}}%
\pgfpathlineto{\pgfqpoint{4.892611in}{0.413320in}}%
\pgfpathlineto{\pgfqpoint{4.889902in}{0.413320in}}%
\pgfpathlineto{\pgfqpoint{4.887211in}{0.413320in}}%
\pgfpathlineto{\pgfqpoint{4.884540in}{0.413320in}}%
\pgfpathlineto{\pgfqpoint{4.881864in}{0.413320in}}%
\pgfpathlineto{\pgfqpoint{4.879180in}{0.413320in}}%
\pgfpathlineto{\pgfqpoint{4.876636in}{0.413320in}}%
\pgfpathlineto{\pgfqpoint{4.873832in}{0.413320in}}%
\pgfpathlineto{\pgfqpoint{4.871209in}{0.413320in}}%
\pgfpathlineto{\pgfqpoint{4.868474in}{0.413320in}}%
\pgfpathlineto{\pgfqpoint{4.865910in}{0.413320in}}%
\pgfpathlineto{\pgfqpoint{4.863116in}{0.413320in}}%
\pgfpathlineto{\pgfqpoint{4.860544in}{0.413320in}}%
\pgfpathlineto{\pgfqpoint{4.857807in}{0.413320in}}%
\pgfpathlineto{\pgfqpoint{4.855070in}{0.413320in}}%
\pgfpathlineto{\pgfqpoint{4.852404in}{0.413320in}}%
\pgfpathlineto{\pgfqpoint{4.849715in}{0.413320in}}%
\pgfpathlineto{\pgfqpoint{4.847127in}{0.413320in}}%
\pgfpathlineto{\pgfqpoint{4.844361in}{0.413320in}}%
\pgfpathlineto{\pgfqpoint{4.842380in}{0.413320in}}%
\pgfpathlineto{\pgfqpoint{4.839922in}{0.413320in}}%
\pgfpathlineto{\pgfqpoint{4.837992in}{0.413320in}}%
\pgfpathlineto{\pgfqpoint{4.833657in}{0.413320in}}%
\pgfpathlineto{\pgfqpoint{4.831045in}{0.413320in}}%
\pgfpathlineto{\pgfqpoint{4.828291in}{0.413320in}}%
\pgfpathlineto{\pgfqpoint{4.825619in}{0.413320in}}%
\pgfpathlineto{\pgfqpoint{4.822945in}{0.413320in}}%
\pgfpathlineto{\pgfqpoint{4.820265in}{0.413320in}}%
\pgfpathlineto{\pgfqpoint{4.817587in}{0.413320in}}%
\pgfpathlineto{\pgfqpoint{4.814907in}{0.413320in}}%
\pgfpathlineto{\pgfqpoint{4.812377in}{0.413320in}}%
\pgfpathlineto{\pgfqpoint{4.809538in}{0.413320in}}%
\pgfpathlineto{\pgfqpoint{4.807017in}{0.413320in}}%
\pgfpathlineto{\pgfqpoint{4.804193in}{0.413320in}}%
\pgfpathlineto{\pgfqpoint{4.801586in}{0.413320in}}%
\pgfpathlineto{\pgfqpoint{4.798830in}{0.413320in}}%
\pgfpathlineto{\pgfqpoint{4.796274in}{0.413320in}}%
\pgfpathlineto{\pgfqpoint{4.793512in}{0.413320in}}%
\pgfpathlineto{\pgfqpoint{4.790798in}{0.413320in}}%
\pgfpathlineto{\pgfqpoint{4.788116in}{0.413320in}}%
\pgfpathlineto{\pgfqpoint{4.785445in}{0.413320in}}%
\pgfpathlineto{\pgfqpoint{4.782872in}{0.413320in}}%
\pgfpathlineto{\pgfqpoint{4.780083in}{0.413320in}}%
\pgfpathlineto{\pgfqpoint{4.777535in}{0.413320in}}%
\pgfpathlineto{\pgfqpoint{4.774732in}{0.413320in}}%
\pgfpathlineto{\pgfqpoint{4.772198in}{0.413320in}}%
\pgfpathlineto{\pgfqpoint{4.769367in}{0.413320in}}%
\pgfpathlineto{\pgfqpoint{4.766783in}{0.413320in}}%
\pgfpathlineto{\pgfqpoint{4.764018in}{0.413320in}}%
\pgfpathlineto{\pgfqpoint{4.761337in}{0.413320in}}%
\pgfpathlineto{\pgfqpoint{4.758653in}{0.413320in}}%
\pgfpathlineto{\pgfqpoint{4.755983in}{0.413320in}}%
\pgfpathlineto{\pgfqpoint{4.753298in}{0.413320in}}%
\pgfpathlineto{\pgfqpoint{4.750627in}{0.413320in}}%
\pgfpathlineto{\pgfqpoint{4.748081in}{0.413320in}}%
\pgfpathlineto{\pgfqpoint{4.745256in}{0.413320in}}%
\pgfpathlineto{\pgfqpoint{4.742696in}{0.413320in}}%
\pgfpathlineto{\pgfqpoint{4.739912in}{0.413320in}}%
\pgfpathlineto{\pgfqpoint{4.737348in}{0.413320in}}%
\pgfpathlineto{\pgfqpoint{4.734552in}{0.413320in}}%
\pgfpathlineto{\pgfqpoint{4.731901in}{0.413320in}}%
\pgfpathlineto{\pgfqpoint{4.729233in}{0.413320in}}%
\pgfpathlineto{\pgfqpoint{4.726508in}{0.413320in}}%
\pgfpathlineto{\pgfqpoint{4.723873in}{0.413320in}}%
\pgfpathlineto{\pgfqpoint{4.721160in}{0.413320in}}%
\pgfpathlineto{\pgfqpoint{4.718486in}{0.413320in}}%
\pgfpathlineto{\pgfqpoint{4.715806in}{0.413320in}}%
\pgfpathlineto{\pgfqpoint{4.713275in}{0.413320in}}%
\pgfpathlineto{\pgfqpoint{4.710437in}{0.413320in}}%
\pgfpathlineto{\pgfqpoint{4.707824in}{0.413320in}}%
\pgfpathlineto{\pgfqpoint{4.705094in}{0.413320in}}%
\pgfpathlineto{\pgfqpoint{4.702517in}{0.413320in}}%
\pgfpathlineto{\pgfqpoint{4.699734in}{0.413320in}}%
\pgfpathlineto{\pgfqpoint{4.697170in}{0.413320in}}%
\pgfpathlineto{\pgfqpoint{4.694381in}{0.413320in}}%
\pgfpathlineto{\pgfqpoint{4.691694in}{0.413320in}}%
\pgfpathlineto{\pgfqpoint{4.689051in}{0.413320in}}%
\pgfpathlineto{\pgfqpoint{4.686337in}{0.413320in}}%
\pgfpathlineto{\pgfqpoint{4.683799in}{0.413320in}}%
\pgfpathlineto{\pgfqpoint{4.680988in}{0.413320in}}%
\pgfpathlineto{\pgfqpoint{4.678448in}{0.413320in}}%
\pgfpathlineto{\pgfqpoint{4.675619in}{0.413320in}}%
\pgfpathlineto{\pgfqpoint{4.673068in}{0.413320in}}%
\pgfpathlineto{\pgfqpoint{4.670261in}{0.413320in}}%
\pgfpathlineto{\pgfqpoint{4.667764in}{0.413320in}}%
\pgfpathlineto{\pgfqpoint{4.664923in}{0.413320in}}%
\pgfpathlineto{\pgfqpoint{4.662237in}{0.413320in}}%
\pgfpathlineto{\pgfqpoint{4.659590in}{0.413320in}}%
\pgfpathlineto{\pgfqpoint{4.656873in}{0.413320in}}%
\pgfpathlineto{\pgfqpoint{4.654203in}{0.413320in}}%
\pgfpathlineto{\pgfqpoint{4.651524in}{0.413320in}}%
\pgfpathlineto{\pgfqpoint{4.648922in}{0.413320in}}%
\pgfpathlineto{\pgfqpoint{4.646169in}{0.413320in}}%
\pgfpathlineto{\pgfqpoint{4.643628in}{0.413320in}}%
\pgfpathlineto{\pgfqpoint{4.640809in}{0.413320in}}%
\pgfpathlineto{\pgfqpoint{4.638204in}{0.413320in}}%
\pgfpathlineto{\pgfqpoint{4.635445in}{0.413320in}}%
\pgfpathlineto{\pgfqpoint{4.632902in}{0.413320in}}%
\pgfpathlineto{\pgfqpoint{4.630096in}{0.413320in}}%
\pgfpathlineto{\pgfqpoint{4.627411in}{0.413320in}}%
\pgfpathlineto{\pgfqpoint{4.624741in}{0.413320in}}%
\pgfpathlineto{\pgfqpoint{4.622056in}{0.413320in}}%
\pgfpathlineto{\pgfqpoint{4.619529in}{0.413320in}}%
\pgfpathlineto{\pgfqpoint{4.616702in}{0.413320in}}%
\pgfpathlineto{\pgfqpoint{4.614134in}{0.413320in}}%
\pgfpathlineto{\pgfqpoint{4.611350in}{0.413320in}}%
\pgfpathlineto{\pgfqpoint{4.608808in}{0.413320in}}%
\pgfpathlineto{\pgfqpoint{4.605990in}{0.413320in}}%
\pgfpathlineto{\pgfqpoint{4.603430in}{0.413320in}}%
\pgfpathlineto{\pgfqpoint{4.600633in}{0.413320in}}%
\pgfpathlineto{\pgfqpoint{4.597951in}{0.413320in}}%
\pgfpathlineto{\pgfqpoint{4.595281in}{0.413320in}}%
\pgfpathlineto{\pgfqpoint{4.592589in}{0.413320in}}%
\pgfpathlineto{\pgfqpoint{4.589920in}{0.413320in}}%
\pgfpathlineto{\pgfqpoint{4.587244in}{0.413320in}}%
\pgfpathlineto{\pgfqpoint{4.584672in}{0.413320in}}%
\pgfpathlineto{\pgfqpoint{4.581888in}{0.413320in}}%
\pgfpathlineto{\pgfqpoint{4.579305in}{0.413320in}}%
\pgfpathlineto{\pgfqpoint{4.576531in}{0.413320in}}%
\pgfpathlineto{\pgfqpoint{4.573947in}{0.413320in}}%
\pgfpathlineto{\pgfqpoint{4.571171in}{0.413320in}}%
\pgfpathlineto{\pgfqpoint{4.568612in}{0.413320in}}%
\pgfpathlineto{\pgfqpoint{4.565820in}{0.413320in}}%
\pgfpathlineto{\pgfqpoint{4.563125in}{0.413320in}}%
\pgfpathlineto{\pgfqpoint{4.560448in}{0.413320in}}%
\pgfpathlineto{\pgfqpoint{4.557777in}{0.413320in}}%
\pgfpathlineto{\pgfqpoint{4.555106in}{0.413320in}}%
\pgfpathlineto{\pgfqpoint{4.552425in}{0.413320in}}%
\pgfpathlineto{\pgfqpoint{4.549822in}{0.413320in}}%
\pgfpathlineto{\pgfqpoint{4.547064in}{0.413320in}}%
\pgfpathlineto{\pgfqpoint{4.544464in}{0.413320in}}%
\pgfpathlineto{\pgfqpoint{4.541711in}{0.413320in}}%
\pgfpathlineto{\pgfqpoint{4.539144in}{0.413320in}}%
\pgfpathlineto{\pgfqpoint{4.536400in}{0.413320in}}%
\pgfpathlineto{\pgfqpoint{4.533764in}{0.413320in}}%
\pgfpathlineto{\pgfqpoint{4.530990in}{0.413320in}}%
\pgfpathlineto{\pgfqpoint{4.528307in}{0.413320in}}%
\pgfpathlineto{\pgfqpoint{4.525640in}{0.413320in}}%
\pgfpathlineto{\pgfqpoint{4.522962in}{0.413320in}}%
\pgfpathlineto{\pgfqpoint{4.520345in}{0.413320in}}%
\pgfpathlineto{\pgfqpoint{4.517598in}{0.413320in}}%
\pgfpathlineto{\pgfqpoint{4.515080in}{0.413320in}}%
\pgfpathlineto{\pgfqpoint{4.512246in}{0.413320in}}%
\pgfpathlineto{\pgfqpoint{4.509643in}{0.413320in}}%
\pgfpathlineto{\pgfqpoint{4.506893in}{0.413320in}}%
\pgfpathlineto{\pgfqpoint{4.504305in}{0.413320in}}%
\pgfpathlineto{\pgfqpoint{4.501529in}{0.413320in}}%
\pgfpathlineto{\pgfqpoint{4.498850in}{0.413320in}}%
\pgfpathlineto{\pgfqpoint{4.496167in}{0.413320in}}%
\pgfpathlineto{\pgfqpoint{4.493492in}{0.413320in}}%
\pgfpathlineto{\pgfqpoint{4.490822in}{0.413320in}}%
\pgfpathlineto{\pgfqpoint{4.488130in}{0.413320in}}%
\pgfpathlineto{\pgfqpoint{4.485581in}{0.413320in}}%
\pgfpathlineto{\pgfqpoint{4.482778in}{0.413320in}}%
\pgfpathlineto{\pgfqpoint{4.480201in}{0.413320in}}%
\pgfpathlineto{\pgfqpoint{4.477430in}{0.413320in}}%
\pgfpathlineto{\pgfqpoint{4.474861in}{0.413320in}}%
\pgfpathlineto{\pgfqpoint{4.472059in}{0.413320in}}%
\pgfpathlineto{\pgfqpoint{4.469492in}{0.413320in}}%
\pgfpathlineto{\pgfqpoint{4.466717in}{0.413320in}}%
\pgfpathlineto{\pgfqpoint{4.464029in}{0.413320in}}%
\pgfpathlineto{\pgfqpoint{4.461367in}{0.413320in}}%
\pgfpathlineto{\pgfqpoint{4.458681in}{0.413320in}}%
\pgfpathlineto{\pgfqpoint{4.456138in}{0.413320in}}%
\pgfpathlineto{\pgfqpoint{4.453312in}{0.413320in}}%
\pgfpathlineto{\pgfqpoint{4.450767in}{0.413320in}}%
\pgfpathlineto{\pgfqpoint{4.447965in}{0.413320in}}%
\pgfpathlineto{\pgfqpoint{4.445423in}{0.413320in}}%
\pgfpathlineto{\pgfqpoint{4.442611in}{0.413320in}}%
\pgfpathlineto{\pgfqpoint{4.440041in}{0.413320in}}%
\pgfpathlineto{\pgfqpoint{4.437253in}{0.413320in}}%
\pgfpathlineto{\pgfqpoint{4.434569in}{0.413320in}}%
\pgfpathlineto{\pgfqpoint{4.431901in}{0.413320in}}%
\pgfpathlineto{\pgfqpoint{4.429220in}{0.413320in}}%
\pgfpathlineto{\pgfqpoint{4.426534in}{0.413320in}}%
\pgfpathlineto{\pgfqpoint{4.423863in}{0.413320in}}%
\pgfpathlineto{\pgfqpoint{4.421292in}{0.413320in}}%
\pgfpathlineto{\pgfqpoint{4.418506in}{0.413320in}}%
\pgfpathlineto{\pgfqpoint{4.415932in}{0.413320in}}%
\pgfpathlineto{\pgfqpoint{4.413149in}{0.413320in}}%
\pgfpathlineto{\pgfqpoint{4.410587in}{0.413320in}}%
\pgfpathlineto{\pgfqpoint{4.407788in}{0.413320in}}%
\pgfpathlineto{\pgfqpoint{4.405234in}{0.413320in}}%
\pgfpathlineto{\pgfqpoint{4.402468in}{0.413320in}}%
\pgfpathlineto{\pgfqpoint{4.399745in}{0.413320in}}%
\pgfpathlineto{\pgfqpoint{4.397076in}{0.413320in}}%
\pgfpathlineto{\pgfqpoint{4.394400in}{0.413320in}}%
\pgfpathlineto{\pgfqpoint{4.391721in}{0.413320in}}%
\pgfpathlineto{\pgfqpoint{4.389044in}{0.413320in}}%
\pgfpathlineto{\pgfqpoint{4.386431in}{0.413320in}}%
\pgfpathlineto{\pgfqpoint{4.383674in}{0.413320in}}%
\pgfpathlineto{\pgfqpoint{4.381097in}{0.413320in}}%
\pgfpathlineto{\pgfqpoint{4.378329in}{0.413320in}}%
\pgfpathlineto{\pgfqpoint{4.375761in}{0.413320in}}%
\pgfpathlineto{\pgfqpoint{4.372976in}{0.413320in}}%
\pgfpathlineto{\pgfqpoint{4.370437in}{0.413320in}}%
\pgfpathlineto{\pgfqpoint{4.367646in}{0.413320in}}%
\pgfpathlineto{\pgfqpoint{4.364936in}{0.413320in}}%
\pgfpathlineto{\pgfqpoint{4.362270in}{0.413320in}}%
\pgfpathlineto{\pgfqpoint{4.359582in}{0.413320in}}%
\pgfpathlineto{\pgfqpoint{4.357014in}{0.413320in}}%
\pgfpathlineto{\pgfqpoint{4.354224in}{0.413320in}}%
\pgfpathlineto{\pgfqpoint{4.351645in}{0.413320in}}%
\pgfpathlineto{\pgfqpoint{4.348868in}{0.413320in}}%
\pgfpathlineto{\pgfqpoint{4.346263in}{0.413320in}}%
\pgfpathlineto{\pgfqpoint{4.343510in}{0.413320in}}%
\pgfpathlineto{\pgfqpoint{4.340976in}{0.413320in}}%
\pgfpathlineto{\pgfqpoint{4.338154in}{0.413320in}}%
\pgfpathlineto{\pgfqpoint{4.335463in}{0.413320in}}%
\pgfpathlineto{\pgfqpoint{4.332796in}{0.413320in}}%
\pgfpathlineto{\pgfqpoint{4.330118in}{0.413320in}}%
\pgfpathlineto{\pgfqpoint{4.327440in}{0.413320in}}%
\pgfpathlineto{\pgfqpoint{4.324760in}{0.413320in}}%
\pgfpathlineto{\pgfqpoint{4.322181in}{0.413320in}}%
\pgfpathlineto{\pgfqpoint{4.319405in}{0.413320in}}%
\pgfpathlineto{\pgfqpoint{4.316856in}{0.413320in}}%
\pgfpathlineto{\pgfqpoint{4.314032in}{0.413320in}}%
\pgfpathlineto{\pgfqpoint{4.311494in}{0.413320in}}%
\pgfpathlineto{\pgfqpoint{4.308691in}{0.413320in}}%
\pgfpathlineto{\pgfqpoint{4.306118in}{0.413320in}}%
\pgfpathlineto{\pgfqpoint{4.303357in}{0.413320in}}%
\pgfpathlineto{\pgfqpoint{4.300656in}{0.413320in}}%
\pgfpathlineto{\pgfqpoint{4.297977in}{0.413320in}}%
\pgfpathlineto{\pgfqpoint{4.295299in}{0.413320in}}%
\pgfpathlineto{\pgfqpoint{4.292786in}{0.413320in}}%
\pgfpathlineto{\pgfqpoint{4.289936in}{0.413320in}}%
\pgfpathlineto{\pgfqpoint{4.287399in}{0.413320in}}%
\pgfpathlineto{\pgfqpoint{4.284586in}{0.413320in}}%
\pgfpathlineto{\pgfqpoint{4.282000in}{0.413320in}}%
\pgfpathlineto{\pgfqpoint{4.279212in}{0.413320in}}%
\pgfpathlineto{\pgfqpoint{4.276635in}{0.413320in}}%
\pgfpathlineto{\pgfqpoint{4.273874in}{0.413320in}}%
\pgfpathlineto{\pgfqpoint{4.271187in}{0.413320in}}%
\pgfpathlineto{\pgfqpoint{4.268590in}{0.413320in}}%
\pgfpathlineto{\pgfqpoint{4.265824in}{0.413320in}}%
\pgfpathlineto{\pgfqpoint{4.263157in}{0.413320in}}%
\pgfpathlineto{\pgfqpoint{4.260477in}{0.413320in}}%
\pgfpathlineto{\pgfqpoint{4.257958in}{0.413320in}}%
\pgfpathlineto{\pgfqpoint{4.255120in}{0.413320in}}%
\pgfpathlineto{\pgfqpoint{4.252581in}{0.413320in}}%
\pgfpathlineto{\pgfqpoint{4.249767in}{0.413320in}}%
\pgfpathlineto{\pgfqpoint{4.247225in}{0.413320in}}%
\pgfpathlineto{\pgfqpoint{4.244394in}{0.413320in}}%
\pgfpathlineto{\pgfqpoint{4.241900in}{0.413320in}}%
\pgfpathlineto{\pgfqpoint{4.239084in}{0.413320in}}%
\pgfpathlineto{\pgfqpoint{4.236375in}{0.413320in}}%
\pgfpathlineto{\pgfqpoint{4.233691in}{0.413320in}}%
\pgfpathlineto{\pgfqpoint{4.231013in}{0.413320in}}%
\pgfpathlineto{\pgfqpoint{4.228331in}{0.413320in}}%
\pgfpathlineto{\pgfqpoint{4.225654in}{0.413320in}}%
\pgfpathlineto{\pgfqpoint{4.223082in}{0.413320in}}%
\pgfpathlineto{\pgfqpoint{4.220304in}{0.413320in}}%
\pgfpathlineto{\pgfqpoint{4.217694in}{0.413320in}}%
\pgfpathlineto{\pgfqpoint{4.214948in}{0.413320in}}%
\pgfpathlineto{\pgfqpoint{4.212383in}{0.413320in}}%
\pgfpathlineto{\pgfqpoint{4.209597in}{0.413320in}}%
\pgfpathlineto{\pgfqpoint{4.207076in}{0.413320in}}%
\pgfpathlineto{\pgfqpoint{4.204240in}{0.413320in}}%
\pgfpathlineto{\pgfqpoint{4.201542in}{0.413320in}}%
\pgfpathlineto{\pgfqpoint{4.198878in}{0.413320in}}%
\pgfpathlineto{\pgfqpoint{4.196186in}{0.413320in}}%
\pgfpathlineto{\pgfqpoint{4.193638in}{0.413320in}}%
\pgfpathlineto{\pgfqpoint{4.190842in}{0.413320in}}%
\pgfpathlineto{\pgfqpoint{4.188318in}{0.413320in}}%
\pgfpathlineto{\pgfqpoint{4.185481in}{0.413320in}}%
\pgfpathlineto{\pgfqpoint{4.182899in}{0.413320in}}%
\pgfpathlineto{\pgfqpoint{4.180129in}{0.413320in}}%
\pgfpathlineto{\pgfqpoint{4.177593in}{0.413320in}}%
\pgfpathlineto{\pgfqpoint{4.174770in}{0.413320in}}%
\pgfpathlineto{\pgfqpoint{4.172093in}{0.413320in}}%
\pgfpathlineto{\pgfqpoint{4.169415in}{0.413320in}}%
\pgfpathlineto{\pgfqpoint{4.166737in}{0.413320in}}%
\pgfpathlineto{\pgfqpoint{4.164059in}{0.413320in}}%
\pgfpathlineto{\pgfqpoint{4.161380in}{0.413320in}}%
\pgfpathlineto{\pgfqpoint{4.158806in}{0.413320in}}%
\pgfpathlineto{\pgfqpoint{4.156016in}{0.413320in}}%
\pgfpathlineto{\pgfqpoint{4.153423in}{0.413320in}}%
\pgfpathlineto{\pgfqpoint{4.150665in}{0.413320in}}%
\pgfpathlineto{\pgfqpoint{4.148082in}{0.413320in}}%
\pgfpathlineto{\pgfqpoint{4.145310in}{0.413320in}}%
\pgfpathlineto{\pgfqpoint{4.142713in}{0.413320in}}%
\pgfpathlineto{\pgfqpoint{4.139963in}{0.413320in}}%
\pgfpathlineto{\pgfqpoint{4.137272in}{0.413320in}}%
\pgfpathlineto{\pgfqpoint{4.134615in}{0.413320in}}%
\pgfpathlineto{\pgfqpoint{4.131920in}{0.413320in}}%
\pgfpathlineto{\pgfqpoint{4.129349in}{0.413320in}}%
\pgfpathlineto{\pgfqpoint{4.126553in}{0.413320in}}%
\pgfpathlineto{\pgfqpoint{4.124019in}{0.413320in}}%
\pgfpathlineto{\pgfqpoint{4.121205in}{0.413320in}}%
\pgfpathlineto{\pgfqpoint{4.118554in}{0.413320in}}%
\pgfpathlineto{\pgfqpoint{4.115844in}{0.413320in}}%
\pgfpathlineto{\pgfqpoint{4.113252in}{0.413320in}}%
\pgfpathlineto{\pgfqpoint{4.110488in}{0.413320in}}%
\pgfpathlineto{\pgfqpoint{4.107814in}{0.413320in}}%
\pgfpathlineto{\pgfqpoint{4.105185in}{0.413320in}}%
\pgfpathlineto{\pgfqpoint{4.102456in}{0.413320in}}%
\pgfpathlineto{\pgfqpoint{4.099777in}{0.413320in}}%
\pgfpathlineto{\pgfqpoint{4.097092in}{0.413320in}}%
\pgfpathlineto{\pgfqpoint{4.094527in}{0.413320in}}%
\pgfpathlineto{\pgfqpoint{4.091729in}{0.413320in}}%
\pgfpathlineto{\pgfqpoint{4.089159in}{0.413320in}}%
\pgfpathlineto{\pgfqpoint{4.086385in}{0.413320in}}%
\pgfpathlineto{\pgfqpoint{4.083870in}{0.413320in}}%
\pgfpathlineto{\pgfqpoint{4.081018in}{0.413320in}}%
\pgfpathlineto{\pgfqpoint{4.078471in}{0.413320in}}%
\pgfpathlineto{\pgfqpoint{4.075705in}{0.413320in}}%
\pgfpathlineto{\pgfqpoint{4.072985in}{0.413320in}}%
\pgfpathlineto{\pgfqpoint{4.070313in}{0.413320in}}%
\pgfpathlineto{\pgfqpoint{4.067636in}{0.413320in}}%
\pgfpathlineto{\pgfqpoint{4.064957in}{0.413320in}}%
\pgfpathlineto{\pgfqpoint{4.062266in}{0.413320in}}%
\pgfpathlineto{\pgfqpoint{4.059702in}{0.413320in}}%
\pgfpathlineto{\pgfqpoint{4.056911in}{0.413320in}}%
\pgfpathlineto{\pgfqpoint{4.054326in}{0.413320in}}%
\pgfpathlineto{\pgfqpoint{4.051557in}{0.413320in}}%
\pgfpathlineto{\pgfqpoint{4.049006in}{0.413320in}}%
\pgfpathlineto{\pgfqpoint{4.046210in}{0.413320in}}%
\pgfpathlineto{\pgfqpoint{4.043667in}{0.413320in}}%
\pgfpathlineto{\pgfqpoint{4.040852in}{0.413320in}}%
\pgfpathlineto{\pgfqpoint{4.038174in}{0.413320in}}%
\pgfpathlineto{\pgfqpoint{4.035492in}{0.413320in}}%
\pgfpathlineto{\pgfqpoint{4.032817in}{0.413320in}}%
\pgfpathlineto{\pgfqpoint{4.030229in}{0.413320in}}%
\pgfpathlineto{\pgfqpoint{4.027447in}{0.413320in}}%
\pgfpathlineto{\pgfqpoint{4.024868in}{0.413320in}}%
\pgfpathlineto{\pgfqpoint{4.022097in}{0.413320in}}%
\pgfpathlineto{\pgfqpoint{4.019518in}{0.413320in}}%
\pgfpathlineto{\pgfqpoint{4.016744in}{0.413320in}}%
\pgfpathlineto{\pgfqpoint{4.014186in}{0.413320in}}%
\pgfpathlineto{\pgfqpoint{4.011394in}{0.413320in}}%
\pgfpathlineto{\pgfqpoint{4.008699in}{0.413320in}}%
\pgfpathlineto{\pgfqpoint{4.006034in}{0.413320in}}%
\pgfpathlineto{\pgfqpoint{4.003348in}{0.413320in}}%
\pgfpathlineto{\pgfqpoint{4.000674in}{0.413320in}}%
\pgfpathlineto{\pgfqpoint{3.997990in}{0.413320in}}%
\pgfpathlineto{\pgfqpoint{3.995417in}{0.413320in}}%
\pgfpathlineto{\pgfqpoint{3.992642in}{0.413320in}}%
\pgfpathlineto{\pgfqpoint{3.990055in}{0.413320in}}%
\pgfpathlineto{\pgfqpoint{3.987270in}{0.413320in}}%
\pgfpathlineto{\pgfqpoint{3.984714in}{0.413320in}}%
\pgfpathlineto{\pgfqpoint{3.981929in}{0.413320in}}%
\pgfpathlineto{\pgfqpoint{3.979389in}{0.413320in}}%
\pgfpathlineto{\pgfqpoint{3.976563in}{0.413320in}}%
\pgfpathlineto{\pgfqpoint{3.973885in}{0.413320in}}%
\pgfpathlineto{\pgfqpoint{3.971250in}{0.413320in}}%
\pgfpathlineto{\pgfqpoint{3.968523in}{0.413320in}}%
\pgfpathlineto{\pgfqpoint{3.966013in}{0.413320in}}%
\pgfpathlineto{\pgfqpoint{3.963176in}{0.413320in}}%
\pgfpathlineto{\pgfqpoint{3.960635in}{0.413320in}}%
\pgfpathlineto{\pgfqpoint{3.957823in}{0.413320in}}%
\pgfpathlineto{\pgfqpoint{3.955211in}{0.413320in}}%
\pgfpathlineto{\pgfqpoint{3.952464in}{0.413320in}}%
\pgfpathlineto{\pgfqpoint{3.949894in}{0.413320in}}%
\pgfpathlineto{\pgfqpoint{3.947101in}{0.413320in}}%
\pgfpathlineto{\pgfqpoint{3.944431in}{0.413320in}}%
\pgfpathlineto{\pgfqpoint{3.941778in}{0.413320in}}%
\pgfpathlineto{\pgfqpoint{3.939075in}{0.413320in}}%
\pgfpathlineto{\pgfqpoint{3.936395in}{0.413320in}}%
\pgfpathlineto{\pgfqpoint{3.933714in}{0.413320in}}%
\pgfpathlineto{\pgfqpoint{3.931202in}{0.413320in}}%
\pgfpathlineto{\pgfqpoint{3.928347in}{0.413320in}}%
\pgfpathlineto{\pgfqpoint{3.925778in}{0.413320in}}%
\pgfpathlineto{\pgfqpoint{3.923005in}{0.413320in}}%
\pgfpathlineto{\pgfqpoint{3.920412in}{0.413320in}}%
\pgfpathlineto{\pgfqpoint{3.917646in}{0.413320in}}%
\pgfpathlineto{\pgfqpoint{3.915107in}{0.413320in}}%
\pgfpathlineto{\pgfqpoint{3.912296in}{0.413320in}}%
\pgfpathlineto{\pgfqpoint{3.909602in}{0.413320in}}%
\pgfpathlineto{\pgfqpoint{3.906918in}{0.413320in}}%
\pgfpathlineto{\pgfqpoint{3.904252in}{0.413320in}}%
\pgfpathlineto{\pgfqpoint{3.901573in}{0.413320in}}%
\pgfpathlineto{\pgfqpoint{3.898891in}{0.413320in}}%
\pgfpathlineto{\pgfqpoint{3.896345in}{0.413320in}}%
\pgfpathlineto{\pgfqpoint{3.893541in}{0.413320in}}%
\pgfpathlineto{\pgfqpoint{3.890926in}{0.413320in}}%
\pgfpathlineto{\pgfqpoint{3.888188in}{0.413320in}}%
\pgfpathlineto{\pgfqpoint{3.885621in}{0.413320in}}%
\pgfpathlineto{\pgfqpoint{3.882850in}{0.413320in}}%
\pgfpathlineto{\pgfqpoint{3.880237in}{0.413320in}}%
\pgfpathlineto{\pgfqpoint{3.877466in}{0.413320in}}%
\pgfpathlineto{\pgfqpoint{3.874790in}{0.413320in}}%
\pgfpathlineto{\pgfqpoint{3.872114in}{0.413320in}}%
\pgfpathlineto{\pgfqpoint{3.869435in}{0.413320in}}%
\pgfpathlineto{\pgfqpoint{3.866815in}{0.413320in}}%
\pgfpathlineto{\pgfqpoint{3.864073in}{0.413320in}}%
\pgfpathlineto{\pgfqpoint{3.861561in}{0.413320in}}%
\pgfpathlineto{\pgfqpoint{3.858720in}{0.413320in}}%
\pgfpathlineto{\pgfqpoint{3.856100in}{0.413320in}}%
\pgfpathlineto{\pgfqpoint{3.853358in}{0.413320in}}%
\pgfpathlineto{\pgfqpoint{3.850814in}{0.413320in}}%
\pgfpathlineto{\pgfqpoint{3.848005in}{0.413320in}}%
\pgfpathlineto{\pgfqpoint{3.845329in}{0.413320in}}%
\pgfpathlineto{\pgfqpoint{3.842641in}{0.413320in}}%
\pgfpathlineto{\pgfqpoint{3.839960in}{0.413320in}}%
\pgfpathlineto{\pgfqpoint{3.837286in}{0.413320in}}%
\pgfpathlineto{\pgfqpoint{3.834616in}{0.413320in}}%
\pgfpathlineto{\pgfqpoint{3.832053in}{0.413320in}}%
\pgfpathlineto{\pgfqpoint{3.829252in}{0.413320in}}%
\pgfpathlineto{\pgfqpoint{3.826679in}{0.413320in}}%
\pgfpathlineto{\pgfqpoint{3.823903in}{0.413320in}}%
\pgfpathlineto{\pgfqpoint{3.821315in}{0.413320in}}%
\pgfpathlineto{\pgfqpoint{3.818546in}{0.413320in}}%
\pgfpathlineto{\pgfqpoint{3.815983in}{0.413320in}}%
\pgfpathlineto{\pgfqpoint{3.813172in}{0.413320in}}%
\pgfpathlineto{\pgfqpoint{3.810510in}{0.413320in}}%
\pgfpathlineto{\pgfqpoint{3.807832in}{0.413320in}}%
\pgfpathlineto{\pgfqpoint{3.805145in}{0.413320in}}%
\pgfpathlineto{\pgfqpoint{3.802569in}{0.413320in}}%
\pgfpathlineto{\pgfqpoint{3.799797in}{0.413320in}}%
\pgfpathlineto{\pgfqpoint{3.797265in}{0.413320in}}%
\pgfpathlineto{\pgfqpoint{3.794435in}{0.413320in}}%
\pgfpathlineto{\pgfqpoint{3.791897in}{0.413320in}}%
\pgfpathlineto{\pgfqpoint{3.789084in}{0.413320in}}%
\pgfpathlineto{\pgfqpoint{3.786504in}{0.413320in}}%
\pgfpathlineto{\pgfqpoint{3.783725in}{0.413320in}}%
\pgfpathlineto{\pgfqpoint{3.781046in}{0.413320in}}%
\pgfpathlineto{\pgfqpoint{3.778370in}{0.413320in}}%
\pgfpathlineto{\pgfqpoint{3.775691in}{0.413320in}}%
\pgfpathlineto{\pgfqpoint{3.773014in}{0.413320in}}%
\pgfpathlineto{\pgfqpoint{3.770323in}{0.413320in}}%
\pgfpathlineto{\pgfqpoint{3.767782in}{0.413320in}}%
\pgfpathlineto{\pgfqpoint{3.764966in}{0.413320in}}%
\pgfpathlineto{\pgfqpoint{3.762389in}{0.413320in}}%
\pgfpathlineto{\pgfqpoint{3.759622in}{0.413320in}}%
\pgfpathlineto{\pgfqpoint{3.757065in}{0.413320in}}%
\pgfpathlineto{\pgfqpoint{3.754265in}{0.413320in}}%
\pgfpathlineto{\pgfqpoint{3.751728in}{0.413320in}}%
\pgfpathlineto{\pgfqpoint{3.748903in}{0.413320in}}%
\pgfpathlineto{\pgfqpoint{3.746229in}{0.413320in}}%
\pgfpathlineto{\pgfqpoint{3.743548in}{0.413320in}}%
\pgfpathlineto{\pgfqpoint{3.740874in}{0.413320in}}%
\pgfpathlineto{\pgfqpoint{3.738194in}{0.413320in}}%
\pgfpathlineto{\pgfqpoint{3.735509in}{0.413320in}}%
\pgfpathlineto{\pgfqpoint{3.732950in}{0.413320in}}%
\pgfpathlineto{\pgfqpoint{3.730158in}{0.413320in}}%
\pgfpathlineto{\pgfqpoint{3.727581in}{0.413320in}}%
\pgfpathlineto{\pgfqpoint{3.724804in}{0.413320in}}%
\pgfpathlineto{\pgfqpoint{3.722228in}{0.413320in}}%
\pgfpathlineto{\pgfqpoint{3.719446in}{0.413320in}}%
\pgfpathlineto{\pgfqpoint{3.716875in}{0.413320in}}%
\pgfpathlineto{\pgfqpoint{3.714086in}{0.413320in}}%
\pgfpathlineto{\pgfqpoint{3.711410in}{0.413320in}}%
\pgfpathlineto{\pgfqpoint{3.708729in}{0.413320in}}%
\pgfpathlineto{\pgfqpoint{3.706053in}{0.413320in}}%
\pgfpathlineto{\pgfqpoint{3.703460in}{0.413320in}}%
\pgfpathlineto{\pgfqpoint{3.700684in}{0.413320in}}%
\pgfpathlineto{\pgfqpoint{3.698125in}{0.413320in}}%
\pgfpathlineto{\pgfqpoint{3.695331in}{0.413320in}}%
\pgfpathlineto{\pgfqpoint{3.692765in}{0.413320in}}%
\pgfpathlineto{\pgfqpoint{3.689983in}{0.413320in}}%
\pgfpathlineto{\pgfqpoint{3.687442in}{0.413320in}}%
\pgfpathlineto{\pgfqpoint{3.684620in}{0.413320in}}%
\pgfpathlineto{\pgfqpoint{3.681948in}{0.413320in}}%
\pgfpathlineto{\pgfqpoint{3.679273in}{0.413320in}}%
\pgfpathlineto{\pgfqpoint{3.676591in}{0.413320in}}%
\pgfpathlineto{\pgfqpoint{3.673911in}{0.413320in}}%
\pgfpathlineto{\pgfqpoint{3.671232in}{0.413320in}}%
\pgfpathlineto{\pgfqpoint{3.668665in}{0.413320in}}%
\pgfpathlineto{\pgfqpoint{3.665864in}{0.413320in}}%
\pgfpathlineto{\pgfqpoint{3.663276in}{0.413320in}}%
\pgfpathlineto{\pgfqpoint{3.660515in}{0.413320in}}%
\pgfpathlineto{\pgfqpoint{3.657917in}{0.413320in}}%
\pgfpathlineto{\pgfqpoint{3.655165in}{0.413320in}}%
\pgfpathlineto{\pgfqpoint{3.652628in}{0.413320in}}%
\pgfpathlineto{\pgfqpoint{3.649837in}{0.413320in}}%
\pgfpathlineto{\pgfqpoint{3.647130in}{0.413320in}}%
\pgfpathlineto{\pgfqpoint{3.644452in}{0.413320in}}%
\pgfpathlineto{\pgfqpoint{3.641773in}{0.413320in}}%
\pgfpathlineto{\pgfqpoint{3.639207in}{0.413320in}}%
\pgfpathlineto{\pgfqpoint{3.636413in}{0.413320in}}%
\pgfpathlineto{\pgfqpoint{3.633858in}{0.413320in}}%
\pgfpathlineto{\pgfqpoint{3.631058in}{0.413320in}}%
\pgfpathlineto{\pgfqpoint{3.628460in}{0.413320in}}%
\pgfpathlineto{\pgfqpoint{3.625689in}{0.413320in}}%
\pgfpathlineto{\pgfqpoint{3.623165in}{0.413320in}}%
\pgfpathlineto{\pgfqpoint{3.620345in}{0.413320in}}%
\pgfpathlineto{\pgfqpoint{3.617667in}{0.413320in}}%
\pgfpathlineto{\pgfqpoint{3.614982in}{0.413320in}}%
\pgfpathlineto{\pgfqpoint{3.612311in}{0.413320in}}%
\pgfpathlineto{\pgfqpoint{3.609632in}{0.413320in}}%
\pgfpathlineto{\pgfqpoint{3.606951in}{0.413320in}}%
\pgfpathlineto{\pgfqpoint{3.604387in}{0.413320in}}%
\pgfpathlineto{\pgfqpoint{3.601590in}{0.413320in}}%
\pgfpathlineto{\pgfqpoint{3.598998in}{0.413320in}}%
\pgfpathlineto{\pgfqpoint{3.596240in}{0.413320in}}%
\pgfpathlineto{\pgfqpoint{3.593620in}{0.413320in}}%
\pgfpathlineto{\pgfqpoint{3.590883in}{0.413320in}}%
\pgfpathlineto{\pgfqpoint{3.588258in}{0.413320in}}%
\pgfpathlineto{\pgfqpoint{3.585532in}{0.413320in}}%
\pgfpathlineto{\pgfqpoint{3.582851in}{0.413320in}}%
\pgfpathlineto{\pgfqpoint{3.580191in}{0.413320in}}%
\pgfpathlineto{\pgfqpoint{3.577487in}{0.413320in}}%
\pgfpathlineto{\pgfqpoint{3.574814in}{0.413320in}}%
\pgfpathlineto{\pgfqpoint{3.572126in}{0.413320in}}%
\pgfpathlineto{\pgfqpoint{3.569584in}{0.413320in}}%
\pgfpathlineto{\pgfqpoint{3.566774in}{0.413320in}}%
\pgfpathlineto{\pgfqpoint{3.564188in}{0.413320in}}%
\pgfpathlineto{\pgfqpoint{3.561420in}{0.413320in}}%
\pgfpathlineto{\pgfqpoint{3.558853in}{0.413320in}}%
\pgfpathlineto{\pgfqpoint{3.556061in}{0.413320in}}%
\pgfpathlineto{\pgfqpoint{3.553498in}{0.413320in}}%
\pgfpathlineto{\pgfqpoint{3.550713in}{0.413320in}}%
\pgfpathlineto{\pgfqpoint{3.548029in}{0.413320in}}%
\pgfpathlineto{\pgfqpoint{3.545349in}{0.413320in}}%
\pgfpathlineto{\pgfqpoint{3.542656in}{0.413320in}}%
\pgfpathlineto{\pgfqpoint{3.540093in}{0.413320in}}%
\pgfpathlineto{\pgfqpoint{3.537309in}{0.413320in}}%
\pgfpathlineto{\pgfqpoint{3.534783in}{0.413320in}}%
\pgfpathlineto{\pgfqpoint{3.531955in}{0.413320in}}%
\pgfpathlineto{\pgfqpoint{3.529327in}{0.413320in}}%
\pgfpathlineto{\pgfqpoint{3.526601in}{0.413320in}}%
\pgfpathlineto{\pgfqpoint{3.524041in}{0.413320in}}%
\pgfpathlineto{\pgfqpoint{3.521244in}{0.413320in}}%
\pgfpathlineto{\pgfqpoint{3.518565in}{0.413320in}}%
\pgfpathlineto{\pgfqpoint{3.515884in}{0.413320in}}%
\pgfpathlineto{\pgfqpoint{3.513209in}{0.413320in}}%
\pgfpathlineto{\pgfqpoint{3.510533in}{0.413320in}}%
\pgfpathlineto{\pgfqpoint{3.507840in}{0.413320in}}%
\pgfpathlineto{\pgfqpoint{3.505262in}{0.413320in}}%
\pgfpathlineto{\pgfqpoint{3.502488in}{0.413320in}}%
\pgfpathlineto{\pgfqpoint{3.499909in}{0.413320in}}%
\pgfpathlineto{\pgfqpoint{3.497139in}{0.413320in}}%
\pgfpathlineto{\pgfqpoint{3.494581in}{0.413320in}}%
\pgfpathlineto{\pgfqpoint{3.491783in}{0.413320in}}%
\pgfpathlineto{\pgfqpoint{3.489223in}{0.413320in}}%
\pgfpathlineto{\pgfqpoint{3.486442in}{0.413320in}}%
\pgfpathlineto{\pgfqpoint{3.483744in}{0.413320in}}%
\pgfpathlineto{\pgfqpoint{3.481072in}{0.413320in}}%
\pgfpathlineto{\pgfqpoint{3.478378in}{0.413320in}}%
\pgfpathlineto{\pgfqpoint{3.475821in}{0.413320in}}%
\pgfpathlineto{\pgfqpoint{3.473021in}{0.413320in}}%
\pgfpathlineto{\pgfqpoint{3.470466in}{0.413320in}}%
\pgfpathlineto{\pgfqpoint{3.467678in}{0.413320in}}%
\pgfpathlineto{\pgfqpoint{3.465072in}{0.413320in}}%
\pgfpathlineto{\pgfqpoint{3.462321in}{0.413320in}}%
\pgfpathlineto{\pgfqpoint{3.459695in}{0.413320in}}%
\pgfpathlineto{\pgfqpoint{3.456960in}{0.413320in}}%
\pgfpathlineto{\pgfqpoint{3.454285in}{0.413320in}}%
\pgfpathlineto{\pgfqpoint{3.451597in}{0.413320in}}%
\pgfpathlineto{\pgfqpoint{3.448926in}{0.413320in}}%
\pgfpathlineto{\pgfqpoint{3.446257in}{0.413320in}}%
\pgfpathlineto{\pgfqpoint{3.443574in}{0.413320in}}%
\pgfpathlineto{\pgfqpoint{3.440996in}{0.413320in}}%
\pgfpathlineto{\pgfqpoint{3.438210in}{0.413320in}}%
\pgfpathlineto{\pgfqpoint{3.435635in}{0.413320in}}%
\pgfpathlineto{\pgfqpoint{3.432851in}{0.413320in}}%
\pgfpathlineto{\pgfqpoint{3.430313in}{0.413320in}}%
\pgfpathlineto{\pgfqpoint{3.427501in}{0.413320in}}%
\pgfpathlineto{\pgfqpoint{3.424887in}{0.413320in}}%
\pgfpathlineto{\pgfqpoint{3.422142in}{0.413320in}}%
\pgfpathlineto{\pgfqpoint{3.419455in}{0.413320in}}%
\pgfpathlineto{\pgfqpoint{3.416780in}{0.413320in}}%
\pgfpathlineto{\pgfqpoint{3.414109in}{0.413320in}}%
\pgfpathlineto{\pgfqpoint{3.411431in}{0.413320in}}%
\pgfpathlineto{\pgfqpoint{3.408752in}{0.413320in}}%
\pgfpathlineto{\pgfqpoint{3.406202in}{0.413320in}}%
\pgfpathlineto{\pgfqpoint{3.403394in}{0.413320in}}%
\pgfpathlineto{\pgfqpoint{3.400783in}{0.413320in}}%
\pgfpathlineto{\pgfqpoint{3.398037in}{0.413320in}}%
\pgfpathlineto{\pgfqpoint{3.395461in}{0.413320in}}%
\pgfpathlineto{\pgfqpoint{3.392681in}{0.413320in}}%
\pgfpathlineto{\pgfqpoint{3.390102in}{0.413320in}}%
\pgfpathlineto{\pgfqpoint{3.387309in}{0.413320in}}%
\pgfpathlineto{\pgfqpoint{3.384647in}{0.413320in}}%
\pgfpathlineto{\pgfqpoint{3.381959in}{0.413320in}}%
\pgfpathlineto{\pgfqpoint{3.379290in}{0.413320in}}%
\pgfpathlineto{\pgfqpoint{3.376735in}{0.413320in}}%
\pgfpathlineto{\pgfqpoint{3.373921in}{0.413320in}}%
\pgfpathlineto{\pgfqpoint{3.371357in}{0.413320in}}%
\pgfpathlineto{\pgfqpoint{3.368577in}{0.413320in}}%
\pgfpathlineto{\pgfqpoint{3.365996in}{0.413320in}}%
\pgfpathlineto{\pgfqpoint{3.363221in}{0.413320in}}%
\pgfpathlineto{\pgfqpoint{3.360620in}{0.413320in}}%
\pgfpathlineto{\pgfqpoint{3.357862in}{0.413320in}}%
\pgfpathlineto{\pgfqpoint{3.355177in}{0.413320in}}%
\pgfpathlineto{\pgfqpoint{3.352505in}{0.413320in}}%
\pgfpathlineto{\pgfqpoint{3.349828in}{0.413320in}}%
\pgfpathlineto{\pgfqpoint{3.347139in}{0.413320in}}%
\pgfpathlineto{\pgfqpoint{3.344468in}{0.413320in}}%
\pgfpathlineto{\pgfqpoint{3.341893in}{0.413320in}}%
\pgfpathlineto{\pgfqpoint{3.339101in}{0.413320in}}%
\pgfpathlineto{\pgfqpoint{3.336541in}{0.413320in}}%
\pgfpathlineto{\pgfqpoint{3.333758in}{0.413320in}}%
\pgfpathlineto{\pgfqpoint{3.331183in}{0.413320in}}%
\pgfpathlineto{\pgfqpoint{3.328401in}{0.413320in}}%
\pgfpathlineto{\pgfqpoint{3.325860in}{0.413320in}}%
\pgfpathlineto{\pgfqpoint{3.323049in}{0.413320in}}%
\pgfpathlineto{\pgfqpoint{3.320366in}{0.413320in}}%
\pgfpathlineto{\pgfqpoint{3.317688in}{0.413320in}}%
\pgfpathlineto{\pgfqpoint{3.315008in}{0.413320in}}%
\pgfpathlineto{\pgfqpoint{3.312480in}{0.413320in}}%
\pgfpathlineto{\pgfqpoint{3.309652in}{0.413320in}}%
\pgfpathlineto{\pgfqpoint{3.307104in}{0.413320in}}%
\pgfpathlineto{\pgfqpoint{3.304295in}{0.413320in}}%
\pgfpathlineto{\pgfqpoint{3.301719in}{0.413320in}}%
\pgfpathlineto{\pgfqpoint{3.298937in}{0.413320in}}%
\pgfpathlineto{\pgfqpoint{3.296376in}{0.413320in}}%
\pgfpathlineto{\pgfqpoint{3.293574in}{0.413320in}}%
\pgfpathlineto{\pgfqpoint{3.290890in}{0.413320in}}%
\pgfpathlineto{\pgfqpoint{3.288225in}{0.413320in}}%
\pgfpathlineto{\pgfqpoint{3.285534in}{0.413320in}}%
\pgfpathlineto{\pgfqpoint{3.282870in}{0.413320in}}%
\pgfpathlineto{\pgfqpoint{3.280189in}{0.413320in}}%
\pgfpathlineto{\pgfqpoint{3.277603in}{0.413320in}}%
\pgfpathlineto{\pgfqpoint{3.274831in}{0.413320in}}%
\pgfpathlineto{\pgfqpoint{3.272254in}{0.413320in}}%
\pgfpathlineto{\pgfqpoint{3.269478in}{0.413320in}}%
\pgfpathlineto{\pgfqpoint{3.266849in}{0.413320in}}%
\pgfpathlineto{\pgfqpoint{3.264119in}{0.413320in}}%
\pgfpathlineto{\pgfqpoint{3.261594in}{0.413320in}}%
\pgfpathlineto{\pgfqpoint{3.258784in}{0.413320in}}%
\pgfpathlineto{\pgfqpoint{3.256083in}{0.413320in}}%
\pgfpathlineto{\pgfqpoint{3.253404in}{0.413320in}}%
\pgfpathlineto{\pgfqpoint{3.250716in}{0.413320in}}%
\pgfpathlineto{\pgfqpoint{3.248049in}{0.413320in}}%
\pgfpathlineto{\pgfqpoint{3.245363in}{0.413320in}}%
\pgfpathlineto{\pgfqpoint{3.242807in}{0.413320in}}%
\pgfpathlineto{\pgfqpoint{3.240010in}{0.413320in}}%
\pgfpathlineto{\pgfqpoint{3.237411in}{0.413320in}}%
\pgfpathlineto{\pgfqpoint{3.234658in}{0.413320in}}%
\pgfpathlineto{\pgfqpoint{3.232069in}{0.413320in}}%
\pgfpathlineto{\pgfqpoint{3.229310in}{0.413320in}}%
\pgfpathlineto{\pgfqpoint{3.226609in}{0.413320in}}%
\pgfpathlineto{\pgfqpoint{3.223942in}{0.413320in}}%
\pgfpathlineto{\pgfqpoint{3.221255in}{0.413320in}}%
\pgfpathlineto{\pgfqpoint{3.218586in}{0.413320in}}%
\pgfpathlineto{\pgfqpoint{3.215908in}{0.413320in}}%
\pgfpathlineto{\pgfqpoint{3.213342in}{0.413320in}}%
\pgfpathlineto{\pgfqpoint{3.210545in}{0.413320in}}%
\pgfpathlineto{\pgfqpoint{3.207984in}{0.413320in}}%
\pgfpathlineto{\pgfqpoint{3.205195in}{0.413320in}}%
\pgfpathlineto{\pgfqpoint{3.202562in}{0.413320in}}%
\pgfpathlineto{\pgfqpoint{3.199823in}{0.413320in}}%
\pgfpathlineto{\pgfqpoint{3.197226in}{0.413320in}}%
\pgfpathlineto{\pgfqpoint{3.194508in}{0.413320in}}%
\pgfpathlineto{\pgfqpoint{3.191796in}{0.413320in}}%
\pgfpathlineto{\pgfqpoint{3.189117in}{0.413320in}}%
\pgfpathlineto{\pgfqpoint{3.186440in}{0.413320in}}%
\pgfpathlineto{\pgfqpoint{3.183760in}{0.413320in}}%
\pgfpathlineto{\pgfqpoint{3.181089in}{0.413320in}}%
\pgfpathlineto{\pgfqpoint{3.178525in}{0.413320in}}%
\pgfpathlineto{\pgfqpoint{3.175724in}{0.413320in}}%
\pgfpathlineto{\pgfqpoint{3.173142in}{0.413320in}}%
\pgfpathlineto{\pgfqpoint{3.170375in}{0.413320in}}%
\pgfpathlineto{\pgfqpoint{3.167776in}{0.413320in}}%
\pgfpathlineto{\pgfqpoint{3.165019in}{0.413320in}}%
\pgfpathlineto{\pgfqpoint{3.162474in}{0.413320in}}%
\pgfpathlineto{\pgfqpoint{3.159675in}{0.413320in}}%
\pgfpathlineto{\pgfqpoint{3.156981in}{0.413320in}}%
\pgfpathlineto{\pgfqpoint{3.154327in}{0.413320in}}%
\pgfpathlineto{\pgfqpoint{3.151612in}{0.413320in}}%
\pgfpathlineto{\pgfqpoint{3.149057in}{0.413320in}}%
\pgfpathlineto{\pgfqpoint{3.146271in}{0.413320in}}%
\pgfpathlineto{\pgfqpoint{3.143740in}{0.413320in}}%
\pgfpathlineto{\pgfqpoint{3.140913in}{0.413320in}}%
\pgfpathlineto{\pgfqpoint{3.138375in}{0.413320in}}%
\pgfpathlineto{\pgfqpoint{3.135550in}{0.413320in}}%
\pgfpathlineto{\pgfqpoint{3.132946in}{0.413320in}}%
\pgfpathlineto{\pgfqpoint{3.130199in}{0.413320in}}%
\pgfpathlineto{\pgfqpoint{3.127512in}{0.413320in}}%
\pgfpathlineto{\pgfqpoint{3.124842in}{0.413320in}}%
\pgfpathlineto{\pgfqpoint{3.122163in}{0.413320in}}%
\pgfpathlineto{\pgfqpoint{3.119487in}{0.413320in}}%
\pgfpathlineto{\pgfqpoint{3.116807in}{0.413320in}}%
\pgfpathlineto{\pgfqpoint{3.114242in}{0.413320in}}%
\pgfpathlineto{\pgfqpoint{3.111451in}{0.413320in}}%
\pgfpathlineto{\pgfqpoint{3.108896in}{0.413320in}}%
\pgfpathlineto{\pgfqpoint{3.106094in}{0.413320in}}%
\pgfpathlineto{\pgfqpoint{3.103508in}{0.413320in}}%
\pgfpathlineto{\pgfqpoint{3.100737in}{0.413320in}}%
\pgfpathlineto{\pgfqpoint{3.098163in}{0.413320in}}%
\pgfpathlineto{\pgfqpoint{3.095388in}{0.413320in}}%
\pgfpathlineto{\pgfqpoint{3.092699in}{0.413320in}}%
\pgfpathlineto{\pgfqpoint{3.090023in}{0.413320in}}%
\pgfpathlineto{\pgfqpoint{3.087343in}{0.413320in}}%
\pgfpathlineto{\pgfqpoint{3.084671in}{0.413320in}}%
\pgfpathlineto{\pgfqpoint{3.081990in}{0.413320in}}%
\pgfpathlineto{\pgfqpoint{3.079381in}{0.413320in}}%
\pgfpathlineto{\pgfqpoint{3.076631in}{0.413320in}}%
\pgfpathlineto{\pgfqpoint{3.074056in}{0.413320in}}%
\pgfpathlineto{\pgfqpoint{3.071266in}{0.413320in}}%
\pgfpathlineto{\pgfqpoint{3.068709in}{0.413320in}}%
\pgfpathlineto{\pgfqpoint{3.065916in}{0.413320in}}%
\pgfpathlineto{\pgfqpoint{3.063230in}{0.413320in}}%
\pgfpathlineto{\pgfqpoint{3.060561in}{0.413320in}}%
\pgfpathlineto{\pgfqpoint{3.057884in}{0.413320in}}%
\pgfpathlineto{\pgfqpoint{3.055202in}{0.413320in}}%
\pgfpathlineto{\pgfqpoint{3.052526in}{0.413320in}}%
\pgfpathlineto{\pgfqpoint{3.049988in}{0.413320in}}%
\pgfpathlineto{\pgfqpoint{3.047157in}{0.413320in}}%
\pgfpathlineto{\pgfqpoint{3.044568in}{0.413320in}}%
\pgfpathlineto{\pgfqpoint{3.041813in}{0.413320in}}%
\pgfpathlineto{\pgfqpoint{3.039262in}{0.413320in}}%
\pgfpathlineto{\pgfqpoint{3.036456in}{0.413320in}}%
\pgfpathlineto{\pgfqpoint{3.033921in}{0.413320in}}%
\pgfpathlineto{\pgfqpoint{3.031091in}{0.413320in}}%
\pgfpathlineto{\pgfqpoint{3.028412in}{0.413320in}}%
\pgfpathlineto{\pgfqpoint{3.025803in}{0.413320in}}%
\pgfpathlineto{\pgfqpoint{3.023058in}{0.413320in}}%
\pgfpathlineto{\pgfqpoint{3.020382in}{0.413320in}}%
\pgfpathlineto{\pgfqpoint{3.017707in}{0.413320in}}%
\pgfpathlineto{\pgfqpoint{3.015097in}{0.413320in}}%
\pgfpathlineto{\pgfqpoint{3.012351in}{0.413320in}}%
\pgfpathlineto{\pgfqpoint{3.009784in}{0.413320in}}%
\pgfpathlineto{\pgfqpoint{3.006993in}{0.413320in}}%
\pgfpathlineto{\pgfqpoint{3.004419in}{0.413320in}}%
\pgfpathlineto{\pgfqpoint{3.001635in}{0.413320in}}%
\pgfpathlineto{\pgfqpoint{2.999103in}{0.413320in}}%
\pgfpathlineto{\pgfqpoint{2.996300in}{0.413320in}}%
\pgfpathlineto{\pgfqpoint{2.993595in}{0.413320in}}%
\pgfpathlineto{\pgfqpoint{2.990978in}{0.413320in}}%
\pgfpathlineto{\pgfqpoint{2.988238in}{0.413320in}}%
\pgfpathlineto{\pgfqpoint{2.985666in}{0.413320in}}%
\pgfpathlineto{\pgfqpoint{2.982885in}{0.413320in}}%
\pgfpathlineto{\pgfqpoint{2.980341in}{0.413320in}}%
\pgfpathlineto{\pgfqpoint{2.977517in}{0.413320in}}%
\pgfpathlineto{\pgfqpoint{2.974972in}{0.413320in}}%
\pgfpathlineto{\pgfqpoint{2.972177in}{0.413320in}}%
\pgfpathlineto{\pgfqpoint{2.969599in}{0.413320in}}%
\pgfpathlineto{\pgfqpoint{2.966812in}{0.413320in}}%
\pgfpathlineto{\pgfqpoint{2.964127in}{0.413320in}}%
\pgfpathlineto{\pgfqpoint{2.961460in}{0.413320in}}%
\pgfpathlineto{\pgfqpoint{2.958782in}{0.413320in}}%
\pgfpathlineto{\pgfqpoint{2.956103in}{0.413320in}}%
\pgfpathlineto{\pgfqpoint{2.953422in}{0.413320in}}%
\pgfpathlineto{\pgfqpoint{2.950884in}{0.413320in}}%
\pgfpathlineto{\pgfqpoint{2.948068in}{0.413320in}}%
\pgfpathlineto{\pgfqpoint{2.945461in}{0.413320in}}%
\pgfpathlineto{\pgfqpoint{2.942711in}{0.413320in}}%
\pgfpathlineto{\pgfqpoint{2.940120in}{0.413320in}}%
\pgfpathlineto{\pgfqpoint{2.937352in}{0.413320in}}%
\pgfpathlineto{\pgfqpoint{2.934759in}{0.413320in}}%
\pgfpathlineto{\pgfqpoint{2.932033in}{0.413320in}}%
\pgfpathlineto{\pgfqpoint{2.929321in}{0.413320in}}%
\pgfpathlineto{\pgfqpoint{2.926655in}{0.413320in}}%
\pgfpathlineto{\pgfqpoint{2.923963in}{0.413320in}}%
\pgfpathlineto{\pgfqpoint{2.921363in}{0.413320in}}%
\pgfpathlineto{\pgfqpoint{2.918606in}{0.413320in}}%
\pgfpathlineto{\pgfqpoint{2.916061in}{0.413320in}}%
\pgfpathlineto{\pgfqpoint{2.913243in}{0.413320in}}%
\pgfpathlineto{\pgfqpoint{2.910631in}{0.413320in}}%
\pgfpathlineto{\pgfqpoint{2.907882in}{0.413320in}}%
\pgfpathlineto{\pgfqpoint{2.905341in}{0.413320in}}%
\pgfpathlineto{\pgfqpoint{2.902535in}{0.413320in}}%
\pgfpathlineto{\pgfqpoint{2.899858in}{0.413320in}}%
\pgfpathlineto{\pgfqpoint{2.897179in}{0.413320in}}%
\pgfpathlineto{\pgfqpoint{2.894487in}{0.413320in}}%
\pgfpathlineto{\pgfqpoint{2.891809in}{0.413320in}}%
\pgfpathlineto{\pgfqpoint{2.889145in}{0.413320in}}%
\pgfpathlineto{\pgfqpoint{2.886578in}{0.413320in}}%
\pgfpathlineto{\pgfqpoint{2.883780in}{0.413320in}}%
\pgfpathlineto{\pgfqpoint{2.881254in}{0.413320in}}%
\pgfpathlineto{\pgfqpoint{2.878431in}{0.413320in}}%
\pgfpathlineto{\pgfqpoint{2.875882in}{0.413320in}}%
\pgfpathlineto{\pgfqpoint{2.873074in}{0.413320in}}%
\pgfpathlineto{\pgfqpoint{2.870475in}{0.413320in}}%
\pgfpathlineto{\pgfqpoint{2.867713in}{0.413320in}}%
\pgfpathlineto{\pgfqpoint{2.865031in}{0.413320in}}%
\pgfpathlineto{\pgfqpoint{2.862402in}{0.413320in}}%
\pgfpathlineto{\pgfqpoint{2.859668in}{0.413320in}}%
\pgfpathlineto{\pgfqpoint{2.857003in}{0.413320in}}%
\pgfpathlineto{\pgfqpoint{2.854325in}{0.413320in}}%
\pgfpathlineto{\pgfqpoint{2.851793in}{0.413320in}}%
\pgfpathlineto{\pgfqpoint{2.848960in}{0.413320in}}%
\pgfpathlineto{\pgfqpoint{2.846408in}{0.413320in}}%
\pgfpathlineto{\pgfqpoint{2.843611in}{0.413320in}}%
\pgfpathlineto{\pgfqpoint{2.841055in}{0.413320in}}%
\pgfpathlineto{\pgfqpoint{2.838254in}{0.413320in}}%
\pgfpathlineto{\pgfqpoint{2.835698in}{0.413320in}}%
\pgfpathlineto{\pgfqpoint{2.832894in}{0.413320in}}%
\pgfpathlineto{\pgfqpoint{2.830219in}{0.413320in}}%
\pgfpathlineto{\pgfqpoint{2.827567in}{0.413320in}}%
\pgfpathlineto{\pgfqpoint{2.824851in}{0.413320in}}%
\pgfpathlineto{\pgfqpoint{2.822303in}{0.413320in}}%
\pgfpathlineto{\pgfqpoint{2.819506in}{0.413320in}}%
\pgfpathlineto{\pgfqpoint{2.816867in}{0.413320in}}%
\pgfpathlineto{\pgfqpoint{2.814141in}{0.413320in}}%
\pgfpathlineto{\pgfqpoint{2.811597in}{0.413320in}}%
\pgfpathlineto{\pgfqpoint{2.808792in}{0.413320in}}%
\pgfpathlineto{\pgfqpoint{2.806175in}{0.413320in}}%
\pgfpathlineto{\pgfqpoint{2.803435in}{0.413320in}}%
\pgfpathlineto{\pgfqpoint{2.800756in}{0.413320in}}%
\pgfpathlineto{\pgfqpoint{2.798070in}{0.413320in}}%
\pgfpathlineto{\pgfqpoint{2.795398in}{0.413320in}}%
\pgfpathlineto{\pgfqpoint{2.792721in}{0.413320in}}%
\pgfpathlineto{\pgfqpoint{2.790044in}{0.413320in}}%
\pgfpathlineto{\pgfqpoint{2.787468in}{0.413320in}}%
\pgfpathlineto{\pgfqpoint{2.784687in}{0.413320in}}%
\pgfpathlineto{\pgfqpoint{2.782113in}{0.413320in}}%
\pgfpathlineto{\pgfqpoint{2.779330in}{0.413320in}}%
\pgfpathlineto{\pgfqpoint{2.776767in}{0.413320in}}%
\pgfpathlineto{\pgfqpoint{2.773972in}{0.413320in}}%
\pgfpathlineto{\pgfqpoint{2.771373in}{0.413320in}}%
\pgfpathlineto{\pgfqpoint{2.768617in}{0.413320in}}%
\pgfpathlineto{\pgfqpoint{2.765935in}{0.413320in}}%
\pgfpathlineto{\pgfqpoint{2.763253in}{0.413320in}}%
\pgfpathlineto{\pgfqpoint{2.760581in}{0.413320in}}%
\pgfpathlineto{\pgfqpoint{2.758028in}{0.413320in}}%
\pgfpathlineto{\pgfqpoint{2.755224in}{0.413320in}}%
\pgfpathlineto{\pgfqpoint{2.752614in}{0.413320in}}%
\pgfpathlineto{\pgfqpoint{2.749868in}{0.413320in}}%
\pgfpathlineto{\pgfqpoint{2.747260in}{0.413320in}}%
\pgfpathlineto{\pgfqpoint{2.744510in}{0.413320in}}%
\pgfpathlineto{\pgfqpoint{2.741928in}{0.413320in}}%
\pgfpathlineto{\pgfqpoint{2.739155in}{0.413320in}}%
\pgfpathlineto{\pgfqpoint{2.736476in}{0.413320in}}%
\pgfpathlineto{\pgfqpoint{2.733798in}{0.413320in}}%
\pgfpathlineto{\pgfqpoint{2.731119in}{0.413320in}}%
\pgfpathlineto{\pgfqpoint{2.728439in}{0.413320in}}%
\pgfpathlineto{\pgfqpoint{2.725760in}{0.413320in}}%
\pgfpathlineto{\pgfqpoint{2.723211in}{0.413320in}}%
\pgfpathlineto{\pgfqpoint{2.720404in}{0.413320in}}%
\pgfpathlineto{\pgfqpoint{2.717773in}{0.413320in}}%
\pgfpathlineto{\pgfqpoint{2.715036in}{0.413320in}}%
\pgfpathlineto{\pgfqpoint{2.712477in}{0.413320in}}%
\pgfpathlineto{\pgfqpoint{2.709683in}{0.413320in}}%
\pgfpathlineto{\pgfqpoint{2.707125in}{0.413320in}}%
\pgfpathlineto{\pgfqpoint{2.704326in}{0.413320in}}%
\pgfpathlineto{\pgfqpoint{2.701657in}{0.413320in}}%
\pgfpathlineto{\pgfqpoint{2.698968in}{0.413320in}}%
\pgfpathlineto{\pgfqpoint{2.696293in}{0.413320in}}%
\pgfpathlineto{\pgfqpoint{2.693611in}{0.413320in}}%
\pgfpathlineto{\pgfqpoint{2.690940in}{0.413320in}}%
\pgfpathlineto{\pgfqpoint{2.688328in}{0.413320in}}%
\pgfpathlineto{\pgfqpoint{2.685586in}{0.413320in}}%
\pgfpathlineto{\pgfqpoint{2.683009in}{0.413320in}}%
\pgfpathlineto{\pgfqpoint{2.680224in}{0.413320in}}%
\pgfpathlineto{\pgfqpoint{2.677650in}{0.413320in}}%
\pgfpathlineto{\pgfqpoint{2.674873in}{0.413320in}}%
\pgfpathlineto{\pgfqpoint{2.672301in}{0.413320in}}%
\pgfpathlineto{\pgfqpoint{2.669506in}{0.413320in}}%
\pgfpathlineto{\pgfqpoint{2.666836in}{0.413320in}}%
\pgfpathlineto{\pgfqpoint{2.664151in}{0.413320in}}%
\pgfpathlineto{\pgfqpoint{2.661481in}{0.413320in}}%
\pgfpathlineto{\pgfqpoint{2.658942in}{0.413320in}}%
\pgfpathlineto{\pgfqpoint{2.656124in}{0.413320in}}%
\pgfpathlineto{\pgfqpoint{2.653567in}{0.413320in}}%
\pgfpathlineto{\pgfqpoint{2.650767in}{0.413320in}}%
\pgfpathlineto{\pgfqpoint{2.648196in}{0.413320in}}%
\pgfpathlineto{\pgfqpoint{2.645408in}{0.413320in}}%
\pgfpathlineto{\pgfqpoint{2.642827in}{0.413320in}}%
\pgfpathlineto{\pgfqpoint{2.640053in}{0.413320in}}%
\pgfpathlineto{\pgfqpoint{2.637369in}{0.413320in}}%
\pgfpathlineto{\pgfqpoint{2.634700in}{0.413320in}}%
\pgfpathlineto{\pgfqpoint{2.632018in}{0.413320in}}%
\pgfpathlineto{\pgfqpoint{2.629340in}{0.413320in}}%
\pgfpathlineto{\pgfqpoint{2.626653in}{0.413320in}}%
\pgfpathlineto{\pgfqpoint{2.624077in}{0.413320in}}%
\pgfpathlineto{\pgfqpoint{2.621304in}{0.413320in}}%
\pgfpathlineto{\pgfqpoint{2.618773in}{0.413320in}}%
\pgfpathlineto{\pgfqpoint{2.615934in}{0.413320in}}%
\pgfpathlineto{\pgfqpoint{2.613393in}{0.413320in}}%
\pgfpathlineto{\pgfqpoint{2.610588in}{0.413320in}}%
\pgfpathlineto{\pgfqpoint{2.608004in}{0.413320in}}%
\pgfpathlineto{\pgfqpoint{2.605232in}{0.413320in}}%
\pgfpathlineto{\pgfqpoint{2.602557in}{0.413320in}}%
\pgfpathlineto{\pgfqpoint{2.599920in}{0.413320in}}%
\pgfpathlineto{\pgfqpoint{2.597196in}{0.413320in}}%
\pgfpathlineto{\pgfqpoint{2.594630in}{0.413320in}}%
\pgfpathlineto{\pgfqpoint{2.591842in}{0.413320in}}%
\pgfpathlineto{\pgfqpoint{2.589248in}{0.413320in}}%
\pgfpathlineto{\pgfqpoint{2.586484in}{0.413320in}}%
\pgfpathlineto{\pgfqpoint{2.583913in}{0.413320in}}%
\pgfpathlineto{\pgfqpoint{2.581129in}{0.413320in}}%
\pgfpathlineto{\pgfqpoint{2.578567in}{0.413320in}}%
\pgfpathlineto{\pgfqpoint{2.575779in}{0.413320in}}%
\pgfpathlineto{\pgfqpoint{2.573082in}{0.413320in}}%
\pgfpathlineto{\pgfqpoint{2.570411in}{0.413320in}}%
\pgfpathlineto{\pgfqpoint{2.567730in}{0.413320in}}%
\pgfpathlineto{\pgfqpoint{2.565045in}{0.413320in}}%
\pgfpathlineto{\pgfqpoint{2.562375in}{0.413320in}}%
\pgfpathlineto{\pgfqpoint{2.559790in}{0.413320in}}%
\pgfpathlineto{\pgfqpoint{2.557009in}{0.413320in}}%
\pgfpathlineto{\pgfqpoint{2.554493in}{0.413320in}}%
\pgfpathlineto{\pgfqpoint{2.551664in}{0.413320in}}%
\pgfpathlineto{\pgfqpoint{2.549114in}{0.413320in}}%
\pgfpathlineto{\pgfqpoint{2.546310in}{0.413320in}}%
\pgfpathlineto{\pgfqpoint{2.543765in}{0.413320in}}%
\pgfpathlineto{\pgfqpoint{2.540949in}{0.413320in}}%
\pgfpathlineto{\pgfqpoint{2.538274in}{0.413320in}}%
\pgfpathlineto{\pgfqpoint{2.535624in}{0.413320in}}%
\pgfpathlineto{\pgfqpoint{2.532917in}{0.413320in}}%
\pgfpathlineto{\pgfqpoint{2.530234in}{0.413320in}}%
\pgfpathlineto{\pgfqpoint{2.527560in}{0.413320in}}%
\pgfpathlineto{\pgfqpoint{2.524988in}{0.413320in}}%
\pgfpathlineto{\pgfqpoint{2.522197in}{0.413320in}}%
\pgfpathlineto{\pgfqpoint{2.519607in}{0.413320in}}%
\pgfpathlineto{\pgfqpoint{2.516845in}{0.413320in}}%
\pgfpathlineto{\pgfqpoint{2.514268in}{0.413320in}}%
\pgfpathlineto{\pgfqpoint{2.511478in}{0.413320in}}%
\pgfpathlineto{\pgfqpoint{2.508917in}{0.413320in}}%
\pgfpathlineto{\pgfqpoint{2.506163in}{0.413320in}}%
\pgfpathlineto{\pgfqpoint{2.503454in}{0.413320in}}%
\pgfpathlineto{\pgfqpoint{2.500801in}{0.413320in}}%
\pgfpathlineto{\pgfqpoint{2.498085in}{0.413320in}}%
\pgfpathlineto{\pgfqpoint{2.495542in}{0.413320in}}%
\pgfpathlineto{\pgfqpoint{2.492729in}{0.413320in}}%
\pgfpathlineto{\pgfqpoint{2.490183in}{0.413320in}}%
\pgfpathlineto{\pgfqpoint{2.487384in}{0.413320in}}%
\pgfpathlineto{\pgfqpoint{2.484870in}{0.413320in}}%
\pgfpathlineto{\pgfqpoint{2.482026in}{0.413320in}}%
\pgfpathlineto{\pgfqpoint{2.479420in}{0.413320in}}%
\pgfpathlineto{\pgfqpoint{2.476671in}{0.413320in}}%
\pgfpathlineto{\pgfqpoint{2.473989in}{0.413320in}}%
\pgfpathlineto{\pgfqpoint{2.471311in}{0.413320in}}%
\pgfpathlineto{\pgfqpoint{2.468635in}{0.413320in}}%
\pgfpathlineto{\pgfqpoint{2.465957in}{0.413320in}}%
\pgfpathlineto{\pgfqpoint{2.463280in}{0.413320in}}%
\pgfpathlineto{\pgfqpoint{2.460711in}{0.413320in}}%
\pgfpathlineto{\pgfqpoint{2.457917in}{0.413320in}}%
\pgfpathlineto{\pgfqpoint{2.455353in}{0.413320in}}%
\pgfpathlineto{\pgfqpoint{2.452562in}{0.413320in}}%
\pgfpathlineto{\pgfqpoint{2.450032in}{0.413320in}}%
\pgfpathlineto{\pgfqpoint{2.447209in}{0.413320in}}%
\pgfpathlineto{\pgfqpoint{2.444677in}{0.413320in}}%
\pgfpathlineto{\pgfqpoint{2.441876in}{0.413320in}}%
\pgfpathlineto{\pgfqpoint{2.439167in}{0.413320in}}%
\pgfpathlineto{\pgfqpoint{2.436518in}{0.413320in}}%
\pgfpathlineto{\pgfqpoint{2.433815in}{0.413320in}}%
\pgfpathlineto{\pgfqpoint{2.431251in}{0.413320in}}%
\pgfpathlineto{\pgfqpoint{2.428453in}{0.413320in}}%
\pgfpathlineto{\pgfqpoint{2.425878in}{0.413320in}}%
\pgfpathlineto{\pgfqpoint{2.423098in}{0.413320in}}%
\pgfpathlineto{\pgfqpoint{2.420528in}{0.413320in}}%
\pgfpathlineto{\pgfqpoint{2.417747in}{0.413320in}}%
\pgfpathlineto{\pgfqpoint{2.415184in}{0.413320in}}%
\pgfpathlineto{\pgfqpoint{2.412389in}{0.413320in}}%
\pgfpathlineto{\pgfqpoint{2.409699in}{0.413320in}}%
\pgfpathlineto{\pgfqpoint{2.407024in}{0.413320in}}%
\pgfpathlineto{\pgfqpoint{2.404352in}{0.413320in}}%
\pgfpathlineto{\pgfqpoint{2.401675in}{0.413320in}}%
\pgfpathlineto{\pgfqpoint{2.398995in}{0.413320in}}%
\pgfpathclose%
\pgfusepath{stroke,fill}%
\end{pgfscope}%
\begin{pgfscope}%
\pgfpathrectangle{\pgfqpoint{2.398995in}{0.319877in}}{\pgfqpoint{3.986877in}{1.993438in}} %
\pgfusepath{clip}%
\pgfsetbuttcap%
\pgfsetroundjoin%
\definecolor{currentfill}{rgb}{1.000000,1.000000,1.000000}%
\pgfsetfillcolor{currentfill}%
\pgfsetlinewidth{1.003750pt}%
\definecolor{currentstroke}{rgb}{0.510531,0.661430,0.193085}%
\pgfsetstrokecolor{currentstroke}%
\pgfsetdash{}{0pt}%
\pgfpathmoveto{\pgfqpoint{2.398995in}{0.413320in}}%
\pgfpathlineto{\pgfqpoint{2.398995in}{1.602041in}}%
\pgfpathlineto{\pgfqpoint{2.401675in}{1.599820in}}%
\pgfpathlineto{\pgfqpoint{2.404352in}{1.598464in}}%
\pgfpathlineto{\pgfqpoint{2.407024in}{1.608260in}}%
\pgfpathlineto{\pgfqpoint{2.409699in}{1.601266in}}%
\pgfpathlineto{\pgfqpoint{2.412389in}{1.598648in}}%
\pgfpathlineto{\pgfqpoint{2.415184in}{1.600742in}}%
\pgfpathlineto{\pgfqpoint{2.417747in}{1.599737in}}%
\pgfpathlineto{\pgfqpoint{2.420528in}{1.603703in}}%
\pgfpathlineto{\pgfqpoint{2.423098in}{1.601251in}}%
\pgfpathlineto{\pgfqpoint{2.425878in}{1.598808in}}%
\pgfpathlineto{\pgfqpoint{2.428453in}{1.601582in}}%
\pgfpathlineto{\pgfqpoint{2.431251in}{1.636013in}}%
\pgfpathlineto{\pgfqpoint{2.433815in}{1.685063in}}%
\pgfpathlineto{\pgfqpoint{2.436518in}{1.702511in}}%
\pgfpathlineto{\pgfqpoint{2.439167in}{1.684033in}}%
\pgfpathlineto{\pgfqpoint{2.441876in}{1.710169in}}%
\pgfpathlineto{\pgfqpoint{2.444677in}{1.703457in}}%
\pgfpathlineto{\pgfqpoint{2.447209in}{1.682820in}}%
\pgfpathlineto{\pgfqpoint{2.450032in}{1.714057in}}%
\pgfpathlineto{\pgfqpoint{2.452562in}{1.727414in}}%
\pgfpathlineto{\pgfqpoint{2.455353in}{1.708483in}}%
\pgfpathlineto{\pgfqpoint{2.457917in}{1.679619in}}%
\pgfpathlineto{\pgfqpoint{2.460711in}{1.673175in}}%
\pgfpathlineto{\pgfqpoint{2.463280in}{1.658686in}}%
\pgfpathlineto{\pgfqpoint{2.465957in}{1.654424in}}%
\pgfpathlineto{\pgfqpoint{2.468635in}{1.648223in}}%
\pgfpathlineto{\pgfqpoint{2.471311in}{1.641933in}}%
\pgfpathlineto{\pgfqpoint{2.473989in}{1.637787in}}%
\pgfpathlineto{\pgfqpoint{2.476671in}{1.636202in}}%
\pgfpathlineto{\pgfqpoint{2.479420in}{1.633307in}}%
\pgfpathlineto{\pgfqpoint{2.482026in}{1.638053in}}%
\pgfpathlineto{\pgfqpoint{2.484870in}{1.704475in}}%
\pgfpathlineto{\pgfqpoint{2.487384in}{1.694713in}}%
\pgfpathlineto{\pgfqpoint{2.490183in}{1.683177in}}%
\pgfpathlineto{\pgfqpoint{2.492729in}{1.670621in}}%
\pgfpathlineto{\pgfqpoint{2.495542in}{1.661412in}}%
\pgfpathlineto{\pgfqpoint{2.498085in}{1.660368in}}%
\pgfpathlineto{\pgfqpoint{2.500801in}{1.652072in}}%
\pgfpathlineto{\pgfqpoint{2.503454in}{1.646809in}}%
\pgfpathlineto{\pgfqpoint{2.506163in}{1.638563in}}%
\pgfpathlineto{\pgfqpoint{2.508917in}{1.633300in}}%
\pgfpathlineto{\pgfqpoint{2.511478in}{1.633440in}}%
\pgfpathlineto{\pgfqpoint{2.514268in}{1.628284in}}%
\pgfpathlineto{\pgfqpoint{2.516845in}{1.622629in}}%
\pgfpathlineto{\pgfqpoint{2.519607in}{1.611028in}}%
\pgfpathlineto{\pgfqpoint{2.522197in}{1.605149in}}%
\pgfpathlineto{\pgfqpoint{2.524988in}{1.605831in}}%
\pgfpathlineto{\pgfqpoint{2.527560in}{1.603274in}}%
\pgfpathlineto{\pgfqpoint{2.530234in}{1.604719in}}%
\pgfpathlineto{\pgfqpoint{2.532917in}{1.608466in}}%
\pgfpathlineto{\pgfqpoint{2.535624in}{1.608290in}}%
\pgfpathlineto{\pgfqpoint{2.538274in}{1.608974in}}%
\pgfpathlineto{\pgfqpoint{2.540949in}{1.608492in}}%
\pgfpathlineto{\pgfqpoint{2.543765in}{1.612025in}}%
\pgfpathlineto{\pgfqpoint{2.546310in}{1.604323in}}%
\pgfpathlineto{\pgfqpoint{2.549114in}{1.606710in}}%
\pgfpathlineto{\pgfqpoint{2.551664in}{1.618949in}}%
\pgfpathlineto{\pgfqpoint{2.554493in}{1.676889in}}%
\pgfpathlineto{\pgfqpoint{2.557009in}{1.699798in}}%
\pgfpathlineto{\pgfqpoint{2.559790in}{1.696771in}}%
\pgfpathlineto{\pgfqpoint{2.562375in}{1.690040in}}%
\pgfpathlineto{\pgfqpoint{2.565045in}{1.687695in}}%
\pgfpathlineto{\pgfqpoint{2.567730in}{1.677421in}}%
\pgfpathlineto{\pgfqpoint{2.570411in}{1.667812in}}%
\pgfpathlineto{\pgfqpoint{2.573082in}{1.650489in}}%
\pgfpathlineto{\pgfqpoint{2.575779in}{1.641139in}}%
\pgfpathlineto{\pgfqpoint{2.578567in}{1.647368in}}%
\pgfpathlineto{\pgfqpoint{2.581129in}{1.656584in}}%
\pgfpathlineto{\pgfqpoint{2.583913in}{1.642875in}}%
\pgfpathlineto{\pgfqpoint{2.586484in}{1.638462in}}%
\pgfpathlineto{\pgfqpoint{2.589248in}{1.635271in}}%
\pgfpathlineto{\pgfqpoint{2.591842in}{1.627746in}}%
\pgfpathlineto{\pgfqpoint{2.594630in}{1.623188in}}%
\pgfpathlineto{\pgfqpoint{2.597196in}{1.609809in}}%
\pgfpathlineto{\pgfqpoint{2.599920in}{1.609604in}}%
\pgfpathlineto{\pgfqpoint{2.602557in}{1.608484in}}%
\pgfpathlineto{\pgfqpoint{2.605232in}{1.608021in}}%
\pgfpathlineto{\pgfqpoint{2.608004in}{1.600404in}}%
\pgfpathlineto{\pgfqpoint{2.610588in}{1.600652in}}%
\pgfpathlineto{\pgfqpoint{2.613393in}{1.600710in}}%
\pgfpathlineto{\pgfqpoint{2.615934in}{1.603755in}}%
\pgfpathlineto{\pgfqpoint{2.618773in}{1.607349in}}%
\pgfpathlineto{\pgfqpoint{2.621304in}{1.607634in}}%
\pgfpathlineto{\pgfqpoint{2.624077in}{1.604719in}}%
\pgfpathlineto{\pgfqpoint{2.626653in}{1.604964in}}%
\pgfpathlineto{\pgfqpoint{2.629340in}{1.604825in}}%
\pgfpathlineto{\pgfqpoint{2.632018in}{1.603506in}}%
\pgfpathlineto{\pgfqpoint{2.634700in}{1.605213in}}%
\pgfpathlineto{\pgfqpoint{2.637369in}{1.603957in}}%
\pgfpathlineto{\pgfqpoint{2.640053in}{1.608973in}}%
\pgfpathlineto{\pgfqpoint{2.642827in}{1.609801in}}%
\pgfpathlineto{\pgfqpoint{2.645408in}{1.616437in}}%
\pgfpathlineto{\pgfqpoint{2.648196in}{1.616437in}}%
\pgfpathlineto{\pgfqpoint{2.650767in}{1.609860in}}%
\pgfpathlineto{\pgfqpoint{2.653567in}{1.606795in}}%
\pgfpathlineto{\pgfqpoint{2.656124in}{1.607986in}}%
\pgfpathlineto{\pgfqpoint{2.658942in}{1.602084in}}%
\pgfpathlineto{\pgfqpoint{2.661481in}{1.598821in}}%
\pgfpathlineto{\pgfqpoint{2.664151in}{1.593191in}}%
\pgfpathlineto{\pgfqpoint{2.666836in}{1.593148in}}%
\pgfpathlineto{\pgfqpoint{2.669506in}{1.599407in}}%
\pgfpathlineto{\pgfqpoint{2.672301in}{1.596626in}}%
\pgfpathlineto{\pgfqpoint{2.674873in}{1.595404in}}%
\pgfpathlineto{\pgfqpoint{2.677650in}{1.603480in}}%
\pgfpathlineto{\pgfqpoint{2.680224in}{1.605730in}}%
\pgfpathlineto{\pgfqpoint{2.683009in}{1.606685in}}%
\pgfpathlineto{\pgfqpoint{2.685586in}{1.609101in}}%
\pgfpathlineto{\pgfqpoint{2.688328in}{1.606103in}}%
\pgfpathlineto{\pgfqpoint{2.690940in}{1.607864in}}%
\pgfpathlineto{\pgfqpoint{2.693611in}{1.599804in}}%
\pgfpathlineto{\pgfqpoint{2.696293in}{1.605262in}}%
\pgfpathlineto{\pgfqpoint{2.698968in}{1.604156in}}%
\pgfpathlineto{\pgfqpoint{2.701657in}{1.603071in}}%
\pgfpathlineto{\pgfqpoint{2.704326in}{1.599775in}}%
\pgfpathlineto{\pgfqpoint{2.707125in}{1.605019in}}%
\pgfpathlineto{\pgfqpoint{2.709683in}{1.598598in}}%
\pgfpathlineto{\pgfqpoint{2.712477in}{1.602472in}}%
\pgfpathlineto{\pgfqpoint{2.715036in}{1.604777in}}%
\pgfpathlineto{\pgfqpoint{2.717773in}{1.607812in}}%
\pgfpathlineto{\pgfqpoint{2.720404in}{1.608549in}}%
\pgfpathlineto{\pgfqpoint{2.723211in}{1.601163in}}%
\pgfpathlineto{\pgfqpoint{2.725760in}{1.600359in}}%
\pgfpathlineto{\pgfqpoint{2.728439in}{1.601170in}}%
\pgfpathlineto{\pgfqpoint{2.731119in}{1.593704in}}%
\pgfpathlineto{\pgfqpoint{2.733798in}{1.591400in}}%
\pgfpathlineto{\pgfqpoint{2.736476in}{1.589568in}}%
\pgfpathlineto{\pgfqpoint{2.739155in}{1.598018in}}%
\pgfpathlineto{\pgfqpoint{2.741928in}{1.591128in}}%
\pgfpathlineto{\pgfqpoint{2.744510in}{1.587456in}}%
\pgfpathlineto{\pgfqpoint{2.747260in}{1.594318in}}%
\pgfpathlineto{\pgfqpoint{2.749868in}{1.595546in}}%
\pgfpathlineto{\pgfqpoint{2.752614in}{1.595684in}}%
\pgfpathlineto{\pgfqpoint{2.755224in}{1.588869in}}%
\pgfpathlineto{\pgfqpoint{2.758028in}{1.591723in}}%
\pgfpathlineto{\pgfqpoint{2.760581in}{1.595196in}}%
\pgfpathlineto{\pgfqpoint{2.763253in}{1.592373in}}%
\pgfpathlineto{\pgfqpoint{2.765935in}{1.596793in}}%
\pgfpathlineto{\pgfqpoint{2.768617in}{1.594514in}}%
\pgfpathlineto{\pgfqpoint{2.771373in}{1.588696in}}%
\pgfpathlineto{\pgfqpoint{2.773972in}{1.588988in}}%
\pgfpathlineto{\pgfqpoint{2.776767in}{1.592992in}}%
\pgfpathlineto{\pgfqpoint{2.779330in}{1.598700in}}%
\pgfpathlineto{\pgfqpoint{2.782113in}{1.597032in}}%
\pgfpathlineto{\pgfqpoint{2.784687in}{1.593782in}}%
\pgfpathlineto{\pgfqpoint{2.787468in}{1.599949in}}%
\pgfpathlineto{\pgfqpoint{2.790044in}{1.601640in}}%
\pgfpathlineto{\pgfqpoint{2.792721in}{1.596657in}}%
\pgfpathlineto{\pgfqpoint{2.795398in}{1.608191in}}%
\pgfpathlineto{\pgfqpoint{2.798070in}{1.596901in}}%
\pgfpathlineto{\pgfqpoint{2.800756in}{1.591381in}}%
\pgfpathlineto{\pgfqpoint{2.803435in}{1.596812in}}%
\pgfpathlineto{\pgfqpoint{2.806175in}{1.595817in}}%
\pgfpathlineto{\pgfqpoint{2.808792in}{1.593410in}}%
\pgfpathlineto{\pgfqpoint{2.811597in}{1.594557in}}%
\pgfpathlineto{\pgfqpoint{2.814141in}{1.590503in}}%
\pgfpathlineto{\pgfqpoint{2.816867in}{1.593103in}}%
\pgfpathlineto{\pgfqpoint{2.819506in}{1.596744in}}%
\pgfpathlineto{\pgfqpoint{2.822303in}{1.593572in}}%
\pgfpathlineto{\pgfqpoint{2.824851in}{1.597078in}}%
\pgfpathlineto{\pgfqpoint{2.827567in}{1.600547in}}%
\pgfpathlineto{\pgfqpoint{2.830219in}{1.599743in}}%
\pgfpathlineto{\pgfqpoint{2.832894in}{1.600187in}}%
\pgfpathlineto{\pgfqpoint{2.835698in}{1.599849in}}%
\pgfpathlineto{\pgfqpoint{2.838254in}{1.608869in}}%
\pgfpathlineto{\pgfqpoint{2.841055in}{1.641203in}}%
\pgfpathlineto{\pgfqpoint{2.843611in}{1.651762in}}%
\pgfpathlineto{\pgfqpoint{2.846408in}{1.636401in}}%
\pgfpathlineto{\pgfqpoint{2.848960in}{1.627105in}}%
\pgfpathlineto{\pgfqpoint{2.851793in}{1.611187in}}%
\pgfpathlineto{\pgfqpoint{2.854325in}{1.614666in}}%
\pgfpathlineto{\pgfqpoint{2.857003in}{1.606586in}}%
\pgfpathlineto{\pgfqpoint{2.859668in}{1.605749in}}%
\pgfpathlineto{\pgfqpoint{2.862402in}{1.608994in}}%
\pgfpathlineto{\pgfqpoint{2.865031in}{1.607465in}}%
\pgfpathlineto{\pgfqpoint{2.867713in}{1.607403in}}%
\pgfpathlineto{\pgfqpoint{2.870475in}{1.603595in}}%
\pgfpathlineto{\pgfqpoint{2.873074in}{1.606138in}}%
\pgfpathlineto{\pgfqpoint{2.875882in}{1.602737in}}%
\pgfpathlineto{\pgfqpoint{2.878431in}{1.603512in}}%
\pgfpathlineto{\pgfqpoint{2.881254in}{1.607907in}}%
\pgfpathlineto{\pgfqpoint{2.883780in}{1.597643in}}%
\pgfpathlineto{\pgfqpoint{2.886578in}{1.599156in}}%
\pgfpathlineto{\pgfqpoint{2.889145in}{1.601857in}}%
\pgfpathlineto{\pgfqpoint{2.891809in}{1.595845in}}%
\pgfpathlineto{\pgfqpoint{2.894487in}{1.601228in}}%
\pgfpathlineto{\pgfqpoint{2.897179in}{1.602431in}}%
\pgfpathlineto{\pgfqpoint{2.899858in}{1.602535in}}%
\pgfpathlineto{\pgfqpoint{2.902535in}{1.597178in}}%
\pgfpathlineto{\pgfqpoint{2.905341in}{1.602669in}}%
\pgfpathlineto{\pgfqpoint{2.907882in}{1.601053in}}%
\pgfpathlineto{\pgfqpoint{2.910631in}{1.603518in}}%
\pgfpathlineto{\pgfqpoint{2.913243in}{1.604271in}}%
\pgfpathlineto{\pgfqpoint{2.916061in}{1.604939in}}%
\pgfpathlineto{\pgfqpoint{2.918606in}{1.601921in}}%
\pgfpathlineto{\pgfqpoint{2.921363in}{1.603922in}}%
\pgfpathlineto{\pgfqpoint{2.923963in}{1.600942in}}%
\pgfpathlineto{\pgfqpoint{2.926655in}{1.601838in}}%
\pgfpathlineto{\pgfqpoint{2.929321in}{1.604132in}}%
\pgfpathlineto{\pgfqpoint{2.932033in}{1.605077in}}%
\pgfpathlineto{\pgfqpoint{2.934759in}{1.606092in}}%
\pgfpathlineto{\pgfqpoint{2.937352in}{1.606375in}}%
\pgfpathlineto{\pgfqpoint{2.940120in}{1.604383in}}%
\pgfpathlineto{\pgfqpoint{2.942711in}{1.602477in}}%
\pgfpathlineto{\pgfqpoint{2.945461in}{1.602962in}}%
\pgfpathlineto{\pgfqpoint{2.948068in}{1.604331in}}%
\pgfpathlineto{\pgfqpoint{2.950884in}{1.603369in}}%
\pgfpathlineto{\pgfqpoint{2.953422in}{1.598919in}}%
\pgfpathlineto{\pgfqpoint{2.956103in}{1.603618in}}%
\pgfpathlineto{\pgfqpoint{2.958782in}{1.600861in}}%
\pgfpathlineto{\pgfqpoint{2.961460in}{1.605500in}}%
\pgfpathlineto{\pgfqpoint{2.964127in}{1.607569in}}%
\pgfpathlineto{\pgfqpoint{2.966812in}{1.602992in}}%
\pgfpathlineto{\pgfqpoint{2.969599in}{1.604290in}}%
\pgfpathlineto{\pgfqpoint{2.972177in}{1.600357in}}%
\pgfpathlineto{\pgfqpoint{2.974972in}{1.602322in}}%
\pgfpathlineto{\pgfqpoint{2.977517in}{1.604140in}}%
\pgfpathlineto{\pgfqpoint{2.980341in}{1.606587in}}%
\pgfpathlineto{\pgfqpoint{2.982885in}{1.602143in}}%
\pgfpathlineto{\pgfqpoint{2.985666in}{1.597638in}}%
\pgfpathlineto{\pgfqpoint{2.988238in}{1.601025in}}%
\pgfpathlineto{\pgfqpoint{2.990978in}{1.599932in}}%
\pgfpathlineto{\pgfqpoint{2.993595in}{1.594681in}}%
\pgfpathlineto{\pgfqpoint{2.996300in}{1.596212in}}%
\pgfpathlineto{\pgfqpoint{2.999103in}{1.599023in}}%
\pgfpathlineto{\pgfqpoint{3.001635in}{1.598089in}}%
\pgfpathlineto{\pgfqpoint{3.004419in}{1.597593in}}%
\pgfpathlineto{\pgfqpoint{3.006993in}{1.599810in}}%
\pgfpathlineto{\pgfqpoint{3.009784in}{1.601683in}}%
\pgfpathlineto{\pgfqpoint{3.012351in}{1.600713in}}%
\pgfpathlineto{\pgfqpoint{3.015097in}{1.597866in}}%
\pgfpathlineto{\pgfqpoint{3.017707in}{1.605047in}}%
\pgfpathlineto{\pgfqpoint{3.020382in}{1.608241in}}%
\pgfpathlineto{\pgfqpoint{3.023058in}{1.609912in}}%
\pgfpathlineto{\pgfqpoint{3.025803in}{1.616327in}}%
\pgfpathlineto{\pgfqpoint{3.028412in}{1.613901in}}%
\pgfpathlineto{\pgfqpoint{3.031091in}{1.604967in}}%
\pgfpathlineto{\pgfqpoint{3.033921in}{1.600933in}}%
\pgfpathlineto{\pgfqpoint{3.036456in}{1.602554in}}%
\pgfpathlineto{\pgfqpoint{3.039262in}{1.600106in}}%
\pgfpathlineto{\pgfqpoint{3.041813in}{1.600659in}}%
\pgfpathlineto{\pgfqpoint{3.044568in}{1.604070in}}%
\pgfpathlineto{\pgfqpoint{3.047157in}{1.605866in}}%
\pgfpathlineto{\pgfqpoint{3.049988in}{1.603473in}}%
\pgfpathlineto{\pgfqpoint{3.052526in}{1.600083in}}%
\pgfpathlineto{\pgfqpoint{3.055202in}{1.597651in}}%
\pgfpathlineto{\pgfqpoint{3.057884in}{1.598925in}}%
\pgfpathlineto{\pgfqpoint{3.060561in}{1.599191in}}%
\pgfpathlineto{\pgfqpoint{3.063230in}{1.603728in}}%
\pgfpathlineto{\pgfqpoint{3.065916in}{1.601293in}}%
\pgfpathlineto{\pgfqpoint{3.068709in}{1.599644in}}%
\pgfpathlineto{\pgfqpoint{3.071266in}{1.604425in}}%
\pgfpathlineto{\pgfqpoint{3.074056in}{1.599762in}}%
\pgfpathlineto{\pgfqpoint{3.076631in}{1.602794in}}%
\pgfpathlineto{\pgfqpoint{3.079381in}{1.600759in}}%
\pgfpathlineto{\pgfqpoint{3.081990in}{1.601618in}}%
\pgfpathlineto{\pgfqpoint{3.084671in}{1.601062in}}%
\pgfpathlineto{\pgfqpoint{3.087343in}{1.599024in}}%
\pgfpathlineto{\pgfqpoint{3.090023in}{1.598971in}}%
\pgfpathlineto{\pgfqpoint{3.092699in}{1.608441in}}%
\pgfpathlineto{\pgfqpoint{3.095388in}{1.600651in}}%
\pgfpathlineto{\pgfqpoint{3.098163in}{1.600765in}}%
\pgfpathlineto{\pgfqpoint{3.100737in}{1.603984in}}%
\pgfpathlineto{\pgfqpoint{3.103508in}{1.616527in}}%
\pgfpathlineto{\pgfqpoint{3.106094in}{1.607017in}}%
\pgfpathlineto{\pgfqpoint{3.108896in}{1.602686in}}%
\pgfpathlineto{\pgfqpoint{3.111451in}{1.598486in}}%
\pgfpathlineto{\pgfqpoint{3.114242in}{1.599245in}}%
\pgfpathlineto{\pgfqpoint{3.116807in}{1.605686in}}%
\pgfpathlineto{\pgfqpoint{3.119487in}{1.599926in}}%
\pgfpathlineto{\pgfqpoint{3.122163in}{1.598497in}}%
\pgfpathlineto{\pgfqpoint{3.124842in}{1.599482in}}%
\pgfpathlineto{\pgfqpoint{3.127512in}{1.603321in}}%
\pgfpathlineto{\pgfqpoint{3.130199in}{1.606067in}}%
\pgfpathlineto{\pgfqpoint{3.132946in}{1.604034in}}%
\pgfpathlineto{\pgfqpoint{3.135550in}{1.599135in}}%
\pgfpathlineto{\pgfqpoint{3.138375in}{1.605298in}}%
\pgfpathlineto{\pgfqpoint{3.140913in}{1.599877in}}%
\pgfpathlineto{\pgfqpoint{3.143740in}{1.596681in}}%
\pgfpathlineto{\pgfqpoint{3.146271in}{1.593515in}}%
\pgfpathlineto{\pgfqpoint{3.149057in}{1.595044in}}%
\pgfpathlineto{\pgfqpoint{3.151612in}{1.599370in}}%
\pgfpathlineto{\pgfqpoint{3.154327in}{1.600448in}}%
\pgfpathlineto{\pgfqpoint{3.156981in}{1.599156in}}%
\pgfpathlineto{\pgfqpoint{3.159675in}{1.600660in}}%
\pgfpathlineto{\pgfqpoint{3.162474in}{1.602163in}}%
\pgfpathlineto{\pgfqpoint{3.165019in}{1.599116in}}%
\pgfpathlineto{\pgfqpoint{3.167776in}{1.600052in}}%
\pgfpathlineto{\pgfqpoint{3.170375in}{1.599710in}}%
\pgfpathlineto{\pgfqpoint{3.173142in}{1.591892in}}%
\pgfpathlineto{\pgfqpoint{3.175724in}{1.586534in}}%
\pgfpathlineto{\pgfqpoint{3.178525in}{1.593860in}}%
\pgfpathlineto{\pgfqpoint{3.181089in}{1.592904in}}%
\pgfpathlineto{\pgfqpoint{3.183760in}{1.599595in}}%
\pgfpathlineto{\pgfqpoint{3.186440in}{1.596331in}}%
\pgfpathlineto{\pgfqpoint{3.189117in}{1.597858in}}%
\pgfpathlineto{\pgfqpoint{3.191796in}{1.594591in}}%
\pgfpathlineto{\pgfqpoint{3.194508in}{1.599961in}}%
\pgfpathlineto{\pgfqpoint{3.197226in}{1.596778in}}%
\pgfpathlineto{\pgfqpoint{3.199823in}{1.599582in}}%
\pgfpathlineto{\pgfqpoint{3.202562in}{1.589832in}}%
\pgfpathlineto{\pgfqpoint{3.205195in}{1.592412in}}%
\pgfpathlineto{\pgfqpoint{3.207984in}{1.590946in}}%
\pgfpathlineto{\pgfqpoint{3.210545in}{1.590949in}}%
\pgfpathlineto{\pgfqpoint{3.213342in}{1.597447in}}%
\pgfpathlineto{\pgfqpoint{3.215908in}{1.596623in}}%
\pgfpathlineto{\pgfqpoint{3.218586in}{1.600702in}}%
\pgfpathlineto{\pgfqpoint{3.221255in}{1.600628in}}%
\pgfpathlineto{\pgfqpoint{3.223942in}{1.599582in}}%
\pgfpathlineto{\pgfqpoint{3.226609in}{1.604542in}}%
\pgfpathlineto{\pgfqpoint{3.229310in}{1.596083in}}%
\pgfpathlineto{\pgfqpoint{3.232069in}{1.598983in}}%
\pgfpathlineto{\pgfqpoint{3.234658in}{1.597287in}}%
\pgfpathlineto{\pgfqpoint{3.237411in}{1.600726in}}%
\pgfpathlineto{\pgfqpoint{3.240010in}{1.601878in}}%
\pgfpathlineto{\pgfqpoint{3.242807in}{1.602577in}}%
\pgfpathlineto{\pgfqpoint{3.245363in}{1.608124in}}%
\pgfpathlineto{\pgfqpoint{3.248049in}{1.601608in}}%
\pgfpathlineto{\pgfqpoint{3.250716in}{1.603512in}}%
\pgfpathlineto{\pgfqpoint{3.253404in}{1.603190in}}%
\pgfpathlineto{\pgfqpoint{3.256083in}{1.601424in}}%
\pgfpathlineto{\pgfqpoint{3.258784in}{1.602149in}}%
\pgfpathlineto{\pgfqpoint{3.261594in}{1.606198in}}%
\pgfpathlineto{\pgfqpoint{3.264119in}{1.603234in}}%
\pgfpathlineto{\pgfqpoint{3.266849in}{1.604546in}}%
\pgfpathlineto{\pgfqpoint{3.269478in}{1.606576in}}%
\pgfpathlineto{\pgfqpoint{3.272254in}{1.604872in}}%
\pgfpathlineto{\pgfqpoint{3.274831in}{1.601265in}}%
\pgfpathlineto{\pgfqpoint{3.277603in}{1.594510in}}%
\pgfpathlineto{\pgfqpoint{3.280189in}{1.596123in}}%
\pgfpathlineto{\pgfqpoint{3.282870in}{1.595561in}}%
\pgfpathlineto{\pgfqpoint{3.285534in}{1.598705in}}%
\pgfpathlineto{\pgfqpoint{3.288225in}{1.602047in}}%
\pgfpathlineto{\pgfqpoint{3.290890in}{1.598074in}}%
\pgfpathlineto{\pgfqpoint{3.293574in}{1.602297in}}%
\pgfpathlineto{\pgfqpoint{3.296376in}{1.609577in}}%
\pgfpathlineto{\pgfqpoint{3.298937in}{1.609497in}}%
\pgfpathlineto{\pgfqpoint{3.301719in}{1.599876in}}%
\pgfpathlineto{\pgfqpoint{3.304295in}{1.598548in}}%
\pgfpathlineto{\pgfqpoint{3.307104in}{1.601953in}}%
\pgfpathlineto{\pgfqpoint{3.309652in}{1.599291in}}%
\pgfpathlineto{\pgfqpoint{3.312480in}{1.603280in}}%
\pgfpathlineto{\pgfqpoint{3.315008in}{1.600388in}}%
\pgfpathlineto{\pgfqpoint{3.317688in}{1.607046in}}%
\pgfpathlineto{\pgfqpoint{3.320366in}{1.607653in}}%
\pgfpathlineto{\pgfqpoint{3.323049in}{1.606213in}}%
\pgfpathlineto{\pgfqpoint{3.325860in}{1.604929in}}%
\pgfpathlineto{\pgfqpoint{3.328401in}{1.606324in}}%
\pgfpathlineto{\pgfqpoint{3.331183in}{1.603828in}}%
\pgfpathlineto{\pgfqpoint{3.333758in}{1.601215in}}%
\pgfpathlineto{\pgfqpoint{3.336541in}{1.606207in}}%
\pgfpathlineto{\pgfqpoint{3.339101in}{1.600824in}}%
\pgfpathlineto{\pgfqpoint{3.341893in}{1.607817in}}%
\pgfpathlineto{\pgfqpoint{3.344468in}{1.604060in}}%
\pgfpathlineto{\pgfqpoint{3.347139in}{1.604232in}}%
\pgfpathlineto{\pgfqpoint{3.349828in}{1.602964in}}%
\pgfpathlineto{\pgfqpoint{3.352505in}{1.604925in}}%
\pgfpathlineto{\pgfqpoint{3.355177in}{1.604013in}}%
\pgfpathlineto{\pgfqpoint{3.357862in}{1.607902in}}%
\pgfpathlineto{\pgfqpoint{3.360620in}{1.611085in}}%
\pgfpathlineto{\pgfqpoint{3.363221in}{1.601258in}}%
\pgfpathlineto{\pgfqpoint{3.365996in}{1.604871in}}%
\pgfpathlineto{\pgfqpoint{3.368577in}{1.606163in}}%
\pgfpathlineto{\pgfqpoint{3.371357in}{1.605691in}}%
\pgfpathlineto{\pgfqpoint{3.373921in}{1.605390in}}%
\pgfpathlineto{\pgfqpoint{3.376735in}{1.606546in}}%
\pgfpathlineto{\pgfqpoint{3.379290in}{1.603891in}}%
\pgfpathlineto{\pgfqpoint{3.381959in}{1.600291in}}%
\pgfpathlineto{\pgfqpoint{3.384647in}{1.607478in}}%
\pgfpathlineto{\pgfqpoint{3.387309in}{1.605962in}}%
\pgfpathlineto{\pgfqpoint{3.390102in}{1.606236in}}%
\pgfpathlineto{\pgfqpoint{3.392681in}{1.603341in}}%
\pgfpathlineto{\pgfqpoint{3.395461in}{1.604770in}}%
\pgfpathlineto{\pgfqpoint{3.398037in}{1.606645in}}%
\pgfpathlineto{\pgfqpoint{3.400783in}{1.606740in}}%
\pgfpathlineto{\pgfqpoint{3.403394in}{1.608086in}}%
\pgfpathlineto{\pgfqpoint{3.406202in}{1.606606in}}%
\pgfpathlineto{\pgfqpoint{3.408752in}{1.605879in}}%
\pgfpathlineto{\pgfqpoint{3.411431in}{1.601114in}}%
\pgfpathlineto{\pgfqpoint{3.414109in}{1.607093in}}%
\pgfpathlineto{\pgfqpoint{3.416780in}{1.610357in}}%
\pgfpathlineto{\pgfqpoint{3.419455in}{1.604303in}}%
\pgfpathlineto{\pgfqpoint{3.422142in}{1.607808in}}%
\pgfpathlineto{\pgfqpoint{3.424887in}{1.605750in}}%
\pgfpathlineto{\pgfqpoint{3.427501in}{1.603153in}}%
\pgfpathlineto{\pgfqpoint{3.430313in}{1.603643in}}%
\pgfpathlineto{\pgfqpoint{3.432851in}{1.602911in}}%
\pgfpathlineto{\pgfqpoint{3.435635in}{1.602529in}}%
\pgfpathlineto{\pgfqpoint{3.438210in}{1.602669in}}%
\pgfpathlineto{\pgfqpoint{3.440996in}{1.601386in}}%
\pgfpathlineto{\pgfqpoint{3.443574in}{1.600055in}}%
\pgfpathlineto{\pgfqpoint{3.446257in}{1.598774in}}%
\pgfpathlineto{\pgfqpoint{3.448926in}{1.599991in}}%
\pgfpathlineto{\pgfqpoint{3.451597in}{1.601602in}}%
\pgfpathlineto{\pgfqpoint{3.454285in}{1.598116in}}%
\pgfpathlineto{\pgfqpoint{3.456960in}{1.599145in}}%
\pgfpathlineto{\pgfqpoint{3.459695in}{1.598233in}}%
\pgfpathlineto{\pgfqpoint{3.462321in}{1.597092in}}%
\pgfpathlineto{\pgfqpoint{3.465072in}{1.600134in}}%
\pgfpathlineto{\pgfqpoint{3.467678in}{1.592491in}}%
\pgfpathlineto{\pgfqpoint{3.470466in}{1.592583in}}%
\pgfpathlineto{\pgfqpoint{3.473021in}{1.601023in}}%
\pgfpathlineto{\pgfqpoint{3.475821in}{1.599294in}}%
\pgfpathlineto{\pgfqpoint{3.478378in}{1.599910in}}%
\pgfpathlineto{\pgfqpoint{3.481072in}{1.604017in}}%
\pgfpathlineto{\pgfqpoint{3.483744in}{1.603665in}}%
\pgfpathlineto{\pgfqpoint{3.486442in}{1.606592in}}%
\pgfpathlineto{\pgfqpoint{3.489223in}{1.604064in}}%
\pgfpathlineto{\pgfqpoint{3.491783in}{1.614019in}}%
\pgfpathlineto{\pgfqpoint{3.494581in}{1.610379in}}%
\pgfpathlineto{\pgfqpoint{3.497139in}{1.619584in}}%
\pgfpathlineto{\pgfqpoint{3.499909in}{1.622250in}}%
\pgfpathlineto{\pgfqpoint{3.502488in}{1.612044in}}%
\pgfpathlineto{\pgfqpoint{3.505262in}{1.609606in}}%
\pgfpathlineto{\pgfqpoint{3.507840in}{1.611125in}}%
\pgfpathlineto{\pgfqpoint{3.510533in}{1.618666in}}%
\pgfpathlineto{\pgfqpoint{3.513209in}{1.612655in}}%
\pgfpathlineto{\pgfqpoint{3.515884in}{1.605604in}}%
\pgfpathlineto{\pgfqpoint{3.518565in}{1.605740in}}%
\pgfpathlineto{\pgfqpoint{3.521244in}{1.602818in}}%
\pgfpathlineto{\pgfqpoint{3.524041in}{1.606218in}}%
\pgfpathlineto{\pgfqpoint{3.526601in}{1.601063in}}%
\pgfpathlineto{\pgfqpoint{3.529327in}{1.601031in}}%
\pgfpathlineto{\pgfqpoint{3.531955in}{1.604598in}}%
\pgfpathlineto{\pgfqpoint{3.534783in}{1.607759in}}%
\pgfpathlineto{\pgfqpoint{3.537309in}{1.605703in}}%
\pgfpathlineto{\pgfqpoint{3.540093in}{1.606241in}}%
\pgfpathlineto{\pgfqpoint{3.542656in}{1.606239in}}%
\pgfpathlineto{\pgfqpoint{3.545349in}{1.609428in}}%
\pgfpathlineto{\pgfqpoint{3.548029in}{1.600744in}}%
\pgfpathlineto{\pgfqpoint{3.550713in}{1.604570in}}%
\pgfpathlineto{\pgfqpoint{3.553498in}{1.605033in}}%
\pgfpathlineto{\pgfqpoint{3.556061in}{1.604652in}}%
\pgfpathlineto{\pgfqpoint{3.558853in}{1.606329in}}%
\pgfpathlineto{\pgfqpoint{3.561420in}{1.605601in}}%
\pgfpathlineto{\pgfqpoint{3.564188in}{1.606202in}}%
\pgfpathlineto{\pgfqpoint{3.566774in}{1.604887in}}%
\pgfpathlineto{\pgfqpoint{3.569584in}{1.607027in}}%
\pgfpathlineto{\pgfqpoint{3.572126in}{1.606465in}}%
\pgfpathlineto{\pgfqpoint{3.574814in}{1.605161in}}%
\pgfpathlineto{\pgfqpoint{3.577487in}{1.602095in}}%
\pgfpathlineto{\pgfqpoint{3.580191in}{1.603885in}}%
\pgfpathlineto{\pgfqpoint{3.582851in}{1.602773in}}%
\pgfpathlineto{\pgfqpoint{3.585532in}{1.605569in}}%
\pgfpathlineto{\pgfqpoint{3.588258in}{1.606026in}}%
\pgfpathlineto{\pgfqpoint{3.590883in}{1.602635in}}%
\pgfpathlineto{\pgfqpoint{3.593620in}{1.602678in}}%
\pgfpathlineto{\pgfqpoint{3.596240in}{1.600984in}}%
\pgfpathlineto{\pgfqpoint{3.598998in}{1.596864in}}%
\pgfpathlineto{\pgfqpoint{3.601590in}{1.602741in}}%
\pgfpathlineto{\pgfqpoint{3.604387in}{1.603426in}}%
\pgfpathlineto{\pgfqpoint{3.606951in}{1.601201in}}%
\pgfpathlineto{\pgfqpoint{3.609632in}{1.611660in}}%
\pgfpathlineto{\pgfqpoint{3.612311in}{1.652004in}}%
\pgfpathlineto{\pgfqpoint{3.614982in}{1.650449in}}%
\pgfpathlineto{\pgfqpoint{3.617667in}{1.633006in}}%
\pgfpathlineto{\pgfqpoint{3.620345in}{1.625940in}}%
\pgfpathlineto{\pgfqpoint{3.623165in}{1.619424in}}%
\pgfpathlineto{\pgfqpoint{3.625689in}{1.612690in}}%
\pgfpathlineto{\pgfqpoint{3.628460in}{1.615967in}}%
\pgfpathlineto{\pgfqpoint{3.631058in}{1.621410in}}%
\pgfpathlineto{\pgfqpoint{3.633858in}{1.619741in}}%
\pgfpathlineto{\pgfqpoint{3.636413in}{1.620054in}}%
\pgfpathlineto{\pgfqpoint{3.639207in}{1.617907in}}%
\pgfpathlineto{\pgfqpoint{3.641773in}{1.616025in}}%
\pgfpathlineto{\pgfqpoint{3.644452in}{1.610662in}}%
\pgfpathlineto{\pgfqpoint{3.647130in}{1.602617in}}%
\pgfpathlineto{\pgfqpoint{3.649837in}{1.607385in}}%
\pgfpathlineto{\pgfqpoint{3.652628in}{1.607865in}}%
\pgfpathlineto{\pgfqpoint{3.655165in}{1.604662in}}%
\pgfpathlineto{\pgfqpoint{3.657917in}{1.601323in}}%
\pgfpathlineto{\pgfqpoint{3.660515in}{1.600966in}}%
\pgfpathlineto{\pgfqpoint{3.663276in}{1.597081in}}%
\pgfpathlineto{\pgfqpoint{3.665864in}{1.597584in}}%
\pgfpathlineto{\pgfqpoint{3.668665in}{1.599957in}}%
\pgfpathlineto{\pgfqpoint{3.671232in}{1.597023in}}%
\pgfpathlineto{\pgfqpoint{3.673911in}{1.598677in}}%
\pgfpathlineto{\pgfqpoint{3.676591in}{1.597638in}}%
\pgfpathlineto{\pgfqpoint{3.679273in}{1.605888in}}%
\pgfpathlineto{\pgfqpoint{3.681948in}{1.607039in}}%
\pgfpathlineto{\pgfqpoint{3.684620in}{1.605724in}}%
\pgfpathlineto{\pgfqpoint{3.687442in}{1.605296in}}%
\pgfpathlineto{\pgfqpoint{3.689983in}{1.611443in}}%
\pgfpathlineto{\pgfqpoint{3.692765in}{1.609322in}}%
\pgfpathlineto{\pgfqpoint{3.695331in}{1.610453in}}%
\pgfpathlineto{\pgfqpoint{3.698125in}{1.600384in}}%
\pgfpathlineto{\pgfqpoint{3.700684in}{1.598845in}}%
\pgfpathlineto{\pgfqpoint{3.703460in}{1.608721in}}%
\pgfpathlineto{\pgfqpoint{3.706053in}{1.602760in}}%
\pgfpathlineto{\pgfqpoint{3.708729in}{1.614237in}}%
\pgfpathlineto{\pgfqpoint{3.711410in}{1.608450in}}%
\pgfpathlineto{\pgfqpoint{3.714086in}{1.604739in}}%
\pgfpathlineto{\pgfqpoint{3.716875in}{1.603180in}}%
\pgfpathlineto{\pgfqpoint{3.719446in}{1.603333in}}%
\pgfpathlineto{\pgfqpoint{3.722228in}{1.609252in}}%
\pgfpathlineto{\pgfqpoint{3.724804in}{1.607369in}}%
\pgfpathlineto{\pgfqpoint{3.727581in}{1.605804in}}%
\pgfpathlineto{\pgfqpoint{3.730158in}{1.604452in}}%
\pgfpathlineto{\pgfqpoint{3.732950in}{1.607561in}}%
\pgfpathlineto{\pgfqpoint{3.735509in}{1.605007in}}%
\pgfpathlineto{\pgfqpoint{3.738194in}{1.601914in}}%
\pgfpathlineto{\pgfqpoint{3.740874in}{1.602214in}}%
\pgfpathlineto{\pgfqpoint{3.743548in}{1.601929in}}%
\pgfpathlineto{\pgfqpoint{3.746229in}{1.600054in}}%
\pgfpathlineto{\pgfqpoint{3.748903in}{1.607772in}}%
\pgfpathlineto{\pgfqpoint{3.751728in}{1.608352in}}%
\pgfpathlineto{\pgfqpoint{3.754265in}{1.607397in}}%
\pgfpathlineto{\pgfqpoint{3.757065in}{1.603643in}}%
\pgfpathlineto{\pgfqpoint{3.759622in}{1.611332in}}%
\pgfpathlineto{\pgfqpoint{3.762389in}{1.604592in}}%
\pgfpathlineto{\pgfqpoint{3.764966in}{1.608986in}}%
\pgfpathlineto{\pgfqpoint{3.767782in}{1.610457in}}%
\pgfpathlineto{\pgfqpoint{3.770323in}{1.636682in}}%
\pgfpathlineto{\pgfqpoint{3.773014in}{1.654192in}}%
\pgfpathlineto{\pgfqpoint{3.775691in}{1.687354in}}%
\pgfpathlineto{\pgfqpoint{3.778370in}{1.712442in}}%
\pgfpathlineto{\pgfqpoint{3.781046in}{1.718638in}}%
\pgfpathlineto{\pgfqpoint{3.783725in}{1.677415in}}%
\pgfpathlineto{\pgfqpoint{3.786504in}{1.653198in}}%
\pgfpathlineto{\pgfqpoint{3.789084in}{1.643511in}}%
\pgfpathlineto{\pgfqpoint{3.791897in}{1.642351in}}%
\pgfpathlineto{\pgfqpoint{3.794435in}{1.630353in}}%
\pgfpathlineto{\pgfqpoint{3.797265in}{1.635236in}}%
\pgfpathlineto{\pgfqpoint{3.799797in}{1.618738in}}%
\pgfpathlineto{\pgfqpoint{3.802569in}{1.616023in}}%
\pgfpathlineto{\pgfqpoint{3.805145in}{1.621784in}}%
\pgfpathlineto{\pgfqpoint{3.807832in}{1.616898in}}%
\pgfpathlineto{\pgfqpoint{3.810510in}{1.612036in}}%
\pgfpathlineto{\pgfqpoint{3.813172in}{1.611804in}}%
\pgfpathlineto{\pgfqpoint{3.815983in}{1.611566in}}%
\pgfpathlineto{\pgfqpoint{3.818546in}{1.612101in}}%
\pgfpathlineto{\pgfqpoint{3.821315in}{1.611261in}}%
\pgfpathlineto{\pgfqpoint{3.823903in}{1.612800in}}%
\pgfpathlineto{\pgfqpoint{3.826679in}{1.609131in}}%
\pgfpathlineto{\pgfqpoint{3.829252in}{1.609406in}}%
\pgfpathlineto{\pgfqpoint{3.832053in}{1.605935in}}%
\pgfpathlineto{\pgfqpoint{3.834616in}{1.609635in}}%
\pgfpathlineto{\pgfqpoint{3.837286in}{1.606183in}}%
\pgfpathlineto{\pgfqpoint{3.839960in}{1.605262in}}%
\pgfpathlineto{\pgfqpoint{3.842641in}{1.603362in}}%
\pgfpathlineto{\pgfqpoint{3.845329in}{1.605876in}}%
\pgfpathlineto{\pgfqpoint{3.848005in}{1.602323in}}%
\pgfpathlineto{\pgfqpoint{3.850814in}{1.599795in}}%
\pgfpathlineto{\pgfqpoint{3.853358in}{1.602964in}}%
\pgfpathlineto{\pgfqpoint{3.856100in}{1.604222in}}%
\pgfpathlineto{\pgfqpoint{3.858720in}{1.604371in}}%
\pgfpathlineto{\pgfqpoint{3.861561in}{1.598689in}}%
\pgfpathlineto{\pgfqpoint{3.864073in}{1.599937in}}%
\pgfpathlineto{\pgfqpoint{3.866815in}{1.601249in}}%
\pgfpathlineto{\pgfqpoint{3.869435in}{1.604548in}}%
\pgfpathlineto{\pgfqpoint{3.872114in}{1.602862in}}%
\pgfpathlineto{\pgfqpoint{3.874790in}{1.602577in}}%
\pgfpathlineto{\pgfqpoint{3.877466in}{1.603288in}}%
\pgfpathlineto{\pgfqpoint{3.880237in}{1.603123in}}%
\pgfpathlineto{\pgfqpoint{3.882850in}{1.605670in}}%
\pgfpathlineto{\pgfqpoint{3.885621in}{1.602944in}}%
\pgfpathlineto{\pgfqpoint{3.888188in}{1.599920in}}%
\pgfpathlineto{\pgfqpoint{3.890926in}{1.600013in}}%
\pgfpathlineto{\pgfqpoint{3.893541in}{1.599352in}}%
\pgfpathlineto{\pgfqpoint{3.896345in}{1.596114in}}%
\pgfpathlineto{\pgfqpoint{3.898891in}{1.595091in}}%
\pgfpathlineto{\pgfqpoint{3.901573in}{1.596293in}}%
\pgfpathlineto{\pgfqpoint{3.904252in}{1.599753in}}%
\pgfpathlineto{\pgfqpoint{3.906918in}{1.600293in}}%
\pgfpathlineto{\pgfqpoint{3.909602in}{1.602208in}}%
\pgfpathlineto{\pgfqpoint{3.912296in}{1.602536in}}%
\pgfpathlineto{\pgfqpoint{3.915107in}{1.602160in}}%
\pgfpathlineto{\pgfqpoint{3.917646in}{1.600518in}}%
\pgfpathlineto{\pgfqpoint{3.920412in}{1.599618in}}%
\pgfpathlineto{\pgfqpoint{3.923005in}{1.599646in}}%
\pgfpathlineto{\pgfqpoint{3.925778in}{1.599897in}}%
\pgfpathlineto{\pgfqpoint{3.928347in}{1.602217in}}%
\pgfpathlineto{\pgfqpoint{3.931202in}{1.606609in}}%
\pgfpathlineto{\pgfqpoint{3.933714in}{1.605352in}}%
\pgfpathlineto{\pgfqpoint{3.936395in}{1.596092in}}%
\pgfpathlineto{\pgfqpoint{3.939075in}{1.601912in}}%
\pgfpathlineto{\pgfqpoint{3.941778in}{1.602843in}}%
\pgfpathlineto{\pgfqpoint{3.944431in}{1.601023in}}%
\pgfpathlineto{\pgfqpoint{3.947101in}{1.600087in}}%
\pgfpathlineto{\pgfqpoint{3.949894in}{1.599853in}}%
\pgfpathlineto{\pgfqpoint{3.952464in}{1.590381in}}%
\pgfpathlineto{\pgfqpoint{3.955211in}{1.591564in}}%
\pgfpathlineto{\pgfqpoint{3.957823in}{1.592232in}}%
\pgfpathlineto{\pgfqpoint{3.960635in}{1.593704in}}%
\pgfpathlineto{\pgfqpoint{3.963176in}{1.601544in}}%
\pgfpathlineto{\pgfqpoint{3.966013in}{1.596671in}}%
\pgfpathlineto{\pgfqpoint{3.968523in}{1.600283in}}%
\pgfpathlineto{\pgfqpoint{3.971250in}{1.602681in}}%
\pgfpathlineto{\pgfqpoint{3.973885in}{1.601938in}}%
\pgfpathlineto{\pgfqpoint{3.976563in}{1.600407in}}%
\pgfpathlineto{\pgfqpoint{3.979389in}{1.594931in}}%
\pgfpathlineto{\pgfqpoint{3.981929in}{1.601781in}}%
\pgfpathlineto{\pgfqpoint{3.984714in}{1.598030in}}%
\pgfpathlineto{\pgfqpoint{3.987270in}{1.600809in}}%
\pgfpathlineto{\pgfqpoint{3.990055in}{1.598621in}}%
\pgfpathlineto{\pgfqpoint{3.992642in}{1.600669in}}%
\pgfpathlineto{\pgfqpoint{3.995417in}{1.600628in}}%
\pgfpathlineto{\pgfqpoint{3.997990in}{1.596710in}}%
\pgfpathlineto{\pgfqpoint{4.000674in}{1.603375in}}%
\pgfpathlineto{\pgfqpoint{4.003348in}{1.601512in}}%
\pgfpathlineto{\pgfqpoint{4.006034in}{1.602431in}}%
\pgfpathlineto{\pgfqpoint{4.008699in}{1.601601in}}%
\pgfpathlineto{\pgfqpoint{4.011394in}{1.599425in}}%
\pgfpathlineto{\pgfqpoint{4.014186in}{1.605141in}}%
\pgfpathlineto{\pgfqpoint{4.016744in}{1.599940in}}%
\pgfpathlineto{\pgfqpoint{4.019518in}{1.598479in}}%
\pgfpathlineto{\pgfqpoint{4.022097in}{1.602065in}}%
\pgfpathlineto{\pgfqpoint{4.024868in}{1.597998in}}%
\pgfpathlineto{\pgfqpoint{4.027447in}{1.600412in}}%
\pgfpathlineto{\pgfqpoint{4.030229in}{1.599023in}}%
\pgfpathlineto{\pgfqpoint{4.032817in}{1.601909in}}%
\pgfpathlineto{\pgfqpoint{4.035492in}{1.600460in}}%
\pgfpathlineto{\pgfqpoint{4.038174in}{1.605039in}}%
\pgfpathlineto{\pgfqpoint{4.040852in}{1.603011in}}%
\pgfpathlineto{\pgfqpoint{4.043667in}{1.605588in}}%
\pgfpathlineto{\pgfqpoint{4.046210in}{1.602064in}}%
\pgfpathlineto{\pgfqpoint{4.049006in}{1.605221in}}%
\pgfpathlineto{\pgfqpoint{4.051557in}{1.603030in}}%
\pgfpathlineto{\pgfqpoint{4.054326in}{1.601557in}}%
\pgfpathlineto{\pgfqpoint{4.056911in}{1.603331in}}%
\pgfpathlineto{\pgfqpoint{4.059702in}{1.603159in}}%
\pgfpathlineto{\pgfqpoint{4.062266in}{1.604988in}}%
\pgfpathlineto{\pgfqpoint{4.064957in}{1.604173in}}%
\pgfpathlineto{\pgfqpoint{4.067636in}{1.608068in}}%
\pgfpathlineto{\pgfqpoint{4.070313in}{1.605548in}}%
\pgfpathlineto{\pgfqpoint{4.072985in}{1.605432in}}%
\pgfpathlineto{\pgfqpoint{4.075705in}{1.602474in}}%
\pgfpathlineto{\pgfqpoint{4.078471in}{1.601296in}}%
\pgfpathlineto{\pgfqpoint{4.081018in}{1.603187in}}%
\pgfpathlineto{\pgfqpoint{4.083870in}{1.604041in}}%
\pgfpathlineto{\pgfqpoint{4.086385in}{1.603874in}}%
\pgfpathlineto{\pgfqpoint{4.089159in}{1.608201in}}%
\pgfpathlineto{\pgfqpoint{4.091729in}{1.598989in}}%
\pgfpathlineto{\pgfqpoint{4.094527in}{1.597809in}}%
\pgfpathlineto{\pgfqpoint{4.097092in}{1.598371in}}%
\pgfpathlineto{\pgfqpoint{4.099777in}{1.598228in}}%
\pgfpathlineto{\pgfqpoint{4.102456in}{1.600741in}}%
\pgfpathlineto{\pgfqpoint{4.105185in}{1.596104in}}%
\pgfpathlineto{\pgfqpoint{4.107814in}{1.594564in}}%
\pgfpathlineto{\pgfqpoint{4.110488in}{1.597859in}}%
\pgfpathlineto{\pgfqpoint{4.113252in}{1.598005in}}%
\pgfpathlineto{\pgfqpoint{4.115844in}{1.602625in}}%
\pgfpathlineto{\pgfqpoint{4.118554in}{1.605344in}}%
\pgfpathlineto{\pgfqpoint{4.121205in}{1.603106in}}%
\pgfpathlineto{\pgfqpoint{4.124019in}{1.606283in}}%
\pgfpathlineto{\pgfqpoint{4.126553in}{1.601367in}}%
\pgfpathlineto{\pgfqpoint{4.129349in}{1.599219in}}%
\pgfpathlineto{\pgfqpoint{4.131920in}{1.596481in}}%
\pgfpathlineto{\pgfqpoint{4.134615in}{1.601606in}}%
\pgfpathlineto{\pgfqpoint{4.137272in}{1.604403in}}%
\pgfpathlineto{\pgfqpoint{4.139963in}{1.605105in}}%
\pgfpathlineto{\pgfqpoint{4.142713in}{1.606945in}}%
\pgfpathlineto{\pgfqpoint{4.145310in}{1.597805in}}%
\pgfpathlineto{\pgfqpoint{4.148082in}{1.600389in}}%
\pgfpathlineto{\pgfqpoint{4.150665in}{1.598743in}}%
\pgfpathlineto{\pgfqpoint{4.153423in}{1.596621in}}%
\pgfpathlineto{\pgfqpoint{4.156016in}{1.596595in}}%
\pgfpathlineto{\pgfqpoint{4.158806in}{1.599587in}}%
\pgfpathlineto{\pgfqpoint{4.161380in}{1.597288in}}%
\pgfpathlineto{\pgfqpoint{4.164059in}{1.600928in}}%
\pgfpathlineto{\pgfqpoint{4.166737in}{1.604814in}}%
\pgfpathlineto{\pgfqpoint{4.169415in}{1.608952in}}%
\pgfpathlineto{\pgfqpoint{4.172093in}{1.605666in}}%
\pgfpathlineto{\pgfqpoint{4.174770in}{1.608522in}}%
\pgfpathlineto{\pgfqpoint{4.177593in}{1.610660in}}%
\pgfpathlineto{\pgfqpoint{4.180129in}{1.606214in}}%
\pgfpathlineto{\pgfqpoint{4.182899in}{1.606596in}}%
\pgfpathlineto{\pgfqpoint{4.185481in}{1.600546in}}%
\pgfpathlineto{\pgfqpoint{4.188318in}{1.597139in}}%
\pgfpathlineto{\pgfqpoint{4.190842in}{1.594752in}}%
\pgfpathlineto{\pgfqpoint{4.193638in}{1.597954in}}%
\pgfpathlineto{\pgfqpoint{4.196186in}{1.598990in}}%
\pgfpathlineto{\pgfqpoint{4.198878in}{1.599648in}}%
\pgfpathlineto{\pgfqpoint{4.201542in}{1.592850in}}%
\pgfpathlineto{\pgfqpoint{4.204240in}{1.597704in}}%
\pgfpathlineto{\pgfqpoint{4.207076in}{1.602074in}}%
\pgfpathlineto{\pgfqpoint{4.209597in}{1.596219in}}%
\pgfpathlineto{\pgfqpoint{4.212383in}{1.596829in}}%
\pgfpathlineto{\pgfqpoint{4.214948in}{1.599449in}}%
\pgfpathlineto{\pgfqpoint{4.217694in}{1.595002in}}%
\pgfpathlineto{\pgfqpoint{4.220304in}{1.599107in}}%
\pgfpathlineto{\pgfqpoint{4.223082in}{1.597157in}}%
\pgfpathlineto{\pgfqpoint{4.225654in}{1.604377in}}%
\pgfpathlineto{\pgfqpoint{4.228331in}{1.613761in}}%
\pgfpathlineto{\pgfqpoint{4.231013in}{1.613890in}}%
\pgfpathlineto{\pgfqpoint{4.233691in}{1.610107in}}%
\pgfpathlineto{\pgfqpoint{4.236375in}{1.610549in}}%
\pgfpathlineto{\pgfqpoint{4.239084in}{1.608170in}}%
\pgfpathlineto{\pgfqpoint{4.241900in}{1.603710in}}%
\pgfpathlineto{\pgfqpoint{4.244394in}{1.604260in}}%
\pgfpathlineto{\pgfqpoint{4.247225in}{1.607906in}}%
\pgfpathlineto{\pgfqpoint{4.249767in}{1.608724in}}%
\pgfpathlineto{\pgfqpoint{4.252581in}{1.602784in}}%
\pgfpathlineto{\pgfqpoint{4.255120in}{1.603934in}}%
\pgfpathlineto{\pgfqpoint{4.257958in}{1.609901in}}%
\pgfpathlineto{\pgfqpoint{4.260477in}{1.606011in}}%
\pgfpathlineto{\pgfqpoint{4.263157in}{1.608960in}}%
\pgfpathlineto{\pgfqpoint{4.265824in}{1.606890in}}%
\pgfpathlineto{\pgfqpoint{4.268590in}{1.612924in}}%
\pgfpathlineto{\pgfqpoint{4.271187in}{1.605638in}}%
\pgfpathlineto{\pgfqpoint{4.273874in}{1.605009in}}%
\pgfpathlineto{\pgfqpoint{4.276635in}{1.600529in}}%
\pgfpathlineto{\pgfqpoint{4.279212in}{1.594606in}}%
\pgfpathlineto{\pgfqpoint{4.282000in}{1.599535in}}%
\pgfpathlineto{\pgfqpoint{4.284586in}{1.595899in}}%
\pgfpathlineto{\pgfqpoint{4.287399in}{1.596926in}}%
\pgfpathlineto{\pgfqpoint{4.289936in}{1.601058in}}%
\pgfpathlineto{\pgfqpoint{4.292786in}{1.594943in}}%
\pgfpathlineto{\pgfqpoint{4.295299in}{1.598583in}}%
\pgfpathlineto{\pgfqpoint{4.297977in}{1.600707in}}%
\pgfpathlineto{\pgfqpoint{4.300656in}{1.601850in}}%
\pgfpathlineto{\pgfqpoint{4.303357in}{1.601896in}}%
\pgfpathlineto{\pgfqpoint{4.306118in}{1.605201in}}%
\pgfpathlineto{\pgfqpoint{4.308691in}{1.600995in}}%
\pgfpathlineto{\pgfqpoint{4.311494in}{1.605745in}}%
\pgfpathlineto{\pgfqpoint{4.314032in}{1.605051in}}%
\pgfpathlineto{\pgfqpoint{4.316856in}{1.607200in}}%
\pgfpathlineto{\pgfqpoint{4.319405in}{1.605017in}}%
\pgfpathlineto{\pgfqpoint{4.322181in}{1.596968in}}%
\pgfpathlineto{\pgfqpoint{4.324760in}{1.600104in}}%
\pgfpathlineto{\pgfqpoint{4.327440in}{1.600135in}}%
\pgfpathlineto{\pgfqpoint{4.330118in}{1.610929in}}%
\pgfpathlineto{\pgfqpoint{4.332796in}{1.610669in}}%
\pgfpathlineto{\pgfqpoint{4.335463in}{1.612407in}}%
\pgfpathlineto{\pgfqpoint{4.338154in}{1.614317in}}%
\pgfpathlineto{\pgfqpoint{4.340976in}{1.612238in}}%
\pgfpathlineto{\pgfqpoint{4.343510in}{1.603879in}}%
\pgfpathlineto{\pgfqpoint{4.346263in}{1.599813in}}%
\pgfpathlineto{\pgfqpoint{4.348868in}{1.602117in}}%
\pgfpathlineto{\pgfqpoint{4.351645in}{1.602903in}}%
\pgfpathlineto{\pgfqpoint{4.354224in}{1.606960in}}%
\pgfpathlineto{\pgfqpoint{4.357014in}{1.604251in}}%
\pgfpathlineto{\pgfqpoint{4.359582in}{1.602802in}}%
\pgfpathlineto{\pgfqpoint{4.362270in}{1.604808in}}%
\pgfpathlineto{\pgfqpoint{4.364936in}{1.608142in}}%
\pgfpathlineto{\pgfqpoint{4.367646in}{1.607972in}}%
\pgfpathlineto{\pgfqpoint{4.370437in}{1.604227in}}%
\pgfpathlineto{\pgfqpoint{4.372976in}{1.598523in}}%
\pgfpathlineto{\pgfqpoint{4.375761in}{1.608097in}}%
\pgfpathlineto{\pgfqpoint{4.378329in}{1.631305in}}%
\pgfpathlineto{\pgfqpoint{4.381097in}{1.637688in}}%
\pgfpathlineto{\pgfqpoint{4.383674in}{1.621736in}}%
\pgfpathlineto{\pgfqpoint{4.386431in}{1.614706in}}%
\pgfpathlineto{\pgfqpoint{4.389044in}{1.610446in}}%
\pgfpathlineto{\pgfqpoint{4.391721in}{1.609949in}}%
\pgfpathlineto{\pgfqpoint{4.394400in}{1.609811in}}%
\pgfpathlineto{\pgfqpoint{4.397076in}{1.614095in}}%
\pgfpathlineto{\pgfqpoint{4.399745in}{1.618373in}}%
\pgfpathlineto{\pgfqpoint{4.402468in}{1.615423in}}%
\pgfpathlineto{\pgfqpoint{4.405234in}{1.608078in}}%
\pgfpathlineto{\pgfqpoint{4.407788in}{1.605698in}}%
\pgfpathlineto{\pgfqpoint{4.410587in}{1.605903in}}%
\pgfpathlineto{\pgfqpoint{4.413149in}{1.601171in}}%
\pgfpathlineto{\pgfqpoint{4.415932in}{1.600588in}}%
\pgfpathlineto{\pgfqpoint{4.418506in}{1.599433in}}%
\pgfpathlineto{\pgfqpoint{4.421292in}{1.601844in}}%
\pgfpathlineto{\pgfqpoint{4.423863in}{1.610928in}}%
\pgfpathlineto{\pgfqpoint{4.426534in}{1.601768in}}%
\pgfpathlineto{\pgfqpoint{4.429220in}{1.602127in}}%
\pgfpathlineto{\pgfqpoint{4.431901in}{1.600693in}}%
\pgfpathlineto{\pgfqpoint{4.434569in}{1.599197in}}%
\pgfpathlineto{\pgfqpoint{4.437253in}{1.600430in}}%
\pgfpathlineto{\pgfqpoint{4.440041in}{1.592954in}}%
\pgfpathlineto{\pgfqpoint{4.442611in}{1.588473in}}%
\pgfpathlineto{\pgfqpoint{4.445423in}{1.595190in}}%
\pgfpathlineto{\pgfqpoint{4.447965in}{1.595640in}}%
\pgfpathlineto{\pgfqpoint{4.450767in}{1.599527in}}%
\pgfpathlineto{\pgfqpoint{4.453312in}{1.598438in}}%
\pgfpathlineto{\pgfqpoint{4.456138in}{1.592145in}}%
\pgfpathlineto{\pgfqpoint{4.458681in}{1.599644in}}%
\pgfpathlineto{\pgfqpoint{4.461367in}{1.606692in}}%
\pgfpathlineto{\pgfqpoint{4.464029in}{1.605804in}}%
\pgfpathlineto{\pgfqpoint{4.466717in}{1.605390in}}%
\pgfpathlineto{\pgfqpoint{4.469492in}{1.603185in}}%
\pgfpathlineto{\pgfqpoint{4.472059in}{1.603031in}}%
\pgfpathlineto{\pgfqpoint{4.474861in}{1.605523in}}%
\pgfpathlineto{\pgfqpoint{4.477430in}{1.605606in}}%
\pgfpathlineto{\pgfqpoint{4.480201in}{1.604352in}}%
\pgfpathlineto{\pgfqpoint{4.482778in}{1.604230in}}%
\pgfpathlineto{\pgfqpoint{4.485581in}{1.607965in}}%
\pgfpathlineto{\pgfqpoint{4.488130in}{1.605829in}}%
\pgfpathlineto{\pgfqpoint{4.490822in}{1.609998in}}%
\pgfpathlineto{\pgfqpoint{4.493492in}{1.610361in}}%
\pgfpathlineto{\pgfqpoint{4.496167in}{1.608874in}}%
\pgfpathlineto{\pgfqpoint{4.498850in}{1.597937in}}%
\pgfpathlineto{\pgfqpoint{4.501529in}{1.609413in}}%
\pgfpathlineto{\pgfqpoint{4.504305in}{1.601324in}}%
\pgfpathlineto{\pgfqpoint{4.506893in}{1.602138in}}%
\pgfpathlineto{\pgfqpoint{4.509643in}{1.603041in}}%
\pgfpathlineto{\pgfqpoint{4.512246in}{1.609919in}}%
\pgfpathlineto{\pgfqpoint{4.515080in}{1.608669in}}%
\pgfpathlineto{\pgfqpoint{4.517598in}{1.605161in}}%
\pgfpathlineto{\pgfqpoint{4.520345in}{1.600476in}}%
\pgfpathlineto{\pgfqpoint{4.522962in}{1.601092in}}%
\pgfpathlineto{\pgfqpoint{4.525640in}{1.609044in}}%
\pgfpathlineto{\pgfqpoint{4.528307in}{1.612174in}}%
\pgfpathlineto{\pgfqpoint{4.530990in}{1.622355in}}%
\pgfpathlineto{\pgfqpoint{4.533764in}{1.622571in}}%
\pgfpathlineto{\pgfqpoint{4.536400in}{1.614976in}}%
\pgfpathlineto{\pgfqpoint{4.539144in}{1.611282in}}%
\pgfpathlineto{\pgfqpoint{4.541711in}{1.601653in}}%
\pgfpathlineto{\pgfqpoint{4.544464in}{1.605120in}}%
\pgfpathlineto{\pgfqpoint{4.547064in}{1.606145in}}%
\pgfpathlineto{\pgfqpoint{4.549822in}{1.604156in}}%
\pgfpathlineto{\pgfqpoint{4.552425in}{1.604178in}}%
\pgfpathlineto{\pgfqpoint{4.555106in}{1.601015in}}%
\pgfpathlineto{\pgfqpoint{4.557777in}{1.601187in}}%
\pgfpathlineto{\pgfqpoint{4.560448in}{1.598715in}}%
\pgfpathlineto{\pgfqpoint{4.563125in}{1.596045in}}%
\pgfpathlineto{\pgfqpoint{4.565820in}{1.597754in}}%
\pgfpathlineto{\pgfqpoint{4.568612in}{1.603063in}}%
\pgfpathlineto{\pgfqpoint{4.571171in}{1.595248in}}%
\pgfpathlineto{\pgfqpoint{4.573947in}{1.593955in}}%
\pgfpathlineto{\pgfqpoint{4.576531in}{1.596264in}}%
\pgfpathlineto{\pgfqpoint{4.579305in}{1.601096in}}%
\pgfpathlineto{\pgfqpoint{4.581888in}{1.598626in}}%
\pgfpathlineto{\pgfqpoint{4.584672in}{1.602813in}}%
\pgfpathlineto{\pgfqpoint{4.587244in}{1.597040in}}%
\pgfpathlineto{\pgfqpoint{4.589920in}{1.595731in}}%
\pgfpathlineto{\pgfqpoint{4.592589in}{1.597930in}}%
\pgfpathlineto{\pgfqpoint{4.595281in}{1.600523in}}%
\pgfpathlineto{\pgfqpoint{4.597951in}{1.604005in}}%
\pgfpathlineto{\pgfqpoint{4.600633in}{1.607455in}}%
\pgfpathlineto{\pgfqpoint{4.603430in}{1.608567in}}%
\pgfpathlineto{\pgfqpoint{4.605990in}{1.608747in}}%
\pgfpathlineto{\pgfqpoint{4.608808in}{1.606806in}}%
\pgfpathlineto{\pgfqpoint{4.611350in}{1.609405in}}%
\pgfpathlineto{\pgfqpoint{4.614134in}{1.605038in}}%
\pgfpathlineto{\pgfqpoint{4.616702in}{1.601327in}}%
\pgfpathlineto{\pgfqpoint{4.619529in}{1.602824in}}%
\pgfpathlineto{\pgfqpoint{4.622056in}{1.605080in}}%
\pgfpathlineto{\pgfqpoint{4.624741in}{1.604501in}}%
\pgfpathlineto{\pgfqpoint{4.627411in}{1.604772in}}%
\pgfpathlineto{\pgfqpoint{4.630096in}{1.607146in}}%
\pgfpathlineto{\pgfqpoint{4.632902in}{1.607890in}}%
\pgfpathlineto{\pgfqpoint{4.635445in}{1.608523in}}%
\pgfpathlineto{\pgfqpoint{4.638204in}{1.604491in}}%
\pgfpathlineto{\pgfqpoint{4.640809in}{1.602416in}}%
\pgfpathlineto{\pgfqpoint{4.643628in}{1.605380in}}%
\pgfpathlineto{\pgfqpoint{4.646169in}{1.599420in}}%
\pgfpathlineto{\pgfqpoint{4.648922in}{1.601629in}}%
\pgfpathlineto{\pgfqpoint{4.651524in}{1.605000in}}%
\pgfpathlineto{\pgfqpoint{4.654203in}{1.602794in}}%
\pgfpathlineto{\pgfqpoint{4.656873in}{1.602420in}}%
\pgfpathlineto{\pgfqpoint{4.659590in}{1.604538in}}%
\pgfpathlineto{\pgfqpoint{4.662237in}{1.602024in}}%
\pgfpathlineto{\pgfqpoint{4.664923in}{1.601583in}}%
\pgfpathlineto{\pgfqpoint{4.667764in}{1.604490in}}%
\pgfpathlineto{\pgfqpoint{4.670261in}{1.600659in}}%
\pgfpathlineto{\pgfqpoint{4.673068in}{1.603110in}}%
\pgfpathlineto{\pgfqpoint{4.675619in}{1.603190in}}%
\pgfpathlineto{\pgfqpoint{4.678448in}{1.606418in}}%
\pgfpathlineto{\pgfqpoint{4.680988in}{1.607918in}}%
\pgfpathlineto{\pgfqpoint{4.683799in}{1.607679in}}%
\pgfpathlineto{\pgfqpoint{4.686337in}{1.604212in}}%
\pgfpathlineto{\pgfqpoint{4.689051in}{1.605992in}}%
\pgfpathlineto{\pgfqpoint{4.691694in}{1.604513in}}%
\pgfpathlineto{\pgfqpoint{4.694381in}{1.604601in}}%
\pgfpathlineto{\pgfqpoint{4.697170in}{1.605522in}}%
\pgfpathlineto{\pgfqpoint{4.699734in}{1.601909in}}%
\pgfpathlineto{\pgfqpoint{4.702517in}{1.601646in}}%
\pgfpathlineto{\pgfqpoint{4.705094in}{1.605483in}}%
\pgfpathlineto{\pgfqpoint{4.707824in}{1.608464in}}%
\pgfpathlineto{\pgfqpoint{4.710437in}{1.603995in}}%
\pgfpathlineto{\pgfqpoint{4.713275in}{1.605732in}}%
\pgfpathlineto{\pgfqpoint{4.715806in}{1.610569in}}%
\pgfpathlineto{\pgfqpoint{4.718486in}{1.608582in}}%
\pgfpathlineto{\pgfqpoint{4.721160in}{1.607274in}}%
\pgfpathlineto{\pgfqpoint{4.723873in}{1.611902in}}%
\pgfpathlineto{\pgfqpoint{4.726508in}{1.613894in}}%
\pgfpathlineto{\pgfqpoint{4.729233in}{1.608757in}}%
\pgfpathlineto{\pgfqpoint{4.731901in}{1.603993in}}%
\pgfpathlineto{\pgfqpoint{4.734552in}{1.605919in}}%
\pgfpathlineto{\pgfqpoint{4.737348in}{1.603359in}}%
\pgfpathlineto{\pgfqpoint{4.739912in}{1.602325in}}%
\pgfpathlineto{\pgfqpoint{4.742696in}{1.605843in}}%
\pgfpathlineto{\pgfqpoint{4.745256in}{1.605728in}}%
\pgfpathlineto{\pgfqpoint{4.748081in}{1.604428in}}%
\pgfpathlineto{\pgfqpoint{4.750627in}{1.607249in}}%
\pgfpathlineto{\pgfqpoint{4.753298in}{1.600972in}}%
\pgfpathlineto{\pgfqpoint{4.755983in}{1.601493in}}%
\pgfpathlineto{\pgfqpoint{4.758653in}{1.600542in}}%
\pgfpathlineto{\pgfqpoint{4.761337in}{1.602849in}}%
\pgfpathlineto{\pgfqpoint{4.764018in}{1.613805in}}%
\pgfpathlineto{\pgfqpoint{4.766783in}{1.605232in}}%
\pgfpathlineto{\pgfqpoint{4.769367in}{1.596382in}}%
\pgfpathlineto{\pgfqpoint{4.772198in}{1.602035in}}%
\pgfpathlineto{\pgfqpoint{4.774732in}{1.596942in}}%
\pgfpathlineto{\pgfqpoint{4.777535in}{1.598975in}}%
\pgfpathlineto{\pgfqpoint{4.780083in}{1.602996in}}%
\pgfpathlineto{\pgfqpoint{4.782872in}{1.599538in}}%
\pgfpathlineto{\pgfqpoint{4.785445in}{1.603851in}}%
\pgfpathlineto{\pgfqpoint{4.788116in}{1.606155in}}%
\pgfpathlineto{\pgfqpoint{4.790798in}{1.600261in}}%
\pgfpathlineto{\pgfqpoint{4.793512in}{1.599554in}}%
\pgfpathlineto{\pgfqpoint{4.796274in}{1.604341in}}%
\pgfpathlineto{\pgfqpoint{4.798830in}{1.601137in}}%
\pgfpathlineto{\pgfqpoint{4.801586in}{1.597511in}}%
\pgfpathlineto{\pgfqpoint{4.804193in}{1.592138in}}%
\pgfpathlineto{\pgfqpoint{4.807017in}{1.599056in}}%
\pgfpathlineto{\pgfqpoint{4.809538in}{1.602759in}}%
\pgfpathlineto{\pgfqpoint{4.812377in}{1.607943in}}%
\pgfpathlineto{\pgfqpoint{4.814907in}{1.595671in}}%
\pgfpathlineto{\pgfqpoint{4.817587in}{1.588658in}}%
\pgfpathlineto{\pgfqpoint{4.820265in}{1.592041in}}%
\pgfpathlineto{\pgfqpoint{4.822945in}{1.599127in}}%
\pgfpathlineto{\pgfqpoint{4.825619in}{1.594708in}}%
\pgfpathlineto{\pgfqpoint{4.828291in}{1.588894in}}%
\pgfpathlineto{\pgfqpoint{4.831045in}{1.590409in}}%
\pgfpathlineto{\pgfqpoint{4.833657in}{1.593645in}}%
\pgfpathlineto{\pgfqpoint{4.837992in}{1.598902in}}%
\pgfpathlineto{\pgfqpoint{4.839922in}{1.607578in}}%
\pgfpathlineto{\pgfqpoint{4.842380in}{1.605437in}}%
\pgfpathlineto{\pgfqpoint{4.844361in}{1.601604in}}%
\pgfpathlineto{\pgfqpoint{4.847127in}{1.601722in}}%
\pgfpathlineto{\pgfqpoint{4.849715in}{1.601534in}}%
\pgfpathlineto{\pgfqpoint{4.852404in}{1.604689in}}%
\pgfpathlineto{\pgfqpoint{4.855070in}{1.602066in}}%
\pgfpathlineto{\pgfqpoint{4.857807in}{1.604266in}}%
\pgfpathlineto{\pgfqpoint{4.860544in}{1.609162in}}%
\pgfpathlineto{\pgfqpoint{4.863116in}{1.606662in}}%
\pgfpathlineto{\pgfqpoint{4.865910in}{1.605725in}}%
\pgfpathlineto{\pgfqpoint{4.868474in}{1.608653in}}%
\pgfpathlineto{\pgfqpoint{4.871209in}{1.604635in}}%
\pgfpathlineto{\pgfqpoint{4.873832in}{1.606487in}}%
\pgfpathlineto{\pgfqpoint{4.876636in}{1.604489in}}%
\pgfpathlineto{\pgfqpoint{4.879180in}{1.602174in}}%
\pgfpathlineto{\pgfqpoint{4.881864in}{1.604673in}}%
\pgfpathlineto{\pgfqpoint{4.884540in}{1.601492in}}%
\pgfpathlineto{\pgfqpoint{4.887211in}{1.597369in}}%
\pgfpathlineto{\pgfqpoint{4.889902in}{1.596784in}}%
\pgfpathlineto{\pgfqpoint{4.892611in}{1.600120in}}%
\pgfpathlineto{\pgfqpoint{4.895399in}{1.600953in}}%
\pgfpathlineto{\pgfqpoint{4.897938in}{1.604467in}}%
\pgfpathlineto{\pgfqpoint{4.900712in}{1.603291in}}%
\pgfpathlineto{\pgfqpoint{4.903295in}{1.601503in}}%
\pgfpathlineto{\pgfqpoint{4.906096in}{1.590911in}}%
\pgfpathlineto{\pgfqpoint{4.908648in}{1.587582in}}%
\pgfpathlineto{\pgfqpoint{4.911435in}{1.590210in}}%
\pgfpathlineto{\pgfqpoint{4.914009in}{1.586892in}}%
\pgfpathlineto{\pgfqpoint{4.916681in}{1.591048in}}%
\pgfpathlineto{\pgfqpoint{4.919352in}{1.593712in}}%
\pgfpathlineto{\pgfqpoint{4.922041in}{1.597633in}}%
\pgfpathlineto{\pgfqpoint{4.924708in}{1.600213in}}%
\pgfpathlineto{\pgfqpoint{4.927400in}{1.598599in}}%
\pgfpathlineto{\pgfqpoint{4.930170in}{1.599830in}}%
\pgfpathlineto{\pgfqpoint{4.932742in}{1.600661in}}%
\pgfpathlineto{\pgfqpoint{4.935515in}{1.601589in}}%
\pgfpathlineto{\pgfqpoint{4.938112in}{1.595179in}}%
\pgfpathlineto{\pgfqpoint{4.940881in}{1.597364in}}%
\pgfpathlineto{\pgfqpoint{4.943466in}{1.606099in}}%
\pgfpathlineto{\pgfqpoint{4.946151in}{1.601114in}}%
\pgfpathlineto{\pgfqpoint{4.948827in}{1.604332in}}%
\pgfpathlineto{\pgfqpoint{4.951504in}{1.607612in}}%
\pgfpathlineto{\pgfqpoint{4.954182in}{1.618294in}}%
\pgfpathlineto{\pgfqpoint{4.956862in}{1.649546in}}%
\pgfpathlineto{\pgfqpoint{4.959689in}{1.659349in}}%
\pgfpathlineto{\pgfqpoint{4.962219in}{1.640594in}}%
\pgfpathlineto{\pgfqpoint{4.965002in}{1.632143in}}%
\pgfpathlineto{\pgfqpoint{4.967575in}{1.629942in}}%
\pgfpathlineto{\pgfqpoint{4.970314in}{1.626882in}}%
\pgfpathlineto{\pgfqpoint{4.972933in}{1.625794in}}%
\pgfpathlineto{\pgfqpoint{4.975703in}{1.623637in}}%
\pgfpathlineto{\pgfqpoint{4.978287in}{1.624613in}}%
\pgfpathlineto{\pgfqpoint{4.980967in}{1.624494in}}%
\pgfpathlineto{\pgfqpoint{4.983637in}{1.620404in}}%
\pgfpathlineto{\pgfqpoint{4.986325in}{1.617777in}}%
\pgfpathlineto{\pgfqpoint{4.989001in}{1.628802in}}%
\pgfpathlineto{\pgfqpoint{4.991683in}{1.616347in}}%
\pgfpathlineto{\pgfqpoint{4.994390in}{1.617131in}}%
\pgfpathlineto{\pgfqpoint{4.997028in}{1.614597in}}%
\pgfpathlineto{\pgfqpoint{4.999780in}{1.612667in}}%
\pgfpathlineto{\pgfqpoint{5.002384in}{1.614606in}}%
\pgfpathlineto{\pgfqpoint{5.005178in}{1.615430in}}%
\pgfpathlineto{\pgfqpoint{5.007751in}{1.612049in}}%
\pgfpathlineto{\pgfqpoint{5.010562in}{1.607518in}}%
\pgfpathlineto{\pgfqpoint{5.013104in}{1.608696in}}%
\pgfpathlineto{\pgfqpoint{5.015820in}{1.611224in}}%
\pgfpathlineto{\pgfqpoint{5.018466in}{1.611218in}}%
\pgfpathlineto{\pgfqpoint{5.021147in}{1.609445in}}%
\pgfpathlineto{\pgfqpoint{5.023927in}{1.611117in}}%
\pgfpathlineto{\pgfqpoint{5.026501in}{1.614506in}}%
\pgfpathlineto{\pgfqpoint{5.029275in}{1.610285in}}%
\pgfpathlineto{\pgfqpoint{5.031849in}{1.609152in}}%
\pgfpathlineto{\pgfqpoint{5.034649in}{1.610563in}}%
\pgfpathlineto{\pgfqpoint{5.037214in}{1.606177in}}%
\pgfpathlineto{\pgfqpoint{5.039962in}{1.601205in}}%
\pgfpathlineto{\pgfqpoint{5.042572in}{1.599152in}}%
\pgfpathlineto{\pgfqpoint{5.045249in}{1.601439in}}%
\pgfpathlineto{\pgfqpoint{5.047924in}{1.604406in}}%
\pgfpathlineto{\pgfqpoint{5.050606in}{1.602944in}}%
\pgfpathlineto{\pgfqpoint{5.053284in}{1.603378in}}%
\pgfpathlineto{\pgfqpoint{5.055952in}{1.605527in}}%
\pgfpathlineto{\pgfqpoint{5.058711in}{1.604711in}}%
\pgfpathlineto{\pgfqpoint{5.061315in}{1.605653in}}%
\pgfpathlineto{\pgfqpoint{5.064144in}{1.604909in}}%
\pgfpathlineto{\pgfqpoint{5.066677in}{1.603823in}}%
\pgfpathlineto{\pgfqpoint{5.069463in}{1.605093in}}%
\pgfpathlineto{\pgfqpoint{5.072030in}{1.602414in}}%
\pgfpathlineto{\pgfqpoint{5.074851in}{1.607016in}}%
\pgfpathlineto{\pgfqpoint{5.077390in}{1.605672in}}%
\pgfpathlineto{\pgfqpoint{5.080067in}{1.598320in}}%
\pgfpathlineto{\pgfqpoint{5.082746in}{1.605386in}}%
\pgfpathlineto{\pgfqpoint{5.085426in}{1.601964in}}%
\pgfpathlineto{\pgfqpoint{5.088103in}{1.602288in}}%
\pgfpathlineto{\pgfqpoint{5.090788in}{1.602538in}}%
\pgfpathlineto{\pgfqpoint{5.093579in}{1.605589in}}%
\pgfpathlineto{\pgfqpoint{5.096142in}{1.602145in}}%
\pgfpathlineto{\pgfqpoint{5.098948in}{1.601296in}}%
\pgfpathlineto{\pgfqpoint{5.101496in}{1.599864in}}%
\pgfpathlineto{\pgfqpoint{5.104312in}{1.603355in}}%
\pgfpathlineto{\pgfqpoint{5.106842in}{1.603854in}}%
\pgfpathlineto{\pgfqpoint{5.109530in}{1.603505in}}%
\pgfpathlineto{\pgfqpoint{5.112209in}{1.604717in}}%
\pgfpathlineto{\pgfqpoint{5.114887in}{1.605262in}}%
\pgfpathlineto{\pgfqpoint{5.117550in}{1.605129in}}%
\pgfpathlineto{\pgfqpoint{5.120243in}{1.609117in}}%
\pgfpathlineto{\pgfqpoint{5.123042in}{1.606631in}}%
\pgfpathlineto{\pgfqpoint{5.125599in}{1.605322in}}%
\pgfpathlineto{\pgfqpoint{5.128421in}{1.606460in}}%
\pgfpathlineto{\pgfqpoint{5.130953in}{1.602873in}}%
\pgfpathlineto{\pgfqpoint{5.133716in}{1.607904in}}%
\pgfpathlineto{\pgfqpoint{5.136311in}{1.608889in}}%
\pgfpathlineto{\pgfqpoint{5.139072in}{1.598253in}}%
\pgfpathlineto{\pgfqpoint{5.141660in}{1.608245in}}%
\pgfpathlineto{\pgfqpoint{5.144349in}{1.606249in}}%
\pgfpathlineto{\pgfqpoint{5.147029in}{1.607075in}}%
\pgfpathlineto{\pgfqpoint{5.149734in}{1.599878in}}%
\pgfpathlineto{\pgfqpoint{5.152382in}{1.590497in}}%
\pgfpathlineto{\pgfqpoint{5.155059in}{1.594456in}}%
\pgfpathlineto{\pgfqpoint{5.157815in}{1.600300in}}%
\pgfpathlineto{\pgfqpoint{5.160420in}{1.594761in}}%
\pgfpathlineto{\pgfqpoint{5.163243in}{1.599343in}}%
\pgfpathlineto{\pgfqpoint{5.165775in}{1.599073in}}%
\pgfpathlineto{\pgfqpoint{5.168591in}{1.603442in}}%
\pgfpathlineto{\pgfqpoint{5.171133in}{1.602845in}}%
\pgfpathlineto{\pgfqpoint{5.173925in}{1.601365in}}%
\pgfpathlineto{\pgfqpoint{5.176477in}{1.604364in}}%
\pgfpathlineto{\pgfqpoint{5.179188in}{1.601880in}}%
\pgfpathlineto{\pgfqpoint{5.181848in}{1.605021in}}%
\pgfpathlineto{\pgfqpoint{5.184522in}{1.603375in}}%
\pgfpathlineto{\pgfqpoint{5.187294in}{1.605329in}}%
\pgfpathlineto{\pgfqpoint{5.189880in}{1.604713in}}%
\pgfpathlineto{\pgfqpoint{5.192680in}{1.599465in}}%
\pgfpathlineto{\pgfqpoint{5.195239in}{1.604369in}}%
\pgfpathlineto{\pgfqpoint{5.198008in}{1.599639in}}%
\pgfpathlineto{\pgfqpoint{5.200594in}{1.604431in}}%
\pgfpathlineto{\pgfqpoint{5.203388in}{1.603543in}}%
\pgfpathlineto{\pgfqpoint{5.205952in}{1.605405in}}%
\pgfpathlineto{\pgfqpoint{5.208630in}{1.599277in}}%
\pgfpathlineto{\pgfqpoint{5.211299in}{1.602046in}}%
\pgfpathlineto{\pgfqpoint{5.214027in}{1.607710in}}%
\pgfpathlineto{\pgfqpoint{5.216667in}{1.606848in}}%
\pgfpathlineto{\pgfqpoint{5.219345in}{1.607225in}}%
\pgfpathlineto{\pgfqpoint{5.222151in}{1.606532in}}%
\pgfpathlineto{\pgfqpoint{5.224695in}{1.607753in}}%
\pgfpathlineto{\pgfqpoint{5.227470in}{1.604222in}}%
\pgfpathlineto{\pgfqpoint{5.230059in}{1.604847in}}%
\pgfpathlineto{\pgfqpoint{5.232855in}{1.608234in}}%
\pgfpathlineto{\pgfqpoint{5.235409in}{1.608390in}}%
\pgfpathlineto{\pgfqpoint{5.238173in}{1.603539in}}%
\pgfpathlineto{\pgfqpoint{5.240777in}{1.605939in}}%
\pgfpathlineto{\pgfqpoint{5.243445in}{1.605395in}}%
\pgfpathlineto{\pgfqpoint{5.246130in}{1.604182in}}%
\pgfpathlineto{\pgfqpoint{5.248816in}{1.602560in}}%
\pgfpathlineto{\pgfqpoint{5.251590in}{1.606924in}}%
\pgfpathlineto{\pgfqpoint{5.254236in}{1.606309in}}%
\pgfpathlineto{\pgfqpoint{5.256973in}{1.603455in}}%
\pgfpathlineto{\pgfqpoint{5.259511in}{1.604689in}}%
\pgfpathlineto{\pgfqpoint{5.262264in}{1.605208in}}%
\pgfpathlineto{\pgfqpoint{5.264876in}{1.606254in}}%
\pgfpathlineto{\pgfqpoint{5.267691in}{1.610612in}}%
\pgfpathlineto{\pgfqpoint{5.270238in}{1.603653in}}%
\pgfpathlineto{\pgfqpoint{5.272913in}{1.603928in}}%
\pgfpathlineto{\pgfqpoint{5.275589in}{1.607664in}}%
\pgfpathlineto{\pgfqpoint{5.278322in}{1.608843in}}%
\pgfpathlineto{\pgfqpoint{5.280947in}{1.623147in}}%
\pgfpathlineto{\pgfqpoint{5.283631in}{1.616737in}}%
\pgfpathlineto{\pgfqpoint{5.286436in}{1.609610in}}%
\pgfpathlineto{\pgfqpoint{5.288984in}{1.609126in}}%
\pgfpathlineto{\pgfqpoint{5.291794in}{1.605742in}}%
\pgfpathlineto{\pgfqpoint{5.294339in}{1.604347in}}%
\pgfpathlineto{\pgfqpoint{5.297140in}{1.608577in}}%
\pgfpathlineto{\pgfqpoint{5.299696in}{1.609331in}}%
\pgfpathlineto{\pgfqpoint{5.302443in}{1.603550in}}%
\pgfpathlineto{\pgfqpoint{5.305054in}{1.604671in}}%
\pgfpathlineto{\pgfqpoint{5.307731in}{1.606575in}}%
\pgfpathlineto{\pgfqpoint{5.310411in}{1.607890in}}%
\pgfpathlineto{\pgfqpoint{5.313089in}{1.608807in}}%
\pgfpathlineto{\pgfqpoint{5.315754in}{1.602526in}}%
\pgfpathlineto{\pgfqpoint{5.318430in}{1.600196in}}%
\pgfpathlineto{\pgfqpoint{5.321256in}{1.599616in}}%
\pgfpathlineto{\pgfqpoint{5.323802in}{1.599873in}}%
\pgfpathlineto{\pgfqpoint{5.326564in}{1.602387in}}%
\pgfpathlineto{\pgfqpoint{5.329159in}{1.610896in}}%
\pgfpathlineto{\pgfqpoint{5.331973in}{1.607186in}}%
\pgfpathlineto{\pgfqpoint{5.334510in}{1.606552in}}%
\pgfpathlineto{\pgfqpoint{5.337353in}{1.599029in}}%
\pgfpathlineto{\pgfqpoint{5.339872in}{1.596770in}}%
\pgfpathlineto{\pgfqpoint{5.342549in}{1.596155in}}%
\pgfpathlineto{\pgfqpoint{5.345224in}{1.590696in}}%
\pgfpathlineto{\pgfqpoint{5.347905in}{1.586534in}}%
\pgfpathlineto{\pgfqpoint{5.350723in}{1.591510in}}%
\pgfpathlineto{\pgfqpoint{5.353262in}{1.594901in}}%
\pgfpathlineto{\pgfqpoint{5.356056in}{1.598437in}}%
\pgfpathlineto{\pgfqpoint{5.358612in}{1.601000in}}%
\pgfpathlineto{\pgfqpoint{5.361370in}{1.602345in}}%
\pgfpathlineto{\pgfqpoint{5.363966in}{1.600897in}}%
\pgfpathlineto{\pgfqpoint{5.366727in}{1.602185in}}%
\pgfpathlineto{\pgfqpoint{5.369335in}{1.605672in}}%
\pgfpathlineto{\pgfqpoint{5.372013in}{1.604817in}}%
\pgfpathlineto{\pgfqpoint{5.374692in}{1.605707in}}%
\pgfpathlineto{\pgfqpoint{5.377370in}{1.608572in}}%
\pgfpathlineto{\pgfqpoint{5.380048in}{1.605341in}}%
\pgfpathlineto{\pgfqpoint{5.382725in}{1.608468in}}%
\pgfpathlineto{\pgfqpoint{5.385550in}{1.608982in}}%
\pgfpathlineto{\pgfqpoint{5.388083in}{1.607459in}}%
\pgfpathlineto{\pgfqpoint{5.390900in}{1.607246in}}%
\pgfpathlineto{\pgfqpoint{5.393441in}{1.607539in}}%
\pgfpathlineto{\pgfqpoint{5.396219in}{1.605873in}}%
\pgfpathlineto{\pgfqpoint{5.398784in}{1.604398in}}%
\pgfpathlineto{\pgfqpoint{5.401576in}{1.597748in}}%
\pgfpathlineto{\pgfqpoint{5.404154in}{1.602037in}}%
\pgfpathlineto{\pgfqpoint{5.406832in}{1.605968in}}%
\pgfpathlineto{\pgfqpoint{5.409507in}{1.606890in}}%
\pgfpathlineto{\pgfqpoint{5.412190in}{1.603886in}}%
\pgfpathlineto{\pgfqpoint{5.414954in}{1.605420in}}%
\pgfpathlineto{\pgfqpoint{5.417547in}{1.605054in}}%
\pgfpathlineto{\pgfqpoint{5.420304in}{1.601304in}}%
\pgfpathlineto{\pgfqpoint{5.422897in}{1.605322in}}%
\pgfpathlineto{\pgfqpoint{5.425661in}{1.610252in}}%
\pgfpathlineto{\pgfqpoint{5.428259in}{1.606555in}}%
\pgfpathlineto{\pgfqpoint{5.431015in}{1.607876in}}%
\pgfpathlineto{\pgfqpoint{5.433616in}{1.604491in}}%
\pgfpathlineto{\pgfqpoint{5.436295in}{1.603575in}}%
\pgfpathlineto{\pgfqpoint{5.438974in}{1.607093in}}%
\pgfpathlineto{\pgfqpoint{5.441698in}{1.603705in}}%
\pgfpathlineto{\pgfqpoint{5.444328in}{1.607232in}}%
\pgfpathlineto{\pgfqpoint{5.447021in}{1.606016in}}%
\pgfpathlineto{\pgfqpoint{5.449769in}{1.595977in}}%
\pgfpathlineto{\pgfqpoint{5.452365in}{1.601414in}}%
\pgfpathlineto{\pgfqpoint{5.455168in}{1.599706in}}%
\pgfpathlineto{\pgfqpoint{5.457721in}{1.600562in}}%
\pgfpathlineto{\pgfqpoint{5.460489in}{1.599333in}}%
\pgfpathlineto{\pgfqpoint{5.463079in}{1.602290in}}%
\pgfpathlineto{\pgfqpoint{5.465888in}{1.597504in}}%
\pgfpathlineto{\pgfqpoint{5.468425in}{1.605262in}}%
\pgfpathlineto{\pgfqpoint{5.471113in}{1.606868in}}%
\pgfpathlineto{\pgfqpoint{5.473792in}{1.609243in}}%
\pgfpathlineto{\pgfqpoint{5.476458in}{1.610950in}}%
\pgfpathlineto{\pgfqpoint{5.479152in}{1.608068in}}%
\pgfpathlineto{\pgfqpoint{5.481825in}{1.603914in}}%
\pgfpathlineto{\pgfqpoint{5.484641in}{1.609354in}}%
\pgfpathlineto{\pgfqpoint{5.487176in}{1.610351in}}%
\pgfpathlineto{\pgfqpoint{5.490000in}{1.608382in}}%
\pgfpathlineto{\pgfqpoint{5.492541in}{1.609132in}}%
\pgfpathlineto{\pgfqpoint{5.495346in}{1.602776in}}%
\pgfpathlineto{\pgfqpoint{5.497898in}{1.605937in}}%
\pgfpathlineto{\pgfqpoint{5.500687in}{1.608730in}}%
\pgfpathlineto{\pgfqpoint{5.503255in}{1.606847in}}%
\pgfpathlineto{\pgfqpoint{5.505933in}{1.605704in}}%
\pgfpathlineto{\pgfqpoint{5.508612in}{1.607607in}}%
\pgfpathlineto{\pgfqpoint{5.511290in}{1.605766in}}%
\pgfpathlineto{\pgfqpoint{5.514080in}{1.600027in}}%
\pgfpathlineto{\pgfqpoint{5.516646in}{1.596414in}}%
\pgfpathlineto{\pgfqpoint{5.519433in}{1.596635in}}%
\pgfpathlineto{\pgfqpoint{5.522003in}{1.598797in}}%
\pgfpathlineto{\pgfqpoint{5.524756in}{1.606274in}}%
\pgfpathlineto{\pgfqpoint{5.527360in}{1.606107in}}%
\pgfpathlineto{\pgfqpoint{5.530148in}{1.601739in}}%
\pgfpathlineto{\pgfqpoint{5.532717in}{1.602222in}}%
\pgfpathlineto{\pgfqpoint{5.535395in}{1.603410in}}%
\pgfpathlineto{\pgfqpoint{5.538074in}{1.603758in}}%
\pgfpathlineto{\pgfqpoint{5.540750in}{1.606603in}}%
\pgfpathlineto{\pgfqpoint{5.543421in}{1.605095in}}%
\pgfpathlineto{\pgfqpoint{5.546110in}{1.600162in}}%
\pgfpathlineto{\pgfqpoint{5.548921in}{1.601771in}}%
\pgfpathlineto{\pgfqpoint{5.551457in}{1.605880in}}%
\pgfpathlineto{\pgfqpoint{5.554198in}{1.605579in}}%
\pgfpathlineto{\pgfqpoint{5.556822in}{1.603440in}}%
\pgfpathlineto{\pgfqpoint{5.559612in}{1.601869in}}%
\pgfpathlineto{\pgfqpoint{5.562180in}{1.598079in}}%
\pgfpathlineto{\pgfqpoint{5.564940in}{1.597280in}}%
\pgfpathlineto{\pgfqpoint{5.567536in}{1.600530in}}%
\pgfpathlineto{\pgfqpoint{5.570215in}{1.598844in}}%
\pgfpathlineto{\pgfqpoint{5.572893in}{1.595400in}}%
\pgfpathlineto{\pgfqpoint{5.575596in}{1.599442in}}%
\pgfpathlineto{\pgfqpoint{5.578342in}{1.602078in}}%
\pgfpathlineto{\pgfqpoint{5.580914in}{1.602064in}}%
\pgfpathlineto{\pgfqpoint{5.583709in}{1.600123in}}%
\pgfpathlineto{\pgfqpoint{5.586269in}{1.598197in}}%
\pgfpathlineto{\pgfqpoint{5.589040in}{1.604146in}}%
\pgfpathlineto{\pgfqpoint{5.591641in}{1.602520in}}%
\pgfpathlineto{\pgfqpoint{5.594368in}{1.605719in}}%
\pgfpathlineto{\pgfqpoint{5.596999in}{1.604034in}}%
\pgfpathlineto{\pgfqpoint{5.599674in}{1.601526in}}%
\pgfpathlineto{\pgfqpoint{5.602352in}{1.603616in}}%
\pgfpathlineto{\pgfqpoint{5.605073in}{1.602004in}}%
\pgfpathlineto{\pgfqpoint{5.607698in}{1.601325in}}%
\pgfpathlineto{\pgfqpoint{5.610389in}{1.602669in}}%
\pgfpathlineto{\pgfqpoint{5.613235in}{1.608293in}}%
\pgfpathlineto{\pgfqpoint{5.615743in}{1.602259in}}%
\pgfpathlineto{\pgfqpoint{5.618526in}{1.601893in}}%
\pgfpathlineto{\pgfqpoint{5.621102in}{1.603084in}}%
\pgfpathlineto{\pgfqpoint{5.623868in}{1.599185in}}%
\pgfpathlineto{\pgfqpoint{5.626460in}{1.600576in}}%
\pgfpathlineto{\pgfqpoint{5.629232in}{1.601660in}}%
\pgfpathlineto{\pgfqpoint{5.631815in}{1.601253in}}%
\pgfpathlineto{\pgfqpoint{5.634496in}{1.602754in}}%
\pgfpathlineto{\pgfqpoint{5.637172in}{1.605240in}}%
\pgfpathlineto{\pgfqpoint{5.639852in}{1.612928in}}%
\pgfpathlineto{\pgfqpoint{5.642518in}{1.619541in}}%
\pgfpathlineto{\pgfqpoint{5.645243in}{1.614965in}}%
\pgfpathlineto{\pgfqpoint{5.648008in}{1.597148in}}%
\pgfpathlineto{\pgfqpoint{5.650563in}{1.603873in}}%
\pgfpathlineto{\pgfqpoint{5.653376in}{1.624296in}}%
\pgfpathlineto{\pgfqpoint{5.655919in}{1.616033in}}%
\pgfpathlineto{\pgfqpoint{5.658723in}{1.620993in}}%
\pgfpathlineto{\pgfqpoint{5.661273in}{1.612025in}}%
\pgfpathlineto{\pgfqpoint{5.664099in}{1.611703in}}%
\pgfpathlineto{\pgfqpoint{5.666632in}{1.607683in}}%
\pgfpathlineto{\pgfqpoint{5.669313in}{1.611896in}}%
\pgfpathlineto{\pgfqpoint{5.671991in}{1.606609in}}%
\pgfpathlineto{\pgfqpoint{5.674667in}{1.608254in}}%
\pgfpathlineto{\pgfqpoint{5.677486in}{1.604556in}}%
\pgfpathlineto{\pgfqpoint{5.680027in}{1.602518in}}%
\pgfpathlineto{\pgfqpoint{5.682836in}{1.599317in}}%
\pgfpathlineto{\pgfqpoint{5.685385in}{1.600357in}}%
\pgfpathlineto{\pgfqpoint{5.688159in}{1.604542in}}%
\pgfpathlineto{\pgfqpoint{5.690730in}{1.605722in}}%
\pgfpathlineto{\pgfqpoint{5.693473in}{1.622080in}}%
\pgfpathlineto{\pgfqpoint{5.696101in}{1.613799in}}%
\pgfpathlineto{\pgfqpoint{5.698775in}{1.616806in}}%
\pgfpathlineto{\pgfqpoint{5.701453in}{1.608922in}}%
\pgfpathlineto{\pgfqpoint{5.704130in}{1.607187in}}%
\pgfpathlineto{\pgfqpoint{5.706800in}{1.600610in}}%
\pgfpathlineto{\pgfqpoint{5.709490in}{1.596344in}}%
\pgfpathlineto{\pgfqpoint{5.712291in}{1.596249in}}%
\pgfpathlineto{\pgfqpoint{5.714834in}{1.597906in}}%
\pgfpathlineto{\pgfqpoint{5.717671in}{1.595263in}}%
\pgfpathlineto{\pgfqpoint{5.720201in}{1.598636in}}%
\pgfpathlineto{\pgfqpoint{5.722950in}{1.601466in}}%
\pgfpathlineto{\pgfqpoint{5.725548in}{1.605719in}}%
\pgfpathlineto{\pgfqpoint{5.728339in}{1.599664in}}%
\pgfpathlineto{\pgfqpoint{5.730919in}{1.598167in}}%
\pgfpathlineto{\pgfqpoint{5.733594in}{1.600536in}}%
\pgfpathlineto{\pgfqpoint{5.736276in}{1.598418in}}%
\pgfpathlineto{\pgfqpoint{5.738974in}{1.600657in}}%
\pgfpathlineto{\pgfqpoint{5.741745in}{1.602424in}}%
\pgfpathlineto{\pgfqpoint{5.744310in}{1.604354in}}%
\pgfpathlineto{\pgfqpoint{5.744310in}{0.413320in}}%
\pgfpathlineto{\pgfqpoint{5.744310in}{0.413320in}}%
\pgfpathlineto{\pgfqpoint{5.741745in}{0.413320in}}%
\pgfpathlineto{\pgfqpoint{5.738974in}{0.413320in}}%
\pgfpathlineto{\pgfqpoint{5.736276in}{0.413320in}}%
\pgfpathlineto{\pgfqpoint{5.733594in}{0.413320in}}%
\pgfpathlineto{\pgfqpoint{5.730919in}{0.413320in}}%
\pgfpathlineto{\pgfqpoint{5.728339in}{0.413320in}}%
\pgfpathlineto{\pgfqpoint{5.725548in}{0.413320in}}%
\pgfpathlineto{\pgfqpoint{5.722950in}{0.413320in}}%
\pgfpathlineto{\pgfqpoint{5.720201in}{0.413320in}}%
\pgfpathlineto{\pgfqpoint{5.717671in}{0.413320in}}%
\pgfpathlineto{\pgfqpoint{5.714834in}{0.413320in}}%
\pgfpathlineto{\pgfqpoint{5.712291in}{0.413320in}}%
\pgfpathlineto{\pgfqpoint{5.709490in}{0.413320in}}%
\pgfpathlineto{\pgfqpoint{5.706800in}{0.413320in}}%
\pgfpathlineto{\pgfqpoint{5.704130in}{0.413320in}}%
\pgfpathlineto{\pgfqpoint{5.701453in}{0.413320in}}%
\pgfpathlineto{\pgfqpoint{5.698775in}{0.413320in}}%
\pgfpathlineto{\pgfqpoint{5.696101in}{0.413320in}}%
\pgfpathlineto{\pgfqpoint{5.693473in}{0.413320in}}%
\pgfpathlineto{\pgfqpoint{5.690730in}{0.413320in}}%
\pgfpathlineto{\pgfqpoint{5.688159in}{0.413320in}}%
\pgfpathlineto{\pgfqpoint{5.685385in}{0.413320in}}%
\pgfpathlineto{\pgfqpoint{5.682836in}{0.413320in}}%
\pgfpathlineto{\pgfqpoint{5.680027in}{0.413320in}}%
\pgfpathlineto{\pgfqpoint{5.677486in}{0.413320in}}%
\pgfpathlineto{\pgfqpoint{5.674667in}{0.413320in}}%
\pgfpathlineto{\pgfqpoint{5.671991in}{0.413320in}}%
\pgfpathlineto{\pgfqpoint{5.669313in}{0.413320in}}%
\pgfpathlineto{\pgfqpoint{5.666632in}{0.413320in}}%
\pgfpathlineto{\pgfqpoint{5.664099in}{0.413320in}}%
\pgfpathlineto{\pgfqpoint{5.661273in}{0.413320in}}%
\pgfpathlineto{\pgfqpoint{5.658723in}{0.413320in}}%
\pgfpathlineto{\pgfqpoint{5.655919in}{0.413320in}}%
\pgfpathlineto{\pgfqpoint{5.653376in}{0.413320in}}%
\pgfpathlineto{\pgfqpoint{5.650563in}{0.413320in}}%
\pgfpathlineto{\pgfqpoint{5.648008in}{0.413320in}}%
\pgfpathlineto{\pgfqpoint{5.645243in}{0.413320in}}%
\pgfpathlineto{\pgfqpoint{5.642518in}{0.413320in}}%
\pgfpathlineto{\pgfqpoint{5.639852in}{0.413320in}}%
\pgfpathlineto{\pgfqpoint{5.637172in}{0.413320in}}%
\pgfpathlineto{\pgfqpoint{5.634496in}{0.413320in}}%
\pgfpathlineto{\pgfqpoint{5.631815in}{0.413320in}}%
\pgfpathlineto{\pgfqpoint{5.629232in}{0.413320in}}%
\pgfpathlineto{\pgfqpoint{5.626460in}{0.413320in}}%
\pgfpathlineto{\pgfqpoint{5.623868in}{0.413320in}}%
\pgfpathlineto{\pgfqpoint{5.621102in}{0.413320in}}%
\pgfpathlineto{\pgfqpoint{5.618526in}{0.413320in}}%
\pgfpathlineto{\pgfqpoint{5.615743in}{0.413320in}}%
\pgfpathlineto{\pgfqpoint{5.613235in}{0.413320in}}%
\pgfpathlineto{\pgfqpoint{5.610389in}{0.413320in}}%
\pgfpathlineto{\pgfqpoint{5.607698in}{0.413320in}}%
\pgfpathlineto{\pgfqpoint{5.605073in}{0.413320in}}%
\pgfpathlineto{\pgfqpoint{5.602352in}{0.413320in}}%
\pgfpathlineto{\pgfqpoint{5.599674in}{0.413320in}}%
\pgfpathlineto{\pgfqpoint{5.596999in}{0.413320in}}%
\pgfpathlineto{\pgfqpoint{5.594368in}{0.413320in}}%
\pgfpathlineto{\pgfqpoint{5.591641in}{0.413320in}}%
\pgfpathlineto{\pgfqpoint{5.589040in}{0.413320in}}%
\pgfpathlineto{\pgfqpoint{5.586269in}{0.413320in}}%
\pgfpathlineto{\pgfqpoint{5.583709in}{0.413320in}}%
\pgfpathlineto{\pgfqpoint{5.580914in}{0.413320in}}%
\pgfpathlineto{\pgfqpoint{5.578342in}{0.413320in}}%
\pgfpathlineto{\pgfqpoint{5.575596in}{0.413320in}}%
\pgfpathlineto{\pgfqpoint{5.572893in}{0.413320in}}%
\pgfpathlineto{\pgfqpoint{5.570215in}{0.413320in}}%
\pgfpathlineto{\pgfqpoint{5.567536in}{0.413320in}}%
\pgfpathlineto{\pgfqpoint{5.564940in}{0.413320in}}%
\pgfpathlineto{\pgfqpoint{5.562180in}{0.413320in}}%
\pgfpathlineto{\pgfqpoint{5.559612in}{0.413320in}}%
\pgfpathlineto{\pgfqpoint{5.556822in}{0.413320in}}%
\pgfpathlineto{\pgfqpoint{5.554198in}{0.413320in}}%
\pgfpathlineto{\pgfqpoint{5.551457in}{0.413320in}}%
\pgfpathlineto{\pgfqpoint{5.548921in}{0.413320in}}%
\pgfpathlineto{\pgfqpoint{5.546110in}{0.413320in}}%
\pgfpathlineto{\pgfqpoint{5.543421in}{0.413320in}}%
\pgfpathlineto{\pgfqpoint{5.540750in}{0.413320in}}%
\pgfpathlineto{\pgfqpoint{5.538074in}{0.413320in}}%
\pgfpathlineto{\pgfqpoint{5.535395in}{0.413320in}}%
\pgfpathlineto{\pgfqpoint{5.532717in}{0.413320in}}%
\pgfpathlineto{\pgfqpoint{5.530148in}{0.413320in}}%
\pgfpathlineto{\pgfqpoint{5.527360in}{0.413320in}}%
\pgfpathlineto{\pgfqpoint{5.524756in}{0.413320in}}%
\pgfpathlineto{\pgfqpoint{5.522003in}{0.413320in}}%
\pgfpathlineto{\pgfqpoint{5.519433in}{0.413320in}}%
\pgfpathlineto{\pgfqpoint{5.516646in}{0.413320in}}%
\pgfpathlineto{\pgfqpoint{5.514080in}{0.413320in}}%
\pgfpathlineto{\pgfqpoint{5.511290in}{0.413320in}}%
\pgfpathlineto{\pgfqpoint{5.508612in}{0.413320in}}%
\pgfpathlineto{\pgfqpoint{5.505933in}{0.413320in}}%
\pgfpathlineto{\pgfqpoint{5.503255in}{0.413320in}}%
\pgfpathlineto{\pgfqpoint{5.500687in}{0.413320in}}%
\pgfpathlineto{\pgfqpoint{5.497898in}{0.413320in}}%
\pgfpathlineto{\pgfqpoint{5.495346in}{0.413320in}}%
\pgfpathlineto{\pgfqpoint{5.492541in}{0.413320in}}%
\pgfpathlineto{\pgfqpoint{5.490000in}{0.413320in}}%
\pgfpathlineto{\pgfqpoint{5.487176in}{0.413320in}}%
\pgfpathlineto{\pgfqpoint{5.484641in}{0.413320in}}%
\pgfpathlineto{\pgfqpoint{5.481825in}{0.413320in}}%
\pgfpathlineto{\pgfqpoint{5.479152in}{0.413320in}}%
\pgfpathlineto{\pgfqpoint{5.476458in}{0.413320in}}%
\pgfpathlineto{\pgfqpoint{5.473792in}{0.413320in}}%
\pgfpathlineto{\pgfqpoint{5.471113in}{0.413320in}}%
\pgfpathlineto{\pgfqpoint{5.468425in}{0.413320in}}%
\pgfpathlineto{\pgfqpoint{5.465888in}{0.413320in}}%
\pgfpathlineto{\pgfqpoint{5.463079in}{0.413320in}}%
\pgfpathlineto{\pgfqpoint{5.460489in}{0.413320in}}%
\pgfpathlineto{\pgfqpoint{5.457721in}{0.413320in}}%
\pgfpathlineto{\pgfqpoint{5.455168in}{0.413320in}}%
\pgfpathlineto{\pgfqpoint{5.452365in}{0.413320in}}%
\pgfpathlineto{\pgfqpoint{5.449769in}{0.413320in}}%
\pgfpathlineto{\pgfqpoint{5.447021in}{0.413320in}}%
\pgfpathlineto{\pgfqpoint{5.444328in}{0.413320in}}%
\pgfpathlineto{\pgfqpoint{5.441698in}{0.413320in}}%
\pgfpathlineto{\pgfqpoint{5.438974in}{0.413320in}}%
\pgfpathlineto{\pgfqpoint{5.436295in}{0.413320in}}%
\pgfpathlineto{\pgfqpoint{5.433616in}{0.413320in}}%
\pgfpathlineto{\pgfqpoint{5.431015in}{0.413320in}}%
\pgfpathlineto{\pgfqpoint{5.428259in}{0.413320in}}%
\pgfpathlineto{\pgfqpoint{5.425661in}{0.413320in}}%
\pgfpathlineto{\pgfqpoint{5.422897in}{0.413320in}}%
\pgfpathlineto{\pgfqpoint{5.420304in}{0.413320in}}%
\pgfpathlineto{\pgfqpoint{5.417547in}{0.413320in}}%
\pgfpathlineto{\pgfqpoint{5.414954in}{0.413320in}}%
\pgfpathlineto{\pgfqpoint{5.412190in}{0.413320in}}%
\pgfpathlineto{\pgfqpoint{5.409507in}{0.413320in}}%
\pgfpathlineto{\pgfqpoint{5.406832in}{0.413320in}}%
\pgfpathlineto{\pgfqpoint{5.404154in}{0.413320in}}%
\pgfpathlineto{\pgfqpoint{5.401576in}{0.413320in}}%
\pgfpathlineto{\pgfqpoint{5.398784in}{0.413320in}}%
\pgfpathlineto{\pgfqpoint{5.396219in}{0.413320in}}%
\pgfpathlineto{\pgfqpoint{5.393441in}{0.413320in}}%
\pgfpathlineto{\pgfqpoint{5.390900in}{0.413320in}}%
\pgfpathlineto{\pgfqpoint{5.388083in}{0.413320in}}%
\pgfpathlineto{\pgfqpoint{5.385550in}{0.413320in}}%
\pgfpathlineto{\pgfqpoint{5.382725in}{0.413320in}}%
\pgfpathlineto{\pgfqpoint{5.380048in}{0.413320in}}%
\pgfpathlineto{\pgfqpoint{5.377370in}{0.413320in}}%
\pgfpathlineto{\pgfqpoint{5.374692in}{0.413320in}}%
\pgfpathlineto{\pgfqpoint{5.372013in}{0.413320in}}%
\pgfpathlineto{\pgfqpoint{5.369335in}{0.413320in}}%
\pgfpathlineto{\pgfqpoint{5.366727in}{0.413320in}}%
\pgfpathlineto{\pgfqpoint{5.363966in}{0.413320in}}%
\pgfpathlineto{\pgfqpoint{5.361370in}{0.413320in}}%
\pgfpathlineto{\pgfqpoint{5.358612in}{0.413320in}}%
\pgfpathlineto{\pgfqpoint{5.356056in}{0.413320in}}%
\pgfpathlineto{\pgfqpoint{5.353262in}{0.413320in}}%
\pgfpathlineto{\pgfqpoint{5.350723in}{0.413320in}}%
\pgfpathlineto{\pgfqpoint{5.347905in}{0.413320in}}%
\pgfpathlineto{\pgfqpoint{5.345224in}{0.413320in}}%
\pgfpathlineto{\pgfqpoint{5.342549in}{0.413320in}}%
\pgfpathlineto{\pgfqpoint{5.339872in}{0.413320in}}%
\pgfpathlineto{\pgfqpoint{5.337353in}{0.413320in}}%
\pgfpathlineto{\pgfqpoint{5.334510in}{0.413320in}}%
\pgfpathlineto{\pgfqpoint{5.331973in}{0.413320in}}%
\pgfpathlineto{\pgfqpoint{5.329159in}{0.413320in}}%
\pgfpathlineto{\pgfqpoint{5.326564in}{0.413320in}}%
\pgfpathlineto{\pgfqpoint{5.323802in}{0.413320in}}%
\pgfpathlineto{\pgfqpoint{5.321256in}{0.413320in}}%
\pgfpathlineto{\pgfqpoint{5.318430in}{0.413320in}}%
\pgfpathlineto{\pgfqpoint{5.315754in}{0.413320in}}%
\pgfpathlineto{\pgfqpoint{5.313089in}{0.413320in}}%
\pgfpathlineto{\pgfqpoint{5.310411in}{0.413320in}}%
\pgfpathlineto{\pgfqpoint{5.307731in}{0.413320in}}%
\pgfpathlineto{\pgfqpoint{5.305054in}{0.413320in}}%
\pgfpathlineto{\pgfqpoint{5.302443in}{0.413320in}}%
\pgfpathlineto{\pgfqpoint{5.299696in}{0.413320in}}%
\pgfpathlineto{\pgfqpoint{5.297140in}{0.413320in}}%
\pgfpathlineto{\pgfqpoint{5.294339in}{0.413320in}}%
\pgfpathlineto{\pgfqpoint{5.291794in}{0.413320in}}%
\pgfpathlineto{\pgfqpoint{5.288984in}{0.413320in}}%
\pgfpathlineto{\pgfqpoint{5.286436in}{0.413320in}}%
\pgfpathlineto{\pgfqpoint{5.283631in}{0.413320in}}%
\pgfpathlineto{\pgfqpoint{5.280947in}{0.413320in}}%
\pgfpathlineto{\pgfqpoint{5.278322in}{0.413320in}}%
\pgfpathlineto{\pgfqpoint{5.275589in}{0.413320in}}%
\pgfpathlineto{\pgfqpoint{5.272913in}{0.413320in}}%
\pgfpathlineto{\pgfqpoint{5.270238in}{0.413320in}}%
\pgfpathlineto{\pgfqpoint{5.267691in}{0.413320in}}%
\pgfpathlineto{\pgfqpoint{5.264876in}{0.413320in}}%
\pgfpathlineto{\pgfqpoint{5.262264in}{0.413320in}}%
\pgfpathlineto{\pgfqpoint{5.259511in}{0.413320in}}%
\pgfpathlineto{\pgfqpoint{5.256973in}{0.413320in}}%
\pgfpathlineto{\pgfqpoint{5.254236in}{0.413320in}}%
\pgfpathlineto{\pgfqpoint{5.251590in}{0.413320in}}%
\pgfpathlineto{\pgfqpoint{5.248816in}{0.413320in}}%
\pgfpathlineto{\pgfqpoint{5.246130in}{0.413320in}}%
\pgfpathlineto{\pgfqpoint{5.243445in}{0.413320in}}%
\pgfpathlineto{\pgfqpoint{5.240777in}{0.413320in}}%
\pgfpathlineto{\pgfqpoint{5.238173in}{0.413320in}}%
\pgfpathlineto{\pgfqpoint{5.235409in}{0.413320in}}%
\pgfpathlineto{\pgfqpoint{5.232855in}{0.413320in}}%
\pgfpathlineto{\pgfqpoint{5.230059in}{0.413320in}}%
\pgfpathlineto{\pgfqpoint{5.227470in}{0.413320in}}%
\pgfpathlineto{\pgfqpoint{5.224695in}{0.413320in}}%
\pgfpathlineto{\pgfqpoint{5.222151in}{0.413320in}}%
\pgfpathlineto{\pgfqpoint{5.219345in}{0.413320in}}%
\pgfpathlineto{\pgfqpoint{5.216667in}{0.413320in}}%
\pgfpathlineto{\pgfqpoint{5.214027in}{0.413320in}}%
\pgfpathlineto{\pgfqpoint{5.211299in}{0.413320in}}%
\pgfpathlineto{\pgfqpoint{5.208630in}{0.413320in}}%
\pgfpathlineto{\pgfqpoint{5.205952in}{0.413320in}}%
\pgfpathlineto{\pgfqpoint{5.203388in}{0.413320in}}%
\pgfpathlineto{\pgfqpoint{5.200594in}{0.413320in}}%
\pgfpathlineto{\pgfqpoint{5.198008in}{0.413320in}}%
\pgfpathlineto{\pgfqpoint{5.195239in}{0.413320in}}%
\pgfpathlineto{\pgfqpoint{5.192680in}{0.413320in}}%
\pgfpathlineto{\pgfqpoint{5.189880in}{0.413320in}}%
\pgfpathlineto{\pgfqpoint{5.187294in}{0.413320in}}%
\pgfpathlineto{\pgfqpoint{5.184522in}{0.413320in}}%
\pgfpathlineto{\pgfqpoint{5.181848in}{0.413320in}}%
\pgfpathlineto{\pgfqpoint{5.179188in}{0.413320in}}%
\pgfpathlineto{\pgfqpoint{5.176477in}{0.413320in}}%
\pgfpathlineto{\pgfqpoint{5.173925in}{0.413320in}}%
\pgfpathlineto{\pgfqpoint{5.171133in}{0.413320in}}%
\pgfpathlineto{\pgfqpoint{5.168591in}{0.413320in}}%
\pgfpathlineto{\pgfqpoint{5.165775in}{0.413320in}}%
\pgfpathlineto{\pgfqpoint{5.163243in}{0.413320in}}%
\pgfpathlineto{\pgfqpoint{5.160420in}{0.413320in}}%
\pgfpathlineto{\pgfqpoint{5.157815in}{0.413320in}}%
\pgfpathlineto{\pgfqpoint{5.155059in}{0.413320in}}%
\pgfpathlineto{\pgfqpoint{5.152382in}{0.413320in}}%
\pgfpathlineto{\pgfqpoint{5.149734in}{0.413320in}}%
\pgfpathlineto{\pgfqpoint{5.147029in}{0.413320in}}%
\pgfpathlineto{\pgfqpoint{5.144349in}{0.413320in}}%
\pgfpathlineto{\pgfqpoint{5.141660in}{0.413320in}}%
\pgfpathlineto{\pgfqpoint{5.139072in}{0.413320in}}%
\pgfpathlineto{\pgfqpoint{5.136311in}{0.413320in}}%
\pgfpathlineto{\pgfqpoint{5.133716in}{0.413320in}}%
\pgfpathlineto{\pgfqpoint{5.130953in}{0.413320in}}%
\pgfpathlineto{\pgfqpoint{5.128421in}{0.413320in}}%
\pgfpathlineto{\pgfqpoint{5.125599in}{0.413320in}}%
\pgfpathlineto{\pgfqpoint{5.123042in}{0.413320in}}%
\pgfpathlineto{\pgfqpoint{5.120243in}{0.413320in}}%
\pgfpathlineto{\pgfqpoint{5.117550in}{0.413320in}}%
\pgfpathlineto{\pgfqpoint{5.114887in}{0.413320in}}%
\pgfpathlineto{\pgfqpoint{5.112209in}{0.413320in}}%
\pgfpathlineto{\pgfqpoint{5.109530in}{0.413320in}}%
\pgfpathlineto{\pgfqpoint{5.106842in}{0.413320in}}%
\pgfpathlineto{\pgfqpoint{5.104312in}{0.413320in}}%
\pgfpathlineto{\pgfqpoint{5.101496in}{0.413320in}}%
\pgfpathlineto{\pgfqpoint{5.098948in}{0.413320in}}%
\pgfpathlineto{\pgfqpoint{5.096142in}{0.413320in}}%
\pgfpathlineto{\pgfqpoint{5.093579in}{0.413320in}}%
\pgfpathlineto{\pgfqpoint{5.090788in}{0.413320in}}%
\pgfpathlineto{\pgfqpoint{5.088103in}{0.413320in}}%
\pgfpathlineto{\pgfqpoint{5.085426in}{0.413320in}}%
\pgfpathlineto{\pgfqpoint{5.082746in}{0.413320in}}%
\pgfpathlineto{\pgfqpoint{5.080067in}{0.413320in}}%
\pgfpathlineto{\pgfqpoint{5.077390in}{0.413320in}}%
\pgfpathlineto{\pgfqpoint{5.074851in}{0.413320in}}%
\pgfpathlineto{\pgfqpoint{5.072030in}{0.413320in}}%
\pgfpathlineto{\pgfqpoint{5.069463in}{0.413320in}}%
\pgfpathlineto{\pgfqpoint{5.066677in}{0.413320in}}%
\pgfpathlineto{\pgfqpoint{5.064144in}{0.413320in}}%
\pgfpathlineto{\pgfqpoint{5.061315in}{0.413320in}}%
\pgfpathlineto{\pgfqpoint{5.058711in}{0.413320in}}%
\pgfpathlineto{\pgfqpoint{5.055952in}{0.413320in}}%
\pgfpathlineto{\pgfqpoint{5.053284in}{0.413320in}}%
\pgfpathlineto{\pgfqpoint{5.050606in}{0.413320in}}%
\pgfpathlineto{\pgfqpoint{5.047924in}{0.413320in}}%
\pgfpathlineto{\pgfqpoint{5.045249in}{0.413320in}}%
\pgfpathlineto{\pgfqpoint{5.042572in}{0.413320in}}%
\pgfpathlineto{\pgfqpoint{5.039962in}{0.413320in}}%
\pgfpathlineto{\pgfqpoint{5.037214in}{0.413320in}}%
\pgfpathlineto{\pgfqpoint{5.034649in}{0.413320in}}%
\pgfpathlineto{\pgfqpoint{5.031849in}{0.413320in}}%
\pgfpathlineto{\pgfqpoint{5.029275in}{0.413320in}}%
\pgfpathlineto{\pgfqpoint{5.026501in}{0.413320in}}%
\pgfpathlineto{\pgfqpoint{5.023927in}{0.413320in}}%
\pgfpathlineto{\pgfqpoint{5.021147in}{0.413320in}}%
\pgfpathlineto{\pgfqpoint{5.018466in}{0.413320in}}%
\pgfpathlineto{\pgfqpoint{5.015820in}{0.413320in}}%
\pgfpathlineto{\pgfqpoint{5.013104in}{0.413320in}}%
\pgfpathlineto{\pgfqpoint{5.010562in}{0.413320in}}%
\pgfpathlineto{\pgfqpoint{5.007751in}{0.413320in}}%
\pgfpathlineto{\pgfqpoint{5.005178in}{0.413320in}}%
\pgfpathlineto{\pgfqpoint{5.002384in}{0.413320in}}%
\pgfpathlineto{\pgfqpoint{4.999780in}{0.413320in}}%
\pgfpathlineto{\pgfqpoint{4.997028in}{0.413320in}}%
\pgfpathlineto{\pgfqpoint{4.994390in}{0.413320in}}%
\pgfpathlineto{\pgfqpoint{4.991683in}{0.413320in}}%
\pgfpathlineto{\pgfqpoint{4.989001in}{0.413320in}}%
\pgfpathlineto{\pgfqpoint{4.986325in}{0.413320in}}%
\pgfpathlineto{\pgfqpoint{4.983637in}{0.413320in}}%
\pgfpathlineto{\pgfqpoint{4.980967in}{0.413320in}}%
\pgfpathlineto{\pgfqpoint{4.978287in}{0.413320in}}%
\pgfpathlineto{\pgfqpoint{4.975703in}{0.413320in}}%
\pgfpathlineto{\pgfqpoint{4.972933in}{0.413320in}}%
\pgfpathlineto{\pgfqpoint{4.970314in}{0.413320in}}%
\pgfpathlineto{\pgfqpoint{4.967575in}{0.413320in}}%
\pgfpathlineto{\pgfqpoint{4.965002in}{0.413320in}}%
\pgfpathlineto{\pgfqpoint{4.962219in}{0.413320in}}%
\pgfpathlineto{\pgfqpoint{4.959689in}{0.413320in}}%
\pgfpathlineto{\pgfqpoint{4.956862in}{0.413320in}}%
\pgfpathlineto{\pgfqpoint{4.954182in}{0.413320in}}%
\pgfpathlineto{\pgfqpoint{4.951504in}{0.413320in}}%
\pgfpathlineto{\pgfqpoint{4.948827in}{0.413320in}}%
\pgfpathlineto{\pgfqpoint{4.946151in}{0.413320in}}%
\pgfpathlineto{\pgfqpoint{4.943466in}{0.413320in}}%
\pgfpathlineto{\pgfqpoint{4.940881in}{0.413320in}}%
\pgfpathlineto{\pgfqpoint{4.938112in}{0.413320in}}%
\pgfpathlineto{\pgfqpoint{4.935515in}{0.413320in}}%
\pgfpathlineto{\pgfqpoint{4.932742in}{0.413320in}}%
\pgfpathlineto{\pgfqpoint{4.930170in}{0.413320in}}%
\pgfpathlineto{\pgfqpoint{4.927400in}{0.413320in}}%
\pgfpathlineto{\pgfqpoint{4.924708in}{0.413320in}}%
\pgfpathlineto{\pgfqpoint{4.922041in}{0.413320in}}%
\pgfpathlineto{\pgfqpoint{4.919352in}{0.413320in}}%
\pgfpathlineto{\pgfqpoint{4.916681in}{0.413320in}}%
\pgfpathlineto{\pgfqpoint{4.914009in}{0.413320in}}%
\pgfpathlineto{\pgfqpoint{4.911435in}{0.413320in}}%
\pgfpathlineto{\pgfqpoint{4.908648in}{0.413320in}}%
\pgfpathlineto{\pgfqpoint{4.906096in}{0.413320in}}%
\pgfpathlineto{\pgfqpoint{4.903295in}{0.413320in}}%
\pgfpathlineto{\pgfqpoint{4.900712in}{0.413320in}}%
\pgfpathlineto{\pgfqpoint{4.897938in}{0.413320in}}%
\pgfpathlineto{\pgfqpoint{4.895399in}{0.413320in}}%
\pgfpathlineto{\pgfqpoint{4.892611in}{0.413320in}}%
\pgfpathlineto{\pgfqpoint{4.889902in}{0.413320in}}%
\pgfpathlineto{\pgfqpoint{4.887211in}{0.413320in}}%
\pgfpathlineto{\pgfqpoint{4.884540in}{0.413320in}}%
\pgfpathlineto{\pgfqpoint{4.881864in}{0.413320in}}%
\pgfpathlineto{\pgfqpoint{4.879180in}{0.413320in}}%
\pgfpathlineto{\pgfqpoint{4.876636in}{0.413320in}}%
\pgfpathlineto{\pgfqpoint{4.873832in}{0.413320in}}%
\pgfpathlineto{\pgfqpoint{4.871209in}{0.413320in}}%
\pgfpathlineto{\pgfqpoint{4.868474in}{0.413320in}}%
\pgfpathlineto{\pgfqpoint{4.865910in}{0.413320in}}%
\pgfpathlineto{\pgfqpoint{4.863116in}{0.413320in}}%
\pgfpathlineto{\pgfqpoint{4.860544in}{0.413320in}}%
\pgfpathlineto{\pgfqpoint{4.857807in}{0.413320in}}%
\pgfpathlineto{\pgfqpoint{4.855070in}{0.413320in}}%
\pgfpathlineto{\pgfqpoint{4.852404in}{0.413320in}}%
\pgfpathlineto{\pgfqpoint{4.849715in}{0.413320in}}%
\pgfpathlineto{\pgfqpoint{4.847127in}{0.413320in}}%
\pgfpathlineto{\pgfqpoint{4.844361in}{0.413320in}}%
\pgfpathlineto{\pgfqpoint{4.842380in}{0.413320in}}%
\pgfpathlineto{\pgfqpoint{4.839922in}{0.413320in}}%
\pgfpathlineto{\pgfqpoint{4.837992in}{0.413320in}}%
\pgfpathlineto{\pgfqpoint{4.833657in}{0.413320in}}%
\pgfpathlineto{\pgfqpoint{4.831045in}{0.413320in}}%
\pgfpathlineto{\pgfqpoint{4.828291in}{0.413320in}}%
\pgfpathlineto{\pgfqpoint{4.825619in}{0.413320in}}%
\pgfpathlineto{\pgfqpoint{4.822945in}{0.413320in}}%
\pgfpathlineto{\pgfqpoint{4.820265in}{0.413320in}}%
\pgfpathlineto{\pgfqpoint{4.817587in}{0.413320in}}%
\pgfpathlineto{\pgfqpoint{4.814907in}{0.413320in}}%
\pgfpathlineto{\pgfqpoint{4.812377in}{0.413320in}}%
\pgfpathlineto{\pgfqpoint{4.809538in}{0.413320in}}%
\pgfpathlineto{\pgfqpoint{4.807017in}{0.413320in}}%
\pgfpathlineto{\pgfqpoint{4.804193in}{0.413320in}}%
\pgfpathlineto{\pgfqpoint{4.801586in}{0.413320in}}%
\pgfpathlineto{\pgfqpoint{4.798830in}{0.413320in}}%
\pgfpathlineto{\pgfqpoint{4.796274in}{0.413320in}}%
\pgfpathlineto{\pgfqpoint{4.793512in}{0.413320in}}%
\pgfpathlineto{\pgfqpoint{4.790798in}{0.413320in}}%
\pgfpathlineto{\pgfqpoint{4.788116in}{0.413320in}}%
\pgfpathlineto{\pgfqpoint{4.785445in}{0.413320in}}%
\pgfpathlineto{\pgfqpoint{4.782872in}{0.413320in}}%
\pgfpathlineto{\pgfqpoint{4.780083in}{0.413320in}}%
\pgfpathlineto{\pgfqpoint{4.777535in}{0.413320in}}%
\pgfpathlineto{\pgfqpoint{4.774732in}{0.413320in}}%
\pgfpathlineto{\pgfqpoint{4.772198in}{0.413320in}}%
\pgfpathlineto{\pgfqpoint{4.769367in}{0.413320in}}%
\pgfpathlineto{\pgfqpoint{4.766783in}{0.413320in}}%
\pgfpathlineto{\pgfqpoint{4.764018in}{0.413320in}}%
\pgfpathlineto{\pgfqpoint{4.761337in}{0.413320in}}%
\pgfpathlineto{\pgfqpoint{4.758653in}{0.413320in}}%
\pgfpathlineto{\pgfqpoint{4.755983in}{0.413320in}}%
\pgfpathlineto{\pgfqpoint{4.753298in}{0.413320in}}%
\pgfpathlineto{\pgfqpoint{4.750627in}{0.413320in}}%
\pgfpathlineto{\pgfqpoint{4.748081in}{0.413320in}}%
\pgfpathlineto{\pgfqpoint{4.745256in}{0.413320in}}%
\pgfpathlineto{\pgfqpoint{4.742696in}{0.413320in}}%
\pgfpathlineto{\pgfqpoint{4.739912in}{0.413320in}}%
\pgfpathlineto{\pgfqpoint{4.737348in}{0.413320in}}%
\pgfpathlineto{\pgfqpoint{4.734552in}{0.413320in}}%
\pgfpathlineto{\pgfqpoint{4.731901in}{0.413320in}}%
\pgfpathlineto{\pgfqpoint{4.729233in}{0.413320in}}%
\pgfpathlineto{\pgfqpoint{4.726508in}{0.413320in}}%
\pgfpathlineto{\pgfqpoint{4.723873in}{0.413320in}}%
\pgfpathlineto{\pgfqpoint{4.721160in}{0.413320in}}%
\pgfpathlineto{\pgfqpoint{4.718486in}{0.413320in}}%
\pgfpathlineto{\pgfqpoint{4.715806in}{0.413320in}}%
\pgfpathlineto{\pgfqpoint{4.713275in}{0.413320in}}%
\pgfpathlineto{\pgfqpoint{4.710437in}{0.413320in}}%
\pgfpathlineto{\pgfqpoint{4.707824in}{0.413320in}}%
\pgfpathlineto{\pgfqpoint{4.705094in}{0.413320in}}%
\pgfpathlineto{\pgfqpoint{4.702517in}{0.413320in}}%
\pgfpathlineto{\pgfqpoint{4.699734in}{0.413320in}}%
\pgfpathlineto{\pgfqpoint{4.697170in}{0.413320in}}%
\pgfpathlineto{\pgfqpoint{4.694381in}{0.413320in}}%
\pgfpathlineto{\pgfqpoint{4.691694in}{0.413320in}}%
\pgfpathlineto{\pgfqpoint{4.689051in}{0.413320in}}%
\pgfpathlineto{\pgfqpoint{4.686337in}{0.413320in}}%
\pgfpathlineto{\pgfqpoint{4.683799in}{0.413320in}}%
\pgfpathlineto{\pgfqpoint{4.680988in}{0.413320in}}%
\pgfpathlineto{\pgfqpoint{4.678448in}{0.413320in}}%
\pgfpathlineto{\pgfqpoint{4.675619in}{0.413320in}}%
\pgfpathlineto{\pgfqpoint{4.673068in}{0.413320in}}%
\pgfpathlineto{\pgfqpoint{4.670261in}{0.413320in}}%
\pgfpathlineto{\pgfqpoint{4.667764in}{0.413320in}}%
\pgfpathlineto{\pgfqpoint{4.664923in}{0.413320in}}%
\pgfpathlineto{\pgfqpoint{4.662237in}{0.413320in}}%
\pgfpathlineto{\pgfqpoint{4.659590in}{0.413320in}}%
\pgfpathlineto{\pgfqpoint{4.656873in}{0.413320in}}%
\pgfpathlineto{\pgfqpoint{4.654203in}{0.413320in}}%
\pgfpathlineto{\pgfqpoint{4.651524in}{0.413320in}}%
\pgfpathlineto{\pgfqpoint{4.648922in}{0.413320in}}%
\pgfpathlineto{\pgfqpoint{4.646169in}{0.413320in}}%
\pgfpathlineto{\pgfqpoint{4.643628in}{0.413320in}}%
\pgfpathlineto{\pgfqpoint{4.640809in}{0.413320in}}%
\pgfpathlineto{\pgfqpoint{4.638204in}{0.413320in}}%
\pgfpathlineto{\pgfqpoint{4.635445in}{0.413320in}}%
\pgfpathlineto{\pgfqpoint{4.632902in}{0.413320in}}%
\pgfpathlineto{\pgfqpoint{4.630096in}{0.413320in}}%
\pgfpathlineto{\pgfqpoint{4.627411in}{0.413320in}}%
\pgfpathlineto{\pgfqpoint{4.624741in}{0.413320in}}%
\pgfpathlineto{\pgfqpoint{4.622056in}{0.413320in}}%
\pgfpathlineto{\pgfqpoint{4.619529in}{0.413320in}}%
\pgfpathlineto{\pgfqpoint{4.616702in}{0.413320in}}%
\pgfpathlineto{\pgfqpoint{4.614134in}{0.413320in}}%
\pgfpathlineto{\pgfqpoint{4.611350in}{0.413320in}}%
\pgfpathlineto{\pgfqpoint{4.608808in}{0.413320in}}%
\pgfpathlineto{\pgfqpoint{4.605990in}{0.413320in}}%
\pgfpathlineto{\pgfqpoint{4.603430in}{0.413320in}}%
\pgfpathlineto{\pgfqpoint{4.600633in}{0.413320in}}%
\pgfpathlineto{\pgfqpoint{4.597951in}{0.413320in}}%
\pgfpathlineto{\pgfqpoint{4.595281in}{0.413320in}}%
\pgfpathlineto{\pgfqpoint{4.592589in}{0.413320in}}%
\pgfpathlineto{\pgfqpoint{4.589920in}{0.413320in}}%
\pgfpathlineto{\pgfqpoint{4.587244in}{0.413320in}}%
\pgfpathlineto{\pgfqpoint{4.584672in}{0.413320in}}%
\pgfpathlineto{\pgfqpoint{4.581888in}{0.413320in}}%
\pgfpathlineto{\pgfqpoint{4.579305in}{0.413320in}}%
\pgfpathlineto{\pgfqpoint{4.576531in}{0.413320in}}%
\pgfpathlineto{\pgfqpoint{4.573947in}{0.413320in}}%
\pgfpathlineto{\pgfqpoint{4.571171in}{0.413320in}}%
\pgfpathlineto{\pgfqpoint{4.568612in}{0.413320in}}%
\pgfpathlineto{\pgfqpoint{4.565820in}{0.413320in}}%
\pgfpathlineto{\pgfqpoint{4.563125in}{0.413320in}}%
\pgfpathlineto{\pgfqpoint{4.560448in}{0.413320in}}%
\pgfpathlineto{\pgfqpoint{4.557777in}{0.413320in}}%
\pgfpathlineto{\pgfqpoint{4.555106in}{0.413320in}}%
\pgfpathlineto{\pgfqpoint{4.552425in}{0.413320in}}%
\pgfpathlineto{\pgfqpoint{4.549822in}{0.413320in}}%
\pgfpathlineto{\pgfqpoint{4.547064in}{0.413320in}}%
\pgfpathlineto{\pgfqpoint{4.544464in}{0.413320in}}%
\pgfpathlineto{\pgfqpoint{4.541711in}{0.413320in}}%
\pgfpathlineto{\pgfqpoint{4.539144in}{0.413320in}}%
\pgfpathlineto{\pgfqpoint{4.536400in}{0.413320in}}%
\pgfpathlineto{\pgfqpoint{4.533764in}{0.413320in}}%
\pgfpathlineto{\pgfqpoint{4.530990in}{0.413320in}}%
\pgfpathlineto{\pgfqpoint{4.528307in}{0.413320in}}%
\pgfpathlineto{\pgfqpoint{4.525640in}{0.413320in}}%
\pgfpathlineto{\pgfqpoint{4.522962in}{0.413320in}}%
\pgfpathlineto{\pgfqpoint{4.520345in}{0.413320in}}%
\pgfpathlineto{\pgfqpoint{4.517598in}{0.413320in}}%
\pgfpathlineto{\pgfqpoint{4.515080in}{0.413320in}}%
\pgfpathlineto{\pgfqpoint{4.512246in}{0.413320in}}%
\pgfpathlineto{\pgfqpoint{4.509643in}{0.413320in}}%
\pgfpathlineto{\pgfqpoint{4.506893in}{0.413320in}}%
\pgfpathlineto{\pgfqpoint{4.504305in}{0.413320in}}%
\pgfpathlineto{\pgfqpoint{4.501529in}{0.413320in}}%
\pgfpathlineto{\pgfqpoint{4.498850in}{0.413320in}}%
\pgfpathlineto{\pgfqpoint{4.496167in}{0.413320in}}%
\pgfpathlineto{\pgfqpoint{4.493492in}{0.413320in}}%
\pgfpathlineto{\pgfqpoint{4.490822in}{0.413320in}}%
\pgfpathlineto{\pgfqpoint{4.488130in}{0.413320in}}%
\pgfpathlineto{\pgfqpoint{4.485581in}{0.413320in}}%
\pgfpathlineto{\pgfqpoint{4.482778in}{0.413320in}}%
\pgfpathlineto{\pgfqpoint{4.480201in}{0.413320in}}%
\pgfpathlineto{\pgfqpoint{4.477430in}{0.413320in}}%
\pgfpathlineto{\pgfqpoint{4.474861in}{0.413320in}}%
\pgfpathlineto{\pgfqpoint{4.472059in}{0.413320in}}%
\pgfpathlineto{\pgfqpoint{4.469492in}{0.413320in}}%
\pgfpathlineto{\pgfqpoint{4.466717in}{0.413320in}}%
\pgfpathlineto{\pgfqpoint{4.464029in}{0.413320in}}%
\pgfpathlineto{\pgfqpoint{4.461367in}{0.413320in}}%
\pgfpathlineto{\pgfqpoint{4.458681in}{0.413320in}}%
\pgfpathlineto{\pgfqpoint{4.456138in}{0.413320in}}%
\pgfpathlineto{\pgfqpoint{4.453312in}{0.413320in}}%
\pgfpathlineto{\pgfqpoint{4.450767in}{0.413320in}}%
\pgfpathlineto{\pgfqpoint{4.447965in}{0.413320in}}%
\pgfpathlineto{\pgfqpoint{4.445423in}{0.413320in}}%
\pgfpathlineto{\pgfqpoint{4.442611in}{0.413320in}}%
\pgfpathlineto{\pgfqpoint{4.440041in}{0.413320in}}%
\pgfpathlineto{\pgfqpoint{4.437253in}{0.413320in}}%
\pgfpathlineto{\pgfqpoint{4.434569in}{0.413320in}}%
\pgfpathlineto{\pgfqpoint{4.431901in}{0.413320in}}%
\pgfpathlineto{\pgfqpoint{4.429220in}{0.413320in}}%
\pgfpathlineto{\pgfqpoint{4.426534in}{0.413320in}}%
\pgfpathlineto{\pgfqpoint{4.423863in}{0.413320in}}%
\pgfpathlineto{\pgfqpoint{4.421292in}{0.413320in}}%
\pgfpathlineto{\pgfqpoint{4.418506in}{0.413320in}}%
\pgfpathlineto{\pgfqpoint{4.415932in}{0.413320in}}%
\pgfpathlineto{\pgfqpoint{4.413149in}{0.413320in}}%
\pgfpathlineto{\pgfqpoint{4.410587in}{0.413320in}}%
\pgfpathlineto{\pgfqpoint{4.407788in}{0.413320in}}%
\pgfpathlineto{\pgfqpoint{4.405234in}{0.413320in}}%
\pgfpathlineto{\pgfqpoint{4.402468in}{0.413320in}}%
\pgfpathlineto{\pgfqpoint{4.399745in}{0.413320in}}%
\pgfpathlineto{\pgfqpoint{4.397076in}{0.413320in}}%
\pgfpathlineto{\pgfqpoint{4.394400in}{0.413320in}}%
\pgfpathlineto{\pgfqpoint{4.391721in}{0.413320in}}%
\pgfpathlineto{\pgfqpoint{4.389044in}{0.413320in}}%
\pgfpathlineto{\pgfqpoint{4.386431in}{0.413320in}}%
\pgfpathlineto{\pgfqpoint{4.383674in}{0.413320in}}%
\pgfpathlineto{\pgfqpoint{4.381097in}{0.413320in}}%
\pgfpathlineto{\pgfqpoint{4.378329in}{0.413320in}}%
\pgfpathlineto{\pgfqpoint{4.375761in}{0.413320in}}%
\pgfpathlineto{\pgfqpoint{4.372976in}{0.413320in}}%
\pgfpathlineto{\pgfqpoint{4.370437in}{0.413320in}}%
\pgfpathlineto{\pgfqpoint{4.367646in}{0.413320in}}%
\pgfpathlineto{\pgfqpoint{4.364936in}{0.413320in}}%
\pgfpathlineto{\pgfqpoint{4.362270in}{0.413320in}}%
\pgfpathlineto{\pgfqpoint{4.359582in}{0.413320in}}%
\pgfpathlineto{\pgfqpoint{4.357014in}{0.413320in}}%
\pgfpathlineto{\pgfqpoint{4.354224in}{0.413320in}}%
\pgfpathlineto{\pgfqpoint{4.351645in}{0.413320in}}%
\pgfpathlineto{\pgfqpoint{4.348868in}{0.413320in}}%
\pgfpathlineto{\pgfqpoint{4.346263in}{0.413320in}}%
\pgfpathlineto{\pgfqpoint{4.343510in}{0.413320in}}%
\pgfpathlineto{\pgfqpoint{4.340976in}{0.413320in}}%
\pgfpathlineto{\pgfqpoint{4.338154in}{0.413320in}}%
\pgfpathlineto{\pgfqpoint{4.335463in}{0.413320in}}%
\pgfpathlineto{\pgfqpoint{4.332796in}{0.413320in}}%
\pgfpathlineto{\pgfqpoint{4.330118in}{0.413320in}}%
\pgfpathlineto{\pgfqpoint{4.327440in}{0.413320in}}%
\pgfpathlineto{\pgfqpoint{4.324760in}{0.413320in}}%
\pgfpathlineto{\pgfqpoint{4.322181in}{0.413320in}}%
\pgfpathlineto{\pgfqpoint{4.319405in}{0.413320in}}%
\pgfpathlineto{\pgfqpoint{4.316856in}{0.413320in}}%
\pgfpathlineto{\pgfqpoint{4.314032in}{0.413320in}}%
\pgfpathlineto{\pgfqpoint{4.311494in}{0.413320in}}%
\pgfpathlineto{\pgfqpoint{4.308691in}{0.413320in}}%
\pgfpathlineto{\pgfqpoint{4.306118in}{0.413320in}}%
\pgfpathlineto{\pgfqpoint{4.303357in}{0.413320in}}%
\pgfpathlineto{\pgfqpoint{4.300656in}{0.413320in}}%
\pgfpathlineto{\pgfqpoint{4.297977in}{0.413320in}}%
\pgfpathlineto{\pgfqpoint{4.295299in}{0.413320in}}%
\pgfpathlineto{\pgfqpoint{4.292786in}{0.413320in}}%
\pgfpathlineto{\pgfqpoint{4.289936in}{0.413320in}}%
\pgfpathlineto{\pgfqpoint{4.287399in}{0.413320in}}%
\pgfpathlineto{\pgfqpoint{4.284586in}{0.413320in}}%
\pgfpathlineto{\pgfqpoint{4.282000in}{0.413320in}}%
\pgfpathlineto{\pgfqpoint{4.279212in}{0.413320in}}%
\pgfpathlineto{\pgfqpoint{4.276635in}{0.413320in}}%
\pgfpathlineto{\pgfqpoint{4.273874in}{0.413320in}}%
\pgfpathlineto{\pgfqpoint{4.271187in}{0.413320in}}%
\pgfpathlineto{\pgfqpoint{4.268590in}{0.413320in}}%
\pgfpathlineto{\pgfqpoint{4.265824in}{0.413320in}}%
\pgfpathlineto{\pgfqpoint{4.263157in}{0.413320in}}%
\pgfpathlineto{\pgfqpoint{4.260477in}{0.413320in}}%
\pgfpathlineto{\pgfqpoint{4.257958in}{0.413320in}}%
\pgfpathlineto{\pgfqpoint{4.255120in}{0.413320in}}%
\pgfpathlineto{\pgfqpoint{4.252581in}{0.413320in}}%
\pgfpathlineto{\pgfqpoint{4.249767in}{0.413320in}}%
\pgfpathlineto{\pgfqpoint{4.247225in}{0.413320in}}%
\pgfpathlineto{\pgfqpoint{4.244394in}{0.413320in}}%
\pgfpathlineto{\pgfqpoint{4.241900in}{0.413320in}}%
\pgfpathlineto{\pgfqpoint{4.239084in}{0.413320in}}%
\pgfpathlineto{\pgfqpoint{4.236375in}{0.413320in}}%
\pgfpathlineto{\pgfqpoint{4.233691in}{0.413320in}}%
\pgfpathlineto{\pgfqpoint{4.231013in}{0.413320in}}%
\pgfpathlineto{\pgfqpoint{4.228331in}{0.413320in}}%
\pgfpathlineto{\pgfqpoint{4.225654in}{0.413320in}}%
\pgfpathlineto{\pgfqpoint{4.223082in}{0.413320in}}%
\pgfpathlineto{\pgfqpoint{4.220304in}{0.413320in}}%
\pgfpathlineto{\pgfqpoint{4.217694in}{0.413320in}}%
\pgfpathlineto{\pgfqpoint{4.214948in}{0.413320in}}%
\pgfpathlineto{\pgfqpoint{4.212383in}{0.413320in}}%
\pgfpathlineto{\pgfqpoint{4.209597in}{0.413320in}}%
\pgfpathlineto{\pgfqpoint{4.207076in}{0.413320in}}%
\pgfpathlineto{\pgfqpoint{4.204240in}{0.413320in}}%
\pgfpathlineto{\pgfqpoint{4.201542in}{0.413320in}}%
\pgfpathlineto{\pgfqpoint{4.198878in}{0.413320in}}%
\pgfpathlineto{\pgfqpoint{4.196186in}{0.413320in}}%
\pgfpathlineto{\pgfqpoint{4.193638in}{0.413320in}}%
\pgfpathlineto{\pgfqpoint{4.190842in}{0.413320in}}%
\pgfpathlineto{\pgfqpoint{4.188318in}{0.413320in}}%
\pgfpathlineto{\pgfqpoint{4.185481in}{0.413320in}}%
\pgfpathlineto{\pgfqpoint{4.182899in}{0.413320in}}%
\pgfpathlineto{\pgfqpoint{4.180129in}{0.413320in}}%
\pgfpathlineto{\pgfqpoint{4.177593in}{0.413320in}}%
\pgfpathlineto{\pgfqpoint{4.174770in}{0.413320in}}%
\pgfpathlineto{\pgfqpoint{4.172093in}{0.413320in}}%
\pgfpathlineto{\pgfqpoint{4.169415in}{0.413320in}}%
\pgfpathlineto{\pgfqpoint{4.166737in}{0.413320in}}%
\pgfpathlineto{\pgfqpoint{4.164059in}{0.413320in}}%
\pgfpathlineto{\pgfqpoint{4.161380in}{0.413320in}}%
\pgfpathlineto{\pgfqpoint{4.158806in}{0.413320in}}%
\pgfpathlineto{\pgfqpoint{4.156016in}{0.413320in}}%
\pgfpathlineto{\pgfqpoint{4.153423in}{0.413320in}}%
\pgfpathlineto{\pgfqpoint{4.150665in}{0.413320in}}%
\pgfpathlineto{\pgfqpoint{4.148082in}{0.413320in}}%
\pgfpathlineto{\pgfqpoint{4.145310in}{0.413320in}}%
\pgfpathlineto{\pgfqpoint{4.142713in}{0.413320in}}%
\pgfpathlineto{\pgfqpoint{4.139963in}{0.413320in}}%
\pgfpathlineto{\pgfqpoint{4.137272in}{0.413320in}}%
\pgfpathlineto{\pgfqpoint{4.134615in}{0.413320in}}%
\pgfpathlineto{\pgfqpoint{4.131920in}{0.413320in}}%
\pgfpathlineto{\pgfqpoint{4.129349in}{0.413320in}}%
\pgfpathlineto{\pgfqpoint{4.126553in}{0.413320in}}%
\pgfpathlineto{\pgfqpoint{4.124019in}{0.413320in}}%
\pgfpathlineto{\pgfqpoint{4.121205in}{0.413320in}}%
\pgfpathlineto{\pgfqpoint{4.118554in}{0.413320in}}%
\pgfpathlineto{\pgfqpoint{4.115844in}{0.413320in}}%
\pgfpathlineto{\pgfqpoint{4.113252in}{0.413320in}}%
\pgfpathlineto{\pgfqpoint{4.110488in}{0.413320in}}%
\pgfpathlineto{\pgfqpoint{4.107814in}{0.413320in}}%
\pgfpathlineto{\pgfqpoint{4.105185in}{0.413320in}}%
\pgfpathlineto{\pgfqpoint{4.102456in}{0.413320in}}%
\pgfpathlineto{\pgfqpoint{4.099777in}{0.413320in}}%
\pgfpathlineto{\pgfqpoint{4.097092in}{0.413320in}}%
\pgfpathlineto{\pgfqpoint{4.094527in}{0.413320in}}%
\pgfpathlineto{\pgfqpoint{4.091729in}{0.413320in}}%
\pgfpathlineto{\pgfqpoint{4.089159in}{0.413320in}}%
\pgfpathlineto{\pgfqpoint{4.086385in}{0.413320in}}%
\pgfpathlineto{\pgfqpoint{4.083870in}{0.413320in}}%
\pgfpathlineto{\pgfqpoint{4.081018in}{0.413320in}}%
\pgfpathlineto{\pgfqpoint{4.078471in}{0.413320in}}%
\pgfpathlineto{\pgfqpoint{4.075705in}{0.413320in}}%
\pgfpathlineto{\pgfqpoint{4.072985in}{0.413320in}}%
\pgfpathlineto{\pgfqpoint{4.070313in}{0.413320in}}%
\pgfpathlineto{\pgfqpoint{4.067636in}{0.413320in}}%
\pgfpathlineto{\pgfqpoint{4.064957in}{0.413320in}}%
\pgfpathlineto{\pgfqpoint{4.062266in}{0.413320in}}%
\pgfpathlineto{\pgfqpoint{4.059702in}{0.413320in}}%
\pgfpathlineto{\pgfqpoint{4.056911in}{0.413320in}}%
\pgfpathlineto{\pgfqpoint{4.054326in}{0.413320in}}%
\pgfpathlineto{\pgfqpoint{4.051557in}{0.413320in}}%
\pgfpathlineto{\pgfqpoint{4.049006in}{0.413320in}}%
\pgfpathlineto{\pgfqpoint{4.046210in}{0.413320in}}%
\pgfpathlineto{\pgfqpoint{4.043667in}{0.413320in}}%
\pgfpathlineto{\pgfqpoint{4.040852in}{0.413320in}}%
\pgfpathlineto{\pgfqpoint{4.038174in}{0.413320in}}%
\pgfpathlineto{\pgfqpoint{4.035492in}{0.413320in}}%
\pgfpathlineto{\pgfqpoint{4.032817in}{0.413320in}}%
\pgfpathlineto{\pgfqpoint{4.030229in}{0.413320in}}%
\pgfpathlineto{\pgfqpoint{4.027447in}{0.413320in}}%
\pgfpathlineto{\pgfqpoint{4.024868in}{0.413320in}}%
\pgfpathlineto{\pgfqpoint{4.022097in}{0.413320in}}%
\pgfpathlineto{\pgfqpoint{4.019518in}{0.413320in}}%
\pgfpathlineto{\pgfqpoint{4.016744in}{0.413320in}}%
\pgfpathlineto{\pgfqpoint{4.014186in}{0.413320in}}%
\pgfpathlineto{\pgfqpoint{4.011394in}{0.413320in}}%
\pgfpathlineto{\pgfqpoint{4.008699in}{0.413320in}}%
\pgfpathlineto{\pgfqpoint{4.006034in}{0.413320in}}%
\pgfpathlineto{\pgfqpoint{4.003348in}{0.413320in}}%
\pgfpathlineto{\pgfqpoint{4.000674in}{0.413320in}}%
\pgfpathlineto{\pgfqpoint{3.997990in}{0.413320in}}%
\pgfpathlineto{\pgfqpoint{3.995417in}{0.413320in}}%
\pgfpathlineto{\pgfqpoint{3.992642in}{0.413320in}}%
\pgfpathlineto{\pgfqpoint{3.990055in}{0.413320in}}%
\pgfpathlineto{\pgfqpoint{3.987270in}{0.413320in}}%
\pgfpathlineto{\pgfqpoint{3.984714in}{0.413320in}}%
\pgfpathlineto{\pgfqpoint{3.981929in}{0.413320in}}%
\pgfpathlineto{\pgfqpoint{3.979389in}{0.413320in}}%
\pgfpathlineto{\pgfqpoint{3.976563in}{0.413320in}}%
\pgfpathlineto{\pgfqpoint{3.973885in}{0.413320in}}%
\pgfpathlineto{\pgfqpoint{3.971250in}{0.413320in}}%
\pgfpathlineto{\pgfqpoint{3.968523in}{0.413320in}}%
\pgfpathlineto{\pgfqpoint{3.966013in}{0.413320in}}%
\pgfpathlineto{\pgfqpoint{3.963176in}{0.413320in}}%
\pgfpathlineto{\pgfqpoint{3.960635in}{0.413320in}}%
\pgfpathlineto{\pgfqpoint{3.957823in}{0.413320in}}%
\pgfpathlineto{\pgfqpoint{3.955211in}{0.413320in}}%
\pgfpathlineto{\pgfqpoint{3.952464in}{0.413320in}}%
\pgfpathlineto{\pgfqpoint{3.949894in}{0.413320in}}%
\pgfpathlineto{\pgfqpoint{3.947101in}{0.413320in}}%
\pgfpathlineto{\pgfqpoint{3.944431in}{0.413320in}}%
\pgfpathlineto{\pgfqpoint{3.941778in}{0.413320in}}%
\pgfpathlineto{\pgfqpoint{3.939075in}{0.413320in}}%
\pgfpathlineto{\pgfqpoint{3.936395in}{0.413320in}}%
\pgfpathlineto{\pgfqpoint{3.933714in}{0.413320in}}%
\pgfpathlineto{\pgfqpoint{3.931202in}{0.413320in}}%
\pgfpathlineto{\pgfqpoint{3.928347in}{0.413320in}}%
\pgfpathlineto{\pgfqpoint{3.925778in}{0.413320in}}%
\pgfpathlineto{\pgfqpoint{3.923005in}{0.413320in}}%
\pgfpathlineto{\pgfqpoint{3.920412in}{0.413320in}}%
\pgfpathlineto{\pgfqpoint{3.917646in}{0.413320in}}%
\pgfpathlineto{\pgfqpoint{3.915107in}{0.413320in}}%
\pgfpathlineto{\pgfqpoint{3.912296in}{0.413320in}}%
\pgfpathlineto{\pgfqpoint{3.909602in}{0.413320in}}%
\pgfpathlineto{\pgfqpoint{3.906918in}{0.413320in}}%
\pgfpathlineto{\pgfqpoint{3.904252in}{0.413320in}}%
\pgfpathlineto{\pgfqpoint{3.901573in}{0.413320in}}%
\pgfpathlineto{\pgfqpoint{3.898891in}{0.413320in}}%
\pgfpathlineto{\pgfqpoint{3.896345in}{0.413320in}}%
\pgfpathlineto{\pgfqpoint{3.893541in}{0.413320in}}%
\pgfpathlineto{\pgfqpoint{3.890926in}{0.413320in}}%
\pgfpathlineto{\pgfqpoint{3.888188in}{0.413320in}}%
\pgfpathlineto{\pgfqpoint{3.885621in}{0.413320in}}%
\pgfpathlineto{\pgfqpoint{3.882850in}{0.413320in}}%
\pgfpathlineto{\pgfqpoint{3.880237in}{0.413320in}}%
\pgfpathlineto{\pgfqpoint{3.877466in}{0.413320in}}%
\pgfpathlineto{\pgfqpoint{3.874790in}{0.413320in}}%
\pgfpathlineto{\pgfqpoint{3.872114in}{0.413320in}}%
\pgfpathlineto{\pgfqpoint{3.869435in}{0.413320in}}%
\pgfpathlineto{\pgfqpoint{3.866815in}{0.413320in}}%
\pgfpathlineto{\pgfqpoint{3.864073in}{0.413320in}}%
\pgfpathlineto{\pgfqpoint{3.861561in}{0.413320in}}%
\pgfpathlineto{\pgfqpoint{3.858720in}{0.413320in}}%
\pgfpathlineto{\pgfqpoint{3.856100in}{0.413320in}}%
\pgfpathlineto{\pgfqpoint{3.853358in}{0.413320in}}%
\pgfpathlineto{\pgfqpoint{3.850814in}{0.413320in}}%
\pgfpathlineto{\pgfqpoint{3.848005in}{0.413320in}}%
\pgfpathlineto{\pgfqpoint{3.845329in}{0.413320in}}%
\pgfpathlineto{\pgfqpoint{3.842641in}{0.413320in}}%
\pgfpathlineto{\pgfqpoint{3.839960in}{0.413320in}}%
\pgfpathlineto{\pgfqpoint{3.837286in}{0.413320in}}%
\pgfpathlineto{\pgfqpoint{3.834616in}{0.413320in}}%
\pgfpathlineto{\pgfqpoint{3.832053in}{0.413320in}}%
\pgfpathlineto{\pgfqpoint{3.829252in}{0.413320in}}%
\pgfpathlineto{\pgfqpoint{3.826679in}{0.413320in}}%
\pgfpathlineto{\pgfqpoint{3.823903in}{0.413320in}}%
\pgfpathlineto{\pgfqpoint{3.821315in}{0.413320in}}%
\pgfpathlineto{\pgfqpoint{3.818546in}{0.413320in}}%
\pgfpathlineto{\pgfqpoint{3.815983in}{0.413320in}}%
\pgfpathlineto{\pgfqpoint{3.813172in}{0.413320in}}%
\pgfpathlineto{\pgfqpoint{3.810510in}{0.413320in}}%
\pgfpathlineto{\pgfqpoint{3.807832in}{0.413320in}}%
\pgfpathlineto{\pgfqpoint{3.805145in}{0.413320in}}%
\pgfpathlineto{\pgfqpoint{3.802569in}{0.413320in}}%
\pgfpathlineto{\pgfqpoint{3.799797in}{0.413320in}}%
\pgfpathlineto{\pgfqpoint{3.797265in}{0.413320in}}%
\pgfpathlineto{\pgfqpoint{3.794435in}{0.413320in}}%
\pgfpathlineto{\pgfqpoint{3.791897in}{0.413320in}}%
\pgfpathlineto{\pgfqpoint{3.789084in}{0.413320in}}%
\pgfpathlineto{\pgfqpoint{3.786504in}{0.413320in}}%
\pgfpathlineto{\pgfqpoint{3.783725in}{0.413320in}}%
\pgfpathlineto{\pgfqpoint{3.781046in}{0.413320in}}%
\pgfpathlineto{\pgfqpoint{3.778370in}{0.413320in}}%
\pgfpathlineto{\pgfqpoint{3.775691in}{0.413320in}}%
\pgfpathlineto{\pgfqpoint{3.773014in}{0.413320in}}%
\pgfpathlineto{\pgfqpoint{3.770323in}{0.413320in}}%
\pgfpathlineto{\pgfqpoint{3.767782in}{0.413320in}}%
\pgfpathlineto{\pgfqpoint{3.764966in}{0.413320in}}%
\pgfpathlineto{\pgfqpoint{3.762389in}{0.413320in}}%
\pgfpathlineto{\pgfqpoint{3.759622in}{0.413320in}}%
\pgfpathlineto{\pgfqpoint{3.757065in}{0.413320in}}%
\pgfpathlineto{\pgfqpoint{3.754265in}{0.413320in}}%
\pgfpathlineto{\pgfqpoint{3.751728in}{0.413320in}}%
\pgfpathlineto{\pgfqpoint{3.748903in}{0.413320in}}%
\pgfpathlineto{\pgfqpoint{3.746229in}{0.413320in}}%
\pgfpathlineto{\pgfqpoint{3.743548in}{0.413320in}}%
\pgfpathlineto{\pgfqpoint{3.740874in}{0.413320in}}%
\pgfpathlineto{\pgfqpoint{3.738194in}{0.413320in}}%
\pgfpathlineto{\pgfqpoint{3.735509in}{0.413320in}}%
\pgfpathlineto{\pgfqpoint{3.732950in}{0.413320in}}%
\pgfpathlineto{\pgfqpoint{3.730158in}{0.413320in}}%
\pgfpathlineto{\pgfqpoint{3.727581in}{0.413320in}}%
\pgfpathlineto{\pgfqpoint{3.724804in}{0.413320in}}%
\pgfpathlineto{\pgfqpoint{3.722228in}{0.413320in}}%
\pgfpathlineto{\pgfqpoint{3.719446in}{0.413320in}}%
\pgfpathlineto{\pgfqpoint{3.716875in}{0.413320in}}%
\pgfpathlineto{\pgfqpoint{3.714086in}{0.413320in}}%
\pgfpathlineto{\pgfqpoint{3.711410in}{0.413320in}}%
\pgfpathlineto{\pgfqpoint{3.708729in}{0.413320in}}%
\pgfpathlineto{\pgfqpoint{3.706053in}{0.413320in}}%
\pgfpathlineto{\pgfqpoint{3.703460in}{0.413320in}}%
\pgfpathlineto{\pgfqpoint{3.700684in}{0.413320in}}%
\pgfpathlineto{\pgfqpoint{3.698125in}{0.413320in}}%
\pgfpathlineto{\pgfqpoint{3.695331in}{0.413320in}}%
\pgfpathlineto{\pgfqpoint{3.692765in}{0.413320in}}%
\pgfpathlineto{\pgfqpoint{3.689983in}{0.413320in}}%
\pgfpathlineto{\pgfqpoint{3.687442in}{0.413320in}}%
\pgfpathlineto{\pgfqpoint{3.684620in}{0.413320in}}%
\pgfpathlineto{\pgfqpoint{3.681948in}{0.413320in}}%
\pgfpathlineto{\pgfqpoint{3.679273in}{0.413320in}}%
\pgfpathlineto{\pgfqpoint{3.676591in}{0.413320in}}%
\pgfpathlineto{\pgfqpoint{3.673911in}{0.413320in}}%
\pgfpathlineto{\pgfqpoint{3.671232in}{0.413320in}}%
\pgfpathlineto{\pgfqpoint{3.668665in}{0.413320in}}%
\pgfpathlineto{\pgfqpoint{3.665864in}{0.413320in}}%
\pgfpathlineto{\pgfqpoint{3.663276in}{0.413320in}}%
\pgfpathlineto{\pgfqpoint{3.660515in}{0.413320in}}%
\pgfpathlineto{\pgfqpoint{3.657917in}{0.413320in}}%
\pgfpathlineto{\pgfqpoint{3.655165in}{0.413320in}}%
\pgfpathlineto{\pgfqpoint{3.652628in}{0.413320in}}%
\pgfpathlineto{\pgfqpoint{3.649837in}{0.413320in}}%
\pgfpathlineto{\pgfqpoint{3.647130in}{0.413320in}}%
\pgfpathlineto{\pgfqpoint{3.644452in}{0.413320in}}%
\pgfpathlineto{\pgfqpoint{3.641773in}{0.413320in}}%
\pgfpathlineto{\pgfqpoint{3.639207in}{0.413320in}}%
\pgfpathlineto{\pgfqpoint{3.636413in}{0.413320in}}%
\pgfpathlineto{\pgfqpoint{3.633858in}{0.413320in}}%
\pgfpathlineto{\pgfqpoint{3.631058in}{0.413320in}}%
\pgfpathlineto{\pgfqpoint{3.628460in}{0.413320in}}%
\pgfpathlineto{\pgfqpoint{3.625689in}{0.413320in}}%
\pgfpathlineto{\pgfqpoint{3.623165in}{0.413320in}}%
\pgfpathlineto{\pgfqpoint{3.620345in}{0.413320in}}%
\pgfpathlineto{\pgfqpoint{3.617667in}{0.413320in}}%
\pgfpathlineto{\pgfqpoint{3.614982in}{0.413320in}}%
\pgfpathlineto{\pgfqpoint{3.612311in}{0.413320in}}%
\pgfpathlineto{\pgfqpoint{3.609632in}{0.413320in}}%
\pgfpathlineto{\pgfqpoint{3.606951in}{0.413320in}}%
\pgfpathlineto{\pgfqpoint{3.604387in}{0.413320in}}%
\pgfpathlineto{\pgfqpoint{3.601590in}{0.413320in}}%
\pgfpathlineto{\pgfqpoint{3.598998in}{0.413320in}}%
\pgfpathlineto{\pgfqpoint{3.596240in}{0.413320in}}%
\pgfpathlineto{\pgfqpoint{3.593620in}{0.413320in}}%
\pgfpathlineto{\pgfqpoint{3.590883in}{0.413320in}}%
\pgfpathlineto{\pgfqpoint{3.588258in}{0.413320in}}%
\pgfpathlineto{\pgfqpoint{3.585532in}{0.413320in}}%
\pgfpathlineto{\pgfqpoint{3.582851in}{0.413320in}}%
\pgfpathlineto{\pgfqpoint{3.580191in}{0.413320in}}%
\pgfpathlineto{\pgfqpoint{3.577487in}{0.413320in}}%
\pgfpathlineto{\pgfqpoint{3.574814in}{0.413320in}}%
\pgfpathlineto{\pgfqpoint{3.572126in}{0.413320in}}%
\pgfpathlineto{\pgfqpoint{3.569584in}{0.413320in}}%
\pgfpathlineto{\pgfqpoint{3.566774in}{0.413320in}}%
\pgfpathlineto{\pgfqpoint{3.564188in}{0.413320in}}%
\pgfpathlineto{\pgfqpoint{3.561420in}{0.413320in}}%
\pgfpathlineto{\pgfqpoint{3.558853in}{0.413320in}}%
\pgfpathlineto{\pgfqpoint{3.556061in}{0.413320in}}%
\pgfpathlineto{\pgfqpoint{3.553498in}{0.413320in}}%
\pgfpathlineto{\pgfqpoint{3.550713in}{0.413320in}}%
\pgfpathlineto{\pgfqpoint{3.548029in}{0.413320in}}%
\pgfpathlineto{\pgfqpoint{3.545349in}{0.413320in}}%
\pgfpathlineto{\pgfqpoint{3.542656in}{0.413320in}}%
\pgfpathlineto{\pgfqpoint{3.540093in}{0.413320in}}%
\pgfpathlineto{\pgfqpoint{3.537309in}{0.413320in}}%
\pgfpathlineto{\pgfqpoint{3.534783in}{0.413320in}}%
\pgfpathlineto{\pgfqpoint{3.531955in}{0.413320in}}%
\pgfpathlineto{\pgfqpoint{3.529327in}{0.413320in}}%
\pgfpathlineto{\pgfqpoint{3.526601in}{0.413320in}}%
\pgfpathlineto{\pgfqpoint{3.524041in}{0.413320in}}%
\pgfpathlineto{\pgfqpoint{3.521244in}{0.413320in}}%
\pgfpathlineto{\pgfqpoint{3.518565in}{0.413320in}}%
\pgfpathlineto{\pgfqpoint{3.515884in}{0.413320in}}%
\pgfpathlineto{\pgfqpoint{3.513209in}{0.413320in}}%
\pgfpathlineto{\pgfqpoint{3.510533in}{0.413320in}}%
\pgfpathlineto{\pgfqpoint{3.507840in}{0.413320in}}%
\pgfpathlineto{\pgfqpoint{3.505262in}{0.413320in}}%
\pgfpathlineto{\pgfqpoint{3.502488in}{0.413320in}}%
\pgfpathlineto{\pgfqpoint{3.499909in}{0.413320in}}%
\pgfpathlineto{\pgfqpoint{3.497139in}{0.413320in}}%
\pgfpathlineto{\pgfqpoint{3.494581in}{0.413320in}}%
\pgfpathlineto{\pgfqpoint{3.491783in}{0.413320in}}%
\pgfpathlineto{\pgfqpoint{3.489223in}{0.413320in}}%
\pgfpathlineto{\pgfqpoint{3.486442in}{0.413320in}}%
\pgfpathlineto{\pgfqpoint{3.483744in}{0.413320in}}%
\pgfpathlineto{\pgfqpoint{3.481072in}{0.413320in}}%
\pgfpathlineto{\pgfqpoint{3.478378in}{0.413320in}}%
\pgfpathlineto{\pgfqpoint{3.475821in}{0.413320in}}%
\pgfpathlineto{\pgfqpoint{3.473021in}{0.413320in}}%
\pgfpathlineto{\pgfqpoint{3.470466in}{0.413320in}}%
\pgfpathlineto{\pgfqpoint{3.467678in}{0.413320in}}%
\pgfpathlineto{\pgfqpoint{3.465072in}{0.413320in}}%
\pgfpathlineto{\pgfqpoint{3.462321in}{0.413320in}}%
\pgfpathlineto{\pgfqpoint{3.459695in}{0.413320in}}%
\pgfpathlineto{\pgfqpoint{3.456960in}{0.413320in}}%
\pgfpathlineto{\pgfqpoint{3.454285in}{0.413320in}}%
\pgfpathlineto{\pgfqpoint{3.451597in}{0.413320in}}%
\pgfpathlineto{\pgfqpoint{3.448926in}{0.413320in}}%
\pgfpathlineto{\pgfqpoint{3.446257in}{0.413320in}}%
\pgfpathlineto{\pgfqpoint{3.443574in}{0.413320in}}%
\pgfpathlineto{\pgfqpoint{3.440996in}{0.413320in}}%
\pgfpathlineto{\pgfqpoint{3.438210in}{0.413320in}}%
\pgfpathlineto{\pgfqpoint{3.435635in}{0.413320in}}%
\pgfpathlineto{\pgfqpoint{3.432851in}{0.413320in}}%
\pgfpathlineto{\pgfqpoint{3.430313in}{0.413320in}}%
\pgfpathlineto{\pgfqpoint{3.427501in}{0.413320in}}%
\pgfpathlineto{\pgfqpoint{3.424887in}{0.413320in}}%
\pgfpathlineto{\pgfqpoint{3.422142in}{0.413320in}}%
\pgfpathlineto{\pgfqpoint{3.419455in}{0.413320in}}%
\pgfpathlineto{\pgfqpoint{3.416780in}{0.413320in}}%
\pgfpathlineto{\pgfqpoint{3.414109in}{0.413320in}}%
\pgfpathlineto{\pgfqpoint{3.411431in}{0.413320in}}%
\pgfpathlineto{\pgfqpoint{3.408752in}{0.413320in}}%
\pgfpathlineto{\pgfqpoint{3.406202in}{0.413320in}}%
\pgfpathlineto{\pgfqpoint{3.403394in}{0.413320in}}%
\pgfpathlineto{\pgfqpoint{3.400783in}{0.413320in}}%
\pgfpathlineto{\pgfqpoint{3.398037in}{0.413320in}}%
\pgfpathlineto{\pgfqpoint{3.395461in}{0.413320in}}%
\pgfpathlineto{\pgfqpoint{3.392681in}{0.413320in}}%
\pgfpathlineto{\pgfqpoint{3.390102in}{0.413320in}}%
\pgfpathlineto{\pgfqpoint{3.387309in}{0.413320in}}%
\pgfpathlineto{\pgfqpoint{3.384647in}{0.413320in}}%
\pgfpathlineto{\pgfqpoint{3.381959in}{0.413320in}}%
\pgfpathlineto{\pgfqpoint{3.379290in}{0.413320in}}%
\pgfpathlineto{\pgfqpoint{3.376735in}{0.413320in}}%
\pgfpathlineto{\pgfqpoint{3.373921in}{0.413320in}}%
\pgfpathlineto{\pgfqpoint{3.371357in}{0.413320in}}%
\pgfpathlineto{\pgfqpoint{3.368577in}{0.413320in}}%
\pgfpathlineto{\pgfqpoint{3.365996in}{0.413320in}}%
\pgfpathlineto{\pgfqpoint{3.363221in}{0.413320in}}%
\pgfpathlineto{\pgfqpoint{3.360620in}{0.413320in}}%
\pgfpathlineto{\pgfqpoint{3.357862in}{0.413320in}}%
\pgfpathlineto{\pgfqpoint{3.355177in}{0.413320in}}%
\pgfpathlineto{\pgfqpoint{3.352505in}{0.413320in}}%
\pgfpathlineto{\pgfqpoint{3.349828in}{0.413320in}}%
\pgfpathlineto{\pgfqpoint{3.347139in}{0.413320in}}%
\pgfpathlineto{\pgfqpoint{3.344468in}{0.413320in}}%
\pgfpathlineto{\pgfqpoint{3.341893in}{0.413320in}}%
\pgfpathlineto{\pgfqpoint{3.339101in}{0.413320in}}%
\pgfpathlineto{\pgfqpoint{3.336541in}{0.413320in}}%
\pgfpathlineto{\pgfqpoint{3.333758in}{0.413320in}}%
\pgfpathlineto{\pgfqpoint{3.331183in}{0.413320in}}%
\pgfpathlineto{\pgfqpoint{3.328401in}{0.413320in}}%
\pgfpathlineto{\pgfqpoint{3.325860in}{0.413320in}}%
\pgfpathlineto{\pgfqpoint{3.323049in}{0.413320in}}%
\pgfpathlineto{\pgfqpoint{3.320366in}{0.413320in}}%
\pgfpathlineto{\pgfqpoint{3.317688in}{0.413320in}}%
\pgfpathlineto{\pgfqpoint{3.315008in}{0.413320in}}%
\pgfpathlineto{\pgfqpoint{3.312480in}{0.413320in}}%
\pgfpathlineto{\pgfqpoint{3.309652in}{0.413320in}}%
\pgfpathlineto{\pgfqpoint{3.307104in}{0.413320in}}%
\pgfpathlineto{\pgfqpoint{3.304295in}{0.413320in}}%
\pgfpathlineto{\pgfqpoint{3.301719in}{0.413320in}}%
\pgfpathlineto{\pgfqpoint{3.298937in}{0.413320in}}%
\pgfpathlineto{\pgfqpoint{3.296376in}{0.413320in}}%
\pgfpathlineto{\pgfqpoint{3.293574in}{0.413320in}}%
\pgfpathlineto{\pgfqpoint{3.290890in}{0.413320in}}%
\pgfpathlineto{\pgfqpoint{3.288225in}{0.413320in}}%
\pgfpathlineto{\pgfqpoint{3.285534in}{0.413320in}}%
\pgfpathlineto{\pgfqpoint{3.282870in}{0.413320in}}%
\pgfpathlineto{\pgfqpoint{3.280189in}{0.413320in}}%
\pgfpathlineto{\pgfqpoint{3.277603in}{0.413320in}}%
\pgfpathlineto{\pgfqpoint{3.274831in}{0.413320in}}%
\pgfpathlineto{\pgfqpoint{3.272254in}{0.413320in}}%
\pgfpathlineto{\pgfqpoint{3.269478in}{0.413320in}}%
\pgfpathlineto{\pgfqpoint{3.266849in}{0.413320in}}%
\pgfpathlineto{\pgfqpoint{3.264119in}{0.413320in}}%
\pgfpathlineto{\pgfqpoint{3.261594in}{0.413320in}}%
\pgfpathlineto{\pgfqpoint{3.258784in}{0.413320in}}%
\pgfpathlineto{\pgfqpoint{3.256083in}{0.413320in}}%
\pgfpathlineto{\pgfqpoint{3.253404in}{0.413320in}}%
\pgfpathlineto{\pgfqpoint{3.250716in}{0.413320in}}%
\pgfpathlineto{\pgfqpoint{3.248049in}{0.413320in}}%
\pgfpathlineto{\pgfqpoint{3.245363in}{0.413320in}}%
\pgfpathlineto{\pgfqpoint{3.242807in}{0.413320in}}%
\pgfpathlineto{\pgfqpoint{3.240010in}{0.413320in}}%
\pgfpathlineto{\pgfqpoint{3.237411in}{0.413320in}}%
\pgfpathlineto{\pgfqpoint{3.234658in}{0.413320in}}%
\pgfpathlineto{\pgfqpoint{3.232069in}{0.413320in}}%
\pgfpathlineto{\pgfqpoint{3.229310in}{0.413320in}}%
\pgfpathlineto{\pgfqpoint{3.226609in}{0.413320in}}%
\pgfpathlineto{\pgfqpoint{3.223942in}{0.413320in}}%
\pgfpathlineto{\pgfqpoint{3.221255in}{0.413320in}}%
\pgfpathlineto{\pgfqpoint{3.218586in}{0.413320in}}%
\pgfpathlineto{\pgfqpoint{3.215908in}{0.413320in}}%
\pgfpathlineto{\pgfqpoint{3.213342in}{0.413320in}}%
\pgfpathlineto{\pgfqpoint{3.210545in}{0.413320in}}%
\pgfpathlineto{\pgfqpoint{3.207984in}{0.413320in}}%
\pgfpathlineto{\pgfqpoint{3.205195in}{0.413320in}}%
\pgfpathlineto{\pgfqpoint{3.202562in}{0.413320in}}%
\pgfpathlineto{\pgfqpoint{3.199823in}{0.413320in}}%
\pgfpathlineto{\pgfqpoint{3.197226in}{0.413320in}}%
\pgfpathlineto{\pgfqpoint{3.194508in}{0.413320in}}%
\pgfpathlineto{\pgfqpoint{3.191796in}{0.413320in}}%
\pgfpathlineto{\pgfqpoint{3.189117in}{0.413320in}}%
\pgfpathlineto{\pgfqpoint{3.186440in}{0.413320in}}%
\pgfpathlineto{\pgfqpoint{3.183760in}{0.413320in}}%
\pgfpathlineto{\pgfqpoint{3.181089in}{0.413320in}}%
\pgfpathlineto{\pgfqpoint{3.178525in}{0.413320in}}%
\pgfpathlineto{\pgfqpoint{3.175724in}{0.413320in}}%
\pgfpathlineto{\pgfqpoint{3.173142in}{0.413320in}}%
\pgfpathlineto{\pgfqpoint{3.170375in}{0.413320in}}%
\pgfpathlineto{\pgfqpoint{3.167776in}{0.413320in}}%
\pgfpathlineto{\pgfqpoint{3.165019in}{0.413320in}}%
\pgfpathlineto{\pgfqpoint{3.162474in}{0.413320in}}%
\pgfpathlineto{\pgfqpoint{3.159675in}{0.413320in}}%
\pgfpathlineto{\pgfqpoint{3.156981in}{0.413320in}}%
\pgfpathlineto{\pgfqpoint{3.154327in}{0.413320in}}%
\pgfpathlineto{\pgfqpoint{3.151612in}{0.413320in}}%
\pgfpathlineto{\pgfqpoint{3.149057in}{0.413320in}}%
\pgfpathlineto{\pgfqpoint{3.146271in}{0.413320in}}%
\pgfpathlineto{\pgfqpoint{3.143740in}{0.413320in}}%
\pgfpathlineto{\pgfqpoint{3.140913in}{0.413320in}}%
\pgfpathlineto{\pgfqpoint{3.138375in}{0.413320in}}%
\pgfpathlineto{\pgfqpoint{3.135550in}{0.413320in}}%
\pgfpathlineto{\pgfqpoint{3.132946in}{0.413320in}}%
\pgfpathlineto{\pgfqpoint{3.130199in}{0.413320in}}%
\pgfpathlineto{\pgfqpoint{3.127512in}{0.413320in}}%
\pgfpathlineto{\pgfqpoint{3.124842in}{0.413320in}}%
\pgfpathlineto{\pgfqpoint{3.122163in}{0.413320in}}%
\pgfpathlineto{\pgfqpoint{3.119487in}{0.413320in}}%
\pgfpathlineto{\pgfqpoint{3.116807in}{0.413320in}}%
\pgfpathlineto{\pgfqpoint{3.114242in}{0.413320in}}%
\pgfpathlineto{\pgfqpoint{3.111451in}{0.413320in}}%
\pgfpathlineto{\pgfqpoint{3.108896in}{0.413320in}}%
\pgfpathlineto{\pgfqpoint{3.106094in}{0.413320in}}%
\pgfpathlineto{\pgfqpoint{3.103508in}{0.413320in}}%
\pgfpathlineto{\pgfqpoint{3.100737in}{0.413320in}}%
\pgfpathlineto{\pgfqpoint{3.098163in}{0.413320in}}%
\pgfpathlineto{\pgfqpoint{3.095388in}{0.413320in}}%
\pgfpathlineto{\pgfqpoint{3.092699in}{0.413320in}}%
\pgfpathlineto{\pgfqpoint{3.090023in}{0.413320in}}%
\pgfpathlineto{\pgfqpoint{3.087343in}{0.413320in}}%
\pgfpathlineto{\pgfqpoint{3.084671in}{0.413320in}}%
\pgfpathlineto{\pgfqpoint{3.081990in}{0.413320in}}%
\pgfpathlineto{\pgfqpoint{3.079381in}{0.413320in}}%
\pgfpathlineto{\pgfqpoint{3.076631in}{0.413320in}}%
\pgfpathlineto{\pgfqpoint{3.074056in}{0.413320in}}%
\pgfpathlineto{\pgfqpoint{3.071266in}{0.413320in}}%
\pgfpathlineto{\pgfqpoint{3.068709in}{0.413320in}}%
\pgfpathlineto{\pgfqpoint{3.065916in}{0.413320in}}%
\pgfpathlineto{\pgfqpoint{3.063230in}{0.413320in}}%
\pgfpathlineto{\pgfqpoint{3.060561in}{0.413320in}}%
\pgfpathlineto{\pgfqpoint{3.057884in}{0.413320in}}%
\pgfpathlineto{\pgfqpoint{3.055202in}{0.413320in}}%
\pgfpathlineto{\pgfqpoint{3.052526in}{0.413320in}}%
\pgfpathlineto{\pgfqpoint{3.049988in}{0.413320in}}%
\pgfpathlineto{\pgfqpoint{3.047157in}{0.413320in}}%
\pgfpathlineto{\pgfqpoint{3.044568in}{0.413320in}}%
\pgfpathlineto{\pgfqpoint{3.041813in}{0.413320in}}%
\pgfpathlineto{\pgfqpoint{3.039262in}{0.413320in}}%
\pgfpathlineto{\pgfqpoint{3.036456in}{0.413320in}}%
\pgfpathlineto{\pgfqpoint{3.033921in}{0.413320in}}%
\pgfpathlineto{\pgfqpoint{3.031091in}{0.413320in}}%
\pgfpathlineto{\pgfqpoint{3.028412in}{0.413320in}}%
\pgfpathlineto{\pgfqpoint{3.025803in}{0.413320in}}%
\pgfpathlineto{\pgfqpoint{3.023058in}{0.413320in}}%
\pgfpathlineto{\pgfqpoint{3.020382in}{0.413320in}}%
\pgfpathlineto{\pgfqpoint{3.017707in}{0.413320in}}%
\pgfpathlineto{\pgfqpoint{3.015097in}{0.413320in}}%
\pgfpathlineto{\pgfqpoint{3.012351in}{0.413320in}}%
\pgfpathlineto{\pgfqpoint{3.009784in}{0.413320in}}%
\pgfpathlineto{\pgfqpoint{3.006993in}{0.413320in}}%
\pgfpathlineto{\pgfqpoint{3.004419in}{0.413320in}}%
\pgfpathlineto{\pgfqpoint{3.001635in}{0.413320in}}%
\pgfpathlineto{\pgfqpoint{2.999103in}{0.413320in}}%
\pgfpathlineto{\pgfqpoint{2.996300in}{0.413320in}}%
\pgfpathlineto{\pgfqpoint{2.993595in}{0.413320in}}%
\pgfpathlineto{\pgfqpoint{2.990978in}{0.413320in}}%
\pgfpathlineto{\pgfqpoint{2.988238in}{0.413320in}}%
\pgfpathlineto{\pgfqpoint{2.985666in}{0.413320in}}%
\pgfpathlineto{\pgfqpoint{2.982885in}{0.413320in}}%
\pgfpathlineto{\pgfqpoint{2.980341in}{0.413320in}}%
\pgfpathlineto{\pgfqpoint{2.977517in}{0.413320in}}%
\pgfpathlineto{\pgfqpoint{2.974972in}{0.413320in}}%
\pgfpathlineto{\pgfqpoint{2.972177in}{0.413320in}}%
\pgfpathlineto{\pgfqpoint{2.969599in}{0.413320in}}%
\pgfpathlineto{\pgfqpoint{2.966812in}{0.413320in}}%
\pgfpathlineto{\pgfqpoint{2.964127in}{0.413320in}}%
\pgfpathlineto{\pgfqpoint{2.961460in}{0.413320in}}%
\pgfpathlineto{\pgfqpoint{2.958782in}{0.413320in}}%
\pgfpathlineto{\pgfqpoint{2.956103in}{0.413320in}}%
\pgfpathlineto{\pgfqpoint{2.953422in}{0.413320in}}%
\pgfpathlineto{\pgfqpoint{2.950884in}{0.413320in}}%
\pgfpathlineto{\pgfqpoint{2.948068in}{0.413320in}}%
\pgfpathlineto{\pgfqpoint{2.945461in}{0.413320in}}%
\pgfpathlineto{\pgfqpoint{2.942711in}{0.413320in}}%
\pgfpathlineto{\pgfqpoint{2.940120in}{0.413320in}}%
\pgfpathlineto{\pgfqpoint{2.937352in}{0.413320in}}%
\pgfpathlineto{\pgfqpoint{2.934759in}{0.413320in}}%
\pgfpathlineto{\pgfqpoint{2.932033in}{0.413320in}}%
\pgfpathlineto{\pgfqpoint{2.929321in}{0.413320in}}%
\pgfpathlineto{\pgfqpoint{2.926655in}{0.413320in}}%
\pgfpathlineto{\pgfqpoint{2.923963in}{0.413320in}}%
\pgfpathlineto{\pgfqpoint{2.921363in}{0.413320in}}%
\pgfpathlineto{\pgfqpoint{2.918606in}{0.413320in}}%
\pgfpathlineto{\pgfqpoint{2.916061in}{0.413320in}}%
\pgfpathlineto{\pgfqpoint{2.913243in}{0.413320in}}%
\pgfpathlineto{\pgfqpoint{2.910631in}{0.413320in}}%
\pgfpathlineto{\pgfqpoint{2.907882in}{0.413320in}}%
\pgfpathlineto{\pgfqpoint{2.905341in}{0.413320in}}%
\pgfpathlineto{\pgfqpoint{2.902535in}{0.413320in}}%
\pgfpathlineto{\pgfqpoint{2.899858in}{0.413320in}}%
\pgfpathlineto{\pgfqpoint{2.897179in}{0.413320in}}%
\pgfpathlineto{\pgfqpoint{2.894487in}{0.413320in}}%
\pgfpathlineto{\pgfqpoint{2.891809in}{0.413320in}}%
\pgfpathlineto{\pgfqpoint{2.889145in}{0.413320in}}%
\pgfpathlineto{\pgfqpoint{2.886578in}{0.413320in}}%
\pgfpathlineto{\pgfqpoint{2.883780in}{0.413320in}}%
\pgfpathlineto{\pgfqpoint{2.881254in}{0.413320in}}%
\pgfpathlineto{\pgfqpoint{2.878431in}{0.413320in}}%
\pgfpathlineto{\pgfqpoint{2.875882in}{0.413320in}}%
\pgfpathlineto{\pgfqpoint{2.873074in}{0.413320in}}%
\pgfpathlineto{\pgfqpoint{2.870475in}{0.413320in}}%
\pgfpathlineto{\pgfqpoint{2.867713in}{0.413320in}}%
\pgfpathlineto{\pgfqpoint{2.865031in}{0.413320in}}%
\pgfpathlineto{\pgfqpoint{2.862402in}{0.413320in}}%
\pgfpathlineto{\pgfqpoint{2.859668in}{0.413320in}}%
\pgfpathlineto{\pgfqpoint{2.857003in}{0.413320in}}%
\pgfpathlineto{\pgfqpoint{2.854325in}{0.413320in}}%
\pgfpathlineto{\pgfqpoint{2.851793in}{0.413320in}}%
\pgfpathlineto{\pgfqpoint{2.848960in}{0.413320in}}%
\pgfpathlineto{\pgfqpoint{2.846408in}{0.413320in}}%
\pgfpathlineto{\pgfqpoint{2.843611in}{0.413320in}}%
\pgfpathlineto{\pgfqpoint{2.841055in}{0.413320in}}%
\pgfpathlineto{\pgfqpoint{2.838254in}{0.413320in}}%
\pgfpathlineto{\pgfqpoint{2.835698in}{0.413320in}}%
\pgfpathlineto{\pgfqpoint{2.832894in}{0.413320in}}%
\pgfpathlineto{\pgfqpoint{2.830219in}{0.413320in}}%
\pgfpathlineto{\pgfqpoint{2.827567in}{0.413320in}}%
\pgfpathlineto{\pgfqpoint{2.824851in}{0.413320in}}%
\pgfpathlineto{\pgfqpoint{2.822303in}{0.413320in}}%
\pgfpathlineto{\pgfqpoint{2.819506in}{0.413320in}}%
\pgfpathlineto{\pgfqpoint{2.816867in}{0.413320in}}%
\pgfpathlineto{\pgfqpoint{2.814141in}{0.413320in}}%
\pgfpathlineto{\pgfqpoint{2.811597in}{0.413320in}}%
\pgfpathlineto{\pgfqpoint{2.808792in}{0.413320in}}%
\pgfpathlineto{\pgfqpoint{2.806175in}{0.413320in}}%
\pgfpathlineto{\pgfqpoint{2.803435in}{0.413320in}}%
\pgfpathlineto{\pgfqpoint{2.800756in}{0.413320in}}%
\pgfpathlineto{\pgfqpoint{2.798070in}{0.413320in}}%
\pgfpathlineto{\pgfqpoint{2.795398in}{0.413320in}}%
\pgfpathlineto{\pgfqpoint{2.792721in}{0.413320in}}%
\pgfpathlineto{\pgfqpoint{2.790044in}{0.413320in}}%
\pgfpathlineto{\pgfqpoint{2.787468in}{0.413320in}}%
\pgfpathlineto{\pgfqpoint{2.784687in}{0.413320in}}%
\pgfpathlineto{\pgfqpoint{2.782113in}{0.413320in}}%
\pgfpathlineto{\pgfqpoint{2.779330in}{0.413320in}}%
\pgfpathlineto{\pgfqpoint{2.776767in}{0.413320in}}%
\pgfpathlineto{\pgfqpoint{2.773972in}{0.413320in}}%
\pgfpathlineto{\pgfqpoint{2.771373in}{0.413320in}}%
\pgfpathlineto{\pgfqpoint{2.768617in}{0.413320in}}%
\pgfpathlineto{\pgfqpoint{2.765935in}{0.413320in}}%
\pgfpathlineto{\pgfqpoint{2.763253in}{0.413320in}}%
\pgfpathlineto{\pgfqpoint{2.760581in}{0.413320in}}%
\pgfpathlineto{\pgfqpoint{2.758028in}{0.413320in}}%
\pgfpathlineto{\pgfqpoint{2.755224in}{0.413320in}}%
\pgfpathlineto{\pgfqpoint{2.752614in}{0.413320in}}%
\pgfpathlineto{\pgfqpoint{2.749868in}{0.413320in}}%
\pgfpathlineto{\pgfqpoint{2.747260in}{0.413320in}}%
\pgfpathlineto{\pgfqpoint{2.744510in}{0.413320in}}%
\pgfpathlineto{\pgfqpoint{2.741928in}{0.413320in}}%
\pgfpathlineto{\pgfqpoint{2.739155in}{0.413320in}}%
\pgfpathlineto{\pgfqpoint{2.736476in}{0.413320in}}%
\pgfpathlineto{\pgfqpoint{2.733798in}{0.413320in}}%
\pgfpathlineto{\pgfqpoint{2.731119in}{0.413320in}}%
\pgfpathlineto{\pgfqpoint{2.728439in}{0.413320in}}%
\pgfpathlineto{\pgfqpoint{2.725760in}{0.413320in}}%
\pgfpathlineto{\pgfqpoint{2.723211in}{0.413320in}}%
\pgfpathlineto{\pgfqpoint{2.720404in}{0.413320in}}%
\pgfpathlineto{\pgfqpoint{2.717773in}{0.413320in}}%
\pgfpathlineto{\pgfqpoint{2.715036in}{0.413320in}}%
\pgfpathlineto{\pgfqpoint{2.712477in}{0.413320in}}%
\pgfpathlineto{\pgfqpoint{2.709683in}{0.413320in}}%
\pgfpathlineto{\pgfqpoint{2.707125in}{0.413320in}}%
\pgfpathlineto{\pgfqpoint{2.704326in}{0.413320in}}%
\pgfpathlineto{\pgfqpoint{2.701657in}{0.413320in}}%
\pgfpathlineto{\pgfqpoint{2.698968in}{0.413320in}}%
\pgfpathlineto{\pgfqpoint{2.696293in}{0.413320in}}%
\pgfpathlineto{\pgfqpoint{2.693611in}{0.413320in}}%
\pgfpathlineto{\pgfqpoint{2.690940in}{0.413320in}}%
\pgfpathlineto{\pgfqpoint{2.688328in}{0.413320in}}%
\pgfpathlineto{\pgfqpoint{2.685586in}{0.413320in}}%
\pgfpathlineto{\pgfqpoint{2.683009in}{0.413320in}}%
\pgfpathlineto{\pgfqpoint{2.680224in}{0.413320in}}%
\pgfpathlineto{\pgfqpoint{2.677650in}{0.413320in}}%
\pgfpathlineto{\pgfqpoint{2.674873in}{0.413320in}}%
\pgfpathlineto{\pgfqpoint{2.672301in}{0.413320in}}%
\pgfpathlineto{\pgfqpoint{2.669506in}{0.413320in}}%
\pgfpathlineto{\pgfqpoint{2.666836in}{0.413320in}}%
\pgfpathlineto{\pgfqpoint{2.664151in}{0.413320in}}%
\pgfpathlineto{\pgfqpoint{2.661481in}{0.413320in}}%
\pgfpathlineto{\pgfqpoint{2.658942in}{0.413320in}}%
\pgfpathlineto{\pgfqpoint{2.656124in}{0.413320in}}%
\pgfpathlineto{\pgfqpoint{2.653567in}{0.413320in}}%
\pgfpathlineto{\pgfqpoint{2.650767in}{0.413320in}}%
\pgfpathlineto{\pgfqpoint{2.648196in}{0.413320in}}%
\pgfpathlineto{\pgfqpoint{2.645408in}{0.413320in}}%
\pgfpathlineto{\pgfqpoint{2.642827in}{0.413320in}}%
\pgfpathlineto{\pgfqpoint{2.640053in}{0.413320in}}%
\pgfpathlineto{\pgfqpoint{2.637369in}{0.413320in}}%
\pgfpathlineto{\pgfqpoint{2.634700in}{0.413320in}}%
\pgfpathlineto{\pgfqpoint{2.632018in}{0.413320in}}%
\pgfpathlineto{\pgfqpoint{2.629340in}{0.413320in}}%
\pgfpathlineto{\pgfqpoint{2.626653in}{0.413320in}}%
\pgfpathlineto{\pgfqpoint{2.624077in}{0.413320in}}%
\pgfpathlineto{\pgfqpoint{2.621304in}{0.413320in}}%
\pgfpathlineto{\pgfqpoint{2.618773in}{0.413320in}}%
\pgfpathlineto{\pgfqpoint{2.615934in}{0.413320in}}%
\pgfpathlineto{\pgfqpoint{2.613393in}{0.413320in}}%
\pgfpathlineto{\pgfqpoint{2.610588in}{0.413320in}}%
\pgfpathlineto{\pgfqpoint{2.608004in}{0.413320in}}%
\pgfpathlineto{\pgfqpoint{2.605232in}{0.413320in}}%
\pgfpathlineto{\pgfqpoint{2.602557in}{0.413320in}}%
\pgfpathlineto{\pgfqpoint{2.599920in}{0.413320in}}%
\pgfpathlineto{\pgfqpoint{2.597196in}{0.413320in}}%
\pgfpathlineto{\pgfqpoint{2.594630in}{0.413320in}}%
\pgfpathlineto{\pgfqpoint{2.591842in}{0.413320in}}%
\pgfpathlineto{\pgfqpoint{2.589248in}{0.413320in}}%
\pgfpathlineto{\pgfqpoint{2.586484in}{0.413320in}}%
\pgfpathlineto{\pgfqpoint{2.583913in}{0.413320in}}%
\pgfpathlineto{\pgfqpoint{2.581129in}{0.413320in}}%
\pgfpathlineto{\pgfqpoint{2.578567in}{0.413320in}}%
\pgfpathlineto{\pgfqpoint{2.575779in}{0.413320in}}%
\pgfpathlineto{\pgfqpoint{2.573082in}{0.413320in}}%
\pgfpathlineto{\pgfqpoint{2.570411in}{0.413320in}}%
\pgfpathlineto{\pgfqpoint{2.567730in}{0.413320in}}%
\pgfpathlineto{\pgfqpoint{2.565045in}{0.413320in}}%
\pgfpathlineto{\pgfqpoint{2.562375in}{0.413320in}}%
\pgfpathlineto{\pgfqpoint{2.559790in}{0.413320in}}%
\pgfpathlineto{\pgfqpoint{2.557009in}{0.413320in}}%
\pgfpathlineto{\pgfqpoint{2.554493in}{0.413320in}}%
\pgfpathlineto{\pgfqpoint{2.551664in}{0.413320in}}%
\pgfpathlineto{\pgfqpoint{2.549114in}{0.413320in}}%
\pgfpathlineto{\pgfqpoint{2.546310in}{0.413320in}}%
\pgfpathlineto{\pgfqpoint{2.543765in}{0.413320in}}%
\pgfpathlineto{\pgfqpoint{2.540949in}{0.413320in}}%
\pgfpathlineto{\pgfqpoint{2.538274in}{0.413320in}}%
\pgfpathlineto{\pgfqpoint{2.535624in}{0.413320in}}%
\pgfpathlineto{\pgfqpoint{2.532917in}{0.413320in}}%
\pgfpathlineto{\pgfqpoint{2.530234in}{0.413320in}}%
\pgfpathlineto{\pgfqpoint{2.527560in}{0.413320in}}%
\pgfpathlineto{\pgfqpoint{2.524988in}{0.413320in}}%
\pgfpathlineto{\pgfqpoint{2.522197in}{0.413320in}}%
\pgfpathlineto{\pgfqpoint{2.519607in}{0.413320in}}%
\pgfpathlineto{\pgfqpoint{2.516845in}{0.413320in}}%
\pgfpathlineto{\pgfqpoint{2.514268in}{0.413320in}}%
\pgfpathlineto{\pgfqpoint{2.511478in}{0.413320in}}%
\pgfpathlineto{\pgfqpoint{2.508917in}{0.413320in}}%
\pgfpathlineto{\pgfqpoint{2.506163in}{0.413320in}}%
\pgfpathlineto{\pgfqpoint{2.503454in}{0.413320in}}%
\pgfpathlineto{\pgfqpoint{2.500801in}{0.413320in}}%
\pgfpathlineto{\pgfqpoint{2.498085in}{0.413320in}}%
\pgfpathlineto{\pgfqpoint{2.495542in}{0.413320in}}%
\pgfpathlineto{\pgfqpoint{2.492729in}{0.413320in}}%
\pgfpathlineto{\pgfqpoint{2.490183in}{0.413320in}}%
\pgfpathlineto{\pgfqpoint{2.487384in}{0.413320in}}%
\pgfpathlineto{\pgfqpoint{2.484870in}{0.413320in}}%
\pgfpathlineto{\pgfqpoint{2.482026in}{0.413320in}}%
\pgfpathlineto{\pgfqpoint{2.479420in}{0.413320in}}%
\pgfpathlineto{\pgfqpoint{2.476671in}{0.413320in}}%
\pgfpathlineto{\pgfqpoint{2.473989in}{0.413320in}}%
\pgfpathlineto{\pgfqpoint{2.471311in}{0.413320in}}%
\pgfpathlineto{\pgfqpoint{2.468635in}{0.413320in}}%
\pgfpathlineto{\pgfqpoint{2.465957in}{0.413320in}}%
\pgfpathlineto{\pgfqpoint{2.463280in}{0.413320in}}%
\pgfpathlineto{\pgfqpoint{2.460711in}{0.413320in}}%
\pgfpathlineto{\pgfqpoint{2.457917in}{0.413320in}}%
\pgfpathlineto{\pgfqpoint{2.455353in}{0.413320in}}%
\pgfpathlineto{\pgfqpoint{2.452562in}{0.413320in}}%
\pgfpathlineto{\pgfqpoint{2.450032in}{0.413320in}}%
\pgfpathlineto{\pgfqpoint{2.447209in}{0.413320in}}%
\pgfpathlineto{\pgfqpoint{2.444677in}{0.413320in}}%
\pgfpathlineto{\pgfqpoint{2.441876in}{0.413320in}}%
\pgfpathlineto{\pgfqpoint{2.439167in}{0.413320in}}%
\pgfpathlineto{\pgfqpoint{2.436518in}{0.413320in}}%
\pgfpathlineto{\pgfqpoint{2.433815in}{0.413320in}}%
\pgfpathlineto{\pgfqpoint{2.431251in}{0.413320in}}%
\pgfpathlineto{\pgfqpoint{2.428453in}{0.413320in}}%
\pgfpathlineto{\pgfqpoint{2.425878in}{0.413320in}}%
\pgfpathlineto{\pgfqpoint{2.423098in}{0.413320in}}%
\pgfpathlineto{\pgfqpoint{2.420528in}{0.413320in}}%
\pgfpathlineto{\pgfqpoint{2.417747in}{0.413320in}}%
\pgfpathlineto{\pgfqpoint{2.415184in}{0.413320in}}%
\pgfpathlineto{\pgfqpoint{2.412389in}{0.413320in}}%
\pgfpathlineto{\pgfqpoint{2.409699in}{0.413320in}}%
\pgfpathlineto{\pgfqpoint{2.407024in}{0.413320in}}%
\pgfpathlineto{\pgfqpoint{2.404352in}{0.413320in}}%
\pgfpathlineto{\pgfqpoint{2.401675in}{0.413320in}}%
\pgfpathlineto{\pgfqpoint{2.398995in}{0.413320in}}%
\pgfpathclose%
\pgfusepath{stroke,fill}%
\end{pgfscope}%
\begin{pgfscope}%
\pgfpathrectangle{\pgfqpoint{2.398995in}{0.319877in}}{\pgfqpoint{3.986877in}{1.993438in}} %
\pgfusepath{clip}%
\pgfsetbuttcap%
\pgfsetroundjoin%
\definecolor{currentfill}{rgb}{1.000000,1.000000,1.000000}%
\pgfsetfillcolor{currentfill}%
\pgfsetlinewidth{1.003750pt}%
\definecolor{currentstroke}{rgb}{0.967798,0.441275,0.535810}%
\pgfsetstrokecolor{currentstroke}%
\pgfsetdash{}{0pt}%
\pgfpathmoveto{\pgfqpoint{2.398995in}{0.413320in}}%
\pgfpathlineto{\pgfqpoint{2.398995in}{1.543038in}}%
\pgfpathlineto{\pgfqpoint{2.401675in}{1.544252in}}%
\pgfpathlineto{\pgfqpoint{2.404352in}{1.544827in}}%
\pgfpathlineto{\pgfqpoint{2.407024in}{1.544008in}}%
\pgfpathlineto{\pgfqpoint{2.409699in}{1.542638in}}%
\pgfpathlineto{\pgfqpoint{2.412389in}{1.545557in}}%
\pgfpathlineto{\pgfqpoint{2.415184in}{1.543292in}}%
\pgfpathlineto{\pgfqpoint{2.417747in}{1.546756in}}%
\pgfpathlineto{\pgfqpoint{2.420528in}{1.544776in}}%
\pgfpathlineto{\pgfqpoint{2.423098in}{1.548835in}}%
\pgfpathlineto{\pgfqpoint{2.425878in}{1.549809in}}%
\pgfpathlineto{\pgfqpoint{2.428453in}{1.548530in}}%
\pgfpathlineto{\pgfqpoint{2.431251in}{1.541833in}}%
\pgfpathlineto{\pgfqpoint{2.433815in}{1.537736in}}%
\pgfpathlineto{\pgfqpoint{2.436518in}{1.542076in}}%
\pgfpathlineto{\pgfqpoint{2.439167in}{1.546925in}}%
\pgfpathlineto{\pgfqpoint{2.441876in}{1.550244in}}%
\pgfpathlineto{\pgfqpoint{2.444677in}{1.550245in}}%
\pgfpathlineto{\pgfqpoint{2.447209in}{1.552233in}}%
\pgfpathlineto{\pgfqpoint{2.450032in}{1.546061in}}%
\pgfpathlineto{\pgfqpoint{2.452562in}{1.547141in}}%
\pgfpathlineto{\pgfqpoint{2.455353in}{1.549271in}}%
\pgfpathlineto{\pgfqpoint{2.457917in}{1.555998in}}%
\pgfpathlineto{\pgfqpoint{2.460711in}{1.560485in}}%
\pgfpathlineto{\pgfqpoint{2.463280in}{1.552674in}}%
\pgfpathlineto{\pgfqpoint{2.465957in}{1.553375in}}%
\pgfpathlineto{\pgfqpoint{2.468635in}{1.547528in}}%
\pgfpathlineto{\pgfqpoint{2.471311in}{1.545700in}}%
\pgfpathlineto{\pgfqpoint{2.473989in}{1.547588in}}%
\pgfpathlineto{\pgfqpoint{2.476671in}{1.545357in}}%
\pgfpathlineto{\pgfqpoint{2.479420in}{1.548052in}}%
\pgfpathlineto{\pgfqpoint{2.482026in}{1.543901in}}%
\pgfpathlineto{\pgfqpoint{2.484870in}{1.547618in}}%
\pgfpathlineto{\pgfqpoint{2.487384in}{1.547313in}}%
\pgfpathlineto{\pgfqpoint{2.490183in}{1.547734in}}%
\pgfpathlineto{\pgfqpoint{2.492729in}{1.547388in}}%
\pgfpathlineto{\pgfqpoint{2.495542in}{1.542896in}}%
\pgfpathlineto{\pgfqpoint{2.498085in}{1.544153in}}%
\pgfpathlineto{\pgfqpoint{2.500801in}{1.542228in}}%
\pgfpathlineto{\pgfqpoint{2.503454in}{1.549467in}}%
\pgfpathlineto{\pgfqpoint{2.506163in}{1.544888in}}%
\pgfpathlineto{\pgfqpoint{2.508917in}{1.546079in}}%
\pgfpathlineto{\pgfqpoint{2.511478in}{1.540341in}}%
\pgfpathlineto{\pgfqpoint{2.514268in}{1.542709in}}%
\pgfpathlineto{\pgfqpoint{2.516845in}{1.552094in}}%
\pgfpathlineto{\pgfqpoint{2.519607in}{1.559030in}}%
\pgfpathlineto{\pgfqpoint{2.522197in}{1.555314in}}%
\pgfpathlineto{\pgfqpoint{2.524988in}{1.545363in}}%
\pgfpathlineto{\pgfqpoint{2.527560in}{1.545190in}}%
\pgfpathlineto{\pgfqpoint{2.530234in}{1.537288in}}%
\pgfpathlineto{\pgfqpoint{2.532917in}{1.546897in}}%
\pgfpathlineto{\pgfqpoint{2.535624in}{1.546353in}}%
\pgfpathlineto{\pgfqpoint{2.538274in}{1.553148in}}%
\pgfpathlineto{\pgfqpoint{2.540949in}{1.546558in}}%
\pgfpathlineto{\pgfqpoint{2.543765in}{1.547634in}}%
\pgfpathlineto{\pgfqpoint{2.546310in}{1.550566in}}%
\pgfpathlineto{\pgfqpoint{2.549114in}{1.550668in}}%
\pgfpathlineto{\pgfqpoint{2.551664in}{1.542629in}}%
\pgfpathlineto{\pgfqpoint{2.554493in}{1.540084in}}%
\pgfpathlineto{\pgfqpoint{2.557009in}{1.535800in}}%
\pgfpathlineto{\pgfqpoint{2.559790in}{1.536427in}}%
\pgfpathlineto{\pgfqpoint{2.562375in}{1.539757in}}%
\pgfpathlineto{\pgfqpoint{2.565045in}{1.541416in}}%
\pgfpathlineto{\pgfqpoint{2.567730in}{1.540438in}}%
\pgfpathlineto{\pgfqpoint{2.570411in}{1.535263in}}%
\pgfpathlineto{\pgfqpoint{2.573082in}{1.532260in}}%
\pgfpathlineto{\pgfqpoint{2.575779in}{1.532637in}}%
\pgfpathlineto{\pgfqpoint{2.578567in}{1.536355in}}%
\pgfpathlineto{\pgfqpoint{2.581129in}{1.532276in}}%
\pgfpathlineto{\pgfqpoint{2.583913in}{1.532807in}}%
\pgfpathlineto{\pgfqpoint{2.586484in}{1.539766in}}%
\pgfpathlineto{\pgfqpoint{2.589248in}{1.544929in}}%
\pgfpathlineto{\pgfqpoint{2.591842in}{1.541276in}}%
\pgfpathlineto{\pgfqpoint{2.594630in}{1.545948in}}%
\pgfpathlineto{\pgfqpoint{2.597196in}{1.544025in}}%
\pgfpathlineto{\pgfqpoint{2.599920in}{1.539735in}}%
\pgfpathlineto{\pgfqpoint{2.602557in}{1.537544in}}%
\pgfpathlineto{\pgfqpoint{2.605232in}{1.540119in}}%
\pgfpathlineto{\pgfqpoint{2.608004in}{1.537013in}}%
\pgfpathlineto{\pgfqpoint{2.610588in}{1.536178in}}%
\pgfpathlineto{\pgfqpoint{2.613393in}{1.541009in}}%
\pgfpathlineto{\pgfqpoint{2.615934in}{1.541133in}}%
\pgfpathlineto{\pgfqpoint{2.618773in}{1.541128in}}%
\pgfpathlineto{\pgfqpoint{2.621304in}{1.537868in}}%
\pgfpathlineto{\pgfqpoint{2.624077in}{1.539001in}}%
\pgfpathlineto{\pgfqpoint{2.626653in}{1.541653in}}%
\pgfpathlineto{\pgfqpoint{2.629340in}{1.538043in}}%
\pgfpathlineto{\pgfqpoint{2.632018in}{1.539210in}}%
\pgfpathlineto{\pgfqpoint{2.634700in}{1.543874in}}%
\pgfpathlineto{\pgfqpoint{2.637369in}{1.544743in}}%
\pgfpathlineto{\pgfqpoint{2.640053in}{1.547921in}}%
\pgfpathlineto{\pgfqpoint{2.642827in}{1.542595in}}%
\pgfpathlineto{\pgfqpoint{2.645408in}{1.543082in}}%
\pgfpathlineto{\pgfqpoint{2.648196in}{1.547541in}}%
\pgfpathlineto{\pgfqpoint{2.650767in}{1.552046in}}%
\pgfpathlineto{\pgfqpoint{2.653567in}{1.544885in}}%
\pgfpathlineto{\pgfqpoint{2.656124in}{1.546818in}}%
\pgfpathlineto{\pgfqpoint{2.658942in}{1.546395in}}%
\pgfpathlineto{\pgfqpoint{2.661481in}{1.549999in}}%
\pgfpathlineto{\pgfqpoint{2.664151in}{1.543631in}}%
\pgfpathlineto{\pgfqpoint{2.666836in}{1.545553in}}%
\pgfpathlineto{\pgfqpoint{2.669506in}{1.549280in}}%
\pgfpathlineto{\pgfqpoint{2.672301in}{1.548631in}}%
\pgfpathlineto{\pgfqpoint{2.674873in}{1.542129in}}%
\pgfpathlineto{\pgfqpoint{2.677650in}{1.540625in}}%
\pgfpathlineto{\pgfqpoint{2.680224in}{1.543899in}}%
\pgfpathlineto{\pgfqpoint{2.683009in}{1.543580in}}%
\pgfpathlineto{\pgfqpoint{2.685586in}{1.543539in}}%
\pgfpathlineto{\pgfqpoint{2.688328in}{1.546501in}}%
\pgfpathlineto{\pgfqpoint{2.690940in}{1.546694in}}%
\pgfpathlineto{\pgfqpoint{2.693611in}{1.539474in}}%
\pgfpathlineto{\pgfqpoint{2.696293in}{1.545118in}}%
\pgfpathlineto{\pgfqpoint{2.698968in}{1.530983in}}%
\pgfpathlineto{\pgfqpoint{2.701657in}{1.531808in}}%
\pgfpathlineto{\pgfqpoint{2.704326in}{1.542858in}}%
\pgfpathlineto{\pgfqpoint{2.707125in}{1.541472in}}%
\pgfpathlineto{\pgfqpoint{2.709683in}{1.537466in}}%
\pgfpathlineto{\pgfqpoint{2.712477in}{1.555845in}}%
\pgfpathlineto{\pgfqpoint{2.715036in}{1.544191in}}%
\pgfpathlineto{\pgfqpoint{2.717773in}{1.537197in}}%
\pgfpathlineto{\pgfqpoint{2.720404in}{1.543927in}}%
\pgfpathlineto{\pgfqpoint{2.723211in}{1.538867in}}%
\pgfpathlineto{\pgfqpoint{2.725760in}{1.537526in}}%
\pgfpathlineto{\pgfqpoint{2.728439in}{1.537263in}}%
\pgfpathlineto{\pgfqpoint{2.731119in}{1.536469in}}%
\pgfpathlineto{\pgfqpoint{2.733798in}{1.535718in}}%
\pgfpathlineto{\pgfqpoint{2.736476in}{1.532721in}}%
\pgfpathlineto{\pgfqpoint{2.739155in}{1.537935in}}%
\pgfpathlineto{\pgfqpoint{2.741928in}{1.534417in}}%
\pgfpathlineto{\pgfqpoint{2.744510in}{1.535251in}}%
\pgfpathlineto{\pgfqpoint{2.747260in}{1.537781in}}%
\pgfpathlineto{\pgfqpoint{2.749868in}{1.535339in}}%
\pgfpathlineto{\pgfqpoint{2.752614in}{1.539228in}}%
\pgfpathlineto{\pgfqpoint{2.755224in}{1.535446in}}%
\pgfpathlineto{\pgfqpoint{2.758028in}{1.544851in}}%
\pgfpathlineto{\pgfqpoint{2.760581in}{1.543477in}}%
\pgfpathlineto{\pgfqpoint{2.763253in}{1.542810in}}%
\pgfpathlineto{\pgfqpoint{2.765935in}{1.545160in}}%
\pgfpathlineto{\pgfqpoint{2.768617in}{1.540843in}}%
\pgfpathlineto{\pgfqpoint{2.771373in}{1.541514in}}%
\pgfpathlineto{\pgfqpoint{2.773972in}{1.538273in}}%
\pgfpathlineto{\pgfqpoint{2.776767in}{1.537124in}}%
\pgfpathlineto{\pgfqpoint{2.779330in}{1.535665in}}%
\pgfpathlineto{\pgfqpoint{2.782113in}{1.538007in}}%
\pgfpathlineto{\pgfqpoint{2.784687in}{1.534315in}}%
\pgfpathlineto{\pgfqpoint{2.787468in}{1.535898in}}%
\pgfpathlineto{\pgfqpoint{2.790044in}{1.533754in}}%
\pgfpathlineto{\pgfqpoint{2.792721in}{1.534237in}}%
\pgfpathlineto{\pgfqpoint{2.795398in}{1.540165in}}%
\pgfpathlineto{\pgfqpoint{2.798070in}{1.542882in}}%
\pgfpathlineto{\pgfqpoint{2.800756in}{1.546770in}}%
\pgfpathlineto{\pgfqpoint{2.803435in}{1.541200in}}%
\pgfpathlineto{\pgfqpoint{2.806175in}{1.543345in}}%
\pgfpathlineto{\pgfqpoint{2.808792in}{1.533038in}}%
\pgfpathlineto{\pgfqpoint{2.811597in}{1.540150in}}%
\pgfpathlineto{\pgfqpoint{2.814141in}{1.546439in}}%
\pgfpathlineto{\pgfqpoint{2.816867in}{1.545312in}}%
\pgfpathlineto{\pgfqpoint{2.819506in}{1.548433in}}%
\pgfpathlineto{\pgfqpoint{2.822303in}{1.553181in}}%
\pgfpathlineto{\pgfqpoint{2.824851in}{1.543387in}}%
\pgfpathlineto{\pgfqpoint{2.827567in}{1.551735in}}%
\pgfpathlineto{\pgfqpoint{2.830219in}{1.545250in}}%
\pgfpathlineto{\pgfqpoint{2.832894in}{1.547231in}}%
\pgfpathlineto{\pgfqpoint{2.835698in}{1.554205in}}%
\pgfpathlineto{\pgfqpoint{2.838254in}{1.549078in}}%
\pgfpathlineto{\pgfqpoint{2.841055in}{1.547533in}}%
\pgfpathlineto{\pgfqpoint{2.843611in}{1.549865in}}%
\pgfpathlineto{\pgfqpoint{2.846408in}{1.551508in}}%
\pgfpathlineto{\pgfqpoint{2.848960in}{1.551570in}}%
\pgfpathlineto{\pgfqpoint{2.851793in}{1.543584in}}%
\pgfpathlineto{\pgfqpoint{2.854325in}{1.546764in}}%
\pgfpathlineto{\pgfqpoint{2.857003in}{1.542377in}}%
\pgfpathlineto{\pgfqpoint{2.859668in}{1.551180in}}%
\pgfpathlineto{\pgfqpoint{2.862402in}{1.552703in}}%
\pgfpathlineto{\pgfqpoint{2.865031in}{1.549288in}}%
\pgfpathlineto{\pgfqpoint{2.867713in}{1.546323in}}%
\pgfpathlineto{\pgfqpoint{2.870475in}{1.538971in}}%
\pgfpathlineto{\pgfqpoint{2.873074in}{1.542853in}}%
\pgfpathlineto{\pgfqpoint{2.875882in}{1.545805in}}%
\pgfpathlineto{\pgfqpoint{2.878431in}{1.550047in}}%
\pgfpathlineto{\pgfqpoint{2.881254in}{1.546862in}}%
\pgfpathlineto{\pgfqpoint{2.883780in}{1.543060in}}%
\pgfpathlineto{\pgfqpoint{2.886578in}{1.537241in}}%
\pgfpathlineto{\pgfqpoint{2.889145in}{1.545148in}}%
\pgfpathlineto{\pgfqpoint{2.891809in}{1.541741in}}%
\pgfpathlineto{\pgfqpoint{2.894487in}{1.542731in}}%
\pgfpathlineto{\pgfqpoint{2.897179in}{1.539796in}}%
\pgfpathlineto{\pgfqpoint{2.899858in}{1.542304in}}%
\pgfpathlineto{\pgfqpoint{2.902535in}{1.539341in}}%
\pgfpathlineto{\pgfqpoint{2.905341in}{1.539063in}}%
\pgfpathlineto{\pgfqpoint{2.907882in}{1.543643in}}%
\pgfpathlineto{\pgfqpoint{2.910631in}{1.543287in}}%
\pgfpathlineto{\pgfqpoint{2.913243in}{1.546674in}}%
\pgfpathlineto{\pgfqpoint{2.916061in}{1.535899in}}%
\pgfpathlineto{\pgfqpoint{2.918606in}{1.542256in}}%
\pgfpathlineto{\pgfqpoint{2.921363in}{1.546406in}}%
\pgfpathlineto{\pgfqpoint{2.923963in}{1.548038in}}%
\pgfpathlineto{\pgfqpoint{2.926655in}{1.546242in}}%
\pgfpathlineto{\pgfqpoint{2.929321in}{1.549431in}}%
\pgfpathlineto{\pgfqpoint{2.932033in}{1.549633in}}%
\pgfpathlineto{\pgfqpoint{2.934759in}{1.548540in}}%
\pgfpathlineto{\pgfqpoint{2.937352in}{1.544724in}}%
\pgfpathlineto{\pgfqpoint{2.940120in}{1.535604in}}%
\pgfpathlineto{\pgfqpoint{2.942711in}{1.536939in}}%
\pgfpathlineto{\pgfqpoint{2.945461in}{1.534046in}}%
\pgfpathlineto{\pgfqpoint{2.948068in}{1.535944in}}%
\pgfpathlineto{\pgfqpoint{2.950884in}{1.538034in}}%
\pgfpathlineto{\pgfqpoint{2.953422in}{1.542993in}}%
\pgfpathlineto{\pgfqpoint{2.956103in}{1.544353in}}%
\pgfpathlineto{\pgfqpoint{2.958782in}{1.541406in}}%
\pgfpathlineto{\pgfqpoint{2.961460in}{1.536344in}}%
\pgfpathlineto{\pgfqpoint{2.964127in}{1.542121in}}%
\pgfpathlineto{\pgfqpoint{2.966812in}{1.539893in}}%
\pgfpathlineto{\pgfqpoint{2.969599in}{1.545148in}}%
\pgfpathlineto{\pgfqpoint{2.972177in}{1.546770in}}%
\pgfpathlineto{\pgfqpoint{2.974972in}{1.545939in}}%
\pgfpathlineto{\pgfqpoint{2.977517in}{1.546570in}}%
\pgfpathlineto{\pgfqpoint{2.980341in}{1.543035in}}%
\pgfpathlineto{\pgfqpoint{2.982885in}{1.546528in}}%
\pgfpathlineto{\pgfqpoint{2.985666in}{1.552124in}}%
\pgfpathlineto{\pgfqpoint{2.988238in}{1.547620in}}%
\pgfpathlineto{\pgfqpoint{2.990978in}{1.544440in}}%
\pgfpathlineto{\pgfqpoint{2.993595in}{1.538763in}}%
\pgfpathlineto{\pgfqpoint{2.996300in}{1.541205in}}%
\pgfpathlineto{\pgfqpoint{2.999103in}{1.551561in}}%
\pgfpathlineto{\pgfqpoint{3.001635in}{1.549923in}}%
\pgfpathlineto{\pgfqpoint{3.004419in}{1.547038in}}%
\pgfpathlineto{\pgfqpoint{3.006993in}{1.545974in}}%
\pgfpathlineto{\pgfqpoint{3.009784in}{1.552667in}}%
\pgfpathlineto{\pgfqpoint{3.012351in}{1.554788in}}%
\pgfpathlineto{\pgfqpoint{3.015097in}{1.554130in}}%
\pgfpathlineto{\pgfqpoint{3.017707in}{1.553756in}}%
\pgfpathlineto{\pgfqpoint{3.020382in}{1.551222in}}%
\pgfpathlineto{\pgfqpoint{3.023058in}{1.548452in}}%
\pgfpathlineto{\pgfqpoint{3.025803in}{1.547597in}}%
\pgfpathlineto{\pgfqpoint{3.028412in}{1.548900in}}%
\pgfpathlineto{\pgfqpoint{3.031091in}{1.545084in}}%
\pgfpathlineto{\pgfqpoint{3.033921in}{1.539464in}}%
\pgfpathlineto{\pgfqpoint{3.036456in}{1.545393in}}%
\pgfpathlineto{\pgfqpoint{3.039262in}{1.538844in}}%
\pgfpathlineto{\pgfqpoint{3.041813in}{1.544519in}}%
\pgfpathlineto{\pgfqpoint{3.044568in}{1.551598in}}%
\pgfpathlineto{\pgfqpoint{3.047157in}{1.553329in}}%
\pgfpathlineto{\pgfqpoint{3.049988in}{1.544107in}}%
\pgfpathlineto{\pgfqpoint{3.052526in}{1.538808in}}%
\pgfpathlineto{\pgfqpoint{3.055202in}{1.533236in}}%
\pgfpathlineto{\pgfqpoint{3.057884in}{1.529571in}}%
\pgfpathlineto{\pgfqpoint{3.060561in}{1.531150in}}%
\pgfpathlineto{\pgfqpoint{3.063230in}{1.542752in}}%
\pgfpathlineto{\pgfqpoint{3.065916in}{1.539708in}}%
\pgfpathlineto{\pgfqpoint{3.068709in}{1.538162in}}%
\pgfpathlineto{\pgfqpoint{3.071266in}{1.544335in}}%
\pgfpathlineto{\pgfqpoint{3.074056in}{1.548942in}}%
\pgfpathlineto{\pgfqpoint{3.076631in}{1.543315in}}%
\pgfpathlineto{\pgfqpoint{3.079381in}{1.546026in}}%
\pgfpathlineto{\pgfqpoint{3.081990in}{1.544469in}}%
\pgfpathlineto{\pgfqpoint{3.084671in}{1.541684in}}%
\pgfpathlineto{\pgfqpoint{3.087343in}{1.543222in}}%
\pgfpathlineto{\pgfqpoint{3.090023in}{1.540530in}}%
\pgfpathlineto{\pgfqpoint{3.092699in}{1.532084in}}%
\pgfpathlineto{\pgfqpoint{3.095388in}{1.534281in}}%
\pgfpathlineto{\pgfqpoint{3.098163in}{1.536686in}}%
\pgfpathlineto{\pgfqpoint{3.100737in}{1.527350in}}%
\pgfpathlineto{\pgfqpoint{3.103508in}{1.533080in}}%
\pgfpathlineto{\pgfqpoint{3.106094in}{1.538450in}}%
\pgfpathlineto{\pgfqpoint{3.108896in}{1.545654in}}%
\pgfpathlineto{\pgfqpoint{3.111451in}{1.540635in}}%
\pgfpathlineto{\pgfqpoint{3.114242in}{1.541760in}}%
\pgfpathlineto{\pgfqpoint{3.116807in}{1.549407in}}%
\pgfpathlineto{\pgfqpoint{3.119487in}{1.541056in}}%
\pgfpathlineto{\pgfqpoint{3.122163in}{1.526149in}}%
\pgfpathlineto{\pgfqpoint{3.124842in}{1.524680in}}%
\pgfpathlineto{\pgfqpoint{3.127512in}{1.530855in}}%
\pgfpathlineto{\pgfqpoint{3.130199in}{1.533251in}}%
\pgfpathlineto{\pgfqpoint{3.132946in}{1.541024in}}%
\pgfpathlineto{\pgfqpoint{3.135550in}{1.535192in}}%
\pgfpathlineto{\pgfqpoint{3.138375in}{1.532902in}}%
\pgfpathlineto{\pgfqpoint{3.140913in}{1.525791in}}%
\pgfpathlineto{\pgfqpoint{3.143740in}{1.526744in}}%
\pgfpathlineto{\pgfqpoint{3.146271in}{1.525625in}}%
\pgfpathlineto{\pgfqpoint{3.149057in}{1.529233in}}%
\pgfpathlineto{\pgfqpoint{3.151612in}{1.524680in}}%
\pgfpathlineto{\pgfqpoint{3.154327in}{1.527170in}}%
\pgfpathlineto{\pgfqpoint{3.156981in}{1.524680in}}%
\pgfpathlineto{\pgfqpoint{3.159675in}{1.526415in}}%
\pgfpathlineto{\pgfqpoint{3.162474in}{1.525007in}}%
\pgfpathlineto{\pgfqpoint{3.165019in}{1.525151in}}%
\pgfpathlineto{\pgfqpoint{3.167776in}{1.532767in}}%
\pgfpathlineto{\pgfqpoint{3.170375in}{1.529609in}}%
\pgfpathlineto{\pgfqpoint{3.173142in}{1.527535in}}%
\pgfpathlineto{\pgfqpoint{3.175724in}{1.529861in}}%
\pgfpathlineto{\pgfqpoint{3.178525in}{1.527213in}}%
\pgfpathlineto{\pgfqpoint{3.181089in}{1.531109in}}%
\pgfpathlineto{\pgfqpoint{3.183760in}{1.531037in}}%
\pgfpathlineto{\pgfqpoint{3.186440in}{1.531618in}}%
\pgfpathlineto{\pgfqpoint{3.189117in}{1.524680in}}%
\pgfpathlineto{\pgfqpoint{3.191796in}{1.527724in}}%
\pgfpathlineto{\pgfqpoint{3.194508in}{1.530272in}}%
\pgfpathlineto{\pgfqpoint{3.197226in}{1.532062in}}%
\pgfpathlineto{\pgfqpoint{3.199823in}{1.535047in}}%
\pgfpathlineto{\pgfqpoint{3.202562in}{1.529880in}}%
\pgfpathlineto{\pgfqpoint{3.205195in}{1.525904in}}%
\pgfpathlineto{\pgfqpoint{3.207984in}{1.524680in}}%
\pgfpathlineto{\pgfqpoint{3.210545in}{1.537133in}}%
\pgfpathlineto{\pgfqpoint{3.213342in}{1.529073in}}%
\pgfpathlineto{\pgfqpoint{3.215908in}{1.533948in}}%
\pgfpathlineto{\pgfqpoint{3.218586in}{1.533341in}}%
\pgfpathlineto{\pgfqpoint{3.221255in}{1.530321in}}%
\pgfpathlineto{\pgfqpoint{3.223942in}{1.528396in}}%
\pgfpathlineto{\pgfqpoint{3.226609in}{1.524680in}}%
\pgfpathlineto{\pgfqpoint{3.229310in}{1.529780in}}%
\pgfpathlineto{\pgfqpoint{3.232069in}{1.528647in}}%
\pgfpathlineto{\pgfqpoint{3.234658in}{1.532984in}}%
\pgfpathlineto{\pgfqpoint{3.237411in}{1.534799in}}%
\pgfpathlineto{\pgfqpoint{3.240010in}{1.535539in}}%
\pgfpathlineto{\pgfqpoint{3.242807in}{1.532458in}}%
\pgfpathlineto{\pgfqpoint{3.245363in}{1.537954in}}%
\pgfpathlineto{\pgfqpoint{3.248049in}{1.537175in}}%
\pgfpathlineto{\pgfqpoint{3.250716in}{1.539972in}}%
\pgfpathlineto{\pgfqpoint{3.253404in}{1.541343in}}%
\pgfpathlineto{\pgfqpoint{3.256083in}{1.539020in}}%
\pgfpathlineto{\pgfqpoint{3.258784in}{1.543134in}}%
\pgfpathlineto{\pgfqpoint{3.261594in}{1.546321in}}%
\pgfpathlineto{\pgfqpoint{3.264119in}{1.543739in}}%
\pgfpathlineto{\pgfqpoint{3.266849in}{1.546327in}}%
\pgfpathlineto{\pgfqpoint{3.269478in}{1.547564in}}%
\pgfpathlineto{\pgfqpoint{3.272254in}{1.539952in}}%
\pgfpathlineto{\pgfqpoint{3.274831in}{1.543596in}}%
\pgfpathlineto{\pgfqpoint{3.277603in}{1.539483in}}%
\pgfpathlineto{\pgfqpoint{3.280189in}{1.543882in}}%
\pgfpathlineto{\pgfqpoint{3.282870in}{1.540157in}}%
\pgfpathlineto{\pgfqpoint{3.285534in}{1.535555in}}%
\pgfpathlineto{\pgfqpoint{3.288225in}{1.538063in}}%
\pgfpathlineto{\pgfqpoint{3.290890in}{1.542419in}}%
\pgfpathlineto{\pgfqpoint{3.293574in}{1.548531in}}%
\pgfpathlineto{\pgfqpoint{3.296376in}{1.540799in}}%
\pgfpathlineto{\pgfqpoint{3.298937in}{1.540414in}}%
\pgfpathlineto{\pgfqpoint{3.301719in}{1.546910in}}%
\pgfpathlineto{\pgfqpoint{3.304295in}{1.544565in}}%
\pgfpathlineto{\pgfqpoint{3.307104in}{1.548005in}}%
\pgfpathlineto{\pgfqpoint{3.309652in}{1.546335in}}%
\pgfpathlineto{\pgfqpoint{3.312480in}{1.549623in}}%
\pgfpathlineto{\pgfqpoint{3.315008in}{1.539991in}}%
\pgfpathlineto{\pgfqpoint{3.317688in}{1.552595in}}%
\pgfpathlineto{\pgfqpoint{3.320366in}{1.546556in}}%
\pgfpathlineto{\pgfqpoint{3.323049in}{1.550492in}}%
\pgfpathlineto{\pgfqpoint{3.325860in}{1.542637in}}%
\pgfpathlineto{\pgfqpoint{3.328401in}{1.548236in}}%
\pgfpathlineto{\pgfqpoint{3.331183in}{1.547181in}}%
\pgfpathlineto{\pgfqpoint{3.333758in}{1.549676in}}%
\pgfpathlineto{\pgfqpoint{3.336541in}{1.552147in}}%
\pgfpathlineto{\pgfqpoint{3.339101in}{1.552188in}}%
\pgfpathlineto{\pgfqpoint{3.341893in}{1.552862in}}%
\pgfpathlineto{\pgfqpoint{3.344468in}{1.549913in}}%
\pgfpathlineto{\pgfqpoint{3.347139in}{1.545187in}}%
\pgfpathlineto{\pgfqpoint{3.349828in}{1.541900in}}%
\pgfpathlineto{\pgfqpoint{3.352505in}{1.538154in}}%
\pgfpathlineto{\pgfqpoint{3.355177in}{1.536425in}}%
\pgfpathlineto{\pgfqpoint{3.357862in}{1.544202in}}%
\pgfpathlineto{\pgfqpoint{3.360620in}{1.545509in}}%
\pgfpathlineto{\pgfqpoint{3.363221in}{1.540715in}}%
\pgfpathlineto{\pgfqpoint{3.365996in}{1.545257in}}%
\pgfpathlineto{\pgfqpoint{3.368577in}{1.541056in}}%
\pgfpathlineto{\pgfqpoint{3.371357in}{1.545110in}}%
\pgfpathlineto{\pgfqpoint{3.373921in}{1.543723in}}%
\pgfpathlineto{\pgfqpoint{3.376735in}{1.547495in}}%
\pgfpathlineto{\pgfqpoint{3.379290in}{1.547324in}}%
\pgfpathlineto{\pgfqpoint{3.381959in}{1.546641in}}%
\pgfpathlineto{\pgfqpoint{3.384647in}{1.539898in}}%
\pgfpathlineto{\pgfqpoint{3.387309in}{1.549243in}}%
\pgfpathlineto{\pgfqpoint{3.390102in}{1.545054in}}%
\pgfpathlineto{\pgfqpoint{3.392681in}{1.537192in}}%
\pgfpathlineto{\pgfqpoint{3.395461in}{1.541404in}}%
\pgfpathlineto{\pgfqpoint{3.398037in}{1.546354in}}%
\pgfpathlineto{\pgfqpoint{3.400783in}{1.552098in}}%
\pgfpathlineto{\pgfqpoint{3.403394in}{1.541718in}}%
\pgfpathlineto{\pgfqpoint{3.406202in}{1.541142in}}%
\pgfpathlineto{\pgfqpoint{3.408752in}{1.546969in}}%
\pgfpathlineto{\pgfqpoint{3.411431in}{1.547145in}}%
\pgfpathlineto{\pgfqpoint{3.414109in}{1.545957in}}%
\pgfpathlineto{\pgfqpoint{3.416780in}{1.541407in}}%
\pgfpathlineto{\pgfqpoint{3.419455in}{1.543332in}}%
\pgfpathlineto{\pgfqpoint{3.422142in}{1.550001in}}%
\pgfpathlineto{\pgfqpoint{3.424887in}{1.547279in}}%
\pgfpathlineto{\pgfqpoint{3.427501in}{1.550609in}}%
\pgfpathlineto{\pgfqpoint{3.430313in}{1.547714in}}%
\pgfpathlineto{\pgfqpoint{3.432851in}{1.548712in}}%
\pgfpathlineto{\pgfqpoint{3.435635in}{1.549747in}}%
\pgfpathlineto{\pgfqpoint{3.438210in}{1.546971in}}%
\pgfpathlineto{\pgfqpoint{3.440996in}{1.548189in}}%
\pgfpathlineto{\pgfqpoint{3.443574in}{1.544910in}}%
\pgfpathlineto{\pgfqpoint{3.446257in}{1.541884in}}%
\pgfpathlineto{\pgfqpoint{3.448926in}{1.546816in}}%
\pgfpathlineto{\pgfqpoint{3.451597in}{1.543462in}}%
\pgfpathlineto{\pgfqpoint{3.454285in}{1.542454in}}%
\pgfpathlineto{\pgfqpoint{3.456960in}{1.546900in}}%
\pgfpathlineto{\pgfqpoint{3.459695in}{1.542404in}}%
\pgfpathlineto{\pgfqpoint{3.462321in}{1.549457in}}%
\pgfpathlineto{\pgfqpoint{3.465072in}{1.546527in}}%
\pgfpathlineto{\pgfqpoint{3.467678in}{1.545402in}}%
\pgfpathlineto{\pgfqpoint{3.470466in}{1.544362in}}%
\pgfpathlineto{\pgfqpoint{3.473021in}{1.538695in}}%
\pgfpathlineto{\pgfqpoint{3.475821in}{1.546088in}}%
\pgfpathlineto{\pgfqpoint{3.478378in}{1.539063in}}%
\pgfpathlineto{\pgfqpoint{3.481072in}{1.534541in}}%
\pgfpathlineto{\pgfqpoint{3.483744in}{1.548498in}}%
\pgfpathlineto{\pgfqpoint{3.486442in}{1.549341in}}%
\pgfpathlineto{\pgfqpoint{3.489223in}{1.545190in}}%
\pgfpathlineto{\pgfqpoint{3.491783in}{1.551331in}}%
\pgfpathlineto{\pgfqpoint{3.494581in}{1.542403in}}%
\pgfpathlineto{\pgfqpoint{3.497139in}{1.547026in}}%
\pgfpathlineto{\pgfqpoint{3.499909in}{1.549815in}}%
\pgfpathlineto{\pgfqpoint{3.502488in}{1.547108in}}%
\pgfpathlineto{\pgfqpoint{3.505262in}{1.551904in}}%
\pgfpathlineto{\pgfqpoint{3.507840in}{1.546671in}}%
\pgfpathlineto{\pgfqpoint{3.510533in}{1.552673in}}%
\pgfpathlineto{\pgfqpoint{3.513209in}{1.545300in}}%
\pgfpathlineto{\pgfqpoint{3.515884in}{1.550878in}}%
\pgfpathlineto{\pgfqpoint{3.518565in}{1.545230in}}%
\pgfpathlineto{\pgfqpoint{3.521244in}{1.542930in}}%
\pgfpathlineto{\pgfqpoint{3.524041in}{1.549672in}}%
\pgfpathlineto{\pgfqpoint{3.526601in}{1.545484in}}%
\pgfpathlineto{\pgfqpoint{3.529327in}{1.550000in}}%
\pgfpathlineto{\pgfqpoint{3.531955in}{1.541136in}}%
\pgfpathlineto{\pgfqpoint{3.534783in}{1.547395in}}%
\pgfpathlineto{\pgfqpoint{3.537309in}{1.542205in}}%
\pgfpathlineto{\pgfqpoint{3.540093in}{1.543317in}}%
\pgfpathlineto{\pgfqpoint{3.542656in}{1.537290in}}%
\pgfpathlineto{\pgfqpoint{3.545349in}{1.542755in}}%
\pgfpathlineto{\pgfqpoint{3.548029in}{1.543018in}}%
\pgfpathlineto{\pgfqpoint{3.550713in}{1.539444in}}%
\pgfpathlineto{\pgfqpoint{3.553498in}{1.538069in}}%
\pgfpathlineto{\pgfqpoint{3.556061in}{1.540850in}}%
\pgfpathlineto{\pgfqpoint{3.558853in}{1.548018in}}%
\pgfpathlineto{\pgfqpoint{3.561420in}{1.543355in}}%
\pgfpathlineto{\pgfqpoint{3.564188in}{1.547424in}}%
\pgfpathlineto{\pgfqpoint{3.566774in}{1.546911in}}%
\pgfpathlineto{\pgfqpoint{3.569584in}{1.544563in}}%
\pgfpathlineto{\pgfqpoint{3.572126in}{1.543669in}}%
\pgfpathlineto{\pgfqpoint{3.574814in}{1.537167in}}%
\pgfpathlineto{\pgfqpoint{3.577487in}{1.541845in}}%
\pgfpathlineto{\pgfqpoint{3.580191in}{1.545664in}}%
\pgfpathlineto{\pgfqpoint{3.582851in}{1.541751in}}%
\pgfpathlineto{\pgfqpoint{3.585532in}{1.546864in}}%
\pgfpathlineto{\pgfqpoint{3.588258in}{1.544160in}}%
\pgfpathlineto{\pgfqpoint{3.590883in}{1.546057in}}%
\pgfpathlineto{\pgfqpoint{3.593620in}{1.545456in}}%
\pgfpathlineto{\pgfqpoint{3.596240in}{1.545675in}}%
\pgfpathlineto{\pgfqpoint{3.598998in}{1.544977in}}%
\pgfpathlineto{\pgfqpoint{3.601590in}{1.543431in}}%
\pgfpathlineto{\pgfqpoint{3.604387in}{1.543556in}}%
\pgfpathlineto{\pgfqpoint{3.606951in}{1.534788in}}%
\pgfpathlineto{\pgfqpoint{3.609632in}{1.541238in}}%
\pgfpathlineto{\pgfqpoint{3.612311in}{1.538971in}}%
\pgfpathlineto{\pgfqpoint{3.614982in}{1.544057in}}%
\pgfpathlineto{\pgfqpoint{3.617667in}{1.545911in}}%
\pgfpathlineto{\pgfqpoint{3.620345in}{1.547791in}}%
\pgfpathlineto{\pgfqpoint{3.623165in}{1.546503in}}%
\pgfpathlineto{\pgfqpoint{3.625689in}{1.545999in}}%
\pgfpathlineto{\pgfqpoint{3.628460in}{1.542364in}}%
\pgfpathlineto{\pgfqpoint{3.631058in}{1.543177in}}%
\pgfpathlineto{\pgfqpoint{3.633858in}{1.539137in}}%
\pgfpathlineto{\pgfqpoint{3.636413in}{1.544073in}}%
\pgfpathlineto{\pgfqpoint{3.639207in}{1.544762in}}%
\pgfpathlineto{\pgfqpoint{3.641773in}{1.542416in}}%
\pgfpathlineto{\pgfqpoint{3.644452in}{1.531700in}}%
\pgfpathlineto{\pgfqpoint{3.647130in}{1.535031in}}%
\pgfpathlineto{\pgfqpoint{3.649837in}{1.538697in}}%
\pgfpathlineto{\pgfqpoint{3.652628in}{1.534417in}}%
\pgfpathlineto{\pgfqpoint{3.655165in}{1.535855in}}%
\pgfpathlineto{\pgfqpoint{3.657917in}{1.553252in}}%
\pgfpathlineto{\pgfqpoint{3.660515in}{1.568606in}}%
\pgfpathlineto{\pgfqpoint{3.663276in}{1.571056in}}%
\pgfpathlineto{\pgfqpoint{3.665864in}{1.570339in}}%
\pgfpathlineto{\pgfqpoint{3.668665in}{1.573450in}}%
\pgfpathlineto{\pgfqpoint{3.671232in}{1.560153in}}%
\pgfpathlineto{\pgfqpoint{3.673911in}{1.538793in}}%
\pgfpathlineto{\pgfqpoint{3.676591in}{1.548482in}}%
\pgfpathlineto{\pgfqpoint{3.679273in}{1.542639in}}%
\pgfpathlineto{\pgfqpoint{3.681948in}{1.542933in}}%
\pgfpathlineto{\pgfqpoint{3.684620in}{1.544594in}}%
\pgfpathlineto{\pgfqpoint{3.687442in}{1.541530in}}%
\pgfpathlineto{\pgfqpoint{3.689983in}{1.544954in}}%
\pgfpathlineto{\pgfqpoint{3.692765in}{1.535036in}}%
\pgfpathlineto{\pgfqpoint{3.695331in}{1.542871in}}%
\pgfpathlineto{\pgfqpoint{3.698125in}{1.541245in}}%
\pgfpathlineto{\pgfqpoint{3.700684in}{1.541610in}}%
\pgfpathlineto{\pgfqpoint{3.703460in}{1.547199in}}%
\pgfpathlineto{\pgfqpoint{3.706053in}{1.550921in}}%
\pgfpathlineto{\pgfqpoint{3.708729in}{1.555027in}}%
\pgfpathlineto{\pgfqpoint{3.711410in}{1.547413in}}%
\pgfpathlineto{\pgfqpoint{3.714086in}{1.550833in}}%
\pgfpathlineto{\pgfqpoint{3.716875in}{1.541399in}}%
\pgfpathlineto{\pgfqpoint{3.719446in}{1.547739in}}%
\pgfpathlineto{\pgfqpoint{3.722228in}{1.554884in}}%
\pgfpathlineto{\pgfqpoint{3.724804in}{1.562056in}}%
\pgfpathlineto{\pgfqpoint{3.727581in}{1.556041in}}%
\pgfpathlineto{\pgfqpoint{3.730158in}{1.555323in}}%
\pgfpathlineto{\pgfqpoint{3.732950in}{1.554012in}}%
\pgfpathlineto{\pgfqpoint{3.735509in}{1.548825in}}%
\pgfpathlineto{\pgfqpoint{3.738194in}{1.551605in}}%
\pgfpathlineto{\pgfqpoint{3.740874in}{1.552197in}}%
\pgfpathlineto{\pgfqpoint{3.743548in}{1.550828in}}%
\pgfpathlineto{\pgfqpoint{3.746229in}{1.547375in}}%
\pgfpathlineto{\pgfqpoint{3.748903in}{1.552255in}}%
\pgfpathlineto{\pgfqpoint{3.751728in}{1.552930in}}%
\pgfpathlineto{\pgfqpoint{3.754265in}{1.554402in}}%
\pgfpathlineto{\pgfqpoint{3.757065in}{1.545045in}}%
\pgfpathlineto{\pgfqpoint{3.759622in}{1.547753in}}%
\pgfpathlineto{\pgfqpoint{3.762389in}{1.544657in}}%
\pgfpathlineto{\pgfqpoint{3.764966in}{1.555283in}}%
\pgfpathlineto{\pgfqpoint{3.767782in}{1.547340in}}%
\pgfpathlineto{\pgfqpoint{3.770323in}{1.550380in}}%
\pgfpathlineto{\pgfqpoint{3.773014in}{1.544230in}}%
\pgfpathlineto{\pgfqpoint{3.775691in}{1.551910in}}%
\pgfpathlineto{\pgfqpoint{3.778370in}{1.549234in}}%
\pgfpathlineto{\pgfqpoint{3.781046in}{1.546637in}}%
\pgfpathlineto{\pgfqpoint{3.783725in}{1.546013in}}%
\pgfpathlineto{\pgfqpoint{3.786504in}{1.536527in}}%
\pgfpathlineto{\pgfqpoint{3.789084in}{1.539835in}}%
\pgfpathlineto{\pgfqpoint{3.791897in}{1.542785in}}%
\pgfpathlineto{\pgfqpoint{3.794435in}{1.537497in}}%
\pgfpathlineto{\pgfqpoint{3.797265in}{1.540175in}}%
\pgfpathlineto{\pgfqpoint{3.799797in}{1.533077in}}%
\pgfpathlineto{\pgfqpoint{3.802569in}{1.539259in}}%
\pgfpathlineto{\pgfqpoint{3.805145in}{1.539738in}}%
\pgfpathlineto{\pgfqpoint{3.807832in}{1.539273in}}%
\pgfpathlineto{\pgfqpoint{3.810510in}{1.538162in}}%
\pgfpathlineto{\pgfqpoint{3.813172in}{1.535140in}}%
\pgfpathlineto{\pgfqpoint{3.815983in}{1.533989in}}%
\pgfpathlineto{\pgfqpoint{3.818546in}{1.536862in}}%
\pgfpathlineto{\pgfqpoint{3.821315in}{1.541249in}}%
\pgfpathlineto{\pgfqpoint{3.823903in}{1.541672in}}%
\pgfpathlineto{\pgfqpoint{3.826679in}{1.540399in}}%
\pgfpathlineto{\pgfqpoint{3.829252in}{1.542599in}}%
\pgfpathlineto{\pgfqpoint{3.832053in}{1.549352in}}%
\pgfpathlineto{\pgfqpoint{3.834616in}{1.546721in}}%
\pgfpathlineto{\pgfqpoint{3.837286in}{1.547808in}}%
\pgfpathlineto{\pgfqpoint{3.839960in}{1.546837in}}%
\pgfpathlineto{\pgfqpoint{3.842641in}{1.546258in}}%
\pgfpathlineto{\pgfqpoint{3.845329in}{1.548366in}}%
\pgfpathlineto{\pgfqpoint{3.848005in}{1.542439in}}%
\pgfpathlineto{\pgfqpoint{3.850814in}{1.532764in}}%
\pgfpathlineto{\pgfqpoint{3.853358in}{1.540267in}}%
\pgfpathlineto{\pgfqpoint{3.856100in}{1.536310in}}%
\pgfpathlineto{\pgfqpoint{3.858720in}{1.540141in}}%
\pgfpathlineto{\pgfqpoint{3.861561in}{1.543034in}}%
\pgfpathlineto{\pgfqpoint{3.864073in}{1.544304in}}%
\pgfpathlineto{\pgfqpoint{3.866815in}{1.546203in}}%
\pgfpathlineto{\pgfqpoint{3.869435in}{1.543194in}}%
\pgfpathlineto{\pgfqpoint{3.872114in}{1.542038in}}%
\pgfpathlineto{\pgfqpoint{3.874790in}{1.547178in}}%
\pgfpathlineto{\pgfqpoint{3.877466in}{1.547105in}}%
\pgfpathlineto{\pgfqpoint{3.880237in}{1.547041in}}%
\pgfpathlineto{\pgfqpoint{3.882850in}{1.545391in}}%
\pgfpathlineto{\pgfqpoint{3.885621in}{1.540610in}}%
\pgfpathlineto{\pgfqpoint{3.888188in}{1.533242in}}%
\pgfpathlineto{\pgfqpoint{3.890926in}{1.533112in}}%
\pgfpathlineto{\pgfqpoint{3.893541in}{1.524680in}}%
\pgfpathlineto{\pgfqpoint{3.896345in}{1.524680in}}%
\pgfpathlineto{\pgfqpoint{3.898891in}{1.528166in}}%
\pgfpathlineto{\pgfqpoint{3.901573in}{1.533648in}}%
\pgfpathlineto{\pgfqpoint{3.904252in}{1.534725in}}%
\pgfpathlineto{\pgfqpoint{3.906918in}{1.535061in}}%
\pgfpathlineto{\pgfqpoint{3.909602in}{1.538249in}}%
\pgfpathlineto{\pgfqpoint{3.912296in}{1.540059in}}%
\pgfpathlineto{\pgfqpoint{3.915107in}{1.542001in}}%
\pgfpathlineto{\pgfqpoint{3.917646in}{1.544685in}}%
\pgfpathlineto{\pgfqpoint{3.920412in}{1.537234in}}%
\pgfpathlineto{\pgfqpoint{3.923005in}{1.549542in}}%
\pgfpathlineto{\pgfqpoint{3.925778in}{1.546615in}}%
\pgfpathlineto{\pgfqpoint{3.928347in}{1.543558in}}%
\pgfpathlineto{\pgfqpoint{3.931202in}{1.550180in}}%
\pgfpathlineto{\pgfqpoint{3.933714in}{1.554657in}}%
\pgfpathlineto{\pgfqpoint{3.936395in}{1.552451in}}%
\pgfpathlineto{\pgfqpoint{3.939075in}{1.550268in}}%
\pgfpathlineto{\pgfqpoint{3.941778in}{1.534751in}}%
\pgfpathlineto{\pgfqpoint{3.944431in}{1.540074in}}%
\pgfpathlineto{\pgfqpoint{3.947101in}{1.531265in}}%
\pgfpathlineto{\pgfqpoint{3.949894in}{1.532895in}}%
\pgfpathlineto{\pgfqpoint{3.952464in}{1.535292in}}%
\pgfpathlineto{\pgfqpoint{3.955211in}{1.539864in}}%
\pgfpathlineto{\pgfqpoint{3.957823in}{1.538458in}}%
\pgfpathlineto{\pgfqpoint{3.960635in}{1.539812in}}%
\pgfpathlineto{\pgfqpoint{3.963176in}{1.545421in}}%
\pgfpathlineto{\pgfqpoint{3.966013in}{1.546825in}}%
\pgfpathlineto{\pgfqpoint{3.968523in}{1.547348in}}%
\pgfpathlineto{\pgfqpoint{3.971250in}{1.547308in}}%
\pgfpathlineto{\pgfqpoint{3.973885in}{1.539431in}}%
\pgfpathlineto{\pgfqpoint{3.976563in}{1.539204in}}%
\pgfpathlineto{\pgfqpoint{3.979389in}{1.541235in}}%
\pgfpathlineto{\pgfqpoint{3.981929in}{1.541046in}}%
\pgfpathlineto{\pgfqpoint{3.984714in}{1.542532in}}%
\pgfpathlineto{\pgfqpoint{3.987270in}{1.538946in}}%
\pgfpathlineto{\pgfqpoint{3.990055in}{1.538326in}}%
\pgfpathlineto{\pgfqpoint{3.992642in}{1.537627in}}%
\pgfpathlineto{\pgfqpoint{3.995417in}{1.534586in}}%
\pgfpathlineto{\pgfqpoint{3.997990in}{1.537770in}}%
\pgfpathlineto{\pgfqpoint{4.000674in}{1.535516in}}%
\pgfpathlineto{\pgfqpoint{4.003348in}{1.540376in}}%
\pgfpathlineto{\pgfqpoint{4.006034in}{1.528441in}}%
\pgfpathlineto{\pgfqpoint{4.008699in}{1.534047in}}%
\pgfpathlineto{\pgfqpoint{4.011394in}{1.536039in}}%
\pgfpathlineto{\pgfqpoint{4.014186in}{1.544375in}}%
\pgfpathlineto{\pgfqpoint{4.016744in}{1.533495in}}%
\pgfpathlineto{\pgfqpoint{4.019518in}{1.537961in}}%
\pgfpathlineto{\pgfqpoint{4.022097in}{1.540182in}}%
\pgfpathlineto{\pgfqpoint{4.024868in}{1.537888in}}%
\pgfpathlineto{\pgfqpoint{4.027447in}{1.541044in}}%
\pgfpathlineto{\pgfqpoint{4.030229in}{1.538240in}}%
\pgfpathlineto{\pgfqpoint{4.032817in}{1.546872in}}%
\pgfpathlineto{\pgfqpoint{4.035492in}{1.551363in}}%
\pgfpathlineto{\pgfqpoint{4.038174in}{1.543898in}}%
\pgfpathlineto{\pgfqpoint{4.040852in}{1.548363in}}%
\pgfpathlineto{\pgfqpoint{4.043667in}{1.538685in}}%
\pgfpathlineto{\pgfqpoint{4.046210in}{1.535432in}}%
\pgfpathlineto{\pgfqpoint{4.049006in}{1.535271in}}%
\pgfpathlineto{\pgfqpoint{4.051557in}{1.541397in}}%
\pgfpathlineto{\pgfqpoint{4.054326in}{1.536202in}}%
\pgfpathlineto{\pgfqpoint{4.056911in}{1.539845in}}%
\pgfpathlineto{\pgfqpoint{4.059702in}{1.544917in}}%
\pgfpathlineto{\pgfqpoint{4.062266in}{1.540437in}}%
\pgfpathlineto{\pgfqpoint{4.064957in}{1.546216in}}%
\pgfpathlineto{\pgfqpoint{4.067636in}{1.545703in}}%
\pgfpathlineto{\pgfqpoint{4.070313in}{1.549068in}}%
\pgfpathlineto{\pgfqpoint{4.072985in}{1.538828in}}%
\pgfpathlineto{\pgfqpoint{4.075705in}{1.547925in}}%
\pgfpathlineto{\pgfqpoint{4.078471in}{1.545099in}}%
\pgfpathlineto{\pgfqpoint{4.081018in}{1.542631in}}%
\pgfpathlineto{\pgfqpoint{4.083870in}{1.543998in}}%
\pgfpathlineto{\pgfqpoint{4.086385in}{1.544727in}}%
\pgfpathlineto{\pgfqpoint{4.089159in}{1.542982in}}%
\pgfpathlineto{\pgfqpoint{4.091729in}{1.549719in}}%
\pgfpathlineto{\pgfqpoint{4.094527in}{1.551848in}}%
\pgfpathlineto{\pgfqpoint{4.097092in}{1.543030in}}%
\pgfpathlineto{\pgfqpoint{4.099777in}{1.543138in}}%
\pgfpathlineto{\pgfqpoint{4.102456in}{1.549666in}}%
\pgfpathlineto{\pgfqpoint{4.105185in}{1.544691in}}%
\pgfpathlineto{\pgfqpoint{4.107814in}{1.548653in}}%
\pgfpathlineto{\pgfqpoint{4.110488in}{1.548196in}}%
\pgfpathlineto{\pgfqpoint{4.113252in}{1.541651in}}%
\pgfpathlineto{\pgfqpoint{4.115844in}{1.551668in}}%
\pgfpathlineto{\pgfqpoint{4.118554in}{1.554270in}}%
\pgfpathlineto{\pgfqpoint{4.121205in}{1.547370in}}%
\pgfpathlineto{\pgfqpoint{4.124019in}{1.550137in}}%
\pgfpathlineto{\pgfqpoint{4.126553in}{1.547921in}}%
\pgfpathlineto{\pgfqpoint{4.129349in}{1.550975in}}%
\pgfpathlineto{\pgfqpoint{4.131920in}{1.549889in}}%
\pgfpathlineto{\pgfqpoint{4.134615in}{1.547081in}}%
\pgfpathlineto{\pgfqpoint{4.137272in}{1.549431in}}%
\pgfpathlineto{\pgfqpoint{4.139963in}{1.548259in}}%
\pgfpathlineto{\pgfqpoint{4.142713in}{1.548838in}}%
\pgfpathlineto{\pgfqpoint{4.145310in}{1.547990in}}%
\pgfpathlineto{\pgfqpoint{4.148082in}{1.538095in}}%
\pgfpathlineto{\pgfqpoint{4.150665in}{1.543092in}}%
\pgfpathlineto{\pgfqpoint{4.153423in}{1.539201in}}%
\pgfpathlineto{\pgfqpoint{4.156016in}{1.542652in}}%
\pgfpathlineto{\pgfqpoint{4.158806in}{1.540510in}}%
\pgfpathlineto{\pgfqpoint{4.161380in}{1.544821in}}%
\pgfpathlineto{\pgfqpoint{4.164059in}{1.547099in}}%
\pgfpathlineto{\pgfqpoint{4.166737in}{1.545546in}}%
\pgfpathlineto{\pgfqpoint{4.169415in}{1.536941in}}%
\pgfpathlineto{\pgfqpoint{4.172093in}{1.540908in}}%
\pgfpathlineto{\pgfqpoint{4.174770in}{1.541791in}}%
\pgfpathlineto{\pgfqpoint{4.177593in}{1.532099in}}%
\pgfpathlineto{\pgfqpoint{4.180129in}{1.546568in}}%
\pgfpathlineto{\pgfqpoint{4.182899in}{1.556189in}}%
\pgfpathlineto{\pgfqpoint{4.185481in}{1.564344in}}%
\pgfpathlineto{\pgfqpoint{4.188318in}{1.564715in}}%
\pgfpathlineto{\pgfqpoint{4.190842in}{1.548649in}}%
\pgfpathlineto{\pgfqpoint{4.193638in}{1.549031in}}%
\pgfpathlineto{\pgfqpoint{4.196186in}{1.551112in}}%
\pgfpathlineto{\pgfqpoint{4.198878in}{1.556240in}}%
\pgfpathlineto{\pgfqpoint{4.201542in}{1.552391in}}%
\pgfpathlineto{\pgfqpoint{4.204240in}{1.549469in}}%
\pgfpathlineto{\pgfqpoint{4.207076in}{1.548340in}}%
\pgfpathlineto{\pgfqpoint{4.209597in}{1.547471in}}%
\pgfpathlineto{\pgfqpoint{4.212383in}{1.538254in}}%
\pgfpathlineto{\pgfqpoint{4.214948in}{1.546957in}}%
\pgfpathlineto{\pgfqpoint{4.217694in}{1.542861in}}%
\pgfpathlineto{\pgfqpoint{4.220304in}{1.548152in}}%
\pgfpathlineto{\pgfqpoint{4.223082in}{1.552821in}}%
\pgfpathlineto{\pgfqpoint{4.225654in}{1.549026in}}%
\pgfpathlineto{\pgfqpoint{4.228331in}{1.552998in}}%
\pgfpathlineto{\pgfqpoint{4.231013in}{1.550802in}}%
\pgfpathlineto{\pgfqpoint{4.233691in}{1.547589in}}%
\pgfpathlineto{\pgfqpoint{4.236375in}{1.546655in}}%
\pgfpathlineto{\pgfqpoint{4.239084in}{1.549815in}}%
\pgfpathlineto{\pgfqpoint{4.241900in}{1.543469in}}%
\pgfpathlineto{\pgfqpoint{4.244394in}{1.539093in}}%
\pgfpathlineto{\pgfqpoint{4.247225in}{1.541481in}}%
\pgfpathlineto{\pgfqpoint{4.249767in}{1.549288in}}%
\pgfpathlineto{\pgfqpoint{4.252581in}{1.552068in}}%
\pgfpathlineto{\pgfqpoint{4.255120in}{1.552018in}}%
\pgfpathlineto{\pgfqpoint{4.257958in}{1.552369in}}%
\pgfpathlineto{\pgfqpoint{4.260477in}{1.549187in}}%
\pgfpathlineto{\pgfqpoint{4.263157in}{1.549498in}}%
\pgfpathlineto{\pgfqpoint{4.265824in}{1.544692in}}%
\pgfpathlineto{\pgfqpoint{4.268590in}{1.544062in}}%
\pgfpathlineto{\pgfqpoint{4.271187in}{1.542169in}}%
\pgfpathlineto{\pgfqpoint{4.273874in}{1.547824in}}%
\pgfpathlineto{\pgfqpoint{4.276635in}{1.543813in}}%
\pgfpathlineto{\pgfqpoint{4.279212in}{1.538230in}}%
\pgfpathlineto{\pgfqpoint{4.282000in}{1.537828in}}%
\pgfpathlineto{\pgfqpoint{4.284586in}{1.540852in}}%
\pgfpathlineto{\pgfqpoint{4.287399in}{1.533058in}}%
\pgfpathlineto{\pgfqpoint{4.289936in}{1.540167in}}%
\pgfpathlineto{\pgfqpoint{4.292786in}{1.544008in}}%
\pgfpathlineto{\pgfqpoint{4.295299in}{1.537155in}}%
\pgfpathlineto{\pgfqpoint{4.297977in}{1.544548in}}%
\pgfpathlineto{\pgfqpoint{4.300656in}{1.551648in}}%
\pgfpathlineto{\pgfqpoint{4.303357in}{1.546690in}}%
\pgfpathlineto{\pgfqpoint{4.306118in}{1.543728in}}%
\pgfpathlineto{\pgfqpoint{4.308691in}{1.545515in}}%
\pgfpathlineto{\pgfqpoint{4.311494in}{1.547633in}}%
\pgfpathlineto{\pgfqpoint{4.314032in}{1.549770in}}%
\pgfpathlineto{\pgfqpoint{4.316856in}{1.547645in}}%
\pgfpathlineto{\pgfqpoint{4.319405in}{1.546467in}}%
\pgfpathlineto{\pgfqpoint{4.322181in}{1.549076in}}%
\pgfpathlineto{\pgfqpoint{4.324760in}{1.547626in}}%
\pgfpathlineto{\pgfqpoint{4.327440in}{1.552202in}}%
\pgfpathlineto{\pgfqpoint{4.330118in}{1.547890in}}%
\pgfpathlineto{\pgfqpoint{4.332796in}{1.537610in}}%
\pgfpathlineto{\pgfqpoint{4.335463in}{1.542292in}}%
\pgfpathlineto{\pgfqpoint{4.338154in}{1.540027in}}%
\pgfpathlineto{\pgfqpoint{4.340976in}{1.532923in}}%
\pgfpathlineto{\pgfqpoint{4.343510in}{1.543909in}}%
\pgfpathlineto{\pgfqpoint{4.346263in}{1.537778in}}%
\pgfpathlineto{\pgfqpoint{4.348868in}{1.539427in}}%
\pgfpathlineto{\pgfqpoint{4.351645in}{1.541361in}}%
\pgfpathlineto{\pgfqpoint{4.354224in}{1.539115in}}%
\pgfpathlineto{\pgfqpoint{4.357014in}{1.538330in}}%
\pgfpathlineto{\pgfqpoint{4.359582in}{1.538292in}}%
\pgfpathlineto{\pgfqpoint{4.362270in}{1.544515in}}%
\pgfpathlineto{\pgfqpoint{4.364936in}{1.539210in}}%
\pgfpathlineto{\pgfqpoint{4.367646in}{1.540676in}}%
\pgfpathlineto{\pgfqpoint{4.370437in}{1.540786in}}%
\pgfpathlineto{\pgfqpoint{4.372976in}{1.541737in}}%
\pgfpathlineto{\pgfqpoint{4.375761in}{1.538422in}}%
\pgfpathlineto{\pgfqpoint{4.378329in}{1.542140in}}%
\pgfpathlineto{\pgfqpoint{4.381097in}{1.536576in}}%
\pgfpathlineto{\pgfqpoint{4.383674in}{1.538909in}}%
\pgfpathlineto{\pgfqpoint{4.386431in}{1.535532in}}%
\pgfpathlineto{\pgfqpoint{4.389044in}{1.539325in}}%
\pgfpathlineto{\pgfqpoint{4.391721in}{1.541037in}}%
\pgfpathlineto{\pgfqpoint{4.394400in}{1.535353in}}%
\pgfpathlineto{\pgfqpoint{4.397076in}{1.535941in}}%
\pgfpathlineto{\pgfqpoint{4.399745in}{1.537651in}}%
\pgfpathlineto{\pgfqpoint{4.402468in}{1.536100in}}%
\pgfpathlineto{\pgfqpoint{4.405234in}{1.538887in}}%
\pgfpathlineto{\pgfqpoint{4.407788in}{1.545102in}}%
\pgfpathlineto{\pgfqpoint{4.410587in}{1.541658in}}%
\pgfpathlineto{\pgfqpoint{4.413149in}{1.538271in}}%
\pgfpathlineto{\pgfqpoint{4.415932in}{1.546327in}}%
\pgfpathlineto{\pgfqpoint{4.418506in}{1.552499in}}%
\pgfpathlineto{\pgfqpoint{4.421292in}{1.547053in}}%
\pgfpathlineto{\pgfqpoint{4.423863in}{1.549898in}}%
\pgfpathlineto{\pgfqpoint{4.426534in}{1.549869in}}%
\pgfpathlineto{\pgfqpoint{4.429220in}{1.541034in}}%
\pgfpathlineto{\pgfqpoint{4.431901in}{1.546548in}}%
\pgfpathlineto{\pgfqpoint{4.434569in}{1.546095in}}%
\pgfpathlineto{\pgfqpoint{4.437253in}{1.550553in}}%
\pgfpathlineto{\pgfqpoint{4.440041in}{1.551769in}}%
\pgfpathlineto{\pgfqpoint{4.442611in}{1.544314in}}%
\pgfpathlineto{\pgfqpoint{4.445423in}{1.536951in}}%
\pgfpathlineto{\pgfqpoint{4.447965in}{1.533934in}}%
\pgfpathlineto{\pgfqpoint{4.450767in}{1.539685in}}%
\pgfpathlineto{\pgfqpoint{4.453312in}{1.539445in}}%
\pgfpathlineto{\pgfqpoint{4.456138in}{1.545293in}}%
\pgfpathlineto{\pgfqpoint{4.458681in}{1.557132in}}%
\pgfpathlineto{\pgfqpoint{4.461367in}{1.555002in}}%
\pgfpathlineto{\pgfqpoint{4.464029in}{1.554041in}}%
\pgfpathlineto{\pgfqpoint{4.466717in}{1.550757in}}%
\pgfpathlineto{\pgfqpoint{4.469492in}{1.543783in}}%
\pgfpathlineto{\pgfqpoint{4.472059in}{1.547303in}}%
\pgfpathlineto{\pgfqpoint{4.474861in}{1.547856in}}%
\pgfpathlineto{\pgfqpoint{4.477430in}{1.554372in}}%
\pgfpathlineto{\pgfqpoint{4.480201in}{1.544666in}}%
\pgfpathlineto{\pgfqpoint{4.482778in}{1.550072in}}%
\pgfpathlineto{\pgfqpoint{4.485581in}{1.545129in}}%
\pgfpathlineto{\pgfqpoint{4.488130in}{1.550541in}}%
\pgfpathlineto{\pgfqpoint{4.490822in}{1.546901in}}%
\pgfpathlineto{\pgfqpoint{4.493492in}{1.545280in}}%
\pgfpathlineto{\pgfqpoint{4.496167in}{1.548736in}}%
\pgfpathlineto{\pgfqpoint{4.498850in}{1.535400in}}%
\pgfpathlineto{\pgfqpoint{4.501529in}{1.545424in}}%
\pgfpathlineto{\pgfqpoint{4.504305in}{1.548264in}}%
\pgfpathlineto{\pgfqpoint{4.506893in}{1.547086in}}%
\pgfpathlineto{\pgfqpoint{4.509643in}{1.551411in}}%
\pgfpathlineto{\pgfqpoint{4.512246in}{1.548285in}}%
\pgfpathlineto{\pgfqpoint{4.515080in}{1.547597in}}%
\pgfpathlineto{\pgfqpoint{4.517598in}{1.544308in}}%
\pgfpathlineto{\pgfqpoint{4.520345in}{1.548509in}}%
\pgfpathlineto{\pgfqpoint{4.522962in}{1.543292in}}%
\pgfpathlineto{\pgfqpoint{4.525640in}{1.553817in}}%
\pgfpathlineto{\pgfqpoint{4.528307in}{1.549580in}}%
\pgfpathlineto{\pgfqpoint{4.530990in}{1.544943in}}%
\pgfpathlineto{\pgfqpoint{4.533764in}{1.548532in}}%
\pgfpathlineto{\pgfqpoint{4.536400in}{1.555429in}}%
\pgfpathlineto{\pgfqpoint{4.539144in}{1.547646in}}%
\pgfpathlineto{\pgfqpoint{4.541711in}{1.539597in}}%
\pgfpathlineto{\pgfqpoint{4.544464in}{1.543176in}}%
\pgfpathlineto{\pgfqpoint{4.547064in}{1.543343in}}%
\pgfpathlineto{\pgfqpoint{4.549822in}{1.550248in}}%
\pgfpathlineto{\pgfqpoint{4.552425in}{1.543186in}}%
\pgfpathlineto{\pgfqpoint{4.555106in}{1.544060in}}%
\pgfpathlineto{\pgfqpoint{4.557777in}{1.543045in}}%
\pgfpathlineto{\pgfqpoint{4.560448in}{1.542723in}}%
\pgfpathlineto{\pgfqpoint{4.563125in}{1.544170in}}%
\pgfpathlineto{\pgfqpoint{4.565820in}{1.540967in}}%
\pgfpathlineto{\pgfqpoint{4.568612in}{1.544862in}}%
\pgfpathlineto{\pgfqpoint{4.571171in}{1.538042in}}%
\pgfpathlineto{\pgfqpoint{4.573947in}{1.543891in}}%
\pgfpathlineto{\pgfqpoint{4.576531in}{1.542773in}}%
\pgfpathlineto{\pgfqpoint{4.579305in}{1.539321in}}%
\pgfpathlineto{\pgfqpoint{4.581888in}{1.542020in}}%
\pgfpathlineto{\pgfqpoint{4.584672in}{1.539854in}}%
\pgfpathlineto{\pgfqpoint{4.587244in}{1.537889in}}%
\pgfpathlineto{\pgfqpoint{4.589920in}{1.541125in}}%
\pgfpathlineto{\pgfqpoint{4.592589in}{1.537022in}}%
\pgfpathlineto{\pgfqpoint{4.595281in}{1.535000in}}%
\pgfpathlineto{\pgfqpoint{4.597951in}{1.538961in}}%
\pgfpathlineto{\pgfqpoint{4.600633in}{1.539371in}}%
\pgfpathlineto{\pgfqpoint{4.603430in}{1.541414in}}%
\pgfpathlineto{\pgfqpoint{4.605990in}{1.543807in}}%
\pgfpathlineto{\pgfqpoint{4.608808in}{1.538154in}}%
\pgfpathlineto{\pgfqpoint{4.611350in}{1.542252in}}%
\pgfpathlineto{\pgfqpoint{4.614134in}{1.550120in}}%
\pgfpathlineto{\pgfqpoint{4.616702in}{1.545914in}}%
\pgfpathlineto{\pgfqpoint{4.619529in}{1.541286in}}%
\pgfpathlineto{\pgfqpoint{4.622056in}{1.547067in}}%
\pgfpathlineto{\pgfqpoint{4.624741in}{1.541924in}}%
\pgfpathlineto{\pgfqpoint{4.627411in}{1.541948in}}%
\pgfpathlineto{\pgfqpoint{4.630096in}{1.537468in}}%
\pgfpathlineto{\pgfqpoint{4.632902in}{1.543671in}}%
\pgfpathlineto{\pgfqpoint{4.635445in}{1.542338in}}%
\pgfpathlineto{\pgfqpoint{4.638204in}{1.537451in}}%
\pgfpathlineto{\pgfqpoint{4.640809in}{1.542335in}}%
\pgfpathlineto{\pgfqpoint{4.643628in}{1.542534in}}%
\pgfpathlineto{\pgfqpoint{4.646169in}{1.540627in}}%
\pgfpathlineto{\pgfqpoint{4.648922in}{1.541786in}}%
\pgfpathlineto{\pgfqpoint{4.651524in}{1.542640in}}%
\pgfpathlineto{\pgfqpoint{4.654203in}{1.545627in}}%
\pgfpathlineto{\pgfqpoint{4.656873in}{1.548024in}}%
\pgfpathlineto{\pgfqpoint{4.659590in}{1.545365in}}%
\pgfpathlineto{\pgfqpoint{4.662237in}{1.539758in}}%
\pgfpathlineto{\pgfqpoint{4.664923in}{1.546630in}}%
\pgfpathlineto{\pgfqpoint{4.667764in}{1.545155in}}%
\pgfpathlineto{\pgfqpoint{4.670261in}{1.543028in}}%
\pgfpathlineto{\pgfqpoint{4.673068in}{1.546746in}}%
\pgfpathlineto{\pgfqpoint{4.675619in}{1.544330in}}%
\pgfpathlineto{\pgfqpoint{4.678448in}{1.542748in}}%
\pgfpathlineto{\pgfqpoint{4.680988in}{1.551543in}}%
\pgfpathlineto{\pgfqpoint{4.683799in}{1.541257in}}%
\pgfpathlineto{\pgfqpoint{4.686337in}{1.541690in}}%
\pgfpathlineto{\pgfqpoint{4.689051in}{1.540102in}}%
\pgfpathlineto{\pgfqpoint{4.691694in}{1.541185in}}%
\pgfpathlineto{\pgfqpoint{4.694381in}{1.545113in}}%
\pgfpathlineto{\pgfqpoint{4.697170in}{1.536290in}}%
\pgfpathlineto{\pgfqpoint{4.699734in}{1.538561in}}%
\pgfpathlineto{\pgfqpoint{4.702517in}{1.546722in}}%
\pgfpathlineto{\pgfqpoint{4.705094in}{1.541974in}}%
\pgfpathlineto{\pgfqpoint{4.707824in}{1.539865in}}%
\pgfpathlineto{\pgfqpoint{4.710437in}{1.533811in}}%
\pgfpathlineto{\pgfqpoint{4.713275in}{1.538186in}}%
\pgfpathlineto{\pgfqpoint{4.715806in}{1.532496in}}%
\pgfpathlineto{\pgfqpoint{4.718486in}{1.541883in}}%
\pgfpathlineto{\pgfqpoint{4.721160in}{1.543824in}}%
\pgfpathlineto{\pgfqpoint{4.723873in}{1.535129in}}%
\pgfpathlineto{\pgfqpoint{4.726508in}{1.543108in}}%
\pgfpathlineto{\pgfqpoint{4.729233in}{1.539629in}}%
\pgfpathlineto{\pgfqpoint{4.731901in}{1.540063in}}%
\pgfpathlineto{\pgfqpoint{4.734552in}{1.542250in}}%
\pgfpathlineto{\pgfqpoint{4.737348in}{1.548983in}}%
\pgfpathlineto{\pgfqpoint{4.739912in}{1.536345in}}%
\pgfpathlineto{\pgfqpoint{4.742696in}{1.545744in}}%
\pgfpathlineto{\pgfqpoint{4.745256in}{1.543133in}}%
\pgfpathlineto{\pgfqpoint{4.748081in}{1.547118in}}%
\pgfpathlineto{\pgfqpoint{4.750627in}{1.549314in}}%
\pgfpathlineto{\pgfqpoint{4.753298in}{1.546091in}}%
\pgfpathlineto{\pgfqpoint{4.755983in}{1.548329in}}%
\pgfpathlineto{\pgfqpoint{4.758653in}{1.550287in}}%
\pgfpathlineto{\pgfqpoint{4.761337in}{1.546952in}}%
\pgfpathlineto{\pgfqpoint{4.764018in}{1.545948in}}%
\pgfpathlineto{\pgfqpoint{4.766783in}{1.545544in}}%
\pgfpathlineto{\pgfqpoint{4.769367in}{1.538768in}}%
\pgfpathlineto{\pgfqpoint{4.772198in}{1.538523in}}%
\pgfpathlineto{\pgfqpoint{4.774732in}{1.537094in}}%
\pgfpathlineto{\pgfqpoint{4.777535in}{1.542773in}}%
\pgfpathlineto{\pgfqpoint{4.780083in}{1.540754in}}%
\pgfpathlineto{\pgfqpoint{4.782872in}{1.539688in}}%
\pgfpathlineto{\pgfqpoint{4.785445in}{1.546648in}}%
\pgfpathlineto{\pgfqpoint{4.788116in}{1.546144in}}%
\pgfpathlineto{\pgfqpoint{4.790798in}{1.545074in}}%
\pgfpathlineto{\pgfqpoint{4.793512in}{1.548698in}}%
\pgfpathlineto{\pgfqpoint{4.796274in}{1.552814in}}%
\pgfpathlineto{\pgfqpoint{4.798830in}{1.550795in}}%
\pgfpathlineto{\pgfqpoint{4.801586in}{1.545505in}}%
\pgfpathlineto{\pgfqpoint{4.804193in}{1.544860in}}%
\pgfpathlineto{\pgfqpoint{4.807017in}{1.539043in}}%
\pgfpathlineto{\pgfqpoint{4.809538in}{1.539902in}}%
\pgfpathlineto{\pgfqpoint{4.812377in}{1.537381in}}%
\pgfpathlineto{\pgfqpoint{4.814907in}{1.546199in}}%
\pgfpathlineto{\pgfqpoint{4.817587in}{1.532207in}}%
\pgfpathlineto{\pgfqpoint{4.820265in}{1.531650in}}%
\pgfpathlineto{\pgfqpoint{4.822945in}{1.539646in}}%
\pgfpathlineto{\pgfqpoint{4.825619in}{1.534619in}}%
\pgfpathlineto{\pgfqpoint{4.828291in}{1.544538in}}%
\pgfpathlineto{\pgfqpoint{4.831045in}{1.539350in}}%
\pgfpathlineto{\pgfqpoint{4.833657in}{1.534630in}}%
\pgfpathlineto{\pgfqpoint{4.837992in}{1.524680in}}%
\pgfpathlineto{\pgfqpoint{4.839922in}{1.537537in}}%
\pgfpathlineto{\pgfqpoint{4.842380in}{1.535220in}}%
\pgfpathlineto{\pgfqpoint{4.844361in}{1.535142in}}%
\pgfpathlineto{\pgfqpoint{4.847127in}{1.536794in}}%
\pgfpathlineto{\pgfqpoint{4.849715in}{1.540139in}}%
\pgfpathlineto{\pgfqpoint{4.852404in}{1.536132in}}%
\pgfpathlineto{\pgfqpoint{4.855070in}{1.535706in}}%
\pgfpathlineto{\pgfqpoint{4.857807in}{1.534099in}}%
\pgfpathlineto{\pgfqpoint{4.860544in}{1.532757in}}%
\pgfpathlineto{\pgfqpoint{4.863116in}{1.531415in}}%
\pgfpathlineto{\pgfqpoint{4.865910in}{1.533434in}}%
\pgfpathlineto{\pgfqpoint{4.868474in}{1.547566in}}%
\pgfpathlineto{\pgfqpoint{4.871209in}{1.541433in}}%
\pgfpathlineto{\pgfqpoint{4.873832in}{1.540548in}}%
\pgfpathlineto{\pgfqpoint{4.876636in}{1.546267in}}%
\pgfpathlineto{\pgfqpoint{4.879180in}{1.545648in}}%
\pgfpathlineto{\pgfqpoint{4.881864in}{1.535775in}}%
\pgfpathlineto{\pgfqpoint{4.884540in}{1.539090in}}%
\pgfpathlineto{\pgfqpoint{4.887211in}{1.542813in}}%
\pgfpathlineto{\pgfqpoint{4.889902in}{1.541275in}}%
\pgfpathlineto{\pgfqpoint{4.892611in}{1.538946in}}%
\pgfpathlineto{\pgfqpoint{4.895399in}{1.538731in}}%
\pgfpathlineto{\pgfqpoint{4.897938in}{1.542304in}}%
\pgfpathlineto{\pgfqpoint{4.900712in}{1.545327in}}%
\pgfpathlineto{\pgfqpoint{4.903295in}{1.548912in}}%
\pgfpathlineto{\pgfqpoint{4.906096in}{1.587526in}}%
\pgfpathlineto{\pgfqpoint{4.908648in}{1.637267in}}%
\pgfpathlineto{\pgfqpoint{4.911435in}{1.641148in}}%
\pgfpathlineto{\pgfqpoint{4.914009in}{1.627886in}}%
\pgfpathlineto{\pgfqpoint{4.916681in}{1.618349in}}%
\pgfpathlineto{\pgfqpoint{4.919352in}{1.605792in}}%
\pgfpathlineto{\pgfqpoint{4.922041in}{1.590648in}}%
\pgfpathlineto{\pgfqpoint{4.924708in}{1.594392in}}%
\pgfpathlineto{\pgfqpoint{4.927400in}{1.587712in}}%
\pgfpathlineto{\pgfqpoint{4.930170in}{1.585001in}}%
\pgfpathlineto{\pgfqpoint{4.932742in}{1.578567in}}%
\pgfpathlineto{\pgfqpoint{4.935515in}{1.574846in}}%
\pgfpathlineto{\pgfqpoint{4.938112in}{1.597742in}}%
\pgfpathlineto{\pgfqpoint{4.940881in}{1.582613in}}%
\pgfpathlineto{\pgfqpoint{4.943466in}{1.579533in}}%
\pgfpathlineto{\pgfqpoint{4.946151in}{1.578807in}}%
\pgfpathlineto{\pgfqpoint{4.948827in}{1.570533in}}%
\pgfpathlineto{\pgfqpoint{4.951504in}{1.571741in}}%
\pgfpathlineto{\pgfqpoint{4.954182in}{1.565283in}}%
\pgfpathlineto{\pgfqpoint{4.956862in}{1.556948in}}%
\pgfpathlineto{\pgfqpoint{4.959689in}{1.552036in}}%
\pgfpathlineto{\pgfqpoint{4.962219in}{1.554986in}}%
\pgfpathlineto{\pgfqpoint{4.965002in}{1.548427in}}%
\pgfpathlineto{\pgfqpoint{4.967575in}{1.544335in}}%
\pgfpathlineto{\pgfqpoint{4.970314in}{1.550027in}}%
\pgfpathlineto{\pgfqpoint{4.972933in}{1.544053in}}%
\pgfpathlineto{\pgfqpoint{4.975703in}{1.546119in}}%
\pgfpathlineto{\pgfqpoint{4.978287in}{1.542252in}}%
\pgfpathlineto{\pgfqpoint{4.980967in}{1.548931in}}%
\pgfpathlineto{\pgfqpoint{4.983637in}{1.547472in}}%
\pgfpathlineto{\pgfqpoint{4.986325in}{1.545184in}}%
\pgfpathlineto{\pgfqpoint{4.989001in}{1.542760in}}%
\pgfpathlineto{\pgfqpoint{4.991683in}{1.541123in}}%
\pgfpathlineto{\pgfqpoint{4.994390in}{1.542836in}}%
\pgfpathlineto{\pgfqpoint{4.997028in}{1.543233in}}%
\pgfpathlineto{\pgfqpoint{4.999780in}{1.538168in}}%
\pgfpathlineto{\pgfqpoint{5.002384in}{1.540401in}}%
\pgfpathlineto{\pgfqpoint{5.005178in}{1.540541in}}%
\pgfpathlineto{\pgfqpoint{5.007751in}{1.547975in}}%
\pgfpathlineto{\pgfqpoint{5.010562in}{1.546088in}}%
\pgfpathlineto{\pgfqpoint{5.013104in}{1.544479in}}%
\pgfpathlineto{\pgfqpoint{5.015820in}{1.540018in}}%
\pgfpathlineto{\pgfqpoint{5.018466in}{1.537262in}}%
\pgfpathlineto{\pgfqpoint{5.021147in}{1.539538in}}%
\pgfpathlineto{\pgfqpoint{5.023927in}{1.535549in}}%
\pgfpathlineto{\pgfqpoint{5.026501in}{1.535758in}}%
\pgfpathlineto{\pgfqpoint{5.029275in}{1.539826in}}%
\pgfpathlineto{\pgfqpoint{5.031849in}{1.534033in}}%
\pgfpathlineto{\pgfqpoint{5.034649in}{1.524680in}}%
\pgfpathlineto{\pgfqpoint{5.037214in}{1.528057in}}%
\pgfpathlineto{\pgfqpoint{5.039962in}{1.531969in}}%
\pgfpathlineto{\pgfqpoint{5.042572in}{1.532921in}}%
\pgfpathlineto{\pgfqpoint{5.045249in}{1.541196in}}%
\pgfpathlineto{\pgfqpoint{5.047924in}{1.541355in}}%
\pgfpathlineto{\pgfqpoint{5.050606in}{1.539516in}}%
\pgfpathlineto{\pgfqpoint{5.053284in}{1.544236in}}%
\pgfpathlineto{\pgfqpoint{5.055952in}{1.544538in}}%
\pgfpathlineto{\pgfqpoint{5.058711in}{1.546844in}}%
\pgfpathlineto{\pgfqpoint{5.061315in}{1.543640in}}%
\pgfpathlineto{\pgfqpoint{5.064144in}{1.542839in}}%
\pgfpathlineto{\pgfqpoint{5.066677in}{1.544750in}}%
\pgfpathlineto{\pgfqpoint{5.069463in}{1.548418in}}%
\pgfpathlineto{\pgfqpoint{5.072030in}{1.545814in}}%
\pgfpathlineto{\pgfqpoint{5.074851in}{1.542478in}}%
\pgfpathlineto{\pgfqpoint{5.077390in}{1.540421in}}%
\pgfpathlineto{\pgfqpoint{5.080067in}{1.539920in}}%
\pgfpathlineto{\pgfqpoint{5.082746in}{1.544140in}}%
\pgfpathlineto{\pgfqpoint{5.085426in}{1.538882in}}%
\pgfpathlineto{\pgfqpoint{5.088103in}{1.542993in}}%
\pgfpathlineto{\pgfqpoint{5.090788in}{1.537619in}}%
\pgfpathlineto{\pgfqpoint{5.093579in}{1.540731in}}%
\pgfpathlineto{\pgfqpoint{5.096142in}{1.540150in}}%
\pgfpathlineto{\pgfqpoint{5.098948in}{1.536196in}}%
\pgfpathlineto{\pgfqpoint{5.101496in}{1.538863in}}%
\pgfpathlineto{\pgfqpoint{5.104312in}{1.538976in}}%
\pgfpathlineto{\pgfqpoint{5.106842in}{1.541377in}}%
\pgfpathlineto{\pgfqpoint{5.109530in}{1.540746in}}%
\pgfpathlineto{\pgfqpoint{5.112209in}{1.536676in}}%
\pgfpathlineto{\pgfqpoint{5.114887in}{1.539450in}}%
\pgfpathlineto{\pgfqpoint{5.117550in}{1.540347in}}%
\pgfpathlineto{\pgfqpoint{5.120243in}{1.536605in}}%
\pgfpathlineto{\pgfqpoint{5.123042in}{1.538964in}}%
\pgfpathlineto{\pgfqpoint{5.125599in}{1.538345in}}%
\pgfpathlineto{\pgfqpoint{5.128421in}{1.538782in}}%
\pgfpathlineto{\pgfqpoint{5.130953in}{1.543236in}}%
\pgfpathlineto{\pgfqpoint{5.133716in}{1.549429in}}%
\pgfpathlineto{\pgfqpoint{5.136311in}{1.535218in}}%
\pgfpathlineto{\pgfqpoint{5.139072in}{1.541674in}}%
\pgfpathlineto{\pgfqpoint{5.141660in}{1.545675in}}%
\pgfpathlineto{\pgfqpoint{5.144349in}{1.539815in}}%
\pgfpathlineto{\pgfqpoint{5.147029in}{1.543772in}}%
\pgfpathlineto{\pgfqpoint{5.149734in}{1.535627in}}%
\pgfpathlineto{\pgfqpoint{5.152382in}{1.524819in}}%
\pgfpathlineto{\pgfqpoint{5.155059in}{1.529394in}}%
\pgfpathlineto{\pgfqpoint{5.157815in}{1.536561in}}%
\pgfpathlineto{\pgfqpoint{5.160420in}{1.535716in}}%
\pgfpathlineto{\pgfqpoint{5.163243in}{1.534815in}}%
\pgfpathlineto{\pgfqpoint{5.165775in}{1.536751in}}%
\pgfpathlineto{\pgfqpoint{5.168591in}{1.540474in}}%
\pgfpathlineto{\pgfqpoint{5.171133in}{1.540156in}}%
\pgfpathlineto{\pgfqpoint{5.173925in}{1.539344in}}%
\pgfpathlineto{\pgfqpoint{5.176477in}{1.541071in}}%
\pgfpathlineto{\pgfqpoint{5.179188in}{1.550826in}}%
\pgfpathlineto{\pgfqpoint{5.181848in}{1.541448in}}%
\pgfpathlineto{\pgfqpoint{5.184522in}{1.545262in}}%
\pgfpathlineto{\pgfqpoint{5.187294in}{1.545026in}}%
\pgfpathlineto{\pgfqpoint{5.189880in}{1.540260in}}%
\pgfpathlineto{\pgfqpoint{5.192680in}{1.542621in}}%
\pgfpathlineto{\pgfqpoint{5.195239in}{1.549218in}}%
\pgfpathlineto{\pgfqpoint{5.198008in}{1.547427in}}%
\pgfpathlineto{\pgfqpoint{5.200594in}{1.542935in}}%
\pgfpathlineto{\pgfqpoint{5.203388in}{1.540011in}}%
\pgfpathlineto{\pgfqpoint{5.205952in}{1.536384in}}%
\pgfpathlineto{\pgfqpoint{5.208630in}{1.539585in}}%
\pgfpathlineto{\pgfqpoint{5.211299in}{1.534687in}}%
\pgfpathlineto{\pgfqpoint{5.214027in}{1.538471in}}%
\pgfpathlineto{\pgfqpoint{5.216667in}{1.542647in}}%
\pgfpathlineto{\pgfqpoint{5.219345in}{1.538135in}}%
\pgfpathlineto{\pgfqpoint{5.222151in}{1.543481in}}%
\pgfpathlineto{\pgfqpoint{5.224695in}{1.546693in}}%
\pgfpathlineto{\pgfqpoint{5.227470in}{1.535098in}}%
\pgfpathlineto{\pgfqpoint{5.230059in}{1.534487in}}%
\pgfpathlineto{\pgfqpoint{5.232855in}{1.535100in}}%
\pgfpathlineto{\pgfqpoint{5.235409in}{1.545756in}}%
\pgfpathlineto{\pgfqpoint{5.238173in}{1.538340in}}%
\pgfpathlineto{\pgfqpoint{5.240777in}{1.538369in}}%
\pgfpathlineto{\pgfqpoint{5.243445in}{1.539262in}}%
\pgfpathlineto{\pgfqpoint{5.246130in}{1.541274in}}%
\pgfpathlineto{\pgfqpoint{5.248816in}{1.535289in}}%
\pgfpathlineto{\pgfqpoint{5.251590in}{1.542631in}}%
\pgfpathlineto{\pgfqpoint{5.254236in}{1.538968in}}%
\pgfpathlineto{\pgfqpoint{5.256973in}{1.536264in}}%
\pgfpathlineto{\pgfqpoint{5.259511in}{1.533191in}}%
\pgfpathlineto{\pgfqpoint{5.262264in}{1.526843in}}%
\pgfpathlineto{\pgfqpoint{5.264876in}{1.537719in}}%
\pgfpathlineto{\pgfqpoint{5.267691in}{1.536497in}}%
\pgfpathlineto{\pgfqpoint{5.270238in}{1.535752in}}%
\pgfpathlineto{\pgfqpoint{5.272913in}{1.538247in}}%
\pgfpathlineto{\pgfqpoint{5.275589in}{1.534918in}}%
\pgfpathlineto{\pgfqpoint{5.278322in}{1.534298in}}%
\pgfpathlineto{\pgfqpoint{5.280947in}{1.535276in}}%
\pgfpathlineto{\pgfqpoint{5.283631in}{1.537798in}}%
\pgfpathlineto{\pgfqpoint{5.286436in}{1.536021in}}%
\pgfpathlineto{\pgfqpoint{5.288984in}{1.541682in}}%
\pgfpathlineto{\pgfqpoint{5.291794in}{1.540179in}}%
\pgfpathlineto{\pgfqpoint{5.294339in}{1.542347in}}%
\pgfpathlineto{\pgfqpoint{5.297140in}{1.546084in}}%
\pgfpathlineto{\pgfqpoint{5.299696in}{1.544591in}}%
\pgfpathlineto{\pgfqpoint{5.302443in}{1.546200in}}%
\pgfpathlineto{\pgfqpoint{5.305054in}{1.550116in}}%
\pgfpathlineto{\pgfqpoint{5.307731in}{1.549141in}}%
\pgfpathlineto{\pgfqpoint{5.310411in}{1.545769in}}%
\pgfpathlineto{\pgfqpoint{5.313089in}{1.550556in}}%
\pgfpathlineto{\pgfqpoint{5.315754in}{1.547033in}}%
\pgfpathlineto{\pgfqpoint{5.318430in}{1.551170in}}%
\pgfpathlineto{\pgfqpoint{5.321256in}{1.547283in}}%
\pgfpathlineto{\pgfqpoint{5.323802in}{1.548284in}}%
\pgfpathlineto{\pgfqpoint{5.326564in}{1.542695in}}%
\pgfpathlineto{\pgfqpoint{5.329159in}{1.543470in}}%
\pgfpathlineto{\pgfqpoint{5.331973in}{1.541162in}}%
\pgfpathlineto{\pgfqpoint{5.334510in}{1.533240in}}%
\pgfpathlineto{\pgfqpoint{5.337353in}{1.534831in}}%
\pgfpathlineto{\pgfqpoint{5.339872in}{1.536429in}}%
\pgfpathlineto{\pgfqpoint{5.342549in}{1.552921in}}%
\pgfpathlineto{\pgfqpoint{5.345224in}{1.574458in}}%
\pgfpathlineto{\pgfqpoint{5.347905in}{1.598719in}}%
\pgfpathlineto{\pgfqpoint{5.350723in}{1.589238in}}%
\pgfpathlineto{\pgfqpoint{5.353262in}{1.567894in}}%
\pgfpathlineto{\pgfqpoint{5.356056in}{1.563635in}}%
\pgfpathlineto{\pgfqpoint{5.358612in}{1.557833in}}%
\pgfpathlineto{\pgfqpoint{5.361370in}{1.554987in}}%
\pgfpathlineto{\pgfqpoint{5.363966in}{1.548889in}}%
\pgfpathlineto{\pgfqpoint{5.366727in}{1.550148in}}%
\pgfpathlineto{\pgfqpoint{5.369335in}{1.548704in}}%
\pgfpathlineto{\pgfqpoint{5.372013in}{1.549739in}}%
\pgfpathlineto{\pgfqpoint{5.374692in}{1.543997in}}%
\pgfpathlineto{\pgfqpoint{5.377370in}{1.542368in}}%
\pgfpathlineto{\pgfqpoint{5.380048in}{1.549788in}}%
\pgfpathlineto{\pgfqpoint{5.382725in}{1.548409in}}%
\pgfpathlineto{\pgfqpoint{5.385550in}{1.545364in}}%
\pgfpathlineto{\pgfqpoint{5.388083in}{1.551782in}}%
\pgfpathlineto{\pgfqpoint{5.390900in}{1.564336in}}%
\pgfpathlineto{\pgfqpoint{5.393441in}{1.564936in}}%
\pgfpathlineto{\pgfqpoint{5.396219in}{1.558021in}}%
\pgfpathlineto{\pgfqpoint{5.398784in}{1.550781in}}%
\pgfpathlineto{\pgfqpoint{5.401576in}{1.550961in}}%
\pgfpathlineto{\pgfqpoint{5.404154in}{1.548638in}}%
\pgfpathlineto{\pgfqpoint{5.406832in}{1.551771in}}%
\pgfpathlineto{\pgfqpoint{5.409507in}{1.544519in}}%
\pgfpathlineto{\pgfqpoint{5.412190in}{1.546007in}}%
\pgfpathlineto{\pgfqpoint{5.414954in}{1.549552in}}%
\pgfpathlineto{\pgfqpoint{5.417547in}{1.549525in}}%
\pgfpathlineto{\pgfqpoint{5.420304in}{1.547598in}}%
\pgfpathlineto{\pgfqpoint{5.422897in}{1.538281in}}%
\pgfpathlineto{\pgfqpoint{5.425661in}{1.544233in}}%
\pgfpathlineto{\pgfqpoint{5.428259in}{1.542871in}}%
\pgfpathlineto{\pgfqpoint{5.431015in}{1.540351in}}%
\pgfpathlineto{\pgfqpoint{5.433616in}{1.542689in}}%
\pgfpathlineto{\pgfqpoint{5.436295in}{1.540182in}}%
\pgfpathlineto{\pgfqpoint{5.438974in}{1.539857in}}%
\pgfpathlineto{\pgfqpoint{5.441698in}{1.547111in}}%
\pgfpathlineto{\pgfqpoint{5.444328in}{1.542639in}}%
\pgfpathlineto{\pgfqpoint{5.447021in}{1.536504in}}%
\pgfpathlineto{\pgfqpoint{5.449769in}{1.545791in}}%
\pgfpathlineto{\pgfqpoint{5.452365in}{1.546007in}}%
\pgfpathlineto{\pgfqpoint{5.455168in}{1.545455in}}%
\pgfpathlineto{\pgfqpoint{5.457721in}{1.541922in}}%
\pgfpathlineto{\pgfqpoint{5.460489in}{1.537307in}}%
\pgfpathlineto{\pgfqpoint{5.463079in}{1.540448in}}%
\pgfpathlineto{\pgfqpoint{5.465888in}{1.539396in}}%
\pgfpathlineto{\pgfqpoint{5.468425in}{1.536983in}}%
\pgfpathlineto{\pgfqpoint{5.471113in}{1.533292in}}%
\pgfpathlineto{\pgfqpoint{5.473792in}{1.539565in}}%
\pgfpathlineto{\pgfqpoint{5.476458in}{1.536477in}}%
\pgfpathlineto{\pgfqpoint{5.479152in}{1.546019in}}%
\pgfpathlineto{\pgfqpoint{5.481825in}{1.544108in}}%
\pgfpathlineto{\pgfqpoint{5.484641in}{1.545267in}}%
\pgfpathlineto{\pgfqpoint{5.487176in}{1.541432in}}%
\pgfpathlineto{\pgfqpoint{5.490000in}{1.543052in}}%
\pgfpathlineto{\pgfqpoint{5.492541in}{1.546801in}}%
\pgfpathlineto{\pgfqpoint{5.495346in}{1.546086in}}%
\pgfpathlineto{\pgfqpoint{5.497898in}{1.543956in}}%
\pgfpathlineto{\pgfqpoint{5.500687in}{1.538345in}}%
\pgfpathlineto{\pgfqpoint{5.503255in}{1.537710in}}%
\pgfpathlineto{\pgfqpoint{5.505933in}{1.536480in}}%
\pgfpathlineto{\pgfqpoint{5.508612in}{1.539308in}}%
\pgfpathlineto{\pgfqpoint{5.511290in}{1.541104in}}%
\pgfpathlineto{\pgfqpoint{5.514080in}{1.542227in}}%
\pgfpathlineto{\pgfqpoint{5.516646in}{1.535791in}}%
\pgfpathlineto{\pgfqpoint{5.519433in}{1.539483in}}%
\pgfpathlineto{\pgfqpoint{5.522003in}{1.536891in}}%
\pgfpathlineto{\pgfqpoint{5.524756in}{1.539179in}}%
\pgfpathlineto{\pgfqpoint{5.527360in}{1.535371in}}%
\pgfpathlineto{\pgfqpoint{5.530148in}{1.537408in}}%
\pgfpathlineto{\pgfqpoint{5.532717in}{1.534805in}}%
\pgfpathlineto{\pgfqpoint{5.535395in}{1.539129in}}%
\pgfpathlineto{\pgfqpoint{5.538074in}{1.537506in}}%
\pgfpathlineto{\pgfqpoint{5.540750in}{1.536767in}}%
\pgfpathlineto{\pgfqpoint{5.543421in}{1.539048in}}%
\pgfpathlineto{\pgfqpoint{5.546110in}{1.545332in}}%
\pgfpathlineto{\pgfqpoint{5.548921in}{1.530875in}}%
\pgfpathlineto{\pgfqpoint{5.551457in}{1.540125in}}%
\pgfpathlineto{\pgfqpoint{5.554198in}{1.531611in}}%
\pgfpathlineto{\pgfqpoint{5.556822in}{1.532608in}}%
\pgfpathlineto{\pgfqpoint{5.559612in}{1.538131in}}%
\pgfpathlineto{\pgfqpoint{5.562180in}{1.530946in}}%
\pgfpathlineto{\pgfqpoint{5.564940in}{1.530363in}}%
\pgfpathlineto{\pgfqpoint{5.567536in}{1.531822in}}%
\pgfpathlineto{\pgfqpoint{5.570215in}{1.530999in}}%
\pgfpathlineto{\pgfqpoint{5.572893in}{1.534058in}}%
\pgfpathlineto{\pgfqpoint{5.575596in}{1.539837in}}%
\pgfpathlineto{\pgfqpoint{5.578342in}{1.543484in}}%
\pgfpathlineto{\pgfqpoint{5.580914in}{1.539065in}}%
\pgfpathlineto{\pgfqpoint{5.583709in}{1.535364in}}%
\pgfpathlineto{\pgfqpoint{5.586269in}{1.538240in}}%
\pgfpathlineto{\pgfqpoint{5.589040in}{1.542646in}}%
\pgfpathlineto{\pgfqpoint{5.591641in}{1.546732in}}%
\pgfpathlineto{\pgfqpoint{5.594368in}{1.538497in}}%
\pgfpathlineto{\pgfqpoint{5.596999in}{1.548337in}}%
\pgfpathlineto{\pgfqpoint{5.599674in}{1.549210in}}%
\pgfpathlineto{\pgfqpoint{5.602352in}{1.540497in}}%
\pgfpathlineto{\pgfqpoint{5.605073in}{1.546177in}}%
\pgfpathlineto{\pgfqpoint{5.607698in}{1.548741in}}%
\pgfpathlineto{\pgfqpoint{5.610389in}{1.543963in}}%
\pgfpathlineto{\pgfqpoint{5.613235in}{1.546723in}}%
\pgfpathlineto{\pgfqpoint{5.615743in}{1.549185in}}%
\pgfpathlineto{\pgfqpoint{5.618526in}{1.544089in}}%
\pgfpathlineto{\pgfqpoint{5.621102in}{1.532813in}}%
\pgfpathlineto{\pgfqpoint{5.623868in}{1.534637in}}%
\pgfpathlineto{\pgfqpoint{5.626460in}{1.540054in}}%
\pgfpathlineto{\pgfqpoint{5.629232in}{1.539949in}}%
\pgfpathlineto{\pgfqpoint{5.631815in}{1.541114in}}%
\pgfpathlineto{\pgfqpoint{5.634496in}{1.543764in}}%
\pgfpathlineto{\pgfqpoint{5.637172in}{1.545199in}}%
\pgfpathlineto{\pgfqpoint{5.639852in}{1.546871in}}%
\pgfpathlineto{\pgfqpoint{5.642518in}{1.554349in}}%
\pgfpathlineto{\pgfqpoint{5.645243in}{1.548773in}}%
\pgfpathlineto{\pgfqpoint{5.648008in}{1.552036in}}%
\pgfpathlineto{\pgfqpoint{5.650563in}{1.551945in}}%
\pgfpathlineto{\pgfqpoint{5.653376in}{1.547729in}}%
\pgfpathlineto{\pgfqpoint{5.655919in}{1.544519in}}%
\pgfpathlineto{\pgfqpoint{5.658723in}{1.539042in}}%
\pgfpathlineto{\pgfqpoint{5.661273in}{1.537067in}}%
\pgfpathlineto{\pgfqpoint{5.664099in}{1.544577in}}%
\pgfpathlineto{\pgfqpoint{5.666632in}{1.540586in}}%
\pgfpathlineto{\pgfqpoint{5.669313in}{1.540433in}}%
\pgfpathlineto{\pgfqpoint{5.671991in}{1.540989in}}%
\pgfpathlineto{\pgfqpoint{5.674667in}{1.545004in}}%
\pgfpathlineto{\pgfqpoint{5.677486in}{1.538960in}}%
\pgfpathlineto{\pgfqpoint{5.680027in}{1.537713in}}%
\pgfpathlineto{\pgfqpoint{5.682836in}{1.535306in}}%
\pgfpathlineto{\pgfqpoint{5.685385in}{1.535874in}}%
\pgfpathlineto{\pgfqpoint{5.688159in}{1.534768in}}%
\pgfpathlineto{\pgfqpoint{5.690730in}{1.532698in}}%
\pgfpathlineto{\pgfqpoint{5.693473in}{1.549543in}}%
\pgfpathlineto{\pgfqpoint{5.696101in}{1.543772in}}%
\pgfpathlineto{\pgfqpoint{5.698775in}{1.544868in}}%
\pgfpathlineto{\pgfqpoint{5.701453in}{1.536309in}}%
\pgfpathlineto{\pgfqpoint{5.704130in}{1.531797in}}%
\pgfpathlineto{\pgfqpoint{5.706800in}{1.524680in}}%
\pgfpathlineto{\pgfqpoint{5.709490in}{1.524680in}}%
\pgfpathlineto{\pgfqpoint{5.712291in}{1.524680in}}%
\pgfpathlineto{\pgfqpoint{5.714834in}{1.524680in}}%
\pgfpathlineto{\pgfqpoint{5.717671in}{1.524680in}}%
\pgfpathlineto{\pgfqpoint{5.720201in}{1.524680in}}%
\pgfpathlineto{\pgfqpoint{5.722950in}{1.528014in}}%
\pgfpathlineto{\pgfqpoint{5.725548in}{1.532311in}}%
\pgfpathlineto{\pgfqpoint{5.728339in}{1.535693in}}%
\pgfpathlineto{\pgfqpoint{5.730919in}{1.533523in}}%
\pgfpathlineto{\pgfqpoint{5.733594in}{1.536413in}}%
\pgfpathlineto{\pgfqpoint{5.736276in}{1.539827in}}%
\pgfpathlineto{\pgfqpoint{5.738974in}{1.537211in}}%
\pgfpathlineto{\pgfqpoint{5.741745in}{1.535874in}}%
\pgfpathlineto{\pgfqpoint{5.744310in}{1.541045in}}%
\pgfpathlineto{\pgfqpoint{5.744310in}{0.413320in}}%
\pgfpathlineto{\pgfqpoint{5.744310in}{0.413320in}}%
\pgfpathlineto{\pgfqpoint{5.741745in}{0.413320in}}%
\pgfpathlineto{\pgfqpoint{5.738974in}{0.413320in}}%
\pgfpathlineto{\pgfqpoint{5.736276in}{0.413320in}}%
\pgfpathlineto{\pgfqpoint{5.733594in}{0.413320in}}%
\pgfpathlineto{\pgfqpoint{5.730919in}{0.413320in}}%
\pgfpathlineto{\pgfqpoint{5.728339in}{0.413320in}}%
\pgfpathlineto{\pgfqpoint{5.725548in}{0.413320in}}%
\pgfpathlineto{\pgfqpoint{5.722950in}{0.413320in}}%
\pgfpathlineto{\pgfqpoint{5.720201in}{0.413320in}}%
\pgfpathlineto{\pgfqpoint{5.717671in}{0.413320in}}%
\pgfpathlineto{\pgfqpoint{5.714834in}{0.413320in}}%
\pgfpathlineto{\pgfqpoint{5.712291in}{0.413320in}}%
\pgfpathlineto{\pgfqpoint{5.709490in}{0.413320in}}%
\pgfpathlineto{\pgfqpoint{5.706800in}{0.413320in}}%
\pgfpathlineto{\pgfqpoint{5.704130in}{0.413320in}}%
\pgfpathlineto{\pgfqpoint{5.701453in}{0.413320in}}%
\pgfpathlineto{\pgfqpoint{5.698775in}{0.413320in}}%
\pgfpathlineto{\pgfqpoint{5.696101in}{0.413320in}}%
\pgfpathlineto{\pgfqpoint{5.693473in}{0.413320in}}%
\pgfpathlineto{\pgfqpoint{5.690730in}{0.413320in}}%
\pgfpathlineto{\pgfqpoint{5.688159in}{0.413320in}}%
\pgfpathlineto{\pgfqpoint{5.685385in}{0.413320in}}%
\pgfpathlineto{\pgfqpoint{5.682836in}{0.413320in}}%
\pgfpathlineto{\pgfqpoint{5.680027in}{0.413320in}}%
\pgfpathlineto{\pgfqpoint{5.677486in}{0.413320in}}%
\pgfpathlineto{\pgfqpoint{5.674667in}{0.413320in}}%
\pgfpathlineto{\pgfqpoint{5.671991in}{0.413320in}}%
\pgfpathlineto{\pgfqpoint{5.669313in}{0.413320in}}%
\pgfpathlineto{\pgfqpoint{5.666632in}{0.413320in}}%
\pgfpathlineto{\pgfqpoint{5.664099in}{0.413320in}}%
\pgfpathlineto{\pgfqpoint{5.661273in}{0.413320in}}%
\pgfpathlineto{\pgfqpoint{5.658723in}{0.413320in}}%
\pgfpathlineto{\pgfqpoint{5.655919in}{0.413320in}}%
\pgfpathlineto{\pgfqpoint{5.653376in}{0.413320in}}%
\pgfpathlineto{\pgfqpoint{5.650563in}{0.413320in}}%
\pgfpathlineto{\pgfqpoint{5.648008in}{0.413320in}}%
\pgfpathlineto{\pgfqpoint{5.645243in}{0.413320in}}%
\pgfpathlineto{\pgfqpoint{5.642518in}{0.413320in}}%
\pgfpathlineto{\pgfqpoint{5.639852in}{0.413320in}}%
\pgfpathlineto{\pgfqpoint{5.637172in}{0.413320in}}%
\pgfpathlineto{\pgfqpoint{5.634496in}{0.413320in}}%
\pgfpathlineto{\pgfqpoint{5.631815in}{0.413320in}}%
\pgfpathlineto{\pgfqpoint{5.629232in}{0.413320in}}%
\pgfpathlineto{\pgfqpoint{5.626460in}{0.413320in}}%
\pgfpathlineto{\pgfqpoint{5.623868in}{0.413320in}}%
\pgfpathlineto{\pgfqpoint{5.621102in}{0.413320in}}%
\pgfpathlineto{\pgfqpoint{5.618526in}{0.413320in}}%
\pgfpathlineto{\pgfqpoint{5.615743in}{0.413320in}}%
\pgfpathlineto{\pgfqpoint{5.613235in}{0.413320in}}%
\pgfpathlineto{\pgfqpoint{5.610389in}{0.413320in}}%
\pgfpathlineto{\pgfqpoint{5.607698in}{0.413320in}}%
\pgfpathlineto{\pgfqpoint{5.605073in}{0.413320in}}%
\pgfpathlineto{\pgfqpoint{5.602352in}{0.413320in}}%
\pgfpathlineto{\pgfqpoint{5.599674in}{0.413320in}}%
\pgfpathlineto{\pgfqpoint{5.596999in}{0.413320in}}%
\pgfpathlineto{\pgfqpoint{5.594368in}{0.413320in}}%
\pgfpathlineto{\pgfqpoint{5.591641in}{0.413320in}}%
\pgfpathlineto{\pgfqpoint{5.589040in}{0.413320in}}%
\pgfpathlineto{\pgfqpoint{5.586269in}{0.413320in}}%
\pgfpathlineto{\pgfqpoint{5.583709in}{0.413320in}}%
\pgfpathlineto{\pgfqpoint{5.580914in}{0.413320in}}%
\pgfpathlineto{\pgfqpoint{5.578342in}{0.413320in}}%
\pgfpathlineto{\pgfqpoint{5.575596in}{0.413320in}}%
\pgfpathlineto{\pgfqpoint{5.572893in}{0.413320in}}%
\pgfpathlineto{\pgfqpoint{5.570215in}{0.413320in}}%
\pgfpathlineto{\pgfqpoint{5.567536in}{0.413320in}}%
\pgfpathlineto{\pgfqpoint{5.564940in}{0.413320in}}%
\pgfpathlineto{\pgfqpoint{5.562180in}{0.413320in}}%
\pgfpathlineto{\pgfqpoint{5.559612in}{0.413320in}}%
\pgfpathlineto{\pgfqpoint{5.556822in}{0.413320in}}%
\pgfpathlineto{\pgfqpoint{5.554198in}{0.413320in}}%
\pgfpathlineto{\pgfqpoint{5.551457in}{0.413320in}}%
\pgfpathlineto{\pgfqpoint{5.548921in}{0.413320in}}%
\pgfpathlineto{\pgfqpoint{5.546110in}{0.413320in}}%
\pgfpathlineto{\pgfqpoint{5.543421in}{0.413320in}}%
\pgfpathlineto{\pgfqpoint{5.540750in}{0.413320in}}%
\pgfpathlineto{\pgfqpoint{5.538074in}{0.413320in}}%
\pgfpathlineto{\pgfqpoint{5.535395in}{0.413320in}}%
\pgfpathlineto{\pgfqpoint{5.532717in}{0.413320in}}%
\pgfpathlineto{\pgfqpoint{5.530148in}{0.413320in}}%
\pgfpathlineto{\pgfqpoint{5.527360in}{0.413320in}}%
\pgfpathlineto{\pgfqpoint{5.524756in}{0.413320in}}%
\pgfpathlineto{\pgfqpoint{5.522003in}{0.413320in}}%
\pgfpathlineto{\pgfqpoint{5.519433in}{0.413320in}}%
\pgfpathlineto{\pgfqpoint{5.516646in}{0.413320in}}%
\pgfpathlineto{\pgfqpoint{5.514080in}{0.413320in}}%
\pgfpathlineto{\pgfqpoint{5.511290in}{0.413320in}}%
\pgfpathlineto{\pgfqpoint{5.508612in}{0.413320in}}%
\pgfpathlineto{\pgfqpoint{5.505933in}{0.413320in}}%
\pgfpathlineto{\pgfqpoint{5.503255in}{0.413320in}}%
\pgfpathlineto{\pgfqpoint{5.500687in}{0.413320in}}%
\pgfpathlineto{\pgfqpoint{5.497898in}{0.413320in}}%
\pgfpathlineto{\pgfqpoint{5.495346in}{0.413320in}}%
\pgfpathlineto{\pgfqpoint{5.492541in}{0.413320in}}%
\pgfpathlineto{\pgfqpoint{5.490000in}{0.413320in}}%
\pgfpathlineto{\pgfqpoint{5.487176in}{0.413320in}}%
\pgfpathlineto{\pgfqpoint{5.484641in}{0.413320in}}%
\pgfpathlineto{\pgfqpoint{5.481825in}{0.413320in}}%
\pgfpathlineto{\pgfqpoint{5.479152in}{0.413320in}}%
\pgfpathlineto{\pgfqpoint{5.476458in}{0.413320in}}%
\pgfpathlineto{\pgfqpoint{5.473792in}{0.413320in}}%
\pgfpathlineto{\pgfqpoint{5.471113in}{0.413320in}}%
\pgfpathlineto{\pgfqpoint{5.468425in}{0.413320in}}%
\pgfpathlineto{\pgfqpoint{5.465888in}{0.413320in}}%
\pgfpathlineto{\pgfqpoint{5.463079in}{0.413320in}}%
\pgfpathlineto{\pgfqpoint{5.460489in}{0.413320in}}%
\pgfpathlineto{\pgfqpoint{5.457721in}{0.413320in}}%
\pgfpathlineto{\pgfqpoint{5.455168in}{0.413320in}}%
\pgfpathlineto{\pgfqpoint{5.452365in}{0.413320in}}%
\pgfpathlineto{\pgfqpoint{5.449769in}{0.413320in}}%
\pgfpathlineto{\pgfqpoint{5.447021in}{0.413320in}}%
\pgfpathlineto{\pgfqpoint{5.444328in}{0.413320in}}%
\pgfpathlineto{\pgfqpoint{5.441698in}{0.413320in}}%
\pgfpathlineto{\pgfqpoint{5.438974in}{0.413320in}}%
\pgfpathlineto{\pgfqpoint{5.436295in}{0.413320in}}%
\pgfpathlineto{\pgfqpoint{5.433616in}{0.413320in}}%
\pgfpathlineto{\pgfqpoint{5.431015in}{0.413320in}}%
\pgfpathlineto{\pgfqpoint{5.428259in}{0.413320in}}%
\pgfpathlineto{\pgfqpoint{5.425661in}{0.413320in}}%
\pgfpathlineto{\pgfqpoint{5.422897in}{0.413320in}}%
\pgfpathlineto{\pgfqpoint{5.420304in}{0.413320in}}%
\pgfpathlineto{\pgfqpoint{5.417547in}{0.413320in}}%
\pgfpathlineto{\pgfqpoint{5.414954in}{0.413320in}}%
\pgfpathlineto{\pgfqpoint{5.412190in}{0.413320in}}%
\pgfpathlineto{\pgfqpoint{5.409507in}{0.413320in}}%
\pgfpathlineto{\pgfqpoint{5.406832in}{0.413320in}}%
\pgfpathlineto{\pgfqpoint{5.404154in}{0.413320in}}%
\pgfpathlineto{\pgfqpoint{5.401576in}{0.413320in}}%
\pgfpathlineto{\pgfqpoint{5.398784in}{0.413320in}}%
\pgfpathlineto{\pgfqpoint{5.396219in}{0.413320in}}%
\pgfpathlineto{\pgfqpoint{5.393441in}{0.413320in}}%
\pgfpathlineto{\pgfqpoint{5.390900in}{0.413320in}}%
\pgfpathlineto{\pgfqpoint{5.388083in}{0.413320in}}%
\pgfpathlineto{\pgfqpoint{5.385550in}{0.413320in}}%
\pgfpathlineto{\pgfqpoint{5.382725in}{0.413320in}}%
\pgfpathlineto{\pgfqpoint{5.380048in}{0.413320in}}%
\pgfpathlineto{\pgfqpoint{5.377370in}{0.413320in}}%
\pgfpathlineto{\pgfqpoint{5.374692in}{0.413320in}}%
\pgfpathlineto{\pgfqpoint{5.372013in}{0.413320in}}%
\pgfpathlineto{\pgfqpoint{5.369335in}{0.413320in}}%
\pgfpathlineto{\pgfqpoint{5.366727in}{0.413320in}}%
\pgfpathlineto{\pgfqpoint{5.363966in}{0.413320in}}%
\pgfpathlineto{\pgfqpoint{5.361370in}{0.413320in}}%
\pgfpathlineto{\pgfqpoint{5.358612in}{0.413320in}}%
\pgfpathlineto{\pgfqpoint{5.356056in}{0.413320in}}%
\pgfpathlineto{\pgfqpoint{5.353262in}{0.413320in}}%
\pgfpathlineto{\pgfqpoint{5.350723in}{0.413320in}}%
\pgfpathlineto{\pgfqpoint{5.347905in}{0.413320in}}%
\pgfpathlineto{\pgfqpoint{5.345224in}{0.413320in}}%
\pgfpathlineto{\pgfqpoint{5.342549in}{0.413320in}}%
\pgfpathlineto{\pgfqpoint{5.339872in}{0.413320in}}%
\pgfpathlineto{\pgfqpoint{5.337353in}{0.413320in}}%
\pgfpathlineto{\pgfqpoint{5.334510in}{0.413320in}}%
\pgfpathlineto{\pgfqpoint{5.331973in}{0.413320in}}%
\pgfpathlineto{\pgfqpoint{5.329159in}{0.413320in}}%
\pgfpathlineto{\pgfqpoint{5.326564in}{0.413320in}}%
\pgfpathlineto{\pgfqpoint{5.323802in}{0.413320in}}%
\pgfpathlineto{\pgfqpoint{5.321256in}{0.413320in}}%
\pgfpathlineto{\pgfqpoint{5.318430in}{0.413320in}}%
\pgfpathlineto{\pgfqpoint{5.315754in}{0.413320in}}%
\pgfpathlineto{\pgfqpoint{5.313089in}{0.413320in}}%
\pgfpathlineto{\pgfqpoint{5.310411in}{0.413320in}}%
\pgfpathlineto{\pgfqpoint{5.307731in}{0.413320in}}%
\pgfpathlineto{\pgfqpoint{5.305054in}{0.413320in}}%
\pgfpathlineto{\pgfqpoint{5.302443in}{0.413320in}}%
\pgfpathlineto{\pgfqpoint{5.299696in}{0.413320in}}%
\pgfpathlineto{\pgfqpoint{5.297140in}{0.413320in}}%
\pgfpathlineto{\pgfqpoint{5.294339in}{0.413320in}}%
\pgfpathlineto{\pgfqpoint{5.291794in}{0.413320in}}%
\pgfpathlineto{\pgfqpoint{5.288984in}{0.413320in}}%
\pgfpathlineto{\pgfqpoint{5.286436in}{0.413320in}}%
\pgfpathlineto{\pgfqpoint{5.283631in}{0.413320in}}%
\pgfpathlineto{\pgfqpoint{5.280947in}{0.413320in}}%
\pgfpathlineto{\pgfqpoint{5.278322in}{0.413320in}}%
\pgfpathlineto{\pgfqpoint{5.275589in}{0.413320in}}%
\pgfpathlineto{\pgfqpoint{5.272913in}{0.413320in}}%
\pgfpathlineto{\pgfqpoint{5.270238in}{0.413320in}}%
\pgfpathlineto{\pgfqpoint{5.267691in}{0.413320in}}%
\pgfpathlineto{\pgfqpoint{5.264876in}{0.413320in}}%
\pgfpathlineto{\pgfqpoint{5.262264in}{0.413320in}}%
\pgfpathlineto{\pgfqpoint{5.259511in}{0.413320in}}%
\pgfpathlineto{\pgfqpoint{5.256973in}{0.413320in}}%
\pgfpathlineto{\pgfqpoint{5.254236in}{0.413320in}}%
\pgfpathlineto{\pgfqpoint{5.251590in}{0.413320in}}%
\pgfpathlineto{\pgfqpoint{5.248816in}{0.413320in}}%
\pgfpathlineto{\pgfqpoint{5.246130in}{0.413320in}}%
\pgfpathlineto{\pgfqpoint{5.243445in}{0.413320in}}%
\pgfpathlineto{\pgfqpoint{5.240777in}{0.413320in}}%
\pgfpathlineto{\pgfqpoint{5.238173in}{0.413320in}}%
\pgfpathlineto{\pgfqpoint{5.235409in}{0.413320in}}%
\pgfpathlineto{\pgfqpoint{5.232855in}{0.413320in}}%
\pgfpathlineto{\pgfqpoint{5.230059in}{0.413320in}}%
\pgfpathlineto{\pgfqpoint{5.227470in}{0.413320in}}%
\pgfpathlineto{\pgfqpoint{5.224695in}{0.413320in}}%
\pgfpathlineto{\pgfqpoint{5.222151in}{0.413320in}}%
\pgfpathlineto{\pgfqpoint{5.219345in}{0.413320in}}%
\pgfpathlineto{\pgfqpoint{5.216667in}{0.413320in}}%
\pgfpathlineto{\pgfqpoint{5.214027in}{0.413320in}}%
\pgfpathlineto{\pgfqpoint{5.211299in}{0.413320in}}%
\pgfpathlineto{\pgfqpoint{5.208630in}{0.413320in}}%
\pgfpathlineto{\pgfqpoint{5.205952in}{0.413320in}}%
\pgfpathlineto{\pgfqpoint{5.203388in}{0.413320in}}%
\pgfpathlineto{\pgfqpoint{5.200594in}{0.413320in}}%
\pgfpathlineto{\pgfqpoint{5.198008in}{0.413320in}}%
\pgfpathlineto{\pgfqpoint{5.195239in}{0.413320in}}%
\pgfpathlineto{\pgfqpoint{5.192680in}{0.413320in}}%
\pgfpathlineto{\pgfqpoint{5.189880in}{0.413320in}}%
\pgfpathlineto{\pgfqpoint{5.187294in}{0.413320in}}%
\pgfpathlineto{\pgfqpoint{5.184522in}{0.413320in}}%
\pgfpathlineto{\pgfqpoint{5.181848in}{0.413320in}}%
\pgfpathlineto{\pgfqpoint{5.179188in}{0.413320in}}%
\pgfpathlineto{\pgfqpoint{5.176477in}{0.413320in}}%
\pgfpathlineto{\pgfqpoint{5.173925in}{0.413320in}}%
\pgfpathlineto{\pgfqpoint{5.171133in}{0.413320in}}%
\pgfpathlineto{\pgfqpoint{5.168591in}{0.413320in}}%
\pgfpathlineto{\pgfqpoint{5.165775in}{0.413320in}}%
\pgfpathlineto{\pgfqpoint{5.163243in}{0.413320in}}%
\pgfpathlineto{\pgfqpoint{5.160420in}{0.413320in}}%
\pgfpathlineto{\pgfqpoint{5.157815in}{0.413320in}}%
\pgfpathlineto{\pgfqpoint{5.155059in}{0.413320in}}%
\pgfpathlineto{\pgfqpoint{5.152382in}{0.413320in}}%
\pgfpathlineto{\pgfqpoint{5.149734in}{0.413320in}}%
\pgfpathlineto{\pgfqpoint{5.147029in}{0.413320in}}%
\pgfpathlineto{\pgfqpoint{5.144349in}{0.413320in}}%
\pgfpathlineto{\pgfqpoint{5.141660in}{0.413320in}}%
\pgfpathlineto{\pgfqpoint{5.139072in}{0.413320in}}%
\pgfpathlineto{\pgfqpoint{5.136311in}{0.413320in}}%
\pgfpathlineto{\pgfqpoint{5.133716in}{0.413320in}}%
\pgfpathlineto{\pgfqpoint{5.130953in}{0.413320in}}%
\pgfpathlineto{\pgfqpoint{5.128421in}{0.413320in}}%
\pgfpathlineto{\pgfqpoint{5.125599in}{0.413320in}}%
\pgfpathlineto{\pgfqpoint{5.123042in}{0.413320in}}%
\pgfpathlineto{\pgfqpoint{5.120243in}{0.413320in}}%
\pgfpathlineto{\pgfqpoint{5.117550in}{0.413320in}}%
\pgfpathlineto{\pgfqpoint{5.114887in}{0.413320in}}%
\pgfpathlineto{\pgfqpoint{5.112209in}{0.413320in}}%
\pgfpathlineto{\pgfqpoint{5.109530in}{0.413320in}}%
\pgfpathlineto{\pgfqpoint{5.106842in}{0.413320in}}%
\pgfpathlineto{\pgfqpoint{5.104312in}{0.413320in}}%
\pgfpathlineto{\pgfqpoint{5.101496in}{0.413320in}}%
\pgfpathlineto{\pgfqpoint{5.098948in}{0.413320in}}%
\pgfpathlineto{\pgfqpoint{5.096142in}{0.413320in}}%
\pgfpathlineto{\pgfqpoint{5.093579in}{0.413320in}}%
\pgfpathlineto{\pgfqpoint{5.090788in}{0.413320in}}%
\pgfpathlineto{\pgfqpoint{5.088103in}{0.413320in}}%
\pgfpathlineto{\pgfqpoint{5.085426in}{0.413320in}}%
\pgfpathlineto{\pgfqpoint{5.082746in}{0.413320in}}%
\pgfpathlineto{\pgfqpoint{5.080067in}{0.413320in}}%
\pgfpathlineto{\pgfqpoint{5.077390in}{0.413320in}}%
\pgfpathlineto{\pgfqpoint{5.074851in}{0.413320in}}%
\pgfpathlineto{\pgfqpoint{5.072030in}{0.413320in}}%
\pgfpathlineto{\pgfqpoint{5.069463in}{0.413320in}}%
\pgfpathlineto{\pgfqpoint{5.066677in}{0.413320in}}%
\pgfpathlineto{\pgfqpoint{5.064144in}{0.413320in}}%
\pgfpathlineto{\pgfqpoint{5.061315in}{0.413320in}}%
\pgfpathlineto{\pgfqpoint{5.058711in}{0.413320in}}%
\pgfpathlineto{\pgfqpoint{5.055952in}{0.413320in}}%
\pgfpathlineto{\pgfqpoint{5.053284in}{0.413320in}}%
\pgfpathlineto{\pgfqpoint{5.050606in}{0.413320in}}%
\pgfpathlineto{\pgfqpoint{5.047924in}{0.413320in}}%
\pgfpathlineto{\pgfqpoint{5.045249in}{0.413320in}}%
\pgfpathlineto{\pgfqpoint{5.042572in}{0.413320in}}%
\pgfpathlineto{\pgfqpoint{5.039962in}{0.413320in}}%
\pgfpathlineto{\pgfqpoint{5.037214in}{0.413320in}}%
\pgfpathlineto{\pgfqpoint{5.034649in}{0.413320in}}%
\pgfpathlineto{\pgfqpoint{5.031849in}{0.413320in}}%
\pgfpathlineto{\pgfqpoint{5.029275in}{0.413320in}}%
\pgfpathlineto{\pgfqpoint{5.026501in}{0.413320in}}%
\pgfpathlineto{\pgfqpoint{5.023927in}{0.413320in}}%
\pgfpathlineto{\pgfqpoint{5.021147in}{0.413320in}}%
\pgfpathlineto{\pgfqpoint{5.018466in}{0.413320in}}%
\pgfpathlineto{\pgfqpoint{5.015820in}{0.413320in}}%
\pgfpathlineto{\pgfqpoint{5.013104in}{0.413320in}}%
\pgfpathlineto{\pgfqpoint{5.010562in}{0.413320in}}%
\pgfpathlineto{\pgfqpoint{5.007751in}{0.413320in}}%
\pgfpathlineto{\pgfqpoint{5.005178in}{0.413320in}}%
\pgfpathlineto{\pgfqpoint{5.002384in}{0.413320in}}%
\pgfpathlineto{\pgfqpoint{4.999780in}{0.413320in}}%
\pgfpathlineto{\pgfqpoint{4.997028in}{0.413320in}}%
\pgfpathlineto{\pgfqpoint{4.994390in}{0.413320in}}%
\pgfpathlineto{\pgfqpoint{4.991683in}{0.413320in}}%
\pgfpathlineto{\pgfqpoint{4.989001in}{0.413320in}}%
\pgfpathlineto{\pgfqpoint{4.986325in}{0.413320in}}%
\pgfpathlineto{\pgfqpoint{4.983637in}{0.413320in}}%
\pgfpathlineto{\pgfqpoint{4.980967in}{0.413320in}}%
\pgfpathlineto{\pgfqpoint{4.978287in}{0.413320in}}%
\pgfpathlineto{\pgfqpoint{4.975703in}{0.413320in}}%
\pgfpathlineto{\pgfqpoint{4.972933in}{0.413320in}}%
\pgfpathlineto{\pgfqpoint{4.970314in}{0.413320in}}%
\pgfpathlineto{\pgfqpoint{4.967575in}{0.413320in}}%
\pgfpathlineto{\pgfqpoint{4.965002in}{0.413320in}}%
\pgfpathlineto{\pgfqpoint{4.962219in}{0.413320in}}%
\pgfpathlineto{\pgfqpoint{4.959689in}{0.413320in}}%
\pgfpathlineto{\pgfqpoint{4.956862in}{0.413320in}}%
\pgfpathlineto{\pgfqpoint{4.954182in}{0.413320in}}%
\pgfpathlineto{\pgfqpoint{4.951504in}{0.413320in}}%
\pgfpathlineto{\pgfqpoint{4.948827in}{0.413320in}}%
\pgfpathlineto{\pgfqpoint{4.946151in}{0.413320in}}%
\pgfpathlineto{\pgfqpoint{4.943466in}{0.413320in}}%
\pgfpathlineto{\pgfqpoint{4.940881in}{0.413320in}}%
\pgfpathlineto{\pgfqpoint{4.938112in}{0.413320in}}%
\pgfpathlineto{\pgfqpoint{4.935515in}{0.413320in}}%
\pgfpathlineto{\pgfqpoint{4.932742in}{0.413320in}}%
\pgfpathlineto{\pgfqpoint{4.930170in}{0.413320in}}%
\pgfpathlineto{\pgfqpoint{4.927400in}{0.413320in}}%
\pgfpathlineto{\pgfqpoint{4.924708in}{0.413320in}}%
\pgfpathlineto{\pgfqpoint{4.922041in}{0.413320in}}%
\pgfpathlineto{\pgfqpoint{4.919352in}{0.413320in}}%
\pgfpathlineto{\pgfqpoint{4.916681in}{0.413320in}}%
\pgfpathlineto{\pgfqpoint{4.914009in}{0.413320in}}%
\pgfpathlineto{\pgfqpoint{4.911435in}{0.413320in}}%
\pgfpathlineto{\pgfqpoint{4.908648in}{0.413320in}}%
\pgfpathlineto{\pgfqpoint{4.906096in}{0.413320in}}%
\pgfpathlineto{\pgfqpoint{4.903295in}{0.413320in}}%
\pgfpathlineto{\pgfqpoint{4.900712in}{0.413320in}}%
\pgfpathlineto{\pgfqpoint{4.897938in}{0.413320in}}%
\pgfpathlineto{\pgfqpoint{4.895399in}{0.413320in}}%
\pgfpathlineto{\pgfqpoint{4.892611in}{0.413320in}}%
\pgfpathlineto{\pgfqpoint{4.889902in}{0.413320in}}%
\pgfpathlineto{\pgfqpoint{4.887211in}{0.413320in}}%
\pgfpathlineto{\pgfqpoint{4.884540in}{0.413320in}}%
\pgfpathlineto{\pgfqpoint{4.881864in}{0.413320in}}%
\pgfpathlineto{\pgfqpoint{4.879180in}{0.413320in}}%
\pgfpathlineto{\pgfqpoint{4.876636in}{0.413320in}}%
\pgfpathlineto{\pgfqpoint{4.873832in}{0.413320in}}%
\pgfpathlineto{\pgfqpoint{4.871209in}{0.413320in}}%
\pgfpathlineto{\pgfqpoint{4.868474in}{0.413320in}}%
\pgfpathlineto{\pgfqpoint{4.865910in}{0.413320in}}%
\pgfpathlineto{\pgfqpoint{4.863116in}{0.413320in}}%
\pgfpathlineto{\pgfqpoint{4.860544in}{0.413320in}}%
\pgfpathlineto{\pgfqpoint{4.857807in}{0.413320in}}%
\pgfpathlineto{\pgfqpoint{4.855070in}{0.413320in}}%
\pgfpathlineto{\pgfqpoint{4.852404in}{0.413320in}}%
\pgfpathlineto{\pgfqpoint{4.849715in}{0.413320in}}%
\pgfpathlineto{\pgfqpoint{4.847127in}{0.413320in}}%
\pgfpathlineto{\pgfqpoint{4.844361in}{0.413320in}}%
\pgfpathlineto{\pgfqpoint{4.842380in}{0.413320in}}%
\pgfpathlineto{\pgfqpoint{4.839922in}{0.413320in}}%
\pgfpathlineto{\pgfqpoint{4.837992in}{0.413320in}}%
\pgfpathlineto{\pgfqpoint{4.833657in}{0.413320in}}%
\pgfpathlineto{\pgfqpoint{4.831045in}{0.413320in}}%
\pgfpathlineto{\pgfqpoint{4.828291in}{0.413320in}}%
\pgfpathlineto{\pgfqpoint{4.825619in}{0.413320in}}%
\pgfpathlineto{\pgfqpoint{4.822945in}{0.413320in}}%
\pgfpathlineto{\pgfqpoint{4.820265in}{0.413320in}}%
\pgfpathlineto{\pgfqpoint{4.817587in}{0.413320in}}%
\pgfpathlineto{\pgfqpoint{4.814907in}{0.413320in}}%
\pgfpathlineto{\pgfqpoint{4.812377in}{0.413320in}}%
\pgfpathlineto{\pgfqpoint{4.809538in}{0.413320in}}%
\pgfpathlineto{\pgfqpoint{4.807017in}{0.413320in}}%
\pgfpathlineto{\pgfqpoint{4.804193in}{0.413320in}}%
\pgfpathlineto{\pgfqpoint{4.801586in}{0.413320in}}%
\pgfpathlineto{\pgfqpoint{4.798830in}{0.413320in}}%
\pgfpathlineto{\pgfqpoint{4.796274in}{0.413320in}}%
\pgfpathlineto{\pgfqpoint{4.793512in}{0.413320in}}%
\pgfpathlineto{\pgfqpoint{4.790798in}{0.413320in}}%
\pgfpathlineto{\pgfqpoint{4.788116in}{0.413320in}}%
\pgfpathlineto{\pgfqpoint{4.785445in}{0.413320in}}%
\pgfpathlineto{\pgfqpoint{4.782872in}{0.413320in}}%
\pgfpathlineto{\pgfqpoint{4.780083in}{0.413320in}}%
\pgfpathlineto{\pgfqpoint{4.777535in}{0.413320in}}%
\pgfpathlineto{\pgfqpoint{4.774732in}{0.413320in}}%
\pgfpathlineto{\pgfqpoint{4.772198in}{0.413320in}}%
\pgfpathlineto{\pgfqpoint{4.769367in}{0.413320in}}%
\pgfpathlineto{\pgfqpoint{4.766783in}{0.413320in}}%
\pgfpathlineto{\pgfqpoint{4.764018in}{0.413320in}}%
\pgfpathlineto{\pgfqpoint{4.761337in}{0.413320in}}%
\pgfpathlineto{\pgfqpoint{4.758653in}{0.413320in}}%
\pgfpathlineto{\pgfqpoint{4.755983in}{0.413320in}}%
\pgfpathlineto{\pgfqpoint{4.753298in}{0.413320in}}%
\pgfpathlineto{\pgfqpoint{4.750627in}{0.413320in}}%
\pgfpathlineto{\pgfqpoint{4.748081in}{0.413320in}}%
\pgfpathlineto{\pgfqpoint{4.745256in}{0.413320in}}%
\pgfpathlineto{\pgfqpoint{4.742696in}{0.413320in}}%
\pgfpathlineto{\pgfqpoint{4.739912in}{0.413320in}}%
\pgfpathlineto{\pgfqpoint{4.737348in}{0.413320in}}%
\pgfpathlineto{\pgfqpoint{4.734552in}{0.413320in}}%
\pgfpathlineto{\pgfqpoint{4.731901in}{0.413320in}}%
\pgfpathlineto{\pgfqpoint{4.729233in}{0.413320in}}%
\pgfpathlineto{\pgfqpoint{4.726508in}{0.413320in}}%
\pgfpathlineto{\pgfqpoint{4.723873in}{0.413320in}}%
\pgfpathlineto{\pgfqpoint{4.721160in}{0.413320in}}%
\pgfpathlineto{\pgfqpoint{4.718486in}{0.413320in}}%
\pgfpathlineto{\pgfqpoint{4.715806in}{0.413320in}}%
\pgfpathlineto{\pgfqpoint{4.713275in}{0.413320in}}%
\pgfpathlineto{\pgfqpoint{4.710437in}{0.413320in}}%
\pgfpathlineto{\pgfqpoint{4.707824in}{0.413320in}}%
\pgfpathlineto{\pgfqpoint{4.705094in}{0.413320in}}%
\pgfpathlineto{\pgfqpoint{4.702517in}{0.413320in}}%
\pgfpathlineto{\pgfqpoint{4.699734in}{0.413320in}}%
\pgfpathlineto{\pgfqpoint{4.697170in}{0.413320in}}%
\pgfpathlineto{\pgfqpoint{4.694381in}{0.413320in}}%
\pgfpathlineto{\pgfqpoint{4.691694in}{0.413320in}}%
\pgfpathlineto{\pgfqpoint{4.689051in}{0.413320in}}%
\pgfpathlineto{\pgfqpoint{4.686337in}{0.413320in}}%
\pgfpathlineto{\pgfqpoint{4.683799in}{0.413320in}}%
\pgfpathlineto{\pgfqpoint{4.680988in}{0.413320in}}%
\pgfpathlineto{\pgfqpoint{4.678448in}{0.413320in}}%
\pgfpathlineto{\pgfqpoint{4.675619in}{0.413320in}}%
\pgfpathlineto{\pgfqpoint{4.673068in}{0.413320in}}%
\pgfpathlineto{\pgfqpoint{4.670261in}{0.413320in}}%
\pgfpathlineto{\pgfqpoint{4.667764in}{0.413320in}}%
\pgfpathlineto{\pgfqpoint{4.664923in}{0.413320in}}%
\pgfpathlineto{\pgfqpoint{4.662237in}{0.413320in}}%
\pgfpathlineto{\pgfqpoint{4.659590in}{0.413320in}}%
\pgfpathlineto{\pgfqpoint{4.656873in}{0.413320in}}%
\pgfpathlineto{\pgfqpoint{4.654203in}{0.413320in}}%
\pgfpathlineto{\pgfqpoint{4.651524in}{0.413320in}}%
\pgfpathlineto{\pgfqpoint{4.648922in}{0.413320in}}%
\pgfpathlineto{\pgfqpoint{4.646169in}{0.413320in}}%
\pgfpathlineto{\pgfqpoint{4.643628in}{0.413320in}}%
\pgfpathlineto{\pgfqpoint{4.640809in}{0.413320in}}%
\pgfpathlineto{\pgfqpoint{4.638204in}{0.413320in}}%
\pgfpathlineto{\pgfqpoint{4.635445in}{0.413320in}}%
\pgfpathlineto{\pgfqpoint{4.632902in}{0.413320in}}%
\pgfpathlineto{\pgfqpoint{4.630096in}{0.413320in}}%
\pgfpathlineto{\pgfqpoint{4.627411in}{0.413320in}}%
\pgfpathlineto{\pgfqpoint{4.624741in}{0.413320in}}%
\pgfpathlineto{\pgfqpoint{4.622056in}{0.413320in}}%
\pgfpathlineto{\pgfqpoint{4.619529in}{0.413320in}}%
\pgfpathlineto{\pgfqpoint{4.616702in}{0.413320in}}%
\pgfpathlineto{\pgfqpoint{4.614134in}{0.413320in}}%
\pgfpathlineto{\pgfqpoint{4.611350in}{0.413320in}}%
\pgfpathlineto{\pgfqpoint{4.608808in}{0.413320in}}%
\pgfpathlineto{\pgfqpoint{4.605990in}{0.413320in}}%
\pgfpathlineto{\pgfqpoint{4.603430in}{0.413320in}}%
\pgfpathlineto{\pgfqpoint{4.600633in}{0.413320in}}%
\pgfpathlineto{\pgfqpoint{4.597951in}{0.413320in}}%
\pgfpathlineto{\pgfqpoint{4.595281in}{0.413320in}}%
\pgfpathlineto{\pgfqpoint{4.592589in}{0.413320in}}%
\pgfpathlineto{\pgfqpoint{4.589920in}{0.413320in}}%
\pgfpathlineto{\pgfqpoint{4.587244in}{0.413320in}}%
\pgfpathlineto{\pgfqpoint{4.584672in}{0.413320in}}%
\pgfpathlineto{\pgfqpoint{4.581888in}{0.413320in}}%
\pgfpathlineto{\pgfqpoint{4.579305in}{0.413320in}}%
\pgfpathlineto{\pgfqpoint{4.576531in}{0.413320in}}%
\pgfpathlineto{\pgfqpoint{4.573947in}{0.413320in}}%
\pgfpathlineto{\pgfqpoint{4.571171in}{0.413320in}}%
\pgfpathlineto{\pgfqpoint{4.568612in}{0.413320in}}%
\pgfpathlineto{\pgfqpoint{4.565820in}{0.413320in}}%
\pgfpathlineto{\pgfqpoint{4.563125in}{0.413320in}}%
\pgfpathlineto{\pgfqpoint{4.560448in}{0.413320in}}%
\pgfpathlineto{\pgfqpoint{4.557777in}{0.413320in}}%
\pgfpathlineto{\pgfqpoint{4.555106in}{0.413320in}}%
\pgfpathlineto{\pgfqpoint{4.552425in}{0.413320in}}%
\pgfpathlineto{\pgfqpoint{4.549822in}{0.413320in}}%
\pgfpathlineto{\pgfqpoint{4.547064in}{0.413320in}}%
\pgfpathlineto{\pgfqpoint{4.544464in}{0.413320in}}%
\pgfpathlineto{\pgfqpoint{4.541711in}{0.413320in}}%
\pgfpathlineto{\pgfqpoint{4.539144in}{0.413320in}}%
\pgfpathlineto{\pgfqpoint{4.536400in}{0.413320in}}%
\pgfpathlineto{\pgfqpoint{4.533764in}{0.413320in}}%
\pgfpathlineto{\pgfqpoint{4.530990in}{0.413320in}}%
\pgfpathlineto{\pgfqpoint{4.528307in}{0.413320in}}%
\pgfpathlineto{\pgfqpoint{4.525640in}{0.413320in}}%
\pgfpathlineto{\pgfqpoint{4.522962in}{0.413320in}}%
\pgfpathlineto{\pgfqpoint{4.520345in}{0.413320in}}%
\pgfpathlineto{\pgfqpoint{4.517598in}{0.413320in}}%
\pgfpathlineto{\pgfqpoint{4.515080in}{0.413320in}}%
\pgfpathlineto{\pgfqpoint{4.512246in}{0.413320in}}%
\pgfpathlineto{\pgfqpoint{4.509643in}{0.413320in}}%
\pgfpathlineto{\pgfqpoint{4.506893in}{0.413320in}}%
\pgfpathlineto{\pgfqpoint{4.504305in}{0.413320in}}%
\pgfpathlineto{\pgfqpoint{4.501529in}{0.413320in}}%
\pgfpathlineto{\pgfqpoint{4.498850in}{0.413320in}}%
\pgfpathlineto{\pgfqpoint{4.496167in}{0.413320in}}%
\pgfpathlineto{\pgfqpoint{4.493492in}{0.413320in}}%
\pgfpathlineto{\pgfqpoint{4.490822in}{0.413320in}}%
\pgfpathlineto{\pgfqpoint{4.488130in}{0.413320in}}%
\pgfpathlineto{\pgfqpoint{4.485581in}{0.413320in}}%
\pgfpathlineto{\pgfqpoint{4.482778in}{0.413320in}}%
\pgfpathlineto{\pgfqpoint{4.480201in}{0.413320in}}%
\pgfpathlineto{\pgfqpoint{4.477430in}{0.413320in}}%
\pgfpathlineto{\pgfqpoint{4.474861in}{0.413320in}}%
\pgfpathlineto{\pgfqpoint{4.472059in}{0.413320in}}%
\pgfpathlineto{\pgfqpoint{4.469492in}{0.413320in}}%
\pgfpathlineto{\pgfqpoint{4.466717in}{0.413320in}}%
\pgfpathlineto{\pgfqpoint{4.464029in}{0.413320in}}%
\pgfpathlineto{\pgfqpoint{4.461367in}{0.413320in}}%
\pgfpathlineto{\pgfqpoint{4.458681in}{0.413320in}}%
\pgfpathlineto{\pgfqpoint{4.456138in}{0.413320in}}%
\pgfpathlineto{\pgfqpoint{4.453312in}{0.413320in}}%
\pgfpathlineto{\pgfqpoint{4.450767in}{0.413320in}}%
\pgfpathlineto{\pgfqpoint{4.447965in}{0.413320in}}%
\pgfpathlineto{\pgfqpoint{4.445423in}{0.413320in}}%
\pgfpathlineto{\pgfqpoint{4.442611in}{0.413320in}}%
\pgfpathlineto{\pgfqpoint{4.440041in}{0.413320in}}%
\pgfpathlineto{\pgfqpoint{4.437253in}{0.413320in}}%
\pgfpathlineto{\pgfqpoint{4.434569in}{0.413320in}}%
\pgfpathlineto{\pgfqpoint{4.431901in}{0.413320in}}%
\pgfpathlineto{\pgfqpoint{4.429220in}{0.413320in}}%
\pgfpathlineto{\pgfqpoint{4.426534in}{0.413320in}}%
\pgfpathlineto{\pgfqpoint{4.423863in}{0.413320in}}%
\pgfpathlineto{\pgfqpoint{4.421292in}{0.413320in}}%
\pgfpathlineto{\pgfqpoint{4.418506in}{0.413320in}}%
\pgfpathlineto{\pgfqpoint{4.415932in}{0.413320in}}%
\pgfpathlineto{\pgfqpoint{4.413149in}{0.413320in}}%
\pgfpathlineto{\pgfqpoint{4.410587in}{0.413320in}}%
\pgfpathlineto{\pgfqpoint{4.407788in}{0.413320in}}%
\pgfpathlineto{\pgfqpoint{4.405234in}{0.413320in}}%
\pgfpathlineto{\pgfqpoint{4.402468in}{0.413320in}}%
\pgfpathlineto{\pgfqpoint{4.399745in}{0.413320in}}%
\pgfpathlineto{\pgfqpoint{4.397076in}{0.413320in}}%
\pgfpathlineto{\pgfqpoint{4.394400in}{0.413320in}}%
\pgfpathlineto{\pgfqpoint{4.391721in}{0.413320in}}%
\pgfpathlineto{\pgfqpoint{4.389044in}{0.413320in}}%
\pgfpathlineto{\pgfqpoint{4.386431in}{0.413320in}}%
\pgfpathlineto{\pgfqpoint{4.383674in}{0.413320in}}%
\pgfpathlineto{\pgfqpoint{4.381097in}{0.413320in}}%
\pgfpathlineto{\pgfqpoint{4.378329in}{0.413320in}}%
\pgfpathlineto{\pgfqpoint{4.375761in}{0.413320in}}%
\pgfpathlineto{\pgfqpoint{4.372976in}{0.413320in}}%
\pgfpathlineto{\pgfqpoint{4.370437in}{0.413320in}}%
\pgfpathlineto{\pgfqpoint{4.367646in}{0.413320in}}%
\pgfpathlineto{\pgfqpoint{4.364936in}{0.413320in}}%
\pgfpathlineto{\pgfqpoint{4.362270in}{0.413320in}}%
\pgfpathlineto{\pgfqpoint{4.359582in}{0.413320in}}%
\pgfpathlineto{\pgfqpoint{4.357014in}{0.413320in}}%
\pgfpathlineto{\pgfqpoint{4.354224in}{0.413320in}}%
\pgfpathlineto{\pgfqpoint{4.351645in}{0.413320in}}%
\pgfpathlineto{\pgfqpoint{4.348868in}{0.413320in}}%
\pgfpathlineto{\pgfqpoint{4.346263in}{0.413320in}}%
\pgfpathlineto{\pgfqpoint{4.343510in}{0.413320in}}%
\pgfpathlineto{\pgfqpoint{4.340976in}{0.413320in}}%
\pgfpathlineto{\pgfqpoint{4.338154in}{0.413320in}}%
\pgfpathlineto{\pgfqpoint{4.335463in}{0.413320in}}%
\pgfpathlineto{\pgfqpoint{4.332796in}{0.413320in}}%
\pgfpathlineto{\pgfqpoint{4.330118in}{0.413320in}}%
\pgfpathlineto{\pgfqpoint{4.327440in}{0.413320in}}%
\pgfpathlineto{\pgfqpoint{4.324760in}{0.413320in}}%
\pgfpathlineto{\pgfqpoint{4.322181in}{0.413320in}}%
\pgfpathlineto{\pgfqpoint{4.319405in}{0.413320in}}%
\pgfpathlineto{\pgfqpoint{4.316856in}{0.413320in}}%
\pgfpathlineto{\pgfqpoint{4.314032in}{0.413320in}}%
\pgfpathlineto{\pgfqpoint{4.311494in}{0.413320in}}%
\pgfpathlineto{\pgfqpoint{4.308691in}{0.413320in}}%
\pgfpathlineto{\pgfqpoint{4.306118in}{0.413320in}}%
\pgfpathlineto{\pgfqpoint{4.303357in}{0.413320in}}%
\pgfpathlineto{\pgfqpoint{4.300656in}{0.413320in}}%
\pgfpathlineto{\pgfqpoint{4.297977in}{0.413320in}}%
\pgfpathlineto{\pgfqpoint{4.295299in}{0.413320in}}%
\pgfpathlineto{\pgfqpoint{4.292786in}{0.413320in}}%
\pgfpathlineto{\pgfqpoint{4.289936in}{0.413320in}}%
\pgfpathlineto{\pgfqpoint{4.287399in}{0.413320in}}%
\pgfpathlineto{\pgfqpoint{4.284586in}{0.413320in}}%
\pgfpathlineto{\pgfqpoint{4.282000in}{0.413320in}}%
\pgfpathlineto{\pgfqpoint{4.279212in}{0.413320in}}%
\pgfpathlineto{\pgfqpoint{4.276635in}{0.413320in}}%
\pgfpathlineto{\pgfqpoint{4.273874in}{0.413320in}}%
\pgfpathlineto{\pgfqpoint{4.271187in}{0.413320in}}%
\pgfpathlineto{\pgfqpoint{4.268590in}{0.413320in}}%
\pgfpathlineto{\pgfqpoint{4.265824in}{0.413320in}}%
\pgfpathlineto{\pgfqpoint{4.263157in}{0.413320in}}%
\pgfpathlineto{\pgfqpoint{4.260477in}{0.413320in}}%
\pgfpathlineto{\pgfqpoint{4.257958in}{0.413320in}}%
\pgfpathlineto{\pgfqpoint{4.255120in}{0.413320in}}%
\pgfpathlineto{\pgfqpoint{4.252581in}{0.413320in}}%
\pgfpathlineto{\pgfqpoint{4.249767in}{0.413320in}}%
\pgfpathlineto{\pgfqpoint{4.247225in}{0.413320in}}%
\pgfpathlineto{\pgfqpoint{4.244394in}{0.413320in}}%
\pgfpathlineto{\pgfqpoint{4.241900in}{0.413320in}}%
\pgfpathlineto{\pgfqpoint{4.239084in}{0.413320in}}%
\pgfpathlineto{\pgfqpoint{4.236375in}{0.413320in}}%
\pgfpathlineto{\pgfqpoint{4.233691in}{0.413320in}}%
\pgfpathlineto{\pgfqpoint{4.231013in}{0.413320in}}%
\pgfpathlineto{\pgfqpoint{4.228331in}{0.413320in}}%
\pgfpathlineto{\pgfqpoint{4.225654in}{0.413320in}}%
\pgfpathlineto{\pgfqpoint{4.223082in}{0.413320in}}%
\pgfpathlineto{\pgfqpoint{4.220304in}{0.413320in}}%
\pgfpathlineto{\pgfqpoint{4.217694in}{0.413320in}}%
\pgfpathlineto{\pgfqpoint{4.214948in}{0.413320in}}%
\pgfpathlineto{\pgfqpoint{4.212383in}{0.413320in}}%
\pgfpathlineto{\pgfqpoint{4.209597in}{0.413320in}}%
\pgfpathlineto{\pgfqpoint{4.207076in}{0.413320in}}%
\pgfpathlineto{\pgfqpoint{4.204240in}{0.413320in}}%
\pgfpathlineto{\pgfqpoint{4.201542in}{0.413320in}}%
\pgfpathlineto{\pgfqpoint{4.198878in}{0.413320in}}%
\pgfpathlineto{\pgfqpoint{4.196186in}{0.413320in}}%
\pgfpathlineto{\pgfqpoint{4.193638in}{0.413320in}}%
\pgfpathlineto{\pgfqpoint{4.190842in}{0.413320in}}%
\pgfpathlineto{\pgfqpoint{4.188318in}{0.413320in}}%
\pgfpathlineto{\pgfqpoint{4.185481in}{0.413320in}}%
\pgfpathlineto{\pgfqpoint{4.182899in}{0.413320in}}%
\pgfpathlineto{\pgfqpoint{4.180129in}{0.413320in}}%
\pgfpathlineto{\pgfqpoint{4.177593in}{0.413320in}}%
\pgfpathlineto{\pgfqpoint{4.174770in}{0.413320in}}%
\pgfpathlineto{\pgfqpoint{4.172093in}{0.413320in}}%
\pgfpathlineto{\pgfqpoint{4.169415in}{0.413320in}}%
\pgfpathlineto{\pgfqpoint{4.166737in}{0.413320in}}%
\pgfpathlineto{\pgfqpoint{4.164059in}{0.413320in}}%
\pgfpathlineto{\pgfqpoint{4.161380in}{0.413320in}}%
\pgfpathlineto{\pgfqpoint{4.158806in}{0.413320in}}%
\pgfpathlineto{\pgfqpoint{4.156016in}{0.413320in}}%
\pgfpathlineto{\pgfqpoint{4.153423in}{0.413320in}}%
\pgfpathlineto{\pgfqpoint{4.150665in}{0.413320in}}%
\pgfpathlineto{\pgfqpoint{4.148082in}{0.413320in}}%
\pgfpathlineto{\pgfqpoint{4.145310in}{0.413320in}}%
\pgfpathlineto{\pgfqpoint{4.142713in}{0.413320in}}%
\pgfpathlineto{\pgfqpoint{4.139963in}{0.413320in}}%
\pgfpathlineto{\pgfqpoint{4.137272in}{0.413320in}}%
\pgfpathlineto{\pgfqpoint{4.134615in}{0.413320in}}%
\pgfpathlineto{\pgfqpoint{4.131920in}{0.413320in}}%
\pgfpathlineto{\pgfqpoint{4.129349in}{0.413320in}}%
\pgfpathlineto{\pgfqpoint{4.126553in}{0.413320in}}%
\pgfpathlineto{\pgfqpoint{4.124019in}{0.413320in}}%
\pgfpathlineto{\pgfqpoint{4.121205in}{0.413320in}}%
\pgfpathlineto{\pgfqpoint{4.118554in}{0.413320in}}%
\pgfpathlineto{\pgfqpoint{4.115844in}{0.413320in}}%
\pgfpathlineto{\pgfqpoint{4.113252in}{0.413320in}}%
\pgfpathlineto{\pgfqpoint{4.110488in}{0.413320in}}%
\pgfpathlineto{\pgfqpoint{4.107814in}{0.413320in}}%
\pgfpathlineto{\pgfqpoint{4.105185in}{0.413320in}}%
\pgfpathlineto{\pgfqpoint{4.102456in}{0.413320in}}%
\pgfpathlineto{\pgfqpoint{4.099777in}{0.413320in}}%
\pgfpathlineto{\pgfqpoint{4.097092in}{0.413320in}}%
\pgfpathlineto{\pgfqpoint{4.094527in}{0.413320in}}%
\pgfpathlineto{\pgfqpoint{4.091729in}{0.413320in}}%
\pgfpathlineto{\pgfqpoint{4.089159in}{0.413320in}}%
\pgfpathlineto{\pgfqpoint{4.086385in}{0.413320in}}%
\pgfpathlineto{\pgfqpoint{4.083870in}{0.413320in}}%
\pgfpathlineto{\pgfqpoint{4.081018in}{0.413320in}}%
\pgfpathlineto{\pgfqpoint{4.078471in}{0.413320in}}%
\pgfpathlineto{\pgfqpoint{4.075705in}{0.413320in}}%
\pgfpathlineto{\pgfqpoint{4.072985in}{0.413320in}}%
\pgfpathlineto{\pgfqpoint{4.070313in}{0.413320in}}%
\pgfpathlineto{\pgfqpoint{4.067636in}{0.413320in}}%
\pgfpathlineto{\pgfqpoint{4.064957in}{0.413320in}}%
\pgfpathlineto{\pgfqpoint{4.062266in}{0.413320in}}%
\pgfpathlineto{\pgfqpoint{4.059702in}{0.413320in}}%
\pgfpathlineto{\pgfqpoint{4.056911in}{0.413320in}}%
\pgfpathlineto{\pgfqpoint{4.054326in}{0.413320in}}%
\pgfpathlineto{\pgfqpoint{4.051557in}{0.413320in}}%
\pgfpathlineto{\pgfqpoint{4.049006in}{0.413320in}}%
\pgfpathlineto{\pgfqpoint{4.046210in}{0.413320in}}%
\pgfpathlineto{\pgfqpoint{4.043667in}{0.413320in}}%
\pgfpathlineto{\pgfqpoint{4.040852in}{0.413320in}}%
\pgfpathlineto{\pgfqpoint{4.038174in}{0.413320in}}%
\pgfpathlineto{\pgfqpoint{4.035492in}{0.413320in}}%
\pgfpathlineto{\pgfqpoint{4.032817in}{0.413320in}}%
\pgfpathlineto{\pgfqpoint{4.030229in}{0.413320in}}%
\pgfpathlineto{\pgfqpoint{4.027447in}{0.413320in}}%
\pgfpathlineto{\pgfqpoint{4.024868in}{0.413320in}}%
\pgfpathlineto{\pgfqpoint{4.022097in}{0.413320in}}%
\pgfpathlineto{\pgfqpoint{4.019518in}{0.413320in}}%
\pgfpathlineto{\pgfqpoint{4.016744in}{0.413320in}}%
\pgfpathlineto{\pgfqpoint{4.014186in}{0.413320in}}%
\pgfpathlineto{\pgfqpoint{4.011394in}{0.413320in}}%
\pgfpathlineto{\pgfqpoint{4.008699in}{0.413320in}}%
\pgfpathlineto{\pgfqpoint{4.006034in}{0.413320in}}%
\pgfpathlineto{\pgfqpoint{4.003348in}{0.413320in}}%
\pgfpathlineto{\pgfqpoint{4.000674in}{0.413320in}}%
\pgfpathlineto{\pgfqpoint{3.997990in}{0.413320in}}%
\pgfpathlineto{\pgfqpoint{3.995417in}{0.413320in}}%
\pgfpathlineto{\pgfqpoint{3.992642in}{0.413320in}}%
\pgfpathlineto{\pgfqpoint{3.990055in}{0.413320in}}%
\pgfpathlineto{\pgfqpoint{3.987270in}{0.413320in}}%
\pgfpathlineto{\pgfqpoint{3.984714in}{0.413320in}}%
\pgfpathlineto{\pgfqpoint{3.981929in}{0.413320in}}%
\pgfpathlineto{\pgfqpoint{3.979389in}{0.413320in}}%
\pgfpathlineto{\pgfqpoint{3.976563in}{0.413320in}}%
\pgfpathlineto{\pgfqpoint{3.973885in}{0.413320in}}%
\pgfpathlineto{\pgfqpoint{3.971250in}{0.413320in}}%
\pgfpathlineto{\pgfqpoint{3.968523in}{0.413320in}}%
\pgfpathlineto{\pgfqpoint{3.966013in}{0.413320in}}%
\pgfpathlineto{\pgfqpoint{3.963176in}{0.413320in}}%
\pgfpathlineto{\pgfqpoint{3.960635in}{0.413320in}}%
\pgfpathlineto{\pgfqpoint{3.957823in}{0.413320in}}%
\pgfpathlineto{\pgfqpoint{3.955211in}{0.413320in}}%
\pgfpathlineto{\pgfqpoint{3.952464in}{0.413320in}}%
\pgfpathlineto{\pgfqpoint{3.949894in}{0.413320in}}%
\pgfpathlineto{\pgfqpoint{3.947101in}{0.413320in}}%
\pgfpathlineto{\pgfqpoint{3.944431in}{0.413320in}}%
\pgfpathlineto{\pgfqpoint{3.941778in}{0.413320in}}%
\pgfpathlineto{\pgfqpoint{3.939075in}{0.413320in}}%
\pgfpathlineto{\pgfqpoint{3.936395in}{0.413320in}}%
\pgfpathlineto{\pgfqpoint{3.933714in}{0.413320in}}%
\pgfpathlineto{\pgfqpoint{3.931202in}{0.413320in}}%
\pgfpathlineto{\pgfqpoint{3.928347in}{0.413320in}}%
\pgfpathlineto{\pgfqpoint{3.925778in}{0.413320in}}%
\pgfpathlineto{\pgfqpoint{3.923005in}{0.413320in}}%
\pgfpathlineto{\pgfqpoint{3.920412in}{0.413320in}}%
\pgfpathlineto{\pgfqpoint{3.917646in}{0.413320in}}%
\pgfpathlineto{\pgfqpoint{3.915107in}{0.413320in}}%
\pgfpathlineto{\pgfqpoint{3.912296in}{0.413320in}}%
\pgfpathlineto{\pgfqpoint{3.909602in}{0.413320in}}%
\pgfpathlineto{\pgfqpoint{3.906918in}{0.413320in}}%
\pgfpathlineto{\pgfqpoint{3.904252in}{0.413320in}}%
\pgfpathlineto{\pgfqpoint{3.901573in}{0.413320in}}%
\pgfpathlineto{\pgfqpoint{3.898891in}{0.413320in}}%
\pgfpathlineto{\pgfqpoint{3.896345in}{0.413320in}}%
\pgfpathlineto{\pgfqpoint{3.893541in}{0.413320in}}%
\pgfpathlineto{\pgfqpoint{3.890926in}{0.413320in}}%
\pgfpathlineto{\pgfqpoint{3.888188in}{0.413320in}}%
\pgfpathlineto{\pgfqpoint{3.885621in}{0.413320in}}%
\pgfpathlineto{\pgfqpoint{3.882850in}{0.413320in}}%
\pgfpathlineto{\pgfqpoint{3.880237in}{0.413320in}}%
\pgfpathlineto{\pgfqpoint{3.877466in}{0.413320in}}%
\pgfpathlineto{\pgfqpoint{3.874790in}{0.413320in}}%
\pgfpathlineto{\pgfqpoint{3.872114in}{0.413320in}}%
\pgfpathlineto{\pgfqpoint{3.869435in}{0.413320in}}%
\pgfpathlineto{\pgfqpoint{3.866815in}{0.413320in}}%
\pgfpathlineto{\pgfqpoint{3.864073in}{0.413320in}}%
\pgfpathlineto{\pgfqpoint{3.861561in}{0.413320in}}%
\pgfpathlineto{\pgfqpoint{3.858720in}{0.413320in}}%
\pgfpathlineto{\pgfqpoint{3.856100in}{0.413320in}}%
\pgfpathlineto{\pgfqpoint{3.853358in}{0.413320in}}%
\pgfpathlineto{\pgfqpoint{3.850814in}{0.413320in}}%
\pgfpathlineto{\pgfqpoint{3.848005in}{0.413320in}}%
\pgfpathlineto{\pgfqpoint{3.845329in}{0.413320in}}%
\pgfpathlineto{\pgfqpoint{3.842641in}{0.413320in}}%
\pgfpathlineto{\pgfqpoint{3.839960in}{0.413320in}}%
\pgfpathlineto{\pgfqpoint{3.837286in}{0.413320in}}%
\pgfpathlineto{\pgfqpoint{3.834616in}{0.413320in}}%
\pgfpathlineto{\pgfqpoint{3.832053in}{0.413320in}}%
\pgfpathlineto{\pgfqpoint{3.829252in}{0.413320in}}%
\pgfpathlineto{\pgfqpoint{3.826679in}{0.413320in}}%
\pgfpathlineto{\pgfqpoint{3.823903in}{0.413320in}}%
\pgfpathlineto{\pgfqpoint{3.821315in}{0.413320in}}%
\pgfpathlineto{\pgfqpoint{3.818546in}{0.413320in}}%
\pgfpathlineto{\pgfqpoint{3.815983in}{0.413320in}}%
\pgfpathlineto{\pgfqpoint{3.813172in}{0.413320in}}%
\pgfpathlineto{\pgfqpoint{3.810510in}{0.413320in}}%
\pgfpathlineto{\pgfqpoint{3.807832in}{0.413320in}}%
\pgfpathlineto{\pgfqpoint{3.805145in}{0.413320in}}%
\pgfpathlineto{\pgfqpoint{3.802569in}{0.413320in}}%
\pgfpathlineto{\pgfqpoint{3.799797in}{0.413320in}}%
\pgfpathlineto{\pgfqpoint{3.797265in}{0.413320in}}%
\pgfpathlineto{\pgfqpoint{3.794435in}{0.413320in}}%
\pgfpathlineto{\pgfqpoint{3.791897in}{0.413320in}}%
\pgfpathlineto{\pgfqpoint{3.789084in}{0.413320in}}%
\pgfpathlineto{\pgfqpoint{3.786504in}{0.413320in}}%
\pgfpathlineto{\pgfqpoint{3.783725in}{0.413320in}}%
\pgfpathlineto{\pgfqpoint{3.781046in}{0.413320in}}%
\pgfpathlineto{\pgfqpoint{3.778370in}{0.413320in}}%
\pgfpathlineto{\pgfqpoint{3.775691in}{0.413320in}}%
\pgfpathlineto{\pgfqpoint{3.773014in}{0.413320in}}%
\pgfpathlineto{\pgfqpoint{3.770323in}{0.413320in}}%
\pgfpathlineto{\pgfqpoint{3.767782in}{0.413320in}}%
\pgfpathlineto{\pgfqpoint{3.764966in}{0.413320in}}%
\pgfpathlineto{\pgfqpoint{3.762389in}{0.413320in}}%
\pgfpathlineto{\pgfqpoint{3.759622in}{0.413320in}}%
\pgfpathlineto{\pgfqpoint{3.757065in}{0.413320in}}%
\pgfpathlineto{\pgfqpoint{3.754265in}{0.413320in}}%
\pgfpathlineto{\pgfqpoint{3.751728in}{0.413320in}}%
\pgfpathlineto{\pgfqpoint{3.748903in}{0.413320in}}%
\pgfpathlineto{\pgfqpoint{3.746229in}{0.413320in}}%
\pgfpathlineto{\pgfqpoint{3.743548in}{0.413320in}}%
\pgfpathlineto{\pgfqpoint{3.740874in}{0.413320in}}%
\pgfpathlineto{\pgfqpoint{3.738194in}{0.413320in}}%
\pgfpathlineto{\pgfqpoint{3.735509in}{0.413320in}}%
\pgfpathlineto{\pgfqpoint{3.732950in}{0.413320in}}%
\pgfpathlineto{\pgfqpoint{3.730158in}{0.413320in}}%
\pgfpathlineto{\pgfqpoint{3.727581in}{0.413320in}}%
\pgfpathlineto{\pgfqpoint{3.724804in}{0.413320in}}%
\pgfpathlineto{\pgfqpoint{3.722228in}{0.413320in}}%
\pgfpathlineto{\pgfqpoint{3.719446in}{0.413320in}}%
\pgfpathlineto{\pgfqpoint{3.716875in}{0.413320in}}%
\pgfpathlineto{\pgfqpoint{3.714086in}{0.413320in}}%
\pgfpathlineto{\pgfqpoint{3.711410in}{0.413320in}}%
\pgfpathlineto{\pgfqpoint{3.708729in}{0.413320in}}%
\pgfpathlineto{\pgfqpoint{3.706053in}{0.413320in}}%
\pgfpathlineto{\pgfqpoint{3.703460in}{0.413320in}}%
\pgfpathlineto{\pgfqpoint{3.700684in}{0.413320in}}%
\pgfpathlineto{\pgfqpoint{3.698125in}{0.413320in}}%
\pgfpathlineto{\pgfqpoint{3.695331in}{0.413320in}}%
\pgfpathlineto{\pgfqpoint{3.692765in}{0.413320in}}%
\pgfpathlineto{\pgfqpoint{3.689983in}{0.413320in}}%
\pgfpathlineto{\pgfqpoint{3.687442in}{0.413320in}}%
\pgfpathlineto{\pgfqpoint{3.684620in}{0.413320in}}%
\pgfpathlineto{\pgfqpoint{3.681948in}{0.413320in}}%
\pgfpathlineto{\pgfqpoint{3.679273in}{0.413320in}}%
\pgfpathlineto{\pgfqpoint{3.676591in}{0.413320in}}%
\pgfpathlineto{\pgfqpoint{3.673911in}{0.413320in}}%
\pgfpathlineto{\pgfqpoint{3.671232in}{0.413320in}}%
\pgfpathlineto{\pgfqpoint{3.668665in}{0.413320in}}%
\pgfpathlineto{\pgfqpoint{3.665864in}{0.413320in}}%
\pgfpathlineto{\pgfqpoint{3.663276in}{0.413320in}}%
\pgfpathlineto{\pgfqpoint{3.660515in}{0.413320in}}%
\pgfpathlineto{\pgfqpoint{3.657917in}{0.413320in}}%
\pgfpathlineto{\pgfqpoint{3.655165in}{0.413320in}}%
\pgfpathlineto{\pgfqpoint{3.652628in}{0.413320in}}%
\pgfpathlineto{\pgfqpoint{3.649837in}{0.413320in}}%
\pgfpathlineto{\pgfqpoint{3.647130in}{0.413320in}}%
\pgfpathlineto{\pgfqpoint{3.644452in}{0.413320in}}%
\pgfpathlineto{\pgfqpoint{3.641773in}{0.413320in}}%
\pgfpathlineto{\pgfqpoint{3.639207in}{0.413320in}}%
\pgfpathlineto{\pgfqpoint{3.636413in}{0.413320in}}%
\pgfpathlineto{\pgfqpoint{3.633858in}{0.413320in}}%
\pgfpathlineto{\pgfqpoint{3.631058in}{0.413320in}}%
\pgfpathlineto{\pgfqpoint{3.628460in}{0.413320in}}%
\pgfpathlineto{\pgfqpoint{3.625689in}{0.413320in}}%
\pgfpathlineto{\pgfqpoint{3.623165in}{0.413320in}}%
\pgfpathlineto{\pgfqpoint{3.620345in}{0.413320in}}%
\pgfpathlineto{\pgfqpoint{3.617667in}{0.413320in}}%
\pgfpathlineto{\pgfqpoint{3.614982in}{0.413320in}}%
\pgfpathlineto{\pgfqpoint{3.612311in}{0.413320in}}%
\pgfpathlineto{\pgfqpoint{3.609632in}{0.413320in}}%
\pgfpathlineto{\pgfqpoint{3.606951in}{0.413320in}}%
\pgfpathlineto{\pgfqpoint{3.604387in}{0.413320in}}%
\pgfpathlineto{\pgfqpoint{3.601590in}{0.413320in}}%
\pgfpathlineto{\pgfqpoint{3.598998in}{0.413320in}}%
\pgfpathlineto{\pgfqpoint{3.596240in}{0.413320in}}%
\pgfpathlineto{\pgfqpoint{3.593620in}{0.413320in}}%
\pgfpathlineto{\pgfqpoint{3.590883in}{0.413320in}}%
\pgfpathlineto{\pgfqpoint{3.588258in}{0.413320in}}%
\pgfpathlineto{\pgfqpoint{3.585532in}{0.413320in}}%
\pgfpathlineto{\pgfqpoint{3.582851in}{0.413320in}}%
\pgfpathlineto{\pgfqpoint{3.580191in}{0.413320in}}%
\pgfpathlineto{\pgfqpoint{3.577487in}{0.413320in}}%
\pgfpathlineto{\pgfqpoint{3.574814in}{0.413320in}}%
\pgfpathlineto{\pgfqpoint{3.572126in}{0.413320in}}%
\pgfpathlineto{\pgfqpoint{3.569584in}{0.413320in}}%
\pgfpathlineto{\pgfqpoint{3.566774in}{0.413320in}}%
\pgfpathlineto{\pgfqpoint{3.564188in}{0.413320in}}%
\pgfpathlineto{\pgfqpoint{3.561420in}{0.413320in}}%
\pgfpathlineto{\pgfqpoint{3.558853in}{0.413320in}}%
\pgfpathlineto{\pgfqpoint{3.556061in}{0.413320in}}%
\pgfpathlineto{\pgfqpoint{3.553498in}{0.413320in}}%
\pgfpathlineto{\pgfqpoint{3.550713in}{0.413320in}}%
\pgfpathlineto{\pgfqpoint{3.548029in}{0.413320in}}%
\pgfpathlineto{\pgfqpoint{3.545349in}{0.413320in}}%
\pgfpathlineto{\pgfqpoint{3.542656in}{0.413320in}}%
\pgfpathlineto{\pgfqpoint{3.540093in}{0.413320in}}%
\pgfpathlineto{\pgfqpoint{3.537309in}{0.413320in}}%
\pgfpathlineto{\pgfqpoint{3.534783in}{0.413320in}}%
\pgfpathlineto{\pgfqpoint{3.531955in}{0.413320in}}%
\pgfpathlineto{\pgfqpoint{3.529327in}{0.413320in}}%
\pgfpathlineto{\pgfqpoint{3.526601in}{0.413320in}}%
\pgfpathlineto{\pgfqpoint{3.524041in}{0.413320in}}%
\pgfpathlineto{\pgfqpoint{3.521244in}{0.413320in}}%
\pgfpathlineto{\pgfqpoint{3.518565in}{0.413320in}}%
\pgfpathlineto{\pgfqpoint{3.515884in}{0.413320in}}%
\pgfpathlineto{\pgfqpoint{3.513209in}{0.413320in}}%
\pgfpathlineto{\pgfqpoint{3.510533in}{0.413320in}}%
\pgfpathlineto{\pgfqpoint{3.507840in}{0.413320in}}%
\pgfpathlineto{\pgfqpoint{3.505262in}{0.413320in}}%
\pgfpathlineto{\pgfqpoint{3.502488in}{0.413320in}}%
\pgfpathlineto{\pgfqpoint{3.499909in}{0.413320in}}%
\pgfpathlineto{\pgfqpoint{3.497139in}{0.413320in}}%
\pgfpathlineto{\pgfqpoint{3.494581in}{0.413320in}}%
\pgfpathlineto{\pgfqpoint{3.491783in}{0.413320in}}%
\pgfpathlineto{\pgfqpoint{3.489223in}{0.413320in}}%
\pgfpathlineto{\pgfqpoint{3.486442in}{0.413320in}}%
\pgfpathlineto{\pgfqpoint{3.483744in}{0.413320in}}%
\pgfpathlineto{\pgfqpoint{3.481072in}{0.413320in}}%
\pgfpathlineto{\pgfqpoint{3.478378in}{0.413320in}}%
\pgfpathlineto{\pgfqpoint{3.475821in}{0.413320in}}%
\pgfpathlineto{\pgfqpoint{3.473021in}{0.413320in}}%
\pgfpathlineto{\pgfqpoint{3.470466in}{0.413320in}}%
\pgfpathlineto{\pgfqpoint{3.467678in}{0.413320in}}%
\pgfpathlineto{\pgfqpoint{3.465072in}{0.413320in}}%
\pgfpathlineto{\pgfqpoint{3.462321in}{0.413320in}}%
\pgfpathlineto{\pgfqpoint{3.459695in}{0.413320in}}%
\pgfpathlineto{\pgfqpoint{3.456960in}{0.413320in}}%
\pgfpathlineto{\pgfqpoint{3.454285in}{0.413320in}}%
\pgfpathlineto{\pgfqpoint{3.451597in}{0.413320in}}%
\pgfpathlineto{\pgfqpoint{3.448926in}{0.413320in}}%
\pgfpathlineto{\pgfqpoint{3.446257in}{0.413320in}}%
\pgfpathlineto{\pgfqpoint{3.443574in}{0.413320in}}%
\pgfpathlineto{\pgfqpoint{3.440996in}{0.413320in}}%
\pgfpathlineto{\pgfqpoint{3.438210in}{0.413320in}}%
\pgfpathlineto{\pgfqpoint{3.435635in}{0.413320in}}%
\pgfpathlineto{\pgfqpoint{3.432851in}{0.413320in}}%
\pgfpathlineto{\pgfqpoint{3.430313in}{0.413320in}}%
\pgfpathlineto{\pgfqpoint{3.427501in}{0.413320in}}%
\pgfpathlineto{\pgfqpoint{3.424887in}{0.413320in}}%
\pgfpathlineto{\pgfqpoint{3.422142in}{0.413320in}}%
\pgfpathlineto{\pgfqpoint{3.419455in}{0.413320in}}%
\pgfpathlineto{\pgfqpoint{3.416780in}{0.413320in}}%
\pgfpathlineto{\pgfqpoint{3.414109in}{0.413320in}}%
\pgfpathlineto{\pgfqpoint{3.411431in}{0.413320in}}%
\pgfpathlineto{\pgfqpoint{3.408752in}{0.413320in}}%
\pgfpathlineto{\pgfqpoint{3.406202in}{0.413320in}}%
\pgfpathlineto{\pgfqpoint{3.403394in}{0.413320in}}%
\pgfpathlineto{\pgfqpoint{3.400783in}{0.413320in}}%
\pgfpathlineto{\pgfqpoint{3.398037in}{0.413320in}}%
\pgfpathlineto{\pgfqpoint{3.395461in}{0.413320in}}%
\pgfpathlineto{\pgfqpoint{3.392681in}{0.413320in}}%
\pgfpathlineto{\pgfqpoint{3.390102in}{0.413320in}}%
\pgfpathlineto{\pgfqpoint{3.387309in}{0.413320in}}%
\pgfpathlineto{\pgfqpoint{3.384647in}{0.413320in}}%
\pgfpathlineto{\pgfqpoint{3.381959in}{0.413320in}}%
\pgfpathlineto{\pgfqpoint{3.379290in}{0.413320in}}%
\pgfpathlineto{\pgfqpoint{3.376735in}{0.413320in}}%
\pgfpathlineto{\pgfqpoint{3.373921in}{0.413320in}}%
\pgfpathlineto{\pgfqpoint{3.371357in}{0.413320in}}%
\pgfpathlineto{\pgfqpoint{3.368577in}{0.413320in}}%
\pgfpathlineto{\pgfqpoint{3.365996in}{0.413320in}}%
\pgfpathlineto{\pgfqpoint{3.363221in}{0.413320in}}%
\pgfpathlineto{\pgfqpoint{3.360620in}{0.413320in}}%
\pgfpathlineto{\pgfqpoint{3.357862in}{0.413320in}}%
\pgfpathlineto{\pgfqpoint{3.355177in}{0.413320in}}%
\pgfpathlineto{\pgfqpoint{3.352505in}{0.413320in}}%
\pgfpathlineto{\pgfqpoint{3.349828in}{0.413320in}}%
\pgfpathlineto{\pgfqpoint{3.347139in}{0.413320in}}%
\pgfpathlineto{\pgfqpoint{3.344468in}{0.413320in}}%
\pgfpathlineto{\pgfqpoint{3.341893in}{0.413320in}}%
\pgfpathlineto{\pgfqpoint{3.339101in}{0.413320in}}%
\pgfpathlineto{\pgfqpoint{3.336541in}{0.413320in}}%
\pgfpathlineto{\pgfqpoint{3.333758in}{0.413320in}}%
\pgfpathlineto{\pgfqpoint{3.331183in}{0.413320in}}%
\pgfpathlineto{\pgfqpoint{3.328401in}{0.413320in}}%
\pgfpathlineto{\pgfqpoint{3.325860in}{0.413320in}}%
\pgfpathlineto{\pgfqpoint{3.323049in}{0.413320in}}%
\pgfpathlineto{\pgfqpoint{3.320366in}{0.413320in}}%
\pgfpathlineto{\pgfqpoint{3.317688in}{0.413320in}}%
\pgfpathlineto{\pgfqpoint{3.315008in}{0.413320in}}%
\pgfpathlineto{\pgfqpoint{3.312480in}{0.413320in}}%
\pgfpathlineto{\pgfqpoint{3.309652in}{0.413320in}}%
\pgfpathlineto{\pgfqpoint{3.307104in}{0.413320in}}%
\pgfpathlineto{\pgfqpoint{3.304295in}{0.413320in}}%
\pgfpathlineto{\pgfqpoint{3.301719in}{0.413320in}}%
\pgfpathlineto{\pgfqpoint{3.298937in}{0.413320in}}%
\pgfpathlineto{\pgfqpoint{3.296376in}{0.413320in}}%
\pgfpathlineto{\pgfqpoint{3.293574in}{0.413320in}}%
\pgfpathlineto{\pgfqpoint{3.290890in}{0.413320in}}%
\pgfpathlineto{\pgfqpoint{3.288225in}{0.413320in}}%
\pgfpathlineto{\pgfqpoint{3.285534in}{0.413320in}}%
\pgfpathlineto{\pgfqpoint{3.282870in}{0.413320in}}%
\pgfpathlineto{\pgfqpoint{3.280189in}{0.413320in}}%
\pgfpathlineto{\pgfqpoint{3.277603in}{0.413320in}}%
\pgfpathlineto{\pgfqpoint{3.274831in}{0.413320in}}%
\pgfpathlineto{\pgfqpoint{3.272254in}{0.413320in}}%
\pgfpathlineto{\pgfqpoint{3.269478in}{0.413320in}}%
\pgfpathlineto{\pgfqpoint{3.266849in}{0.413320in}}%
\pgfpathlineto{\pgfqpoint{3.264119in}{0.413320in}}%
\pgfpathlineto{\pgfqpoint{3.261594in}{0.413320in}}%
\pgfpathlineto{\pgfqpoint{3.258784in}{0.413320in}}%
\pgfpathlineto{\pgfqpoint{3.256083in}{0.413320in}}%
\pgfpathlineto{\pgfqpoint{3.253404in}{0.413320in}}%
\pgfpathlineto{\pgfqpoint{3.250716in}{0.413320in}}%
\pgfpathlineto{\pgfqpoint{3.248049in}{0.413320in}}%
\pgfpathlineto{\pgfqpoint{3.245363in}{0.413320in}}%
\pgfpathlineto{\pgfqpoint{3.242807in}{0.413320in}}%
\pgfpathlineto{\pgfqpoint{3.240010in}{0.413320in}}%
\pgfpathlineto{\pgfqpoint{3.237411in}{0.413320in}}%
\pgfpathlineto{\pgfqpoint{3.234658in}{0.413320in}}%
\pgfpathlineto{\pgfqpoint{3.232069in}{0.413320in}}%
\pgfpathlineto{\pgfqpoint{3.229310in}{0.413320in}}%
\pgfpathlineto{\pgfqpoint{3.226609in}{0.413320in}}%
\pgfpathlineto{\pgfqpoint{3.223942in}{0.413320in}}%
\pgfpathlineto{\pgfqpoint{3.221255in}{0.413320in}}%
\pgfpathlineto{\pgfqpoint{3.218586in}{0.413320in}}%
\pgfpathlineto{\pgfqpoint{3.215908in}{0.413320in}}%
\pgfpathlineto{\pgfqpoint{3.213342in}{0.413320in}}%
\pgfpathlineto{\pgfqpoint{3.210545in}{0.413320in}}%
\pgfpathlineto{\pgfqpoint{3.207984in}{0.413320in}}%
\pgfpathlineto{\pgfqpoint{3.205195in}{0.413320in}}%
\pgfpathlineto{\pgfqpoint{3.202562in}{0.413320in}}%
\pgfpathlineto{\pgfqpoint{3.199823in}{0.413320in}}%
\pgfpathlineto{\pgfqpoint{3.197226in}{0.413320in}}%
\pgfpathlineto{\pgfqpoint{3.194508in}{0.413320in}}%
\pgfpathlineto{\pgfqpoint{3.191796in}{0.413320in}}%
\pgfpathlineto{\pgfqpoint{3.189117in}{0.413320in}}%
\pgfpathlineto{\pgfqpoint{3.186440in}{0.413320in}}%
\pgfpathlineto{\pgfqpoint{3.183760in}{0.413320in}}%
\pgfpathlineto{\pgfqpoint{3.181089in}{0.413320in}}%
\pgfpathlineto{\pgfqpoint{3.178525in}{0.413320in}}%
\pgfpathlineto{\pgfqpoint{3.175724in}{0.413320in}}%
\pgfpathlineto{\pgfqpoint{3.173142in}{0.413320in}}%
\pgfpathlineto{\pgfqpoint{3.170375in}{0.413320in}}%
\pgfpathlineto{\pgfqpoint{3.167776in}{0.413320in}}%
\pgfpathlineto{\pgfqpoint{3.165019in}{0.413320in}}%
\pgfpathlineto{\pgfqpoint{3.162474in}{0.413320in}}%
\pgfpathlineto{\pgfqpoint{3.159675in}{0.413320in}}%
\pgfpathlineto{\pgfqpoint{3.156981in}{0.413320in}}%
\pgfpathlineto{\pgfqpoint{3.154327in}{0.413320in}}%
\pgfpathlineto{\pgfqpoint{3.151612in}{0.413320in}}%
\pgfpathlineto{\pgfqpoint{3.149057in}{0.413320in}}%
\pgfpathlineto{\pgfqpoint{3.146271in}{0.413320in}}%
\pgfpathlineto{\pgfqpoint{3.143740in}{0.413320in}}%
\pgfpathlineto{\pgfqpoint{3.140913in}{0.413320in}}%
\pgfpathlineto{\pgfqpoint{3.138375in}{0.413320in}}%
\pgfpathlineto{\pgfqpoint{3.135550in}{0.413320in}}%
\pgfpathlineto{\pgfqpoint{3.132946in}{0.413320in}}%
\pgfpathlineto{\pgfqpoint{3.130199in}{0.413320in}}%
\pgfpathlineto{\pgfqpoint{3.127512in}{0.413320in}}%
\pgfpathlineto{\pgfqpoint{3.124842in}{0.413320in}}%
\pgfpathlineto{\pgfqpoint{3.122163in}{0.413320in}}%
\pgfpathlineto{\pgfqpoint{3.119487in}{0.413320in}}%
\pgfpathlineto{\pgfqpoint{3.116807in}{0.413320in}}%
\pgfpathlineto{\pgfqpoint{3.114242in}{0.413320in}}%
\pgfpathlineto{\pgfqpoint{3.111451in}{0.413320in}}%
\pgfpathlineto{\pgfqpoint{3.108896in}{0.413320in}}%
\pgfpathlineto{\pgfqpoint{3.106094in}{0.413320in}}%
\pgfpathlineto{\pgfqpoint{3.103508in}{0.413320in}}%
\pgfpathlineto{\pgfqpoint{3.100737in}{0.413320in}}%
\pgfpathlineto{\pgfqpoint{3.098163in}{0.413320in}}%
\pgfpathlineto{\pgfqpoint{3.095388in}{0.413320in}}%
\pgfpathlineto{\pgfqpoint{3.092699in}{0.413320in}}%
\pgfpathlineto{\pgfqpoint{3.090023in}{0.413320in}}%
\pgfpathlineto{\pgfqpoint{3.087343in}{0.413320in}}%
\pgfpathlineto{\pgfqpoint{3.084671in}{0.413320in}}%
\pgfpathlineto{\pgfqpoint{3.081990in}{0.413320in}}%
\pgfpathlineto{\pgfqpoint{3.079381in}{0.413320in}}%
\pgfpathlineto{\pgfqpoint{3.076631in}{0.413320in}}%
\pgfpathlineto{\pgfqpoint{3.074056in}{0.413320in}}%
\pgfpathlineto{\pgfqpoint{3.071266in}{0.413320in}}%
\pgfpathlineto{\pgfqpoint{3.068709in}{0.413320in}}%
\pgfpathlineto{\pgfqpoint{3.065916in}{0.413320in}}%
\pgfpathlineto{\pgfqpoint{3.063230in}{0.413320in}}%
\pgfpathlineto{\pgfqpoint{3.060561in}{0.413320in}}%
\pgfpathlineto{\pgfqpoint{3.057884in}{0.413320in}}%
\pgfpathlineto{\pgfqpoint{3.055202in}{0.413320in}}%
\pgfpathlineto{\pgfqpoint{3.052526in}{0.413320in}}%
\pgfpathlineto{\pgfqpoint{3.049988in}{0.413320in}}%
\pgfpathlineto{\pgfqpoint{3.047157in}{0.413320in}}%
\pgfpathlineto{\pgfqpoint{3.044568in}{0.413320in}}%
\pgfpathlineto{\pgfqpoint{3.041813in}{0.413320in}}%
\pgfpathlineto{\pgfqpoint{3.039262in}{0.413320in}}%
\pgfpathlineto{\pgfqpoint{3.036456in}{0.413320in}}%
\pgfpathlineto{\pgfqpoint{3.033921in}{0.413320in}}%
\pgfpathlineto{\pgfqpoint{3.031091in}{0.413320in}}%
\pgfpathlineto{\pgfqpoint{3.028412in}{0.413320in}}%
\pgfpathlineto{\pgfqpoint{3.025803in}{0.413320in}}%
\pgfpathlineto{\pgfqpoint{3.023058in}{0.413320in}}%
\pgfpathlineto{\pgfqpoint{3.020382in}{0.413320in}}%
\pgfpathlineto{\pgfqpoint{3.017707in}{0.413320in}}%
\pgfpathlineto{\pgfqpoint{3.015097in}{0.413320in}}%
\pgfpathlineto{\pgfqpoint{3.012351in}{0.413320in}}%
\pgfpathlineto{\pgfqpoint{3.009784in}{0.413320in}}%
\pgfpathlineto{\pgfqpoint{3.006993in}{0.413320in}}%
\pgfpathlineto{\pgfqpoint{3.004419in}{0.413320in}}%
\pgfpathlineto{\pgfqpoint{3.001635in}{0.413320in}}%
\pgfpathlineto{\pgfqpoint{2.999103in}{0.413320in}}%
\pgfpathlineto{\pgfqpoint{2.996300in}{0.413320in}}%
\pgfpathlineto{\pgfqpoint{2.993595in}{0.413320in}}%
\pgfpathlineto{\pgfqpoint{2.990978in}{0.413320in}}%
\pgfpathlineto{\pgfqpoint{2.988238in}{0.413320in}}%
\pgfpathlineto{\pgfqpoint{2.985666in}{0.413320in}}%
\pgfpathlineto{\pgfqpoint{2.982885in}{0.413320in}}%
\pgfpathlineto{\pgfqpoint{2.980341in}{0.413320in}}%
\pgfpathlineto{\pgfqpoint{2.977517in}{0.413320in}}%
\pgfpathlineto{\pgfqpoint{2.974972in}{0.413320in}}%
\pgfpathlineto{\pgfqpoint{2.972177in}{0.413320in}}%
\pgfpathlineto{\pgfqpoint{2.969599in}{0.413320in}}%
\pgfpathlineto{\pgfqpoint{2.966812in}{0.413320in}}%
\pgfpathlineto{\pgfqpoint{2.964127in}{0.413320in}}%
\pgfpathlineto{\pgfqpoint{2.961460in}{0.413320in}}%
\pgfpathlineto{\pgfqpoint{2.958782in}{0.413320in}}%
\pgfpathlineto{\pgfqpoint{2.956103in}{0.413320in}}%
\pgfpathlineto{\pgfqpoint{2.953422in}{0.413320in}}%
\pgfpathlineto{\pgfqpoint{2.950884in}{0.413320in}}%
\pgfpathlineto{\pgfqpoint{2.948068in}{0.413320in}}%
\pgfpathlineto{\pgfqpoint{2.945461in}{0.413320in}}%
\pgfpathlineto{\pgfqpoint{2.942711in}{0.413320in}}%
\pgfpathlineto{\pgfqpoint{2.940120in}{0.413320in}}%
\pgfpathlineto{\pgfqpoint{2.937352in}{0.413320in}}%
\pgfpathlineto{\pgfqpoint{2.934759in}{0.413320in}}%
\pgfpathlineto{\pgfqpoint{2.932033in}{0.413320in}}%
\pgfpathlineto{\pgfqpoint{2.929321in}{0.413320in}}%
\pgfpathlineto{\pgfqpoint{2.926655in}{0.413320in}}%
\pgfpathlineto{\pgfqpoint{2.923963in}{0.413320in}}%
\pgfpathlineto{\pgfqpoint{2.921363in}{0.413320in}}%
\pgfpathlineto{\pgfqpoint{2.918606in}{0.413320in}}%
\pgfpathlineto{\pgfqpoint{2.916061in}{0.413320in}}%
\pgfpathlineto{\pgfqpoint{2.913243in}{0.413320in}}%
\pgfpathlineto{\pgfqpoint{2.910631in}{0.413320in}}%
\pgfpathlineto{\pgfqpoint{2.907882in}{0.413320in}}%
\pgfpathlineto{\pgfqpoint{2.905341in}{0.413320in}}%
\pgfpathlineto{\pgfqpoint{2.902535in}{0.413320in}}%
\pgfpathlineto{\pgfqpoint{2.899858in}{0.413320in}}%
\pgfpathlineto{\pgfqpoint{2.897179in}{0.413320in}}%
\pgfpathlineto{\pgfqpoint{2.894487in}{0.413320in}}%
\pgfpathlineto{\pgfqpoint{2.891809in}{0.413320in}}%
\pgfpathlineto{\pgfqpoint{2.889145in}{0.413320in}}%
\pgfpathlineto{\pgfqpoint{2.886578in}{0.413320in}}%
\pgfpathlineto{\pgfqpoint{2.883780in}{0.413320in}}%
\pgfpathlineto{\pgfqpoint{2.881254in}{0.413320in}}%
\pgfpathlineto{\pgfqpoint{2.878431in}{0.413320in}}%
\pgfpathlineto{\pgfqpoint{2.875882in}{0.413320in}}%
\pgfpathlineto{\pgfqpoint{2.873074in}{0.413320in}}%
\pgfpathlineto{\pgfqpoint{2.870475in}{0.413320in}}%
\pgfpathlineto{\pgfqpoint{2.867713in}{0.413320in}}%
\pgfpathlineto{\pgfqpoint{2.865031in}{0.413320in}}%
\pgfpathlineto{\pgfqpoint{2.862402in}{0.413320in}}%
\pgfpathlineto{\pgfqpoint{2.859668in}{0.413320in}}%
\pgfpathlineto{\pgfqpoint{2.857003in}{0.413320in}}%
\pgfpathlineto{\pgfqpoint{2.854325in}{0.413320in}}%
\pgfpathlineto{\pgfqpoint{2.851793in}{0.413320in}}%
\pgfpathlineto{\pgfqpoint{2.848960in}{0.413320in}}%
\pgfpathlineto{\pgfqpoint{2.846408in}{0.413320in}}%
\pgfpathlineto{\pgfqpoint{2.843611in}{0.413320in}}%
\pgfpathlineto{\pgfqpoint{2.841055in}{0.413320in}}%
\pgfpathlineto{\pgfqpoint{2.838254in}{0.413320in}}%
\pgfpathlineto{\pgfqpoint{2.835698in}{0.413320in}}%
\pgfpathlineto{\pgfqpoint{2.832894in}{0.413320in}}%
\pgfpathlineto{\pgfqpoint{2.830219in}{0.413320in}}%
\pgfpathlineto{\pgfqpoint{2.827567in}{0.413320in}}%
\pgfpathlineto{\pgfqpoint{2.824851in}{0.413320in}}%
\pgfpathlineto{\pgfqpoint{2.822303in}{0.413320in}}%
\pgfpathlineto{\pgfqpoint{2.819506in}{0.413320in}}%
\pgfpathlineto{\pgfqpoint{2.816867in}{0.413320in}}%
\pgfpathlineto{\pgfqpoint{2.814141in}{0.413320in}}%
\pgfpathlineto{\pgfqpoint{2.811597in}{0.413320in}}%
\pgfpathlineto{\pgfqpoint{2.808792in}{0.413320in}}%
\pgfpathlineto{\pgfqpoint{2.806175in}{0.413320in}}%
\pgfpathlineto{\pgfqpoint{2.803435in}{0.413320in}}%
\pgfpathlineto{\pgfqpoint{2.800756in}{0.413320in}}%
\pgfpathlineto{\pgfqpoint{2.798070in}{0.413320in}}%
\pgfpathlineto{\pgfqpoint{2.795398in}{0.413320in}}%
\pgfpathlineto{\pgfqpoint{2.792721in}{0.413320in}}%
\pgfpathlineto{\pgfqpoint{2.790044in}{0.413320in}}%
\pgfpathlineto{\pgfqpoint{2.787468in}{0.413320in}}%
\pgfpathlineto{\pgfqpoint{2.784687in}{0.413320in}}%
\pgfpathlineto{\pgfqpoint{2.782113in}{0.413320in}}%
\pgfpathlineto{\pgfqpoint{2.779330in}{0.413320in}}%
\pgfpathlineto{\pgfqpoint{2.776767in}{0.413320in}}%
\pgfpathlineto{\pgfqpoint{2.773972in}{0.413320in}}%
\pgfpathlineto{\pgfqpoint{2.771373in}{0.413320in}}%
\pgfpathlineto{\pgfqpoint{2.768617in}{0.413320in}}%
\pgfpathlineto{\pgfqpoint{2.765935in}{0.413320in}}%
\pgfpathlineto{\pgfqpoint{2.763253in}{0.413320in}}%
\pgfpathlineto{\pgfqpoint{2.760581in}{0.413320in}}%
\pgfpathlineto{\pgfqpoint{2.758028in}{0.413320in}}%
\pgfpathlineto{\pgfqpoint{2.755224in}{0.413320in}}%
\pgfpathlineto{\pgfqpoint{2.752614in}{0.413320in}}%
\pgfpathlineto{\pgfqpoint{2.749868in}{0.413320in}}%
\pgfpathlineto{\pgfqpoint{2.747260in}{0.413320in}}%
\pgfpathlineto{\pgfqpoint{2.744510in}{0.413320in}}%
\pgfpathlineto{\pgfqpoint{2.741928in}{0.413320in}}%
\pgfpathlineto{\pgfqpoint{2.739155in}{0.413320in}}%
\pgfpathlineto{\pgfqpoint{2.736476in}{0.413320in}}%
\pgfpathlineto{\pgfqpoint{2.733798in}{0.413320in}}%
\pgfpathlineto{\pgfqpoint{2.731119in}{0.413320in}}%
\pgfpathlineto{\pgfqpoint{2.728439in}{0.413320in}}%
\pgfpathlineto{\pgfqpoint{2.725760in}{0.413320in}}%
\pgfpathlineto{\pgfqpoint{2.723211in}{0.413320in}}%
\pgfpathlineto{\pgfqpoint{2.720404in}{0.413320in}}%
\pgfpathlineto{\pgfqpoint{2.717773in}{0.413320in}}%
\pgfpathlineto{\pgfqpoint{2.715036in}{0.413320in}}%
\pgfpathlineto{\pgfqpoint{2.712477in}{0.413320in}}%
\pgfpathlineto{\pgfqpoint{2.709683in}{0.413320in}}%
\pgfpathlineto{\pgfqpoint{2.707125in}{0.413320in}}%
\pgfpathlineto{\pgfqpoint{2.704326in}{0.413320in}}%
\pgfpathlineto{\pgfqpoint{2.701657in}{0.413320in}}%
\pgfpathlineto{\pgfqpoint{2.698968in}{0.413320in}}%
\pgfpathlineto{\pgfqpoint{2.696293in}{0.413320in}}%
\pgfpathlineto{\pgfqpoint{2.693611in}{0.413320in}}%
\pgfpathlineto{\pgfqpoint{2.690940in}{0.413320in}}%
\pgfpathlineto{\pgfqpoint{2.688328in}{0.413320in}}%
\pgfpathlineto{\pgfqpoint{2.685586in}{0.413320in}}%
\pgfpathlineto{\pgfqpoint{2.683009in}{0.413320in}}%
\pgfpathlineto{\pgfqpoint{2.680224in}{0.413320in}}%
\pgfpathlineto{\pgfqpoint{2.677650in}{0.413320in}}%
\pgfpathlineto{\pgfqpoint{2.674873in}{0.413320in}}%
\pgfpathlineto{\pgfqpoint{2.672301in}{0.413320in}}%
\pgfpathlineto{\pgfqpoint{2.669506in}{0.413320in}}%
\pgfpathlineto{\pgfqpoint{2.666836in}{0.413320in}}%
\pgfpathlineto{\pgfqpoint{2.664151in}{0.413320in}}%
\pgfpathlineto{\pgfqpoint{2.661481in}{0.413320in}}%
\pgfpathlineto{\pgfqpoint{2.658942in}{0.413320in}}%
\pgfpathlineto{\pgfqpoint{2.656124in}{0.413320in}}%
\pgfpathlineto{\pgfqpoint{2.653567in}{0.413320in}}%
\pgfpathlineto{\pgfqpoint{2.650767in}{0.413320in}}%
\pgfpathlineto{\pgfqpoint{2.648196in}{0.413320in}}%
\pgfpathlineto{\pgfqpoint{2.645408in}{0.413320in}}%
\pgfpathlineto{\pgfqpoint{2.642827in}{0.413320in}}%
\pgfpathlineto{\pgfqpoint{2.640053in}{0.413320in}}%
\pgfpathlineto{\pgfqpoint{2.637369in}{0.413320in}}%
\pgfpathlineto{\pgfqpoint{2.634700in}{0.413320in}}%
\pgfpathlineto{\pgfqpoint{2.632018in}{0.413320in}}%
\pgfpathlineto{\pgfqpoint{2.629340in}{0.413320in}}%
\pgfpathlineto{\pgfqpoint{2.626653in}{0.413320in}}%
\pgfpathlineto{\pgfqpoint{2.624077in}{0.413320in}}%
\pgfpathlineto{\pgfqpoint{2.621304in}{0.413320in}}%
\pgfpathlineto{\pgfqpoint{2.618773in}{0.413320in}}%
\pgfpathlineto{\pgfqpoint{2.615934in}{0.413320in}}%
\pgfpathlineto{\pgfqpoint{2.613393in}{0.413320in}}%
\pgfpathlineto{\pgfqpoint{2.610588in}{0.413320in}}%
\pgfpathlineto{\pgfqpoint{2.608004in}{0.413320in}}%
\pgfpathlineto{\pgfqpoint{2.605232in}{0.413320in}}%
\pgfpathlineto{\pgfqpoint{2.602557in}{0.413320in}}%
\pgfpathlineto{\pgfqpoint{2.599920in}{0.413320in}}%
\pgfpathlineto{\pgfqpoint{2.597196in}{0.413320in}}%
\pgfpathlineto{\pgfqpoint{2.594630in}{0.413320in}}%
\pgfpathlineto{\pgfqpoint{2.591842in}{0.413320in}}%
\pgfpathlineto{\pgfqpoint{2.589248in}{0.413320in}}%
\pgfpathlineto{\pgfqpoint{2.586484in}{0.413320in}}%
\pgfpathlineto{\pgfqpoint{2.583913in}{0.413320in}}%
\pgfpathlineto{\pgfqpoint{2.581129in}{0.413320in}}%
\pgfpathlineto{\pgfqpoint{2.578567in}{0.413320in}}%
\pgfpathlineto{\pgfqpoint{2.575779in}{0.413320in}}%
\pgfpathlineto{\pgfqpoint{2.573082in}{0.413320in}}%
\pgfpathlineto{\pgfqpoint{2.570411in}{0.413320in}}%
\pgfpathlineto{\pgfqpoint{2.567730in}{0.413320in}}%
\pgfpathlineto{\pgfqpoint{2.565045in}{0.413320in}}%
\pgfpathlineto{\pgfqpoint{2.562375in}{0.413320in}}%
\pgfpathlineto{\pgfqpoint{2.559790in}{0.413320in}}%
\pgfpathlineto{\pgfqpoint{2.557009in}{0.413320in}}%
\pgfpathlineto{\pgfqpoint{2.554493in}{0.413320in}}%
\pgfpathlineto{\pgfqpoint{2.551664in}{0.413320in}}%
\pgfpathlineto{\pgfqpoint{2.549114in}{0.413320in}}%
\pgfpathlineto{\pgfqpoint{2.546310in}{0.413320in}}%
\pgfpathlineto{\pgfqpoint{2.543765in}{0.413320in}}%
\pgfpathlineto{\pgfqpoint{2.540949in}{0.413320in}}%
\pgfpathlineto{\pgfqpoint{2.538274in}{0.413320in}}%
\pgfpathlineto{\pgfqpoint{2.535624in}{0.413320in}}%
\pgfpathlineto{\pgfqpoint{2.532917in}{0.413320in}}%
\pgfpathlineto{\pgfqpoint{2.530234in}{0.413320in}}%
\pgfpathlineto{\pgfqpoint{2.527560in}{0.413320in}}%
\pgfpathlineto{\pgfqpoint{2.524988in}{0.413320in}}%
\pgfpathlineto{\pgfqpoint{2.522197in}{0.413320in}}%
\pgfpathlineto{\pgfqpoint{2.519607in}{0.413320in}}%
\pgfpathlineto{\pgfqpoint{2.516845in}{0.413320in}}%
\pgfpathlineto{\pgfqpoint{2.514268in}{0.413320in}}%
\pgfpathlineto{\pgfqpoint{2.511478in}{0.413320in}}%
\pgfpathlineto{\pgfqpoint{2.508917in}{0.413320in}}%
\pgfpathlineto{\pgfqpoint{2.506163in}{0.413320in}}%
\pgfpathlineto{\pgfqpoint{2.503454in}{0.413320in}}%
\pgfpathlineto{\pgfqpoint{2.500801in}{0.413320in}}%
\pgfpathlineto{\pgfqpoint{2.498085in}{0.413320in}}%
\pgfpathlineto{\pgfqpoint{2.495542in}{0.413320in}}%
\pgfpathlineto{\pgfqpoint{2.492729in}{0.413320in}}%
\pgfpathlineto{\pgfqpoint{2.490183in}{0.413320in}}%
\pgfpathlineto{\pgfqpoint{2.487384in}{0.413320in}}%
\pgfpathlineto{\pgfqpoint{2.484870in}{0.413320in}}%
\pgfpathlineto{\pgfqpoint{2.482026in}{0.413320in}}%
\pgfpathlineto{\pgfqpoint{2.479420in}{0.413320in}}%
\pgfpathlineto{\pgfqpoint{2.476671in}{0.413320in}}%
\pgfpathlineto{\pgfqpoint{2.473989in}{0.413320in}}%
\pgfpathlineto{\pgfqpoint{2.471311in}{0.413320in}}%
\pgfpathlineto{\pgfqpoint{2.468635in}{0.413320in}}%
\pgfpathlineto{\pgfqpoint{2.465957in}{0.413320in}}%
\pgfpathlineto{\pgfqpoint{2.463280in}{0.413320in}}%
\pgfpathlineto{\pgfqpoint{2.460711in}{0.413320in}}%
\pgfpathlineto{\pgfqpoint{2.457917in}{0.413320in}}%
\pgfpathlineto{\pgfqpoint{2.455353in}{0.413320in}}%
\pgfpathlineto{\pgfqpoint{2.452562in}{0.413320in}}%
\pgfpathlineto{\pgfqpoint{2.450032in}{0.413320in}}%
\pgfpathlineto{\pgfqpoint{2.447209in}{0.413320in}}%
\pgfpathlineto{\pgfqpoint{2.444677in}{0.413320in}}%
\pgfpathlineto{\pgfqpoint{2.441876in}{0.413320in}}%
\pgfpathlineto{\pgfqpoint{2.439167in}{0.413320in}}%
\pgfpathlineto{\pgfqpoint{2.436518in}{0.413320in}}%
\pgfpathlineto{\pgfqpoint{2.433815in}{0.413320in}}%
\pgfpathlineto{\pgfqpoint{2.431251in}{0.413320in}}%
\pgfpathlineto{\pgfqpoint{2.428453in}{0.413320in}}%
\pgfpathlineto{\pgfqpoint{2.425878in}{0.413320in}}%
\pgfpathlineto{\pgfqpoint{2.423098in}{0.413320in}}%
\pgfpathlineto{\pgfqpoint{2.420528in}{0.413320in}}%
\pgfpathlineto{\pgfqpoint{2.417747in}{0.413320in}}%
\pgfpathlineto{\pgfqpoint{2.415184in}{0.413320in}}%
\pgfpathlineto{\pgfqpoint{2.412389in}{0.413320in}}%
\pgfpathlineto{\pgfqpoint{2.409699in}{0.413320in}}%
\pgfpathlineto{\pgfqpoint{2.407024in}{0.413320in}}%
\pgfpathlineto{\pgfqpoint{2.404352in}{0.413320in}}%
\pgfpathlineto{\pgfqpoint{2.401675in}{0.413320in}}%
\pgfpathlineto{\pgfqpoint{2.398995in}{0.413320in}}%
\pgfpathclose%
\pgfusepath{stroke,fill}%
\end{pgfscope}%
\begin{pgfscope}%
\pgfpathrectangle{\pgfqpoint{2.398995in}{0.319877in}}{\pgfqpoint{3.986877in}{1.993438in}} %
\pgfusepath{clip}%
\pgfsetbuttcap%
\pgfsetroundjoin%
\definecolor{currentfill}{rgb}{1.000000,1.000000,1.000000}%
\pgfsetfillcolor{currentfill}%
\pgfsetlinewidth{1.003750pt}%
\definecolor{currentstroke}{rgb}{0.592089,0.641847,0.193507}%
\pgfsetstrokecolor{currentstroke}%
\pgfsetdash{}{0pt}%
\pgfpathmoveto{\pgfqpoint{2.398995in}{0.413320in}}%
\pgfpathlineto{\pgfqpoint{2.398995in}{1.487689in}}%
\pgfpathlineto{\pgfqpoint{2.401675in}{1.485317in}}%
\pgfpathlineto{\pgfqpoint{2.404352in}{1.485934in}}%
\pgfpathlineto{\pgfqpoint{2.407024in}{1.485673in}}%
\pgfpathlineto{\pgfqpoint{2.409699in}{1.481501in}}%
\pgfpathlineto{\pgfqpoint{2.412389in}{1.482435in}}%
\pgfpathlineto{\pgfqpoint{2.415184in}{1.477368in}}%
\pgfpathlineto{\pgfqpoint{2.417747in}{1.478690in}}%
\pgfpathlineto{\pgfqpoint{2.420528in}{1.480114in}}%
\pgfpathlineto{\pgfqpoint{2.423098in}{1.479155in}}%
\pgfpathlineto{\pgfqpoint{2.425878in}{1.481548in}}%
\pgfpathlineto{\pgfqpoint{2.428453in}{1.483028in}}%
\pgfpathlineto{\pgfqpoint{2.431251in}{1.487342in}}%
\pgfpathlineto{\pgfqpoint{2.433815in}{1.490336in}}%
\pgfpathlineto{\pgfqpoint{2.436518in}{1.488252in}}%
\pgfpathlineto{\pgfqpoint{2.439167in}{1.484134in}}%
\pgfpathlineto{\pgfqpoint{2.441876in}{1.480257in}}%
\pgfpathlineto{\pgfqpoint{2.444677in}{1.481817in}}%
\pgfpathlineto{\pgfqpoint{2.447209in}{1.468354in}}%
\pgfpathlineto{\pgfqpoint{2.450032in}{1.470984in}}%
\pgfpathlineto{\pgfqpoint{2.452562in}{1.463161in}}%
\pgfpathlineto{\pgfqpoint{2.455353in}{1.469121in}}%
\pgfpathlineto{\pgfqpoint{2.457917in}{1.467071in}}%
\pgfpathlineto{\pgfqpoint{2.460711in}{1.468670in}}%
\pgfpathlineto{\pgfqpoint{2.463280in}{1.476995in}}%
\pgfpathlineto{\pgfqpoint{2.465957in}{1.474815in}}%
\pgfpathlineto{\pgfqpoint{2.468635in}{1.478495in}}%
\pgfpathlineto{\pgfqpoint{2.471311in}{1.476625in}}%
\pgfpathlineto{\pgfqpoint{2.473989in}{1.478497in}}%
\pgfpathlineto{\pgfqpoint{2.476671in}{1.478696in}}%
\pgfpathlineto{\pgfqpoint{2.479420in}{1.478802in}}%
\pgfpathlineto{\pgfqpoint{2.482026in}{1.472370in}}%
\pgfpathlineto{\pgfqpoint{2.484870in}{1.467977in}}%
\pgfpathlineto{\pgfqpoint{2.487384in}{1.468325in}}%
\pgfpathlineto{\pgfqpoint{2.490183in}{1.471302in}}%
\pgfpathlineto{\pgfqpoint{2.492729in}{1.473449in}}%
\pgfpathlineto{\pgfqpoint{2.495542in}{1.471057in}}%
\pgfpathlineto{\pgfqpoint{2.498085in}{1.475956in}}%
\pgfpathlineto{\pgfqpoint{2.500801in}{1.469556in}}%
\pgfpathlineto{\pgfqpoint{2.503454in}{1.480764in}}%
\pgfpathlineto{\pgfqpoint{2.506163in}{1.476716in}}%
\pgfpathlineto{\pgfqpoint{2.508917in}{1.474088in}}%
\pgfpathlineto{\pgfqpoint{2.511478in}{1.476500in}}%
\pgfpathlineto{\pgfqpoint{2.514268in}{1.477128in}}%
\pgfpathlineto{\pgfqpoint{2.516845in}{1.477665in}}%
\pgfpathlineto{\pgfqpoint{2.519607in}{1.479577in}}%
\pgfpathlineto{\pgfqpoint{2.522197in}{1.476661in}}%
\pgfpathlineto{\pgfqpoint{2.524988in}{1.471334in}}%
\pgfpathlineto{\pgfqpoint{2.527560in}{1.471494in}}%
\pgfpathlineto{\pgfqpoint{2.530234in}{1.465512in}}%
\pgfpathlineto{\pgfqpoint{2.532917in}{1.472291in}}%
\pgfpathlineto{\pgfqpoint{2.535624in}{1.480546in}}%
\pgfpathlineto{\pgfqpoint{2.538274in}{1.474602in}}%
\pgfpathlineto{\pgfqpoint{2.540949in}{1.475083in}}%
\pgfpathlineto{\pgfqpoint{2.543765in}{1.475430in}}%
\pgfpathlineto{\pgfqpoint{2.546310in}{1.480908in}}%
\pgfpathlineto{\pgfqpoint{2.549114in}{1.476739in}}%
\pgfpathlineto{\pgfqpoint{2.551664in}{1.496154in}}%
\pgfpathlineto{\pgfqpoint{2.554493in}{1.528960in}}%
\pgfpathlineto{\pgfqpoint{2.557009in}{1.530979in}}%
\pgfpathlineto{\pgfqpoint{2.559790in}{1.504089in}}%
\pgfpathlineto{\pgfqpoint{2.562375in}{1.516494in}}%
\pgfpathlineto{\pgfqpoint{2.565045in}{1.526315in}}%
\pgfpathlineto{\pgfqpoint{2.567730in}{1.518313in}}%
\pgfpathlineto{\pgfqpoint{2.570411in}{1.516134in}}%
\pgfpathlineto{\pgfqpoint{2.573082in}{1.517002in}}%
\pgfpathlineto{\pgfqpoint{2.575779in}{1.502435in}}%
\pgfpathlineto{\pgfqpoint{2.578567in}{1.508357in}}%
\pgfpathlineto{\pgfqpoint{2.581129in}{1.501886in}}%
\pgfpathlineto{\pgfqpoint{2.583913in}{1.485774in}}%
\pgfpathlineto{\pgfqpoint{2.586484in}{1.485136in}}%
\pgfpathlineto{\pgfqpoint{2.589248in}{1.486498in}}%
\pgfpathlineto{\pgfqpoint{2.591842in}{1.484125in}}%
\pgfpathlineto{\pgfqpoint{2.594630in}{1.479382in}}%
\pgfpathlineto{\pgfqpoint{2.597196in}{1.479310in}}%
\pgfpathlineto{\pgfqpoint{2.599920in}{1.469987in}}%
\pgfpathlineto{\pgfqpoint{2.602557in}{1.466633in}}%
\pgfpathlineto{\pgfqpoint{2.605232in}{1.473832in}}%
\pgfpathlineto{\pgfqpoint{2.608004in}{1.473767in}}%
\pgfpathlineto{\pgfqpoint{2.610588in}{1.478306in}}%
\pgfpathlineto{\pgfqpoint{2.613393in}{1.475106in}}%
\pgfpathlineto{\pgfqpoint{2.615934in}{1.484563in}}%
\pgfpathlineto{\pgfqpoint{2.618773in}{1.481859in}}%
\pgfpathlineto{\pgfqpoint{2.621304in}{1.481029in}}%
\pgfpathlineto{\pgfqpoint{2.624077in}{1.476104in}}%
\pgfpathlineto{\pgfqpoint{2.626653in}{1.475953in}}%
\pgfpathlineto{\pgfqpoint{2.629340in}{1.473025in}}%
\pgfpathlineto{\pgfqpoint{2.632018in}{1.473716in}}%
\pgfpathlineto{\pgfqpoint{2.634700in}{1.475719in}}%
\pgfpathlineto{\pgfqpoint{2.637369in}{1.477768in}}%
\pgfpathlineto{\pgfqpoint{2.640053in}{1.475672in}}%
\pgfpathlineto{\pgfqpoint{2.642827in}{1.478907in}}%
\pgfpathlineto{\pgfqpoint{2.645408in}{1.479099in}}%
\pgfpathlineto{\pgfqpoint{2.648196in}{1.490098in}}%
\pgfpathlineto{\pgfqpoint{2.650767in}{1.504643in}}%
\pgfpathlineto{\pgfqpoint{2.653567in}{1.502362in}}%
\pgfpathlineto{\pgfqpoint{2.656124in}{1.497596in}}%
\pgfpathlineto{\pgfqpoint{2.658942in}{1.487755in}}%
\pgfpathlineto{\pgfqpoint{2.661481in}{1.482914in}}%
\pgfpathlineto{\pgfqpoint{2.664151in}{1.478481in}}%
\pgfpathlineto{\pgfqpoint{2.666836in}{1.480190in}}%
\pgfpathlineto{\pgfqpoint{2.669506in}{1.484352in}}%
\pgfpathlineto{\pgfqpoint{2.672301in}{1.480743in}}%
\pgfpathlineto{\pgfqpoint{2.674873in}{1.479265in}}%
\pgfpathlineto{\pgfqpoint{2.677650in}{1.479862in}}%
\pgfpathlineto{\pgfqpoint{2.680224in}{1.478266in}}%
\pgfpathlineto{\pgfqpoint{2.683009in}{1.477625in}}%
\pgfpathlineto{\pgfqpoint{2.685586in}{1.474524in}}%
\pgfpathlineto{\pgfqpoint{2.688328in}{1.482447in}}%
\pgfpathlineto{\pgfqpoint{2.690940in}{1.477521in}}%
\pgfpathlineto{\pgfqpoint{2.693611in}{1.481190in}}%
\pgfpathlineto{\pgfqpoint{2.696293in}{1.474971in}}%
\pgfpathlineto{\pgfqpoint{2.698968in}{1.481866in}}%
\pgfpathlineto{\pgfqpoint{2.701657in}{1.479758in}}%
\pgfpathlineto{\pgfqpoint{2.704326in}{1.477686in}}%
\pgfpathlineto{\pgfqpoint{2.707125in}{1.479571in}}%
\pgfpathlineto{\pgfqpoint{2.709683in}{1.478838in}}%
\pgfpathlineto{\pgfqpoint{2.712477in}{1.486384in}}%
\pgfpathlineto{\pgfqpoint{2.715036in}{1.481209in}}%
\pgfpathlineto{\pgfqpoint{2.717773in}{1.481121in}}%
\pgfpathlineto{\pgfqpoint{2.720404in}{1.482894in}}%
\pgfpathlineto{\pgfqpoint{2.723211in}{1.480230in}}%
\pgfpathlineto{\pgfqpoint{2.725760in}{1.474125in}}%
\pgfpathlineto{\pgfqpoint{2.728439in}{1.473021in}}%
\pgfpathlineto{\pgfqpoint{2.731119in}{1.466084in}}%
\pgfpathlineto{\pgfqpoint{2.733798in}{1.469198in}}%
\pgfpathlineto{\pgfqpoint{2.736476in}{1.469202in}}%
\pgfpathlineto{\pgfqpoint{2.739155in}{1.467662in}}%
\pgfpathlineto{\pgfqpoint{2.741928in}{1.466341in}}%
\pgfpathlineto{\pgfqpoint{2.744510in}{1.473158in}}%
\pgfpathlineto{\pgfqpoint{2.747260in}{1.473260in}}%
\pgfpathlineto{\pgfqpoint{2.749868in}{1.478788in}}%
\pgfpathlineto{\pgfqpoint{2.752614in}{1.478364in}}%
\pgfpathlineto{\pgfqpoint{2.755224in}{1.477165in}}%
\pgfpathlineto{\pgfqpoint{2.758028in}{1.477338in}}%
\pgfpathlineto{\pgfqpoint{2.760581in}{1.476793in}}%
\pgfpathlineto{\pgfqpoint{2.763253in}{1.481632in}}%
\pgfpathlineto{\pgfqpoint{2.765935in}{1.467461in}}%
\pgfpathlineto{\pgfqpoint{2.768617in}{1.467816in}}%
\pgfpathlineto{\pgfqpoint{2.771373in}{1.466464in}}%
\pgfpathlineto{\pgfqpoint{2.773972in}{1.471476in}}%
\pgfpathlineto{\pgfqpoint{2.776767in}{1.469990in}}%
\pgfpathlineto{\pgfqpoint{2.779330in}{1.473397in}}%
\pgfpathlineto{\pgfqpoint{2.782113in}{1.477557in}}%
\pgfpathlineto{\pgfqpoint{2.784687in}{1.476296in}}%
\pgfpathlineto{\pgfqpoint{2.787468in}{1.468249in}}%
\pgfpathlineto{\pgfqpoint{2.790044in}{1.472182in}}%
\pgfpathlineto{\pgfqpoint{2.792721in}{1.473108in}}%
\pgfpathlineto{\pgfqpoint{2.795398in}{1.471740in}}%
\pgfpathlineto{\pgfqpoint{2.798070in}{1.468416in}}%
\pgfpathlineto{\pgfqpoint{2.800756in}{1.467916in}}%
\pgfpathlineto{\pgfqpoint{2.803435in}{1.468583in}}%
\pgfpathlineto{\pgfqpoint{2.806175in}{1.473731in}}%
\pgfpathlineto{\pgfqpoint{2.808792in}{1.467386in}}%
\pgfpathlineto{\pgfqpoint{2.811597in}{1.468374in}}%
\pgfpathlineto{\pgfqpoint{2.814141in}{1.479458in}}%
\pgfpathlineto{\pgfqpoint{2.816867in}{1.474179in}}%
\pgfpathlineto{\pgfqpoint{2.819506in}{1.476241in}}%
\pgfpathlineto{\pgfqpoint{2.822303in}{1.477105in}}%
\pgfpathlineto{\pgfqpoint{2.824851in}{1.477769in}}%
\pgfpathlineto{\pgfqpoint{2.827567in}{1.483887in}}%
\pgfpathlineto{\pgfqpoint{2.830219in}{1.481129in}}%
\pgfpathlineto{\pgfqpoint{2.832894in}{1.481047in}}%
\pgfpathlineto{\pgfqpoint{2.835698in}{1.477290in}}%
\pgfpathlineto{\pgfqpoint{2.838254in}{1.486304in}}%
\pgfpathlineto{\pgfqpoint{2.841055in}{1.489880in}}%
\pgfpathlineto{\pgfqpoint{2.843611in}{1.494377in}}%
\pgfpathlineto{\pgfqpoint{2.846408in}{1.490376in}}%
\pgfpathlineto{\pgfqpoint{2.848960in}{1.487937in}}%
\pgfpathlineto{\pgfqpoint{2.851793in}{1.478495in}}%
\pgfpathlineto{\pgfqpoint{2.854325in}{1.479767in}}%
\pgfpathlineto{\pgfqpoint{2.857003in}{1.478357in}}%
\pgfpathlineto{\pgfqpoint{2.859668in}{1.478456in}}%
\pgfpathlineto{\pgfqpoint{2.862402in}{1.478395in}}%
\pgfpathlineto{\pgfqpoint{2.865031in}{1.481457in}}%
\pgfpathlineto{\pgfqpoint{2.867713in}{1.476715in}}%
\pgfpathlineto{\pgfqpoint{2.870475in}{1.476489in}}%
\pgfpathlineto{\pgfqpoint{2.873074in}{1.474607in}}%
\pgfpathlineto{\pgfqpoint{2.875882in}{1.474057in}}%
\pgfpathlineto{\pgfqpoint{2.878431in}{1.481363in}}%
\pgfpathlineto{\pgfqpoint{2.881254in}{1.474986in}}%
\pgfpathlineto{\pgfqpoint{2.883780in}{1.463794in}}%
\pgfpathlineto{\pgfqpoint{2.886578in}{1.469245in}}%
\pgfpathlineto{\pgfqpoint{2.889145in}{1.478022in}}%
\pgfpathlineto{\pgfqpoint{2.891809in}{1.478310in}}%
\pgfpathlineto{\pgfqpoint{2.894487in}{1.477025in}}%
\pgfpathlineto{\pgfqpoint{2.897179in}{1.477800in}}%
\pgfpathlineto{\pgfqpoint{2.899858in}{1.479396in}}%
\pgfpathlineto{\pgfqpoint{2.902535in}{1.477286in}}%
\pgfpathlineto{\pgfqpoint{2.905341in}{1.477339in}}%
\pgfpathlineto{\pgfqpoint{2.907882in}{1.484962in}}%
\pgfpathlineto{\pgfqpoint{2.910631in}{1.478934in}}%
\pgfpathlineto{\pgfqpoint{2.913243in}{1.477407in}}%
\pgfpathlineto{\pgfqpoint{2.916061in}{1.478339in}}%
\pgfpathlineto{\pgfqpoint{2.918606in}{1.479761in}}%
\pgfpathlineto{\pgfqpoint{2.921363in}{1.483966in}}%
\pgfpathlineto{\pgfqpoint{2.923963in}{1.479918in}}%
\pgfpathlineto{\pgfqpoint{2.926655in}{1.479145in}}%
\pgfpathlineto{\pgfqpoint{2.929321in}{1.483573in}}%
\pgfpathlineto{\pgfqpoint{2.932033in}{1.479294in}}%
\pgfpathlineto{\pgfqpoint{2.934759in}{1.476215in}}%
\pgfpathlineto{\pgfqpoint{2.937352in}{1.481539in}}%
\pgfpathlineto{\pgfqpoint{2.940120in}{1.480540in}}%
\pgfpathlineto{\pgfqpoint{2.942711in}{1.476067in}}%
\pgfpathlineto{\pgfqpoint{2.945461in}{1.474890in}}%
\pgfpathlineto{\pgfqpoint{2.948068in}{1.471081in}}%
\pgfpathlineto{\pgfqpoint{2.950884in}{1.474988in}}%
\pgfpathlineto{\pgfqpoint{2.953422in}{1.468674in}}%
\pgfpathlineto{\pgfqpoint{2.956103in}{1.473871in}}%
\pgfpathlineto{\pgfqpoint{2.958782in}{1.474337in}}%
\pgfpathlineto{\pgfqpoint{2.961460in}{1.474903in}}%
\pgfpathlineto{\pgfqpoint{2.964127in}{1.476840in}}%
\pgfpathlineto{\pgfqpoint{2.966812in}{1.480624in}}%
\pgfpathlineto{\pgfqpoint{2.969599in}{1.475552in}}%
\pgfpathlineto{\pgfqpoint{2.972177in}{1.476319in}}%
\pgfpathlineto{\pgfqpoint{2.974972in}{1.477059in}}%
\pgfpathlineto{\pgfqpoint{2.977517in}{1.479008in}}%
\pgfpathlineto{\pgfqpoint{2.980341in}{1.476747in}}%
\pgfpathlineto{\pgfqpoint{2.982885in}{1.479200in}}%
\pgfpathlineto{\pgfqpoint{2.985666in}{1.476582in}}%
\pgfpathlineto{\pgfqpoint{2.988238in}{1.479449in}}%
\pgfpathlineto{\pgfqpoint{2.990978in}{1.484550in}}%
\pgfpathlineto{\pgfqpoint{2.993595in}{1.478288in}}%
\pgfpathlineto{\pgfqpoint{2.996300in}{1.486395in}}%
\pgfpathlineto{\pgfqpoint{2.999103in}{1.519925in}}%
\pgfpathlineto{\pgfqpoint{3.001635in}{1.510728in}}%
\pgfpathlineto{\pgfqpoint{3.004419in}{1.498723in}}%
\pgfpathlineto{\pgfqpoint{3.006993in}{1.491999in}}%
\pgfpathlineto{\pgfqpoint{3.009784in}{1.484294in}}%
\pgfpathlineto{\pgfqpoint{3.012351in}{1.491477in}}%
\pgfpathlineto{\pgfqpoint{3.015097in}{1.487471in}}%
\pgfpathlineto{\pgfqpoint{3.017707in}{1.482994in}}%
\pgfpathlineto{\pgfqpoint{3.020382in}{1.488807in}}%
\pgfpathlineto{\pgfqpoint{3.023058in}{1.485655in}}%
\pgfpathlineto{\pgfqpoint{3.025803in}{1.505111in}}%
\pgfpathlineto{\pgfqpoint{3.028412in}{1.513482in}}%
\pgfpathlineto{\pgfqpoint{3.031091in}{1.509124in}}%
\pgfpathlineto{\pgfqpoint{3.033921in}{1.500642in}}%
\pgfpathlineto{\pgfqpoint{3.036456in}{1.494587in}}%
\pgfpathlineto{\pgfqpoint{3.039262in}{1.503112in}}%
\pgfpathlineto{\pgfqpoint{3.041813in}{1.490888in}}%
\pgfpathlineto{\pgfqpoint{3.044568in}{1.481822in}}%
\pgfpathlineto{\pgfqpoint{3.047157in}{1.478480in}}%
\pgfpathlineto{\pgfqpoint{3.049988in}{1.483852in}}%
\pgfpathlineto{\pgfqpoint{3.052526in}{1.484844in}}%
\pgfpathlineto{\pgfqpoint{3.055202in}{1.481187in}}%
\pgfpathlineto{\pgfqpoint{3.057884in}{1.480609in}}%
\pgfpathlineto{\pgfqpoint{3.060561in}{1.475727in}}%
\pgfpathlineto{\pgfqpoint{3.063230in}{1.474549in}}%
\pgfpathlineto{\pgfqpoint{3.065916in}{1.471103in}}%
\pgfpathlineto{\pgfqpoint{3.068709in}{1.472338in}}%
\pgfpathlineto{\pgfqpoint{3.071266in}{1.473326in}}%
\pgfpathlineto{\pgfqpoint{3.074056in}{1.477632in}}%
\pgfpathlineto{\pgfqpoint{3.076631in}{1.476918in}}%
\pgfpathlineto{\pgfqpoint{3.079381in}{1.471256in}}%
\pgfpathlineto{\pgfqpoint{3.081990in}{1.475930in}}%
\pgfpathlineto{\pgfqpoint{3.084671in}{1.472614in}}%
\pgfpathlineto{\pgfqpoint{3.087343in}{1.473697in}}%
\pgfpathlineto{\pgfqpoint{3.090023in}{1.478035in}}%
\pgfpathlineto{\pgfqpoint{3.092699in}{1.482818in}}%
\pgfpathlineto{\pgfqpoint{3.095388in}{1.478634in}}%
\pgfpathlineto{\pgfqpoint{3.098163in}{1.468189in}}%
\pgfpathlineto{\pgfqpoint{3.100737in}{1.473569in}}%
\pgfpathlineto{\pgfqpoint{3.103508in}{1.486022in}}%
\pgfpathlineto{\pgfqpoint{3.106094in}{1.475222in}}%
\pgfpathlineto{\pgfqpoint{3.108896in}{1.472955in}}%
\pgfpathlineto{\pgfqpoint{3.111451in}{1.467728in}}%
\pgfpathlineto{\pgfqpoint{3.114242in}{1.464305in}}%
\pgfpathlineto{\pgfqpoint{3.116807in}{1.473368in}}%
\pgfpathlineto{\pgfqpoint{3.119487in}{1.474554in}}%
\pgfpathlineto{\pgfqpoint{3.122163in}{1.473259in}}%
\pgfpathlineto{\pgfqpoint{3.124842in}{1.473231in}}%
\pgfpathlineto{\pgfqpoint{3.127512in}{1.474815in}}%
\pgfpathlineto{\pgfqpoint{3.130199in}{1.475466in}}%
\pgfpathlineto{\pgfqpoint{3.132946in}{1.477100in}}%
\pgfpathlineto{\pgfqpoint{3.135550in}{1.476333in}}%
\pgfpathlineto{\pgfqpoint{3.138375in}{1.473014in}}%
\pgfpathlineto{\pgfqpoint{3.140913in}{1.461938in}}%
\pgfpathlineto{\pgfqpoint{3.143740in}{1.461819in}}%
\pgfpathlineto{\pgfqpoint{3.146271in}{1.461819in}}%
\pgfpathlineto{\pgfqpoint{3.149057in}{1.461819in}}%
\pgfpathlineto{\pgfqpoint{3.151612in}{1.461819in}}%
\pgfpathlineto{\pgfqpoint{3.154327in}{1.470042in}}%
\pgfpathlineto{\pgfqpoint{3.156981in}{1.472150in}}%
\pgfpathlineto{\pgfqpoint{3.159675in}{1.471465in}}%
\pgfpathlineto{\pgfqpoint{3.162474in}{1.467530in}}%
\pgfpathlineto{\pgfqpoint{3.165019in}{1.464622in}}%
\pgfpathlineto{\pgfqpoint{3.167776in}{1.461819in}}%
\pgfpathlineto{\pgfqpoint{3.170375in}{1.461819in}}%
\pgfpathlineto{\pgfqpoint{3.173142in}{1.461819in}}%
\pgfpathlineto{\pgfqpoint{3.175724in}{1.462152in}}%
\pgfpathlineto{\pgfqpoint{3.178525in}{1.472053in}}%
\pgfpathlineto{\pgfqpoint{3.181089in}{1.472934in}}%
\pgfpathlineto{\pgfqpoint{3.183760in}{1.466394in}}%
\pgfpathlineto{\pgfqpoint{3.186440in}{1.471963in}}%
\pgfpathlineto{\pgfqpoint{3.189117in}{1.474085in}}%
\pgfpathlineto{\pgfqpoint{3.191796in}{1.477079in}}%
\pgfpathlineto{\pgfqpoint{3.194508in}{1.469274in}}%
\pgfpathlineto{\pgfqpoint{3.197226in}{1.472092in}}%
\pgfpathlineto{\pgfqpoint{3.199823in}{1.468333in}}%
\pgfpathlineto{\pgfqpoint{3.202562in}{1.461819in}}%
\pgfpathlineto{\pgfqpoint{3.205195in}{1.470213in}}%
\pgfpathlineto{\pgfqpoint{3.207984in}{1.471680in}}%
\pgfpathlineto{\pgfqpoint{3.210545in}{1.471329in}}%
\pgfpathlineto{\pgfqpoint{3.213342in}{1.472759in}}%
\pgfpathlineto{\pgfqpoint{3.215908in}{1.473432in}}%
\pgfpathlineto{\pgfqpoint{3.218586in}{1.468064in}}%
\pgfpathlineto{\pgfqpoint{3.221255in}{1.477942in}}%
\pgfpathlineto{\pgfqpoint{3.223942in}{1.480979in}}%
\pgfpathlineto{\pgfqpoint{3.226609in}{1.477693in}}%
\pgfpathlineto{\pgfqpoint{3.229310in}{1.471069in}}%
\pgfpathlineto{\pgfqpoint{3.232069in}{1.470936in}}%
\pgfpathlineto{\pgfqpoint{3.234658in}{1.472057in}}%
\pgfpathlineto{\pgfqpoint{3.237411in}{1.471861in}}%
\pgfpathlineto{\pgfqpoint{3.240010in}{1.476921in}}%
\pgfpathlineto{\pgfqpoint{3.242807in}{1.473907in}}%
\pgfpathlineto{\pgfqpoint{3.245363in}{1.473764in}}%
\pgfpathlineto{\pgfqpoint{3.248049in}{1.473521in}}%
\pgfpathlineto{\pgfqpoint{3.250716in}{1.471614in}}%
\pgfpathlineto{\pgfqpoint{3.253404in}{1.474560in}}%
\pgfpathlineto{\pgfqpoint{3.256083in}{1.472289in}}%
\pgfpathlineto{\pgfqpoint{3.258784in}{1.480725in}}%
\pgfpathlineto{\pgfqpoint{3.261594in}{1.482096in}}%
\pgfpathlineto{\pgfqpoint{3.264119in}{1.487544in}}%
\pgfpathlineto{\pgfqpoint{3.266849in}{1.490514in}}%
\pgfpathlineto{\pgfqpoint{3.269478in}{1.495349in}}%
\pgfpathlineto{\pgfqpoint{3.272254in}{1.487645in}}%
\pgfpathlineto{\pgfqpoint{3.274831in}{1.484135in}}%
\pgfpathlineto{\pgfqpoint{3.277603in}{1.482826in}}%
\pgfpathlineto{\pgfqpoint{3.280189in}{1.484857in}}%
\pgfpathlineto{\pgfqpoint{3.282870in}{1.487926in}}%
\pgfpathlineto{\pgfqpoint{3.285534in}{1.483387in}}%
\pgfpathlineto{\pgfqpoint{3.288225in}{1.484342in}}%
\pgfpathlineto{\pgfqpoint{3.290890in}{1.486932in}}%
\pgfpathlineto{\pgfqpoint{3.293574in}{1.489540in}}%
\pgfpathlineto{\pgfqpoint{3.296376in}{1.484844in}}%
\pgfpathlineto{\pgfqpoint{3.298937in}{1.488018in}}%
\pgfpathlineto{\pgfqpoint{3.301719in}{1.488428in}}%
\pgfpathlineto{\pgfqpoint{3.304295in}{1.483599in}}%
\pgfpathlineto{\pgfqpoint{3.307104in}{1.487707in}}%
\pgfpathlineto{\pgfqpoint{3.309652in}{1.488741in}}%
\pgfpathlineto{\pgfqpoint{3.312480in}{1.487400in}}%
\pgfpathlineto{\pgfqpoint{3.315008in}{1.488450in}}%
\pgfpathlineto{\pgfqpoint{3.317688in}{1.490416in}}%
\pgfpathlineto{\pgfqpoint{3.320366in}{1.484930in}}%
\pgfpathlineto{\pgfqpoint{3.323049in}{1.483915in}}%
\pgfpathlineto{\pgfqpoint{3.325860in}{1.482990in}}%
\pgfpathlineto{\pgfqpoint{3.328401in}{1.481130in}}%
\pgfpathlineto{\pgfqpoint{3.331183in}{1.482406in}}%
\pgfpathlineto{\pgfqpoint{3.333758in}{1.486746in}}%
\pgfpathlineto{\pgfqpoint{3.336541in}{1.490870in}}%
\pgfpathlineto{\pgfqpoint{3.339101in}{1.487360in}}%
\pgfpathlineto{\pgfqpoint{3.341893in}{1.486091in}}%
\pgfpathlineto{\pgfqpoint{3.344468in}{1.485474in}}%
\pgfpathlineto{\pgfqpoint{3.347139in}{1.481673in}}%
\pgfpathlineto{\pgfqpoint{3.349828in}{1.481428in}}%
\pgfpathlineto{\pgfqpoint{3.352505in}{1.484289in}}%
\pgfpathlineto{\pgfqpoint{3.355177in}{1.482547in}}%
\pgfpathlineto{\pgfqpoint{3.357862in}{1.486952in}}%
\pgfpathlineto{\pgfqpoint{3.360620in}{1.485514in}}%
\pgfpathlineto{\pgfqpoint{3.363221in}{1.480203in}}%
\pgfpathlineto{\pgfqpoint{3.365996in}{1.483433in}}%
\pgfpathlineto{\pgfqpoint{3.368577in}{1.480901in}}%
\pgfpathlineto{\pgfqpoint{3.371357in}{1.481876in}}%
\pgfpathlineto{\pgfqpoint{3.373921in}{1.481231in}}%
\pgfpathlineto{\pgfqpoint{3.376735in}{1.480772in}}%
\pgfpathlineto{\pgfqpoint{3.379290in}{1.486014in}}%
\pgfpathlineto{\pgfqpoint{3.381959in}{1.479823in}}%
\pgfpathlineto{\pgfqpoint{3.384647in}{1.481373in}}%
\pgfpathlineto{\pgfqpoint{3.387309in}{1.480259in}}%
\pgfpathlineto{\pgfqpoint{3.390102in}{1.481810in}}%
\pgfpathlineto{\pgfqpoint{3.392681in}{1.477194in}}%
\pgfpathlineto{\pgfqpoint{3.395461in}{1.477193in}}%
\pgfpathlineto{\pgfqpoint{3.398037in}{1.476293in}}%
\pgfpathlineto{\pgfqpoint{3.400783in}{1.481564in}}%
\pgfpathlineto{\pgfqpoint{3.403394in}{1.479599in}}%
\pgfpathlineto{\pgfqpoint{3.406202in}{1.480056in}}%
\pgfpathlineto{\pgfqpoint{3.408752in}{1.483041in}}%
\pgfpathlineto{\pgfqpoint{3.411431in}{1.481576in}}%
\pgfpathlineto{\pgfqpoint{3.414109in}{1.479333in}}%
\pgfpathlineto{\pgfqpoint{3.416780in}{1.476524in}}%
\pgfpathlineto{\pgfqpoint{3.419455in}{1.483981in}}%
\pgfpathlineto{\pgfqpoint{3.422142in}{1.483571in}}%
\pgfpathlineto{\pgfqpoint{3.424887in}{1.484263in}}%
\pgfpathlineto{\pgfqpoint{3.427501in}{1.489528in}}%
\pgfpathlineto{\pgfqpoint{3.430313in}{1.490263in}}%
\pgfpathlineto{\pgfqpoint{3.432851in}{1.485021in}}%
\pgfpathlineto{\pgfqpoint{3.435635in}{1.485275in}}%
\pgfpathlineto{\pgfqpoint{3.438210in}{1.486323in}}%
\pgfpathlineto{\pgfqpoint{3.440996in}{1.482753in}}%
\pgfpathlineto{\pgfqpoint{3.443574in}{1.482398in}}%
\pgfpathlineto{\pgfqpoint{3.446257in}{1.483038in}}%
\pgfpathlineto{\pgfqpoint{3.448926in}{1.488121in}}%
\pgfpathlineto{\pgfqpoint{3.451597in}{1.488003in}}%
\pgfpathlineto{\pgfqpoint{3.454285in}{1.484357in}}%
\pgfpathlineto{\pgfqpoint{3.456960in}{1.483810in}}%
\pgfpathlineto{\pgfqpoint{3.459695in}{1.493780in}}%
\pgfpathlineto{\pgfqpoint{3.462321in}{1.489118in}}%
\pgfpathlineto{\pgfqpoint{3.465072in}{1.489541in}}%
\pgfpathlineto{\pgfqpoint{3.467678in}{1.480809in}}%
\pgfpathlineto{\pgfqpoint{3.470466in}{1.485346in}}%
\pgfpathlineto{\pgfqpoint{3.473021in}{1.493000in}}%
\pgfpathlineto{\pgfqpoint{3.475821in}{1.489871in}}%
\pgfpathlineto{\pgfqpoint{3.478378in}{1.494228in}}%
\pgfpathlineto{\pgfqpoint{3.481072in}{1.490244in}}%
\pgfpathlineto{\pgfqpoint{3.483744in}{1.491697in}}%
\pgfpathlineto{\pgfqpoint{3.486442in}{1.481696in}}%
\pgfpathlineto{\pgfqpoint{3.489223in}{1.486639in}}%
\pgfpathlineto{\pgfqpoint{3.491783in}{1.487788in}}%
\pgfpathlineto{\pgfqpoint{3.494581in}{1.484214in}}%
\pgfpathlineto{\pgfqpoint{3.497139in}{1.477169in}}%
\pgfpathlineto{\pgfqpoint{3.499909in}{1.486355in}}%
\pgfpathlineto{\pgfqpoint{3.502488in}{1.485045in}}%
\pgfpathlineto{\pgfqpoint{3.505262in}{1.478847in}}%
\pgfpathlineto{\pgfqpoint{3.507840in}{1.478244in}}%
\pgfpathlineto{\pgfqpoint{3.510533in}{1.491532in}}%
\pgfpathlineto{\pgfqpoint{3.513209in}{1.486216in}}%
\pgfpathlineto{\pgfqpoint{3.515884in}{1.480696in}}%
\pgfpathlineto{\pgfqpoint{3.518565in}{1.480277in}}%
\pgfpathlineto{\pgfqpoint{3.521244in}{1.481240in}}%
\pgfpathlineto{\pgfqpoint{3.524041in}{1.477818in}}%
\pgfpathlineto{\pgfqpoint{3.526601in}{1.472103in}}%
\pgfpathlineto{\pgfqpoint{3.529327in}{1.477975in}}%
\pgfpathlineto{\pgfqpoint{3.531955in}{1.474304in}}%
\pgfpathlineto{\pgfqpoint{3.534783in}{1.473660in}}%
\pgfpathlineto{\pgfqpoint{3.537309in}{1.475449in}}%
\pgfpathlineto{\pgfqpoint{3.540093in}{1.481264in}}%
\pgfpathlineto{\pgfqpoint{3.542656in}{1.485333in}}%
\pgfpathlineto{\pgfqpoint{3.545349in}{1.474781in}}%
\pgfpathlineto{\pgfqpoint{3.548029in}{1.482243in}}%
\pgfpathlineto{\pgfqpoint{3.550713in}{1.477075in}}%
\pgfpathlineto{\pgfqpoint{3.553498in}{1.475435in}}%
\pgfpathlineto{\pgfqpoint{3.556061in}{1.478765in}}%
\pgfpathlineto{\pgfqpoint{3.558853in}{1.478981in}}%
\pgfpathlineto{\pgfqpoint{3.561420in}{1.479813in}}%
\pgfpathlineto{\pgfqpoint{3.564188in}{1.479298in}}%
\pgfpathlineto{\pgfqpoint{3.566774in}{1.480045in}}%
\pgfpathlineto{\pgfqpoint{3.569584in}{1.478010in}}%
\pgfpathlineto{\pgfqpoint{3.572126in}{1.482805in}}%
\pgfpathlineto{\pgfqpoint{3.574814in}{1.482797in}}%
\pgfpathlineto{\pgfqpoint{3.577487in}{1.476324in}}%
\pgfpathlineto{\pgfqpoint{3.580191in}{1.479398in}}%
\pgfpathlineto{\pgfqpoint{3.582851in}{1.484853in}}%
\pgfpathlineto{\pgfqpoint{3.585532in}{1.481352in}}%
\pgfpathlineto{\pgfqpoint{3.588258in}{1.483436in}}%
\pgfpathlineto{\pgfqpoint{3.590883in}{1.479761in}}%
\pgfpathlineto{\pgfqpoint{3.593620in}{1.481515in}}%
\pgfpathlineto{\pgfqpoint{3.596240in}{1.482460in}}%
\pgfpathlineto{\pgfqpoint{3.598998in}{1.482723in}}%
\pgfpathlineto{\pgfqpoint{3.601590in}{1.483625in}}%
\pgfpathlineto{\pgfqpoint{3.604387in}{1.480545in}}%
\pgfpathlineto{\pgfqpoint{3.606951in}{1.481183in}}%
\pgfpathlineto{\pgfqpoint{3.609632in}{1.478754in}}%
\pgfpathlineto{\pgfqpoint{3.612311in}{1.484365in}}%
\pgfpathlineto{\pgfqpoint{3.614982in}{1.482582in}}%
\pgfpathlineto{\pgfqpoint{3.617667in}{1.481449in}}%
\pgfpathlineto{\pgfqpoint{3.620345in}{1.478374in}}%
\pgfpathlineto{\pgfqpoint{3.623165in}{1.477579in}}%
\pgfpathlineto{\pgfqpoint{3.625689in}{1.478857in}}%
\pgfpathlineto{\pgfqpoint{3.628460in}{1.475570in}}%
\pgfpathlineto{\pgfqpoint{3.631058in}{1.481908in}}%
\pgfpathlineto{\pgfqpoint{3.633858in}{1.479843in}}%
\pgfpathlineto{\pgfqpoint{3.636413in}{1.480908in}}%
\pgfpathlineto{\pgfqpoint{3.639207in}{1.479782in}}%
\pgfpathlineto{\pgfqpoint{3.641773in}{1.484638in}}%
\pgfpathlineto{\pgfqpoint{3.644452in}{1.516418in}}%
\pgfpathlineto{\pgfqpoint{3.647130in}{1.522702in}}%
\pgfpathlineto{\pgfqpoint{3.649837in}{1.520259in}}%
\pgfpathlineto{\pgfqpoint{3.652628in}{1.517948in}}%
\pgfpathlineto{\pgfqpoint{3.655165in}{1.507599in}}%
\pgfpathlineto{\pgfqpoint{3.657917in}{1.492967in}}%
\pgfpathlineto{\pgfqpoint{3.660515in}{1.473400in}}%
\pgfpathlineto{\pgfqpoint{3.663276in}{1.472504in}}%
\pgfpathlineto{\pgfqpoint{3.665864in}{1.477894in}}%
\pgfpathlineto{\pgfqpoint{3.668665in}{1.476907in}}%
\pgfpathlineto{\pgfqpoint{3.671232in}{1.480512in}}%
\pgfpathlineto{\pgfqpoint{3.673911in}{1.482894in}}%
\pgfpathlineto{\pgfqpoint{3.676591in}{1.476004in}}%
\pgfpathlineto{\pgfqpoint{3.679273in}{1.499186in}}%
\pgfpathlineto{\pgfqpoint{3.681948in}{1.507606in}}%
\pgfpathlineto{\pgfqpoint{3.684620in}{1.516050in}}%
\pgfpathlineto{\pgfqpoint{3.687442in}{1.510265in}}%
\pgfpathlineto{\pgfqpoint{3.689983in}{1.503333in}}%
\pgfpathlineto{\pgfqpoint{3.692765in}{1.499002in}}%
\pgfpathlineto{\pgfqpoint{3.695331in}{1.501393in}}%
\pgfpathlineto{\pgfqpoint{3.698125in}{1.500127in}}%
\pgfpathlineto{\pgfqpoint{3.700684in}{1.496955in}}%
\pgfpathlineto{\pgfqpoint{3.703460in}{1.485048in}}%
\pgfpathlineto{\pgfqpoint{3.706053in}{1.484117in}}%
\pgfpathlineto{\pgfqpoint{3.708729in}{1.476533in}}%
\pgfpathlineto{\pgfqpoint{3.711410in}{1.479854in}}%
\pgfpathlineto{\pgfqpoint{3.714086in}{1.482356in}}%
\pgfpathlineto{\pgfqpoint{3.716875in}{1.485280in}}%
\pgfpathlineto{\pgfqpoint{3.719446in}{1.481048in}}%
\pgfpathlineto{\pgfqpoint{3.722228in}{1.486097in}}%
\pgfpathlineto{\pgfqpoint{3.724804in}{1.494358in}}%
\pgfpathlineto{\pgfqpoint{3.727581in}{1.494673in}}%
\pgfpathlineto{\pgfqpoint{3.730158in}{1.491595in}}%
\pgfpathlineto{\pgfqpoint{3.732950in}{1.488296in}}%
\pgfpathlineto{\pgfqpoint{3.735509in}{1.481285in}}%
\pgfpathlineto{\pgfqpoint{3.738194in}{1.489570in}}%
\pgfpathlineto{\pgfqpoint{3.740874in}{1.489785in}}%
\pgfpathlineto{\pgfqpoint{3.743548in}{1.486415in}}%
\pgfpathlineto{\pgfqpoint{3.746229in}{1.481768in}}%
\pgfpathlineto{\pgfqpoint{3.748903in}{1.487596in}}%
\pgfpathlineto{\pgfqpoint{3.751728in}{1.482715in}}%
\pgfpathlineto{\pgfqpoint{3.754265in}{1.484837in}}%
\pgfpathlineto{\pgfqpoint{3.757065in}{1.487407in}}%
\pgfpathlineto{\pgfqpoint{3.759622in}{1.488248in}}%
\pgfpathlineto{\pgfqpoint{3.762389in}{1.484578in}}%
\pgfpathlineto{\pgfqpoint{3.764966in}{1.497707in}}%
\pgfpathlineto{\pgfqpoint{3.767782in}{1.493760in}}%
\pgfpathlineto{\pgfqpoint{3.770323in}{1.514097in}}%
\pgfpathlineto{\pgfqpoint{3.773014in}{1.549371in}}%
\pgfpathlineto{\pgfqpoint{3.775691in}{1.655610in}}%
\pgfpathlineto{\pgfqpoint{3.778370in}{1.675090in}}%
\pgfpathlineto{\pgfqpoint{3.781046in}{1.688129in}}%
\pgfpathlineto{\pgfqpoint{3.783725in}{1.633151in}}%
\pgfpathlineto{\pgfqpoint{3.786504in}{1.590358in}}%
\pgfpathlineto{\pgfqpoint{3.789084in}{1.553536in}}%
\pgfpathlineto{\pgfqpoint{3.791897in}{1.543381in}}%
\pgfpathlineto{\pgfqpoint{3.794435in}{1.531678in}}%
\pgfpathlineto{\pgfqpoint{3.797265in}{1.507319in}}%
\pgfpathlineto{\pgfqpoint{3.799797in}{1.490432in}}%
\pgfpathlineto{\pgfqpoint{3.802569in}{1.489192in}}%
\pgfpathlineto{\pgfqpoint{3.805145in}{1.482129in}}%
\pgfpathlineto{\pgfqpoint{3.807832in}{1.472003in}}%
\pgfpathlineto{\pgfqpoint{3.810510in}{1.472810in}}%
\pgfpathlineto{\pgfqpoint{3.813172in}{1.483160in}}%
\pgfpathlineto{\pgfqpoint{3.815983in}{1.475643in}}%
\pgfpathlineto{\pgfqpoint{3.818546in}{1.476234in}}%
\pgfpathlineto{\pgfqpoint{3.821315in}{1.529357in}}%
\pgfpathlineto{\pgfqpoint{3.823903in}{1.547555in}}%
\pgfpathlineto{\pgfqpoint{3.826679in}{1.519680in}}%
\pgfpathlineto{\pgfqpoint{3.829252in}{1.513061in}}%
\pgfpathlineto{\pgfqpoint{3.832053in}{1.506687in}}%
\pgfpathlineto{\pgfqpoint{3.834616in}{1.504723in}}%
\pgfpathlineto{\pgfqpoint{3.837286in}{1.504355in}}%
\pgfpathlineto{\pgfqpoint{3.839960in}{1.497121in}}%
\pgfpathlineto{\pgfqpoint{3.842641in}{1.495616in}}%
\pgfpathlineto{\pgfqpoint{3.845329in}{1.503200in}}%
\pgfpathlineto{\pgfqpoint{3.848005in}{1.495976in}}%
\pgfpathlineto{\pgfqpoint{3.850814in}{1.486350in}}%
\pgfpathlineto{\pgfqpoint{3.853358in}{1.495392in}}%
\pgfpathlineto{\pgfqpoint{3.856100in}{1.494544in}}%
\pgfpathlineto{\pgfqpoint{3.858720in}{1.486772in}}%
\pgfpathlineto{\pgfqpoint{3.861561in}{1.484937in}}%
\pgfpathlineto{\pgfqpoint{3.864073in}{1.487673in}}%
\pgfpathlineto{\pgfqpoint{3.866815in}{1.480443in}}%
\pgfpathlineto{\pgfqpoint{3.869435in}{1.483757in}}%
\pgfpathlineto{\pgfqpoint{3.872114in}{1.480196in}}%
\pgfpathlineto{\pgfqpoint{3.874790in}{1.480621in}}%
\pgfpathlineto{\pgfqpoint{3.877466in}{1.479877in}}%
\pgfpathlineto{\pgfqpoint{3.880237in}{1.483905in}}%
\pgfpathlineto{\pgfqpoint{3.882850in}{1.481651in}}%
\pgfpathlineto{\pgfqpoint{3.885621in}{1.483643in}}%
\pgfpathlineto{\pgfqpoint{3.888188in}{1.481535in}}%
\pgfpathlineto{\pgfqpoint{3.890926in}{1.483539in}}%
\pgfpathlineto{\pgfqpoint{3.893541in}{1.480014in}}%
\pgfpathlineto{\pgfqpoint{3.896345in}{1.482200in}}%
\pgfpathlineto{\pgfqpoint{3.898891in}{1.477164in}}%
\pgfpathlineto{\pgfqpoint{3.901573in}{1.474257in}}%
\pgfpathlineto{\pgfqpoint{3.904252in}{1.477033in}}%
\pgfpathlineto{\pgfqpoint{3.906918in}{1.476594in}}%
\pgfpathlineto{\pgfqpoint{3.909602in}{1.478670in}}%
\pgfpathlineto{\pgfqpoint{3.912296in}{1.478791in}}%
\pgfpathlineto{\pgfqpoint{3.915107in}{1.474243in}}%
\pgfpathlineto{\pgfqpoint{3.917646in}{1.477497in}}%
\pgfpathlineto{\pgfqpoint{3.920412in}{1.477715in}}%
\pgfpathlineto{\pgfqpoint{3.923005in}{1.482131in}}%
\pgfpathlineto{\pgfqpoint{3.925778in}{1.477700in}}%
\pgfpathlineto{\pgfqpoint{3.928347in}{1.482542in}}%
\pgfpathlineto{\pgfqpoint{3.931202in}{1.487022in}}%
\pgfpathlineto{\pgfqpoint{3.933714in}{1.483623in}}%
\pgfpathlineto{\pgfqpoint{3.936395in}{1.488989in}}%
\pgfpathlineto{\pgfqpoint{3.939075in}{1.487328in}}%
\pgfpathlineto{\pgfqpoint{3.941778in}{1.477601in}}%
\pgfpathlineto{\pgfqpoint{3.944431in}{1.486347in}}%
\pgfpathlineto{\pgfqpoint{3.947101in}{1.484621in}}%
\pgfpathlineto{\pgfqpoint{3.949894in}{1.485968in}}%
\pgfpathlineto{\pgfqpoint{3.952464in}{1.472815in}}%
\pgfpathlineto{\pgfqpoint{3.955211in}{1.478060in}}%
\pgfpathlineto{\pgfqpoint{3.957823in}{1.480267in}}%
\pgfpathlineto{\pgfqpoint{3.960635in}{1.477892in}}%
\pgfpathlineto{\pgfqpoint{3.963176in}{1.476283in}}%
\pgfpathlineto{\pgfqpoint{3.966013in}{1.478292in}}%
\pgfpathlineto{\pgfqpoint{3.968523in}{1.482158in}}%
\pgfpathlineto{\pgfqpoint{3.971250in}{1.482751in}}%
\pgfpathlineto{\pgfqpoint{3.973885in}{1.478376in}}%
\pgfpathlineto{\pgfqpoint{3.976563in}{1.471602in}}%
\pgfpathlineto{\pgfqpoint{3.979389in}{1.475654in}}%
\pgfpathlineto{\pgfqpoint{3.981929in}{1.478126in}}%
\pgfpathlineto{\pgfqpoint{3.984714in}{1.480424in}}%
\pgfpathlineto{\pgfqpoint{3.987270in}{1.478506in}}%
\pgfpathlineto{\pgfqpoint{3.990055in}{1.482299in}}%
\pgfpathlineto{\pgfqpoint{3.992642in}{1.474852in}}%
\pgfpathlineto{\pgfqpoint{3.995417in}{1.480268in}}%
\pgfpathlineto{\pgfqpoint{3.997990in}{1.481405in}}%
\pgfpathlineto{\pgfqpoint{4.000674in}{1.481089in}}%
\pgfpathlineto{\pgfqpoint{4.003348in}{1.483138in}}%
\pgfpathlineto{\pgfqpoint{4.006034in}{1.481592in}}%
\pgfpathlineto{\pgfqpoint{4.008699in}{1.482985in}}%
\pgfpathlineto{\pgfqpoint{4.011394in}{1.482257in}}%
\pgfpathlineto{\pgfqpoint{4.014186in}{1.475697in}}%
\pgfpathlineto{\pgfqpoint{4.016744in}{1.479925in}}%
\pgfpathlineto{\pgfqpoint{4.019518in}{1.477587in}}%
\pgfpathlineto{\pgfqpoint{4.022097in}{1.478469in}}%
\pgfpathlineto{\pgfqpoint{4.024868in}{1.479895in}}%
\pgfpathlineto{\pgfqpoint{4.027447in}{1.481216in}}%
\pgfpathlineto{\pgfqpoint{4.030229in}{1.486744in}}%
\pgfpathlineto{\pgfqpoint{4.032817in}{1.480179in}}%
\pgfpathlineto{\pgfqpoint{4.035492in}{1.478214in}}%
\pgfpathlineto{\pgfqpoint{4.038174in}{1.480326in}}%
\pgfpathlineto{\pgfqpoint{4.040852in}{1.481252in}}%
\pgfpathlineto{\pgfqpoint{4.043667in}{1.480247in}}%
\pgfpathlineto{\pgfqpoint{4.046210in}{1.477778in}}%
\pgfpathlineto{\pgfqpoint{4.049006in}{1.481264in}}%
\pgfpathlineto{\pgfqpoint{4.051557in}{1.482064in}}%
\pgfpathlineto{\pgfqpoint{4.054326in}{1.482295in}}%
\pgfpathlineto{\pgfqpoint{4.056911in}{1.487394in}}%
\pgfpathlineto{\pgfqpoint{4.059702in}{1.486607in}}%
\pgfpathlineto{\pgfqpoint{4.062266in}{1.484093in}}%
\pgfpathlineto{\pgfqpoint{4.064957in}{1.483712in}}%
\pgfpathlineto{\pgfqpoint{4.067636in}{1.483790in}}%
\pgfpathlineto{\pgfqpoint{4.070313in}{1.484054in}}%
\pgfpathlineto{\pgfqpoint{4.072985in}{1.483552in}}%
\pgfpathlineto{\pgfqpoint{4.075705in}{1.487039in}}%
\pgfpathlineto{\pgfqpoint{4.078471in}{1.489911in}}%
\pgfpathlineto{\pgfqpoint{4.081018in}{1.487020in}}%
\pgfpathlineto{\pgfqpoint{4.083870in}{1.485911in}}%
\pgfpathlineto{\pgfqpoint{4.086385in}{1.490549in}}%
\pgfpathlineto{\pgfqpoint{4.089159in}{1.494968in}}%
\pgfpathlineto{\pgfqpoint{4.091729in}{1.488611in}}%
\pgfpathlineto{\pgfqpoint{4.094527in}{1.489696in}}%
\pgfpathlineto{\pgfqpoint{4.097092in}{1.483158in}}%
\pgfpathlineto{\pgfqpoint{4.099777in}{1.490808in}}%
\pgfpathlineto{\pgfqpoint{4.102456in}{1.484387in}}%
\pgfpathlineto{\pgfqpoint{4.105185in}{1.477911in}}%
\pgfpathlineto{\pgfqpoint{4.107814in}{1.481076in}}%
\pgfpathlineto{\pgfqpoint{4.110488in}{1.481935in}}%
\pgfpathlineto{\pgfqpoint{4.113252in}{1.485494in}}%
\pgfpathlineto{\pgfqpoint{4.115844in}{1.482818in}}%
\pgfpathlineto{\pgfqpoint{4.118554in}{1.488881in}}%
\pgfpathlineto{\pgfqpoint{4.121205in}{1.481372in}}%
\pgfpathlineto{\pgfqpoint{4.124019in}{1.484732in}}%
\pgfpathlineto{\pgfqpoint{4.126553in}{1.482380in}}%
\pgfpathlineto{\pgfqpoint{4.129349in}{1.483505in}}%
\pgfpathlineto{\pgfqpoint{4.131920in}{1.483807in}}%
\pgfpathlineto{\pgfqpoint{4.134615in}{1.481658in}}%
\pgfpathlineto{\pgfqpoint{4.137272in}{1.485411in}}%
\pgfpathlineto{\pgfqpoint{4.139963in}{1.484491in}}%
\pgfpathlineto{\pgfqpoint{4.142713in}{1.486792in}}%
\pgfpathlineto{\pgfqpoint{4.145310in}{1.484203in}}%
\pgfpathlineto{\pgfqpoint{4.148082in}{1.483340in}}%
\pgfpathlineto{\pgfqpoint{4.150665in}{1.479416in}}%
\pgfpathlineto{\pgfqpoint{4.153423in}{1.490302in}}%
\pgfpathlineto{\pgfqpoint{4.156016in}{1.486776in}}%
\pgfpathlineto{\pgfqpoint{4.158806in}{1.479971in}}%
\pgfpathlineto{\pgfqpoint{4.161380in}{1.477218in}}%
\pgfpathlineto{\pgfqpoint{4.164059in}{1.478115in}}%
\pgfpathlineto{\pgfqpoint{4.166737in}{1.481088in}}%
\pgfpathlineto{\pgfqpoint{4.169415in}{1.481493in}}%
\pgfpathlineto{\pgfqpoint{4.172093in}{1.485882in}}%
\pgfpathlineto{\pgfqpoint{4.174770in}{1.477688in}}%
\pgfpathlineto{\pgfqpoint{4.177593in}{1.483662in}}%
\pgfpathlineto{\pgfqpoint{4.180129in}{1.486437in}}%
\pgfpathlineto{\pgfqpoint{4.182899in}{1.482786in}}%
\pgfpathlineto{\pgfqpoint{4.185481in}{1.489466in}}%
\pgfpathlineto{\pgfqpoint{4.188318in}{1.480137in}}%
\pgfpathlineto{\pgfqpoint{4.190842in}{1.477586in}}%
\pgfpathlineto{\pgfqpoint{4.193638in}{1.484139in}}%
\pgfpathlineto{\pgfqpoint{4.196186in}{1.482981in}}%
\pgfpathlineto{\pgfqpoint{4.198878in}{1.482538in}}%
\pgfpathlineto{\pgfqpoint{4.201542in}{1.481668in}}%
\pgfpathlineto{\pgfqpoint{4.204240in}{1.477574in}}%
\pgfpathlineto{\pgfqpoint{4.207076in}{1.467609in}}%
\pgfpathlineto{\pgfqpoint{4.209597in}{1.470008in}}%
\pgfpathlineto{\pgfqpoint{4.212383in}{1.468770in}}%
\pgfpathlineto{\pgfqpoint{4.214948in}{1.470073in}}%
\pgfpathlineto{\pgfqpoint{4.217694in}{1.473896in}}%
\pgfpathlineto{\pgfqpoint{4.220304in}{1.480404in}}%
\pgfpathlineto{\pgfqpoint{4.223082in}{1.489414in}}%
\pgfpathlineto{\pgfqpoint{4.225654in}{1.499133in}}%
\pgfpathlineto{\pgfqpoint{4.228331in}{1.519271in}}%
\pgfpathlineto{\pgfqpoint{4.231013in}{1.510451in}}%
\pgfpathlineto{\pgfqpoint{4.233691in}{1.505279in}}%
\pgfpathlineto{\pgfqpoint{4.236375in}{1.503382in}}%
\pgfpathlineto{\pgfqpoint{4.239084in}{1.493295in}}%
\pgfpathlineto{\pgfqpoint{4.241900in}{1.480906in}}%
\pgfpathlineto{\pgfqpoint{4.244394in}{1.482985in}}%
\pgfpathlineto{\pgfqpoint{4.247225in}{1.493731in}}%
\pgfpathlineto{\pgfqpoint{4.249767in}{1.493062in}}%
\pgfpathlineto{\pgfqpoint{4.252581in}{1.493537in}}%
\pgfpathlineto{\pgfqpoint{4.255120in}{1.487503in}}%
\pgfpathlineto{\pgfqpoint{4.257958in}{1.491439in}}%
\pgfpathlineto{\pgfqpoint{4.260477in}{1.486626in}}%
\pgfpathlineto{\pgfqpoint{4.263157in}{1.494713in}}%
\pgfpathlineto{\pgfqpoint{4.265824in}{1.487010in}}%
\pgfpathlineto{\pgfqpoint{4.268590in}{1.492438in}}%
\pgfpathlineto{\pgfqpoint{4.271187in}{1.486520in}}%
\pgfpathlineto{\pgfqpoint{4.273874in}{1.480401in}}%
\pgfpathlineto{\pgfqpoint{4.276635in}{1.475263in}}%
\pgfpathlineto{\pgfqpoint{4.279212in}{1.474763in}}%
\pgfpathlineto{\pgfqpoint{4.282000in}{1.489260in}}%
\pgfpathlineto{\pgfqpoint{4.284586in}{1.486777in}}%
\pgfpathlineto{\pgfqpoint{4.287399in}{1.484939in}}%
\pgfpathlineto{\pgfqpoint{4.289936in}{1.478127in}}%
\pgfpathlineto{\pgfqpoint{4.292786in}{1.488053in}}%
\pgfpathlineto{\pgfqpoint{4.295299in}{1.477491in}}%
\pgfpathlineto{\pgfqpoint{4.297977in}{1.474677in}}%
\pgfpathlineto{\pgfqpoint{4.300656in}{1.471994in}}%
\pgfpathlineto{\pgfqpoint{4.303357in}{1.477928in}}%
\pgfpathlineto{\pgfqpoint{4.306118in}{1.475974in}}%
\pgfpathlineto{\pgfqpoint{4.308691in}{1.474121in}}%
\pgfpathlineto{\pgfqpoint{4.311494in}{1.470470in}}%
\pgfpathlineto{\pgfqpoint{4.314032in}{1.480589in}}%
\pgfpathlineto{\pgfqpoint{4.316856in}{1.475785in}}%
\pgfpathlineto{\pgfqpoint{4.319405in}{1.474216in}}%
\pgfpathlineto{\pgfqpoint{4.322181in}{1.469988in}}%
\pgfpathlineto{\pgfqpoint{4.324760in}{1.474931in}}%
\pgfpathlineto{\pgfqpoint{4.327440in}{1.480344in}}%
\pgfpathlineto{\pgfqpoint{4.330118in}{1.475715in}}%
\pgfpathlineto{\pgfqpoint{4.332796in}{1.472340in}}%
\pgfpathlineto{\pgfqpoint{4.335463in}{1.486671in}}%
\pgfpathlineto{\pgfqpoint{4.338154in}{1.482827in}}%
\pgfpathlineto{\pgfqpoint{4.340976in}{1.486094in}}%
\pgfpathlineto{\pgfqpoint{4.343510in}{1.480893in}}%
\pgfpathlineto{\pgfqpoint{4.346263in}{1.479221in}}%
\pgfpathlineto{\pgfqpoint{4.348868in}{1.481013in}}%
\pgfpathlineto{\pgfqpoint{4.351645in}{1.480414in}}%
\pgfpathlineto{\pgfqpoint{4.354224in}{1.477977in}}%
\pgfpathlineto{\pgfqpoint{4.357014in}{1.482844in}}%
\pgfpathlineto{\pgfqpoint{4.359582in}{1.482460in}}%
\pgfpathlineto{\pgfqpoint{4.362270in}{1.483934in}}%
\pgfpathlineto{\pgfqpoint{4.364936in}{1.479661in}}%
\pgfpathlineto{\pgfqpoint{4.367646in}{1.486771in}}%
\pgfpathlineto{\pgfqpoint{4.370437in}{1.483947in}}%
\pgfpathlineto{\pgfqpoint{4.372976in}{1.480639in}}%
\pgfpathlineto{\pgfqpoint{4.375761in}{1.479391in}}%
\pgfpathlineto{\pgfqpoint{4.378329in}{1.478085in}}%
\pgfpathlineto{\pgfqpoint{4.381097in}{1.485949in}}%
\pgfpathlineto{\pgfqpoint{4.383674in}{1.489834in}}%
\pgfpathlineto{\pgfqpoint{4.386431in}{1.487673in}}%
\pgfpathlineto{\pgfqpoint{4.389044in}{1.482956in}}%
\pgfpathlineto{\pgfqpoint{4.391721in}{1.481626in}}%
\pgfpathlineto{\pgfqpoint{4.394400in}{1.479409in}}%
\pgfpathlineto{\pgfqpoint{4.397076in}{1.475029in}}%
\pgfpathlineto{\pgfqpoint{4.399745in}{1.478443in}}%
\pgfpathlineto{\pgfqpoint{4.402468in}{1.480417in}}%
\pgfpathlineto{\pgfqpoint{4.405234in}{1.482677in}}%
\pgfpathlineto{\pgfqpoint{4.407788in}{1.489083in}}%
\pgfpathlineto{\pgfqpoint{4.410587in}{1.483162in}}%
\pgfpathlineto{\pgfqpoint{4.413149in}{1.481872in}}%
\pgfpathlineto{\pgfqpoint{4.415932in}{1.487273in}}%
\pgfpathlineto{\pgfqpoint{4.418506in}{1.485956in}}%
\pgfpathlineto{\pgfqpoint{4.421292in}{1.488394in}}%
\pgfpathlineto{\pgfqpoint{4.423863in}{1.496417in}}%
\pgfpathlineto{\pgfqpoint{4.426534in}{1.483890in}}%
\pgfpathlineto{\pgfqpoint{4.429220in}{1.484996in}}%
\pgfpathlineto{\pgfqpoint{4.431901in}{1.487124in}}%
\pgfpathlineto{\pgfqpoint{4.434569in}{1.487113in}}%
\pgfpathlineto{\pgfqpoint{4.437253in}{1.484951in}}%
\pgfpathlineto{\pgfqpoint{4.440041in}{1.480887in}}%
\pgfpathlineto{\pgfqpoint{4.442611in}{1.483659in}}%
\pgfpathlineto{\pgfqpoint{4.445423in}{1.482622in}}%
\pgfpathlineto{\pgfqpoint{4.447965in}{1.487283in}}%
\pgfpathlineto{\pgfqpoint{4.450767in}{1.489676in}}%
\pgfpathlineto{\pgfqpoint{4.453312in}{1.485612in}}%
\pgfpathlineto{\pgfqpoint{4.456138in}{1.472824in}}%
\pgfpathlineto{\pgfqpoint{4.458681in}{1.481858in}}%
\pgfpathlineto{\pgfqpoint{4.461367in}{1.480916in}}%
\pgfpathlineto{\pgfqpoint{4.464029in}{1.482216in}}%
\pgfpathlineto{\pgfqpoint{4.466717in}{1.481634in}}%
\pgfpathlineto{\pgfqpoint{4.469492in}{1.483668in}}%
\pgfpathlineto{\pgfqpoint{4.472059in}{1.480564in}}%
\pgfpathlineto{\pgfqpoint{4.474861in}{1.483246in}}%
\pgfpathlineto{\pgfqpoint{4.477430in}{1.480732in}}%
\pgfpathlineto{\pgfqpoint{4.480201in}{1.480645in}}%
\pgfpathlineto{\pgfqpoint{4.482778in}{1.473287in}}%
\pgfpathlineto{\pgfqpoint{4.485581in}{1.475894in}}%
\pgfpathlineto{\pgfqpoint{4.488130in}{1.475779in}}%
\pgfpathlineto{\pgfqpoint{4.490822in}{1.476749in}}%
\pgfpathlineto{\pgfqpoint{4.493492in}{1.472803in}}%
\pgfpathlineto{\pgfqpoint{4.496167in}{1.478427in}}%
\pgfpathlineto{\pgfqpoint{4.498850in}{1.475233in}}%
\pgfpathlineto{\pgfqpoint{4.501529in}{1.484998in}}%
\pgfpathlineto{\pgfqpoint{4.504305in}{1.483458in}}%
\pgfpathlineto{\pgfqpoint{4.506893in}{1.477539in}}%
\pgfpathlineto{\pgfqpoint{4.509643in}{1.481866in}}%
\pgfpathlineto{\pgfqpoint{4.512246in}{1.487463in}}%
\pgfpathlineto{\pgfqpoint{4.515080in}{1.478852in}}%
\pgfpathlineto{\pgfqpoint{4.517598in}{1.474852in}}%
\pgfpathlineto{\pgfqpoint{4.520345in}{1.475612in}}%
\pgfpathlineto{\pgfqpoint{4.522962in}{1.475826in}}%
\pgfpathlineto{\pgfqpoint{4.525640in}{1.485543in}}%
\pgfpathlineto{\pgfqpoint{4.528307in}{1.486413in}}%
\pgfpathlineto{\pgfqpoint{4.530990in}{1.484757in}}%
\pgfpathlineto{\pgfqpoint{4.533764in}{1.484396in}}%
\pgfpathlineto{\pgfqpoint{4.536400in}{1.477655in}}%
\pgfpathlineto{\pgfqpoint{4.539144in}{1.479844in}}%
\pgfpathlineto{\pgfqpoint{4.541711in}{1.480408in}}%
\pgfpathlineto{\pgfqpoint{4.544464in}{1.479265in}}%
\pgfpathlineto{\pgfqpoint{4.547064in}{1.477724in}}%
\pgfpathlineto{\pgfqpoint{4.549822in}{1.483721in}}%
\pgfpathlineto{\pgfqpoint{4.552425in}{1.481884in}}%
\pgfpathlineto{\pgfqpoint{4.555106in}{1.479212in}}%
\pgfpathlineto{\pgfqpoint{4.557777in}{1.474444in}}%
\pgfpathlineto{\pgfqpoint{4.560448in}{1.475848in}}%
\pgfpathlineto{\pgfqpoint{4.563125in}{1.480629in}}%
\pgfpathlineto{\pgfqpoint{4.565820in}{1.485412in}}%
\pgfpathlineto{\pgfqpoint{4.568612in}{1.480662in}}%
\pgfpathlineto{\pgfqpoint{4.571171in}{1.482556in}}%
\pgfpathlineto{\pgfqpoint{4.573947in}{1.482947in}}%
\pgfpathlineto{\pgfqpoint{4.576531in}{1.477077in}}%
\pgfpathlineto{\pgfqpoint{4.579305in}{1.479823in}}%
\pgfpathlineto{\pgfqpoint{4.581888in}{1.478127in}}%
\pgfpathlineto{\pgfqpoint{4.584672in}{1.484250in}}%
\pgfpathlineto{\pgfqpoint{4.587244in}{1.478987in}}%
\pgfpathlineto{\pgfqpoint{4.589920in}{1.473281in}}%
\pgfpathlineto{\pgfqpoint{4.592589in}{1.510211in}}%
\pgfpathlineto{\pgfqpoint{4.595281in}{1.498746in}}%
\pgfpathlineto{\pgfqpoint{4.597951in}{1.489361in}}%
\pgfpathlineto{\pgfqpoint{4.600633in}{1.482358in}}%
\pgfpathlineto{\pgfqpoint{4.603430in}{1.479368in}}%
\pgfpathlineto{\pgfqpoint{4.605990in}{1.474584in}}%
\pgfpathlineto{\pgfqpoint{4.608808in}{1.481956in}}%
\pgfpathlineto{\pgfqpoint{4.611350in}{1.480848in}}%
\pgfpathlineto{\pgfqpoint{4.614134in}{1.475341in}}%
\pgfpathlineto{\pgfqpoint{4.616702in}{1.469817in}}%
\pgfpathlineto{\pgfqpoint{4.619529in}{1.476743in}}%
\pgfpathlineto{\pgfqpoint{4.622056in}{1.474809in}}%
\pgfpathlineto{\pgfqpoint{4.624741in}{1.481299in}}%
\pgfpathlineto{\pgfqpoint{4.627411in}{1.480684in}}%
\pgfpathlineto{\pgfqpoint{4.630096in}{1.476802in}}%
\pgfpathlineto{\pgfqpoint{4.632902in}{1.473165in}}%
\pgfpathlineto{\pgfqpoint{4.635445in}{1.472704in}}%
\pgfpathlineto{\pgfqpoint{4.638204in}{1.477626in}}%
\pgfpathlineto{\pgfqpoint{4.640809in}{1.477437in}}%
\pgfpathlineto{\pgfqpoint{4.643628in}{1.473551in}}%
\pgfpathlineto{\pgfqpoint{4.646169in}{1.475527in}}%
\pgfpathlineto{\pgfqpoint{4.648922in}{1.476472in}}%
\pgfpathlineto{\pgfqpoint{4.651524in}{1.478480in}}%
\pgfpathlineto{\pgfqpoint{4.654203in}{1.479993in}}%
\pgfpathlineto{\pgfqpoint{4.656873in}{1.479883in}}%
\pgfpathlineto{\pgfqpoint{4.659590in}{1.473343in}}%
\pgfpathlineto{\pgfqpoint{4.662237in}{1.476432in}}%
\pgfpathlineto{\pgfqpoint{4.664923in}{1.481196in}}%
\pgfpathlineto{\pgfqpoint{4.667764in}{1.478932in}}%
\pgfpathlineto{\pgfqpoint{4.670261in}{1.484220in}}%
\pgfpathlineto{\pgfqpoint{4.673068in}{1.480667in}}%
\pgfpathlineto{\pgfqpoint{4.675619in}{1.482078in}}%
\pgfpathlineto{\pgfqpoint{4.678448in}{1.484939in}}%
\pgfpathlineto{\pgfqpoint{4.680988in}{1.490542in}}%
\pgfpathlineto{\pgfqpoint{4.683799in}{1.481424in}}%
\pgfpathlineto{\pgfqpoint{4.686337in}{1.479774in}}%
\pgfpathlineto{\pgfqpoint{4.689051in}{1.490988in}}%
\pgfpathlineto{\pgfqpoint{4.691694in}{1.483231in}}%
\pgfpathlineto{\pgfqpoint{4.694381in}{1.480185in}}%
\pgfpathlineto{\pgfqpoint{4.697170in}{1.481250in}}%
\pgfpathlineto{\pgfqpoint{4.699734in}{1.478005in}}%
\pgfpathlineto{\pgfqpoint{4.702517in}{1.481188in}}%
\pgfpathlineto{\pgfqpoint{4.705094in}{1.482409in}}%
\pgfpathlineto{\pgfqpoint{4.707824in}{1.481667in}}%
\pgfpathlineto{\pgfqpoint{4.710437in}{1.474579in}}%
\pgfpathlineto{\pgfqpoint{4.713275in}{1.469340in}}%
\pgfpathlineto{\pgfqpoint{4.715806in}{1.471504in}}%
\pgfpathlineto{\pgfqpoint{4.718486in}{1.479090in}}%
\pgfpathlineto{\pgfqpoint{4.721160in}{1.476544in}}%
\pgfpathlineto{\pgfqpoint{4.723873in}{1.478672in}}%
\pgfpathlineto{\pgfqpoint{4.726508in}{1.478012in}}%
\pgfpathlineto{\pgfqpoint{4.729233in}{1.485835in}}%
\pgfpathlineto{\pgfqpoint{4.731901in}{1.481669in}}%
\pgfpathlineto{\pgfqpoint{4.734552in}{1.480295in}}%
\pgfpathlineto{\pgfqpoint{4.737348in}{1.481390in}}%
\pgfpathlineto{\pgfqpoint{4.739912in}{1.482188in}}%
\pgfpathlineto{\pgfqpoint{4.742696in}{1.491229in}}%
\pgfpathlineto{\pgfqpoint{4.745256in}{1.481257in}}%
\pgfpathlineto{\pgfqpoint{4.748081in}{1.482147in}}%
\pgfpathlineto{\pgfqpoint{4.750627in}{1.476556in}}%
\pgfpathlineto{\pgfqpoint{4.753298in}{1.476572in}}%
\pgfpathlineto{\pgfqpoint{4.755983in}{1.475009in}}%
\pgfpathlineto{\pgfqpoint{4.758653in}{1.488886in}}%
\pgfpathlineto{\pgfqpoint{4.761337in}{1.516699in}}%
\pgfpathlineto{\pgfqpoint{4.764018in}{1.507057in}}%
\pgfpathlineto{\pgfqpoint{4.766783in}{1.498926in}}%
\pgfpathlineto{\pgfqpoint{4.769367in}{1.487702in}}%
\pgfpathlineto{\pgfqpoint{4.772198in}{1.488046in}}%
\pgfpathlineto{\pgfqpoint{4.774732in}{1.492161in}}%
\pgfpathlineto{\pgfqpoint{4.777535in}{1.481217in}}%
\pgfpathlineto{\pgfqpoint{4.780083in}{1.486563in}}%
\pgfpathlineto{\pgfqpoint{4.782872in}{1.483647in}}%
\pgfpathlineto{\pgfqpoint{4.785445in}{1.482037in}}%
\pgfpathlineto{\pgfqpoint{4.788116in}{1.480718in}}%
\pgfpathlineto{\pgfqpoint{4.790798in}{1.480353in}}%
\pgfpathlineto{\pgfqpoint{4.793512in}{1.481776in}}%
\pgfpathlineto{\pgfqpoint{4.796274in}{1.478733in}}%
\pgfpathlineto{\pgfqpoint{4.798830in}{1.485695in}}%
\pgfpathlineto{\pgfqpoint{4.801586in}{1.482061in}}%
\pgfpathlineto{\pgfqpoint{4.804193in}{1.476143in}}%
\pgfpathlineto{\pgfqpoint{4.807017in}{1.473215in}}%
\pgfpathlineto{\pgfqpoint{4.809538in}{1.467057in}}%
\pgfpathlineto{\pgfqpoint{4.812377in}{1.468963in}}%
\pgfpathlineto{\pgfqpoint{4.814907in}{1.466626in}}%
\pgfpathlineto{\pgfqpoint{4.817587in}{1.461819in}}%
\pgfpathlineto{\pgfqpoint{4.820265in}{1.461819in}}%
\pgfpathlineto{\pgfqpoint{4.822945in}{1.466297in}}%
\pgfpathlineto{\pgfqpoint{4.825619in}{1.466239in}}%
\pgfpathlineto{\pgfqpoint{4.828291in}{1.461819in}}%
\pgfpathlineto{\pgfqpoint{4.831045in}{1.461819in}}%
\pgfpathlineto{\pgfqpoint{4.833657in}{1.465671in}}%
\pgfpathlineto{\pgfqpoint{4.837992in}{1.472297in}}%
\pgfpathlineto{\pgfqpoint{4.839922in}{1.469087in}}%
\pgfpathlineto{\pgfqpoint{4.842380in}{1.465676in}}%
\pgfpathlineto{\pgfqpoint{4.844361in}{1.461819in}}%
\pgfpathlineto{\pgfqpoint{4.847127in}{1.463630in}}%
\pgfpathlineto{\pgfqpoint{4.849715in}{1.463784in}}%
\pgfpathlineto{\pgfqpoint{4.852404in}{1.461819in}}%
\pgfpathlineto{\pgfqpoint{4.855070in}{1.461819in}}%
\pgfpathlineto{\pgfqpoint{4.857807in}{1.464183in}}%
\pgfpathlineto{\pgfqpoint{4.860544in}{1.465599in}}%
\pgfpathlineto{\pgfqpoint{4.863116in}{1.471070in}}%
\pgfpathlineto{\pgfqpoint{4.865910in}{1.477646in}}%
\pgfpathlineto{\pgfqpoint{4.868474in}{1.476494in}}%
\pgfpathlineto{\pgfqpoint{4.871209in}{1.474072in}}%
\pgfpathlineto{\pgfqpoint{4.873832in}{1.479408in}}%
\pgfpathlineto{\pgfqpoint{4.876636in}{1.483045in}}%
\pgfpathlineto{\pgfqpoint{4.879180in}{1.475748in}}%
\pgfpathlineto{\pgfqpoint{4.881864in}{1.476524in}}%
\pgfpathlineto{\pgfqpoint{4.884540in}{1.481218in}}%
\pgfpathlineto{\pgfqpoint{4.887211in}{1.481315in}}%
\pgfpathlineto{\pgfqpoint{4.889902in}{1.485078in}}%
\pgfpathlineto{\pgfqpoint{4.892611in}{1.481366in}}%
\pgfpathlineto{\pgfqpoint{4.895399in}{1.478068in}}%
\pgfpathlineto{\pgfqpoint{4.897938in}{1.487155in}}%
\pgfpathlineto{\pgfqpoint{4.900712in}{1.491404in}}%
\pgfpathlineto{\pgfqpoint{4.903295in}{1.487196in}}%
\pgfpathlineto{\pgfqpoint{4.906096in}{1.480514in}}%
\pgfpathlineto{\pgfqpoint{4.908648in}{1.475504in}}%
\pgfpathlineto{\pgfqpoint{4.911435in}{1.486289in}}%
\pgfpathlineto{\pgfqpoint{4.914009in}{1.488875in}}%
\pgfpathlineto{\pgfqpoint{4.916681in}{1.484164in}}%
\pgfpathlineto{\pgfqpoint{4.919352in}{1.487619in}}%
\pgfpathlineto{\pgfqpoint{4.922041in}{1.490994in}}%
\pgfpathlineto{\pgfqpoint{4.924708in}{1.485518in}}%
\pgfpathlineto{\pgfqpoint{4.927400in}{1.484695in}}%
\pgfpathlineto{\pgfqpoint{4.930170in}{1.484475in}}%
\pgfpathlineto{\pgfqpoint{4.932742in}{1.482039in}}%
\pgfpathlineto{\pgfqpoint{4.935515in}{1.482528in}}%
\pgfpathlineto{\pgfqpoint{4.938112in}{1.481127in}}%
\pgfpathlineto{\pgfqpoint{4.940881in}{1.481442in}}%
\pgfpathlineto{\pgfqpoint{4.943466in}{1.480688in}}%
\pgfpathlineto{\pgfqpoint{4.946151in}{1.483885in}}%
\pgfpathlineto{\pgfqpoint{4.948827in}{1.484892in}}%
\pgfpathlineto{\pgfqpoint{4.951504in}{1.503532in}}%
\pgfpathlineto{\pgfqpoint{4.954182in}{1.510122in}}%
\pgfpathlineto{\pgfqpoint{4.956862in}{1.527596in}}%
\pgfpathlineto{\pgfqpoint{4.959689in}{1.513026in}}%
\pgfpathlineto{\pgfqpoint{4.962219in}{1.507447in}}%
\pgfpathlineto{\pgfqpoint{4.965002in}{1.501498in}}%
\pgfpathlineto{\pgfqpoint{4.967575in}{1.501282in}}%
\pgfpathlineto{\pgfqpoint{4.970314in}{1.500066in}}%
\pgfpathlineto{\pgfqpoint{4.972933in}{1.493489in}}%
\pgfpathlineto{\pgfqpoint{4.975703in}{1.495272in}}%
\pgfpathlineto{\pgfqpoint{4.978287in}{1.484679in}}%
\pgfpathlineto{\pgfqpoint{4.980967in}{1.482947in}}%
\pgfpathlineto{\pgfqpoint{4.983637in}{1.482074in}}%
\pgfpathlineto{\pgfqpoint{4.986325in}{1.483104in}}%
\pgfpathlineto{\pgfqpoint{4.989001in}{1.481898in}}%
\pgfpathlineto{\pgfqpoint{4.991683in}{1.485012in}}%
\pgfpathlineto{\pgfqpoint{4.994390in}{1.482849in}}%
\pgfpathlineto{\pgfqpoint{4.997028in}{1.478446in}}%
\pgfpathlineto{\pgfqpoint{4.999780in}{1.477664in}}%
\pgfpathlineto{\pgfqpoint{5.002384in}{1.473257in}}%
\pgfpathlineto{\pgfqpoint{5.005178in}{1.477369in}}%
\pgfpathlineto{\pgfqpoint{5.007751in}{1.475330in}}%
\pgfpathlineto{\pgfqpoint{5.010562in}{1.474104in}}%
\pgfpathlineto{\pgfqpoint{5.013104in}{1.477045in}}%
\pgfpathlineto{\pgfqpoint{5.015820in}{1.472366in}}%
\pgfpathlineto{\pgfqpoint{5.018466in}{1.472950in}}%
\pgfpathlineto{\pgfqpoint{5.021147in}{1.475502in}}%
\pgfpathlineto{\pgfqpoint{5.023927in}{1.475145in}}%
\pgfpathlineto{\pgfqpoint{5.026501in}{1.475317in}}%
\pgfpathlineto{\pgfqpoint{5.029275in}{1.482159in}}%
\pgfpathlineto{\pgfqpoint{5.031849in}{1.473316in}}%
\pgfpathlineto{\pgfqpoint{5.034649in}{1.491587in}}%
\pgfpathlineto{\pgfqpoint{5.037214in}{1.479034in}}%
\pgfpathlineto{\pgfqpoint{5.039962in}{1.477499in}}%
\pgfpathlineto{\pgfqpoint{5.042572in}{1.480419in}}%
\pgfpathlineto{\pgfqpoint{5.045249in}{1.473757in}}%
\pgfpathlineto{\pgfqpoint{5.047924in}{1.474305in}}%
\pgfpathlineto{\pgfqpoint{5.050606in}{1.476068in}}%
\pgfpathlineto{\pgfqpoint{5.053284in}{1.481993in}}%
\pgfpathlineto{\pgfqpoint{5.055952in}{1.476524in}}%
\pgfpathlineto{\pgfqpoint{5.058711in}{1.480937in}}%
\pgfpathlineto{\pgfqpoint{5.061315in}{1.474154in}}%
\pgfpathlineto{\pgfqpoint{5.064144in}{1.477312in}}%
\pgfpathlineto{\pgfqpoint{5.066677in}{1.477019in}}%
\pgfpathlineto{\pgfqpoint{5.069463in}{1.483659in}}%
\pgfpathlineto{\pgfqpoint{5.072030in}{1.484675in}}%
\pgfpathlineto{\pgfqpoint{5.074851in}{1.482476in}}%
\pgfpathlineto{\pgfqpoint{5.077390in}{1.472773in}}%
\pgfpathlineto{\pgfqpoint{5.080067in}{1.477132in}}%
\pgfpathlineto{\pgfqpoint{5.082746in}{1.482325in}}%
\pgfpathlineto{\pgfqpoint{5.085426in}{1.481900in}}%
\pgfpathlineto{\pgfqpoint{5.088103in}{1.482667in}}%
\pgfpathlineto{\pgfqpoint{5.090788in}{1.482890in}}%
\pgfpathlineto{\pgfqpoint{5.093579in}{1.470919in}}%
\pgfpathlineto{\pgfqpoint{5.096142in}{1.471935in}}%
\pgfpathlineto{\pgfqpoint{5.098948in}{1.474285in}}%
\pgfpathlineto{\pgfqpoint{5.101496in}{1.475384in}}%
\pgfpathlineto{\pgfqpoint{5.104312in}{1.479966in}}%
\pgfpathlineto{\pgfqpoint{5.106842in}{1.480330in}}%
\pgfpathlineto{\pgfqpoint{5.109530in}{1.471832in}}%
\pgfpathlineto{\pgfqpoint{5.112209in}{1.476816in}}%
\pgfpathlineto{\pgfqpoint{5.114887in}{1.481884in}}%
\pgfpathlineto{\pgfqpoint{5.117550in}{1.475806in}}%
\pgfpathlineto{\pgfqpoint{5.120243in}{1.483634in}}%
\pgfpathlineto{\pgfqpoint{5.123042in}{1.473478in}}%
\pgfpathlineto{\pgfqpoint{5.125599in}{1.483914in}}%
\pgfpathlineto{\pgfqpoint{5.128421in}{1.474489in}}%
\pgfpathlineto{\pgfqpoint{5.130953in}{1.476807in}}%
\pgfpathlineto{\pgfqpoint{5.133716in}{1.483262in}}%
\pgfpathlineto{\pgfqpoint{5.136311in}{1.476711in}}%
\pgfpathlineto{\pgfqpoint{5.139072in}{1.478925in}}%
\pgfpathlineto{\pgfqpoint{5.141660in}{1.480375in}}%
\pgfpathlineto{\pgfqpoint{5.144349in}{1.476135in}}%
\pgfpathlineto{\pgfqpoint{5.147029in}{1.480088in}}%
\pgfpathlineto{\pgfqpoint{5.149734in}{1.474121in}}%
\pgfpathlineto{\pgfqpoint{5.152382in}{1.462070in}}%
\pgfpathlineto{\pgfqpoint{5.155059in}{1.462672in}}%
\pgfpathlineto{\pgfqpoint{5.157815in}{1.468244in}}%
\pgfpathlineto{\pgfqpoint{5.160420in}{1.472903in}}%
\pgfpathlineto{\pgfqpoint{5.163243in}{1.472796in}}%
\pgfpathlineto{\pgfqpoint{5.165775in}{1.474520in}}%
\pgfpathlineto{\pgfqpoint{5.168591in}{1.474324in}}%
\pgfpathlineto{\pgfqpoint{5.171133in}{1.472883in}}%
\pgfpathlineto{\pgfqpoint{5.173925in}{1.474321in}}%
\pgfpathlineto{\pgfqpoint{5.176477in}{1.475646in}}%
\pgfpathlineto{\pgfqpoint{5.179188in}{1.476297in}}%
\pgfpathlineto{\pgfqpoint{5.181848in}{1.470348in}}%
\pgfpathlineto{\pgfqpoint{5.184522in}{1.476246in}}%
\pgfpathlineto{\pgfqpoint{5.187294in}{1.477193in}}%
\pgfpathlineto{\pgfqpoint{5.189880in}{1.480622in}}%
\pgfpathlineto{\pgfqpoint{5.192680in}{1.483987in}}%
\pgfpathlineto{\pgfqpoint{5.195239in}{1.483735in}}%
\pgfpathlineto{\pgfqpoint{5.198008in}{1.488585in}}%
\pgfpathlineto{\pgfqpoint{5.200594in}{1.486318in}}%
\pgfpathlineto{\pgfqpoint{5.203388in}{1.487711in}}%
\pgfpathlineto{\pgfqpoint{5.205952in}{1.490765in}}%
\pgfpathlineto{\pgfqpoint{5.208630in}{1.484690in}}%
\pgfpathlineto{\pgfqpoint{5.211299in}{1.480741in}}%
\pgfpathlineto{\pgfqpoint{5.214027in}{1.481016in}}%
\pgfpathlineto{\pgfqpoint{5.216667in}{1.481462in}}%
\pgfpathlineto{\pgfqpoint{5.219345in}{1.477600in}}%
\pgfpathlineto{\pgfqpoint{5.222151in}{1.477208in}}%
\pgfpathlineto{\pgfqpoint{5.224695in}{1.470556in}}%
\pgfpathlineto{\pgfqpoint{5.227470in}{1.475861in}}%
\pgfpathlineto{\pgfqpoint{5.230059in}{1.478291in}}%
\pgfpathlineto{\pgfqpoint{5.232855in}{1.480147in}}%
\pgfpathlineto{\pgfqpoint{5.235409in}{1.468786in}}%
\pgfpathlineto{\pgfqpoint{5.238173in}{1.478147in}}%
\pgfpathlineto{\pgfqpoint{5.240777in}{1.478634in}}%
\pgfpathlineto{\pgfqpoint{5.243445in}{1.477961in}}%
\pgfpathlineto{\pgfqpoint{5.246130in}{1.486068in}}%
\pgfpathlineto{\pgfqpoint{5.248816in}{1.482173in}}%
\pgfpathlineto{\pgfqpoint{5.251590in}{1.477413in}}%
\pgfpathlineto{\pgfqpoint{5.254236in}{1.477527in}}%
\pgfpathlineto{\pgfqpoint{5.256973in}{1.480547in}}%
\pgfpathlineto{\pgfqpoint{5.259511in}{1.479333in}}%
\pgfpathlineto{\pgfqpoint{5.262264in}{1.476450in}}%
\pgfpathlineto{\pgfqpoint{5.264876in}{1.490115in}}%
\pgfpathlineto{\pgfqpoint{5.267691in}{1.488236in}}%
\pgfpathlineto{\pgfqpoint{5.270238in}{1.481690in}}%
\pgfpathlineto{\pgfqpoint{5.272913in}{1.492024in}}%
\pgfpathlineto{\pgfqpoint{5.275589in}{1.496655in}}%
\pgfpathlineto{\pgfqpoint{5.278322in}{1.490228in}}%
\pgfpathlineto{\pgfqpoint{5.280947in}{1.495602in}}%
\pgfpathlineto{\pgfqpoint{5.283631in}{1.491816in}}%
\pgfpathlineto{\pgfqpoint{5.286436in}{1.503059in}}%
\pgfpathlineto{\pgfqpoint{5.288984in}{1.495214in}}%
\pgfpathlineto{\pgfqpoint{5.291794in}{1.493337in}}%
\pgfpathlineto{\pgfqpoint{5.294339in}{1.494877in}}%
\pgfpathlineto{\pgfqpoint{5.297140in}{1.494353in}}%
\pgfpathlineto{\pgfqpoint{5.299696in}{1.485548in}}%
\pgfpathlineto{\pgfqpoint{5.302443in}{1.483573in}}%
\pgfpathlineto{\pgfqpoint{5.305054in}{1.488872in}}%
\pgfpathlineto{\pgfqpoint{5.307731in}{1.491259in}}%
\pgfpathlineto{\pgfqpoint{5.310411in}{1.488715in}}%
\pgfpathlineto{\pgfqpoint{5.313089in}{1.487546in}}%
\pgfpathlineto{\pgfqpoint{5.315754in}{1.487722in}}%
\pgfpathlineto{\pgfqpoint{5.318430in}{1.487579in}}%
\pgfpathlineto{\pgfqpoint{5.321256in}{1.487820in}}%
\pgfpathlineto{\pgfqpoint{5.323802in}{1.480107in}}%
\pgfpathlineto{\pgfqpoint{5.326564in}{1.484655in}}%
\pgfpathlineto{\pgfqpoint{5.329159in}{1.481115in}}%
\pgfpathlineto{\pgfqpoint{5.331973in}{1.468011in}}%
\pgfpathlineto{\pgfqpoint{5.334510in}{1.473683in}}%
\pgfpathlineto{\pgfqpoint{5.337353in}{1.478694in}}%
\pgfpathlineto{\pgfqpoint{5.339872in}{1.477097in}}%
\pgfpathlineto{\pgfqpoint{5.342549in}{1.465320in}}%
\pgfpathlineto{\pgfqpoint{5.345224in}{1.461819in}}%
\pgfpathlineto{\pgfqpoint{5.347905in}{1.465737in}}%
\pgfpathlineto{\pgfqpoint{5.350723in}{1.474857in}}%
\pgfpathlineto{\pgfqpoint{5.353262in}{1.478815in}}%
\pgfpathlineto{\pgfqpoint{5.356056in}{1.479074in}}%
\pgfpathlineto{\pgfqpoint{5.358612in}{1.480109in}}%
\pgfpathlineto{\pgfqpoint{5.361370in}{1.483939in}}%
\pgfpathlineto{\pgfqpoint{5.363966in}{1.482242in}}%
\pgfpathlineto{\pgfqpoint{5.366727in}{1.479194in}}%
\pgfpathlineto{\pgfqpoint{5.369335in}{1.480145in}}%
\pgfpathlineto{\pgfqpoint{5.372013in}{1.475883in}}%
\pgfpathlineto{\pgfqpoint{5.374692in}{1.479725in}}%
\pgfpathlineto{\pgfqpoint{5.377370in}{1.481341in}}%
\pgfpathlineto{\pgfqpoint{5.380048in}{1.481843in}}%
\pgfpathlineto{\pgfqpoint{5.382725in}{1.479177in}}%
\pgfpathlineto{\pgfqpoint{5.385550in}{1.480409in}}%
\pgfpathlineto{\pgfqpoint{5.388083in}{1.480843in}}%
\pgfpathlineto{\pgfqpoint{5.390900in}{1.482329in}}%
\pgfpathlineto{\pgfqpoint{5.393441in}{1.469786in}}%
\pgfpathlineto{\pgfqpoint{5.396219in}{1.474116in}}%
\pgfpathlineto{\pgfqpoint{5.398784in}{1.469770in}}%
\pgfpathlineto{\pgfqpoint{5.401576in}{1.474630in}}%
\pgfpathlineto{\pgfqpoint{5.404154in}{1.475439in}}%
\pgfpathlineto{\pgfqpoint{5.406832in}{1.479461in}}%
\pgfpathlineto{\pgfqpoint{5.409507in}{1.475827in}}%
\pgfpathlineto{\pgfqpoint{5.412190in}{1.478002in}}%
\pgfpathlineto{\pgfqpoint{5.414954in}{1.482812in}}%
\pgfpathlineto{\pgfqpoint{5.417547in}{1.476624in}}%
\pgfpathlineto{\pgfqpoint{5.420304in}{1.478641in}}%
\pgfpathlineto{\pgfqpoint{5.422897in}{1.487657in}}%
\pgfpathlineto{\pgfqpoint{5.425661in}{1.483047in}}%
\pgfpathlineto{\pgfqpoint{5.428259in}{1.484460in}}%
\pgfpathlineto{\pgfqpoint{5.431015in}{1.482193in}}%
\pgfpathlineto{\pgfqpoint{5.433616in}{1.480508in}}%
\pgfpathlineto{\pgfqpoint{5.436295in}{1.479396in}}%
\pgfpathlineto{\pgfqpoint{5.438974in}{1.482931in}}%
\pgfpathlineto{\pgfqpoint{5.441698in}{1.478229in}}%
\pgfpathlineto{\pgfqpoint{5.444328in}{1.476691in}}%
\pgfpathlineto{\pgfqpoint{5.447021in}{1.476873in}}%
\pgfpathlineto{\pgfqpoint{5.449769in}{1.483481in}}%
\pgfpathlineto{\pgfqpoint{5.452365in}{1.486588in}}%
\pgfpathlineto{\pgfqpoint{5.455168in}{1.485316in}}%
\pgfpathlineto{\pgfqpoint{5.457721in}{1.478559in}}%
\pgfpathlineto{\pgfqpoint{5.460489in}{1.483462in}}%
\pgfpathlineto{\pgfqpoint{5.463079in}{1.480021in}}%
\pgfpathlineto{\pgfqpoint{5.465888in}{1.477268in}}%
\pgfpathlineto{\pgfqpoint{5.468425in}{1.476517in}}%
\pgfpathlineto{\pgfqpoint{5.471113in}{1.481823in}}%
\pgfpathlineto{\pgfqpoint{5.473792in}{1.475184in}}%
\pgfpathlineto{\pgfqpoint{5.476458in}{1.475647in}}%
\pgfpathlineto{\pgfqpoint{5.479152in}{1.478101in}}%
\pgfpathlineto{\pgfqpoint{5.481825in}{1.478981in}}%
\pgfpathlineto{\pgfqpoint{5.484641in}{1.480988in}}%
\pgfpathlineto{\pgfqpoint{5.487176in}{1.477461in}}%
\pgfpathlineto{\pgfqpoint{5.490000in}{1.479130in}}%
\pgfpathlineto{\pgfqpoint{5.492541in}{1.480050in}}%
\pgfpathlineto{\pgfqpoint{5.495346in}{1.476595in}}%
\pgfpathlineto{\pgfqpoint{5.497898in}{1.478059in}}%
\pgfpathlineto{\pgfqpoint{5.500687in}{1.478538in}}%
\pgfpathlineto{\pgfqpoint{5.503255in}{1.478249in}}%
\pgfpathlineto{\pgfqpoint{5.505933in}{1.486210in}}%
\pgfpathlineto{\pgfqpoint{5.508612in}{1.481842in}}%
\pgfpathlineto{\pgfqpoint{5.511290in}{1.476004in}}%
\pgfpathlineto{\pgfqpoint{5.514080in}{1.483634in}}%
\pgfpathlineto{\pgfqpoint{5.516646in}{1.485898in}}%
\pgfpathlineto{\pgfqpoint{5.519433in}{1.483667in}}%
\pgfpathlineto{\pgfqpoint{5.522003in}{1.484939in}}%
\pgfpathlineto{\pgfqpoint{5.524756in}{1.488720in}}%
\pgfpathlineto{\pgfqpoint{5.527360in}{1.479684in}}%
\pgfpathlineto{\pgfqpoint{5.530148in}{1.480516in}}%
\pgfpathlineto{\pgfqpoint{5.532717in}{1.477854in}}%
\pgfpathlineto{\pgfqpoint{5.535395in}{1.484429in}}%
\pgfpathlineto{\pgfqpoint{5.538074in}{1.481565in}}%
\pgfpathlineto{\pgfqpoint{5.540750in}{1.475137in}}%
\pgfpathlineto{\pgfqpoint{5.543421in}{1.484383in}}%
\pgfpathlineto{\pgfqpoint{5.546110in}{1.478874in}}%
\pgfpathlineto{\pgfqpoint{5.548921in}{1.483530in}}%
\pgfpathlineto{\pgfqpoint{5.551457in}{1.474391in}}%
\pgfpathlineto{\pgfqpoint{5.554198in}{1.477838in}}%
\pgfpathlineto{\pgfqpoint{5.556822in}{1.477263in}}%
\pgfpathlineto{\pgfqpoint{5.559612in}{1.478465in}}%
\pgfpathlineto{\pgfqpoint{5.562180in}{1.472902in}}%
\pgfpathlineto{\pgfqpoint{5.564940in}{1.469968in}}%
\pgfpathlineto{\pgfqpoint{5.567536in}{1.474855in}}%
\pgfpathlineto{\pgfqpoint{5.570215in}{1.477325in}}%
\pgfpathlineto{\pgfqpoint{5.572893in}{1.474259in}}%
\pgfpathlineto{\pgfqpoint{5.575596in}{1.478620in}}%
\pgfpathlineto{\pgfqpoint{5.578342in}{1.475727in}}%
\pgfpathlineto{\pgfqpoint{5.580914in}{1.477439in}}%
\pgfpathlineto{\pgfqpoint{5.583709in}{1.477580in}}%
\pgfpathlineto{\pgfqpoint{5.586269in}{1.482506in}}%
\pgfpathlineto{\pgfqpoint{5.589040in}{1.485334in}}%
\pgfpathlineto{\pgfqpoint{5.591641in}{1.482868in}}%
\pgfpathlineto{\pgfqpoint{5.594368in}{1.482800in}}%
\pgfpathlineto{\pgfqpoint{5.596999in}{1.486218in}}%
\pgfpathlineto{\pgfqpoint{5.599674in}{1.482884in}}%
\pgfpathlineto{\pgfqpoint{5.602352in}{1.483186in}}%
\pgfpathlineto{\pgfqpoint{5.605073in}{1.486783in}}%
\pgfpathlineto{\pgfqpoint{5.607698in}{1.487996in}}%
\pgfpathlineto{\pgfqpoint{5.610389in}{1.490668in}}%
\pgfpathlineto{\pgfqpoint{5.613235in}{1.484286in}}%
\pgfpathlineto{\pgfqpoint{5.615743in}{1.481732in}}%
\pgfpathlineto{\pgfqpoint{5.618526in}{1.486861in}}%
\pgfpathlineto{\pgfqpoint{5.621102in}{1.479051in}}%
\pgfpathlineto{\pgfqpoint{5.623868in}{1.482223in}}%
\pgfpathlineto{\pgfqpoint{5.626460in}{1.482741in}}%
\pgfpathlineto{\pgfqpoint{5.629232in}{1.481891in}}%
\pgfpathlineto{\pgfqpoint{5.631815in}{1.482173in}}%
\pgfpathlineto{\pgfqpoint{5.634496in}{1.485190in}}%
\pgfpathlineto{\pgfqpoint{5.637172in}{1.491949in}}%
\pgfpathlineto{\pgfqpoint{5.639852in}{1.514758in}}%
\pgfpathlineto{\pgfqpoint{5.642518in}{1.546802in}}%
\pgfpathlineto{\pgfqpoint{5.645243in}{1.534525in}}%
\pgfpathlineto{\pgfqpoint{5.648008in}{1.514114in}}%
\pgfpathlineto{\pgfqpoint{5.650563in}{1.528218in}}%
\pgfpathlineto{\pgfqpoint{5.653376in}{1.549105in}}%
\pgfpathlineto{\pgfqpoint{5.655919in}{1.532230in}}%
\pgfpathlineto{\pgfqpoint{5.658723in}{1.522553in}}%
\pgfpathlineto{\pgfqpoint{5.661273in}{1.510365in}}%
\pgfpathlineto{\pgfqpoint{5.664099in}{1.496253in}}%
\pgfpathlineto{\pgfqpoint{5.666632in}{1.497457in}}%
\pgfpathlineto{\pgfqpoint{5.669313in}{1.501656in}}%
\pgfpathlineto{\pgfqpoint{5.671991in}{1.495593in}}%
\pgfpathlineto{\pgfqpoint{5.674667in}{1.489502in}}%
\pgfpathlineto{\pgfqpoint{5.677486in}{1.489671in}}%
\pgfpathlineto{\pgfqpoint{5.680027in}{1.488999in}}%
\pgfpathlineto{\pgfqpoint{5.682836in}{1.485443in}}%
\pgfpathlineto{\pgfqpoint{5.685385in}{1.488653in}}%
\pgfpathlineto{\pgfqpoint{5.688159in}{1.482098in}}%
\pgfpathlineto{\pgfqpoint{5.690730in}{1.484911in}}%
\pgfpathlineto{\pgfqpoint{5.693473in}{1.505009in}}%
\pgfpathlineto{\pgfqpoint{5.696101in}{1.511487in}}%
\pgfpathlineto{\pgfqpoint{5.698775in}{1.498348in}}%
\pgfpathlineto{\pgfqpoint{5.701453in}{1.491084in}}%
\pgfpathlineto{\pgfqpoint{5.704130in}{1.496599in}}%
\pgfpathlineto{\pgfqpoint{5.706800in}{1.485514in}}%
\pgfpathlineto{\pgfqpoint{5.709490in}{1.485934in}}%
\pgfpathlineto{\pgfqpoint{5.712291in}{1.484960in}}%
\pgfpathlineto{\pgfqpoint{5.714834in}{1.474462in}}%
\pgfpathlineto{\pgfqpoint{5.717671in}{1.479947in}}%
\pgfpathlineto{\pgfqpoint{5.720201in}{1.473609in}}%
\pgfpathlineto{\pgfqpoint{5.722950in}{1.472566in}}%
\pgfpathlineto{\pgfqpoint{5.725548in}{1.479508in}}%
\pgfpathlineto{\pgfqpoint{5.728339in}{1.481760in}}%
\pgfpathlineto{\pgfqpoint{5.730919in}{1.486891in}}%
\pgfpathlineto{\pgfqpoint{5.733594in}{1.481103in}}%
\pgfpathlineto{\pgfqpoint{5.736276in}{1.478673in}}%
\pgfpathlineto{\pgfqpoint{5.738974in}{1.483026in}}%
\pgfpathlineto{\pgfqpoint{5.741745in}{1.488485in}}%
\pgfpathlineto{\pgfqpoint{5.744310in}{1.485165in}}%
\pgfpathlineto{\pgfqpoint{5.744310in}{0.413320in}}%
\pgfpathlineto{\pgfqpoint{5.744310in}{0.413320in}}%
\pgfpathlineto{\pgfqpoint{5.741745in}{0.413320in}}%
\pgfpathlineto{\pgfqpoint{5.738974in}{0.413320in}}%
\pgfpathlineto{\pgfqpoint{5.736276in}{0.413320in}}%
\pgfpathlineto{\pgfqpoint{5.733594in}{0.413320in}}%
\pgfpathlineto{\pgfqpoint{5.730919in}{0.413320in}}%
\pgfpathlineto{\pgfqpoint{5.728339in}{0.413320in}}%
\pgfpathlineto{\pgfqpoint{5.725548in}{0.413320in}}%
\pgfpathlineto{\pgfqpoint{5.722950in}{0.413320in}}%
\pgfpathlineto{\pgfqpoint{5.720201in}{0.413320in}}%
\pgfpathlineto{\pgfqpoint{5.717671in}{0.413320in}}%
\pgfpathlineto{\pgfqpoint{5.714834in}{0.413320in}}%
\pgfpathlineto{\pgfqpoint{5.712291in}{0.413320in}}%
\pgfpathlineto{\pgfqpoint{5.709490in}{0.413320in}}%
\pgfpathlineto{\pgfqpoint{5.706800in}{0.413320in}}%
\pgfpathlineto{\pgfqpoint{5.704130in}{0.413320in}}%
\pgfpathlineto{\pgfqpoint{5.701453in}{0.413320in}}%
\pgfpathlineto{\pgfqpoint{5.698775in}{0.413320in}}%
\pgfpathlineto{\pgfqpoint{5.696101in}{0.413320in}}%
\pgfpathlineto{\pgfqpoint{5.693473in}{0.413320in}}%
\pgfpathlineto{\pgfqpoint{5.690730in}{0.413320in}}%
\pgfpathlineto{\pgfqpoint{5.688159in}{0.413320in}}%
\pgfpathlineto{\pgfqpoint{5.685385in}{0.413320in}}%
\pgfpathlineto{\pgfqpoint{5.682836in}{0.413320in}}%
\pgfpathlineto{\pgfqpoint{5.680027in}{0.413320in}}%
\pgfpathlineto{\pgfqpoint{5.677486in}{0.413320in}}%
\pgfpathlineto{\pgfqpoint{5.674667in}{0.413320in}}%
\pgfpathlineto{\pgfqpoint{5.671991in}{0.413320in}}%
\pgfpathlineto{\pgfqpoint{5.669313in}{0.413320in}}%
\pgfpathlineto{\pgfqpoint{5.666632in}{0.413320in}}%
\pgfpathlineto{\pgfqpoint{5.664099in}{0.413320in}}%
\pgfpathlineto{\pgfqpoint{5.661273in}{0.413320in}}%
\pgfpathlineto{\pgfqpoint{5.658723in}{0.413320in}}%
\pgfpathlineto{\pgfqpoint{5.655919in}{0.413320in}}%
\pgfpathlineto{\pgfqpoint{5.653376in}{0.413320in}}%
\pgfpathlineto{\pgfqpoint{5.650563in}{0.413320in}}%
\pgfpathlineto{\pgfqpoint{5.648008in}{0.413320in}}%
\pgfpathlineto{\pgfqpoint{5.645243in}{0.413320in}}%
\pgfpathlineto{\pgfqpoint{5.642518in}{0.413320in}}%
\pgfpathlineto{\pgfqpoint{5.639852in}{0.413320in}}%
\pgfpathlineto{\pgfqpoint{5.637172in}{0.413320in}}%
\pgfpathlineto{\pgfqpoint{5.634496in}{0.413320in}}%
\pgfpathlineto{\pgfqpoint{5.631815in}{0.413320in}}%
\pgfpathlineto{\pgfqpoint{5.629232in}{0.413320in}}%
\pgfpathlineto{\pgfqpoint{5.626460in}{0.413320in}}%
\pgfpathlineto{\pgfqpoint{5.623868in}{0.413320in}}%
\pgfpathlineto{\pgfqpoint{5.621102in}{0.413320in}}%
\pgfpathlineto{\pgfqpoint{5.618526in}{0.413320in}}%
\pgfpathlineto{\pgfqpoint{5.615743in}{0.413320in}}%
\pgfpathlineto{\pgfqpoint{5.613235in}{0.413320in}}%
\pgfpathlineto{\pgfqpoint{5.610389in}{0.413320in}}%
\pgfpathlineto{\pgfqpoint{5.607698in}{0.413320in}}%
\pgfpathlineto{\pgfqpoint{5.605073in}{0.413320in}}%
\pgfpathlineto{\pgfqpoint{5.602352in}{0.413320in}}%
\pgfpathlineto{\pgfqpoint{5.599674in}{0.413320in}}%
\pgfpathlineto{\pgfqpoint{5.596999in}{0.413320in}}%
\pgfpathlineto{\pgfqpoint{5.594368in}{0.413320in}}%
\pgfpathlineto{\pgfqpoint{5.591641in}{0.413320in}}%
\pgfpathlineto{\pgfqpoint{5.589040in}{0.413320in}}%
\pgfpathlineto{\pgfqpoint{5.586269in}{0.413320in}}%
\pgfpathlineto{\pgfqpoint{5.583709in}{0.413320in}}%
\pgfpathlineto{\pgfqpoint{5.580914in}{0.413320in}}%
\pgfpathlineto{\pgfqpoint{5.578342in}{0.413320in}}%
\pgfpathlineto{\pgfqpoint{5.575596in}{0.413320in}}%
\pgfpathlineto{\pgfqpoint{5.572893in}{0.413320in}}%
\pgfpathlineto{\pgfqpoint{5.570215in}{0.413320in}}%
\pgfpathlineto{\pgfqpoint{5.567536in}{0.413320in}}%
\pgfpathlineto{\pgfqpoint{5.564940in}{0.413320in}}%
\pgfpathlineto{\pgfqpoint{5.562180in}{0.413320in}}%
\pgfpathlineto{\pgfqpoint{5.559612in}{0.413320in}}%
\pgfpathlineto{\pgfqpoint{5.556822in}{0.413320in}}%
\pgfpathlineto{\pgfqpoint{5.554198in}{0.413320in}}%
\pgfpathlineto{\pgfqpoint{5.551457in}{0.413320in}}%
\pgfpathlineto{\pgfqpoint{5.548921in}{0.413320in}}%
\pgfpathlineto{\pgfqpoint{5.546110in}{0.413320in}}%
\pgfpathlineto{\pgfqpoint{5.543421in}{0.413320in}}%
\pgfpathlineto{\pgfqpoint{5.540750in}{0.413320in}}%
\pgfpathlineto{\pgfqpoint{5.538074in}{0.413320in}}%
\pgfpathlineto{\pgfqpoint{5.535395in}{0.413320in}}%
\pgfpathlineto{\pgfqpoint{5.532717in}{0.413320in}}%
\pgfpathlineto{\pgfqpoint{5.530148in}{0.413320in}}%
\pgfpathlineto{\pgfqpoint{5.527360in}{0.413320in}}%
\pgfpathlineto{\pgfqpoint{5.524756in}{0.413320in}}%
\pgfpathlineto{\pgfqpoint{5.522003in}{0.413320in}}%
\pgfpathlineto{\pgfqpoint{5.519433in}{0.413320in}}%
\pgfpathlineto{\pgfqpoint{5.516646in}{0.413320in}}%
\pgfpathlineto{\pgfqpoint{5.514080in}{0.413320in}}%
\pgfpathlineto{\pgfqpoint{5.511290in}{0.413320in}}%
\pgfpathlineto{\pgfqpoint{5.508612in}{0.413320in}}%
\pgfpathlineto{\pgfqpoint{5.505933in}{0.413320in}}%
\pgfpathlineto{\pgfqpoint{5.503255in}{0.413320in}}%
\pgfpathlineto{\pgfqpoint{5.500687in}{0.413320in}}%
\pgfpathlineto{\pgfqpoint{5.497898in}{0.413320in}}%
\pgfpathlineto{\pgfqpoint{5.495346in}{0.413320in}}%
\pgfpathlineto{\pgfqpoint{5.492541in}{0.413320in}}%
\pgfpathlineto{\pgfqpoint{5.490000in}{0.413320in}}%
\pgfpathlineto{\pgfqpoint{5.487176in}{0.413320in}}%
\pgfpathlineto{\pgfqpoint{5.484641in}{0.413320in}}%
\pgfpathlineto{\pgfqpoint{5.481825in}{0.413320in}}%
\pgfpathlineto{\pgfqpoint{5.479152in}{0.413320in}}%
\pgfpathlineto{\pgfqpoint{5.476458in}{0.413320in}}%
\pgfpathlineto{\pgfqpoint{5.473792in}{0.413320in}}%
\pgfpathlineto{\pgfqpoint{5.471113in}{0.413320in}}%
\pgfpathlineto{\pgfqpoint{5.468425in}{0.413320in}}%
\pgfpathlineto{\pgfqpoint{5.465888in}{0.413320in}}%
\pgfpathlineto{\pgfqpoint{5.463079in}{0.413320in}}%
\pgfpathlineto{\pgfqpoint{5.460489in}{0.413320in}}%
\pgfpathlineto{\pgfqpoint{5.457721in}{0.413320in}}%
\pgfpathlineto{\pgfqpoint{5.455168in}{0.413320in}}%
\pgfpathlineto{\pgfqpoint{5.452365in}{0.413320in}}%
\pgfpathlineto{\pgfqpoint{5.449769in}{0.413320in}}%
\pgfpathlineto{\pgfqpoint{5.447021in}{0.413320in}}%
\pgfpathlineto{\pgfqpoint{5.444328in}{0.413320in}}%
\pgfpathlineto{\pgfqpoint{5.441698in}{0.413320in}}%
\pgfpathlineto{\pgfqpoint{5.438974in}{0.413320in}}%
\pgfpathlineto{\pgfqpoint{5.436295in}{0.413320in}}%
\pgfpathlineto{\pgfqpoint{5.433616in}{0.413320in}}%
\pgfpathlineto{\pgfqpoint{5.431015in}{0.413320in}}%
\pgfpathlineto{\pgfqpoint{5.428259in}{0.413320in}}%
\pgfpathlineto{\pgfqpoint{5.425661in}{0.413320in}}%
\pgfpathlineto{\pgfqpoint{5.422897in}{0.413320in}}%
\pgfpathlineto{\pgfqpoint{5.420304in}{0.413320in}}%
\pgfpathlineto{\pgfqpoint{5.417547in}{0.413320in}}%
\pgfpathlineto{\pgfqpoint{5.414954in}{0.413320in}}%
\pgfpathlineto{\pgfqpoint{5.412190in}{0.413320in}}%
\pgfpathlineto{\pgfqpoint{5.409507in}{0.413320in}}%
\pgfpathlineto{\pgfqpoint{5.406832in}{0.413320in}}%
\pgfpathlineto{\pgfqpoint{5.404154in}{0.413320in}}%
\pgfpathlineto{\pgfqpoint{5.401576in}{0.413320in}}%
\pgfpathlineto{\pgfqpoint{5.398784in}{0.413320in}}%
\pgfpathlineto{\pgfqpoint{5.396219in}{0.413320in}}%
\pgfpathlineto{\pgfqpoint{5.393441in}{0.413320in}}%
\pgfpathlineto{\pgfqpoint{5.390900in}{0.413320in}}%
\pgfpathlineto{\pgfqpoint{5.388083in}{0.413320in}}%
\pgfpathlineto{\pgfqpoint{5.385550in}{0.413320in}}%
\pgfpathlineto{\pgfqpoint{5.382725in}{0.413320in}}%
\pgfpathlineto{\pgfqpoint{5.380048in}{0.413320in}}%
\pgfpathlineto{\pgfqpoint{5.377370in}{0.413320in}}%
\pgfpathlineto{\pgfqpoint{5.374692in}{0.413320in}}%
\pgfpathlineto{\pgfqpoint{5.372013in}{0.413320in}}%
\pgfpathlineto{\pgfqpoint{5.369335in}{0.413320in}}%
\pgfpathlineto{\pgfqpoint{5.366727in}{0.413320in}}%
\pgfpathlineto{\pgfqpoint{5.363966in}{0.413320in}}%
\pgfpathlineto{\pgfqpoint{5.361370in}{0.413320in}}%
\pgfpathlineto{\pgfqpoint{5.358612in}{0.413320in}}%
\pgfpathlineto{\pgfqpoint{5.356056in}{0.413320in}}%
\pgfpathlineto{\pgfqpoint{5.353262in}{0.413320in}}%
\pgfpathlineto{\pgfqpoint{5.350723in}{0.413320in}}%
\pgfpathlineto{\pgfqpoint{5.347905in}{0.413320in}}%
\pgfpathlineto{\pgfqpoint{5.345224in}{0.413320in}}%
\pgfpathlineto{\pgfqpoint{5.342549in}{0.413320in}}%
\pgfpathlineto{\pgfqpoint{5.339872in}{0.413320in}}%
\pgfpathlineto{\pgfqpoint{5.337353in}{0.413320in}}%
\pgfpathlineto{\pgfqpoint{5.334510in}{0.413320in}}%
\pgfpathlineto{\pgfqpoint{5.331973in}{0.413320in}}%
\pgfpathlineto{\pgfqpoint{5.329159in}{0.413320in}}%
\pgfpathlineto{\pgfqpoint{5.326564in}{0.413320in}}%
\pgfpathlineto{\pgfqpoint{5.323802in}{0.413320in}}%
\pgfpathlineto{\pgfqpoint{5.321256in}{0.413320in}}%
\pgfpathlineto{\pgfqpoint{5.318430in}{0.413320in}}%
\pgfpathlineto{\pgfqpoint{5.315754in}{0.413320in}}%
\pgfpathlineto{\pgfqpoint{5.313089in}{0.413320in}}%
\pgfpathlineto{\pgfqpoint{5.310411in}{0.413320in}}%
\pgfpathlineto{\pgfqpoint{5.307731in}{0.413320in}}%
\pgfpathlineto{\pgfqpoint{5.305054in}{0.413320in}}%
\pgfpathlineto{\pgfqpoint{5.302443in}{0.413320in}}%
\pgfpathlineto{\pgfqpoint{5.299696in}{0.413320in}}%
\pgfpathlineto{\pgfqpoint{5.297140in}{0.413320in}}%
\pgfpathlineto{\pgfqpoint{5.294339in}{0.413320in}}%
\pgfpathlineto{\pgfqpoint{5.291794in}{0.413320in}}%
\pgfpathlineto{\pgfqpoint{5.288984in}{0.413320in}}%
\pgfpathlineto{\pgfqpoint{5.286436in}{0.413320in}}%
\pgfpathlineto{\pgfqpoint{5.283631in}{0.413320in}}%
\pgfpathlineto{\pgfqpoint{5.280947in}{0.413320in}}%
\pgfpathlineto{\pgfqpoint{5.278322in}{0.413320in}}%
\pgfpathlineto{\pgfqpoint{5.275589in}{0.413320in}}%
\pgfpathlineto{\pgfqpoint{5.272913in}{0.413320in}}%
\pgfpathlineto{\pgfqpoint{5.270238in}{0.413320in}}%
\pgfpathlineto{\pgfqpoint{5.267691in}{0.413320in}}%
\pgfpathlineto{\pgfqpoint{5.264876in}{0.413320in}}%
\pgfpathlineto{\pgfqpoint{5.262264in}{0.413320in}}%
\pgfpathlineto{\pgfqpoint{5.259511in}{0.413320in}}%
\pgfpathlineto{\pgfqpoint{5.256973in}{0.413320in}}%
\pgfpathlineto{\pgfqpoint{5.254236in}{0.413320in}}%
\pgfpathlineto{\pgfqpoint{5.251590in}{0.413320in}}%
\pgfpathlineto{\pgfqpoint{5.248816in}{0.413320in}}%
\pgfpathlineto{\pgfqpoint{5.246130in}{0.413320in}}%
\pgfpathlineto{\pgfqpoint{5.243445in}{0.413320in}}%
\pgfpathlineto{\pgfqpoint{5.240777in}{0.413320in}}%
\pgfpathlineto{\pgfqpoint{5.238173in}{0.413320in}}%
\pgfpathlineto{\pgfqpoint{5.235409in}{0.413320in}}%
\pgfpathlineto{\pgfqpoint{5.232855in}{0.413320in}}%
\pgfpathlineto{\pgfqpoint{5.230059in}{0.413320in}}%
\pgfpathlineto{\pgfqpoint{5.227470in}{0.413320in}}%
\pgfpathlineto{\pgfqpoint{5.224695in}{0.413320in}}%
\pgfpathlineto{\pgfqpoint{5.222151in}{0.413320in}}%
\pgfpathlineto{\pgfqpoint{5.219345in}{0.413320in}}%
\pgfpathlineto{\pgfqpoint{5.216667in}{0.413320in}}%
\pgfpathlineto{\pgfqpoint{5.214027in}{0.413320in}}%
\pgfpathlineto{\pgfqpoint{5.211299in}{0.413320in}}%
\pgfpathlineto{\pgfqpoint{5.208630in}{0.413320in}}%
\pgfpathlineto{\pgfqpoint{5.205952in}{0.413320in}}%
\pgfpathlineto{\pgfqpoint{5.203388in}{0.413320in}}%
\pgfpathlineto{\pgfqpoint{5.200594in}{0.413320in}}%
\pgfpathlineto{\pgfqpoint{5.198008in}{0.413320in}}%
\pgfpathlineto{\pgfqpoint{5.195239in}{0.413320in}}%
\pgfpathlineto{\pgfqpoint{5.192680in}{0.413320in}}%
\pgfpathlineto{\pgfqpoint{5.189880in}{0.413320in}}%
\pgfpathlineto{\pgfqpoint{5.187294in}{0.413320in}}%
\pgfpathlineto{\pgfqpoint{5.184522in}{0.413320in}}%
\pgfpathlineto{\pgfqpoint{5.181848in}{0.413320in}}%
\pgfpathlineto{\pgfqpoint{5.179188in}{0.413320in}}%
\pgfpathlineto{\pgfqpoint{5.176477in}{0.413320in}}%
\pgfpathlineto{\pgfqpoint{5.173925in}{0.413320in}}%
\pgfpathlineto{\pgfqpoint{5.171133in}{0.413320in}}%
\pgfpathlineto{\pgfqpoint{5.168591in}{0.413320in}}%
\pgfpathlineto{\pgfqpoint{5.165775in}{0.413320in}}%
\pgfpathlineto{\pgfqpoint{5.163243in}{0.413320in}}%
\pgfpathlineto{\pgfqpoint{5.160420in}{0.413320in}}%
\pgfpathlineto{\pgfqpoint{5.157815in}{0.413320in}}%
\pgfpathlineto{\pgfqpoint{5.155059in}{0.413320in}}%
\pgfpathlineto{\pgfqpoint{5.152382in}{0.413320in}}%
\pgfpathlineto{\pgfqpoint{5.149734in}{0.413320in}}%
\pgfpathlineto{\pgfqpoint{5.147029in}{0.413320in}}%
\pgfpathlineto{\pgfqpoint{5.144349in}{0.413320in}}%
\pgfpathlineto{\pgfqpoint{5.141660in}{0.413320in}}%
\pgfpathlineto{\pgfqpoint{5.139072in}{0.413320in}}%
\pgfpathlineto{\pgfqpoint{5.136311in}{0.413320in}}%
\pgfpathlineto{\pgfqpoint{5.133716in}{0.413320in}}%
\pgfpathlineto{\pgfqpoint{5.130953in}{0.413320in}}%
\pgfpathlineto{\pgfqpoint{5.128421in}{0.413320in}}%
\pgfpathlineto{\pgfqpoint{5.125599in}{0.413320in}}%
\pgfpathlineto{\pgfqpoint{5.123042in}{0.413320in}}%
\pgfpathlineto{\pgfqpoint{5.120243in}{0.413320in}}%
\pgfpathlineto{\pgfqpoint{5.117550in}{0.413320in}}%
\pgfpathlineto{\pgfqpoint{5.114887in}{0.413320in}}%
\pgfpathlineto{\pgfqpoint{5.112209in}{0.413320in}}%
\pgfpathlineto{\pgfqpoint{5.109530in}{0.413320in}}%
\pgfpathlineto{\pgfqpoint{5.106842in}{0.413320in}}%
\pgfpathlineto{\pgfqpoint{5.104312in}{0.413320in}}%
\pgfpathlineto{\pgfqpoint{5.101496in}{0.413320in}}%
\pgfpathlineto{\pgfqpoint{5.098948in}{0.413320in}}%
\pgfpathlineto{\pgfqpoint{5.096142in}{0.413320in}}%
\pgfpathlineto{\pgfqpoint{5.093579in}{0.413320in}}%
\pgfpathlineto{\pgfqpoint{5.090788in}{0.413320in}}%
\pgfpathlineto{\pgfqpoint{5.088103in}{0.413320in}}%
\pgfpathlineto{\pgfqpoint{5.085426in}{0.413320in}}%
\pgfpathlineto{\pgfqpoint{5.082746in}{0.413320in}}%
\pgfpathlineto{\pgfqpoint{5.080067in}{0.413320in}}%
\pgfpathlineto{\pgfqpoint{5.077390in}{0.413320in}}%
\pgfpathlineto{\pgfqpoint{5.074851in}{0.413320in}}%
\pgfpathlineto{\pgfqpoint{5.072030in}{0.413320in}}%
\pgfpathlineto{\pgfqpoint{5.069463in}{0.413320in}}%
\pgfpathlineto{\pgfqpoint{5.066677in}{0.413320in}}%
\pgfpathlineto{\pgfqpoint{5.064144in}{0.413320in}}%
\pgfpathlineto{\pgfqpoint{5.061315in}{0.413320in}}%
\pgfpathlineto{\pgfqpoint{5.058711in}{0.413320in}}%
\pgfpathlineto{\pgfqpoint{5.055952in}{0.413320in}}%
\pgfpathlineto{\pgfqpoint{5.053284in}{0.413320in}}%
\pgfpathlineto{\pgfqpoint{5.050606in}{0.413320in}}%
\pgfpathlineto{\pgfqpoint{5.047924in}{0.413320in}}%
\pgfpathlineto{\pgfqpoint{5.045249in}{0.413320in}}%
\pgfpathlineto{\pgfqpoint{5.042572in}{0.413320in}}%
\pgfpathlineto{\pgfqpoint{5.039962in}{0.413320in}}%
\pgfpathlineto{\pgfqpoint{5.037214in}{0.413320in}}%
\pgfpathlineto{\pgfqpoint{5.034649in}{0.413320in}}%
\pgfpathlineto{\pgfqpoint{5.031849in}{0.413320in}}%
\pgfpathlineto{\pgfqpoint{5.029275in}{0.413320in}}%
\pgfpathlineto{\pgfqpoint{5.026501in}{0.413320in}}%
\pgfpathlineto{\pgfqpoint{5.023927in}{0.413320in}}%
\pgfpathlineto{\pgfqpoint{5.021147in}{0.413320in}}%
\pgfpathlineto{\pgfqpoint{5.018466in}{0.413320in}}%
\pgfpathlineto{\pgfqpoint{5.015820in}{0.413320in}}%
\pgfpathlineto{\pgfqpoint{5.013104in}{0.413320in}}%
\pgfpathlineto{\pgfqpoint{5.010562in}{0.413320in}}%
\pgfpathlineto{\pgfqpoint{5.007751in}{0.413320in}}%
\pgfpathlineto{\pgfqpoint{5.005178in}{0.413320in}}%
\pgfpathlineto{\pgfqpoint{5.002384in}{0.413320in}}%
\pgfpathlineto{\pgfqpoint{4.999780in}{0.413320in}}%
\pgfpathlineto{\pgfqpoint{4.997028in}{0.413320in}}%
\pgfpathlineto{\pgfqpoint{4.994390in}{0.413320in}}%
\pgfpathlineto{\pgfqpoint{4.991683in}{0.413320in}}%
\pgfpathlineto{\pgfqpoint{4.989001in}{0.413320in}}%
\pgfpathlineto{\pgfqpoint{4.986325in}{0.413320in}}%
\pgfpathlineto{\pgfqpoint{4.983637in}{0.413320in}}%
\pgfpathlineto{\pgfqpoint{4.980967in}{0.413320in}}%
\pgfpathlineto{\pgfqpoint{4.978287in}{0.413320in}}%
\pgfpathlineto{\pgfqpoint{4.975703in}{0.413320in}}%
\pgfpathlineto{\pgfqpoint{4.972933in}{0.413320in}}%
\pgfpathlineto{\pgfqpoint{4.970314in}{0.413320in}}%
\pgfpathlineto{\pgfqpoint{4.967575in}{0.413320in}}%
\pgfpathlineto{\pgfqpoint{4.965002in}{0.413320in}}%
\pgfpathlineto{\pgfqpoint{4.962219in}{0.413320in}}%
\pgfpathlineto{\pgfqpoint{4.959689in}{0.413320in}}%
\pgfpathlineto{\pgfqpoint{4.956862in}{0.413320in}}%
\pgfpathlineto{\pgfqpoint{4.954182in}{0.413320in}}%
\pgfpathlineto{\pgfqpoint{4.951504in}{0.413320in}}%
\pgfpathlineto{\pgfqpoint{4.948827in}{0.413320in}}%
\pgfpathlineto{\pgfqpoint{4.946151in}{0.413320in}}%
\pgfpathlineto{\pgfqpoint{4.943466in}{0.413320in}}%
\pgfpathlineto{\pgfqpoint{4.940881in}{0.413320in}}%
\pgfpathlineto{\pgfqpoint{4.938112in}{0.413320in}}%
\pgfpathlineto{\pgfqpoint{4.935515in}{0.413320in}}%
\pgfpathlineto{\pgfqpoint{4.932742in}{0.413320in}}%
\pgfpathlineto{\pgfqpoint{4.930170in}{0.413320in}}%
\pgfpathlineto{\pgfqpoint{4.927400in}{0.413320in}}%
\pgfpathlineto{\pgfqpoint{4.924708in}{0.413320in}}%
\pgfpathlineto{\pgfqpoint{4.922041in}{0.413320in}}%
\pgfpathlineto{\pgfqpoint{4.919352in}{0.413320in}}%
\pgfpathlineto{\pgfqpoint{4.916681in}{0.413320in}}%
\pgfpathlineto{\pgfqpoint{4.914009in}{0.413320in}}%
\pgfpathlineto{\pgfqpoint{4.911435in}{0.413320in}}%
\pgfpathlineto{\pgfqpoint{4.908648in}{0.413320in}}%
\pgfpathlineto{\pgfqpoint{4.906096in}{0.413320in}}%
\pgfpathlineto{\pgfqpoint{4.903295in}{0.413320in}}%
\pgfpathlineto{\pgfqpoint{4.900712in}{0.413320in}}%
\pgfpathlineto{\pgfqpoint{4.897938in}{0.413320in}}%
\pgfpathlineto{\pgfqpoint{4.895399in}{0.413320in}}%
\pgfpathlineto{\pgfqpoint{4.892611in}{0.413320in}}%
\pgfpathlineto{\pgfqpoint{4.889902in}{0.413320in}}%
\pgfpathlineto{\pgfqpoint{4.887211in}{0.413320in}}%
\pgfpathlineto{\pgfqpoint{4.884540in}{0.413320in}}%
\pgfpathlineto{\pgfqpoint{4.881864in}{0.413320in}}%
\pgfpathlineto{\pgfqpoint{4.879180in}{0.413320in}}%
\pgfpathlineto{\pgfqpoint{4.876636in}{0.413320in}}%
\pgfpathlineto{\pgfqpoint{4.873832in}{0.413320in}}%
\pgfpathlineto{\pgfqpoint{4.871209in}{0.413320in}}%
\pgfpathlineto{\pgfqpoint{4.868474in}{0.413320in}}%
\pgfpathlineto{\pgfqpoint{4.865910in}{0.413320in}}%
\pgfpathlineto{\pgfqpoint{4.863116in}{0.413320in}}%
\pgfpathlineto{\pgfqpoint{4.860544in}{0.413320in}}%
\pgfpathlineto{\pgfqpoint{4.857807in}{0.413320in}}%
\pgfpathlineto{\pgfqpoint{4.855070in}{0.413320in}}%
\pgfpathlineto{\pgfqpoint{4.852404in}{0.413320in}}%
\pgfpathlineto{\pgfqpoint{4.849715in}{0.413320in}}%
\pgfpathlineto{\pgfqpoint{4.847127in}{0.413320in}}%
\pgfpathlineto{\pgfqpoint{4.844361in}{0.413320in}}%
\pgfpathlineto{\pgfqpoint{4.842380in}{0.413320in}}%
\pgfpathlineto{\pgfqpoint{4.839922in}{0.413320in}}%
\pgfpathlineto{\pgfqpoint{4.837992in}{0.413320in}}%
\pgfpathlineto{\pgfqpoint{4.833657in}{0.413320in}}%
\pgfpathlineto{\pgfqpoint{4.831045in}{0.413320in}}%
\pgfpathlineto{\pgfqpoint{4.828291in}{0.413320in}}%
\pgfpathlineto{\pgfqpoint{4.825619in}{0.413320in}}%
\pgfpathlineto{\pgfqpoint{4.822945in}{0.413320in}}%
\pgfpathlineto{\pgfqpoint{4.820265in}{0.413320in}}%
\pgfpathlineto{\pgfqpoint{4.817587in}{0.413320in}}%
\pgfpathlineto{\pgfqpoint{4.814907in}{0.413320in}}%
\pgfpathlineto{\pgfqpoint{4.812377in}{0.413320in}}%
\pgfpathlineto{\pgfqpoint{4.809538in}{0.413320in}}%
\pgfpathlineto{\pgfqpoint{4.807017in}{0.413320in}}%
\pgfpathlineto{\pgfqpoint{4.804193in}{0.413320in}}%
\pgfpathlineto{\pgfqpoint{4.801586in}{0.413320in}}%
\pgfpathlineto{\pgfqpoint{4.798830in}{0.413320in}}%
\pgfpathlineto{\pgfqpoint{4.796274in}{0.413320in}}%
\pgfpathlineto{\pgfqpoint{4.793512in}{0.413320in}}%
\pgfpathlineto{\pgfqpoint{4.790798in}{0.413320in}}%
\pgfpathlineto{\pgfqpoint{4.788116in}{0.413320in}}%
\pgfpathlineto{\pgfqpoint{4.785445in}{0.413320in}}%
\pgfpathlineto{\pgfqpoint{4.782872in}{0.413320in}}%
\pgfpathlineto{\pgfqpoint{4.780083in}{0.413320in}}%
\pgfpathlineto{\pgfqpoint{4.777535in}{0.413320in}}%
\pgfpathlineto{\pgfqpoint{4.774732in}{0.413320in}}%
\pgfpathlineto{\pgfqpoint{4.772198in}{0.413320in}}%
\pgfpathlineto{\pgfqpoint{4.769367in}{0.413320in}}%
\pgfpathlineto{\pgfqpoint{4.766783in}{0.413320in}}%
\pgfpathlineto{\pgfqpoint{4.764018in}{0.413320in}}%
\pgfpathlineto{\pgfqpoint{4.761337in}{0.413320in}}%
\pgfpathlineto{\pgfqpoint{4.758653in}{0.413320in}}%
\pgfpathlineto{\pgfqpoint{4.755983in}{0.413320in}}%
\pgfpathlineto{\pgfqpoint{4.753298in}{0.413320in}}%
\pgfpathlineto{\pgfqpoint{4.750627in}{0.413320in}}%
\pgfpathlineto{\pgfqpoint{4.748081in}{0.413320in}}%
\pgfpathlineto{\pgfqpoint{4.745256in}{0.413320in}}%
\pgfpathlineto{\pgfqpoint{4.742696in}{0.413320in}}%
\pgfpathlineto{\pgfqpoint{4.739912in}{0.413320in}}%
\pgfpathlineto{\pgfqpoint{4.737348in}{0.413320in}}%
\pgfpathlineto{\pgfqpoint{4.734552in}{0.413320in}}%
\pgfpathlineto{\pgfqpoint{4.731901in}{0.413320in}}%
\pgfpathlineto{\pgfqpoint{4.729233in}{0.413320in}}%
\pgfpathlineto{\pgfqpoint{4.726508in}{0.413320in}}%
\pgfpathlineto{\pgfqpoint{4.723873in}{0.413320in}}%
\pgfpathlineto{\pgfqpoint{4.721160in}{0.413320in}}%
\pgfpathlineto{\pgfqpoint{4.718486in}{0.413320in}}%
\pgfpathlineto{\pgfqpoint{4.715806in}{0.413320in}}%
\pgfpathlineto{\pgfqpoint{4.713275in}{0.413320in}}%
\pgfpathlineto{\pgfqpoint{4.710437in}{0.413320in}}%
\pgfpathlineto{\pgfqpoint{4.707824in}{0.413320in}}%
\pgfpathlineto{\pgfqpoint{4.705094in}{0.413320in}}%
\pgfpathlineto{\pgfqpoint{4.702517in}{0.413320in}}%
\pgfpathlineto{\pgfqpoint{4.699734in}{0.413320in}}%
\pgfpathlineto{\pgfqpoint{4.697170in}{0.413320in}}%
\pgfpathlineto{\pgfqpoint{4.694381in}{0.413320in}}%
\pgfpathlineto{\pgfqpoint{4.691694in}{0.413320in}}%
\pgfpathlineto{\pgfqpoint{4.689051in}{0.413320in}}%
\pgfpathlineto{\pgfqpoint{4.686337in}{0.413320in}}%
\pgfpathlineto{\pgfqpoint{4.683799in}{0.413320in}}%
\pgfpathlineto{\pgfqpoint{4.680988in}{0.413320in}}%
\pgfpathlineto{\pgfqpoint{4.678448in}{0.413320in}}%
\pgfpathlineto{\pgfqpoint{4.675619in}{0.413320in}}%
\pgfpathlineto{\pgfqpoint{4.673068in}{0.413320in}}%
\pgfpathlineto{\pgfqpoint{4.670261in}{0.413320in}}%
\pgfpathlineto{\pgfqpoint{4.667764in}{0.413320in}}%
\pgfpathlineto{\pgfqpoint{4.664923in}{0.413320in}}%
\pgfpathlineto{\pgfqpoint{4.662237in}{0.413320in}}%
\pgfpathlineto{\pgfqpoint{4.659590in}{0.413320in}}%
\pgfpathlineto{\pgfqpoint{4.656873in}{0.413320in}}%
\pgfpathlineto{\pgfqpoint{4.654203in}{0.413320in}}%
\pgfpathlineto{\pgfqpoint{4.651524in}{0.413320in}}%
\pgfpathlineto{\pgfqpoint{4.648922in}{0.413320in}}%
\pgfpathlineto{\pgfqpoint{4.646169in}{0.413320in}}%
\pgfpathlineto{\pgfqpoint{4.643628in}{0.413320in}}%
\pgfpathlineto{\pgfqpoint{4.640809in}{0.413320in}}%
\pgfpathlineto{\pgfqpoint{4.638204in}{0.413320in}}%
\pgfpathlineto{\pgfqpoint{4.635445in}{0.413320in}}%
\pgfpathlineto{\pgfqpoint{4.632902in}{0.413320in}}%
\pgfpathlineto{\pgfqpoint{4.630096in}{0.413320in}}%
\pgfpathlineto{\pgfqpoint{4.627411in}{0.413320in}}%
\pgfpathlineto{\pgfqpoint{4.624741in}{0.413320in}}%
\pgfpathlineto{\pgfqpoint{4.622056in}{0.413320in}}%
\pgfpathlineto{\pgfqpoint{4.619529in}{0.413320in}}%
\pgfpathlineto{\pgfqpoint{4.616702in}{0.413320in}}%
\pgfpathlineto{\pgfqpoint{4.614134in}{0.413320in}}%
\pgfpathlineto{\pgfqpoint{4.611350in}{0.413320in}}%
\pgfpathlineto{\pgfqpoint{4.608808in}{0.413320in}}%
\pgfpathlineto{\pgfqpoint{4.605990in}{0.413320in}}%
\pgfpathlineto{\pgfqpoint{4.603430in}{0.413320in}}%
\pgfpathlineto{\pgfqpoint{4.600633in}{0.413320in}}%
\pgfpathlineto{\pgfqpoint{4.597951in}{0.413320in}}%
\pgfpathlineto{\pgfqpoint{4.595281in}{0.413320in}}%
\pgfpathlineto{\pgfqpoint{4.592589in}{0.413320in}}%
\pgfpathlineto{\pgfqpoint{4.589920in}{0.413320in}}%
\pgfpathlineto{\pgfqpoint{4.587244in}{0.413320in}}%
\pgfpathlineto{\pgfqpoint{4.584672in}{0.413320in}}%
\pgfpathlineto{\pgfqpoint{4.581888in}{0.413320in}}%
\pgfpathlineto{\pgfqpoint{4.579305in}{0.413320in}}%
\pgfpathlineto{\pgfqpoint{4.576531in}{0.413320in}}%
\pgfpathlineto{\pgfqpoint{4.573947in}{0.413320in}}%
\pgfpathlineto{\pgfqpoint{4.571171in}{0.413320in}}%
\pgfpathlineto{\pgfqpoint{4.568612in}{0.413320in}}%
\pgfpathlineto{\pgfqpoint{4.565820in}{0.413320in}}%
\pgfpathlineto{\pgfqpoint{4.563125in}{0.413320in}}%
\pgfpathlineto{\pgfqpoint{4.560448in}{0.413320in}}%
\pgfpathlineto{\pgfqpoint{4.557777in}{0.413320in}}%
\pgfpathlineto{\pgfqpoint{4.555106in}{0.413320in}}%
\pgfpathlineto{\pgfqpoint{4.552425in}{0.413320in}}%
\pgfpathlineto{\pgfqpoint{4.549822in}{0.413320in}}%
\pgfpathlineto{\pgfqpoint{4.547064in}{0.413320in}}%
\pgfpathlineto{\pgfqpoint{4.544464in}{0.413320in}}%
\pgfpathlineto{\pgfqpoint{4.541711in}{0.413320in}}%
\pgfpathlineto{\pgfqpoint{4.539144in}{0.413320in}}%
\pgfpathlineto{\pgfqpoint{4.536400in}{0.413320in}}%
\pgfpathlineto{\pgfqpoint{4.533764in}{0.413320in}}%
\pgfpathlineto{\pgfqpoint{4.530990in}{0.413320in}}%
\pgfpathlineto{\pgfqpoint{4.528307in}{0.413320in}}%
\pgfpathlineto{\pgfqpoint{4.525640in}{0.413320in}}%
\pgfpathlineto{\pgfqpoint{4.522962in}{0.413320in}}%
\pgfpathlineto{\pgfqpoint{4.520345in}{0.413320in}}%
\pgfpathlineto{\pgfqpoint{4.517598in}{0.413320in}}%
\pgfpathlineto{\pgfqpoint{4.515080in}{0.413320in}}%
\pgfpathlineto{\pgfqpoint{4.512246in}{0.413320in}}%
\pgfpathlineto{\pgfqpoint{4.509643in}{0.413320in}}%
\pgfpathlineto{\pgfqpoint{4.506893in}{0.413320in}}%
\pgfpathlineto{\pgfqpoint{4.504305in}{0.413320in}}%
\pgfpathlineto{\pgfqpoint{4.501529in}{0.413320in}}%
\pgfpathlineto{\pgfqpoint{4.498850in}{0.413320in}}%
\pgfpathlineto{\pgfqpoint{4.496167in}{0.413320in}}%
\pgfpathlineto{\pgfqpoint{4.493492in}{0.413320in}}%
\pgfpathlineto{\pgfqpoint{4.490822in}{0.413320in}}%
\pgfpathlineto{\pgfqpoint{4.488130in}{0.413320in}}%
\pgfpathlineto{\pgfqpoint{4.485581in}{0.413320in}}%
\pgfpathlineto{\pgfqpoint{4.482778in}{0.413320in}}%
\pgfpathlineto{\pgfqpoint{4.480201in}{0.413320in}}%
\pgfpathlineto{\pgfqpoint{4.477430in}{0.413320in}}%
\pgfpathlineto{\pgfqpoint{4.474861in}{0.413320in}}%
\pgfpathlineto{\pgfqpoint{4.472059in}{0.413320in}}%
\pgfpathlineto{\pgfqpoint{4.469492in}{0.413320in}}%
\pgfpathlineto{\pgfqpoint{4.466717in}{0.413320in}}%
\pgfpathlineto{\pgfqpoint{4.464029in}{0.413320in}}%
\pgfpathlineto{\pgfqpoint{4.461367in}{0.413320in}}%
\pgfpathlineto{\pgfqpoint{4.458681in}{0.413320in}}%
\pgfpathlineto{\pgfqpoint{4.456138in}{0.413320in}}%
\pgfpathlineto{\pgfqpoint{4.453312in}{0.413320in}}%
\pgfpathlineto{\pgfqpoint{4.450767in}{0.413320in}}%
\pgfpathlineto{\pgfqpoint{4.447965in}{0.413320in}}%
\pgfpathlineto{\pgfqpoint{4.445423in}{0.413320in}}%
\pgfpathlineto{\pgfqpoint{4.442611in}{0.413320in}}%
\pgfpathlineto{\pgfqpoint{4.440041in}{0.413320in}}%
\pgfpathlineto{\pgfqpoint{4.437253in}{0.413320in}}%
\pgfpathlineto{\pgfqpoint{4.434569in}{0.413320in}}%
\pgfpathlineto{\pgfqpoint{4.431901in}{0.413320in}}%
\pgfpathlineto{\pgfqpoint{4.429220in}{0.413320in}}%
\pgfpathlineto{\pgfqpoint{4.426534in}{0.413320in}}%
\pgfpathlineto{\pgfqpoint{4.423863in}{0.413320in}}%
\pgfpathlineto{\pgfqpoint{4.421292in}{0.413320in}}%
\pgfpathlineto{\pgfqpoint{4.418506in}{0.413320in}}%
\pgfpathlineto{\pgfqpoint{4.415932in}{0.413320in}}%
\pgfpathlineto{\pgfqpoint{4.413149in}{0.413320in}}%
\pgfpathlineto{\pgfqpoint{4.410587in}{0.413320in}}%
\pgfpathlineto{\pgfqpoint{4.407788in}{0.413320in}}%
\pgfpathlineto{\pgfqpoint{4.405234in}{0.413320in}}%
\pgfpathlineto{\pgfqpoint{4.402468in}{0.413320in}}%
\pgfpathlineto{\pgfqpoint{4.399745in}{0.413320in}}%
\pgfpathlineto{\pgfqpoint{4.397076in}{0.413320in}}%
\pgfpathlineto{\pgfqpoint{4.394400in}{0.413320in}}%
\pgfpathlineto{\pgfqpoint{4.391721in}{0.413320in}}%
\pgfpathlineto{\pgfqpoint{4.389044in}{0.413320in}}%
\pgfpathlineto{\pgfqpoint{4.386431in}{0.413320in}}%
\pgfpathlineto{\pgfqpoint{4.383674in}{0.413320in}}%
\pgfpathlineto{\pgfqpoint{4.381097in}{0.413320in}}%
\pgfpathlineto{\pgfqpoint{4.378329in}{0.413320in}}%
\pgfpathlineto{\pgfqpoint{4.375761in}{0.413320in}}%
\pgfpathlineto{\pgfqpoint{4.372976in}{0.413320in}}%
\pgfpathlineto{\pgfqpoint{4.370437in}{0.413320in}}%
\pgfpathlineto{\pgfqpoint{4.367646in}{0.413320in}}%
\pgfpathlineto{\pgfqpoint{4.364936in}{0.413320in}}%
\pgfpathlineto{\pgfqpoint{4.362270in}{0.413320in}}%
\pgfpathlineto{\pgfqpoint{4.359582in}{0.413320in}}%
\pgfpathlineto{\pgfqpoint{4.357014in}{0.413320in}}%
\pgfpathlineto{\pgfqpoint{4.354224in}{0.413320in}}%
\pgfpathlineto{\pgfqpoint{4.351645in}{0.413320in}}%
\pgfpathlineto{\pgfqpoint{4.348868in}{0.413320in}}%
\pgfpathlineto{\pgfqpoint{4.346263in}{0.413320in}}%
\pgfpathlineto{\pgfqpoint{4.343510in}{0.413320in}}%
\pgfpathlineto{\pgfqpoint{4.340976in}{0.413320in}}%
\pgfpathlineto{\pgfqpoint{4.338154in}{0.413320in}}%
\pgfpathlineto{\pgfqpoint{4.335463in}{0.413320in}}%
\pgfpathlineto{\pgfqpoint{4.332796in}{0.413320in}}%
\pgfpathlineto{\pgfqpoint{4.330118in}{0.413320in}}%
\pgfpathlineto{\pgfqpoint{4.327440in}{0.413320in}}%
\pgfpathlineto{\pgfqpoint{4.324760in}{0.413320in}}%
\pgfpathlineto{\pgfqpoint{4.322181in}{0.413320in}}%
\pgfpathlineto{\pgfqpoint{4.319405in}{0.413320in}}%
\pgfpathlineto{\pgfqpoint{4.316856in}{0.413320in}}%
\pgfpathlineto{\pgfqpoint{4.314032in}{0.413320in}}%
\pgfpathlineto{\pgfqpoint{4.311494in}{0.413320in}}%
\pgfpathlineto{\pgfqpoint{4.308691in}{0.413320in}}%
\pgfpathlineto{\pgfqpoint{4.306118in}{0.413320in}}%
\pgfpathlineto{\pgfqpoint{4.303357in}{0.413320in}}%
\pgfpathlineto{\pgfqpoint{4.300656in}{0.413320in}}%
\pgfpathlineto{\pgfqpoint{4.297977in}{0.413320in}}%
\pgfpathlineto{\pgfqpoint{4.295299in}{0.413320in}}%
\pgfpathlineto{\pgfqpoint{4.292786in}{0.413320in}}%
\pgfpathlineto{\pgfqpoint{4.289936in}{0.413320in}}%
\pgfpathlineto{\pgfqpoint{4.287399in}{0.413320in}}%
\pgfpathlineto{\pgfqpoint{4.284586in}{0.413320in}}%
\pgfpathlineto{\pgfqpoint{4.282000in}{0.413320in}}%
\pgfpathlineto{\pgfqpoint{4.279212in}{0.413320in}}%
\pgfpathlineto{\pgfqpoint{4.276635in}{0.413320in}}%
\pgfpathlineto{\pgfqpoint{4.273874in}{0.413320in}}%
\pgfpathlineto{\pgfqpoint{4.271187in}{0.413320in}}%
\pgfpathlineto{\pgfqpoint{4.268590in}{0.413320in}}%
\pgfpathlineto{\pgfqpoint{4.265824in}{0.413320in}}%
\pgfpathlineto{\pgfqpoint{4.263157in}{0.413320in}}%
\pgfpathlineto{\pgfqpoint{4.260477in}{0.413320in}}%
\pgfpathlineto{\pgfqpoint{4.257958in}{0.413320in}}%
\pgfpathlineto{\pgfqpoint{4.255120in}{0.413320in}}%
\pgfpathlineto{\pgfqpoint{4.252581in}{0.413320in}}%
\pgfpathlineto{\pgfqpoint{4.249767in}{0.413320in}}%
\pgfpathlineto{\pgfqpoint{4.247225in}{0.413320in}}%
\pgfpathlineto{\pgfqpoint{4.244394in}{0.413320in}}%
\pgfpathlineto{\pgfqpoint{4.241900in}{0.413320in}}%
\pgfpathlineto{\pgfqpoint{4.239084in}{0.413320in}}%
\pgfpathlineto{\pgfqpoint{4.236375in}{0.413320in}}%
\pgfpathlineto{\pgfqpoint{4.233691in}{0.413320in}}%
\pgfpathlineto{\pgfqpoint{4.231013in}{0.413320in}}%
\pgfpathlineto{\pgfqpoint{4.228331in}{0.413320in}}%
\pgfpathlineto{\pgfqpoint{4.225654in}{0.413320in}}%
\pgfpathlineto{\pgfqpoint{4.223082in}{0.413320in}}%
\pgfpathlineto{\pgfqpoint{4.220304in}{0.413320in}}%
\pgfpathlineto{\pgfqpoint{4.217694in}{0.413320in}}%
\pgfpathlineto{\pgfqpoint{4.214948in}{0.413320in}}%
\pgfpathlineto{\pgfqpoint{4.212383in}{0.413320in}}%
\pgfpathlineto{\pgfqpoint{4.209597in}{0.413320in}}%
\pgfpathlineto{\pgfqpoint{4.207076in}{0.413320in}}%
\pgfpathlineto{\pgfqpoint{4.204240in}{0.413320in}}%
\pgfpathlineto{\pgfqpoint{4.201542in}{0.413320in}}%
\pgfpathlineto{\pgfqpoint{4.198878in}{0.413320in}}%
\pgfpathlineto{\pgfqpoint{4.196186in}{0.413320in}}%
\pgfpathlineto{\pgfqpoint{4.193638in}{0.413320in}}%
\pgfpathlineto{\pgfqpoint{4.190842in}{0.413320in}}%
\pgfpathlineto{\pgfqpoint{4.188318in}{0.413320in}}%
\pgfpathlineto{\pgfqpoint{4.185481in}{0.413320in}}%
\pgfpathlineto{\pgfqpoint{4.182899in}{0.413320in}}%
\pgfpathlineto{\pgfqpoint{4.180129in}{0.413320in}}%
\pgfpathlineto{\pgfqpoint{4.177593in}{0.413320in}}%
\pgfpathlineto{\pgfqpoint{4.174770in}{0.413320in}}%
\pgfpathlineto{\pgfqpoint{4.172093in}{0.413320in}}%
\pgfpathlineto{\pgfqpoint{4.169415in}{0.413320in}}%
\pgfpathlineto{\pgfqpoint{4.166737in}{0.413320in}}%
\pgfpathlineto{\pgfqpoint{4.164059in}{0.413320in}}%
\pgfpathlineto{\pgfqpoint{4.161380in}{0.413320in}}%
\pgfpathlineto{\pgfqpoint{4.158806in}{0.413320in}}%
\pgfpathlineto{\pgfqpoint{4.156016in}{0.413320in}}%
\pgfpathlineto{\pgfqpoint{4.153423in}{0.413320in}}%
\pgfpathlineto{\pgfqpoint{4.150665in}{0.413320in}}%
\pgfpathlineto{\pgfqpoint{4.148082in}{0.413320in}}%
\pgfpathlineto{\pgfqpoint{4.145310in}{0.413320in}}%
\pgfpathlineto{\pgfqpoint{4.142713in}{0.413320in}}%
\pgfpathlineto{\pgfqpoint{4.139963in}{0.413320in}}%
\pgfpathlineto{\pgfqpoint{4.137272in}{0.413320in}}%
\pgfpathlineto{\pgfqpoint{4.134615in}{0.413320in}}%
\pgfpathlineto{\pgfqpoint{4.131920in}{0.413320in}}%
\pgfpathlineto{\pgfqpoint{4.129349in}{0.413320in}}%
\pgfpathlineto{\pgfqpoint{4.126553in}{0.413320in}}%
\pgfpathlineto{\pgfqpoint{4.124019in}{0.413320in}}%
\pgfpathlineto{\pgfqpoint{4.121205in}{0.413320in}}%
\pgfpathlineto{\pgfqpoint{4.118554in}{0.413320in}}%
\pgfpathlineto{\pgfqpoint{4.115844in}{0.413320in}}%
\pgfpathlineto{\pgfqpoint{4.113252in}{0.413320in}}%
\pgfpathlineto{\pgfqpoint{4.110488in}{0.413320in}}%
\pgfpathlineto{\pgfqpoint{4.107814in}{0.413320in}}%
\pgfpathlineto{\pgfqpoint{4.105185in}{0.413320in}}%
\pgfpathlineto{\pgfqpoint{4.102456in}{0.413320in}}%
\pgfpathlineto{\pgfqpoint{4.099777in}{0.413320in}}%
\pgfpathlineto{\pgfqpoint{4.097092in}{0.413320in}}%
\pgfpathlineto{\pgfqpoint{4.094527in}{0.413320in}}%
\pgfpathlineto{\pgfqpoint{4.091729in}{0.413320in}}%
\pgfpathlineto{\pgfqpoint{4.089159in}{0.413320in}}%
\pgfpathlineto{\pgfqpoint{4.086385in}{0.413320in}}%
\pgfpathlineto{\pgfqpoint{4.083870in}{0.413320in}}%
\pgfpathlineto{\pgfqpoint{4.081018in}{0.413320in}}%
\pgfpathlineto{\pgfqpoint{4.078471in}{0.413320in}}%
\pgfpathlineto{\pgfqpoint{4.075705in}{0.413320in}}%
\pgfpathlineto{\pgfqpoint{4.072985in}{0.413320in}}%
\pgfpathlineto{\pgfqpoint{4.070313in}{0.413320in}}%
\pgfpathlineto{\pgfqpoint{4.067636in}{0.413320in}}%
\pgfpathlineto{\pgfqpoint{4.064957in}{0.413320in}}%
\pgfpathlineto{\pgfqpoint{4.062266in}{0.413320in}}%
\pgfpathlineto{\pgfqpoint{4.059702in}{0.413320in}}%
\pgfpathlineto{\pgfqpoint{4.056911in}{0.413320in}}%
\pgfpathlineto{\pgfqpoint{4.054326in}{0.413320in}}%
\pgfpathlineto{\pgfqpoint{4.051557in}{0.413320in}}%
\pgfpathlineto{\pgfqpoint{4.049006in}{0.413320in}}%
\pgfpathlineto{\pgfqpoint{4.046210in}{0.413320in}}%
\pgfpathlineto{\pgfqpoint{4.043667in}{0.413320in}}%
\pgfpathlineto{\pgfqpoint{4.040852in}{0.413320in}}%
\pgfpathlineto{\pgfqpoint{4.038174in}{0.413320in}}%
\pgfpathlineto{\pgfqpoint{4.035492in}{0.413320in}}%
\pgfpathlineto{\pgfqpoint{4.032817in}{0.413320in}}%
\pgfpathlineto{\pgfqpoint{4.030229in}{0.413320in}}%
\pgfpathlineto{\pgfqpoint{4.027447in}{0.413320in}}%
\pgfpathlineto{\pgfqpoint{4.024868in}{0.413320in}}%
\pgfpathlineto{\pgfqpoint{4.022097in}{0.413320in}}%
\pgfpathlineto{\pgfqpoint{4.019518in}{0.413320in}}%
\pgfpathlineto{\pgfqpoint{4.016744in}{0.413320in}}%
\pgfpathlineto{\pgfqpoint{4.014186in}{0.413320in}}%
\pgfpathlineto{\pgfqpoint{4.011394in}{0.413320in}}%
\pgfpathlineto{\pgfqpoint{4.008699in}{0.413320in}}%
\pgfpathlineto{\pgfqpoint{4.006034in}{0.413320in}}%
\pgfpathlineto{\pgfqpoint{4.003348in}{0.413320in}}%
\pgfpathlineto{\pgfqpoint{4.000674in}{0.413320in}}%
\pgfpathlineto{\pgfqpoint{3.997990in}{0.413320in}}%
\pgfpathlineto{\pgfqpoint{3.995417in}{0.413320in}}%
\pgfpathlineto{\pgfqpoint{3.992642in}{0.413320in}}%
\pgfpathlineto{\pgfqpoint{3.990055in}{0.413320in}}%
\pgfpathlineto{\pgfqpoint{3.987270in}{0.413320in}}%
\pgfpathlineto{\pgfqpoint{3.984714in}{0.413320in}}%
\pgfpathlineto{\pgfqpoint{3.981929in}{0.413320in}}%
\pgfpathlineto{\pgfqpoint{3.979389in}{0.413320in}}%
\pgfpathlineto{\pgfqpoint{3.976563in}{0.413320in}}%
\pgfpathlineto{\pgfqpoint{3.973885in}{0.413320in}}%
\pgfpathlineto{\pgfqpoint{3.971250in}{0.413320in}}%
\pgfpathlineto{\pgfqpoint{3.968523in}{0.413320in}}%
\pgfpathlineto{\pgfqpoint{3.966013in}{0.413320in}}%
\pgfpathlineto{\pgfqpoint{3.963176in}{0.413320in}}%
\pgfpathlineto{\pgfqpoint{3.960635in}{0.413320in}}%
\pgfpathlineto{\pgfqpoint{3.957823in}{0.413320in}}%
\pgfpathlineto{\pgfqpoint{3.955211in}{0.413320in}}%
\pgfpathlineto{\pgfqpoint{3.952464in}{0.413320in}}%
\pgfpathlineto{\pgfqpoint{3.949894in}{0.413320in}}%
\pgfpathlineto{\pgfqpoint{3.947101in}{0.413320in}}%
\pgfpathlineto{\pgfqpoint{3.944431in}{0.413320in}}%
\pgfpathlineto{\pgfqpoint{3.941778in}{0.413320in}}%
\pgfpathlineto{\pgfqpoint{3.939075in}{0.413320in}}%
\pgfpathlineto{\pgfqpoint{3.936395in}{0.413320in}}%
\pgfpathlineto{\pgfqpoint{3.933714in}{0.413320in}}%
\pgfpathlineto{\pgfqpoint{3.931202in}{0.413320in}}%
\pgfpathlineto{\pgfqpoint{3.928347in}{0.413320in}}%
\pgfpathlineto{\pgfqpoint{3.925778in}{0.413320in}}%
\pgfpathlineto{\pgfqpoint{3.923005in}{0.413320in}}%
\pgfpathlineto{\pgfqpoint{3.920412in}{0.413320in}}%
\pgfpathlineto{\pgfqpoint{3.917646in}{0.413320in}}%
\pgfpathlineto{\pgfqpoint{3.915107in}{0.413320in}}%
\pgfpathlineto{\pgfqpoint{3.912296in}{0.413320in}}%
\pgfpathlineto{\pgfqpoint{3.909602in}{0.413320in}}%
\pgfpathlineto{\pgfqpoint{3.906918in}{0.413320in}}%
\pgfpathlineto{\pgfqpoint{3.904252in}{0.413320in}}%
\pgfpathlineto{\pgfqpoint{3.901573in}{0.413320in}}%
\pgfpathlineto{\pgfqpoint{3.898891in}{0.413320in}}%
\pgfpathlineto{\pgfqpoint{3.896345in}{0.413320in}}%
\pgfpathlineto{\pgfqpoint{3.893541in}{0.413320in}}%
\pgfpathlineto{\pgfqpoint{3.890926in}{0.413320in}}%
\pgfpathlineto{\pgfqpoint{3.888188in}{0.413320in}}%
\pgfpathlineto{\pgfqpoint{3.885621in}{0.413320in}}%
\pgfpathlineto{\pgfqpoint{3.882850in}{0.413320in}}%
\pgfpathlineto{\pgfqpoint{3.880237in}{0.413320in}}%
\pgfpathlineto{\pgfqpoint{3.877466in}{0.413320in}}%
\pgfpathlineto{\pgfqpoint{3.874790in}{0.413320in}}%
\pgfpathlineto{\pgfqpoint{3.872114in}{0.413320in}}%
\pgfpathlineto{\pgfqpoint{3.869435in}{0.413320in}}%
\pgfpathlineto{\pgfqpoint{3.866815in}{0.413320in}}%
\pgfpathlineto{\pgfqpoint{3.864073in}{0.413320in}}%
\pgfpathlineto{\pgfqpoint{3.861561in}{0.413320in}}%
\pgfpathlineto{\pgfqpoint{3.858720in}{0.413320in}}%
\pgfpathlineto{\pgfqpoint{3.856100in}{0.413320in}}%
\pgfpathlineto{\pgfqpoint{3.853358in}{0.413320in}}%
\pgfpathlineto{\pgfqpoint{3.850814in}{0.413320in}}%
\pgfpathlineto{\pgfqpoint{3.848005in}{0.413320in}}%
\pgfpathlineto{\pgfqpoint{3.845329in}{0.413320in}}%
\pgfpathlineto{\pgfqpoint{3.842641in}{0.413320in}}%
\pgfpathlineto{\pgfqpoint{3.839960in}{0.413320in}}%
\pgfpathlineto{\pgfqpoint{3.837286in}{0.413320in}}%
\pgfpathlineto{\pgfqpoint{3.834616in}{0.413320in}}%
\pgfpathlineto{\pgfqpoint{3.832053in}{0.413320in}}%
\pgfpathlineto{\pgfqpoint{3.829252in}{0.413320in}}%
\pgfpathlineto{\pgfqpoint{3.826679in}{0.413320in}}%
\pgfpathlineto{\pgfqpoint{3.823903in}{0.413320in}}%
\pgfpathlineto{\pgfqpoint{3.821315in}{0.413320in}}%
\pgfpathlineto{\pgfqpoint{3.818546in}{0.413320in}}%
\pgfpathlineto{\pgfqpoint{3.815983in}{0.413320in}}%
\pgfpathlineto{\pgfqpoint{3.813172in}{0.413320in}}%
\pgfpathlineto{\pgfqpoint{3.810510in}{0.413320in}}%
\pgfpathlineto{\pgfqpoint{3.807832in}{0.413320in}}%
\pgfpathlineto{\pgfqpoint{3.805145in}{0.413320in}}%
\pgfpathlineto{\pgfqpoint{3.802569in}{0.413320in}}%
\pgfpathlineto{\pgfqpoint{3.799797in}{0.413320in}}%
\pgfpathlineto{\pgfqpoint{3.797265in}{0.413320in}}%
\pgfpathlineto{\pgfqpoint{3.794435in}{0.413320in}}%
\pgfpathlineto{\pgfqpoint{3.791897in}{0.413320in}}%
\pgfpathlineto{\pgfqpoint{3.789084in}{0.413320in}}%
\pgfpathlineto{\pgfqpoint{3.786504in}{0.413320in}}%
\pgfpathlineto{\pgfqpoint{3.783725in}{0.413320in}}%
\pgfpathlineto{\pgfqpoint{3.781046in}{0.413320in}}%
\pgfpathlineto{\pgfqpoint{3.778370in}{0.413320in}}%
\pgfpathlineto{\pgfqpoint{3.775691in}{0.413320in}}%
\pgfpathlineto{\pgfqpoint{3.773014in}{0.413320in}}%
\pgfpathlineto{\pgfqpoint{3.770323in}{0.413320in}}%
\pgfpathlineto{\pgfqpoint{3.767782in}{0.413320in}}%
\pgfpathlineto{\pgfqpoint{3.764966in}{0.413320in}}%
\pgfpathlineto{\pgfqpoint{3.762389in}{0.413320in}}%
\pgfpathlineto{\pgfqpoint{3.759622in}{0.413320in}}%
\pgfpathlineto{\pgfqpoint{3.757065in}{0.413320in}}%
\pgfpathlineto{\pgfqpoint{3.754265in}{0.413320in}}%
\pgfpathlineto{\pgfqpoint{3.751728in}{0.413320in}}%
\pgfpathlineto{\pgfqpoint{3.748903in}{0.413320in}}%
\pgfpathlineto{\pgfqpoint{3.746229in}{0.413320in}}%
\pgfpathlineto{\pgfqpoint{3.743548in}{0.413320in}}%
\pgfpathlineto{\pgfqpoint{3.740874in}{0.413320in}}%
\pgfpathlineto{\pgfqpoint{3.738194in}{0.413320in}}%
\pgfpathlineto{\pgfqpoint{3.735509in}{0.413320in}}%
\pgfpathlineto{\pgfqpoint{3.732950in}{0.413320in}}%
\pgfpathlineto{\pgfqpoint{3.730158in}{0.413320in}}%
\pgfpathlineto{\pgfqpoint{3.727581in}{0.413320in}}%
\pgfpathlineto{\pgfqpoint{3.724804in}{0.413320in}}%
\pgfpathlineto{\pgfqpoint{3.722228in}{0.413320in}}%
\pgfpathlineto{\pgfqpoint{3.719446in}{0.413320in}}%
\pgfpathlineto{\pgfqpoint{3.716875in}{0.413320in}}%
\pgfpathlineto{\pgfqpoint{3.714086in}{0.413320in}}%
\pgfpathlineto{\pgfqpoint{3.711410in}{0.413320in}}%
\pgfpathlineto{\pgfqpoint{3.708729in}{0.413320in}}%
\pgfpathlineto{\pgfqpoint{3.706053in}{0.413320in}}%
\pgfpathlineto{\pgfqpoint{3.703460in}{0.413320in}}%
\pgfpathlineto{\pgfqpoint{3.700684in}{0.413320in}}%
\pgfpathlineto{\pgfqpoint{3.698125in}{0.413320in}}%
\pgfpathlineto{\pgfqpoint{3.695331in}{0.413320in}}%
\pgfpathlineto{\pgfqpoint{3.692765in}{0.413320in}}%
\pgfpathlineto{\pgfqpoint{3.689983in}{0.413320in}}%
\pgfpathlineto{\pgfqpoint{3.687442in}{0.413320in}}%
\pgfpathlineto{\pgfqpoint{3.684620in}{0.413320in}}%
\pgfpathlineto{\pgfqpoint{3.681948in}{0.413320in}}%
\pgfpathlineto{\pgfqpoint{3.679273in}{0.413320in}}%
\pgfpathlineto{\pgfqpoint{3.676591in}{0.413320in}}%
\pgfpathlineto{\pgfqpoint{3.673911in}{0.413320in}}%
\pgfpathlineto{\pgfqpoint{3.671232in}{0.413320in}}%
\pgfpathlineto{\pgfqpoint{3.668665in}{0.413320in}}%
\pgfpathlineto{\pgfqpoint{3.665864in}{0.413320in}}%
\pgfpathlineto{\pgfqpoint{3.663276in}{0.413320in}}%
\pgfpathlineto{\pgfqpoint{3.660515in}{0.413320in}}%
\pgfpathlineto{\pgfqpoint{3.657917in}{0.413320in}}%
\pgfpathlineto{\pgfqpoint{3.655165in}{0.413320in}}%
\pgfpathlineto{\pgfqpoint{3.652628in}{0.413320in}}%
\pgfpathlineto{\pgfqpoint{3.649837in}{0.413320in}}%
\pgfpathlineto{\pgfqpoint{3.647130in}{0.413320in}}%
\pgfpathlineto{\pgfqpoint{3.644452in}{0.413320in}}%
\pgfpathlineto{\pgfqpoint{3.641773in}{0.413320in}}%
\pgfpathlineto{\pgfqpoint{3.639207in}{0.413320in}}%
\pgfpathlineto{\pgfqpoint{3.636413in}{0.413320in}}%
\pgfpathlineto{\pgfqpoint{3.633858in}{0.413320in}}%
\pgfpathlineto{\pgfqpoint{3.631058in}{0.413320in}}%
\pgfpathlineto{\pgfqpoint{3.628460in}{0.413320in}}%
\pgfpathlineto{\pgfqpoint{3.625689in}{0.413320in}}%
\pgfpathlineto{\pgfqpoint{3.623165in}{0.413320in}}%
\pgfpathlineto{\pgfqpoint{3.620345in}{0.413320in}}%
\pgfpathlineto{\pgfqpoint{3.617667in}{0.413320in}}%
\pgfpathlineto{\pgfqpoint{3.614982in}{0.413320in}}%
\pgfpathlineto{\pgfqpoint{3.612311in}{0.413320in}}%
\pgfpathlineto{\pgfqpoint{3.609632in}{0.413320in}}%
\pgfpathlineto{\pgfqpoint{3.606951in}{0.413320in}}%
\pgfpathlineto{\pgfqpoint{3.604387in}{0.413320in}}%
\pgfpathlineto{\pgfqpoint{3.601590in}{0.413320in}}%
\pgfpathlineto{\pgfqpoint{3.598998in}{0.413320in}}%
\pgfpathlineto{\pgfqpoint{3.596240in}{0.413320in}}%
\pgfpathlineto{\pgfqpoint{3.593620in}{0.413320in}}%
\pgfpathlineto{\pgfqpoint{3.590883in}{0.413320in}}%
\pgfpathlineto{\pgfqpoint{3.588258in}{0.413320in}}%
\pgfpathlineto{\pgfqpoint{3.585532in}{0.413320in}}%
\pgfpathlineto{\pgfqpoint{3.582851in}{0.413320in}}%
\pgfpathlineto{\pgfqpoint{3.580191in}{0.413320in}}%
\pgfpathlineto{\pgfqpoint{3.577487in}{0.413320in}}%
\pgfpathlineto{\pgfqpoint{3.574814in}{0.413320in}}%
\pgfpathlineto{\pgfqpoint{3.572126in}{0.413320in}}%
\pgfpathlineto{\pgfqpoint{3.569584in}{0.413320in}}%
\pgfpathlineto{\pgfqpoint{3.566774in}{0.413320in}}%
\pgfpathlineto{\pgfqpoint{3.564188in}{0.413320in}}%
\pgfpathlineto{\pgfqpoint{3.561420in}{0.413320in}}%
\pgfpathlineto{\pgfqpoint{3.558853in}{0.413320in}}%
\pgfpathlineto{\pgfqpoint{3.556061in}{0.413320in}}%
\pgfpathlineto{\pgfqpoint{3.553498in}{0.413320in}}%
\pgfpathlineto{\pgfqpoint{3.550713in}{0.413320in}}%
\pgfpathlineto{\pgfqpoint{3.548029in}{0.413320in}}%
\pgfpathlineto{\pgfqpoint{3.545349in}{0.413320in}}%
\pgfpathlineto{\pgfqpoint{3.542656in}{0.413320in}}%
\pgfpathlineto{\pgfqpoint{3.540093in}{0.413320in}}%
\pgfpathlineto{\pgfqpoint{3.537309in}{0.413320in}}%
\pgfpathlineto{\pgfqpoint{3.534783in}{0.413320in}}%
\pgfpathlineto{\pgfqpoint{3.531955in}{0.413320in}}%
\pgfpathlineto{\pgfqpoint{3.529327in}{0.413320in}}%
\pgfpathlineto{\pgfqpoint{3.526601in}{0.413320in}}%
\pgfpathlineto{\pgfqpoint{3.524041in}{0.413320in}}%
\pgfpathlineto{\pgfqpoint{3.521244in}{0.413320in}}%
\pgfpathlineto{\pgfqpoint{3.518565in}{0.413320in}}%
\pgfpathlineto{\pgfqpoint{3.515884in}{0.413320in}}%
\pgfpathlineto{\pgfqpoint{3.513209in}{0.413320in}}%
\pgfpathlineto{\pgfqpoint{3.510533in}{0.413320in}}%
\pgfpathlineto{\pgfqpoint{3.507840in}{0.413320in}}%
\pgfpathlineto{\pgfqpoint{3.505262in}{0.413320in}}%
\pgfpathlineto{\pgfqpoint{3.502488in}{0.413320in}}%
\pgfpathlineto{\pgfqpoint{3.499909in}{0.413320in}}%
\pgfpathlineto{\pgfqpoint{3.497139in}{0.413320in}}%
\pgfpathlineto{\pgfqpoint{3.494581in}{0.413320in}}%
\pgfpathlineto{\pgfqpoint{3.491783in}{0.413320in}}%
\pgfpathlineto{\pgfqpoint{3.489223in}{0.413320in}}%
\pgfpathlineto{\pgfqpoint{3.486442in}{0.413320in}}%
\pgfpathlineto{\pgfqpoint{3.483744in}{0.413320in}}%
\pgfpathlineto{\pgfqpoint{3.481072in}{0.413320in}}%
\pgfpathlineto{\pgfqpoint{3.478378in}{0.413320in}}%
\pgfpathlineto{\pgfqpoint{3.475821in}{0.413320in}}%
\pgfpathlineto{\pgfqpoint{3.473021in}{0.413320in}}%
\pgfpathlineto{\pgfqpoint{3.470466in}{0.413320in}}%
\pgfpathlineto{\pgfqpoint{3.467678in}{0.413320in}}%
\pgfpathlineto{\pgfqpoint{3.465072in}{0.413320in}}%
\pgfpathlineto{\pgfqpoint{3.462321in}{0.413320in}}%
\pgfpathlineto{\pgfqpoint{3.459695in}{0.413320in}}%
\pgfpathlineto{\pgfqpoint{3.456960in}{0.413320in}}%
\pgfpathlineto{\pgfqpoint{3.454285in}{0.413320in}}%
\pgfpathlineto{\pgfqpoint{3.451597in}{0.413320in}}%
\pgfpathlineto{\pgfqpoint{3.448926in}{0.413320in}}%
\pgfpathlineto{\pgfqpoint{3.446257in}{0.413320in}}%
\pgfpathlineto{\pgfqpoint{3.443574in}{0.413320in}}%
\pgfpathlineto{\pgfqpoint{3.440996in}{0.413320in}}%
\pgfpathlineto{\pgfqpoint{3.438210in}{0.413320in}}%
\pgfpathlineto{\pgfqpoint{3.435635in}{0.413320in}}%
\pgfpathlineto{\pgfqpoint{3.432851in}{0.413320in}}%
\pgfpathlineto{\pgfqpoint{3.430313in}{0.413320in}}%
\pgfpathlineto{\pgfqpoint{3.427501in}{0.413320in}}%
\pgfpathlineto{\pgfqpoint{3.424887in}{0.413320in}}%
\pgfpathlineto{\pgfqpoint{3.422142in}{0.413320in}}%
\pgfpathlineto{\pgfqpoint{3.419455in}{0.413320in}}%
\pgfpathlineto{\pgfqpoint{3.416780in}{0.413320in}}%
\pgfpathlineto{\pgfqpoint{3.414109in}{0.413320in}}%
\pgfpathlineto{\pgfqpoint{3.411431in}{0.413320in}}%
\pgfpathlineto{\pgfqpoint{3.408752in}{0.413320in}}%
\pgfpathlineto{\pgfqpoint{3.406202in}{0.413320in}}%
\pgfpathlineto{\pgfqpoint{3.403394in}{0.413320in}}%
\pgfpathlineto{\pgfqpoint{3.400783in}{0.413320in}}%
\pgfpathlineto{\pgfqpoint{3.398037in}{0.413320in}}%
\pgfpathlineto{\pgfqpoint{3.395461in}{0.413320in}}%
\pgfpathlineto{\pgfqpoint{3.392681in}{0.413320in}}%
\pgfpathlineto{\pgfqpoint{3.390102in}{0.413320in}}%
\pgfpathlineto{\pgfqpoint{3.387309in}{0.413320in}}%
\pgfpathlineto{\pgfqpoint{3.384647in}{0.413320in}}%
\pgfpathlineto{\pgfqpoint{3.381959in}{0.413320in}}%
\pgfpathlineto{\pgfqpoint{3.379290in}{0.413320in}}%
\pgfpathlineto{\pgfqpoint{3.376735in}{0.413320in}}%
\pgfpathlineto{\pgfqpoint{3.373921in}{0.413320in}}%
\pgfpathlineto{\pgfqpoint{3.371357in}{0.413320in}}%
\pgfpathlineto{\pgfqpoint{3.368577in}{0.413320in}}%
\pgfpathlineto{\pgfqpoint{3.365996in}{0.413320in}}%
\pgfpathlineto{\pgfqpoint{3.363221in}{0.413320in}}%
\pgfpathlineto{\pgfqpoint{3.360620in}{0.413320in}}%
\pgfpathlineto{\pgfqpoint{3.357862in}{0.413320in}}%
\pgfpathlineto{\pgfqpoint{3.355177in}{0.413320in}}%
\pgfpathlineto{\pgfqpoint{3.352505in}{0.413320in}}%
\pgfpathlineto{\pgfqpoint{3.349828in}{0.413320in}}%
\pgfpathlineto{\pgfqpoint{3.347139in}{0.413320in}}%
\pgfpathlineto{\pgfqpoint{3.344468in}{0.413320in}}%
\pgfpathlineto{\pgfqpoint{3.341893in}{0.413320in}}%
\pgfpathlineto{\pgfqpoint{3.339101in}{0.413320in}}%
\pgfpathlineto{\pgfqpoint{3.336541in}{0.413320in}}%
\pgfpathlineto{\pgfqpoint{3.333758in}{0.413320in}}%
\pgfpathlineto{\pgfqpoint{3.331183in}{0.413320in}}%
\pgfpathlineto{\pgfqpoint{3.328401in}{0.413320in}}%
\pgfpathlineto{\pgfqpoint{3.325860in}{0.413320in}}%
\pgfpathlineto{\pgfqpoint{3.323049in}{0.413320in}}%
\pgfpathlineto{\pgfqpoint{3.320366in}{0.413320in}}%
\pgfpathlineto{\pgfqpoint{3.317688in}{0.413320in}}%
\pgfpathlineto{\pgfqpoint{3.315008in}{0.413320in}}%
\pgfpathlineto{\pgfqpoint{3.312480in}{0.413320in}}%
\pgfpathlineto{\pgfqpoint{3.309652in}{0.413320in}}%
\pgfpathlineto{\pgfqpoint{3.307104in}{0.413320in}}%
\pgfpathlineto{\pgfqpoint{3.304295in}{0.413320in}}%
\pgfpathlineto{\pgfqpoint{3.301719in}{0.413320in}}%
\pgfpathlineto{\pgfqpoint{3.298937in}{0.413320in}}%
\pgfpathlineto{\pgfqpoint{3.296376in}{0.413320in}}%
\pgfpathlineto{\pgfqpoint{3.293574in}{0.413320in}}%
\pgfpathlineto{\pgfqpoint{3.290890in}{0.413320in}}%
\pgfpathlineto{\pgfqpoint{3.288225in}{0.413320in}}%
\pgfpathlineto{\pgfqpoint{3.285534in}{0.413320in}}%
\pgfpathlineto{\pgfqpoint{3.282870in}{0.413320in}}%
\pgfpathlineto{\pgfqpoint{3.280189in}{0.413320in}}%
\pgfpathlineto{\pgfqpoint{3.277603in}{0.413320in}}%
\pgfpathlineto{\pgfqpoint{3.274831in}{0.413320in}}%
\pgfpathlineto{\pgfqpoint{3.272254in}{0.413320in}}%
\pgfpathlineto{\pgfqpoint{3.269478in}{0.413320in}}%
\pgfpathlineto{\pgfqpoint{3.266849in}{0.413320in}}%
\pgfpathlineto{\pgfqpoint{3.264119in}{0.413320in}}%
\pgfpathlineto{\pgfqpoint{3.261594in}{0.413320in}}%
\pgfpathlineto{\pgfqpoint{3.258784in}{0.413320in}}%
\pgfpathlineto{\pgfqpoint{3.256083in}{0.413320in}}%
\pgfpathlineto{\pgfqpoint{3.253404in}{0.413320in}}%
\pgfpathlineto{\pgfqpoint{3.250716in}{0.413320in}}%
\pgfpathlineto{\pgfqpoint{3.248049in}{0.413320in}}%
\pgfpathlineto{\pgfqpoint{3.245363in}{0.413320in}}%
\pgfpathlineto{\pgfqpoint{3.242807in}{0.413320in}}%
\pgfpathlineto{\pgfqpoint{3.240010in}{0.413320in}}%
\pgfpathlineto{\pgfqpoint{3.237411in}{0.413320in}}%
\pgfpathlineto{\pgfqpoint{3.234658in}{0.413320in}}%
\pgfpathlineto{\pgfqpoint{3.232069in}{0.413320in}}%
\pgfpathlineto{\pgfqpoint{3.229310in}{0.413320in}}%
\pgfpathlineto{\pgfqpoint{3.226609in}{0.413320in}}%
\pgfpathlineto{\pgfqpoint{3.223942in}{0.413320in}}%
\pgfpathlineto{\pgfqpoint{3.221255in}{0.413320in}}%
\pgfpathlineto{\pgfqpoint{3.218586in}{0.413320in}}%
\pgfpathlineto{\pgfqpoint{3.215908in}{0.413320in}}%
\pgfpathlineto{\pgfqpoint{3.213342in}{0.413320in}}%
\pgfpathlineto{\pgfqpoint{3.210545in}{0.413320in}}%
\pgfpathlineto{\pgfqpoint{3.207984in}{0.413320in}}%
\pgfpathlineto{\pgfqpoint{3.205195in}{0.413320in}}%
\pgfpathlineto{\pgfqpoint{3.202562in}{0.413320in}}%
\pgfpathlineto{\pgfqpoint{3.199823in}{0.413320in}}%
\pgfpathlineto{\pgfqpoint{3.197226in}{0.413320in}}%
\pgfpathlineto{\pgfqpoint{3.194508in}{0.413320in}}%
\pgfpathlineto{\pgfqpoint{3.191796in}{0.413320in}}%
\pgfpathlineto{\pgfqpoint{3.189117in}{0.413320in}}%
\pgfpathlineto{\pgfqpoint{3.186440in}{0.413320in}}%
\pgfpathlineto{\pgfqpoint{3.183760in}{0.413320in}}%
\pgfpathlineto{\pgfqpoint{3.181089in}{0.413320in}}%
\pgfpathlineto{\pgfqpoint{3.178525in}{0.413320in}}%
\pgfpathlineto{\pgfqpoint{3.175724in}{0.413320in}}%
\pgfpathlineto{\pgfqpoint{3.173142in}{0.413320in}}%
\pgfpathlineto{\pgfqpoint{3.170375in}{0.413320in}}%
\pgfpathlineto{\pgfqpoint{3.167776in}{0.413320in}}%
\pgfpathlineto{\pgfqpoint{3.165019in}{0.413320in}}%
\pgfpathlineto{\pgfqpoint{3.162474in}{0.413320in}}%
\pgfpathlineto{\pgfqpoint{3.159675in}{0.413320in}}%
\pgfpathlineto{\pgfqpoint{3.156981in}{0.413320in}}%
\pgfpathlineto{\pgfqpoint{3.154327in}{0.413320in}}%
\pgfpathlineto{\pgfqpoint{3.151612in}{0.413320in}}%
\pgfpathlineto{\pgfqpoint{3.149057in}{0.413320in}}%
\pgfpathlineto{\pgfqpoint{3.146271in}{0.413320in}}%
\pgfpathlineto{\pgfqpoint{3.143740in}{0.413320in}}%
\pgfpathlineto{\pgfqpoint{3.140913in}{0.413320in}}%
\pgfpathlineto{\pgfqpoint{3.138375in}{0.413320in}}%
\pgfpathlineto{\pgfqpoint{3.135550in}{0.413320in}}%
\pgfpathlineto{\pgfqpoint{3.132946in}{0.413320in}}%
\pgfpathlineto{\pgfqpoint{3.130199in}{0.413320in}}%
\pgfpathlineto{\pgfqpoint{3.127512in}{0.413320in}}%
\pgfpathlineto{\pgfqpoint{3.124842in}{0.413320in}}%
\pgfpathlineto{\pgfqpoint{3.122163in}{0.413320in}}%
\pgfpathlineto{\pgfqpoint{3.119487in}{0.413320in}}%
\pgfpathlineto{\pgfqpoint{3.116807in}{0.413320in}}%
\pgfpathlineto{\pgfqpoint{3.114242in}{0.413320in}}%
\pgfpathlineto{\pgfqpoint{3.111451in}{0.413320in}}%
\pgfpathlineto{\pgfqpoint{3.108896in}{0.413320in}}%
\pgfpathlineto{\pgfqpoint{3.106094in}{0.413320in}}%
\pgfpathlineto{\pgfqpoint{3.103508in}{0.413320in}}%
\pgfpathlineto{\pgfqpoint{3.100737in}{0.413320in}}%
\pgfpathlineto{\pgfqpoint{3.098163in}{0.413320in}}%
\pgfpathlineto{\pgfqpoint{3.095388in}{0.413320in}}%
\pgfpathlineto{\pgfqpoint{3.092699in}{0.413320in}}%
\pgfpathlineto{\pgfqpoint{3.090023in}{0.413320in}}%
\pgfpathlineto{\pgfqpoint{3.087343in}{0.413320in}}%
\pgfpathlineto{\pgfqpoint{3.084671in}{0.413320in}}%
\pgfpathlineto{\pgfqpoint{3.081990in}{0.413320in}}%
\pgfpathlineto{\pgfqpoint{3.079381in}{0.413320in}}%
\pgfpathlineto{\pgfqpoint{3.076631in}{0.413320in}}%
\pgfpathlineto{\pgfqpoint{3.074056in}{0.413320in}}%
\pgfpathlineto{\pgfqpoint{3.071266in}{0.413320in}}%
\pgfpathlineto{\pgfqpoint{3.068709in}{0.413320in}}%
\pgfpathlineto{\pgfqpoint{3.065916in}{0.413320in}}%
\pgfpathlineto{\pgfqpoint{3.063230in}{0.413320in}}%
\pgfpathlineto{\pgfqpoint{3.060561in}{0.413320in}}%
\pgfpathlineto{\pgfqpoint{3.057884in}{0.413320in}}%
\pgfpathlineto{\pgfqpoint{3.055202in}{0.413320in}}%
\pgfpathlineto{\pgfqpoint{3.052526in}{0.413320in}}%
\pgfpathlineto{\pgfqpoint{3.049988in}{0.413320in}}%
\pgfpathlineto{\pgfqpoint{3.047157in}{0.413320in}}%
\pgfpathlineto{\pgfqpoint{3.044568in}{0.413320in}}%
\pgfpathlineto{\pgfqpoint{3.041813in}{0.413320in}}%
\pgfpathlineto{\pgfqpoint{3.039262in}{0.413320in}}%
\pgfpathlineto{\pgfqpoint{3.036456in}{0.413320in}}%
\pgfpathlineto{\pgfqpoint{3.033921in}{0.413320in}}%
\pgfpathlineto{\pgfqpoint{3.031091in}{0.413320in}}%
\pgfpathlineto{\pgfqpoint{3.028412in}{0.413320in}}%
\pgfpathlineto{\pgfqpoint{3.025803in}{0.413320in}}%
\pgfpathlineto{\pgfqpoint{3.023058in}{0.413320in}}%
\pgfpathlineto{\pgfqpoint{3.020382in}{0.413320in}}%
\pgfpathlineto{\pgfqpoint{3.017707in}{0.413320in}}%
\pgfpathlineto{\pgfqpoint{3.015097in}{0.413320in}}%
\pgfpathlineto{\pgfqpoint{3.012351in}{0.413320in}}%
\pgfpathlineto{\pgfqpoint{3.009784in}{0.413320in}}%
\pgfpathlineto{\pgfqpoint{3.006993in}{0.413320in}}%
\pgfpathlineto{\pgfqpoint{3.004419in}{0.413320in}}%
\pgfpathlineto{\pgfqpoint{3.001635in}{0.413320in}}%
\pgfpathlineto{\pgfqpoint{2.999103in}{0.413320in}}%
\pgfpathlineto{\pgfqpoint{2.996300in}{0.413320in}}%
\pgfpathlineto{\pgfqpoint{2.993595in}{0.413320in}}%
\pgfpathlineto{\pgfqpoint{2.990978in}{0.413320in}}%
\pgfpathlineto{\pgfqpoint{2.988238in}{0.413320in}}%
\pgfpathlineto{\pgfqpoint{2.985666in}{0.413320in}}%
\pgfpathlineto{\pgfqpoint{2.982885in}{0.413320in}}%
\pgfpathlineto{\pgfqpoint{2.980341in}{0.413320in}}%
\pgfpathlineto{\pgfqpoint{2.977517in}{0.413320in}}%
\pgfpathlineto{\pgfqpoint{2.974972in}{0.413320in}}%
\pgfpathlineto{\pgfqpoint{2.972177in}{0.413320in}}%
\pgfpathlineto{\pgfqpoint{2.969599in}{0.413320in}}%
\pgfpathlineto{\pgfqpoint{2.966812in}{0.413320in}}%
\pgfpathlineto{\pgfqpoint{2.964127in}{0.413320in}}%
\pgfpathlineto{\pgfqpoint{2.961460in}{0.413320in}}%
\pgfpathlineto{\pgfqpoint{2.958782in}{0.413320in}}%
\pgfpathlineto{\pgfqpoint{2.956103in}{0.413320in}}%
\pgfpathlineto{\pgfqpoint{2.953422in}{0.413320in}}%
\pgfpathlineto{\pgfqpoint{2.950884in}{0.413320in}}%
\pgfpathlineto{\pgfqpoint{2.948068in}{0.413320in}}%
\pgfpathlineto{\pgfqpoint{2.945461in}{0.413320in}}%
\pgfpathlineto{\pgfqpoint{2.942711in}{0.413320in}}%
\pgfpathlineto{\pgfqpoint{2.940120in}{0.413320in}}%
\pgfpathlineto{\pgfqpoint{2.937352in}{0.413320in}}%
\pgfpathlineto{\pgfqpoint{2.934759in}{0.413320in}}%
\pgfpathlineto{\pgfqpoint{2.932033in}{0.413320in}}%
\pgfpathlineto{\pgfqpoint{2.929321in}{0.413320in}}%
\pgfpathlineto{\pgfqpoint{2.926655in}{0.413320in}}%
\pgfpathlineto{\pgfqpoint{2.923963in}{0.413320in}}%
\pgfpathlineto{\pgfqpoint{2.921363in}{0.413320in}}%
\pgfpathlineto{\pgfqpoint{2.918606in}{0.413320in}}%
\pgfpathlineto{\pgfqpoint{2.916061in}{0.413320in}}%
\pgfpathlineto{\pgfqpoint{2.913243in}{0.413320in}}%
\pgfpathlineto{\pgfqpoint{2.910631in}{0.413320in}}%
\pgfpathlineto{\pgfqpoint{2.907882in}{0.413320in}}%
\pgfpathlineto{\pgfqpoint{2.905341in}{0.413320in}}%
\pgfpathlineto{\pgfqpoint{2.902535in}{0.413320in}}%
\pgfpathlineto{\pgfqpoint{2.899858in}{0.413320in}}%
\pgfpathlineto{\pgfqpoint{2.897179in}{0.413320in}}%
\pgfpathlineto{\pgfqpoint{2.894487in}{0.413320in}}%
\pgfpathlineto{\pgfqpoint{2.891809in}{0.413320in}}%
\pgfpathlineto{\pgfqpoint{2.889145in}{0.413320in}}%
\pgfpathlineto{\pgfqpoint{2.886578in}{0.413320in}}%
\pgfpathlineto{\pgfqpoint{2.883780in}{0.413320in}}%
\pgfpathlineto{\pgfqpoint{2.881254in}{0.413320in}}%
\pgfpathlineto{\pgfqpoint{2.878431in}{0.413320in}}%
\pgfpathlineto{\pgfqpoint{2.875882in}{0.413320in}}%
\pgfpathlineto{\pgfqpoint{2.873074in}{0.413320in}}%
\pgfpathlineto{\pgfqpoint{2.870475in}{0.413320in}}%
\pgfpathlineto{\pgfqpoint{2.867713in}{0.413320in}}%
\pgfpathlineto{\pgfqpoint{2.865031in}{0.413320in}}%
\pgfpathlineto{\pgfqpoint{2.862402in}{0.413320in}}%
\pgfpathlineto{\pgfqpoint{2.859668in}{0.413320in}}%
\pgfpathlineto{\pgfqpoint{2.857003in}{0.413320in}}%
\pgfpathlineto{\pgfqpoint{2.854325in}{0.413320in}}%
\pgfpathlineto{\pgfqpoint{2.851793in}{0.413320in}}%
\pgfpathlineto{\pgfqpoint{2.848960in}{0.413320in}}%
\pgfpathlineto{\pgfqpoint{2.846408in}{0.413320in}}%
\pgfpathlineto{\pgfqpoint{2.843611in}{0.413320in}}%
\pgfpathlineto{\pgfqpoint{2.841055in}{0.413320in}}%
\pgfpathlineto{\pgfqpoint{2.838254in}{0.413320in}}%
\pgfpathlineto{\pgfqpoint{2.835698in}{0.413320in}}%
\pgfpathlineto{\pgfqpoint{2.832894in}{0.413320in}}%
\pgfpathlineto{\pgfqpoint{2.830219in}{0.413320in}}%
\pgfpathlineto{\pgfqpoint{2.827567in}{0.413320in}}%
\pgfpathlineto{\pgfqpoint{2.824851in}{0.413320in}}%
\pgfpathlineto{\pgfqpoint{2.822303in}{0.413320in}}%
\pgfpathlineto{\pgfqpoint{2.819506in}{0.413320in}}%
\pgfpathlineto{\pgfqpoint{2.816867in}{0.413320in}}%
\pgfpathlineto{\pgfqpoint{2.814141in}{0.413320in}}%
\pgfpathlineto{\pgfqpoint{2.811597in}{0.413320in}}%
\pgfpathlineto{\pgfqpoint{2.808792in}{0.413320in}}%
\pgfpathlineto{\pgfqpoint{2.806175in}{0.413320in}}%
\pgfpathlineto{\pgfqpoint{2.803435in}{0.413320in}}%
\pgfpathlineto{\pgfqpoint{2.800756in}{0.413320in}}%
\pgfpathlineto{\pgfqpoint{2.798070in}{0.413320in}}%
\pgfpathlineto{\pgfqpoint{2.795398in}{0.413320in}}%
\pgfpathlineto{\pgfqpoint{2.792721in}{0.413320in}}%
\pgfpathlineto{\pgfqpoint{2.790044in}{0.413320in}}%
\pgfpathlineto{\pgfqpoint{2.787468in}{0.413320in}}%
\pgfpathlineto{\pgfqpoint{2.784687in}{0.413320in}}%
\pgfpathlineto{\pgfqpoint{2.782113in}{0.413320in}}%
\pgfpathlineto{\pgfqpoint{2.779330in}{0.413320in}}%
\pgfpathlineto{\pgfqpoint{2.776767in}{0.413320in}}%
\pgfpathlineto{\pgfqpoint{2.773972in}{0.413320in}}%
\pgfpathlineto{\pgfqpoint{2.771373in}{0.413320in}}%
\pgfpathlineto{\pgfqpoint{2.768617in}{0.413320in}}%
\pgfpathlineto{\pgfqpoint{2.765935in}{0.413320in}}%
\pgfpathlineto{\pgfqpoint{2.763253in}{0.413320in}}%
\pgfpathlineto{\pgfqpoint{2.760581in}{0.413320in}}%
\pgfpathlineto{\pgfqpoint{2.758028in}{0.413320in}}%
\pgfpathlineto{\pgfqpoint{2.755224in}{0.413320in}}%
\pgfpathlineto{\pgfqpoint{2.752614in}{0.413320in}}%
\pgfpathlineto{\pgfqpoint{2.749868in}{0.413320in}}%
\pgfpathlineto{\pgfqpoint{2.747260in}{0.413320in}}%
\pgfpathlineto{\pgfqpoint{2.744510in}{0.413320in}}%
\pgfpathlineto{\pgfqpoint{2.741928in}{0.413320in}}%
\pgfpathlineto{\pgfqpoint{2.739155in}{0.413320in}}%
\pgfpathlineto{\pgfqpoint{2.736476in}{0.413320in}}%
\pgfpathlineto{\pgfqpoint{2.733798in}{0.413320in}}%
\pgfpathlineto{\pgfqpoint{2.731119in}{0.413320in}}%
\pgfpathlineto{\pgfqpoint{2.728439in}{0.413320in}}%
\pgfpathlineto{\pgfqpoint{2.725760in}{0.413320in}}%
\pgfpathlineto{\pgfqpoint{2.723211in}{0.413320in}}%
\pgfpathlineto{\pgfqpoint{2.720404in}{0.413320in}}%
\pgfpathlineto{\pgfqpoint{2.717773in}{0.413320in}}%
\pgfpathlineto{\pgfqpoint{2.715036in}{0.413320in}}%
\pgfpathlineto{\pgfqpoint{2.712477in}{0.413320in}}%
\pgfpathlineto{\pgfqpoint{2.709683in}{0.413320in}}%
\pgfpathlineto{\pgfqpoint{2.707125in}{0.413320in}}%
\pgfpathlineto{\pgfqpoint{2.704326in}{0.413320in}}%
\pgfpathlineto{\pgfqpoint{2.701657in}{0.413320in}}%
\pgfpathlineto{\pgfqpoint{2.698968in}{0.413320in}}%
\pgfpathlineto{\pgfqpoint{2.696293in}{0.413320in}}%
\pgfpathlineto{\pgfqpoint{2.693611in}{0.413320in}}%
\pgfpathlineto{\pgfqpoint{2.690940in}{0.413320in}}%
\pgfpathlineto{\pgfqpoint{2.688328in}{0.413320in}}%
\pgfpathlineto{\pgfqpoint{2.685586in}{0.413320in}}%
\pgfpathlineto{\pgfqpoint{2.683009in}{0.413320in}}%
\pgfpathlineto{\pgfqpoint{2.680224in}{0.413320in}}%
\pgfpathlineto{\pgfqpoint{2.677650in}{0.413320in}}%
\pgfpathlineto{\pgfqpoint{2.674873in}{0.413320in}}%
\pgfpathlineto{\pgfqpoint{2.672301in}{0.413320in}}%
\pgfpathlineto{\pgfqpoint{2.669506in}{0.413320in}}%
\pgfpathlineto{\pgfqpoint{2.666836in}{0.413320in}}%
\pgfpathlineto{\pgfqpoint{2.664151in}{0.413320in}}%
\pgfpathlineto{\pgfqpoint{2.661481in}{0.413320in}}%
\pgfpathlineto{\pgfqpoint{2.658942in}{0.413320in}}%
\pgfpathlineto{\pgfqpoint{2.656124in}{0.413320in}}%
\pgfpathlineto{\pgfqpoint{2.653567in}{0.413320in}}%
\pgfpathlineto{\pgfqpoint{2.650767in}{0.413320in}}%
\pgfpathlineto{\pgfqpoint{2.648196in}{0.413320in}}%
\pgfpathlineto{\pgfqpoint{2.645408in}{0.413320in}}%
\pgfpathlineto{\pgfqpoint{2.642827in}{0.413320in}}%
\pgfpathlineto{\pgfqpoint{2.640053in}{0.413320in}}%
\pgfpathlineto{\pgfqpoint{2.637369in}{0.413320in}}%
\pgfpathlineto{\pgfqpoint{2.634700in}{0.413320in}}%
\pgfpathlineto{\pgfqpoint{2.632018in}{0.413320in}}%
\pgfpathlineto{\pgfqpoint{2.629340in}{0.413320in}}%
\pgfpathlineto{\pgfqpoint{2.626653in}{0.413320in}}%
\pgfpathlineto{\pgfqpoint{2.624077in}{0.413320in}}%
\pgfpathlineto{\pgfqpoint{2.621304in}{0.413320in}}%
\pgfpathlineto{\pgfqpoint{2.618773in}{0.413320in}}%
\pgfpathlineto{\pgfqpoint{2.615934in}{0.413320in}}%
\pgfpathlineto{\pgfqpoint{2.613393in}{0.413320in}}%
\pgfpathlineto{\pgfqpoint{2.610588in}{0.413320in}}%
\pgfpathlineto{\pgfqpoint{2.608004in}{0.413320in}}%
\pgfpathlineto{\pgfqpoint{2.605232in}{0.413320in}}%
\pgfpathlineto{\pgfqpoint{2.602557in}{0.413320in}}%
\pgfpathlineto{\pgfqpoint{2.599920in}{0.413320in}}%
\pgfpathlineto{\pgfqpoint{2.597196in}{0.413320in}}%
\pgfpathlineto{\pgfqpoint{2.594630in}{0.413320in}}%
\pgfpathlineto{\pgfqpoint{2.591842in}{0.413320in}}%
\pgfpathlineto{\pgfqpoint{2.589248in}{0.413320in}}%
\pgfpathlineto{\pgfqpoint{2.586484in}{0.413320in}}%
\pgfpathlineto{\pgfqpoint{2.583913in}{0.413320in}}%
\pgfpathlineto{\pgfqpoint{2.581129in}{0.413320in}}%
\pgfpathlineto{\pgfqpoint{2.578567in}{0.413320in}}%
\pgfpathlineto{\pgfqpoint{2.575779in}{0.413320in}}%
\pgfpathlineto{\pgfqpoint{2.573082in}{0.413320in}}%
\pgfpathlineto{\pgfqpoint{2.570411in}{0.413320in}}%
\pgfpathlineto{\pgfqpoint{2.567730in}{0.413320in}}%
\pgfpathlineto{\pgfqpoint{2.565045in}{0.413320in}}%
\pgfpathlineto{\pgfqpoint{2.562375in}{0.413320in}}%
\pgfpathlineto{\pgfqpoint{2.559790in}{0.413320in}}%
\pgfpathlineto{\pgfqpoint{2.557009in}{0.413320in}}%
\pgfpathlineto{\pgfqpoint{2.554493in}{0.413320in}}%
\pgfpathlineto{\pgfqpoint{2.551664in}{0.413320in}}%
\pgfpathlineto{\pgfqpoint{2.549114in}{0.413320in}}%
\pgfpathlineto{\pgfqpoint{2.546310in}{0.413320in}}%
\pgfpathlineto{\pgfqpoint{2.543765in}{0.413320in}}%
\pgfpathlineto{\pgfqpoint{2.540949in}{0.413320in}}%
\pgfpathlineto{\pgfqpoint{2.538274in}{0.413320in}}%
\pgfpathlineto{\pgfqpoint{2.535624in}{0.413320in}}%
\pgfpathlineto{\pgfqpoint{2.532917in}{0.413320in}}%
\pgfpathlineto{\pgfqpoint{2.530234in}{0.413320in}}%
\pgfpathlineto{\pgfqpoint{2.527560in}{0.413320in}}%
\pgfpathlineto{\pgfqpoint{2.524988in}{0.413320in}}%
\pgfpathlineto{\pgfqpoint{2.522197in}{0.413320in}}%
\pgfpathlineto{\pgfqpoint{2.519607in}{0.413320in}}%
\pgfpathlineto{\pgfqpoint{2.516845in}{0.413320in}}%
\pgfpathlineto{\pgfqpoint{2.514268in}{0.413320in}}%
\pgfpathlineto{\pgfqpoint{2.511478in}{0.413320in}}%
\pgfpathlineto{\pgfqpoint{2.508917in}{0.413320in}}%
\pgfpathlineto{\pgfqpoint{2.506163in}{0.413320in}}%
\pgfpathlineto{\pgfqpoint{2.503454in}{0.413320in}}%
\pgfpathlineto{\pgfqpoint{2.500801in}{0.413320in}}%
\pgfpathlineto{\pgfqpoint{2.498085in}{0.413320in}}%
\pgfpathlineto{\pgfqpoint{2.495542in}{0.413320in}}%
\pgfpathlineto{\pgfqpoint{2.492729in}{0.413320in}}%
\pgfpathlineto{\pgfqpoint{2.490183in}{0.413320in}}%
\pgfpathlineto{\pgfqpoint{2.487384in}{0.413320in}}%
\pgfpathlineto{\pgfqpoint{2.484870in}{0.413320in}}%
\pgfpathlineto{\pgfqpoint{2.482026in}{0.413320in}}%
\pgfpathlineto{\pgfqpoint{2.479420in}{0.413320in}}%
\pgfpathlineto{\pgfqpoint{2.476671in}{0.413320in}}%
\pgfpathlineto{\pgfqpoint{2.473989in}{0.413320in}}%
\pgfpathlineto{\pgfqpoint{2.471311in}{0.413320in}}%
\pgfpathlineto{\pgfqpoint{2.468635in}{0.413320in}}%
\pgfpathlineto{\pgfqpoint{2.465957in}{0.413320in}}%
\pgfpathlineto{\pgfqpoint{2.463280in}{0.413320in}}%
\pgfpathlineto{\pgfqpoint{2.460711in}{0.413320in}}%
\pgfpathlineto{\pgfqpoint{2.457917in}{0.413320in}}%
\pgfpathlineto{\pgfqpoint{2.455353in}{0.413320in}}%
\pgfpathlineto{\pgfqpoint{2.452562in}{0.413320in}}%
\pgfpathlineto{\pgfqpoint{2.450032in}{0.413320in}}%
\pgfpathlineto{\pgfqpoint{2.447209in}{0.413320in}}%
\pgfpathlineto{\pgfqpoint{2.444677in}{0.413320in}}%
\pgfpathlineto{\pgfqpoint{2.441876in}{0.413320in}}%
\pgfpathlineto{\pgfqpoint{2.439167in}{0.413320in}}%
\pgfpathlineto{\pgfqpoint{2.436518in}{0.413320in}}%
\pgfpathlineto{\pgfqpoint{2.433815in}{0.413320in}}%
\pgfpathlineto{\pgfqpoint{2.431251in}{0.413320in}}%
\pgfpathlineto{\pgfqpoint{2.428453in}{0.413320in}}%
\pgfpathlineto{\pgfqpoint{2.425878in}{0.413320in}}%
\pgfpathlineto{\pgfqpoint{2.423098in}{0.413320in}}%
\pgfpathlineto{\pgfqpoint{2.420528in}{0.413320in}}%
\pgfpathlineto{\pgfqpoint{2.417747in}{0.413320in}}%
\pgfpathlineto{\pgfqpoint{2.415184in}{0.413320in}}%
\pgfpathlineto{\pgfqpoint{2.412389in}{0.413320in}}%
\pgfpathlineto{\pgfqpoint{2.409699in}{0.413320in}}%
\pgfpathlineto{\pgfqpoint{2.407024in}{0.413320in}}%
\pgfpathlineto{\pgfqpoint{2.404352in}{0.413320in}}%
\pgfpathlineto{\pgfqpoint{2.401675in}{0.413320in}}%
\pgfpathlineto{\pgfqpoint{2.398995in}{0.413320in}}%
\pgfpathclose%
\pgfusepath{stroke,fill}%
\end{pgfscope}%
\begin{pgfscope}%
\pgfpathrectangle{\pgfqpoint{2.398995in}{0.319877in}}{\pgfqpoint{3.986877in}{1.993438in}} %
\pgfusepath{clip}%
\pgfsetbuttcap%
\pgfsetroundjoin%
\definecolor{currentfill}{rgb}{1.000000,1.000000,1.000000}%
\pgfsetfillcolor{currentfill}%
\pgfsetlinewidth{1.003750pt}%
\definecolor{currentstroke}{rgb}{0.656648,0.622956,0.193898}%
\pgfsetstrokecolor{currentstroke}%
\pgfsetdash{}{0pt}%
\pgfpathmoveto{\pgfqpoint{2.398995in}{0.413320in}}%
\pgfpathlineto{\pgfqpoint{2.398995in}{1.412999in}}%
\pgfpathlineto{\pgfqpoint{2.401675in}{1.413326in}}%
\pgfpathlineto{\pgfqpoint{2.404352in}{1.411259in}}%
\pgfpathlineto{\pgfqpoint{2.407024in}{1.413410in}}%
\pgfpathlineto{\pgfqpoint{2.409699in}{1.418167in}}%
\pgfpathlineto{\pgfqpoint{2.412389in}{1.413447in}}%
\pgfpathlineto{\pgfqpoint{2.415184in}{1.411302in}}%
\pgfpathlineto{\pgfqpoint{2.417747in}{1.408094in}}%
\pgfpathlineto{\pgfqpoint{2.420528in}{1.413138in}}%
\pgfpathlineto{\pgfqpoint{2.423098in}{1.412442in}}%
\pgfpathlineto{\pgfqpoint{2.425878in}{1.412217in}}%
\pgfpathlineto{\pgfqpoint{2.428453in}{1.416609in}}%
\pgfpathlineto{\pgfqpoint{2.431251in}{1.416008in}}%
\pgfpathlineto{\pgfqpoint{2.433815in}{1.422077in}}%
\pgfpathlineto{\pgfqpoint{2.436518in}{1.418228in}}%
\pgfpathlineto{\pgfqpoint{2.439167in}{1.411875in}}%
\pgfpathlineto{\pgfqpoint{2.441876in}{1.410497in}}%
\pgfpathlineto{\pgfqpoint{2.444677in}{1.407295in}}%
\pgfpathlineto{\pgfqpoint{2.447209in}{1.409725in}}%
\pgfpathlineto{\pgfqpoint{2.450032in}{1.405602in}}%
\pgfpathlineto{\pgfqpoint{2.452562in}{1.408567in}}%
\pgfpathlineto{\pgfqpoint{2.455353in}{1.407872in}}%
\pgfpathlineto{\pgfqpoint{2.457917in}{1.411100in}}%
\pgfpathlineto{\pgfqpoint{2.460711in}{1.409573in}}%
\pgfpathlineto{\pgfqpoint{2.463280in}{1.417049in}}%
\pgfpathlineto{\pgfqpoint{2.465957in}{1.413528in}}%
\pgfpathlineto{\pgfqpoint{2.468635in}{1.416604in}}%
\pgfpathlineto{\pgfqpoint{2.471311in}{1.415749in}}%
\pgfpathlineto{\pgfqpoint{2.473989in}{1.412551in}}%
\pgfpathlineto{\pgfqpoint{2.476671in}{1.412409in}}%
\pgfpathlineto{\pgfqpoint{2.479420in}{1.412566in}}%
\pgfpathlineto{\pgfqpoint{2.482026in}{1.418550in}}%
\pgfpathlineto{\pgfqpoint{2.484870in}{1.421438in}}%
\pgfpathlineto{\pgfqpoint{2.487384in}{1.409847in}}%
\pgfpathlineto{\pgfqpoint{2.490183in}{1.412683in}}%
\pgfpathlineto{\pgfqpoint{2.492729in}{1.409768in}}%
\pgfpathlineto{\pgfqpoint{2.495542in}{1.409828in}}%
\pgfpathlineto{\pgfqpoint{2.498085in}{1.410043in}}%
\pgfpathlineto{\pgfqpoint{2.500801in}{1.412083in}}%
\pgfpathlineto{\pgfqpoint{2.503454in}{1.414899in}}%
\pgfpathlineto{\pgfqpoint{2.506163in}{1.419265in}}%
\pgfpathlineto{\pgfqpoint{2.508917in}{1.409656in}}%
\pgfpathlineto{\pgfqpoint{2.511478in}{1.415035in}}%
\pgfpathlineto{\pgfqpoint{2.514268in}{1.411457in}}%
\pgfpathlineto{\pgfqpoint{2.516845in}{1.410336in}}%
\pgfpathlineto{\pgfqpoint{2.519607in}{1.418442in}}%
\pgfpathlineto{\pgfqpoint{2.522197in}{1.419043in}}%
\pgfpathlineto{\pgfqpoint{2.524988in}{1.413377in}}%
\pgfpathlineto{\pgfqpoint{2.527560in}{1.406288in}}%
\pgfpathlineto{\pgfqpoint{2.530234in}{1.406504in}}%
\pgfpathlineto{\pgfqpoint{2.532917in}{1.409244in}}%
\pgfpathlineto{\pgfqpoint{2.535624in}{1.405514in}}%
\pgfpathlineto{\pgfqpoint{2.538274in}{1.408032in}}%
\pgfpathlineto{\pgfqpoint{2.540949in}{1.407146in}}%
\pgfpathlineto{\pgfqpoint{2.543765in}{1.413321in}}%
\pgfpathlineto{\pgfqpoint{2.546310in}{1.410474in}}%
\pgfpathlineto{\pgfqpoint{2.549114in}{1.414891in}}%
\pgfpathlineto{\pgfqpoint{2.551664in}{1.412680in}}%
\pgfpathlineto{\pgfqpoint{2.554493in}{1.407378in}}%
\pgfpathlineto{\pgfqpoint{2.557009in}{1.412468in}}%
\pgfpathlineto{\pgfqpoint{2.559790in}{1.415452in}}%
\pgfpathlineto{\pgfqpoint{2.562375in}{1.410811in}}%
\pgfpathlineto{\pgfqpoint{2.565045in}{1.408956in}}%
\pgfpathlineto{\pgfqpoint{2.567730in}{1.410303in}}%
\pgfpathlineto{\pgfqpoint{2.570411in}{1.411641in}}%
\pgfpathlineto{\pgfqpoint{2.573082in}{1.412972in}}%
\pgfpathlineto{\pgfqpoint{2.575779in}{1.409291in}}%
\pgfpathlineto{\pgfqpoint{2.578567in}{1.407071in}}%
\pgfpathlineto{\pgfqpoint{2.581129in}{1.411237in}}%
\pgfpathlineto{\pgfqpoint{2.583913in}{1.414057in}}%
\pgfpathlineto{\pgfqpoint{2.586484in}{1.414878in}}%
\pgfpathlineto{\pgfqpoint{2.589248in}{1.411058in}}%
\pgfpathlineto{\pgfqpoint{2.591842in}{1.418742in}}%
\pgfpathlineto{\pgfqpoint{2.594630in}{1.413564in}}%
\pgfpathlineto{\pgfqpoint{2.597196in}{1.413308in}}%
\pgfpathlineto{\pgfqpoint{2.599920in}{1.407737in}}%
\pgfpathlineto{\pgfqpoint{2.602557in}{1.413026in}}%
\pgfpathlineto{\pgfqpoint{2.605232in}{1.413086in}}%
\pgfpathlineto{\pgfqpoint{2.608004in}{1.415864in}}%
\pgfpathlineto{\pgfqpoint{2.610588in}{1.415057in}}%
\pgfpathlineto{\pgfqpoint{2.613393in}{1.418240in}}%
\pgfpathlineto{\pgfqpoint{2.615934in}{1.413969in}}%
\pgfpathlineto{\pgfqpoint{2.618773in}{1.413652in}}%
\pgfpathlineto{\pgfqpoint{2.621304in}{1.411248in}}%
\pgfpathlineto{\pgfqpoint{2.624077in}{1.419536in}}%
\pgfpathlineto{\pgfqpoint{2.626653in}{1.415049in}}%
\pgfpathlineto{\pgfqpoint{2.629340in}{1.412872in}}%
\pgfpathlineto{\pgfqpoint{2.632018in}{1.414857in}}%
\pgfpathlineto{\pgfqpoint{2.634700in}{1.419115in}}%
\pgfpathlineto{\pgfqpoint{2.637369in}{1.414923in}}%
\pgfpathlineto{\pgfqpoint{2.640053in}{1.414665in}}%
\pgfpathlineto{\pgfqpoint{2.642827in}{1.418053in}}%
\pgfpathlineto{\pgfqpoint{2.645408in}{1.418022in}}%
\pgfpathlineto{\pgfqpoint{2.648196in}{1.416274in}}%
\pgfpathlineto{\pgfqpoint{2.650767in}{1.420396in}}%
\pgfpathlineto{\pgfqpoint{2.653567in}{1.420050in}}%
\pgfpathlineto{\pgfqpoint{2.656124in}{1.422407in}}%
\pgfpathlineto{\pgfqpoint{2.658942in}{1.419250in}}%
\pgfpathlineto{\pgfqpoint{2.661481in}{1.417426in}}%
\pgfpathlineto{\pgfqpoint{2.664151in}{1.414890in}}%
\pgfpathlineto{\pgfqpoint{2.666836in}{1.413030in}}%
\pgfpathlineto{\pgfqpoint{2.669506in}{1.413516in}}%
\pgfpathlineto{\pgfqpoint{2.672301in}{1.415690in}}%
\pgfpathlineto{\pgfqpoint{2.674873in}{1.421323in}}%
\pgfpathlineto{\pgfqpoint{2.677650in}{1.418537in}}%
\pgfpathlineto{\pgfqpoint{2.680224in}{1.413355in}}%
\pgfpathlineto{\pgfqpoint{2.683009in}{1.417515in}}%
\pgfpathlineto{\pgfqpoint{2.685586in}{1.417179in}}%
\pgfpathlineto{\pgfqpoint{2.688328in}{1.418474in}}%
\pgfpathlineto{\pgfqpoint{2.690940in}{1.419582in}}%
\pgfpathlineto{\pgfqpoint{2.693611in}{1.415797in}}%
\pgfpathlineto{\pgfqpoint{2.696293in}{1.413426in}}%
\pgfpathlineto{\pgfqpoint{2.698968in}{1.413366in}}%
\pgfpathlineto{\pgfqpoint{2.701657in}{1.416943in}}%
\pgfpathlineto{\pgfqpoint{2.704326in}{1.414022in}}%
\pgfpathlineto{\pgfqpoint{2.707125in}{1.414495in}}%
\pgfpathlineto{\pgfqpoint{2.709683in}{1.415676in}}%
\pgfpathlineto{\pgfqpoint{2.712477in}{1.417211in}}%
\pgfpathlineto{\pgfqpoint{2.715036in}{1.413645in}}%
\pgfpathlineto{\pgfqpoint{2.717773in}{1.410413in}}%
\pgfpathlineto{\pgfqpoint{2.720404in}{1.410007in}}%
\pgfpathlineto{\pgfqpoint{2.723211in}{1.409986in}}%
\pgfpathlineto{\pgfqpoint{2.725760in}{1.409992in}}%
\pgfpathlineto{\pgfqpoint{2.728439in}{1.412491in}}%
\pgfpathlineto{\pgfqpoint{2.731119in}{1.408881in}}%
\pgfpathlineto{\pgfqpoint{2.733798in}{1.405514in}}%
\pgfpathlineto{\pgfqpoint{2.736476in}{1.405514in}}%
\pgfpathlineto{\pgfqpoint{2.739155in}{1.405514in}}%
\pgfpathlineto{\pgfqpoint{2.741928in}{1.405946in}}%
\pgfpathlineto{\pgfqpoint{2.744510in}{1.408393in}}%
\pgfpathlineto{\pgfqpoint{2.747260in}{1.415359in}}%
\pgfpathlineto{\pgfqpoint{2.749868in}{1.412024in}}%
\pgfpathlineto{\pgfqpoint{2.752614in}{1.412450in}}%
\pgfpathlineto{\pgfqpoint{2.755224in}{1.413348in}}%
\pgfpathlineto{\pgfqpoint{2.758028in}{1.413958in}}%
\pgfpathlineto{\pgfqpoint{2.760581in}{1.414478in}}%
\pgfpathlineto{\pgfqpoint{2.763253in}{1.413243in}}%
\pgfpathlineto{\pgfqpoint{2.765935in}{1.419371in}}%
\pgfpathlineto{\pgfqpoint{2.768617in}{1.418822in}}%
\pgfpathlineto{\pgfqpoint{2.771373in}{1.412578in}}%
\pgfpathlineto{\pgfqpoint{2.773972in}{1.412256in}}%
\pgfpathlineto{\pgfqpoint{2.776767in}{1.408190in}}%
\pgfpathlineto{\pgfqpoint{2.779330in}{1.407239in}}%
\pgfpathlineto{\pgfqpoint{2.782113in}{1.410693in}}%
\pgfpathlineto{\pgfqpoint{2.784687in}{1.410706in}}%
\pgfpathlineto{\pgfqpoint{2.787468in}{1.411923in}}%
\pgfpathlineto{\pgfqpoint{2.790044in}{1.410818in}}%
\pgfpathlineto{\pgfqpoint{2.792721in}{1.415712in}}%
\pgfpathlineto{\pgfqpoint{2.795398in}{1.414510in}}%
\pgfpathlineto{\pgfqpoint{2.798070in}{1.414504in}}%
\pgfpathlineto{\pgfqpoint{2.800756in}{1.417360in}}%
\pgfpathlineto{\pgfqpoint{2.803435in}{1.414819in}}%
\pgfpathlineto{\pgfqpoint{2.806175in}{1.416773in}}%
\pgfpathlineto{\pgfqpoint{2.808792in}{1.414428in}}%
\pgfpathlineto{\pgfqpoint{2.811597in}{1.413343in}}%
\pgfpathlineto{\pgfqpoint{2.814141in}{1.416266in}}%
\pgfpathlineto{\pgfqpoint{2.816867in}{1.412196in}}%
\pgfpathlineto{\pgfqpoint{2.819506in}{1.419443in}}%
\pgfpathlineto{\pgfqpoint{2.822303in}{1.416997in}}%
\pgfpathlineto{\pgfqpoint{2.824851in}{1.415877in}}%
\pgfpathlineto{\pgfqpoint{2.827567in}{1.415512in}}%
\pgfpathlineto{\pgfqpoint{2.830219in}{1.414910in}}%
\pgfpathlineto{\pgfqpoint{2.832894in}{1.416521in}}%
\pgfpathlineto{\pgfqpoint{2.835698in}{1.413209in}}%
\pgfpathlineto{\pgfqpoint{2.838254in}{1.410276in}}%
\pgfpathlineto{\pgfqpoint{2.841055in}{1.414668in}}%
\pgfpathlineto{\pgfqpoint{2.843611in}{1.417896in}}%
\pgfpathlineto{\pgfqpoint{2.846408in}{1.415733in}}%
\pgfpathlineto{\pgfqpoint{2.848960in}{1.417573in}}%
\pgfpathlineto{\pgfqpoint{2.851793in}{1.413677in}}%
\pgfpathlineto{\pgfqpoint{2.854325in}{1.413802in}}%
\pgfpathlineto{\pgfqpoint{2.857003in}{1.410482in}}%
\pgfpathlineto{\pgfqpoint{2.859668in}{1.413280in}}%
\pgfpathlineto{\pgfqpoint{2.862402in}{1.413569in}}%
\pgfpathlineto{\pgfqpoint{2.865031in}{1.418778in}}%
\pgfpathlineto{\pgfqpoint{2.867713in}{1.411973in}}%
\pgfpathlineto{\pgfqpoint{2.870475in}{1.411307in}}%
\pgfpathlineto{\pgfqpoint{2.873074in}{1.418197in}}%
\pgfpathlineto{\pgfqpoint{2.875882in}{1.413168in}}%
\pgfpathlineto{\pgfqpoint{2.878431in}{1.414572in}}%
\pgfpathlineto{\pgfqpoint{2.881254in}{1.409510in}}%
\pgfpathlineto{\pgfqpoint{2.883780in}{1.413731in}}%
\pgfpathlineto{\pgfqpoint{2.886578in}{1.414763in}}%
\pgfpathlineto{\pgfqpoint{2.889145in}{1.412935in}}%
\pgfpathlineto{\pgfqpoint{2.891809in}{1.413937in}}%
\pgfpathlineto{\pgfqpoint{2.894487in}{1.413041in}}%
\pgfpathlineto{\pgfqpoint{2.897179in}{1.416224in}}%
\pgfpathlineto{\pgfqpoint{2.899858in}{1.412638in}}%
\pgfpathlineto{\pgfqpoint{2.902535in}{1.416890in}}%
\pgfpathlineto{\pgfqpoint{2.905341in}{1.413964in}}%
\pgfpathlineto{\pgfqpoint{2.907882in}{1.410875in}}%
\pgfpathlineto{\pgfqpoint{2.910631in}{1.414679in}}%
\pgfpathlineto{\pgfqpoint{2.913243in}{1.414749in}}%
\pgfpathlineto{\pgfqpoint{2.916061in}{1.414426in}}%
\pgfpathlineto{\pgfqpoint{2.918606in}{1.414332in}}%
\pgfpathlineto{\pgfqpoint{2.921363in}{1.417969in}}%
\pgfpathlineto{\pgfqpoint{2.923963in}{1.417627in}}%
\pgfpathlineto{\pgfqpoint{2.926655in}{1.412426in}}%
\pgfpathlineto{\pgfqpoint{2.929321in}{1.414530in}}%
\pgfpathlineto{\pgfqpoint{2.932033in}{1.413843in}}%
\pgfpathlineto{\pgfqpoint{2.934759in}{1.415064in}}%
\pgfpathlineto{\pgfqpoint{2.937352in}{1.410124in}}%
\pgfpathlineto{\pgfqpoint{2.940120in}{1.411644in}}%
\pgfpathlineto{\pgfqpoint{2.942711in}{1.414134in}}%
\pgfpathlineto{\pgfqpoint{2.945461in}{1.412752in}}%
\pgfpathlineto{\pgfqpoint{2.948068in}{1.410888in}}%
\pgfpathlineto{\pgfqpoint{2.950884in}{1.412199in}}%
\pgfpathlineto{\pgfqpoint{2.953422in}{1.419804in}}%
\pgfpathlineto{\pgfqpoint{2.956103in}{1.413079in}}%
\pgfpathlineto{\pgfqpoint{2.958782in}{1.412154in}}%
\pgfpathlineto{\pgfqpoint{2.961460in}{1.412258in}}%
\pgfpathlineto{\pgfqpoint{2.964127in}{1.415904in}}%
\pgfpathlineto{\pgfqpoint{2.966812in}{1.413179in}}%
\pgfpathlineto{\pgfqpoint{2.969599in}{1.417717in}}%
\pgfpathlineto{\pgfqpoint{2.972177in}{1.416671in}}%
\pgfpathlineto{\pgfqpoint{2.974972in}{1.411389in}}%
\pgfpathlineto{\pgfqpoint{2.977517in}{1.414652in}}%
\pgfpathlineto{\pgfqpoint{2.980341in}{1.411368in}}%
\pgfpathlineto{\pgfqpoint{2.982885in}{1.407846in}}%
\pgfpathlineto{\pgfqpoint{2.985666in}{1.412663in}}%
\pgfpathlineto{\pgfqpoint{2.988238in}{1.408287in}}%
\pgfpathlineto{\pgfqpoint{2.990978in}{1.418276in}}%
\pgfpathlineto{\pgfqpoint{2.993595in}{1.416588in}}%
\pgfpathlineto{\pgfqpoint{2.996300in}{1.429537in}}%
\pgfpathlineto{\pgfqpoint{2.999103in}{1.503173in}}%
\pgfpathlineto{\pgfqpoint{3.001635in}{1.474660in}}%
\pgfpathlineto{\pgfqpoint{3.004419in}{1.447447in}}%
\pgfpathlineto{\pgfqpoint{3.006993in}{1.435928in}}%
\pgfpathlineto{\pgfqpoint{3.009784in}{1.433825in}}%
\pgfpathlineto{\pgfqpoint{3.012351in}{1.428610in}}%
\pgfpathlineto{\pgfqpoint{3.015097in}{1.433517in}}%
\pgfpathlineto{\pgfqpoint{3.017707in}{1.425341in}}%
\pgfpathlineto{\pgfqpoint{3.020382in}{1.425023in}}%
\pgfpathlineto{\pgfqpoint{3.023058in}{1.428045in}}%
\pgfpathlineto{\pgfqpoint{3.025803in}{1.431991in}}%
\pgfpathlineto{\pgfqpoint{3.028412in}{1.431209in}}%
\pgfpathlineto{\pgfqpoint{3.031091in}{1.428280in}}%
\pgfpathlineto{\pgfqpoint{3.033921in}{1.420899in}}%
\pgfpathlineto{\pgfqpoint{3.036456in}{1.416598in}}%
\pgfpathlineto{\pgfqpoint{3.039262in}{1.416637in}}%
\pgfpathlineto{\pgfqpoint{3.041813in}{1.416003in}}%
\pgfpathlineto{\pgfqpoint{3.044568in}{1.422162in}}%
\pgfpathlineto{\pgfqpoint{3.047157in}{1.428254in}}%
\pgfpathlineto{\pgfqpoint{3.049988in}{1.455042in}}%
\pgfpathlineto{\pgfqpoint{3.052526in}{1.448251in}}%
\pgfpathlineto{\pgfqpoint{3.055202in}{1.429618in}}%
\pgfpathlineto{\pgfqpoint{3.057884in}{1.423865in}}%
\pgfpathlineto{\pgfqpoint{3.060561in}{1.419032in}}%
\pgfpathlineto{\pgfqpoint{3.063230in}{1.415767in}}%
\pgfpathlineto{\pgfqpoint{3.065916in}{1.415823in}}%
\pgfpathlineto{\pgfqpoint{3.068709in}{1.415997in}}%
\pgfpathlineto{\pgfqpoint{3.071266in}{1.413803in}}%
\pgfpathlineto{\pgfqpoint{3.074056in}{1.416613in}}%
\pgfpathlineto{\pgfqpoint{3.076631in}{1.411845in}}%
\pgfpathlineto{\pgfqpoint{3.079381in}{1.412003in}}%
\pgfpathlineto{\pgfqpoint{3.081990in}{1.407399in}}%
\pgfpathlineto{\pgfqpoint{3.084671in}{1.409066in}}%
\pgfpathlineto{\pgfqpoint{3.087343in}{1.414274in}}%
\pgfpathlineto{\pgfqpoint{3.090023in}{1.411510in}}%
\pgfpathlineto{\pgfqpoint{3.092699in}{1.405823in}}%
\pgfpathlineto{\pgfqpoint{3.095388in}{1.407393in}}%
\pgfpathlineto{\pgfqpoint{3.098163in}{1.407778in}}%
\pgfpathlineto{\pgfqpoint{3.100737in}{1.405514in}}%
\pgfpathlineto{\pgfqpoint{3.103508in}{1.405514in}}%
\pgfpathlineto{\pgfqpoint{3.106094in}{1.405514in}}%
\pgfpathlineto{\pgfqpoint{3.108896in}{1.405872in}}%
\pgfpathlineto{\pgfqpoint{3.111451in}{1.405514in}}%
\pgfpathlineto{\pgfqpoint{3.114242in}{1.405514in}}%
\pgfpathlineto{\pgfqpoint{3.116807in}{1.405852in}}%
\pgfpathlineto{\pgfqpoint{3.119487in}{1.405514in}}%
\pgfpathlineto{\pgfqpoint{3.122163in}{1.405514in}}%
\pgfpathlineto{\pgfqpoint{3.124842in}{1.407692in}}%
\pgfpathlineto{\pgfqpoint{3.127512in}{1.405514in}}%
\pgfpathlineto{\pgfqpoint{3.130199in}{1.410890in}}%
\pgfpathlineto{\pgfqpoint{3.132946in}{1.407795in}}%
\pgfpathlineto{\pgfqpoint{3.135550in}{1.405514in}}%
\pgfpathlineto{\pgfqpoint{3.138375in}{1.405514in}}%
\pgfpathlineto{\pgfqpoint{3.140913in}{1.405514in}}%
\pgfpathlineto{\pgfqpoint{3.143740in}{1.405514in}}%
\pgfpathlineto{\pgfqpoint{3.146271in}{1.405514in}}%
\pgfpathlineto{\pgfqpoint{3.149057in}{1.405514in}}%
\pgfpathlineto{\pgfqpoint{3.151612in}{1.405514in}}%
\pgfpathlineto{\pgfqpoint{3.154327in}{1.405514in}}%
\pgfpathlineto{\pgfqpoint{3.156981in}{1.407601in}}%
\pgfpathlineto{\pgfqpoint{3.159675in}{1.405514in}}%
\pgfpathlineto{\pgfqpoint{3.162474in}{1.405514in}}%
\pgfpathlineto{\pgfqpoint{3.165019in}{1.405514in}}%
\pgfpathlineto{\pgfqpoint{3.167776in}{1.405514in}}%
\pgfpathlineto{\pgfqpoint{3.170375in}{1.405514in}}%
\pgfpathlineto{\pgfqpoint{3.173142in}{1.405514in}}%
\pgfpathlineto{\pgfqpoint{3.175724in}{1.405514in}}%
\pgfpathlineto{\pgfqpoint{3.178525in}{1.405514in}}%
\pgfpathlineto{\pgfqpoint{3.181089in}{1.405514in}}%
\pgfpathlineto{\pgfqpoint{3.183760in}{1.405514in}}%
\pgfpathlineto{\pgfqpoint{3.186440in}{1.405514in}}%
\pgfpathlineto{\pgfqpoint{3.189117in}{1.405514in}}%
\pgfpathlineto{\pgfqpoint{3.191796in}{1.405514in}}%
\pgfpathlineto{\pgfqpoint{3.194508in}{1.405514in}}%
\pgfpathlineto{\pgfqpoint{3.197226in}{1.407889in}}%
\pgfpathlineto{\pgfqpoint{3.199823in}{1.405514in}}%
\pgfpathlineto{\pgfqpoint{3.202562in}{1.405514in}}%
\pgfpathlineto{\pgfqpoint{3.205195in}{1.405514in}}%
\pgfpathlineto{\pgfqpoint{3.207984in}{1.405514in}}%
\pgfpathlineto{\pgfqpoint{3.210545in}{1.405514in}}%
\pgfpathlineto{\pgfqpoint{3.213342in}{1.405514in}}%
\pgfpathlineto{\pgfqpoint{3.215908in}{1.405514in}}%
\pgfpathlineto{\pgfqpoint{3.218586in}{1.405514in}}%
\pgfpathlineto{\pgfqpoint{3.221255in}{1.411854in}}%
\pgfpathlineto{\pgfqpoint{3.223942in}{1.414432in}}%
\pgfpathlineto{\pgfqpoint{3.226609in}{1.409501in}}%
\pgfpathlineto{\pgfqpoint{3.229310in}{1.406612in}}%
\pgfpathlineto{\pgfqpoint{3.232069in}{1.412255in}}%
\pgfpathlineto{\pgfqpoint{3.234658in}{1.412146in}}%
\pgfpathlineto{\pgfqpoint{3.237411in}{1.410683in}}%
\pgfpathlineto{\pgfqpoint{3.240010in}{1.406718in}}%
\pgfpathlineto{\pgfqpoint{3.242807in}{1.408813in}}%
\pgfpathlineto{\pgfqpoint{3.245363in}{1.409777in}}%
\pgfpathlineto{\pgfqpoint{3.248049in}{1.407391in}}%
\pgfpathlineto{\pgfqpoint{3.250716in}{1.413526in}}%
\pgfpathlineto{\pgfqpoint{3.253404in}{1.406871in}}%
\pgfpathlineto{\pgfqpoint{3.256083in}{1.411690in}}%
\pgfpathlineto{\pgfqpoint{3.258784in}{1.410851in}}%
\pgfpathlineto{\pgfqpoint{3.261594in}{1.414041in}}%
\pgfpathlineto{\pgfqpoint{3.264119in}{1.417468in}}%
\pgfpathlineto{\pgfqpoint{3.266849in}{1.421975in}}%
\pgfpathlineto{\pgfqpoint{3.269478in}{1.422512in}}%
\pgfpathlineto{\pgfqpoint{3.272254in}{1.423933in}}%
\pgfpathlineto{\pgfqpoint{3.274831in}{1.420119in}}%
\pgfpathlineto{\pgfqpoint{3.277603in}{1.415965in}}%
\pgfpathlineto{\pgfqpoint{3.280189in}{1.421449in}}%
\pgfpathlineto{\pgfqpoint{3.282870in}{1.418898in}}%
\pgfpathlineto{\pgfqpoint{3.285534in}{1.423728in}}%
\pgfpathlineto{\pgfqpoint{3.288225in}{1.416734in}}%
\pgfpathlineto{\pgfqpoint{3.290890in}{1.420269in}}%
\pgfpathlineto{\pgfqpoint{3.293574in}{1.418329in}}%
\pgfpathlineto{\pgfqpoint{3.296376in}{1.413466in}}%
\pgfpathlineto{\pgfqpoint{3.298937in}{1.414796in}}%
\pgfpathlineto{\pgfqpoint{3.301719in}{1.412100in}}%
\pgfpathlineto{\pgfqpoint{3.304295in}{1.414786in}}%
\pgfpathlineto{\pgfqpoint{3.307104in}{1.414469in}}%
\pgfpathlineto{\pgfqpoint{3.309652in}{1.414625in}}%
\pgfpathlineto{\pgfqpoint{3.312480in}{1.417760in}}%
\pgfpathlineto{\pgfqpoint{3.315008in}{1.417785in}}%
\pgfpathlineto{\pgfqpoint{3.317688in}{1.420538in}}%
\pgfpathlineto{\pgfqpoint{3.320366in}{1.417246in}}%
\pgfpathlineto{\pgfqpoint{3.323049in}{1.413078in}}%
\pgfpathlineto{\pgfqpoint{3.325860in}{1.413813in}}%
\pgfpathlineto{\pgfqpoint{3.328401in}{1.416306in}}%
\pgfpathlineto{\pgfqpoint{3.331183in}{1.416279in}}%
\pgfpathlineto{\pgfqpoint{3.333758in}{1.420635in}}%
\pgfpathlineto{\pgfqpoint{3.336541in}{1.413848in}}%
\pgfpathlineto{\pgfqpoint{3.339101in}{1.421248in}}%
\pgfpathlineto{\pgfqpoint{3.341893in}{1.416860in}}%
\pgfpathlineto{\pgfqpoint{3.344468in}{1.414872in}}%
\pgfpathlineto{\pgfqpoint{3.347139in}{1.418252in}}%
\pgfpathlineto{\pgfqpoint{3.349828in}{1.423947in}}%
\pgfpathlineto{\pgfqpoint{3.352505in}{1.419148in}}%
\pgfpathlineto{\pgfqpoint{3.355177in}{1.417866in}}%
\pgfpathlineto{\pgfqpoint{3.357862in}{1.419988in}}%
\pgfpathlineto{\pgfqpoint{3.360620in}{1.419178in}}%
\pgfpathlineto{\pgfqpoint{3.363221in}{1.417826in}}%
\pgfpathlineto{\pgfqpoint{3.365996in}{1.419998in}}%
\pgfpathlineto{\pgfqpoint{3.368577in}{1.419939in}}%
\pgfpathlineto{\pgfqpoint{3.371357in}{1.417867in}}%
\pgfpathlineto{\pgfqpoint{3.373921in}{1.416830in}}%
\pgfpathlineto{\pgfqpoint{3.376735in}{1.421620in}}%
\pgfpathlineto{\pgfqpoint{3.379290in}{1.421014in}}%
\pgfpathlineto{\pgfqpoint{3.381959in}{1.418136in}}%
\pgfpathlineto{\pgfqpoint{3.384647in}{1.423123in}}%
\pgfpathlineto{\pgfqpoint{3.387309in}{1.421982in}}%
\pgfpathlineto{\pgfqpoint{3.390102in}{1.419703in}}%
\pgfpathlineto{\pgfqpoint{3.392681in}{1.416302in}}%
\pgfpathlineto{\pgfqpoint{3.395461in}{1.415455in}}%
\pgfpathlineto{\pgfqpoint{3.398037in}{1.414731in}}%
\pgfpathlineto{\pgfqpoint{3.400783in}{1.418671in}}%
\pgfpathlineto{\pgfqpoint{3.403394in}{1.413845in}}%
\pgfpathlineto{\pgfqpoint{3.406202in}{1.415893in}}%
\pgfpathlineto{\pgfqpoint{3.408752in}{1.419136in}}%
\pgfpathlineto{\pgfqpoint{3.411431in}{1.418055in}}%
\pgfpathlineto{\pgfqpoint{3.414109in}{1.420441in}}%
\pgfpathlineto{\pgfqpoint{3.416780in}{1.417633in}}%
\pgfpathlineto{\pgfqpoint{3.419455in}{1.416020in}}%
\pgfpathlineto{\pgfqpoint{3.422142in}{1.416424in}}%
\pgfpathlineto{\pgfqpoint{3.424887in}{1.417645in}}%
\pgfpathlineto{\pgfqpoint{3.427501in}{1.418276in}}%
\pgfpathlineto{\pgfqpoint{3.430313in}{1.419074in}}%
\pgfpathlineto{\pgfqpoint{3.432851in}{1.422124in}}%
\pgfpathlineto{\pgfqpoint{3.435635in}{1.419487in}}%
\pgfpathlineto{\pgfqpoint{3.438210in}{1.424437in}}%
\pgfpathlineto{\pgfqpoint{3.440996in}{1.424925in}}%
\pgfpathlineto{\pgfqpoint{3.443574in}{1.436797in}}%
\pgfpathlineto{\pgfqpoint{3.446257in}{1.432350in}}%
\pgfpathlineto{\pgfqpoint{3.448926in}{1.425904in}}%
\pgfpathlineto{\pgfqpoint{3.451597in}{1.421365in}}%
\pgfpathlineto{\pgfqpoint{3.454285in}{1.427028in}}%
\pgfpathlineto{\pgfqpoint{3.456960in}{1.426199in}}%
\pgfpathlineto{\pgfqpoint{3.459695in}{1.421532in}}%
\pgfpathlineto{\pgfqpoint{3.462321in}{1.422548in}}%
\pgfpathlineto{\pgfqpoint{3.465072in}{1.426878in}}%
\pgfpathlineto{\pgfqpoint{3.467678in}{1.432986in}}%
\pgfpathlineto{\pgfqpoint{3.470466in}{1.439062in}}%
\pgfpathlineto{\pgfqpoint{3.473021in}{1.428789in}}%
\pgfpathlineto{\pgfqpoint{3.475821in}{1.426677in}}%
\pgfpathlineto{\pgfqpoint{3.478378in}{1.426240in}}%
\pgfpathlineto{\pgfqpoint{3.481072in}{1.427699in}}%
\pgfpathlineto{\pgfqpoint{3.483744in}{1.421976in}}%
\pgfpathlineto{\pgfqpoint{3.486442in}{1.417990in}}%
\pgfpathlineto{\pgfqpoint{3.489223in}{1.422181in}}%
\pgfpathlineto{\pgfqpoint{3.491783in}{1.418045in}}%
\pgfpathlineto{\pgfqpoint{3.494581in}{1.420343in}}%
\pgfpathlineto{\pgfqpoint{3.497139in}{1.419884in}}%
\pgfpathlineto{\pgfqpoint{3.499909in}{1.416420in}}%
\pgfpathlineto{\pgfqpoint{3.502488in}{1.423449in}}%
\pgfpathlineto{\pgfqpoint{3.505262in}{1.424091in}}%
\pgfpathlineto{\pgfqpoint{3.507840in}{1.425486in}}%
\pgfpathlineto{\pgfqpoint{3.510533in}{1.428544in}}%
\pgfpathlineto{\pgfqpoint{3.513209in}{1.420963in}}%
\pgfpathlineto{\pgfqpoint{3.515884in}{1.421087in}}%
\pgfpathlineto{\pgfqpoint{3.518565in}{1.419506in}}%
\pgfpathlineto{\pgfqpoint{3.521244in}{1.421344in}}%
\pgfpathlineto{\pgfqpoint{3.524041in}{1.419683in}}%
\pgfpathlineto{\pgfqpoint{3.526601in}{1.426879in}}%
\pgfpathlineto{\pgfqpoint{3.529327in}{1.425059in}}%
\pgfpathlineto{\pgfqpoint{3.531955in}{1.417623in}}%
\pgfpathlineto{\pgfqpoint{3.534783in}{1.422694in}}%
\pgfpathlineto{\pgfqpoint{3.537309in}{1.419562in}}%
\pgfpathlineto{\pgfqpoint{3.540093in}{1.417414in}}%
\pgfpathlineto{\pgfqpoint{3.542656in}{1.421678in}}%
\pgfpathlineto{\pgfqpoint{3.545349in}{1.416032in}}%
\pgfpathlineto{\pgfqpoint{3.548029in}{1.410126in}}%
\pgfpathlineto{\pgfqpoint{3.550713in}{1.413069in}}%
\pgfpathlineto{\pgfqpoint{3.553498in}{1.416023in}}%
\pgfpathlineto{\pgfqpoint{3.556061in}{1.416832in}}%
\pgfpathlineto{\pgfqpoint{3.558853in}{1.414332in}}%
\pgfpathlineto{\pgfqpoint{3.561420in}{1.417668in}}%
\pgfpathlineto{\pgfqpoint{3.564188in}{1.421828in}}%
\pgfpathlineto{\pgfqpoint{3.566774in}{1.422004in}}%
\pgfpathlineto{\pgfqpoint{3.569584in}{1.421595in}}%
\pgfpathlineto{\pgfqpoint{3.572126in}{1.417935in}}%
\pgfpathlineto{\pgfqpoint{3.574814in}{1.415325in}}%
\pgfpathlineto{\pgfqpoint{3.577487in}{1.413703in}}%
\pgfpathlineto{\pgfqpoint{3.580191in}{1.415600in}}%
\pgfpathlineto{\pgfqpoint{3.582851in}{1.413723in}}%
\pgfpathlineto{\pgfqpoint{3.585532in}{1.413631in}}%
\pgfpathlineto{\pgfqpoint{3.588258in}{1.412155in}}%
\pgfpathlineto{\pgfqpoint{3.590883in}{1.417268in}}%
\pgfpathlineto{\pgfqpoint{3.593620in}{1.419712in}}%
\pgfpathlineto{\pgfqpoint{3.596240in}{1.418130in}}%
\pgfpathlineto{\pgfqpoint{3.598998in}{1.418765in}}%
\pgfpathlineto{\pgfqpoint{3.601590in}{1.416560in}}%
\pgfpathlineto{\pgfqpoint{3.604387in}{1.418573in}}%
\pgfpathlineto{\pgfqpoint{3.606951in}{1.411914in}}%
\pgfpathlineto{\pgfqpoint{3.609632in}{1.413714in}}%
\pgfpathlineto{\pgfqpoint{3.612311in}{1.408777in}}%
\pgfpathlineto{\pgfqpoint{3.614982in}{1.405514in}}%
\pgfpathlineto{\pgfqpoint{3.617667in}{1.408759in}}%
\pgfpathlineto{\pgfqpoint{3.620345in}{1.410702in}}%
\pgfpathlineto{\pgfqpoint{3.623165in}{1.413076in}}%
\pgfpathlineto{\pgfqpoint{3.625689in}{1.408210in}}%
\pgfpathlineto{\pgfqpoint{3.628460in}{1.419560in}}%
\pgfpathlineto{\pgfqpoint{3.631058in}{1.415278in}}%
\pgfpathlineto{\pgfqpoint{3.633858in}{1.420624in}}%
\pgfpathlineto{\pgfqpoint{3.636413in}{1.418388in}}%
\pgfpathlineto{\pgfqpoint{3.639207in}{1.427346in}}%
\pgfpathlineto{\pgfqpoint{3.641773in}{1.425691in}}%
\pgfpathlineto{\pgfqpoint{3.644452in}{1.416965in}}%
\pgfpathlineto{\pgfqpoint{3.647130in}{1.411011in}}%
\pgfpathlineto{\pgfqpoint{3.649837in}{1.412842in}}%
\pgfpathlineto{\pgfqpoint{3.652628in}{1.415806in}}%
\pgfpathlineto{\pgfqpoint{3.655165in}{1.420499in}}%
\pgfpathlineto{\pgfqpoint{3.657917in}{1.433144in}}%
\pgfpathlineto{\pgfqpoint{3.660515in}{1.461308in}}%
\pgfpathlineto{\pgfqpoint{3.663276in}{1.482361in}}%
\pgfpathlineto{\pgfqpoint{3.665864in}{1.470988in}}%
\pgfpathlineto{\pgfqpoint{3.668665in}{1.449975in}}%
\pgfpathlineto{\pgfqpoint{3.671232in}{1.436116in}}%
\pgfpathlineto{\pgfqpoint{3.673911in}{1.426973in}}%
\pgfpathlineto{\pgfqpoint{3.676591in}{1.433198in}}%
\pgfpathlineto{\pgfqpoint{3.679273in}{1.446579in}}%
\pgfpathlineto{\pgfqpoint{3.681948in}{1.441128in}}%
\pgfpathlineto{\pgfqpoint{3.684620in}{1.431000in}}%
\pgfpathlineto{\pgfqpoint{3.687442in}{1.417220in}}%
\pgfpathlineto{\pgfqpoint{3.689983in}{1.411443in}}%
\pgfpathlineto{\pgfqpoint{3.692765in}{1.413734in}}%
\pgfpathlineto{\pgfqpoint{3.695331in}{1.414470in}}%
\pgfpathlineto{\pgfqpoint{3.698125in}{1.411589in}}%
\pgfpathlineto{\pgfqpoint{3.700684in}{1.411103in}}%
\pgfpathlineto{\pgfqpoint{3.703460in}{1.409508in}}%
\pgfpathlineto{\pgfqpoint{3.706053in}{1.412012in}}%
\pgfpathlineto{\pgfqpoint{3.708729in}{1.412828in}}%
\pgfpathlineto{\pgfqpoint{3.711410in}{1.411188in}}%
\pgfpathlineto{\pgfqpoint{3.714086in}{1.411920in}}%
\pgfpathlineto{\pgfqpoint{3.716875in}{1.410440in}}%
\pgfpathlineto{\pgfqpoint{3.719446in}{1.414330in}}%
\pgfpathlineto{\pgfqpoint{3.722228in}{1.411086in}}%
\pgfpathlineto{\pgfqpoint{3.724804in}{1.408424in}}%
\pgfpathlineto{\pgfqpoint{3.727581in}{1.409748in}}%
\pgfpathlineto{\pgfqpoint{3.730158in}{1.415185in}}%
\pgfpathlineto{\pgfqpoint{3.732950in}{1.414878in}}%
\pgfpathlineto{\pgfqpoint{3.735509in}{1.417638in}}%
\pgfpathlineto{\pgfqpoint{3.738194in}{1.414112in}}%
\pgfpathlineto{\pgfqpoint{3.740874in}{1.417248in}}%
\pgfpathlineto{\pgfqpoint{3.743548in}{1.418120in}}%
\pgfpathlineto{\pgfqpoint{3.746229in}{1.419068in}}%
\pgfpathlineto{\pgfqpoint{3.748903in}{1.419299in}}%
\pgfpathlineto{\pgfqpoint{3.751728in}{1.414285in}}%
\pgfpathlineto{\pgfqpoint{3.754265in}{1.422487in}}%
\pgfpathlineto{\pgfqpoint{3.757065in}{1.420489in}}%
\pgfpathlineto{\pgfqpoint{3.759622in}{1.418854in}}%
\pgfpathlineto{\pgfqpoint{3.762389in}{1.421860in}}%
\pgfpathlineto{\pgfqpoint{3.764966in}{1.415397in}}%
\pgfpathlineto{\pgfqpoint{3.767782in}{1.413488in}}%
\pgfpathlineto{\pgfqpoint{3.770323in}{1.413616in}}%
\pgfpathlineto{\pgfqpoint{3.773014in}{1.426676in}}%
\pgfpathlineto{\pgfqpoint{3.775691in}{1.492106in}}%
\pgfpathlineto{\pgfqpoint{3.778370in}{1.484636in}}%
\pgfpathlineto{\pgfqpoint{3.781046in}{1.454774in}}%
\pgfpathlineto{\pgfqpoint{3.783725in}{1.438018in}}%
\pgfpathlineto{\pgfqpoint{3.786504in}{1.430281in}}%
\pgfpathlineto{\pgfqpoint{3.789084in}{1.422474in}}%
\pgfpathlineto{\pgfqpoint{3.791897in}{1.419642in}}%
\pgfpathlineto{\pgfqpoint{3.794435in}{1.417283in}}%
\pgfpathlineto{\pgfqpoint{3.797265in}{1.409009in}}%
\pgfpathlineto{\pgfqpoint{3.799797in}{1.405514in}}%
\pgfpathlineto{\pgfqpoint{3.802569in}{1.405514in}}%
\pgfpathlineto{\pgfqpoint{3.805145in}{1.406391in}}%
\pgfpathlineto{\pgfqpoint{3.807832in}{1.407809in}}%
\pgfpathlineto{\pgfqpoint{3.810510in}{1.407356in}}%
\pgfpathlineto{\pgfqpoint{3.813172in}{1.412334in}}%
\pgfpathlineto{\pgfqpoint{3.815983in}{1.416402in}}%
\pgfpathlineto{\pgfqpoint{3.818546in}{1.413243in}}%
\pgfpathlineto{\pgfqpoint{3.821315in}{1.410644in}}%
\pgfpathlineto{\pgfqpoint{3.823903in}{1.410048in}}%
\pgfpathlineto{\pgfqpoint{3.826679in}{1.410844in}}%
\pgfpathlineto{\pgfqpoint{3.829252in}{1.415603in}}%
\pgfpathlineto{\pgfqpoint{3.832053in}{1.414424in}}%
\pgfpathlineto{\pgfqpoint{3.834616in}{1.425038in}}%
\pgfpathlineto{\pgfqpoint{3.837286in}{1.419041in}}%
\pgfpathlineto{\pgfqpoint{3.839960in}{1.419963in}}%
\pgfpathlineto{\pgfqpoint{3.842641in}{1.420814in}}%
\pgfpathlineto{\pgfqpoint{3.845329in}{1.420370in}}%
\pgfpathlineto{\pgfqpoint{3.848005in}{1.417090in}}%
\pgfpathlineto{\pgfqpoint{3.850814in}{1.419583in}}%
\pgfpathlineto{\pgfqpoint{3.853358in}{1.420618in}}%
\pgfpathlineto{\pgfqpoint{3.856100in}{1.413912in}}%
\pgfpathlineto{\pgfqpoint{3.858720in}{1.418942in}}%
\pgfpathlineto{\pgfqpoint{3.861561in}{1.415781in}}%
\pgfpathlineto{\pgfqpoint{3.864073in}{1.420309in}}%
\pgfpathlineto{\pgfqpoint{3.866815in}{1.423521in}}%
\pgfpathlineto{\pgfqpoint{3.869435in}{1.422755in}}%
\pgfpathlineto{\pgfqpoint{3.872114in}{1.422967in}}%
\pgfpathlineto{\pgfqpoint{3.874790in}{1.422582in}}%
\pgfpathlineto{\pgfqpoint{3.877466in}{1.426215in}}%
\pgfpathlineto{\pgfqpoint{3.880237in}{1.430892in}}%
\pgfpathlineto{\pgfqpoint{3.882850in}{1.423409in}}%
\pgfpathlineto{\pgfqpoint{3.885621in}{1.426796in}}%
\pgfpathlineto{\pgfqpoint{3.888188in}{1.417702in}}%
\pgfpathlineto{\pgfqpoint{3.890926in}{1.415481in}}%
\pgfpathlineto{\pgfqpoint{3.893541in}{1.417398in}}%
\pgfpathlineto{\pgfqpoint{3.896345in}{1.421115in}}%
\pgfpathlineto{\pgfqpoint{3.898891in}{1.420436in}}%
\pgfpathlineto{\pgfqpoint{3.901573in}{1.416468in}}%
\pgfpathlineto{\pgfqpoint{3.904252in}{1.418885in}}%
\pgfpathlineto{\pgfqpoint{3.906918in}{1.421017in}}%
\pgfpathlineto{\pgfqpoint{3.909602in}{1.421011in}}%
\pgfpathlineto{\pgfqpoint{3.912296in}{1.418684in}}%
\pgfpathlineto{\pgfqpoint{3.915107in}{1.418748in}}%
\pgfpathlineto{\pgfqpoint{3.917646in}{1.421674in}}%
\pgfpathlineto{\pgfqpoint{3.920412in}{1.423590in}}%
\pgfpathlineto{\pgfqpoint{3.923005in}{1.422468in}}%
\pgfpathlineto{\pgfqpoint{3.925778in}{1.421658in}}%
\pgfpathlineto{\pgfqpoint{3.928347in}{1.422778in}}%
\pgfpathlineto{\pgfqpoint{3.931202in}{1.422860in}}%
\pgfpathlineto{\pgfqpoint{3.933714in}{1.420050in}}%
\pgfpathlineto{\pgfqpoint{3.936395in}{1.420727in}}%
\pgfpathlineto{\pgfqpoint{3.939075in}{1.423547in}}%
\pgfpathlineto{\pgfqpoint{3.941778in}{1.421752in}}%
\pgfpathlineto{\pgfqpoint{3.944431in}{1.420862in}}%
\pgfpathlineto{\pgfqpoint{3.947101in}{1.425708in}}%
\pgfpathlineto{\pgfqpoint{3.949894in}{1.420716in}}%
\pgfpathlineto{\pgfqpoint{3.952464in}{1.414074in}}%
\pgfpathlineto{\pgfqpoint{3.955211in}{1.415140in}}%
\pgfpathlineto{\pgfqpoint{3.957823in}{1.418388in}}%
\pgfpathlineto{\pgfqpoint{3.960635in}{1.419218in}}%
\pgfpathlineto{\pgfqpoint{3.963176in}{1.423762in}}%
\pgfpathlineto{\pgfqpoint{3.966013in}{1.423075in}}%
\pgfpathlineto{\pgfqpoint{3.968523in}{1.424029in}}%
\pgfpathlineto{\pgfqpoint{3.971250in}{1.426593in}}%
\pgfpathlineto{\pgfqpoint{3.973885in}{1.425887in}}%
\pgfpathlineto{\pgfqpoint{3.976563in}{1.421464in}}%
\pgfpathlineto{\pgfqpoint{3.979389in}{1.419857in}}%
\pgfpathlineto{\pgfqpoint{3.981929in}{1.424645in}}%
\pgfpathlineto{\pgfqpoint{3.984714in}{1.418062in}}%
\pgfpathlineto{\pgfqpoint{3.987270in}{1.425557in}}%
\pgfpathlineto{\pgfqpoint{3.990055in}{1.423559in}}%
\pgfpathlineto{\pgfqpoint{3.992642in}{1.425084in}}%
\pgfpathlineto{\pgfqpoint{3.995417in}{1.428291in}}%
\pgfpathlineto{\pgfqpoint{3.997990in}{1.424402in}}%
\pgfpathlineto{\pgfqpoint{4.000674in}{1.428184in}}%
\pgfpathlineto{\pgfqpoint{4.003348in}{1.422593in}}%
\pgfpathlineto{\pgfqpoint{4.006034in}{1.425070in}}%
\pgfpathlineto{\pgfqpoint{4.008699in}{1.424653in}}%
\pgfpathlineto{\pgfqpoint{4.011394in}{1.426102in}}%
\pgfpathlineto{\pgfqpoint{4.014186in}{1.425055in}}%
\pgfpathlineto{\pgfqpoint{4.016744in}{1.426591in}}%
\pgfpathlineto{\pgfqpoint{4.019518in}{1.424114in}}%
\pgfpathlineto{\pgfqpoint{4.022097in}{1.424350in}}%
\pgfpathlineto{\pgfqpoint{4.024868in}{1.425484in}}%
\pgfpathlineto{\pgfqpoint{4.027447in}{1.422785in}}%
\pgfpathlineto{\pgfqpoint{4.030229in}{1.418923in}}%
\pgfpathlineto{\pgfqpoint{4.032817in}{1.422424in}}%
\pgfpathlineto{\pgfqpoint{4.035492in}{1.422088in}}%
\pgfpathlineto{\pgfqpoint{4.038174in}{1.420009in}}%
\pgfpathlineto{\pgfqpoint{4.040852in}{1.425339in}}%
\pgfpathlineto{\pgfqpoint{4.043667in}{1.423852in}}%
\pgfpathlineto{\pgfqpoint{4.046210in}{1.429064in}}%
\pgfpathlineto{\pgfqpoint{4.049006in}{1.424916in}}%
\pgfpathlineto{\pgfqpoint{4.051557in}{1.425792in}}%
\pgfpathlineto{\pgfqpoint{4.054326in}{1.422181in}}%
\pgfpathlineto{\pgfqpoint{4.056911in}{1.426798in}}%
\pgfpathlineto{\pgfqpoint{4.059702in}{1.416437in}}%
\pgfpathlineto{\pgfqpoint{4.062266in}{1.423935in}}%
\pgfpathlineto{\pgfqpoint{4.064957in}{1.429949in}}%
\pgfpathlineto{\pgfqpoint{4.067636in}{1.426528in}}%
\pgfpathlineto{\pgfqpoint{4.070313in}{1.427492in}}%
\pgfpathlineto{\pgfqpoint{4.072985in}{1.427933in}}%
\pgfpathlineto{\pgfqpoint{4.075705in}{1.427198in}}%
\pgfpathlineto{\pgfqpoint{4.078471in}{1.429125in}}%
\pgfpathlineto{\pgfqpoint{4.081018in}{1.422468in}}%
\pgfpathlineto{\pgfqpoint{4.083870in}{1.421896in}}%
\pgfpathlineto{\pgfqpoint{4.086385in}{1.421083in}}%
\pgfpathlineto{\pgfqpoint{4.089159in}{1.422332in}}%
\pgfpathlineto{\pgfqpoint{4.091729in}{1.426024in}}%
\pgfpathlineto{\pgfqpoint{4.094527in}{1.428521in}}%
\pgfpathlineto{\pgfqpoint{4.097092in}{1.423673in}}%
\pgfpathlineto{\pgfqpoint{4.099777in}{1.423673in}}%
\pgfpathlineto{\pgfqpoint{4.102456in}{1.425442in}}%
\pgfpathlineto{\pgfqpoint{4.105185in}{1.422945in}}%
\pgfpathlineto{\pgfqpoint{4.107814in}{1.426830in}}%
\pgfpathlineto{\pgfqpoint{4.110488in}{1.422757in}}%
\pgfpathlineto{\pgfqpoint{4.113252in}{1.428848in}}%
\pgfpathlineto{\pgfqpoint{4.115844in}{1.424416in}}%
\pgfpathlineto{\pgfqpoint{4.118554in}{1.422865in}}%
\pgfpathlineto{\pgfqpoint{4.121205in}{1.424503in}}%
\pgfpathlineto{\pgfqpoint{4.124019in}{1.419032in}}%
\pgfpathlineto{\pgfqpoint{4.126553in}{1.418179in}}%
\pgfpathlineto{\pgfqpoint{4.129349in}{1.418258in}}%
\pgfpathlineto{\pgfqpoint{4.131920in}{1.421800in}}%
\pgfpathlineto{\pgfqpoint{4.134615in}{1.418988in}}%
\pgfpathlineto{\pgfqpoint{4.137272in}{1.420998in}}%
\pgfpathlineto{\pgfqpoint{4.139963in}{1.421115in}}%
\pgfpathlineto{\pgfqpoint{4.142713in}{1.417654in}}%
\pgfpathlineto{\pgfqpoint{4.145310in}{1.421616in}}%
\pgfpathlineto{\pgfqpoint{4.148082in}{1.418597in}}%
\pgfpathlineto{\pgfqpoint{4.150665in}{1.419351in}}%
\pgfpathlineto{\pgfqpoint{4.153423in}{1.417077in}}%
\pgfpathlineto{\pgfqpoint{4.156016in}{1.420618in}}%
\pgfpathlineto{\pgfqpoint{4.158806in}{1.426962in}}%
\pgfpathlineto{\pgfqpoint{4.161380in}{1.422461in}}%
\pgfpathlineto{\pgfqpoint{4.164059in}{1.419742in}}%
\pgfpathlineto{\pgfqpoint{4.166737in}{1.415476in}}%
\pgfpathlineto{\pgfqpoint{4.169415in}{1.420667in}}%
\pgfpathlineto{\pgfqpoint{4.172093in}{1.417537in}}%
\pgfpathlineto{\pgfqpoint{4.174770in}{1.419636in}}%
\pgfpathlineto{\pgfqpoint{4.177593in}{1.420725in}}%
\pgfpathlineto{\pgfqpoint{4.180129in}{1.421580in}}%
\pgfpathlineto{\pgfqpoint{4.182899in}{1.417639in}}%
\pgfpathlineto{\pgfqpoint{4.185481in}{1.406536in}}%
\pgfpathlineto{\pgfqpoint{4.188318in}{1.407083in}}%
\pgfpathlineto{\pgfqpoint{4.190842in}{1.412944in}}%
\pgfpathlineto{\pgfqpoint{4.193638in}{1.411792in}}%
\pgfpathlineto{\pgfqpoint{4.196186in}{1.418855in}}%
\pgfpathlineto{\pgfqpoint{4.198878in}{1.416558in}}%
\pgfpathlineto{\pgfqpoint{4.201542in}{1.412389in}}%
\pgfpathlineto{\pgfqpoint{4.204240in}{1.413431in}}%
\pgfpathlineto{\pgfqpoint{4.207076in}{1.405514in}}%
\pgfpathlineto{\pgfqpoint{4.209597in}{1.405514in}}%
\pgfpathlineto{\pgfqpoint{4.212383in}{1.405514in}}%
\pgfpathlineto{\pgfqpoint{4.214948in}{1.406702in}}%
\pgfpathlineto{\pgfqpoint{4.217694in}{1.413241in}}%
\pgfpathlineto{\pgfqpoint{4.220304in}{1.412784in}}%
\pgfpathlineto{\pgfqpoint{4.223082in}{1.418050in}}%
\pgfpathlineto{\pgfqpoint{4.225654in}{1.412137in}}%
\pgfpathlineto{\pgfqpoint{4.228331in}{1.416096in}}%
\pgfpathlineto{\pgfqpoint{4.231013in}{1.416665in}}%
\pgfpathlineto{\pgfqpoint{4.233691in}{1.425088in}}%
\pgfpathlineto{\pgfqpoint{4.236375in}{1.428666in}}%
\pgfpathlineto{\pgfqpoint{4.239084in}{1.429420in}}%
\pgfpathlineto{\pgfqpoint{4.241900in}{1.421112in}}%
\pgfpathlineto{\pgfqpoint{4.244394in}{1.418484in}}%
\pgfpathlineto{\pgfqpoint{4.247225in}{1.419958in}}%
\pgfpathlineto{\pgfqpoint{4.249767in}{1.424604in}}%
\pgfpathlineto{\pgfqpoint{4.252581in}{1.426407in}}%
\pgfpathlineto{\pgfqpoint{4.255120in}{1.426566in}}%
\pgfpathlineto{\pgfqpoint{4.257958in}{1.425945in}}%
\pgfpathlineto{\pgfqpoint{4.260477in}{1.428683in}}%
\pgfpathlineto{\pgfqpoint{4.263157in}{1.429288in}}%
\pgfpathlineto{\pgfqpoint{4.265824in}{1.427758in}}%
\pgfpathlineto{\pgfqpoint{4.268590in}{1.429866in}}%
\pgfpathlineto{\pgfqpoint{4.271187in}{1.426169in}}%
\pgfpathlineto{\pgfqpoint{4.273874in}{1.418482in}}%
\pgfpathlineto{\pgfqpoint{4.276635in}{1.417126in}}%
\pgfpathlineto{\pgfqpoint{4.279212in}{1.417397in}}%
\pgfpathlineto{\pgfqpoint{4.282000in}{1.426320in}}%
\pgfpathlineto{\pgfqpoint{4.284586in}{1.423822in}}%
\pgfpathlineto{\pgfqpoint{4.287399in}{1.420341in}}%
\pgfpathlineto{\pgfqpoint{4.289936in}{1.418243in}}%
\pgfpathlineto{\pgfqpoint{4.292786in}{1.418991in}}%
\pgfpathlineto{\pgfqpoint{4.295299in}{1.413604in}}%
\pgfpathlineto{\pgfqpoint{4.297977in}{1.420504in}}%
\pgfpathlineto{\pgfqpoint{4.300656in}{1.421894in}}%
\pgfpathlineto{\pgfqpoint{4.303357in}{1.417399in}}%
\pgfpathlineto{\pgfqpoint{4.306118in}{1.416669in}}%
\pgfpathlineto{\pgfqpoint{4.308691in}{1.420824in}}%
\pgfpathlineto{\pgfqpoint{4.311494in}{1.418002in}}%
\pgfpathlineto{\pgfqpoint{4.314032in}{1.417126in}}%
\pgfpathlineto{\pgfqpoint{4.316856in}{1.412777in}}%
\pgfpathlineto{\pgfqpoint{4.319405in}{1.416541in}}%
\pgfpathlineto{\pgfqpoint{4.322181in}{1.415029in}}%
\pgfpathlineto{\pgfqpoint{4.324760in}{1.413175in}}%
\pgfpathlineto{\pgfqpoint{4.327440in}{1.415051in}}%
\pgfpathlineto{\pgfqpoint{4.330118in}{1.417150in}}%
\pgfpathlineto{\pgfqpoint{4.332796in}{1.417990in}}%
\pgfpathlineto{\pgfqpoint{4.335463in}{1.415756in}}%
\pgfpathlineto{\pgfqpoint{4.338154in}{1.414461in}}%
\pgfpathlineto{\pgfqpoint{4.340976in}{1.413846in}}%
\pgfpathlineto{\pgfqpoint{4.343510in}{1.417754in}}%
\pgfpathlineto{\pgfqpoint{4.346263in}{1.414366in}}%
\pgfpathlineto{\pgfqpoint{4.348868in}{1.416051in}}%
\pgfpathlineto{\pgfqpoint{4.351645in}{1.422383in}}%
\pgfpathlineto{\pgfqpoint{4.354224in}{1.422092in}}%
\pgfpathlineto{\pgfqpoint{4.357014in}{1.420708in}}%
\pgfpathlineto{\pgfqpoint{4.359582in}{1.415770in}}%
\pgfpathlineto{\pgfqpoint{4.362270in}{1.418069in}}%
\pgfpathlineto{\pgfqpoint{4.364936in}{1.417505in}}%
\pgfpathlineto{\pgfqpoint{4.367646in}{1.421023in}}%
\pgfpathlineto{\pgfqpoint{4.370437in}{1.420307in}}%
\pgfpathlineto{\pgfqpoint{4.372976in}{1.422184in}}%
\pgfpathlineto{\pgfqpoint{4.375761in}{1.417140in}}%
\pgfpathlineto{\pgfqpoint{4.378329in}{1.419228in}}%
\pgfpathlineto{\pgfqpoint{4.381097in}{1.415219in}}%
\pgfpathlineto{\pgfqpoint{4.383674in}{1.416563in}}%
\pgfpathlineto{\pgfqpoint{4.386431in}{1.414070in}}%
\pgfpathlineto{\pgfqpoint{4.389044in}{1.410676in}}%
\pgfpathlineto{\pgfqpoint{4.391721in}{1.418743in}}%
\pgfpathlineto{\pgfqpoint{4.394400in}{1.416356in}}%
\pgfpathlineto{\pgfqpoint{4.397076in}{1.408073in}}%
\pgfpathlineto{\pgfqpoint{4.399745in}{1.412659in}}%
\pgfpathlineto{\pgfqpoint{4.402468in}{1.412537in}}%
\pgfpathlineto{\pgfqpoint{4.405234in}{1.416543in}}%
\pgfpathlineto{\pgfqpoint{4.407788in}{1.419499in}}%
\pgfpathlineto{\pgfqpoint{4.410587in}{1.414002in}}%
\pgfpathlineto{\pgfqpoint{4.413149in}{1.415594in}}%
\pgfpathlineto{\pgfqpoint{4.415932in}{1.417145in}}%
\pgfpathlineto{\pgfqpoint{4.418506in}{1.415128in}}%
\pgfpathlineto{\pgfqpoint{4.421292in}{1.416008in}}%
\pgfpathlineto{\pgfqpoint{4.423863in}{1.418196in}}%
\pgfpathlineto{\pgfqpoint{4.426534in}{1.416874in}}%
\pgfpathlineto{\pgfqpoint{4.429220in}{1.418345in}}%
\pgfpathlineto{\pgfqpoint{4.431901in}{1.419406in}}%
\pgfpathlineto{\pgfqpoint{4.434569in}{1.421485in}}%
\pgfpathlineto{\pgfqpoint{4.437253in}{1.431998in}}%
\pgfpathlineto{\pgfqpoint{4.440041in}{1.442301in}}%
\pgfpathlineto{\pgfqpoint{4.442611in}{1.429459in}}%
\pgfpathlineto{\pgfqpoint{4.445423in}{1.424391in}}%
\pgfpathlineto{\pgfqpoint{4.447965in}{1.439812in}}%
\pgfpathlineto{\pgfqpoint{4.450767in}{1.447696in}}%
\pgfpathlineto{\pgfqpoint{4.453312in}{1.432942in}}%
\pgfpathlineto{\pgfqpoint{4.456138in}{1.433268in}}%
\pgfpathlineto{\pgfqpoint{4.458681in}{1.443436in}}%
\pgfpathlineto{\pgfqpoint{4.461367in}{1.439220in}}%
\pgfpathlineto{\pgfqpoint{4.464029in}{1.431109in}}%
\pgfpathlineto{\pgfqpoint{4.466717in}{1.435383in}}%
\pgfpathlineto{\pgfqpoint{4.469492in}{1.428719in}}%
\pgfpathlineto{\pgfqpoint{4.472059in}{1.423584in}}%
\pgfpathlineto{\pgfqpoint{4.474861in}{1.425707in}}%
\pgfpathlineto{\pgfqpoint{4.477430in}{1.423493in}}%
\pgfpathlineto{\pgfqpoint{4.480201in}{1.425194in}}%
\pgfpathlineto{\pgfqpoint{4.482778in}{1.422693in}}%
\pgfpathlineto{\pgfqpoint{4.485581in}{1.422359in}}%
\pgfpathlineto{\pgfqpoint{4.488130in}{1.420506in}}%
\pgfpathlineto{\pgfqpoint{4.490822in}{1.423897in}}%
\pgfpathlineto{\pgfqpoint{4.493492in}{1.419376in}}%
\pgfpathlineto{\pgfqpoint{4.496167in}{1.422241in}}%
\pgfpathlineto{\pgfqpoint{4.498850in}{1.419615in}}%
\pgfpathlineto{\pgfqpoint{4.501529in}{1.420382in}}%
\pgfpathlineto{\pgfqpoint{4.504305in}{1.426437in}}%
\pgfpathlineto{\pgfqpoint{4.506893in}{1.434200in}}%
\pgfpathlineto{\pgfqpoint{4.509643in}{1.434649in}}%
\pgfpathlineto{\pgfqpoint{4.512246in}{1.441822in}}%
\pgfpathlineto{\pgfqpoint{4.515080in}{1.430207in}}%
\pgfpathlineto{\pgfqpoint{4.517598in}{1.428450in}}%
\pgfpathlineto{\pgfqpoint{4.520345in}{1.423849in}}%
\pgfpathlineto{\pgfqpoint{4.522962in}{1.426567in}}%
\pgfpathlineto{\pgfqpoint{4.525640in}{1.424601in}}%
\pgfpathlineto{\pgfqpoint{4.528307in}{1.422919in}}%
\pgfpathlineto{\pgfqpoint{4.530990in}{1.422641in}}%
\pgfpathlineto{\pgfqpoint{4.533764in}{1.423787in}}%
\pgfpathlineto{\pgfqpoint{4.536400in}{1.433771in}}%
\pgfpathlineto{\pgfqpoint{4.539144in}{1.422541in}}%
\pgfpathlineto{\pgfqpoint{4.541711in}{1.422516in}}%
\pgfpathlineto{\pgfqpoint{4.544464in}{1.417331in}}%
\pgfpathlineto{\pgfqpoint{4.547064in}{1.416892in}}%
\pgfpathlineto{\pgfqpoint{4.549822in}{1.413866in}}%
\pgfpathlineto{\pgfqpoint{4.552425in}{1.416938in}}%
\pgfpathlineto{\pgfqpoint{4.555106in}{1.413845in}}%
\pgfpathlineto{\pgfqpoint{4.557777in}{1.420580in}}%
\pgfpathlineto{\pgfqpoint{4.560448in}{1.421598in}}%
\pgfpathlineto{\pgfqpoint{4.563125in}{1.420345in}}%
\pgfpathlineto{\pgfqpoint{4.565820in}{1.425549in}}%
\pgfpathlineto{\pgfqpoint{4.568612in}{1.418818in}}%
\pgfpathlineto{\pgfqpoint{4.571171in}{1.426369in}}%
\pgfpathlineto{\pgfqpoint{4.573947in}{1.427432in}}%
\pgfpathlineto{\pgfqpoint{4.576531in}{1.424470in}}%
\pgfpathlineto{\pgfqpoint{4.579305in}{1.419930in}}%
\pgfpathlineto{\pgfqpoint{4.581888in}{1.425410in}}%
\pgfpathlineto{\pgfqpoint{4.584672in}{1.425432in}}%
\pgfpathlineto{\pgfqpoint{4.587244in}{1.423831in}}%
\pgfpathlineto{\pgfqpoint{4.589920in}{1.422925in}}%
\pgfpathlineto{\pgfqpoint{4.592589in}{1.434422in}}%
\pgfpathlineto{\pgfqpoint{4.595281in}{1.429439in}}%
\pgfpathlineto{\pgfqpoint{4.597951in}{1.436431in}}%
\pgfpathlineto{\pgfqpoint{4.600633in}{1.435474in}}%
\pgfpathlineto{\pgfqpoint{4.603430in}{1.427549in}}%
\pgfpathlineto{\pgfqpoint{4.605990in}{1.426373in}}%
\pgfpathlineto{\pgfqpoint{4.608808in}{1.423817in}}%
\pgfpathlineto{\pgfqpoint{4.611350in}{1.422232in}}%
\pgfpathlineto{\pgfqpoint{4.614134in}{1.419096in}}%
\pgfpathlineto{\pgfqpoint{4.616702in}{1.422509in}}%
\pgfpathlineto{\pgfqpoint{4.619529in}{1.424314in}}%
\pgfpathlineto{\pgfqpoint{4.622056in}{1.423446in}}%
\pgfpathlineto{\pgfqpoint{4.624741in}{1.425275in}}%
\pgfpathlineto{\pgfqpoint{4.627411in}{1.422064in}}%
\pgfpathlineto{\pgfqpoint{4.630096in}{1.424731in}}%
\pgfpathlineto{\pgfqpoint{4.632902in}{1.423740in}}%
\pgfpathlineto{\pgfqpoint{4.635445in}{1.414796in}}%
\pgfpathlineto{\pgfqpoint{4.638204in}{1.418450in}}%
\pgfpathlineto{\pgfqpoint{4.640809in}{1.418589in}}%
\pgfpathlineto{\pgfqpoint{4.643628in}{1.423620in}}%
\pgfpathlineto{\pgfqpoint{4.646169in}{1.416475in}}%
\pgfpathlineto{\pgfqpoint{4.648922in}{1.418385in}}%
\pgfpathlineto{\pgfqpoint{4.651524in}{1.418305in}}%
\pgfpathlineto{\pgfqpoint{4.654203in}{1.420230in}}%
\pgfpathlineto{\pgfqpoint{4.656873in}{1.417947in}}%
\pgfpathlineto{\pgfqpoint{4.659590in}{1.421163in}}%
\pgfpathlineto{\pgfqpoint{4.662237in}{1.418953in}}%
\pgfpathlineto{\pgfqpoint{4.664923in}{1.418411in}}%
\pgfpathlineto{\pgfqpoint{4.667764in}{1.414268in}}%
\pgfpathlineto{\pgfqpoint{4.670261in}{1.419098in}}%
\pgfpathlineto{\pgfqpoint{4.673068in}{1.417326in}}%
\pgfpathlineto{\pgfqpoint{4.675619in}{1.413172in}}%
\pgfpathlineto{\pgfqpoint{4.678448in}{1.415601in}}%
\pgfpathlineto{\pgfqpoint{4.680988in}{1.415408in}}%
\pgfpathlineto{\pgfqpoint{4.683799in}{1.417930in}}%
\pgfpathlineto{\pgfqpoint{4.686337in}{1.412744in}}%
\pgfpathlineto{\pgfqpoint{4.689051in}{1.411222in}}%
\pgfpathlineto{\pgfqpoint{4.691694in}{1.408491in}}%
\pgfpathlineto{\pgfqpoint{4.694381in}{1.412714in}}%
\pgfpathlineto{\pgfqpoint{4.697170in}{1.412040in}}%
\pgfpathlineto{\pgfqpoint{4.699734in}{1.416500in}}%
\pgfpathlineto{\pgfqpoint{4.702517in}{1.414790in}}%
\pgfpathlineto{\pgfqpoint{4.705094in}{1.414055in}}%
\pgfpathlineto{\pgfqpoint{4.707824in}{1.416785in}}%
\pgfpathlineto{\pgfqpoint{4.710437in}{1.414731in}}%
\pgfpathlineto{\pgfqpoint{4.713275in}{1.413294in}}%
\pgfpathlineto{\pgfqpoint{4.715806in}{1.411739in}}%
\pgfpathlineto{\pgfqpoint{4.718486in}{1.415948in}}%
\pgfpathlineto{\pgfqpoint{4.721160in}{1.412123in}}%
\pgfpathlineto{\pgfqpoint{4.723873in}{1.421425in}}%
\pgfpathlineto{\pgfqpoint{4.726508in}{1.421425in}}%
\pgfpathlineto{\pgfqpoint{4.729233in}{1.421514in}}%
\pgfpathlineto{\pgfqpoint{4.731901in}{1.420737in}}%
\pgfpathlineto{\pgfqpoint{4.734552in}{1.420239in}}%
\pgfpathlineto{\pgfqpoint{4.737348in}{1.421321in}}%
\pgfpathlineto{\pgfqpoint{4.739912in}{1.420623in}}%
\pgfpathlineto{\pgfqpoint{4.742696in}{1.418129in}}%
\pgfpathlineto{\pgfqpoint{4.745256in}{1.420869in}}%
\pgfpathlineto{\pgfqpoint{4.748081in}{1.422267in}}%
\pgfpathlineto{\pgfqpoint{4.750627in}{1.422848in}}%
\pgfpathlineto{\pgfqpoint{4.753298in}{1.422060in}}%
\pgfpathlineto{\pgfqpoint{4.755983in}{1.422760in}}%
\pgfpathlineto{\pgfqpoint{4.758653in}{1.416443in}}%
\pgfpathlineto{\pgfqpoint{4.761337in}{1.414452in}}%
\pgfpathlineto{\pgfqpoint{4.764018in}{1.412197in}}%
\pgfpathlineto{\pgfqpoint{4.766783in}{1.418131in}}%
\pgfpathlineto{\pgfqpoint{4.769367in}{1.422794in}}%
\pgfpathlineto{\pgfqpoint{4.772198in}{1.422285in}}%
\pgfpathlineto{\pgfqpoint{4.774732in}{1.422245in}}%
\pgfpathlineto{\pgfqpoint{4.777535in}{1.424598in}}%
\pgfpathlineto{\pgfqpoint{4.780083in}{1.426313in}}%
\pgfpathlineto{\pgfqpoint{4.782872in}{1.420236in}}%
\pgfpathlineto{\pgfqpoint{4.785445in}{1.424286in}}%
\pgfpathlineto{\pgfqpoint{4.788116in}{1.420631in}}%
\pgfpathlineto{\pgfqpoint{4.790798in}{1.420046in}}%
\pgfpathlineto{\pgfqpoint{4.793512in}{1.425737in}}%
\pgfpathlineto{\pgfqpoint{4.796274in}{1.420171in}}%
\pgfpathlineto{\pgfqpoint{4.798830in}{1.423089in}}%
\pgfpathlineto{\pgfqpoint{4.801586in}{1.419953in}}%
\pgfpathlineto{\pgfqpoint{4.804193in}{1.414043in}}%
\pgfpathlineto{\pgfqpoint{4.807017in}{1.413551in}}%
\pgfpathlineto{\pgfqpoint{4.809538in}{1.414835in}}%
\pgfpathlineto{\pgfqpoint{4.812377in}{1.421738in}}%
\pgfpathlineto{\pgfqpoint{4.814907in}{1.414501in}}%
\pgfpathlineto{\pgfqpoint{4.817587in}{1.406050in}}%
\pgfpathlineto{\pgfqpoint{4.820265in}{1.405590in}}%
\pgfpathlineto{\pgfqpoint{4.822945in}{1.405514in}}%
\pgfpathlineto{\pgfqpoint{4.825619in}{1.411851in}}%
\pgfpathlineto{\pgfqpoint{4.828291in}{1.405514in}}%
\pgfpathlineto{\pgfqpoint{4.831045in}{1.415013in}}%
\pgfpathlineto{\pgfqpoint{4.833657in}{1.414787in}}%
\pgfpathlineto{\pgfqpoint{4.837992in}{1.413252in}}%
\pgfpathlineto{\pgfqpoint{4.839922in}{1.417482in}}%
\pgfpathlineto{\pgfqpoint{4.842380in}{1.417442in}}%
\pgfpathlineto{\pgfqpoint{4.844361in}{1.419502in}}%
\pgfpathlineto{\pgfqpoint{4.847127in}{1.421947in}}%
\pgfpathlineto{\pgfqpoint{4.849715in}{1.415750in}}%
\pgfpathlineto{\pgfqpoint{4.852404in}{1.410167in}}%
\pgfpathlineto{\pgfqpoint{4.855070in}{1.409971in}}%
\pgfpathlineto{\pgfqpoint{4.857807in}{1.418958in}}%
\pgfpathlineto{\pgfqpoint{4.860544in}{1.415496in}}%
\pgfpathlineto{\pgfqpoint{4.863116in}{1.419584in}}%
\pgfpathlineto{\pgfqpoint{4.865910in}{1.419601in}}%
\pgfpathlineto{\pgfqpoint{4.868474in}{1.417935in}}%
\pgfpathlineto{\pgfqpoint{4.871209in}{1.420201in}}%
\pgfpathlineto{\pgfqpoint{4.873832in}{1.419610in}}%
\pgfpathlineto{\pgfqpoint{4.876636in}{1.420358in}}%
\pgfpathlineto{\pgfqpoint{4.879180in}{1.417091in}}%
\pgfpathlineto{\pgfqpoint{4.881864in}{1.418003in}}%
\pgfpathlineto{\pgfqpoint{4.884540in}{1.419568in}}%
\pgfpathlineto{\pgfqpoint{4.887211in}{1.425712in}}%
\pgfpathlineto{\pgfqpoint{4.889902in}{1.421776in}}%
\pgfpathlineto{\pgfqpoint{4.892611in}{1.422849in}}%
\pgfpathlineto{\pgfqpoint{4.895399in}{1.421151in}}%
\pgfpathlineto{\pgfqpoint{4.897938in}{1.424895in}}%
\pgfpathlineto{\pgfqpoint{4.900712in}{1.422702in}}%
\pgfpathlineto{\pgfqpoint{4.903295in}{1.425719in}}%
\pgfpathlineto{\pgfqpoint{4.906096in}{1.444737in}}%
\pgfpathlineto{\pgfqpoint{4.908648in}{1.439769in}}%
\pgfpathlineto{\pgfqpoint{4.911435in}{1.444171in}}%
\pgfpathlineto{\pgfqpoint{4.914009in}{1.438760in}}%
\pgfpathlineto{\pgfqpoint{4.916681in}{1.440754in}}%
\pgfpathlineto{\pgfqpoint{4.919352in}{1.445825in}}%
\pgfpathlineto{\pgfqpoint{4.922041in}{1.434143in}}%
\pgfpathlineto{\pgfqpoint{4.924708in}{1.437509in}}%
\pgfpathlineto{\pgfqpoint{4.927400in}{1.432101in}}%
\pgfpathlineto{\pgfqpoint{4.930170in}{1.436364in}}%
\pgfpathlineto{\pgfqpoint{4.932742in}{1.435902in}}%
\pgfpathlineto{\pgfqpoint{4.935515in}{1.425212in}}%
\pgfpathlineto{\pgfqpoint{4.938112in}{1.419301in}}%
\pgfpathlineto{\pgfqpoint{4.940881in}{1.423401in}}%
\pgfpathlineto{\pgfqpoint{4.943466in}{1.425912in}}%
\pgfpathlineto{\pgfqpoint{4.946151in}{1.420153in}}%
\pgfpathlineto{\pgfqpoint{4.948827in}{1.415185in}}%
\pgfpathlineto{\pgfqpoint{4.951504in}{1.418340in}}%
\pgfpathlineto{\pgfqpoint{4.954182in}{1.427100in}}%
\pgfpathlineto{\pgfqpoint{4.956862in}{1.437193in}}%
\pgfpathlineto{\pgfqpoint{4.959689in}{1.433380in}}%
\pgfpathlineto{\pgfqpoint{4.962219in}{1.430993in}}%
\pgfpathlineto{\pgfqpoint{4.965002in}{1.416919in}}%
\pgfpathlineto{\pgfqpoint{4.967575in}{1.414631in}}%
\pgfpathlineto{\pgfqpoint{4.970314in}{1.411669in}}%
\pgfpathlineto{\pgfqpoint{4.972933in}{1.411513in}}%
\pgfpathlineto{\pgfqpoint{4.975703in}{1.413215in}}%
\pgfpathlineto{\pgfqpoint{4.978287in}{1.411401in}}%
\pgfpathlineto{\pgfqpoint{4.980967in}{1.411192in}}%
\pgfpathlineto{\pgfqpoint{4.983637in}{1.414250in}}%
\pgfpathlineto{\pgfqpoint{4.986325in}{1.414888in}}%
\pgfpathlineto{\pgfqpoint{4.989001in}{1.412138in}}%
\pgfpathlineto{\pgfqpoint{4.991683in}{1.413875in}}%
\pgfpathlineto{\pgfqpoint{4.994390in}{1.413101in}}%
\pgfpathlineto{\pgfqpoint{4.997028in}{1.415709in}}%
\pgfpathlineto{\pgfqpoint{4.999780in}{1.416226in}}%
\pgfpathlineto{\pgfqpoint{5.002384in}{1.412457in}}%
\pgfpathlineto{\pgfqpoint{5.005178in}{1.413404in}}%
\pgfpathlineto{\pgfqpoint{5.007751in}{1.417510in}}%
\pgfpathlineto{\pgfqpoint{5.010562in}{1.418874in}}%
\pgfpathlineto{\pgfqpoint{5.013104in}{1.418055in}}%
\pgfpathlineto{\pgfqpoint{5.015820in}{1.419755in}}%
\pgfpathlineto{\pgfqpoint{5.018466in}{1.420849in}}%
\pgfpathlineto{\pgfqpoint{5.021147in}{1.421384in}}%
\pgfpathlineto{\pgfqpoint{5.023927in}{1.418803in}}%
\pgfpathlineto{\pgfqpoint{5.026501in}{1.421799in}}%
\pgfpathlineto{\pgfqpoint{5.029275in}{1.417033in}}%
\pgfpathlineto{\pgfqpoint{5.031849in}{1.422705in}}%
\pgfpathlineto{\pgfqpoint{5.034649in}{1.425298in}}%
\pgfpathlineto{\pgfqpoint{5.037214in}{1.422917in}}%
\pgfpathlineto{\pgfqpoint{5.039962in}{1.419705in}}%
\pgfpathlineto{\pgfqpoint{5.042572in}{1.426372in}}%
\pgfpathlineto{\pgfqpoint{5.045249in}{1.420006in}}%
\pgfpathlineto{\pgfqpoint{5.047924in}{1.420811in}}%
\pgfpathlineto{\pgfqpoint{5.050606in}{1.418118in}}%
\pgfpathlineto{\pgfqpoint{5.053284in}{1.417124in}}%
\pgfpathlineto{\pgfqpoint{5.055952in}{1.421070in}}%
\pgfpathlineto{\pgfqpoint{5.058711in}{1.418298in}}%
\pgfpathlineto{\pgfqpoint{5.061315in}{1.421296in}}%
\pgfpathlineto{\pgfqpoint{5.064144in}{1.418633in}}%
\pgfpathlineto{\pgfqpoint{5.066677in}{1.421870in}}%
\pgfpathlineto{\pgfqpoint{5.069463in}{1.417554in}}%
\pgfpathlineto{\pgfqpoint{5.072030in}{1.414575in}}%
\pgfpathlineto{\pgfqpoint{5.074851in}{1.412561in}}%
\pgfpathlineto{\pgfqpoint{5.077390in}{1.419020in}}%
\pgfpathlineto{\pgfqpoint{5.080067in}{1.419874in}}%
\pgfpathlineto{\pgfqpoint{5.082746in}{1.419530in}}%
\pgfpathlineto{\pgfqpoint{5.085426in}{1.419259in}}%
\pgfpathlineto{\pgfqpoint{5.088103in}{1.416208in}}%
\pgfpathlineto{\pgfqpoint{5.090788in}{1.422081in}}%
\pgfpathlineto{\pgfqpoint{5.093579in}{1.423830in}}%
\pgfpathlineto{\pgfqpoint{5.096142in}{1.419791in}}%
\pgfpathlineto{\pgfqpoint{5.098948in}{1.422843in}}%
\pgfpathlineto{\pgfqpoint{5.101496in}{1.423385in}}%
\pgfpathlineto{\pgfqpoint{5.104312in}{1.420902in}}%
\pgfpathlineto{\pgfqpoint{5.106842in}{1.422270in}}%
\pgfpathlineto{\pgfqpoint{5.109530in}{1.417090in}}%
\pgfpathlineto{\pgfqpoint{5.112209in}{1.416873in}}%
\pgfpathlineto{\pgfqpoint{5.114887in}{1.421677in}}%
\pgfpathlineto{\pgfqpoint{5.117550in}{1.422326in}}%
\pgfpathlineto{\pgfqpoint{5.120243in}{1.422330in}}%
\pgfpathlineto{\pgfqpoint{5.123042in}{1.419492in}}%
\pgfpathlineto{\pgfqpoint{5.125599in}{1.419142in}}%
\pgfpathlineto{\pgfqpoint{5.128421in}{1.419887in}}%
\pgfpathlineto{\pgfqpoint{5.130953in}{1.417934in}}%
\pgfpathlineto{\pgfqpoint{5.133716in}{1.420583in}}%
\pgfpathlineto{\pgfqpoint{5.136311in}{1.418137in}}%
\pgfpathlineto{\pgfqpoint{5.139072in}{1.418393in}}%
\pgfpathlineto{\pgfqpoint{5.141660in}{1.424142in}}%
\pgfpathlineto{\pgfqpoint{5.144349in}{1.420890in}}%
\pgfpathlineto{\pgfqpoint{5.147029in}{1.421263in}}%
\pgfpathlineto{\pgfqpoint{5.149734in}{1.443996in}}%
\pgfpathlineto{\pgfqpoint{5.152382in}{1.488869in}}%
\pgfpathlineto{\pgfqpoint{5.155059in}{1.470102in}}%
\pgfpathlineto{\pgfqpoint{5.157815in}{1.456704in}}%
\pgfpathlineto{\pgfqpoint{5.160420in}{1.450709in}}%
\pgfpathlineto{\pgfqpoint{5.163243in}{1.439222in}}%
\pgfpathlineto{\pgfqpoint{5.165775in}{1.430343in}}%
\pgfpathlineto{\pgfqpoint{5.168591in}{1.427716in}}%
\pgfpathlineto{\pgfqpoint{5.171133in}{1.426097in}}%
\pgfpathlineto{\pgfqpoint{5.173925in}{1.430854in}}%
\pgfpathlineto{\pgfqpoint{5.176477in}{1.422016in}}%
\pgfpathlineto{\pgfqpoint{5.179188in}{1.419730in}}%
\pgfpathlineto{\pgfqpoint{5.181848in}{1.413963in}}%
\pgfpathlineto{\pgfqpoint{5.184522in}{1.411820in}}%
\pgfpathlineto{\pgfqpoint{5.187294in}{1.414877in}}%
\pgfpathlineto{\pgfqpoint{5.189880in}{1.419472in}}%
\pgfpathlineto{\pgfqpoint{5.192680in}{1.417772in}}%
\pgfpathlineto{\pgfqpoint{5.195239in}{1.421003in}}%
\pgfpathlineto{\pgfqpoint{5.198008in}{1.423973in}}%
\pgfpathlineto{\pgfqpoint{5.200594in}{1.427956in}}%
\pgfpathlineto{\pgfqpoint{5.203388in}{1.422811in}}%
\pgfpathlineto{\pgfqpoint{5.205952in}{1.426244in}}%
\pgfpathlineto{\pgfqpoint{5.208630in}{1.428032in}}%
\pgfpathlineto{\pgfqpoint{5.211299in}{1.425513in}}%
\pgfpathlineto{\pgfqpoint{5.214027in}{1.428133in}}%
\pgfpathlineto{\pgfqpoint{5.216667in}{1.421230in}}%
\pgfpathlineto{\pgfqpoint{5.219345in}{1.425778in}}%
\pgfpathlineto{\pgfqpoint{5.222151in}{1.425248in}}%
\pgfpathlineto{\pgfqpoint{5.224695in}{1.424470in}}%
\pgfpathlineto{\pgfqpoint{5.227470in}{1.423634in}}%
\pgfpathlineto{\pgfqpoint{5.230059in}{1.422088in}}%
\pgfpathlineto{\pgfqpoint{5.232855in}{1.421995in}}%
\pgfpathlineto{\pgfqpoint{5.235409in}{1.422328in}}%
\pgfpathlineto{\pgfqpoint{5.238173in}{1.423592in}}%
\pgfpathlineto{\pgfqpoint{5.240777in}{1.418618in}}%
\pgfpathlineto{\pgfqpoint{5.243445in}{1.422375in}}%
\pgfpathlineto{\pgfqpoint{5.246130in}{1.417623in}}%
\pgfpathlineto{\pgfqpoint{5.248816in}{1.422330in}}%
\pgfpathlineto{\pgfqpoint{5.251590in}{1.420006in}}%
\pgfpathlineto{\pgfqpoint{5.254236in}{1.418269in}}%
\pgfpathlineto{\pgfqpoint{5.256973in}{1.413859in}}%
\pgfpathlineto{\pgfqpoint{5.259511in}{1.416699in}}%
\pgfpathlineto{\pgfqpoint{5.262264in}{1.417373in}}%
\pgfpathlineto{\pgfqpoint{5.264876in}{1.409234in}}%
\pgfpathlineto{\pgfqpoint{5.267691in}{1.414383in}}%
\pgfpathlineto{\pgfqpoint{5.270238in}{1.423933in}}%
\pgfpathlineto{\pgfqpoint{5.272913in}{1.424436in}}%
\pgfpathlineto{\pgfqpoint{5.275589in}{1.420834in}}%
\pgfpathlineto{\pgfqpoint{5.278322in}{1.406785in}}%
\pgfpathlineto{\pgfqpoint{5.280947in}{1.405514in}}%
\pgfpathlineto{\pgfqpoint{5.283631in}{1.408075in}}%
\pgfpathlineto{\pgfqpoint{5.286436in}{1.413498in}}%
\pgfpathlineto{\pgfqpoint{5.288984in}{1.414850in}}%
\pgfpathlineto{\pgfqpoint{5.291794in}{1.417185in}}%
\pgfpathlineto{\pgfqpoint{5.294339in}{1.412908in}}%
\pgfpathlineto{\pgfqpoint{5.297140in}{1.416168in}}%
\pgfpathlineto{\pgfqpoint{5.299696in}{1.414786in}}%
\pgfpathlineto{\pgfqpoint{5.302443in}{1.420097in}}%
\pgfpathlineto{\pgfqpoint{5.305054in}{1.424171in}}%
\pgfpathlineto{\pgfqpoint{5.307731in}{1.421977in}}%
\pgfpathlineto{\pgfqpoint{5.310411in}{1.421397in}}%
\pgfpathlineto{\pgfqpoint{5.313089in}{1.426043in}}%
\pgfpathlineto{\pgfqpoint{5.315754in}{1.428134in}}%
\pgfpathlineto{\pgfqpoint{5.318430in}{1.431902in}}%
\pgfpathlineto{\pgfqpoint{5.321256in}{1.433570in}}%
\pgfpathlineto{\pgfqpoint{5.323802in}{1.434941in}}%
\pgfpathlineto{\pgfqpoint{5.326564in}{1.424143in}}%
\pgfpathlineto{\pgfqpoint{5.329159in}{1.418116in}}%
\pgfpathlineto{\pgfqpoint{5.331973in}{1.425152in}}%
\pgfpathlineto{\pgfqpoint{5.334510in}{1.421463in}}%
\pgfpathlineto{\pgfqpoint{5.337353in}{1.426000in}}%
\pgfpathlineto{\pgfqpoint{5.339872in}{1.423202in}}%
\pgfpathlineto{\pgfqpoint{5.342549in}{1.414179in}}%
\pgfpathlineto{\pgfqpoint{5.345224in}{1.416448in}}%
\pgfpathlineto{\pgfqpoint{5.347905in}{1.416363in}}%
\pgfpathlineto{\pgfqpoint{5.350723in}{1.420534in}}%
\pgfpathlineto{\pgfqpoint{5.353262in}{1.419994in}}%
\pgfpathlineto{\pgfqpoint{5.356056in}{1.422876in}}%
\pgfpathlineto{\pgfqpoint{5.358612in}{1.425075in}}%
\pgfpathlineto{\pgfqpoint{5.361370in}{1.416515in}}%
\pgfpathlineto{\pgfqpoint{5.363966in}{1.421130in}}%
\pgfpathlineto{\pgfqpoint{5.366727in}{1.416462in}}%
\pgfpathlineto{\pgfqpoint{5.369335in}{1.416239in}}%
\pgfpathlineto{\pgfqpoint{5.372013in}{1.418789in}}%
\pgfpathlineto{\pgfqpoint{5.374692in}{1.417438in}}%
\pgfpathlineto{\pgfqpoint{5.377370in}{1.420736in}}%
\pgfpathlineto{\pgfqpoint{5.380048in}{1.414387in}}%
\pgfpathlineto{\pgfqpoint{5.382725in}{1.416916in}}%
\pgfpathlineto{\pgfqpoint{5.385550in}{1.409434in}}%
\pgfpathlineto{\pgfqpoint{5.388083in}{1.416306in}}%
\pgfpathlineto{\pgfqpoint{5.390900in}{1.409031in}}%
\pgfpathlineto{\pgfqpoint{5.393441in}{1.407332in}}%
\pgfpathlineto{\pgfqpoint{5.396219in}{1.410783in}}%
\pgfpathlineto{\pgfqpoint{5.398784in}{1.420211in}}%
\pgfpathlineto{\pgfqpoint{5.401576in}{1.419032in}}%
\pgfpathlineto{\pgfqpoint{5.404154in}{1.412644in}}%
\pgfpathlineto{\pgfqpoint{5.406832in}{1.416726in}}%
\pgfpathlineto{\pgfqpoint{5.409507in}{1.414051in}}%
\pgfpathlineto{\pgfqpoint{5.412190in}{1.417006in}}%
\pgfpathlineto{\pgfqpoint{5.414954in}{1.416801in}}%
\pgfpathlineto{\pgfqpoint{5.417547in}{1.415417in}}%
\pgfpathlineto{\pgfqpoint{5.420304in}{1.416105in}}%
\pgfpathlineto{\pgfqpoint{5.422897in}{1.417333in}}%
\pgfpathlineto{\pgfqpoint{5.425661in}{1.417353in}}%
\pgfpathlineto{\pgfqpoint{5.428259in}{1.420228in}}%
\pgfpathlineto{\pgfqpoint{5.431015in}{1.418476in}}%
\pgfpathlineto{\pgfqpoint{5.433616in}{1.419908in}}%
\pgfpathlineto{\pgfqpoint{5.436295in}{1.419556in}}%
\pgfpathlineto{\pgfqpoint{5.438974in}{1.420077in}}%
\pgfpathlineto{\pgfqpoint{5.441698in}{1.419047in}}%
\pgfpathlineto{\pgfqpoint{5.444328in}{1.426915in}}%
\pgfpathlineto{\pgfqpoint{5.447021in}{1.418816in}}%
\pgfpathlineto{\pgfqpoint{5.449769in}{1.432829in}}%
\pgfpathlineto{\pgfqpoint{5.452365in}{1.425338in}}%
\pgfpathlineto{\pgfqpoint{5.455168in}{1.423658in}}%
\pgfpathlineto{\pgfqpoint{5.457721in}{1.423502in}}%
\pgfpathlineto{\pgfqpoint{5.460489in}{1.426497in}}%
\pgfpathlineto{\pgfqpoint{5.463079in}{1.423008in}}%
\pgfpathlineto{\pgfqpoint{5.465888in}{1.418432in}}%
\pgfpathlineto{\pgfqpoint{5.468425in}{1.415969in}}%
\pgfpathlineto{\pgfqpoint{5.471113in}{1.420040in}}%
\pgfpathlineto{\pgfqpoint{5.473792in}{1.419064in}}%
\pgfpathlineto{\pgfqpoint{5.476458in}{1.418763in}}%
\pgfpathlineto{\pgfqpoint{5.479152in}{1.414459in}}%
\pgfpathlineto{\pgfqpoint{5.481825in}{1.419307in}}%
\pgfpathlineto{\pgfqpoint{5.484641in}{1.415043in}}%
\pgfpathlineto{\pgfqpoint{5.487176in}{1.416705in}}%
\pgfpathlineto{\pgfqpoint{5.490000in}{1.419790in}}%
\pgfpathlineto{\pgfqpoint{5.492541in}{1.420000in}}%
\pgfpathlineto{\pgfqpoint{5.495346in}{1.420878in}}%
\pgfpathlineto{\pgfqpoint{5.497898in}{1.424100in}}%
\pgfpathlineto{\pgfqpoint{5.500687in}{1.416821in}}%
\pgfpathlineto{\pgfqpoint{5.503255in}{1.420088in}}%
\pgfpathlineto{\pgfqpoint{5.505933in}{1.425101in}}%
\pgfpathlineto{\pgfqpoint{5.508612in}{1.424326in}}%
\pgfpathlineto{\pgfqpoint{5.511290in}{1.418529in}}%
\pgfpathlineto{\pgfqpoint{5.514080in}{1.426335in}}%
\pgfpathlineto{\pgfqpoint{5.516646in}{1.431069in}}%
\pgfpathlineto{\pgfqpoint{5.519433in}{1.430803in}}%
\pgfpathlineto{\pgfqpoint{5.522003in}{1.423044in}}%
\pgfpathlineto{\pgfqpoint{5.524756in}{1.421717in}}%
\pgfpathlineto{\pgfqpoint{5.527360in}{1.422344in}}%
\pgfpathlineto{\pgfqpoint{5.530148in}{1.423845in}}%
\pgfpathlineto{\pgfqpoint{5.532717in}{1.418282in}}%
\pgfpathlineto{\pgfqpoint{5.535395in}{1.422022in}}%
\pgfpathlineto{\pgfqpoint{5.538074in}{1.418629in}}%
\pgfpathlineto{\pgfqpoint{5.540750in}{1.421091in}}%
\pgfpathlineto{\pgfqpoint{5.543421in}{1.415411in}}%
\pgfpathlineto{\pgfqpoint{5.546110in}{1.422598in}}%
\pgfpathlineto{\pgfqpoint{5.548921in}{1.421258in}}%
\pgfpathlineto{\pgfqpoint{5.551457in}{1.417623in}}%
\pgfpathlineto{\pgfqpoint{5.554198in}{1.427331in}}%
\pgfpathlineto{\pgfqpoint{5.556822in}{1.425993in}}%
\pgfpathlineto{\pgfqpoint{5.559612in}{1.430385in}}%
\pgfpathlineto{\pgfqpoint{5.562180in}{1.429932in}}%
\pgfpathlineto{\pgfqpoint{5.564940in}{1.430700in}}%
\pgfpathlineto{\pgfqpoint{5.567536in}{1.426495in}}%
\pgfpathlineto{\pgfqpoint{5.570215in}{1.424922in}}%
\pgfpathlineto{\pgfqpoint{5.572893in}{1.423624in}}%
\pgfpathlineto{\pgfqpoint{5.575596in}{1.422429in}}%
\pgfpathlineto{\pgfqpoint{5.578342in}{1.423442in}}%
\pgfpathlineto{\pgfqpoint{5.580914in}{1.422682in}}%
\pgfpathlineto{\pgfqpoint{5.583709in}{1.424698in}}%
\pgfpathlineto{\pgfqpoint{5.586269in}{1.422430in}}%
\pgfpathlineto{\pgfqpoint{5.589040in}{1.421222in}}%
\pgfpathlineto{\pgfqpoint{5.591641in}{1.418708in}}%
\pgfpathlineto{\pgfqpoint{5.594368in}{1.420362in}}%
\pgfpathlineto{\pgfqpoint{5.596999in}{1.420500in}}%
\pgfpathlineto{\pgfqpoint{5.599674in}{1.418735in}}%
\pgfpathlineto{\pgfqpoint{5.602352in}{1.422659in}}%
\pgfpathlineto{\pgfqpoint{5.605073in}{1.422487in}}%
\pgfpathlineto{\pgfqpoint{5.607698in}{1.425603in}}%
\pgfpathlineto{\pgfqpoint{5.610389in}{1.430978in}}%
\pgfpathlineto{\pgfqpoint{5.613235in}{1.425315in}}%
\pgfpathlineto{\pgfqpoint{5.615743in}{1.420826in}}%
\pgfpathlineto{\pgfqpoint{5.618526in}{1.419604in}}%
\pgfpathlineto{\pgfqpoint{5.621102in}{1.417763in}}%
\pgfpathlineto{\pgfqpoint{5.623868in}{1.419477in}}%
\pgfpathlineto{\pgfqpoint{5.626460in}{1.417757in}}%
\pgfpathlineto{\pgfqpoint{5.629232in}{1.417965in}}%
\pgfpathlineto{\pgfqpoint{5.631815in}{1.421013in}}%
\pgfpathlineto{\pgfqpoint{5.634496in}{1.427271in}}%
\pgfpathlineto{\pgfqpoint{5.637172in}{1.430241in}}%
\pgfpathlineto{\pgfqpoint{5.639852in}{1.439097in}}%
\pgfpathlineto{\pgfqpoint{5.642518in}{1.466693in}}%
\pgfpathlineto{\pgfqpoint{5.645243in}{1.474791in}}%
\pgfpathlineto{\pgfqpoint{5.648008in}{1.534185in}}%
\pgfpathlineto{\pgfqpoint{5.650563in}{1.539056in}}%
\pgfpathlineto{\pgfqpoint{5.653376in}{1.488497in}}%
\pgfpathlineto{\pgfqpoint{5.655919in}{1.448589in}}%
\pgfpathlineto{\pgfqpoint{5.658723in}{1.434242in}}%
\pgfpathlineto{\pgfqpoint{5.661273in}{1.425703in}}%
\pgfpathlineto{\pgfqpoint{5.664099in}{1.424994in}}%
\pgfpathlineto{\pgfqpoint{5.666632in}{1.438647in}}%
\pgfpathlineto{\pgfqpoint{5.669313in}{1.447857in}}%
\pgfpathlineto{\pgfqpoint{5.671991in}{1.444369in}}%
\pgfpathlineto{\pgfqpoint{5.674667in}{1.434760in}}%
\pgfpathlineto{\pgfqpoint{5.677486in}{1.427879in}}%
\pgfpathlineto{\pgfqpoint{5.680027in}{1.423195in}}%
\pgfpathlineto{\pgfqpoint{5.682836in}{1.421840in}}%
\pgfpathlineto{\pgfqpoint{5.685385in}{1.422757in}}%
\pgfpathlineto{\pgfqpoint{5.688159in}{1.420694in}}%
\pgfpathlineto{\pgfqpoint{5.690730in}{1.424116in}}%
\pgfpathlineto{\pgfqpoint{5.693473in}{1.434776in}}%
\pgfpathlineto{\pgfqpoint{5.696101in}{1.435203in}}%
\pgfpathlineto{\pgfqpoint{5.698775in}{1.428847in}}%
\pgfpathlineto{\pgfqpoint{5.701453in}{1.425405in}}%
\pgfpathlineto{\pgfqpoint{5.704130in}{1.451370in}}%
\pgfpathlineto{\pgfqpoint{5.706800in}{1.451497in}}%
\pgfpathlineto{\pgfqpoint{5.709490in}{1.459768in}}%
\pgfpathlineto{\pgfqpoint{5.712291in}{1.450183in}}%
\pgfpathlineto{\pgfqpoint{5.714834in}{1.446991in}}%
\pgfpathlineto{\pgfqpoint{5.717671in}{1.437174in}}%
\pgfpathlineto{\pgfqpoint{5.720201in}{1.433492in}}%
\pgfpathlineto{\pgfqpoint{5.722950in}{1.427168in}}%
\pgfpathlineto{\pgfqpoint{5.725548in}{1.419157in}}%
\pgfpathlineto{\pgfqpoint{5.728339in}{1.419345in}}%
\pgfpathlineto{\pgfqpoint{5.730919in}{1.415517in}}%
\pgfpathlineto{\pgfqpoint{5.733594in}{1.419279in}}%
\pgfpathlineto{\pgfqpoint{5.736276in}{1.415854in}}%
\pgfpathlineto{\pgfqpoint{5.738974in}{1.419919in}}%
\pgfpathlineto{\pgfqpoint{5.741745in}{1.420681in}}%
\pgfpathlineto{\pgfqpoint{5.744310in}{1.424571in}}%
\pgfpathlineto{\pgfqpoint{5.744310in}{0.413320in}}%
\pgfpathlineto{\pgfqpoint{5.744310in}{0.413320in}}%
\pgfpathlineto{\pgfqpoint{5.741745in}{0.413320in}}%
\pgfpathlineto{\pgfqpoint{5.738974in}{0.413320in}}%
\pgfpathlineto{\pgfqpoint{5.736276in}{0.413320in}}%
\pgfpathlineto{\pgfqpoint{5.733594in}{0.413320in}}%
\pgfpathlineto{\pgfqpoint{5.730919in}{0.413320in}}%
\pgfpathlineto{\pgfqpoint{5.728339in}{0.413320in}}%
\pgfpathlineto{\pgfqpoint{5.725548in}{0.413320in}}%
\pgfpathlineto{\pgfqpoint{5.722950in}{0.413320in}}%
\pgfpathlineto{\pgfqpoint{5.720201in}{0.413320in}}%
\pgfpathlineto{\pgfqpoint{5.717671in}{0.413320in}}%
\pgfpathlineto{\pgfqpoint{5.714834in}{0.413320in}}%
\pgfpathlineto{\pgfqpoint{5.712291in}{0.413320in}}%
\pgfpathlineto{\pgfqpoint{5.709490in}{0.413320in}}%
\pgfpathlineto{\pgfqpoint{5.706800in}{0.413320in}}%
\pgfpathlineto{\pgfqpoint{5.704130in}{0.413320in}}%
\pgfpathlineto{\pgfqpoint{5.701453in}{0.413320in}}%
\pgfpathlineto{\pgfqpoint{5.698775in}{0.413320in}}%
\pgfpathlineto{\pgfqpoint{5.696101in}{0.413320in}}%
\pgfpathlineto{\pgfqpoint{5.693473in}{0.413320in}}%
\pgfpathlineto{\pgfqpoint{5.690730in}{0.413320in}}%
\pgfpathlineto{\pgfqpoint{5.688159in}{0.413320in}}%
\pgfpathlineto{\pgfqpoint{5.685385in}{0.413320in}}%
\pgfpathlineto{\pgfqpoint{5.682836in}{0.413320in}}%
\pgfpathlineto{\pgfqpoint{5.680027in}{0.413320in}}%
\pgfpathlineto{\pgfqpoint{5.677486in}{0.413320in}}%
\pgfpathlineto{\pgfqpoint{5.674667in}{0.413320in}}%
\pgfpathlineto{\pgfqpoint{5.671991in}{0.413320in}}%
\pgfpathlineto{\pgfqpoint{5.669313in}{0.413320in}}%
\pgfpathlineto{\pgfqpoint{5.666632in}{0.413320in}}%
\pgfpathlineto{\pgfqpoint{5.664099in}{0.413320in}}%
\pgfpathlineto{\pgfqpoint{5.661273in}{0.413320in}}%
\pgfpathlineto{\pgfqpoint{5.658723in}{0.413320in}}%
\pgfpathlineto{\pgfqpoint{5.655919in}{0.413320in}}%
\pgfpathlineto{\pgfqpoint{5.653376in}{0.413320in}}%
\pgfpathlineto{\pgfqpoint{5.650563in}{0.413320in}}%
\pgfpathlineto{\pgfqpoint{5.648008in}{0.413320in}}%
\pgfpathlineto{\pgfqpoint{5.645243in}{0.413320in}}%
\pgfpathlineto{\pgfqpoint{5.642518in}{0.413320in}}%
\pgfpathlineto{\pgfqpoint{5.639852in}{0.413320in}}%
\pgfpathlineto{\pgfqpoint{5.637172in}{0.413320in}}%
\pgfpathlineto{\pgfqpoint{5.634496in}{0.413320in}}%
\pgfpathlineto{\pgfqpoint{5.631815in}{0.413320in}}%
\pgfpathlineto{\pgfqpoint{5.629232in}{0.413320in}}%
\pgfpathlineto{\pgfqpoint{5.626460in}{0.413320in}}%
\pgfpathlineto{\pgfqpoint{5.623868in}{0.413320in}}%
\pgfpathlineto{\pgfqpoint{5.621102in}{0.413320in}}%
\pgfpathlineto{\pgfqpoint{5.618526in}{0.413320in}}%
\pgfpathlineto{\pgfqpoint{5.615743in}{0.413320in}}%
\pgfpathlineto{\pgfqpoint{5.613235in}{0.413320in}}%
\pgfpathlineto{\pgfqpoint{5.610389in}{0.413320in}}%
\pgfpathlineto{\pgfqpoint{5.607698in}{0.413320in}}%
\pgfpathlineto{\pgfqpoint{5.605073in}{0.413320in}}%
\pgfpathlineto{\pgfqpoint{5.602352in}{0.413320in}}%
\pgfpathlineto{\pgfqpoint{5.599674in}{0.413320in}}%
\pgfpathlineto{\pgfqpoint{5.596999in}{0.413320in}}%
\pgfpathlineto{\pgfqpoint{5.594368in}{0.413320in}}%
\pgfpathlineto{\pgfqpoint{5.591641in}{0.413320in}}%
\pgfpathlineto{\pgfqpoint{5.589040in}{0.413320in}}%
\pgfpathlineto{\pgfqpoint{5.586269in}{0.413320in}}%
\pgfpathlineto{\pgfqpoint{5.583709in}{0.413320in}}%
\pgfpathlineto{\pgfqpoint{5.580914in}{0.413320in}}%
\pgfpathlineto{\pgfqpoint{5.578342in}{0.413320in}}%
\pgfpathlineto{\pgfqpoint{5.575596in}{0.413320in}}%
\pgfpathlineto{\pgfqpoint{5.572893in}{0.413320in}}%
\pgfpathlineto{\pgfqpoint{5.570215in}{0.413320in}}%
\pgfpathlineto{\pgfqpoint{5.567536in}{0.413320in}}%
\pgfpathlineto{\pgfqpoint{5.564940in}{0.413320in}}%
\pgfpathlineto{\pgfqpoint{5.562180in}{0.413320in}}%
\pgfpathlineto{\pgfqpoint{5.559612in}{0.413320in}}%
\pgfpathlineto{\pgfqpoint{5.556822in}{0.413320in}}%
\pgfpathlineto{\pgfqpoint{5.554198in}{0.413320in}}%
\pgfpathlineto{\pgfqpoint{5.551457in}{0.413320in}}%
\pgfpathlineto{\pgfqpoint{5.548921in}{0.413320in}}%
\pgfpathlineto{\pgfqpoint{5.546110in}{0.413320in}}%
\pgfpathlineto{\pgfqpoint{5.543421in}{0.413320in}}%
\pgfpathlineto{\pgfqpoint{5.540750in}{0.413320in}}%
\pgfpathlineto{\pgfqpoint{5.538074in}{0.413320in}}%
\pgfpathlineto{\pgfqpoint{5.535395in}{0.413320in}}%
\pgfpathlineto{\pgfqpoint{5.532717in}{0.413320in}}%
\pgfpathlineto{\pgfqpoint{5.530148in}{0.413320in}}%
\pgfpathlineto{\pgfqpoint{5.527360in}{0.413320in}}%
\pgfpathlineto{\pgfqpoint{5.524756in}{0.413320in}}%
\pgfpathlineto{\pgfqpoint{5.522003in}{0.413320in}}%
\pgfpathlineto{\pgfqpoint{5.519433in}{0.413320in}}%
\pgfpathlineto{\pgfqpoint{5.516646in}{0.413320in}}%
\pgfpathlineto{\pgfqpoint{5.514080in}{0.413320in}}%
\pgfpathlineto{\pgfqpoint{5.511290in}{0.413320in}}%
\pgfpathlineto{\pgfqpoint{5.508612in}{0.413320in}}%
\pgfpathlineto{\pgfqpoint{5.505933in}{0.413320in}}%
\pgfpathlineto{\pgfqpoint{5.503255in}{0.413320in}}%
\pgfpathlineto{\pgfqpoint{5.500687in}{0.413320in}}%
\pgfpathlineto{\pgfqpoint{5.497898in}{0.413320in}}%
\pgfpathlineto{\pgfqpoint{5.495346in}{0.413320in}}%
\pgfpathlineto{\pgfqpoint{5.492541in}{0.413320in}}%
\pgfpathlineto{\pgfqpoint{5.490000in}{0.413320in}}%
\pgfpathlineto{\pgfqpoint{5.487176in}{0.413320in}}%
\pgfpathlineto{\pgfqpoint{5.484641in}{0.413320in}}%
\pgfpathlineto{\pgfqpoint{5.481825in}{0.413320in}}%
\pgfpathlineto{\pgfqpoint{5.479152in}{0.413320in}}%
\pgfpathlineto{\pgfqpoint{5.476458in}{0.413320in}}%
\pgfpathlineto{\pgfqpoint{5.473792in}{0.413320in}}%
\pgfpathlineto{\pgfqpoint{5.471113in}{0.413320in}}%
\pgfpathlineto{\pgfqpoint{5.468425in}{0.413320in}}%
\pgfpathlineto{\pgfqpoint{5.465888in}{0.413320in}}%
\pgfpathlineto{\pgfqpoint{5.463079in}{0.413320in}}%
\pgfpathlineto{\pgfqpoint{5.460489in}{0.413320in}}%
\pgfpathlineto{\pgfqpoint{5.457721in}{0.413320in}}%
\pgfpathlineto{\pgfqpoint{5.455168in}{0.413320in}}%
\pgfpathlineto{\pgfqpoint{5.452365in}{0.413320in}}%
\pgfpathlineto{\pgfqpoint{5.449769in}{0.413320in}}%
\pgfpathlineto{\pgfqpoint{5.447021in}{0.413320in}}%
\pgfpathlineto{\pgfqpoint{5.444328in}{0.413320in}}%
\pgfpathlineto{\pgfqpoint{5.441698in}{0.413320in}}%
\pgfpathlineto{\pgfqpoint{5.438974in}{0.413320in}}%
\pgfpathlineto{\pgfqpoint{5.436295in}{0.413320in}}%
\pgfpathlineto{\pgfqpoint{5.433616in}{0.413320in}}%
\pgfpathlineto{\pgfqpoint{5.431015in}{0.413320in}}%
\pgfpathlineto{\pgfqpoint{5.428259in}{0.413320in}}%
\pgfpathlineto{\pgfqpoint{5.425661in}{0.413320in}}%
\pgfpathlineto{\pgfqpoint{5.422897in}{0.413320in}}%
\pgfpathlineto{\pgfqpoint{5.420304in}{0.413320in}}%
\pgfpathlineto{\pgfqpoint{5.417547in}{0.413320in}}%
\pgfpathlineto{\pgfqpoint{5.414954in}{0.413320in}}%
\pgfpathlineto{\pgfqpoint{5.412190in}{0.413320in}}%
\pgfpathlineto{\pgfqpoint{5.409507in}{0.413320in}}%
\pgfpathlineto{\pgfqpoint{5.406832in}{0.413320in}}%
\pgfpathlineto{\pgfqpoint{5.404154in}{0.413320in}}%
\pgfpathlineto{\pgfqpoint{5.401576in}{0.413320in}}%
\pgfpathlineto{\pgfqpoint{5.398784in}{0.413320in}}%
\pgfpathlineto{\pgfqpoint{5.396219in}{0.413320in}}%
\pgfpathlineto{\pgfqpoint{5.393441in}{0.413320in}}%
\pgfpathlineto{\pgfqpoint{5.390900in}{0.413320in}}%
\pgfpathlineto{\pgfqpoint{5.388083in}{0.413320in}}%
\pgfpathlineto{\pgfqpoint{5.385550in}{0.413320in}}%
\pgfpathlineto{\pgfqpoint{5.382725in}{0.413320in}}%
\pgfpathlineto{\pgfqpoint{5.380048in}{0.413320in}}%
\pgfpathlineto{\pgfqpoint{5.377370in}{0.413320in}}%
\pgfpathlineto{\pgfqpoint{5.374692in}{0.413320in}}%
\pgfpathlineto{\pgfqpoint{5.372013in}{0.413320in}}%
\pgfpathlineto{\pgfqpoint{5.369335in}{0.413320in}}%
\pgfpathlineto{\pgfqpoint{5.366727in}{0.413320in}}%
\pgfpathlineto{\pgfqpoint{5.363966in}{0.413320in}}%
\pgfpathlineto{\pgfqpoint{5.361370in}{0.413320in}}%
\pgfpathlineto{\pgfqpoint{5.358612in}{0.413320in}}%
\pgfpathlineto{\pgfqpoint{5.356056in}{0.413320in}}%
\pgfpathlineto{\pgfqpoint{5.353262in}{0.413320in}}%
\pgfpathlineto{\pgfqpoint{5.350723in}{0.413320in}}%
\pgfpathlineto{\pgfqpoint{5.347905in}{0.413320in}}%
\pgfpathlineto{\pgfqpoint{5.345224in}{0.413320in}}%
\pgfpathlineto{\pgfqpoint{5.342549in}{0.413320in}}%
\pgfpathlineto{\pgfqpoint{5.339872in}{0.413320in}}%
\pgfpathlineto{\pgfqpoint{5.337353in}{0.413320in}}%
\pgfpathlineto{\pgfqpoint{5.334510in}{0.413320in}}%
\pgfpathlineto{\pgfqpoint{5.331973in}{0.413320in}}%
\pgfpathlineto{\pgfqpoint{5.329159in}{0.413320in}}%
\pgfpathlineto{\pgfqpoint{5.326564in}{0.413320in}}%
\pgfpathlineto{\pgfqpoint{5.323802in}{0.413320in}}%
\pgfpathlineto{\pgfqpoint{5.321256in}{0.413320in}}%
\pgfpathlineto{\pgfqpoint{5.318430in}{0.413320in}}%
\pgfpathlineto{\pgfqpoint{5.315754in}{0.413320in}}%
\pgfpathlineto{\pgfqpoint{5.313089in}{0.413320in}}%
\pgfpathlineto{\pgfqpoint{5.310411in}{0.413320in}}%
\pgfpathlineto{\pgfqpoint{5.307731in}{0.413320in}}%
\pgfpathlineto{\pgfqpoint{5.305054in}{0.413320in}}%
\pgfpathlineto{\pgfqpoint{5.302443in}{0.413320in}}%
\pgfpathlineto{\pgfqpoint{5.299696in}{0.413320in}}%
\pgfpathlineto{\pgfqpoint{5.297140in}{0.413320in}}%
\pgfpathlineto{\pgfqpoint{5.294339in}{0.413320in}}%
\pgfpathlineto{\pgfqpoint{5.291794in}{0.413320in}}%
\pgfpathlineto{\pgfqpoint{5.288984in}{0.413320in}}%
\pgfpathlineto{\pgfqpoint{5.286436in}{0.413320in}}%
\pgfpathlineto{\pgfqpoint{5.283631in}{0.413320in}}%
\pgfpathlineto{\pgfqpoint{5.280947in}{0.413320in}}%
\pgfpathlineto{\pgfqpoint{5.278322in}{0.413320in}}%
\pgfpathlineto{\pgfqpoint{5.275589in}{0.413320in}}%
\pgfpathlineto{\pgfqpoint{5.272913in}{0.413320in}}%
\pgfpathlineto{\pgfqpoint{5.270238in}{0.413320in}}%
\pgfpathlineto{\pgfqpoint{5.267691in}{0.413320in}}%
\pgfpathlineto{\pgfqpoint{5.264876in}{0.413320in}}%
\pgfpathlineto{\pgfqpoint{5.262264in}{0.413320in}}%
\pgfpathlineto{\pgfqpoint{5.259511in}{0.413320in}}%
\pgfpathlineto{\pgfqpoint{5.256973in}{0.413320in}}%
\pgfpathlineto{\pgfqpoint{5.254236in}{0.413320in}}%
\pgfpathlineto{\pgfqpoint{5.251590in}{0.413320in}}%
\pgfpathlineto{\pgfqpoint{5.248816in}{0.413320in}}%
\pgfpathlineto{\pgfqpoint{5.246130in}{0.413320in}}%
\pgfpathlineto{\pgfqpoint{5.243445in}{0.413320in}}%
\pgfpathlineto{\pgfqpoint{5.240777in}{0.413320in}}%
\pgfpathlineto{\pgfqpoint{5.238173in}{0.413320in}}%
\pgfpathlineto{\pgfqpoint{5.235409in}{0.413320in}}%
\pgfpathlineto{\pgfqpoint{5.232855in}{0.413320in}}%
\pgfpathlineto{\pgfqpoint{5.230059in}{0.413320in}}%
\pgfpathlineto{\pgfqpoint{5.227470in}{0.413320in}}%
\pgfpathlineto{\pgfqpoint{5.224695in}{0.413320in}}%
\pgfpathlineto{\pgfqpoint{5.222151in}{0.413320in}}%
\pgfpathlineto{\pgfqpoint{5.219345in}{0.413320in}}%
\pgfpathlineto{\pgfqpoint{5.216667in}{0.413320in}}%
\pgfpathlineto{\pgfqpoint{5.214027in}{0.413320in}}%
\pgfpathlineto{\pgfqpoint{5.211299in}{0.413320in}}%
\pgfpathlineto{\pgfqpoint{5.208630in}{0.413320in}}%
\pgfpathlineto{\pgfqpoint{5.205952in}{0.413320in}}%
\pgfpathlineto{\pgfqpoint{5.203388in}{0.413320in}}%
\pgfpathlineto{\pgfqpoint{5.200594in}{0.413320in}}%
\pgfpathlineto{\pgfqpoint{5.198008in}{0.413320in}}%
\pgfpathlineto{\pgfqpoint{5.195239in}{0.413320in}}%
\pgfpathlineto{\pgfqpoint{5.192680in}{0.413320in}}%
\pgfpathlineto{\pgfqpoint{5.189880in}{0.413320in}}%
\pgfpathlineto{\pgfqpoint{5.187294in}{0.413320in}}%
\pgfpathlineto{\pgfqpoint{5.184522in}{0.413320in}}%
\pgfpathlineto{\pgfqpoint{5.181848in}{0.413320in}}%
\pgfpathlineto{\pgfqpoint{5.179188in}{0.413320in}}%
\pgfpathlineto{\pgfqpoint{5.176477in}{0.413320in}}%
\pgfpathlineto{\pgfqpoint{5.173925in}{0.413320in}}%
\pgfpathlineto{\pgfqpoint{5.171133in}{0.413320in}}%
\pgfpathlineto{\pgfqpoint{5.168591in}{0.413320in}}%
\pgfpathlineto{\pgfqpoint{5.165775in}{0.413320in}}%
\pgfpathlineto{\pgfqpoint{5.163243in}{0.413320in}}%
\pgfpathlineto{\pgfqpoint{5.160420in}{0.413320in}}%
\pgfpathlineto{\pgfqpoint{5.157815in}{0.413320in}}%
\pgfpathlineto{\pgfqpoint{5.155059in}{0.413320in}}%
\pgfpathlineto{\pgfqpoint{5.152382in}{0.413320in}}%
\pgfpathlineto{\pgfqpoint{5.149734in}{0.413320in}}%
\pgfpathlineto{\pgfqpoint{5.147029in}{0.413320in}}%
\pgfpathlineto{\pgfqpoint{5.144349in}{0.413320in}}%
\pgfpathlineto{\pgfqpoint{5.141660in}{0.413320in}}%
\pgfpathlineto{\pgfqpoint{5.139072in}{0.413320in}}%
\pgfpathlineto{\pgfqpoint{5.136311in}{0.413320in}}%
\pgfpathlineto{\pgfqpoint{5.133716in}{0.413320in}}%
\pgfpathlineto{\pgfqpoint{5.130953in}{0.413320in}}%
\pgfpathlineto{\pgfqpoint{5.128421in}{0.413320in}}%
\pgfpathlineto{\pgfqpoint{5.125599in}{0.413320in}}%
\pgfpathlineto{\pgfqpoint{5.123042in}{0.413320in}}%
\pgfpathlineto{\pgfqpoint{5.120243in}{0.413320in}}%
\pgfpathlineto{\pgfqpoint{5.117550in}{0.413320in}}%
\pgfpathlineto{\pgfqpoint{5.114887in}{0.413320in}}%
\pgfpathlineto{\pgfqpoint{5.112209in}{0.413320in}}%
\pgfpathlineto{\pgfqpoint{5.109530in}{0.413320in}}%
\pgfpathlineto{\pgfqpoint{5.106842in}{0.413320in}}%
\pgfpathlineto{\pgfqpoint{5.104312in}{0.413320in}}%
\pgfpathlineto{\pgfqpoint{5.101496in}{0.413320in}}%
\pgfpathlineto{\pgfqpoint{5.098948in}{0.413320in}}%
\pgfpathlineto{\pgfqpoint{5.096142in}{0.413320in}}%
\pgfpathlineto{\pgfqpoint{5.093579in}{0.413320in}}%
\pgfpathlineto{\pgfqpoint{5.090788in}{0.413320in}}%
\pgfpathlineto{\pgfqpoint{5.088103in}{0.413320in}}%
\pgfpathlineto{\pgfqpoint{5.085426in}{0.413320in}}%
\pgfpathlineto{\pgfqpoint{5.082746in}{0.413320in}}%
\pgfpathlineto{\pgfqpoint{5.080067in}{0.413320in}}%
\pgfpathlineto{\pgfqpoint{5.077390in}{0.413320in}}%
\pgfpathlineto{\pgfqpoint{5.074851in}{0.413320in}}%
\pgfpathlineto{\pgfqpoint{5.072030in}{0.413320in}}%
\pgfpathlineto{\pgfqpoint{5.069463in}{0.413320in}}%
\pgfpathlineto{\pgfqpoint{5.066677in}{0.413320in}}%
\pgfpathlineto{\pgfqpoint{5.064144in}{0.413320in}}%
\pgfpathlineto{\pgfqpoint{5.061315in}{0.413320in}}%
\pgfpathlineto{\pgfqpoint{5.058711in}{0.413320in}}%
\pgfpathlineto{\pgfqpoint{5.055952in}{0.413320in}}%
\pgfpathlineto{\pgfqpoint{5.053284in}{0.413320in}}%
\pgfpathlineto{\pgfqpoint{5.050606in}{0.413320in}}%
\pgfpathlineto{\pgfqpoint{5.047924in}{0.413320in}}%
\pgfpathlineto{\pgfqpoint{5.045249in}{0.413320in}}%
\pgfpathlineto{\pgfqpoint{5.042572in}{0.413320in}}%
\pgfpathlineto{\pgfqpoint{5.039962in}{0.413320in}}%
\pgfpathlineto{\pgfqpoint{5.037214in}{0.413320in}}%
\pgfpathlineto{\pgfqpoint{5.034649in}{0.413320in}}%
\pgfpathlineto{\pgfqpoint{5.031849in}{0.413320in}}%
\pgfpathlineto{\pgfqpoint{5.029275in}{0.413320in}}%
\pgfpathlineto{\pgfqpoint{5.026501in}{0.413320in}}%
\pgfpathlineto{\pgfqpoint{5.023927in}{0.413320in}}%
\pgfpathlineto{\pgfqpoint{5.021147in}{0.413320in}}%
\pgfpathlineto{\pgfqpoint{5.018466in}{0.413320in}}%
\pgfpathlineto{\pgfqpoint{5.015820in}{0.413320in}}%
\pgfpathlineto{\pgfqpoint{5.013104in}{0.413320in}}%
\pgfpathlineto{\pgfqpoint{5.010562in}{0.413320in}}%
\pgfpathlineto{\pgfqpoint{5.007751in}{0.413320in}}%
\pgfpathlineto{\pgfqpoint{5.005178in}{0.413320in}}%
\pgfpathlineto{\pgfqpoint{5.002384in}{0.413320in}}%
\pgfpathlineto{\pgfqpoint{4.999780in}{0.413320in}}%
\pgfpathlineto{\pgfqpoint{4.997028in}{0.413320in}}%
\pgfpathlineto{\pgfqpoint{4.994390in}{0.413320in}}%
\pgfpathlineto{\pgfqpoint{4.991683in}{0.413320in}}%
\pgfpathlineto{\pgfqpoint{4.989001in}{0.413320in}}%
\pgfpathlineto{\pgfqpoint{4.986325in}{0.413320in}}%
\pgfpathlineto{\pgfqpoint{4.983637in}{0.413320in}}%
\pgfpathlineto{\pgfqpoint{4.980967in}{0.413320in}}%
\pgfpathlineto{\pgfqpoint{4.978287in}{0.413320in}}%
\pgfpathlineto{\pgfqpoint{4.975703in}{0.413320in}}%
\pgfpathlineto{\pgfqpoint{4.972933in}{0.413320in}}%
\pgfpathlineto{\pgfqpoint{4.970314in}{0.413320in}}%
\pgfpathlineto{\pgfqpoint{4.967575in}{0.413320in}}%
\pgfpathlineto{\pgfqpoint{4.965002in}{0.413320in}}%
\pgfpathlineto{\pgfqpoint{4.962219in}{0.413320in}}%
\pgfpathlineto{\pgfqpoint{4.959689in}{0.413320in}}%
\pgfpathlineto{\pgfqpoint{4.956862in}{0.413320in}}%
\pgfpathlineto{\pgfqpoint{4.954182in}{0.413320in}}%
\pgfpathlineto{\pgfqpoint{4.951504in}{0.413320in}}%
\pgfpathlineto{\pgfqpoint{4.948827in}{0.413320in}}%
\pgfpathlineto{\pgfqpoint{4.946151in}{0.413320in}}%
\pgfpathlineto{\pgfqpoint{4.943466in}{0.413320in}}%
\pgfpathlineto{\pgfqpoint{4.940881in}{0.413320in}}%
\pgfpathlineto{\pgfqpoint{4.938112in}{0.413320in}}%
\pgfpathlineto{\pgfqpoint{4.935515in}{0.413320in}}%
\pgfpathlineto{\pgfqpoint{4.932742in}{0.413320in}}%
\pgfpathlineto{\pgfqpoint{4.930170in}{0.413320in}}%
\pgfpathlineto{\pgfqpoint{4.927400in}{0.413320in}}%
\pgfpathlineto{\pgfqpoint{4.924708in}{0.413320in}}%
\pgfpathlineto{\pgfqpoint{4.922041in}{0.413320in}}%
\pgfpathlineto{\pgfqpoint{4.919352in}{0.413320in}}%
\pgfpathlineto{\pgfqpoint{4.916681in}{0.413320in}}%
\pgfpathlineto{\pgfqpoint{4.914009in}{0.413320in}}%
\pgfpathlineto{\pgfqpoint{4.911435in}{0.413320in}}%
\pgfpathlineto{\pgfqpoint{4.908648in}{0.413320in}}%
\pgfpathlineto{\pgfqpoint{4.906096in}{0.413320in}}%
\pgfpathlineto{\pgfqpoint{4.903295in}{0.413320in}}%
\pgfpathlineto{\pgfqpoint{4.900712in}{0.413320in}}%
\pgfpathlineto{\pgfqpoint{4.897938in}{0.413320in}}%
\pgfpathlineto{\pgfqpoint{4.895399in}{0.413320in}}%
\pgfpathlineto{\pgfqpoint{4.892611in}{0.413320in}}%
\pgfpathlineto{\pgfqpoint{4.889902in}{0.413320in}}%
\pgfpathlineto{\pgfqpoint{4.887211in}{0.413320in}}%
\pgfpathlineto{\pgfqpoint{4.884540in}{0.413320in}}%
\pgfpathlineto{\pgfqpoint{4.881864in}{0.413320in}}%
\pgfpathlineto{\pgfqpoint{4.879180in}{0.413320in}}%
\pgfpathlineto{\pgfqpoint{4.876636in}{0.413320in}}%
\pgfpathlineto{\pgfqpoint{4.873832in}{0.413320in}}%
\pgfpathlineto{\pgfqpoint{4.871209in}{0.413320in}}%
\pgfpathlineto{\pgfqpoint{4.868474in}{0.413320in}}%
\pgfpathlineto{\pgfqpoint{4.865910in}{0.413320in}}%
\pgfpathlineto{\pgfqpoint{4.863116in}{0.413320in}}%
\pgfpathlineto{\pgfqpoint{4.860544in}{0.413320in}}%
\pgfpathlineto{\pgfqpoint{4.857807in}{0.413320in}}%
\pgfpathlineto{\pgfqpoint{4.855070in}{0.413320in}}%
\pgfpathlineto{\pgfqpoint{4.852404in}{0.413320in}}%
\pgfpathlineto{\pgfqpoint{4.849715in}{0.413320in}}%
\pgfpathlineto{\pgfqpoint{4.847127in}{0.413320in}}%
\pgfpathlineto{\pgfqpoint{4.844361in}{0.413320in}}%
\pgfpathlineto{\pgfqpoint{4.842380in}{0.413320in}}%
\pgfpathlineto{\pgfqpoint{4.839922in}{0.413320in}}%
\pgfpathlineto{\pgfqpoint{4.837992in}{0.413320in}}%
\pgfpathlineto{\pgfqpoint{4.833657in}{0.413320in}}%
\pgfpathlineto{\pgfqpoint{4.831045in}{0.413320in}}%
\pgfpathlineto{\pgfqpoint{4.828291in}{0.413320in}}%
\pgfpathlineto{\pgfqpoint{4.825619in}{0.413320in}}%
\pgfpathlineto{\pgfqpoint{4.822945in}{0.413320in}}%
\pgfpathlineto{\pgfqpoint{4.820265in}{0.413320in}}%
\pgfpathlineto{\pgfqpoint{4.817587in}{0.413320in}}%
\pgfpathlineto{\pgfqpoint{4.814907in}{0.413320in}}%
\pgfpathlineto{\pgfqpoint{4.812377in}{0.413320in}}%
\pgfpathlineto{\pgfqpoint{4.809538in}{0.413320in}}%
\pgfpathlineto{\pgfqpoint{4.807017in}{0.413320in}}%
\pgfpathlineto{\pgfqpoint{4.804193in}{0.413320in}}%
\pgfpathlineto{\pgfqpoint{4.801586in}{0.413320in}}%
\pgfpathlineto{\pgfqpoint{4.798830in}{0.413320in}}%
\pgfpathlineto{\pgfqpoint{4.796274in}{0.413320in}}%
\pgfpathlineto{\pgfqpoint{4.793512in}{0.413320in}}%
\pgfpathlineto{\pgfqpoint{4.790798in}{0.413320in}}%
\pgfpathlineto{\pgfqpoint{4.788116in}{0.413320in}}%
\pgfpathlineto{\pgfqpoint{4.785445in}{0.413320in}}%
\pgfpathlineto{\pgfqpoint{4.782872in}{0.413320in}}%
\pgfpathlineto{\pgfqpoint{4.780083in}{0.413320in}}%
\pgfpathlineto{\pgfqpoint{4.777535in}{0.413320in}}%
\pgfpathlineto{\pgfqpoint{4.774732in}{0.413320in}}%
\pgfpathlineto{\pgfqpoint{4.772198in}{0.413320in}}%
\pgfpathlineto{\pgfqpoint{4.769367in}{0.413320in}}%
\pgfpathlineto{\pgfqpoint{4.766783in}{0.413320in}}%
\pgfpathlineto{\pgfqpoint{4.764018in}{0.413320in}}%
\pgfpathlineto{\pgfqpoint{4.761337in}{0.413320in}}%
\pgfpathlineto{\pgfqpoint{4.758653in}{0.413320in}}%
\pgfpathlineto{\pgfqpoint{4.755983in}{0.413320in}}%
\pgfpathlineto{\pgfqpoint{4.753298in}{0.413320in}}%
\pgfpathlineto{\pgfqpoint{4.750627in}{0.413320in}}%
\pgfpathlineto{\pgfqpoint{4.748081in}{0.413320in}}%
\pgfpathlineto{\pgfqpoint{4.745256in}{0.413320in}}%
\pgfpathlineto{\pgfqpoint{4.742696in}{0.413320in}}%
\pgfpathlineto{\pgfqpoint{4.739912in}{0.413320in}}%
\pgfpathlineto{\pgfqpoint{4.737348in}{0.413320in}}%
\pgfpathlineto{\pgfqpoint{4.734552in}{0.413320in}}%
\pgfpathlineto{\pgfqpoint{4.731901in}{0.413320in}}%
\pgfpathlineto{\pgfqpoint{4.729233in}{0.413320in}}%
\pgfpathlineto{\pgfqpoint{4.726508in}{0.413320in}}%
\pgfpathlineto{\pgfqpoint{4.723873in}{0.413320in}}%
\pgfpathlineto{\pgfqpoint{4.721160in}{0.413320in}}%
\pgfpathlineto{\pgfqpoint{4.718486in}{0.413320in}}%
\pgfpathlineto{\pgfqpoint{4.715806in}{0.413320in}}%
\pgfpathlineto{\pgfqpoint{4.713275in}{0.413320in}}%
\pgfpathlineto{\pgfqpoint{4.710437in}{0.413320in}}%
\pgfpathlineto{\pgfqpoint{4.707824in}{0.413320in}}%
\pgfpathlineto{\pgfqpoint{4.705094in}{0.413320in}}%
\pgfpathlineto{\pgfqpoint{4.702517in}{0.413320in}}%
\pgfpathlineto{\pgfqpoint{4.699734in}{0.413320in}}%
\pgfpathlineto{\pgfqpoint{4.697170in}{0.413320in}}%
\pgfpathlineto{\pgfqpoint{4.694381in}{0.413320in}}%
\pgfpathlineto{\pgfqpoint{4.691694in}{0.413320in}}%
\pgfpathlineto{\pgfqpoint{4.689051in}{0.413320in}}%
\pgfpathlineto{\pgfqpoint{4.686337in}{0.413320in}}%
\pgfpathlineto{\pgfqpoint{4.683799in}{0.413320in}}%
\pgfpathlineto{\pgfqpoint{4.680988in}{0.413320in}}%
\pgfpathlineto{\pgfqpoint{4.678448in}{0.413320in}}%
\pgfpathlineto{\pgfqpoint{4.675619in}{0.413320in}}%
\pgfpathlineto{\pgfqpoint{4.673068in}{0.413320in}}%
\pgfpathlineto{\pgfqpoint{4.670261in}{0.413320in}}%
\pgfpathlineto{\pgfqpoint{4.667764in}{0.413320in}}%
\pgfpathlineto{\pgfqpoint{4.664923in}{0.413320in}}%
\pgfpathlineto{\pgfqpoint{4.662237in}{0.413320in}}%
\pgfpathlineto{\pgfqpoint{4.659590in}{0.413320in}}%
\pgfpathlineto{\pgfqpoint{4.656873in}{0.413320in}}%
\pgfpathlineto{\pgfqpoint{4.654203in}{0.413320in}}%
\pgfpathlineto{\pgfqpoint{4.651524in}{0.413320in}}%
\pgfpathlineto{\pgfqpoint{4.648922in}{0.413320in}}%
\pgfpathlineto{\pgfqpoint{4.646169in}{0.413320in}}%
\pgfpathlineto{\pgfqpoint{4.643628in}{0.413320in}}%
\pgfpathlineto{\pgfqpoint{4.640809in}{0.413320in}}%
\pgfpathlineto{\pgfqpoint{4.638204in}{0.413320in}}%
\pgfpathlineto{\pgfqpoint{4.635445in}{0.413320in}}%
\pgfpathlineto{\pgfqpoint{4.632902in}{0.413320in}}%
\pgfpathlineto{\pgfqpoint{4.630096in}{0.413320in}}%
\pgfpathlineto{\pgfqpoint{4.627411in}{0.413320in}}%
\pgfpathlineto{\pgfqpoint{4.624741in}{0.413320in}}%
\pgfpathlineto{\pgfqpoint{4.622056in}{0.413320in}}%
\pgfpathlineto{\pgfqpoint{4.619529in}{0.413320in}}%
\pgfpathlineto{\pgfqpoint{4.616702in}{0.413320in}}%
\pgfpathlineto{\pgfqpoint{4.614134in}{0.413320in}}%
\pgfpathlineto{\pgfqpoint{4.611350in}{0.413320in}}%
\pgfpathlineto{\pgfqpoint{4.608808in}{0.413320in}}%
\pgfpathlineto{\pgfqpoint{4.605990in}{0.413320in}}%
\pgfpathlineto{\pgfqpoint{4.603430in}{0.413320in}}%
\pgfpathlineto{\pgfqpoint{4.600633in}{0.413320in}}%
\pgfpathlineto{\pgfqpoint{4.597951in}{0.413320in}}%
\pgfpathlineto{\pgfqpoint{4.595281in}{0.413320in}}%
\pgfpathlineto{\pgfqpoint{4.592589in}{0.413320in}}%
\pgfpathlineto{\pgfqpoint{4.589920in}{0.413320in}}%
\pgfpathlineto{\pgfqpoint{4.587244in}{0.413320in}}%
\pgfpathlineto{\pgfqpoint{4.584672in}{0.413320in}}%
\pgfpathlineto{\pgfqpoint{4.581888in}{0.413320in}}%
\pgfpathlineto{\pgfqpoint{4.579305in}{0.413320in}}%
\pgfpathlineto{\pgfqpoint{4.576531in}{0.413320in}}%
\pgfpathlineto{\pgfqpoint{4.573947in}{0.413320in}}%
\pgfpathlineto{\pgfqpoint{4.571171in}{0.413320in}}%
\pgfpathlineto{\pgfqpoint{4.568612in}{0.413320in}}%
\pgfpathlineto{\pgfqpoint{4.565820in}{0.413320in}}%
\pgfpathlineto{\pgfqpoint{4.563125in}{0.413320in}}%
\pgfpathlineto{\pgfqpoint{4.560448in}{0.413320in}}%
\pgfpathlineto{\pgfqpoint{4.557777in}{0.413320in}}%
\pgfpathlineto{\pgfqpoint{4.555106in}{0.413320in}}%
\pgfpathlineto{\pgfqpoint{4.552425in}{0.413320in}}%
\pgfpathlineto{\pgfqpoint{4.549822in}{0.413320in}}%
\pgfpathlineto{\pgfqpoint{4.547064in}{0.413320in}}%
\pgfpathlineto{\pgfqpoint{4.544464in}{0.413320in}}%
\pgfpathlineto{\pgfqpoint{4.541711in}{0.413320in}}%
\pgfpathlineto{\pgfqpoint{4.539144in}{0.413320in}}%
\pgfpathlineto{\pgfqpoint{4.536400in}{0.413320in}}%
\pgfpathlineto{\pgfqpoint{4.533764in}{0.413320in}}%
\pgfpathlineto{\pgfqpoint{4.530990in}{0.413320in}}%
\pgfpathlineto{\pgfqpoint{4.528307in}{0.413320in}}%
\pgfpathlineto{\pgfqpoint{4.525640in}{0.413320in}}%
\pgfpathlineto{\pgfqpoint{4.522962in}{0.413320in}}%
\pgfpathlineto{\pgfqpoint{4.520345in}{0.413320in}}%
\pgfpathlineto{\pgfqpoint{4.517598in}{0.413320in}}%
\pgfpathlineto{\pgfqpoint{4.515080in}{0.413320in}}%
\pgfpathlineto{\pgfqpoint{4.512246in}{0.413320in}}%
\pgfpathlineto{\pgfqpoint{4.509643in}{0.413320in}}%
\pgfpathlineto{\pgfqpoint{4.506893in}{0.413320in}}%
\pgfpathlineto{\pgfqpoint{4.504305in}{0.413320in}}%
\pgfpathlineto{\pgfqpoint{4.501529in}{0.413320in}}%
\pgfpathlineto{\pgfqpoint{4.498850in}{0.413320in}}%
\pgfpathlineto{\pgfqpoint{4.496167in}{0.413320in}}%
\pgfpathlineto{\pgfqpoint{4.493492in}{0.413320in}}%
\pgfpathlineto{\pgfqpoint{4.490822in}{0.413320in}}%
\pgfpathlineto{\pgfqpoint{4.488130in}{0.413320in}}%
\pgfpathlineto{\pgfqpoint{4.485581in}{0.413320in}}%
\pgfpathlineto{\pgfqpoint{4.482778in}{0.413320in}}%
\pgfpathlineto{\pgfqpoint{4.480201in}{0.413320in}}%
\pgfpathlineto{\pgfqpoint{4.477430in}{0.413320in}}%
\pgfpathlineto{\pgfqpoint{4.474861in}{0.413320in}}%
\pgfpathlineto{\pgfqpoint{4.472059in}{0.413320in}}%
\pgfpathlineto{\pgfqpoint{4.469492in}{0.413320in}}%
\pgfpathlineto{\pgfqpoint{4.466717in}{0.413320in}}%
\pgfpathlineto{\pgfqpoint{4.464029in}{0.413320in}}%
\pgfpathlineto{\pgfqpoint{4.461367in}{0.413320in}}%
\pgfpathlineto{\pgfqpoint{4.458681in}{0.413320in}}%
\pgfpathlineto{\pgfqpoint{4.456138in}{0.413320in}}%
\pgfpathlineto{\pgfqpoint{4.453312in}{0.413320in}}%
\pgfpathlineto{\pgfqpoint{4.450767in}{0.413320in}}%
\pgfpathlineto{\pgfqpoint{4.447965in}{0.413320in}}%
\pgfpathlineto{\pgfqpoint{4.445423in}{0.413320in}}%
\pgfpathlineto{\pgfqpoint{4.442611in}{0.413320in}}%
\pgfpathlineto{\pgfqpoint{4.440041in}{0.413320in}}%
\pgfpathlineto{\pgfqpoint{4.437253in}{0.413320in}}%
\pgfpathlineto{\pgfqpoint{4.434569in}{0.413320in}}%
\pgfpathlineto{\pgfqpoint{4.431901in}{0.413320in}}%
\pgfpathlineto{\pgfqpoint{4.429220in}{0.413320in}}%
\pgfpathlineto{\pgfqpoint{4.426534in}{0.413320in}}%
\pgfpathlineto{\pgfqpoint{4.423863in}{0.413320in}}%
\pgfpathlineto{\pgfqpoint{4.421292in}{0.413320in}}%
\pgfpathlineto{\pgfqpoint{4.418506in}{0.413320in}}%
\pgfpathlineto{\pgfqpoint{4.415932in}{0.413320in}}%
\pgfpathlineto{\pgfqpoint{4.413149in}{0.413320in}}%
\pgfpathlineto{\pgfqpoint{4.410587in}{0.413320in}}%
\pgfpathlineto{\pgfqpoint{4.407788in}{0.413320in}}%
\pgfpathlineto{\pgfqpoint{4.405234in}{0.413320in}}%
\pgfpathlineto{\pgfqpoint{4.402468in}{0.413320in}}%
\pgfpathlineto{\pgfqpoint{4.399745in}{0.413320in}}%
\pgfpathlineto{\pgfqpoint{4.397076in}{0.413320in}}%
\pgfpathlineto{\pgfqpoint{4.394400in}{0.413320in}}%
\pgfpathlineto{\pgfqpoint{4.391721in}{0.413320in}}%
\pgfpathlineto{\pgfqpoint{4.389044in}{0.413320in}}%
\pgfpathlineto{\pgfqpoint{4.386431in}{0.413320in}}%
\pgfpathlineto{\pgfqpoint{4.383674in}{0.413320in}}%
\pgfpathlineto{\pgfqpoint{4.381097in}{0.413320in}}%
\pgfpathlineto{\pgfqpoint{4.378329in}{0.413320in}}%
\pgfpathlineto{\pgfqpoint{4.375761in}{0.413320in}}%
\pgfpathlineto{\pgfqpoint{4.372976in}{0.413320in}}%
\pgfpathlineto{\pgfqpoint{4.370437in}{0.413320in}}%
\pgfpathlineto{\pgfqpoint{4.367646in}{0.413320in}}%
\pgfpathlineto{\pgfqpoint{4.364936in}{0.413320in}}%
\pgfpathlineto{\pgfqpoint{4.362270in}{0.413320in}}%
\pgfpathlineto{\pgfqpoint{4.359582in}{0.413320in}}%
\pgfpathlineto{\pgfqpoint{4.357014in}{0.413320in}}%
\pgfpathlineto{\pgfqpoint{4.354224in}{0.413320in}}%
\pgfpathlineto{\pgfqpoint{4.351645in}{0.413320in}}%
\pgfpathlineto{\pgfqpoint{4.348868in}{0.413320in}}%
\pgfpathlineto{\pgfqpoint{4.346263in}{0.413320in}}%
\pgfpathlineto{\pgfqpoint{4.343510in}{0.413320in}}%
\pgfpathlineto{\pgfqpoint{4.340976in}{0.413320in}}%
\pgfpathlineto{\pgfqpoint{4.338154in}{0.413320in}}%
\pgfpathlineto{\pgfqpoint{4.335463in}{0.413320in}}%
\pgfpathlineto{\pgfqpoint{4.332796in}{0.413320in}}%
\pgfpathlineto{\pgfqpoint{4.330118in}{0.413320in}}%
\pgfpathlineto{\pgfqpoint{4.327440in}{0.413320in}}%
\pgfpathlineto{\pgfqpoint{4.324760in}{0.413320in}}%
\pgfpathlineto{\pgfqpoint{4.322181in}{0.413320in}}%
\pgfpathlineto{\pgfqpoint{4.319405in}{0.413320in}}%
\pgfpathlineto{\pgfqpoint{4.316856in}{0.413320in}}%
\pgfpathlineto{\pgfqpoint{4.314032in}{0.413320in}}%
\pgfpathlineto{\pgfqpoint{4.311494in}{0.413320in}}%
\pgfpathlineto{\pgfqpoint{4.308691in}{0.413320in}}%
\pgfpathlineto{\pgfqpoint{4.306118in}{0.413320in}}%
\pgfpathlineto{\pgfqpoint{4.303357in}{0.413320in}}%
\pgfpathlineto{\pgfqpoint{4.300656in}{0.413320in}}%
\pgfpathlineto{\pgfqpoint{4.297977in}{0.413320in}}%
\pgfpathlineto{\pgfqpoint{4.295299in}{0.413320in}}%
\pgfpathlineto{\pgfqpoint{4.292786in}{0.413320in}}%
\pgfpathlineto{\pgfqpoint{4.289936in}{0.413320in}}%
\pgfpathlineto{\pgfqpoint{4.287399in}{0.413320in}}%
\pgfpathlineto{\pgfqpoint{4.284586in}{0.413320in}}%
\pgfpathlineto{\pgfqpoint{4.282000in}{0.413320in}}%
\pgfpathlineto{\pgfqpoint{4.279212in}{0.413320in}}%
\pgfpathlineto{\pgfqpoint{4.276635in}{0.413320in}}%
\pgfpathlineto{\pgfqpoint{4.273874in}{0.413320in}}%
\pgfpathlineto{\pgfqpoint{4.271187in}{0.413320in}}%
\pgfpathlineto{\pgfqpoint{4.268590in}{0.413320in}}%
\pgfpathlineto{\pgfqpoint{4.265824in}{0.413320in}}%
\pgfpathlineto{\pgfqpoint{4.263157in}{0.413320in}}%
\pgfpathlineto{\pgfqpoint{4.260477in}{0.413320in}}%
\pgfpathlineto{\pgfqpoint{4.257958in}{0.413320in}}%
\pgfpathlineto{\pgfqpoint{4.255120in}{0.413320in}}%
\pgfpathlineto{\pgfqpoint{4.252581in}{0.413320in}}%
\pgfpathlineto{\pgfqpoint{4.249767in}{0.413320in}}%
\pgfpathlineto{\pgfqpoint{4.247225in}{0.413320in}}%
\pgfpathlineto{\pgfqpoint{4.244394in}{0.413320in}}%
\pgfpathlineto{\pgfqpoint{4.241900in}{0.413320in}}%
\pgfpathlineto{\pgfqpoint{4.239084in}{0.413320in}}%
\pgfpathlineto{\pgfqpoint{4.236375in}{0.413320in}}%
\pgfpathlineto{\pgfqpoint{4.233691in}{0.413320in}}%
\pgfpathlineto{\pgfqpoint{4.231013in}{0.413320in}}%
\pgfpathlineto{\pgfqpoint{4.228331in}{0.413320in}}%
\pgfpathlineto{\pgfqpoint{4.225654in}{0.413320in}}%
\pgfpathlineto{\pgfqpoint{4.223082in}{0.413320in}}%
\pgfpathlineto{\pgfqpoint{4.220304in}{0.413320in}}%
\pgfpathlineto{\pgfqpoint{4.217694in}{0.413320in}}%
\pgfpathlineto{\pgfqpoint{4.214948in}{0.413320in}}%
\pgfpathlineto{\pgfqpoint{4.212383in}{0.413320in}}%
\pgfpathlineto{\pgfqpoint{4.209597in}{0.413320in}}%
\pgfpathlineto{\pgfqpoint{4.207076in}{0.413320in}}%
\pgfpathlineto{\pgfqpoint{4.204240in}{0.413320in}}%
\pgfpathlineto{\pgfqpoint{4.201542in}{0.413320in}}%
\pgfpathlineto{\pgfqpoint{4.198878in}{0.413320in}}%
\pgfpathlineto{\pgfqpoint{4.196186in}{0.413320in}}%
\pgfpathlineto{\pgfqpoint{4.193638in}{0.413320in}}%
\pgfpathlineto{\pgfqpoint{4.190842in}{0.413320in}}%
\pgfpathlineto{\pgfqpoint{4.188318in}{0.413320in}}%
\pgfpathlineto{\pgfqpoint{4.185481in}{0.413320in}}%
\pgfpathlineto{\pgfqpoint{4.182899in}{0.413320in}}%
\pgfpathlineto{\pgfqpoint{4.180129in}{0.413320in}}%
\pgfpathlineto{\pgfqpoint{4.177593in}{0.413320in}}%
\pgfpathlineto{\pgfqpoint{4.174770in}{0.413320in}}%
\pgfpathlineto{\pgfqpoint{4.172093in}{0.413320in}}%
\pgfpathlineto{\pgfqpoint{4.169415in}{0.413320in}}%
\pgfpathlineto{\pgfqpoint{4.166737in}{0.413320in}}%
\pgfpathlineto{\pgfqpoint{4.164059in}{0.413320in}}%
\pgfpathlineto{\pgfqpoint{4.161380in}{0.413320in}}%
\pgfpathlineto{\pgfqpoint{4.158806in}{0.413320in}}%
\pgfpathlineto{\pgfqpoint{4.156016in}{0.413320in}}%
\pgfpathlineto{\pgfqpoint{4.153423in}{0.413320in}}%
\pgfpathlineto{\pgfqpoint{4.150665in}{0.413320in}}%
\pgfpathlineto{\pgfqpoint{4.148082in}{0.413320in}}%
\pgfpathlineto{\pgfqpoint{4.145310in}{0.413320in}}%
\pgfpathlineto{\pgfqpoint{4.142713in}{0.413320in}}%
\pgfpathlineto{\pgfqpoint{4.139963in}{0.413320in}}%
\pgfpathlineto{\pgfqpoint{4.137272in}{0.413320in}}%
\pgfpathlineto{\pgfqpoint{4.134615in}{0.413320in}}%
\pgfpathlineto{\pgfqpoint{4.131920in}{0.413320in}}%
\pgfpathlineto{\pgfqpoint{4.129349in}{0.413320in}}%
\pgfpathlineto{\pgfqpoint{4.126553in}{0.413320in}}%
\pgfpathlineto{\pgfqpoint{4.124019in}{0.413320in}}%
\pgfpathlineto{\pgfqpoint{4.121205in}{0.413320in}}%
\pgfpathlineto{\pgfqpoint{4.118554in}{0.413320in}}%
\pgfpathlineto{\pgfqpoint{4.115844in}{0.413320in}}%
\pgfpathlineto{\pgfqpoint{4.113252in}{0.413320in}}%
\pgfpathlineto{\pgfqpoint{4.110488in}{0.413320in}}%
\pgfpathlineto{\pgfqpoint{4.107814in}{0.413320in}}%
\pgfpathlineto{\pgfqpoint{4.105185in}{0.413320in}}%
\pgfpathlineto{\pgfqpoint{4.102456in}{0.413320in}}%
\pgfpathlineto{\pgfqpoint{4.099777in}{0.413320in}}%
\pgfpathlineto{\pgfqpoint{4.097092in}{0.413320in}}%
\pgfpathlineto{\pgfqpoint{4.094527in}{0.413320in}}%
\pgfpathlineto{\pgfqpoint{4.091729in}{0.413320in}}%
\pgfpathlineto{\pgfqpoint{4.089159in}{0.413320in}}%
\pgfpathlineto{\pgfqpoint{4.086385in}{0.413320in}}%
\pgfpathlineto{\pgfqpoint{4.083870in}{0.413320in}}%
\pgfpathlineto{\pgfqpoint{4.081018in}{0.413320in}}%
\pgfpathlineto{\pgfqpoint{4.078471in}{0.413320in}}%
\pgfpathlineto{\pgfqpoint{4.075705in}{0.413320in}}%
\pgfpathlineto{\pgfqpoint{4.072985in}{0.413320in}}%
\pgfpathlineto{\pgfqpoint{4.070313in}{0.413320in}}%
\pgfpathlineto{\pgfqpoint{4.067636in}{0.413320in}}%
\pgfpathlineto{\pgfqpoint{4.064957in}{0.413320in}}%
\pgfpathlineto{\pgfqpoint{4.062266in}{0.413320in}}%
\pgfpathlineto{\pgfqpoint{4.059702in}{0.413320in}}%
\pgfpathlineto{\pgfqpoint{4.056911in}{0.413320in}}%
\pgfpathlineto{\pgfqpoint{4.054326in}{0.413320in}}%
\pgfpathlineto{\pgfqpoint{4.051557in}{0.413320in}}%
\pgfpathlineto{\pgfqpoint{4.049006in}{0.413320in}}%
\pgfpathlineto{\pgfqpoint{4.046210in}{0.413320in}}%
\pgfpathlineto{\pgfqpoint{4.043667in}{0.413320in}}%
\pgfpathlineto{\pgfqpoint{4.040852in}{0.413320in}}%
\pgfpathlineto{\pgfqpoint{4.038174in}{0.413320in}}%
\pgfpathlineto{\pgfqpoint{4.035492in}{0.413320in}}%
\pgfpathlineto{\pgfqpoint{4.032817in}{0.413320in}}%
\pgfpathlineto{\pgfqpoint{4.030229in}{0.413320in}}%
\pgfpathlineto{\pgfqpoint{4.027447in}{0.413320in}}%
\pgfpathlineto{\pgfqpoint{4.024868in}{0.413320in}}%
\pgfpathlineto{\pgfqpoint{4.022097in}{0.413320in}}%
\pgfpathlineto{\pgfqpoint{4.019518in}{0.413320in}}%
\pgfpathlineto{\pgfqpoint{4.016744in}{0.413320in}}%
\pgfpathlineto{\pgfqpoint{4.014186in}{0.413320in}}%
\pgfpathlineto{\pgfqpoint{4.011394in}{0.413320in}}%
\pgfpathlineto{\pgfqpoint{4.008699in}{0.413320in}}%
\pgfpathlineto{\pgfqpoint{4.006034in}{0.413320in}}%
\pgfpathlineto{\pgfqpoint{4.003348in}{0.413320in}}%
\pgfpathlineto{\pgfqpoint{4.000674in}{0.413320in}}%
\pgfpathlineto{\pgfqpoint{3.997990in}{0.413320in}}%
\pgfpathlineto{\pgfqpoint{3.995417in}{0.413320in}}%
\pgfpathlineto{\pgfqpoint{3.992642in}{0.413320in}}%
\pgfpathlineto{\pgfqpoint{3.990055in}{0.413320in}}%
\pgfpathlineto{\pgfqpoint{3.987270in}{0.413320in}}%
\pgfpathlineto{\pgfqpoint{3.984714in}{0.413320in}}%
\pgfpathlineto{\pgfqpoint{3.981929in}{0.413320in}}%
\pgfpathlineto{\pgfqpoint{3.979389in}{0.413320in}}%
\pgfpathlineto{\pgfqpoint{3.976563in}{0.413320in}}%
\pgfpathlineto{\pgfqpoint{3.973885in}{0.413320in}}%
\pgfpathlineto{\pgfqpoint{3.971250in}{0.413320in}}%
\pgfpathlineto{\pgfqpoint{3.968523in}{0.413320in}}%
\pgfpathlineto{\pgfqpoint{3.966013in}{0.413320in}}%
\pgfpathlineto{\pgfqpoint{3.963176in}{0.413320in}}%
\pgfpathlineto{\pgfqpoint{3.960635in}{0.413320in}}%
\pgfpathlineto{\pgfqpoint{3.957823in}{0.413320in}}%
\pgfpathlineto{\pgfqpoint{3.955211in}{0.413320in}}%
\pgfpathlineto{\pgfqpoint{3.952464in}{0.413320in}}%
\pgfpathlineto{\pgfqpoint{3.949894in}{0.413320in}}%
\pgfpathlineto{\pgfqpoint{3.947101in}{0.413320in}}%
\pgfpathlineto{\pgfqpoint{3.944431in}{0.413320in}}%
\pgfpathlineto{\pgfqpoint{3.941778in}{0.413320in}}%
\pgfpathlineto{\pgfqpoint{3.939075in}{0.413320in}}%
\pgfpathlineto{\pgfqpoint{3.936395in}{0.413320in}}%
\pgfpathlineto{\pgfqpoint{3.933714in}{0.413320in}}%
\pgfpathlineto{\pgfqpoint{3.931202in}{0.413320in}}%
\pgfpathlineto{\pgfqpoint{3.928347in}{0.413320in}}%
\pgfpathlineto{\pgfqpoint{3.925778in}{0.413320in}}%
\pgfpathlineto{\pgfqpoint{3.923005in}{0.413320in}}%
\pgfpathlineto{\pgfqpoint{3.920412in}{0.413320in}}%
\pgfpathlineto{\pgfqpoint{3.917646in}{0.413320in}}%
\pgfpathlineto{\pgfqpoint{3.915107in}{0.413320in}}%
\pgfpathlineto{\pgfqpoint{3.912296in}{0.413320in}}%
\pgfpathlineto{\pgfqpoint{3.909602in}{0.413320in}}%
\pgfpathlineto{\pgfqpoint{3.906918in}{0.413320in}}%
\pgfpathlineto{\pgfqpoint{3.904252in}{0.413320in}}%
\pgfpathlineto{\pgfqpoint{3.901573in}{0.413320in}}%
\pgfpathlineto{\pgfqpoint{3.898891in}{0.413320in}}%
\pgfpathlineto{\pgfqpoint{3.896345in}{0.413320in}}%
\pgfpathlineto{\pgfqpoint{3.893541in}{0.413320in}}%
\pgfpathlineto{\pgfqpoint{3.890926in}{0.413320in}}%
\pgfpathlineto{\pgfqpoint{3.888188in}{0.413320in}}%
\pgfpathlineto{\pgfqpoint{3.885621in}{0.413320in}}%
\pgfpathlineto{\pgfqpoint{3.882850in}{0.413320in}}%
\pgfpathlineto{\pgfqpoint{3.880237in}{0.413320in}}%
\pgfpathlineto{\pgfqpoint{3.877466in}{0.413320in}}%
\pgfpathlineto{\pgfqpoint{3.874790in}{0.413320in}}%
\pgfpathlineto{\pgfqpoint{3.872114in}{0.413320in}}%
\pgfpathlineto{\pgfqpoint{3.869435in}{0.413320in}}%
\pgfpathlineto{\pgfqpoint{3.866815in}{0.413320in}}%
\pgfpathlineto{\pgfqpoint{3.864073in}{0.413320in}}%
\pgfpathlineto{\pgfqpoint{3.861561in}{0.413320in}}%
\pgfpathlineto{\pgfqpoint{3.858720in}{0.413320in}}%
\pgfpathlineto{\pgfqpoint{3.856100in}{0.413320in}}%
\pgfpathlineto{\pgfqpoint{3.853358in}{0.413320in}}%
\pgfpathlineto{\pgfqpoint{3.850814in}{0.413320in}}%
\pgfpathlineto{\pgfqpoint{3.848005in}{0.413320in}}%
\pgfpathlineto{\pgfqpoint{3.845329in}{0.413320in}}%
\pgfpathlineto{\pgfqpoint{3.842641in}{0.413320in}}%
\pgfpathlineto{\pgfqpoint{3.839960in}{0.413320in}}%
\pgfpathlineto{\pgfqpoint{3.837286in}{0.413320in}}%
\pgfpathlineto{\pgfqpoint{3.834616in}{0.413320in}}%
\pgfpathlineto{\pgfqpoint{3.832053in}{0.413320in}}%
\pgfpathlineto{\pgfqpoint{3.829252in}{0.413320in}}%
\pgfpathlineto{\pgfqpoint{3.826679in}{0.413320in}}%
\pgfpathlineto{\pgfqpoint{3.823903in}{0.413320in}}%
\pgfpathlineto{\pgfqpoint{3.821315in}{0.413320in}}%
\pgfpathlineto{\pgfqpoint{3.818546in}{0.413320in}}%
\pgfpathlineto{\pgfqpoint{3.815983in}{0.413320in}}%
\pgfpathlineto{\pgfqpoint{3.813172in}{0.413320in}}%
\pgfpathlineto{\pgfqpoint{3.810510in}{0.413320in}}%
\pgfpathlineto{\pgfqpoint{3.807832in}{0.413320in}}%
\pgfpathlineto{\pgfqpoint{3.805145in}{0.413320in}}%
\pgfpathlineto{\pgfqpoint{3.802569in}{0.413320in}}%
\pgfpathlineto{\pgfqpoint{3.799797in}{0.413320in}}%
\pgfpathlineto{\pgfqpoint{3.797265in}{0.413320in}}%
\pgfpathlineto{\pgfqpoint{3.794435in}{0.413320in}}%
\pgfpathlineto{\pgfqpoint{3.791897in}{0.413320in}}%
\pgfpathlineto{\pgfqpoint{3.789084in}{0.413320in}}%
\pgfpathlineto{\pgfqpoint{3.786504in}{0.413320in}}%
\pgfpathlineto{\pgfqpoint{3.783725in}{0.413320in}}%
\pgfpathlineto{\pgfqpoint{3.781046in}{0.413320in}}%
\pgfpathlineto{\pgfqpoint{3.778370in}{0.413320in}}%
\pgfpathlineto{\pgfqpoint{3.775691in}{0.413320in}}%
\pgfpathlineto{\pgfqpoint{3.773014in}{0.413320in}}%
\pgfpathlineto{\pgfqpoint{3.770323in}{0.413320in}}%
\pgfpathlineto{\pgfqpoint{3.767782in}{0.413320in}}%
\pgfpathlineto{\pgfqpoint{3.764966in}{0.413320in}}%
\pgfpathlineto{\pgfqpoint{3.762389in}{0.413320in}}%
\pgfpathlineto{\pgfqpoint{3.759622in}{0.413320in}}%
\pgfpathlineto{\pgfqpoint{3.757065in}{0.413320in}}%
\pgfpathlineto{\pgfqpoint{3.754265in}{0.413320in}}%
\pgfpathlineto{\pgfqpoint{3.751728in}{0.413320in}}%
\pgfpathlineto{\pgfqpoint{3.748903in}{0.413320in}}%
\pgfpathlineto{\pgfqpoint{3.746229in}{0.413320in}}%
\pgfpathlineto{\pgfqpoint{3.743548in}{0.413320in}}%
\pgfpathlineto{\pgfqpoint{3.740874in}{0.413320in}}%
\pgfpathlineto{\pgfqpoint{3.738194in}{0.413320in}}%
\pgfpathlineto{\pgfqpoint{3.735509in}{0.413320in}}%
\pgfpathlineto{\pgfqpoint{3.732950in}{0.413320in}}%
\pgfpathlineto{\pgfqpoint{3.730158in}{0.413320in}}%
\pgfpathlineto{\pgfqpoint{3.727581in}{0.413320in}}%
\pgfpathlineto{\pgfqpoint{3.724804in}{0.413320in}}%
\pgfpathlineto{\pgfqpoint{3.722228in}{0.413320in}}%
\pgfpathlineto{\pgfqpoint{3.719446in}{0.413320in}}%
\pgfpathlineto{\pgfqpoint{3.716875in}{0.413320in}}%
\pgfpathlineto{\pgfqpoint{3.714086in}{0.413320in}}%
\pgfpathlineto{\pgfqpoint{3.711410in}{0.413320in}}%
\pgfpathlineto{\pgfqpoint{3.708729in}{0.413320in}}%
\pgfpathlineto{\pgfqpoint{3.706053in}{0.413320in}}%
\pgfpathlineto{\pgfqpoint{3.703460in}{0.413320in}}%
\pgfpathlineto{\pgfqpoint{3.700684in}{0.413320in}}%
\pgfpathlineto{\pgfqpoint{3.698125in}{0.413320in}}%
\pgfpathlineto{\pgfqpoint{3.695331in}{0.413320in}}%
\pgfpathlineto{\pgfqpoint{3.692765in}{0.413320in}}%
\pgfpathlineto{\pgfqpoint{3.689983in}{0.413320in}}%
\pgfpathlineto{\pgfqpoint{3.687442in}{0.413320in}}%
\pgfpathlineto{\pgfqpoint{3.684620in}{0.413320in}}%
\pgfpathlineto{\pgfqpoint{3.681948in}{0.413320in}}%
\pgfpathlineto{\pgfqpoint{3.679273in}{0.413320in}}%
\pgfpathlineto{\pgfqpoint{3.676591in}{0.413320in}}%
\pgfpathlineto{\pgfqpoint{3.673911in}{0.413320in}}%
\pgfpathlineto{\pgfqpoint{3.671232in}{0.413320in}}%
\pgfpathlineto{\pgfqpoint{3.668665in}{0.413320in}}%
\pgfpathlineto{\pgfqpoint{3.665864in}{0.413320in}}%
\pgfpathlineto{\pgfqpoint{3.663276in}{0.413320in}}%
\pgfpathlineto{\pgfqpoint{3.660515in}{0.413320in}}%
\pgfpathlineto{\pgfqpoint{3.657917in}{0.413320in}}%
\pgfpathlineto{\pgfqpoint{3.655165in}{0.413320in}}%
\pgfpathlineto{\pgfqpoint{3.652628in}{0.413320in}}%
\pgfpathlineto{\pgfqpoint{3.649837in}{0.413320in}}%
\pgfpathlineto{\pgfqpoint{3.647130in}{0.413320in}}%
\pgfpathlineto{\pgfqpoint{3.644452in}{0.413320in}}%
\pgfpathlineto{\pgfqpoint{3.641773in}{0.413320in}}%
\pgfpathlineto{\pgfqpoint{3.639207in}{0.413320in}}%
\pgfpathlineto{\pgfqpoint{3.636413in}{0.413320in}}%
\pgfpathlineto{\pgfqpoint{3.633858in}{0.413320in}}%
\pgfpathlineto{\pgfqpoint{3.631058in}{0.413320in}}%
\pgfpathlineto{\pgfqpoint{3.628460in}{0.413320in}}%
\pgfpathlineto{\pgfqpoint{3.625689in}{0.413320in}}%
\pgfpathlineto{\pgfqpoint{3.623165in}{0.413320in}}%
\pgfpathlineto{\pgfqpoint{3.620345in}{0.413320in}}%
\pgfpathlineto{\pgfqpoint{3.617667in}{0.413320in}}%
\pgfpathlineto{\pgfqpoint{3.614982in}{0.413320in}}%
\pgfpathlineto{\pgfqpoint{3.612311in}{0.413320in}}%
\pgfpathlineto{\pgfqpoint{3.609632in}{0.413320in}}%
\pgfpathlineto{\pgfqpoint{3.606951in}{0.413320in}}%
\pgfpathlineto{\pgfqpoint{3.604387in}{0.413320in}}%
\pgfpathlineto{\pgfqpoint{3.601590in}{0.413320in}}%
\pgfpathlineto{\pgfqpoint{3.598998in}{0.413320in}}%
\pgfpathlineto{\pgfqpoint{3.596240in}{0.413320in}}%
\pgfpathlineto{\pgfqpoint{3.593620in}{0.413320in}}%
\pgfpathlineto{\pgfqpoint{3.590883in}{0.413320in}}%
\pgfpathlineto{\pgfqpoint{3.588258in}{0.413320in}}%
\pgfpathlineto{\pgfqpoint{3.585532in}{0.413320in}}%
\pgfpathlineto{\pgfqpoint{3.582851in}{0.413320in}}%
\pgfpathlineto{\pgfqpoint{3.580191in}{0.413320in}}%
\pgfpathlineto{\pgfqpoint{3.577487in}{0.413320in}}%
\pgfpathlineto{\pgfqpoint{3.574814in}{0.413320in}}%
\pgfpathlineto{\pgfqpoint{3.572126in}{0.413320in}}%
\pgfpathlineto{\pgfqpoint{3.569584in}{0.413320in}}%
\pgfpathlineto{\pgfqpoint{3.566774in}{0.413320in}}%
\pgfpathlineto{\pgfqpoint{3.564188in}{0.413320in}}%
\pgfpathlineto{\pgfqpoint{3.561420in}{0.413320in}}%
\pgfpathlineto{\pgfqpoint{3.558853in}{0.413320in}}%
\pgfpathlineto{\pgfqpoint{3.556061in}{0.413320in}}%
\pgfpathlineto{\pgfqpoint{3.553498in}{0.413320in}}%
\pgfpathlineto{\pgfqpoint{3.550713in}{0.413320in}}%
\pgfpathlineto{\pgfqpoint{3.548029in}{0.413320in}}%
\pgfpathlineto{\pgfqpoint{3.545349in}{0.413320in}}%
\pgfpathlineto{\pgfqpoint{3.542656in}{0.413320in}}%
\pgfpathlineto{\pgfqpoint{3.540093in}{0.413320in}}%
\pgfpathlineto{\pgfqpoint{3.537309in}{0.413320in}}%
\pgfpathlineto{\pgfqpoint{3.534783in}{0.413320in}}%
\pgfpathlineto{\pgfqpoint{3.531955in}{0.413320in}}%
\pgfpathlineto{\pgfqpoint{3.529327in}{0.413320in}}%
\pgfpathlineto{\pgfqpoint{3.526601in}{0.413320in}}%
\pgfpathlineto{\pgfqpoint{3.524041in}{0.413320in}}%
\pgfpathlineto{\pgfqpoint{3.521244in}{0.413320in}}%
\pgfpathlineto{\pgfqpoint{3.518565in}{0.413320in}}%
\pgfpathlineto{\pgfqpoint{3.515884in}{0.413320in}}%
\pgfpathlineto{\pgfqpoint{3.513209in}{0.413320in}}%
\pgfpathlineto{\pgfqpoint{3.510533in}{0.413320in}}%
\pgfpathlineto{\pgfqpoint{3.507840in}{0.413320in}}%
\pgfpathlineto{\pgfqpoint{3.505262in}{0.413320in}}%
\pgfpathlineto{\pgfqpoint{3.502488in}{0.413320in}}%
\pgfpathlineto{\pgfqpoint{3.499909in}{0.413320in}}%
\pgfpathlineto{\pgfqpoint{3.497139in}{0.413320in}}%
\pgfpathlineto{\pgfqpoint{3.494581in}{0.413320in}}%
\pgfpathlineto{\pgfqpoint{3.491783in}{0.413320in}}%
\pgfpathlineto{\pgfqpoint{3.489223in}{0.413320in}}%
\pgfpathlineto{\pgfqpoint{3.486442in}{0.413320in}}%
\pgfpathlineto{\pgfqpoint{3.483744in}{0.413320in}}%
\pgfpathlineto{\pgfqpoint{3.481072in}{0.413320in}}%
\pgfpathlineto{\pgfqpoint{3.478378in}{0.413320in}}%
\pgfpathlineto{\pgfqpoint{3.475821in}{0.413320in}}%
\pgfpathlineto{\pgfqpoint{3.473021in}{0.413320in}}%
\pgfpathlineto{\pgfqpoint{3.470466in}{0.413320in}}%
\pgfpathlineto{\pgfqpoint{3.467678in}{0.413320in}}%
\pgfpathlineto{\pgfqpoint{3.465072in}{0.413320in}}%
\pgfpathlineto{\pgfqpoint{3.462321in}{0.413320in}}%
\pgfpathlineto{\pgfqpoint{3.459695in}{0.413320in}}%
\pgfpathlineto{\pgfqpoint{3.456960in}{0.413320in}}%
\pgfpathlineto{\pgfqpoint{3.454285in}{0.413320in}}%
\pgfpathlineto{\pgfqpoint{3.451597in}{0.413320in}}%
\pgfpathlineto{\pgfqpoint{3.448926in}{0.413320in}}%
\pgfpathlineto{\pgfqpoint{3.446257in}{0.413320in}}%
\pgfpathlineto{\pgfqpoint{3.443574in}{0.413320in}}%
\pgfpathlineto{\pgfqpoint{3.440996in}{0.413320in}}%
\pgfpathlineto{\pgfqpoint{3.438210in}{0.413320in}}%
\pgfpathlineto{\pgfqpoint{3.435635in}{0.413320in}}%
\pgfpathlineto{\pgfqpoint{3.432851in}{0.413320in}}%
\pgfpathlineto{\pgfqpoint{3.430313in}{0.413320in}}%
\pgfpathlineto{\pgfqpoint{3.427501in}{0.413320in}}%
\pgfpathlineto{\pgfqpoint{3.424887in}{0.413320in}}%
\pgfpathlineto{\pgfqpoint{3.422142in}{0.413320in}}%
\pgfpathlineto{\pgfqpoint{3.419455in}{0.413320in}}%
\pgfpathlineto{\pgfqpoint{3.416780in}{0.413320in}}%
\pgfpathlineto{\pgfqpoint{3.414109in}{0.413320in}}%
\pgfpathlineto{\pgfqpoint{3.411431in}{0.413320in}}%
\pgfpathlineto{\pgfqpoint{3.408752in}{0.413320in}}%
\pgfpathlineto{\pgfqpoint{3.406202in}{0.413320in}}%
\pgfpathlineto{\pgfqpoint{3.403394in}{0.413320in}}%
\pgfpathlineto{\pgfqpoint{3.400783in}{0.413320in}}%
\pgfpathlineto{\pgfqpoint{3.398037in}{0.413320in}}%
\pgfpathlineto{\pgfqpoint{3.395461in}{0.413320in}}%
\pgfpathlineto{\pgfqpoint{3.392681in}{0.413320in}}%
\pgfpathlineto{\pgfqpoint{3.390102in}{0.413320in}}%
\pgfpathlineto{\pgfqpoint{3.387309in}{0.413320in}}%
\pgfpathlineto{\pgfqpoint{3.384647in}{0.413320in}}%
\pgfpathlineto{\pgfqpoint{3.381959in}{0.413320in}}%
\pgfpathlineto{\pgfqpoint{3.379290in}{0.413320in}}%
\pgfpathlineto{\pgfqpoint{3.376735in}{0.413320in}}%
\pgfpathlineto{\pgfqpoint{3.373921in}{0.413320in}}%
\pgfpathlineto{\pgfqpoint{3.371357in}{0.413320in}}%
\pgfpathlineto{\pgfqpoint{3.368577in}{0.413320in}}%
\pgfpathlineto{\pgfqpoint{3.365996in}{0.413320in}}%
\pgfpathlineto{\pgfqpoint{3.363221in}{0.413320in}}%
\pgfpathlineto{\pgfqpoint{3.360620in}{0.413320in}}%
\pgfpathlineto{\pgfqpoint{3.357862in}{0.413320in}}%
\pgfpathlineto{\pgfqpoint{3.355177in}{0.413320in}}%
\pgfpathlineto{\pgfqpoint{3.352505in}{0.413320in}}%
\pgfpathlineto{\pgfqpoint{3.349828in}{0.413320in}}%
\pgfpathlineto{\pgfqpoint{3.347139in}{0.413320in}}%
\pgfpathlineto{\pgfqpoint{3.344468in}{0.413320in}}%
\pgfpathlineto{\pgfqpoint{3.341893in}{0.413320in}}%
\pgfpathlineto{\pgfqpoint{3.339101in}{0.413320in}}%
\pgfpathlineto{\pgfqpoint{3.336541in}{0.413320in}}%
\pgfpathlineto{\pgfqpoint{3.333758in}{0.413320in}}%
\pgfpathlineto{\pgfqpoint{3.331183in}{0.413320in}}%
\pgfpathlineto{\pgfqpoint{3.328401in}{0.413320in}}%
\pgfpathlineto{\pgfqpoint{3.325860in}{0.413320in}}%
\pgfpathlineto{\pgfqpoint{3.323049in}{0.413320in}}%
\pgfpathlineto{\pgfqpoint{3.320366in}{0.413320in}}%
\pgfpathlineto{\pgfqpoint{3.317688in}{0.413320in}}%
\pgfpathlineto{\pgfqpoint{3.315008in}{0.413320in}}%
\pgfpathlineto{\pgfqpoint{3.312480in}{0.413320in}}%
\pgfpathlineto{\pgfqpoint{3.309652in}{0.413320in}}%
\pgfpathlineto{\pgfqpoint{3.307104in}{0.413320in}}%
\pgfpathlineto{\pgfqpoint{3.304295in}{0.413320in}}%
\pgfpathlineto{\pgfqpoint{3.301719in}{0.413320in}}%
\pgfpathlineto{\pgfqpoint{3.298937in}{0.413320in}}%
\pgfpathlineto{\pgfqpoint{3.296376in}{0.413320in}}%
\pgfpathlineto{\pgfqpoint{3.293574in}{0.413320in}}%
\pgfpathlineto{\pgfqpoint{3.290890in}{0.413320in}}%
\pgfpathlineto{\pgfqpoint{3.288225in}{0.413320in}}%
\pgfpathlineto{\pgfqpoint{3.285534in}{0.413320in}}%
\pgfpathlineto{\pgfqpoint{3.282870in}{0.413320in}}%
\pgfpathlineto{\pgfqpoint{3.280189in}{0.413320in}}%
\pgfpathlineto{\pgfqpoint{3.277603in}{0.413320in}}%
\pgfpathlineto{\pgfqpoint{3.274831in}{0.413320in}}%
\pgfpathlineto{\pgfqpoint{3.272254in}{0.413320in}}%
\pgfpathlineto{\pgfqpoint{3.269478in}{0.413320in}}%
\pgfpathlineto{\pgfqpoint{3.266849in}{0.413320in}}%
\pgfpathlineto{\pgfqpoint{3.264119in}{0.413320in}}%
\pgfpathlineto{\pgfqpoint{3.261594in}{0.413320in}}%
\pgfpathlineto{\pgfqpoint{3.258784in}{0.413320in}}%
\pgfpathlineto{\pgfqpoint{3.256083in}{0.413320in}}%
\pgfpathlineto{\pgfqpoint{3.253404in}{0.413320in}}%
\pgfpathlineto{\pgfqpoint{3.250716in}{0.413320in}}%
\pgfpathlineto{\pgfqpoint{3.248049in}{0.413320in}}%
\pgfpathlineto{\pgfqpoint{3.245363in}{0.413320in}}%
\pgfpathlineto{\pgfqpoint{3.242807in}{0.413320in}}%
\pgfpathlineto{\pgfqpoint{3.240010in}{0.413320in}}%
\pgfpathlineto{\pgfqpoint{3.237411in}{0.413320in}}%
\pgfpathlineto{\pgfqpoint{3.234658in}{0.413320in}}%
\pgfpathlineto{\pgfqpoint{3.232069in}{0.413320in}}%
\pgfpathlineto{\pgfqpoint{3.229310in}{0.413320in}}%
\pgfpathlineto{\pgfqpoint{3.226609in}{0.413320in}}%
\pgfpathlineto{\pgfqpoint{3.223942in}{0.413320in}}%
\pgfpathlineto{\pgfqpoint{3.221255in}{0.413320in}}%
\pgfpathlineto{\pgfqpoint{3.218586in}{0.413320in}}%
\pgfpathlineto{\pgfqpoint{3.215908in}{0.413320in}}%
\pgfpathlineto{\pgfqpoint{3.213342in}{0.413320in}}%
\pgfpathlineto{\pgfqpoint{3.210545in}{0.413320in}}%
\pgfpathlineto{\pgfqpoint{3.207984in}{0.413320in}}%
\pgfpathlineto{\pgfqpoint{3.205195in}{0.413320in}}%
\pgfpathlineto{\pgfqpoint{3.202562in}{0.413320in}}%
\pgfpathlineto{\pgfqpoint{3.199823in}{0.413320in}}%
\pgfpathlineto{\pgfqpoint{3.197226in}{0.413320in}}%
\pgfpathlineto{\pgfqpoint{3.194508in}{0.413320in}}%
\pgfpathlineto{\pgfqpoint{3.191796in}{0.413320in}}%
\pgfpathlineto{\pgfqpoint{3.189117in}{0.413320in}}%
\pgfpathlineto{\pgfqpoint{3.186440in}{0.413320in}}%
\pgfpathlineto{\pgfqpoint{3.183760in}{0.413320in}}%
\pgfpathlineto{\pgfqpoint{3.181089in}{0.413320in}}%
\pgfpathlineto{\pgfqpoint{3.178525in}{0.413320in}}%
\pgfpathlineto{\pgfqpoint{3.175724in}{0.413320in}}%
\pgfpathlineto{\pgfqpoint{3.173142in}{0.413320in}}%
\pgfpathlineto{\pgfqpoint{3.170375in}{0.413320in}}%
\pgfpathlineto{\pgfqpoint{3.167776in}{0.413320in}}%
\pgfpathlineto{\pgfqpoint{3.165019in}{0.413320in}}%
\pgfpathlineto{\pgfqpoint{3.162474in}{0.413320in}}%
\pgfpathlineto{\pgfqpoint{3.159675in}{0.413320in}}%
\pgfpathlineto{\pgfqpoint{3.156981in}{0.413320in}}%
\pgfpathlineto{\pgfqpoint{3.154327in}{0.413320in}}%
\pgfpathlineto{\pgfqpoint{3.151612in}{0.413320in}}%
\pgfpathlineto{\pgfqpoint{3.149057in}{0.413320in}}%
\pgfpathlineto{\pgfqpoint{3.146271in}{0.413320in}}%
\pgfpathlineto{\pgfqpoint{3.143740in}{0.413320in}}%
\pgfpathlineto{\pgfqpoint{3.140913in}{0.413320in}}%
\pgfpathlineto{\pgfqpoint{3.138375in}{0.413320in}}%
\pgfpathlineto{\pgfqpoint{3.135550in}{0.413320in}}%
\pgfpathlineto{\pgfqpoint{3.132946in}{0.413320in}}%
\pgfpathlineto{\pgfqpoint{3.130199in}{0.413320in}}%
\pgfpathlineto{\pgfqpoint{3.127512in}{0.413320in}}%
\pgfpathlineto{\pgfqpoint{3.124842in}{0.413320in}}%
\pgfpathlineto{\pgfqpoint{3.122163in}{0.413320in}}%
\pgfpathlineto{\pgfqpoint{3.119487in}{0.413320in}}%
\pgfpathlineto{\pgfqpoint{3.116807in}{0.413320in}}%
\pgfpathlineto{\pgfqpoint{3.114242in}{0.413320in}}%
\pgfpathlineto{\pgfqpoint{3.111451in}{0.413320in}}%
\pgfpathlineto{\pgfqpoint{3.108896in}{0.413320in}}%
\pgfpathlineto{\pgfqpoint{3.106094in}{0.413320in}}%
\pgfpathlineto{\pgfqpoint{3.103508in}{0.413320in}}%
\pgfpathlineto{\pgfqpoint{3.100737in}{0.413320in}}%
\pgfpathlineto{\pgfqpoint{3.098163in}{0.413320in}}%
\pgfpathlineto{\pgfqpoint{3.095388in}{0.413320in}}%
\pgfpathlineto{\pgfqpoint{3.092699in}{0.413320in}}%
\pgfpathlineto{\pgfqpoint{3.090023in}{0.413320in}}%
\pgfpathlineto{\pgfqpoint{3.087343in}{0.413320in}}%
\pgfpathlineto{\pgfqpoint{3.084671in}{0.413320in}}%
\pgfpathlineto{\pgfqpoint{3.081990in}{0.413320in}}%
\pgfpathlineto{\pgfqpoint{3.079381in}{0.413320in}}%
\pgfpathlineto{\pgfqpoint{3.076631in}{0.413320in}}%
\pgfpathlineto{\pgfqpoint{3.074056in}{0.413320in}}%
\pgfpathlineto{\pgfqpoint{3.071266in}{0.413320in}}%
\pgfpathlineto{\pgfqpoint{3.068709in}{0.413320in}}%
\pgfpathlineto{\pgfqpoint{3.065916in}{0.413320in}}%
\pgfpathlineto{\pgfqpoint{3.063230in}{0.413320in}}%
\pgfpathlineto{\pgfqpoint{3.060561in}{0.413320in}}%
\pgfpathlineto{\pgfqpoint{3.057884in}{0.413320in}}%
\pgfpathlineto{\pgfqpoint{3.055202in}{0.413320in}}%
\pgfpathlineto{\pgfqpoint{3.052526in}{0.413320in}}%
\pgfpathlineto{\pgfqpoint{3.049988in}{0.413320in}}%
\pgfpathlineto{\pgfqpoint{3.047157in}{0.413320in}}%
\pgfpathlineto{\pgfqpoint{3.044568in}{0.413320in}}%
\pgfpathlineto{\pgfqpoint{3.041813in}{0.413320in}}%
\pgfpathlineto{\pgfqpoint{3.039262in}{0.413320in}}%
\pgfpathlineto{\pgfqpoint{3.036456in}{0.413320in}}%
\pgfpathlineto{\pgfqpoint{3.033921in}{0.413320in}}%
\pgfpathlineto{\pgfqpoint{3.031091in}{0.413320in}}%
\pgfpathlineto{\pgfqpoint{3.028412in}{0.413320in}}%
\pgfpathlineto{\pgfqpoint{3.025803in}{0.413320in}}%
\pgfpathlineto{\pgfqpoint{3.023058in}{0.413320in}}%
\pgfpathlineto{\pgfqpoint{3.020382in}{0.413320in}}%
\pgfpathlineto{\pgfqpoint{3.017707in}{0.413320in}}%
\pgfpathlineto{\pgfqpoint{3.015097in}{0.413320in}}%
\pgfpathlineto{\pgfqpoint{3.012351in}{0.413320in}}%
\pgfpathlineto{\pgfqpoint{3.009784in}{0.413320in}}%
\pgfpathlineto{\pgfqpoint{3.006993in}{0.413320in}}%
\pgfpathlineto{\pgfqpoint{3.004419in}{0.413320in}}%
\pgfpathlineto{\pgfqpoint{3.001635in}{0.413320in}}%
\pgfpathlineto{\pgfqpoint{2.999103in}{0.413320in}}%
\pgfpathlineto{\pgfqpoint{2.996300in}{0.413320in}}%
\pgfpathlineto{\pgfqpoint{2.993595in}{0.413320in}}%
\pgfpathlineto{\pgfqpoint{2.990978in}{0.413320in}}%
\pgfpathlineto{\pgfqpoint{2.988238in}{0.413320in}}%
\pgfpathlineto{\pgfqpoint{2.985666in}{0.413320in}}%
\pgfpathlineto{\pgfqpoint{2.982885in}{0.413320in}}%
\pgfpathlineto{\pgfqpoint{2.980341in}{0.413320in}}%
\pgfpathlineto{\pgfqpoint{2.977517in}{0.413320in}}%
\pgfpathlineto{\pgfqpoint{2.974972in}{0.413320in}}%
\pgfpathlineto{\pgfqpoint{2.972177in}{0.413320in}}%
\pgfpathlineto{\pgfqpoint{2.969599in}{0.413320in}}%
\pgfpathlineto{\pgfqpoint{2.966812in}{0.413320in}}%
\pgfpathlineto{\pgfqpoint{2.964127in}{0.413320in}}%
\pgfpathlineto{\pgfqpoint{2.961460in}{0.413320in}}%
\pgfpathlineto{\pgfqpoint{2.958782in}{0.413320in}}%
\pgfpathlineto{\pgfqpoint{2.956103in}{0.413320in}}%
\pgfpathlineto{\pgfqpoint{2.953422in}{0.413320in}}%
\pgfpathlineto{\pgfqpoint{2.950884in}{0.413320in}}%
\pgfpathlineto{\pgfqpoint{2.948068in}{0.413320in}}%
\pgfpathlineto{\pgfqpoint{2.945461in}{0.413320in}}%
\pgfpathlineto{\pgfqpoint{2.942711in}{0.413320in}}%
\pgfpathlineto{\pgfqpoint{2.940120in}{0.413320in}}%
\pgfpathlineto{\pgfqpoint{2.937352in}{0.413320in}}%
\pgfpathlineto{\pgfqpoint{2.934759in}{0.413320in}}%
\pgfpathlineto{\pgfqpoint{2.932033in}{0.413320in}}%
\pgfpathlineto{\pgfqpoint{2.929321in}{0.413320in}}%
\pgfpathlineto{\pgfqpoint{2.926655in}{0.413320in}}%
\pgfpathlineto{\pgfqpoint{2.923963in}{0.413320in}}%
\pgfpathlineto{\pgfqpoint{2.921363in}{0.413320in}}%
\pgfpathlineto{\pgfqpoint{2.918606in}{0.413320in}}%
\pgfpathlineto{\pgfqpoint{2.916061in}{0.413320in}}%
\pgfpathlineto{\pgfqpoint{2.913243in}{0.413320in}}%
\pgfpathlineto{\pgfqpoint{2.910631in}{0.413320in}}%
\pgfpathlineto{\pgfqpoint{2.907882in}{0.413320in}}%
\pgfpathlineto{\pgfqpoint{2.905341in}{0.413320in}}%
\pgfpathlineto{\pgfqpoint{2.902535in}{0.413320in}}%
\pgfpathlineto{\pgfqpoint{2.899858in}{0.413320in}}%
\pgfpathlineto{\pgfqpoint{2.897179in}{0.413320in}}%
\pgfpathlineto{\pgfqpoint{2.894487in}{0.413320in}}%
\pgfpathlineto{\pgfqpoint{2.891809in}{0.413320in}}%
\pgfpathlineto{\pgfqpoint{2.889145in}{0.413320in}}%
\pgfpathlineto{\pgfqpoint{2.886578in}{0.413320in}}%
\pgfpathlineto{\pgfqpoint{2.883780in}{0.413320in}}%
\pgfpathlineto{\pgfqpoint{2.881254in}{0.413320in}}%
\pgfpathlineto{\pgfqpoint{2.878431in}{0.413320in}}%
\pgfpathlineto{\pgfqpoint{2.875882in}{0.413320in}}%
\pgfpathlineto{\pgfqpoint{2.873074in}{0.413320in}}%
\pgfpathlineto{\pgfqpoint{2.870475in}{0.413320in}}%
\pgfpathlineto{\pgfqpoint{2.867713in}{0.413320in}}%
\pgfpathlineto{\pgfqpoint{2.865031in}{0.413320in}}%
\pgfpathlineto{\pgfqpoint{2.862402in}{0.413320in}}%
\pgfpathlineto{\pgfqpoint{2.859668in}{0.413320in}}%
\pgfpathlineto{\pgfqpoint{2.857003in}{0.413320in}}%
\pgfpathlineto{\pgfqpoint{2.854325in}{0.413320in}}%
\pgfpathlineto{\pgfqpoint{2.851793in}{0.413320in}}%
\pgfpathlineto{\pgfqpoint{2.848960in}{0.413320in}}%
\pgfpathlineto{\pgfqpoint{2.846408in}{0.413320in}}%
\pgfpathlineto{\pgfqpoint{2.843611in}{0.413320in}}%
\pgfpathlineto{\pgfqpoint{2.841055in}{0.413320in}}%
\pgfpathlineto{\pgfqpoint{2.838254in}{0.413320in}}%
\pgfpathlineto{\pgfqpoint{2.835698in}{0.413320in}}%
\pgfpathlineto{\pgfqpoint{2.832894in}{0.413320in}}%
\pgfpathlineto{\pgfqpoint{2.830219in}{0.413320in}}%
\pgfpathlineto{\pgfqpoint{2.827567in}{0.413320in}}%
\pgfpathlineto{\pgfqpoint{2.824851in}{0.413320in}}%
\pgfpathlineto{\pgfqpoint{2.822303in}{0.413320in}}%
\pgfpathlineto{\pgfqpoint{2.819506in}{0.413320in}}%
\pgfpathlineto{\pgfqpoint{2.816867in}{0.413320in}}%
\pgfpathlineto{\pgfqpoint{2.814141in}{0.413320in}}%
\pgfpathlineto{\pgfqpoint{2.811597in}{0.413320in}}%
\pgfpathlineto{\pgfqpoint{2.808792in}{0.413320in}}%
\pgfpathlineto{\pgfqpoint{2.806175in}{0.413320in}}%
\pgfpathlineto{\pgfqpoint{2.803435in}{0.413320in}}%
\pgfpathlineto{\pgfqpoint{2.800756in}{0.413320in}}%
\pgfpathlineto{\pgfqpoint{2.798070in}{0.413320in}}%
\pgfpathlineto{\pgfqpoint{2.795398in}{0.413320in}}%
\pgfpathlineto{\pgfqpoint{2.792721in}{0.413320in}}%
\pgfpathlineto{\pgfqpoint{2.790044in}{0.413320in}}%
\pgfpathlineto{\pgfqpoint{2.787468in}{0.413320in}}%
\pgfpathlineto{\pgfqpoint{2.784687in}{0.413320in}}%
\pgfpathlineto{\pgfqpoint{2.782113in}{0.413320in}}%
\pgfpathlineto{\pgfqpoint{2.779330in}{0.413320in}}%
\pgfpathlineto{\pgfqpoint{2.776767in}{0.413320in}}%
\pgfpathlineto{\pgfqpoint{2.773972in}{0.413320in}}%
\pgfpathlineto{\pgfqpoint{2.771373in}{0.413320in}}%
\pgfpathlineto{\pgfqpoint{2.768617in}{0.413320in}}%
\pgfpathlineto{\pgfqpoint{2.765935in}{0.413320in}}%
\pgfpathlineto{\pgfqpoint{2.763253in}{0.413320in}}%
\pgfpathlineto{\pgfqpoint{2.760581in}{0.413320in}}%
\pgfpathlineto{\pgfqpoint{2.758028in}{0.413320in}}%
\pgfpathlineto{\pgfqpoint{2.755224in}{0.413320in}}%
\pgfpathlineto{\pgfqpoint{2.752614in}{0.413320in}}%
\pgfpathlineto{\pgfqpoint{2.749868in}{0.413320in}}%
\pgfpathlineto{\pgfqpoint{2.747260in}{0.413320in}}%
\pgfpathlineto{\pgfqpoint{2.744510in}{0.413320in}}%
\pgfpathlineto{\pgfqpoint{2.741928in}{0.413320in}}%
\pgfpathlineto{\pgfqpoint{2.739155in}{0.413320in}}%
\pgfpathlineto{\pgfqpoint{2.736476in}{0.413320in}}%
\pgfpathlineto{\pgfqpoint{2.733798in}{0.413320in}}%
\pgfpathlineto{\pgfqpoint{2.731119in}{0.413320in}}%
\pgfpathlineto{\pgfqpoint{2.728439in}{0.413320in}}%
\pgfpathlineto{\pgfqpoint{2.725760in}{0.413320in}}%
\pgfpathlineto{\pgfqpoint{2.723211in}{0.413320in}}%
\pgfpathlineto{\pgfqpoint{2.720404in}{0.413320in}}%
\pgfpathlineto{\pgfqpoint{2.717773in}{0.413320in}}%
\pgfpathlineto{\pgfqpoint{2.715036in}{0.413320in}}%
\pgfpathlineto{\pgfqpoint{2.712477in}{0.413320in}}%
\pgfpathlineto{\pgfqpoint{2.709683in}{0.413320in}}%
\pgfpathlineto{\pgfqpoint{2.707125in}{0.413320in}}%
\pgfpathlineto{\pgfqpoint{2.704326in}{0.413320in}}%
\pgfpathlineto{\pgfqpoint{2.701657in}{0.413320in}}%
\pgfpathlineto{\pgfqpoint{2.698968in}{0.413320in}}%
\pgfpathlineto{\pgfqpoint{2.696293in}{0.413320in}}%
\pgfpathlineto{\pgfqpoint{2.693611in}{0.413320in}}%
\pgfpathlineto{\pgfqpoint{2.690940in}{0.413320in}}%
\pgfpathlineto{\pgfqpoint{2.688328in}{0.413320in}}%
\pgfpathlineto{\pgfqpoint{2.685586in}{0.413320in}}%
\pgfpathlineto{\pgfqpoint{2.683009in}{0.413320in}}%
\pgfpathlineto{\pgfqpoint{2.680224in}{0.413320in}}%
\pgfpathlineto{\pgfqpoint{2.677650in}{0.413320in}}%
\pgfpathlineto{\pgfqpoint{2.674873in}{0.413320in}}%
\pgfpathlineto{\pgfqpoint{2.672301in}{0.413320in}}%
\pgfpathlineto{\pgfqpoint{2.669506in}{0.413320in}}%
\pgfpathlineto{\pgfqpoint{2.666836in}{0.413320in}}%
\pgfpathlineto{\pgfqpoint{2.664151in}{0.413320in}}%
\pgfpathlineto{\pgfqpoint{2.661481in}{0.413320in}}%
\pgfpathlineto{\pgfqpoint{2.658942in}{0.413320in}}%
\pgfpathlineto{\pgfqpoint{2.656124in}{0.413320in}}%
\pgfpathlineto{\pgfqpoint{2.653567in}{0.413320in}}%
\pgfpathlineto{\pgfqpoint{2.650767in}{0.413320in}}%
\pgfpathlineto{\pgfqpoint{2.648196in}{0.413320in}}%
\pgfpathlineto{\pgfqpoint{2.645408in}{0.413320in}}%
\pgfpathlineto{\pgfqpoint{2.642827in}{0.413320in}}%
\pgfpathlineto{\pgfqpoint{2.640053in}{0.413320in}}%
\pgfpathlineto{\pgfqpoint{2.637369in}{0.413320in}}%
\pgfpathlineto{\pgfqpoint{2.634700in}{0.413320in}}%
\pgfpathlineto{\pgfqpoint{2.632018in}{0.413320in}}%
\pgfpathlineto{\pgfqpoint{2.629340in}{0.413320in}}%
\pgfpathlineto{\pgfqpoint{2.626653in}{0.413320in}}%
\pgfpathlineto{\pgfqpoint{2.624077in}{0.413320in}}%
\pgfpathlineto{\pgfqpoint{2.621304in}{0.413320in}}%
\pgfpathlineto{\pgfqpoint{2.618773in}{0.413320in}}%
\pgfpathlineto{\pgfqpoint{2.615934in}{0.413320in}}%
\pgfpathlineto{\pgfqpoint{2.613393in}{0.413320in}}%
\pgfpathlineto{\pgfqpoint{2.610588in}{0.413320in}}%
\pgfpathlineto{\pgfqpoint{2.608004in}{0.413320in}}%
\pgfpathlineto{\pgfqpoint{2.605232in}{0.413320in}}%
\pgfpathlineto{\pgfqpoint{2.602557in}{0.413320in}}%
\pgfpathlineto{\pgfqpoint{2.599920in}{0.413320in}}%
\pgfpathlineto{\pgfqpoint{2.597196in}{0.413320in}}%
\pgfpathlineto{\pgfqpoint{2.594630in}{0.413320in}}%
\pgfpathlineto{\pgfqpoint{2.591842in}{0.413320in}}%
\pgfpathlineto{\pgfqpoint{2.589248in}{0.413320in}}%
\pgfpathlineto{\pgfqpoint{2.586484in}{0.413320in}}%
\pgfpathlineto{\pgfqpoint{2.583913in}{0.413320in}}%
\pgfpathlineto{\pgfqpoint{2.581129in}{0.413320in}}%
\pgfpathlineto{\pgfqpoint{2.578567in}{0.413320in}}%
\pgfpathlineto{\pgfqpoint{2.575779in}{0.413320in}}%
\pgfpathlineto{\pgfqpoint{2.573082in}{0.413320in}}%
\pgfpathlineto{\pgfqpoint{2.570411in}{0.413320in}}%
\pgfpathlineto{\pgfqpoint{2.567730in}{0.413320in}}%
\pgfpathlineto{\pgfqpoint{2.565045in}{0.413320in}}%
\pgfpathlineto{\pgfqpoint{2.562375in}{0.413320in}}%
\pgfpathlineto{\pgfqpoint{2.559790in}{0.413320in}}%
\pgfpathlineto{\pgfqpoint{2.557009in}{0.413320in}}%
\pgfpathlineto{\pgfqpoint{2.554493in}{0.413320in}}%
\pgfpathlineto{\pgfqpoint{2.551664in}{0.413320in}}%
\pgfpathlineto{\pgfqpoint{2.549114in}{0.413320in}}%
\pgfpathlineto{\pgfqpoint{2.546310in}{0.413320in}}%
\pgfpathlineto{\pgfqpoint{2.543765in}{0.413320in}}%
\pgfpathlineto{\pgfqpoint{2.540949in}{0.413320in}}%
\pgfpathlineto{\pgfqpoint{2.538274in}{0.413320in}}%
\pgfpathlineto{\pgfqpoint{2.535624in}{0.413320in}}%
\pgfpathlineto{\pgfqpoint{2.532917in}{0.413320in}}%
\pgfpathlineto{\pgfqpoint{2.530234in}{0.413320in}}%
\pgfpathlineto{\pgfqpoint{2.527560in}{0.413320in}}%
\pgfpathlineto{\pgfqpoint{2.524988in}{0.413320in}}%
\pgfpathlineto{\pgfqpoint{2.522197in}{0.413320in}}%
\pgfpathlineto{\pgfqpoint{2.519607in}{0.413320in}}%
\pgfpathlineto{\pgfqpoint{2.516845in}{0.413320in}}%
\pgfpathlineto{\pgfqpoint{2.514268in}{0.413320in}}%
\pgfpathlineto{\pgfqpoint{2.511478in}{0.413320in}}%
\pgfpathlineto{\pgfqpoint{2.508917in}{0.413320in}}%
\pgfpathlineto{\pgfqpoint{2.506163in}{0.413320in}}%
\pgfpathlineto{\pgfqpoint{2.503454in}{0.413320in}}%
\pgfpathlineto{\pgfqpoint{2.500801in}{0.413320in}}%
\pgfpathlineto{\pgfqpoint{2.498085in}{0.413320in}}%
\pgfpathlineto{\pgfqpoint{2.495542in}{0.413320in}}%
\pgfpathlineto{\pgfqpoint{2.492729in}{0.413320in}}%
\pgfpathlineto{\pgfqpoint{2.490183in}{0.413320in}}%
\pgfpathlineto{\pgfqpoint{2.487384in}{0.413320in}}%
\pgfpathlineto{\pgfqpoint{2.484870in}{0.413320in}}%
\pgfpathlineto{\pgfqpoint{2.482026in}{0.413320in}}%
\pgfpathlineto{\pgfqpoint{2.479420in}{0.413320in}}%
\pgfpathlineto{\pgfqpoint{2.476671in}{0.413320in}}%
\pgfpathlineto{\pgfqpoint{2.473989in}{0.413320in}}%
\pgfpathlineto{\pgfqpoint{2.471311in}{0.413320in}}%
\pgfpathlineto{\pgfqpoint{2.468635in}{0.413320in}}%
\pgfpathlineto{\pgfqpoint{2.465957in}{0.413320in}}%
\pgfpathlineto{\pgfqpoint{2.463280in}{0.413320in}}%
\pgfpathlineto{\pgfqpoint{2.460711in}{0.413320in}}%
\pgfpathlineto{\pgfqpoint{2.457917in}{0.413320in}}%
\pgfpathlineto{\pgfqpoint{2.455353in}{0.413320in}}%
\pgfpathlineto{\pgfqpoint{2.452562in}{0.413320in}}%
\pgfpathlineto{\pgfqpoint{2.450032in}{0.413320in}}%
\pgfpathlineto{\pgfqpoint{2.447209in}{0.413320in}}%
\pgfpathlineto{\pgfqpoint{2.444677in}{0.413320in}}%
\pgfpathlineto{\pgfqpoint{2.441876in}{0.413320in}}%
\pgfpathlineto{\pgfqpoint{2.439167in}{0.413320in}}%
\pgfpathlineto{\pgfqpoint{2.436518in}{0.413320in}}%
\pgfpathlineto{\pgfqpoint{2.433815in}{0.413320in}}%
\pgfpathlineto{\pgfqpoint{2.431251in}{0.413320in}}%
\pgfpathlineto{\pgfqpoint{2.428453in}{0.413320in}}%
\pgfpathlineto{\pgfqpoint{2.425878in}{0.413320in}}%
\pgfpathlineto{\pgfqpoint{2.423098in}{0.413320in}}%
\pgfpathlineto{\pgfqpoint{2.420528in}{0.413320in}}%
\pgfpathlineto{\pgfqpoint{2.417747in}{0.413320in}}%
\pgfpathlineto{\pgfqpoint{2.415184in}{0.413320in}}%
\pgfpathlineto{\pgfqpoint{2.412389in}{0.413320in}}%
\pgfpathlineto{\pgfqpoint{2.409699in}{0.413320in}}%
\pgfpathlineto{\pgfqpoint{2.407024in}{0.413320in}}%
\pgfpathlineto{\pgfqpoint{2.404352in}{0.413320in}}%
\pgfpathlineto{\pgfqpoint{2.401675in}{0.413320in}}%
\pgfpathlineto{\pgfqpoint{2.398995in}{0.413320in}}%
\pgfpathclose%
\pgfusepath{stroke,fill}%
\end{pgfscope}%
\begin{pgfscope}%
\pgfpathrectangle{\pgfqpoint{2.398995in}{0.319877in}}{\pgfqpoint{3.986877in}{1.993438in}} %
\pgfusepath{clip}%
\pgfsetbuttcap%
\pgfsetroundjoin%
\definecolor{currentfill}{rgb}{1.000000,1.000000,1.000000}%
\pgfsetfillcolor{currentfill}%
\pgfsetlinewidth{1.003750pt}%
\definecolor{currentstroke}{rgb}{0.216630,0.667659,0.731870}%
\pgfsetstrokecolor{currentstroke}%
\pgfsetdash{}{0pt}%
\pgfpathmoveto{\pgfqpoint{2.398995in}{0.413320in}}%
\pgfpathlineto{\pgfqpoint{2.398995in}{1.357554in}}%
\pgfpathlineto{\pgfqpoint{2.401675in}{1.359763in}}%
\pgfpathlineto{\pgfqpoint{2.404352in}{1.355141in}}%
\pgfpathlineto{\pgfqpoint{2.407024in}{1.362862in}}%
\pgfpathlineto{\pgfqpoint{2.409699in}{1.351010in}}%
\pgfpathlineto{\pgfqpoint{2.412389in}{1.353550in}}%
\pgfpathlineto{\pgfqpoint{2.415184in}{1.355900in}}%
\pgfpathlineto{\pgfqpoint{2.417747in}{1.356814in}}%
\pgfpathlineto{\pgfqpoint{2.420528in}{1.360869in}}%
\pgfpathlineto{\pgfqpoint{2.423098in}{1.360159in}}%
\pgfpathlineto{\pgfqpoint{2.425878in}{1.364077in}}%
\pgfpathlineto{\pgfqpoint{2.428453in}{1.350283in}}%
\pgfpathlineto{\pgfqpoint{2.431251in}{1.348379in}}%
\pgfpathlineto{\pgfqpoint{2.433815in}{1.351139in}}%
\pgfpathlineto{\pgfqpoint{2.436518in}{1.357623in}}%
\pgfpathlineto{\pgfqpoint{2.439167in}{1.356058in}}%
\pgfpathlineto{\pgfqpoint{2.441876in}{1.367528in}}%
\pgfpathlineto{\pgfqpoint{2.444677in}{1.365366in}}%
\pgfpathlineto{\pgfqpoint{2.447209in}{1.359820in}}%
\pgfpathlineto{\pgfqpoint{2.450032in}{1.360585in}}%
\pgfpathlineto{\pgfqpoint{2.452562in}{1.361293in}}%
\pgfpathlineto{\pgfqpoint{2.455353in}{1.363413in}}%
\pgfpathlineto{\pgfqpoint{2.457917in}{1.357130in}}%
\pgfpathlineto{\pgfqpoint{2.460711in}{1.358425in}}%
\pgfpathlineto{\pgfqpoint{2.463280in}{1.353450in}}%
\pgfpathlineto{\pgfqpoint{2.465957in}{1.365860in}}%
\pgfpathlineto{\pgfqpoint{2.468635in}{1.360783in}}%
\pgfpathlineto{\pgfqpoint{2.471311in}{1.364224in}}%
\pgfpathlineto{\pgfqpoint{2.473989in}{1.360404in}}%
\pgfpathlineto{\pgfqpoint{2.476671in}{1.361136in}}%
\pgfpathlineto{\pgfqpoint{2.479420in}{1.358105in}}%
\pgfpathlineto{\pgfqpoint{2.482026in}{1.359312in}}%
\pgfpathlineto{\pgfqpoint{2.484870in}{1.366358in}}%
\pgfpathlineto{\pgfqpoint{2.487384in}{1.355681in}}%
\pgfpathlineto{\pgfqpoint{2.490183in}{1.354171in}}%
\pgfpathlineto{\pgfqpoint{2.492729in}{1.361333in}}%
\pgfpathlineto{\pgfqpoint{2.495542in}{1.358101in}}%
\pgfpathlineto{\pgfqpoint{2.498085in}{1.357788in}}%
\pgfpathlineto{\pgfqpoint{2.500801in}{1.358429in}}%
\pgfpathlineto{\pgfqpoint{2.503454in}{1.359977in}}%
\pgfpathlineto{\pgfqpoint{2.506163in}{1.358660in}}%
\pgfpathlineto{\pgfqpoint{2.508917in}{1.358547in}}%
\pgfpathlineto{\pgfqpoint{2.511478in}{1.359574in}}%
\pgfpathlineto{\pgfqpoint{2.514268in}{1.363836in}}%
\pgfpathlineto{\pgfqpoint{2.516845in}{1.352321in}}%
\pgfpathlineto{\pgfqpoint{2.519607in}{1.358104in}}%
\pgfpathlineto{\pgfqpoint{2.522197in}{1.358893in}}%
\pgfpathlineto{\pgfqpoint{2.524988in}{1.358785in}}%
\pgfpathlineto{\pgfqpoint{2.527560in}{1.358625in}}%
\pgfpathlineto{\pgfqpoint{2.530234in}{1.354943in}}%
\pgfpathlineto{\pgfqpoint{2.532917in}{1.356244in}}%
\pgfpathlineto{\pgfqpoint{2.535624in}{1.357589in}}%
\pgfpathlineto{\pgfqpoint{2.538274in}{1.358165in}}%
\pgfpathlineto{\pgfqpoint{2.540949in}{1.357556in}}%
\pgfpathlineto{\pgfqpoint{2.543765in}{1.356770in}}%
\pgfpathlineto{\pgfqpoint{2.546310in}{1.359355in}}%
\pgfpathlineto{\pgfqpoint{2.549114in}{1.361553in}}%
\pgfpathlineto{\pgfqpoint{2.551664in}{1.361896in}}%
\pgfpathlineto{\pgfqpoint{2.554493in}{1.359277in}}%
\pgfpathlineto{\pgfqpoint{2.557009in}{1.355488in}}%
\pgfpathlineto{\pgfqpoint{2.559790in}{1.351334in}}%
\pgfpathlineto{\pgfqpoint{2.562375in}{1.357402in}}%
\pgfpathlineto{\pgfqpoint{2.565045in}{1.356000in}}%
\pgfpathlineto{\pgfqpoint{2.567730in}{1.356913in}}%
\pgfpathlineto{\pgfqpoint{2.570411in}{1.366066in}}%
\pgfpathlineto{\pgfqpoint{2.573082in}{1.366459in}}%
\pgfpathlineto{\pgfqpoint{2.575779in}{1.370686in}}%
\pgfpathlineto{\pgfqpoint{2.578567in}{1.357557in}}%
\pgfpathlineto{\pgfqpoint{2.581129in}{1.354168in}}%
\pgfpathlineto{\pgfqpoint{2.583913in}{1.352041in}}%
\pgfpathlineto{\pgfqpoint{2.586484in}{1.350950in}}%
\pgfpathlineto{\pgfqpoint{2.589248in}{1.353539in}}%
\pgfpathlineto{\pgfqpoint{2.591842in}{1.354073in}}%
\pgfpathlineto{\pgfqpoint{2.594630in}{1.356383in}}%
\pgfpathlineto{\pgfqpoint{2.597196in}{1.354853in}}%
\pgfpathlineto{\pgfqpoint{2.599920in}{1.355155in}}%
\pgfpathlineto{\pgfqpoint{2.602557in}{1.352056in}}%
\pgfpathlineto{\pgfqpoint{2.605232in}{1.346787in}}%
\pgfpathlineto{\pgfqpoint{2.608004in}{1.348686in}}%
\pgfpathlineto{\pgfqpoint{2.610588in}{1.345819in}}%
\pgfpathlineto{\pgfqpoint{2.613393in}{1.346415in}}%
\pgfpathlineto{\pgfqpoint{2.615934in}{1.341047in}}%
\pgfpathlineto{\pgfqpoint{2.618773in}{1.345761in}}%
\pgfpathlineto{\pgfqpoint{2.621304in}{1.347220in}}%
\pgfpathlineto{\pgfqpoint{2.624077in}{1.353073in}}%
\pgfpathlineto{\pgfqpoint{2.626653in}{1.352107in}}%
\pgfpathlineto{\pgfqpoint{2.629340in}{1.349798in}}%
\pgfpathlineto{\pgfqpoint{2.632018in}{1.352997in}}%
\pgfpathlineto{\pgfqpoint{2.634700in}{1.354082in}}%
\pgfpathlineto{\pgfqpoint{2.637369in}{1.358280in}}%
\pgfpathlineto{\pgfqpoint{2.640053in}{1.356883in}}%
\pgfpathlineto{\pgfqpoint{2.642827in}{1.356913in}}%
\pgfpathlineto{\pgfqpoint{2.645408in}{1.361359in}}%
\pgfpathlineto{\pgfqpoint{2.648196in}{1.358879in}}%
\pgfpathlineto{\pgfqpoint{2.650767in}{1.354979in}}%
\pgfpathlineto{\pgfqpoint{2.653567in}{1.349813in}}%
\pgfpathlineto{\pgfqpoint{2.656124in}{1.349301in}}%
\pgfpathlineto{\pgfqpoint{2.658942in}{1.346020in}}%
\pgfpathlineto{\pgfqpoint{2.661481in}{1.344623in}}%
\pgfpathlineto{\pgfqpoint{2.664151in}{1.340716in}}%
\pgfpathlineto{\pgfqpoint{2.666836in}{1.338973in}}%
\pgfpathlineto{\pgfqpoint{2.669506in}{1.346449in}}%
\pgfpathlineto{\pgfqpoint{2.672301in}{1.349963in}}%
\pgfpathlineto{\pgfqpoint{2.674873in}{1.356426in}}%
\pgfpathlineto{\pgfqpoint{2.677650in}{1.359607in}}%
\pgfpathlineto{\pgfqpoint{2.680224in}{1.360201in}}%
\pgfpathlineto{\pgfqpoint{2.683009in}{1.363029in}}%
\pgfpathlineto{\pgfqpoint{2.685586in}{1.357135in}}%
\pgfpathlineto{\pgfqpoint{2.688328in}{1.361826in}}%
\pgfpathlineto{\pgfqpoint{2.690940in}{1.353541in}}%
\pgfpathlineto{\pgfqpoint{2.693611in}{1.358925in}}%
\pgfpathlineto{\pgfqpoint{2.696293in}{1.362104in}}%
\pgfpathlineto{\pgfqpoint{2.698968in}{1.359749in}}%
\pgfpathlineto{\pgfqpoint{2.701657in}{1.358644in}}%
\pgfpathlineto{\pgfqpoint{2.704326in}{1.356426in}}%
\pgfpathlineto{\pgfqpoint{2.707125in}{1.359336in}}%
\pgfpathlineto{\pgfqpoint{2.709683in}{1.361507in}}%
\pgfpathlineto{\pgfqpoint{2.712477in}{1.353849in}}%
\pgfpathlineto{\pgfqpoint{2.715036in}{1.362957in}}%
\pgfpathlineto{\pgfqpoint{2.717773in}{1.362065in}}%
\pgfpathlineto{\pgfqpoint{2.720404in}{1.355586in}}%
\pgfpathlineto{\pgfqpoint{2.723211in}{1.359541in}}%
\pgfpathlineto{\pgfqpoint{2.725760in}{1.356345in}}%
\pgfpathlineto{\pgfqpoint{2.728439in}{1.358633in}}%
\pgfpathlineto{\pgfqpoint{2.731119in}{1.349713in}}%
\pgfpathlineto{\pgfqpoint{2.733798in}{1.351432in}}%
\pgfpathlineto{\pgfqpoint{2.736476in}{1.340544in}}%
\pgfpathlineto{\pgfqpoint{2.739155in}{1.343115in}}%
\pgfpathlineto{\pgfqpoint{2.741928in}{1.346441in}}%
\pgfpathlineto{\pgfqpoint{2.744510in}{1.347560in}}%
\pgfpathlineto{\pgfqpoint{2.747260in}{1.351187in}}%
\pgfpathlineto{\pgfqpoint{2.749868in}{1.349710in}}%
\pgfpathlineto{\pgfqpoint{2.752614in}{1.352263in}}%
\pgfpathlineto{\pgfqpoint{2.755224in}{1.355150in}}%
\pgfpathlineto{\pgfqpoint{2.758028in}{1.347989in}}%
\pgfpathlineto{\pgfqpoint{2.760581in}{1.343824in}}%
\pgfpathlineto{\pgfqpoint{2.763253in}{1.344675in}}%
\pgfpathlineto{\pgfqpoint{2.765935in}{1.340516in}}%
\pgfpathlineto{\pgfqpoint{2.768617in}{1.341943in}}%
\pgfpathlineto{\pgfqpoint{2.771373in}{1.347247in}}%
\pgfpathlineto{\pgfqpoint{2.773972in}{1.353987in}}%
\pgfpathlineto{\pgfqpoint{2.776767in}{1.350987in}}%
\pgfpathlineto{\pgfqpoint{2.779330in}{1.350385in}}%
\pgfpathlineto{\pgfqpoint{2.782113in}{1.349379in}}%
\pgfpathlineto{\pgfqpoint{2.784687in}{1.351802in}}%
\pgfpathlineto{\pgfqpoint{2.787468in}{1.360904in}}%
\pgfpathlineto{\pgfqpoint{2.790044in}{1.358722in}}%
\pgfpathlineto{\pgfqpoint{2.792721in}{1.360382in}}%
\pgfpathlineto{\pgfqpoint{2.795398in}{1.355958in}}%
\pgfpathlineto{\pgfqpoint{2.798070in}{1.357975in}}%
\pgfpathlineto{\pgfqpoint{2.800756in}{1.355337in}}%
\pgfpathlineto{\pgfqpoint{2.803435in}{1.360076in}}%
\pgfpathlineto{\pgfqpoint{2.806175in}{1.359563in}}%
\pgfpathlineto{\pgfqpoint{2.808792in}{1.356407in}}%
\pgfpathlineto{\pgfqpoint{2.811597in}{1.362948in}}%
\pgfpathlineto{\pgfqpoint{2.814141in}{1.363030in}}%
\pgfpathlineto{\pgfqpoint{2.816867in}{1.361470in}}%
\pgfpathlineto{\pgfqpoint{2.819506in}{1.356201in}}%
\pgfpathlineto{\pgfqpoint{2.822303in}{1.358956in}}%
\pgfpathlineto{\pgfqpoint{2.824851in}{1.353102in}}%
\pgfpathlineto{\pgfqpoint{2.827567in}{1.357668in}}%
\pgfpathlineto{\pgfqpoint{2.830219in}{1.356806in}}%
\pgfpathlineto{\pgfqpoint{2.832894in}{1.359155in}}%
\pgfpathlineto{\pgfqpoint{2.835698in}{1.351626in}}%
\pgfpathlineto{\pgfqpoint{2.838254in}{1.356864in}}%
\pgfpathlineto{\pgfqpoint{2.841055in}{1.347293in}}%
\pgfpathlineto{\pgfqpoint{2.843611in}{1.344250in}}%
\pgfpathlineto{\pgfqpoint{2.846408in}{1.345595in}}%
\pgfpathlineto{\pgfqpoint{2.848960in}{1.346221in}}%
\pgfpathlineto{\pgfqpoint{2.851793in}{1.358724in}}%
\pgfpathlineto{\pgfqpoint{2.854325in}{1.357040in}}%
\pgfpathlineto{\pgfqpoint{2.857003in}{1.358694in}}%
\pgfpathlineto{\pgfqpoint{2.859668in}{1.353547in}}%
\pgfpathlineto{\pgfqpoint{2.862402in}{1.356603in}}%
\pgfpathlineto{\pgfqpoint{2.865031in}{1.358226in}}%
\pgfpathlineto{\pgfqpoint{2.867713in}{1.362128in}}%
\pgfpathlineto{\pgfqpoint{2.870475in}{1.360927in}}%
\pgfpathlineto{\pgfqpoint{2.873074in}{1.357340in}}%
\pgfpathlineto{\pgfqpoint{2.875882in}{1.361768in}}%
\pgfpathlineto{\pgfqpoint{2.878431in}{1.358286in}}%
\pgfpathlineto{\pgfqpoint{2.881254in}{1.355356in}}%
\pgfpathlineto{\pgfqpoint{2.883780in}{1.361612in}}%
\pgfpathlineto{\pgfqpoint{2.886578in}{1.362589in}}%
\pgfpathlineto{\pgfqpoint{2.889145in}{1.353223in}}%
\pgfpathlineto{\pgfqpoint{2.891809in}{1.353157in}}%
\pgfpathlineto{\pgfqpoint{2.894487in}{1.361187in}}%
\pgfpathlineto{\pgfqpoint{2.897179in}{1.361815in}}%
\pgfpathlineto{\pgfqpoint{2.899858in}{1.358581in}}%
\pgfpathlineto{\pgfqpoint{2.902535in}{1.363465in}}%
\pgfpathlineto{\pgfqpoint{2.905341in}{1.362948in}}%
\pgfpathlineto{\pgfqpoint{2.907882in}{1.364882in}}%
\pgfpathlineto{\pgfqpoint{2.910631in}{1.366656in}}%
\pgfpathlineto{\pgfqpoint{2.913243in}{1.363295in}}%
\pgfpathlineto{\pgfqpoint{2.916061in}{1.364028in}}%
\pgfpathlineto{\pgfqpoint{2.918606in}{1.360575in}}%
\pgfpathlineto{\pgfqpoint{2.921363in}{1.356822in}}%
\pgfpathlineto{\pgfqpoint{2.923963in}{1.359530in}}%
\pgfpathlineto{\pgfqpoint{2.926655in}{1.360582in}}%
\pgfpathlineto{\pgfqpoint{2.929321in}{1.361634in}}%
\pgfpathlineto{\pgfqpoint{2.932033in}{1.364125in}}%
\pgfpathlineto{\pgfqpoint{2.934759in}{1.367282in}}%
\pgfpathlineto{\pgfqpoint{2.937352in}{1.367298in}}%
\pgfpathlineto{\pgfqpoint{2.940120in}{1.358922in}}%
\pgfpathlineto{\pgfqpoint{2.942711in}{1.359827in}}%
\pgfpathlineto{\pgfqpoint{2.945461in}{1.358125in}}%
\pgfpathlineto{\pgfqpoint{2.948068in}{1.361250in}}%
\pgfpathlineto{\pgfqpoint{2.950884in}{1.359130in}}%
\pgfpathlineto{\pgfqpoint{2.953422in}{1.365588in}}%
\pgfpathlineto{\pgfqpoint{2.956103in}{1.355274in}}%
\pgfpathlineto{\pgfqpoint{2.958782in}{1.358005in}}%
\pgfpathlineto{\pgfqpoint{2.961460in}{1.354539in}}%
\pgfpathlineto{\pgfqpoint{2.964127in}{1.355603in}}%
\pgfpathlineto{\pgfqpoint{2.966812in}{1.355884in}}%
\pgfpathlineto{\pgfqpoint{2.969599in}{1.354666in}}%
\pgfpathlineto{\pgfqpoint{2.972177in}{1.355155in}}%
\pgfpathlineto{\pgfqpoint{2.974972in}{1.352894in}}%
\pgfpathlineto{\pgfqpoint{2.977517in}{1.357258in}}%
\pgfpathlineto{\pgfqpoint{2.980341in}{1.356471in}}%
\pgfpathlineto{\pgfqpoint{2.982885in}{1.355233in}}%
\pgfpathlineto{\pgfqpoint{2.985666in}{1.353398in}}%
\pgfpathlineto{\pgfqpoint{2.988238in}{1.350777in}}%
\pgfpathlineto{\pgfqpoint{2.990978in}{1.356641in}}%
\pgfpathlineto{\pgfqpoint{2.993595in}{1.356162in}}%
\pgfpathlineto{\pgfqpoint{2.996300in}{1.353935in}}%
\pgfpathlineto{\pgfqpoint{2.999103in}{1.349024in}}%
\pgfpathlineto{\pgfqpoint{3.001635in}{1.349086in}}%
\pgfpathlineto{\pgfqpoint{3.004419in}{1.348306in}}%
\pgfpathlineto{\pgfqpoint{3.006993in}{1.354792in}}%
\pgfpathlineto{\pgfqpoint{3.009784in}{1.352277in}}%
\pgfpathlineto{\pgfqpoint{3.012351in}{1.350934in}}%
\pgfpathlineto{\pgfqpoint{3.015097in}{1.351802in}}%
\pgfpathlineto{\pgfqpoint{3.017707in}{1.350122in}}%
\pgfpathlineto{\pgfqpoint{3.020382in}{1.349497in}}%
\pgfpathlineto{\pgfqpoint{3.023058in}{1.350031in}}%
\pgfpathlineto{\pgfqpoint{3.025803in}{1.354102in}}%
\pgfpathlineto{\pgfqpoint{3.028412in}{1.353659in}}%
\pgfpathlineto{\pgfqpoint{3.031091in}{1.361053in}}%
\pgfpathlineto{\pgfqpoint{3.033921in}{1.361272in}}%
\pgfpathlineto{\pgfqpoint{3.036456in}{1.359785in}}%
\pgfpathlineto{\pgfqpoint{3.039262in}{1.355824in}}%
\pgfpathlineto{\pgfqpoint{3.041813in}{1.360844in}}%
\pgfpathlineto{\pgfqpoint{3.044568in}{1.372477in}}%
\pgfpathlineto{\pgfqpoint{3.047157in}{1.383991in}}%
\pgfpathlineto{\pgfqpoint{3.049988in}{1.397040in}}%
\pgfpathlineto{\pgfqpoint{3.052526in}{1.369033in}}%
\pgfpathlineto{\pgfqpoint{3.055202in}{1.366021in}}%
\pgfpathlineto{\pgfqpoint{3.057884in}{1.362004in}}%
\pgfpathlineto{\pgfqpoint{3.060561in}{1.360915in}}%
\pgfpathlineto{\pgfqpoint{3.063230in}{1.364566in}}%
\pgfpathlineto{\pgfqpoint{3.065916in}{1.363369in}}%
\pgfpathlineto{\pgfqpoint{3.068709in}{1.368832in}}%
\pgfpathlineto{\pgfqpoint{3.071266in}{1.359767in}}%
\pgfpathlineto{\pgfqpoint{3.074056in}{1.352225in}}%
\pgfpathlineto{\pgfqpoint{3.076631in}{1.362249in}}%
\pgfpathlineto{\pgfqpoint{3.079381in}{1.366420in}}%
\pgfpathlineto{\pgfqpoint{3.081990in}{1.379257in}}%
\pgfpathlineto{\pgfqpoint{3.084671in}{1.377126in}}%
\pgfpathlineto{\pgfqpoint{3.087343in}{1.370683in}}%
\pgfpathlineto{\pgfqpoint{3.090023in}{1.364091in}}%
\pgfpathlineto{\pgfqpoint{3.092699in}{1.358562in}}%
\pgfpathlineto{\pgfqpoint{3.095388in}{1.359982in}}%
\pgfpathlineto{\pgfqpoint{3.098163in}{1.351493in}}%
\pgfpathlineto{\pgfqpoint{3.100737in}{1.342579in}}%
\pgfpathlineto{\pgfqpoint{3.103508in}{1.351344in}}%
\pgfpathlineto{\pgfqpoint{3.106094in}{1.369576in}}%
\pgfpathlineto{\pgfqpoint{3.108896in}{1.364279in}}%
\pgfpathlineto{\pgfqpoint{3.111451in}{1.352621in}}%
\pgfpathlineto{\pgfqpoint{3.114242in}{1.356703in}}%
\pgfpathlineto{\pgfqpoint{3.116807in}{1.358114in}}%
\pgfpathlineto{\pgfqpoint{3.119487in}{1.352291in}}%
\pgfpathlineto{\pgfqpoint{3.122163in}{1.349668in}}%
\pgfpathlineto{\pgfqpoint{3.124842in}{1.353340in}}%
\pgfpathlineto{\pgfqpoint{3.127512in}{1.359821in}}%
\pgfpathlineto{\pgfqpoint{3.130199in}{1.358590in}}%
\pgfpathlineto{\pgfqpoint{3.132946in}{1.359883in}}%
\pgfpathlineto{\pgfqpoint{3.135550in}{1.356112in}}%
\pgfpathlineto{\pgfqpoint{3.138375in}{1.351351in}}%
\pgfpathlineto{\pgfqpoint{3.140913in}{1.359771in}}%
\pgfpathlineto{\pgfqpoint{3.143740in}{1.362996in}}%
\pgfpathlineto{\pgfqpoint{3.146271in}{1.365717in}}%
\pgfpathlineto{\pgfqpoint{3.149057in}{1.347750in}}%
\pgfpathlineto{\pgfqpoint{3.151612in}{1.352741in}}%
\pgfpathlineto{\pgfqpoint{3.154327in}{1.346194in}}%
\pgfpathlineto{\pgfqpoint{3.156981in}{1.346759in}}%
\pgfpathlineto{\pgfqpoint{3.159675in}{1.348428in}}%
\pgfpathlineto{\pgfqpoint{3.162474in}{1.349245in}}%
\pgfpathlineto{\pgfqpoint{3.165019in}{1.336791in}}%
\pgfpathlineto{\pgfqpoint{3.167776in}{1.334887in}}%
\pgfpathlineto{\pgfqpoint{3.170375in}{1.334887in}}%
\pgfpathlineto{\pgfqpoint{3.173142in}{1.341682in}}%
\pgfpathlineto{\pgfqpoint{3.175724in}{1.335881in}}%
\pgfpathlineto{\pgfqpoint{3.178525in}{1.344907in}}%
\pgfpathlineto{\pgfqpoint{3.181089in}{1.341828in}}%
\pgfpathlineto{\pgfqpoint{3.183760in}{1.340279in}}%
\pgfpathlineto{\pgfqpoint{3.186440in}{1.338633in}}%
\pgfpathlineto{\pgfqpoint{3.189117in}{1.337185in}}%
\pgfpathlineto{\pgfqpoint{3.191796in}{1.341249in}}%
\pgfpathlineto{\pgfqpoint{3.194508in}{1.354294in}}%
\pgfpathlineto{\pgfqpoint{3.197226in}{1.355052in}}%
\pgfpathlineto{\pgfqpoint{3.199823in}{1.358768in}}%
\pgfpathlineto{\pgfqpoint{3.202562in}{1.356555in}}%
\pgfpathlineto{\pgfqpoint{3.205195in}{1.348605in}}%
\pgfpathlineto{\pgfqpoint{3.207984in}{1.343501in}}%
\pgfpathlineto{\pgfqpoint{3.210545in}{1.341160in}}%
\pgfpathlineto{\pgfqpoint{3.213342in}{1.345686in}}%
\pgfpathlineto{\pgfqpoint{3.215908in}{1.343263in}}%
\pgfpathlineto{\pgfqpoint{3.218586in}{1.344465in}}%
\pgfpathlineto{\pgfqpoint{3.221255in}{1.349887in}}%
\pgfpathlineto{\pgfqpoint{3.223942in}{1.349229in}}%
\pgfpathlineto{\pgfqpoint{3.226609in}{1.347194in}}%
\pgfpathlineto{\pgfqpoint{3.229310in}{1.346069in}}%
\pgfpathlineto{\pgfqpoint{3.232069in}{1.345378in}}%
\pgfpathlineto{\pgfqpoint{3.234658in}{1.350677in}}%
\pgfpathlineto{\pgfqpoint{3.237411in}{1.347642in}}%
\pgfpathlineto{\pgfqpoint{3.240010in}{1.349747in}}%
\pgfpathlineto{\pgfqpoint{3.242807in}{1.353608in}}%
\pgfpathlineto{\pgfqpoint{3.245363in}{1.350118in}}%
\pgfpathlineto{\pgfqpoint{3.248049in}{1.353680in}}%
\pgfpathlineto{\pgfqpoint{3.250716in}{1.354722in}}%
\pgfpathlineto{\pgfqpoint{3.253404in}{1.354476in}}%
\pgfpathlineto{\pgfqpoint{3.256083in}{1.350065in}}%
\pgfpathlineto{\pgfqpoint{3.258784in}{1.351198in}}%
\pgfpathlineto{\pgfqpoint{3.261594in}{1.352410in}}%
\pgfpathlineto{\pgfqpoint{3.264119in}{1.347231in}}%
\pgfpathlineto{\pgfqpoint{3.266849in}{1.342595in}}%
\pgfpathlineto{\pgfqpoint{3.269478in}{1.339144in}}%
\pgfpathlineto{\pgfqpoint{3.272254in}{1.340161in}}%
\pgfpathlineto{\pgfqpoint{3.274831in}{1.343004in}}%
\pgfpathlineto{\pgfqpoint{3.277603in}{1.351495in}}%
\pgfpathlineto{\pgfqpoint{3.280189in}{1.350179in}}%
\pgfpathlineto{\pgfqpoint{3.282870in}{1.353017in}}%
\pgfpathlineto{\pgfqpoint{3.285534in}{1.355452in}}%
\pgfpathlineto{\pgfqpoint{3.288225in}{1.359940in}}%
\pgfpathlineto{\pgfqpoint{3.290890in}{1.358841in}}%
\pgfpathlineto{\pgfqpoint{3.293574in}{1.357489in}}%
\pgfpathlineto{\pgfqpoint{3.296376in}{1.355321in}}%
\pgfpathlineto{\pgfqpoint{3.298937in}{1.363647in}}%
\pgfpathlineto{\pgfqpoint{3.301719in}{1.360940in}}%
\pgfpathlineto{\pgfqpoint{3.304295in}{1.360231in}}%
\pgfpathlineto{\pgfqpoint{3.307104in}{1.357680in}}%
\pgfpathlineto{\pgfqpoint{3.309652in}{1.358758in}}%
\pgfpathlineto{\pgfqpoint{3.312480in}{1.360667in}}%
\pgfpathlineto{\pgfqpoint{3.315008in}{1.358654in}}%
\pgfpathlineto{\pgfqpoint{3.317688in}{1.358407in}}%
\pgfpathlineto{\pgfqpoint{3.320366in}{1.358644in}}%
\pgfpathlineto{\pgfqpoint{3.323049in}{1.358430in}}%
\pgfpathlineto{\pgfqpoint{3.325860in}{1.360487in}}%
\pgfpathlineto{\pgfqpoint{3.328401in}{1.360842in}}%
\pgfpathlineto{\pgfqpoint{3.331183in}{1.359489in}}%
\pgfpathlineto{\pgfqpoint{3.333758in}{1.360052in}}%
\pgfpathlineto{\pgfqpoint{3.336541in}{1.362798in}}%
\pgfpathlineto{\pgfqpoint{3.339101in}{1.362942in}}%
\pgfpathlineto{\pgfqpoint{3.341893in}{1.360229in}}%
\pgfpathlineto{\pgfqpoint{3.344468in}{1.356968in}}%
\pgfpathlineto{\pgfqpoint{3.347139in}{1.357673in}}%
\pgfpathlineto{\pgfqpoint{3.349828in}{1.360920in}}%
\pgfpathlineto{\pgfqpoint{3.352505in}{1.364335in}}%
\pgfpathlineto{\pgfqpoint{3.355177in}{1.361494in}}%
\pgfpathlineto{\pgfqpoint{3.357862in}{1.367422in}}%
\pgfpathlineto{\pgfqpoint{3.360620in}{1.362374in}}%
\pgfpathlineto{\pgfqpoint{3.363221in}{1.363395in}}%
\pgfpathlineto{\pgfqpoint{3.365996in}{1.359915in}}%
\pgfpathlineto{\pgfqpoint{3.368577in}{1.360471in}}%
\pgfpathlineto{\pgfqpoint{3.371357in}{1.357007in}}%
\pgfpathlineto{\pgfqpoint{3.373921in}{1.358070in}}%
\pgfpathlineto{\pgfqpoint{3.376735in}{1.358627in}}%
\pgfpathlineto{\pgfqpoint{3.379290in}{1.357719in}}%
\pgfpathlineto{\pgfqpoint{3.381959in}{1.358580in}}%
\pgfpathlineto{\pgfqpoint{3.384647in}{1.362929in}}%
\pgfpathlineto{\pgfqpoint{3.387309in}{1.358483in}}%
\pgfpathlineto{\pgfqpoint{3.390102in}{1.364588in}}%
\pgfpathlineto{\pgfqpoint{3.392681in}{1.356145in}}%
\pgfpathlineto{\pgfqpoint{3.395461in}{1.357543in}}%
\pgfpathlineto{\pgfqpoint{3.398037in}{1.358781in}}%
\pgfpathlineto{\pgfqpoint{3.400783in}{1.355638in}}%
\pgfpathlineto{\pgfqpoint{3.403394in}{1.361457in}}%
\pgfpathlineto{\pgfqpoint{3.406202in}{1.361739in}}%
\pgfpathlineto{\pgfqpoint{3.408752in}{1.357223in}}%
\pgfpathlineto{\pgfqpoint{3.411431in}{1.361365in}}%
\pgfpathlineto{\pgfqpoint{3.414109in}{1.358647in}}%
\pgfpathlineto{\pgfqpoint{3.416780in}{1.358949in}}%
\pgfpathlineto{\pgfqpoint{3.419455in}{1.359858in}}%
\pgfpathlineto{\pgfqpoint{3.422142in}{1.358141in}}%
\pgfpathlineto{\pgfqpoint{3.424887in}{1.358979in}}%
\pgfpathlineto{\pgfqpoint{3.427501in}{1.349169in}}%
\pgfpathlineto{\pgfqpoint{3.430313in}{1.353599in}}%
\pgfpathlineto{\pgfqpoint{3.432851in}{1.352226in}}%
\pgfpathlineto{\pgfqpoint{3.435635in}{1.355304in}}%
\pgfpathlineto{\pgfqpoint{3.438210in}{1.354168in}}%
\pgfpathlineto{\pgfqpoint{3.440996in}{1.355476in}}%
\pgfpathlineto{\pgfqpoint{3.443574in}{1.356412in}}%
\pgfpathlineto{\pgfqpoint{3.446257in}{1.350349in}}%
\pgfpathlineto{\pgfqpoint{3.448926in}{1.354786in}}%
\pgfpathlineto{\pgfqpoint{3.451597in}{1.353790in}}%
\pgfpathlineto{\pgfqpoint{3.454285in}{1.357642in}}%
\pgfpathlineto{\pgfqpoint{3.456960in}{1.358033in}}%
\pgfpathlineto{\pgfqpoint{3.459695in}{1.354767in}}%
\pgfpathlineto{\pgfqpoint{3.462321in}{1.357708in}}%
\pgfpathlineto{\pgfqpoint{3.465072in}{1.356896in}}%
\pgfpathlineto{\pgfqpoint{3.467678in}{1.352251in}}%
\pgfpathlineto{\pgfqpoint{3.470466in}{1.346063in}}%
\pgfpathlineto{\pgfqpoint{3.473021in}{1.350228in}}%
\pgfpathlineto{\pgfqpoint{3.475821in}{1.353649in}}%
\pgfpathlineto{\pgfqpoint{3.478378in}{1.361134in}}%
\pgfpathlineto{\pgfqpoint{3.481072in}{1.358001in}}%
\pgfpathlineto{\pgfqpoint{3.483744in}{1.367473in}}%
\pgfpathlineto{\pgfqpoint{3.486442in}{1.363453in}}%
\pgfpathlineto{\pgfqpoint{3.489223in}{1.357813in}}%
\pgfpathlineto{\pgfqpoint{3.491783in}{1.364831in}}%
\pgfpathlineto{\pgfqpoint{3.494581in}{1.361311in}}%
\pgfpathlineto{\pgfqpoint{3.497139in}{1.355868in}}%
\pgfpathlineto{\pgfqpoint{3.499909in}{1.355839in}}%
\pgfpathlineto{\pgfqpoint{3.502488in}{1.355462in}}%
\pgfpathlineto{\pgfqpoint{3.505262in}{1.358329in}}%
\pgfpathlineto{\pgfqpoint{3.507840in}{1.355947in}}%
\pgfpathlineto{\pgfqpoint{3.510533in}{1.355793in}}%
\pgfpathlineto{\pgfqpoint{3.513209in}{1.353825in}}%
\pgfpathlineto{\pgfqpoint{3.515884in}{1.354435in}}%
\pgfpathlineto{\pgfqpoint{3.518565in}{1.358240in}}%
\pgfpathlineto{\pgfqpoint{3.521244in}{1.355599in}}%
\pgfpathlineto{\pgfqpoint{3.524041in}{1.353835in}}%
\pgfpathlineto{\pgfqpoint{3.526601in}{1.352232in}}%
\pgfpathlineto{\pgfqpoint{3.529327in}{1.350130in}}%
\pgfpathlineto{\pgfqpoint{3.531955in}{1.347196in}}%
\pgfpathlineto{\pgfqpoint{3.534783in}{1.354512in}}%
\pgfpathlineto{\pgfqpoint{3.537309in}{1.356267in}}%
\pgfpathlineto{\pgfqpoint{3.540093in}{1.357211in}}%
\pgfpathlineto{\pgfqpoint{3.542656in}{1.357594in}}%
\pgfpathlineto{\pgfqpoint{3.545349in}{1.356125in}}%
\pgfpathlineto{\pgfqpoint{3.548029in}{1.364879in}}%
\pgfpathlineto{\pgfqpoint{3.550713in}{1.364682in}}%
\pgfpathlineto{\pgfqpoint{3.553498in}{1.364738in}}%
\pgfpathlineto{\pgfqpoint{3.556061in}{1.358404in}}%
\pgfpathlineto{\pgfqpoint{3.558853in}{1.355996in}}%
\pgfpathlineto{\pgfqpoint{3.561420in}{1.356582in}}%
\pgfpathlineto{\pgfqpoint{3.564188in}{1.356889in}}%
\pgfpathlineto{\pgfqpoint{3.566774in}{1.358224in}}%
\pgfpathlineto{\pgfqpoint{3.569584in}{1.357746in}}%
\pgfpathlineto{\pgfqpoint{3.572126in}{1.359968in}}%
\pgfpathlineto{\pgfqpoint{3.574814in}{1.355439in}}%
\pgfpathlineto{\pgfqpoint{3.577487in}{1.360068in}}%
\pgfpathlineto{\pgfqpoint{3.580191in}{1.363444in}}%
\pgfpathlineto{\pgfqpoint{3.582851in}{1.363999in}}%
\pgfpathlineto{\pgfqpoint{3.585532in}{1.361989in}}%
\pgfpathlineto{\pgfqpoint{3.588258in}{1.359369in}}%
\pgfpathlineto{\pgfqpoint{3.590883in}{1.350501in}}%
\pgfpathlineto{\pgfqpoint{3.593620in}{1.357470in}}%
\pgfpathlineto{\pgfqpoint{3.596240in}{1.356175in}}%
\pgfpathlineto{\pgfqpoint{3.598998in}{1.356189in}}%
\pgfpathlineto{\pgfqpoint{3.601590in}{1.353107in}}%
\pgfpathlineto{\pgfqpoint{3.604387in}{1.356047in}}%
\pgfpathlineto{\pgfqpoint{3.606951in}{1.360800in}}%
\pgfpathlineto{\pgfqpoint{3.609632in}{1.353784in}}%
\pgfpathlineto{\pgfqpoint{3.612311in}{1.347475in}}%
\pgfpathlineto{\pgfqpoint{3.614982in}{1.348786in}}%
\pgfpathlineto{\pgfqpoint{3.617667in}{1.355449in}}%
\pgfpathlineto{\pgfqpoint{3.620345in}{1.352103in}}%
\pgfpathlineto{\pgfqpoint{3.623165in}{1.353713in}}%
\pgfpathlineto{\pgfqpoint{3.625689in}{1.354118in}}%
\pgfpathlineto{\pgfqpoint{3.628460in}{1.358859in}}%
\pgfpathlineto{\pgfqpoint{3.631058in}{1.352951in}}%
\pgfpathlineto{\pgfqpoint{3.633858in}{1.356044in}}%
\pgfpathlineto{\pgfqpoint{3.636413in}{1.351530in}}%
\pgfpathlineto{\pgfqpoint{3.639207in}{1.355758in}}%
\pgfpathlineto{\pgfqpoint{3.641773in}{1.351591in}}%
\pgfpathlineto{\pgfqpoint{3.644452in}{1.341149in}}%
\pgfpathlineto{\pgfqpoint{3.647130in}{1.337809in}}%
\pgfpathlineto{\pgfqpoint{3.649837in}{1.341284in}}%
\pgfpathlineto{\pgfqpoint{3.652628in}{1.345627in}}%
\pgfpathlineto{\pgfqpoint{3.655165in}{1.344028in}}%
\pgfpathlineto{\pgfqpoint{3.657917in}{1.341588in}}%
\pgfpathlineto{\pgfqpoint{3.660515in}{1.352147in}}%
\pgfpathlineto{\pgfqpoint{3.663276in}{1.351803in}}%
\pgfpathlineto{\pgfqpoint{3.665864in}{1.334887in}}%
\pgfpathlineto{\pgfqpoint{3.668665in}{1.334887in}}%
\pgfpathlineto{\pgfqpoint{3.671232in}{1.336757in}}%
\pgfpathlineto{\pgfqpoint{3.673911in}{1.347872in}}%
\pgfpathlineto{\pgfqpoint{3.676591in}{1.355400in}}%
\pgfpathlineto{\pgfqpoint{3.679273in}{1.362200in}}%
\pgfpathlineto{\pgfqpoint{3.681948in}{1.352799in}}%
\pgfpathlineto{\pgfqpoint{3.684620in}{1.351374in}}%
\pgfpathlineto{\pgfqpoint{3.687442in}{1.358707in}}%
\pgfpathlineto{\pgfqpoint{3.689983in}{1.362300in}}%
\pgfpathlineto{\pgfqpoint{3.692765in}{1.362174in}}%
\pgfpathlineto{\pgfqpoint{3.695331in}{1.361803in}}%
\pgfpathlineto{\pgfqpoint{3.698125in}{1.358545in}}%
\pgfpathlineto{\pgfqpoint{3.700684in}{1.360907in}}%
\pgfpathlineto{\pgfqpoint{3.703460in}{1.360553in}}%
\pgfpathlineto{\pgfqpoint{3.706053in}{1.359369in}}%
\pgfpathlineto{\pgfqpoint{3.708729in}{1.368352in}}%
\pgfpathlineto{\pgfqpoint{3.711410in}{1.367766in}}%
\pgfpathlineto{\pgfqpoint{3.714086in}{1.367939in}}%
\pgfpathlineto{\pgfqpoint{3.716875in}{1.366702in}}%
\pgfpathlineto{\pgfqpoint{3.719446in}{1.362656in}}%
\pgfpathlineto{\pgfqpoint{3.722228in}{1.364490in}}%
\pgfpathlineto{\pgfqpoint{3.724804in}{1.364366in}}%
\pgfpathlineto{\pgfqpoint{3.727581in}{1.365096in}}%
\pgfpathlineto{\pgfqpoint{3.730158in}{1.366952in}}%
\pgfpathlineto{\pgfqpoint{3.732950in}{1.360808in}}%
\pgfpathlineto{\pgfqpoint{3.735509in}{1.363847in}}%
\pgfpathlineto{\pgfqpoint{3.738194in}{1.364542in}}%
\pgfpathlineto{\pgfqpoint{3.740874in}{1.362692in}}%
\pgfpathlineto{\pgfqpoint{3.743548in}{1.362814in}}%
\pgfpathlineto{\pgfqpoint{3.746229in}{1.362013in}}%
\pgfpathlineto{\pgfqpoint{3.748903in}{1.372475in}}%
\pgfpathlineto{\pgfqpoint{3.751728in}{1.360543in}}%
\pgfpathlineto{\pgfqpoint{3.754265in}{1.364178in}}%
\pgfpathlineto{\pgfqpoint{3.757065in}{1.365470in}}%
\pgfpathlineto{\pgfqpoint{3.759622in}{1.368800in}}%
\pgfpathlineto{\pgfqpoint{3.762389in}{1.368494in}}%
\pgfpathlineto{\pgfqpoint{3.764966in}{1.382026in}}%
\pgfpathlineto{\pgfqpoint{3.767782in}{1.381445in}}%
\pgfpathlineto{\pgfqpoint{3.770323in}{1.364618in}}%
\pgfpathlineto{\pgfqpoint{3.773014in}{1.364429in}}%
\pgfpathlineto{\pgfqpoint{3.775691in}{1.363298in}}%
\pgfpathlineto{\pgfqpoint{3.778370in}{1.356433in}}%
\pgfpathlineto{\pgfqpoint{3.781046in}{1.356014in}}%
\pgfpathlineto{\pgfqpoint{3.783725in}{1.352612in}}%
\pgfpathlineto{\pgfqpoint{3.786504in}{1.357469in}}%
\pgfpathlineto{\pgfqpoint{3.789084in}{1.355852in}}%
\pgfpathlineto{\pgfqpoint{3.791897in}{1.358158in}}%
\pgfpathlineto{\pgfqpoint{3.794435in}{1.357587in}}%
\pgfpathlineto{\pgfqpoint{3.797265in}{1.355514in}}%
\pgfpathlineto{\pgfqpoint{3.799797in}{1.355507in}}%
\pgfpathlineto{\pgfqpoint{3.802569in}{1.352394in}}%
\pgfpathlineto{\pgfqpoint{3.805145in}{1.353715in}}%
\pgfpathlineto{\pgfqpoint{3.807832in}{1.356531in}}%
\pgfpathlineto{\pgfqpoint{3.810510in}{1.359323in}}%
\pgfpathlineto{\pgfqpoint{3.813172in}{1.359433in}}%
\pgfpathlineto{\pgfqpoint{3.815983in}{1.358991in}}%
\pgfpathlineto{\pgfqpoint{3.818546in}{1.361657in}}%
\pgfpathlineto{\pgfqpoint{3.821315in}{1.358668in}}%
\pgfpathlineto{\pgfqpoint{3.823903in}{1.358205in}}%
\pgfpathlineto{\pgfqpoint{3.826679in}{1.360704in}}%
\pgfpathlineto{\pgfqpoint{3.829252in}{1.351277in}}%
\pgfpathlineto{\pgfqpoint{3.832053in}{1.350315in}}%
\pgfpathlineto{\pgfqpoint{3.834616in}{1.351941in}}%
\pgfpathlineto{\pgfqpoint{3.837286in}{1.354161in}}%
\pgfpathlineto{\pgfqpoint{3.839960in}{1.350385in}}%
\pgfpathlineto{\pgfqpoint{3.842641in}{1.349271in}}%
\pgfpathlineto{\pgfqpoint{3.845329in}{1.351487in}}%
\pgfpathlineto{\pgfqpoint{3.848005in}{1.357894in}}%
\pgfpathlineto{\pgfqpoint{3.850814in}{1.360301in}}%
\pgfpathlineto{\pgfqpoint{3.853358in}{1.359850in}}%
\pgfpathlineto{\pgfqpoint{3.856100in}{1.356853in}}%
\pgfpathlineto{\pgfqpoint{3.858720in}{1.355380in}}%
\pgfpathlineto{\pgfqpoint{3.861561in}{1.357030in}}%
\pgfpathlineto{\pgfqpoint{3.864073in}{1.357440in}}%
\pgfpathlineto{\pgfqpoint{3.866815in}{1.356067in}}%
\pgfpathlineto{\pgfqpoint{3.869435in}{1.352936in}}%
\pgfpathlineto{\pgfqpoint{3.872114in}{1.354823in}}%
\pgfpathlineto{\pgfqpoint{3.874790in}{1.356645in}}%
\pgfpathlineto{\pgfqpoint{3.877466in}{1.349544in}}%
\pgfpathlineto{\pgfqpoint{3.880237in}{1.352922in}}%
\pgfpathlineto{\pgfqpoint{3.882850in}{1.349840in}}%
\pgfpathlineto{\pgfqpoint{3.885621in}{1.357773in}}%
\pgfpathlineto{\pgfqpoint{3.888188in}{1.351650in}}%
\pgfpathlineto{\pgfqpoint{3.890926in}{1.347248in}}%
\pgfpathlineto{\pgfqpoint{3.893541in}{1.345291in}}%
\pgfpathlineto{\pgfqpoint{3.896345in}{1.342219in}}%
\pgfpathlineto{\pgfqpoint{3.898891in}{1.347136in}}%
\pgfpathlineto{\pgfqpoint{3.901573in}{1.344596in}}%
\pgfpathlineto{\pgfqpoint{3.904252in}{1.351084in}}%
\pgfpathlineto{\pgfqpoint{3.906918in}{1.352006in}}%
\pgfpathlineto{\pgfqpoint{3.909602in}{1.353577in}}%
\pgfpathlineto{\pgfqpoint{3.912296in}{1.356056in}}%
\pgfpathlineto{\pgfqpoint{3.915107in}{1.353247in}}%
\pgfpathlineto{\pgfqpoint{3.917646in}{1.350987in}}%
\pgfpathlineto{\pgfqpoint{3.920412in}{1.353631in}}%
\pgfpathlineto{\pgfqpoint{3.923005in}{1.348840in}}%
\pgfpathlineto{\pgfqpoint{3.925778in}{1.353662in}}%
\pgfpathlineto{\pgfqpoint{3.928347in}{1.351620in}}%
\pgfpathlineto{\pgfqpoint{3.931202in}{1.353722in}}%
\pgfpathlineto{\pgfqpoint{3.933714in}{1.354111in}}%
\pgfpathlineto{\pgfqpoint{3.936395in}{1.348743in}}%
\pgfpathlineto{\pgfqpoint{3.939075in}{1.350035in}}%
\pgfpathlineto{\pgfqpoint{3.941778in}{1.343648in}}%
\pgfpathlineto{\pgfqpoint{3.944431in}{1.339600in}}%
\pgfpathlineto{\pgfqpoint{3.947101in}{1.344179in}}%
\pgfpathlineto{\pgfqpoint{3.949894in}{1.349055in}}%
\pgfpathlineto{\pgfqpoint{3.952464in}{1.358919in}}%
\pgfpathlineto{\pgfqpoint{3.955211in}{1.357519in}}%
\pgfpathlineto{\pgfqpoint{3.957823in}{1.357949in}}%
\pgfpathlineto{\pgfqpoint{3.960635in}{1.355427in}}%
\pgfpathlineto{\pgfqpoint{3.963176in}{1.355057in}}%
\pgfpathlineto{\pgfqpoint{3.966013in}{1.358951in}}%
\pgfpathlineto{\pgfqpoint{3.968523in}{1.364704in}}%
\pgfpathlineto{\pgfqpoint{3.971250in}{1.360560in}}%
\pgfpathlineto{\pgfqpoint{3.973885in}{1.361993in}}%
\pgfpathlineto{\pgfqpoint{3.976563in}{1.354190in}}%
\pgfpathlineto{\pgfqpoint{3.979389in}{1.356162in}}%
\pgfpathlineto{\pgfqpoint{3.981929in}{1.357826in}}%
\pgfpathlineto{\pgfqpoint{3.984714in}{1.357760in}}%
\pgfpathlineto{\pgfqpoint{3.987270in}{1.366777in}}%
\pgfpathlineto{\pgfqpoint{3.990055in}{1.358591in}}%
\pgfpathlineto{\pgfqpoint{3.992642in}{1.358940in}}%
\pgfpathlineto{\pgfqpoint{3.995417in}{1.363700in}}%
\pgfpathlineto{\pgfqpoint{3.997990in}{1.357075in}}%
\pgfpathlineto{\pgfqpoint{4.000674in}{1.357547in}}%
\pgfpathlineto{\pgfqpoint{4.003348in}{1.359963in}}%
\pgfpathlineto{\pgfqpoint{4.006034in}{1.358827in}}%
\pgfpathlineto{\pgfqpoint{4.008699in}{1.358667in}}%
\pgfpathlineto{\pgfqpoint{4.011394in}{1.359208in}}%
\pgfpathlineto{\pgfqpoint{4.014186in}{1.356906in}}%
\pgfpathlineto{\pgfqpoint{4.016744in}{1.359932in}}%
\pgfpathlineto{\pgfqpoint{4.019518in}{1.356156in}}%
\pgfpathlineto{\pgfqpoint{4.022097in}{1.358872in}}%
\pgfpathlineto{\pgfqpoint{4.024868in}{1.358693in}}%
\pgfpathlineto{\pgfqpoint{4.027447in}{1.354973in}}%
\pgfpathlineto{\pgfqpoint{4.030229in}{1.358639in}}%
\pgfpathlineto{\pgfqpoint{4.032817in}{1.359874in}}%
\pgfpathlineto{\pgfqpoint{4.035492in}{1.375412in}}%
\pgfpathlineto{\pgfqpoint{4.038174in}{1.367382in}}%
\pgfpathlineto{\pgfqpoint{4.040852in}{1.354156in}}%
\pgfpathlineto{\pgfqpoint{4.043667in}{1.359608in}}%
\pgfpathlineto{\pgfqpoint{4.046210in}{1.359285in}}%
\pgfpathlineto{\pgfqpoint{4.049006in}{1.355543in}}%
\pgfpathlineto{\pgfqpoint{4.051557in}{1.358031in}}%
\pgfpathlineto{\pgfqpoint{4.054326in}{1.356938in}}%
\pgfpathlineto{\pgfqpoint{4.056911in}{1.360908in}}%
\pgfpathlineto{\pgfqpoint{4.059702in}{1.355952in}}%
\pgfpathlineto{\pgfqpoint{4.062266in}{1.351157in}}%
\pgfpathlineto{\pgfqpoint{4.064957in}{1.354401in}}%
\pgfpathlineto{\pgfqpoint{4.067636in}{1.356011in}}%
\pgfpathlineto{\pgfqpoint{4.070313in}{1.354194in}}%
\pgfpathlineto{\pgfqpoint{4.072985in}{1.358933in}}%
\pgfpathlineto{\pgfqpoint{4.075705in}{1.352803in}}%
\pgfpathlineto{\pgfqpoint{4.078471in}{1.356080in}}%
\pgfpathlineto{\pgfqpoint{4.081018in}{1.353072in}}%
\pgfpathlineto{\pgfqpoint{4.083870in}{1.357440in}}%
\pgfpathlineto{\pgfqpoint{4.086385in}{1.354046in}}%
\pgfpathlineto{\pgfqpoint{4.089159in}{1.355875in}}%
\pgfpathlineto{\pgfqpoint{4.091729in}{1.350601in}}%
\pgfpathlineto{\pgfqpoint{4.094527in}{1.351822in}}%
\pgfpathlineto{\pgfqpoint{4.097092in}{1.352004in}}%
\pgfpathlineto{\pgfqpoint{4.099777in}{1.357488in}}%
\pgfpathlineto{\pgfqpoint{4.102456in}{1.353653in}}%
\pgfpathlineto{\pgfqpoint{4.105185in}{1.356295in}}%
\pgfpathlineto{\pgfqpoint{4.107814in}{1.358475in}}%
\pgfpathlineto{\pgfqpoint{4.110488in}{1.355172in}}%
\pgfpathlineto{\pgfqpoint{4.113252in}{1.360475in}}%
\pgfpathlineto{\pgfqpoint{4.115844in}{1.360546in}}%
\pgfpathlineto{\pgfqpoint{4.118554in}{1.358480in}}%
\pgfpathlineto{\pgfqpoint{4.121205in}{1.356057in}}%
\pgfpathlineto{\pgfqpoint{4.124019in}{1.354292in}}%
\pgfpathlineto{\pgfqpoint{4.126553in}{1.352335in}}%
\pgfpathlineto{\pgfqpoint{4.129349in}{1.349052in}}%
\pgfpathlineto{\pgfqpoint{4.131920in}{1.352638in}}%
\pgfpathlineto{\pgfqpoint{4.134615in}{1.358256in}}%
\pgfpathlineto{\pgfqpoint{4.137272in}{1.351185in}}%
\pgfpathlineto{\pgfqpoint{4.139963in}{1.356214in}}%
\pgfpathlineto{\pgfqpoint{4.142713in}{1.358392in}}%
\pgfpathlineto{\pgfqpoint{4.145310in}{1.351770in}}%
\pgfpathlineto{\pgfqpoint{4.148082in}{1.357155in}}%
\pgfpathlineto{\pgfqpoint{4.150665in}{1.356670in}}%
\pgfpathlineto{\pgfqpoint{4.153423in}{1.358105in}}%
\pgfpathlineto{\pgfqpoint{4.156016in}{1.359492in}}%
\pgfpathlineto{\pgfqpoint{4.158806in}{1.361563in}}%
\pgfpathlineto{\pgfqpoint{4.161380in}{1.364881in}}%
\pgfpathlineto{\pgfqpoint{4.164059in}{1.357313in}}%
\pgfpathlineto{\pgfqpoint{4.166737in}{1.354243in}}%
\pgfpathlineto{\pgfqpoint{4.169415in}{1.354444in}}%
\pgfpathlineto{\pgfqpoint{4.172093in}{1.358114in}}%
\pgfpathlineto{\pgfqpoint{4.174770in}{1.358086in}}%
\pgfpathlineto{\pgfqpoint{4.177593in}{1.356788in}}%
\pgfpathlineto{\pgfqpoint{4.180129in}{1.358220in}}%
\pgfpathlineto{\pgfqpoint{4.182899in}{1.350975in}}%
\pgfpathlineto{\pgfqpoint{4.185481in}{1.348908in}}%
\pgfpathlineto{\pgfqpoint{4.188318in}{1.348237in}}%
\pgfpathlineto{\pgfqpoint{4.190842in}{1.351615in}}%
\pgfpathlineto{\pgfqpoint{4.193638in}{1.351651in}}%
\pgfpathlineto{\pgfqpoint{4.196186in}{1.351481in}}%
\pgfpathlineto{\pgfqpoint{4.198878in}{1.351897in}}%
\pgfpathlineto{\pgfqpoint{4.201542in}{1.352981in}}%
\pgfpathlineto{\pgfqpoint{4.204240in}{1.355845in}}%
\pgfpathlineto{\pgfqpoint{4.207076in}{1.355103in}}%
\pgfpathlineto{\pgfqpoint{4.209597in}{1.356120in}}%
\pgfpathlineto{\pgfqpoint{4.212383in}{1.358743in}}%
\pgfpathlineto{\pgfqpoint{4.214948in}{1.355853in}}%
\pgfpathlineto{\pgfqpoint{4.217694in}{1.361982in}}%
\pgfpathlineto{\pgfqpoint{4.220304in}{1.362568in}}%
\pgfpathlineto{\pgfqpoint{4.223082in}{1.360073in}}%
\pgfpathlineto{\pgfqpoint{4.225654in}{1.360039in}}%
\pgfpathlineto{\pgfqpoint{4.228331in}{1.361563in}}%
\pgfpathlineto{\pgfqpoint{4.231013in}{1.355667in}}%
\pgfpathlineto{\pgfqpoint{4.233691in}{1.350049in}}%
\pgfpathlineto{\pgfqpoint{4.236375in}{1.350609in}}%
\pgfpathlineto{\pgfqpoint{4.239084in}{1.353875in}}%
\pgfpathlineto{\pgfqpoint{4.241900in}{1.361417in}}%
\pgfpathlineto{\pgfqpoint{4.244394in}{1.357677in}}%
\pgfpathlineto{\pgfqpoint{4.247225in}{1.354183in}}%
\pgfpathlineto{\pgfqpoint{4.249767in}{1.353348in}}%
\pgfpathlineto{\pgfqpoint{4.252581in}{1.350007in}}%
\pgfpathlineto{\pgfqpoint{4.255120in}{1.348894in}}%
\pgfpathlineto{\pgfqpoint{4.257958in}{1.347692in}}%
\pgfpathlineto{\pgfqpoint{4.260477in}{1.352520in}}%
\pgfpathlineto{\pgfqpoint{4.263157in}{1.346291in}}%
\pgfpathlineto{\pgfqpoint{4.265824in}{1.342713in}}%
\pgfpathlineto{\pgfqpoint{4.268590in}{1.342286in}}%
\pgfpathlineto{\pgfqpoint{4.271187in}{1.341359in}}%
\pgfpathlineto{\pgfqpoint{4.273874in}{1.343556in}}%
\pgfpathlineto{\pgfqpoint{4.276635in}{1.344508in}}%
\pgfpathlineto{\pgfqpoint{4.279212in}{1.348503in}}%
\pgfpathlineto{\pgfqpoint{4.282000in}{1.351810in}}%
\pgfpathlineto{\pgfqpoint{4.284586in}{1.358905in}}%
\pgfpathlineto{\pgfqpoint{4.287399in}{1.353017in}}%
\pgfpathlineto{\pgfqpoint{4.289936in}{1.354155in}}%
\pgfpathlineto{\pgfqpoint{4.292786in}{1.351885in}}%
\pgfpathlineto{\pgfqpoint{4.295299in}{1.349682in}}%
\pgfpathlineto{\pgfqpoint{4.297977in}{1.357257in}}%
\pgfpathlineto{\pgfqpoint{4.300656in}{1.355233in}}%
\pgfpathlineto{\pgfqpoint{4.303357in}{1.354761in}}%
\pgfpathlineto{\pgfqpoint{4.306118in}{1.358215in}}%
\pgfpathlineto{\pgfqpoint{4.308691in}{1.360565in}}%
\pgfpathlineto{\pgfqpoint{4.311494in}{1.357674in}}%
\pgfpathlineto{\pgfqpoint{4.314032in}{1.354544in}}%
\pgfpathlineto{\pgfqpoint{4.316856in}{1.356476in}}%
\pgfpathlineto{\pgfqpoint{4.319405in}{1.360364in}}%
\pgfpathlineto{\pgfqpoint{4.322181in}{1.360714in}}%
\pgfpathlineto{\pgfqpoint{4.324760in}{1.355343in}}%
\pgfpathlineto{\pgfqpoint{4.327440in}{1.353489in}}%
\pgfpathlineto{\pgfqpoint{4.330118in}{1.358917in}}%
\pgfpathlineto{\pgfqpoint{4.332796in}{1.361773in}}%
\pgfpathlineto{\pgfqpoint{4.335463in}{1.360632in}}%
\pgfpathlineto{\pgfqpoint{4.338154in}{1.350591in}}%
\pgfpathlineto{\pgfqpoint{4.340976in}{1.358961in}}%
\pgfpathlineto{\pgfqpoint{4.343510in}{1.356209in}}%
\pgfpathlineto{\pgfqpoint{4.346263in}{1.357062in}}%
\pgfpathlineto{\pgfqpoint{4.348868in}{1.362166in}}%
\pgfpathlineto{\pgfqpoint{4.351645in}{1.358913in}}%
\pgfpathlineto{\pgfqpoint{4.354224in}{1.362693in}}%
\pgfpathlineto{\pgfqpoint{4.357014in}{1.362024in}}%
\pgfpathlineto{\pgfqpoint{4.359582in}{1.361658in}}%
\pgfpathlineto{\pgfqpoint{4.362270in}{1.361473in}}%
\pgfpathlineto{\pgfqpoint{4.364936in}{1.360537in}}%
\pgfpathlineto{\pgfqpoint{4.367646in}{1.360122in}}%
\pgfpathlineto{\pgfqpoint{4.370437in}{1.359404in}}%
\pgfpathlineto{\pgfqpoint{4.372976in}{1.357195in}}%
\pgfpathlineto{\pgfqpoint{4.375761in}{1.360455in}}%
\pgfpathlineto{\pgfqpoint{4.378329in}{1.358105in}}%
\pgfpathlineto{\pgfqpoint{4.381097in}{1.355279in}}%
\pgfpathlineto{\pgfqpoint{4.383674in}{1.360617in}}%
\pgfpathlineto{\pgfqpoint{4.386431in}{1.360993in}}%
\pgfpathlineto{\pgfqpoint{4.389044in}{1.361239in}}%
\pgfpathlineto{\pgfqpoint{4.391721in}{1.365147in}}%
\pgfpathlineto{\pgfqpoint{4.394400in}{1.363231in}}%
\pgfpathlineto{\pgfqpoint{4.397076in}{1.364036in}}%
\pgfpathlineto{\pgfqpoint{4.399745in}{1.364182in}}%
\pgfpathlineto{\pgfqpoint{4.402468in}{1.358616in}}%
\pgfpathlineto{\pgfqpoint{4.405234in}{1.355664in}}%
\pgfpathlineto{\pgfqpoint{4.407788in}{1.361638in}}%
\pgfpathlineto{\pgfqpoint{4.410587in}{1.358718in}}%
\pgfpathlineto{\pgfqpoint{4.413149in}{1.359234in}}%
\pgfpathlineto{\pgfqpoint{4.415932in}{1.357646in}}%
\pgfpathlineto{\pgfqpoint{4.418506in}{1.353296in}}%
\pgfpathlineto{\pgfqpoint{4.421292in}{1.351647in}}%
\pgfpathlineto{\pgfqpoint{4.423863in}{1.351103in}}%
\pgfpathlineto{\pgfqpoint{4.426534in}{1.353767in}}%
\pgfpathlineto{\pgfqpoint{4.429220in}{1.354773in}}%
\pgfpathlineto{\pgfqpoint{4.431901in}{1.358452in}}%
\pgfpathlineto{\pgfqpoint{4.434569in}{1.356919in}}%
\pgfpathlineto{\pgfqpoint{4.437253in}{1.361227in}}%
\pgfpathlineto{\pgfqpoint{4.440041in}{1.360808in}}%
\pgfpathlineto{\pgfqpoint{4.442611in}{1.361136in}}%
\pgfpathlineto{\pgfqpoint{4.445423in}{1.361848in}}%
\pgfpathlineto{\pgfqpoint{4.447965in}{1.360621in}}%
\pgfpathlineto{\pgfqpoint{4.450767in}{1.354853in}}%
\pgfpathlineto{\pgfqpoint{4.453312in}{1.353396in}}%
\pgfpathlineto{\pgfqpoint{4.456138in}{1.358014in}}%
\pgfpathlineto{\pgfqpoint{4.458681in}{1.354037in}}%
\pgfpathlineto{\pgfqpoint{4.461367in}{1.353111in}}%
\pgfpathlineto{\pgfqpoint{4.464029in}{1.347001in}}%
\pgfpathlineto{\pgfqpoint{4.466717in}{1.347921in}}%
\pgfpathlineto{\pgfqpoint{4.469492in}{1.353193in}}%
\pgfpathlineto{\pgfqpoint{4.472059in}{1.351794in}}%
\pgfpathlineto{\pgfqpoint{4.474861in}{1.352615in}}%
\pgfpathlineto{\pgfqpoint{4.477430in}{1.349233in}}%
\pgfpathlineto{\pgfqpoint{4.480201in}{1.343221in}}%
\pgfpathlineto{\pgfqpoint{4.482778in}{1.349706in}}%
\pgfpathlineto{\pgfqpoint{4.485581in}{1.356045in}}%
\pgfpathlineto{\pgfqpoint{4.488130in}{1.356748in}}%
\pgfpathlineto{\pgfqpoint{4.490822in}{1.356430in}}%
\pgfpathlineto{\pgfqpoint{4.493492in}{1.358763in}}%
\pgfpathlineto{\pgfqpoint{4.496167in}{1.359562in}}%
\pgfpathlineto{\pgfqpoint{4.498850in}{1.362360in}}%
\pgfpathlineto{\pgfqpoint{4.501529in}{1.354890in}}%
\pgfpathlineto{\pgfqpoint{4.504305in}{1.356305in}}%
\pgfpathlineto{\pgfqpoint{4.506893in}{1.357809in}}%
\pgfpathlineto{\pgfqpoint{4.509643in}{1.358163in}}%
\pgfpathlineto{\pgfqpoint{4.512246in}{1.357650in}}%
\pgfpathlineto{\pgfqpoint{4.515080in}{1.359714in}}%
\pgfpathlineto{\pgfqpoint{4.517598in}{1.355998in}}%
\pgfpathlineto{\pgfqpoint{4.520345in}{1.362268in}}%
\pgfpathlineto{\pgfqpoint{4.522962in}{1.368252in}}%
\pgfpathlineto{\pgfqpoint{4.525640in}{1.356018in}}%
\pgfpathlineto{\pgfqpoint{4.528307in}{1.360458in}}%
\pgfpathlineto{\pgfqpoint{4.530990in}{1.361298in}}%
\pgfpathlineto{\pgfqpoint{4.533764in}{1.362827in}}%
\pgfpathlineto{\pgfqpoint{4.536400in}{1.369571in}}%
\pgfpathlineto{\pgfqpoint{4.539144in}{1.363753in}}%
\pgfpathlineto{\pgfqpoint{4.541711in}{1.357529in}}%
\pgfpathlineto{\pgfqpoint{4.544464in}{1.358797in}}%
\pgfpathlineto{\pgfqpoint{4.547064in}{1.360345in}}%
\pgfpathlineto{\pgfqpoint{4.549822in}{1.358144in}}%
\pgfpathlineto{\pgfqpoint{4.552425in}{1.356833in}}%
\pgfpathlineto{\pgfqpoint{4.555106in}{1.356054in}}%
\pgfpathlineto{\pgfqpoint{4.557777in}{1.358196in}}%
\pgfpathlineto{\pgfqpoint{4.560448in}{1.356399in}}%
\pgfpathlineto{\pgfqpoint{4.563125in}{1.359082in}}%
\pgfpathlineto{\pgfqpoint{4.565820in}{1.358282in}}%
\pgfpathlineto{\pgfqpoint{4.568612in}{1.358858in}}%
\pgfpathlineto{\pgfqpoint{4.571171in}{1.352905in}}%
\pgfpathlineto{\pgfqpoint{4.573947in}{1.353870in}}%
\pgfpathlineto{\pgfqpoint{4.576531in}{1.354155in}}%
\pgfpathlineto{\pgfqpoint{4.579305in}{1.358768in}}%
\pgfpathlineto{\pgfqpoint{4.581888in}{1.356300in}}%
\pgfpathlineto{\pgfqpoint{4.584672in}{1.356513in}}%
\pgfpathlineto{\pgfqpoint{4.587244in}{1.359165in}}%
\pgfpathlineto{\pgfqpoint{4.589920in}{1.351834in}}%
\pgfpathlineto{\pgfqpoint{4.592589in}{1.351785in}}%
\pgfpathlineto{\pgfqpoint{4.595281in}{1.356665in}}%
\pgfpathlineto{\pgfqpoint{4.597951in}{1.359030in}}%
\pgfpathlineto{\pgfqpoint{4.600633in}{1.360400in}}%
\pgfpathlineto{\pgfqpoint{4.603430in}{1.358480in}}%
\pgfpathlineto{\pgfqpoint{4.605990in}{1.358639in}}%
\pgfpathlineto{\pgfqpoint{4.608808in}{1.361913in}}%
\pgfpathlineto{\pgfqpoint{4.611350in}{1.355760in}}%
\pgfpathlineto{\pgfqpoint{4.614134in}{1.358131in}}%
\pgfpathlineto{\pgfqpoint{4.616702in}{1.360277in}}%
\pgfpathlineto{\pgfqpoint{4.619529in}{1.361334in}}%
\pgfpathlineto{\pgfqpoint{4.622056in}{1.355511in}}%
\pgfpathlineto{\pgfqpoint{4.624741in}{1.359988in}}%
\pgfpathlineto{\pgfqpoint{4.627411in}{1.360356in}}%
\pgfpathlineto{\pgfqpoint{4.630096in}{1.360392in}}%
\pgfpathlineto{\pgfqpoint{4.632902in}{1.358222in}}%
\pgfpathlineto{\pgfqpoint{4.635445in}{1.360790in}}%
\pgfpathlineto{\pgfqpoint{4.638204in}{1.360597in}}%
\pgfpathlineto{\pgfqpoint{4.640809in}{1.355527in}}%
\pgfpathlineto{\pgfqpoint{4.643628in}{1.360745in}}%
\pgfpathlineto{\pgfqpoint{4.646169in}{1.361447in}}%
\pgfpathlineto{\pgfqpoint{4.648922in}{1.359718in}}%
\pgfpathlineto{\pgfqpoint{4.651524in}{1.360338in}}%
\pgfpathlineto{\pgfqpoint{4.654203in}{1.356976in}}%
\pgfpathlineto{\pgfqpoint{4.656873in}{1.362988in}}%
\pgfpathlineto{\pgfqpoint{4.659590in}{1.361720in}}%
\pgfpathlineto{\pgfqpoint{4.662237in}{1.364092in}}%
\pgfpathlineto{\pgfqpoint{4.664923in}{1.359945in}}%
\pgfpathlineto{\pgfqpoint{4.667764in}{1.360535in}}%
\pgfpathlineto{\pgfqpoint{4.670261in}{1.358720in}}%
\pgfpathlineto{\pgfqpoint{4.673068in}{1.355914in}}%
\pgfpathlineto{\pgfqpoint{4.675619in}{1.358112in}}%
\pgfpathlineto{\pgfqpoint{4.678448in}{1.359551in}}%
\pgfpathlineto{\pgfqpoint{4.680988in}{1.359435in}}%
\pgfpathlineto{\pgfqpoint{4.683799in}{1.361751in}}%
\pgfpathlineto{\pgfqpoint{4.686337in}{1.383604in}}%
\pgfpathlineto{\pgfqpoint{4.689051in}{1.373030in}}%
\pgfpathlineto{\pgfqpoint{4.691694in}{1.368317in}}%
\pgfpathlineto{\pgfqpoint{4.694381in}{1.363448in}}%
\pgfpathlineto{\pgfqpoint{4.697170in}{1.367681in}}%
\pgfpathlineto{\pgfqpoint{4.699734in}{1.358974in}}%
\pgfpathlineto{\pgfqpoint{4.702517in}{1.358884in}}%
\pgfpathlineto{\pgfqpoint{4.705094in}{1.356785in}}%
\pgfpathlineto{\pgfqpoint{4.707824in}{1.361339in}}%
\pgfpathlineto{\pgfqpoint{4.710437in}{1.375507in}}%
\pgfpathlineto{\pgfqpoint{4.713275in}{1.386147in}}%
\pgfpathlineto{\pgfqpoint{4.715806in}{1.374102in}}%
\pgfpathlineto{\pgfqpoint{4.718486in}{1.364135in}}%
\pgfpathlineto{\pgfqpoint{4.721160in}{1.359298in}}%
\pgfpathlineto{\pgfqpoint{4.723873in}{1.360357in}}%
\pgfpathlineto{\pgfqpoint{4.726508in}{1.353592in}}%
\pgfpathlineto{\pgfqpoint{4.729233in}{1.352908in}}%
\pgfpathlineto{\pgfqpoint{4.731901in}{1.353089in}}%
\pgfpathlineto{\pgfqpoint{4.734552in}{1.352792in}}%
\pgfpathlineto{\pgfqpoint{4.737348in}{1.350034in}}%
\pgfpathlineto{\pgfqpoint{4.739912in}{1.360346in}}%
\pgfpathlineto{\pgfqpoint{4.742696in}{1.351654in}}%
\pgfpathlineto{\pgfqpoint{4.745256in}{1.350701in}}%
\pgfpathlineto{\pgfqpoint{4.748081in}{1.350618in}}%
\pgfpathlineto{\pgfqpoint{4.750627in}{1.346994in}}%
\pgfpathlineto{\pgfqpoint{4.753298in}{1.351152in}}%
\pgfpathlineto{\pgfqpoint{4.755983in}{1.350777in}}%
\pgfpathlineto{\pgfqpoint{4.758653in}{1.359809in}}%
\pgfpathlineto{\pgfqpoint{4.761337in}{1.382337in}}%
\pgfpathlineto{\pgfqpoint{4.764018in}{1.375050in}}%
\pgfpathlineto{\pgfqpoint{4.766783in}{1.362279in}}%
\pgfpathlineto{\pgfqpoint{4.769367in}{1.361986in}}%
\pgfpathlineto{\pgfqpoint{4.772198in}{1.361868in}}%
\pgfpathlineto{\pgfqpoint{4.774732in}{1.359628in}}%
\pgfpathlineto{\pgfqpoint{4.777535in}{1.360230in}}%
\pgfpathlineto{\pgfqpoint{4.780083in}{1.361468in}}%
\pgfpathlineto{\pgfqpoint{4.782872in}{1.356645in}}%
\pgfpathlineto{\pgfqpoint{4.785445in}{1.360360in}}%
\pgfpathlineto{\pgfqpoint{4.788116in}{1.357752in}}%
\pgfpathlineto{\pgfqpoint{4.790798in}{1.359431in}}%
\pgfpathlineto{\pgfqpoint{4.793512in}{1.360579in}}%
\pgfpathlineto{\pgfqpoint{4.796274in}{1.357592in}}%
\pgfpathlineto{\pgfqpoint{4.798830in}{1.357887in}}%
\pgfpathlineto{\pgfqpoint{4.801586in}{1.363285in}}%
\pgfpathlineto{\pgfqpoint{4.804193in}{1.365605in}}%
\pgfpathlineto{\pgfqpoint{4.807017in}{1.365572in}}%
\pgfpathlineto{\pgfqpoint{4.809538in}{1.369178in}}%
\pgfpathlineto{\pgfqpoint{4.812377in}{1.373235in}}%
\pgfpathlineto{\pgfqpoint{4.814907in}{1.372459in}}%
\pgfpathlineto{\pgfqpoint{4.817587in}{1.377458in}}%
\pgfpathlineto{\pgfqpoint{4.820265in}{1.372302in}}%
\pgfpathlineto{\pgfqpoint{4.822945in}{1.364779in}}%
\pgfpathlineto{\pgfqpoint{4.825619in}{1.360859in}}%
\pgfpathlineto{\pgfqpoint{4.828291in}{1.359487in}}%
\pgfpathlineto{\pgfqpoint{4.831045in}{1.355571in}}%
\pgfpathlineto{\pgfqpoint{4.833657in}{1.360370in}}%
\pgfpathlineto{\pgfqpoint{4.837992in}{1.363836in}}%
\pgfpathlineto{\pgfqpoint{4.839922in}{1.366883in}}%
\pgfpathlineto{\pgfqpoint{4.842380in}{1.365211in}}%
\pgfpathlineto{\pgfqpoint{4.844361in}{1.367097in}}%
\pgfpathlineto{\pgfqpoint{4.847127in}{1.373955in}}%
\pgfpathlineto{\pgfqpoint{4.849715in}{1.376636in}}%
\pgfpathlineto{\pgfqpoint{4.852404in}{1.393417in}}%
\pgfpathlineto{\pgfqpoint{4.855070in}{1.384843in}}%
\pgfpathlineto{\pgfqpoint{4.857807in}{1.379184in}}%
\pgfpathlineto{\pgfqpoint{4.860544in}{1.371429in}}%
\pgfpathlineto{\pgfqpoint{4.863116in}{1.371092in}}%
\pgfpathlineto{\pgfqpoint{4.865910in}{1.369783in}}%
\pgfpathlineto{\pgfqpoint{4.868474in}{1.363268in}}%
\pgfpathlineto{\pgfqpoint{4.871209in}{1.359648in}}%
\pgfpathlineto{\pgfqpoint{4.873832in}{1.356421in}}%
\pgfpathlineto{\pgfqpoint{4.876636in}{1.354807in}}%
\pgfpathlineto{\pgfqpoint{4.879180in}{1.351678in}}%
\pgfpathlineto{\pgfqpoint{4.881864in}{1.357958in}}%
\pgfpathlineto{\pgfqpoint{4.884540in}{1.354122in}}%
\pgfpathlineto{\pgfqpoint{4.887211in}{1.348333in}}%
\pgfpathlineto{\pgfqpoint{4.889902in}{1.348175in}}%
\pgfpathlineto{\pgfqpoint{4.892611in}{1.352495in}}%
\pgfpathlineto{\pgfqpoint{4.895399in}{1.353080in}}%
\pgfpathlineto{\pgfqpoint{4.897938in}{1.342695in}}%
\pgfpathlineto{\pgfqpoint{4.900712in}{1.349132in}}%
\pgfpathlineto{\pgfqpoint{4.903295in}{1.354639in}}%
\pgfpathlineto{\pgfqpoint{4.906096in}{1.364755in}}%
\pgfpathlineto{\pgfqpoint{4.908648in}{1.364099in}}%
\pgfpathlineto{\pgfqpoint{4.911435in}{1.357961in}}%
\pgfpathlineto{\pgfqpoint{4.914009in}{1.349283in}}%
\pgfpathlineto{\pgfqpoint{4.916681in}{1.346893in}}%
\pgfpathlineto{\pgfqpoint{4.919352in}{1.345752in}}%
\pgfpathlineto{\pgfqpoint{4.922041in}{1.346723in}}%
\pgfpathlineto{\pgfqpoint{4.924708in}{1.345259in}}%
\pgfpathlineto{\pgfqpoint{4.927400in}{1.346122in}}%
\pgfpathlineto{\pgfqpoint{4.930170in}{1.341832in}}%
\pgfpathlineto{\pgfqpoint{4.932742in}{1.334887in}}%
\pgfpathlineto{\pgfqpoint{4.935515in}{1.354834in}}%
\pgfpathlineto{\pgfqpoint{4.938112in}{1.367426in}}%
\pgfpathlineto{\pgfqpoint{4.940881in}{1.372109in}}%
\pgfpathlineto{\pgfqpoint{4.943466in}{1.361631in}}%
\pgfpathlineto{\pgfqpoint{4.946151in}{1.357650in}}%
\pgfpathlineto{\pgfqpoint{4.948827in}{1.358750in}}%
\pgfpathlineto{\pgfqpoint{4.951504in}{1.354643in}}%
\pgfpathlineto{\pgfqpoint{4.954182in}{1.348086in}}%
\pgfpathlineto{\pgfqpoint{4.956862in}{1.348214in}}%
\pgfpathlineto{\pgfqpoint{4.959689in}{1.343626in}}%
\pgfpathlineto{\pgfqpoint{4.962219in}{1.344163in}}%
\pgfpathlineto{\pgfqpoint{4.965002in}{1.348170in}}%
\pgfpathlineto{\pgfqpoint{4.967575in}{1.357366in}}%
\pgfpathlineto{\pgfqpoint{4.970314in}{1.358215in}}%
\pgfpathlineto{\pgfqpoint{4.972933in}{1.363495in}}%
\pgfpathlineto{\pgfqpoint{4.975703in}{1.361941in}}%
\pgfpathlineto{\pgfqpoint{4.978287in}{1.363455in}}%
\pgfpathlineto{\pgfqpoint{4.980967in}{1.361569in}}%
\pgfpathlineto{\pgfqpoint{4.983637in}{1.367758in}}%
\pgfpathlineto{\pgfqpoint{4.986325in}{1.368760in}}%
\pgfpathlineto{\pgfqpoint{4.989001in}{1.363747in}}%
\pgfpathlineto{\pgfqpoint{4.991683in}{1.362979in}}%
\pgfpathlineto{\pgfqpoint{4.994390in}{1.360811in}}%
\pgfpathlineto{\pgfqpoint{4.997028in}{1.364405in}}%
\pgfpathlineto{\pgfqpoint{4.999780in}{1.366335in}}%
\pgfpathlineto{\pgfqpoint{5.002384in}{1.359494in}}%
\pgfpathlineto{\pgfqpoint{5.005178in}{1.358040in}}%
\pgfpathlineto{\pgfqpoint{5.007751in}{1.358274in}}%
\pgfpathlineto{\pgfqpoint{5.010562in}{1.361673in}}%
\pgfpathlineto{\pgfqpoint{5.013104in}{1.363132in}}%
\pgfpathlineto{\pgfqpoint{5.015820in}{1.355246in}}%
\pgfpathlineto{\pgfqpoint{5.018466in}{1.351306in}}%
\pgfpathlineto{\pgfqpoint{5.021147in}{1.350552in}}%
\pgfpathlineto{\pgfqpoint{5.023927in}{1.355641in}}%
\pgfpathlineto{\pgfqpoint{5.026501in}{1.357748in}}%
\pgfpathlineto{\pgfqpoint{5.029275in}{1.354818in}}%
\pgfpathlineto{\pgfqpoint{5.031849in}{1.352785in}}%
\pgfpathlineto{\pgfqpoint{5.034649in}{1.346830in}}%
\pgfpathlineto{\pgfqpoint{5.037214in}{1.346219in}}%
\pgfpathlineto{\pgfqpoint{5.039962in}{1.350524in}}%
\pgfpathlineto{\pgfqpoint{5.042572in}{1.350339in}}%
\pgfpathlineto{\pgfqpoint{5.045249in}{1.351850in}}%
\pgfpathlineto{\pgfqpoint{5.047924in}{1.352963in}}%
\pgfpathlineto{\pgfqpoint{5.050606in}{1.357245in}}%
\pgfpathlineto{\pgfqpoint{5.053284in}{1.357922in}}%
\pgfpathlineto{\pgfqpoint{5.055952in}{1.361216in}}%
\pgfpathlineto{\pgfqpoint{5.058711in}{1.357590in}}%
\pgfpathlineto{\pgfqpoint{5.061315in}{1.362798in}}%
\pgfpathlineto{\pgfqpoint{5.064144in}{1.358642in}}%
\pgfpathlineto{\pgfqpoint{5.066677in}{1.355113in}}%
\pgfpathlineto{\pgfqpoint{5.069463in}{1.356139in}}%
\pgfpathlineto{\pgfqpoint{5.072030in}{1.356818in}}%
\pgfpathlineto{\pgfqpoint{5.074851in}{1.352735in}}%
\pgfpathlineto{\pgfqpoint{5.077390in}{1.359465in}}%
\pgfpathlineto{\pgfqpoint{5.080067in}{1.360470in}}%
\pgfpathlineto{\pgfqpoint{5.082746in}{1.359305in}}%
\pgfpathlineto{\pgfqpoint{5.085426in}{1.361883in}}%
\pgfpathlineto{\pgfqpoint{5.088103in}{1.359526in}}%
\pgfpathlineto{\pgfqpoint{5.090788in}{1.360655in}}%
\pgfpathlineto{\pgfqpoint{5.093579in}{1.360067in}}%
\pgfpathlineto{\pgfqpoint{5.096142in}{1.361661in}}%
\pgfpathlineto{\pgfqpoint{5.098948in}{1.358817in}}%
\pgfpathlineto{\pgfqpoint{5.101496in}{1.364189in}}%
\pgfpathlineto{\pgfqpoint{5.104312in}{1.360511in}}%
\pgfpathlineto{\pgfqpoint{5.106842in}{1.362854in}}%
\pgfpathlineto{\pgfqpoint{5.109530in}{1.362885in}}%
\pgfpathlineto{\pgfqpoint{5.112209in}{1.361188in}}%
\pgfpathlineto{\pgfqpoint{5.114887in}{1.360266in}}%
\pgfpathlineto{\pgfqpoint{5.117550in}{1.361332in}}%
\pgfpathlineto{\pgfqpoint{5.120243in}{1.361954in}}%
\pgfpathlineto{\pgfqpoint{5.123042in}{1.359349in}}%
\pgfpathlineto{\pgfqpoint{5.125599in}{1.358950in}}%
\pgfpathlineto{\pgfqpoint{5.128421in}{1.369937in}}%
\pgfpathlineto{\pgfqpoint{5.130953in}{1.359402in}}%
\pgfpathlineto{\pgfqpoint{5.133716in}{1.356538in}}%
\pgfpathlineto{\pgfqpoint{5.136311in}{1.357199in}}%
\pgfpathlineto{\pgfqpoint{5.139072in}{1.356859in}}%
\pgfpathlineto{\pgfqpoint{5.141660in}{1.353491in}}%
\pgfpathlineto{\pgfqpoint{5.144349in}{1.360460in}}%
\pgfpathlineto{\pgfqpoint{5.147029in}{1.362817in}}%
\pgfpathlineto{\pgfqpoint{5.149734in}{1.348106in}}%
\pgfpathlineto{\pgfqpoint{5.152382in}{1.344136in}}%
\pgfpathlineto{\pgfqpoint{5.155059in}{1.344102in}}%
\pgfpathlineto{\pgfqpoint{5.157815in}{1.357227in}}%
\pgfpathlineto{\pgfqpoint{5.160420in}{1.352667in}}%
\pgfpathlineto{\pgfqpoint{5.163243in}{1.359703in}}%
\pgfpathlineto{\pgfqpoint{5.165775in}{1.355147in}}%
\pgfpathlineto{\pgfqpoint{5.168591in}{1.356658in}}%
\pgfpathlineto{\pgfqpoint{5.171133in}{1.359586in}}%
\pgfpathlineto{\pgfqpoint{5.173925in}{1.360141in}}%
\pgfpathlineto{\pgfqpoint{5.176477in}{1.360561in}}%
\pgfpathlineto{\pgfqpoint{5.179188in}{1.359592in}}%
\pgfpathlineto{\pgfqpoint{5.181848in}{1.364140in}}%
\pgfpathlineto{\pgfqpoint{5.184522in}{1.355625in}}%
\pgfpathlineto{\pgfqpoint{5.187294in}{1.357615in}}%
\pgfpathlineto{\pgfqpoint{5.189880in}{1.360862in}}%
\pgfpathlineto{\pgfqpoint{5.192680in}{1.356546in}}%
\pgfpathlineto{\pgfqpoint{5.195239in}{1.353425in}}%
\pgfpathlineto{\pgfqpoint{5.198008in}{1.354442in}}%
\pgfpathlineto{\pgfqpoint{5.200594in}{1.351236in}}%
\pgfpathlineto{\pgfqpoint{5.203388in}{1.341600in}}%
\pgfpathlineto{\pgfqpoint{5.205952in}{1.350699in}}%
\pgfpathlineto{\pgfqpoint{5.208630in}{1.350780in}}%
\pgfpathlineto{\pgfqpoint{5.211299in}{1.350588in}}%
\pgfpathlineto{\pgfqpoint{5.214027in}{1.354385in}}%
\pgfpathlineto{\pgfqpoint{5.216667in}{1.353434in}}%
\pgfpathlineto{\pgfqpoint{5.219345in}{1.359609in}}%
\pgfpathlineto{\pgfqpoint{5.222151in}{1.354846in}}%
\pgfpathlineto{\pgfqpoint{5.224695in}{1.354934in}}%
\pgfpathlineto{\pgfqpoint{5.227470in}{1.358860in}}%
\pgfpathlineto{\pgfqpoint{5.230059in}{1.357924in}}%
\pgfpathlineto{\pgfqpoint{5.232855in}{1.357299in}}%
\pgfpathlineto{\pgfqpoint{5.235409in}{1.357229in}}%
\pgfpathlineto{\pgfqpoint{5.238173in}{1.353019in}}%
\pgfpathlineto{\pgfqpoint{5.240777in}{1.355666in}}%
\pgfpathlineto{\pgfqpoint{5.243445in}{1.358566in}}%
\pgfpathlineto{\pgfqpoint{5.246130in}{1.352830in}}%
\pgfpathlineto{\pgfqpoint{5.248816in}{1.357162in}}%
\pgfpathlineto{\pgfqpoint{5.251590in}{1.353779in}}%
\pgfpathlineto{\pgfqpoint{5.254236in}{1.359369in}}%
\pgfpathlineto{\pgfqpoint{5.256973in}{1.360859in}}%
\pgfpathlineto{\pgfqpoint{5.259511in}{1.358871in}}%
\pgfpathlineto{\pgfqpoint{5.262264in}{1.362899in}}%
\pgfpathlineto{\pgfqpoint{5.264876in}{1.354695in}}%
\pgfpathlineto{\pgfqpoint{5.267691in}{1.359175in}}%
\pgfpathlineto{\pgfqpoint{5.270238in}{1.357389in}}%
\pgfpathlineto{\pgfqpoint{5.272913in}{1.354010in}}%
\pgfpathlineto{\pgfqpoint{5.275589in}{1.348011in}}%
\pgfpathlineto{\pgfqpoint{5.278322in}{1.345238in}}%
\pgfpathlineto{\pgfqpoint{5.280947in}{1.346770in}}%
\pgfpathlineto{\pgfqpoint{5.283631in}{1.345642in}}%
\pgfpathlineto{\pgfqpoint{5.286436in}{1.349312in}}%
\pgfpathlineto{\pgfqpoint{5.288984in}{1.351326in}}%
\pgfpathlineto{\pgfqpoint{5.291794in}{1.355181in}}%
\pgfpathlineto{\pgfqpoint{5.294339in}{1.355831in}}%
\pgfpathlineto{\pgfqpoint{5.297140in}{1.356548in}}%
\pgfpathlineto{\pgfqpoint{5.299696in}{1.360179in}}%
\pgfpathlineto{\pgfqpoint{5.302443in}{1.356303in}}%
\pgfpathlineto{\pgfqpoint{5.305054in}{1.351689in}}%
\pgfpathlineto{\pgfqpoint{5.307731in}{1.357116in}}%
\pgfpathlineto{\pgfqpoint{5.310411in}{1.359759in}}%
\pgfpathlineto{\pgfqpoint{5.313089in}{1.354607in}}%
\pgfpathlineto{\pgfqpoint{5.315754in}{1.356091in}}%
\pgfpathlineto{\pgfqpoint{5.318430in}{1.347231in}}%
\pgfpathlineto{\pgfqpoint{5.321256in}{1.347936in}}%
\pgfpathlineto{\pgfqpoint{5.323802in}{1.340978in}}%
\pgfpathlineto{\pgfqpoint{5.326564in}{1.341502in}}%
\pgfpathlineto{\pgfqpoint{5.329159in}{1.334887in}}%
\pgfpathlineto{\pgfqpoint{5.331973in}{1.338731in}}%
\pgfpathlineto{\pgfqpoint{5.334510in}{1.357075in}}%
\pgfpathlineto{\pgfqpoint{5.337353in}{1.345776in}}%
\pgfpathlineto{\pgfqpoint{5.339872in}{1.347438in}}%
\pgfpathlineto{\pgfqpoint{5.342549in}{1.346148in}}%
\pgfpathlineto{\pgfqpoint{5.345224in}{1.363004in}}%
\pgfpathlineto{\pgfqpoint{5.347905in}{1.371667in}}%
\pgfpathlineto{\pgfqpoint{5.350723in}{1.367952in}}%
\pgfpathlineto{\pgfqpoint{5.353262in}{1.357782in}}%
\pgfpathlineto{\pgfqpoint{5.356056in}{1.357048in}}%
\pgfpathlineto{\pgfqpoint{5.358612in}{1.351830in}}%
\pgfpathlineto{\pgfqpoint{5.361370in}{1.352670in}}%
\pgfpathlineto{\pgfqpoint{5.363966in}{1.346512in}}%
\pgfpathlineto{\pgfqpoint{5.366727in}{1.347246in}}%
\pgfpathlineto{\pgfqpoint{5.369335in}{1.351803in}}%
\pgfpathlineto{\pgfqpoint{5.372013in}{1.346076in}}%
\pgfpathlineto{\pgfqpoint{5.374692in}{1.350924in}}%
\pgfpathlineto{\pgfqpoint{5.377370in}{1.353150in}}%
\pgfpathlineto{\pgfqpoint{5.380048in}{1.357872in}}%
\pgfpathlineto{\pgfqpoint{5.382725in}{1.359528in}}%
\pgfpathlineto{\pgfqpoint{5.385550in}{1.360251in}}%
\pgfpathlineto{\pgfqpoint{5.388083in}{1.388709in}}%
\pgfpathlineto{\pgfqpoint{5.390900in}{1.470534in}}%
\pgfpathlineto{\pgfqpoint{5.393441in}{1.500566in}}%
\pgfpathlineto{\pgfqpoint{5.396219in}{1.470652in}}%
\pgfpathlineto{\pgfqpoint{5.398784in}{1.439821in}}%
\pgfpathlineto{\pgfqpoint{5.401576in}{1.428745in}}%
\pgfpathlineto{\pgfqpoint{5.404154in}{1.412965in}}%
\pgfpathlineto{\pgfqpoint{5.406832in}{1.400525in}}%
\pgfpathlineto{\pgfqpoint{5.409507in}{1.396406in}}%
\pgfpathlineto{\pgfqpoint{5.412190in}{1.392463in}}%
\pgfpathlineto{\pgfqpoint{5.414954in}{1.388254in}}%
\pgfpathlineto{\pgfqpoint{5.417547in}{1.386729in}}%
\pgfpathlineto{\pgfqpoint{5.420304in}{1.381461in}}%
\pgfpathlineto{\pgfqpoint{5.422897in}{1.382857in}}%
\pgfpathlineto{\pgfqpoint{5.425661in}{1.377088in}}%
\pgfpathlineto{\pgfqpoint{5.428259in}{1.379936in}}%
\pgfpathlineto{\pgfqpoint{5.431015in}{1.374452in}}%
\pgfpathlineto{\pgfqpoint{5.433616in}{1.370044in}}%
\pgfpathlineto{\pgfqpoint{5.436295in}{1.370008in}}%
\pgfpathlineto{\pgfqpoint{5.438974in}{1.374552in}}%
\pgfpathlineto{\pgfqpoint{5.441698in}{1.364604in}}%
\pgfpathlineto{\pgfqpoint{5.444328in}{1.366383in}}%
\pgfpathlineto{\pgfqpoint{5.447021in}{1.361790in}}%
\pgfpathlineto{\pgfqpoint{5.449769in}{1.356185in}}%
\pgfpathlineto{\pgfqpoint{5.452365in}{1.359349in}}%
\pgfpathlineto{\pgfqpoint{5.455168in}{1.358771in}}%
\pgfpathlineto{\pgfqpoint{5.457721in}{1.352122in}}%
\pgfpathlineto{\pgfqpoint{5.460489in}{1.353950in}}%
\pgfpathlineto{\pgfqpoint{5.463079in}{1.358584in}}%
\pgfpathlineto{\pgfqpoint{5.465888in}{1.355163in}}%
\pgfpathlineto{\pgfqpoint{5.468425in}{1.361021in}}%
\pgfpathlineto{\pgfqpoint{5.471113in}{1.364040in}}%
\pgfpathlineto{\pgfqpoint{5.473792in}{1.364272in}}%
\pgfpathlineto{\pgfqpoint{5.476458in}{1.367212in}}%
\pgfpathlineto{\pgfqpoint{5.479152in}{1.366222in}}%
\pgfpathlineto{\pgfqpoint{5.481825in}{1.362896in}}%
\pgfpathlineto{\pgfqpoint{5.484641in}{1.370127in}}%
\pgfpathlineto{\pgfqpoint{5.487176in}{1.368240in}}%
\pgfpathlineto{\pgfqpoint{5.490000in}{1.364709in}}%
\pgfpathlineto{\pgfqpoint{5.492541in}{1.364205in}}%
\pgfpathlineto{\pgfqpoint{5.495346in}{1.365759in}}%
\pgfpathlineto{\pgfqpoint{5.497898in}{1.364762in}}%
\pgfpathlineto{\pgfqpoint{5.500687in}{1.369559in}}%
\pgfpathlineto{\pgfqpoint{5.503255in}{1.366217in}}%
\pgfpathlineto{\pgfqpoint{5.505933in}{1.366854in}}%
\pgfpathlineto{\pgfqpoint{5.508612in}{1.363474in}}%
\pgfpathlineto{\pgfqpoint{5.511290in}{1.361422in}}%
\pgfpathlineto{\pgfqpoint{5.514080in}{1.363675in}}%
\pgfpathlineto{\pgfqpoint{5.516646in}{1.369444in}}%
\pgfpathlineto{\pgfqpoint{5.519433in}{1.366416in}}%
\pgfpathlineto{\pgfqpoint{5.522003in}{1.366539in}}%
\pgfpathlineto{\pgfqpoint{5.524756in}{1.366919in}}%
\pgfpathlineto{\pgfqpoint{5.527360in}{1.364317in}}%
\pgfpathlineto{\pgfqpoint{5.530148in}{1.365296in}}%
\pgfpathlineto{\pgfqpoint{5.532717in}{1.369338in}}%
\pgfpathlineto{\pgfqpoint{5.535395in}{1.364291in}}%
\pgfpathlineto{\pgfqpoint{5.538074in}{1.368150in}}%
\pgfpathlineto{\pgfqpoint{5.540750in}{1.361605in}}%
\pgfpathlineto{\pgfqpoint{5.543421in}{1.362774in}}%
\pgfpathlineto{\pgfqpoint{5.546110in}{1.360466in}}%
\pgfpathlineto{\pgfqpoint{5.548921in}{1.364306in}}%
\pgfpathlineto{\pgfqpoint{5.551457in}{1.360381in}}%
\pgfpathlineto{\pgfqpoint{5.554198in}{1.360081in}}%
\pgfpathlineto{\pgfqpoint{5.556822in}{1.357501in}}%
\pgfpathlineto{\pgfqpoint{5.559612in}{1.360096in}}%
\pgfpathlineto{\pgfqpoint{5.562180in}{1.353067in}}%
\pgfpathlineto{\pgfqpoint{5.564940in}{1.354826in}}%
\pgfpathlineto{\pgfqpoint{5.567536in}{1.358949in}}%
\pgfpathlineto{\pgfqpoint{5.570215in}{1.358014in}}%
\pgfpathlineto{\pgfqpoint{5.572893in}{1.358227in}}%
\pgfpathlineto{\pgfqpoint{5.575596in}{1.362524in}}%
\pgfpathlineto{\pgfqpoint{5.578342in}{1.356604in}}%
\pgfpathlineto{\pgfqpoint{5.580914in}{1.355495in}}%
\pgfpathlineto{\pgfqpoint{5.583709in}{1.363044in}}%
\pgfpathlineto{\pgfqpoint{5.586269in}{1.360759in}}%
\pgfpathlineto{\pgfqpoint{5.589040in}{1.353862in}}%
\pgfpathlineto{\pgfqpoint{5.591641in}{1.357851in}}%
\pgfpathlineto{\pgfqpoint{5.594368in}{1.360155in}}%
\pgfpathlineto{\pgfqpoint{5.596999in}{1.353025in}}%
\pgfpathlineto{\pgfqpoint{5.599674in}{1.357980in}}%
\pgfpathlineto{\pgfqpoint{5.602352in}{1.352114in}}%
\pgfpathlineto{\pgfqpoint{5.605073in}{1.348203in}}%
\pgfpathlineto{\pgfqpoint{5.607698in}{1.342623in}}%
\pgfpathlineto{\pgfqpoint{5.610389in}{1.340954in}}%
\pgfpathlineto{\pgfqpoint{5.613235in}{1.347794in}}%
\pgfpathlineto{\pgfqpoint{5.615743in}{1.346106in}}%
\pgfpathlineto{\pgfqpoint{5.618526in}{1.344254in}}%
\pgfpathlineto{\pgfqpoint{5.621102in}{1.353969in}}%
\pgfpathlineto{\pgfqpoint{5.623868in}{1.356158in}}%
\pgfpathlineto{\pgfqpoint{5.626460in}{1.349802in}}%
\pgfpathlineto{\pgfqpoint{5.629232in}{1.351583in}}%
\pgfpathlineto{\pgfqpoint{5.631815in}{1.352003in}}%
\pgfpathlineto{\pgfqpoint{5.634496in}{1.356115in}}%
\pgfpathlineto{\pgfqpoint{5.637172in}{1.356344in}}%
\pgfpathlineto{\pgfqpoint{5.639852in}{1.354162in}}%
\pgfpathlineto{\pgfqpoint{5.642518in}{1.360943in}}%
\pgfpathlineto{\pgfqpoint{5.645243in}{1.351749in}}%
\pgfpathlineto{\pgfqpoint{5.648008in}{1.368096in}}%
\pgfpathlineto{\pgfqpoint{5.650563in}{1.353606in}}%
\pgfpathlineto{\pgfqpoint{5.653376in}{1.338168in}}%
\pgfpathlineto{\pgfqpoint{5.655919in}{1.335676in}}%
\pgfpathlineto{\pgfqpoint{5.658723in}{1.334887in}}%
\pgfpathlineto{\pgfqpoint{5.661273in}{1.334887in}}%
\pgfpathlineto{\pgfqpoint{5.664099in}{1.337286in}}%
\pgfpathlineto{\pgfqpoint{5.666632in}{1.353397in}}%
\pgfpathlineto{\pgfqpoint{5.669313in}{1.348959in}}%
\pgfpathlineto{\pgfqpoint{5.671991in}{1.350696in}}%
\pgfpathlineto{\pgfqpoint{5.674667in}{1.345857in}}%
\pgfpathlineto{\pgfqpoint{5.677486in}{1.352608in}}%
\pgfpathlineto{\pgfqpoint{5.680027in}{1.353999in}}%
\pgfpathlineto{\pgfqpoint{5.682836in}{1.357129in}}%
\pgfpathlineto{\pgfqpoint{5.685385in}{1.355144in}}%
\pgfpathlineto{\pgfqpoint{5.688159in}{1.354496in}}%
\pgfpathlineto{\pgfqpoint{5.690730in}{1.363335in}}%
\pgfpathlineto{\pgfqpoint{5.693473in}{1.361441in}}%
\pgfpathlineto{\pgfqpoint{5.696101in}{1.358123in}}%
\pgfpathlineto{\pgfqpoint{5.698775in}{1.357749in}}%
\pgfpathlineto{\pgfqpoint{5.701453in}{1.353374in}}%
\pgfpathlineto{\pgfqpoint{5.704130in}{1.343794in}}%
\pgfpathlineto{\pgfqpoint{5.706800in}{1.347320in}}%
\pgfpathlineto{\pgfqpoint{5.709490in}{1.347522in}}%
\pgfpathlineto{\pgfqpoint{5.712291in}{1.346837in}}%
\pgfpathlineto{\pgfqpoint{5.714834in}{1.354622in}}%
\pgfpathlineto{\pgfqpoint{5.717671in}{1.356732in}}%
\pgfpathlineto{\pgfqpoint{5.720201in}{1.355806in}}%
\pgfpathlineto{\pgfqpoint{5.722950in}{1.355674in}}%
\pgfpathlineto{\pgfqpoint{5.725548in}{1.359882in}}%
\pgfpathlineto{\pgfqpoint{5.728339in}{1.362721in}}%
\pgfpathlineto{\pgfqpoint{5.730919in}{1.370219in}}%
\pgfpathlineto{\pgfqpoint{5.733594in}{1.363010in}}%
\pgfpathlineto{\pgfqpoint{5.736276in}{1.357178in}}%
\pgfpathlineto{\pgfqpoint{5.738974in}{1.358412in}}%
\pgfpathlineto{\pgfqpoint{5.741745in}{1.359120in}}%
\pgfpathlineto{\pgfqpoint{5.744310in}{1.358396in}}%
\pgfpathlineto{\pgfqpoint{5.744310in}{0.413320in}}%
\pgfpathlineto{\pgfqpoint{5.744310in}{0.413320in}}%
\pgfpathlineto{\pgfqpoint{5.741745in}{0.413320in}}%
\pgfpathlineto{\pgfqpoint{5.738974in}{0.413320in}}%
\pgfpathlineto{\pgfqpoint{5.736276in}{0.413320in}}%
\pgfpathlineto{\pgfqpoint{5.733594in}{0.413320in}}%
\pgfpathlineto{\pgfqpoint{5.730919in}{0.413320in}}%
\pgfpathlineto{\pgfqpoint{5.728339in}{0.413320in}}%
\pgfpathlineto{\pgfqpoint{5.725548in}{0.413320in}}%
\pgfpathlineto{\pgfqpoint{5.722950in}{0.413320in}}%
\pgfpathlineto{\pgfqpoint{5.720201in}{0.413320in}}%
\pgfpathlineto{\pgfqpoint{5.717671in}{0.413320in}}%
\pgfpathlineto{\pgfqpoint{5.714834in}{0.413320in}}%
\pgfpathlineto{\pgfqpoint{5.712291in}{0.413320in}}%
\pgfpathlineto{\pgfqpoint{5.709490in}{0.413320in}}%
\pgfpathlineto{\pgfqpoint{5.706800in}{0.413320in}}%
\pgfpathlineto{\pgfqpoint{5.704130in}{0.413320in}}%
\pgfpathlineto{\pgfqpoint{5.701453in}{0.413320in}}%
\pgfpathlineto{\pgfqpoint{5.698775in}{0.413320in}}%
\pgfpathlineto{\pgfqpoint{5.696101in}{0.413320in}}%
\pgfpathlineto{\pgfqpoint{5.693473in}{0.413320in}}%
\pgfpathlineto{\pgfqpoint{5.690730in}{0.413320in}}%
\pgfpathlineto{\pgfqpoint{5.688159in}{0.413320in}}%
\pgfpathlineto{\pgfqpoint{5.685385in}{0.413320in}}%
\pgfpathlineto{\pgfqpoint{5.682836in}{0.413320in}}%
\pgfpathlineto{\pgfqpoint{5.680027in}{0.413320in}}%
\pgfpathlineto{\pgfqpoint{5.677486in}{0.413320in}}%
\pgfpathlineto{\pgfqpoint{5.674667in}{0.413320in}}%
\pgfpathlineto{\pgfqpoint{5.671991in}{0.413320in}}%
\pgfpathlineto{\pgfqpoint{5.669313in}{0.413320in}}%
\pgfpathlineto{\pgfqpoint{5.666632in}{0.413320in}}%
\pgfpathlineto{\pgfqpoint{5.664099in}{0.413320in}}%
\pgfpathlineto{\pgfqpoint{5.661273in}{0.413320in}}%
\pgfpathlineto{\pgfqpoint{5.658723in}{0.413320in}}%
\pgfpathlineto{\pgfqpoint{5.655919in}{0.413320in}}%
\pgfpathlineto{\pgfqpoint{5.653376in}{0.413320in}}%
\pgfpathlineto{\pgfqpoint{5.650563in}{0.413320in}}%
\pgfpathlineto{\pgfqpoint{5.648008in}{0.413320in}}%
\pgfpathlineto{\pgfqpoint{5.645243in}{0.413320in}}%
\pgfpathlineto{\pgfqpoint{5.642518in}{0.413320in}}%
\pgfpathlineto{\pgfqpoint{5.639852in}{0.413320in}}%
\pgfpathlineto{\pgfqpoint{5.637172in}{0.413320in}}%
\pgfpathlineto{\pgfqpoint{5.634496in}{0.413320in}}%
\pgfpathlineto{\pgfqpoint{5.631815in}{0.413320in}}%
\pgfpathlineto{\pgfqpoint{5.629232in}{0.413320in}}%
\pgfpathlineto{\pgfqpoint{5.626460in}{0.413320in}}%
\pgfpathlineto{\pgfqpoint{5.623868in}{0.413320in}}%
\pgfpathlineto{\pgfqpoint{5.621102in}{0.413320in}}%
\pgfpathlineto{\pgfqpoint{5.618526in}{0.413320in}}%
\pgfpathlineto{\pgfqpoint{5.615743in}{0.413320in}}%
\pgfpathlineto{\pgfqpoint{5.613235in}{0.413320in}}%
\pgfpathlineto{\pgfqpoint{5.610389in}{0.413320in}}%
\pgfpathlineto{\pgfqpoint{5.607698in}{0.413320in}}%
\pgfpathlineto{\pgfqpoint{5.605073in}{0.413320in}}%
\pgfpathlineto{\pgfqpoint{5.602352in}{0.413320in}}%
\pgfpathlineto{\pgfqpoint{5.599674in}{0.413320in}}%
\pgfpathlineto{\pgfqpoint{5.596999in}{0.413320in}}%
\pgfpathlineto{\pgfqpoint{5.594368in}{0.413320in}}%
\pgfpathlineto{\pgfqpoint{5.591641in}{0.413320in}}%
\pgfpathlineto{\pgfqpoint{5.589040in}{0.413320in}}%
\pgfpathlineto{\pgfqpoint{5.586269in}{0.413320in}}%
\pgfpathlineto{\pgfqpoint{5.583709in}{0.413320in}}%
\pgfpathlineto{\pgfqpoint{5.580914in}{0.413320in}}%
\pgfpathlineto{\pgfqpoint{5.578342in}{0.413320in}}%
\pgfpathlineto{\pgfqpoint{5.575596in}{0.413320in}}%
\pgfpathlineto{\pgfqpoint{5.572893in}{0.413320in}}%
\pgfpathlineto{\pgfqpoint{5.570215in}{0.413320in}}%
\pgfpathlineto{\pgfqpoint{5.567536in}{0.413320in}}%
\pgfpathlineto{\pgfqpoint{5.564940in}{0.413320in}}%
\pgfpathlineto{\pgfqpoint{5.562180in}{0.413320in}}%
\pgfpathlineto{\pgfqpoint{5.559612in}{0.413320in}}%
\pgfpathlineto{\pgfqpoint{5.556822in}{0.413320in}}%
\pgfpathlineto{\pgfqpoint{5.554198in}{0.413320in}}%
\pgfpathlineto{\pgfqpoint{5.551457in}{0.413320in}}%
\pgfpathlineto{\pgfqpoint{5.548921in}{0.413320in}}%
\pgfpathlineto{\pgfqpoint{5.546110in}{0.413320in}}%
\pgfpathlineto{\pgfqpoint{5.543421in}{0.413320in}}%
\pgfpathlineto{\pgfqpoint{5.540750in}{0.413320in}}%
\pgfpathlineto{\pgfqpoint{5.538074in}{0.413320in}}%
\pgfpathlineto{\pgfqpoint{5.535395in}{0.413320in}}%
\pgfpathlineto{\pgfqpoint{5.532717in}{0.413320in}}%
\pgfpathlineto{\pgfqpoint{5.530148in}{0.413320in}}%
\pgfpathlineto{\pgfqpoint{5.527360in}{0.413320in}}%
\pgfpathlineto{\pgfqpoint{5.524756in}{0.413320in}}%
\pgfpathlineto{\pgfqpoint{5.522003in}{0.413320in}}%
\pgfpathlineto{\pgfqpoint{5.519433in}{0.413320in}}%
\pgfpathlineto{\pgfqpoint{5.516646in}{0.413320in}}%
\pgfpathlineto{\pgfqpoint{5.514080in}{0.413320in}}%
\pgfpathlineto{\pgfqpoint{5.511290in}{0.413320in}}%
\pgfpathlineto{\pgfqpoint{5.508612in}{0.413320in}}%
\pgfpathlineto{\pgfqpoint{5.505933in}{0.413320in}}%
\pgfpathlineto{\pgfqpoint{5.503255in}{0.413320in}}%
\pgfpathlineto{\pgfqpoint{5.500687in}{0.413320in}}%
\pgfpathlineto{\pgfqpoint{5.497898in}{0.413320in}}%
\pgfpathlineto{\pgfqpoint{5.495346in}{0.413320in}}%
\pgfpathlineto{\pgfqpoint{5.492541in}{0.413320in}}%
\pgfpathlineto{\pgfqpoint{5.490000in}{0.413320in}}%
\pgfpathlineto{\pgfqpoint{5.487176in}{0.413320in}}%
\pgfpathlineto{\pgfqpoint{5.484641in}{0.413320in}}%
\pgfpathlineto{\pgfqpoint{5.481825in}{0.413320in}}%
\pgfpathlineto{\pgfqpoint{5.479152in}{0.413320in}}%
\pgfpathlineto{\pgfqpoint{5.476458in}{0.413320in}}%
\pgfpathlineto{\pgfqpoint{5.473792in}{0.413320in}}%
\pgfpathlineto{\pgfqpoint{5.471113in}{0.413320in}}%
\pgfpathlineto{\pgfqpoint{5.468425in}{0.413320in}}%
\pgfpathlineto{\pgfqpoint{5.465888in}{0.413320in}}%
\pgfpathlineto{\pgfqpoint{5.463079in}{0.413320in}}%
\pgfpathlineto{\pgfqpoint{5.460489in}{0.413320in}}%
\pgfpathlineto{\pgfqpoint{5.457721in}{0.413320in}}%
\pgfpathlineto{\pgfqpoint{5.455168in}{0.413320in}}%
\pgfpathlineto{\pgfqpoint{5.452365in}{0.413320in}}%
\pgfpathlineto{\pgfqpoint{5.449769in}{0.413320in}}%
\pgfpathlineto{\pgfqpoint{5.447021in}{0.413320in}}%
\pgfpathlineto{\pgfqpoint{5.444328in}{0.413320in}}%
\pgfpathlineto{\pgfqpoint{5.441698in}{0.413320in}}%
\pgfpathlineto{\pgfqpoint{5.438974in}{0.413320in}}%
\pgfpathlineto{\pgfqpoint{5.436295in}{0.413320in}}%
\pgfpathlineto{\pgfqpoint{5.433616in}{0.413320in}}%
\pgfpathlineto{\pgfqpoint{5.431015in}{0.413320in}}%
\pgfpathlineto{\pgfqpoint{5.428259in}{0.413320in}}%
\pgfpathlineto{\pgfqpoint{5.425661in}{0.413320in}}%
\pgfpathlineto{\pgfqpoint{5.422897in}{0.413320in}}%
\pgfpathlineto{\pgfqpoint{5.420304in}{0.413320in}}%
\pgfpathlineto{\pgfqpoint{5.417547in}{0.413320in}}%
\pgfpathlineto{\pgfqpoint{5.414954in}{0.413320in}}%
\pgfpathlineto{\pgfqpoint{5.412190in}{0.413320in}}%
\pgfpathlineto{\pgfqpoint{5.409507in}{0.413320in}}%
\pgfpathlineto{\pgfqpoint{5.406832in}{0.413320in}}%
\pgfpathlineto{\pgfqpoint{5.404154in}{0.413320in}}%
\pgfpathlineto{\pgfqpoint{5.401576in}{0.413320in}}%
\pgfpathlineto{\pgfqpoint{5.398784in}{0.413320in}}%
\pgfpathlineto{\pgfqpoint{5.396219in}{0.413320in}}%
\pgfpathlineto{\pgfqpoint{5.393441in}{0.413320in}}%
\pgfpathlineto{\pgfqpoint{5.390900in}{0.413320in}}%
\pgfpathlineto{\pgfqpoint{5.388083in}{0.413320in}}%
\pgfpathlineto{\pgfqpoint{5.385550in}{0.413320in}}%
\pgfpathlineto{\pgfqpoint{5.382725in}{0.413320in}}%
\pgfpathlineto{\pgfqpoint{5.380048in}{0.413320in}}%
\pgfpathlineto{\pgfqpoint{5.377370in}{0.413320in}}%
\pgfpathlineto{\pgfqpoint{5.374692in}{0.413320in}}%
\pgfpathlineto{\pgfqpoint{5.372013in}{0.413320in}}%
\pgfpathlineto{\pgfqpoint{5.369335in}{0.413320in}}%
\pgfpathlineto{\pgfqpoint{5.366727in}{0.413320in}}%
\pgfpathlineto{\pgfqpoint{5.363966in}{0.413320in}}%
\pgfpathlineto{\pgfqpoint{5.361370in}{0.413320in}}%
\pgfpathlineto{\pgfqpoint{5.358612in}{0.413320in}}%
\pgfpathlineto{\pgfqpoint{5.356056in}{0.413320in}}%
\pgfpathlineto{\pgfqpoint{5.353262in}{0.413320in}}%
\pgfpathlineto{\pgfqpoint{5.350723in}{0.413320in}}%
\pgfpathlineto{\pgfqpoint{5.347905in}{0.413320in}}%
\pgfpathlineto{\pgfqpoint{5.345224in}{0.413320in}}%
\pgfpathlineto{\pgfqpoint{5.342549in}{0.413320in}}%
\pgfpathlineto{\pgfqpoint{5.339872in}{0.413320in}}%
\pgfpathlineto{\pgfqpoint{5.337353in}{0.413320in}}%
\pgfpathlineto{\pgfqpoint{5.334510in}{0.413320in}}%
\pgfpathlineto{\pgfqpoint{5.331973in}{0.413320in}}%
\pgfpathlineto{\pgfqpoint{5.329159in}{0.413320in}}%
\pgfpathlineto{\pgfqpoint{5.326564in}{0.413320in}}%
\pgfpathlineto{\pgfqpoint{5.323802in}{0.413320in}}%
\pgfpathlineto{\pgfqpoint{5.321256in}{0.413320in}}%
\pgfpathlineto{\pgfqpoint{5.318430in}{0.413320in}}%
\pgfpathlineto{\pgfqpoint{5.315754in}{0.413320in}}%
\pgfpathlineto{\pgfqpoint{5.313089in}{0.413320in}}%
\pgfpathlineto{\pgfqpoint{5.310411in}{0.413320in}}%
\pgfpathlineto{\pgfqpoint{5.307731in}{0.413320in}}%
\pgfpathlineto{\pgfqpoint{5.305054in}{0.413320in}}%
\pgfpathlineto{\pgfqpoint{5.302443in}{0.413320in}}%
\pgfpathlineto{\pgfqpoint{5.299696in}{0.413320in}}%
\pgfpathlineto{\pgfqpoint{5.297140in}{0.413320in}}%
\pgfpathlineto{\pgfqpoint{5.294339in}{0.413320in}}%
\pgfpathlineto{\pgfqpoint{5.291794in}{0.413320in}}%
\pgfpathlineto{\pgfqpoint{5.288984in}{0.413320in}}%
\pgfpathlineto{\pgfqpoint{5.286436in}{0.413320in}}%
\pgfpathlineto{\pgfqpoint{5.283631in}{0.413320in}}%
\pgfpathlineto{\pgfqpoint{5.280947in}{0.413320in}}%
\pgfpathlineto{\pgfqpoint{5.278322in}{0.413320in}}%
\pgfpathlineto{\pgfqpoint{5.275589in}{0.413320in}}%
\pgfpathlineto{\pgfqpoint{5.272913in}{0.413320in}}%
\pgfpathlineto{\pgfqpoint{5.270238in}{0.413320in}}%
\pgfpathlineto{\pgfqpoint{5.267691in}{0.413320in}}%
\pgfpathlineto{\pgfqpoint{5.264876in}{0.413320in}}%
\pgfpathlineto{\pgfqpoint{5.262264in}{0.413320in}}%
\pgfpathlineto{\pgfqpoint{5.259511in}{0.413320in}}%
\pgfpathlineto{\pgfqpoint{5.256973in}{0.413320in}}%
\pgfpathlineto{\pgfqpoint{5.254236in}{0.413320in}}%
\pgfpathlineto{\pgfqpoint{5.251590in}{0.413320in}}%
\pgfpathlineto{\pgfqpoint{5.248816in}{0.413320in}}%
\pgfpathlineto{\pgfqpoint{5.246130in}{0.413320in}}%
\pgfpathlineto{\pgfqpoint{5.243445in}{0.413320in}}%
\pgfpathlineto{\pgfqpoint{5.240777in}{0.413320in}}%
\pgfpathlineto{\pgfqpoint{5.238173in}{0.413320in}}%
\pgfpathlineto{\pgfqpoint{5.235409in}{0.413320in}}%
\pgfpathlineto{\pgfqpoint{5.232855in}{0.413320in}}%
\pgfpathlineto{\pgfqpoint{5.230059in}{0.413320in}}%
\pgfpathlineto{\pgfqpoint{5.227470in}{0.413320in}}%
\pgfpathlineto{\pgfqpoint{5.224695in}{0.413320in}}%
\pgfpathlineto{\pgfqpoint{5.222151in}{0.413320in}}%
\pgfpathlineto{\pgfqpoint{5.219345in}{0.413320in}}%
\pgfpathlineto{\pgfqpoint{5.216667in}{0.413320in}}%
\pgfpathlineto{\pgfqpoint{5.214027in}{0.413320in}}%
\pgfpathlineto{\pgfqpoint{5.211299in}{0.413320in}}%
\pgfpathlineto{\pgfqpoint{5.208630in}{0.413320in}}%
\pgfpathlineto{\pgfqpoint{5.205952in}{0.413320in}}%
\pgfpathlineto{\pgfqpoint{5.203388in}{0.413320in}}%
\pgfpathlineto{\pgfqpoint{5.200594in}{0.413320in}}%
\pgfpathlineto{\pgfqpoint{5.198008in}{0.413320in}}%
\pgfpathlineto{\pgfqpoint{5.195239in}{0.413320in}}%
\pgfpathlineto{\pgfqpoint{5.192680in}{0.413320in}}%
\pgfpathlineto{\pgfqpoint{5.189880in}{0.413320in}}%
\pgfpathlineto{\pgfqpoint{5.187294in}{0.413320in}}%
\pgfpathlineto{\pgfqpoint{5.184522in}{0.413320in}}%
\pgfpathlineto{\pgfqpoint{5.181848in}{0.413320in}}%
\pgfpathlineto{\pgfqpoint{5.179188in}{0.413320in}}%
\pgfpathlineto{\pgfqpoint{5.176477in}{0.413320in}}%
\pgfpathlineto{\pgfqpoint{5.173925in}{0.413320in}}%
\pgfpathlineto{\pgfqpoint{5.171133in}{0.413320in}}%
\pgfpathlineto{\pgfqpoint{5.168591in}{0.413320in}}%
\pgfpathlineto{\pgfqpoint{5.165775in}{0.413320in}}%
\pgfpathlineto{\pgfqpoint{5.163243in}{0.413320in}}%
\pgfpathlineto{\pgfqpoint{5.160420in}{0.413320in}}%
\pgfpathlineto{\pgfqpoint{5.157815in}{0.413320in}}%
\pgfpathlineto{\pgfqpoint{5.155059in}{0.413320in}}%
\pgfpathlineto{\pgfqpoint{5.152382in}{0.413320in}}%
\pgfpathlineto{\pgfqpoint{5.149734in}{0.413320in}}%
\pgfpathlineto{\pgfqpoint{5.147029in}{0.413320in}}%
\pgfpathlineto{\pgfqpoint{5.144349in}{0.413320in}}%
\pgfpathlineto{\pgfqpoint{5.141660in}{0.413320in}}%
\pgfpathlineto{\pgfqpoint{5.139072in}{0.413320in}}%
\pgfpathlineto{\pgfqpoint{5.136311in}{0.413320in}}%
\pgfpathlineto{\pgfqpoint{5.133716in}{0.413320in}}%
\pgfpathlineto{\pgfqpoint{5.130953in}{0.413320in}}%
\pgfpathlineto{\pgfqpoint{5.128421in}{0.413320in}}%
\pgfpathlineto{\pgfqpoint{5.125599in}{0.413320in}}%
\pgfpathlineto{\pgfqpoint{5.123042in}{0.413320in}}%
\pgfpathlineto{\pgfqpoint{5.120243in}{0.413320in}}%
\pgfpathlineto{\pgfqpoint{5.117550in}{0.413320in}}%
\pgfpathlineto{\pgfqpoint{5.114887in}{0.413320in}}%
\pgfpathlineto{\pgfqpoint{5.112209in}{0.413320in}}%
\pgfpathlineto{\pgfqpoint{5.109530in}{0.413320in}}%
\pgfpathlineto{\pgfqpoint{5.106842in}{0.413320in}}%
\pgfpathlineto{\pgfqpoint{5.104312in}{0.413320in}}%
\pgfpathlineto{\pgfqpoint{5.101496in}{0.413320in}}%
\pgfpathlineto{\pgfqpoint{5.098948in}{0.413320in}}%
\pgfpathlineto{\pgfqpoint{5.096142in}{0.413320in}}%
\pgfpathlineto{\pgfqpoint{5.093579in}{0.413320in}}%
\pgfpathlineto{\pgfqpoint{5.090788in}{0.413320in}}%
\pgfpathlineto{\pgfqpoint{5.088103in}{0.413320in}}%
\pgfpathlineto{\pgfqpoint{5.085426in}{0.413320in}}%
\pgfpathlineto{\pgfqpoint{5.082746in}{0.413320in}}%
\pgfpathlineto{\pgfqpoint{5.080067in}{0.413320in}}%
\pgfpathlineto{\pgfqpoint{5.077390in}{0.413320in}}%
\pgfpathlineto{\pgfqpoint{5.074851in}{0.413320in}}%
\pgfpathlineto{\pgfqpoint{5.072030in}{0.413320in}}%
\pgfpathlineto{\pgfqpoint{5.069463in}{0.413320in}}%
\pgfpathlineto{\pgfqpoint{5.066677in}{0.413320in}}%
\pgfpathlineto{\pgfqpoint{5.064144in}{0.413320in}}%
\pgfpathlineto{\pgfqpoint{5.061315in}{0.413320in}}%
\pgfpathlineto{\pgfqpoint{5.058711in}{0.413320in}}%
\pgfpathlineto{\pgfqpoint{5.055952in}{0.413320in}}%
\pgfpathlineto{\pgfqpoint{5.053284in}{0.413320in}}%
\pgfpathlineto{\pgfqpoint{5.050606in}{0.413320in}}%
\pgfpathlineto{\pgfqpoint{5.047924in}{0.413320in}}%
\pgfpathlineto{\pgfqpoint{5.045249in}{0.413320in}}%
\pgfpathlineto{\pgfqpoint{5.042572in}{0.413320in}}%
\pgfpathlineto{\pgfqpoint{5.039962in}{0.413320in}}%
\pgfpathlineto{\pgfqpoint{5.037214in}{0.413320in}}%
\pgfpathlineto{\pgfqpoint{5.034649in}{0.413320in}}%
\pgfpathlineto{\pgfqpoint{5.031849in}{0.413320in}}%
\pgfpathlineto{\pgfqpoint{5.029275in}{0.413320in}}%
\pgfpathlineto{\pgfqpoint{5.026501in}{0.413320in}}%
\pgfpathlineto{\pgfqpoint{5.023927in}{0.413320in}}%
\pgfpathlineto{\pgfqpoint{5.021147in}{0.413320in}}%
\pgfpathlineto{\pgfqpoint{5.018466in}{0.413320in}}%
\pgfpathlineto{\pgfqpoint{5.015820in}{0.413320in}}%
\pgfpathlineto{\pgfqpoint{5.013104in}{0.413320in}}%
\pgfpathlineto{\pgfqpoint{5.010562in}{0.413320in}}%
\pgfpathlineto{\pgfqpoint{5.007751in}{0.413320in}}%
\pgfpathlineto{\pgfqpoint{5.005178in}{0.413320in}}%
\pgfpathlineto{\pgfqpoint{5.002384in}{0.413320in}}%
\pgfpathlineto{\pgfqpoint{4.999780in}{0.413320in}}%
\pgfpathlineto{\pgfqpoint{4.997028in}{0.413320in}}%
\pgfpathlineto{\pgfqpoint{4.994390in}{0.413320in}}%
\pgfpathlineto{\pgfqpoint{4.991683in}{0.413320in}}%
\pgfpathlineto{\pgfqpoint{4.989001in}{0.413320in}}%
\pgfpathlineto{\pgfqpoint{4.986325in}{0.413320in}}%
\pgfpathlineto{\pgfqpoint{4.983637in}{0.413320in}}%
\pgfpathlineto{\pgfqpoint{4.980967in}{0.413320in}}%
\pgfpathlineto{\pgfqpoint{4.978287in}{0.413320in}}%
\pgfpathlineto{\pgfqpoint{4.975703in}{0.413320in}}%
\pgfpathlineto{\pgfqpoint{4.972933in}{0.413320in}}%
\pgfpathlineto{\pgfqpoint{4.970314in}{0.413320in}}%
\pgfpathlineto{\pgfqpoint{4.967575in}{0.413320in}}%
\pgfpathlineto{\pgfqpoint{4.965002in}{0.413320in}}%
\pgfpathlineto{\pgfqpoint{4.962219in}{0.413320in}}%
\pgfpathlineto{\pgfqpoint{4.959689in}{0.413320in}}%
\pgfpathlineto{\pgfqpoint{4.956862in}{0.413320in}}%
\pgfpathlineto{\pgfqpoint{4.954182in}{0.413320in}}%
\pgfpathlineto{\pgfqpoint{4.951504in}{0.413320in}}%
\pgfpathlineto{\pgfqpoint{4.948827in}{0.413320in}}%
\pgfpathlineto{\pgfqpoint{4.946151in}{0.413320in}}%
\pgfpathlineto{\pgfqpoint{4.943466in}{0.413320in}}%
\pgfpathlineto{\pgfqpoint{4.940881in}{0.413320in}}%
\pgfpathlineto{\pgfqpoint{4.938112in}{0.413320in}}%
\pgfpathlineto{\pgfqpoint{4.935515in}{0.413320in}}%
\pgfpathlineto{\pgfqpoint{4.932742in}{0.413320in}}%
\pgfpathlineto{\pgfqpoint{4.930170in}{0.413320in}}%
\pgfpathlineto{\pgfqpoint{4.927400in}{0.413320in}}%
\pgfpathlineto{\pgfqpoint{4.924708in}{0.413320in}}%
\pgfpathlineto{\pgfqpoint{4.922041in}{0.413320in}}%
\pgfpathlineto{\pgfqpoint{4.919352in}{0.413320in}}%
\pgfpathlineto{\pgfqpoint{4.916681in}{0.413320in}}%
\pgfpathlineto{\pgfqpoint{4.914009in}{0.413320in}}%
\pgfpathlineto{\pgfqpoint{4.911435in}{0.413320in}}%
\pgfpathlineto{\pgfqpoint{4.908648in}{0.413320in}}%
\pgfpathlineto{\pgfqpoint{4.906096in}{0.413320in}}%
\pgfpathlineto{\pgfqpoint{4.903295in}{0.413320in}}%
\pgfpathlineto{\pgfqpoint{4.900712in}{0.413320in}}%
\pgfpathlineto{\pgfqpoint{4.897938in}{0.413320in}}%
\pgfpathlineto{\pgfqpoint{4.895399in}{0.413320in}}%
\pgfpathlineto{\pgfqpoint{4.892611in}{0.413320in}}%
\pgfpathlineto{\pgfqpoint{4.889902in}{0.413320in}}%
\pgfpathlineto{\pgfqpoint{4.887211in}{0.413320in}}%
\pgfpathlineto{\pgfqpoint{4.884540in}{0.413320in}}%
\pgfpathlineto{\pgfqpoint{4.881864in}{0.413320in}}%
\pgfpathlineto{\pgfqpoint{4.879180in}{0.413320in}}%
\pgfpathlineto{\pgfqpoint{4.876636in}{0.413320in}}%
\pgfpathlineto{\pgfqpoint{4.873832in}{0.413320in}}%
\pgfpathlineto{\pgfqpoint{4.871209in}{0.413320in}}%
\pgfpathlineto{\pgfqpoint{4.868474in}{0.413320in}}%
\pgfpathlineto{\pgfqpoint{4.865910in}{0.413320in}}%
\pgfpathlineto{\pgfqpoint{4.863116in}{0.413320in}}%
\pgfpathlineto{\pgfqpoint{4.860544in}{0.413320in}}%
\pgfpathlineto{\pgfqpoint{4.857807in}{0.413320in}}%
\pgfpathlineto{\pgfqpoint{4.855070in}{0.413320in}}%
\pgfpathlineto{\pgfqpoint{4.852404in}{0.413320in}}%
\pgfpathlineto{\pgfqpoint{4.849715in}{0.413320in}}%
\pgfpathlineto{\pgfqpoint{4.847127in}{0.413320in}}%
\pgfpathlineto{\pgfqpoint{4.844361in}{0.413320in}}%
\pgfpathlineto{\pgfqpoint{4.842380in}{0.413320in}}%
\pgfpathlineto{\pgfqpoint{4.839922in}{0.413320in}}%
\pgfpathlineto{\pgfqpoint{4.837992in}{0.413320in}}%
\pgfpathlineto{\pgfqpoint{4.833657in}{0.413320in}}%
\pgfpathlineto{\pgfqpoint{4.831045in}{0.413320in}}%
\pgfpathlineto{\pgfqpoint{4.828291in}{0.413320in}}%
\pgfpathlineto{\pgfqpoint{4.825619in}{0.413320in}}%
\pgfpathlineto{\pgfqpoint{4.822945in}{0.413320in}}%
\pgfpathlineto{\pgfqpoint{4.820265in}{0.413320in}}%
\pgfpathlineto{\pgfqpoint{4.817587in}{0.413320in}}%
\pgfpathlineto{\pgfqpoint{4.814907in}{0.413320in}}%
\pgfpathlineto{\pgfqpoint{4.812377in}{0.413320in}}%
\pgfpathlineto{\pgfqpoint{4.809538in}{0.413320in}}%
\pgfpathlineto{\pgfqpoint{4.807017in}{0.413320in}}%
\pgfpathlineto{\pgfqpoint{4.804193in}{0.413320in}}%
\pgfpathlineto{\pgfqpoint{4.801586in}{0.413320in}}%
\pgfpathlineto{\pgfqpoint{4.798830in}{0.413320in}}%
\pgfpathlineto{\pgfqpoint{4.796274in}{0.413320in}}%
\pgfpathlineto{\pgfqpoint{4.793512in}{0.413320in}}%
\pgfpathlineto{\pgfqpoint{4.790798in}{0.413320in}}%
\pgfpathlineto{\pgfqpoint{4.788116in}{0.413320in}}%
\pgfpathlineto{\pgfqpoint{4.785445in}{0.413320in}}%
\pgfpathlineto{\pgfqpoint{4.782872in}{0.413320in}}%
\pgfpathlineto{\pgfqpoint{4.780083in}{0.413320in}}%
\pgfpathlineto{\pgfqpoint{4.777535in}{0.413320in}}%
\pgfpathlineto{\pgfqpoint{4.774732in}{0.413320in}}%
\pgfpathlineto{\pgfqpoint{4.772198in}{0.413320in}}%
\pgfpathlineto{\pgfqpoint{4.769367in}{0.413320in}}%
\pgfpathlineto{\pgfqpoint{4.766783in}{0.413320in}}%
\pgfpathlineto{\pgfqpoint{4.764018in}{0.413320in}}%
\pgfpathlineto{\pgfqpoint{4.761337in}{0.413320in}}%
\pgfpathlineto{\pgfqpoint{4.758653in}{0.413320in}}%
\pgfpathlineto{\pgfqpoint{4.755983in}{0.413320in}}%
\pgfpathlineto{\pgfqpoint{4.753298in}{0.413320in}}%
\pgfpathlineto{\pgfqpoint{4.750627in}{0.413320in}}%
\pgfpathlineto{\pgfqpoint{4.748081in}{0.413320in}}%
\pgfpathlineto{\pgfqpoint{4.745256in}{0.413320in}}%
\pgfpathlineto{\pgfqpoint{4.742696in}{0.413320in}}%
\pgfpathlineto{\pgfqpoint{4.739912in}{0.413320in}}%
\pgfpathlineto{\pgfqpoint{4.737348in}{0.413320in}}%
\pgfpathlineto{\pgfqpoint{4.734552in}{0.413320in}}%
\pgfpathlineto{\pgfqpoint{4.731901in}{0.413320in}}%
\pgfpathlineto{\pgfqpoint{4.729233in}{0.413320in}}%
\pgfpathlineto{\pgfqpoint{4.726508in}{0.413320in}}%
\pgfpathlineto{\pgfqpoint{4.723873in}{0.413320in}}%
\pgfpathlineto{\pgfqpoint{4.721160in}{0.413320in}}%
\pgfpathlineto{\pgfqpoint{4.718486in}{0.413320in}}%
\pgfpathlineto{\pgfqpoint{4.715806in}{0.413320in}}%
\pgfpathlineto{\pgfqpoint{4.713275in}{0.413320in}}%
\pgfpathlineto{\pgfqpoint{4.710437in}{0.413320in}}%
\pgfpathlineto{\pgfqpoint{4.707824in}{0.413320in}}%
\pgfpathlineto{\pgfqpoint{4.705094in}{0.413320in}}%
\pgfpathlineto{\pgfqpoint{4.702517in}{0.413320in}}%
\pgfpathlineto{\pgfqpoint{4.699734in}{0.413320in}}%
\pgfpathlineto{\pgfqpoint{4.697170in}{0.413320in}}%
\pgfpathlineto{\pgfqpoint{4.694381in}{0.413320in}}%
\pgfpathlineto{\pgfqpoint{4.691694in}{0.413320in}}%
\pgfpathlineto{\pgfqpoint{4.689051in}{0.413320in}}%
\pgfpathlineto{\pgfqpoint{4.686337in}{0.413320in}}%
\pgfpathlineto{\pgfqpoint{4.683799in}{0.413320in}}%
\pgfpathlineto{\pgfqpoint{4.680988in}{0.413320in}}%
\pgfpathlineto{\pgfqpoint{4.678448in}{0.413320in}}%
\pgfpathlineto{\pgfqpoint{4.675619in}{0.413320in}}%
\pgfpathlineto{\pgfqpoint{4.673068in}{0.413320in}}%
\pgfpathlineto{\pgfqpoint{4.670261in}{0.413320in}}%
\pgfpathlineto{\pgfqpoint{4.667764in}{0.413320in}}%
\pgfpathlineto{\pgfqpoint{4.664923in}{0.413320in}}%
\pgfpathlineto{\pgfqpoint{4.662237in}{0.413320in}}%
\pgfpathlineto{\pgfqpoint{4.659590in}{0.413320in}}%
\pgfpathlineto{\pgfqpoint{4.656873in}{0.413320in}}%
\pgfpathlineto{\pgfqpoint{4.654203in}{0.413320in}}%
\pgfpathlineto{\pgfqpoint{4.651524in}{0.413320in}}%
\pgfpathlineto{\pgfqpoint{4.648922in}{0.413320in}}%
\pgfpathlineto{\pgfqpoint{4.646169in}{0.413320in}}%
\pgfpathlineto{\pgfqpoint{4.643628in}{0.413320in}}%
\pgfpathlineto{\pgfqpoint{4.640809in}{0.413320in}}%
\pgfpathlineto{\pgfqpoint{4.638204in}{0.413320in}}%
\pgfpathlineto{\pgfqpoint{4.635445in}{0.413320in}}%
\pgfpathlineto{\pgfqpoint{4.632902in}{0.413320in}}%
\pgfpathlineto{\pgfqpoint{4.630096in}{0.413320in}}%
\pgfpathlineto{\pgfqpoint{4.627411in}{0.413320in}}%
\pgfpathlineto{\pgfqpoint{4.624741in}{0.413320in}}%
\pgfpathlineto{\pgfqpoint{4.622056in}{0.413320in}}%
\pgfpathlineto{\pgfqpoint{4.619529in}{0.413320in}}%
\pgfpathlineto{\pgfqpoint{4.616702in}{0.413320in}}%
\pgfpathlineto{\pgfqpoint{4.614134in}{0.413320in}}%
\pgfpathlineto{\pgfqpoint{4.611350in}{0.413320in}}%
\pgfpathlineto{\pgfqpoint{4.608808in}{0.413320in}}%
\pgfpathlineto{\pgfqpoint{4.605990in}{0.413320in}}%
\pgfpathlineto{\pgfqpoint{4.603430in}{0.413320in}}%
\pgfpathlineto{\pgfqpoint{4.600633in}{0.413320in}}%
\pgfpathlineto{\pgfqpoint{4.597951in}{0.413320in}}%
\pgfpathlineto{\pgfqpoint{4.595281in}{0.413320in}}%
\pgfpathlineto{\pgfqpoint{4.592589in}{0.413320in}}%
\pgfpathlineto{\pgfqpoint{4.589920in}{0.413320in}}%
\pgfpathlineto{\pgfqpoint{4.587244in}{0.413320in}}%
\pgfpathlineto{\pgfqpoint{4.584672in}{0.413320in}}%
\pgfpathlineto{\pgfqpoint{4.581888in}{0.413320in}}%
\pgfpathlineto{\pgfqpoint{4.579305in}{0.413320in}}%
\pgfpathlineto{\pgfqpoint{4.576531in}{0.413320in}}%
\pgfpathlineto{\pgfqpoint{4.573947in}{0.413320in}}%
\pgfpathlineto{\pgfqpoint{4.571171in}{0.413320in}}%
\pgfpathlineto{\pgfqpoint{4.568612in}{0.413320in}}%
\pgfpathlineto{\pgfqpoint{4.565820in}{0.413320in}}%
\pgfpathlineto{\pgfqpoint{4.563125in}{0.413320in}}%
\pgfpathlineto{\pgfqpoint{4.560448in}{0.413320in}}%
\pgfpathlineto{\pgfqpoint{4.557777in}{0.413320in}}%
\pgfpathlineto{\pgfqpoint{4.555106in}{0.413320in}}%
\pgfpathlineto{\pgfqpoint{4.552425in}{0.413320in}}%
\pgfpathlineto{\pgfqpoint{4.549822in}{0.413320in}}%
\pgfpathlineto{\pgfqpoint{4.547064in}{0.413320in}}%
\pgfpathlineto{\pgfqpoint{4.544464in}{0.413320in}}%
\pgfpathlineto{\pgfqpoint{4.541711in}{0.413320in}}%
\pgfpathlineto{\pgfqpoint{4.539144in}{0.413320in}}%
\pgfpathlineto{\pgfqpoint{4.536400in}{0.413320in}}%
\pgfpathlineto{\pgfqpoint{4.533764in}{0.413320in}}%
\pgfpathlineto{\pgfqpoint{4.530990in}{0.413320in}}%
\pgfpathlineto{\pgfqpoint{4.528307in}{0.413320in}}%
\pgfpathlineto{\pgfqpoint{4.525640in}{0.413320in}}%
\pgfpathlineto{\pgfqpoint{4.522962in}{0.413320in}}%
\pgfpathlineto{\pgfqpoint{4.520345in}{0.413320in}}%
\pgfpathlineto{\pgfqpoint{4.517598in}{0.413320in}}%
\pgfpathlineto{\pgfqpoint{4.515080in}{0.413320in}}%
\pgfpathlineto{\pgfqpoint{4.512246in}{0.413320in}}%
\pgfpathlineto{\pgfqpoint{4.509643in}{0.413320in}}%
\pgfpathlineto{\pgfqpoint{4.506893in}{0.413320in}}%
\pgfpathlineto{\pgfqpoint{4.504305in}{0.413320in}}%
\pgfpathlineto{\pgfqpoint{4.501529in}{0.413320in}}%
\pgfpathlineto{\pgfqpoint{4.498850in}{0.413320in}}%
\pgfpathlineto{\pgfqpoint{4.496167in}{0.413320in}}%
\pgfpathlineto{\pgfqpoint{4.493492in}{0.413320in}}%
\pgfpathlineto{\pgfqpoint{4.490822in}{0.413320in}}%
\pgfpathlineto{\pgfqpoint{4.488130in}{0.413320in}}%
\pgfpathlineto{\pgfqpoint{4.485581in}{0.413320in}}%
\pgfpathlineto{\pgfqpoint{4.482778in}{0.413320in}}%
\pgfpathlineto{\pgfqpoint{4.480201in}{0.413320in}}%
\pgfpathlineto{\pgfqpoint{4.477430in}{0.413320in}}%
\pgfpathlineto{\pgfqpoint{4.474861in}{0.413320in}}%
\pgfpathlineto{\pgfqpoint{4.472059in}{0.413320in}}%
\pgfpathlineto{\pgfqpoint{4.469492in}{0.413320in}}%
\pgfpathlineto{\pgfqpoint{4.466717in}{0.413320in}}%
\pgfpathlineto{\pgfqpoint{4.464029in}{0.413320in}}%
\pgfpathlineto{\pgfqpoint{4.461367in}{0.413320in}}%
\pgfpathlineto{\pgfqpoint{4.458681in}{0.413320in}}%
\pgfpathlineto{\pgfqpoint{4.456138in}{0.413320in}}%
\pgfpathlineto{\pgfqpoint{4.453312in}{0.413320in}}%
\pgfpathlineto{\pgfqpoint{4.450767in}{0.413320in}}%
\pgfpathlineto{\pgfqpoint{4.447965in}{0.413320in}}%
\pgfpathlineto{\pgfqpoint{4.445423in}{0.413320in}}%
\pgfpathlineto{\pgfqpoint{4.442611in}{0.413320in}}%
\pgfpathlineto{\pgfqpoint{4.440041in}{0.413320in}}%
\pgfpathlineto{\pgfqpoint{4.437253in}{0.413320in}}%
\pgfpathlineto{\pgfqpoint{4.434569in}{0.413320in}}%
\pgfpathlineto{\pgfqpoint{4.431901in}{0.413320in}}%
\pgfpathlineto{\pgfqpoint{4.429220in}{0.413320in}}%
\pgfpathlineto{\pgfqpoint{4.426534in}{0.413320in}}%
\pgfpathlineto{\pgfqpoint{4.423863in}{0.413320in}}%
\pgfpathlineto{\pgfqpoint{4.421292in}{0.413320in}}%
\pgfpathlineto{\pgfqpoint{4.418506in}{0.413320in}}%
\pgfpathlineto{\pgfqpoint{4.415932in}{0.413320in}}%
\pgfpathlineto{\pgfqpoint{4.413149in}{0.413320in}}%
\pgfpathlineto{\pgfqpoint{4.410587in}{0.413320in}}%
\pgfpathlineto{\pgfqpoint{4.407788in}{0.413320in}}%
\pgfpathlineto{\pgfqpoint{4.405234in}{0.413320in}}%
\pgfpathlineto{\pgfqpoint{4.402468in}{0.413320in}}%
\pgfpathlineto{\pgfqpoint{4.399745in}{0.413320in}}%
\pgfpathlineto{\pgfqpoint{4.397076in}{0.413320in}}%
\pgfpathlineto{\pgfqpoint{4.394400in}{0.413320in}}%
\pgfpathlineto{\pgfqpoint{4.391721in}{0.413320in}}%
\pgfpathlineto{\pgfqpoint{4.389044in}{0.413320in}}%
\pgfpathlineto{\pgfqpoint{4.386431in}{0.413320in}}%
\pgfpathlineto{\pgfqpoint{4.383674in}{0.413320in}}%
\pgfpathlineto{\pgfqpoint{4.381097in}{0.413320in}}%
\pgfpathlineto{\pgfqpoint{4.378329in}{0.413320in}}%
\pgfpathlineto{\pgfqpoint{4.375761in}{0.413320in}}%
\pgfpathlineto{\pgfqpoint{4.372976in}{0.413320in}}%
\pgfpathlineto{\pgfqpoint{4.370437in}{0.413320in}}%
\pgfpathlineto{\pgfqpoint{4.367646in}{0.413320in}}%
\pgfpathlineto{\pgfqpoint{4.364936in}{0.413320in}}%
\pgfpathlineto{\pgfqpoint{4.362270in}{0.413320in}}%
\pgfpathlineto{\pgfqpoint{4.359582in}{0.413320in}}%
\pgfpathlineto{\pgfqpoint{4.357014in}{0.413320in}}%
\pgfpathlineto{\pgfqpoint{4.354224in}{0.413320in}}%
\pgfpathlineto{\pgfqpoint{4.351645in}{0.413320in}}%
\pgfpathlineto{\pgfqpoint{4.348868in}{0.413320in}}%
\pgfpathlineto{\pgfqpoint{4.346263in}{0.413320in}}%
\pgfpathlineto{\pgfqpoint{4.343510in}{0.413320in}}%
\pgfpathlineto{\pgfqpoint{4.340976in}{0.413320in}}%
\pgfpathlineto{\pgfqpoint{4.338154in}{0.413320in}}%
\pgfpathlineto{\pgfqpoint{4.335463in}{0.413320in}}%
\pgfpathlineto{\pgfqpoint{4.332796in}{0.413320in}}%
\pgfpathlineto{\pgfqpoint{4.330118in}{0.413320in}}%
\pgfpathlineto{\pgfqpoint{4.327440in}{0.413320in}}%
\pgfpathlineto{\pgfqpoint{4.324760in}{0.413320in}}%
\pgfpathlineto{\pgfqpoint{4.322181in}{0.413320in}}%
\pgfpathlineto{\pgfqpoint{4.319405in}{0.413320in}}%
\pgfpathlineto{\pgfqpoint{4.316856in}{0.413320in}}%
\pgfpathlineto{\pgfqpoint{4.314032in}{0.413320in}}%
\pgfpathlineto{\pgfqpoint{4.311494in}{0.413320in}}%
\pgfpathlineto{\pgfqpoint{4.308691in}{0.413320in}}%
\pgfpathlineto{\pgfqpoint{4.306118in}{0.413320in}}%
\pgfpathlineto{\pgfqpoint{4.303357in}{0.413320in}}%
\pgfpathlineto{\pgfqpoint{4.300656in}{0.413320in}}%
\pgfpathlineto{\pgfqpoint{4.297977in}{0.413320in}}%
\pgfpathlineto{\pgfqpoint{4.295299in}{0.413320in}}%
\pgfpathlineto{\pgfqpoint{4.292786in}{0.413320in}}%
\pgfpathlineto{\pgfqpoint{4.289936in}{0.413320in}}%
\pgfpathlineto{\pgfqpoint{4.287399in}{0.413320in}}%
\pgfpathlineto{\pgfqpoint{4.284586in}{0.413320in}}%
\pgfpathlineto{\pgfqpoint{4.282000in}{0.413320in}}%
\pgfpathlineto{\pgfqpoint{4.279212in}{0.413320in}}%
\pgfpathlineto{\pgfqpoint{4.276635in}{0.413320in}}%
\pgfpathlineto{\pgfqpoint{4.273874in}{0.413320in}}%
\pgfpathlineto{\pgfqpoint{4.271187in}{0.413320in}}%
\pgfpathlineto{\pgfqpoint{4.268590in}{0.413320in}}%
\pgfpathlineto{\pgfqpoint{4.265824in}{0.413320in}}%
\pgfpathlineto{\pgfqpoint{4.263157in}{0.413320in}}%
\pgfpathlineto{\pgfqpoint{4.260477in}{0.413320in}}%
\pgfpathlineto{\pgfqpoint{4.257958in}{0.413320in}}%
\pgfpathlineto{\pgfqpoint{4.255120in}{0.413320in}}%
\pgfpathlineto{\pgfqpoint{4.252581in}{0.413320in}}%
\pgfpathlineto{\pgfqpoint{4.249767in}{0.413320in}}%
\pgfpathlineto{\pgfqpoint{4.247225in}{0.413320in}}%
\pgfpathlineto{\pgfqpoint{4.244394in}{0.413320in}}%
\pgfpathlineto{\pgfqpoint{4.241900in}{0.413320in}}%
\pgfpathlineto{\pgfqpoint{4.239084in}{0.413320in}}%
\pgfpathlineto{\pgfqpoint{4.236375in}{0.413320in}}%
\pgfpathlineto{\pgfqpoint{4.233691in}{0.413320in}}%
\pgfpathlineto{\pgfqpoint{4.231013in}{0.413320in}}%
\pgfpathlineto{\pgfqpoint{4.228331in}{0.413320in}}%
\pgfpathlineto{\pgfqpoint{4.225654in}{0.413320in}}%
\pgfpathlineto{\pgfqpoint{4.223082in}{0.413320in}}%
\pgfpathlineto{\pgfqpoint{4.220304in}{0.413320in}}%
\pgfpathlineto{\pgfqpoint{4.217694in}{0.413320in}}%
\pgfpathlineto{\pgfqpoint{4.214948in}{0.413320in}}%
\pgfpathlineto{\pgfqpoint{4.212383in}{0.413320in}}%
\pgfpathlineto{\pgfqpoint{4.209597in}{0.413320in}}%
\pgfpathlineto{\pgfqpoint{4.207076in}{0.413320in}}%
\pgfpathlineto{\pgfqpoint{4.204240in}{0.413320in}}%
\pgfpathlineto{\pgfqpoint{4.201542in}{0.413320in}}%
\pgfpathlineto{\pgfqpoint{4.198878in}{0.413320in}}%
\pgfpathlineto{\pgfqpoint{4.196186in}{0.413320in}}%
\pgfpathlineto{\pgfqpoint{4.193638in}{0.413320in}}%
\pgfpathlineto{\pgfqpoint{4.190842in}{0.413320in}}%
\pgfpathlineto{\pgfqpoint{4.188318in}{0.413320in}}%
\pgfpathlineto{\pgfqpoint{4.185481in}{0.413320in}}%
\pgfpathlineto{\pgfqpoint{4.182899in}{0.413320in}}%
\pgfpathlineto{\pgfqpoint{4.180129in}{0.413320in}}%
\pgfpathlineto{\pgfqpoint{4.177593in}{0.413320in}}%
\pgfpathlineto{\pgfqpoint{4.174770in}{0.413320in}}%
\pgfpathlineto{\pgfqpoint{4.172093in}{0.413320in}}%
\pgfpathlineto{\pgfqpoint{4.169415in}{0.413320in}}%
\pgfpathlineto{\pgfqpoint{4.166737in}{0.413320in}}%
\pgfpathlineto{\pgfqpoint{4.164059in}{0.413320in}}%
\pgfpathlineto{\pgfqpoint{4.161380in}{0.413320in}}%
\pgfpathlineto{\pgfqpoint{4.158806in}{0.413320in}}%
\pgfpathlineto{\pgfqpoint{4.156016in}{0.413320in}}%
\pgfpathlineto{\pgfqpoint{4.153423in}{0.413320in}}%
\pgfpathlineto{\pgfqpoint{4.150665in}{0.413320in}}%
\pgfpathlineto{\pgfqpoint{4.148082in}{0.413320in}}%
\pgfpathlineto{\pgfqpoint{4.145310in}{0.413320in}}%
\pgfpathlineto{\pgfqpoint{4.142713in}{0.413320in}}%
\pgfpathlineto{\pgfqpoint{4.139963in}{0.413320in}}%
\pgfpathlineto{\pgfqpoint{4.137272in}{0.413320in}}%
\pgfpathlineto{\pgfqpoint{4.134615in}{0.413320in}}%
\pgfpathlineto{\pgfqpoint{4.131920in}{0.413320in}}%
\pgfpathlineto{\pgfqpoint{4.129349in}{0.413320in}}%
\pgfpathlineto{\pgfqpoint{4.126553in}{0.413320in}}%
\pgfpathlineto{\pgfqpoint{4.124019in}{0.413320in}}%
\pgfpathlineto{\pgfqpoint{4.121205in}{0.413320in}}%
\pgfpathlineto{\pgfqpoint{4.118554in}{0.413320in}}%
\pgfpathlineto{\pgfqpoint{4.115844in}{0.413320in}}%
\pgfpathlineto{\pgfqpoint{4.113252in}{0.413320in}}%
\pgfpathlineto{\pgfqpoint{4.110488in}{0.413320in}}%
\pgfpathlineto{\pgfqpoint{4.107814in}{0.413320in}}%
\pgfpathlineto{\pgfqpoint{4.105185in}{0.413320in}}%
\pgfpathlineto{\pgfqpoint{4.102456in}{0.413320in}}%
\pgfpathlineto{\pgfqpoint{4.099777in}{0.413320in}}%
\pgfpathlineto{\pgfqpoint{4.097092in}{0.413320in}}%
\pgfpathlineto{\pgfqpoint{4.094527in}{0.413320in}}%
\pgfpathlineto{\pgfqpoint{4.091729in}{0.413320in}}%
\pgfpathlineto{\pgfqpoint{4.089159in}{0.413320in}}%
\pgfpathlineto{\pgfqpoint{4.086385in}{0.413320in}}%
\pgfpathlineto{\pgfqpoint{4.083870in}{0.413320in}}%
\pgfpathlineto{\pgfqpoint{4.081018in}{0.413320in}}%
\pgfpathlineto{\pgfqpoint{4.078471in}{0.413320in}}%
\pgfpathlineto{\pgfqpoint{4.075705in}{0.413320in}}%
\pgfpathlineto{\pgfqpoint{4.072985in}{0.413320in}}%
\pgfpathlineto{\pgfqpoint{4.070313in}{0.413320in}}%
\pgfpathlineto{\pgfqpoint{4.067636in}{0.413320in}}%
\pgfpathlineto{\pgfqpoint{4.064957in}{0.413320in}}%
\pgfpathlineto{\pgfqpoint{4.062266in}{0.413320in}}%
\pgfpathlineto{\pgfqpoint{4.059702in}{0.413320in}}%
\pgfpathlineto{\pgfqpoint{4.056911in}{0.413320in}}%
\pgfpathlineto{\pgfqpoint{4.054326in}{0.413320in}}%
\pgfpathlineto{\pgfqpoint{4.051557in}{0.413320in}}%
\pgfpathlineto{\pgfqpoint{4.049006in}{0.413320in}}%
\pgfpathlineto{\pgfqpoint{4.046210in}{0.413320in}}%
\pgfpathlineto{\pgfqpoint{4.043667in}{0.413320in}}%
\pgfpathlineto{\pgfqpoint{4.040852in}{0.413320in}}%
\pgfpathlineto{\pgfqpoint{4.038174in}{0.413320in}}%
\pgfpathlineto{\pgfqpoint{4.035492in}{0.413320in}}%
\pgfpathlineto{\pgfqpoint{4.032817in}{0.413320in}}%
\pgfpathlineto{\pgfqpoint{4.030229in}{0.413320in}}%
\pgfpathlineto{\pgfqpoint{4.027447in}{0.413320in}}%
\pgfpathlineto{\pgfqpoint{4.024868in}{0.413320in}}%
\pgfpathlineto{\pgfqpoint{4.022097in}{0.413320in}}%
\pgfpathlineto{\pgfqpoint{4.019518in}{0.413320in}}%
\pgfpathlineto{\pgfqpoint{4.016744in}{0.413320in}}%
\pgfpathlineto{\pgfqpoint{4.014186in}{0.413320in}}%
\pgfpathlineto{\pgfqpoint{4.011394in}{0.413320in}}%
\pgfpathlineto{\pgfqpoint{4.008699in}{0.413320in}}%
\pgfpathlineto{\pgfqpoint{4.006034in}{0.413320in}}%
\pgfpathlineto{\pgfqpoint{4.003348in}{0.413320in}}%
\pgfpathlineto{\pgfqpoint{4.000674in}{0.413320in}}%
\pgfpathlineto{\pgfqpoint{3.997990in}{0.413320in}}%
\pgfpathlineto{\pgfqpoint{3.995417in}{0.413320in}}%
\pgfpathlineto{\pgfqpoint{3.992642in}{0.413320in}}%
\pgfpathlineto{\pgfqpoint{3.990055in}{0.413320in}}%
\pgfpathlineto{\pgfqpoint{3.987270in}{0.413320in}}%
\pgfpathlineto{\pgfqpoint{3.984714in}{0.413320in}}%
\pgfpathlineto{\pgfqpoint{3.981929in}{0.413320in}}%
\pgfpathlineto{\pgfqpoint{3.979389in}{0.413320in}}%
\pgfpathlineto{\pgfqpoint{3.976563in}{0.413320in}}%
\pgfpathlineto{\pgfqpoint{3.973885in}{0.413320in}}%
\pgfpathlineto{\pgfqpoint{3.971250in}{0.413320in}}%
\pgfpathlineto{\pgfqpoint{3.968523in}{0.413320in}}%
\pgfpathlineto{\pgfqpoint{3.966013in}{0.413320in}}%
\pgfpathlineto{\pgfqpoint{3.963176in}{0.413320in}}%
\pgfpathlineto{\pgfqpoint{3.960635in}{0.413320in}}%
\pgfpathlineto{\pgfqpoint{3.957823in}{0.413320in}}%
\pgfpathlineto{\pgfqpoint{3.955211in}{0.413320in}}%
\pgfpathlineto{\pgfqpoint{3.952464in}{0.413320in}}%
\pgfpathlineto{\pgfqpoint{3.949894in}{0.413320in}}%
\pgfpathlineto{\pgfqpoint{3.947101in}{0.413320in}}%
\pgfpathlineto{\pgfqpoint{3.944431in}{0.413320in}}%
\pgfpathlineto{\pgfqpoint{3.941778in}{0.413320in}}%
\pgfpathlineto{\pgfqpoint{3.939075in}{0.413320in}}%
\pgfpathlineto{\pgfqpoint{3.936395in}{0.413320in}}%
\pgfpathlineto{\pgfqpoint{3.933714in}{0.413320in}}%
\pgfpathlineto{\pgfqpoint{3.931202in}{0.413320in}}%
\pgfpathlineto{\pgfqpoint{3.928347in}{0.413320in}}%
\pgfpathlineto{\pgfqpoint{3.925778in}{0.413320in}}%
\pgfpathlineto{\pgfqpoint{3.923005in}{0.413320in}}%
\pgfpathlineto{\pgfqpoint{3.920412in}{0.413320in}}%
\pgfpathlineto{\pgfqpoint{3.917646in}{0.413320in}}%
\pgfpathlineto{\pgfqpoint{3.915107in}{0.413320in}}%
\pgfpathlineto{\pgfqpoint{3.912296in}{0.413320in}}%
\pgfpathlineto{\pgfqpoint{3.909602in}{0.413320in}}%
\pgfpathlineto{\pgfqpoint{3.906918in}{0.413320in}}%
\pgfpathlineto{\pgfqpoint{3.904252in}{0.413320in}}%
\pgfpathlineto{\pgfqpoint{3.901573in}{0.413320in}}%
\pgfpathlineto{\pgfqpoint{3.898891in}{0.413320in}}%
\pgfpathlineto{\pgfqpoint{3.896345in}{0.413320in}}%
\pgfpathlineto{\pgfqpoint{3.893541in}{0.413320in}}%
\pgfpathlineto{\pgfqpoint{3.890926in}{0.413320in}}%
\pgfpathlineto{\pgfqpoint{3.888188in}{0.413320in}}%
\pgfpathlineto{\pgfqpoint{3.885621in}{0.413320in}}%
\pgfpathlineto{\pgfqpoint{3.882850in}{0.413320in}}%
\pgfpathlineto{\pgfqpoint{3.880237in}{0.413320in}}%
\pgfpathlineto{\pgfqpoint{3.877466in}{0.413320in}}%
\pgfpathlineto{\pgfqpoint{3.874790in}{0.413320in}}%
\pgfpathlineto{\pgfqpoint{3.872114in}{0.413320in}}%
\pgfpathlineto{\pgfqpoint{3.869435in}{0.413320in}}%
\pgfpathlineto{\pgfqpoint{3.866815in}{0.413320in}}%
\pgfpathlineto{\pgfqpoint{3.864073in}{0.413320in}}%
\pgfpathlineto{\pgfqpoint{3.861561in}{0.413320in}}%
\pgfpathlineto{\pgfqpoint{3.858720in}{0.413320in}}%
\pgfpathlineto{\pgfqpoint{3.856100in}{0.413320in}}%
\pgfpathlineto{\pgfqpoint{3.853358in}{0.413320in}}%
\pgfpathlineto{\pgfqpoint{3.850814in}{0.413320in}}%
\pgfpathlineto{\pgfqpoint{3.848005in}{0.413320in}}%
\pgfpathlineto{\pgfqpoint{3.845329in}{0.413320in}}%
\pgfpathlineto{\pgfqpoint{3.842641in}{0.413320in}}%
\pgfpathlineto{\pgfqpoint{3.839960in}{0.413320in}}%
\pgfpathlineto{\pgfqpoint{3.837286in}{0.413320in}}%
\pgfpathlineto{\pgfqpoint{3.834616in}{0.413320in}}%
\pgfpathlineto{\pgfqpoint{3.832053in}{0.413320in}}%
\pgfpathlineto{\pgfqpoint{3.829252in}{0.413320in}}%
\pgfpathlineto{\pgfqpoint{3.826679in}{0.413320in}}%
\pgfpathlineto{\pgfqpoint{3.823903in}{0.413320in}}%
\pgfpathlineto{\pgfqpoint{3.821315in}{0.413320in}}%
\pgfpathlineto{\pgfqpoint{3.818546in}{0.413320in}}%
\pgfpathlineto{\pgfqpoint{3.815983in}{0.413320in}}%
\pgfpathlineto{\pgfqpoint{3.813172in}{0.413320in}}%
\pgfpathlineto{\pgfqpoint{3.810510in}{0.413320in}}%
\pgfpathlineto{\pgfqpoint{3.807832in}{0.413320in}}%
\pgfpathlineto{\pgfqpoint{3.805145in}{0.413320in}}%
\pgfpathlineto{\pgfqpoint{3.802569in}{0.413320in}}%
\pgfpathlineto{\pgfqpoint{3.799797in}{0.413320in}}%
\pgfpathlineto{\pgfqpoint{3.797265in}{0.413320in}}%
\pgfpathlineto{\pgfqpoint{3.794435in}{0.413320in}}%
\pgfpathlineto{\pgfqpoint{3.791897in}{0.413320in}}%
\pgfpathlineto{\pgfqpoint{3.789084in}{0.413320in}}%
\pgfpathlineto{\pgfqpoint{3.786504in}{0.413320in}}%
\pgfpathlineto{\pgfqpoint{3.783725in}{0.413320in}}%
\pgfpathlineto{\pgfqpoint{3.781046in}{0.413320in}}%
\pgfpathlineto{\pgfqpoint{3.778370in}{0.413320in}}%
\pgfpathlineto{\pgfqpoint{3.775691in}{0.413320in}}%
\pgfpathlineto{\pgfqpoint{3.773014in}{0.413320in}}%
\pgfpathlineto{\pgfqpoint{3.770323in}{0.413320in}}%
\pgfpathlineto{\pgfqpoint{3.767782in}{0.413320in}}%
\pgfpathlineto{\pgfqpoint{3.764966in}{0.413320in}}%
\pgfpathlineto{\pgfqpoint{3.762389in}{0.413320in}}%
\pgfpathlineto{\pgfqpoint{3.759622in}{0.413320in}}%
\pgfpathlineto{\pgfqpoint{3.757065in}{0.413320in}}%
\pgfpathlineto{\pgfqpoint{3.754265in}{0.413320in}}%
\pgfpathlineto{\pgfqpoint{3.751728in}{0.413320in}}%
\pgfpathlineto{\pgfqpoint{3.748903in}{0.413320in}}%
\pgfpathlineto{\pgfqpoint{3.746229in}{0.413320in}}%
\pgfpathlineto{\pgfqpoint{3.743548in}{0.413320in}}%
\pgfpathlineto{\pgfqpoint{3.740874in}{0.413320in}}%
\pgfpathlineto{\pgfqpoint{3.738194in}{0.413320in}}%
\pgfpathlineto{\pgfqpoint{3.735509in}{0.413320in}}%
\pgfpathlineto{\pgfqpoint{3.732950in}{0.413320in}}%
\pgfpathlineto{\pgfqpoint{3.730158in}{0.413320in}}%
\pgfpathlineto{\pgfqpoint{3.727581in}{0.413320in}}%
\pgfpathlineto{\pgfqpoint{3.724804in}{0.413320in}}%
\pgfpathlineto{\pgfqpoint{3.722228in}{0.413320in}}%
\pgfpathlineto{\pgfqpoint{3.719446in}{0.413320in}}%
\pgfpathlineto{\pgfqpoint{3.716875in}{0.413320in}}%
\pgfpathlineto{\pgfqpoint{3.714086in}{0.413320in}}%
\pgfpathlineto{\pgfqpoint{3.711410in}{0.413320in}}%
\pgfpathlineto{\pgfqpoint{3.708729in}{0.413320in}}%
\pgfpathlineto{\pgfqpoint{3.706053in}{0.413320in}}%
\pgfpathlineto{\pgfqpoint{3.703460in}{0.413320in}}%
\pgfpathlineto{\pgfqpoint{3.700684in}{0.413320in}}%
\pgfpathlineto{\pgfqpoint{3.698125in}{0.413320in}}%
\pgfpathlineto{\pgfqpoint{3.695331in}{0.413320in}}%
\pgfpathlineto{\pgfqpoint{3.692765in}{0.413320in}}%
\pgfpathlineto{\pgfqpoint{3.689983in}{0.413320in}}%
\pgfpathlineto{\pgfqpoint{3.687442in}{0.413320in}}%
\pgfpathlineto{\pgfqpoint{3.684620in}{0.413320in}}%
\pgfpathlineto{\pgfqpoint{3.681948in}{0.413320in}}%
\pgfpathlineto{\pgfqpoint{3.679273in}{0.413320in}}%
\pgfpathlineto{\pgfqpoint{3.676591in}{0.413320in}}%
\pgfpathlineto{\pgfqpoint{3.673911in}{0.413320in}}%
\pgfpathlineto{\pgfqpoint{3.671232in}{0.413320in}}%
\pgfpathlineto{\pgfqpoint{3.668665in}{0.413320in}}%
\pgfpathlineto{\pgfqpoint{3.665864in}{0.413320in}}%
\pgfpathlineto{\pgfqpoint{3.663276in}{0.413320in}}%
\pgfpathlineto{\pgfqpoint{3.660515in}{0.413320in}}%
\pgfpathlineto{\pgfqpoint{3.657917in}{0.413320in}}%
\pgfpathlineto{\pgfqpoint{3.655165in}{0.413320in}}%
\pgfpathlineto{\pgfqpoint{3.652628in}{0.413320in}}%
\pgfpathlineto{\pgfqpoint{3.649837in}{0.413320in}}%
\pgfpathlineto{\pgfqpoint{3.647130in}{0.413320in}}%
\pgfpathlineto{\pgfqpoint{3.644452in}{0.413320in}}%
\pgfpathlineto{\pgfqpoint{3.641773in}{0.413320in}}%
\pgfpathlineto{\pgfqpoint{3.639207in}{0.413320in}}%
\pgfpathlineto{\pgfqpoint{3.636413in}{0.413320in}}%
\pgfpathlineto{\pgfqpoint{3.633858in}{0.413320in}}%
\pgfpathlineto{\pgfqpoint{3.631058in}{0.413320in}}%
\pgfpathlineto{\pgfqpoint{3.628460in}{0.413320in}}%
\pgfpathlineto{\pgfqpoint{3.625689in}{0.413320in}}%
\pgfpathlineto{\pgfqpoint{3.623165in}{0.413320in}}%
\pgfpathlineto{\pgfqpoint{3.620345in}{0.413320in}}%
\pgfpathlineto{\pgfqpoint{3.617667in}{0.413320in}}%
\pgfpathlineto{\pgfqpoint{3.614982in}{0.413320in}}%
\pgfpathlineto{\pgfqpoint{3.612311in}{0.413320in}}%
\pgfpathlineto{\pgfqpoint{3.609632in}{0.413320in}}%
\pgfpathlineto{\pgfqpoint{3.606951in}{0.413320in}}%
\pgfpathlineto{\pgfqpoint{3.604387in}{0.413320in}}%
\pgfpathlineto{\pgfqpoint{3.601590in}{0.413320in}}%
\pgfpathlineto{\pgfqpoint{3.598998in}{0.413320in}}%
\pgfpathlineto{\pgfqpoint{3.596240in}{0.413320in}}%
\pgfpathlineto{\pgfqpoint{3.593620in}{0.413320in}}%
\pgfpathlineto{\pgfqpoint{3.590883in}{0.413320in}}%
\pgfpathlineto{\pgfqpoint{3.588258in}{0.413320in}}%
\pgfpathlineto{\pgfqpoint{3.585532in}{0.413320in}}%
\pgfpathlineto{\pgfqpoint{3.582851in}{0.413320in}}%
\pgfpathlineto{\pgfqpoint{3.580191in}{0.413320in}}%
\pgfpathlineto{\pgfqpoint{3.577487in}{0.413320in}}%
\pgfpathlineto{\pgfqpoint{3.574814in}{0.413320in}}%
\pgfpathlineto{\pgfqpoint{3.572126in}{0.413320in}}%
\pgfpathlineto{\pgfqpoint{3.569584in}{0.413320in}}%
\pgfpathlineto{\pgfqpoint{3.566774in}{0.413320in}}%
\pgfpathlineto{\pgfqpoint{3.564188in}{0.413320in}}%
\pgfpathlineto{\pgfqpoint{3.561420in}{0.413320in}}%
\pgfpathlineto{\pgfqpoint{3.558853in}{0.413320in}}%
\pgfpathlineto{\pgfqpoint{3.556061in}{0.413320in}}%
\pgfpathlineto{\pgfqpoint{3.553498in}{0.413320in}}%
\pgfpathlineto{\pgfqpoint{3.550713in}{0.413320in}}%
\pgfpathlineto{\pgfqpoint{3.548029in}{0.413320in}}%
\pgfpathlineto{\pgfqpoint{3.545349in}{0.413320in}}%
\pgfpathlineto{\pgfqpoint{3.542656in}{0.413320in}}%
\pgfpathlineto{\pgfqpoint{3.540093in}{0.413320in}}%
\pgfpathlineto{\pgfqpoint{3.537309in}{0.413320in}}%
\pgfpathlineto{\pgfqpoint{3.534783in}{0.413320in}}%
\pgfpathlineto{\pgfqpoint{3.531955in}{0.413320in}}%
\pgfpathlineto{\pgfqpoint{3.529327in}{0.413320in}}%
\pgfpathlineto{\pgfqpoint{3.526601in}{0.413320in}}%
\pgfpathlineto{\pgfqpoint{3.524041in}{0.413320in}}%
\pgfpathlineto{\pgfqpoint{3.521244in}{0.413320in}}%
\pgfpathlineto{\pgfqpoint{3.518565in}{0.413320in}}%
\pgfpathlineto{\pgfqpoint{3.515884in}{0.413320in}}%
\pgfpathlineto{\pgfqpoint{3.513209in}{0.413320in}}%
\pgfpathlineto{\pgfqpoint{3.510533in}{0.413320in}}%
\pgfpathlineto{\pgfqpoint{3.507840in}{0.413320in}}%
\pgfpathlineto{\pgfqpoint{3.505262in}{0.413320in}}%
\pgfpathlineto{\pgfqpoint{3.502488in}{0.413320in}}%
\pgfpathlineto{\pgfqpoint{3.499909in}{0.413320in}}%
\pgfpathlineto{\pgfqpoint{3.497139in}{0.413320in}}%
\pgfpathlineto{\pgfqpoint{3.494581in}{0.413320in}}%
\pgfpathlineto{\pgfqpoint{3.491783in}{0.413320in}}%
\pgfpathlineto{\pgfqpoint{3.489223in}{0.413320in}}%
\pgfpathlineto{\pgfqpoint{3.486442in}{0.413320in}}%
\pgfpathlineto{\pgfqpoint{3.483744in}{0.413320in}}%
\pgfpathlineto{\pgfqpoint{3.481072in}{0.413320in}}%
\pgfpathlineto{\pgfqpoint{3.478378in}{0.413320in}}%
\pgfpathlineto{\pgfqpoint{3.475821in}{0.413320in}}%
\pgfpathlineto{\pgfqpoint{3.473021in}{0.413320in}}%
\pgfpathlineto{\pgfqpoint{3.470466in}{0.413320in}}%
\pgfpathlineto{\pgfqpoint{3.467678in}{0.413320in}}%
\pgfpathlineto{\pgfqpoint{3.465072in}{0.413320in}}%
\pgfpathlineto{\pgfqpoint{3.462321in}{0.413320in}}%
\pgfpathlineto{\pgfqpoint{3.459695in}{0.413320in}}%
\pgfpathlineto{\pgfqpoint{3.456960in}{0.413320in}}%
\pgfpathlineto{\pgfqpoint{3.454285in}{0.413320in}}%
\pgfpathlineto{\pgfqpoint{3.451597in}{0.413320in}}%
\pgfpathlineto{\pgfqpoint{3.448926in}{0.413320in}}%
\pgfpathlineto{\pgfqpoint{3.446257in}{0.413320in}}%
\pgfpathlineto{\pgfqpoint{3.443574in}{0.413320in}}%
\pgfpathlineto{\pgfqpoint{3.440996in}{0.413320in}}%
\pgfpathlineto{\pgfqpoint{3.438210in}{0.413320in}}%
\pgfpathlineto{\pgfqpoint{3.435635in}{0.413320in}}%
\pgfpathlineto{\pgfqpoint{3.432851in}{0.413320in}}%
\pgfpathlineto{\pgfqpoint{3.430313in}{0.413320in}}%
\pgfpathlineto{\pgfqpoint{3.427501in}{0.413320in}}%
\pgfpathlineto{\pgfqpoint{3.424887in}{0.413320in}}%
\pgfpathlineto{\pgfqpoint{3.422142in}{0.413320in}}%
\pgfpathlineto{\pgfqpoint{3.419455in}{0.413320in}}%
\pgfpathlineto{\pgfqpoint{3.416780in}{0.413320in}}%
\pgfpathlineto{\pgfqpoint{3.414109in}{0.413320in}}%
\pgfpathlineto{\pgfqpoint{3.411431in}{0.413320in}}%
\pgfpathlineto{\pgfqpoint{3.408752in}{0.413320in}}%
\pgfpathlineto{\pgfqpoint{3.406202in}{0.413320in}}%
\pgfpathlineto{\pgfqpoint{3.403394in}{0.413320in}}%
\pgfpathlineto{\pgfqpoint{3.400783in}{0.413320in}}%
\pgfpathlineto{\pgfqpoint{3.398037in}{0.413320in}}%
\pgfpathlineto{\pgfqpoint{3.395461in}{0.413320in}}%
\pgfpathlineto{\pgfqpoint{3.392681in}{0.413320in}}%
\pgfpathlineto{\pgfqpoint{3.390102in}{0.413320in}}%
\pgfpathlineto{\pgfqpoint{3.387309in}{0.413320in}}%
\pgfpathlineto{\pgfqpoint{3.384647in}{0.413320in}}%
\pgfpathlineto{\pgfqpoint{3.381959in}{0.413320in}}%
\pgfpathlineto{\pgfqpoint{3.379290in}{0.413320in}}%
\pgfpathlineto{\pgfqpoint{3.376735in}{0.413320in}}%
\pgfpathlineto{\pgfqpoint{3.373921in}{0.413320in}}%
\pgfpathlineto{\pgfqpoint{3.371357in}{0.413320in}}%
\pgfpathlineto{\pgfqpoint{3.368577in}{0.413320in}}%
\pgfpathlineto{\pgfqpoint{3.365996in}{0.413320in}}%
\pgfpathlineto{\pgfqpoint{3.363221in}{0.413320in}}%
\pgfpathlineto{\pgfqpoint{3.360620in}{0.413320in}}%
\pgfpathlineto{\pgfqpoint{3.357862in}{0.413320in}}%
\pgfpathlineto{\pgfqpoint{3.355177in}{0.413320in}}%
\pgfpathlineto{\pgfqpoint{3.352505in}{0.413320in}}%
\pgfpathlineto{\pgfqpoint{3.349828in}{0.413320in}}%
\pgfpathlineto{\pgfqpoint{3.347139in}{0.413320in}}%
\pgfpathlineto{\pgfqpoint{3.344468in}{0.413320in}}%
\pgfpathlineto{\pgfqpoint{3.341893in}{0.413320in}}%
\pgfpathlineto{\pgfqpoint{3.339101in}{0.413320in}}%
\pgfpathlineto{\pgfqpoint{3.336541in}{0.413320in}}%
\pgfpathlineto{\pgfqpoint{3.333758in}{0.413320in}}%
\pgfpathlineto{\pgfqpoint{3.331183in}{0.413320in}}%
\pgfpathlineto{\pgfqpoint{3.328401in}{0.413320in}}%
\pgfpathlineto{\pgfqpoint{3.325860in}{0.413320in}}%
\pgfpathlineto{\pgfqpoint{3.323049in}{0.413320in}}%
\pgfpathlineto{\pgfqpoint{3.320366in}{0.413320in}}%
\pgfpathlineto{\pgfqpoint{3.317688in}{0.413320in}}%
\pgfpathlineto{\pgfqpoint{3.315008in}{0.413320in}}%
\pgfpathlineto{\pgfqpoint{3.312480in}{0.413320in}}%
\pgfpathlineto{\pgfqpoint{3.309652in}{0.413320in}}%
\pgfpathlineto{\pgfqpoint{3.307104in}{0.413320in}}%
\pgfpathlineto{\pgfqpoint{3.304295in}{0.413320in}}%
\pgfpathlineto{\pgfqpoint{3.301719in}{0.413320in}}%
\pgfpathlineto{\pgfqpoint{3.298937in}{0.413320in}}%
\pgfpathlineto{\pgfqpoint{3.296376in}{0.413320in}}%
\pgfpathlineto{\pgfqpoint{3.293574in}{0.413320in}}%
\pgfpathlineto{\pgfqpoint{3.290890in}{0.413320in}}%
\pgfpathlineto{\pgfqpoint{3.288225in}{0.413320in}}%
\pgfpathlineto{\pgfqpoint{3.285534in}{0.413320in}}%
\pgfpathlineto{\pgfqpoint{3.282870in}{0.413320in}}%
\pgfpathlineto{\pgfqpoint{3.280189in}{0.413320in}}%
\pgfpathlineto{\pgfqpoint{3.277603in}{0.413320in}}%
\pgfpathlineto{\pgfqpoint{3.274831in}{0.413320in}}%
\pgfpathlineto{\pgfqpoint{3.272254in}{0.413320in}}%
\pgfpathlineto{\pgfqpoint{3.269478in}{0.413320in}}%
\pgfpathlineto{\pgfqpoint{3.266849in}{0.413320in}}%
\pgfpathlineto{\pgfqpoint{3.264119in}{0.413320in}}%
\pgfpathlineto{\pgfqpoint{3.261594in}{0.413320in}}%
\pgfpathlineto{\pgfqpoint{3.258784in}{0.413320in}}%
\pgfpathlineto{\pgfqpoint{3.256083in}{0.413320in}}%
\pgfpathlineto{\pgfqpoint{3.253404in}{0.413320in}}%
\pgfpathlineto{\pgfqpoint{3.250716in}{0.413320in}}%
\pgfpathlineto{\pgfqpoint{3.248049in}{0.413320in}}%
\pgfpathlineto{\pgfqpoint{3.245363in}{0.413320in}}%
\pgfpathlineto{\pgfqpoint{3.242807in}{0.413320in}}%
\pgfpathlineto{\pgfqpoint{3.240010in}{0.413320in}}%
\pgfpathlineto{\pgfqpoint{3.237411in}{0.413320in}}%
\pgfpathlineto{\pgfqpoint{3.234658in}{0.413320in}}%
\pgfpathlineto{\pgfqpoint{3.232069in}{0.413320in}}%
\pgfpathlineto{\pgfqpoint{3.229310in}{0.413320in}}%
\pgfpathlineto{\pgfqpoint{3.226609in}{0.413320in}}%
\pgfpathlineto{\pgfqpoint{3.223942in}{0.413320in}}%
\pgfpathlineto{\pgfqpoint{3.221255in}{0.413320in}}%
\pgfpathlineto{\pgfqpoint{3.218586in}{0.413320in}}%
\pgfpathlineto{\pgfqpoint{3.215908in}{0.413320in}}%
\pgfpathlineto{\pgfqpoint{3.213342in}{0.413320in}}%
\pgfpathlineto{\pgfqpoint{3.210545in}{0.413320in}}%
\pgfpathlineto{\pgfqpoint{3.207984in}{0.413320in}}%
\pgfpathlineto{\pgfqpoint{3.205195in}{0.413320in}}%
\pgfpathlineto{\pgfqpoint{3.202562in}{0.413320in}}%
\pgfpathlineto{\pgfqpoint{3.199823in}{0.413320in}}%
\pgfpathlineto{\pgfqpoint{3.197226in}{0.413320in}}%
\pgfpathlineto{\pgfqpoint{3.194508in}{0.413320in}}%
\pgfpathlineto{\pgfqpoint{3.191796in}{0.413320in}}%
\pgfpathlineto{\pgfqpoint{3.189117in}{0.413320in}}%
\pgfpathlineto{\pgfqpoint{3.186440in}{0.413320in}}%
\pgfpathlineto{\pgfqpoint{3.183760in}{0.413320in}}%
\pgfpathlineto{\pgfqpoint{3.181089in}{0.413320in}}%
\pgfpathlineto{\pgfqpoint{3.178525in}{0.413320in}}%
\pgfpathlineto{\pgfqpoint{3.175724in}{0.413320in}}%
\pgfpathlineto{\pgfqpoint{3.173142in}{0.413320in}}%
\pgfpathlineto{\pgfqpoint{3.170375in}{0.413320in}}%
\pgfpathlineto{\pgfqpoint{3.167776in}{0.413320in}}%
\pgfpathlineto{\pgfqpoint{3.165019in}{0.413320in}}%
\pgfpathlineto{\pgfqpoint{3.162474in}{0.413320in}}%
\pgfpathlineto{\pgfqpoint{3.159675in}{0.413320in}}%
\pgfpathlineto{\pgfqpoint{3.156981in}{0.413320in}}%
\pgfpathlineto{\pgfqpoint{3.154327in}{0.413320in}}%
\pgfpathlineto{\pgfqpoint{3.151612in}{0.413320in}}%
\pgfpathlineto{\pgfqpoint{3.149057in}{0.413320in}}%
\pgfpathlineto{\pgfqpoint{3.146271in}{0.413320in}}%
\pgfpathlineto{\pgfqpoint{3.143740in}{0.413320in}}%
\pgfpathlineto{\pgfqpoint{3.140913in}{0.413320in}}%
\pgfpathlineto{\pgfqpoint{3.138375in}{0.413320in}}%
\pgfpathlineto{\pgfqpoint{3.135550in}{0.413320in}}%
\pgfpathlineto{\pgfqpoint{3.132946in}{0.413320in}}%
\pgfpathlineto{\pgfqpoint{3.130199in}{0.413320in}}%
\pgfpathlineto{\pgfqpoint{3.127512in}{0.413320in}}%
\pgfpathlineto{\pgfqpoint{3.124842in}{0.413320in}}%
\pgfpathlineto{\pgfqpoint{3.122163in}{0.413320in}}%
\pgfpathlineto{\pgfqpoint{3.119487in}{0.413320in}}%
\pgfpathlineto{\pgfqpoint{3.116807in}{0.413320in}}%
\pgfpathlineto{\pgfqpoint{3.114242in}{0.413320in}}%
\pgfpathlineto{\pgfqpoint{3.111451in}{0.413320in}}%
\pgfpathlineto{\pgfqpoint{3.108896in}{0.413320in}}%
\pgfpathlineto{\pgfqpoint{3.106094in}{0.413320in}}%
\pgfpathlineto{\pgfqpoint{3.103508in}{0.413320in}}%
\pgfpathlineto{\pgfqpoint{3.100737in}{0.413320in}}%
\pgfpathlineto{\pgfqpoint{3.098163in}{0.413320in}}%
\pgfpathlineto{\pgfqpoint{3.095388in}{0.413320in}}%
\pgfpathlineto{\pgfqpoint{3.092699in}{0.413320in}}%
\pgfpathlineto{\pgfqpoint{3.090023in}{0.413320in}}%
\pgfpathlineto{\pgfqpoint{3.087343in}{0.413320in}}%
\pgfpathlineto{\pgfqpoint{3.084671in}{0.413320in}}%
\pgfpathlineto{\pgfqpoint{3.081990in}{0.413320in}}%
\pgfpathlineto{\pgfqpoint{3.079381in}{0.413320in}}%
\pgfpathlineto{\pgfqpoint{3.076631in}{0.413320in}}%
\pgfpathlineto{\pgfqpoint{3.074056in}{0.413320in}}%
\pgfpathlineto{\pgfqpoint{3.071266in}{0.413320in}}%
\pgfpathlineto{\pgfqpoint{3.068709in}{0.413320in}}%
\pgfpathlineto{\pgfqpoint{3.065916in}{0.413320in}}%
\pgfpathlineto{\pgfqpoint{3.063230in}{0.413320in}}%
\pgfpathlineto{\pgfqpoint{3.060561in}{0.413320in}}%
\pgfpathlineto{\pgfqpoint{3.057884in}{0.413320in}}%
\pgfpathlineto{\pgfqpoint{3.055202in}{0.413320in}}%
\pgfpathlineto{\pgfqpoint{3.052526in}{0.413320in}}%
\pgfpathlineto{\pgfqpoint{3.049988in}{0.413320in}}%
\pgfpathlineto{\pgfqpoint{3.047157in}{0.413320in}}%
\pgfpathlineto{\pgfqpoint{3.044568in}{0.413320in}}%
\pgfpathlineto{\pgfqpoint{3.041813in}{0.413320in}}%
\pgfpathlineto{\pgfqpoint{3.039262in}{0.413320in}}%
\pgfpathlineto{\pgfqpoint{3.036456in}{0.413320in}}%
\pgfpathlineto{\pgfqpoint{3.033921in}{0.413320in}}%
\pgfpathlineto{\pgfqpoint{3.031091in}{0.413320in}}%
\pgfpathlineto{\pgfqpoint{3.028412in}{0.413320in}}%
\pgfpathlineto{\pgfqpoint{3.025803in}{0.413320in}}%
\pgfpathlineto{\pgfqpoint{3.023058in}{0.413320in}}%
\pgfpathlineto{\pgfqpoint{3.020382in}{0.413320in}}%
\pgfpathlineto{\pgfqpoint{3.017707in}{0.413320in}}%
\pgfpathlineto{\pgfqpoint{3.015097in}{0.413320in}}%
\pgfpathlineto{\pgfqpoint{3.012351in}{0.413320in}}%
\pgfpathlineto{\pgfqpoint{3.009784in}{0.413320in}}%
\pgfpathlineto{\pgfqpoint{3.006993in}{0.413320in}}%
\pgfpathlineto{\pgfqpoint{3.004419in}{0.413320in}}%
\pgfpathlineto{\pgfqpoint{3.001635in}{0.413320in}}%
\pgfpathlineto{\pgfqpoint{2.999103in}{0.413320in}}%
\pgfpathlineto{\pgfqpoint{2.996300in}{0.413320in}}%
\pgfpathlineto{\pgfqpoint{2.993595in}{0.413320in}}%
\pgfpathlineto{\pgfqpoint{2.990978in}{0.413320in}}%
\pgfpathlineto{\pgfqpoint{2.988238in}{0.413320in}}%
\pgfpathlineto{\pgfqpoint{2.985666in}{0.413320in}}%
\pgfpathlineto{\pgfqpoint{2.982885in}{0.413320in}}%
\pgfpathlineto{\pgfqpoint{2.980341in}{0.413320in}}%
\pgfpathlineto{\pgfqpoint{2.977517in}{0.413320in}}%
\pgfpathlineto{\pgfqpoint{2.974972in}{0.413320in}}%
\pgfpathlineto{\pgfqpoint{2.972177in}{0.413320in}}%
\pgfpathlineto{\pgfqpoint{2.969599in}{0.413320in}}%
\pgfpathlineto{\pgfqpoint{2.966812in}{0.413320in}}%
\pgfpathlineto{\pgfqpoint{2.964127in}{0.413320in}}%
\pgfpathlineto{\pgfqpoint{2.961460in}{0.413320in}}%
\pgfpathlineto{\pgfqpoint{2.958782in}{0.413320in}}%
\pgfpathlineto{\pgfqpoint{2.956103in}{0.413320in}}%
\pgfpathlineto{\pgfqpoint{2.953422in}{0.413320in}}%
\pgfpathlineto{\pgfqpoint{2.950884in}{0.413320in}}%
\pgfpathlineto{\pgfqpoint{2.948068in}{0.413320in}}%
\pgfpathlineto{\pgfqpoint{2.945461in}{0.413320in}}%
\pgfpathlineto{\pgfqpoint{2.942711in}{0.413320in}}%
\pgfpathlineto{\pgfqpoint{2.940120in}{0.413320in}}%
\pgfpathlineto{\pgfqpoint{2.937352in}{0.413320in}}%
\pgfpathlineto{\pgfqpoint{2.934759in}{0.413320in}}%
\pgfpathlineto{\pgfqpoint{2.932033in}{0.413320in}}%
\pgfpathlineto{\pgfqpoint{2.929321in}{0.413320in}}%
\pgfpathlineto{\pgfqpoint{2.926655in}{0.413320in}}%
\pgfpathlineto{\pgfqpoint{2.923963in}{0.413320in}}%
\pgfpathlineto{\pgfqpoint{2.921363in}{0.413320in}}%
\pgfpathlineto{\pgfqpoint{2.918606in}{0.413320in}}%
\pgfpathlineto{\pgfqpoint{2.916061in}{0.413320in}}%
\pgfpathlineto{\pgfqpoint{2.913243in}{0.413320in}}%
\pgfpathlineto{\pgfqpoint{2.910631in}{0.413320in}}%
\pgfpathlineto{\pgfqpoint{2.907882in}{0.413320in}}%
\pgfpathlineto{\pgfqpoint{2.905341in}{0.413320in}}%
\pgfpathlineto{\pgfqpoint{2.902535in}{0.413320in}}%
\pgfpathlineto{\pgfqpoint{2.899858in}{0.413320in}}%
\pgfpathlineto{\pgfqpoint{2.897179in}{0.413320in}}%
\pgfpathlineto{\pgfqpoint{2.894487in}{0.413320in}}%
\pgfpathlineto{\pgfqpoint{2.891809in}{0.413320in}}%
\pgfpathlineto{\pgfqpoint{2.889145in}{0.413320in}}%
\pgfpathlineto{\pgfqpoint{2.886578in}{0.413320in}}%
\pgfpathlineto{\pgfqpoint{2.883780in}{0.413320in}}%
\pgfpathlineto{\pgfqpoint{2.881254in}{0.413320in}}%
\pgfpathlineto{\pgfqpoint{2.878431in}{0.413320in}}%
\pgfpathlineto{\pgfqpoint{2.875882in}{0.413320in}}%
\pgfpathlineto{\pgfqpoint{2.873074in}{0.413320in}}%
\pgfpathlineto{\pgfqpoint{2.870475in}{0.413320in}}%
\pgfpathlineto{\pgfqpoint{2.867713in}{0.413320in}}%
\pgfpathlineto{\pgfqpoint{2.865031in}{0.413320in}}%
\pgfpathlineto{\pgfqpoint{2.862402in}{0.413320in}}%
\pgfpathlineto{\pgfqpoint{2.859668in}{0.413320in}}%
\pgfpathlineto{\pgfqpoint{2.857003in}{0.413320in}}%
\pgfpathlineto{\pgfqpoint{2.854325in}{0.413320in}}%
\pgfpathlineto{\pgfqpoint{2.851793in}{0.413320in}}%
\pgfpathlineto{\pgfqpoint{2.848960in}{0.413320in}}%
\pgfpathlineto{\pgfqpoint{2.846408in}{0.413320in}}%
\pgfpathlineto{\pgfqpoint{2.843611in}{0.413320in}}%
\pgfpathlineto{\pgfqpoint{2.841055in}{0.413320in}}%
\pgfpathlineto{\pgfqpoint{2.838254in}{0.413320in}}%
\pgfpathlineto{\pgfqpoint{2.835698in}{0.413320in}}%
\pgfpathlineto{\pgfqpoint{2.832894in}{0.413320in}}%
\pgfpathlineto{\pgfqpoint{2.830219in}{0.413320in}}%
\pgfpathlineto{\pgfqpoint{2.827567in}{0.413320in}}%
\pgfpathlineto{\pgfqpoint{2.824851in}{0.413320in}}%
\pgfpathlineto{\pgfqpoint{2.822303in}{0.413320in}}%
\pgfpathlineto{\pgfqpoint{2.819506in}{0.413320in}}%
\pgfpathlineto{\pgfqpoint{2.816867in}{0.413320in}}%
\pgfpathlineto{\pgfqpoint{2.814141in}{0.413320in}}%
\pgfpathlineto{\pgfqpoint{2.811597in}{0.413320in}}%
\pgfpathlineto{\pgfqpoint{2.808792in}{0.413320in}}%
\pgfpathlineto{\pgfqpoint{2.806175in}{0.413320in}}%
\pgfpathlineto{\pgfqpoint{2.803435in}{0.413320in}}%
\pgfpathlineto{\pgfqpoint{2.800756in}{0.413320in}}%
\pgfpathlineto{\pgfqpoint{2.798070in}{0.413320in}}%
\pgfpathlineto{\pgfqpoint{2.795398in}{0.413320in}}%
\pgfpathlineto{\pgfqpoint{2.792721in}{0.413320in}}%
\pgfpathlineto{\pgfqpoint{2.790044in}{0.413320in}}%
\pgfpathlineto{\pgfqpoint{2.787468in}{0.413320in}}%
\pgfpathlineto{\pgfqpoint{2.784687in}{0.413320in}}%
\pgfpathlineto{\pgfqpoint{2.782113in}{0.413320in}}%
\pgfpathlineto{\pgfqpoint{2.779330in}{0.413320in}}%
\pgfpathlineto{\pgfqpoint{2.776767in}{0.413320in}}%
\pgfpathlineto{\pgfqpoint{2.773972in}{0.413320in}}%
\pgfpathlineto{\pgfqpoint{2.771373in}{0.413320in}}%
\pgfpathlineto{\pgfqpoint{2.768617in}{0.413320in}}%
\pgfpathlineto{\pgfqpoint{2.765935in}{0.413320in}}%
\pgfpathlineto{\pgfqpoint{2.763253in}{0.413320in}}%
\pgfpathlineto{\pgfqpoint{2.760581in}{0.413320in}}%
\pgfpathlineto{\pgfqpoint{2.758028in}{0.413320in}}%
\pgfpathlineto{\pgfqpoint{2.755224in}{0.413320in}}%
\pgfpathlineto{\pgfqpoint{2.752614in}{0.413320in}}%
\pgfpathlineto{\pgfqpoint{2.749868in}{0.413320in}}%
\pgfpathlineto{\pgfqpoint{2.747260in}{0.413320in}}%
\pgfpathlineto{\pgfqpoint{2.744510in}{0.413320in}}%
\pgfpathlineto{\pgfqpoint{2.741928in}{0.413320in}}%
\pgfpathlineto{\pgfqpoint{2.739155in}{0.413320in}}%
\pgfpathlineto{\pgfqpoint{2.736476in}{0.413320in}}%
\pgfpathlineto{\pgfqpoint{2.733798in}{0.413320in}}%
\pgfpathlineto{\pgfqpoint{2.731119in}{0.413320in}}%
\pgfpathlineto{\pgfqpoint{2.728439in}{0.413320in}}%
\pgfpathlineto{\pgfqpoint{2.725760in}{0.413320in}}%
\pgfpathlineto{\pgfqpoint{2.723211in}{0.413320in}}%
\pgfpathlineto{\pgfqpoint{2.720404in}{0.413320in}}%
\pgfpathlineto{\pgfqpoint{2.717773in}{0.413320in}}%
\pgfpathlineto{\pgfqpoint{2.715036in}{0.413320in}}%
\pgfpathlineto{\pgfqpoint{2.712477in}{0.413320in}}%
\pgfpathlineto{\pgfqpoint{2.709683in}{0.413320in}}%
\pgfpathlineto{\pgfqpoint{2.707125in}{0.413320in}}%
\pgfpathlineto{\pgfqpoint{2.704326in}{0.413320in}}%
\pgfpathlineto{\pgfqpoint{2.701657in}{0.413320in}}%
\pgfpathlineto{\pgfqpoint{2.698968in}{0.413320in}}%
\pgfpathlineto{\pgfqpoint{2.696293in}{0.413320in}}%
\pgfpathlineto{\pgfqpoint{2.693611in}{0.413320in}}%
\pgfpathlineto{\pgfqpoint{2.690940in}{0.413320in}}%
\pgfpathlineto{\pgfqpoint{2.688328in}{0.413320in}}%
\pgfpathlineto{\pgfqpoint{2.685586in}{0.413320in}}%
\pgfpathlineto{\pgfqpoint{2.683009in}{0.413320in}}%
\pgfpathlineto{\pgfqpoint{2.680224in}{0.413320in}}%
\pgfpathlineto{\pgfqpoint{2.677650in}{0.413320in}}%
\pgfpathlineto{\pgfqpoint{2.674873in}{0.413320in}}%
\pgfpathlineto{\pgfqpoint{2.672301in}{0.413320in}}%
\pgfpathlineto{\pgfqpoint{2.669506in}{0.413320in}}%
\pgfpathlineto{\pgfqpoint{2.666836in}{0.413320in}}%
\pgfpathlineto{\pgfqpoint{2.664151in}{0.413320in}}%
\pgfpathlineto{\pgfqpoint{2.661481in}{0.413320in}}%
\pgfpathlineto{\pgfqpoint{2.658942in}{0.413320in}}%
\pgfpathlineto{\pgfqpoint{2.656124in}{0.413320in}}%
\pgfpathlineto{\pgfqpoint{2.653567in}{0.413320in}}%
\pgfpathlineto{\pgfqpoint{2.650767in}{0.413320in}}%
\pgfpathlineto{\pgfqpoint{2.648196in}{0.413320in}}%
\pgfpathlineto{\pgfqpoint{2.645408in}{0.413320in}}%
\pgfpathlineto{\pgfqpoint{2.642827in}{0.413320in}}%
\pgfpathlineto{\pgfqpoint{2.640053in}{0.413320in}}%
\pgfpathlineto{\pgfqpoint{2.637369in}{0.413320in}}%
\pgfpathlineto{\pgfqpoint{2.634700in}{0.413320in}}%
\pgfpathlineto{\pgfqpoint{2.632018in}{0.413320in}}%
\pgfpathlineto{\pgfqpoint{2.629340in}{0.413320in}}%
\pgfpathlineto{\pgfqpoint{2.626653in}{0.413320in}}%
\pgfpathlineto{\pgfqpoint{2.624077in}{0.413320in}}%
\pgfpathlineto{\pgfqpoint{2.621304in}{0.413320in}}%
\pgfpathlineto{\pgfqpoint{2.618773in}{0.413320in}}%
\pgfpathlineto{\pgfqpoint{2.615934in}{0.413320in}}%
\pgfpathlineto{\pgfqpoint{2.613393in}{0.413320in}}%
\pgfpathlineto{\pgfqpoint{2.610588in}{0.413320in}}%
\pgfpathlineto{\pgfqpoint{2.608004in}{0.413320in}}%
\pgfpathlineto{\pgfqpoint{2.605232in}{0.413320in}}%
\pgfpathlineto{\pgfqpoint{2.602557in}{0.413320in}}%
\pgfpathlineto{\pgfqpoint{2.599920in}{0.413320in}}%
\pgfpathlineto{\pgfqpoint{2.597196in}{0.413320in}}%
\pgfpathlineto{\pgfqpoint{2.594630in}{0.413320in}}%
\pgfpathlineto{\pgfqpoint{2.591842in}{0.413320in}}%
\pgfpathlineto{\pgfqpoint{2.589248in}{0.413320in}}%
\pgfpathlineto{\pgfqpoint{2.586484in}{0.413320in}}%
\pgfpathlineto{\pgfqpoint{2.583913in}{0.413320in}}%
\pgfpathlineto{\pgfqpoint{2.581129in}{0.413320in}}%
\pgfpathlineto{\pgfqpoint{2.578567in}{0.413320in}}%
\pgfpathlineto{\pgfqpoint{2.575779in}{0.413320in}}%
\pgfpathlineto{\pgfqpoint{2.573082in}{0.413320in}}%
\pgfpathlineto{\pgfqpoint{2.570411in}{0.413320in}}%
\pgfpathlineto{\pgfqpoint{2.567730in}{0.413320in}}%
\pgfpathlineto{\pgfqpoint{2.565045in}{0.413320in}}%
\pgfpathlineto{\pgfqpoint{2.562375in}{0.413320in}}%
\pgfpathlineto{\pgfqpoint{2.559790in}{0.413320in}}%
\pgfpathlineto{\pgfqpoint{2.557009in}{0.413320in}}%
\pgfpathlineto{\pgfqpoint{2.554493in}{0.413320in}}%
\pgfpathlineto{\pgfqpoint{2.551664in}{0.413320in}}%
\pgfpathlineto{\pgfqpoint{2.549114in}{0.413320in}}%
\pgfpathlineto{\pgfqpoint{2.546310in}{0.413320in}}%
\pgfpathlineto{\pgfqpoint{2.543765in}{0.413320in}}%
\pgfpathlineto{\pgfqpoint{2.540949in}{0.413320in}}%
\pgfpathlineto{\pgfqpoint{2.538274in}{0.413320in}}%
\pgfpathlineto{\pgfqpoint{2.535624in}{0.413320in}}%
\pgfpathlineto{\pgfqpoint{2.532917in}{0.413320in}}%
\pgfpathlineto{\pgfqpoint{2.530234in}{0.413320in}}%
\pgfpathlineto{\pgfqpoint{2.527560in}{0.413320in}}%
\pgfpathlineto{\pgfqpoint{2.524988in}{0.413320in}}%
\pgfpathlineto{\pgfqpoint{2.522197in}{0.413320in}}%
\pgfpathlineto{\pgfqpoint{2.519607in}{0.413320in}}%
\pgfpathlineto{\pgfqpoint{2.516845in}{0.413320in}}%
\pgfpathlineto{\pgfqpoint{2.514268in}{0.413320in}}%
\pgfpathlineto{\pgfqpoint{2.511478in}{0.413320in}}%
\pgfpathlineto{\pgfqpoint{2.508917in}{0.413320in}}%
\pgfpathlineto{\pgfqpoint{2.506163in}{0.413320in}}%
\pgfpathlineto{\pgfqpoint{2.503454in}{0.413320in}}%
\pgfpathlineto{\pgfqpoint{2.500801in}{0.413320in}}%
\pgfpathlineto{\pgfqpoint{2.498085in}{0.413320in}}%
\pgfpathlineto{\pgfqpoint{2.495542in}{0.413320in}}%
\pgfpathlineto{\pgfqpoint{2.492729in}{0.413320in}}%
\pgfpathlineto{\pgfqpoint{2.490183in}{0.413320in}}%
\pgfpathlineto{\pgfqpoint{2.487384in}{0.413320in}}%
\pgfpathlineto{\pgfqpoint{2.484870in}{0.413320in}}%
\pgfpathlineto{\pgfqpoint{2.482026in}{0.413320in}}%
\pgfpathlineto{\pgfqpoint{2.479420in}{0.413320in}}%
\pgfpathlineto{\pgfqpoint{2.476671in}{0.413320in}}%
\pgfpathlineto{\pgfqpoint{2.473989in}{0.413320in}}%
\pgfpathlineto{\pgfqpoint{2.471311in}{0.413320in}}%
\pgfpathlineto{\pgfqpoint{2.468635in}{0.413320in}}%
\pgfpathlineto{\pgfqpoint{2.465957in}{0.413320in}}%
\pgfpathlineto{\pgfqpoint{2.463280in}{0.413320in}}%
\pgfpathlineto{\pgfqpoint{2.460711in}{0.413320in}}%
\pgfpathlineto{\pgfqpoint{2.457917in}{0.413320in}}%
\pgfpathlineto{\pgfqpoint{2.455353in}{0.413320in}}%
\pgfpathlineto{\pgfqpoint{2.452562in}{0.413320in}}%
\pgfpathlineto{\pgfqpoint{2.450032in}{0.413320in}}%
\pgfpathlineto{\pgfqpoint{2.447209in}{0.413320in}}%
\pgfpathlineto{\pgfqpoint{2.444677in}{0.413320in}}%
\pgfpathlineto{\pgfqpoint{2.441876in}{0.413320in}}%
\pgfpathlineto{\pgfqpoint{2.439167in}{0.413320in}}%
\pgfpathlineto{\pgfqpoint{2.436518in}{0.413320in}}%
\pgfpathlineto{\pgfqpoint{2.433815in}{0.413320in}}%
\pgfpathlineto{\pgfqpoint{2.431251in}{0.413320in}}%
\pgfpathlineto{\pgfqpoint{2.428453in}{0.413320in}}%
\pgfpathlineto{\pgfqpoint{2.425878in}{0.413320in}}%
\pgfpathlineto{\pgfqpoint{2.423098in}{0.413320in}}%
\pgfpathlineto{\pgfqpoint{2.420528in}{0.413320in}}%
\pgfpathlineto{\pgfqpoint{2.417747in}{0.413320in}}%
\pgfpathlineto{\pgfqpoint{2.415184in}{0.413320in}}%
\pgfpathlineto{\pgfqpoint{2.412389in}{0.413320in}}%
\pgfpathlineto{\pgfqpoint{2.409699in}{0.413320in}}%
\pgfpathlineto{\pgfqpoint{2.407024in}{0.413320in}}%
\pgfpathlineto{\pgfqpoint{2.404352in}{0.413320in}}%
\pgfpathlineto{\pgfqpoint{2.401675in}{0.413320in}}%
\pgfpathlineto{\pgfqpoint{2.398995in}{0.413320in}}%
\pgfpathclose%
\pgfusepath{stroke,fill}%
\end{pgfscope}%
\begin{pgfscope}%
\pgfpathrectangle{\pgfqpoint{2.398995in}{0.319877in}}{\pgfqpoint{3.986877in}{1.993438in}} %
\pgfusepath{clip}%
\pgfsetbuttcap%
\pgfsetroundjoin%
\definecolor{currentfill}{rgb}{1.000000,1.000000,1.000000}%
\pgfsetfillcolor{currentfill}%
\pgfsetlinewidth{1.003750pt}%
\definecolor{currentstroke}{rgb}{0.275658,0.629942,0.958673}%
\pgfsetstrokecolor{currentstroke}%
\pgfsetdash{}{0pt}%
\pgfpathmoveto{\pgfqpoint{2.398995in}{0.413320in}}%
\pgfpathlineto{\pgfqpoint{2.398995in}{1.290562in}}%
\pgfpathlineto{\pgfqpoint{2.401675in}{1.289064in}}%
\pgfpathlineto{\pgfqpoint{2.404352in}{1.293289in}}%
\pgfpathlineto{\pgfqpoint{2.407024in}{1.292502in}}%
\pgfpathlineto{\pgfqpoint{2.409699in}{1.294363in}}%
\pgfpathlineto{\pgfqpoint{2.412389in}{1.285545in}}%
\pgfpathlineto{\pgfqpoint{2.415184in}{1.287219in}}%
\pgfpathlineto{\pgfqpoint{2.417747in}{1.284970in}}%
\pgfpathlineto{\pgfqpoint{2.420528in}{1.287122in}}%
\pgfpathlineto{\pgfqpoint{2.423098in}{1.287597in}}%
\pgfpathlineto{\pgfqpoint{2.425878in}{1.291190in}}%
\pgfpathlineto{\pgfqpoint{2.428453in}{1.292933in}}%
\pgfpathlineto{\pgfqpoint{2.431251in}{1.284970in}}%
\pgfpathlineto{\pgfqpoint{2.433815in}{1.284970in}}%
\pgfpathlineto{\pgfqpoint{2.436518in}{1.284970in}}%
\pgfpathlineto{\pgfqpoint{2.439167in}{1.284970in}}%
\pgfpathlineto{\pgfqpoint{2.441876in}{1.284970in}}%
\pgfpathlineto{\pgfqpoint{2.444677in}{1.284970in}}%
\pgfpathlineto{\pgfqpoint{2.447209in}{1.284970in}}%
\pgfpathlineto{\pgfqpoint{2.450032in}{1.284970in}}%
\pgfpathlineto{\pgfqpoint{2.452562in}{1.287121in}}%
\pgfpathlineto{\pgfqpoint{2.455353in}{1.284970in}}%
\pgfpathlineto{\pgfqpoint{2.457917in}{1.285520in}}%
\pgfpathlineto{\pgfqpoint{2.460711in}{1.284970in}}%
\pgfpathlineto{\pgfqpoint{2.463280in}{1.287812in}}%
\pgfpathlineto{\pgfqpoint{2.465957in}{1.284970in}}%
\pgfpathlineto{\pgfqpoint{2.468635in}{1.286556in}}%
\pgfpathlineto{\pgfqpoint{2.471311in}{1.285551in}}%
\pgfpathlineto{\pgfqpoint{2.473989in}{1.285254in}}%
\pgfpathlineto{\pgfqpoint{2.476671in}{1.284970in}}%
\pgfpathlineto{\pgfqpoint{2.479420in}{1.286296in}}%
\pgfpathlineto{\pgfqpoint{2.482026in}{1.284970in}}%
\pgfpathlineto{\pgfqpoint{2.484870in}{1.284970in}}%
\pgfpathlineto{\pgfqpoint{2.487384in}{1.284970in}}%
\pgfpathlineto{\pgfqpoint{2.490183in}{1.284970in}}%
\pgfpathlineto{\pgfqpoint{2.492729in}{1.284970in}}%
\pgfpathlineto{\pgfqpoint{2.495542in}{1.286879in}}%
\pgfpathlineto{\pgfqpoint{2.498085in}{1.286325in}}%
\pgfpathlineto{\pgfqpoint{2.500801in}{1.288187in}}%
\pgfpathlineto{\pgfqpoint{2.503454in}{1.285122in}}%
\pgfpathlineto{\pgfqpoint{2.506163in}{1.287770in}}%
\pgfpathlineto{\pgfqpoint{2.508917in}{1.288708in}}%
\pgfpathlineto{\pgfqpoint{2.511478in}{1.290604in}}%
\pgfpathlineto{\pgfqpoint{2.514268in}{1.287521in}}%
\pgfpathlineto{\pgfqpoint{2.516845in}{1.288623in}}%
\pgfpathlineto{\pgfqpoint{2.519607in}{1.286738in}}%
\pgfpathlineto{\pgfqpoint{2.522197in}{1.288270in}}%
\pgfpathlineto{\pgfqpoint{2.524988in}{1.286168in}}%
\pgfpathlineto{\pgfqpoint{2.527560in}{1.287934in}}%
\pgfpathlineto{\pgfqpoint{2.530234in}{1.289228in}}%
\pgfpathlineto{\pgfqpoint{2.532917in}{1.285970in}}%
\pgfpathlineto{\pgfqpoint{2.535624in}{1.284998in}}%
\pgfpathlineto{\pgfqpoint{2.538274in}{1.284970in}}%
\pgfpathlineto{\pgfqpoint{2.540949in}{1.286327in}}%
\pgfpathlineto{\pgfqpoint{2.543765in}{1.286990in}}%
\pgfpathlineto{\pgfqpoint{2.546310in}{1.289085in}}%
\pgfpathlineto{\pgfqpoint{2.549114in}{1.289017in}}%
\pgfpathlineto{\pgfqpoint{2.551664in}{1.284970in}}%
\pgfpathlineto{\pgfqpoint{2.554493in}{1.284970in}}%
\pgfpathlineto{\pgfqpoint{2.557009in}{1.284970in}}%
\pgfpathlineto{\pgfqpoint{2.559790in}{1.284970in}}%
\pgfpathlineto{\pgfqpoint{2.562375in}{1.284970in}}%
\pgfpathlineto{\pgfqpoint{2.565045in}{1.284970in}}%
\pgfpathlineto{\pgfqpoint{2.567730in}{1.284970in}}%
\pgfpathlineto{\pgfqpoint{2.570411in}{1.284970in}}%
\pgfpathlineto{\pgfqpoint{2.573082in}{1.284970in}}%
\pgfpathlineto{\pgfqpoint{2.575779in}{1.284970in}}%
\pgfpathlineto{\pgfqpoint{2.578567in}{1.284970in}}%
\pgfpathlineto{\pgfqpoint{2.581129in}{1.284970in}}%
\pgfpathlineto{\pgfqpoint{2.583913in}{1.285897in}}%
\pgfpathlineto{\pgfqpoint{2.586484in}{1.285605in}}%
\pgfpathlineto{\pgfqpoint{2.589248in}{1.286638in}}%
\pgfpathlineto{\pgfqpoint{2.591842in}{1.286994in}}%
\pgfpathlineto{\pgfqpoint{2.594630in}{1.287696in}}%
\pgfpathlineto{\pgfqpoint{2.597196in}{1.286198in}}%
\pgfpathlineto{\pgfqpoint{2.599920in}{1.284970in}}%
\pgfpathlineto{\pgfqpoint{2.602557in}{1.287090in}}%
\pgfpathlineto{\pgfqpoint{2.605232in}{1.284970in}}%
\pgfpathlineto{\pgfqpoint{2.608004in}{1.287026in}}%
\pgfpathlineto{\pgfqpoint{2.610588in}{1.287235in}}%
\pgfpathlineto{\pgfqpoint{2.613393in}{1.287358in}}%
\pgfpathlineto{\pgfqpoint{2.615934in}{1.284970in}}%
\pgfpathlineto{\pgfqpoint{2.618773in}{1.289583in}}%
\pgfpathlineto{\pgfqpoint{2.621304in}{1.288956in}}%
\pgfpathlineto{\pgfqpoint{2.624077in}{1.289421in}}%
\pgfpathlineto{\pgfqpoint{2.626653in}{1.289869in}}%
\pgfpathlineto{\pgfqpoint{2.629340in}{1.288421in}}%
\pgfpathlineto{\pgfqpoint{2.632018in}{1.292591in}}%
\pgfpathlineto{\pgfqpoint{2.634700in}{1.287440in}}%
\pgfpathlineto{\pgfqpoint{2.637369in}{1.287462in}}%
\pgfpathlineto{\pgfqpoint{2.640053in}{1.284970in}}%
\pgfpathlineto{\pgfqpoint{2.642827in}{1.290874in}}%
\pgfpathlineto{\pgfqpoint{2.645408in}{1.290284in}}%
\pgfpathlineto{\pgfqpoint{2.648196in}{1.293441in}}%
\pgfpathlineto{\pgfqpoint{2.650767in}{1.298186in}}%
\pgfpathlineto{\pgfqpoint{2.653567in}{1.299155in}}%
\pgfpathlineto{\pgfqpoint{2.656124in}{1.299525in}}%
\pgfpathlineto{\pgfqpoint{2.658942in}{1.288910in}}%
\pgfpathlineto{\pgfqpoint{2.661481in}{1.289120in}}%
\pgfpathlineto{\pgfqpoint{2.664151in}{1.288609in}}%
\pgfpathlineto{\pgfqpoint{2.666836in}{1.289752in}}%
\pgfpathlineto{\pgfqpoint{2.669506in}{1.288441in}}%
\pgfpathlineto{\pgfqpoint{2.672301in}{1.288304in}}%
\pgfpathlineto{\pgfqpoint{2.674873in}{1.291312in}}%
\pgfpathlineto{\pgfqpoint{2.677650in}{1.289692in}}%
\pgfpathlineto{\pgfqpoint{2.680224in}{1.288148in}}%
\pgfpathlineto{\pgfqpoint{2.683009in}{1.290120in}}%
\pgfpathlineto{\pgfqpoint{2.685586in}{1.289897in}}%
\pgfpathlineto{\pgfqpoint{2.688328in}{1.288723in}}%
\pgfpathlineto{\pgfqpoint{2.690940in}{1.286029in}}%
\pgfpathlineto{\pgfqpoint{2.693611in}{1.292312in}}%
\pgfpathlineto{\pgfqpoint{2.696293in}{1.287505in}}%
\pgfpathlineto{\pgfqpoint{2.698968in}{1.289862in}}%
\pgfpathlineto{\pgfqpoint{2.701657in}{1.287071in}}%
\pgfpathlineto{\pgfqpoint{2.704326in}{1.286901in}}%
\pgfpathlineto{\pgfqpoint{2.707125in}{1.284970in}}%
\pgfpathlineto{\pgfqpoint{2.709683in}{1.285989in}}%
\pgfpathlineto{\pgfqpoint{2.712477in}{1.286924in}}%
\pgfpathlineto{\pgfqpoint{2.715036in}{1.293029in}}%
\pgfpathlineto{\pgfqpoint{2.717773in}{1.286637in}}%
\pgfpathlineto{\pgfqpoint{2.720404in}{1.284970in}}%
\pgfpathlineto{\pgfqpoint{2.723211in}{1.286984in}}%
\pgfpathlineto{\pgfqpoint{2.725760in}{1.286798in}}%
\pgfpathlineto{\pgfqpoint{2.728439in}{1.284970in}}%
\pgfpathlineto{\pgfqpoint{2.731119in}{1.284970in}}%
\pgfpathlineto{\pgfqpoint{2.733798in}{1.284970in}}%
\pgfpathlineto{\pgfqpoint{2.736476in}{1.284970in}}%
\pgfpathlineto{\pgfqpoint{2.739155in}{1.284970in}}%
\pgfpathlineto{\pgfqpoint{2.741928in}{1.284970in}}%
\pgfpathlineto{\pgfqpoint{2.744510in}{1.284970in}}%
\pgfpathlineto{\pgfqpoint{2.747260in}{1.284970in}}%
\pgfpathlineto{\pgfqpoint{2.749868in}{1.286618in}}%
\pgfpathlineto{\pgfqpoint{2.752614in}{1.285147in}}%
\pgfpathlineto{\pgfqpoint{2.755224in}{1.285010in}}%
\pgfpathlineto{\pgfqpoint{2.758028in}{1.285754in}}%
\pgfpathlineto{\pgfqpoint{2.760581in}{1.287282in}}%
\pgfpathlineto{\pgfqpoint{2.763253in}{1.287137in}}%
\pgfpathlineto{\pgfqpoint{2.765935in}{1.287194in}}%
\pgfpathlineto{\pgfqpoint{2.768617in}{1.287703in}}%
\pgfpathlineto{\pgfqpoint{2.771373in}{1.285254in}}%
\pgfpathlineto{\pgfqpoint{2.773972in}{1.287149in}}%
\pgfpathlineto{\pgfqpoint{2.776767in}{1.284970in}}%
\pgfpathlineto{\pgfqpoint{2.779330in}{1.284970in}}%
\pgfpathlineto{\pgfqpoint{2.782113in}{1.284970in}}%
\pgfpathlineto{\pgfqpoint{2.784687in}{1.284970in}}%
\pgfpathlineto{\pgfqpoint{2.787468in}{1.284970in}}%
\pgfpathlineto{\pgfqpoint{2.790044in}{1.284970in}}%
\pgfpathlineto{\pgfqpoint{2.792721in}{1.286001in}}%
\pgfpathlineto{\pgfqpoint{2.795398in}{1.284970in}}%
\pgfpathlineto{\pgfqpoint{2.798070in}{1.284970in}}%
\pgfpathlineto{\pgfqpoint{2.800756in}{1.290945in}}%
\pgfpathlineto{\pgfqpoint{2.803435in}{1.285833in}}%
\pgfpathlineto{\pgfqpoint{2.806175in}{1.284970in}}%
\pgfpathlineto{\pgfqpoint{2.808792in}{1.293255in}}%
\pgfpathlineto{\pgfqpoint{2.811597in}{1.289989in}}%
\pgfpathlineto{\pgfqpoint{2.814141in}{1.290761in}}%
\pgfpathlineto{\pgfqpoint{2.816867in}{1.290929in}}%
\pgfpathlineto{\pgfqpoint{2.819506in}{1.286662in}}%
\pgfpathlineto{\pgfqpoint{2.822303in}{1.288907in}}%
\pgfpathlineto{\pgfqpoint{2.824851in}{1.290477in}}%
\pgfpathlineto{\pgfqpoint{2.827567in}{1.286524in}}%
\pgfpathlineto{\pgfqpoint{2.830219in}{1.290116in}}%
\pgfpathlineto{\pgfqpoint{2.832894in}{1.290417in}}%
\pgfpathlineto{\pgfqpoint{2.835698in}{1.289773in}}%
\pgfpathlineto{\pgfqpoint{2.838254in}{1.294282in}}%
\pgfpathlineto{\pgfqpoint{2.841055in}{1.301927in}}%
\pgfpathlineto{\pgfqpoint{2.843611in}{1.297955in}}%
\pgfpathlineto{\pgfqpoint{2.846408in}{1.294339in}}%
\pgfpathlineto{\pgfqpoint{2.848960in}{1.294044in}}%
\pgfpathlineto{\pgfqpoint{2.851793in}{1.292110in}}%
\pgfpathlineto{\pgfqpoint{2.854325in}{1.293490in}}%
\pgfpathlineto{\pgfqpoint{2.857003in}{1.301396in}}%
\pgfpathlineto{\pgfqpoint{2.859668in}{1.300833in}}%
\pgfpathlineto{\pgfqpoint{2.862402in}{1.301332in}}%
\pgfpathlineto{\pgfqpoint{2.865031in}{1.296088in}}%
\pgfpathlineto{\pgfqpoint{2.867713in}{1.295974in}}%
\pgfpathlineto{\pgfqpoint{2.870475in}{1.292751in}}%
\pgfpathlineto{\pgfqpoint{2.873074in}{1.290983in}}%
\pgfpathlineto{\pgfqpoint{2.875882in}{1.287817in}}%
\pgfpathlineto{\pgfqpoint{2.878431in}{1.288898in}}%
\pgfpathlineto{\pgfqpoint{2.881254in}{1.295954in}}%
\pgfpathlineto{\pgfqpoint{2.883780in}{1.308291in}}%
\pgfpathlineto{\pgfqpoint{2.886578in}{1.293113in}}%
\pgfpathlineto{\pgfqpoint{2.889145in}{1.289347in}}%
\pgfpathlineto{\pgfqpoint{2.891809in}{1.287390in}}%
\pgfpathlineto{\pgfqpoint{2.894487in}{1.287364in}}%
\pgfpathlineto{\pgfqpoint{2.897179in}{1.284970in}}%
\pgfpathlineto{\pgfqpoint{2.899858in}{1.284970in}}%
\pgfpathlineto{\pgfqpoint{2.902535in}{1.286448in}}%
\pgfpathlineto{\pgfqpoint{2.905341in}{1.284970in}}%
\pgfpathlineto{\pgfqpoint{2.907882in}{1.287252in}}%
\pgfpathlineto{\pgfqpoint{2.910631in}{1.287589in}}%
\pgfpathlineto{\pgfqpoint{2.913243in}{1.290699in}}%
\pgfpathlineto{\pgfqpoint{2.916061in}{1.297253in}}%
\pgfpathlineto{\pgfqpoint{2.918606in}{1.293776in}}%
\pgfpathlineto{\pgfqpoint{2.921363in}{1.286871in}}%
\pgfpathlineto{\pgfqpoint{2.923963in}{1.288608in}}%
\pgfpathlineto{\pgfqpoint{2.926655in}{1.288945in}}%
\pgfpathlineto{\pgfqpoint{2.929321in}{1.284970in}}%
\pgfpathlineto{\pgfqpoint{2.932033in}{1.286706in}}%
\pgfpathlineto{\pgfqpoint{2.934759in}{1.285935in}}%
\pgfpathlineto{\pgfqpoint{2.937352in}{1.289097in}}%
\pgfpathlineto{\pgfqpoint{2.940120in}{1.289683in}}%
\pgfpathlineto{\pgfqpoint{2.942711in}{1.290749in}}%
\pgfpathlineto{\pgfqpoint{2.945461in}{1.292414in}}%
\pgfpathlineto{\pgfqpoint{2.948068in}{1.289860in}}%
\pgfpathlineto{\pgfqpoint{2.950884in}{1.289254in}}%
\pgfpathlineto{\pgfqpoint{2.953422in}{1.289052in}}%
\pgfpathlineto{\pgfqpoint{2.956103in}{1.290241in}}%
\pgfpathlineto{\pgfqpoint{2.958782in}{1.290225in}}%
\pgfpathlineto{\pgfqpoint{2.961460in}{1.291554in}}%
\pgfpathlineto{\pgfqpoint{2.964127in}{1.291661in}}%
\pgfpathlineto{\pgfqpoint{2.966812in}{1.291942in}}%
\pgfpathlineto{\pgfqpoint{2.969599in}{1.291899in}}%
\pgfpathlineto{\pgfqpoint{2.972177in}{1.291904in}}%
\pgfpathlineto{\pgfqpoint{2.974972in}{1.293265in}}%
\pgfpathlineto{\pgfqpoint{2.977517in}{1.290287in}}%
\pgfpathlineto{\pgfqpoint{2.980341in}{1.289939in}}%
\pgfpathlineto{\pgfqpoint{2.982885in}{1.289433in}}%
\pgfpathlineto{\pgfqpoint{2.985666in}{1.291300in}}%
\pgfpathlineto{\pgfqpoint{2.988238in}{1.288313in}}%
\pgfpathlineto{\pgfqpoint{2.990978in}{1.289468in}}%
\pgfpathlineto{\pgfqpoint{2.993595in}{1.294707in}}%
\pgfpathlineto{\pgfqpoint{2.996300in}{1.286462in}}%
\pgfpathlineto{\pgfqpoint{2.999103in}{1.284970in}}%
\pgfpathlineto{\pgfqpoint{3.001635in}{1.288017in}}%
\pgfpathlineto{\pgfqpoint{3.004419in}{1.289462in}}%
\pgfpathlineto{\pgfqpoint{3.006993in}{1.290333in}}%
\pgfpathlineto{\pgfqpoint{3.009784in}{1.292836in}}%
\pgfpathlineto{\pgfqpoint{3.012351in}{1.291843in}}%
\pgfpathlineto{\pgfqpoint{3.015097in}{1.292535in}}%
\pgfpathlineto{\pgfqpoint{3.017707in}{1.294144in}}%
\pgfpathlineto{\pgfqpoint{3.020382in}{1.293606in}}%
\pgfpathlineto{\pgfqpoint{3.023058in}{1.292730in}}%
\pgfpathlineto{\pgfqpoint{3.025803in}{1.289133in}}%
\pgfpathlineto{\pgfqpoint{3.028412in}{1.286547in}}%
\pgfpathlineto{\pgfqpoint{3.031091in}{1.284970in}}%
\pgfpathlineto{\pgfqpoint{3.033921in}{1.286955in}}%
\pgfpathlineto{\pgfqpoint{3.036456in}{1.287973in}}%
\pgfpathlineto{\pgfqpoint{3.039262in}{1.287164in}}%
\pgfpathlineto{\pgfqpoint{3.041813in}{1.288340in}}%
\pgfpathlineto{\pgfqpoint{3.044568in}{1.286270in}}%
\pgfpathlineto{\pgfqpoint{3.047157in}{1.291801in}}%
\pgfpathlineto{\pgfqpoint{3.049988in}{1.305148in}}%
\pgfpathlineto{\pgfqpoint{3.052526in}{1.294285in}}%
\pgfpathlineto{\pgfqpoint{3.055202in}{1.287280in}}%
\pgfpathlineto{\pgfqpoint{3.057884in}{1.284970in}}%
\pgfpathlineto{\pgfqpoint{3.060561in}{1.288199in}}%
\pgfpathlineto{\pgfqpoint{3.063230in}{1.296868in}}%
\pgfpathlineto{\pgfqpoint{3.065916in}{1.292058in}}%
\pgfpathlineto{\pgfqpoint{3.068709in}{1.295700in}}%
\pgfpathlineto{\pgfqpoint{3.071266in}{1.289494in}}%
\pgfpathlineto{\pgfqpoint{3.074056in}{1.287304in}}%
\pgfpathlineto{\pgfqpoint{3.076631in}{1.284970in}}%
\pgfpathlineto{\pgfqpoint{3.079381in}{1.307645in}}%
\pgfpathlineto{\pgfqpoint{3.081990in}{1.333242in}}%
\pgfpathlineto{\pgfqpoint{3.084671in}{1.322821in}}%
\pgfpathlineto{\pgfqpoint{3.087343in}{1.312397in}}%
\pgfpathlineto{\pgfqpoint{3.090023in}{1.308336in}}%
\pgfpathlineto{\pgfqpoint{3.092699in}{1.314790in}}%
\pgfpathlineto{\pgfqpoint{3.095388in}{1.313221in}}%
\pgfpathlineto{\pgfqpoint{3.098163in}{1.318144in}}%
\pgfpathlineto{\pgfqpoint{3.100737in}{1.362082in}}%
\pgfpathlineto{\pgfqpoint{3.103508in}{1.337692in}}%
\pgfpathlineto{\pgfqpoint{3.106094in}{1.335812in}}%
\pgfpathlineto{\pgfqpoint{3.108896in}{1.370413in}}%
\pgfpathlineto{\pgfqpoint{3.111451in}{1.396214in}}%
\pgfpathlineto{\pgfqpoint{3.114242in}{1.373764in}}%
\pgfpathlineto{\pgfqpoint{3.116807in}{1.350559in}}%
\pgfpathlineto{\pgfqpoint{3.119487in}{1.351797in}}%
\pgfpathlineto{\pgfqpoint{3.122163in}{1.344192in}}%
\pgfpathlineto{\pgfqpoint{3.124842in}{1.326302in}}%
\pgfpathlineto{\pgfqpoint{3.127512in}{1.311646in}}%
\pgfpathlineto{\pgfqpoint{3.130199in}{1.305085in}}%
\pgfpathlineto{\pgfqpoint{3.132946in}{1.303126in}}%
\pgfpathlineto{\pgfqpoint{3.135550in}{1.336491in}}%
\pgfpathlineto{\pgfqpoint{3.138375in}{1.377475in}}%
\pgfpathlineto{\pgfqpoint{3.140913in}{1.409413in}}%
\pgfpathlineto{\pgfqpoint{3.143740in}{1.438012in}}%
\pgfpathlineto{\pgfqpoint{3.146271in}{1.429530in}}%
\pgfpathlineto{\pgfqpoint{3.149057in}{1.418920in}}%
\pgfpathlineto{\pgfqpoint{3.151612in}{1.419352in}}%
\pgfpathlineto{\pgfqpoint{3.154327in}{1.364297in}}%
\pgfpathlineto{\pgfqpoint{3.156981in}{1.341859in}}%
\pgfpathlineto{\pgfqpoint{3.159675in}{1.377158in}}%
\pgfpathlineto{\pgfqpoint{3.162474in}{1.377350in}}%
\pgfpathlineto{\pgfqpoint{3.165019in}{1.397850in}}%
\pgfpathlineto{\pgfqpoint{3.167776in}{1.449976in}}%
\pgfpathlineto{\pgfqpoint{3.170375in}{1.454756in}}%
\pgfpathlineto{\pgfqpoint{3.173142in}{1.424376in}}%
\pgfpathlineto{\pgfqpoint{3.175724in}{1.367919in}}%
\pgfpathlineto{\pgfqpoint{3.178525in}{1.338772in}}%
\pgfpathlineto{\pgfqpoint{3.181089in}{1.325700in}}%
\pgfpathlineto{\pgfqpoint{3.183760in}{1.317575in}}%
\pgfpathlineto{\pgfqpoint{3.186440in}{1.332582in}}%
\pgfpathlineto{\pgfqpoint{3.189117in}{1.349086in}}%
\pgfpathlineto{\pgfqpoint{3.191796in}{1.362651in}}%
\pgfpathlineto{\pgfqpoint{3.194508in}{1.342255in}}%
\pgfpathlineto{\pgfqpoint{3.197226in}{1.336963in}}%
\pgfpathlineto{\pgfqpoint{3.199823in}{1.390395in}}%
\pgfpathlineto{\pgfqpoint{3.202562in}{1.382857in}}%
\pgfpathlineto{\pgfqpoint{3.205195in}{1.343941in}}%
\pgfpathlineto{\pgfqpoint{3.207984in}{1.323909in}}%
\pgfpathlineto{\pgfqpoint{3.210545in}{1.352559in}}%
\pgfpathlineto{\pgfqpoint{3.213342in}{1.350849in}}%
\pgfpathlineto{\pgfqpoint{3.215908in}{1.350078in}}%
\pgfpathlineto{\pgfqpoint{3.218586in}{1.345716in}}%
\pgfpathlineto{\pgfqpoint{3.221255in}{1.334332in}}%
\pgfpathlineto{\pgfqpoint{3.223942in}{1.328679in}}%
\pgfpathlineto{\pgfqpoint{3.226609in}{1.331851in}}%
\pgfpathlineto{\pgfqpoint{3.229310in}{1.323247in}}%
\pgfpathlineto{\pgfqpoint{3.232069in}{1.331612in}}%
\pgfpathlineto{\pgfqpoint{3.234658in}{1.336920in}}%
\pgfpathlineto{\pgfqpoint{3.237411in}{1.342742in}}%
\pgfpathlineto{\pgfqpoint{3.240010in}{1.339377in}}%
\pgfpathlineto{\pgfqpoint{3.242807in}{1.329376in}}%
\pgfpathlineto{\pgfqpoint{3.245363in}{1.323203in}}%
\pgfpathlineto{\pgfqpoint{3.248049in}{1.315469in}}%
\pgfpathlineto{\pgfqpoint{3.250716in}{1.316228in}}%
\pgfpathlineto{\pgfqpoint{3.253404in}{1.307607in}}%
\pgfpathlineto{\pgfqpoint{3.256083in}{1.303049in}}%
\pgfpathlineto{\pgfqpoint{3.258784in}{1.299524in}}%
\pgfpathlineto{\pgfqpoint{3.261594in}{1.295981in}}%
\pgfpathlineto{\pgfqpoint{3.264119in}{1.296159in}}%
\pgfpathlineto{\pgfqpoint{3.266849in}{1.291496in}}%
\pgfpathlineto{\pgfqpoint{3.269478in}{1.289355in}}%
\pgfpathlineto{\pgfqpoint{3.272254in}{1.289480in}}%
\pgfpathlineto{\pgfqpoint{3.274831in}{1.289457in}}%
\pgfpathlineto{\pgfqpoint{3.277603in}{1.290664in}}%
\pgfpathlineto{\pgfqpoint{3.280189in}{1.287874in}}%
\pgfpathlineto{\pgfqpoint{3.282870in}{1.289466in}}%
\pgfpathlineto{\pgfqpoint{3.285534in}{1.289235in}}%
\pgfpathlineto{\pgfqpoint{3.288225in}{1.290936in}}%
\pgfpathlineto{\pgfqpoint{3.290890in}{1.284970in}}%
\pgfpathlineto{\pgfqpoint{3.293574in}{1.288858in}}%
\pgfpathlineto{\pgfqpoint{3.296376in}{1.286871in}}%
\pgfpathlineto{\pgfqpoint{3.298937in}{1.285472in}}%
\pgfpathlineto{\pgfqpoint{3.301719in}{1.287242in}}%
\pgfpathlineto{\pgfqpoint{3.304295in}{1.287936in}}%
\pgfpathlineto{\pgfqpoint{3.307104in}{1.284970in}}%
\pgfpathlineto{\pgfqpoint{3.309652in}{1.285752in}}%
\pgfpathlineto{\pgfqpoint{3.312480in}{1.285263in}}%
\pgfpathlineto{\pgfqpoint{3.315008in}{1.285134in}}%
\pgfpathlineto{\pgfqpoint{3.317688in}{1.285427in}}%
\pgfpathlineto{\pgfqpoint{3.320366in}{1.284970in}}%
\pgfpathlineto{\pgfqpoint{3.323049in}{1.284970in}}%
\pgfpathlineto{\pgfqpoint{3.325860in}{1.284970in}}%
\pgfpathlineto{\pgfqpoint{3.328401in}{1.284970in}}%
\pgfpathlineto{\pgfqpoint{3.331183in}{1.284970in}}%
\pgfpathlineto{\pgfqpoint{3.333758in}{1.286446in}}%
\pgfpathlineto{\pgfqpoint{3.336541in}{1.284970in}}%
\pgfpathlineto{\pgfqpoint{3.339101in}{1.284970in}}%
\pgfpathlineto{\pgfqpoint{3.341893in}{1.284970in}}%
\pgfpathlineto{\pgfqpoint{3.344468in}{1.284970in}}%
\pgfpathlineto{\pgfqpoint{3.347139in}{1.284970in}}%
\pgfpathlineto{\pgfqpoint{3.349828in}{1.286230in}}%
\pgfpathlineto{\pgfqpoint{3.352505in}{1.284970in}}%
\pgfpathlineto{\pgfqpoint{3.355177in}{1.284970in}}%
\pgfpathlineto{\pgfqpoint{3.357862in}{1.284970in}}%
\pgfpathlineto{\pgfqpoint{3.360620in}{1.284970in}}%
\pgfpathlineto{\pgfqpoint{3.363221in}{1.288440in}}%
\pgfpathlineto{\pgfqpoint{3.365996in}{1.285761in}}%
\pgfpathlineto{\pgfqpoint{3.368577in}{1.285105in}}%
\pgfpathlineto{\pgfqpoint{3.371357in}{1.286791in}}%
\pgfpathlineto{\pgfqpoint{3.373921in}{1.286962in}}%
\pgfpathlineto{\pgfqpoint{3.376735in}{1.287745in}}%
\pgfpathlineto{\pgfqpoint{3.379290in}{1.287713in}}%
\pgfpathlineto{\pgfqpoint{3.381959in}{1.287249in}}%
\pgfpathlineto{\pgfqpoint{3.384647in}{1.286951in}}%
\pgfpathlineto{\pgfqpoint{3.387309in}{1.287074in}}%
\pgfpathlineto{\pgfqpoint{3.390102in}{1.289104in}}%
\pgfpathlineto{\pgfqpoint{3.392681in}{1.287421in}}%
\pgfpathlineto{\pgfqpoint{3.395461in}{1.284970in}}%
\pgfpathlineto{\pgfqpoint{3.398037in}{1.284970in}}%
\pgfpathlineto{\pgfqpoint{3.400783in}{1.284970in}}%
\pgfpathlineto{\pgfqpoint{3.403394in}{1.288392in}}%
\pgfpathlineto{\pgfqpoint{3.406202in}{1.284970in}}%
\pgfpathlineto{\pgfqpoint{3.408752in}{1.288094in}}%
\pgfpathlineto{\pgfqpoint{3.411431in}{1.285888in}}%
\pgfpathlineto{\pgfqpoint{3.414109in}{1.284970in}}%
\pgfpathlineto{\pgfqpoint{3.416780in}{1.289518in}}%
\pgfpathlineto{\pgfqpoint{3.419455in}{1.285865in}}%
\pgfpathlineto{\pgfqpoint{3.422142in}{1.286874in}}%
\pgfpathlineto{\pgfqpoint{3.424887in}{1.286439in}}%
\pgfpathlineto{\pgfqpoint{3.427501in}{1.290172in}}%
\pgfpathlineto{\pgfqpoint{3.430313in}{1.286831in}}%
\pgfpathlineto{\pgfqpoint{3.432851in}{1.287810in}}%
\pgfpathlineto{\pgfqpoint{3.435635in}{1.291046in}}%
\pgfpathlineto{\pgfqpoint{3.438210in}{1.289646in}}%
\pgfpathlineto{\pgfqpoint{3.440996in}{1.289715in}}%
\pgfpathlineto{\pgfqpoint{3.443574in}{1.290593in}}%
\pgfpathlineto{\pgfqpoint{3.446257in}{1.286847in}}%
\pgfpathlineto{\pgfqpoint{3.448926in}{1.287984in}}%
\pgfpathlineto{\pgfqpoint{3.451597in}{1.290851in}}%
\pgfpathlineto{\pgfqpoint{3.454285in}{1.287933in}}%
\pgfpathlineto{\pgfqpoint{3.456960in}{1.288067in}}%
\pgfpathlineto{\pgfqpoint{3.459695in}{1.286948in}}%
\pgfpathlineto{\pgfqpoint{3.462321in}{1.289785in}}%
\pgfpathlineto{\pgfqpoint{3.465072in}{1.287233in}}%
\pgfpathlineto{\pgfqpoint{3.467678in}{1.286777in}}%
\pgfpathlineto{\pgfqpoint{3.470466in}{1.287955in}}%
\pgfpathlineto{\pgfqpoint{3.473021in}{1.291593in}}%
\pgfpathlineto{\pgfqpoint{3.475821in}{1.289292in}}%
\pgfpathlineto{\pgfqpoint{3.478378in}{1.287874in}}%
\pgfpathlineto{\pgfqpoint{3.481072in}{1.288600in}}%
\pgfpathlineto{\pgfqpoint{3.483744in}{1.287810in}}%
\pgfpathlineto{\pgfqpoint{3.486442in}{1.290494in}}%
\pgfpathlineto{\pgfqpoint{3.489223in}{1.291371in}}%
\pgfpathlineto{\pgfqpoint{3.491783in}{1.286949in}}%
\pgfpathlineto{\pgfqpoint{3.494581in}{1.290626in}}%
\pgfpathlineto{\pgfqpoint{3.497139in}{1.286675in}}%
\pgfpathlineto{\pgfqpoint{3.499909in}{1.286355in}}%
\pgfpathlineto{\pgfqpoint{3.502488in}{1.284970in}}%
\pgfpathlineto{\pgfqpoint{3.505262in}{1.288791in}}%
\pgfpathlineto{\pgfqpoint{3.507840in}{1.288614in}}%
\pgfpathlineto{\pgfqpoint{3.510533in}{1.287782in}}%
\pgfpathlineto{\pgfqpoint{3.513209in}{1.286480in}}%
\pgfpathlineto{\pgfqpoint{3.515884in}{1.287928in}}%
\pgfpathlineto{\pgfqpoint{3.518565in}{1.285044in}}%
\pgfpathlineto{\pgfqpoint{3.521244in}{1.287367in}}%
\pgfpathlineto{\pgfqpoint{3.524041in}{1.290525in}}%
\pgfpathlineto{\pgfqpoint{3.526601in}{1.294351in}}%
\pgfpathlineto{\pgfqpoint{3.529327in}{1.293793in}}%
\pgfpathlineto{\pgfqpoint{3.531955in}{1.301556in}}%
\pgfpathlineto{\pgfqpoint{3.534783in}{1.297397in}}%
\pgfpathlineto{\pgfqpoint{3.537309in}{1.291463in}}%
\pgfpathlineto{\pgfqpoint{3.540093in}{1.289510in}}%
\pgfpathlineto{\pgfqpoint{3.542656in}{1.288593in}}%
\pgfpathlineto{\pgfqpoint{3.545349in}{1.290624in}}%
\pgfpathlineto{\pgfqpoint{3.548029in}{1.291743in}}%
\pgfpathlineto{\pgfqpoint{3.550713in}{1.293878in}}%
\pgfpathlineto{\pgfqpoint{3.553498in}{1.292702in}}%
\pgfpathlineto{\pgfqpoint{3.556061in}{1.291255in}}%
\pgfpathlineto{\pgfqpoint{3.558853in}{1.289357in}}%
\pgfpathlineto{\pgfqpoint{3.561420in}{1.291827in}}%
\pgfpathlineto{\pgfqpoint{3.564188in}{1.290681in}}%
\pgfpathlineto{\pgfqpoint{3.566774in}{1.289495in}}%
\pgfpathlineto{\pgfqpoint{3.569584in}{1.286791in}}%
\pgfpathlineto{\pgfqpoint{3.572126in}{1.289199in}}%
\pgfpathlineto{\pgfqpoint{3.574814in}{1.290258in}}%
\pgfpathlineto{\pgfqpoint{3.577487in}{1.290131in}}%
\pgfpathlineto{\pgfqpoint{3.580191in}{1.286842in}}%
\pgfpathlineto{\pgfqpoint{3.582851in}{1.286396in}}%
\pgfpathlineto{\pgfqpoint{3.585532in}{1.286147in}}%
\pgfpathlineto{\pgfqpoint{3.588258in}{1.287742in}}%
\pgfpathlineto{\pgfqpoint{3.590883in}{1.288819in}}%
\pgfpathlineto{\pgfqpoint{3.593620in}{1.289352in}}%
\pgfpathlineto{\pgfqpoint{3.596240in}{1.288814in}}%
\pgfpathlineto{\pgfqpoint{3.598998in}{1.291441in}}%
\pgfpathlineto{\pgfqpoint{3.601590in}{1.293429in}}%
\pgfpathlineto{\pgfqpoint{3.604387in}{1.290459in}}%
\pgfpathlineto{\pgfqpoint{3.606951in}{1.290895in}}%
\pgfpathlineto{\pgfqpoint{3.609632in}{1.287379in}}%
\pgfpathlineto{\pgfqpoint{3.612311in}{1.285711in}}%
\pgfpathlineto{\pgfqpoint{3.614982in}{1.289393in}}%
\pgfpathlineto{\pgfqpoint{3.617667in}{1.289962in}}%
\pgfpathlineto{\pgfqpoint{3.620345in}{1.288570in}}%
\pgfpathlineto{\pgfqpoint{3.623165in}{1.288170in}}%
\pgfpathlineto{\pgfqpoint{3.625689in}{1.289552in}}%
\pgfpathlineto{\pgfqpoint{3.628460in}{1.290790in}}%
\pgfpathlineto{\pgfqpoint{3.631058in}{1.290147in}}%
\pgfpathlineto{\pgfqpoint{3.633858in}{1.286228in}}%
\pgfpathlineto{\pgfqpoint{3.636413in}{1.287900in}}%
\pgfpathlineto{\pgfqpoint{3.639207in}{1.290151in}}%
\pgfpathlineto{\pgfqpoint{3.641773in}{1.288053in}}%
\pgfpathlineto{\pgfqpoint{3.644452in}{1.284970in}}%
\pgfpathlineto{\pgfqpoint{3.647130in}{1.285136in}}%
\pgfpathlineto{\pgfqpoint{3.649837in}{1.284970in}}%
\pgfpathlineto{\pgfqpoint{3.652628in}{1.286061in}}%
\pgfpathlineto{\pgfqpoint{3.655165in}{1.286411in}}%
\pgfpathlineto{\pgfqpoint{3.657917in}{1.289592in}}%
\pgfpathlineto{\pgfqpoint{3.660515in}{1.290274in}}%
\pgfpathlineto{\pgfqpoint{3.663276in}{1.284970in}}%
\pgfpathlineto{\pgfqpoint{3.665864in}{1.284970in}}%
\pgfpathlineto{\pgfqpoint{3.668665in}{1.285049in}}%
\pgfpathlineto{\pgfqpoint{3.671232in}{1.286329in}}%
\pgfpathlineto{\pgfqpoint{3.673911in}{1.285584in}}%
\pgfpathlineto{\pgfqpoint{3.676591in}{1.284970in}}%
\pgfpathlineto{\pgfqpoint{3.679273in}{1.284970in}}%
\pgfpathlineto{\pgfqpoint{3.681948in}{1.284970in}}%
\pgfpathlineto{\pgfqpoint{3.684620in}{1.284970in}}%
\pgfpathlineto{\pgfqpoint{3.687442in}{1.284970in}}%
\pgfpathlineto{\pgfqpoint{3.689983in}{1.284970in}}%
\pgfpathlineto{\pgfqpoint{3.692765in}{1.284970in}}%
\pgfpathlineto{\pgfqpoint{3.695331in}{1.284970in}}%
\pgfpathlineto{\pgfqpoint{3.698125in}{1.286736in}}%
\pgfpathlineto{\pgfqpoint{3.700684in}{1.284970in}}%
\pgfpathlineto{\pgfqpoint{3.703460in}{1.291129in}}%
\pgfpathlineto{\pgfqpoint{3.706053in}{1.297847in}}%
\pgfpathlineto{\pgfqpoint{3.708729in}{1.306509in}}%
\pgfpathlineto{\pgfqpoint{3.711410in}{1.308467in}}%
\pgfpathlineto{\pgfqpoint{3.714086in}{1.303401in}}%
\pgfpathlineto{\pgfqpoint{3.716875in}{1.296303in}}%
\pgfpathlineto{\pgfqpoint{3.719446in}{1.295394in}}%
\pgfpathlineto{\pgfqpoint{3.722228in}{1.292660in}}%
\pgfpathlineto{\pgfqpoint{3.724804in}{1.295185in}}%
\pgfpathlineto{\pgfqpoint{3.727581in}{1.288141in}}%
\pgfpathlineto{\pgfqpoint{3.730158in}{1.289433in}}%
\pgfpathlineto{\pgfqpoint{3.732950in}{1.289239in}}%
\pgfpathlineto{\pgfqpoint{3.735509in}{1.300059in}}%
\pgfpathlineto{\pgfqpoint{3.738194in}{1.299655in}}%
\pgfpathlineto{\pgfqpoint{3.740874in}{1.292634in}}%
\pgfpathlineto{\pgfqpoint{3.743548in}{1.292362in}}%
\pgfpathlineto{\pgfqpoint{3.746229in}{1.292713in}}%
\pgfpathlineto{\pgfqpoint{3.748903in}{1.290338in}}%
\pgfpathlineto{\pgfqpoint{3.751728in}{1.296617in}}%
\pgfpathlineto{\pgfqpoint{3.754265in}{1.291086in}}%
\pgfpathlineto{\pgfqpoint{3.757065in}{1.290254in}}%
\pgfpathlineto{\pgfqpoint{3.759622in}{1.288280in}}%
\pgfpathlineto{\pgfqpoint{3.762389in}{1.294130in}}%
\pgfpathlineto{\pgfqpoint{3.764966in}{1.340706in}}%
\pgfpathlineto{\pgfqpoint{3.767782in}{1.370754in}}%
\pgfpathlineto{\pgfqpoint{3.770323in}{1.350585in}}%
\pgfpathlineto{\pgfqpoint{3.773014in}{1.324877in}}%
\pgfpathlineto{\pgfqpoint{3.775691in}{1.297720in}}%
\pgfpathlineto{\pgfqpoint{3.778370in}{1.306722in}}%
\pgfpathlineto{\pgfqpoint{3.781046in}{1.363962in}}%
\pgfpathlineto{\pgfqpoint{3.783725in}{1.378017in}}%
\pgfpathlineto{\pgfqpoint{3.786504in}{1.367134in}}%
\pgfpathlineto{\pgfqpoint{3.789084in}{1.355551in}}%
\pgfpathlineto{\pgfqpoint{3.791897in}{1.348361in}}%
\pgfpathlineto{\pgfqpoint{3.794435in}{1.343739in}}%
\pgfpathlineto{\pgfqpoint{3.797265in}{1.335801in}}%
\pgfpathlineto{\pgfqpoint{3.799797in}{1.333311in}}%
\pgfpathlineto{\pgfqpoint{3.802569in}{1.329569in}}%
\pgfpathlineto{\pgfqpoint{3.805145in}{1.315670in}}%
\pgfpathlineto{\pgfqpoint{3.807832in}{1.309803in}}%
\pgfpathlineto{\pgfqpoint{3.810510in}{1.303248in}}%
\pgfpathlineto{\pgfqpoint{3.813172in}{1.305603in}}%
\pgfpathlineto{\pgfqpoint{3.815983in}{1.318604in}}%
\pgfpathlineto{\pgfqpoint{3.818546in}{1.329951in}}%
\pgfpathlineto{\pgfqpoint{3.821315in}{1.325733in}}%
\pgfpathlineto{\pgfqpoint{3.823903in}{1.331599in}}%
\pgfpathlineto{\pgfqpoint{3.826679in}{1.324598in}}%
\pgfpathlineto{\pgfqpoint{3.829252in}{1.320822in}}%
\pgfpathlineto{\pgfqpoint{3.832053in}{1.322015in}}%
\pgfpathlineto{\pgfqpoint{3.834616in}{1.320360in}}%
\pgfpathlineto{\pgfqpoint{3.837286in}{1.315204in}}%
\pgfpathlineto{\pgfqpoint{3.839960in}{1.314277in}}%
\pgfpathlineto{\pgfqpoint{3.842641in}{1.328089in}}%
\pgfpathlineto{\pgfqpoint{3.845329in}{1.319415in}}%
\pgfpathlineto{\pgfqpoint{3.848005in}{1.320070in}}%
\pgfpathlineto{\pgfqpoint{3.850814in}{1.322235in}}%
\pgfpathlineto{\pgfqpoint{3.853358in}{1.327104in}}%
\pgfpathlineto{\pgfqpoint{3.856100in}{1.357863in}}%
\pgfpathlineto{\pgfqpoint{3.858720in}{1.379227in}}%
\pgfpathlineto{\pgfqpoint{3.861561in}{1.382828in}}%
\pgfpathlineto{\pgfqpoint{3.864073in}{1.384829in}}%
\pgfpathlineto{\pgfqpoint{3.866815in}{1.372285in}}%
\pgfpathlineto{\pgfqpoint{3.869435in}{1.365870in}}%
\pgfpathlineto{\pgfqpoint{3.872114in}{1.357146in}}%
\pgfpathlineto{\pgfqpoint{3.874790in}{1.345253in}}%
\pgfpathlineto{\pgfqpoint{3.877466in}{1.330441in}}%
\pgfpathlineto{\pgfqpoint{3.880237in}{1.327891in}}%
\pgfpathlineto{\pgfqpoint{3.882850in}{1.318934in}}%
\pgfpathlineto{\pgfqpoint{3.885621in}{1.318378in}}%
\pgfpathlineto{\pgfqpoint{3.888188in}{1.313590in}}%
\pgfpathlineto{\pgfqpoint{3.890926in}{1.315114in}}%
\pgfpathlineto{\pgfqpoint{3.893541in}{1.322598in}}%
\pgfpathlineto{\pgfqpoint{3.896345in}{1.300868in}}%
\pgfpathlineto{\pgfqpoint{3.898891in}{1.300745in}}%
\pgfpathlineto{\pgfqpoint{3.901573in}{1.301876in}}%
\pgfpathlineto{\pgfqpoint{3.904252in}{1.296497in}}%
\pgfpathlineto{\pgfqpoint{3.906918in}{1.292952in}}%
\pgfpathlineto{\pgfqpoint{3.909602in}{1.290274in}}%
\pgfpathlineto{\pgfqpoint{3.912296in}{1.292098in}}%
\pgfpathlineto{\pgfqpoint{3.915107in}{1.297768in}}%
\pgfpathlineto{\pgfqpoint{3.917646in}{1.297719in}}%
\pgfpathlineto{\pgfqpoint{3.920412in}{1.304237in}}%
\pgfpathlineto{\pgfqpoint{3.923005in}{1.296540in}}%
\pgfpathlineto{\pgfqpoint{3.925778in}{1.297412in}}%
\pgfpathlineto{\pgfqpoint{3.928347in}{1.296223in}}%
\pgfpathlineto{\pgfqpoint{3.931202in}{1.294632in}}%
\pgfpathlineto{\pgfqpoint{3.933714in}{1.293806in}}%
\pgfpathlineto{\pgfqpoint{3.936395in}{1.291794in}}%
\pgfpathlineto{\pgfqpoint{3.939075in}{1.288818in}}%
\pgfpathlineto{\pgfqpoint{3.941778in}{1.284970in}}%
\pgfpathlineto{\pgfqpoint{3.944431in}{1.287448in}}%
\pgfpathlineto{\pgfqpoint{3.947101in}{1.288132in}}%
\pgfpathlineto{\pgfqpoint{3.949894in}{1.285453in}}%
\pgfpathlineto{\pgfqpoint{3.952464in}{1.286821in}}%
\pgfpathlineto{\pgfqpoint{3.955211in}{1.291606in}}%
\pgfpathlineto{\pgfqpoint{3.957823in}{1.292428in}}%
\pgfpathlineto{\pgfqpoint{3.960635in}{1.291469in}}%
\pgfpathlineto{\pgfqpoint{3.963176in}{1.289604in}}%
\pgfpathlineto{\pgfqpoint{3.966013in}{1.291155in}}%
\pgfpathlineto{\pgfqpoint{3.968523in}{1.288531in}}%
\pgfpathlineto{\pgfqpoint{3.971250in}{1.285760in}}%
\pgfpathlineto{\pgfqpoint{3.973885in}{1.284970in}}%
\pgfpathlineto{\pgfqpoint{3.976563in}{1.284970in}}%
\pgfpathlineto{\pgfqpoint{3.979389in}{1.289118in}}%
\pgfpathlineto{\pgfqpoint{3.981929in}{1.286220in}}%
\pgfpathlineto{\pgfqpoint{3.984714in}{1.289177in}}%
\pgfpathlineto{\pgfqpoint{3.987270in}{1.290292in}}%
\pgfpathlineto{\pgfqpoint{3.990055in}{1.288014in}}%
\pgfpathlineto{\pgfqpoint{3.992642in}{1.286814in}}%
\pgfpathlineto{\pgfqpoint{3.995417in}{1.288761in}}%
\pgfpathlineto{\pgfqpoint{3.997990in}{1.291367in}}%
\pgfpathlineto{\pgfqpoint{4.000674in}{1.286870in}}%
\pgfpathlineto{\pgfqpoint{4.003348in}{1.289427in}}%
\pgfpathlineto{\pgfqpoint{4.006034in}{1.286982in}}%
\pgfpathlineto{\pgfqpoint{4.008699in}{1.287421in}}%
\pgfpathlineto{\pgfqpoint{4.011394in}{1.285592in}}%
\pgfpathlineto{\pgfqpoint{4.014186in}{1.287567in}}%
\pgfpathlineto{\pgfqpoint{4.016744in}{1.288743in}}%
\pgfpathlineto{\pgfqpoint{4.019518in}{1.291596in}}%
\pgfpathlineto{\pgfqpoint{4.022097in}{1.286858in}}%
\pgfpathlineto{\pgfqpoint{4.024868in}{1.291537in}}%
\pgfpathlineto{\pgfqpoint{4.027447in}{1.289385in}}%
\pgfpathlineto{\pgfqpoint{4.030229in}{1.290533in}}%
\pgfpathlineto{\pgfqpoint{4.032817in}{1.289552in}}%
\pgfpathlineto{\pgfqpoint{4.035492in}{1.289239in}}%
\pgfpathlineto{\pgfqpoint{4.038174in}{1.289707in}}%
\pgfpathlineto{\pgfqpoint{4.040852in}{1.287321in}}%
\pgfpathlineto{\pgfqpoint{4.043667in}{1.288576in}}%
\pgfpathlineto{\pgfqpoint{4.046210in}{1.287473in}}%
\pgfpathlineto{\pgfqpoint{4.049006in}{1.289102in}}%
\pgfpathlineto{\pgfqpoint{4.051557in}{1.298966in}}%
\pgfpathlineto{\pgfqpoint{4.054326in}{1.294278in}}%
\pgfpathlineto{\pgfqpoint{4.056911in}{1.289514in}}%
\pgfpathlineto{\pgfqpoint{4.059702in}{1.287784in}}%
\pgfpathlineto{\pgfqpoint{4.062266in}{1.295976in}}%
\pgfpathlineto{\pgfqpoint{4.064957in}{1.293578in}}%
\pgfpathlineto{\pgfqpoint{4.067636in}{1.293742in}}%
\pgfpathlineto{\pgfqpoint{4.070313in}{1.294159in}}%
\pgfpathlineto{\pgfqpoint{4.072985in}{1.294174in}}%
\pgfpathlineto{\pgfqpoint{4.075705in}{1.295497in}}%
\pgfpathlineto{\pgfqpoint{4.078471in}{1.293128in}}%
\pgfpathlineto{\pgfqpoint{4.081018in}{1.290810in}}%
\pgfpathlineto{\pgfqpoint{4.083870in}{1.290246in}}%
\pgfpathlineto{\pgfqpoint{4.086385in}{1.292168in}}%
\pgfpathlineto{\pgfqpoint{4.089159in}{1.294088in}}%
\pgfpathlineto{\pgfqpoint{4.091729in}{1.291804in}}%
\pgfpathlineto{\pgfqpoint{4.094527in}{1.288885in}}%
\pgfpathlineto{\pgfqpoint{4.097092in}{1.297025in}}%
\pgfpathlineto{\pgfqpoint{4.099777in}{1.293211in}}%
\pgfpathlineto{\pgfqpoint{4.102456in}{1.294220in}}%
\pgfpathlineto{\pgfqpoint{4.105185in}{1.290996in}}%
\pgfpathlineto{\pgfqpoint{4.107814in}{1.293623in}}%
\pgfpathlineto{\pgfqpoint{4.110488in}{1.293988in}}%
\pgfpathlineto{\pgfqpoint{4.113252in}{1.294698in}}%
\pgfpathlineto{\pgfqpoint{4.115844in}{1.292073in}}%
\pgfpathlineto{\pgfqpoint{4.118554in}{1.290778in}}%
\pgfpathlineto{\pgfqpoint{4.121205in}{1.288006in}}%
\pgfpathlineto{\pgfqpoint{4.124019in}{1.286737in}}%
\pgfpathlineto{\pgfqpoint{4.126553in}{1.284970in}}%
\pgfpathlineto{\pgfqpoint{4.129349in}{1.285452in}}%
\pgfpathlineto{\pgfqpoint{4.131920in}{1.291229in}}%
\pgfpathlineto{\pgfqpoint{4.134615in}{1.288485in}}%
\pgfpathlineto{\pgfqpoint{4.137272in}{1.285779in}}%
\pgfpathlineto{\pgfqpoint{4.139963in}{1.285167in}}%
\pgfpathlineto{\pgfqpoint{4.142713in}{1.286218in}}%
\pgfpathlineto{\pgfqpoint{4.145310in}{1.289764in}}%
\pgfpathlineto{\pgfqpoint{4.148082in}{1.287141in}}%
\pgfpathlineto{\pgfqpoint{4.150665in}{1.284970in}}%
\pgfpathlineto{\pgfqpoint{4.153423in}{1.288955in}}%
\pgfpathlineto{\pgfqpoint{4.156016in}{1.287562in}}%
\pgfpathlineto{\pgfqpoint{4.158806in}{1.284970in}}%
\pgfpathlineto{\pgfqpoint{4.161380in}{1.286557in}}%
\pgfpathlineto{\pgfqpoint{4.164059in}{1.287143in}}%
\pgfpathlineto{\pgfqpoint{4.166737in}{1.284970in}}%
\pgfpathlineto{\pgfqpoint{4.169415in}{1.284970in}}%
\pgfpathlineto{\pgfqpoint{4.172093in}{1.284970in}}%
\pgfpathlineto{\pgfqpoint{4.174770in}{1.284970in}}%
\pgfpathlineto{\pgfqpoint{4.177593in}{1.285375in}}%
\pgfpathlineto{\pgfqpoint{4.180129in}{1.285632in}}%
\pgfpathlineto{\pgfqpoint{4.182899in}{1.286079in}}%
\pgfpathlineto{\pgfqpoint{4.185481in}{1.287587in}}%
\pgfpathlineto{\pgfqpoint{4.188318in}{1.289049in}}%
\pgfpathlineto{\pgfqpoint{4.190842in}{1.289618in}}%
\pgfpathlineto{\pgfqpoint{4.193638in}{1.293285in}}%
\pgfpathlineto{\pgfqpoint{4.196186in}{1.291986in}}%
\pgfpathlineto{\pgfqpoint{4.198878in}{1.291444in}}%
\pgfpathlineto{\pgfqpoint{4.201542in}{1.293358in}}%
\pgfpathlineto{\pgfqpoint{4.204240in}{1.295883in}}%
\pgfpathlineto{\pgfqpoint{4.207076in}{1.290146in}}%
\pgfpathlineto{\pgfqpoint{4.209597in}{1.294153in}}%
\pgfpathlineto{\pgfqpoint{4.212383in}{1.292033in}}%
\pgfpathlineto{\pgfqpoint{4.214948in}{1.288807in}}%
\pgfpathlineto{\pgfqpoint{4.217694in}{1.290774in}}%
\pgfpathlineto{\pgfqpoint{4.220304in}{1.288372in}}%
\pgfpathlineto{\pgfqpoint{4.223082in}{1.293505in}}%
\pgfpathlineto{\pgfqpoint{4.225654in}{1.288949in}}%
\pgfpathlineto{\pgfqpoint{4.228331in}{1.289708in}}%
\pgfpathlineto{\pgfqpoint{4.231013in}{1.288841in}}%
\pgfpathlineto{\pgfqpoint{4.233691in}{1.290299in}}%
\pgfpathlineto{\pgfqpoint{4.236375in}{1.289967in}}%
\pgfpathlineto{\pgfqpoint{4.239084in}{1.290615in}}%
\pgfpathlineto{\pgfqpoint{4.241900in}{1.291004in}}%
\pgfpathlineto{\pgfqpoint{4.244394in}{1.295341in}}%
\pgfpathlineto{\pgfqpoint{4.247225in}{1.293008in}}%
\pgfpathlineto{\pgfqpoint{4.249767in}{1.291003in}}%
\pgfpathlineto{\pgfqpoint{4.252581in}{1.292209in}}%
\pgfpathlineto{\pgfqpoint{4.255120in}{1.291690in}}%
\pgfpathlineto{\pgfqpoint{4.257958in}{1.294407in}}%
\pgfpathlineto{\pgfqpoint{4.260477in}{1.293489in}}%
\pgfpathlineto{\pgfqpoint{4.263157in}{1.293741in}}%
\pgfpathlineto{\pgfqpoint{4.265824in}{1.294543in}}%
\pgfpathlineto{\pgfqpoint{4.268590in}{1.294930in}}%
\pgfpathlineto{\pgfqpoint{4.271187in}{1.296107in}}%
\pgfpathlineto{\pgfqpoint{4.273874in}{1.291943in}}%
\pgfpathlineto{\pgfqpoint{4.276635in}{1.294625in}}%
\pgfpathlineto{\pgfqpoint{4.279212in}{1.292472in}}%
\pgfpathlineto{\pgfqpoint{4.282000in}{1.288119in}}%
\pgfpathlineto{\pgfqpoint{4.284586in}{1.286273in}}%
\pgfpathlineto{\pgfqpoint{4.287399in}{1.290303in}}%
\pgfpathlineto{\pgfqpoint{4.289936in}{1.288494in}}%
\pgfpathlineto{\pgfqpoint{4.292786in}{1.292590in}}%
\pgfpathlineto{\pgfqpoint{4.295299in}{1.290438in}}%
\pgfpathlineto{\pgfqpoint{4.297977in}{1.287506in}}%
\pgfpathlineto{\pgfqpoint{4.300656in}{1.289168in}}%
\pgfpathlineto{\pgfqpoint{4.303357in}{1.285137in}}%
\pgfpathlineto{\pgfqpoint{4.306118in}{1.290952in}}%
\pgfpathlineto{\pgfqpoint{4.308691in}{1.289162in}}%
\pgfpathlineto{\pgfqpoint{4.311494in}{1.289780in}}%
\pgfpathlineto{\pgfqpoint{4.314032in}{1.305288in}}%
\pgfpathlineto{\pgfqpoint{4.316856in}{1.294564in}}%
\pgfpathlineto{\pgfqpoint{4.319405in}{1.292689in}}%
\pgfpathlineto{\pgfqpoint{4.322181in}{1.290136in}}%
\pgfpathlineto{\pgfqpoint{4.324760in}{1.290940in}}%
\pgfpathlineto{\pgfqpoint{4.327440in}{1.294196in}}%
\pgfpathlineto{\pgfqpoint{4.330118in}{1.293741in}}%
\pgfpathlineto{\pgfqpoint{4.332796in}{1.296045in}}%
\pgfpathlineto{\pgfqpoint{4.335463in}{1.289525in}}%
\pgfpathlineto{\pgfqpoint{4.338154in}{1.290369in}}%
\pgfpathlineto{\pgfqpoint{4.340976in}{1.288193in}}%
\pgfpathlineto{\pgfqpoint{4.343510in}{1.288586in}}%
\pgfpathlineto{\pgfqpoint{4.346263in}{1.287902in}}%
\pgfpathlineto{\pgfqpoint{4.348868in}{1.288241in}}%
\pgfpathlineto{\pgfqpoint{4.351645in}{1.287412in}}%
\pgfpathlineto{\pgfqpoint{4.354224in}{1.288129in}}%
\pgfpathlineto{\pgfqpoint{4.357014in}{1.288151in}}%
\pgfpathlineto{\pgfqpoint{4.359582in}{1.287442in}}%
\pgfpathlineto{\pgfqpoint{4.362270in}{1.286773in}}%
\pgfpathlineto{\pgfqpoint{4.364936in}{1.291046in}}%
\pgfpathlineto{\pgfqpoint{4.367646in}{1.287618in}}%
\pgfpathlineto{\pgfqpoint{4.370437in}{1.285421in}}%
\pgfpathlineto{\pgfqpoint{4.372976in}{1.285283in}}%
\pgfpathlineto{\pgfqpoint{4.375761in}{1.287453in}}%
\pgfpathlineto{\pgfqpoint{4.378329in}{1.286106in}}%
\pgfpathlineto{\pgfqpoint{4.381097in}{1.286464in}}%
\pgfpathlineto{\pgfqpoint{4.383674in}{1.288660in}}%
\pgfpathlineto{\pgfqpoint{4.386431in}{1.290487in}}%
\pgfpathlineto{\pgfqpoint{4.389044in}{1.293419in}}%
\pgfpathlineto{\pgfqpoint{4.391721in}{1.295154in}}%
\pgfpathlineto{\pgfqpoint{4.394400in}{1.298476in}}%
\pgfpathlineto{\pgfqpoint{4.397076in}{1.312399in}}%
\pgfpathlineto{\pgfqpoint{4.399745in}{1.311987in}}%
\pgfpathlineto{\pgfqpoint{4.402468in}{1.295651in}}%
\pgfpathlineto{\pgfqpoint{4.405234in}{1.295583in}}%
\pgfpathlineto{\pgfqpoint{4.407788in}{1.289348in}}%
\pgfpathlineto{\pgfqpoint{4.410587in}{1.292469in}}%
\pgfpathlineto{\pgfqpoint{4.413149in}{1.291255in}}%
\pgfpathlineto{\pgfqpoint{4.415932in}{1.290551in}}%
\pgfpathlineto{\pgfqpoint{4.418506in}{1.289613in}}%
\pgfpathlineto{\pgfqpoint{4.421292in}{1.291807in}}%
\pgfpathlineto{\pgfqpoint{4.423863in}{1.289433in}}%
\pgfpathlineto{\pgfqpoint{4.426534in}{1.290186in}}%
\pgfpathlineto{\pgfqpoint{4.429220in}{1.292747in}}%
\pgfpathlineto{\pgfqpoint{4.431901in}{1.290059in}}%
\pgfpathlineto{\pgfqpoint{4.434569in}{1.291837in}}%
\pgfpathlineto{\pgfqpoint{4.437253in}{1.292789in}}%
\pgfpathlineto{\pgfqpoint{4.440041in}{1.292763in}}%
\pgfpathlineto{\pgfqpoint{4.442611in}{1.289421in}}%
\pgfpathlineto{\pgfqpoint{4.445423in}{1.293810in}}%
\pgfpathlineto{\pgfqpoint{4.447965in}{1.293571in}}%
\pgfpathlineto{\pgfqpoint{4.450767in}{1.292525in}}%
\pgfpathlineto{\pgfqpoint{4.453312in}{1.294826in}}%
\pgfpathlineto{\pgfqpoint{4.456138in}{1.294003in}}%
\pgfpathlineto{\pgfqpoint{4.458681in}{1.292467in}}%
\pgfpathlineto{\pgfqpoint{4.461367in}{1.291096in}}%
\pgfpathlineto{\pgfqpoint{4.464029in}{1.292763in}}%
\pgfpathlineto{\pgfqpoint{4.466717in}{1.294070in}}%
\pgfpathlineto{\pgfqpoint{4.469492in}{1.293548in}}%
\pgfpathlineto{\pgfqpoint{4.472059in}{1.292031in}}%
\pgfpathlineto{\pgfqpoint{4.474861in}{1.295775in}}%
\pgfpathlineto{\pgfqpoint{4.477430in}{1.295157in}}%
\pgfpathlineto{\pgfqpoint{4.480201in}{1.293734in}}%
\pgfpathlineto{\pgfqpoint{4.482778in}{1.294870in}}%
\pgfpathlineto{\pgfqpoint{4.485581in}{1.292893in}}%
\pgfpathlineto{\pgfqpoint{4.488130in}{1.294292in}}%
\pgfpathlineto{\pgfqpoint{4.490822in}{1.295267in}}%
\pgfpathlineto{\pgfqpoint{4.493492in}{1.298201in}}%
\pgfpathlineto{\pgfqpoint{4.496167in}{1.294208in}}%
\pgfpathlineto{\pgfqpoint{4.498850in}{1.298360in}}%
\pgfpathlineto{\pgfqpoint{4.501529in}{1.296558in}}%
\pgfpathlineto{\pgfqpoint{4.504305in}{1.294785in}}%
\pgfpathlineto{\pgfqpoint{4.506893in}{1.295061in}}%
\pgfpathlineto{\pgfqpoint{4.509643in}{1.294110in}}%
\pgfpathlineto{\pgfqpoint{4.512246in}{1.292573in}}%
\pgfpathlineto{\pgfqpoint{4.515080in}{1.294185in}}%
\pgfpathlineto{\pgfqpoint{4.517598in}{1.295322in}}%
\pgfpathlineto{\pgfqpoint{4.520345in}{1.297137in}}%
\pgfpathlineto{\pgfqpoint{4.522962in}{1.294187in}}%
\pgfpathlineto{\pgfqpoint{4.525640in}{1.292448in}}%
\pgfpathlineto{\pgfqpoint{4.528307in}{1.291318in}}%
\pgfpathlineto{\pgfqpoint{4.530990in}{1.293220in}}%
\pgfpathlineto{\pgfqpoint{4.533764in}{1.290822in}}%
\pgfpathlineto{\pgfqpoint{4.536400in}{1.288110in}}%
\pgfpathlineto{\pgfqpoint{4.539144in}{1.290306in}}%
\pgfpathlineto{\pgfqpoint{4.541711in}{1.293381in}}%
\pgfpathlineto{\pgfqpoint{4.544464in}{1.293849in}}%
\pgfpathlineto{\pgfqpoint{4.547064in}{1.293477in}}%
\pgfpathlineto{\pgfqpoint{4.549822in}{1.290635in}}%
\pgfpathlineto{\pgfqpoint{4.552425in}{1.292395in}}%
\pgfpathlineto{\pgfqpoint{4.555106in}{1.291060in}}%
\pgfpathlineto{\pgfqpoint{4.557777in}{1.292012in}}%
\pgfpathlineto{\pgfqpoint{4.560448in}{1.289986in}}%
\pgfpathlineto{\pgfqpoint{4.563125in}{1.289356in}}%
\pgfpathlineto{\pgfqpoint{4.565820in}{1.288424in}}%
\pgfpathlineto{\pgfqpoint{4.568612in}{1.289709in}}%
\pgfpathlineto{\pgfqpoint{4.571171in}{1.286951in}}%
\pgfpathlineto{\pgfqpoint{4.573947in}{1.288562in}}%
\pgfpathlineto{\pgfqpoint{4.576531in}{1.291610in}}%
\pgfpathlineto{\pgfqpoint{4.579305in}{1.291407in}}%
\pgfpathlineto{\pgfqpoint{4.581888in}{1.292283in}}%
\pgfpathlineto{\pgfqpoint{4.584672in}{1.290689in}}%
\pgfpathlineto{\pgfqpoint{4.587244in}{1.292019in}}%
\pgfpathlineto{\pgfqpoint{4.589920in}{1.287325in}}%
\pgfpathlineto{\pgfqpoint{4.592589in}{1.287125in}}%
\pgfpathlineto{\pgfqpoint{4.595281in}{1.290743in}}%
\pgfpathlineto{\pgfqpoint{4.597951in}{1.288705in}}%
\pgfpathlineto{\pgfqpoint{4.600633in}{1.290078in}}%
\pgfpathlineto{\pgfqpoint{4.603430in}{1.294844in}}%
\pgfpathlineto{\pgfqpoint{4.605990in}{1.292563in}}%
\pgfpathlineto{\pgfqpoint{4.608808in}{1.293935in}}%
\pgfpathlineto{\pgfqpoint{4.611350in}{1.293735in}}%
\pgfpathlineto{\pgfqpoint{4.614134in}{1.291866in}}%
\pgfpathlineto{\pgfqpoint{4.616702in}{1.293512in}}%
\pgfpathlineto{\pgfqpoint{4.619529in}{1.294860in}}%
\pgfpathlineto{\pgfqpoint{4.622056in}{1.295201in}}%
\pgfpathlineto{\pgfqpoint{4.624741in}{1.295804in}}%
\pgfpathlineto{\pgfqpoint{4.627411in}{1.293291in}}%
\pgfpathlineto{\pgfqpoint{4.630096in}{1.295022in}}%
\pgfpathlineto{\pgfqpoint{4.632902in}{1.296947in}}%
\pgfpathlineto{\pgfqpoint{4.635445in}{1.296292in}}%
\pgfpathlineto{\pgfqpoint{4.638204in}{1.296282in}}%
\pgfpathlineto{\pgfqpoint{4.640809in}{1.299163in}}%
\pgfpathlineto{\pgfqpoint{4.643628in}{1.295957in}}%
\pgfpathlineto{\pgfqpoint{4.646169in}{1.295726in}}%
\pgfpathlineto{\pgfqpoint{4.648922in}{1.295795in}}%
\pgfpathlineto{\pgfqpoint{4.651524in}{1.296383in}}%
\pgfpathlineto{\pgfqpoint{4.654203in}{1.295695in}}%
\pgfpathlineto{\pgfqpoint{4.656873in}{1.291420in}}%
\pgfpathlineto{\pgfqpoint{4.659590in}{1.292852in}}%
\pgfpathlineto{\pgfqpoint{4.662237in}{1.292076in}}%
\pgfpathlineto{\pgfqpoint{4.664923in}{1.294258in}}%
\pgfpathlineto{\pgfqpoint{4.667764in}{1.292396in}}%
\pgfpathlineto{\pgfqpoint{4.670261in}{1.292597in}}%
\pgfpathlineto{\pgfqpoint{4.673068in}{1.291936in}}%
\pgfpathlineto{\pgfqpoint{4.675619in}{1.285800in}}%
\pgfpathlineto{\pgfqpoint{4.678448in}{1.289705in}}%
\pgfpathlineto{\pgfqpoint{4.680988in}{1.287493in}}%
\pgfpathlineto{\pgfqpoint{4.683799in}{1.288678in}}%
\pgfpathlineto{\pgfqpoint{4.686337in}{1.300935in}}%
\pgfpathlineto{\pgfqpoint{4.689051in}{1.293892in}}%
\pgfpathlineto{\pgfqpoint{4.691694in}{1.295024in}}%
\pgfpathlineto{\pgfqpoint{4.694381in}{1.294210in}}%
\pgfpathlineto{\pgfqpoint{4.697170in}{1.296684in}}%
\pgfpathlineto{\pgfqpoint{4.699734in}{1.293029in}}%
\pgfpathlineto{\pgfqpoint{4.702517in}{1.289627in}}%
\pgfpathlineto{\pgfqpoint{4.705094in}{1.295870in}}%
\pgfpathlineto{\pgfqpoint{4.707824in}{1.292361in}}%
\pgfpathlineto{\pgfqpoint{4.710437in}{1.302494in}}%
\pgfpathlineto{\pgfqpoint{4.713275in}{1.303641in}}%
\pgfpathlineto{\pgfqpoint{4.715806in}{1.300167in}}%
\pgfpathlineto{\pgfqpoint{4.718486in}{1.292948in}}%
\pgfpathlineto{\pgfqpoint{4.721160in}{1.293764in}}%
\pgfpathlineto{\pgfqpoint{4.723873in}{1.297105in}}%
\pgfpathlineto{\pgfqpoint{4.726508in}{1.293676in}}%
\pgfpathlineto{\pgfqpoint{4.729233in}{1.291799in}}%
\pgfpathlineto{\pgfqpoint{4.731901in}{1.294317in}}%
\pgfpathlineto{\pgfqpoint{4.734552in}{1.300088in}}%
\pgfpathlineto{\pgfqpoint{4.737348in}{1.295083in}}%
\pgfpathlineto{\pgfqpoint{4.739912in}{1.296440in}}%
\pgfpathlineto{\pgfqpoint{4.742696in}{1.295337in}}%
\pgfpathlineto{\pgfqpoint{4.745256in}{1.299692in}}%
\pgfpathlineto{\pgfqpoint{4.748081in}{1.298589in}}%
\pgfpathlineto{\pgfqpoint{4.750627in}{1.298105in}}%
\pgfpathlineto{\pgfqpoint{4.753298in}{1.298869in}}%
\pgfpathlineto{\pgfqpoint{4.755983in}{1.297622in}}%
\pgfpathlineto{\pgfqpoint{4.758653in}{1.291456in}}%
\pgfpathlineto{\pgfqpoint{4.761337in}{1.289848in}}%
\pgfpathlineto{\pgfqpoint{4.764018in}{1.292624in}}%
\pgfpathlineto{\pgfqpoint{4.766783in}{1.297780in}}%
\pgfpathlineto{\pgfqpoint{4.769367in}{1.293187in}}%
\pgfpathlineto{\pgfqpoint{4.772198in}{1.295202in}}%
\pgfpathlineto{\pgfqpoint{4.774732in}{1.297318in}}%
\pgfpathlineto{\pgfqpoint{4.777535in}{1.296017in}}%
\pgfpathlineto{\pgfqpoint{4.780083in}{1.294335in}}%
\pgfpathlineto{\pgfqpoint{4.782872in}{1.292521in}}%
\pgfpathlineto{\pgfqpoint{4.785445in}{1.292671in}}%
\pgfpathlineto{\pgfqpoint{4.788116in}{1.290679in}}%
\pgfpathlineto{\pgfqpoint{4.790798in}{1.293410in}}%
\pgfpathlineto{\pgfqpoint{4.793512in}{1.294983in}}%
\pgfpathlineto{\pgfqpoint{4.796274in}{1.292071in}}%
\pgfpathlineto{\pgfqpoint{4.798830in}{1.293290in}}%
\pgfpathlineto{\pgfqpoint{4.801586in}{1.293123in}}%
\pgfpathlineto{\pgfqpoint{4.804193in}{1.297780in}}%
\pgfpathlineto{\pgfqpoint{4.807017in}{1.295606in}}%
\pgfpathlineto{\pgfqpoint{4.809538in}{1.292067in}}%
\pgfpathlineto{\pgfqpoint{4.812377in}{1.292177in}}%
\pgfpathlineto{\pgfqpoint{4.814907in}{1.300873in}}%
\pgfpathlineto{\pgfqpoint{4.817587in}{1.317608in}}%
\pgfpathlineto{\pgfqpoint{4.820265in}{1.303569in}}%
\pgfpathlineto{\pgfqpoint{4.822945in}{1.294400in}}%
\pgfpathlineto{\pgfqpoint{4.825619in}{1.294075in}}%
\pgfpathlineto{\pgfqpoint{4.828291in}{1.291539in}}%
\pgfpathlineto{\pgfqpoint{4.831045in}{1.292308in}}%
\pgfpathlineto{\pgfqpoint{4.833657in}{1.299314in}}%
\pgfpathlineto{\pgfqpoint{4.837992in}{1.297214in}}%
\pgfpathlineto{\pgfqpoint{4.839922in}{1.296200in}}%
\pgfpathlineto{\pgfqpoint{4.842380in}{1.296995in}}%
\pgfpathlineto{\pgfqpoint{4.844361in}{1.295426in}}%
\pgfpathlineto{\pgfqpoint{4.847127in}{1.294591in}}%
\pgfpathlineto{\pgfqpoint{4.849715in}{1.294486in}}%
\pgfpathlineto{\pgfqpoint{4.852404in}{1.294976in}}%
\pgfpathlineto{\pgfqpoint{4.855070in}{1.293156in}}%
\pgfpathlineto{\pgfqpoint{4.857807in}{1.294212in}}%
\pgfpathlineto{\pgfqpoint{4.860544in}{1.294347in}}%
\pgfpathlineto{\pgfqpoint{4.863116in}{1.294023in}}%
\pgfpathlineto{\pgfqpoint{4.865910in}{1.292249in}}%
\pgfpathlineto{\pgfqpoint{4.868474in}{1.292982in}}%
\pgfpathlineto{\pgfqpoint{4.871209in}{1.293470in}}%
\pgfpathlineto{\pgfqpoint{4.873832in}{1.291901in}}%
\pgfpathlineto{\pgfqpoint{4.876636in}{1.294215in}}%
\pgfpathlineto{\pgfqpoint{4.879180in}{1.294531in}}%
\pgfpathlineto{\pgfqpoint{4.881864in}{1.293151in}}%
\pgfpathlineto{\pgfqpoint{4.884540in}{1.291211in}}%
\pgfpathlineto{\pgfqpoint{4.887211in}{1.293866in}}%
\pgfpathlineto{\pgfqpoint{4.889902in}{1.294919in}}%
\pgfpathlineto{\pgfqpoint{4.892611in}{1.292710in}}%
\pgfpathlineto{\pgfqpoint{4.895399in}{1.292650in}}%
\pgfpathlineto{\pgfqpoint{4.897938in}{1.296630in}}%
\pgfpathlineto{\pgfqpoint{4.900712in}{1.298411in}}%
\pgfpathlineto{\pgfqpoint{4.903295in}{1.295735in}}%
\pgfpathlineto{\pgfqpoint{4.906096in}{1.295543in}}%
\pgfpathlineto{\pgfqpoint{4.908648in}{1.296592in}}%
\pgfpathlineto{\pgfqpoint{4.911435in}{1.291278in}}%
\pgfpathlineto{\pgfqpoint{4.914009in}{1.290436in}}%
\pgfpathlineto{\pgfqpoint{4.916681in}{1.290237in}}%
\pgfpathlineto{\pgfqpoint{4.919352in}{1.292311in}}%
\pgfpathlineto{\pgfqpoint{4.922041in}{1.292216in}}%
\pgfpathlineto{\pgfqpoint{4.924708in}{1.296907in}}%
\pgfpathlineto{\pgfqpoint{4.927400in}{1.294391in}}%
\pgfpathlineto{\pgfqpoint{4.930170in}{1.298418in}}%
\pgfpathlineto{\pgfqpoint{4.932742in}{1.299221in}}%
\pgfpathlineto{\pgfqpoint{4.935515in}{1.296572in}}%
\pgfpathlineto{\pgfqpoint{4.938112in}{1.294685in}}%
\pgfpathlineto{\pgfqpoint{4.940881in}{1.295942in}}%
\pgfpathlineto{\pgfqpoint{4.943466in}{1.294671in}}%
\pgfpathlineto{\pgfqpoint{4.946151in}{1.293317in}}%
\pgfpathlineto{\pgfqpoint{4.948827in}{1.292355in}}%
\pgfpathlineto{\pgfqpoint{4.951504in}{1.295094in}}%
\pgfpathlineto{\pgfqpoint{4.954182in}{1.295940in}}%
\pgfpathlineto{\pgfqpoint{4.956862in}{1.290644in}}%
\pgfpathlineto{\pgfqpoint{4.959689in}{1.292163in}}%
\pgfpathlineto{\pgfqpoint{4.962219in}{1.290671in}}%
\pgfpathlineto{\pgfqpoint{4.965002in}{1.292588in}}%
\pgfpathlineto{\pgfqpoint{4.967575in}{1.291701in}}%
\pgfpathlineto{\pgfqpoint{4.970314in}{1.287923in}}%
\pgfpathlineto{\pgfqpoint{4.972933in}{1.292687in}}%
\pgfpathlineto{\pgfqpoint{4.975703in}{1.290282in}}%
\pgfpathlineto{\pgfqpoint{4.978287in}{1.292285in}}%
\pgfpathlineto{\pgfqpoint{4.980967in}{1.293237in}}%
\pgfpathlineto{\pgfqpoint{4.983637in}{1.296911in}}%
\pgfpathlineto{\pgfqpoint{4.986325in}{1.291740in}}%
\pgfpathlineto{\pgfqpoint{4.989001in}{1.312683in}}%
\pgfpathlineto{\pgfqpoint{4.991683in}{1.343784in}}%
\pgfpathlineto{\pgfqpoint{4.994390in}{1.333313in}}%
\pgfpathlineto{\pgfqpoint{4.997028in}{1.322329in}}%
\pgfpathlineto{\pgfqpoint{4.999780in}{1.324544in}}%
\pgfpathlineto{\pgfqpoint{5.002384in}{1.351750in}}%
\pgfpathlineto{\pgfqpoint{5.005178in}{1.344874in}}%
\pgfpathlineto{\pgfqpoint{5.007751in}{1.335329in}}%
\pgfpathlineto{\pgfqpoint{5.010562in}{1.324000in}}%
\pgfpathlineto{\pgfqpoint{5.013104in}{1.327558in}}%
\pgfpathlineto{\pgfqpoint{5.015820in}{1.342523in}}%
\pgfpathlineto{\pgfqpoint{5.018466in}{1.328071in}}%
\pgfpathlineto{\pgfqpoint{5.021147in}{1.314840in}}%
\pgfpathlineto{\pgfqpoint{5.023927in}{1.317291in}}%
\pgfpathlineto{\pgfqpoint{5.026501in}{1.316971in}}%
\pgfpathlineto{\pgfqpoint{5.029275in}{1.316129in}}%
\pgfpathlineto{\pgfqpoint{5.031849in}{1.313893in}}%
\pgfpathlineto{\pgfqpoint{5.034649in}{1.368601in}}%
\pgfpathlineto{\pgfqpoint{5.037214in}{1.368858in}}%
\pgfpathlineto{\pgfqpoint{5.039962in}{1.361774in}}%
\pgfpathlineto{\pgfqpoint{5.042572in}{1.348991in}}%
\pgfpathlineto{\pgfqpoint{5.045249in}{1.337146in}}%
\pgfpathlineto{\pgfqpoint{5.047924in}{1.327619in}}%
\pgfpathlineto{\pgfqpoint{5.050606in}{1.333862in}}%
\pgfpathlineto{\pgfqpoint{5.053284in}{1.353592in}}%
\pgfpathlineto{\pgfqpoint{5.055952in}{1.344427in}}%
\pgfpathlineto{\pgfqpoint{5.058711in}{1.343199in}}%
\pgfpathlineto{\pgfqpoint{5.061315in}{1.334410in}}%
\pgfpathlineto{\pgfqpoint{5.064144in}{1.324574in}}%
\pgfpathlineto{\pgfqpoint{5.066677in}{1.315187in}}%
\pgfpathlineto{\pgfqpoint{5.069463in}{1.308303in}}%
\pgfpathlineto{\pgfqpoint{5.072030in}{1.301707in}}%
\pgfpathlineto{\pgfqpoint{5.074851in}{1.304271in}}%
\pgfpathlineto{\pgfqpoint{5.077390in}{1.300508in}}%
\pgfpathlineto{\pgfqpoint{5.080067in}{1.302957in}}%
\pgfpathlineto{\pgfqpoint{5.082746in}{1.297150in}}%
\pgfpathlineto{\pgfqpoint{5.085426in}{1.295838in}}%
\pgfpathlineto{\pgfqpoint{5.088103in}{1.295421in}}%
\pgfpathlineto{\pgfqpoint{5.090788in}{1.294898in}}%
\pgfpathlineto{\pgfqpoint{5.093579in}{1.303136in}}%
\pgfpathlineto{\pgfqpoint{5.096142in}{1.305503in}}%
\pgfpathlineto{\pgfqpoint{5.098948in}{1.298373in}}%
\pgfpathlineto{\pgfqpoint{5.101496in}{1.295604in}}%
\pgfpathlineto{\pgfqpoint{5.104312in}{1.293080in}}%
\pgfpathlineto{\pgfqpoint{5.106842in}{1.292270in}}%
\pgfpathlineto{\pgfqpoint{5.109530in}{1.289566in}}%
\pgfpathlineto{\pgfqpoint{5.112209in}{1.291561in}}%
\pgfpathlineto{\pgfqpoint{5.114887in}{1.293568in}}%
\pgfpathlineto{\pgfqpoint{5.117550in}{1.289206in}}%
\pgfpathlineto{\pgfqpoint{5.120243in}{1.289259in}}%
\pgfpathlineto{\pgfqpoint{5.123042in}{1.292805in}}%
\pgfpathlineto{\pgfqpoint{5.125599in}{1.291923in}}%
\pgfpathlineto{\pgfqpoint{5.128421in}{1.294233in}}%
\pgfpathlineto{\pgfqpoint{5.130953in}{1.294264in}}%
\pgfpathlineto{\pgfqpoint{5.133716in}{1.293485in}}%
\pgfpathlineto{\pgfqpoint{5.136311in}{1.295701in}}%
\pgfpathlineto{\pgfqpoint{5.139072in}{1.292669in}}%
\pgfpathlineto{\pgfqpoint{5.141660in}{1.292347in}}%
\pgfpathlineto{\pgfqpoint{5.144349in}{1.297191in}}%
\pgfpathlineto{\pgfqpoint{5.147029in}{1.297380in}}%
\pgfpathlineto{\pgfqpoint{5.149734in}{1.297448in}}%
\pgfpathlineto{\pgfqpoint{5.152382in}{1.301313in}}%
\pgfpathlineto{\pgfqpoint{5.155059in}{1.304125in}}%
\pgfpathlineto{\pgfqpoint{5.157815in}{1.299563in}}%
\pgfpathlineto{\pgfqpoint{5.160420in}{1.294285in}}%
\pgfpathlineto{\pgfqpoint{5.163243in}{1.293641in}}%
\pgfpathlineto{\pgfqpoint{5.165775in}{1.293069in}}%
\pgfpathlineto{\pgfqpoint{5.168591in}{1.290705in}}%
\pgfpathlineto{\pgfqpoint{5.171133in}{1.290894in}}%
\pgfpathlineto{\pgfqpoint{5.173925in}{1.296004in}}%
\pgfpathlineto{\pgfqpoint{5.176477in}{1.294083in}}%
\pgfpathlineto{\pgfqpoint{5.179188in}{1.291290in}}%
\pgfpathlineto{\pgfqpoint{5.181848in}{1.288890in}}%
\pgfpathlineto{\pgfqpoint{5.184522in}{1.290974in}}%
\pgfpathlineto{\pgfqpoint{5.187294in}{1.291897in}}%
\pgfpathlineto{\pgfqpoint{5.189880in}{1.294715in}}%
\pgfpathlineto{\pgfqpoint{5.192680in}{1.292556in}}%
\pgfpathlineto{\pgfqpoint{5.195239in}{1.292279in}}%
\pgfpathlineto{\pgfqpoint{5.198008in}{1.294109in}}%
\pgfpathlineto{\pgfqpoint{5.200594in}{1.295223in}}%
\pgfpathlineto{\pgfqpoint{5.203388in}{1.288295in}}%
\pgfpathlineto{\pgfqpoint{5.205952in}{1.287264in}}%
\pgfpathlineto{\pgfqpoint{5.208630in}{1.290682in}}%
\pgfpathlineto{\pgfqpoint{5.211299in}{1.287360in}}%
\pgfpathlineto{\pgfqpoint{5.214027in}{1.291338in}}%
\pgfpathlineto{\pgfqpoint{5.216667in}{1.293401in}}%
\pgfpathlineto{\pgfqpoint{5.219345in}{1.291739in}}%
\pgfpathlineto{\pgfqpoint{5.222151in}{1.292862in}}%
\pgfpathlineto{\pgfqpoint{5.224695in}{1.293039in}}%
\pgfpathlineto{\pgfqpoint{5.227470in}{1.294577in}}%
\pgfpathlineto{\pgfqpoint{5.230059in}{1.292350in}}%
\pgfpathlineto{\pgfqpoint{5.232855in}{1.295229in}}%
\pgfpathlineto{\pgfqpoint{5.235409in}{1.295183in}}%
\pgfpathlineto{\pgfqpoint{5.238173in}{1.295600in}}%
\pgfpathlineto{\pgfqpoint{5.240777in}{1.294857in}}%
\pgfpathlineto{\pgfqpoint{5.243445in}{1.295871in}}%
\pgfpathlineto{\pgfqpoint{5.246130in}{1.292225in}}%
\pgfpathlineto{\pgfqpoint{5.248816in}{1.296592in}}%
\pgfpathlineto{\pgfqpoint{5.251590in}{1.294755in}}%
\pgfpathlineto{\pgfqpoint{5.254236in}{1.296690in}}%
\pgfpathlineto{\pgfqpoint{5.256973in}{1.296159in}}%
\pgfpathlineto{\pgfqpoint{5.259511in}{1.297606in}}%
\pgfpathlineto{\pgfqpoint{5.262264in}{1.290104in}}%
\pgfpathlineto{\pgfqpoint{5.264876in}{1.292103in}}%
\pgfpathlineto{\pgfqpoint{5.267691in}{1.293014in}}%
\pgfpathlineto{\pgfqpoint{5.270238in}{1.295692in}}%
\pgfpathlineto{\pgfqpoint{5.272913in}{1.307784in}}%
\pgfpathlineto{\pgfqpoint{5.275589in}{1.315750in}}%
\pgfpathlineto{\pgfqpoint{5.278322in}{1.303179in}}%
\pgfpathlineto{\pgfqpoint{5.280947in}{1.300681in}}%
\pgfpathlineto{\pgfqpoint{5.283631in}{1.292585in}}%
\pgfpathlineto{\pgfqpoint{5.286436in}{1.286022in}}%
\pgfpathlineto{\pgfqpoint{5.288984in}{1.286998in}}%
\pgfpathlineto{\pgfqpoint{5.291794in}{1.288296in}}%
\pgfpathlineto{\pgfqpoint{5.294339in}{1.289392in}}%
\pgfpathlineto{\pgfqpoint{5.297140in}{1.289884in}}%
\pgfpathlineto{\pgfqpoint{5.299696in}{1.290489in}}%
\pgfpathlineto{\pgfqpoint{5.302443in}{1.294796in}}%
\pgfpathlineto{\pgfqpoint{5.305054in}{1.291651in}}%
\pgfpathlineto{\pgfqpoint{5.307731in}{1.293905in}}%
\pgfpathlineto{\pgfqpoint{5.310411in}{1.293956in}}%
\pgfpathlineto{\pgfqpoint{5.313089in}{1.295391in}}%
\pgfpathlineto{\pgfqpoint{5.315754in}{1.296076in}}%
\pgfpathlineto{\pgfqpoint{5.318430in}{1.296178in}}%
\pgfpathlineto{\pgfqpoint{5.321256in}{1.293208in}}%
\pgfpathlineto{\pgfqpoint{5.323802in}{1.288594in}}%
\pgfpathlineto{\pgfqpoint{5.326564in}{1.291875in}}%
\pgfpathlineto{\pgfqpoint{5.329159in}{1.289582in}}%
\pgfpathlineto{\pgfqpoint{5.331973in}{1.288720in}}%
\pgfpathlineto{\pgfqpoint{5.334510in}{1.291099in}}%
\pgfpathlineto{\pgfqpoint{5.337353in}{1.291017in}}%
\pgfpathlineto{\pgfqpoint{5.339872in}{1.295205in}}%
\pgfpathlineto{\pgfqpoint{5.342549in}{1.295830in}}%
\pgfpathlineto{\pgfqpoint{5.345224in}{1.305381in}}%
\pgfpathlineto{\pgfqpoint{5.347905in}{1.297139in}}%
\pgfpathlineto{\pgfqpoint{5.350723in}{1.295073in}}%
\pgfpathlineto{\pgfqpoint{5.353262in}{1.295869in}}%
\pgfpathlineto{\pgfqpoint{5.356056in}{1.293407in}}%
\pgfpathlineto{\pgfqpoint{5.358612in}{1.293668in}}%
\pgfpathlineto{\pgfqpoint{5.361370in}{1.293528in}}%
\pgfpathlineto{\pgfqpoint{5.363966in}{1.293556in}}%
\pgfpathlineto{\pgfqpoint{5.366727in}{1.294756in}}%
\pgfpathlineto{\pgfqpoint{5.369335in}{1.291331in}}%
\pgfpathlineto{\pgfqpoint{5.372013in}{1.289787in}}%
\pgfpathlineto{\pgfqpoint{5.374692in}{1.291619in}}%
\pgfpathlineto{\pgfqpoint{5.377370in}{1.291817in}}%
\pgfpathlineto{\pgfqpoint{5.380048in}{1.292648in}}%
\pgfpathlineto{\pgfqpoint{5.382725in}{1.293717in}}%
\pgfpathlineto{\pgfqpoint{5.385550in}{1.302483in}}%
\pgfpathlineto{\pgfqpoint{5.388083in}{1.302402in}}%
\pgfpathlineto{\pgfqpoint{5.390900in}{1.321634in}}%
\pgfpathlineto{\pgfqpoint{5.393441in}{1.346187in}}%
\pgfpathlineto{\pgfqpoint{5.396219in}{1.325748in}}%
\pgfpathlineto{\pgfqpoint{5.398784in}{1.311456in}}%
\pgfpathlineto{\pgfqpoint{5.401576in}{1.309082in}}%
\pgfpathlineto{\pgfqpoint{5.404154in}{1.305269in}}%
\pgfpathlineto{\pgfqpoint{5.406832in}{1.301743in}}%
\pgfpathlineto{\pgfqpoint{5.409507in}{1.305026in}}%
\pgfpathlineto{\pgfqpoint{5.412190in}{1.303360in}}%
\pgfpathlineto{\pgfqpoint{5.414954in}{1.295392in}}%
\pgfpathlineto{\pgfqpoint{5.417547in}{1.295996in}}%
\pgfpathlineto{\pgfqpoint{5.420304in}{1.297991in}}%
\pgfpathlineto{\pgfqpoint{5.422897in}{1.295972in}}%
\pgfpathlineto{\pgfqpoint{5.425661in}{1.290411in}}%
\pgfpathlineto{\pgfqpoint{5.428259in}{1.292071in}}%
\pgfpathlineto{\pgfqpoint{5.431015in}{1.291816in}}%
\pgfpathlineto{\pgfqpoint{5.433616in}{1.292288in}}%
\pgfpathlineto{\pgfqpoint{5.436295in}{1.296256in}}%
\pgfpathlineto{\pgfqpoint{5.438974in}{1.293206in}}%
\pgfpathlineto{\pgfqpoint{5.441698in}{1.290103in}}%
\pgfpathlineto{\pgfqpoint{5.444328in}{1.287484in}}%
\pgfpathlineto{\pgfqpoint{5.447021in}{1.291317in}}%
\pgfpathlineto{\pgfqpoint{5.449769in}{1.291726in}}%
\pgfpathlineto{\pgfqpoint{5.452365in}{1.298717in}}%
\pgfpathlineto{\pgfqpoint{5.455168in}{1.303999in}}%
\pgfpathlineto{\pgfqpoint{5.457721in}{1.303099in}}%
\pgfpathlineto{\pgfqpoint{5.460489in}{1.306459in}}%
\pgfpathlineto{\pgfqpoint{5.463079in}{1.314382in}}%
\pgfpathlineto{\pgfqpoint{5.465888in}{1.302192in}}%
\pgfpathlineto{\pgfqpoint{5.468425in}{1.308382in}}%
\pgfpathlineto{\pgfqpoint{5.471113in}{1.309662in}}%
\pgfpathlineto{\pgfqpoint{5.473792in}{1.315279in}}%
\pgfpathlineto{\pgfqpoint{5.476458in}{1.316186in}}%
\pgfpathlineto{\pgfqpoint{5.479152in}{1.316870in}}%
\pgfpathlineto{\pgfqpoint{5.481825in}{1.315512in}}%
\pgfpathlineto{\pgfqpoint{5.484641in}{1.315258in}}%
\pgfpathlineto{\pgfqpoint{5.487176in}{1.309722in}}%
\pgfpathlineto{\pgfqpoint{5.490000in}{1.304698in}}%
\pgfpathlineto{\pgfqpoint{5.492541in}{1.302086in}}%
\pgfpathlineto{\pgfqpoint{5.495346in}{1.303624in}}%
\pgfpathlineto{\pgfqpoint{5.497898in}{1.299633in}}%
\pgfpathlineto{\pgfqpoint{5.500687in}{1.303975in}}%
\pgfpathlineto{\pgfqpoint{5.503255in}{1.299889in}}%
\pgfpathlineto{\pgfqpoint{5.505933in}{1.298677in}}%
\pgfpathlineto{\pgfqpoint{5.508612in}{1.298729in}}%
\pgfpathlineto{\pgfqpoint{5.511290in}{1.298411in}}%
\pgfpathlineto{\pgfqpoint{5.514080in}{1.299245in}}%
\pgfpathlineto{\pgfqpoint{5.516646in}{1.298874in}}%
\pgfpathlineto{\pgfqpoint{5.519433in}{1.295365in}}%
\pgfpathlineto{\pgfqpoint{5.522003in}{1.293271in}}%
\pgfpathlineto{\pgfqpoint{5.524756in}{1.295398in}}%
\pgfpathlineto{\pgfqpoint{5.527360in}{1.291712in}}%
\pgfpathlineto{\pgfqpoint{5.530148in}{1.291930in}}%
\pgfpathlineto{\pgfqpoint{5.532717in}{1.293649in}}%
\pgfpathlineto{\pgfqpoint{5.535395in}{1.294691in}}%
\pgfpathlineto{\pgfqpoint{5.538074in}{1.293063in}}%
\pgfpathlineto{\pgfqpoint{5.540750in}{1.294265in}}%
\pgfpathlineto{\pgfqpoint{5.543421in}{1.295120in}}%
\pgfpathlineto{\pgfqpoint{5.546110in}{1.290494in}}%
\pgfpathlineto{\pgfqpoint{5.548921in}{1.293914in}}%
\pgfpathlineto{\pgfqpoint{5.551457in}{1.291430in}}%
\pgfpathlineto{\pgfqpoint{5.554198in}{1.291860in}}%
\pgfpathlineto{\pgfqpoint{5.556822in}{1.291503in}}%
\pgfpathlineto{\pgfqpoint{5.559612in}{1.298158in}}%
\pgfpathlineto{\pgfqpoint{5.562180in}{1.290851in}}%
\pgfpathlineto{\pgfqpoint{5.564940in}{1.293339in}}%
\pgfpathlineto{\pgfqpoint{5.567536in}{1.293117in}}%
\pgfpathlineto{\pgfqpoint{5.570215in}{1.295996in}}%
\pgfpathlineto{\pgfqpoint{5.572893in}{1.291501in}}%
\pgfpathlineto{\pgfqpoint{5.575596in}{1.294397in}}%
\pgfpathlineto{\pgfqpoint{5.578342in}{1.287625in}}%
\pgfpathlineto{\pgfqpoint{5.580914in}{1.293306in}}%
\pgfpathlineto{\pgfqpoint{5.583709in}{1.296554in}}%
\pgfpathlineto{\pgfqpoint{5.586269in}{1.294075in}}%
\pgfpathlineto{\pgfqpoint{5.589040in}{1.292567in}}%
\pgfpathlineto{\pgfqpoint{5.591641in}{1.291090in}}%
\pgfpathlineto{\pgfqpoint{5.594368in}{1.290530in}}%
\pgfpathlineto{\pgfqpoint{5.596999in}{1.288707in}}%
\pgfpathlineto{\pgfqpoint{5.599674in}{1.291064in}}%
\pgfpathlineto{\pgfqpoint{5.602352in}{1.293275in}}%
\pgfpathlineto{\pgfqpoint{5.605073in}{1.294232in}}%
\pgfpathlineto{\pgfqpoint{5.607698in}{1.295483in}}%
\pgfpathlineto{\pgfqpoint{5.610389in}{1.301023in}}%
\pgfpathlineto{\pgfqpoint{5.613235in}{1.321325in}}%
\pgfpathlineto{\pgfqpoint{5.615743in}{1.336640in}}%
\pgfpathlineto{\pgfqpoint{5.618526in}{1.328699in}}%
\pgfpathlineto{\pgfqpoint{5.621102in}{1.326433in}}%
\pgfpathlineto{\pgfqpoint{5.623868in}{1.323327in}}%
\pgfpathlineto{\pgfqpoint{5.626460in}{1.320075in}}%
\pgfpathlineto{\pgfqpoint{5.629232in}{1.314114in}}%
\pgfpathlineto{\pgfqpoint{5.631815in}{1.309088in}}%
\pgfpathlineto{\pgfqpoint{5.634496in}{1.304010in}}%
\pgfpathlineto{\pgfqpoint{5.637172in}{1.303264in}}%
\pgfpathlineto{\pgfqpoint{5.639852in}{1.302389in}}%
\pgfpathlineto{\pgfqpoint{5.642518in}{1.298627in}}%
\pgfpathlineto{\pgfqpoint{5.645243in}{1.297850in}}%
\pgfpathlineto{\pgfqpoint{5.648008in}{1.290692in}}%
\pgfpathlineto{\pgfqpoint{5.650563in}{1.291716in}}%
\pgfpathlineto{\pgfqpoint{5.653376in}{1.297901in}}%
\pgfpathlineto{\pgfqpoint{5.655919in}{1.297605in}}%
\pgfpathlineto{\pgfqpoint{5.658723in}{1.301246in}}%
\pgfpathlineto{\pgfqpoint{5.661273in}{1.300318in}}%
\pgfpathlineto{\pgfqpoint{5.664099in}{1.300371in}}%
\pgfpathlineto{\pgfqpoint{5.666632in}{1.294131in}}%
\pgfpathlineto{\pgfqpoint{5.669313in}{1.295794in}}%
\pgfpathlineto{\pgfqpoint{5.671991in}{1.298036in}}%
\pgfpathlineto{\pgfqpoint{5.674667in}{1.294639in}}%
\pgfpathlineto{\pgfqpoint{5.677486in}{1.296789in}}%
\pgfpathlineto{\pgfqpoint{5.680027in}{1.294683in}}%
\pgfpathlineto{\pgfqpoint{5.682836in}{1.294883in}}%
\pgfpathlineto{\pgfqpoint{5.685385in}{1.299958in}}%
\pgfpathlineto{\pgfqpoint{5.688159in}{1.296362in}}%
\pgfpathlineto{\pgfqpoint{5.690730in}{1.296427in}}%
\pgfpathlineto{\pgfqpoint{5.693473in}{1.296237in}}%
\pgfpathlineto{\pgfqpoint{5.696101in}{1.296316in}}%
\pgfpathlineto{\pgfqpoint{5.698775in}{1.293101in}}%
\pgfpathlineto{\pgfqpoint{5.701453in}{1.293879in}}%
\pgfpathlineto{\pgfqpoint{5.704130in}{1.292598in}}%
\pgfpathlineto{\pgfqpoint{5.706800in}{1.292082in}}%
\pgfpathlineto{\pgfqpoint{5.709490in}{1.292134in}}%
\pgfpathlineto{\pgfqpoint{5.712291in}{1.290245in}}%
\pgfpathlineto{\pgfqpoint{5.714834in}{1.296964in}}%
\pgfpathlineto{\pgfqpoint{5.717671in}{1.293584in}}%
\pgfpathlineto{\pgfqpoint{5.720201in}{1.299955in}}%
\pgfpathlineto{\pgfqpoint{5.722950in}{1.301098in}}%
\pgfpathlineto{\pgfqpoint{5.725548in}{1.293899in}}%
\pgfpathlineto{\pgfqpoint{5.728339in}{1.296257in}}%
\pgfpathlineto{\pgfqpoint{5.730919in}{1.295619in}}%
\pgfpathlineto{\pgfqpoint{5.733594in}{1.295130in}}%
\pgfpathlineto{\pgfqpoint{5.736276in}{1.298058in}}%
\pgfpathlineto{\pgfqpoint{5.738974in}{1.295345in}}%
\pgfpathlineto{\pgfqpoint{5.741745in}{1.293774in}}%
\pgfpathlineto{\pgfqpoint{5.744310in}{1.298448in}}%
\pgfpathlineto{\pgfqpoint{5.744310in}{0.413320in}}%
\pgfpathlineto{\pgfqpoint{5.744310in}{0.413320in}}%
\pgfpathlineto{\pgfqpoint{5.741745in}{0.413320in}}%
\pgfpathlineto{\pgfqpoint{5.738974in}{0.413320in}}%
\pgfpathlineto{\pgfqpoint{5.736276in}{0.413320in}}%
\pgfpathlineto{\pgfqpoint{5.733594in}{0.413320in}}%
\pgfpathlineto{\pgfqpoint{5.730919in}{0.413320in}}%
\pgfpathlineto{\pgfqpoint{5.728339in}{0.413320in}}%
\pgfpathlineto{\pgfqpoint{5.725548in}{0.413320in}}%
\pgfpathlineto{\pgfqpoint{5.722950in}{0.413320in}}%
\pgfpathlineto{\pgfqpoint{5.720201in}{0.413320in}}%
\pgfpathlineto{\pgfqpoint{5.717671in}{0.413320in}}%
\pgfpathlineto{\pgfqpoint{5.714834in}{0.413320in}}%
\pgfpathlineto{\pgfqpoint{5.712291in}{0.413320in}}%
\pgfpathlineto{\pgfqpoint{5.709490in}{0.413320in}}%
\pgfpathlineto{\pgfqpoint{5.706800in}{0.413320in}}%
\pgfpathlineto{\pgfqpoint{5.704130in}{0.413320in}}%
\pgfpathlineto{\pgfqpoint{5.701453in}{0.413320in}}%
\pgfpathlineto{\pgfqpoint{5.698775in}{0.413320in}}%
\pgfpathlineto{\pgfqpoint{5.696101in}{0.413320in}}%
\pgfpathlineto{\pgfqpoint{5.693473in}{0.413320in}}%
\pgfpathlineto{\pgfqpoint{5.690730in}{0.413320in}}%
\pgfpathlineto{\pgfqpoint{5.688159in}{0.413320in}}%
\pgfpathlineto{\pgfqpoint{5.685385in}{0.413320in}}%
\pgfpathlineto{\pgfqpoint{5.682836in}{0.413320in}}%
\pgfpathlineto{\pgfqpoint{5.680027in}{0.413320in}}%
\pgfpathlineto{\pgfqpoint{5.677486in}{0.413320in}}%
\pgfpathlineto{\pgfqpoint{5.674667in}{0.413320in}}%
\pgfpathlineto{\pgfqpoint{5.671991in}{0.413320in}}%
\pgfpathlineto{\pgfqpoint{5.669313in}{0.413320in}}%
\pgfpathlineto{\pgfqpoint{5.666632in}{0.413320in}}%
\pgfpathlineto{\pgfqpoint{5.664099in}{0.413320in}}%
\pgfpathlineto{\pgfqpoint{5.661273in}{0.413320in}}%
\pgfpathlineto{\pgfqpoint{5.658723in}{0.413320in}}%
\pgfpathlineto{\pgfqpoint{5.655919in}{0.413320in}}%
\pgfpathlineto{\pgfqpoint{5.653376in}{0.413320in}}%
\pgfpathlineto{\pgfqpoint{5.650563in}{0.413320in}}%
\pgfpathlineto{\pgfqpoint{5.648008in}{0.413320in}}%
\pgfpathlineto{\pgfqpoint{5.645243in}{0.413320in}}%
\pgfpathlineto{\pgfqpoint{5.642518in}{0.413320in}}%
\pgfpathlineto{\pgfqpoint{5.639852in}{0.413320in}}%
\pgfpathlineto{\pgfqpoint{5.637172in}{0.413320in}}%
\pgfpathlineto{\pgfqpoint{5.634496in}{0.413320in}}%
\pgfpathlineto{\pgfqpoint{5.631815in}{0.413320in}}%
\pgfpathlineto{\pgfqpoint{5.629232in}{0.413320in}}%
\pgfpathlineto{\pgfqpoint{5.626460in}{0.413320in}}%
\pgfpathlineto{\pgfqpoint{5.623868in}{0.413320in}}%
\pgfpathlineto{\pgfqpoint{5.621102in}{0.413320in}}%
\pgfpathlineto{\pgfqpoint{5.618526in}{0.413320in}}%
\pgfpathlineto{\pgfqpoint{5.615743in}{0.413320in}}%
\pgfpathlineto{\pgfqpoint{5.613235in}{0.413320in}}%
\pgfpathlineto{\pgfqpoint{5.610389in}{0.413320in}}%
\pgfpathlineto{\pgfqpoint{5.607698in}{0.413320in}}%
\pgfpathlineto{\pgfqpoint{5.605073in}{0.413320in}}%
\pgfpathlineto{\pgfqpoint{5.602352in}{0.413320in}}%
\pgfpathlineto{\pgfqpoint{5.599674in}{0.413320in}}%
\pgfpathlineto{\pgfqpoint{5.596999in}{0.413320in}}%
\pgfpathlineto{\pgfqpoint{5.594368in}{0.413320in}}%
\pgfpathlineto{\pgfqpoint{5.591641in}{0.413320in}}%
\pgfpathlineto{\pgfqpoint{5.589040in}{0.413320in}}%
\pgfpathlineto{\pgfqpoint{5.586269in}{0.413320in}}%
\pgfpathlineto{\pgfqpoint{5.583709in}{0.413320in}}%
\pgfpathlineto{\pgfqpoint{5.580914in}{0.413320in}}%
\pgfpathlineto{\pgfqpoint{5.578342in}{0.413320in}}%
\pgfpathlineto{\pgfqpoint{5.575596in}{0.413320in}}%
\pgfpathlineto{\pgfqpoint{5.572893in}{0.413320in}}%
\pgfpathlineto{\pgfqpoint{5.570215in}{0.413320in}}%
\pgfpathlineto{\pgfqpoint{5.567536in}{0.413320in}}%
\pgfpathlineto{\pgfqpoint{5.564940in}{0.413320in}}%
\pgfpathlineto{\pgfqpoint{5.562180in}{0.413320in}}%
\pgfpathlineto{\pgfqpoint{5.559612in}{0.413320in}}%
\pgfpathlineto{\pgfqpoint{5.556822in}{0.413320in}}%
\pgfpathlineto{\pgfqpoint{5.554198in}{0.413320in}}%
\pgfpathlineto{\pgfqpoint{5.551457in}{0.413320in}}%
\pgfpathlineto{\pgfqpoint{5.548921in}{0.413320in}}%
\pgfpathlineto{\pgfqpoint{5.546110in}{0.413320in}}%
\pgfpathlineto{\pgfqpoint{5.543421in}{0.413320in}}%
\pgfpathlineto{\pgfqpoint{5.540750in}{0.413320in}}%
\pgfpathlineto{\pgfqpoint{5.538074in}{0.413320in}}%
\pgfpathlineto{\pgfqpoint{5.535395in}{0.413320in}}%
\pgfpathlineto{\pgfqpoint{5.532717in}{0.413320in}}%
\pgfpathlineto{\pgfqpoint{5.530148in}{0.413320in}}%
\pgfpathlineto{\pgfqpoint{5.527360in}{0.413320in}}%
\pgfpathlineto{\pgfqpoint{5.524756in}{0.413320in}}%
\pgfpathlineto{\pgfqpoint{5.522003in}{0.413320in}}%
\pgfpathlineto{\pgfqpoint{5.519433in}{0.413320in}}%
\pgfpathlineto{\pgfqpoint{5.516646in}{0.413320in}}%
\pgfpathlineto{\pgfqpoint{5.514080in}{0.413320in}}%
\pgfpathlineto{\pgfqpoint{5.511290in}{0.413320in}}%
\pgfpathlineto{\pgfqpoint{5.508612in}{0.413320in}}%
\pgfpathlineto{\pgfqpoint{5.505933in}{0.413320in}}%
\pgfpathlineto{\pgfqpoint{5.503255in}{0.413320in}}%
\pgfpathlineto{\pgfqpoint{5.500687in}{0.413320in}}%
\pgfpathlineto{\pgfqpoint{5.497898in}{0.413320in}}%
\pgfpathlineto{\pgfqpoint{5.495346in}{0.413320in}}%
\pgfpathlineto{\pgfqpoint{5.492541in}{0.413320in}}%
\pgfpathlineto{\pgfqpoint{5.490000in}{0.413320in}}%
\pgfpathlineto{\pgfqpoint{5.487176in}{0.413320in}}%
\pgfpathlineto{\pgfqpoint{5.484641in}{0.413320in}}%
\pgfpathlineto{\pgfqpoint{5.481825in}{0.413320in}}%
\pgfpathlineto{\pgfqpoint{5.479152in}{0.413320in}}%
\pgfpathlineto{\pgfqpoint{5.476458in}{0.413320in}}%
\pgfpathlineto{\pgfqpoint{5.473792in}{0.413320in}}%
\pgfpathlineto{\pgfqpoint{5.471113in}{0.413320in}}%
\pgfpathlineto{\pgfqpoint{5.468425in}{0.413320in}}%
\pgfpathlineto{\pgfqpoint{5.465888in}{0.413320in}}%
\pgfpathlineto{\pgfqpoint{5.463079in}{0.413320in}}%
\pgfpathlineto{\pgfqpoint{5.460489in}{0.413320in}}%
\pgfpathlineto{\pgfqpoint{5.457721in}{0.413320in}}%
\pgfpathlineto{\pgfqpoint{5.455168in}{0.413320in}}%
\pgfpathlineto{\pgfqpoint{5.452365in}{0.413320in}}%
\pgfpathlineto{\pgfqpoint{5.449769in}{0.413320in}}%
\pgfpathlineto{\pgfqpoint{5.447021in}{0.413320in}}%
\pgfpathlineto{\pgfqpoint{5.444328in}{0.413320in}}%
\pgfpathlineto{\pgfqpoint{5.441698in}{0.413320in}}%
\pgfpathlineto{\pgfqpoint{5.438974in}{0.413320in}}%
\pgfpathlineto{\pgfqpoint{5.436295in}{0.413320in}}%
\pgfpathlineto{\pgfqpoint{5.433616in}{0.413320in}}%
\pgfpathlineto{\pgfqpoint{5.431015in}{0.413320in}}%
\pgfpathlineto{\pgfqpoint{5.428259in}{0.413320in}}%
\pgfpathlineto{\pgfqpoint{5.425661in}{0.413320in}}%
\pgfpathlineto{\pgfqpoint{5.422897in}{0.413320in}}%
\pgfpathlineto{\pgfqpoint{5.420304in}{0.413320in}}%
\pgfpathlineto{\pgfqpoint{5.417547in}{0.413320in}}%
\pgfpathlineto{\pgfqpoint{5.414954in}{0.413320in}}%
\pgfpathlineto{\pgfqpoint{5.412190in}{0.413320in}}%
\pgfpathlineto{\pgfqpoint{5.409507in}{0.413320in}}%
\pgfpathlineto{\pgfqpoint{5.406832in}{0.413320in}}%
\pgfpathlineto{\pgfqpoint{5.404154in}{0.413320in}}%
\pgfpathlineto{\pgfqpoint{5.401576in}{0.413320in}}%
\pgfpathlineto{\pgfqpoint{5.398784in}{0.413320in}}%
\pgfpathlineto{\pgfqpoint{5.396219in}{0.413320in}}%
\pgfpathlineto{\pgfqpoint{5.393441in}{0.413320in}}%
\pgfpathlineto{\pgfqpoint{5.390900in}{0.413320in}}%
\pgfpathlineto{\pgfqpoint{5.388083in}{0.413320in}}%
\pgfpathlineto{\pgfqpoint{5.385550in}{0.413320in}}%
\pgfpathlineto{\pgfqpoint{5.382725in}{0.413320in}}%
\pgfpathlineto{\pgfqpoint{5.380048in}{0.413320in}}%
\pgfpathlineto{\pgfqpoint{5.377370in}{0.413320in}}%
\pgfpathlineto{\pgfqpoint{5.374692in}{0.413320in}}%
\pgfpathlineto{\pgfqpoint{5.372013in}{0.413320in}}%
\pgfpathlineto{\pgfqpoint{5.369335in}{0.413320in}}%
\pgfpathlineto{\pgfqpoint{5.366727in}{0.413320in}}%
\pgfpathlineto{\pgfqpoint{5.363966in}{0.413320in}}%
\pgfpathlineto{\pgfqpoint{5.361370in}{0.413320in}}%
\pgfpathlineto{\pgfqpoint{5.358612in}{0.413320in}}%
\pgfpathlineto{\pgfqpoint{5.356056in}{0.413320in}}%
\pgfpathlineto{\pgfqpoint{5.353262in}{0.413320in}}%
\pgfpathlineto{\pgfqpoint{5.350723in}{0.413320in}}%
\pgfpathlineto{\pgfqpoint{5.347905in}{0.413320in}}%
\pgfpathlineto{\pgfqpoint{5.345224in}{0.413320in}}%
\pgfpathlineto{\pgfqpoint{5.342549in}{0.413320in}}%
\pgfpathlineto{\pgfqpoint{5.339872in}{0.413320in}}%
\pgfpathlineto{\pgfqpoint{5.337353in}{0.413320in}}%
\pgfpathlineto{\pgfqpoint{5.334510in}{0.413320in}}%
\pgfpathlineto{\pgfqpoint{5.331973in}{0.413320in}}%
\pgfpathlineto{\pgfqpoint{5.329159in}{0.413320in}}%
\pgfpathlineto{\pgfqpoint{5.326564in}{0.413320in}}%
\pgfpathlineto{\pgfqpoint{5.323802in}{0.413320in}}%
\pgfpathlineto{\pgfqpoint{5.321256in}{0.413320in}}%
\pgfpathlineto{\pgfqpoint{5.318430in}{0.413320in}}%
\pgfpathlineto{\pgfqpoint{5.315754in}{0.413320in}}%
\pgfpathlineto{\pgfqpoint{5.313089in}{0.413320in}}%
\pgfpathlineto{\pgfqpoint{5.310411in}{0.413320in}}%
\pgfpathlineto{\pgfqpoint{5.307731in}{0.413320in}}%
\pgfpathlineto{\pgfqpoint{5.305054in}{0.413320in}}%
\pgfpathlineto{\pgfqpoint{5.302443in}{0.413320in}}%
\pgfpathlineto{\pgfqpoint{5.299696in}{0.413320in}}%
\pgfpathlineto{\pgfqpoint{5.297140in}{0.413320in}}%
\pgfpathlineto{\pgfqpoint{5.294339in}{0.413320in}}%
\pgfpathlineto{\pgfqpoint{5.291794in}{0.413320in}}%
\pgfpathlineto{\pgfqpoint{5.288984in}{0.413320in}}%
\pgfpathlineto{\pgfqpoint{5.286436in}{0.413320in}}%
\pgfpathlineto{\pgfqpoint{5.283631in}{0.413320in}}%
\pgfpathlineto{\pgfqpoint{5.280947in}{0.413320in}}%
\pgfpathlineto{\pgfqpoint{5.278322in}{0.413320in}}%
\pgfpathlineto{\pgfqpoint{5.275589in}{0.413320in}}%
\pgfpathlineto{\pgfqpoint{5.272913in}{0.413320in}}%
\pgfpathlineto{\pgfqpoint{5.270238in}{0.413320in}}%
\pgfpathlineto{\pgfqpoint{5.267691in}{0.413320in}}%
\pgfpathlineto{\pgfqpoint{5.264876in}{0.413320in}}%
\pgfpathlineto{\pgfqpoint{5.262264in}{0.413320in}}%
\pgfpathlineto{\pgfqpoint{5.259511in}{0.413320in}}%
\pgfpathlineto{\pgfqpoint{5.256973in}{0.413320in}}%
\pgfpathlineto{\pgfqpoint{5.254236in}{0.413320in}}%
\pgfpathlineto{\pgfqpoint{5.251590in}{0.413320in}}%
\pgfpathlineto{\pgfqpoint{5.248816in}{0.413320in}}%
\pgfpathlineto{\pgfqpoint{5.246130in}{0.413320in}}%
\pgfpathlineto{\pgfqpoint{5.243445in}{0.413320in}}%
\pgfpathlineto{\pgfqpoint{5.240777in}{0.413320in}}%
\pgfpathlineto{\pgfqpoint{5.238173in}{0.413320in}}%
\pgfpathlineto{\pgfqpoint{5.235409in}{0.413320in}}%
\pgfpathlineto{\pgfqpoint{5.232855in}{0.413320in}}%
\pgfpathlineto{\pgfqpoint{5.230059in}{0.413320in}}%
\pgfpathlineto{\pgfqpoint{5.227470in}{0.413320in}}%
\pgfpathlineto{\pgfqpoint{5.224695in}{0.413320in}}%
\pgfpathlineto{\pgfqpoint{5.222151in}{0.413320in}}%
\pgfpathlineto{\pgfqpoint{5.219345in}{0.413320in}}%
\pgfpathlineto{\pgfqpoint{5.216667in}{0.413320in}}%
\pgfpathlineto{\pgfqpoint{5.214027in}{0.413320in}}%
\pgfpathlineto{\pgfqpoint{5.211299in}{0.413320in}}%
\pgfpathlineto{\pgfqpoint{5.208630in}{0.413320in}}%
\pgfpathlineto{\pgfqpoint{5.205952in}{0.413320in}}%
\pgfpathlineto{\pgfqpoint{5.203388in}{0.413320in}}%
\pgfpathlineto{\pgfqpoint{5.200594in}{0.413320in}}%
\pgfpathlineto{\pgfqpoint{5.198008in}{0.413320in}}%
\pgfpathlineto{\pgfqpoint{5.195239in}{0.413320in}}%
\pgfpathlineto{\pgfqpoint{5.192680in}{0.413320in}}%
\pgfpathlineto{\pgfqpoint{5.189880in}{0.413320in}}%
\pgfpathlineto{\pgfqpoint{5.187294in}{0.413320in}}%
\pgfpathlineto{\pgfqpoint{5.184522in}{0.413320in}}%
\pgfpathlineto{\pgfqpoint{5.181848in}{0.413320in}}%
\pgfpathlineto{\pgfqpoint{5.179188in}{0.413320in}}%
\pgfpathlineto{\pgfqpoint{5.176477in}{0.413320in}}%
\pgfpathlineto{\pgfqpoint{5.173925in}{0.413320in}}%
\pgfpathlineto{\pgfqpoint{5.171133in}{0.413320in}}%
\pgfpathlineto{\pgfqpoint{5.168591in}{0.413320in}}%
\pgfpathlineto{\pgfqpoint{5.165775in}{0.413320in}}%
\pgfpathlineto{\pgfqpoint{5.163243in}{0.413320in}}%
\pgfpathlineto{\pgfqpoint{5.160420in}{0.413320in}}%
\pgfpathlineto{\pgfqpoint{5.157815in}{0.413320in}}%
\pgfpathlineto{\pgfqpoint{5.155059in}{0.413320in}}%
\pgfpathlineto{\pgfqpoint{5.152382in}{0.413320in}}%
\pgfpathlineto{\pgfqpoint{5.149734in}{0.413320in}}%
\pgfpathlineto{\pgfqpoint{5.147029in}{0.413320in}}%
\pgfpathlineto{\pgfqpoint{5.144349in}{0.413320in}}%
\pgfpathlineto{\pgfqpoint{5.141660in}{0.413320in}}%
\pgfpathlineto{\pgfqpoint{5.139072in}{0.413320in}}%
\pgfpathlineto{\pgfqpoint{5.136311in}{0.413320in}}%
\pgfpathlineto{\pgfqpoint{5.133716in}{0.413320in}}%
\pgfpathlineto{\pgfqpoint{5.130953in}{0.413320in}}%
\pgfpathlineto{\pgfqpoint{5.128421in}{0.413320in}}%
\pgfpathlineto{\pgfqpoint{5.125599in}{0.413320in}}%
\pgfpathlineto{\pgfqpoint{5.123042in}{0.413320in}}%
\pgfpathlineto{\pgfqpoint{5.120243in}{0.413320in}}%
\pgfpathlineto{\pgfqpoint{5.117550in}{0.413320in}}%
\pgfpathlineto{\pgfqpoint{5.114887in}{0.413320in}}%
\pgfpathlineto{\pgfqpoint{5.112209in}{0.413320in}}%
\pgfpathlineto{\pgfqpoint{5.109530in}{0.413320in}}%
\pgfpathlineto{\pgfqpoint{5.106842in}{0.413320in}}%
\pgfpathlineto{\pgfqpoint{5.104312in}{0.413320in}}%
\pgfpathlineto{\pgfqpoint{5.101496in}{0.413320in}}%
\pgfpathlineto{\pgfqpoint{5.098948in}{0.413320in}}%
\pgfpathlineto{\pgfqpoint{5.096142in}{0.413320in}}%
\pgfpathlineto{\pgfqpoint{5.093579in}{0.413320in}}%
\pgfpathlineto{\pgfqpoint{5.090788in}{0.413320in}}%
\pgfpathlineto{\pgfqpoint{5.088103in}{0.413320in}}%
\pgfpathlineto{\pgfqpoint{5.085426in}{0.413320in}}%
\pgfpathlineto{\pgfqpoint{5.082746in}{0.413320in}}%
\pgfpathlineto{\pgfqpoint{5.080067in}{0.413320in}}%
\pgfpathlineto{\pgfqpoint{5.077390in}{0.413320in}}%
\pgfpathlineto{\pgfqpoint{5.074851in}{0.413320in}}%
\pgfpathlineto{\pgfqpoint{5.072030in}{0.413320in}}%
\pgfpathlineto{\pgfqpoint{5.069463in}{0.413320in}}%
\pgfpathlineto{\pgfqpoint{5.066677in}{0.413320in}}%
\pgfpathlineto{\pgfqpoint{5.064144in}{0.413320in}}%
\pgfpathlineto{\pgfqpoint{5.061315in}{0.413320in}}%
\pgfpathlineto{\pgfqpoint{5.058711in}{0.413320in}}%
\pgfpathlineto{\pgfqpoint{5.055952in}{0.413320in}}%
\pgfpathlineto{\pgfqpoint{5.053284in}{0.413320in}}%
\pgfpathlineto{\pgfqpoint{5.050606in}{0.413320in}}%
\pgfpathlineto{\pgfqpoint{5.047924in}{0.413320in}}%
\pgfpathlineto{\pgfqpoint{5.045249in}{0.413320in}}%
\pgfpathlineto{\pgfqpoint{5.042572in}{0.413320in}}%
\pgfpathlineto{\pgfqpoint{5.039962in}{0.413320in}}%
\pgfpathlineto{\pgfqpoint{5.037214in}{0.413320in}}%
\pgfpathlineto{\pgfqpoint{5.034649in}{0.413320in}}%
\pgfpathlineto{\pgfqpoint{5.031849in}{0.413320in}}%
\pgfpathlineto{\pgfqpoint{5.029275in}{0.413320in}}%
\pgfpathlineto{\pgfqpoint{5.026501in}{0.413320in}}%
\pgfpathlineto{\pgfqpoint{5.023927in}{0.413320in}}%
\pgfpathlineto{\pgfqpoint{5.021147in}{0.413320in}}%
\pgfpathlineto{\pgfqpoint{5.018466in}{0.413320in}}%
\pgfpathlineto{\pgfqpoint{5.015820in}{0.413320in}}%
\pgfpathlineto{\pgfqpoint{5.013104in}{0.413320in}}%
\pgfpathlineto{\pgfqpoint{5.010562in}{0.413320in}}%
\pgfpathlineto{\pgfqpoint{5.007751in}{0.413320in}}%
\pgfpathlineto{\pgfqpoint{5.005178in}{0.413320in}}%
\pgfpathlineto{\pgfqpoint{5.002384in}{0.413320in}}%
\pgfpathlineto{\pgfqpoint{4.999780in}{0.413320in}}%
\pgfpathlineto{\pgfqpoint{4.997028in}{0.413320in}}%
\pgfpathlineto{\pgfqpoint{4.994390in}{0.413320in}}%
\pgfpathlineto{\pgfqpoint{4.991683in}{0.413320in}}%
\pgfpathlineto{\pgfqpoint{4.989001in}{0.413320in}}%
\pgfpathlineto{\pgfqpoint{4.986325in}{0.413320in}}%
\pgfpathlineto{\pgfqpoint{4.983637in}{0.413320in}}%
\pgfpathlineto{\pgfqpoint{4.980967in}{0.413320in}}%
\pgfpathlineto{\pgfqpoint{4.978287in}{0.413320in}}%
\pgfpathlineto{\pgfqpoint{4.975703in}{0.413320in}}%
\pgfpathlineto{\pgfqpoint{4.972933in}{0.413320in}}%
\pgfpathlineto{\pgfqpoint{4.970314in}{0.413320in}}%
\pgfpathlineto{\pgfqpoint{4.967575in}{0.413320in}}%
\pgfpathlineto{\pgfqpoint{4.965002in}{0.413320in}}%
\pgfpathlineto{\pgfqpoint{4.962219in}{0.413320in}}%
\pgfpathlineto{\pgfqpoint{4.959689in}{0.413320in}}%
\pgfpathlineto{\pgfqpoint{4.956862in}{0.413320in}}%
\pgfpathlineto{\pgfqpoint{4.954182in}{0.413320in}}%
\pgfpathlineto{\pgfqpoint{4.951504in}{0.413320in}}%
\pgfpathlineto{\pgfqpoint{4.948827in}{0.413320in}}%
\pgfpathlineto{\pgfqpoint{4.946151in}{0.413320in}}%
\pgfpathlineto{\pgfqpoint{4.943466in}{0.413320in}}%
\pgfpathlineto{\pgfqpoint{4.940881in}{0.413320in}}%
\pgfpathlineto{\pgfqpoint{4.938112in}{0.413320in}}%
\pgfpathlineto{\pgfqpoint{4.935515in}{0.413320in}}%
\pgfpathlineto{\pgfqpoint{4.932742in}{0.413320in}}%
\pgfpathlineto{\pgfqpoint{4.930170in}{0.413320in}}%
\pgfpathlineto{\pgfqpoint{4.927400in}{0.413320in}}%
\pgfpathlineto{\pgfqpoint{4.924708in}{0.413320in}}%
\pgfpathlineto{\pgfqpoint{4.922041in}{0.413320in}}%
\pgfpathlineto{\pgfqpoint{4.919352in}{0.413320in}}%
\pgfpathlineto{\pgfqpoint{4.916681in}{0.413320in}}%
\pgfpathlineto{\pgfqpoint{4.914009in}{0.413320in}}%
\pgfpathlineto{\pgfqpoint{4.911435in}{0.413320in}}%
\pgfpathlineto{\pgfqpoint{4.908648in}{0.413320in}}%
\pgfpathlineto{\pgfqpoint{4.906096in}{0.413320in}}%
\pgfpathlineto{\pgfqpoint{4.903295in}{0.413320in}}%
\pgfpathlineto{\pgfqpoint{4.900712in}{0.413320in}}%
\pgfpathlineto{\pgfqpoint{4.897938in}{0.413320in}}%
\pgfpathlineto{\pgfqpoint{4.895399in}{0.413320in}}%
\pgfpathlineto{\pgfqpoint{4.892611in}{0.413320in}}%
\pgfpathlineto{\pgfqpoint{4.889902in}{0.413320in}}%
\pgfpathlineto{\pgfqpoint{4.887211in}{0.413320in}}%
\pgfpathlineto{\pgfqpoint{4.884540in}{0.413320in}}%
\pgfpathlineto{\pgfqpoint{4.881864in}{0.413320in}}%
\pgfpathlineto{\pgfqpoint{4.879180in}{0.413320in}}%
\pgfpathlineto{\pgfqpoint{4.876636in}{0.413320in}}%
\pgfpathlineto{\pgfqpoint{4.873832in}{0.413320in}}%
\pgfpathlineto{\pgfqpoint{4.871209in}{0.413320in}}%
\pgfpathlineto{\pgfqpoint{4.868474in}{0.413320in}}%
\pgfpathlineto{\pgfqpoint{4.865910in}{0.413320in}}%
\pgfpathlineto{\pgfqpoint{4.863116in}{0.413320in}}%
\pgfpathlineto{\pgfqpoint{4.860544in}{0.413320in}}%
\pgfpathlineto{\pgfqpoint{4.857807in}{0.413320in}}%
\pgfpathlineto{\pgfqpoint{4.855070in}{0.413320in}}%
\pgfpathlineto{\pgfqpoint{4.852404in}{0.413320in}}%
\pgfpathlineto{\pgfqpoint{4.849715in}{0.413320in}}%
\pgfpathlineto{\pgfqpoint{4.847127in}{0.413320in}}%
\pgfpathlineto{\pgfqpoint{4.844361in}{0.413320in}}%
\pgfpathlineto{\pgfqpoint{4.842380in}{0.413320in}}%
\pgfpathlineto{\pgfqpoint{4.839922in}{0.413320in}}%
\pgfpathlineto{\pgfqpoint{4.837992in}{0.413320in}}%
\pgfpathlineto{\pgfqpoint{4.833657in}{0.413320in}}%
\pgfpathlineto{\pgfqpoint{4.831045in}{0.413320in}}%
\pgfpathlineto{\pgfqpoint{4.828291in}{0.413320in}}%
\pgfpathlineto{\pgfqpoint{4.825619in}{0.413320in}}%
\pgfpathlineto{\pgfqpoint{4.822945in}{0.413320in}}%
\pgfpathlineto{\pgfqpoint{4.820265in}{0.413320in}}%
\pgfpathlineto{\pgfqpoint{4.817587in}{0.413320in}}%
\pgfpathlineto{\pgfqpoint{4.814907in}{0.413320in}}%
\pgfpathlineto{\pgfqpoint{4.812377in}{0.413320in}}%
\pgfpathlineto{\pgfqpoint{4.809538in}{0.413320in}}%
\pgfpathlineto{\pgfqpoint{4.807017in}{0.413320in}}%
\pgfpathlineto{\pgfqpoint{4.804193in}{0.413320in}}%
\pgfpathlineto{\pgfqpoint{4.801586in}{0.413320in}}%
\pgfpathlineto{\pgfqpoint{4.798830in}{0.413320in}}%
\pgfpathlineto{\pgfqpoint{4.796274in}{0.413320in}}%
\pgfpathlineto{\pgfqpoint{4.793512in}{0.413320in}}%
\pgfpathlineto{\pgfqpoint{4.790798in}{0.413320in}}%
\pgfpathlineto{\pgfqpoint{4.788116in}{0.413320in}}%
\pgfpathlineto{\pgfqpoint{4.785445in}{0.413320in}}%
\pgfpathlineto{\pgfqpoint{4.782872in}{0.413320in}}%
\pgfpathlineto{\pgfqpoint{4.780083in}{0.413320in}}%
\pgfpathlineto{\pgfqpoint{4.777535in}{0.413320in}}%
\pgfpathlineto{\pgfqpoint{4.774732in}{0.413320in}}%
\pgfpathlineto{\pgfqpoint{4.772198in}{0.413320in}}%
\pgfpathlineto{\pgfqpoint{4.769367in}{0.413320in}}%
\pgfpathlineto{\pgfqpoint{4.766783in}{0.413320in}}%
\pgfpathlineto{\pgfqpoint{4.764018in}{0.413320in}}%
\pgfpathlineto{\pgfqpoint{4.761337in}{0.413320in}}%
\pgfpathlineto{\pgfqpoint{4.758653in}{0.413320in}}%
\pgfpathlineto{\pgfqpoint{4.755983in}{0.413320in}}%
\pgfpathlineto{\pgfqpoint{4.753298in}{0.413320in}}%
\pgfpathlineto{\pgfqpoint{4.750627in}{0.413320in}}%
\pgfpathlineto{\pgfqpoint{4.748081in}{0.413320in}}%
\pgfpathlineto{\pgfqpoint{4.745256in}{0.413320in}}%
\pgfpathlineto{\pgfqpoint{4.742696in}{0.413320in}}%
\pgfpathlineto{\pgfqpoint{4.739912in}{0.413320in}}%
\pgfpathlineto{\pgfqpoint{4.737348in}{0.413320in}}%
\pgfpathlineto{\pgfqpoint{4.734552in}{0.413320in}}%
\pgfpathlineto{\pgfqpoint{4.731901in}{0.413320in}}%
\pgfpathlineto{\pgfqpoint{4.729233in}{0.413320in}}%
\pgfpathlineto{\pgfqpoint{4.726508in}{0.413320in}}%
\pgfpathlineto{\pgfqpoint{4.723873in}{0.413320in}}%
\pgfpathlineto{\pgfqpoint{4.721160in}{0.413320in}}%
\pgfpathlineto{\pgfqpoint{4.718486in}{0.413320in}}%
\pgfpathlineto{\pgfqpoint{4.715806in}{0.413320in}}%
\pgfpathlineto{\pgfqpoint{4.713275in}{0.413320in}}%
\pgfpathlineto{\pgfqpoint{4.710437in}{0.413320in}}%
\pgfpathlineto{\pgfqpoint{4.707824in}{0.413320in}}%
\pgfpathlineto{\pgfqpoint{4.705094in}{0.413320in}}%
\pgfpathlineto{\pgfqpoint{4.702517in}{0.413320in}}%
\pgfpathlineto{\pgfqpoint{4.699734in}{0.413320in}}%
\pgfpathlineto{\pgfqpoint{4.697170in}{0.413320in}}%
\pgfpathlineto{\pgfqpoint{4.694381in}{0.413320in}}%
\pgfpathlineto{\pgfqpoint{4.691694in}{0.413320in}}%
\pgfpathlineto{\pgfqpoint{4.689051in}{0.413320in}}%
\pgfpathlineto{\pgfqpoint{4.686337in}{0.413320in}}%
\pgfpathlineto{\pgfqpoint{4.683799in}{0.413320in}}%
\pgfpathlineto{\pgfqpoint{4.680988in}{0.413320in}}%
\pgfpathlineto{\pgfqpoint{4.678448in}{0.413320in}}%
\pgfpathlineto{\pgfqpoint{4.675619in}{0.413320in}}%
\pgfpathlineto{\pgfqpoint{4.673068in}{0.413320in}}%
\pgfpathlineto{\pgfqpoint{4.670261in}{0.413320in}}%
\pgfpathlineto{\pgfqpoint{4.667764in}{0.413320in}}%
\pgfpathlineto{\pgfqpoint{4.664923in}{0.413320in}}%
\pgfpathlineto{\pgfqpoint{4.662237in}{0.413320in}}%
\pgfpathlineto{\pgfqpoint{4.659590in}{0.413320in}}%
\pgfpathlineto{\pgfqpoint{4.656873in}{0.413320in}}%
\pgfpathlineto{\pgfqpoint{4.654203in}{0.413320in}}%
\pgfpathlineto{\pgfqpoint{4.651524in}{0.413320in}}%
\pgfpathlineto{\pgfqpoint{4.648922in}{0.413320in}}%
\pgfpathlineto{\pgfqpoint{4.646169in}{0.413320in}}%
\pgfpathlineto{\pgfqpoint{4.643628in}{0.413320in}}%
\pgfpathlineto{\pgfqpoint{4.640809in}{0.413320in}}%
\pgfpathlineto{\pgfqpoint{4.638204in}{0.413320in}}%
\pgfpathlineto{\pgfqpoint{4.635445in}{0.413320in}}%
\pgfpathlineto{\pgfqpoint{4.632902in}{0.413320in}}%
\pgfpathlineto{\pgfqpoint{4.630096in}{0.413320in}}%
\pgfpathlineto{\pgfqpoint{4.627411in}{0.413320in}}%
\pgfpathlineto{\pgfqpoint{4.624741in}{0.413320in}}%
\pgfpathlineto{\pgfqpoint{4.622056in}{0.413320in}}%
\pgfpathlineto{\pgfqpoint{4.619529in}{0.413320in}}%
\pgfpathlineto{\pgfqpoint{4.616702in}{0.413320in}}%
\pgfpathlineto{\pgfqpoint{4.614134in}{0.413320in}}%
\pgfpathlineto{\pgfqpoint{4.611350in}{0.413320in}}%
\pgfpathlineto{\pgfqpoint{4.608808in}{0.413320in}}%
\pgfpathlineto{\pgfqpoint{4.605990in}{0.413320in}}%
\pgfpathlineto{\pgfqpoint{4.603430in}{0.413320in}}%
\pgfpathlineto{\pgfqpoint{4.600633in}{0.413320in}}%
\pgfpathlineto{\pgfqpoint{4.597951in}{0.413320in}}%
\pgfpathlineto{\pgfqpoint{4.595281in}{0.413320in}}%
\pgfpathlineto{\pgfqpoint{4.592589in}{0.413320in}}%
\pgfpathlineto{\pgfqpoint{4.589920in}{0.413320in}}%
\pgfpathlineto{\pgfqpoint{4.587244in}{0.413320in}}%
\pgfpathlineto{\pgfqpoint{4.584672in}{0.413320in}}%
\pgfpathlineto{\pgfqpoint{4.581888in}{0.413320in}}%
\pgfpathlineto{\pgfqpoint{4.579305in}{0.413320in}}%
\pgfpathlineto{\pgfqpoint{4.576531in}{0.413320in}}%
\pgfpathlineto{\pgfqpoint{4.573947in}{0.413320in}}%
\pgfpathlineto{\pgfqpoint{4.571171in}{0.413320in}}%
\pgfpathlineto{\pgfqpoint{4.568612in}{0.413320in}}%
\pgfpathlineto{\pgfqpoint{4.565820in}{0.413320in}}%
\pgfpathlineto{\pgfqpoint{4.563125in}{0.413320in}}%
\pgfpathlineto{\pgfqpoint{4.560448in}{0.413320in}}%
\pgfpathlineto{\pgfqpoint{4.557777in}{0.413320in}}%
\pgfpathlineto{\pgfqpoint{4.555106in}{0.413320in}}%
\pgfpathlineto{\pgfqpoint{4.552425in}{0.413320in}}%
\pgfpathlineto{\pgfqpoint{4.549822in}{0.413320in}}%
\pgfpathlineto{\pgfqpoint{4.547064in}{0.413320in}}%
\pgfpathlineto{\pgfqpoint{4.544464in}{0.413320in}}%
\pgfpathlineto{\pgfqpoint{4.541711in}{0.413320in}}%
\pgfpathlineto{\pgfqpoint{4.539144in}{0.413320in}}%
\pgfpathlineto{\pgfqpoint{4.536400in}{0.413320in}}%
\pgfpathlineto{\pgfqpoint{4.533764in}{0.413320in}}%
\pgfpathlineto{\pgfqpoint{4.530990in}{0.413320in}}%
\pgfpathlineto{\pgfqpoint{4.528307in}{0.413320in}}%
\pgfpathlineto{\pgfqpoint{4.525640in}{0.413320in}}%
\pgfpathlineto{\pgfqpoint{4.522962in}{0.413320in}}%
\pgfpathlineto{\pgfqpoint{4.520345in}{0.413320in}}%
\pgfpathlineto{\pgfqpoint{4.517598in}{0.413320in}}%
\pgfpathlineto{\pgfqpoint{4.515080in}{0.413320in}}%
\pgfpathlineto{\pgfqpoint{4.512246in}{0.413320in}}%
\pgfpathlineto{\pgfqpoint{4.509643in}{0.413320in}}%
\pgfpathlineto{\pgfqpoint{4.506893in}{0.413320in}}%
\pgfpathlineto{\pgfqpoint{4.504305in}{0.413320in}}%
\pgfpathlineto{\pgfqpoint{4.501529in}{0.413320in}}%
\pgfpathlineto{\pgfqpoint{4.498850in}{0.413320in}}%
\pgfpathlineto{\pgfqpoint{4.496167in}{0.413320in}}%
\pgfpathlineto{\pgfqpoint{4.493492in}{0.413320in}}%
\pgfpathlineto{\pgfqpoint{4.490822in}{0.413320in}}%
\pgfpathlineto{\pgfqpoint{4.488130in}{0.413320in}}%
\pgfpathlineto{\pgfqpoint{4.485581in}{0.413320in}}%
\pgfpathlineto{\pgfqpoint{4.482778in}{0.413320in}}%
\pgfpathlineto{\pgfqpoint{4.480201in}{0.413320in}}%
\pgfpathlineto{\pgfqpoint{4.477430in}{0.413320in}}%
\pgfpathlineto{\pgfqpoint{4.474861in}{0.413320in}}%
\pgfpathlineto{\pgfqpoint{4.472059in}{0.413320in}}%
\pgfpathlineto{\pgfqpoint{4.469492in}{0.413320in}}%
\pgfpathlineto{\pgfqpoint{4.466717in}{0.413320in}}%
\pgfpathlineto{\pgfqpoint{4.464029in}{0.413320in}}%
\pgfpathlineto{\pgfqpoint{4.461367in}{0.413320in}}%
\pgfpathlineto{\pgfqpoint{4.458681in}{0.413320in}}%
\pgfpathlineto{\pgfqpoint{4.456138in}{0.413320in}}%
\pgfpathlineto{\pgfqpoint{4.453312in}{0.413320in}}%
\pgfpathlineto{\pgfqpoint{4.450767in}{0.413320in}}%
\pgfpathlineto{\pgfqpoint{4.447965in}{0.413320in}}%
\pgfpathlineto{\pgfqpoint{4.445423in}{0.413320in}}%
\pgfpathlineto{\pgfqpoint{4.442611in}{0.413320in}}%
\pgfpathlineto{\pgfqpoint{4.440041in}{0.413320in}}%
\pgfpathlineto{\pgfqpoint{4.437253in}{0.413320in}}%
\pgfpathlineto{\pgfqpoint{4.434569in}{0.413320in}}%
\pgfpathlineto{\pgfqpoint{4.431901in}{0.413320in}}%
\pgfpathlineto{\pgfqpoint{4.429220in}{0.413320in}}%
\pgfpathlineto{\pgfqpoint{4.426534in}{0.413320in}}%
\pgfpathlineto{\pgfqpoint{4.423863in}{0.413320in}}%
\pgfpathlineto{\pgfqpoint{4.421292in}{0.413320in}}%
\pgfpathlineto{\pgfqpoint{4.418506in}{0.413320in}}%
\pgfpathlineto{\pgfqpoint{4.415932in}{0.413320in}}%
\pgfpathlineto{\pgfqpoint{4.413149in}{0.413320in}}%
\pgfpathlineto{\pgfqpoint{4.410587in}{0.413320in}}%
\pgfpathlineto{\pgfqpoint{4.407788in}{0.413320in}}%
\pgfpathlineto{\pgfqpoint{4.405234in}{0.413320in}}%
\pgfpathlineto{\pgfqpoint{4.402468in}{0.413320in}}%
\pgfpathlineto{\pgfqpoint{4.399745in}{0.413320in}}%
\pgfpathlineto{\pgfqpoint{4.397076in}{0.413320in}}%
\pgfpathlineto{\pgfqpoint{4.394400in}{0.413320in}}%
\pgfpathlineto{\pgfqpoint{4.391721in}{0.413320in}}%
\pgfpathlineto{\pgfqpoint{4.389044in}{0.413320in}}%
\pgfpathlineto{\pgfqpoint{4.386431in}{0.413320in}}%
\pgfpathlineto{\pgfqpoint{4.383674in}{0.413320in}}%
\pgfpathlineto{\pgfqpoint{4.381097in}{0.413320in}}%
\pgfpathlineto{\pgfqpoint{4.378329in}{0.413320in}}%
\pgfpathlineto{\pgfqpoint{4.375761in}{0.413320in}}%
\pgfpathlineto{\pgfqpoint{4.372976in}{0.413320in}}%
\pgfpathlineto{\pgfqpoint{4.370437in}{0.413320in}}%
\pgfpathlineto{\pgfqpoint{4.367646in}{0.413320in}}%
\pgfpathlineto{\pgfqpoint{4.364936in}{0.413320in}}%
\pgfpathlineto{\pgfqpoint{4.362270in}{0.413320in}}%
\pgfpathlineto{\pgfqpoint{4.359582in}{0.413320in}}%
\pgfpathlineto{\pgfqpoint{4.357014in}{0.413320in}}%
\pgfpathlineto{\pgfqpoint{4.354224in}{0.413320in}}%
\pgfpathlineto{\pgfqpoint{4.351645in}{0.413320in}}%
\pgfpathlineto{\pgfqpoint{4.348868in}{0.413320in}}%
\pgfpathlineto{\pgfqpoint{4.346263in}{0.413320in}}%
\pgfpathlineto{\pgfqpoint{4.343510in}{0.413320in}}%
\pgfpathlineto{\pgfqpoint{4.340976in}{0.413320in}}%
\pgfpathlineto{\pgfqpoint{4.338154in}{0.413320in}}%
\pgfpathlineto{\pgfqpoint{4.335463in}{0.413320in}}%
\pgfpathlineto{\pgfqpoint{4.332796in}{0.413320in}}%
\pgfpathlineto{\pgfqpoint{4.330118in}{0.413320in}}%
\pgfpathlineto{\pgfqpoint{4.327440in}{0.413320in}}%
\pgfpathlineto{\pgfqpoint{4.324760in}{0.413320in}}%
\pgfpathlineto{\pgfqpoint{4.322181in}{0.413320in}}%
\pgfpathlineto{\pgfqpoint{4.319405in}{0.413320in}}%
\pgfpathlineto{\pgfqpoint{4.316856in}{0.413320in}}%
\pgfpathlineto{\pgfqpoint{4.314032in}{0.413320in}}%
\pgfpathlineto{\pgfqpoint{4.311494in}{0.413320in}}%
\pgfpathlineto{\pgfqpoint{4.308691in}{0.413320in}}%
\pgfpathlineto{\pgfqpoint{4.306118in}{0.413320in}}%
\pgfpathlineto{\pgfqpoint{4.303357in}{0.413320in}}%
\pgfpathlineto{\pgfqpoint{4.300656in}{0.413320in}}%
\pgfpathlineto{\pgfqpoint{4.297977in}{0.413320in}}%
\pgfpathlineto{\pgfqpoint{4.295299in}{0.413320in}}%
\pgfpathlineto{\pgfqpoint{4.292786in}{0.413320in}}%
\pgfpathlineto{\pgfqpoint{4.289936in}{0.413320in}}%
\pgfpathlineto{\pgfqpoint{4.287399in}{0.413320in}}%
\pgfpathlineto{\pgfqpoint{4.284586in}{0.413320in}}%
\pgfpathlineto{\pgfqpoint{4.282000in}{0.413320in}}%
\pgfpathlineto{\pgfqpoint{4.279212in}{0.413320in}}%
\pgfpathlineto{\pgfqpoint{4.276635in}{0.413320in}}%
\pgfpathlineto{\pgfqpoint{4.273874in}{0.413320in}}%
\pgfpathlineto{\pgfqpoint{4.271187in}{0.413320in}}%
\pgfpathlineto{\pgfqpoint{4.268590in}{0.413320in}}%
\pgfpathlineto{\pgfqpoint{4.265824in}{0.413320in}}%
\pgfpathlineto{\pgfqpoint{4.263157in}{0.413320in}}%
\pgfpathlineto{\pgfqpoint{4.260477in}{0.413320in}}%
\pgfpathlineto{\pgfqpoint{4.257958in}{0.413320in}}%
\pgfpathlineto{\pgfqpoint{4.255120in}{0.413320in}}%
\pgfpathlineto{\pgfqpoint{4.252581in}{0.413320in}}%
\pgfpathlineto{\pgfqpoint{4.249767in}{0.413320in}}%
\pgfpathlineto{\pgfqpoint{4.247225in}{0.413320in}}%
\pgfpathlineto{\pgfqpoint{4.244394in}{0.413320in}}%
\pgfpathlineto{\pgfqpoint{4.241900in}{0.413320in}}%
\pgfpathlineto{\pgfqpoint{4.239084in}{0.413320in}}%
\pgfpathlineto{\pgfqpoint{4.236375in}{0.413320in}}%
\pgfpathlineto{\pgfqpoint{4.233691in}{0.413320in}}%
\pgfpathlineto{\pgfqpoint{4.231013in}{0.413320in}}%
\pgfpathlineto{\pgfqpoint{4.228331in}{0.413320in}}%
\pgfpathlineto{\pgfqpoint{4.225654in}{0.413320in}}%
\pgfpathlineto{\pgfqpoint{4.223082in}{0.413320in}}%
\pgfpathlineto{\pgfqpoint{4.220304in}{0.413320in}}%
\pgfpathlineto{\pgfqpoint{4.217694in}{0.413320in}}%
\pgfpathlineto{\pgfqpoint{4.214948in}{0.413320in}}%
\pgfpathlineto{\pgfqpoint{4.212383in}{0.413320in}}%
\pgfpathlineto{\pgfqpoint{4.209597in}{0.413320in}}%
\pgfpathlineto{\pgfqpoint{4.207076in}{0.413320in}}%
\pgfpathlineto{\pgfqpoint{4.204240in}{0.413320in}}%
\pgfpathlineto{\pgfqpoint{4.201542in}{0.413320in}}%
\pgfpathlineto{\pgfqpoint{4.198878in}{0.413320in}}%
\pgfpathlineto{\pgfqpoint{4.196186in}{0.413320in}}%
\pgfpathlineto{\pgfqpoint{4.193638in}{0.413320in}}%
\pgfpathlineto{\pgfqpoint{4.190842in}{0.413320in}}%
\pgfpathlineto{\pgfqpoint{4.188318in}{0.413320in}}%
\pgfpathlineto{\pgfqpoint{4.185481in}{0.413320in}}%
\pgfpathlineto{\pgfqpoint{4.182899in}{0.413320in}}%
\pgfpathlineto{\pgfqpoint{4.180129in}{0.413320in}}%
\pgfpathlineto{\pgfqpoint{4.177593in}{0.413320in}}%
\pgfpathlineto{\pgfqpoint{4.174770in}{0.413320in}}%
\pgfpathlineto{\pgfqpoint{4.172093in}{0.413320in}}%
\pgfpathlineto{\pgfqpoint{4.169415in}{0.413320in}}%
\pgfpathlineto{\pgfqpoint{4.166737in}{0.413320in}}%
\pgfpathlineto{\pgfqpoint{4.164059in}{0.413320in}}%
\pgfpathlineto{\pgfqpoint{4.161380in}{0.413320in}}%
\pgfpathlineto{\pgfqpoint{4.158806in}{0.413320in}}%
\pgfpathlineto{\pgfqpoint{4.156016in}{0.413320in}}%
\pgfpathlineto{\pgfqpoint{4.153423in}{0.413320in}}%
\pgfpathlineto{\pgfqpoint{4.150665in}{0.413320in}}%
\pgfpathlineto{\pgfqpoint{4.148082in}{0.413320in}}%
\pgfpathlineto{\pgfqpoint{4.145310in}{0.413320in}}%
\pgfpathlineto{\pgfqpoint{4.142713in}{0.413320in}}%
\pgfpathlineto{\pgfqpoint{4.139963in}{0.413320in}}%
\pgfpathlineto{\pgfqpoint{4.137272in}{0.413320in}}%
\pgfpathlineto{\pgfqpoint{4.134615in}{0.413320in}}%
\pgfpathlineto{\pgfqpoint{4.131920in}{0.413320in}}%
\pgfpathlineto{\pgfqpoint{4.129349in}{0.413320in}}%
\pgfpathlineto{\pgfqpoint{4.126553in}{0.413320in}}%
\pgfpathlineto{\pgfqpoint{4.124019in}{0.413320in}}%
\pgfpathlineto{\pgfqpoint{4.121205in}{0.413320in}}%
\pgfpathlineto{\pgfqpoint{4.118554in}{0.413320in}}%
\pgfpathlineto{\pgfqpoint{4.115844in}{0.413320in}}%
\pgfpathlineto{\pgfqpoint{4.113252in}{0.413320in}}%
\pgfpathlineto{\pgfqpoint{4.110488in}{0.413320in}}%
\pgfpathlineto{\pgfqpoint{4.107814in}{0.413320in}}%
\pgfpathlineto{\pgfqpoint{4.105185in}{0.413320in}}%
\pgfpathlineto{\pgfqpoint{4.102456in}{0.413320in}}%
\pgfpathlineto{\pgfqpoint{4.099777in}{0.413320in}}%
\pgfpathlineto{\pgfqpoint{4.097092in}{0.413320in}}%
\pgfpathlineto{\pgfqpoint{4.094527in}{0.413320in}}%
\pgfpathlineto{\pgfqpoint{4.091729in}{0.413320in}}%
\pgfpathlineto{\pgfqpoint{4.089159in}{0.413320in}}%
\pgfpathlineto{\pgfqpoint{4.086385in}{0.413320in}}%
\pgfpathlineto{\pgfqpoint{4.083870in}{0.413320in}}%
\pgfpathlineto{\pgfqpoint{4.081018in}{0.413320in}}%
\pgfpathlineto{\pgfqpoint{4.078471in}{0.413320in}}%
\pgfpathlineto{\pgfqpoint{4.075705in}{0.413320in}}%
\pgfpathlineto{\pgfqpoint{4.072985in}{0.413320in}}%
\pgfpathlineto{\pgfqpoint{4.070313in}{0.413320in}}%
\pgfpathlineto{\pgfqpoint{4.067636in}{0.413320in}}%
\pgfpathlineto{\pgfqpoint{4.064957in}{0.413320in}}%
\pgfpathlineto{\pgfqpoint{4.062266in}{0.413320in}}%
\pgfpathlineto{\pgfqpoint{4.059702in}{0.413320in}}%
\pgfpathlineto{\pgfqpoint{4.056911in}{0.413320in}}%
\pgfpathlineto{\pgfqpoint{4.054326in}{0.413320in}}%
\pgfpathlineto{\pgfqpoint{4.051557in}{0.413320in}}%
\pgfpathlineto{\pgfqpoint{4.049006in}{0.413320in}}%
\pgfpathlineto{\pgfqpoint{4.046210in}{0.413320in}}%
\pgfpathlineto{\pgfqpoint{4.043667in}{0.413320in}}%
\pgfpathlineto{\pgfqpoint{4.040852in}{0.413320in}}%
\pgfpathlineto{\pgfqpoint{4.038174in}{0.413320in}}%
\pgfpathlineto{\pgfqpoint{4.035492in}{0.413320in}}%
\pgfpathlineto{\pgfqpoint{4.032817in}{0.413320in}}%
\pgfpathlineto{\pgfqpoint{4.030229in}{0.413320in}}%
\pgfpathlineto{\pgfqpoint{4.027447in}{0.413320in}}%
\pgfpathlineto{\pgfqpoint{4.024868in}{0.413320in}}%
\pgfpathlineto{\pgfqpoint{4.022097in}{0.413320in}}%
\pgfpathlineto{\pgfqpoint{4.019518in}{0.413320in}}%
\pgfpathlineto{\pgfqpoint{4.016744in}{0.413320in}}%
\pgfpathlineto{\pgfqpoint{4.014186in}{0.413320in}}%
\pgfpathlineto{\pgfqpoint{4.011394in}{0.413320in}}%
\pgfpathlineto{\pgfqpoint{4.008699in}{0.413320in}}%
\pgfpathlineto{\pgfqpoint{4.006034in}{0.413320in}}%
\pgfpathlineto{\pgfqpoint{4.003348in}{0.413320in}}%
\pgfpathlineto{\pgfqpoint{4.000674in}{0.413320in}}%
\pgfpathlineto{\pgfqpoint{3.997990in}{0.413320in}}%
\pgfpathlineto{\pgfqpoint{3.995417in}{0.413320in}}%
\pgfpathlineto{\pgfqpoint{3.992642in}{0.413320in}}%
\pgfpathlineto{\pgfqpoint{3.990055in}{0.413320in}}%
\pgfpathlineto{\pgfqpoint{3.987270in}{0.413320in}}%
\pgfpathlineto{\pgfqpoint{3.984714in}{0.413320in}}%
\pgfpathlineto{\pgfqpoint{3.981929in}{0.413320in}}%
\pgfpathlineto{\pgfqpoint{3.979389in}{0.413320in}}%
\pgfpathlineto{\pgfqpoint{3.976563in}{0.413320in}}%
\pgfpathlineto{\pgfqpoint{3.973885in}{0.413320in}}%
\pgfpathlineto{\pgfqpoint{3.971250in}{0.413320in}}%
\pgfpathlineto{\pgfqpoint{3.968523in}{0.413320in}}%
\pgfpathlineto{\pgfqpoint{3.966013in}{0.413320in}}%
\pgfpathlineto{\pgfqpoint{3.963176in}{0.413320in}}%
\pgfpathlineto{\pgfqpoint{3.960635in}{0.413320in}}%
\pgfpathlineto{\pgfqpoint{3.957823in}{0.413320in}}%
\pgfpathlineto{\pgfqpoint{3.955211in}{0.413320in}}%
\pgfpathlineto{\pgfqpoint{3.952464in}{0.413320in}}%
\pgfpathlineto{\pgfqpoint{3.949894in}{0.413320in}}%
\pgfpathlineto{\pgfqpoint{3.947101in}{0.413320in}}%
\pgfpathlineto{\pgfqpoint{3.944431in}{0.413320in}}%
\pgfpathlineto{\pgfqpoint{3.941778in}{0.413320in}}%
\pgfpathlineto{\pgfqpoint{3.939075in}{0.413320in}}%
\pgfpathlineto{\pgfqpoint{3.936395in}{0.413320in}}%
\pgfpathlineto{\pgfqpoint{3.933714in}{0.413320in}}%
\pgfpathlineto{\pgfqpoint{3.931202in}{0.413320in}}%
\pgfpathlineto{\pgfqpoint{3.928347in}{0.413320in}}%
\pgfpathlineto{\pgfqpoint{3.925778in}{0.413320in}}%
\pgfpathlineto{\pgfqpoint{3.923005in}{0.413320in}}%
\pgfpathlineto{\pgfqpoint{3.920412in}{0.413320in}}%
\pgfpathlineto{\pgfqpoint{3.917646in}{0.413320in}}%
\pgfpathlineto{\pgfqpoint{3.915107in}{0.413320in}}%
\pgfpathlineto{\pgfqpoint{3.912296in}{0.413320in}}%
\pgfpathlineto{\pgfqpoint{3.909602in}{0.413320in}}%
\pgfpathlineto{\pgfqpoint{3.906918in}{0.413320in}}%
\pgfpathlineto{\pgfqpoint{3.904252in}{0.413320in}}%
\pgfpathlineto{\pgfqpoint{3.901573in}{0.413320in}}%
\pgfpathlineto{\pgfqpoint{3.898891in}{0.413320in}}%
\pgfpathlineto{\pgfqpoint{3.896345in}{0.413320in}}%
\pgfpathlineto{\pgfqpoint{3.893541in}{0.413320in}}%
\pgfpathlineto{\pgfqpoint{3.890926in}{0.413320in}}%
\pgfpathlineto{\pgfqpoint{3.888188in}{0.413320in}}%
\pgfpathlineto{\pgfqpoint{3.885621in}{0.413320in}}%
\pgfpathlineto{\pgfqpoint{3.882850in}{0.413320in}}%
\pgfpathlineto{\pgfqpoint{3.880237in}{0.413320in}}%
\pgfpathlineto{\pgfqpoint{3.877466in}{0.413320in}}%
\pgfpathlineto{\pgfqpoint{3.874790in}{0.413320in}}%
\pgfpathlineto{\pgfqpoint{3.872114in}{0.413320in}}%
\pgfpathlineto{\pgfqpoint{3.869435in}{0.413320in}}%
\pgfpathlineto{\pgfqpoint{3.866815in}{0.413320in}}%
\pgfpathlineto{\pgfqpoint{3.864073in}{0.413320in}}%
\pgfpathlineto{\pgfqpoint{3.861561in}{0.413320in}}%
\pgfpathlineto{\pgfqpoint{3.858720in}{0.413320in}}%
\pgfpathlineto{\pgfqpoint{3.856100in}{0.413320in}}%
\pgfpathlineto{\pgfqpoint{3.853358in}{0.413320in}}%
\pgfpathlineto{\pgfqpoint{3.850814in}{0.413320in}}%
\pgfpathlineto{\pgfqpoint{3.848005in}{0.413320in}}%
\pgfpathlineto{\pgfqpoint{3.845329in}{0.413320in}}%
\pgfpathlineto{\pgfqpoint{3.842641in}{0.413320in}}%
\pgfpathlineto{\pgfqpoint{3.839960in}{0.413320in}}%
\pgfpathlineto{\pgfqpoint{3.837286in}{0.413320in}}%
\pgfpathlineto{\pgfqpoint{3.834616in}{0.413320in}}%
\pgfpathlineto{\pgfqpoint{3.832053in}{0.413320in}}%
\pgfpathlineto{\pgfqpoint{3.829252in}{0.413320in}}%
\pgfpathlineto{\pgfqpoint{3.826679in}{0.413320in}}%
\pgfpathlineto{\pgfqpoint{3.823903in}{0.413320in}}%
\pgfpathlineto{\pgfqpoint{3.821315in}{0.413320in}}%
\pgfpathlineto{\pgfqpoint{3.818546in}{0.413320in}}%
\pgfpathlineto{\pgfqpoint{3.815983in}{0.413320in}}%
\pgfpathlineto{\pgfqpoint{3.813172in}{0.413320in}}%
\pgfpathlineto{\pgfqpoint{3.810510in}{0.413320in}}%
\pgfpathlineto{\pgfqpoint{3.807832in}{0.413320in}}%
\pgfpathlineto{\pgfqpoint{3.805145in}{0.413320in}}%
\pgfpathlineto{\pgfqpoint{3.802569in}{0.413320in}}%
\pgfpathlineto{\pgfqpoint{3.799797in}{0.413320in}}%
\pgfpathlineto{\pgfqpoint{3.797265in}{0.413320in}}%
\pgfpathlineto{\pgfqpoint{3.794435in}{0.413320in}}%
\pgfpathlineto{\pgfqpoint{3.791897in}{0.413320in}}%
\pgfpathlineto{\pgfqpoint{3.789084in}{0.413320in}}%
\pgfpathlineto{\pgfqpoint{3.786504in}{0.413320in}}%
\pgfpathlineto{\pgfqpoint{3.783725in}{0.413320in}}%
\pgfpathlineto{\pgfqpoint{3.781046in}{0.413320in}}%
\pgfpathlineto{\pgfqpoint{3.778370in}{0.413320in}}%
\pgfpathlineto{\pgfqpoint{3.775691in}{0.413320in}}%
\pgfpathlineto{\pgfqpoint{3.773014in}{0.413320in}}%
\pgfpathlineto{\pgfqpoint{3.770323in}{0.413320in}}%
\pgfpathlineto{\pgfqpoint{3.767782in}{0.413320in}}%
\pgfpathlineto{\pgfqpoint{3.764966in}{0.413320in}}%
\pgfpathlineto{\pgfqpoint{3.762389in}{0.413320in}}%
\pgfpathlineto{\pgfqpoint{3.759622in}{0.413320in}}%
\pgfpathlineto{\pgfqpoint{3.757065in}{0.413320in}}%
\pgfpathlineto{\pgfqpoint{3.754265in}{0.413320in}}%
\pgfpathlineto{\pgfqpoint{3.751728in}{0.413320in}}%
\pgfpathlineto{\pgfqpoint{3.748903in}{0.413320in}}%
\pgfpathlineto{\pgfqpoint{3.746229in}{0.413320in}}%
\pgfpathlineto{\pgfqpoint{3.743548in}{0.413320in}}%
\pgfpathlineto{\pgfqpoint{3.740874in}{0.413320in}}%
\pgfpathlineto{\pgfqpoint{3.738194in}{0.413320in}}%
\pgfpathlineto{\pgfqpoint{3.735509in}{0.413320in}}%
\pgfpathlineto{\pgfqpoint{3.732950in}{0.413320in}}%
\pgfpathlineto{\pgfqpoint{3.730158in}{0.413320in}}%
\pgfpathlineto{\pgfqpoint{3.727581in}{0.413320in}}%
\pgfpathlineto{\pgfqpoint{3.724804in}{0.413320in}}%
\pgfpathlineto{\pgfqpoint{3.722228in}{0.413320in}}%
\pgfpathlineto{\pgfqpoint{3.719446in}{0.413320in}}%
\pgfpathlineto{\pgfqpoint{3.716875in}{0.413320in}}%
\pgfpathlineto{\pgfqpoint{3.714086in}{0.413320in}}%
\pgfpathlineto{\pgfqpoint{3.711410in}{0.413320in}}%
\pgfpathlineto{\pgfqpoint{3.708729in}{0.413320in}}%
\pgfpathlineto{\pgfqpoint{3.706053in}{0.413320in}}%
\pgfpathlineto{\pgfqpoint{3.703460in}{0.413320in}}%
\pgfpathlineto{\pgfqpoint{3.700684in}{0.413320in}}%
\pgfpathlineto{\pgfqpoint{3.698125in}{0.413320in}}%
\pgfpathlineto{\pgfqpoint{3.695331in}{0.413320in}}%
\pgfpathlineto{\pgfqpoint{3.692765in}{0.413320in}}%
\pgfpathlineto{\pgfqpoint{3.689983in}{0.413320in}}%
\pgfpathlineto{\pgfqpoint{3.687442in}{0.413320in}}%
\pgfpathlineto{\pgfqpoint{3.684620in}{0.413320in}}%
\pgfpathlineto{\pgfqpoint{3.681948in}{0.413320in}}%
\pgfpathlineto{\pgfqpoint{3.679273in}{0.413320in}}%
\pgfpathlineto{\pgfqpoint{3.676591in}{0.413320in}}%
\pgfpathlineto{\pgfqpoint{3.673911in}{0.413320in}}%
\pgfpathlineto{\pgfqpoint{3.671232in}{0.413320in}}%
\pgfpathlineto{\pgfqpoint{3.668665in}{0.413320in}}%
\pgfpathlineto{\pgfqpoint{3.665864in}{0.413320in}}%
\pgfpathlineto{\pgfqpoint{3.663276in}{0.413320in}}%
\pgfpathlineto{\pgfqpoint{3.660515in}{0.413320in}}%
\pgfpathlineto{\pgfqpoint{3.657917in}{0.413320in}}%
\pgfpathlineto{\pgfqpoint{3.655165in}{0.413320in}}%
\pgfpathlineto{\pgfqpoint{3.652628in}{0.413320in}}%
\pgfpathlineto{\pgfqpoint{3.649837in}{0.413320in}}%
\pgfpathlineto{\pgfqpoint{3.647130in}{0.413320in}}%
\pgfpathlineto{\pgfqpoint{3.644452in}{0.413320in}}%
\pgfpathlineto{\pgfqpoint{3.641773in}{0.413320in}}%
\pgfpathlineto{\pgfqpoint{3.639207in}{0.413320in}}%
\pgfpathlineto{\pgfqpoint{3.636413in}{0.413320in}}%
\pgfpathlineto{\pgfqpoint{3.633858in}{0.413320in}}%
\pgfpathlineto{\pgfqpoint{3.631058in}{0.413320in}}%
\pgfpathlineto{\pgfqpoint{3.628460in}{0.413320in}}%
\pgfpathlineto{\pgfqpoint{3.625689in}{0.413320in}}%
\pgfpathlineto{\pgfqpoint{3.623165in}{0.413320in}}%
\pgfpathlineto{\pgfqpoint{3.620345in}{0.413320in}}%
\pgfpathlineto{\pgfqpoint{3.617667in}{0.413320in}}%
\pgfpathlineto{\pgfqpoint{3.614982in}{0.413320in}}%
\pgfpathlineto{\pgfqpoint{3.612311in}{0.413320in}}%
\pgfpathlineto{\pgfqpoint{3.609632in}{0.413320in}}%
\pgfpathlineto{\pgfqpoint{3.606951in}{0.413320in}}%
\pgfpathlineto{\pgfqpoint{3.604387in}{0.413320in}}%
\pgfpathlineto{\pgfqpoint{3.601590in}{0.413320in}}%
\pgfpathlineto{\pgfqpoint{3.598998in}{0.413320in}}%
\pgfpathlineto{\pgfqpoint{3.596240in}{0.413320in}}%
\pgfpathlineto{\pgfqpoint{3.593620in}{0.413320in}}%
\pgfpathlineto{\pgfqpoint{3.590883in}{0.413320in}}%
\pgfpathlineto{\pgfqpoint{3.588258in}{0.413320in}}%
\pgfpathlineto{\pgfqpoint{3.585532in}{0.413320in}}%
\pgfpathlineto{\pgfqpoint{3.582851in}{0.413320in}}%
\pgfpathlineto{\pgfqpoint{3.580191in}{0.413320in}}%
\pgfpathlineto{\pgfqpoint{3.577487in}{0.413320in}}%
\pgfpathlineto{\pgfqpoint{3.574814in}{0.413320in}}%
\pgfpathlineto{\pgfqpoint{3.572126in}{0.413320in}}%
\pgfpathlineto{\pgfqpoint{3.569584in}{0.413320in}}%
\pgfpathlineto{\pgfqpoint{3.566774in}{0.413320in}}%
\pgfpathlineto{\pgfqpoint{3.564188in}{0.413320in}}%
\pgfpathlineto{\pgfqpoint{3.561420in}{0.413320in}}%
\pgfpathlineto{\pgfqpoint{3.558853in}{0.413320in}}%
\pgfpathlineto{\pgfqpoint{3.556061in}{0.413320in}}%
\pgfpathlineto{\pgfqpoint{3.553498in}{0.413320in}}%
\pgfpathlineto{\pgfqpoint{3.550713in}{0.413320in}}%
\pgfpathlineto{\pgfqpoint{3.548029in}{0.413320in}}%
\pgfpathlineto{\pgfqpoint{3.545349in}{0.413320in}}%
\pgfpathlineto{\pgfqpoint{3.542656in}{0.413320in}}%
\pgfpathlineto{\pgfqpoint{3.540093in}{0.413320in}}%
\pgfpathlineto{\pgfqpoint{3.537309in}{0.413320in}}%
\pgfpathlineto{\pgfqpoint{3.534783in}{0.413320in}}%
\pgfpathlineto{\pgfqpoint{3.531955in}{0.413320in}}%
\pgfpathlineto{\pgfqpoint{3.529327in}{0.413320in}}%
\pgfpathlineto{\pgfqpoint{3.526601in}{0.413320in}}%
\pgfpathlineto{\pgfqpoint{3.524041in}{0.413320in}}%
\pgfpathlineto{\pgfqpoint{3.521244in}{0.413320in}}%
\pgfpathlineto{\pgfqpoint{3.518565in}{0.413320in}}%
\pgfpathlineto{\pgfqpoint{3.515884in}{0.413320in}}%
\pgfpathlineto{\pgfqpoint{3.513209in}{0.413320in}}%
\pgfpathlineto{\pgfqpoint{3.510533in}{0.413320in}}%
\pgfpathlineto{\pgfqpoint{3.507840in}{0.413320in}}%
\pgfpathlineto{\pgfqpoint{3.505262in}{0.413320in}}%
\pgfpathlineto{\pgfqpoint{3.502488in}{0.413320in}}%
\pgfpathlineto{\pgfqpoint{3.499909in}{0.413320in}}%
\pgfpathlineto{\pgfqpoint{3.497139in}{0.413320in}}%
\pgfpathlineto{\pgfqpoint{3.494581in}{0.413320in}}%
\pgfpathlineto{\pgfqpoint{3.491783in}{0.413320in}}%
\pgfpathlineto{\pgfqpoint{3.489223in}{0.413320in}}%
\pgfpathlineto{\pgfqpoint{3.486442in}{0.413320in}}%
\pgfpathlineto{\pgfqpoint{3.483744in}{0.413320in}}%
\pgfpathlineto{\pgfqpoint{3.481072in}{0.413320in}}%
\pgfpathlineto{\pgfqpoint{3.478378in}{0.413320in}}%
\pgfpathlineto{\pgfqpoint{3.475821in}{0.413320in}}%
\pgfpathlineto{\pgfqpoint{3.473021in}{0.413320in}}%
\pgfpathlineto{\pgfqpoint{3.470466in}{0.413320in}}%
\pgfpathlineto{\pgfqpoint{3.467678in}{0.413320in}}%
\pgfpathlineto{\pgfqpoint{3.465072in}{0.413320in}}%
\pgfpathlineto{\pgfqpoint{3.462321in}{0.413320in}}%
\pgfpathlineto{\pgfqpoint{3.459695in}{0.413320in}}%
\pgfpathlineto{\pgfqpoint{3.456960in}{0.413320in}}%
\pgfpathlineto{\pgfqpoint{3.454285in}{0.413320in}}%
\pgfpathlineto{\pgfqpoint{3.451597in}{0.413320in}}%
\pgfpathlineto{\pgfqpoint{3.448926in}{0.413320in}}%
\pgfpathlineto{\pgfqpoint{3.446257in}{0.413320in}}%
\pgfpathlineto{\pgfqpoint{3.443574in}{0.413320in}}%
\pgfpathlineto{\pgfqpoint{3.440996in}{0.413320in}}%
\pgfpathlineto{\pgfqpoint{3.438210in}{0.413320in}}%
\pgfpathlineto{\pgfqpoint{3.435635in}{0.413320in}}%
\pgfpathlineto{\pgfqpoint{3.432851in}{0.413320in}}%
\pgfpathlineto{\pgfqpoint{3.430313in}{0.413320in}}%
\pgfpathlineto{\pgfqpoint{3.427501in}{0.413320in}}%
\pgfpathlineto{\pgfqpoint{3.424887in}{0.413320in}}%
\pgfpathlineto{\pgfqpoint{3.422142in}{0.413320in}}%
\pgfpathlineto{\pgfqpoint{3.419455in}{0.413320in}}%
\pgfpathlineto{\pgfqpoint{3.416780in}{0.413320in}}%
\pgfpathlineto{\pgfqpoint{3.414109in}{0.413320in}}%
\pgfpathlineto{\pgfqpoint{3.411431in}{0.413320in}}%
\pgfpathlineto{\pgfqpoint{3.408752in}{0.413320in}}%
\pgfpathlineto{\pgfqpoint{3.406202in}{0.413320in}}%
\pgfpathlineto{\pgfqpoint{3.403394in}{0.413320in}}%
\pgfpathlineto{\pgfqpoint{3.400783in}{0.413320in}}%
\pgfpathlineto{\pgfqpoint{3.398037in}{0.413320in}}%
\pgfpathlineto{\pgfqpoint{3.395461in}{0.413320in}}%
\pgfpathlineto{\pgfqpoint{3.392681in}{0.413320in}}%
\pgfpathlineto{\pgfqpoint{3.390102in}{0.413320in}}%
\pgfpathlineto{\pgfqpoint{3.387309in}{0.413320in}}%
\pgfpathlineto{\pgfqpoint{3.384647in}{0.413320in}}%
\pgfpathlineto{\pgfqpoint{3.381959in}{0.413320in}}%
\pgfpathlineto{\pgfqpoint{3.379290in}{0.413320in}}%
\pgfpathlineto{\pgfqpoint{3.376735in}{0.413320in}}%
\pgfpathlineto{\pgfqpoint{3.373921in}{0.413320in}}%
\pgfpathlineto{\pgfqpoint{3.371357in}{0.413320in}}%
\pgfpathlineto{\pgfqpoint{3.368577in}{0.413320in}}%
\pgfpathlineto{\pgfqpoint{3.365996in}{0.413320in}}%
\pgfpathlineto{\pgfqpoint{3.363221in}{0.413320in}}%
\pgfpathlineto{\pgfqpoint{3.360620in}{0.413320in}}%
\pgfpathlineto{\pgfqpoint{3.357862in}{0.413320in}}%
\pgfpathlineto{\pgfqpoint{3.355177in}{0.413320in}}%
\pgfpathlineto{\pgfqpoint{3.352505in}{0.413320in}}%
\pgfpathlineto{\pgfqpoint{3.349828in}{0.413320in}}%
\pgfpathlineto{\pgfqpoint{3.347139in}{0.413320in}}%
\pgfpathlineto{\pgfqpoint{3.344468in}{0.413320in}}%
\pgfpathlineto{\pgfqpoint{3.341893in}{0.413320in}}%
\pgfpathlineto{\pgfqpoint{3.339101in}{0.413320in}}%
\pgfpathlineto{\pgfqpoint{3.336541in}{0.413320in}}%
\pgfpathlineto{\pgfqpoint{3.333758in}{0.413320in}}%
\pgfpathlineto{\pgfqpoint{3.331183in}{0.413320in}}%
\pgfpathlineto{\pgfqpoint{3.328401in}{0.413320in}}%
\pgfpathlineto{\pgfqpoint{3.325860in}{0.413320in}}%
\pgfpathlineto{\pgfqpoint{3.323049in}{0.413320in}}%
\pgfpathlineto{\pgfqpoint{3.320366in}{0.413320in}}%
\pgfpathlineto{\pgfqpoint{3.317688in}{0.413320in}}%
\pgfpathlineto{\pgfqpoint{3.315008in}{0.413320in}}%
\pgfpathlineto{\pgfqpoint{3.312480in}{0.413320in}}%
\pgfpathlineto{\pgfqpoint{3.309652in}{0.413320in}}%
\pgfpathlineto{\pgfqpoint{3.307104in}{0.413320in}}%
\pgfpathlineto{\pgfqpoint{3.304295in}{0.413320in}}%
\pgfpathlineto{\pgfqpoint{3.301719in}{0.413320in}}%
\pgfpathlineto{\pgfqpoint{3.298937in}{0.413320in}}%
\pgfpathlineto{\pgfqpoint{3.296376in}{0.413320in}}%
\pgfpathlineto{\pgfqpoint{3.293574in}{0.413320in}}%
\pgfpathlineto{\pgfqpoint{3.290890in}{0.413320in}}%
\pgfpathlineto{\pgfqpoint{3.288225in}{0.413320in}}%
\pgfpathlineto{\pgfqpoint{3.285534in}{0.413320in}}%
\pgfpathlineto{\pgfqpoint{3.282870in}{0.413320in}}%
\pgfpathlineto{\pgfqpoint{3.280189in}{0.413320in}}%
\pgfpathlineto{\pgfqpoint{3.277603in}{0.413320in}}%
\pgfpathlineto{\pgfqpoint{3.274831in}{0.413320in}}%
\pgfpathlineto{\pgfqpoint{3.272254in}{0.413320in}}%
\pgfpathlineto{\pgfqpoint{3.269478in}{0.413320in}}%
\pgfpathlineto{\pgfqpoint{3.266849in}{0.413320in}}%
\pgfpathlineto{\pgfqpoint{3.264119in}{0.413320in}}%
\pgfpathlineto{\pgfqpoint{3.261594in}{0.413320in}}%
\pgfpathlineto{\pgfqpoint{3.258784in}{0.413320in}}%
\pgfpathlineto{\pgfqpoint{3.256083in}{0.413320in}}%
\pgfpathlineto{\pgfqpoint{3.253404in}{0.413320in}}%
\pgfpathlineto{\pgfqpoint{3.250716in}{0.413320in}}%
\pgfpathlineto{\pgfqpoint{3.248049in}{0.413320in}}%
\pgfpathlineto{\pgfqpoint{3.245363in}{0.413320in}}%
\pgfpathlineto{\pgfqpoint{3.242807in}{0.413320in}}%
\pgfpathlineto{\pgfqpoint{3.240010in}{0.413320in}}%
\pgfpathlineto{\pgfqpoint{3.237411in}{0.413320in}}%
\pgfpathlineto{\pgfqpoint{3.234658in}{0.413320in}}%
\pgfpathlineto{\pgfqpoint{3.232069in}{0.413320in}}%
\pgfpathlineto{\pgfqpoint{3.229310in}{0.413320in}}%
\pgfpathlineto{\pgfqpoint{3.226609in}{0.413320in}}%
\pgfpathlineto{\pgfqpoint{3.223942in}{0.413320in}}%
\pgfpathlineto{\pgfqpoint{3.221255in}{0.413320in}}%
\pgfpathlineto{\pgfqpoint{3.218586in}{0.413320in}}%
\pgfpathlineto{\pgfqpoint{3.215908in}{0.413320in}}%
\pgfpathlineto{\pgfqpoint{3.213342in}{0.413320in}}%
\pgfpathlineto{\pgfqpoint{3.210545in}{0.413320in}}%
\pgfpathlineto{\pgfqpoint{3.207984in}{0.413320in}}%
\pgfpathlineto{\pgfqpoint{3.205195in}{0.413320in}}%
\pgfpathlineto{\pgfqpoint{3.202562in}{0.413320in}}%
\pgfpathlineto{\pgfqpoint{3.199823in}{0.413320in}}%
\pgfpathlineto{\pgfqpoint{3.197226in}{0.413320in}}%
\pgfpathlineto{\pgfqpoint{3.194508in}{0.413320in}}%
\pgfpathlineto{\pgfqpoint{3.191796in}{0.413320in}}%
\pgfpathlineto{\pgfqpoint{3.189117in}{0.413320in}}%
\pgfpathlineto{\pgfqpoint{3.186440in}{0.413320in}}%
\pgfpathlineto{\pgfqpoint{3.183760in}{0.413320in}}%
\pgfpathlineto{\pgfqpoint{3.181089in}{0.413320in}}%
\pgfpathlineto{\pgfqpoint{3.178525in}{0.413320in}}%
\pgfpathlineto{\pgfqpoint{3.175724in}{0.413320in}}%
\pgfpathlineto{\pgfqpoint{3.173142in}{0.413320in}}%
\pgfpathlineto{\pgfqpoint{3.170375in}{0.413320in}}%
\pgfpathlineto{\pgfqpoint{3.167776in}{0.413320in}}%
\pgfpathlineto{\pgfqpoint{3.165019in}{0.413320in}}%
\pgfpathlineto{\pgfqpoint{3.162474in}{0.413320in}}%
\pgfpathlineto{\pgfqpoint{3.159675in}{0.413320in}}%
\pgfpathlineto{\pgfqpoint{3.156981in}{0.413320in}}%
\pgfpathlineto{\pgfqpoint{3.154327in}{0.413320in}}%
\pgfpathlineto{\pgfqpoint{3.151612in}{0.413320in}}%
\pgfpathlineto{\pgfqpoint{3.149057in}{0.413320in}}%
\pgfpathlineto{\pgfqpoint{3.146271in}{0.413320in}}%
\pgfpathlineto{\pgfqpoint{3.143740in}{0.413320in}}%
\pgfpathlineto{\pgfqpoint{3.140913in}{0.413320in}}%
\pgfpathlineto{\pgfqpoint{3.138375in}{0.413320in}}%
\pgfpathlineto{\pgfqpoint{3.135550in}{0.413320in}}%
\pgfpathlineto{\pgfqpoint{3.132946in}{0.413320in}}%
\pgfpathlineto{\pgfqpoint{3.130199in}{0.413320in}}%
\pgfpathlineto{\pgfqpoint{3.127512in}{0.413320in}}%
\pgfpathlineto{\pgfqpoint{3.124842in}{0.413320in}}%
\pgfpathlineto{\pgfqpoint{3.122163in}{0.413320in}}%
\pgfpathlineto{\pgfqpoint{3.119487in}{0.413320in}}%
\pgfpathlineto{\pgfqpoint{3.116807in}{0.413320in}}%
\pgfpathlineto{\pgfqpoint{3.114242in}{0.413320in}}%
\pgfpathlineto{\pgfqpoint{3.111451in}{0.413320in}}%
\pgfpathlineto{\pgfqpoint{3.108896in}{0.413320in}}%
\pgfpathlineto{\pgfqpoint{3.106094in}{0.413320in}}%
\pgfpathlineto{\pgfqpoint{3.103508in}{0.413320in}}%
\pgfpathlineto{\pgfqpoint{3.100737in}{0.413320in}}%
\pgfpathlineto{\pgfqpoint{3.098163in}{0.413320in}}%
\pgfpathlineto{\pgfqpoint{3.095388in}{0.413320in}}%
\pgfpathlineto{\pgfqpoint{3.092699in}{0.413320in}}%
\pgfpathlineto{\pgfqpoint{3.090023in}{0.413320in}}%
\pgfpathlineto{\pgfqpoint{3.087343in}{0.413320in}}%
\pgfpathlineto{\pgfqpoint{3.084671in}{0.413320in}}%
\pgfpathlineto{\pgfqpoint{3.081990in}{0.413320in}}%
\pgfpathlineto{\pgfqpoint{3.079381in}{0.413320in}}%
\pgfpathlineto{\pgfqpoint{3.076631in}{0.413320in}}%
\pgfpathlineto{\pgfqpoint{3.074056in}{0.413320in}}%
\pgfpathlineto{\pgfqpoint{3.071266in}{0.413320in}}%
\pgfpathlineto{\pgfqpoint{3.068709in}{0.413320in}}%
\pgfpathlineto{\pgfqpoint{3.065916in}{0.413320in}}%
\pgfpathlineto{\pgfqpoint{3.063230in}{0.413320in}}%
\pgfpathlineto{\pgfqpoint{3.060561in}{0.413320in}}%
\pgfpathlineto{\pgfqpoint{3.057884in}{0.413320in}}%
\pgfpathlineto{\pgfqpoint{3.055202in}{0.413320in}}%
\pgfpathlineto{\pgfqpoint{3.052526in}{0.413320in}}%
\pgfpathlineto{\pgfqpoint{3.049988in}{0.413320in}}%
\pgfpathlineto{\pgfqpoint{3.047157in}{0.413320in}}%
\pgfpathlineto{\pgfqpoint{3.044568in}{0.413320in}}%
\pgfpathlineto{\pgfqpoint{3.041813in}{0.413320in}}%
\pgfpathlineto{\pgfqpoint{3.039262in}{0.413320in}}%
\pgfpathlineto{\pgfqpoint{3.036456in}{0.413320in}}%
\pgfpathlineto{\pgfqpoint{3.033921in}{0.413320in}}%
\pgfpathlineto{\pgfqpoint{3.031091in}{0.413320in}}%
\pgfpathlineto{\pgfqpoint{3.028412in}{0.413320in}}%
\pgfpathlineto{\pgfqpoint{3.025803in}{0.413320in}}%
\pgfpathlineto{\pgfqpoint{3.023058in}{0.413320in}}%
\pgfpathlineto{\pgfqpoint{3.020382in}{0.413320in}}%
\pgfpathlineto{\pgfqpoint{3.017707in}{0.413320in}}%
\pgfpathlineto{\pgfqpoint{3.015097in}{0.413320in}}%
\pgfpathlineto{\pgfqpoint{3.012351in}{0.413320in}}%
\pgfpathlineto{\pgfqpoint{3.009784in}{0.413320in}}%
\pgfpathlineto{\pgfqpoint{3.006993in}{0.413320in}}%
\pgfpathlineto{\pgfqpoint{3.004419in}{0.413320in}}%
\pgfpathlineto{\pgfqpoint{3.001635in}{0.413320in}}%
\pgfpathlineto{\pgfqpoint{2.999103in}{0.413320in}}%
\pgfpathlineto{\pgfqpoint{2.996300in}{0.413320in}}%
\pgfpathlineto{\pgfqpoint{2.993595in}{0.413320in}}%
\pgfpathlineto{\pgfqpoint{2.990978in}{0.413320in}}%
\pgfpathlineto{\pgfqpoint{2.988238in}{0.413320in}}%
\pgfpathlineto{\pgfqpoint{2.985666in}{0.413320in}}%
\pgfpathlineto{\pgfqpoint{2.982885in}{0.413320in}}%
\pgfpathlineto{\pgfqpoint{2.980341in}{0.413320in}}%
\pgfpathlineto{\pgfqpoint{2.977517in}{0.413320in}}%
\pgfpathlineto{\pgfqpoint{2.974972in}{0.413320in}}%
\pgfpathlineto{\pgfqpoint{2.972177in}{0.413320in}}%
\pgfpathlineto{\pgfqpoint{2.969599in}{0.413320in}}%
\pgfpathlineto{\pgfqpoint{2.966812in}{0.413320in}}%
\pgfpathlineto{\pgfqpoint{2.964127in}{0.413320in}}%
\pgfpathlineto{\pgfqpoint{2.961460in}{0.413320in}}%
\pgfpathlineto{\pgfqpoint{2.958782in}{0.413320in}}%
\pgfpathlineto{\pgfqpoint{2.956103in}{0.413320in}}%
\pgfpathlineto{\pgfqpoint{2.953422in}{0.413320in}}%
\pgfpathlineto{\pgfqpoint{2.950884in}{0.413320in}}%
\pgfpathlineto{\pgfqpoint{2.948068in}{0.413320in}}%
\pgfpathlineto{\pgfqpoint{2.945461in}{0.413320in}}%
\pgfpathlineto{\pgfqpoint{2.942711in}{0.413320in}}%
\pgfpathlineto{\pgfqpoint{2.940120in}{0.413320in}}%
\pgfpathlineto{\pgfqpoint{2.937352in}{0.413320in}}%
\pgfpathlineto{\pgfqpoint{2.934759in}{0.413320in}}%
\pgfpathlineto{\pgfqpoint{2.932033in}{0.413320in}}%
\pgfpathlineto{\pgfqpoint{2.929321in}{0.413320in}}%
\pgfpathlineto{\pgfqpoint{2.926655in}{0.413320in}}%
\pgfpathlineto{\pgfqpoint{2.923963in}{0.413320in}}%
\pgfpathlineto{\pgfqpoint{2.921363in}{0.413320in}}%
\pgfpathlineto{\pgfqpoint{2.918606in}{0.413320in}}%
\pgfpathlineto{\pgfqpoint{2.916061in}{0.413320in}}%
\pgfpathlineto{\pgfqpoint{2.913243in}{0.413320in}}%
\pgfpathlineto{\pgfqpoint{2.910631in}{0.413320in}}%
\pgfpathlineto{\pgfqpoint{2.907882in}{0.413320in}}%
\pgfpathlineto{\pgfqpoint{2.905341in}{0.413320in}}%
\pgfpathlineto{\pgfqpoint{2.902535in}{0.413320in}}%
\pgfpathlineto{\pgfqpoint{2.899858in}{0.413320in}}%
\pgfpathlineto{\pgfqpoint{2.897179in}{0.413320in}}%
\pgfpathlineto{\pgfqpoint{2.894487in}{0.413320in}}%
\pgfpathlineto{\pgfqpoint{2.891809in}{0.413320in}}%
\pgfpathlineto{\pgfqpoint{2.889145in}{0.413320in}}%
\pgfpathlineto{\pgfqpoint{2.886578in}{0.413320in}}%
\pgfpathlineto{\pgfqpoint{2.883780in}{0.413320in}}%
\pgfpathlineto{\pgfqpoint{2.881254in}{0.413320in}}%
\pgfpathlineto{\pgfqpoint{2.878431in}{0.413320in}}%
\pgfpathlineto{\pgfqpoint{2.875882in}{0.413320in}}%
\pgfpathlineto{\pgfqpoint{2.873074in}{0.413320in}}%
\pgfpathlineto{\pgfqpoint{2.870475in}{0.413320in}}%
\pgfpathlineto{\pgfqpoint{2.867713in}{0.413320in}}%
\pgfpathlineto{\pgfqpoint{2.865031in}{0.413320in}}%
\pgfpathlineto{\pgfqpoint{2.862402in}{0.413320in}}%
\pgfpathlineto{\pgfqpoint{2.859668in}{0.413320in}}%
\pgfpathlineto{\pgfqpoint{2.857003in}{0.413320in}}%
\pgfpathlineto{\pgfqpoint{2.854325in}{0.413320in}}%
\pgfpathlineto{\pgfqpoint{2.851793in}{0.413320in}}%
\pgfpathlineto{\pgfqpoint{2.848960in}{0.413320in}}%
\pgfpathlineto{\pgfqpoint{2.846408in}{0.413320in}}%
\pgfpathlineto{\pgfqpoint{2.843611in}{0.413320in}}%
\pgfpathlineto{\pgfqpoint{2.841055in}{0.413320in}}%
\pgfpathlineto{\pgfqpoint{2.838254in}{0.413320in}}%
\pgfpathlineto{\pgfqpoint{2.835698in}{0.413320in}}%
\pgfpathlineto{\pgfqpoint{2.832894in}{0.413320in}}%
\pgfpathlineto{\pgfqpoint{2.830219in}{0.413320in}}%
\pgfpathlineto{\pgfqpoint{2.827567in}{0.413320in}}%
\pgfpathlineto{\pgfqpoint{2.824851in}{0.413320in}}%
\pgfpathlineto{\pgfqpoint{2.822303in}{0.413320in}}%
\pgfpathlineto{\pgfqpoint{2.819506in}{0.413320in}}%
\pgfpathlineto{\pgfqpoint{2.816867in}{0.413320in}}%
\pgfpathlineto{\pgfqpoint{2.814141in}{0.413320in}}%
\pgfpathlineto{\pgfqpoint{2.811597in}{0.413320in}}%
\pgfpathlineto{\pgfqpoint{2.808792in}{0.413320in}}%
\pgfpathlineto{\pgfqpoint{2.806175in}{0.413320in}}%
\pgfpathlineto{\pgfqpoint{2.803435in}{0.413320in}}%
\pgfpathlineto{\pgfqpoint{2.800756in}{0.413320in}}%
\pgfpathlineto{\pgfqpoint{2.798070in}{0.413320in}}%
\pgfpathlineto{\pgfqpoint{2.795398in}{0.413320in}}%
\pgfpathlineto{\pgfqpoint{2.792721in}{0.413320in}}%
\pgfpathlineto{\pgfqpoint{2.790044in}{0.413320in}}%
\pgfpathlineto{\pgfqpoint{2.787468in}{0.413320in}}%
\pgfpathlineto{\pgfqpoint{2.784687in}{0.413320in}}%
\pgfpathlineto{\pgfqpoint{2.782113in}{0.413320in}}%
\pgfpathlineto{\pgfqpoint{2.779330in}{0.413320in}}%
\pgfpathlineto{\pgfqpoint{2.776767in}{0.413320in}}%
\pgfpathlineto{\pgfqpoint{2.773972in}{0.413320in}}%
\pgfpathlineto{\pgfqpoint{2.771373in}{0.413320in}}%
\pgfpathlineto{\pgfqpoint{2.768617in}{0.413320in}}%
\pgfpathlineto{\pgfqpoint{2.765935in}{0.413320in}}%
\pgfpathlineto{\pgfqpoint{2.763253in}{0.413320in}}%
\pgfpathlineto{\pgfqpoint{2.760581in}{0.413320in}}%
\pgfpathlineto{\pgfqpoint{2.758028in}{0.413320in}}%
\pgfpathlineto{\pgfqpoint{2.755224in}{0.413320in}}%
\pgfpathlineto{\pgfqpoint{2.752614in}{0.413320in}}%
\pgfpathlineto{\pgfqpoint{2.749868in}{0.413320in}}%
\pgfpathlineto{\pgfqpoint{2.747260in}{0.413320in}}%
\pgfpathlineto{\pgfqpoint{2.744510in}{0.413320in}}%
\pgfpathlineto{\pgfqpoint{2.741928in}{0.413320in}}%
\pgfpathlineto{\pgfqpoint{2.739155in}{0.413320in}}%
\pgfpathlineto{\pgfqpoint{2.736476in}{0.413320in}}%
\pgfpathlineto{\pgfqpoint{2.733798in}{0.413320in}}%
\pgfpathlineto{\pgfqpoint{2.731119in}{0.413320in}}%
\pgfpathlineto{\pgfqpoint{2.728439in}{0.413320in}}%
\pgfpathlineto{\pgfqpoint{2.725760in}{0.413320in}}%
\pgfpathlineto{\pgfqpoint{2.723211in}{0.413320in}}%
\pgfpathlineto{\pgfqpoint{2.720404in}{0.413320in}}%
\pgfpathlineto{\pgfqpoint{2.717773in}{0.413320in}}%
\pgfpathlineto{\pgfqpoint{2.715036in}{0.413320in}}%
\pgfpathlineto{\pgfqpoint{2.712477in}{0.413320in}}%
\pgfpathlineto{\pgfqpoint{2.709683in}{0.413320in}}%
\pgfpathlineto{\pgfqpoint{2.707125in}{0.413320in}}%
\pgfpathlineto{\pgfqpoint{2.704326in}{0.413320in}}%
\pgfpathlineto{\pgfqpoint{2.701657in}{0.413320in}}%
\pgfpathlineto{\pgfqpoint{2.698968in}{0.413320in}}%
\pgfpathlineto{\pgfqpoint{2.696293in}{0.413320in}}%
\pgfpathlineto{\pgfqpoint{2.693611in}{0.413320in}}%
\pgfpathlineto{\pgfqpoint{2.690940in}{0.413320in}}%
\pgfpathlineto{\pgfqpoint{2.688328in}{0.413320in}}%
\pgfpathlineto{\pgfqpoint{2.685586in}{0.413320in}}%
\pgfpathlineto{\pgfqpoint{2.683009in}{0.413320in}}%
\pgfpathlineto{\pgfqpoint{2.680224in}{0.413320in}}%
\pgfpathlineto{\pgfqpoint{2.677650in}{0.413320in}}%
\pgfpathlineto{\pgfqpoint{2.674873in}{0.413320in}}%
\pgfpathlineto{\pgfqpoint{2.672301in}{0.413320in}}%
\pgfpathlineto{\pgfqpoint{2.669506in}{0.413320in}}%
\pgfpathlineto{\pgfqpoint{2.666836in}{0.413320in}}%
\pgfpathlineto{\pgfqpoint{2.664151in}{0.413320in}}%
\pgfpathlineto{\pgfqpoint{2.661481in}{0.413320in}}%
\pgfpathlineto{\pgfqpoint{2.658942in}{0.413320in}}%
\pgfpathlineto{\pgfqpoint{2.656124in}{0.413320in}}%
\pgfpathlineto{\pgfqpoint{2.653567in}{0.413320in}}%
\pgfpathlineto{\pgfqpoint{2.650767in}{0.413320in}}%
\pgfpathlineto{\pgfqpoint{2.648196in}{0.413320in}}%
\pgfpathlineto{\pgfqpoint{2.645408in}{0.413320in}}%
\pgfpathlineto{\pgfqpoint{2.642827in}{0.413320in}}%
\pgfpathlineto{\pgfqpoint{2.640053in}{0.413320in}}%
\pgfpathlineto{\pgfqpoint{2.637369in}{0.413320in}}%
\pgfpathlineto{\pgfqpoint{2.634700in}{0.413320in}}%
\pgfpathlineto{\pgfqpoint{2.632018in}{0.413320in}}%
\pgfpathlineto{\pgfqpoint{2.629340in}{0.413320in}}%
\pgfpathlineto{\pgfqpoint{2.626653in}{0.413320in}}%
\pgfpathlineto{\pgfqpoint{2.624077in}{0.413320in}}%
\pgfpathlineto{\pgfqpoint{2.621304in}{0.413320in}}%
\pgfpathlineto{\pgfqpoint{2.618773in}{0.413320in}}%
\pgfpathlineto{\pgfqpoint{2.615934in}{0.413320in}}%
\pgfpathlineto{\pgfqpoint{2.613393in}{0.413320in}}%
\pgfpathlineto{\pgfqpoint{2.610588in}{0.413320in}}%
\pgfpathlineto{\pgfqpoint{2.608004in}{0.413320in}}%
\pgfpathlineto{\pgfqpoint{2.605232in}{0.413320in}}%
\pgfpathlineto{\pgfqpoint{2.602557in}{0.413320in}}%
\pgfpathlineto{\pgfqpoint{2.599920in}{0.413320in}}%
\pgfpathlineto{\pgfqpoint{2.597196in}{0.413320in}}%
\pgfpathlineto{\pgfqpoint{2.594630in}{0.413320in}}%
\pgfpathlineto{\pgfqpoint{2.591842in}{0.413320in}}%
\pgfpathlineto{\pgfqpoint{2.589248in}{0.413320in}}%
\pgfpathlineto{\pgfqpoint{2.586484in}{0.413320in}}%
\pgfpathlineto{\pgfqpoint{2.583913in}{0.413320in}}%
\pgfpathlineto{\pgfqpoint{2.581129in}{0.413320in}}%
\pgfpathlineto{\pgfqpoint{2.578567in}{0.413320in}}%
\pgfpathlineto{\pgfqpoint{2.575779in}{0.413320in}}%
\pgfpathlineto{\pgfqpoint{2.573082in}{0.413320in}}%
\pgfpathlineto{\pgfqpoint{2.570411in}{0.413320in}}%
\pgfpathlineto{\pgfqpoint{2.567730in}{0.413320in}}%
\pgfpathlineto{\pgfqpoint{2.565045in}{0.413320in}}%
\pgfpathlineto{\pgfqpoint{2.562375in}{0.413320in}}%
\pgfpathlineto{\pgfqpoint{2.559790in}{0.413320in}}%
\pgfpathlineto{\pgfqpoint{2.557009in}{0.413320in}}%
\pgfpathlineto{\pgfqpoint{2.554493in}{0.413320in}}%
\pgfpathlineto{\pgfqpoint{2.551664in}{0.413320in}}%
\pgfpathlineto{\pgfqpoint{2.549114in}{0.413320in}}%
\pgfpathlineto{\pgfqpoint{2.546310in}{0.413320in}}%
\pgfpathlineto{\pgfqpoint{2.543765in}{0.413320in}}%
\pgfpathlineto{\pgfqpoint{2.540949in}{0.413320in}}%
\pgfpathlineto{\pgfqpoint{2.538274in}{0.413320in}}%
\pgfpathlineto{\pgfqpoint{2.535624in}{0.413320in}}%
\pgfpathlineto{\pgfqpoint{2.532917in}{0.413320in}}%
\pgfpathlineto{\pgfqpoint{2.530234in}{0.413320in}}%
\pgfpathlineto{\pgfqpoint{2.527560in}{0.413320in}}%
\pgfpathlineto{\pgfqpoint{2.524988in}{0.413320in}}%
\pgfpathlineto{\pgfqpoint{2.522197in}{0.413320in}}%
\pgfpathlineto{\pgfqpoint{2.519607in}{0.413320in}}%
\pgfpathlineto{\pgfqpoint{2.516845in}{0.413320in}}%
\pgfpathlineto{\pgfqpoint{2.514268in}{0.413320in}}%
\pgfpathlineto{\pgfqpoint{2.511478in}{0.413320in}}%
\pgfpathlineto{\pgfqpoint{2.508917in}{0.413320in}}%
\pgfpathlineto{\pgfqpoint{2.506163in}{0.413320in}}%
\pgfpathlineto{\pgfqpoint{2.503454in}{0.413320in}}%
\pgfpathlineto{\pgfqpoint{2.500801in}{0.413320in}}%
\pgfpathlineto{\pgfqpoint{2.498085in}{0.413320in}}%
\pgfpathlineto{\pgfqpoint{2.495542in}{0.413320in}}%
\pgfpathlineto{\pgfqpoint{2.492729in}{0.413320in}}%
\pgfpathlineto{\pgfqpoint{2.490183in}{0.413320in}}%
\pgfpathlineto{\pgfqpoint{2.487384in}{0.413320in}}%
\pgfpathlineto{\pgfqpoint{2.484870in}{0.413320in}}%
\pgfpathlineto{\pgfqpoint{2.482026in}{0.413320in}}%
\pgfpathlineto{\pgfqpoint{2.479420in}{0.413320in}}%
\pgfpathlineto{\pgfqpoint{2.476671in}{0.413320in}}%
\pgfpathlineto{\pgfqpoint{2.473989in}{0.413320in}}%
\pgfpathlineto{\pgfqpoint{2.471311in}{0.413320in}}%
\pgfpathlineto{\pgfqpoint{2.468635in}{0.413320in}}%
\pgfpathlineto{\pgfqpoint{2.465957in}{0.413320in}}%
\pgfpathlineto{\pgfqpoint{2.463280in}{0.413320in}}%
\pgfpathlineto{\pgfqpoint{2.460711in}{0.413320in}}%
\pgfpathlineto{\pgfqpoint{2.457917in}{0.413320in}}%
\pgfpathlineto{\pgfqpoint{2.455353in}{0.413320in}}%
\pgfpathlineto{\pgfqpoint{2.452562in}{0.413320in}}%
\pgfpathlineto{\pgfqpoint{2.450032in}{0.413320in}}%
\pgfpathlineto{\pgfqpoint{2.447209in}{0.413320in}}%
\pgfpathlineto{\pgfqpoint{2.444677in}{0.413320in}}%
\pgfpathlineto{\pgfqpoint{2.441876in}{0.413320in}}%
\pgfpathlineto{\pgfqpoint{2.439167in}{0.413320in}}%
\pgfpathlineto{\pgfqpoint{2.436518in}{0.413320in}}%
\pgfpathlineto{\pgfqpoint{2.433815in}{0.413320in}}%
\pgfpathlineto{\pgfqpoint{2.431251in}{0.413320in}}%
\pgfpathlineto{\pgfqpoint{2.428453in}{0.413320in}}%
\pgfpathlineto{\pgfqpoint{2.425878in}{0.413320in}}%
\pgfpathlineto{\pgfqpoint{2.423098in}{0.413320in}}%
\pgfpathlineto{\pgfqpoint{2.420528in}{0.413320in}}%
\pgfpathlineto{\pgfqpoint{2.417747in}{0.413320in}}%
\pgfpathlineto{\pgfqpoint{2.415184in}{0.413320in}}%
\pgfpathlineto{\pgfqpoint{2.412389in}{0.413320in}}%
\pgfpathlineto{\pgfqpoint{2.409699in}{0.413320in}}%
\pgfpathlineto{\pgfqpoint{2.407024in}{0.413320in}}%
\pgfpathlineto{\pgfqpoint{2.404352in}{0.413320in}}%
\pgfpathlineto{\pgfqpoint{2.401675in}{0.413320in}}%
\pgfpathlineto{\pgfqpoint{2.398995in}{0.413320in}}%
\pgfpathclose%
\pgfusepath{stroke,fill}%
\end{pgfscope}%
\begin{pgfscope}%
\pgfpathrectangle{\pgfqpoint{2.398995in}{0.319877in}}{\pgfqpoint{3.986877in}{1.993438in}} %
\pgfusepath{clip}%
\pgfsetbuttcap%
\pgfsetroundjoin%
\definecolor{currentfill}{rgb}{1.000000,1.000000,1.000000}%
\pgfsetfillcolor{currentfill}%
\pgfsetlinewidth{1.003750pt}%
\definecolor{currentstroke}{rgb}{0.194683,0.699828,0.304148}%
\pgfsetstrokecolor{currentstroke}%
\pgfsetdash{}{0pt}%
\pgfpathmoveto{\pgfqpoint{2.398995in}{0.413320in}}%
\pgfpathlineto{\pgfqpoint{2.398995in}{1.226116in}}%
\pgfpathlineto{\pgfqpoint{2.401675in}{1.231924in}}%
\pgfpathlineto{\pgfqpoint{2.404352in}{1.231214in}}%
\pgfpathlineto{\pgfqpoint{2.407024in}{1.233086in}}%
\pgfpathlineto{\pgfqpoint{2.409699in}{1.219812in}}%
\pgfpathlineto{\pgfqpoint{2.412389in}{1.220332in}}%
\pgfpathlineto{\pgfqpoint{2.415184in}{1.225368in}}%
\pgfpathlineto{\pgfqpoint{2.417747in}{1.225006in}}%
\pgfpathlineto{\pgfqpoint{2.420528in}{1.224803in}}%
\pgfpathlineto{\pgfqpoint{2.423098in}{1.220338in}}%
\pgfpathlineto{\pgfqpoint{2.425878in}{1.219782in}}%
\pgfpathlineto{\pgfqpoint{2.428453in}{1.230370in}}%
\pgfpathlineto{\pgfqpoint{2.431251in}{1.231260in}}%
\pgfpathlineto{\pgfqpoint{2.433815in}{1.220317in}}%
\pgfpathlineto{\pgfqpoint{2.436518in}{1.219763in}}%
\pgfpathlineto{\pgfqpoint{2.439167in}{1.222778in}}%
\pgfpathlineto{\pgfqpoint{2.441876in}{1.219702in}}%
\pgfpathlineto{\pgfqpoint{2.444677in}{1.219702in}}%
\pgfpathlineto{\pgfqpoint{2.447209in}{1.219702in}}%
\pgfpathlineto{\pgfqpoint{2.450032in}{1.222260in}}%
\pgfpathlineto{\pgfqpoint{2.452562in}{1.224221in}}%
\pgfpathlineto{\pgfqpoint{2.455353in}{1.219702in}}%
\pgfpathlineto{\pgfqpoint{2.457917in}{1.220652in}}%
\pgfpathlineto{\pgfqpoint{2.460711in}{1.223603in}}%
\pgfpathlineto{\pgfqpoint{2.463280in}{1.223939in}}%
\pgfpathlineto{\pgfqpoint{2.465957in}{1.219903in}}%
\pgfpathlineto{\pgfqpoint{2.468635in}{1.225832in}}%
\pgfpathlineto{\pgfqpoint{2.471311in}{1.226501in}}%
\pgfpathlineto{\pgfqpoint{2.473989in}{1.224122in}}%
\pgfpathlineto{\pgfqpoint{2.476671in}{1.225579in}}%
\pgfpathlineto{\pgfqpoint{2.479420in}{1.225891in}}%
\pgfpathlineto{\pgfqpoint{2.482026in}{1.229755in}}%
\pgfpathlineto{\pgfqpoint{2.484870in}{1.235397in}}%
\pgfpathlineto{\pgfqpoint{2.487384in}{1.232451in}}%
\pgfpathlineto{\pgfqpoint{2.490183in}{1.233626in}}%
\pgfpathlineto{\pgfqpoint{2.492729in}{1.229462in}}%
\pgfpathlineto{\pgfqpoint{2.495542in}{1.230052in}}%
\pgfpathlineto{\pgfqpoint{2.498085in}{1.231947in}}%
\pgfpathlineto{\pgfqpoint{2.500801in}{1.227396in}}%
\pgfpathlineto{\pgfqpoint{2.503454in}{1.225205in}}%
\pgfpathlineto{\pgfqpoint{2.506163in}{1.229954in}}%
\pgfpathlineto{\pgfqpoint{2.508917in}{1.235338in}}%
\pgfpathlineto{\pgfqpoint{2.511478in}{1.235186in}}%
\pgfpathlineto{\pgfqpoint{2.514268in}{1.235647in}}%
\pgfpathlineto{\pgfqpoint{2.516845in}{1.226699in}}%
\pgfpathlineto{\pgfqpoint{2.519607in}{1.228594in}}%
\pgfpathlineto{\pgfqpoint{2.522197in}{1.238304in}}%
\pgfpathlineto{\pgfqpoint{2.524988in}{1.244943in}}%
\pgfpathlineto{\pgfqpoint{2.527560in}{1.242949in}}%
\pgfpathlineto{\pgfqpoint{2.530234in}{1.244563in}}%
\pgfpathlineto{\pgfqpoint{2.532917in}{1.235814in}}%
\pgfpathlineto{\pgfqpoint{2.535624in}{1.232486in}}%
\pgfpathlineto{\pgfqpoint{2.538274in}{1.229352in}}%
\pgfpathlineto{\pgfqpoint{2.540949in}{1.229063in}}%
\pgfpathlineto{\pgfqpoint{2.543765in}{1.231433in}}%
\pgfpathlineto{\pgfqpoint{2.546310in}{1.231912in}}%
\pgfpathlineto{\pgfqpoint{2.549114in}{1.228562in}}%
\pgfpathlineto{\pgfqpoint{2.551664in}{1.225729in}}%
\pgfpathlineto{\pgfqpoint{2.554493in}{1.219702in}}%
\pgfpathlineto{\pgfqpoint{2.557009in}{1.219702in}}%
\pgfpathlineto{\pgfqpoint{2.559790in}{1.224978in}}%
\pgfpathlineto{\pgfqpoint{2.562375in}{1.222878in}}%
\pgfpathlineto{\pgfqpoint{2.565045in}{1.219702in}}%
\pgfpathlineto{\pgfqpoint{2.567730in}{1.225818in}}%
\pgfpathlineto{\pgfqpoint{2.570411in}{1.225633in}}%
\pgfpathlineto{\pgfqpoint{2.573082in}{1.230230in}}%
\pgfpathlineto{\pgfqpoint{2.575779in}{1.229753in}}%
\pgfpathlineto{\pgfqpoint{2.578567in}{1.219702in}}%
\pgfpathlineto{\pgfqpoint{2.581129in}{1.220054in}}%
\pgfpathlineto{\pgfqpoint{2.583913in}{1.227119in}}%
\pgfpathlineto{\pgfqpoint{2.586484in}{1.219702in}}%
\pgfpathlineto{\pgfqpoint{2.589248in}{1.222839in}}%
\pgfpathlineto{\pgfqpoint{2.591842in}{1.223776in}}%
\pgfpathlineto{\pgfqpoint{2.594630in}{1.220319in}}%
\pgfpathlineto{\pgfqpoint{2.597196in}{1.221387in}}%
\pgfpathlineto{\pgfqpoint{2.599920in}{1.219833in}}%
\pgfpathlineto{\pgfqpoint{2.602557in}{1.225171in}}%
\pgfpathlineto{\pgfqpoint{2.605232in}{1.220965in}}%
\pgfpathlineto{\pgfqpoint{2.608004in}{1.222939in}}%
\pgfpathlineto{\pgfqpoint{2.610588in}{1.225889in}}%
\pgfpathlineto{\pgfqpoint{2.613393in}{1.224972in}}%
\pgfpathlineto{\pgfqpoint{2.615934in}{1.226710in}}%
\pgfpathlineto{\pgfqpoint{2.618773in}{1.228207in}}%
\pgfpathlineto{\pgfqpoint{2.621304in}{1.224126in}}%
\pgfpathlineto{\pgfqpoint{2.624077in}{1.223535in}}%
\pgfpathlineto{\pgfqpoint{2.626653in}{1.233732in}}%
\pgfpathlineto{\pgfqpoint{2.629340in}{1.227723in}}%
\pgfpathlineto{\pgfqpoint{2.632018in}{1.231149in}}%
\pgfpathlineto{\pgfqpoint{2.634700in}{1.229300in}}%
\pgfpathlineto{\pgfqpoint{2.637369in}{1.230752in}}%
\pgfpathlineto{\pgfqpoint{2.640053in}{1.226734in}}%
\pgfpathlineto{\pgfqpoint{2.642827in}{1.227013in}}%
\pgfpathlineto{\pgfqpoint{2.645408in}{1.228893in}}%
\pgfpathlineto{\pgfqpoint{2.648196in}{1.225236in}}%
\pgfpathlineto{\pgfqpoint{2.650767in}{1.226565in}}%
\pgfpathlineto{\pgfqpoint{2.653567in}{1.225128in}}%
\pgfpathlineto{\pgfqpoint{2.656124in}{1.219750in}}%
\pgfpathlineto{\pgfqpoint{2.658942in}{1.223567in}}%
\pgfpathlineto{\pgfqpoint{2.661481in}{1.221795in}}%
\pgfpathlineto{\pgfqpoint{2.664151in}{1.221095in}}%
\pgfpathlineto{\pgfqpoint{2.666836in}{1.227760in}}%
\pgfpathlineto{\pgfqpoint{2.669506in}{1.228482in}}%
\pgfpathlineto{\pgfqpoint{2.672301in}{1.228153in}}%
\pgfpathlineto{\pgfqpoint{2.674873in}{1.231763in}}%
\pgfpathlineto{\pgfqpoint{2.677650in}{1.229166in}}%
\pgfpathlineto{\pgfqpoint{2.680224in}{1.231066in}}%
\pgfpathlineto{\pgfqpoint{2.683009in}{1.234344in}}%
\pgfpathlineto{\pgfqpoint{2.685586in}{1.230753in}}%
\pgfpathlineto{\pgfqpoint{2.688328in}{1.229135in}}%
\pgfpathlineto{\pgfqpoint{2.690940in}{1.229230in}}%
\pgfpathlineto{\pgfqpoint{2.693611in}{1.234637in}}%
\pgfpathlineto{\pgfqpoint{2.696293in}{1.231336in}}%
\pgfpathlineto{\pgfqpoint{2.698968in}{1.230512in}}%
\pgfpathlineto{\pgfqpoint{2.701657in}{1.231401in}}%
\pgfpathlineto{\pgfqpoint{2.704326in}{1.233442in}}%
\pgfpathlineto{\pgfqpoint{2.707125in}{1.228751in}}%
\pgfpathlineto{\pgfqpoint{2.709683in}{1.227769in}}%
\pgfpathlineto{\pgfqpoint{2.712477in}{1.223878in}}%
\pgfpathlineto{\pgfqpoint{2.715036in}{1.228824in}}%
\pgfpathlineto{\pgfqpoint{2.717773in}{1.226190in}}%
\pgfpathlineto{\pgfqpoint{2.720404in}{1.222911in}}%
\pgfpathlineto{\pgfqpoint{2.723211in}{1.226095in}}%
\pgfpathlineto{\pgfqpoint{2.725760in}{1.224810in}}%
\pgfpathlineto{\pgfqpoint{2.728439in}{1.225889in}}%
\pgfpathlineto{\pgfqpoint{2.731119in}{1.223231in}}%
\pgfpathlineto{\pgfqpoint{2.733798in}{1.219702in}}%
\pgfpathlineto{\pgfqpoint{2.736476in}{1.219702in}}%
\pgfpathlineto{\pgfqpoint{2.739155in}{1.219702in}}%
\pgfpathlineto{\pgfqpoint{2.741928in}{1.219702in}}%
\pgfpathlineto{\pgfqpoint{2.744510in}{1.222472in}}%
\pgfpathlineto{\pgfqpoint{2.747260in}{1.222637in}}%
\pgfpathlineto{\pgfqpoint{2.749868in}{1.225140in}}%
\pgfpathlineto{\pgfqpoint{2.752614in}{1.226880in}}%
\pgfpathlineto{\pgfqpoint{2.755224in}{1.225876in}}%
\pgfpathlineto{\pgfqpoint{2.758028in}{1.226664in}}%
\pgfpathlineto{\pgfqpoint{2.760581in}{1.226057in}}%
\pgfpathlineto{\pgfqpoint{2.763253in}{1.220892in}}%
\pgfpathlineto{\pgfqpoint{2.765935in}{1.219702in}}%
\pgfpathlineto{\pgfqpoint{2.768617in}{1.228657in}}%
\pgfpathlineto{\pgfqpoint{2.771373in}{1.219702in}}%
\pgfpathlineto{\pgfqpoint{2.773972in}{1.222946in}}%
\pgfpathlineto{\pgfqpoint{2.776767in}{1.219702in}}%
\pgfpathlineto{\pgfqpoint{2.779330in}{1.220365in}}%
\pgfpathlineto{\pgfqpoint{2.782113in}{1.223936in}}%
\pgfpathlineto{\pgfqpoint{2.784687in}{1.220892in}}%
\pgfpathlineto{\pgfqpoint{2.787468in}{1.220808in}}%
\pgfpathlineto{\pgfqpoint{2.790044in}{1.224216in}}%
\pgfpathlineto{\pgfqpoint{2.792721in}{1.225564in}}%
\pgfpathlineto{\pgfqpoint{2.795398in}{1.222220in}}%
\pgfpathlineto{\pgfqpoint{2.798070in}{1.226116in}}%
\pgfpathlineto{\pgfqpoint{2.800756in}{1.226228in}}%
\pgfpathlineto{\pgfqpoint{2.803435in}{1.227139in}}%
\pgfpathlineto{\pgfqpoint{2.806175in}{1.225143in}}%
\pgfpathlineto{\pgfqpoint{2.808792in}{1.232214in}}%
\pgfpathlineto{\pgfqpoint{2.811597in}{1.232929in}}%
\pgfpathlineto{\pgfqpoint{2.814141in}{1.234599in}}%
\pgfpathlineto{\pgfqpoint{2.816867in}{1.232409in}}%
\pgfpathlineto{\pgfqpoint{2.819506in}{1.229371in}}%
\pgfpathlineto{\pgfqpoint{2.822303in}{1.232249in}}%
\pgfpathlineto{\pgfqpoint{2.824851in}{1.237513in}}%
\pgfpathlineto{\pgfqpoint{2.827567in}{1.231864in}}%
\pgfpathlineto{\pgfqpoint{2.830219in}{1.223671in}}%
\pgfpathlineto{\pgfqpoint{2.832894in}{1.231409in}}%
\pgfpathlineto{\pgfqpoint{2.835698in}{1.227961in}}%
\pgfpathlineto{\pgfqpoint{2.838254in}{1.226887in}}%
\pgfpathlineto{\pgfqpoint{2.841055in}{1.225127in}}%
\pgfpathlineto{\pgfqpoint{2.843611in}{1.226218in}}%
\pgfpathlineto{\pgfqpoint{2.846408in}{1.221773in}}%
\pgfpathlineto{\pgfqpoint{2.848960in}{1.224135in}}%
\pgfpathlineto{\pgfqpoint{2.851793in}{1.229639in}}%
\pgfpathlineto{\pgfqpoint{2.854325in}{1.230583in}}%
\pgfpathlineto{\pgfqpoint{2.857003in}{1.229290in}}%
\pgfpathlineto{\pgfqpoint{2.859668in}{1.232290in}}%
\pgfpathlineto{\pgfqpoint{2.862402in}{1.233969in}}%
\pgfpathlineto{\pgfqpoint{2.865031in}{1.226403in}}%
\pgfpathlineto{\pgfqpoint{2.867713in}{1.236417in}}%
\pgfpathlineto{\pgfqpoint{2.870475in}{1.236892in}}%
\pgfpathlineto{\pgfqpoint{2.873074in}{1.236764in}}%
\pgfpathlineto{\pgfqpoint{2.875882in}{1.231424in}}%
\pgfpathlineto{\pgfqpoint{2.878431in}{1.226191in}}%
\pgfpathlineto{\pgfqpoint{2.881254in}{1.227018in}}%
\pgfpathlineto{\pgfqpoint{2.883780in}{1.236615in}}%
\pgfpathlineto{\pgfqpoint{2.886578in}{1.232471in}}%
\pgfpathlineto{\pgfqpoint{2.889145in}{1.229406in}}%
\pgfpathlineto{\pgfqpoint{2.891809in}{1.231150in}}%
\pgfpathlineto{\pgfqpoint{2.894487in}{1.230500in}}%
\pgfpathlineto{\pgfqpoint{2.897179in}{1.226227in}}%
\pgfpathlineto{\pgfqpoint{2.899858in}{1.226614in}}%
\pgfpathlineto{\pgfqpoint{2.902535in}{1.232417in}}%
\pgfpathlineto{\pgfqpoint{2.905341in}{1.232904in}}%
\pgfpathlineto{\pgfqpoint{2.907882in}{1.229600in}}%
\pgfpathlineto{\pgfqpoint{2.910631in}{1.230519in}}%
\pgfpathlineto{\pgfqpoint{2.913243in}{1.227227in}}%
\pgfpathlineto{\pgfqpoint{2.916061in}{1.229824in}}%
\pgfpathlineto{\pgfqpoint{2.918606in}{1.230523in}}%
\pgfpathlineto{\pgfqpoint{2.921363in}{1.230817in}}%
\pgfpathlineto{\pgfqpoint{2.923963in}{1.227616in}}%
\pgfpathlineto{\pgfqpoint{2.926655in}{1.228369in}}%
\pgfpathlineto{\pgfqpoint{2.929321in}{1.224397in}}%
\pgfpathlineto{\pgfqpoint{2.932033in}{1.223205in}}%
\pgfpathlineto{\pgfqpoint{2.934759in}{1.222936in}}%
\pgfpathlineto{\pgfqpoint{2.937352in}{1.225460in}}%
\pgfpathlineto{\pgfqpoint{2.940120in}{1.226971in}}%
\pgfpathlineto{\pgfqpoint{2.942711in}{1.230497in}}%
\pgfpathlineto{\pgfqpoint{2.945461in}{1.232370in}}%
\pgfpathlineto{\pgfqpoint{2.948068in}{1.228074in}}%
\pgfpathlineto{\pgfqpoint{2.950884in}{1.233093in}}%
\pgfpathlineto{\pgfqpoint{2.953422in}{1.227745in}}%
\pgfpathlineto{\pgfqpoint{2.956103in}{1.225512in}}%
\pgfpathlineto{\pgfqpoint{2.958782in}{1.229883in}}%
\pgfpathlineto{\pgfqpoint{2.961460in}{1.229616in}}%
\pgfpathlineto{\pgfqpoint{2.964127in}{1.229206in}}%
\pgfpathlineto{\pgfqpoint{2.966812in}{1.232180in}}%
\pgfpathlineto{\pgfqpoint{2.969599in}{1.228935in}}%
\pgfpathlineto{\pgfqpoint{2.972177in}{1.233252in}}%
\pgfpathlineto{\pgfqpoint{2.974972in}{1.230180in}}%
\pgfpathlineto{\pgfqpoint{2.977517in}{1.228555in}}%
\pgfpathlineto{\pgfqpoint{2.980341in}{1.227860in}}%
\pgfpathlineto{\pgfqpoint{2.982885in}{1.231924in}}%
\pgfpathlineto{\pgfqpoint{2.985666in}{1.231242in}}%
\pgfpathlineto{\pgfqpoint{2.988238in}{1.232407in}}%
\pgfpathlineto{\pgfqpoint{2.990978in}{1.232037in}}%
\pgfpathlineto{\pgfqpoint{2.993595in}{1.230789in}}%
\pgfpathlineto{\pgfqpoint{2.996300in}{1.226675in}}%
\pgfpathlineto{\pgfqpoint{2.999103in}{1.219702in}}%
\pgfpathlineto{\pgfqpoint{3.001635in}{1.219702in}}%
\pgfpathlineto{\pgfqpoint{3.004419in}{1.222170in}}%
\pgfpathlineto{\pgfqpoint{3.006993in}{1.225573in}}%
\pgfpathlineto{\pgfqpoint{3.009784in}{1.230234in}}%
\pgfpathlineto{\pgfqpoint{3.012351in}{1.228915in}}%
\pgfpathlineto{\pgfqpoint{3.015097in}{1.230421in}}%
\pgfpathlineto{\pgfqpoint{3.017707in}{1.231547in}}%
\pgfpathlineto{\pgfqpoint{3.020382in}{1.238792in}}%
\pgfpathlineto{\pgfqpoint{3.023058in}{1.245981in}}%
\pgfpathlineto{\pgfqpoint{3.025803in}{1.234900in}}%
\pgfpathlineto{\pgfqpoint{3.028412in}{1.221201in}}%
\pgfpathlineto{\pgfqpoint{3.031091in}{1.223992in}}%
\pgfpathlineto{\pgfqpoint{3.033921in}{1.223926in}}%
\pgfpathlineto{\pgfqpoint{3.036456in}{1.228903in}}%
\pgfpathlineto{\pgfqpoint{3.039262in}{1.228453in}}%
\pgfpathlineto{\pgfqpoint{3.041813in}{1.233815in}}%
\pgfpathlineto{\pgfqpoint{3.044568in}{1.236335in}}%
\pgfpathlineto{\pgfqpoint{3.047157in}{1.248220in}}%
\pgfpathlineto{\pgfqpoint{3.049988in}{1.253577in}}%
\pgfpathlineto{\pgfqpoint{3.052526in}{1.249322in}}%
\pgfpathlineto{\pgfqpoint{3.055202in}{1.250922in}}%
\pgfpathlineto{\pgfqpoint{3.057884in}{1.254974in}}%
\pgfpathlineto{\pgfqpoint{3.060561in}{1.260984in}}%
\pgfpathlineto{\pgfqpoint{3.063230in}{1.256342in}}%
\pgfpathlineto{\pgfqpoint{3.065916in}{1.256806in}}%
\pgfpathlineto{\pgfqpoint{3.068709in}{1.270369in}}%
\pgfpathlineto{\pgfqpoint{3.071266in}{1.260961in}}%
\pgfpathlineto{\pgfqpoint{3.074056in}{1.268114in}}%
\pgfpathlineto{\pgfqpoint{3.076631in}{1.259509in}}%
\pgfpathlineto{\pgfqpoint{3.079381in}{1.304324in}}%
\pgfpathlineto{\pgfqpoint{3.081990in}{1.296281in}}%
\pgfpathlineto{\pgfqpoint{3.084671in}{1.304815in}}%
\pgfpathlineto{\pgfqpoint{3.087343in}{1.309156in}}%
\pgfpathlineto{\pgfqpoint{3.090023in}{1.304561in}}%
\pgfpathlineto{\pgfqpoint{3.092699in}{1.290632in}}%
\pgfpathlineto{\pgfqpoint{3.095388in}{1.290976in}}%
\pgfpathlineto{\pgfqpoint{3.098163in}{1.281999in}}%
\pgfpathlineto{\pgfqpoint{3.100737in}{1.260607in}}%
\pgfpathlineto{\pgfqpoint{3.103508in}{1.266692in}}%
\pgfpathlineto{\pgfqpoint{3.106094in}{1.288831in}}%
\pgfpathlineto{\pgfqpoint{3.108896in}{1.292160in}}%
\pgfpathlineto{\pgfqpoint{3.111451in}{1.303914in}}%
\pgfpathlineto{\pgfqpoint{3.114242in}{1.293282in}}%
\pgfpathlineto{\pgfqpoint{3.116807in}{1.278361in}}%
\pgfpathlineto{\pgfqpoint{3.119487in}{1.269720in}}%
\pgfpathlineto{\pgfqpoint{3.122163in}{1.275798in}}%
\pgfpathlineto{\pgfqpoint{3.124842in}{1.273909in}}%
\pgfpathlineto{\pgfqpoint{3.127512in}{1.271151in}}%
\pgfpathlineto{\pgfqpoint{3.130199in}{1.257471in}}%
\pgfpathlineto{\pgfqpoint{3.132946in}{1.254691in}}%
\pgfpathlineto{\pgfqpoint{3.135550in}{1.246855in}}%
\pgfpathlineto{\pgfqpoint{3.138375in}{1.255676in}}%
\pgfpathlineto{\pgfqpoint{3.140913in}{1.290628in}}%
\pgfpathlineto{\pgfqpoint{3.143740in}{1.311384in}}%
\pgfpathlineto{\pgfqpoint{3.146271in}{1.312629in}}%
\pgfpathlineto{\pgfqpoint{3.149057in}{1.315807in}}%
\pgfpathlineto{\pgfqpoint{3.151612in}{1.312453in}}%
\pgfpathlineto{\pgfqpoint{3.154327in}{1.300323in}}%
\pgfpathlineto{\pgfqpoint{3.156981in}{1.294588in}}%
\pgfpathlineto{\pgfqpoint{3.159675in}{1.282031in}}%
\pgfpathlineto{\pgfqpoint{3.162474in}{1.268562in}}%
\pgfpathlineto{\pgfqpoint{3.165019in}{1.275939in}}%
\pgfpathlineto{\pgfqpoint{3.167776in}{1.261120in}}%
\pgfpathlineto{\pgfqpoint{3.170375in}{1.260058in}}%
\pgfpathlineto{\pgfqpoint{3.173142in}{1.247549in}}%
\pgfpathlineto{\pgfqpoint{3.175724in}{1.254827in}}%
\pgfpathlineto{\pgfqpoint{3.178525in}{1.257623in}}%
\pgfpathlineto{\pgfqpoint{3.181089in}{1.260340in}}%
\pgfpathlineto{\pgfqpoint{3.183760in}{1.251978in}}%
\pgfpathlineto{\pgfqpoint{3.186440in}{1.243732in}}%
\pgfpathlineto{\pgfqpoint{3.189117in}{1.248838in}}%
\pgfpathlineto{\pgfqpoint{3.191796in}{1.263410in}}%
\pgfpathlineto{\pgfqpoint{3.194508in}{1.284786in}}%
\pgfpathlineto{\pgfqpoint{3.197226in}{1.292891in}}%
\pgfpathlineto{\pgfqpoint{3.199823in}{1.292954in}}%
\pgfpathlineto{\pgfqpoint{3.202562in}{1.313524in}}%
\pgfpathlineto{\pgfqpoint{3.205195in}{1.293400in}}%
\pgfpathlineto{\pgfqpoint{3.207984in}{1.299785in}}%
\pgfpathlineto{\pgfqpoint{3.210545in}{1.302757in}}%
\pgfpathlineto{\pgfqpoint{3.213342in}{1.296076in}}%
\pgfpathlineto{\pgfqpoint{3.215908in}{1.291698in}}%
\pgfpathlineto{\pgfqpoint{3.218586in}{1.278311in}}%
\pgfpathlineto{\pgfqpoint{3.221255in}{1.271116in}}%
\pgfpathlineto{\pgfqpoint{3.223942in}{1.275622in}}%
\pgfpathlineto{\pgfqpoint{3.226609in}{1.264029in}}%
\pgfpathlineto{\pgfqpoint{3.229310in}{1.259726in}}%
\pgfpathlineto{\pgfqpoint{3.232069in}{1.268008in}}%
\pgfpathlineto{\pgfqpoint{3.234658in}{1.260435in}}%
\pgfpathlineto{\pgfqpoint{3.237411in}{1.256602in}}%
\pgfpathlineto{\pgfqpoint{3.240010in}{1.258382in}}%
\pgfpathlineto{\pgfqpoint{3.242807in}{1.250707in}}%
\pgfpathlineto{\pgfqpoint{3.245363in}{1.250084in}}%
\pgfpathlineto{\pgfqpoint{3.248049in}{1.248064in}}%
\pgfpathlineto{\pgfqpoint{3.250716in}{1.250386in}}%
\pgfpathlineto{\pgfqpoint{3.253404in}{1.244716in}}%
\pgfpathlineto{\pgfqpoint{3.256083in}{1.246159in}}%
\pgfpathlineto{\pgfqpoint{3.258784in}{1.242872in}}%
\pgfpathlineto{\pgfqpoint{3.261594in}{1.243699in}}%
\pgfpathlineto{\pgfqpoint{3.264119in}{1.241442in}}%
\pgfpathlineto{\pgfqpoint{3.266849in}{1.241304in}}%
\pgfpathlineto{\pgfqpoint{3.269478in}{1.232849in}}%
\pgfpathlineto{\pgfqpoint{3.272254in}{1.234109in}}%
\pgfpathlineto{\pgfqpoint{3.274831in}{1.236062in}}%
\pgfpathlineto{\pgfqpoint{3.277603in}{1.234883in}}%
\pgfpathlineto{\pgfqpoint{3.280189in}{1.234338in}}%
\pgfpathlineto{\pgfqpoint{3.282870in}{1.230886in}}%
\pgfpathlineto{\pgfqpoint{3.285534in}{1.225242in}}%
\pgfpathlineto{\pgfqpoint{3.288225in}{1.230451in}}%
\pgfpathlineto{\pgfqpoint{3.290890in}{1.225976in}}%
\pgfpathlineto{\pgfqpoint{3.293574in}{1.228895in}}%
\pgfpathlineto{\pgfqpoint{3.296376in}{1.227699in}}%
\pgfpathlineto{\pgfqpoint{3.298937in}{1.226781in}}%
\pgfpathlineto{\pgfqpoint{3.301719in}{1.230971in}}%
\pgfpathlineto{\pgfqpoint{3.304295in}{1.229626in}}%
\pgfpathlineto{\pgfqpoint{3.307104in}{1.227495in}}%
\pgfpathlineto{\pgfqpoint{3.309652in}{1.226808in}}%
\pgfpathlineto{\pgfqpoint{3.312480in}{1.230750in}}%
\pgfpathlineto{\pgfqpoint{3.315008in}{1.229187in}}%
\pgfpathlineto{\pgfqpoint{3.317688in}{1.224583in}}%
\pgfpathlineto{\pgfqpoint{3.320366in}{1.229098in}}%
\pgfpathlineto{\pgfqpoint{3.323049in}{1.222847in}}%
\pgfpathlineto{\pgfqpoint{3.325860in}{1.233225in}}%
\pgfpathlineto{\pgfqpoint{3.328401in}{1.227206in}}%
\pgfpathlineto{\pgfqpoint{3.331183in}{1.226757in}}%
\pgfpathlineto{\pgfqpoint{3.333758in}{1.228405in}}%
\pgfpathlineto{\pgfqpoint{3.336541in}{1.224206in}}%
\pgfpathlineto{\pgfqpoint{3.339101in}{1.221600in}}%
\pgfpathlineto{\pgfqpoint{3.341893in}{1.227236in}}%
\pgfpathlineto{\pgfqpoint{3.344468in}{1.227845in}}%
\pgfpathlineto{\pgfqpoint{3.347139in}{1.229096in}}%
\pgfpathlineto{\pgfqpoint{3.349828in}{1.232441in}}%
\pgfpathlineto{\pgfqpoint{3.352505in}{1.231836in}}%
\pgfpathlineto{\pgfqpoint{3.355177in}{1.224398in}}%
\pgfpathlineto{\pgfqpoint{3.357862in}{1.224425in}}%
\pgfpathlineto{\pgfqpoint{3.360620in}{1.220793in}}%
\pgfpathlineto{\pgfqpoint{3.363221in}{1.227974in}}%
\pgfpathlineto{\pgfqpoint{3.365996in}{1.228418in}}%
\pgfpathlineto{\pgfqpoint{3.368577in}{1.224344in}}%
\pgfpathlineto{\pgfqpoint{3.371357in}{1.229076in}}%
\pgfpathlineto{\pgfqpoint{3.373921in}{1.230394in}}%
\pgfpathlineto{\pgfqpoint{3.376735in}{1.228503in}}%
\pgfpathlineto{\pgfqpoint{3.379290in}{1.226534in}}%
\pgfpathlineto{\pgfqpoint{3.381959in}{1.229293in}}%
\pgfpathlineto{\pgfqpoint{3.384647in}{1.231714in}}%
\pgfpathlineto{\pgfqpoint{3.387309in}{1.230075in}}%
\pgfpathlineto{\pgfqpoint{3.390102in}{1.232388in}}%
\pgfpathlineto{\pgfqpoint{3.392681in}{1.234186in}}%
\pgfpathlineto{\pgfqpoint{3.395461in}{1.233204in}}%
\pgfpathlineto{\pgfqpoint{3.398037in}{1.232356in}}%
\pgfpathlineto{\pgfqpoint{3.400783in}{1.230329in}}%
\pgfpathlineto{\pgfqpoint{3.403394in}{1.233490in}}%
\pgfpathlineto{\pgfqpoint{3.406202in}{1.226906in}}%
\pgfpathlineto{\pgfqpoint{3.408752in}{1.226934in}}%
\pgfpathlineto{\pgfqpoint{3.411431in}{1.231459in}}%
\pgfpathlineto{\pgfqpoint{3.414109in}{1.227094in}}%
\pgfpathlineto{\pgfqpoint{3.416780in}{1.229104in}}%
\pgfpathlineto{\pgfqpoint{3.419455in}{1.227209in}}%
\pgfpathlineto{\pgfqpoint{3.422142in}{1.226373in}}%
\pgfpathlineto{\pgfqpoint{3.424887in}{1.226954in}}%
\pgfpathlineto{\pgfqpoint{3.427501in}{1.227128in}}%
\pgfpathlineto{\pgfqpoint{3.430313in}{1.230072in}}%
\pgfpathlineto{\pgfqpoint{3.432851in}{1.225845in}}%
\pgfpathlineto{\pgfqpoint{3.435635in}{1.228814in}}%
\pgfpathlineto{\pgfqpoint{3.438210in}{1.227168in}}%
\pgfpathlineto{\pgfqpoint{3.440996in}{1.228570in}}%
\pgfpathlineto{\pgfqpoint{3.443574in}{1.235487in}}%
\pgfpathlineto{\pgfqpoint{3.446257in}{1.226093in}}%
\pgfpathlineto{\pgfqpoint{3.448926in}{1.225116in}}%
\pgfpathlineto{\pgfqpoint{3.451597in}{1.224314in}}%
\pgfpathlineto{\pgfqpoint{3.454285in}{1.224703in}}%
\pgfpathlineto{\pgfqpoint{3.456960in}{1.220027in}}%
\pgfpathlineto{\pgfqpoint{3.459695in}{1.220415in}}%
\pgfpathlineto{\pgfqpoint{3.462321in}{1.223617in}}%
\pgfpathlineto{\pgfqpoint{3.465072in}{1.223753in}}%
\pgfpathlineto{\pgfqpoint{3.467678in}{1.219702in}}%
\pgfpathlineto{\pgfqpoint{3.470466in}{1.223358in}}%
\pgfpathlineto{\pgfqpoint{3.473021in}{1.227525in}}%
\pgfpathlineto{\pgfqpoint{3.475821in}{1.227486in}}%
\pgfpathlineto{\pgfqpoint{3.478378in}{1.222933in}}%
\pgfpathlineto{\pgfqpoint{3.481072in}{1.227293in}}%
\pgfpathlineto{\pgfqpoint{3.483744in}{1.227422in}}%
\pgfpathlineto{\pgfqpoint{3.486442in}{1.226244in}}%
\pgfpathlineto{\pgfqpoint{3.489223in}{1.225351in}}%
\pgfpathlineto{\pgfqpoint{3.491783in}{1.224173in}}%
\pgfpathlineto{\pgfqpoint{3.494581in}{1.225987in}}%
\pgfpathlineto{\pgfqpoint{3.497139in}{1.227005in}}%
\pgfpathlineto{\pgfqpoint{3.499909in}{1.227030in}}%
\pgfpathlineto{\pgfqpoint{3.502488in}{1.222410in}}%
\pgfpathlineto{\pgfqpoint{3.505262in}{1.236384in}}%
\pgfpathlineto{\pgfqpoint{3.507840in}{1.237642in}}%
\pgfpathlineto{\pgfqpoint{3.510533in}{1.223638in}}%
\pgfpathlineto{\pgfqpoint{3.513209in}{1.221722in}}%
\pgfpathlineto{\pgfqpoint{3.515884in}{1.225592in}}%
\pgfpathlineto{\pgfqpoint{3.518565in}{1.228334in}}%
\pgfpathlineto{\pgfqpoint{3.521244in}{1.231017in}}%
\pgfpathlineto{\pgfqpoint{3.524041in}{1.230786in}}%
\pgfpathlineto{\pgfqpoint{3.526601in}{1.234964in}}%
\pgfpathlineto{\pgfqpoint{3.529327in}{1.229907in}}%
\pgfpathlineto{\pgfqpoint{3.531955in}{1.231871in}}%
\pgfpathlineto{\pgfqpoint{3.534783in}{1.228290in}}%
\pgfpathlineto{\pgfqpoint{3.537309in}{1.230557in}}%
\pgfpathlineto{\pgfqpoint{3.540093in}{1.230012in}}%
\pgfpathlineto{\pgfqpoint{3.542656in}{1.232075in}}%
\pgfpathlineto{\pgfqpoint{3.545349in}{1.236296in}}%
\pgfpathlineto{\pgfqpoint{3.548029in}{1.238535in}}%
\pgfpathlineto{\pgfqpoint{3.550713in}{1.238232in}}%
\pgfpathlineto{\pgfqpoint{3.553498in}{1.239952in}}%
\pgfpathlineto{\pgfqpoint{3.556061in}{1.233378in}}%
\pgfpathlineto{\pgfqpoint{3.558853in}{1.231251in}}%
\pgfpathlineto{\pgfqpoint{3.561420in}{1.226856in}}%
\pgfpathlineto{\pgfqpoint{3.564188in}{1.226808in}}%
\pgfpathlineto{\pgfqpoint{3.566774in}{1.225903in}}%
\pgfpathlineto{\pgfqpoint{3.569584in}{1.229014in}}%
\pgfpathlineto{\pgfqpoint{3.572126in}{1.228278in}}%
\pgfpathlineto{\pgfqpoint{3.574814in}{1.232658in}}%
\pgfpathlineto{\pgfqpoint{3.577487in}{1.236147in}}%
\pgfpathlineto{\pgfqpoint{3.580191in}{1.239165in}}%
\pgfpathlineto{\pgfqpoint{3.582851in}{1.231280in}}%
\pgfpathlineto{\pgfqpoint{3.585532in}{1.228599in}}%
\pgfpathlineto{\pgfqpoint{3.588258in}{1.233365in}}%
\pgfpathlineto{\pgfqpoint{3.590883in}{1.224435in}}%
\pgfpathlineto{\pgfqpoint{3.593620in}{1.230119in}}%
\pgfpathlineto{\pgfqpoint{3.596240in}{1.227993in}}%
\pgfpathlineto{\pgfqpoint{3.598998in}{1.225957in}}%
\pgfpathlineto{\pgfqpoint{3.601590in}{1.227511in}}%
\pgfpathlineto{\pgfqpoint{3.604387in}{1.226623in}}%
\pgfpathlineto{\pgfqpoint{3.606951in}{1.231579in}}%
\pgfpathlineto{\pgfqpoint{3.609632in}{1.231377in}}%
\pgfpathlineto{\pgfqpoint{3.612311in}{1.231801in}}%
\pgfpathlineto{\pgfqpoint{3.614982in}{1.229890in}}%
\pgfpathlineto{\pgfqpoint{3.617667in}{1.228177in}}%
\pgfpathlineto{\pgfqpoint{3.620345in}{1.227609in}}%
\pgfpathlineto{\pgfqpoint{3.623165in}{1.230877in}}%
\pgfpathlineto{\pgfqpoint{3.625689in}{1.228743in}}%
\pgfpathlineto{\pgfqpoint{3.628460in}{1.229475in}}%
\pgfpathlineto{\pgfqpoint{3.631058in}{1.228148in}}%
\pgfpathlineto{\pgfqpoint{3.633858in}{1.227981in}}%
\pgfpathlineto{\pgfqpoint{3.636413in}{1.228417in}}%
\pgfpathlineto{\pgfqpoint{3.639207in}{1.229636in}}%
\pgfpathlineto{\pgfqpoint{3.641773in}{1.227942in}}%
\pgfpathlineto{\pgfqpoint{3.644452in}{1.225213in}}%
\pgfpathlineto{\pgfqpoint{3.647130in}{1.225365in}}%
\pgfpathlineto{\pgfqpoint{3.649837in}{1.223382in}}%
\pgfpathlineto{\pgfqpoint{3.652628in}{1.219785in}}%
\pgfpathlineto{\pgfqpoint{3.655165in}{1.219702in}}%
\pgfpathlineto{\pgfqpoint{3.657917in}{1.232034in}}%
\pgfpathlineto{\pgfqpoint{3.660515in}{1.250986in}}%
\pgfpathlineto{\pgfqpoint{3.663276in}{1.271626in}}%
\pgfpathlineto{\pgfqpoint{3.665864in}{1.272345in}}%
\pgfpathlineto{\pgfqpoint{3.668665in}{1.271307in}}%
\pgfpathlineto{\pgfqpoint{3.671232in}{1.260556in}}%
\pgfpathlineto{\pgfqpoint{3.673911in}{1.250457in}}%
\pgfpathlineto{\pgfqpoint{3.676591in}{1.262991in}}%
\pgfpathlineto{\pgfqpoint{3.679273in}{1.274141in}}%
\pgfpathlineto{\pgfqpoint{3.681948in}{1.283762in}}%
\pgfpathlineto{\pgfqpoint{3.684620in}{1.293691in}}%
\pgfpathlineto{\pgfqpoint{3.687442in}{1.279575in}}%
\pgfpathlineto{\pgfqpoint{3.689983in}{1.274876in}}%
\pgfpathlineto{\pgfqpoint{3.692765in}{1.267832in}}%
\pgfpathlineto{\pgfqpoint{3.695331in}{1.260836in}}%
\pgfpathlineto{\pgfqpoint{3.698125in}{1.257950in}}%
\pgfpathlineto{\pgfqpoint{3.700684in}{1.250634in}}%
\pgfpathlineto{\pgfqpoint{3.703460in}{1.247893in}}%
\pgfpathlineto{\pgfqpoint{3.706053in}{1.248737in}}%
\pgfpathlineto{\pgfqpoint{3.708729in}{1.273002in}}%
\pgfpathlineto{\pgfqpoint{3.711410in}{1.267256in}}%
\pgfpathlineto{\pgfqpoint{3.714086in}{1.265949in}}%
\pgfpathlineto{\pgfqpoint{3.716875in}{1.258633in}}%
\pgfpathlineto{\pgfqpoint{3.719446in}{1.249157in}}%
\pgfpathlineto{\pgfqpoint{3.722228in}{1.245756in}}%
\pgfpathlineto{\pgfqpoint{3.724804in}{1.252570in}}%
\pgfpathlineto{\pgfqpoint{3.727581in}{1.246814in}}%
\pgfpathlineto{\pgfqpoint{3.730158in}{1.248275in}}%
\pgfpathlineto{\pgfqpoint{3.732950in}{1.243568in}}%
\pgfpathlineto{\pgfqpoint{3.735509in}{1.243505in}}%
\pgfpathlineto{\pgfqpoint{3.738194in}{1.237647in}}%
\pgfpathlineto{\pgfqpoint{3.740874in}{1.240792in}}%
\pgfpathlineto{\pgfqpoint{3.743548in}{1.245234in}}%
\pgfpathlineto{\pgfqpoint{3.746229in}{1.241636in}}%
\pgfpathlineto{\pgfqpoint{3.748903in}{1.238962in}}%
\pgfpathlineto{\pgfqpoint{3.751728in}{1.237880in}}%
\pgfpathlineto{\pgfqpoint{3.754265in}{1.235905in}}%
\pgfpathlineto{\pgfqpoint{3.757065in}{1.232725in}}%
\pgfpathlineto{\pgfqpoint{3.759622in}{1.231233in}}%
\pgfpathlineto{\pgfqpoint{3.762389in}{1.234851in}}%
\pgfpathlineto{\pgfqpoint{3.764966in}{1.252091in}}%
\pgfpathlineto{\pgfqpoint{3.767782in}{1.259459in}}%
\pgfpathlineto{\pgfqpoint{3.770323in}{1.242979in}}%
\pgfpathlineto{\pgfqpoint{3.773014in}{1.251553in}}%
\pgfpathlineto{\pgfqpoint{3.775691in}{1.261397in}}%
\pgfpathlineto{\pgfqpoint{3.778370in}{1.265017in}}%
\pgfpathlineto{\pgfqpoint{3.781046in}{1.257698in}}%
\pgfpathlineto{\pgfqpoint{3.783725in}{1.247746in}}%
\pgfpathlineto{\pgfqpoint{3.786504in}{1.245419in}}%
\pgfpathlineto{\pgfqpoint{3.789084in}{1.244852in}}%
\pgfpathlineto{\pgfqpoint{3.791897in}{1.244978in}}%
\pgfpathlineto{\pgfqpoint{3.794435in}{1.237743in}}%
\pgfpathlineto{\pgfqpoint{3.797265in}{1.235684in}}%
\pgfpathlineto{\pgfqpoint{3.799797in}{1.241163in}}%
\pgfpathlineto{\pgfqpoint{3.802569in}{1.242385in}}%
\pgfpathlineto{\pgfqpoint{3.805145in}{1.235369in}}%
\pgfpathlineto{\pgfqpoint{3.807832in}{1.234026in}}%
\pgfpathlineto{\pgfqpoint{3.810510in}{1.234025in}}%
\pgfpathlineto{\pgfqpoint{3.813172in}{1.240189in}}%
\pgfpathlineto{\pgfqpoint{3.815983in}{1.244556in}}%
\pgfpathlineto{\pgfqpoint{3.818546in}{1.240321in}}%
\pgfpathlineto{\pgfqpoint{3.821315in}{1.235301in}}%
\pgfpathlineto{\pgfqpoint{3.823903in}{1.227080in}}%
\pgfpathlineto{\pgfqpoint{3.826679in}{1.226236in}}%
\pgfpathlineto{\pgfqpoint{3.829252in}{1.222806in}}%
\pgfpathlineto{\pgfqpoint{3.832053in}{1.231512in}}%
\pgfpathlineto{\pgfqpoint{3.834616in}{1.231670in}}%
\pgfpathlineto{\pgfqpoint{3.837286in}{1.232825in}}%
\pgfpathlineto{\pgfqpoint{3.839960in}{1.227504in}}%
\pgfpathlineto{\pgfqpoint{3.842641in}{1.224997in}}%
\pgfpathlineto{\pgfqpoint{3.845329in}{1.228981in}}%
\pgfpathlineto{\pgfqpoint{3.848005in}{1.231615in}}%
\pgfpathlineto{\pgfqpoint{3.850814in}{1.236373in}}%
\pgfpathlineto{\pgfqpoint{3.853358in}{1.226156in}}%
\pgfpathlineto{\pgfqpoint{3.856100in}{1.230912in}}%
\pgfpathlineto{\pgfqpoint{3.858720in}{1.234922in}}%
\pgfpathlineto{\pgfqpoint{3.861561in}{1.232210in}}%
\pgfpathlineto{\pgfqpoint{3.864073in}{1.237283in}}%
\pgfpathlineto{\pgfqpoint{3.866815in}{1.231177in}}%
\pgfpathlineto{\pgfqpoint{3.869435in}{1.231285in}}%
\pgfpathlineto{\pgfqpoint{3.872114in}{1.236093in}}%
\pgfpathlineto{\pgfqpoint{3.874790in}{1.231134in}}%
\pgfpathlineto{\pgfqpoint{3.877466in}{1.229610in}}%
\pgfpathlineto{\pgfqpoint{3.880237in}{1.227691in}}%
\pgfpathlineto{\pgfqpoint{3.882850in}{1.227168in}}%
\pgfpathlineto{\pgfqpoint{3.885621in}{1.227491in}}%
\pgfpathlineto{\pgfqpoint{3.888188in}{1.227139in}}%
\pgfpathlineto{\pgfqpoint{3.890926in}{1.224449in}}%
\pgfpathlineto{\pgfqpoint{3.893541in}{1.220189in}}%
\pgfpathlineto{\pgfqpoint{3.896345in}{1.221816in}}%
\pgfpathlineto{\pgfqpoint{3.898891in}{1.227749in}}%
\pgfpathlineto{\pgfqpoint{3.901573in}{1.228427in}}%
\pgfpathlineto{\pgfqpoint{3.904252in}{1.229985in}}%
\pgfpathlineto{\pgfqpoint{3.906918in}{1.241633in}}%
\pgfpathlineto{\pgfqpoint{3.909602in}{1.228169in}}%
\pgfpathlineto{\pgfqpoint{3.912296in}{1.229784in}}%
\pgfpathlineto{\pgfqpoint{3.915107in}{1.231680in}}%
\pgfpathlineto{\pgfqpoint{3.917646in}{1.232146in}}%
\pgfpathlineto{\pgfqpoint{3.920412in}{1.235543in}}%
\pgfpathlineto{\pgfqpoint{3.923005in}{1.229944in}}%
\pgfpathlineto{\pgfqpoint{3.925778in}{1.232252in}}%
\pgfpathlineto{\pgfqpoint{3.928347in}{1.228781in}}%
\pgfpathlineto{\pgfqpoint{3.931202in}{1.230697in}}%
\pgfpathlineto{\pgfqpoint{3.933714in}{1.225919in}}%
\pgfpathlineto{\pgfqpoint{3.936395in}{1.224462in}}%
\pgfpathlineto{\pgfqpoint{3.939075in}{1.228312in}}%
\pgfpathlineto{\pgfqpoint{3.941778in}{1.229826in}}%
\pgfpathlineto{\pgfqpoint{3.944431in}{1.230287in}}%
\pgfpathlineto{\pgfqpoint{3.947101in}{1.223876in}}%
\pgfpathlineto{\pgfqpoint{3.949894in}{1.219702in}}%
\pgfpathlineto{\pgfqpoint{3.952464in}{1.222411in}}%
\pgfpathlineto{\pgfqpoint{3.955211in}{1.228042in}}%
\pgfpathlineto{\pgfqpoint{3.957823in}{1.227048in}}%
\pgfpathlineto{\pgfqpoint{3.960635in}{1.232449in}}%
\pgfpathlineto{\pgfqpoint{3.963176in}{1.227758in}}%
\pgfpathlineto{\pgfqpoint{3.966013in}{1.224910in}}%
\pgfpathlineto{\pgfqpoint{3.968523in}{1.219702in}}%
\pgfpathlineto{\pgfqpoint{3.971250in}{1.221848in}}%
\pgfpathlineto{\pgfqpoint{3.973885in}{1.227071in}}%
\pgfpathlineto{\pgfqpoint{3.976563in}{1.233758in}}%
\pgfpathlineto{\pgfqpoint{3.979389in}{1.231774in}}%
\pgfpathlineto{\pgfqpoint{3.981929in}{1.228439in}}%
\pgfpathlineto{\pgfqpoint{3.984714in}{1.227032in}}%
\pgfpathlineto{\pgfqpoint{3.987270in}{1.226230in}}%
\pgfpathlineto{\pgfqpoint{3.990055in}{1.226987in}}%
\pgfpathlineto{\pgfqpoint{3.992642in}{1.231269in}}%
\pgfpathlineto{\pgfqpoint{3.995417in}{1.225043in}}%
\pgfpathlineto{\pgfqpoint{3.997990in}{1.225548in}}%
\pgfpathlineto{\pgfqpoint{4.000674in}{1.233095in}}%
\pgfpathlineto{\pgfqpoint{4.003348in}{1.233321in}}%
\pgfpathlineto{\pgfqpoint{4.006034in}{1.227782in}}%
\pgfpathlineto{\pgfqpoint{4.008699in}{1.227197in}}%
\pgfpathlineto{\pgfqpoint{4.011394in}{1.231797in}}%
\pgfpathlineto{\pgfqpoint{4.014186in}{1.229577in}}%
\pgfpathlineto{\pgfqpoint{4.016744in}{1.231719in}}%
\pgfpathlineto{\pgfqpoint{4.019518in}{1.231384in}}%
\pgfpathlineto{\pgfqpoint{4.022097in}{1.229404in}}%
\pgfpathlineto{\pgfqpoint{4.024868in}{1.232671in}}%
\pgfpathlineto{\pgfqpoint{4.027447in}{1.228061in}}%
\pgfpathlineto{\pgfqpoint{4.030229in}{1.229406in}}%
\pgfpathlineto{\pgfqpoint{4.032817in}{1.231142in}}%
\pgfpathlineto{\pgfqpoint{4.035492in}{1.224852in}}%
\pgfpathlineto{\pgfqpoint{4.038174in}{1.225662in}}%
\pgfpathlineto{\pgfqpoint{4.040852in}{1.226468in}}%
\pgfpathlineto{\pgfqpoint{4.043667in}{1.226952in}}%
\pgfpathlineto{\pgfqpoint{4.046210in}{1.223498in}}%
\pgfpathlineto{\pgfqpoint{4.049006in}{1.226809in}}%
\pgfpathlineto{\pgfqpoint{4.051557in}{1.227071in}}%
\pgfpathlineto{\pgfqpoint{4.054326in}{1.229786in}}%
\pgfpathlineto{\pgfqpoint{4.056911in}{1.219702in}}%
\pgfpathlineto{\pgfqpoint{4.059702in}{1.219702in}}%
\pgfpathlineto{\pgfqpoint{4.062266in}{1.223051in}}%
\pgfpathlineto{\pgfqpoint{4.064957in}{1.226259in}}%
\pgfpathlineto{\pgfqpoint{4.067636in}{1.225340in}}%
\pgfpathlineto{\pgfqpoint{4.070313in}{1.225435in}}%
\pgfpathlineto{\pgfqpoint{4.072985in}{1.227264in}}%
\pgfpathlineto{\pgfqpoint{4.075705in}{1.226874in}}%
\pgfpathlineto{\pgfqpoint{4.078471in}{1.226491in}}%
\pgfpathlineto{\pgfqpoint{4.081018in}{1.227958in}}%
\pgfpathlineto{\pgfqpoint{4.083870in}{1.226436in}}%
\pgfpathlineto{\pgfqpoint{4.086385in}{1.230614in}}%
\pgfpathlineto{\pgfqpoint{4.089159in}{1.233009in}}%
\pgfpathlineto{\pgfqpoint{4.091729in}{1.232194in}}%
\pgfpathlineto{\pgfqpoint{4.094527in}{1.230628in}}%
\pgfpathlineto{\pgfqpoint{4.097092in}{1.229644in}}%
\pgfpathlineto{\pgfqpoint{4.099777in}{1.234297in}}%
\pgfpathlineto{\pgfqpoint{4.102456in}{1.229316in}}%
\pgfpathlineto{\pgfqpoint{4.105185in}{1.231158in}}%
\pgfpathlineto{\pgfqpoint{4.107814in}{1.228955in}}%
\pgfpathlineto{\pgfqpoint{4.110488in}{1.232345in}}%
\pgfpathlineto{\pgfqpoint{4.113252in}{1.230099in}}%
\pgfpathlineto{\pgfqpoint{4.115844in}{1.228421in}}%
\pgfpathlineto{\pgfqpoint{4.118554in}{1.228711in}}%
\pgfpathlineto{\pgfqpoint{4.121205in}{1.227555in}}%
\pgfpathlineto{\pgfqpoint{4.124019in}{1.228362in}}%
\pgfpathlineto{\pgfqpoint{4.126553in}{1.222626in}}%
\pgfpathlineto{\pgfqpoint{4.129349in}{1.227973in}}%
\pgfpathlineto{\pgfqpoint{4.131920in}{1.231807in}}%
\pgfpathlineto{\pgfqpoint{4.134615in}{1.230800in}}%
\pgfpathlineto{\pgfqpoint{4.137272in}{1.231041in}}%
\pgfpathlineto{\pgfqpoint{4.139963in}{1.222087in}}%
\pgfpathlineto{\pgfqpoint{4.142713in}{1.224278in}}%
\pgfpathlineto{\pgfqpoint{4.145310in}{1.223154in}}%
\pgfpathlineto{\pgfqpoint{4.148082in}{1.222861in}}%
\pgfpathlineto{\pgfqpoint{4.150665in}{1.224711in}}%
\pgfpathlineto{\pgfqpoint{4.153423in}{1.225594in}}%
\pgfpathlineto{\pgfqpoint{4.156016in}{1.224982in}}%
\pgfpathlineto{\pgfqpoint{4.158806in}{1.229941in}}%
\pgfpathlineto{\pgfqpoint{4.161380in}{1.225735in}}%
\pgfpathlineto{\pgfqpoint{4.164059in}{1.225259in}}%
\pgfpathlineto{\pgfqpoint{4.166737in}{1.225002in}}%
\pgfpathlineto{\pgfqpoint{4.169415in}{1.220838in}}%
\pgfpathlineto{\pgfqpoint{4.172093in}{1.223010in}}%
\pgfpathlineto{\pgfqpoint{4.174770in}{1.224545in}}%
\pgfpathlineto{\pgfqpoint{4.177593in}{1.228256in}}%
\pgfpathlineto{\pgfqpoint{4.180129in}{1.226965in}}%
\pgfpathlineto{\pgfqpoint{4.182899in}{1.222982in}}%
\pgfpathlineto{\pgfqpoint{4.185481in}{1.221963in}}%
\pgfpathlineto{\pgfqpoint{4.188318in}{1.225574in}}%
\pgfpathlineto{\pgfqpoint{4.190842in}{1.223635in}}%
\pgfpathlineto{\pgfqpoint{4.193638in}{1.228599in}}%
\pgfpathlineto{\pgfqpoint{4.196186in}{1.230248in}}%
\pgfpathlineto{\pgfqpoint{4.198878in}{1.226619in}}%
\pgfpathlineto{\pgfqpoint{4.201542in}{1.229007in}}%
\pgfpathlineto{\pgfqpoint{4.204240in}{1.233227in}}%
\pgfpathlineto{\pgfqpoint{4.207076in}{1.230448in}}%
\pgfpathlineto{\pgfqpoint{4.209597in}{1.240890in}}%
\pgfpathlineto{\pgfqpoint{4.212383in}{1.239858in}}%
\pgfpathlineto{\pgfqpoint{4.214948in}{1.238963in}}%
\pgfpathlineto{\pgfqpoint{4.217694in}{1.234695in}}%
\pgfpathlineto{\pgfqpoint{4.220304in}{1.236207in}}%
\pgfpathlineto{\pgfqpoint{4.223082in}{1.242076in}}%
\pgfpathlineto{\pgfqpoint{4.225654in}{1.238178in}}%
\pgfpathlineto{\pgfqpoint{4.228331in}{1.237230in}}%
\pgfpathlineto{\pgfqpoint{4.231013in}{1.231504in}}%
\pgfpathlineto{\pgfqpoint{4.233691in}{1.229922in}}%
\pgfpathlineto{\pgfqpoint{4.236375in}{1.233410in}}%
\pgfpathlineto{\pgfqpoint{4.239084in}{1.229371in}}%
\pgfpathlineto{\pgfqpoint{4.241900in}{1.229973in}}%
\pgfpathlineto{\pgfqpoint{4.244394in}{1.230867in}}%
\pgfpathlineto{\pgfqpoint{4.247225in}{1.229129in}}%
\pgfpathlineto{\pgfqpoint{4.249767in}{1.231190in}}%
\pgfpathlineto{\pgfqpoint{4.252581in}{1.230826in}}%
\pgfpathlineto{\pgfqpoint{4.255120in}{1.228492in}}%
\pgfpathlineto{\pgfqpoint{4.257958in}{1.228468in}}%
\pgfpathlineto{\pgfqpoint{4.260477in}{1.228780in}}%
\pgfpathlineto{\pgfqpoint{4.263157in}{1.228223in}}%
\pgfpathlineto{\pgfqpoint{4.265824in}{1.231251in}}%
\pgfpathlineto{\pgfqpoint{4.268590in}{1.234604in}}%
\pgfpathlineto{\pgfqpoint{4.271187in}{1.233541in}}%
\pgfpathlineto{\pgfqpoint{4.273874in}{1.232483in}}%
\pgfpathlineto{\pgfqpoint{4.276635in}{1.234276in}}%
\pgfpathlineto{\pgfqpoint{4.279212in}{1.223902in}}%
\pgfpathlineto{\pgfqpoint{4.282000in}{1.230585in}}%
\pgfpathlineto{\pgfqpoint{4.284586in}{1.235401in}}%
\pgfpathlineto{\pgfqpoint{4.287399in}{1.233805in}}%
\pgfpathlineto{\pgfqpoint{4.289936in}{1.231469in}}%
\pgfpathlineto{\pgfqpoint{4.292786in}{1.234194in}}%
\pgfpathlineto{\pgfqpoint{4.295299in}{1.231973in}}%
\pgfpathlineto{\pgfqpoint{4.297977in}{1.228820in}}%
\pgfpathlineto{\pgfqpoint{4.300656in}{1.227731in}}%
\pgfpathlineto{\pgfqpoint{4.303357in}{1.228712in}}%
\pgfpathlineto{\pgfqpoint{4.306118in}{1.226950in}}%
\pgfpathlineto{\pgfqpoint{4.308691in}{1.230790in}}%
\pgfpathlineto{\pgfqpoint{4.311494in}{1.228515in}}%
\pgfpathlineto{\pgfqpoint{4.314032in}{1.224495in}}%
\pgfpathlineto{\pgfqpoint{4.316856in}{1.231922in}}%
\pgfpathlineto{\pgfqpoint{4.319405in}{1.229815in}}%
\pgfpathlineto{\pgfqpoint{4.322181in}{1.233338in}}%
\pgfpathlineto{\pgfqpoint{4.324760in}{1.228971in}}%
\pgfpathlineto{\pgfqpoint{4.327440in}{1.222196in}}%
\pgfpathlineto{\pgfqpoint{4.330118in}{1.224759in}}%
\pgfpathlineto{\pgfqpoint{4.332796in}{1.228414in}}%
\pgfpathlineto{\pgfqpoint{4.335463in}{1.223490in}}%
\pgfpathlineto{\pgfqpoint{4.338154in}{1.227483in}}%
\pgfpathlineto{\pgfqpoint{4.340976in}{1.223087in}}%
\pgfpathlineto{\pgfqpoint{4.343510in}{1.226073in}}%
\pgfpathlineto{\pgfqpoint{4.346263in}{1.224158in}}%
\pgfpathlineto{\pgfqpoint{4.348868in}{1.227960in}}%
\pgfpathlineto{\pgfqpoint{4.351645in}{1.226393in}}%
\pgfpathlineto{\pgfqpoint{4.354224in}{1.223398in}}%
\pgfpathlineto{\pgfqpoint{4.357014in}{1.222376in}}%
\pgfpathlineto{\pgfqpoint{4.359582in}{1.225433in}}%
\pgfpathlineto{\pgfqpoint{4.362270in}{1.222217in}}%
\pgfpathlineto{\pgfqpoint{4.364936in}{1.224937in}}%
\pgfpathlineto{\pgfqpoint{4.367646in}{1.229106in}}%
\pgfpathlineto{\pgfqpoint{4.370437in}{1.222844in}}%
\pgfpathlineto{\pgfqpoint{4.372976in}{1.226974in}}%
\pgfpathlineto{\pgfqpoint{4.375761in}{1.225531in}}%
\pgfpathlineto{\pgfqpoint{4.378329in}{1.223198in}}%
\pgfpathlineto{\pgfqpoint{4.381097in}{1.225816in}}%
\pgfpathlineto{\pgfqpoint{4.383674in}{1.220743in}}%
\pgfpathlineto{\pgfqpoint{4.386431in}{1.219702in}}%
\pgfpathlineto{\pgfqpoint{4.389044in}{1.229128in}}%
\pgfpathlineto{\pgfqpoint{4.391721in}{1.228559in}}%
\pgfpathlineto{\pgfqpoint{4.394400in}{1.246128in}}%
\pgfpathlineto{\pgfqpoint{4.397076in}{1.264793in}}%
\pgfpathlineto{\pgfqpoint{4.399745in}{1.255733in}}%
\pgfpathlineto{\pgfqpoint{4.402468in}{1.241801in}}%
\pgfpathlineto{\pgfqpoint{4.405234in}{1.237418in}}%
\pgfpathlineto{\pgfqpoint{4.407788in}{1.232433in}}%
\pgfpathlineto{\pgfqpoint{4.410587in}{1.238990in}}%
\pgfpathlineto{\pgfqpoint{4.413149in}{1.233882in}}%
\pgfpathlineto{\pgfqpoint{4.415932in}{1.229598in}}%
\pgfpathlineto{\pgfqpoint{4.418506in}{1.223175in}}%
\pgfpathlineto{\pgfqpoint{4.421292in}{1.219763in}}%
\pgfpathlineto{\pgfqpoint{4.423863in}{1.221866in}}%
\pgfpathlineto{\pgfqpoint{4.426534in}{1.223481in}}%
\pgfpathlineto{\pgfqpoint{4.429220in}{1.222653in}}%
\pgfpathlineto{\pgfqpoint{4.431901in}{1.223621in}}%
\pgfpathlineto{\pgfqpoint{4.434569in}{1.222919in}}%
\pgfpathlineto{\pgfqpoint{4.437253in}{1.221120in}}%
\pgfpathlineto{\pgfqpoint{4.440041in}{1.221290in}}%
\pgfpathlineto{\pgfqpoint{4.442611in}{1.223362in}}%
\pgfpathlineto{\pgfqpoint{4.445423in}{1.220739in}}%
\pgfpathlineto{\pgfqpoint{4.447965in}{1.222212in}}%
\pgfpathlineto{\pgfqpoint{4.450767in}{1.223507in}}%
\pgfpathlineto{\pgfqpoint{4.453312in}{1.227340in}}%
\pgfpathlineto{\pgfqpoint{4.456138in}{1.225197in}}%
\pgfpathlineto{\pgfqpoint{4.458681in}{1.225007in}}%
\pgfpathlineto{\pgfqpoint{4.461367in}{1.229362in}}%
\pgfpathlineto{\pgfqpoint{4.464029in}{1.229538in}}%
\pgfpathlineto{\pgfqpoint{4.466717in}{1.230102in}}%
\pgfpathlineto{\pgfqpoint{4.469492in}{1.227147in}}%
\pgfpathlineto{\pgfqpoint{4.472059in}{1.228279in}}%
\pgfpathlineto{\pgfqpoint{4.474861in}{1.229216in}}%
\pgfpathlineto{\pgfqpoint{4.477430in}{1.230362in}}%
\pgfpathlineto{\pgfqpoint{4.480201in}{1.230588in}}%
\pgfpathlineto{\pgfqpoint{4.482778in}{1.235692in}}%
\pgfpathlineto{\pgfqpoint{4.485581in}{1.233319in}}%
\pgfpathlineto{\pgfqpoint{4.488130in}{1.230053in}}%
\pgfpathlineto{\pgfqpoint{4.490822in}{1.234510in}}%
\pgfpathlineto{\pgfqpoint{4.493492in}{1.236812in}}%
\pgfpathlineto{\pgfqpoint{4.496167in}{1.232246in}}%
\pgfpathlineto{\pgfqpoint{4.498850in}{1.237922in}}%
\pgfpathlineto{\pgfqpoint{4.501529in}{1.229745in}}%
\pgfpathlineto{\pgfqpoint{4.504305in}{1.237095in}}%
\pgfpathlineto{\pgfqpoint{4.506893in}{1.235265in}}%
\pgfpathlineto{\pgfqpoint{4.509643in}{1.228823in}}%
\pgfpathlineto{\pgfqpoint{4.512246in}{1.229759in}}%
\pgfpathlineto{\pgfqpoint{4.515080in}{1.229878in}}%
\pgfpathlineto{\pgfqpoint{4.517598in}{1.236778in}}%
\pgfpathlineto{\pgfqpoint{4.520345in}{1.244809in}}%
\pgfpathlineto{\pgfqpoint{4.522962in}{1.254268in}}%
\pgfpathlineto{\pgfqpoint{4.525640in}{1.236839in}}%
\pgfpathlineto{\pgfqpoint{4.528307in}{1.233672in}}%
\pgfpathlineto{\pgfqpoint{4.530990in}{1.236976in}}%
\pgfpathlineto{\pgfqpoint{4.533764in}{1.246758in}}%
\pgfpathlineto{\pgfqpoint{4.536400in}{1.263165in}}%
\pgfpathlineto{\pgfqpoint{4.539144in}{1.254255in}}%
\pgfpathlineto{\pgfqpoint{4.541711in}{1.246993in}}%
\pgfpathlineto{\pgfqpoint{4.544464in}{1.238275in}}%
\pgfpathlineto{\pgfqpoint{4.547064in}{1.237065in}}%
\pgfpathlineto{\pgfqpoint{4.549822in}{1.232992in}}%
\pgfpathlineto{\pgfqpoint{4.552425in}{1.224801in}}%
\pgfpathlineto{\pgfqpoint{4.555106in}{1.223152in}}%
\pgfpathlineto{\pgfqpoint{4.557777in}{1.226063in}}%
\pgfpathlineto{\pgfqpoint{4.560448in}{1.224330in}}%
\pgfpathlineto{\pgfqpoint{4.563125in}{1.225812in}}%
\pgfpathlineto{\pgfqpoint{4.565820in}{1.225464in}}%
\pgfpathlineto{\pgfqpoint{4.568612in}{1.223432in}}%
\pgfpathlineto{\pgfqpoint{4.571171in}{1.223712in}}%
\pgfpathlineto{\pgfqpoint{4.573947in}{1.219866in}}%
\pgfpathlineto{\pgfqpoint{4.576531in}{1.220811in}}%
\pgfpathlineto{\pgfqpoint{4.579305in}{1.228711in}}%
\pgfpathlineto{\pgfqpoint{4.581888in}{1.224138in}}%
\pgfpathlineto{\pgfqpoint{4.584672in}{1.227268in}}%
\pgfpathlineto{\pgfqpoint{4.587244in}{1.224471in}}%
\pgfpathlineto{\pgfqpoint{4.589920in}{1.222809in}}%
\pgfpathlineto{\pgfqpoint{4.592589in}{1.219702in}}%
\pgfpathlineto{\pgfqpoint{4.595281in}{1.224365in}}%
\pgfpathlineto{\pgfqpoint{4.597951in}{1.225416in}}%
\pgfpathlineto{\pgfqpoint{4.600633in}{1.228982in}}%
\pgfpathlineto{\pgfqpoint{4.603430in}{1.230288in}}%
\pgfpathlineto{\pgfqpoint{4.605990in}{1.227576in}}%
\pgfpathlineto{\pgfqpoint{4.608808in}{1.231678in}}%
\pgfpathlineto{\pgfqpoint{4.611350in}{1.226096in}}%
\pgfpathlineto{\pgfqpoint{4.614134in}{1.229347in}}%
\pgfpathlineto{\pgfqpoint{4.616702in}{1.229333in}}%
\pgfpathlineto{\pgfqpoint{4.619529in}{1.231593in}}%
\pgfpathlineto{\pgfqpoint{4.622056in}{1.227363in}}%
\pgfpathlineto{\pgfqpoint{4.624741in}{1.230739in}}%
\pgfpathlineto{\pgfqpoint{4.627411in}{1.230333in}}%
\pgfpathlineto{\pgfqpoint{4.630096in}{1.235427in}}%
\pgfpathlineto{\pgfqpoint{4.632902in}{1.232926in}}%
\pgfpathlineto{\pgfqpoint{4.635445in}{1.233120in}}%
\pgfpathlineto{\pgfqpoint{4.638204in}{1.230367in}}%
\pgfpathlineto{\pgfqpoint{4.640809in}{1.238177in}}%
\pgfpathlineto{\pgfqpoint{4.643628in}{1.236755in}}%
\pgfpathlineto{\pgfqpoint{4.646169in}{1.234583in}}%
\pgfpathlineto{\pgfqpoint{4.648922in}{1.236209in}}%
\pgfpathlineto{\pgfqpoint{4.651524in}{1.232616in}}%
\pgfpathlineto{\pgfqpoint{4.654203in}{1.233092in}}%
\pgfpathlineto{\pgfqpoint{4.656873in}{1.233387in}}%
\pgfpathlineto{\pgfqpoint{4.659590in}{1.229940in}}%
\pgfpathlineto{\pgfqpoint{4.662237in}{1.226264in}}%
\pgfpathlineto{\pgfqpoint{4.664923in}{1.226727in}}%
\pgfpathlineto{\pgfqpoint{4.667764in}{1.230324in}}%
\pgfpathlineto{\pgfqpoint{4.670261in}{1.228729in}}%
\pgfpathlineto{\pgfqpoint{4.673068in}{1.229973in}}%
\pgfpathlineto{\pgfqpoint{4.675619in}{1.223233in}}%
\pgfpathlineto{\pgfqpoint{4.678448in}{1.223046in}}%
\pgfpathlineto{\pgfqpoint{4.680988in}{1.221672in}}%
\pgfpathlineto{\pgfqpoint{4.683799in}{1.224926in}}%
\pgfpathlineto{\pgfqpoint{4.686337in}{1.253376in}}%
\pgfpathlineto{\pgfqpoint{4.689051in}{1.241709in}}%
\pgfpathlineto{\pgfqpoint{4.691694in}{1.241809in}}%
\pgfpathlineto{\pgfqpoint{4.694381in}{1.247334in}}%
\pgfpathlineto{\pgfqpoint{4.697170in}{1.237600in}}%
\pgfpathlineto{\pgfqpoint{4.699734in}{1.237031in}}%
\pgfpathlineto{\pgfqpoint{4.702517in}{1.239043in}}%
\pgfpathlineto{\pgfqpoint{4.705094in}{1.250060in}}%
\pgfpathlineto{\pgfqpoint{4.707824in}{1.246501in}}%
\pgfpathlineto{\pgfqpoint{4.710437in}{1.271284in}}%
\pgfpathlineto{\pgfqpoint{4.713275in}{1.302521in}}%
\pgfpathlineto{\pgfqpoint{4.715806in}{1.304309in}}%
\pgfpathlineto{\pgfqpoint{4.718486in}{1.293952in}}%
\pgfpathlineto{\pgfqpoint{4.721160in}{1.283045in}}%
\pgfpathlineto{\pgfqpoint{4.723873in}{1.288434in}}%
\pgfpathlineto{\pgfqpoint{4.726508in}{1.276344in}}%
\pgfpathlineto{\pgfqpoint{4.729233in}{1.264490in}}%
\pgfpathlineto{\pgfqpoint{4.731901in}{1.263930in}}%
\pgfpathlineto{\pgfqpoint{4.734552in}{1.259627in}}%
\pgfpathlineto{\pgfqpoint{4.737348in}{1.256534in}}%
\pgfpathlineto{\pgfqpoint{4.739912in}{1.251648in}}%
\pgfpathlineto{\pgfqpoint{4.742696in}{1.250869in}}%
\pgfpathlineto{\pgfqpoint{4.745256in}{1.250063in}}%
\pgfpathlineto{\pgfqpoint{4.748081in}{1.246656in}}%
\pgfpathlineto{\pgfqpoint{4.750627in}{1.240208in}}%
\pgfpathlineto{\pgfqpoint{4.753298in}{1.239113in}}%
\pgfpathlineto{\pgfqpoint{4.755983in}{1.239905in}}%
\pgfpathlineto{\pgfqpoint{4.758653in}{1.236268in}}%
\pgfpathlineto{\pgfqpoint{4.761337in}{1.227019in}}%
\pgfpathlineto{\pgfqpoint{4.764018in}{1.232581in}}%
\pgfpathlineto{\pgfqpoint{4.766783in}{1.237002in}}%
\pgfpathlineto{\pgfqpoint{4.769367in}{1.238783in}}%
\pgfpathlineto{\pgfqpoint{4.772198in}{1.240623in}}%
\pgfpathlineto{\pgfqpoint{4.774732in}{1.233658in}}%
\pgfpathlineto{\pgfqpoint{4.777535in}{1.231259in}}%
\pgfpathlineto{\pgfqpoint{4.780083in}{1.232508in}}%
\pgfpathlineto{\pgfqpoint{4.782872in}{1.232933in}}%
\pgfpathlineto{\pgfqpoint{4.785445in}{1.230786in}}%
\pgfpathlineto{\pgfqpoint{4.788116in}{1.230071in}}%
\pgfpathlineto{\pgfqpoint{4.790798in}{1.231509in}}%
\pgfpathlineto{\pgfqpoint{4.793512in}{1.229950in}}%
\pgfpathlineto{\pgfqpoint{4.796274in}{1.233406in}}%
\pgfpathlineto{\pgfqpoint{4.798830in}{1.228349in}}%
\pgfpathlineto{\pgfqpoint{4.801586in}{1.231559in}}%
\pgfpathlineto{\pgfqpoint{4.804193in}{1.236166in}}%
\pgfpathlineto{\pgfqpoint{4.807017in}{1.226360in}}%
\pgfpathlineto{\pgfqpoint{4.809538in}{1.221998in}}%
\pgfpathlineto{\pgfqpoint{4.812377in}{1.225207in}}%
\pgfpathlineto{\pgfqpoint{4.814907in}{1.222108in}}%
\pgfpathlineto{\pgfqpoint{4.817587in}{1.237695in}}%
\pgfpathlineto{\pgfqpoint{4.820265in}{1.233831in}}%
\pgfpathlineto{\pgfqpoint{4.822945in}{1.230048in}}%
\pgfpathlineto{\pgfqpoint{4.825619in}{1.226350in}}%
\pgfpathlineto{\pgfqpoint{4.828291in}{1.223375in}}%
\pgfpathlineto{\pgfqpoint{4.831045in}{1.224170in}}%
\pgfpathlineto{\pgfqpoint{4.833657in}{1.230825in}}%
\pgfpathlineto{\pgfqpoint{4.837992in}{1.234648in}}%
\pgfpathlineto{\pgfqpoint{4.839922in}{1.241735in}}%
\pgfpathlineto{\pgfqpoint{4.842380in}{1.233184in}}%
\pgfpathlineto{\pgfqpoint{4.844361in}{1.229859in}}%
\pgfpathlineto{\pgfqpoint{4.847127in}{1.232721in}}%
\pgfpathlineto{\pgfqpoint{4.849715in}{1.229169in}}%
\pgfpathlineto{\pgfqpoint{4.852404in}{1.231048in}}%
\pgfpathlineto{\pgfqpoint{4.855070in}{1.229007in}}%
\pgfpathlineto{\pgfqpoint{4.857807in}{1.233403in}}%
\pgfpathlineto{\pgfqpoint{4.860544in}{1.232906in}}%
\pgfpathlineto{\pgfqpoint{4.863116in}{1.229969in}}%
\pgfpathlineto{\pgfqpoint{4.865910in}{1.233487in}}%
\pgfpathlineto{\pgfqpoint{4.868474in}{1.232770in}}%
\pgfpathlineto{\pgfqpoint{4.871209in}{1.230670in}}%
\pgfpathlineto{\pgfqpoint{4.873832in}{1.232238in}}%
\pgfpathlineto{\pgfqpoint{4.876636in}{1.234020in}}%
\pgfpathlineto{\pgfqpoint{4.879180in}{1.232989in}}%
\pgfpathlineto{\pgfqpoint{4.881864in}{1.233695in}}%
\pgfpathlineto{\pgfqpoint{4.884540in}{1.229723in}}%
\pgfpathlineto{\pgfqpoint{4.887211in}{1.234049in}}%
\pgfpathlineto{\pgfqpoint{4.889902in}{1.230689in}}%
\pgfpathlineto{\pgfqpoint{4.892611in}{1.231266in}}%
\pgfpathlineto{\pgfqpoint{4.895399in}{1.229455in}}%
\pgfpathlineto{\pgfqpoint{4.897938in}{1.231060in}}%
\pgfpathlineto{\pgfqpoint{4.900712in}{1.232209in}}%
\pgfpathlineto{\pgfqpoint{4.903295in}{1.235438in}}%
\pgfpathlineto{\pgfqpoint{4.906096in}{1.224489in}}%
\pgfpathlineto{\pgfqpoint{4.908648in}{1.236736in}}%
\pgfpathlineto{\pgfqpoint{4.911435in}{1.228039in}}%
\pgfpathlineto{\pgfqpoint{4.914009in}{1.225979in}}%
\pgfpathlineto{\pgfqpoint{4.916681in}{1.227916in}}%
\pgfpathlineto{\pgfqpoint{4.919352in}{1.226319in}}%
\pgfpathlineto{\pgfqpoint{4.922041in}{1.228290in}}%
\pgfpathlineto{\pgfqpoint{4.924708in}{1.229894in}}%
\pgfpathlineto{\pgfqpoint{4.927400in}{1.233827in}}%
\pgfpathlineto{\pgfqpoint{4.930170in}{1.234862in}}%
\pgfpathlineto{\pgfqpoint{4.932742in}{1.238320in}}%
\pgfpathlineto{\pgfqpoint{4.935515in}{1.229787in}}%
\pgfpathlineto{\pgfqpoint{4.938112in}{1.235441in}}%
\pgfpathlineto{\pgfqpoint{4.940881in}{1.227575in}}%
\pgfpathlineto{\pgfqpoint{4.943466in}{1.230928in}}%
\pgfpathlineto{\pgfqpoint{4.946151in}{1.232327in}}%
\pgfpathlineto{\pgfqpoint{4.948827in}{1.231717in}}%
\pgfpathlineto{\pgfqpoint{4.951504in}{1.235026in}}%
\pgfpathlineto{\pgfqpoint{4.954182in}{1.227937in}}%
\pgfpathlineto{\pgfqpoint{4.956862in}{1.224912in}}%
\pgfpathlineto{\pgfqpoint{4.959689in}{1.221213in}}%
\pgfpathlineto{\pgfqpoint{4.962219in}{1.222933in}}%
\pgfpathlineto{\pgfqpoint{4.965002in}{1.228290in}}%
\pgfpathlineto{\pgfqpoint{4.967575in}{1.230068in}}%
\pgfpathlineto{\pgfqpoint{4.970314in}{1.224198in}}%
\pgfpathlineto{\pgfqpoint{4.972933in}{1.230291in}}%
\pgfpathlineto{\pgfqpoint{4.975703in}{1.219744in}}%
\pgfpathlineto{\pgfqpoint{4.978287in}{1.222886in}}%
\pgfpathlineto{\pgfqpoint{4.980967in}{1.224243in}}%
\pgfpathlineto{\pgfqpoint{4.983637in}{1.227986in}}%
\pgfpathlineto{\pgfqpoint{4.986325in}{1.224267in}}%
\pgfpathlineto{\pgfqpoint{4.989001in}{1.227582in}}%
\pgfpathlineto{\pgfqpoint{4.991683in}{1.222868in}}%
\pgfpathlineto{\pgfqpoint{4.994390in}{1.224296in}}%
\pgfpathlineto{\pgfqpoint{4.997028in}{1.225716in}}%
\pgfpathlineto{\pgfqpoint{4.999780in}{1.222474in}}%
\pgfpathlineto{\pgfqpoint{5.002384in}{1.231642in}}%
\pgfpathlineto{\pgfqpoint{5.005178in}{1.227082in}}%
\pgfpathlineto{\pgfqpoint{5.007751in}{1.225586in}}%
\pgfpathlineto{\pgfqpoint{5.010562in}{1.230240in}}%
\pgfpathlineto{\pgfqpoint{5.013104in}{1.229036in}}%
\pgfpathlineto{\pgfqpoint{5.015820in}{1.223925in}}%
\pgfpathlineto{\pgfqpoint{5.018466in}{1.222391in}}%
\pgfpathlineto{\pgfqpoint{5.021147in}{1.224329in}}%
\pgfpathlineto{\pgfqpoint{5.023927in}{1.227604in}}%
\pgfpathlineto{\pgfqpoint{5.026501in}{1.232685in}}%
\pgfpathlineto{\pgfqpoint{5.029275in}{1.227164in}}%
\pgfpathlineto{\pgfqpoint{5.031849in}{1.231929in}}%
\pgfpathlineto{\pgfqpoint{5.034649in}{1.230053in}}%
\pgfpathlineto{\pgfqpoint{5.037214in}{1.226935in}}%
\pgfpathlineto{\pgfqpoint{5.039962in}{1.221603in}}%
\pgfpathlineto{\pgfqpoint{5.042572in}{1.223012in}}%
\pgfpathlineto{\pgfqpoint{5.045249in}{1.225392in}}%
\pgfpathlineto{\pgfqpoint{5.047924in}{1.224481in}}%
\pgfpathlineto{\pgfqpoint{5.050606in}{1.226948in}}%
\pgfpathlineto{\pgfqpoint{5.053284in}{1.231489in}}%
\pgfpathlineto{\pgfqpoint{5.055952in}{1.238837in}}%
\pgfpathlineto{\pgfqpoint{5.058711in}{1.236351in}}%
\pgfpathlineto{\pgfqpoint{5.061315in}{1.244406in}}%
\pgfpathlineto{\pgfqpoint{5.064144in}{1.234175in}}%
\pgfpathlineto{\pgfqpoint{5.066677in}{1.236089in}}%
\pgfpathlineto{\pgfqpoint{5.069463in}{1.232581in}}%
\pgfpathlineto{\pgfqpoint{5.072030in}{1.230349in}}%
\pgfpathlineto{\pgfqpoint{5.074851in}{1.225278in}}%
\pgfpathlineto{\pgfqpoint{5.077390in}{1.227854in}}%
\pgfpathlineto{\pgfqpoint{5.080067in}{1.230878in}}%
\pgfpathlineto{\pgfqpoint{5.082746in}{1.233087in}}%
\pgfpathlineto{\pgfqpoint{5.085426in}{1.230530in}}%
\pgfpathlineto{\pgfqpoint{5.088103in}{1.231302in}}%
\pgfpathlineto{\pgfqpoint{5.090788in}{1.232374in}}%
\pgfpathlineto{\pgfqpoint{5.093579in}{1.226378in}}%
\pgfpathlineto{\pgfqpoint{5.096142in}{1.222687in}}%
\pgfpathlineto{\pgfqpoint{5.098948in}{1.226636in}}%
\pgfpathlineto{\pgfqpoint{5.101496in}{1.231047in}}%
\pgfpathlineto{\pgfqpoint{5.104312in}{1.231041in}}%
\pgfpathlineto{\pgfqpoint{5.106842in}{1.225578in}}%
\pgfpathlineto{\pgfqpoint{5.109530in}{1.228883in}}%
\pgfpathlineto{\pgfqpoint{5.112209in}{1.228385in}}%
\pgfpathlineto{\pgfqpoint{5.114887in}{1.228120in}}%
\pgfpathlineto{\pgfqpoint{5.117550in}{1.229423in}}%
\pgfpathlineto{\pgfqpoint{5.120243in}{1.225306in}}%
\pgfpathlineto{\pgfqpoint{5.123042in}{1.231883in}}%
\pgfpathlineto{\pgfqpoint{5.125599in}{1.230776in}}%
\pgfpathlineto{\pgfqpoint{5.128421in}{1.228712in}}%
\pgfpathlineto{\pgfqpoint{5.130953in}{1.225263in}}%
\pgfpathlineto{\pgfqpoint{5.133716in}{1.229275in}}%
\pgfpathlineto{\pgfqpoint{5.136311in}{1.237710in}}%
\pgfpathlineto{\pgfqpoint{5.139072in}{1.236150in}}%
\pgfpathlineto{\pgfqpoint{5.141660in}{1.234742in}}%
\pgfpathlineto{\pgfqpoint{5.144349in}{1.235124in}}%
\pgfpathlineto{\pgfqpoint{5.147029in}{1.236395in}}%
\pgfpathlineto{\pgfqpoint{5.149734in}{1.239516in}}%
\pgfpathlineto{\pgfqpoint{5.152382in}{1.228488in}}%
\pgfpathlineto{\pgfqpoint{5.155059in}{1.232583in}}%
\pgfpathlineto{\pgfqpoint{5.157815in}{1.227021in}}%
\pgfpathlineto{\pgfqpoint{5.160420in}{1.235809in}}%
\pgfpathlineto{\pgfqpoint{5.163243in}{1.232355in}}%
\pgfpathlineto{\pgfqpoint{5.165775in}{1.236129in}}%
\pgfpathlineto{\pgfqpoint{5.168591in}{1.232827in}}%
\pgfpathlineto{\pgfqpoint{5.171133in}{1.231320in}}%
\pgfpathlineto{\pgfqpoint{5.173925in}{1.232260in}}%
\pgfpathlineto{\pgfqpoint{5.176477in}{1.234368in}}%
\pgfpathlineto{\pgfqpoint{5.179188in}{1.228567in}}%
\pgfpathlineto{\pgfqpoint{5.181848in}{1.229677in}}%
\pgfpathlineto{\pgfqpoint{5.184522in}{1.227438in}}%
\pgfpathlineto{\pgfqpoint{5.187294in}{1.230322in}}%
\pgfpathlineto{\pgfqpoint{5.189880in}{1.232115in}}%
\pgfpathlineto{\pgfqpoint{5.192680in}{1.228052in}}%
\pgfpathlineto{\pgfqpoint{5.195239in}{1.229467in}}%
\pgfpathlineto{\pgfqpoint{5.198008in}{1.226920in}}%
\pgfpathlineto{\pgfqpoint{5.200594in}{1.238091in}}%
\pgfpathlineto{\pgfqpoint{5.203388in}{1.231133in}}%
\pgfpathlineto{\pgfqpoint{5.205952in}{1.230076in}}%
\pgfpathlineto{\pgfqpoint{5.208630in}{1.229619in}}%
\pgfpathlineto{\pgfqpoint{5.211299in}{1.230424in}}%
\pgfpathlineto{\pgfqpoint{5.214027in}{1.233012in}}%
\pgfpathlineto{\pgfqpoint{5.216667in}{1.230066in}}%
\pgfpathlineto{\pgfqpoint{5.219345in}{1.234572in}}%
\pgfpathlineto{\pgfqpoint{5.222151in}{1.235421in}}%
\pgfpathlineto{\pgfqpoint{5.224695in}{1.235417in}}%
\pgfpathlineto{\pgfqpoint{5.227470in}{1.241320in}}%
\pgfpathlineto{\pgfqpoint{5.230059in}{1.236644in}}%
\pgfpathlineto{\pgfqpoint{5.232855in}{1.235370in}}%
\pgfpathlineto{\pgfqpoint{5.235409in}{1.236785in}}%
\pgfpathlineto{\pgfqpoint{5.238173in}{1.232027in}}%
\pgfpathlineto{\pgfqpoint{5.240777in}{1.234951in}}%
\pgfpathlineto{\pgfqpoint{5.243445in}{1.235343in}}%
\pgfpathlineto{\pgfqpoint{5.246130in}{1.236739in}}%
\pgfpathlineto{\pgfqpoint{5.248816in}{1.234052in}}%
\pgfpathlineto{\pgfqpoint{5.251590in}{1.224853in}}%
\pgfpathlineto{\pgfqpoint{5.254236in}{1.233171in}}%
\pgfpathlineto{\pgfqpoint{5.256973in}{1.233730in}}%
\pgfpathlineto{\pgfqpoint{5.259511in}{1.239402in}}%
\pgfpathlineto{\pgfqpoint{5.262264in}{1.242328in}}%
\pgfpathlineto{\pgfqpoint{5.264876in}{1.238878in}}%
\pgfpathlineto{\pgfqpoint{5.267691in}{1.234551in}}%
\pgfpathlineto{\pgfqpoint{5.270238in}{1.235069in}}%
\pgfpathlineto{\pgfqpoint{5.272913in}{1.243771in}}%
\pgfpathlineto{\pgfqpoint{5.275589in}{1.275628in}}%
\pgfpathlineto{\pgfqpoint{5.278322in}{1.255300in}}%
\pgfpathlineto{\pgfqpoint{5.280947in}{1.249317in}}%
\pgfpathlineto{\pgfqpoint{5.283631in}{1.231828in}}%
\pgfpathlineto{\pgfqpoint{5.286436in}{1.227716in}}%
\pgfpathlineto{\pgfqpoint{5.288984in}{1.223753in}}%
\pgfpathlineto{\pgfqpoint{5.291794in}{1.223228in}}%
\pgfpathlineto{\pgfqpoint{5.294339in}{1.227220in}}%
\pgfpathlineto{\pgfqpoint{5.297140in}{1.225725in}}%
\pgfpathlineto{\pgfqpoint{5.299696in}{1.221990in}}%
\pgfpathlineto{\pgfqpoint{5.302443in}{1.227326in}}%
\pgfpathlineto{\pgfqpoint{5.305054in}{1.225578in}}%
\pgfpathlineto{\pgfqpoint{5.307731in}{1.226825in}}%
\pgfpathlineto{\pgfqpoint{5.310411in}{1.231646in}}%
\pgfpathlineto{\pgfqpoint{5.313089in}{1.229320in}}%
\pgfpathlineto{\pgfqpoint{5.315754in}{1.232681in}}%
\pgfpathlineto{\pgfqpoint{5.318430in}{1.235329in}}%
\pgfpathlineto{\pgfqpoint{5.321256in}{1.233609in}}%
\pgfpathlineto{\pgfqpoint{5.323802in}{1.239361in}}%
\pgfpathlineto{\pgfqpoint{5.326564in}{1.232810in}}%
\pgfpathlineto{\pgfqpoint{5.329159in}{1.228662in}}%
\pgfpathlineto{\pgfqpoint{5.331973in}{1.230737in}}%
\pgfpathlineto{\pgfqpoint{5.334510in}{1.229798in}}%
\pgfpathlineto{\pgfqpoint{5.337353in}{1.227221in}}%
\pgfpathlineto{\pgfqpoint{5.339872in}{1.229272in}}%
\pgfpathlineto{\pgfqpoint{5.342549in}{1.234139in}}%
\pgfpathlineto{\pgfqpoint{5.345224in}{1.271718in}}%
\pgfpathlineto{\pgfqpoint{5.347905in}{1.304778in}}%
\pgfpathlineto{\pgfqpoint{5.350723in}{1.318085in}}%
\pgfpathlineto{\pgfqpoint{5.353262in}{1.315018in}}%
\pgfpathlineto{\pgfqpoint{5.356056in}{1.319672in}}%
\pgfpathlineto{\pgfqpoint{5.358612in}{1.301273in}}%
\pgfpathlineto{\pgfqpoint{5.361370in}{1.283017in}}%
\pgfpathlineto{\pgfqpoint{5.363966in}{1.283902in}}%
\pgfpathlineto{\pgfqpoint{5.366727in}{1.275829in}}%
\pgfpathlineto{\pgfqpoint{5.369335in}{1.263997in}}%
\pgfpathlineto{\pgfqpoint{5.372013in}{1.249636in}}%
\pgfpathlineto{\pgfqpoint{5.374692in}{1.243946in}}%
\pgfpathlineto{\pgfqpoint{5.377370in}{1.243590in}}%
\pgfpathlineto{\pgfqpoint{5.380048in}{1.245087in}}%
\pgfpathlineto{\pgfqpoint{5.382725in}{1.245572in}}%
\pgfpathlineto{\pgfqpoint{5.385550in}{1.250516in}}%
\pgfpathlineto{\pgfqpoint{5.388083in}{1.243019in}}%
\pgfpathlineto{\pgfqpoint{5.390900in}{1.236520in}}%
\pgfpathlineto{\pgfqpoint{5.393441in}{1.238608in}}%
\pgfpathlineto{\pgfqpoint{5.396219in}{1.237306in}}%
\pgfpathlineto{\pgfqpoint{5.398784in}{1.226560in}}%
\pgfpathlineto{\pgfqpoint{5.401576in}{1.223770in}}%
\pgfpathlineto{\pgfqpoint{5.404154in}{1.225431in}}%
\pgfpathlineto{\pgfqpoint{5.406832in}{1.230667in}}%
\pgfpathlineto{\pgfqpoint{5.409507in}{1.236037in}}%
\pgfpathlineto{\pgfqpoint{5.412190in}{1.231058in}}%
\pgfpathlineto{\pgfqpoint{5.414954in}{1.229288in}}%
\pgfpathlineto{\pgfqpoint{5.417547in}{1.237649in}}%
\pgfpathlineto{\pgfqpoint{5.420304in}{1.227945in}}%
\pgfpathlineto{\pgfqpoint{5.422897in}{1.225578in}}%
\pgfpathlineto{\pgfqpoint{5.425661in}{1.222828in}}%
\pgfpathlineto{\pgfqpoint{5.428259in}{1.225434in}}%
\pgfpathlineto{\pgfqpoint{5.431015in}{1.227164in}}%
\pgfpathlineto{\pgfqpoint{5.433616in}{1.229048in}}%
\pgfpathlineto{\pgfqpoint{5.436295in}{1.227700in}}%
\pgfpathlineto{\pgfqpoint{5.438974in}{1.222263in}}%
\pgfpathlineto{\pgfqpoint{5.441698in}{1.226005in}}%
\pgfpathlineto{\pgfqpoint{5.444328in}{1.219702in}}%
\pgfpathlineto{\pgfqpoint{5.447021in}{1.220596in}}%
\pgfpathlineto{\pgfqpoint{5.449769in}{1.221447in}}%
\pgfpathlineto{\pgfqpoint{5.452365in}{1.224034in}}%
\pgfpathlineto{\pgfqpoint{5.455168in}{1.223208in}}%
\pgfpathlineto{\pgfqpoint{5.457721in}{1.226708in}}%
\pgfpathlineto{\pgfqpoint{5.460489in}{1.229312in}}%
\pgfpathlineto{\pgfqpoint{5.463079in}{1.224849in}}%
\pgfpathlineto{\pgfqpoint{5.465888in}{1.226095in}}%
\pgfpathlineto{\pgfqpoint{5.468425in}{1.224201in}}%
\pgfpathlineto{\pgfqpoint{5.471113in}{1.226352in}}%
\pgfpathlineto{\pgfqpoint{5.473792in}{1.229220in}}%
\pgfpathlineto{\pgfqpoint{5.476458in}{1.229876in}}%
\pgfpathlineto{\pgfqpoint{5.479152in}{1.224251in}}%
\pgfpathlineto{\pgfqpoint{5.481825in}{1.231399in}}%
\pgfpathlineto{\pgfqpoint{5.484641in}{1.235925in}}%
\pgfpathlineto{\pgfqpoint{5.487176in}{1.232619in}}%
\pgfpathlineto{\pgfqpoint{5.490000in}{1.233435in}}%
\pgfpathlineto{\pgfqpoint{5.492541in}{1.234915in}}%
\pgfpathlineto{\pgfqpoint{5.495346in}{1.232890in}}%
\pgfpathlineto{\pgfqpoint{5.497898in}{1.236089in}}%
\pgfpathlineto{\pgfqpoint{5.500687in}{1.238259in}}%
\pgfpathlineto{\pgfqpoint{5.503255in}{1.239966in}}%
\pgfpathlineto{\pgfqpoint{5.505933in}{1.236876in}}%
\pgfpathlineto{\pgfqpoint{5.508612in}{1.239227in}}%
\pgfpathlineto{\pgfqpoint{5.511290in}{1.239223in}}%
\pgfpathlineto{\pgfqpoint{5.514080in}{1.242341in}}%
\pgfpathlineto{\pgfqpoint{5.516646in}{1.250681in}}%
\pgfpathlineto{\pgfqpoint{5.519433in}{1.244288in}}%
\pgfpathlineto{\pgfqpoint{5.522003in}{1.237160in}}%
\pgfpathlineto{\pgfqpoint{5.524756in}{1.238820in}}%
\pgfpathlineto{\pgfqpoint{5.527360in}{1.237854in}}%
\pgfpathlineto{\pgfqpoint{5.530148in}{1.237484in}}%
\pgfpathlineto{\pgfqpoint{5.532717in}{1.233904in}}%
\pgfpathlineto{\pgfqpoint{5.535395in}{1.232076in}}%
\pgfpathlineto{\pgfqpoint{5.538074in}{1.233346in}}%
\pgfpathlineto{\pgfqpoint{5.540750in}{1.234092in}}%
\pgfpathlineto{\pgfqpoint{5.543421in}{1.233146in}}%
\pgfpathlineto{\pgfqpoint{5.546110in}{1.231408in}}%
\pgfpathlineto{\pgfqpoint{5.548921in}{1.236185in}}%
\pgfpathlineto{\pgfqpoint{5.551457in}{1.233241in}}%
\pgfpathlineto{\pgfqpoint{5.554198in}{1.226052in}}%
\pgfpathlineto{\pgfqpoint{5.556822in}{1.231104in}}%
\pgfpathlineto{\pgfqpoint{5.559612in}{1.236561in}}%
\pgfpathlineto{\pgfqpoint{5.562180in}{1.237023in}}%
\pgfpathlineto{\pgfqpoint{5.564940in}{1.236958in}}%
\pgfpathlineto{\pgfqpoint{5.567536in}{1.233812in}}%
\pgfpathlineto{\pgfqpoint{5.570215in}{1.236110in}}%
\pgfpathlineto{\pgfqpoint{5.572893in}{1.231256in}}%
\pgfpathlineto{\pgfqpoint{5.575596in}{1.229981in}}%
\pgfpathlineto{\pgfqpoint{5.578342in}{1.224700in}}%
\pgfpathlineto{\pgfqpoint{5.580914in}{1.236787in}}%
\pgfpathlineto{\pgfqpoint{5.583709in}{1.243590in}}%
\pgfpathlineto{\pgfqpoint{5.586269in}{1.235769in}}%
\pgfpathlineto{\pgfqpoint{5.589040in}{1.229466in}}%
\pgfpathlineto{\pgfqpoint{5.591641in}{1.230532in}}%
\pgfpathlineto{\pgfqpoint{5.594368in}{1.227432in}}%
\pgfpathlineto{\pgfqpoint{5.596999in}{1.221239in}}%
\pgfpathlineto{\pgfqpoint{5.599674in}{1.228948in}}%
\pgfpathlineto{\pgfqpoint{5.602352in}{1.228324in}}%
\pgfpathlineto{\pgfqpoint{5.605073in}{1.230761in}}%
\pgfpathlineto{\pgfqpoint{5.607698in}{1.236406in}}%
\pgfpathlineto{\pgfqpoint{5.610389in}{1.238496in}}%
\pgfpathlineto{\pgfqpoint{5.613235in}{1.233611in}}%
\pgfpathlineto{\pgfqpoint{5.615743in}{1.234364in}}%
\pgfpathlineto{\pgfqpoint{5.618526in}{1.229080in}}%
\pgfpathlineto{\pgfqpoint{5.621102in}{1.232741in}}%
\pgfpathlineto{\pgfqpoint{5.623868in}{1.224977in}}%
\pgfpathlineto{\pgfqpoint{5.626460in}{1.228771in}}%
\pgfpathlineto{\pgfqpoint{5.629232in}{1.236543in}}%
\pgfpathlineto{\pgfqpoint{5.631815in}{1.231807in}}%
\pgfpathlineto{\pgfqpoint{5.634496in}{1.231851in}}%
\pgfpathlineto{\pgfqpoint{5.637172in}{1.225825in}}%
\pgfpathlineto{\pgfqpoint{5.639852in}{1.224500in}}%
\pgfpathlineto{\pgfqpoint{5.642518in}{1.226089in}}%
\pgfpathlineto{\pgfqpoint{5.645243in}{1.223684in}}%
\pgfpathlineto{\pgfqpoint{5.648008in}{1.220630in}}%
\pgfpathlineto{\pgfqpoint{5.650563in}{1.219702in}}%
\pgfpathlineto{\pgfqpoint{5.653376in}{1.219702in}}%
\pgfpathlineto{\pgfqpoint{5.655919in}{1.222361in}}%
\pgfpathlineto{\pgfqpoint{5.658723in}{1.225188in}}%
\pgfpathlineto{\pgfqpoint{5.661273in}{1.230981in}}%
\pgfpathlineto{\pgfqpoint{5.664099in}{1.232007in}}%
\pgfpathlineto{\pgfqpoint{5.666632in}{1.232056in}}%
\pgfpathlineto{\pgfqpoint{5.669313in}{1.232813in}}%
\pgfpathlineto{\pgfqpoint{5.671991in}{1.224442in}}%
\pgfpathlineto{\pgfqpoint{5.674667in}{1.228649in}}%
\pgfpathlineto{\pgfqpoint{5.677486in}{1.231225in}}%
\pgfpathlineto{\pgfqpoint{5.680027in}{1.229951in}}%
\pgfpathlineto{\pgfqpoint{5.682836in}{1.232792in}}%
\pgfpathlineto{\pgfqpoint{5.685385in}{1.230493in}}%
\pgfpathlineto{\pgfqpoint{5.688159in}{1.229776in}}%
\pgfpathlineto{\pgfqpoint{5.690730in}{1.233512in}}%
\pgfpathlineto{\pgfqpoint{5.693473in}{1.230701in}}%
\pgfpathlineto{\pgfqpoint{5.696101in}{1.231729in}}%
\pgfpathlineto{\pgfqpoint{5.698775in}{1.228389in}}%
\pgfpathlineto{\pgfqpoint{5.701453in}{1.223672in}}%
\pgfpathlineto{\pgfqpoint{5.704130in}{1.226607in}}%
\pgfpathlineto{\pgfqpoint{5.706800in}{1.226302in}}%
\pgfpathlineto{\pgfqpoint{5.709490in}{1.228568in}}%
\pgfpathlineto{\pgfqpoint{5.712291in}{1.229167in}}%
\pgfpathlineto{\pgfqpoint{5.714834in}{1.235836in}}%
\pgfpathlineto{\pgfqpoint{5.717671in}{1.232735in}}%
\pgfpathlineto{\pgfqpoint{5.720201in}{1.234716in}}%
\pgfpathlineto{\pgfqpoint{5.722950in}{1.233898in}}%
\pgfpathlineto{\pgfqpoint{5.725548in}{1.235152in}}%
\pgfpathlineto{\pgfqpoint{5.728339in}{1.239892in}}%
\pgfpathlineto{\pgfqpoint{5.730919in}{1.241170in}}%
\pgfpathlineto{\pgfqpoint{5.733594in}{1.229701in}}%
\pgfpathlineto{\pgfqpoint{5.736276in}{1.237834in}}%
\pgfpathlineto{\pgfqpoint{5.738974in}{1.238016in}}%
\pgfpathlineto{\pgfqpoint{5.741745in}{1.241109in}}%
\pgfpathlineto{\pgfqpoint{5.744310in}{1.236379in}}%
\pgfpathlineto{\pgfqpoint{5.744310in}{0.413320in}}%
\pgfpathlineto{\pgfqpoint{5.744310in}{0.413320in}}%
\pgfpathlineto{\pgfqpoint{5.741745in}{0.413320in}}%
\pgfpathlineto{\pgfqpoint{5.738974in}{0.413320in}}%
\pgfpathlineto{\pgfqpoint{5.736276in}{0.413320in}}%
\pgfpathlineto{\pgfqpoint{5.733594in}{0.413320in}}%
\pgfpathlineto{\pgfqpoint{5.730919in}{0.413320in}}%
\pgfpathlineto{\pgfqpoint{5.728339in}{0.413320in}}%
\pgfpathlineto{\pgfqpoint{5.725548in}{0.413320in}}%
\pgfpathlineto{\pgfqpoint{5.722950in}{0.413320in}}%
\pgfpathlineto{\pgfqpoint{5.720201in}{0.413320in}}%
\pgfpathlineto{\pgfqpoint{5.717671in}{0.413320in}}%
\pgfpathlineto{\pgfqpoint{5.714834in}{0.413320in}}%
\pgfpathlineto{\pgfqpoint{5.712291in}{0.413320in}}%
\pgfpathlineto{\pgfqpoint{5.709490in}{0.413320in}}%
\pgfpathlineto{\pgfqpoint{5.706800in}{0.413320in}}%
\pgfpathlineto{\pgfqpoint{5.704130in}{0.413320in}}%
\pgfpathlineto{\pgfqpoint{5.701453in}{0.413320in}}%
\pgfpathlineto{\pgfqpoint{5.698775in}{0.413320in}}%
\pgfpathlineto{\pgfqpoint{5.696101in}{0.413320in}}%
\pgfpathlineto{\pgfqpoint{5.693473in}{0.413320in}}%
\pgfpathlineto{\pgfqpoint{5.690730in}{0.413320in}}%
\pgfpathlineto{\pgfqpoint{5.688159in}{0.413320in}}%
\pgfpathlineto{\pgfqpoint{5.685385in}{0.413320in}}%
\pgfpathlineto{\pgfqpoint{5.682836in}{0.413320in}}%
\pgfpathlineto{\pgfqpoint{5.680027in}{0.413320in}}%
\pgfpathlineto{\pgfqpoint{5.677486in}{0.413320in}}%
\pgfpathlineto{\pgfqpoint{5.674667in}{0.413320in}}%
\pgfpathlineto{\pgfqpoint{5.671991in}{0.413320in}}%
\pgfpathlineto{\pgfqpoint{5.669313in}{0.413320in}}%
\pgfpathlineto{\pgfqpoint{5.666632in}{0.413320in}}%
\pgfpathlineto{\pgfqpoint{5.664099in}{0.413320in}}%
\pgfpathlineto{\pgfqpoint{5.661273in}{0.413320in}}%
\pgfpathlineto{\pgfqpoint{5.658723in}{0.413320in}}%
\pgfpathlineto{\pgfqpoint{5.655919in}{0.413320in}}%
\pgfpathlineto{\pgfqpoint{5.653376in}{0.413320in}}%
\pgfpathlineto{\pgfqpoint{5.650563in}{0.413320in}}%
\pgfpathlineto{\pgfqpoint{5.648008in}{0.413320in}}%
\pgfpathlineto{\pgfqpoint{5.645243in}{0.413320in}}%
\pgfpathlineto{\pgfqpoint{5.642518in}{0.413320in}}%
\pgfpathlineto{\pgfqpoint{5.639852in}{0.413320in}}%
\pgfpathlineto{\pgfqpoint{5.637172in}{0.413320in}}%
\pgfpathlineto{\pgfqpoint{5.634496in}{0.413320in}}%
\pgfpathlineto{\pgfqpoint{5.631815in}{0.413320in}}%
\pgfpathlineto{\pgfqpoint{5.629232in}{0.413320in}}%
\pgfpathlineto{\pgfqpoint{5.626460in}{0.413320in}}%
\pgfpathlineto{\pgfqpoint{5.623868in}{0.413320in}}%
\pgfpathlineto{\pgfqpoint{5.621102in}{0.413320in}}%
\pgfpathlineto{\pgfqpoint{5.618526in}{0.413320in}}%
\pgfpathlineto{\pgfqpoint{5.615743in}{0.413320in}}%
\pgfpathlineto{\pgfqpoint{5.613235in}{0.413320in}}%
\pgfpathlineto{\pgfqpoint{5.610389in}{0.413320in}}%
\pgfpathlineto{\pgfqpoint{5.607698in}{0.413320in}}%
\pgfpathlineto{\pgfqpoint{5.605073in}{0.413320in}}%
\pgfpathlineto{\pgfqpoint{5.602352in}{0.413320in}}%
\pgfpathlineto{\pgfqpoint{5.599674in}{0.413320in}}%
\pgfpathlineto{\pgfqpoint{5.596999in}{0.413320in}}%
\pgfpathlineto{\pgfqpoint{5.594368in}{0.413320in}}%
\pgfpathlineto{\pgfqpoint{5.591641in}{0.413320in}}%
\pgfpathlineto{\pgfqpoint{5.589040in}{0.413320in}}%
\pgfpathlineto{\pgfqpoint{5.586269in}{0.413320in}}%
\pgfpathlineto{\pgfqpoint{5.583709in}{0.413320in}}%
\pgfpathlineto{\pgfqpoint{5.580914in}{0.413320in}}%
\pgfpathlineto{\pgfqpoint{5.578342in}{0.413320in}}%
\pgfpathlineto{\pgfqpoint{5.575596in}{0.413320in}}%
\pgfpathlineto{\pgfqpoint{5.572893in}{0.413320in}}%
\pgfpathlineto{\pgfqpoint{5.570215in}{0.413320in}}%
\pgfpathlineto{\pgfqpoint{5.567536in}{0.413320in}}%
\pgfpathlineto{\pgfqpoint{5.564940in}{0.413320in}}%
\pgfpathlineto{\pgfqpoint{5.562180in}{0.413320in}}%
\pgfpathlineto{\pgfqpoint{5.559612in}{0.413320in}}%
\pgfpathlineto{\pgfqpoint{5.556822in}{0.413320in}}%
\pgfpathlineto{\pgfqpoint{5.554198in}{0.413320in}}%
\pgfpathlineto{\pgfqpoint{5.551457in}{0.413320in}}%
\pgfpathlineto{\pgfqpoint{5.548921in}{0.413320in}}%
\pgfpathlineto{\pgfqpoint{5.546110in}{0.413320in}}%
\pgfpathlineto{\pgfqpoint{5.543421in}{0.413320in}}%
\pgfpathlineto{\pgfqpoint{5.540750in}{0.413320in}}%
\pgfpathlineto{\pgfqpoint{5.538074in}{0.413320in}}%
\pgfpathlineto{\pgfqpoint{5.535395in}{0.413320in}}%
\pgfpathlineto{\pgfqpoint{5.532717in}{0.413320in}}%
\pgfpathlineto{\pgfqpoint{5.530148in}{0.413320in}}%
\pgfpathlineto{\pgfqpoint{5.527360in}{0.413320in}}%
\pgfpathlineto{\pgfqpoint{5.524756in}{0.413320in}}%
\pgfpathlineto{\pgfqpoint{5.522003in}{0.413320in}}%
\pgfpathlineto{\pgfqpoint{5.519433in}{0.413320in}}%
\pgfpathlineto{\pgfqpoint{5.516646in}{0.413320in}}%
\pgfpathlineto{\pgfqpoint{5.514080in}{0.413320in}}%
\pgfpathlineto{\pgfqpoint{5.511290in}{0.413320in}}%
\pgfpathlineto{\pgfqpoint{5.508612in}{0.413320in}}%
\pgfpathlineto{\pgfqpoint{5.505933in}{0.413320in}}%
\pgfpathlineto{\pgfqpoint{5.503255in}{0.413320in}}%
\pgfpathlineto{\pgfqpoint{5.500687in}{0.413320in}}%
\pgfpathlineto{\pgfqpoint{5.497898in}{0.413320in}}%
\pgfpathlineto{\pgfqpoint{5.495346in}{0.413320in}}%
\pgfpathlineto{\pgfqpoint{5.492541in}{0.413320in}}%
\pgfpathlineto{\pgfqpoint{5.490000in}{0.413320in}}%
\pgfpathlineto{\pgfqpoint{5.487176in}{0.413320in}}%
\pgfpathlineto{\pgfqpoint{5.484641in}{0.413320in}}%
\pgfpathlineto{\pgfqpoint{5.481825in}{0.413320in}}%
\pgfpathlineto{\pgfqpoint{5.479152in}{0.413320in}}%
\pgfpathlineto{\pgfqpoint{5.476458in}{0.413320in}}%
\pgfpathlineto{\pgfqpoint{5.473792in}{0.413320in}}%
\pgfpathlineto{\pgfqpoint{5.471113in}{0.413320in}}%
\pgfpathlineto{\pgfqpoint{5.468425in}{0.413320in}}%
\pgfpathlineto{\pgfqpoint{5.465888in}{0.413320in}}%
\pgfpathlineto{\pgfqpoint{5.463079in}{0.413320in}}%
\pgfpathlineto{\pgfqpoint{5.460489in}{0.413320in}}%
\pgfpathlineto{\pgfqpoint{5.457721in}{0.413320in}}%
\pgfpathlineto{\pgfqpoint{5.455168in}{0.413320in}}%
\pgfpathlineto{\pgfqpoint{5.452365in}{0.413320in}}%
\pgfpathlineto{\pgfqpoint{5.449769in}{0.413320in}}%
\pgfpathlineto{\pgfqpoint{5.447021in}{0.413320in}}%
\pgfpathlineto{\pgfqpoint{5.444328in}{0.413320in}}%
\pgfpathlineto{\pgfqpoint{5.441698in}{0.413320in}}%
\pgfpathlineto{\pgfqpoint{5.438974in}{0.413320in}}%
\pgfpathlineto{\pgfqpoint{5.436295in}{0.413320in}}%
\pgfpathlineto{\pgfqpoint{5.433616in}{0.413320in}}%
\pgfpathlineto{\pgfqpoint{5.431015in}{0.413320in}}%
\pgfpathlineto{\pgfqpoint{5.428259in}{0.413320in}}%
\pgfpathlineto{\pgfqpoint{5.425661in}{0.413320in}}%
\pgfpathlineto{\pgfqpoint{5.422897in}{0.413320in}}%
\pgfpathlineto{\pgfqpoint{5.420304in}{0.413320in}}%
\pgfpathlineto{\pgfqpoint{5.417547in}{0.413320in}}%
\pgfpathlineto{\pgfqpoint{5.414954in}{0.413320in}}%
\pgfpathlineto{\pgfqpoint{5.412190in}{0.413320in}}%
\pgfpathlineto{\pgfqpoint{5.409507in}{0.413320in}}%
\pgfpathlineto{\pgfqpoint{5.406832in}{0.413320in}}%
\pgfpathlineto{\pgfqpoint{5.404154in}{0.413320in}}%
\pgfpathlineto{\pgfqpoint{5.401576in}{0.413320in}}%
\pgfpathlineto{\pgfqpoint{5.398784in}{0.413320in}}%
\pgfpathlineto{\pgfqpoint{5.396219in}{0.413320in}}%
\pgfpathlineto{\pgfqpoint{5.393441in}{0.413320in}}%
\pgfpathlineto{\pgfqpoint{5.390900in}{0.413320in}}%
\pgfpathlineto{\pgfqpoint{5.388083in}{0.413320in}}%
\pgfpathlineto{\pgfqpoint{5.385550in}{0.413320in}}%
\pgfpathlineto{\pgfqpoint{5.382725in}{0.413320in}}%
\pgfpathlineto{\pgfqpoint{5.380048in}{0.413320in}}%
\pgfpathlineto{\pgfqpoint{5.377370in}{0.413320in}}%
\pgfpathlineto{\pgfqpoint{5.374692in}{0.413320in}}%
\pgfpathlineto{\pgfqpoint{5.372013in}{0.413320in}}%
\pgfpathlineto{\pgfqpoint{5.369335in}{0.413320in}}%
\pgfpathlineto{\pgfqpoint{5.366727in}{0.413320in}}%
\pgfpathlineto{\pgfqpoint{5.363966in}{0.413320in}}%
\pgfpathlineto{\pgfqpoint{5.361370in}{0.413320in}}%
\pgfpathlineto{\pgfqpoint{5.358612in}{0.413320in}}%
\pgfpathlineto{\pgfqpoint{5.356056in}{0.413320in}}%
\pgfpathlineto{\pgfqpoint{5.353262in}{0.413320in}}%
\pgfpathlineto{\pgfqpoint{5.350723in}{0.413320in}}%
\pgfpathlineto{\pgfqpoint{5.347905in}{0.413320in}}%
\pgfpathlineto{\pgfqpoint{5.345224in}{0.413320in}}%
\pgfpathlineto{\pgfqpoint{5.342549in}{0.413320in}}%
\pgfpathlineto{\pgfqpoint{5.339872in}{0.413320in}}%
\pgfpathlineto{\pgfqpoint{5.337353in}{0.413320in}}%
\pgfpathlineto{\pgfqpoint{5.334510in}{0.413320in}}%
\pgfpathlineto{\pgfqpoint{5.331973in}{0.413320in}}%
\pgfpathlineto{\pgfqpoint{5.329159in}{0.413320in}}%
\pgfpathlineto{\pgfqpoint{5.326564in}{0.413320in}}%
\pgfpathlineto{\pgfqpoint{5.323802in}{0.413320in}}%
\pgfpathlineto{\pgfqpoint{5.321256in}{0.413320in}}%
\pgfpathlineto{\pgfqpoint{5.318430in}{0.413320in}}%
\pgfpathlineto{\pgfqpoint{5.315754in}{0.413320in}}%
\pgfpathlineto{\pgfqpoint{5.313089in}{0.413320in}}%
\pgfpathlineto{\pgfqpoint{5.310411in}{0.413320in}}%
\pgfpathlineto{\pgfqpoint{5.307731in}{0.413320in}}%
\pgfpathlineto{\pgfqpoint{5.305054in}{0.413320in}}%
\pgfpathlineto{\pgfqpoint{5.302443in}{0.413320in}}%
\pgfpathlineto{\pgfqpoint{5.299696in}{0.413320in}}%
\pgfpathlineto{\pgfqpoint{5.297140in}{0.413320in}}%
\pgfpathlineto{\pgfqpoint{5.294339in}{0.413320in}}%
\pgfpathlineto{\pgfqpoint{5.291794in}{0.413320in}}%
\pgfpathlineto{\pgfqpoint{5.288984in}{0.413320in}}%
\pgfpathlineto{\pgfqpoint{5.286436in}{0.413320in}}%
\pgfpathlineto{\pgfqpoint{5.283631in}{0.413320in}}%
\pgfpathlineto{\pgfqpoint{5.280947in}{0.413320in}}%
\pgfpathlineto{\pgfqpoint{5.278322in}{0.413320in}}%
\pgfpathlineto{\pgfqpoint{5.275589in}{0.413320in}}%
\pgfpathlineto{\pgfqpoint{5.272913in}{0.413320in}}%
\pgfpathlineto{\pgfqpoint{5.270238in}{0.413320in}}%
\pgfpathlineto{\pgfqpoint{5.267691in}{0.413320in}}%
\pgfpathlineto{\pgfqpoint{5.264876in}{0.413320in}}%
\pgfpathlineto{\pgfqpoint{5.262264in}{0.413320in}}%
\pgfpathlineto{\pgfqpoint{5.259511in}{0.413320in}}%
\pgfpathlineto{\pgfqpoint{5.256973in}{0.413320in}}%
\pgfpathlineto{\pgfqpoint{5.254236in}{0.413320in}}%
\pgfpathlineto{\pgfqpoint{5.251590in}{0.413320in}}%
\pgfpathlineto{\pgfqpoint{5.248816in}{0.413320in}}%
\pgfpathlineto{\pgfqpoint{5.246130in}{0.413320in}}%
\pgfpathlineto{\pgfqpoint{5.243445in}{0.413320in}}%
\pgfpathlineto{\pgfqpoint{5.240777in}{0.413320in}}%
\pgfpathlineto{\pgfqpoint{5.238173in}{0.413320in}}%
\pgfpathlineto{\pgfqpoint{5.235409in}{0.413320in}}%
\pgfpathlineto{\pgfqpoint{5.232855in}{0.413320in}}%
\pgfpathlineto{\pgfqpoint{5.230059in}{0.413320in}}%
\pgfpathlineto{\pgfqpoint{5.227470in}{0.413320in}}%
\pgfpathlineto{\pgfqpoint{5.224695in}{0.413320in}}%
\pgfpathlineto{\pgfqpoint{5.222151in}{0.413320in}}%
\pgfpathlineto{\pgfqpoint{5.219345in}{0.413320in}}%
\pgfpathlineto{\pgfqpoint{5.216667in}{0.413320in}}%
\pgfpathlineto{\pgfqpoint{5.214027in}{0.413320in}}%
\pgfpathlineto{\pgfqpoint{5.211299in}{0.413320in}}%
\pgfpathlineto{\pgfqpoint{5.208630in}{0.413320in}}%
\pgfpathlineto{\pgfqpoint{5.205952in}{0.413320in}}%
\pgfpathlineto{\pgfqpoint{5.203388in}{0.413320in}}%
\pgfpathlineto{\pgfqpoint{5.200594in}{0.413320in}}%
\pgfpathlineto{\pgfqpoint{5.198008in}{0.413320in}}%
\pgfpathlineto{\pgfqpoint{5.195239in}{0.413320in}}%
\pgfpathlineto{\pgfqpoint{5.192680in}{0.413320in}}%
\pgfpathlineto{\pgfqpoint{5.189880in}{0.413320in}}%
\pgfpathlineto{\pgfqpoint{5.187294in}{0.413320in}}%
\pgfpathlineto{\pgfqpoint{5.184522in}{0.413320in}}%
\pgfpathlineto{\pgfqpoint{5.181848in}{0.413320in}}%
\pgfpathlineto{\pgfqpoint{5.179188in}{0.413320in}}%
\pgfpathlineto{\pgfqpoint{5.176477in}{0.413320in}}%
\pgfpathlineto{\pgfqpoint{5.173925in}{0.413320in}}%
\pgfpathlineto{\pgfqpoint{5.171133in}{0.413320in}}%
\pgfpathlineto{\pgfqpoint{5.168591in}{0.413320in}}%
\pgfpathlineto{\pgfqpoint{5.165775in}{0.413320in}}%
\pgfpathlineto{\pgfqpoint{5.163243in}{0.413320in}}%
\pgfpathlineto{\pgfqpoint{5.160420in}{0.413320in}}%
\pgfpathlineto{\pgfqpoint{5.157815in}{0.413320in}}%
\pgfpathlineto{\pgfqpoint{5.155059in}{0.413320in}}%
\pgfpathlineto{\pgfqpoint{5.152382in}{0.413320in}}%
\pgfpathlineto{\pgfqpoint{5.149734in}{0.413320in}}%
\pgfpathlineto{\pgfqpoint{5.147029in}{0.413320in}}%
\pgfpathlineto{\pgfqpoint{5.144349in}{0.413320in}}%
\pgfpathlineto{\pgfqpoint{5.141660in}{0.413320in}}%
\pgfpathlineto{\pgfqpoint{5.139072in}{0.413320in}}%
\pgfpathlineto{\pgfqpoint{5.136311in}{0.413320in}}%
\pgfpathlineto{\pgfqpoint{5.133716in}{0.413320in}}%
\pgfpathlineto{\pgfqpoint{5.130953in}{0.413320in}}%
\pgfpathlineto{\pgfqpoint{5.128421in}{0.413320in}}%
\pgfpathlineto{\pgfqpoint{5.125599in}{0.413320in}}%
\pgfpathlineto{\pgfqpoint{5.123042in}{0.413320in}}%
\pgfpathlineto{\pgfqpoint{5.120243in}{0.413320in}}%
\pgfpathlineto{\pgfqpoint{5.117550in}{0.413320in}}%
\pgfpathlineto{\pgfqpoint{5.114887in}{0.413320in}}%
\pgfpathlineto{\pgfqpoint{5.112209in}{0.413320in}}%
\pgfpathlineto{\pgfqpoint{5.109530in}{0.413320in}}%
\pgfpathlineto{\pgfqpoint{5.106842in}{0.413320in}}%
\pgfpathlineto{\pgfqpoint{5.104312in}{0.413320in}}%
\pgfpathlineto{\pgfqpoint{5.101496in}{0.413320in}}%
\pgfpathlineto{\pgfqpoint{5.098948in}{0.413320in}}%
\pgfpathlineto{\pgfqpoint{5.096142in}{0.413320in}}%
\pgfpathlineto{\pgfqpoint{5.093579in}{0.413320in}}%
\pgfpathlineto{\pgfqpoint{5.090788in}{0.413320in}}%
\pgfpathlineto{\pgfqpoint{5.088103in}{0.413320in}}%
\pgfpathlineto{\pgfqpoint{5.085426in}{0.413320in}}%
\pgfpathlineto{\pgfqpoint{5.082746in}{0.413320in}}%
\pgfpathlineto{\pgfqpoint{5.080067in}{0.413320in}}%
\pgfpathlineto{\pgfqpoint{5.077390in}{0.413320in}}%
\pgfpathlineto{\pgfqpoint{5.074851in}{0.413320in}}%
\pgfpathlineto{\pgfqpoint{5.072030in}{0.413320in}}%
\pgfpathlineto{\pgfqpoint{5.069463in}{0.413320in}}%
\pgfpathlineto{\pgfqpoint{5.066677in}{0.413320in}}%
\pgfpathlineto{\pgfqpoint{5.064144in}{0.413320in}}%
\pgfpathlineto{\pgfqpoint{5.061315in}{0.413320in}}%
\pgfpathlineto{\pgfqpoint{5.058711in}{0.413320in}}%
\pgfpathlineto{\pgfqpoint{5.055952in}{0.413320in}}%
\pgfpathlineto{\pgfqpoint{5.053284in}{0.413320in}}%
\pgfpathlineto{\pgfqpoint{5.050606in}{0.413320in}}%
\pgfpathlineto{\pgfqpoint{5.047924in}{0.413320in}}%
\pgfpathlineto{\pgfqpoint{5.045249in}{0.413320in}}%
\pgfpathlineto{\pgfqpoint{5.042572in}{0.413320in}}%
\pgfpathlineto{\pgfqpoint{5.039962in}{0.413320in}}%
\pgfpathlineto{\pgfqpoint{5.037214in}{0.413320in}}%
\pgfpathlineto{\pgfqpoint{5.034649in}{0.413320in}}%
\pgfpathlineto{\pgfqpoint{5.031849in}{0.413320in}}%
\pgfpathlineto{\pgfqpoint{5.029275in}{0.413320in}}%
\pgfpathlineto{\pgfqpoint{5.026501in}{0.413320in}}%
\pgfpathlineto{\pgfqpoint{5.023927in}{0.413320in}}%
\pgfpathlineto{\pgfqpoint{5.021147in}{0.413320in}}%
\pgfpathlineto{\pgfqpoint{5.018466in}{0.413320in}}%
\pgfpathlineto{\pgfqpoint{5.015820in}{0.413320in}}%
\pgfpathlineto{\pgfqpoint{5.013104in}{0.413320in}}%
\pgfpathlineto{\pgfqpoint{5.010562in}{0.413320in}}%
\pgfpathlineto{\pgfqpoint{5.007751in}{0.413320in}}%
\pgfpathlineto{\pgfqpoint{5.005178in}{0.413320in}}%
\pgfpathlineto{\pgfqpoint{5.002384in}{0.413320in}}%
\pgfpathlineto{\pgfqpoint{4.999780in}{0.413320in}}%
\pgfpathlineto{\pgfqpoint{4.997028in}{0.413320in}}%
\pgfpathlineto{\pgfqpoint{4.994390in}{0.413320in}}%
\pgfpathlineto{\pgfqpoint{4.991683in}{0.413320in}}%
\pgfpathlineto{\pgfqpoint{4.989001in}{0.413320in}}%
\pgfpathlineto{\pgfqpoint{4.986325in}{0.413320in}}%
\pgfpathlineto{\pgfqpoint{4.983637in}{0.413320in}}%
\pgfpathlineto{\pgfqpoint{4.980967in}{0.413320in}}%
\pgfpathlineto{\pgfqpoint{4.978287in}{0.413320in}}%
\pgfpathlineto{\pgfqpoint{4.975703in}{0.413320in}}%
\pgfpathlineto{\pgfqpoint{4.972933in}{0.413320in}}%
\pgfpathlineto{\pgfqpoint{4.970314in}{0.413320in}}%
\pgfpathlineto{\pgfqpoint{4.967575in}{0.413320in}}%
\pgfpathlineto{\pgfqpoint{4.965002in}{0.413320in}}%
\pgfpathlineto{\pgfqpoint{4.962219in}{0.413320in}}%
\pgfpathlineto{\pgfqpoint{4.959689in}{0.413320in}}%
\pgfpathlineto{\pgfqpoint{4.956862in}{0.413320in}}%
\pgfpathlineto{\pgfqpoint{4.954182in}{0.413320in}}%
\pgfpathlineto{\pgfqpoint{4.951504in}{0.413320in}}%
\pgfpathlineto{\pgfqpoint{4.948827in}{0.413320in}}%
\pgfpathlineto{\pgfqpoint{4.946151in}{0.413320in}}%
\pgfpathlineto{\pgfqpoint{4.943466in}{0.413320in}}%
\pgfpathlineto{\pgfqpoint{4.940881in}{0.413320in}}%
\pgfpathlineto{\pgfqpoint{4.938112in}{0.413320in}}%
\pgfpathlineto{\pgfqpoint{4.935515in}{0.413320in}}%
\pgfpathlineto{\pgfqpoint{4.932742in}{0.413320in}}%
\pgfpathlineto{\pgfqpoint{4.930170in}{0.413320in}}%
\pgfpathlineto{\pgfqpoint{4.927400in}{0.413320in}}%
\pgfpathlineto{\pgfqpoint{4.924708in}{0.413320in}}%
\pgfpathlineto{\pgfqpoint{4.922041in}{0.413320in}}%
\pgfpathlineto{\pgfqpoint{4.919352in}{0.413320in}}%
\pgfpathlineto{\pgfqpoint{4.916681in}{0.413320in}}%
\pgfpathlineto{\pgfqpoint{4.914009in}{0.413320in}}%
\pgfpathlineto{\pgfqpoint{4.911435in}{0.413320in}}%
\pgfpathlineto{\pgfqpoint{4.908648in}{0.413320in}}%
\pgfpathlineto{\pgfqpoint{4.906096in}{0.413320in}}%
\pgfpathlineto{\pgfqpoint{4.903295in}{0.413320in}}%
\pgfpathlineto{\pgfqpoint{4.900712in}{0.413320in}}%
\pgfpathlineto{\pgfqpoint{4.897938in}{0.413320in}}%
\pgfpathlineto{\pgfqpoint{4.895399in}{0.413320in}}%
\pgfpathlineto{\pgfqpoint{4.892611in}{0.413320in}}%
\pgfpathlineto{\pgfqpoint{4.889902in}{0.413320in}}%
\pgfpathlineto{\pgfqpoint{4.887211in}{0.413320in}}%
\pgfpathlineto{\pgfqpoint{4.884540in}{0.413320in}}%
\pgfpathlineto{\pgfqpoint{4.881864in}{0.413320in}}%
\pgfpathlineto{\pgfqpoint{4.879180in}{0.413320in}}%
\pgfpathlineto{\pgfqpoint{4.876636in}{0.413320in}}%
\pgfpathlineto{\pgfqpoint{4.873832in}{0.413320in}}%
\pgfpathlineto{\pgfqpoint{4.871209in}{0.413320in}}%
\pgfpathlineto{\pgfqpoint{4.868474in}{0.413320in}}%
\pgfpathlineto{\pgfqpoint{4.865910in}{0.413320in}}%
\pgfpathlineto{\pgfqpoint{4.863116in}{0.413320in}}%
\pgfpathlineto{\pgfqpoint{4.860544in}{0.413320in}}%
\pgfpathlineto{\pgfqpoint{4.857807in}{0.413320in}}%
\pgfpathlineto{\pgfqpoint{4.855070in}{0.413320in}}%
\pgfpathlineto{\pgfqpoint{4.852404in}{0.413320in}}%
\pgfpathlineto{\pgfqpoint{4.849715in}{0.413320in}}%
\pgfpathlineto{\pgfqpoint{4.847127in}{0.413320in}}%
\pgfpathlineto{\pgfqpoint{4.844361in}{0.413320in}}%
\pgfpathlineto{\pgfqpoint{4.842380in}{0.413320in}}%
\pgfpathlineto{\pgfqpoint{4.839922in}{0.413320in}}%
\pgfpathlineto{\pgfqpoint{4.837992in}{0.413320in}}%
\pgfpathlineto{\pgfqpoint{4.833657in}{0.413320in}}%
\pgfpathlineto{\pgfqpoint{4.831045in}{0.413320in}}%
\pgfpathlineto{\pgfqpoint{4.828291in}{0.413320in}}%
\pgfpathlineto{\pgfqpoint{4.825619in}{0.413320in}}%
\pgfpathlineto{\pgfqpoint{4.822945in}{0.413320in}}%
\pgfpathlineto{\pgfqpoint{4.820265in}{0.413320in}}%
\pgfpathlineto{\pgfqpoint{4.817587in}{0.413320in}}%
\pgfpathlineto{\pgfqpoint{4.814907in}{0.413320in}}%
\pgfpathlineto{\pgfqpoint{4.812377in}{0.413320in}}%
\pgfpathlineto{\pgfqpoint{4.809538in}{0.413320in}}%
\pgfpathlineto{\pgfqpoint{4.807017in}{0.413320in}}%
\pgfpathlineto{\pgfqpoint{4.804193in}{0.413320in}}%
\pgfpathlineto{\pgfqpoint{4.801586in}{0.413320in}}%
\pgfpathlineto{\pgfqpoint{4.798830in}{0.413320in}}%
\pgfpathlineto{\pgfqpoint{4.796274in}{0.413320in}}%
\pgfpathlineto{\pgfqpoint{4.793512in}{0.413320in}}%
\pgfpathlineto{\pgfqpoint{4.790798in}{0.413320in}}%
\pgfpathlineto{\pgfqpoint{4.788116in}{0.413320in}}%
\pgfpathlineto{\pgfqpoint{4.785445in}{0.413320in}}%
\pgfpathlineto{\pgfqpoint{4.782872in}{0.413320in}}%
\pgfpathlineto{\pgfqpoint{4.780083in}{0.413320in}}%
\pgfpathlineto{\pgfqpoint{4.777535in}{0.413320in}}%
\pgfpathlineto{\pgfqpoint{4.774732in}{0.413320in}}%
\pgfpathlineto{\pgfqpoint{4.772198in}{0.413320in}}%
\pgfpathlineto{\pgfqpoint{4.769367in}{0.413320in}}%
\pgfpathlineto{\pgfqpoint{4.766783in}{0.413320in}}%
\pgfpathlineto{\pgfqpoint{4.764018in}{0.413320in}}%
\pgfpathlineto{\pgfqpoint{4.761337in}{0.413320in}}%
\pgfpathlineto{\pgfqpoint{4.758653in}{0.413320in}}%
\pgfpathlineto{\pgfqpoint{4.755983in}{0.413320in}}%
\pgfpathlineto{\pgfqpoint{4.753298in}{0.413320in}}%
\pgfpathlineto{\pgfqpoint{4.750627in}{0.413320in}}%
\pgfpathlineto{\pgfqpoint{4.748081in}{0.413320in}}%
\pgfpathlineto{\pgfqpoint{4.745256in}{0.413320in}}%
\pgfpathlineto{\pgfqpoint{4.742696in}{0.413320in}}%
\pgfpathlineto{\pgfqpoint{4.739912in}{0.413320in}}%
\pgfpathlineto{\pgfqpoint{4.737348in}{0.413320in}}%
\pgfpathlineto{\pgfqpoint{4.734552in}{0.413320in}}%
\pgfpathlineto{\pgfqpoint{4.731901in}{0.413320in}}%
\pgfpathlineto{\pgfqpoint{4.729233in}{0.413320in}}%
\pgfpathlineto{\pgfqpoint{4.726508in}{0.413320in}}%
\pgfpathlineto{\pgfqpoint{4.723873in}{0.413320in}}%
\pgfpathlineto{\pgfqpoint{4.721160in}{0.413320in}}%
\pgfpathlineto{\pgfqpoint{4.718486in}{0.413320in}}%
\pgfpathlineto{\pgfqpoint{4.715806in}{0.413320in}}%
\pgfpathlineto{\pgfqpoint{4.713275in}{0.413320in}}%
\pgfpathlineto{\pgfqpoint{4.710437in}{0.413320in}}%
\pgfpathlineto{\pgfqpoint{4.707824in}{0.413320in}}%
\pgfpathlineto{\pgfqpoint{4.705094in}{0.413320in}}%
\pgfpathlineto{\pgfqpoint{4.702517in}{0.413320in}}%
\pgfpathlineto{\pgfqpoint{4.699734in}{0.413320in}}%
\pgfpathlineto{\pgfqpoint{4.697170in}{0.413320in}}%
\pgfpathlineto{\pgfqpoint{4.694381in}{0.413320in}}%
\pgfpathlineto{\pgfqpoint{4.691694in}{0.413320in}}%
\pgfpathlineto{\pgfqpoint{4.689051in}{0.413320in}}%
\pgfpathlineto{\pgfqpoint{4.686337in}{0.413320in}}%
\pgfpathlineto{\pgfqpoint{4.683799in}{0.413320in}}%
\pgfpathlineto{\pgfqpoint{4.680988in}{0.413320in}}%
\pgfpathlineto{\pgfqpoint{4.678448in}{0.413320in}}%
\pgfpathlineto{\pgfqpoint{4.675619in}{0.413320in}}%
\pgfpathlineto{\pgfqpoint{4.673068in}{0.413320in}}%
\pgfpathlineto{\pgfqpoint{4.670261in}{0.413320in}}%
\pgfpathlineto{\pgfqpoint{4.667764in}{0.413320in}}%
\pgfpathlineto{\pgfqpoint{4.664923in}{0.413320in}}%
\pgfpathlineto{\pgfqpoint{4.662237in}{0.413320in}}%
\pgfpathlineto{\pgfqpoint{4.659590in}{0.413320in}}%
\pgfpathlineto{\pgfqpoint{4.656873in}{0.413320in}}%
\pgfpathlineto{\pgfqpoint{4.654203in}{0.413320in}}%
\pgfpathlineto{\pgfqpoint{4.651524in}{0.413320in}}%
\pgfpathlineto{\pgfqpoint{4.648922in}{0.413320in}}%
\pgfpathlineto{\pgfqpoint{4.646169in}{0.413320in}}%
\pgfpathlineto{\pgfqpoint{4.643628in}{0.413320in}}%
\pgfpathlineto{\pgfqpoint{4.640809in}{0.413320in}}%
\pgfpathlineto{\pgfqpoint{4.638204in}{0.413320in}}%
\pgfpathlineto{\pgfqpoint{4.635445in}{0.413320in}}%
\pgfpathlineto{\pgfqpoint{4.632902in}{0.413320in}}%
\pgfpathlineto{\pgfqpoint{4.630096in}{0.413320in}}%
\pgfpathlineto{\pgfqpoint{4.627411in}{0.413320in}}%
\pgfpathlineto{\pgfqpoint{4.624741in}{0.413320in}}%
\pgfpathlineto{\pgfqpoint{4.622056in}{0.413320in}}%
\pgfpathlineto{\pgfqpoint{4.619529in}{0.413320in}}%
\pgfpathlineto{\pgfqpoint{4.616702in}{0.413320in}}%
\pgfpathlineto{\pgfqpoint{4.614134in}{0.413320in}}%
\pgfpathlineto{\pgfqpoint{4.611350in}{0.413320in}}%
\pgfpathlineto{\pgfqpoint{4.608808in}{0.413320in}}%
\pgfpathlineto{\pgfqpoint{4.605990in}{0.413320in}}%
\pgfpathlineto{\pgfqpoint{4.603430in}{0.413320in}}%
\pgfpathlineto{\pgfqpoint{4.600633in}{0.413320in}}%
\pgfpathlineto{\pgfqpoint{4.597951in}{0.413320in}}%
\pgfpathlineto{\pgfqpoint{4.595281in}{0.413320in}}%
\pgfpathlineto{\pgfqpoint{4.592589in}{0.413320in}}%
\pgfpathlineto{\pgfqpoint{4.589920in}{0.413320in}}%
\pgfpathlineto{\pgfqpoint{4.587244in}{0.413320in}}%
\pgfpathlineto{\pgfqpoint{4.584672in}{0.413320in}}%
\pgfpathlineto{\pgfqpoint{4.581888in}{0.413320in}}%
\pgfpathlineto{\pgfqpoint{4.579305in}{0.413320in}}%
\pgfpathlineto{\pgfqpoint{4.576531in}{0.413320in}}%
\pgfpathlineto{\pgfqpoint{4.573947in}{0.413320in}}%
\pgfpathlineto{\pgfqpoint{4.571171in}{0.413320in}}%
\pgfpathlineto{\pgfqpoint{4.568612in}{0.413320in}}%
\pgfpathlineto{\pgfqpoint{4.565820in}{0.413320in}}%
\pgfpathlineto{\pgfqpoint{4.563125in}{0.413320in}}%
\pgfpathlineto{\pgfqpoint{4.560448in}{0.413320in}}%
\pgfpathlineto{\pgfqpoint{4.557777in}{0.413320in}}%
\pgfpathlineto{\pgfqpoint{4.555106in}{0.413320in}}%
\pgfpathlineto{\pgfqpoint{4.552425in}{0.413320in}}%
\pgfpathlineto{\pgfqpoint{4.549822in}{0.413320in}}%
\pgfpathlineto{\pgfqpoint{4.547064in}{0.413320in}}%
\pgfpathlineto{\pgfqpoint{4.544464in}{0.413320in}}%
\pgfpathlineto{\pgfqpoint{4.541711in}{0.413320in}}%
\pgfpathlineto{\pgfqpoint{4.539144in}{0.413320in}}%
\pgfpathlineto{\pgfqpoint{4.536400in}{0.413320in}}%
\pgfpathlineto{\pgfqpoint{4.533764in}{0.413320in}}%
\pgfpathlineto{\pgfqpoint{4.530990in}{0.413320in}}%
\pgfpathlineto{\pgfqpoint{4.528307in}{0.413320in}}%
\pgfpathlineto{\pgfqpoint{4.525640in}{0.413320in}}%
\pgfpathlineto{\pgfqpoint{4.522962in}{0.413320in}}%
\pgfpathlineto{\pgfqpoint{4.520345in}{0.413320in}}%
\pgfpathlineto{\pgfqpoint{4.517598in}{0.413320in}}%
\pgfpathlineto{\pgfqpoint{4.515080in}{0.413320in}}%
\pgfpathlineto{\pgfqpoint{4.512246in}{0.413320in}}%
\pgfpathlineto{\pgfqpoint{4.509643in}{0.413320in}}%
\pgfpathlineto{\pgfqpoint{4.506893in}{0.413320in}}%
\pgfpathlineto{\pgfqpoint{4.504305in}{0.413320in}}%
\pgfpathlineto{\pgfqpoint{4.501529in}{0.413320in}}%
\pgfpathlineto{\pgfqpoint{4.498850in}{0.413320in}}%
\pgfpathlineto{\pgfqpoint{4.496167in}{0.413320in}}%
\pgfpathlineto{\pgfqpoint{4.493492in}{0.413320in}}%
\pgfpathlineto{\pgfqpoint{4.490822in}{0.413320in}}%
\pgfpathlineto{\pgfqpoint{4.488130in}{0.413320in}}%
\pgfpathlineto{\pgfqpoint{4.485581in}{0.413320in}}%
\pgfpathlineto{\pgfqpoint{4.482778in}{0.413320in}}%
\pgfpathlineto{\pgfqpoint{4.480201in}{0.413320in}}%
\pgfpathlineto{\pgfqpoint{4.477430in}{0.413320in}}%
\pgfpathlineto{\pgfqpoint{4.474861in}{0.413320in}}%
\pgfpathlineto{\pgfqpoint{4.472059in}{0.413320in}}%
\pgfpathlineto{\pgfqpoint{4.469492in}{0.413320in}}%
\pgfpathlineto{\pgfqpoint{4.466717in}{0.413320in}}%
\pgfpathlineto{\pgfqpoint{4.464029in}{0.413320in}}%
\pgfpathlineto{\pgfqpoint{4.461367in}{0.413320in}}%
\pgfpathlineto{\pgfqpoint{4.458681in}{0.413320in}}%
\pgfpathlineto{\pgfqpoint{4.456138in}{0.413320in}}%
\pgfpathlineto{\pgfqpoint{4.453312in}{0.413320in}}%
\pgfpathlineto{\pgfqpoint{4.450767in}{0.413320in}}%
\pgfpathlineto{\pgfqpoint{4.447965in}{0.413320in}}%
\pgfpathlineto{\pgfqpoint{4.445423in}{0.413320in}}%
\pgfpathlineto{\pgfqpoint{4.442611in}{0.413320in}}%
\pgfpathlineto{\pgfqpoint{4.440041in}{0.413320in}}%
\pgfpathlineto{\pgfqpoint{4.437253in}{0.413320in}}%
\pgfpathlineto{\pgfqpoint{4.434569in}{0.413320in}}%
\pgfpathlineto{\pgfqpoint{4.431901in}{0.413320in}}%
\pgfpathlineto{\pgfqpoint{4.429220in}{0.413320in}}%
\pgfpathlineto{\pgfqpoint{4.426534in}{0.413320in}}%
\pgfpathlineto{\pgfqpoint{4.423863in}{0.413320in}}%
\pgfpathlineto{\pgfqpoint{4.421292in}{0.413320in}}%
\pgfpathlineto{\pgfqpoint{4.418506in}{0.413320in}}%
\pgfpathlineto{\pgfqpoint{4.415932in}{0.413320in}}%
\pgfpathlineto{\pgfqpoint{4.413149in}{0.413320in}}%
\pgfpathlineto{\pgfqpoint{4.410587in}{0.413320in}}%
\pgfpathlineto{\pgfqpoint{4.407788in}{0.413320in}}%
\pgfpathlineto{\pgfqpoint{4.405234in}{0.413320in}}%
\pgfpathlineto{\pgfqpoint{4.402468in}{0.413320in}}%
\pgfpathlineto{\pgfqpoint{4.399745in}{0.413320in}}%
\pgfpathlineto{\pgfqpoint{4.397076in}{0.413320in}}%
\pgfpathlineto{\pgfqpoint{4.394400in}{0.413320in}}%
\pgfpathlineto{\pgfqpoint{4.391721in}{0.413320in}}%
\pgfpathlineto{\pgfqpoint{4.389044in}{0.413320in}}%
\pgfpathlineto{\pgfqpoint{4.386431in}{0.413320in}}%
\pgfpathlineto{\pgfqpoint{4.383674in}{0.413320in}}%
\pgfpathlineto{\pgfqpoint{4.381097in}{0.413320in}}%
\pgfpathlineto{\pgfqpoint{4.378329in}{0.413320in}}%
\pgfpathlineto{\pgfqpoint{4.375761in}{0.413320in}}%
\pgfpathlineto{\pgfqpoint{4.372976in}{0.413320in}}%
\pgfpathlineto{\pgfqpoint{4.370437in}{0.413320in}}%
\pgfpathlineto{\pgfqpoint{4.367646in}{0.413320in}}%
\pgfpathlineto{\pgfqpoint{4.364936in}{0.413320in}}%
\pgfpathlineto{\pgfqpoint{4.362270in}{0.413320in}}%
\pgfpathlineto{\pgfqpoint{4.359582in}{0.413320in}}%
\pgfpathlineto{\pgfqpoint{4.357014in}{0.413320in}}%
\pgfpathlineto{\pgfqpoint{4.354224in}{0.413320in}}%
\pgfpathlineto{\pgfqpoint{4.351645in}{0.413320in}}%
\pgfpathlineto{\pgfqpoint{4.348868in}{0.413320in}}%
\pgfpathlineto{\pgfqpoint{4.346263in}{0.413320in}}%
\pgfpathlineto{\pgfqpoint{4.343510in}{0.413320in}}%
\pgfpathlineto{\pgfqpoint{4.340976in}{0.413320in}}%
\pgfpathlineto{\pgfqpoint{4.338154in}{0.413320in}}%
\pgfpathlineto{\pgfqpoint{4.335463in}{0.413320in}}%
\pgfpathlineto{\pgfqpoint{4.332796in}{0.413320in}}%
\pgfpathlineto{\pgfqpoint{4.330118in}{0.413320in}}%
\pgfpathlineto{\pgfqpoint{4.327440in}{0.413320in}}%
\pgfpathlineto{\pgfqpoint{4.324760in}{0.413320in}}%
\pgfpathlineto{\pgfqpoint{4.322181in}{0.413320in}}%
\pgfpathlineto{\pgfqpoint{4.319405in}{0.413320in}}%
\pgfpathlineto{\pgfqpoint{4.316856in}{0.413320in}}%
\pgfpathlineto{\pgfqpoint{4.314032in}{0.413320in}}%
\pgfpathlineto{\pgfqpoint{4.311494in}{0.413320in}}%
\pgfpathlineto{\pgfqpoint{4.308691in}{0.413320in}}%
\pgfpathlineto{\pgfqpoint{4.306118in}{0.413320in}}%
\pgfpathlineto{\pgfqpoint{4.303357in}{0.413320in}}%
\pgfpathlineto{\pgfqpoint{4.300656in}{0.413320in}}%
\pgfpathlineto{\pgfqpoint{4.297977in}{0.413320in}}%
\pgfpathlineto{\pgfqpoint{4.295299in}{0.413320in}}%
\pgfpathlineto{\pgfqpoint{4.292786in}{0.413320in}}%
\pgfpathlineto{\pgfqpoint{4.289936in}{0.413320in}}%
\pgfpathlineto{\pgfqpoint{4.287399in}{0.413320in}}%
\pgfpathlineto{\pgfqpoint{4.284586in}{0.413320in}}%
\pgfpathlineto{\pgfqpoint{4.282000in}{0.413320in}}%
\pgfpathlineto{\pgfqpoint{4.279212in}{0.413320in}}%
\pgfpathlineto{\pgfqpoint{4.276635in}{0.413320in}}%
\pgfpathlineto{\pgfqpoint{4.273874in}{0.413320in}}%
\pgfpathlineto{\pgfqpoint{4.271187in}{0.413320in}}%
\pgfpathlineto{\pgfqpoint{4.268590in}{0.413320in}}%
\pgfpathlineto{\pgfqpoint{4.265824in}{0.413320in}}%
\pgfpathlineto{\pgfqpoint{4.263157in}{0.413320in}}%
\pgfpathlineto{\pgfqpoint{4.260477in}{0.413320in}}%
\pgfpathlineto{\pgfqpoint{4.257958in}{0.413320in}}%
\pgfpathlineto{\pgfqpoint{4.255120in}{0.413320in}}%
\pgfpathlineto{\pgfqpoint{4.252581in}{0.413320in}}%
\pgfpathlineto{\pgfqpoint{4.249767in}{0.413320in}}%
\pgfpathlineto{\pgfqpoint{4.247225in}{0.413320in}}%
\pgfpathlineto{\pgfqpoint{4.244394in}{0.413320in}}%
\pgfpathlineto{\pgfqpoint{4.241900in}{0.413320in}}%
\pgfpathlineto{\pgfqpoint{4.239084in}{0.413320in}}%
\pgfpathlineto{\pgfqpoint{4.236375in}{0.413320in}}%
\pgfpathlineto{\pgfqpoint{4.233691in}{0.413320in}}%
\pgfpathlineto{\pgfqpoint{4.231013in}{0.413320in}}%
\pgfpathlineto{\pgfqpoint{4.228331in}{0.413320in}}%
\pgfpathlineto{\pgfqpoint{4.225654in}{0.413320in}}%
\pgfpathlineto{\pgfqpoint{4.223082in}{0.413320in}}%
\pgfpathlineto{\pgfqpoint{4.220304in}{0.413320in}}%
\pgfpathlineto{\pgfqpoint{4.217694in}{0.413320in}}%
\pgfpathlineto{\pgfqpoint{4.214948in}{0.413320in}}%
\pgfpathlineto{\pgfqpoint{4.212383in}{0.413320in}}%
\pgfpathlineto{\pgfqpoint{4.209597in}{0.413320in}}%
\pgfpathlineto{\pgfqpoint{4.207076in}{0.413320in}}%
\pgfpathlineto{\pgfqpoint{4.204240in}{0.413320in}}%
\pgfpathlineto{\pgfqpoint{4.201542in}{0.413320in}}%
\pgfpathlineto{\pgfqpoint{4.198878in}{0.413320in}}%
\pgfpathlineto{\pgfqpoint{4.196186in}{0.413320in}}%
\pgfpathlineto{\pgfqpoint{4.193638in}{0.413320in}}%
\pgfpathlineto{\pgfqpoint{4.190842in}{0.413320in}}%
\pgfpathlineto{\pgfqpoint{4.188318in}{0.413320in}}%
\pgfpathlineto{\pgfqpoint{4.185481in}{0.413320in}}%
\pgfpathlineto{\pgfqpoint{4.182899in}{0.413320in}}%
\pgfpathlineto{\pgfqpoint{4.180129in}{0.413320in}}%
\pgfpathlineto{\pgfqpoint{4.177593in}{0.413320in}}%
\pgfpathlineto{\pgfqpoint{4.174770in}{0.413320in}}%
\pgfpathlineto{\pgfqpoint{4.172093in}{0.413320in}}%
\pgfpathlineto{\pgfqpoint{4.169415in}{0.413320in}}%
\pgfpathlineto{\pgfqpoint{4.166737in}{0.413320in}}%
\pgfpathlineto{\pgfqpoint{4.164059in}{0.413320in}}%
\pgfpathlineto{\pgfqpoint{4.161380in}{0.413320in}}%
\pgfpathlineto{\pgfqpoint{4.158806in}{0.413320in}}%
\pgfpathlineto{\pgfqpoint{4.156016in}{0.413320in}}%
\pgfpathlineto{\pgfqpoint{4.153423in}{0.413320in}}%
\pgfpathlineto{\pgfqpoint{4.150665in}{0.413320in}}%
\pgfpathlineto{\pgfqpoint{4.148082in}{0.413320in}}%
\pgfpathlineto{\pgfqpoint{4.145310in}{0.413320in}}%
\pgfpathlineto{\pgfqpoint{4.142713in}{0.413320in}}%
\pgfpathlineto{\pgfqpoint{4.139963in}{0.413320in}}%
\pgfpathlineto{\pgfqpoint{4.137272in}{0.413320in}}%
\pgfpathlineto{\pgfqpoint{4.134615in}{0.413320in}}%
\pgfpathlineto{\pgfqpoint{4.131920in}{0.413320in}}%
\pgfpathlineto{\pgfqpoint{4.129349in}{0.413320in}}%
\pgfpathlineto{\pgfqpoint{4.126553in}{0.413320in}}%
\pgfpathlineto{\pgfqpoint{4.124019in}{0.413320in}}%
\pgfpathlineto{\pgfqpoint{4.121205in}{0.413320in}}%
\pgfpathlineto{\pgfqpoint{4.118554in}{0.413320in}}%
\pgfpathlineto{\pgfqpoint{4.115844in}{0.413320in}}%
\pgfpathlineto{\pgfqpoint{4.113252in}{0.413320in}}%
\pgfpathlineto{\pgfqpoint{4.110488in}{0.413320in}}%
\pgfpathlineto{\pgfqpoint{4.107814in}{0.413320in}}%
\pgfpathlineto{\pgfqpoint{4.105185in}{0.413320in}}%
\pgfpathlineto{\pgfqpoint{4.102456in}{0.413320in}}%
\pgfpathlineto{\pgfqpoint{4.099777in}{0.413320in}}%
\pgfpathlineto{\pgfqpoint{4.097092in}{0.413320in}}%
\pgfpathlineto{\pgfqpoint{4.094527in}{0.413320in}}%
\pgfpathlineto{\pgfqpoint{4.091729in}{0.413320in}}%
\pgfpathlineto{\pgfqpoint{4.089159in}{0.413320in}}%
\pgfpathlineto{\pgfqpoint{4.086385in}{0.413320in}}%
\pgfpathlineto{\pgfqpoint{4.083870in}{0.413320in}}%
\pgfpathlineto{\pgfqpoint{4.081018in}{0.413320in}}%
\pgfpathlineto{\pgfqpoint{4.078471in}{0.413320in}}%
\pgfpathlineto{\pgfqpoint{4.075705in}{0.413320in}}%
\pgfpathlineto{\pgfqpoint{4.072985in}{0.413320in}}%
\pgfpathlineto{\pgfqpoint{4.070313in}{0.413320in}}%
\pgfpathlineto{\pgfqpoint{4.067636in}{0.413320in}}%
\pgfpathlineto{\pgfqpoint{4.064957in}{0.413320in}}%
\pgfpathlineto{\pgfqpoint{4.062266in}{0.413320in}}%
\pgfpathlineto{\pgfqpoint{4.059702in}{0.413320in}}%
\pgfpathlineto{\pgfqpoint{4.056911in}{0.413320in}}%
\pgfpathlineto{\pgfqpoint{4.054326in}{0.413320in}}%
\pgfpathlineto{\pgfqpoint{4.051557in}{0.413320in}}%
\pgfpathlineto{\pgfqpoint{4.049006in}{0.413320in}}%
\pgfpathlineto{\pgfqpoint{4.046210in}{0.413320in}}%
\pgfpathlineto{\pgfqpoint{4.043667in}{0.413320in}}%
\pgfpathlineto{\pgfqpoint{4.040852in}{0.413320in}}%
\pgfpathlineto{\pgfqpoint{4.038174in}{0.413320in}}%
\pgfpathlineto{\pgfqpoint{4.035492in}{0.413320in}}%
\pgfpathlineto{\pgfqpoint{4.032817in}{0.413320in}}%
\pgfpathlineto{\pgfqpoint{4.030229in}{0.413320in}}%
\pgfpathlineto{\pgfqpoint{4.027447in}{0.413320in}}%
\pgfpathlineto{\pgfqpoint{4.024868in}{0.413320in}}%
\pgfpathlineto{\pgfqpoint{4.022097in}{0.413320in}}%
\pgfpathlineto{\pgfqpoint{4.019518in}{0.413320in}}%
\pgfpathlineto{\pgfqpoint{4.016744in}{0.413320in}}%
\pgfpathlineto{\pgfqpoint{4.014186in}{0.413320in}}%
\pgfpathlineto{\pgfqpoint{4.011394in}{0.413320in}}%
\pgfpathlineto{\pgfqpoint{4.008699in}{0.413320in}}%
\pgfpathlineto{\pgfqpoint{4.006034in}{0.413320in}}%
\pgfpathlineto{\pgfqpoint{4.003348in}{0.413320in}}%
\pgfpathlineto{\pgfqpoint{4.000674in}{0.413320in}}%
\pgfpathlineto{\pgfqpoint{3.997990in}{0.413320in}}%
\pgfpathlineto{\pgfqpoint{3.995417in}{0.413320in}}%
\pgfpathlineto{\pgfqpoint{3.992642in}{0.413320in}}%
\pgfpathlineto{\pgfqpoint{3.990055in}{0.413320in}}%
\pgfpathlineto{\pgfqpoint{3.987270in}{0.413320in}}%
\pgfpathlineto{\pgfqpoint{3.984714in}{0.413320in}}%
\pgfpathlineto{\pgfqpoint{3.981929in}{0.413320in}}%
\pgfpathlineto{\pgfqpoint{3.979389in}{0.413320in}}%
\pgfpathlineto{\pgfqpoint{3.976563in}{0.413320in}}%
\pgfpathlineto{\pgfqpoint{3.973885in}{0.413320in}}%
\pgfpathlineto{\pgfqpoint{3.971250in}{0.413320in}}%
\pgfpathlineto{\pgfqpoint{3.968523in}{0.413320in}}%
\pgfpathlineto{\pgfqpoint{3.966013in}{0.413320in}}%
\pgfpathlineto{\pgfqpoint{3.963176in}{0.413320in}}%
\pgfpathlineto{\pgfqpoint{3.960635in}{0.413320in}}%
\pgfpathlineto{\pgfqpoint{3.957823in}{0.413320in}}%
\pgfpathlineto{\pgfqpoint{3.955211in}{0.413320in}}%
\pgfpathlineto{\pgfqpoint{3.952464in}{0.413320in}}%
\pgfpathlineto{\pgfqpoint{3.949894in}{0.413320in}}%
\pgfpathlineto{\pgfqpoint{3.947101in}{0.413320in}}%
\pgfpathlineto{\pgfqpoint{3.944431in}{0.413320in}}%
\pgfpathlineto{\pgfqpoint{3.941778in}{0.413320in}}%
\pgfpathlineto{\pgfqpoint{3.939075in}{0.413320in}}%
\pgfpathlineto{\pgfqpoint{3.936395in}{0.413320in}}%
\pgfpathlineto{\pgfqpoint{3.933714in}{0.413320in}}%
\pgfpathlineto{\pgfqpoint{3.931202in}{0.413320in}}%
\pgfpathlineto{\pgfqpoint{3.928347in}{0.413320in}}%
\pgfpathlineto{\pgfqpoint{3.925778in}{0.413320in}}%
\pgfpathlineto{\pgfqpoint{3.923005in}{0.413320in}}%
\pgfpathlineto{\pgfqpoint{3.920412in}{0.413320in}}%
\pgfpathlineto{\pgfqpoint{3.917646in}{0.413320in}}%
\pgfpathlineto{\pgfqpoint{3.915107in}{0.413320in}}%
\pgfpathlineto{\pgfqpoint{3.912296in}{0.413320in}}%
\pgfpathlineto{\pgfqpoint{3.909602in}{0.413320in}}%
\pgfpathlineto{\pgfqpoint{3.906918in}{0.413320in}}%
\pgfpathlineto{\pgfqpoint{3.904252in}{0.413320in}}%
\pgfpathlineto{\pgfqpoint{3.901573in}{0.413320in}}%
\pgfpathlineto{\pgfqpoint{3.898891in}{0.413320in}}%
\pgfpathlineto{\pgfqpoint{3.896345in}{0.413320in}}%
\pgfpathlineto{\pgfqpoint{3.893541in}{0.413320in}}%
\pgfpathlineto{\pgfqpoint{3.890926in}{0.413320in}}%
\pgfpathlineto{\pgfqpoint{3.888188in}{0.413320in}}%
\pgfpathlineto{\pgfqpoint{3.885621in}{0.413320in}}%
\pgfpathlineto{\pgfqpoint{3.882850in}{0.413320in}}%
\pgfpathlineto{\pgfqpoint{3.880237in}{0.413320in}}%
\pgfpathlineto{\pgfqpoint{3.877466in}{0.413320in}}%
\pgfpathlineto{\pgfqpoint{3.874790in}{0.413320in}}%
\pgfpathlineto{\pgfqpoint{3.872114in}{0.413320in}}%
\pgfpathlineto{\pgfqpoint{3.869435in}{0.413320in}}%
\pgfpathlineto{\pgfqpoint{3.866815in}{0.413320in}}%
\pgfpathlineto{\pgfqpoint{3.864073in}{0.413320in}}%
\pgfpathlineto{\pgfqpoint{3.861561in}{0.413320in}}%
\pgfpathlineto{\pgfqpoint{3.858720in}{0.413320in}}%
\pgfpathlineto{\pgfqpoint{3.856100in}{0.413320in}}%
\pgfpathlineto{\pgfqpoint{3.853358in}{0.413320in}}%
\pgfpathlineto{\pgfqpoint{3.850814in}{0.413320in}}%
\pgfpathlineto{\pgfqpoint{3.848005in}{0.413320in}}%
\pgfpathlineto{\pgfqpoint{3.845329in}{0.413320in}}%
\pgfpathlineto{\pgfqpoint{3.842641in}{0.413320in}}%
\pgfpathlineto{\pgfqpoint{3.839960in}{0.413320in}}%
\pgfpathlineto{\pgfqpoint{3.837286in}{0.413320in}}%
\pgfpathlineto{\pgfqpoint{3.834616in}{0.413320in}}%
\pgfpathlineto{\pgfqpoint{3.832053in}{0.413320in}}%
\pgfpathlineto{\pgfqpoint{3.829252in}{0.413320in}}%
\pgfpathlineto{\pgfqpoint{3.826679in}{0.413320in}}%
\pgfpathlineto{\pgfqpoint{3.823903in}{0.413320in}}%
\pgfpathlineto{\pgfqpoint{3.821315in}{0.413320in}}%
\pgfpathlineto{\pgfqpoint{3.818546in}{0.413320in}}%
\pgfpathlineto{\pgfqpoint{3.815983in}{0.413320in}}%
\pgfpathlineto{\pgfqpoint{3.813172in}{0.413320in}}%
\pgfpathlineto{\pgfqpoint{3.810510in}{0.413320in}}%
\pgfpathlineto{\pgfqpoint{3.807832in}{0.413320in}}%
\pgfpathlineto{\pgfqpoint{3.805145in}{0.413320in}}%
\pgfpathlineto{\pgfqpoint{3.802569in}{0.413320in}}%
\pgfpathlineto{\pgfqpoint{3.799797in}{0.413320in}}%
\pgfpathlineto{\pgfqpoint{3.797265in}{0.413320in}}%
\pgfpathlineto{\pgfqpoint{3.794435in}{0.413320in}}%
\pgfpathlineto{\pgfqpoint{3.791897in}{0.413320in}}%
\pgfpathlineto{\pgfqpoint{3.789084in}{0.413320in}}%
\pgfpathlineto{\pgfqpoint{3.786504in}{0.413320in}}%
\pgfpathlineto{\pgfqpoint{3.783725in}{0.413320in}}%
\pgfpathlineto{\pgfqpoint{3.781046in}{0.413320in}}%
\pgfpathlineto{\pgfqpoint{3.778370in}{0.413320in}}%
\pgfpathlineto{\pgfqpoint{3.775691in}{0.413320in}}%
\pgfpathlineto{\pgfqpoint{3.773014in}{0.413320in}}%
\pgfpathlineto{\pgfqpoint{3.770323in}{0.413320in}}%
\pgfpathlineto{\pgfqpoint{3.767782in}{0.413320in}}%
\pgfpathlineto{\pgfqpoint{3.764966in}{0.413320in}}%
\pgfpathlineto{\pgfqpoint{3.762389in}{0.413320in}}%
\pgfpathlineto{\pgfqpoint{3.759622in}{0.413320in}}%
\pgfpathlineto{\pgfqpoint{3.757065in}{0.413320in}}%
\pgfpathlineto{\pgfqpoint{3.754265in}{0.413320in}}%
\pgfpathlineto{\pgfqpoint{3.751728in}{0.413320in}}%
\pgfpathlineto{\pgfqpoint{3.748903in}{0.413320in}}%
\pgfpathlineto{\pgfqpoint{3.746229in}{0.413320in}}%
\pgfpathlineto{\pgfqpoint{3.743548in}{0.413320in}}%
\pgfpathlineto{\pgfqpoint{3.740874in}{0.413320in}}%
\pgfpathlineto{\pgfqpoint{3.738194in}{0.413320in}}%
\pgfpathlineto{\pgfqpoint{3.735509in}{0.413320in}}%
\pgfpathlineto{\pgfqpoint{3.732950in}{0.413320in}}%
\pgfpathlineto{\pgfqpoint{3.730158in}{0.413320in}}%
\pgfpathlineto{\pgfqpoint{3.727581in}{0.413320in}}%
\pgfpathlineto{\pgfqpoint{3.724804in}{0.413320in}}%
\pgfpathlineto{\pgfqpoint{3.722228in}{0.413320in}}%
\pgfpathlineto{\pgfqpoint{3.719446in}{0.413320in}}%
\pgfpathlineto{\pgfqpoint{3.716875in}{0.413320in}}%
\pgfpathlineto{\pgfqpoint{3.714086in}{0.413320in}}%
\pgfpathlineto{\pgfqpoint{3.711410in}{0.413320in}}%
\pgfpathlineto{\pgfqpoint{3.708729in}{0.413320in}}%
\pgfpathlineto{\pgfqpoint{3.706053in}{0.413320in}}%
\pgfpathlineto{\pgfqpoint{3.703460in}{0.413320in}}%
\pgfpathlineto{\pgfqpoint{3.700684in}{0.413320in}}%
\pgfpathlineto{\pgfqpoint{3.698125in}{0.413320in}}%
\pgfpathlineto{\pgfqpoint{3.695331in}{0.413320in}}%
\pgfpathlineto{\pgfqpoint{3.692765in}{0.413320in}}%
\pgfpathlineto{\pgfqpoint{3.689983in}{0.413320in}}%
\pgfpathlineto{\pgfqpoint{3.687442in}{0.413320in}}%
\pgfpathlineto{\pgfqpoint{3.684620in}{0.413320in}}%
\pgfpathlineto{\pgfqpoint{3.681948in}{0.413320in}}%
\pgfpathlineto{\pgfqpoint{3.679273in}{0.413320in}}%
\pgfpathlineto{\pgfqpoint{3.676591in}{0.413320in}}%
\pgfpathlineto{\pgfqpoint{3.673911in}{0.413320in}}%
\pgfpathlineto{\pgfqpoint{3.671232in}{0.413320in}}%
\pgfpathlineto{\pgfqpoint{3.668665in}{0.413320in}}%
\pgfpathlineto{\pgfqpoint{3.665864in}{0.413320in}}%
\pgfpathlineto{\pgfqpoint{3.663276in}{0.413320in}}%
\pgfpathlineto{\pgfqpoint{3.660515in}{0.413320in}}%
\pgfpathlineto{\pgfqpoint{3.657917in}{0.413320in}}%
\pgfpathlineto{\pgfqpoint{3.655165in}{0.413320in}}%
\pgfpathlineto{\pgfqpoint{3.652628in}{0.413320in}}%
\pgfpathlineto{\pgfqpoint{3.649837in}{0.413320in}}%
\pgfpathlineto{\pgfqpoint{3.647130in}{0.413320in}}%
\pgfpathlineto{\pgfqpoint{3.644452in}{0.413320in}}%
\pgfpathlineto{\pgfqpoint{3.641773in}{0.413320in}}%
\pgfpathlineto{\pgfqpoint{3.639207in}{0.413320in}}%
\pgfpathlineto{\pgfqpoint{3.636413in}{0.413320in}}%
\pgfpathlineto{\pgfqpoint{3.633858in}{0.413320in}}%
\pgfpathlineto{\pgfqpoint{3.631058in}{0.413320in}}%
\pgfpathlineto{\pgfqpoint{3.628460in}{0.413320in}}%
\pgfpathlineto{\pgfqpoint{3.625689in}{0.413320in}}%
\pgfpathlineto{\pgfqpoint{3.623165in}{0.413320in}}%
\pgfpathlineto{\pgfqpoint{3.620345in}{0.413320in}}%
\pgfpathlineto{\pgfqpoint{3.617667in}{0.413320in}}%
\pgfpathlineto{\pgfqpoint{3.614982in}{0.413320in}}%
\pgfpathlineto{\pgfqpoint{3.612311in}{0.413320in}}%
\pgfpathlineto{\pgfqpoint{3.609632in}{0.413320in}}%
\pgfpathlineto{\pgfqpoint{3.606951in}{0.413320in}}%
\pgfpathlineto{\pgfqpoint{3.604387in}{0.413320in}}%
\pgfpathlineto{\pgfqpoint{3.601590in}{0.413320in}}%
\pgfpathlineto{\pgfqpoint{3.598998in}{0.413320in}}%
\pgfpathlineto{\pgfqpoint{3.596240in}{0.413320in}}%
\pgfpathlineto{\pgfqpoint{3.593620in}{0.413320in}}%
\pgfpathlineto{\pgfqpoint{3.590883in}{0.413320in}}%
\pgfpathlineto{\pgfqpoint{3.588258in}{0.413320in}}%
\pgfpathlineto{\pgfqpoint{3.585532in}{0.413320in}}%
\pgfpathlineto{\pgfqpoint{3.582851in}{0.413320in}}%
\pgfpathlineto{\pgfqpoint{3.580191in}{0.413320in}}%
\pgfpathlineto{\pgfqpoint{3.577487in}{0.413320in}}%
\pgfpathlineto{\pgfqpoint{3.574814in}{0.413320in}}%
\pgfpathlineto{\pgfqpoint{3.572126in}{0.413320in}}%
\pgfpathlineto{\pgfqpoint{3.569584in}{0.413320in}}%
\pgfpathlineto{\pgfqpoint{3.566774in}{0.413320in}}%
\pgfpathlineto{\pgfqpoint{3.564188in}{0.413320in}}%
\pgfpathlineto{\pgfqpoint{3.561420in}{0.413320in}}%
\pgfpathlineto{\pgfqpoint{3.558853in}{0.413320in}}%
\pgfpathlineto{\pgfqpoint{3.556061in}{0.413320in}}%
\pgfpathlineto{\pgfqpoint{3.553498in}{0.413320in}}%
\pgfpathlineto{\pgfqpoint{3.550713in}{0.413320in}}%
\pgfpathlineto{\pgfqpoint{3.548029in}{0.413320in}}%
\pgfpathlineto{\pgfqpoint{3.545349in}{0.413320in}}%
\pgfpathlineto{\pgfqpoint{3.542656in}{0.413320in}}%
\pgfpathlineto{\pgfqpoint{3.540093in}{0.413320in}}%
\pgfpathlineto{\pgfqpoint{3.537309in}{0.413320in}}%
\pgfpathlineto{\pgfqpoint{3.534783in}{0.413320in}}%
\pgfpathlineto{\pgfqpoint{3.531955in}{0.413320in}}%
\pgfpathlineto{\pgfqpoint{3.529327in}{0.413320in}}%
\pgfpathlineto{\pgfqpoint{3.526601in}{0.413320in}}%
\pgfpathlineto{\pgfqpoint{3.524041in}{0.413320in}}%
\pgfpathlineto{\pgfqpoint{3.521244in}{0.413320in}}%
\pgfpathlineto{\pgfqpoint{3.518565in}{0.413320in}}%
\pgfpathlineto{\pgfqpoint{3.515884in}{0.413320in}}%
\pgfpathlineto{\pgfqpoint{3.513209in}{0.413320in}}%
\pgfpathlineto{\pgfqpoint{3.510533in}{0.413320in}}%
\pgfpathlineto{\pgfqpoint{3.507840in}{0.413320in}}%
\pgfpathlineto{\pgfqpoint{3.505262in}{0.413320in}}%
\pgfpathlineto{\pgfqpoint{3.502488in}{0.413320in}}%
\pgfpathlineto{\pgfqpoint{3.499909in}{0.413320in}}%
\pgfpathlineto{\pgfqpoint{3.497139in}{0.413320in}}%
\pgfpathlineto{\pgfqpoint{3.494581in}{0.413320in}}%
\pgfpathlineto{\pgfqpoint{3.491783in}{0.413320in}}%
\pgfpathlineto{\pgfqpoint{3.489223in}{0.413320in}}%
\pgfpathlineto{\pgfqpoint{3.486442in}{0.413320in}}%
\pgfpathlineto{\pgfqpoint{3.483744in}{0.413320in}}%
\pgfpathlineto{\pgfqpoint{3.481072in}{0.413320in}}%
\pgfpathlineto{\pgfqpoint{3.478378in}{0.413320in}}%
\pgfpathlineto{\pgfqpoint{3.475821in}{0.413320in}}%
\pgfpathlineto{\pgfqpoint{3.473021in}{0.413320in}}%
\pgfpathlineto{\pgfqpoint{3.470466in}{0.413320in}}%
\pgfpathlineto{\pgfqpoint{3.467678in}{0.413320in}}%
\pgfpathlineto{\pgfqpoint{3.465072in}{0.413320in}}%
\pgfpathlineto{\pgfqpoint{3.462321in}{0.413320in}}%
\pgfpathlineto{\pgfqpoint{3.459695in}{0.413320in}}%
\pgfpathlineto{\pgfqpoint{3.456960in}{0.413320in}}%
\pgfpathlineto{\pgfqpoint{3.454285in}{0.413320in}}%
\pgfpathlineto{\pgfqpoint{3.451597in}{0.413320in}}%
\pgfpathlineto{\pgfqpoint{3.448926in}{0.413320in}}%
\pgfpathlineto{\pgfqpoint{3.446257in}{0.413320in}}%
\pgfpathlineto{\pgfqpoint{3.443574in}{0.413320in}}%
\pgfpathlineto{\pgfqpoint{3.440996in}{0.413320in}}%
\pgfpathlineto{\pgfqpoint{3.438210in}{0.413320in}}%
\pgfpathlineto{\pgfqpoint{3.435635in}{0.413320in}}%
\pgfpathlineto{\pgfqpoint{3.432851in}{0.413320in}}%
\pgfpathlineto{\pgfqpoint{3.430313in}{0.413320in}}%
\pgfpathlineto{\pgfqpoint{3.427501in}{0.413320in}}%
\pgfpathlineto{\pgfqpoint{3.424887in}{0.413320in}}%
\pgfpathlineto{\pgfqpoint{3.422142in}{0.413320in}}%
\pgfpathlineto{\pgfqpoint{3.419455in}{0.413320in}}%
\pgfpathlineto{\pgfqpoint{3.416780in}{0.413320in}}%
\pgfpathlineto{\pgfqpoint{3.414109in}{0.413320in}}%
\pgfpathlineto{\pgfqpoint{3.411431in}{0.413320in}}%
\pgfpathlineto{\pgfqpoint{3.408752in}{0.413320in}}%
\pgfpathlineto{\pgfqpoint{3.406202in}{0.413320in}}%
\pgfpathlineto{\pgfqpoint{3.403394in}{0.413320in}}%
\pgfpathlineto{\pgfqpoint{3.400783in}{0.413320in}}%
\pgfpathlineto{\pgfqpoint{3.398037in}{0.413320in}}%
\pgfpathlineto{\pgfqpoint{3.395461in}{0.413320in}}%
\pgfpathlineto{\pgfqpoint{3.392681in}{0.413320in}}%
\pgfpathlineto{\pgfqpoint{3.390102in}{0.413320in}}%
\pgfpathlineto{\pgfqpoint{3.387309in}{0.413320in}}%
\pgfpathlineto{\pgfqpoint{3.384647in}{0.413320in}}%
\pgfpathlineto{\pgfqpoint{3.381959in}{0.413320in}}%
\pgfpathlineto{\pgfqpoint{3.379290in}{0.413320in}}%
\pgfpathlineto{\pgfqpoint{3.376735in}{0.413320in}}%
\pgfpathlineto{\pgfqpoint{3.373921in}{0.413320in}}%
\pgfpathlineto{\pgfqpoint{3.371357in}{0.413320in}}%
\pgfpathlineto{\pgfqpoint{3.368577in}{0.413320in}}%
\pgfpathlineto{\pgfqpoint{3.365996in}{0.413320in}}%
\pgfpathlineto{\pgfqpoint{3.363221in}{0.413320in}}%
\pgfpathlineto{\pgfqpoint{3.360620in}{0.413320in}}%
\pgfpathlineto{\pgfqpoint{3.357862in}{0.413320in}}%
\pgfpathlineto{\pgfqpoint{3.355177in}{0.413320in}}%
\pgfpathlineto{\pgfqpoint{3.352505in}{0.413320in}}%
\pgfpathlineto{\pgfqpoint{3.349828in}{0.413320in}}%
\pgfpathlineto{\pgfqpoint{3.347139in}{0.413320in}}%
\pgfpathlineto{\pgfqpoint{3.344468in}{0.413320in}}%
\pgfpathlineto{\pgfqpoint{3.341893in}{0.413320in}}%
\pgfpathlineto{\pgfqpoint{3.339101in}{0.413320in}}%
\pgfpathlineto{\pgfqpoint{3.336541in}{0.413320in}}%
\pgfpathlineto{\pgfqpoint{3.333758in}{0.413320in}}%
\pgfpathlineto{\pgfqpoint{3.331183in}{0.413320in}}%
\pgfpathlineto{\pgfqpoint{3.328401in}{0.413320in}}%
\pgfpathlineto{\pgfqpoint{3.325860in}{0.413320in}}%
\pgfpathlineto{\pgfqpoint{3.323049in}{0.413320in}}%
\pgfpathlineto{\pgfqpoint{3.320366in}{0.413320in}}%
\pgfpathlineto{\pgfqpoint{3.317688in}{0.413320in}}%
\pgfpathlineto{\pgfqpoint{3.315008in}{0.413320in}}%
\pgfpathlineto{\pgfqpoint{3.312480in}{0.413320in}}%
\pgfpathlineto{\pgfqpoint{3.309652in}{0.413320in}}%
\pgfpathlineto{\pgfqpoint{3.307104in}{0.413320in}}%
\pgfpathlineto{\pgfqpoint{3.304295in}{0.413320in}}%
\pgfpathlineto{\pgfqpoint{3.301719in}{0.413320in}}%
\pgfpathlineto{\pgfqpoint{3.298937in}{0.413320in}}%
\pgfpathlineto{\pgfqpoint{3.296376in}{0.413320in}}%
\pgfpathlineto{\pgfqpoint{3.293574in}{0.413320in}}%
\pgfpathlineto{\pgfqpoint{3.290890in}{0.413320in}}%
\pgfpathlineto{\pgfqpoint{3.288225in}{0.413320in}}%
\pgfpathlineto{\pgfqpoint{3.285534in}{0.413320in}}%
\pgfpathlineto{\pgfqpoint{3.282870in}{0.413320in}}%
\pgfpathlineto{\pgfqpoint{3.280189in}{0.413320in}}%
\pgfpathlineto{\pgfqpoint{3.277603in}{0.413320in}}%
\pgfpathlineto{\pgfqpoint{3.274831in}{0.413320in}}%
\pgfpathlineto{\pgfqpoint{3.272254in}{0.413320in}}%
\pgfpathlineto{\pgfqpoint{3.269478in}{0.413320in}}%
\pgfpathlineto{\pgfqpoint{3.266849in}{0.413320in}}%
\pgfpathlineto{\pgfqpoint{3.264119in}{0.413320in}}%
\pgfpathlineto{\pgfqpoint{3.261594in}{0.413320in}}%
\pgfpathlineto{\pgfqpoint{3.258784in}{0.413320in}}%
\pgfpathlineto{\pgfqpoint{3.256083in}{0.413320in}}%
\pgfpathlineto{\pgfqpoint{3.253404in}{0.413320in}}%
\pgfpathlineto{\pgfqpoint{3.250716in}{0.413320in}}%
\pgfpathlineto{\pgfqpoint{3.248049in}{0.413320in}}%
\pgfpathlineto{\pgfqpoint{3.245363in}{0.413320in}}%
\pgfpathlineto{\pgfqpoint{3.242807in}{0.413320in}}%
\pgfpathlineto{\pgfqpoint{3.240010in}{0.413320in}}%
\pgfpathlineto{\pgfqpoint{3.237411in}{0.413320in}}%
\pgfpathlineto{\pgfqpoint{3.234658in}{0.413320in}}%
\pgfpathlineto{\pgfqpoint{3.232069in}{0.413320in}}%
\pgfpathlineto{\pgfqpoint{3.229310in}{0.413320in}}%
\pgfpathlineto{\pgfqpoint{3.226609in}{0.413320in}}%
\pgfpathlineto{\pgfqpoint{3.223942in}{0.413320in}}%
\pgfpathlineto{\pgfqpoint{3.221255in}{0.413320in}}%
\pgfpathlineto{\pgfqpoint{3.218586in}{0.413320in}}%
\pgfpathlineto{\pgfqpoint{3.215908in}{0.413320in}}%
\pgfpathlineto{\pgfqpoint{3.213342in}{0.413320in}}%
\pgfpathlineto{\pgfqpoint{3.210545in}{0.413320in}}%
\pgfpathlineto{\pgfqpoint{3.207984in}{0.413320in}}%
\pgfpathlineto{\pgfqpoint{3.205195in}{0.413320in}}%
\pgfpathlineto{\pgfqpoint{3.202562in}{0.413320in}}%
\pgfpathlineto{\pgfqpoint{3.199823in}{0.413320in}}%
\pgfpathlineto{\pgfqpoint{3.197226in}{0.413320in}}%
\pgfpathlineto{\pgfqpoint{3.194508in}{0.413320in}}%
\pgfpathlineto{\pgfqpoint{3.191796in}{0.413320in}}%
\pgfpathlineto{\pgfqpoint{3.189117in}{0.413320in}}%
\pgfpathlineto{\pgfqpoint{3.186440in}{0.413320in}}%
\pgfpathlineto{\pgfqpoint{3.183760in}{0.413320in}}%
\pgfpathlineto{\pgfqpoint{3.181089in}{0.413320in}}%
\pgfpathlineto{\pgfqpoint{3.178525in}{0.413320in}}%
\pgfpathlineto{\pgfqpoint{3.175724in}{0.413320in}}%
\pgfpathlineto{\pgfqpoint{3.173142in}{0.413320in}}%
\pgfpathlineto{\pgfqpoint{3.170375in}{0.413320in}}%
\pgfpathlineto{\pgfqpoint{3.167776in}{0.413320in}}%
\pgfpathlineto{\pgfqpoint{3.165019in}{0.413320in}}%
\pgfpathlineto{\pgfqpoint{3.162474in}{0.413320in}}%
\pgfpathlineto{\pgfqpoint{3.159675in}{0.413320in}}%
\pgfpathlineto{\pgfqpoint{3.156981in}{0.413320in}}%
\pgfpathlineto{\pgfqpoint{3.154327in}{0.413320in}}%
\pgfpathlineto{\pgfqpoint{3.151612in}{0.413320in}}%
\pgfpathlineto{\pgfqpoint{3.149057in}{0.413320in}}%
\pgfpathlineto{\pgfqpoint{3.146271in}{0.413320in}}%
\pgfpathlineto{\pgfqpoint{3.143740in}{0.413320in}}%
\pgfpathlineto{\pgfqpoint{3.140913in}{0.413320in}}%
\pgfpathlineto{\pgfqpoint{3.138375in}{0.413320in}}%
\pgfpathlineto{\pgfqpoint{3.135550in}{0.413320in}}%
\pgfpathlineto{\pgfqpoint{3.132946in}{0.413320in}}%
\pgfpathlineto{\pgfqpoint{3.130199in}{0.413320in}}%
\pgfpathlineto{\pgfqpoint{3.127512in}{0.413320in}}%
\pgfpathlineto{\pgfqpoint{3.124842in}{0.413320in}}%
\pgfpathlineto{\pgfqpoint{3.122163in}{0.413320in}}%
\pgfpathlineto{\pgfqpoint{3.119487in}{0.413320in}}%
\pgfpathlineto{\pgfqpoint{3.116807in}{0.413320in}}%
\pgfpathlineto{\pgfqpoint{3.114242in}{0.413320in}}%
\pgfpathlineto{\pgfqpoint{3.111451in}{0.413320in}}%
\pgfpathlineto{\pgfqpoint{3.108896in}{0.413320in}}%
\pgfpathlineto{\pgfqpoint{3.106094in}{0.413320in}}%
\pgfpathlineto{\pgfqpoint{3.103508in}{0.413320in}}%
\pgfpathlineto{\pgfqpoint{3.100737in}{0.413320in}}%
\pgfpathlineto{\pgfqpoint{3.098163in}{0.413320in}}%
\pgfpathlineto{\pgfqpoint{3.095388in}{0.413320in}}%
\pgfpathlineto{\pgfqpoint{3.092699in}{0.413320in}}%
\pgfpathlineto{\pgfqpoint{3.090023in}{0.413320in}}%
\pgfpathlineto{\pgfqpoint{3.087343in}{0.413320in}}%
\pgfpathlineto{\pgfqpoint{3.084671in}{0.413320in}}%
\pgfpathlineto{\pgfqpoint{3.081990in}{0.413320in}}%
\pgfpathlineto{\pgfqpoint{3.079381in}{0.413320in}}%
\pgfpathlineto{\pgfqpoint{3.076631in}{0.413320in}}%
\pgfpathlineto{\pgfqpoint{3.074056in}{0.413320in}}%
\pgfpathlineto{\pgfqpoint{3.071266in}{0.413320in}}%
\pgfpathlineto{\pgfqpoint{3.068709in}{0.413320in}}%
\pgfpathlineto{\pgfqpoint{3.065916in}{0.413320in}}%
\pgfpathlineto{\pgfqpoint{3.063230in}{0.413320in}}%
\pgfpathlineto{\pgfqpoint{3.060561in}{0.413320in}}%
\pgfpathlineto{\pgfqpoint{3.057884in}{0.413320in}}%
\pgfpathlineto{\pgfqpoint{3.055202in}{0.413320in}}%
\pgfpathlineto{\pgfqpoint{3.052526in}{0.413320in}}%
\pgfpathlineto{\pgfqpoint{3.049988in}{0.413320in}}%
\pgfpathlineto{\pgfqpoint{3.047157in}{0.413320in}}%
\pgfpathlineto{\pgfqpoint{3.044568in}{0.413320in}}%
\pgfpathlineto{\pgfqpoint{3.041813in}{0.413320in}}%
\pgfpathlineto{\pgfqpoint{3.039262in}{0.413320in}}%
\pgfpathlineto{\pgfqpoint{3.036456in}{0.413320in}}%
\pgfpathlineto{\pgfqpoint{3.033921in}{0.413320in}}%
\pgfpathlineto{\pgfqpoint{3.031091in}{0.413320in}}%
\pgfpathlineto{\pgfqpoint{3.028412in}{0.413320in}}%
\pgfpathlineto{\pgfqpoint{3.025803in}{0.413320in}}%
\pgfpathlineto{\pgfqpoint{3.023058in}{0.413320in}}%
\pgfpathlineto{\pgfqpoint{3.020382in}{0.413320in}}%
\pgfpathlineto{\pgfqpoint{3.017707in}{0.413320in}}%
\pgfpathlineto{\pgfqpoint{3.015097in}{0.413320in}}%
\pgfpathlineto{\pgfqpoint{3.012351in}{0.413320in}}%
\pgfpathlineto{\pgfqpoint{3.009784in}{0.413320in}}%
\pgfpathlineto{\pgfqpoint{3.006993in}{0.413320in}}%
\pgfpathlineto{\pgfqpoint{3.004419in}{0.413320in}}%
\pgfpathlineto{\pgfqpoint{3.001635in}{0.413320in}}%
\pgfpathlineto{\pgfqpoint{2.999103in}{0.413320in}}%
\pgfpathlineto{\pgfqpoint{2.996300in}{0.413320in}}%
\pgfpathlineto{\pgfqpoint{2.993595in}{0.413320in}}%
\pgfpathlineto{\pgfqpoint{2.990978in}{0.413320in}}%
\pgfpathlineto{\pgfqpoint{2.988238in}{0.413320in}}%
\pgfpathlineto{\pgfqpoint{2.985666in}{0.413320in}}%
\pgfpathlineto{\pgfqpoint{2.982885in}{0.413320in}}%
\pgfpathlineto{\pgfqpoint{2.980341in}{0.413320in}}%
\pgfpathlineto{\pgfqpoint{2.977517in}{0.413320in}}%
\pgfpathlineto{\pgfqpoint{2.974972in}{0.413320in}}%
\pgfpathlineto{\pgfqpoint{2.972177in}{0.413320in}}%
\pgfpathlineto{\pgfqpoint{2.969599in}{0.413320in}}%
\pgfpathlineto{\pgfqpoint{2.966812in}{0.413320in}}%
\pgfpathlineto{\pgfqpoint{2.964127in}{0.413320in}}%
\pgfpathlineto{\pgfqpoint{2.961460in}{0.413320in}}%
\pgfpathlineto{\pgfqpoint{2.958782in}{0.413320in}}%
\pgfpathlineto{\pgfqpoint{2.956103in}{0.413320in}}%
\pgfpathlineto{\pgfqpoint{2.953422in}{0.413320in}}%
\pgfpathlineto{\pgfqpoint{2.950884in}{0.413320in}}%
\pgfpathlineto{\pgfqpoint{2.948068in}{0.413320in}}%
\pgfpathlineto{\pgfqpoint{2.945461in}{0.413320in}}%
\pgfpathlineto{\pgfqpoint{2.942711in}{0.413320in}}%
\pgfpathlineto{\pgfqpoint{2.940120in}{0.413320in}}%
\pgfpathlineto{\pgfqpoint{2.937352in}{0.413320in}}%
\pgfpathlineto{\pgfqpoint{2.934759in}{0.413320in}}%
\pgfpathlineto{\pgfqpoint{2.932033in}{0.413320in}}%
\pgfpathlineto{\pgfqpoint{2.929321in}{0.413320in}}%
\pgfpathlineto{\pgfqpoint{2.926655in}{0.413320in}}%
\pgfpathlineto{\pgfqpoint{2.923963in}{0.413320in}}%
\pgfpathlineto{\pgfqpoint{2.921363in}{0.413320in}}%
\pgfpathlineto{\pgfqpoint{2.918606in}{0.413320in}}%
\pgfpathlineto{\pgfqpoint{2.916061in}{0.413320in}}%
\pgfpathlineto{\pgfqpoint{2.913243in}{0.413320in}}%
\pgfpathlineto{\pgfqpoint{2.910631in}{0.413320in}}%
\pgfpathlineto{\pgfqpoint{2.907882in}{0.413320in}}%
\pgfpathlineto{\pgfqpoint{2.905341in}{0.413320in}}%
\pgfpathlineto{\pgfqpoint{2.902535in}{0.413320in}}%
\pgfpathlineto{\pgfqpoint{2.899858in}{0.413320in}}%
\pgfpathlineto{\pgfqpoint{2.897179in}{0.413320in}}%
\pgfpathlineto{\pgfqpoint{2.894487in}{0.413320in}}%
\pgfpathlineto{\pgfqpoint{2.891809in}{0.413320in}}%
\pgfpathlineto{\pgfqpoint{2.889145in}{0.413320in}}%
\pgfpathlineto{\pgfqpoint{2.886578in}{0.413320in}}%
\pgfpathlineto{\pgfqpoint{2.883780in}{0.413320in}}%
\pgfpathlineto{\pgfqpoint{2.881254in}{0.413320in}}%
\pgfpathlineto{\pgfqpoint{2.878431in}{0.413320in}}%
\pgfpathlineto{\pgfqpoint{2.875882in}{0.413320in}}%
\pgfpathlineto{\pgfqpoint{2.873074in}{0.413320in}}%
\pgfpathlineto{\pgfqpoint{2.870475in}{0.413320in}}%
\pgfpathlineto{\pgfqpoint{2.867713in}{0.413320in}}%
\pgfpathlineto{\pgfqpoint{2.865031in}{0.413320in}}%
\pgfpathlineto{\pgfqpoint{2.862402in}{0.413320in}}%
\pgfpathlineto{\pgfqpoint{2.859668in}{0.413320in}}%
\pgfpathlineto{\pgfqpoint{2.857003in}{0.413320in}}%
\pgfpathlineto{\pgfqpoint{2.854325in}{0.413320in}}%
\pgfpathlineto{\pgfqpoint{2.851793in}{0.413320in}}%
\pgfpathlineto{\pgfqpoint{2.848960in}{0.413320in}}%
\pgfpathlineto{\pgfqpoint{2.846408in}{0.413320in}}%
\pgfpathlineto{\pgfqpoint{2.843611in}{0.413320in}}%
\pgfpathlineto{\pgfqpoint{2.841055in}{0.413320in}}%
\pgfpathlineto{\pgfqpoint{2.838254in}{0.413320in}}%
\pgfpathlineto{\pgfqpoint{2.835698in}{0.413320in}}%
\pgfpathlineto{\pgfqpoint{2.832894in}{0.413320in}}%
\pgfpathlineto{\pgfqpoint{2.830219in}{0.413320in}}%
\pgfpathlineto{\pgfqpoint{2.827567in}{0.413320in}}%
\pgfpathlineto{\pgfqpoint{2.824851in}{0.413320in}}%
\pgfpathlineto{\pgfqpoint{2.822303in}{0.413320in}}%
\pgfpathlineto{\pgfqpoint{2.819506in}{0.413320in}}%
\pgfpathlineto{\pgfqpoint{2.816867in}{0.413320in}}%
\pgfpathlineto{\pgfqpoint{2.814141in}{0.413320in}}%
\pgfpathlineto{\pgfqpoint{2.811597in}{0.413320in}}%
\pgfpathlineto{\pgfqpoint{2.808792in}{0.413320in}}%
\pgfpathlineto{\pgfqpoint{2.806175in}{0.413320in}}%
\pgfpathlineto{\pgfqpoint{2.803435in}{0.413320in}}%
\pgfpathlineto{\pgfqpoint{2.800756in}{0.413320in}}%
\pgfpathlineto{\pgfqpoint{2.798070in}{0.413320in}}%
\pgfpathlineto{\pgfqpoint{2.795398in}{0.413320in}}%
\pgfpathlineto{\pgfqpoint{2.792721in}{0.413320in}}%
\pgfpathlineto{\pgfqpoint{2.790044in}{0.413320in}}%
\pgfpathlineto{\pgfqpoint{2.787468in}{0.413320in}}%
\pgfpathlineto{\pgfqpoint{2.784687in}{0.413320in}}%
\pgfpathlineto{\pgfqpoint{2.782113in}{0.413320in}}%
\pgfpathlineto{\pgfqpoint{2.779330in}{0.413320in}}%
\pgfpathlineto{\pgfqpoint{2.776767in}{0.413320in}}%
\pgfpathlineto{\pgfqpoint{2.773972in}{0.413320in}}%
\pgfpathlineto{\pgfqpoint{2.771373in}{0.413320in}}%
\pgfpathlineto{\pgfqpoint{2.768617in}{0.413320in}}%
\pgfpathlineto{\pgfqpoint{2.765935in}{0.413320in}}%
\pgfpathlineto{\pgfqpoint{2.763253in}{0.413320in}}%
\pgfpathlineto{\pgfqpoint{2.760581in}{0.413320in}}%
\pgfpathlineto{\pgfqpoint{2.758028in}{0.413320in}}%
\pgfpathlineto{\pgfqpoint{2.755224in}{0.413320in}}%
\pgfpathlineto{\pgfqpoint{2.752614in}{0.413320in}}%
\pgfpathlineto{\pgfqpoint{2.749868in}{0.413320in}}%
\pgfpathlineto{\pgfqpoint{2.747260in}{0.413320in}}%
\pgfpathlineto{\pgfqpoint{2.744510in}{0.413320in}}%
\pgfpathlineto{\pgfqpoint{2.741928in}{0.413320in}}%
\pgfpathlineto{\pgfqpoint{2.739155in}{0.413320in}}%
\pgfpathlineto{\pgfqpoint{2.736476in}{0.413320in}}%
\pgfpathlineto{\pgfqpoint{2.733798in}{0.413320in}}%
\pgfpathlineto{\pgfqpoint{2.731119in}{0.413320in}}%
\pgfpathlineto{\pgfqpoint{2.728439in}{0.413320in}}%
\pgfpathlineto{\pgfqpoint{2.725760in}{0.413320in}}%
\pgfpathlineto{\pgfqpoint{2.723211in}{0.413320in}}%
\pgfpathlineto{\pgfqpoint{2.720404in}{0.413320in}}%
\pgfpathlineto{\pgfqpoint{2.717773in}{0.413320in}}%
\pgfpathlineto{\pgfqpoint{2.715036in}{0.413320in}}%
\pgfpathlineto{\pgfqpoint{2.712477in}{0.413320in}}%
\pgfpathlineto{\pgfqpoint{2.709683in}{0.413320in}}%
\pgfpathlineto{\pgfqpoint{2.707125in}{0.413320in}}%
\pgfpathlineto{\pgfqpoint{2.704326in}{0.413320in}}%
\pgfpathlineto{\pgfqpoint{2.701657in}{0.413320in}}%
\pgfpathlineto{\pgfqpoint{2.698968in}{0.413320in}}%
\pgfpathlineto{\pgfqpoint{2.696293in}{0.413320in}}%
\pgfpathlineto{\pgfqpoint{2.693611in}{0.413320in}}%
\pgfpathlineto{\pgfqpoint{2.690940in}{0.413320in}}%
\pgfpathlineto{\pgfqpoint{2.688328in}{0.413320in}}%
\pgfpathlineto{\pgfqpoint{2.685586in}{0.413320in}}%
\pgfpathlineto{\pgfqpoint{2.683009in}{0.413320in}}%
\pgfpathlineto{\pgfqpoint{2.680224in}{0.413320in}}%
\pgfpathlineto{\pgfqpoint{2.677650in}{0.413320in}}%
\pgfpathlineto{\pgfqpoint{2.674873in}{0.413320in}}%
\pgfpathlineto{\pgfqpoint{2.672301in}{0.413320in}}%
\pgfpathlineto{\pgfqpoint{2.669506in}{0.413320in}}%
\pgfpathlineto{\pgfqpoint{2.666836in}{0.413320in}}%
\pgfpathlineto{\pgfqpoint{2.664151in}{0.413320in}}%
\pgfpathlineto{\pgfqpoint{2.661481in}{0.413320in}}%
\pgfpathlineto{\pgfqpoint{2.658942in}{0.413320in}}%
\pgfpathlineto{\pgfqpoint{2.656124in}{0.413320in}}%
\pgfpathlineto{\pgfqpoint{2.653567in}{0.413320in}}%
\pgfpathlineto{\pgfqpoint{2.650767in}{0.413320in}}%
\pgfpathlineto{\pgfqpoint{2.648196in}{0.413320in}}%
\pgfpathlineto{\pgfqpoint{2.645408in}{0.413320in}}%
\pgfpathlineto{\pgfqpoint{2.642827in}{0.413320in}}%
\pgfpathlineto{\pgfqpoint{2.640053in}{0.413320in}}%
\pgfpathlineto{\pgfqpoint{2.637369in}{0.413320in}}%
\pgfpathlineto{\pgfqpoint{2.634700in}{0.413320in}}%
\pgfpathlineto{\pgfqpoint{2.632018in}{0.413320in}}%
\pgfpathlineto{\pgfqpoint{2.629340in}{0.413320in}}%
\pgfpathlineto{\pgfqpoint{2.626653in}{0.413320in}}%
\pgfpathlineto{\pgfqpoint{2.624077in}{0.413320in}}%
\pgfpathlineto{\pgfqpoint{2.621304in}{0.413320in}}%
\pgfpathlineto{\pgfqpoint{2.618773in}{0.413320in}}%
\pgfpathlineto{\pgfqpoint{2.615934in}{0.413320in}}%
\pgfpathlineto{\pgfqpoint{2.613393in}{0.413320in}}%
\pgfpathlineto{\pgfqpoint{2.610588in}{0.413320in}}%
\pgfpathlineto{\pgfqpoint{2.608004in}{0.413320in}}%
\pgfpathlineto{\pgfqpoint{2.605232in}{0.413320in}}%
\pgfpathlineto{\pgfqpoint{2.602557in}{0.413320in}}%
\pgfpathlineto{\pgfqpoint{2.599920in}{0.413320in}}%
\pgfpathlineto{\pgfqpoint{2.597196in}{0.413320in}}%
\pgfpathlineto{\pgfqpoint{2.594630in}{0.413320in}}%
\pgfpathlineto{\pgfqpoint{2.591842in}{0.413320in}}%
\pgfpathlineto{\pgfqpoint{2.589248in}{0.413320in}}%
\pgfpathlineto{\pgfqpoint{2.586484in}{0.413320in}}%
\pgfpathlineto{\pgfqpoint{2.583913in}{0.413320in}}%
\pgfpathlineto{\pgfqpoint{2.581129in}{0.413320in}}%
\pgfpathlineto{\pgfqpoint{2.578567in}{0.413320in}}%
\pgfpathlineto{\pgfqpoint{2.575779in}{0.413320in}}%
\pgfpathlineto{\pgfqpoint{2.573082in}{0.413320in}}%
\pgfpathlineto{\pgfqpoint{2.570411in}{0.413320in}}%
\pgfpathlineto{\pgfqpoint{2.567730in}{0.413320in}}%
\pgfpathlineto{\pgfqpoint{2.565045in}{0.413320in}}%
\pgfpathlineto{\pgfqpoint{2.562375in}{0.413320in}}%
\pgfpathlineto{\pgfqpoint{2.559790in}{0.413320in}}%
\pgfpathlineto{\pgfqpoint{2.557009in}{0.413320in}}%
\pgfpathlineto{\pgfqpoint{2.554493in}{0.413320in}}%
\pgfpathlineto{\pgfqpoint{2.551664in}{0.413320in}}%
\pgfpathlineto{\pgfqpoint{2.549114in}{0.413320in}}%
\pgfpathlineto{\pgfqpoint{2.546310in}{0.413320in}}%
\pgfpathlineto{\pgfqpoint{2.543765in}{0.413320in}}%
\pgfpathlineto{\pgfqpoint{2.540949in}{0.413320in}}%
\pgfpathlineto{\pgfqpoint{2.538274in}{0.413320in}}%
\pgfpathlineto{\pgfqpoint{2.535624in}{0.413320in}}%
\pgfpathlineto{\pgfqpoint{2.532917in}{0.413320in}}%
\pgfpathlineto{\pgfqpoint{2.530234in}{0.413320in}}%
\pgfpathlineto{\pgfqpoint{2.527560in}{0.413320in}}%
\pgfpathlineto{\pgfqpoint{2.524988in}{0.413320in}}%
\pgfpathlineto{\pgfqpoint{2.522197in}{0.413320in}}%
\pgfpathlineto{\pgfqpoint{2.519607in}{0.413320in}}%
\pgfpathlineto{\pgfqpoint{2.516845in}{0.413320in}}%
\pgfpathlineto{\pgfqpoint{2.514268in}{0.413320in}}%
\pgfpathlineto{\pgfqpoint{2.511478in}{0.413320in}}%
\pgfpathlineto{\pgfqpoint{2.508917in}{0.413320in}}%
\pgfpathlineto{\pgfqpoint{2.506163in}{0.413320in}}%
\pgfpathlineto{\pgfqpoint{2.503454in}{0.413320in}}%
\pgfpathlineto{\pgfqpoint{2.500801in}{0.413320in}}%
\pgfpathlineto{\pgfqpoint{2.498085in}{0.413320in}}%
\pgfpathlineto{\pgfqpoint{2.495542in}{0.413320in}}%
\pgfpathlineto{\pgfqpoint{2.492729in}{0.413320in}}%
\pgfpathlineto{\pgfqpoint{2.490183in}{0.413320in}}%
\pgfpathlineto{\pgfqpoint{2.487384in}{0.413320in}}%
\pgfpathlineto{\pgfqpoint{2.484870in}{0.413320in}}%
\pgfpathlineto{\pgfqpoint{2.482026in}{0.413320in}}%
\pgfpathlineto{\pgfqpoint{2.479420in}{0.413320in}}%
\pgfpathlineto{\pgfqpoint{2.476671in}{0.413320in}}%
\pgfpathlineto{\pgfqpoint{2.473989in}{0.413320in}}%
\pgfpathlineto{\pgfqpoint{2.471311in}{0.413320in}}%
\pgfpathlineto{\pgfqpoint{2.468635in}{0.413320in}}%
\pgfpathlineto{\pgfqpoint{2.465957in}{0.413320in}}%
\pgfpathlineto{\pgfqpoint{2.463280in}{0.413320in}}%
\pgfpathlineto{\pgfqpoint{2.460711in}{0.413320in}}%
\pgfpathlineto{\pgfqpoint{2.457917in}{0.413320in}}%
\pgfpathlineto{\pgfqpoint{2.455353in}{0.413320in}}%
\pgfpathlineto{\pgfqpoint{2.452562in}{0.413320in}}%
\pgfpathlineto{\pgfqpoint{2.450032in}{0.413320in}}%
\pgfpathlineto{\pgfqpoint{2.447209in}{0.413320in}}%
\pgfpathlineto{\pgfqpoint{2.444677in}{0.413320in}}%
\pgfpathlineto{\pgfqpoint{2.441876in}{0.413320in}}%
\pgfpathlineto{\pgfqpoint{2.439167in}{0.413320in}}%
\pgfpathlineto{\pgfqpoint{2.436518in}{0.413320in}}%
\pgfpathlineto{\pgfqpoint{2.433815in}{0.413320in}}%
\pgfpathlineto{\pgfqpoint{2.431251in}{0.413320in}}%
\pgfpathlineto{\pgfqpoint{2.428453in}{0.413320in}}%
\pgfpathlineto{\pgfqpoint{2.425878in}{0.413320in}}%
\pgfpathlineto{\pgfqpoint{2.423098in}{0.413320in}}%
\pgfpathlineto{\pgfqpoint{2.420528in}{0.413320in}}%
\pgfpathlineto{\pgfqpoint{2.417747in}{0.413320in}}%
\pgfpathlineto{\pgfqpoint{2.415184in}{0.413320in}}%
\pgfpathlineto{\pgfqpoint{2.412389in}{0.413320in}}%
\pgfpathlineto{\pgfqpoint{2.409699in}{0.413320in}}%
\pgfpathlineto{\pgfqpoint{2.407024in}{0.413320in}}%
\pgfpathlineto{\pgfqpoint{2.404352in}{0.413320in}}%
\pgfpathlineto{\pgfqpoint{2.401675in}{0.413320in}}%
\pgfpathlineto{\pgfqpoint{2.398995in}{0.413320in}}%
\pgfpathclose%
\pgfusepath{stroke,fill}%
\end{pgfscope}%
\begin{pgfscope}%
\pgfpathrectangle{\pgfqpoint{2.398995in}{0.319877in}}{\pgfqpoint{3.986877in}{1.993438in}} %
\pgfusepath{clip}%
\pgfsetbuttcap%
\pgfsetroundjoin%
\definecolor{currentfill}{rgb}{1.000000,1.000000,1.000000}%
\pgfsetfillcolor{currentfill}%
\pgfsetlinewidth{1.003750pt}%
\definecolor{currentstroke}{rgb}{0.210448,0.677311,0.643394}%
\pgfsetstrokecolor{currentstroke}%
\pgfsetdash{}{0pt}%
\pgfpathmoveto{\pgfqpoint{2.398995in}{0.413320in}}%
\pgfpathlineto{\pgfqpoint{2.398995in}{1.168327in}}%
\pgfpathlineto{\pgfqpoint{2.401675in}{1.166325in}}%
\pgfpathlineto{\pgfqpoint{2.404352in}{1.166671in}}%
\pgfpathlineto{\pgfqpoint{2.407024in}{1.165988in}}%
\pgfpathlineto{\pgfqpoint{2.409699in}{1.171333in}}%
\pgfpathlineto{\pgfqpoint{2.412389in}{1.164293in}}%
\pgfpathlineto{\pgfqpoint{2.415184in}{1.165993in}}%
\pgfpathlineto{\pgfqpoint{2.417747in}{1.166495in}}%
\pgfpathlineto{\pgfqpoint{2.420528in}{1.172695in}}%
\pgfpathlineto{\pgfqpoint{2.423098in}{1.175564in}}%
\pgfpathlineto{\pgfqpoint{2.425878in}{1.172843in}}%
\pgfpathlineto{\pgfqpoint{2.428453in}{1.175322in}}%
\pgfpathlineto{\pgfqpoint{2.431251in}{1.182463in}}%
\pgfpathlineto{\pgfqpoint{2.433815in}{1.201786in}}%
\pgfpathlineto{\pgfqpoint{2.436518in}{1.198237in}}%
\pgfpathlineto{\pgfqpoint{2.439167in}{1.188968in}}%
\pgfpathlineto{\pgfqpoint{2.441876in}{1.188937in}}%
\pgfpathlineto{\pgfqpoint{2.444677in}{1.186125in}}%
\pgfpathlineto{\pgfqpoint{2.447209in}{1.180683in}}%
\pgfpathlineto{\pgfqpoint{2.450032in}{1.175579in}}%
\pgfpathlineto{\pgfqpoint{2.452562in}{1.170041in}}%
\pgfpathlineto{\pgfqpoint{2.455353in}{1.166826in}}%
\pgfpathlineto{\pgfqpoint{2.457917in}{1.160346in}}%
\pgfpathlineto{\pgfqpoint{2.460711in}{1.162869in}}%
\pgfpathlineto{\pgfqpoint{2.463280in}{1.162589in}}%
\pgfpathlineto{\pgfqpoint{2.465957in}{1.158409in}}%
\pgfpathlineto{\pgfqpoint{2.468635in}{1.159541in}}%
\pgfpathlineto{\pgfqpoint{2.471311in}{1.162555in}}%
\pgfpathlineto{\pgfqpoint{2.473989in}{1.158224in}}%
\pgfpathlineto{\pgfqpoint{2.476671in}{1.159860in}}%
\pgfpathlineto{\pgfqpoint{2.479420in}{1.161740in}}%
\pgfpathlineto{\pgfqpoint{2.482026in}{1.160670in}}%
\pgfpathlineto{\pgfqpoint{2.484870in}{1.168949in}}%
\pgfpathlineto{\pgfqpoint{2.487384in}{1.164490in}}%
\pgfpathlineto{\pgfqpoint{2.490183in}{1.159526in}}%
\pgfpathlineto{\pgfqpoint{2.492729in}{1.159525in}}%
\pgfpathlineto{\pgfqpoint{2.495542in}{1.163694in}}%
\pgfpathlineto{\pgfqpoint{2.498085in}{1.165033in}}%
\pgfpathlineto{\pgfqpoint{2.500801in}{1.164208in}}%
\pgfpathlineto{\pgfqpoint{2.503454in}{1.165896in}}%
\pgfpathlineto{\pgfqpoint{2.506163in}{1.165898in}}%
\pgfpathlineto{\pgfqpoint{2.508917in}{1.160916in}}%
\pgfpathlineto{\pgfqpoint{2.511478in}{1.160828in}}%
\pgfpathlineto{\pgfqpoint{2.514268in}{1.166132in}}%
\pgfpathlineto{\pgfqpoint{2.516845in}{1.162291in}}%
\pgfpathlineto{\pgfqpoint{2.519607in}{1.165797in}}%
\pgfpathlineto{\pgfqpoint{2.522197in}{1.164639in}}%
\pgfpathlineto{\pgfqpoint{2.524988in}{1.158630in}}%
\pgfpathlineto{\pgfqpoint{2.527560in}{1.157863in}}%
\pgfpathlineto{\pgfqpoint{2.530234in}{1.160406in}}%
\pgfpathlineto{\pgfqpoint{2.532917in}{1.160280in}}%
\pgfpathlineto{\pgfqpoint{2.535624in}{1.161244in}}%
\pgfpathlineto{\pgfqpoint{2.538274in}{1.162246in}}%
\pgfpathlineto{\pgfqpoint{2.540949in}{1.169345in}}%
\pgfpathlineto{\pgfqpoint{2.543765in}{1.166822in}}%
\pgfpathlineto{\pgfqpoint{2.546310in}{1.167820in}}%
\pgfpathlineto{\pgfqpoint{2.549114in}{1.165252in}}%
\pgfpathlineto{\pgfqpoint{2.551664in}{1.170168in}}%
\pgfpathlineto{\pgfqpoint{2.554493in}{1.173344in}}%
\pgfpathlineto{\pgfqpoint{2.557009in}{1.182318in}}%
\pgfpathlineto{\pgfqpoint{2.559790in}{1.178273in}}%
\pgfpathlineto{\pgfqpoint{2.562375in}{1.173997in}}%
\pgfpathlineto{\pgfqpoint{2.565045in}{1.169130in}}%
\pgfpathlineto{\pgfqpoint{2.567730in}{1.166993in}}%
\pgfpathlineto{\pgfqpoint{2.570411in}{1.168026in}}%
\pgfpathlineto{\pgfqpoint{2.573082in}{1.192281in}}%
\pgfpathlineto{\pgfqpoint{2.575779in}{1.201471in}}%
\pgfpathlineto{\pgfqpoint{2.578567in}{1.187940in}}%
\pgfpathlineto{\pgfqpoint{2.581129in}{1.180796in}}%
\pgfpathlineto{\pgfqpoint{2.583913in}{1.172547in}}%
\pgfpathlineto{\pgfqpoint{2.586484in}{1.172563in}}%
\pgfpathlineto{\pgfqpoint{2.589248in}{1.171161in}}%
\pgfpathlineto{\pgfqpoint{2.591842in}{1.168848in}}%
\pgfpathlineto{\pgfqpoint{2.594630in}{1.166302in}}%
\pgfpathlineto{\pgfqpoint{2.597196in}{1.157342in}}%
\pgfpathlineto{\pgfqpoint{2.599920in}{1.157226in}}%
\pgfpathlineto{\pgfqpoint{2.602557in}{1.158169in}}%
\pgfpathlineto{\pgfqpoint{2.605232in}{1.165941in}}%
\pgfpathlineto{\pgfqpoint{2.608004in}{1.167527in}}%
\pgfpathlineto{\pgfqpoint{2.610588in}{1.161847in}}%
\pgfpathlineto{\pgfqpoint{2.613393in}{1.166905in}}%
\pgfpathlineto{\pgfqpoint{2.615934in}{1.158857in}}%
\pgfpathlineto{\pgfqpoint{2.618773in}{1.160109in}}%
\pgfpathlineto{\pgfqpoint{2.621304in}{1.158034in}}%
\pgfpathlineto{\pgfqpoint{2.624077in}{1.155004in}}%
\pgfpathlineto{\pgfqpoint{2.626653in}{1.158570in}}%
\pgfpathlineto{\pgfqpoint{2.629340in}{1.157024in}}%
\pgfpathlineto{\pgfqpoint{2.632018in}{1.164671in}}%
\pgfpathlineto{\pgfqpoint{2.634700in}{1.158210in}}%
\pgfpathlineto{\pgfqpoint{2.637369in}{1.159024in}}%
\pgfpathlineto{\pgfqpoint{2.640053in}{1.168788in}}%
\pgfpathlineto{\pgfqpoint{2.642827in}{1.163770in}}%
\pgfpathlineto{\pgfqpoint{2.645408in}{1.162736in}}%
\pgfpathlineto{\pgfqpoint{2.648196in}{1.164753in}}%
\pgfpathlineto{\pgfqpoint{2.650767in}{1.169685in}}%
\pgfpathlineto{\pgfqpoint{2.653567in}{1.164314in}}%
\pgfpathlineto{\pgfqpoint{2.656124in}{1.163547in}}%
\pgfpathlineto{\pgfqpoint{2.658942in}{1.159860in}}%
\pgfpathlineto{\pgfqpoint{2.661481in}{1.157406in}}%
\pgfpathlineto{\pgfqpoint{2.664151in}{1.156537in}}%
\pgfpathlineto{\pgfqpoint{2.666836in}{1.165535in}}%
\pgfpathlineto{\pgfqpoint{2.669506in}{1.162108in}}%
\pgfpathlineto{\pgfqpoint{2.672301in}{1.163796in}}%
\pgfpathlineto{\pgfqpoint{2.674873in}{1.156752in}}%
\pgfpathlineto{\pgfqpoint{2.677650in}{1.157361in}}%
\pgfpathlineto{\pgfqpoint{2.680224in}{1.162219in}}%
\pgfpathlineto{\pgfqpoint{2.683009in}{1.159735in}}%
\pgfpathlineto{\pgfqpoint{2.685586in}{1.163844in}}%
\pgfpathlineto{\pgfqpoint{2.688328in}{1.168427in}}%
\pgfpathlineto{\pgfqpoint{2.690940in}{1.166558in}}%
\pgfpathlineto{\pgfqpoint{2.693611in}{1.168635in}}%
\pgfpathlineto{\pgfqpoint{2.696293in}{1.167334in}}%
\pgfpathlineto{\pgfqpoint{2.698968in}{1.167943in}}%
\pgfpathlineto{\pgfqpoint{2.701657in}{1.167974in}}%
\pgfpathlineto{\pgfqpoint{2.704326in}{1.166823in}}%
\pgfpathlineto{\pgfqpoint{2.707125in}{1.169573in}}%
\pgfpathlineto{\pgfqpoint{2.709683in}{1.172546in}}%
\pgfpathlineto{\pgfqpoint{2.712477in}{1.174103in}}%
\pgfpathlineto{\pgfqpoint{2.715036in}{1.171559in}}%
\pgfpathlineto{\pgfqpoint{2.717773in}{1.170184in}}%
\pgfpathlineto{\pgfqpoint{2.720404in}{1.169532in}}%
\pgfpathlineto{\pgfqpoint{2.723211in}{1.168063in}}%
\pgfpathlineto{\pgfqpoint{2.725760in}{1.163297in}}%
\pgfpathlineto{\pgfqpoint{2.728439in}{1.165855in}}%
\pgfpathlineto{\pgfqpoint{2.731119in}{1.169150in}}%
\pgfpathlineto{\pgfqpoint{2.733798in}{1.163029in}}%
\pgfpathlineto{\pgfqpoint{2.736476in}{1.164715in}}%
\pgfpathlineto{\pgfqpoint{2.739155in}{1.161711in}}%
\pgfpathlineto{\pgfqpoint{2.741928in}{1.167032in}}%
\pgfpathlineto{\pgfqpoint{2.744510in}{1.163268in}}%
\pgfpathlineto{\pgfqpoint{2.747260in}{1.163668in}}%
\pgfpathlineto{\pgfqpoint{2.749868in}{1.161129in}}%
\pgfpathlineto{\pgfqpoint{2.752614in}{1.165500in}}%
\pgfpathlineto{\pgfqpoint{2.755224in}{1.165918in}}%
\pgfpathlineto{\pgfqpoint{2.758028in}{1.168752in}}%
\pgfpathlineto{\pgfqpoint{2.760581in}{1.160902in}}%
\pgfpathlineto{\pgfqpoint{2.763253in}{1.162810in}}%
\pgfpathlineto{\pgfqpoint{2.765935in}{1.157380in}}%
\pgfpathlineto{\pgfqpoint{2.768617in}{1.166288in}}%
\pgfpathlineto{\pgfqpoint{2.771373in}{1.161622in}}%
\pgfpathlineto{\pgfqpoint{2.773972in}{1.163387in}}%
\pgfpathlineto{\pgfqpoint{2.776767in}{1.166185in}}%
\pgfpathlineto{\pgfqpoint{2.779330in}{1.163333in}}%
\pgfpathlineto{\pgfqpoint{2.782113in}{1.162943in}}%
\pgfpathlineto{\pgfqpoint{2.784687in}{1.166054in}}%
\pgfpathlineto{\pgfqpoint{2.787468in}{1.164457in}}%
\pgfpathlineto{\pgfqpoint{2.790044in}{1.162642in}}%
\pgfpathlineto{\pgfqpoint{2.792721in}{1.162102in}}%
\pgfpathlineto{\pgfqpoint{2.795398in}{1.163385in}}%
\pgfpathlineto{\pgfqpoint{2.798070in}{1.165124in}}%
\pgfpathlineto{\pgfqpoint{2.800756in}{1.176112in}}%
\pgfpathlineto{\pgfqpoint{2.803435in}{1.167317in}}%
\pgfpathlineto{\pgfqpoint{2.806175in}{1.168330in}}%
\pgfpathlineto{\pgfqpoint{2.808792in}{1.163167in}}%
\pgfpathlineto{\pgfqpoint{2.811597in}{1.159811in}}%
\pgfpathlineto{\pgfqpoint{2.814141in}{1.161846in}}%
\pgfpathlineto{\pgfqpoint{2.816867in}{1.163177in}}%
\pgfpathlineto{\pgfqpoint{2.819506in}{1.161398in}}%
\pgfpathlineto{\pgfqpoint{2.822303in}{1.161116in}}%
\pgfpathlineto{\pgfqpoint{2.824851in}{1.163113in}}%
\pgfpathlineto{\pgfqpoint{2.827567in}{1.163609in}}%
\pgfpathlineto{\pgfqpoint{2.830219in}{1.162893in}}%
\pgfpathlineto{\pgfqpoint{2.832894in}{1.165896in}}%
\pgfpathlineto{\pgfqpoint{2.835698in}{1.163623in}}%
\pgfpathlineto{\pgfqpoint{2.838254in}{1.163898in}}%
\pgfpathlineto{\pgfqpoint{2.841055in}{1.158842in}}%
\pgfpathlineto{\pgfqpoint{2.843611in}{1.162938in}}%
\pgfpathlineto{\pgfqpoint{2.846408in}{1.167267in}}%
\pgfpathlineto{\pgfqpoint{2.848960in}{1.169367in}}%
\pgfpathlineto{\pgfqpoint{2.851793in}{1.160559in}}%
\pgfpathlineto{\pgfqpoint{2.854325in}{1.162505in}}%
\pgfpathlineto{\pgfqpoint{2.857003in}{1.165170in}}%
\pgfpathlineto{\pgfqpoint{2.859668in}{1.164738in}}%
\pgfpathlineto{\pgfqpoint{2.862402in}{1.167503in}}%
\pgfpathlineto{\pgfqpoint{2.865031in}{1.163262in}}%
\pgfpathlineto{\pgfqpoint{2.867713in}{1.168652in}}%
\pgfpathlineto{\pgfqpoint{2.870475in}{1.166486in}}%
\pgfpathlineto{\pgfqpoint{2.873074in}{1.166776in}}%
\pgfpathlineto{\pgfqpoint{2.875882in}{1.171246in}}%
\pgfpathlineto{\pgfqpoint{2.878431in}{1.171998in}}%
\pgfpathlineto{\pgfqpoint{2.881254in}{1.171698in}}%
\pgfpathlineto{\pgfqpoint{2.883780in}{1.170168in}}%
\pgfpathlineto{\pgfqpoint{2.886578in}{1.162948in}}%
\pgfpathlineto{\pgfqpoint{2.889145in}{1.173870in}}%
\pgfpathlineto{\pgfqpoint{2.891809in}{1.169491in}}%
\pgfpathlineto{\pgfqpoint{2.894487in}{1.168515in}}%
\pgfpathlineto{\pgfqpoint{2.897179in}{1.171852in}}%
\pgfpathlineto{\pgfqpoint{2.899858in}{1.169109in}}%
\pgfpathlineto{\pgfqpoint{2.902535in}{1.170713in}}%
\pgfpathlineto{\pgfqpoint{2.905341in}{1.171542in}}%
\pgfpathlineto{\pgfqpoint{2.907882in}{1.168099in}}%
\pgfpathlineto{\pgfqpoint{2.910631in}{1.174463in}}%
\pgfpathlineto{\pgfqpoint{2.913243in}{1.170524in}}%
\pgfpathlineto{\pgfqpoint{2.916061in}{1.169848in}}%
\pgfpathlineto{\pgfqpoint{2.918606in}{1.165494in}}%
\pgfpathlineto{\pgfqpoint{2.921363in}{1.169500in}}%
\pgfpathlineto{\pgfqpoint{2.923963in}{1.167921in}}%
\pgfpathlineto{\pgfqpoint{2.926655in}{1.172588in}}%
\pgfpathlineto{\pgfqpoint{2.929321in}{1.172238in}}%
\pgfpathlineto{\pgfqpoint{2.932033in}{1.165567in}}%
\pgfpathlineto{\pgfqpoint{2.934759in}{1.161800in}}%
\pgfpathlineto{\pgfqpoint{2.937352in}{1.161709in}}%
\pgfpathlineto{\pgfqpoint{2.940120in}{1.157010in}}%
\pgfpathlineto{\pgfqpoint{2.942711in}{1.157084in}}%
\pgfpathlineto{\pgfqpoint{2.945461in}{1.161919in}}%
\pgfpathlineto{\pgfqpoint{2.948068in}{1.161464in}}%
\pgfpathlineto{\pgfqpoint{2.950884in}{1.162472in}}%
\pgfpathlineto{\pgfqpoint{2.953422in}{1.167180in}}%
\pgfpathlineto{\pgfqpoint{2.956103in}{1.165578in}}%
\pgfpathlineto{\pgfqpoint{2.958782in}{1.165193in}}%
\pgfpathlineto{\pgfqpoint{2.961460in}{1.169955in}}%
\pgfpathlineto{\pgfqpoint{2.964127in}{1.171268in}}%
\pgfpathlineto{\pgfqpoint{2.966812in}{1.164982in}}%
\pgfpathlineto{\pgfqpoint{2.969599in}{1.166284in}}%
\pgfpathlineto{\pgfqpoint{2.972177in}{1.167613in}}%
\pgfpathlineto{\pgfqpoint{2.974972in}{1.171115in}}%
\pgfpathlineto{\pgfqpoint{2.977517in}{1.172117in}}%
\pgfpathlineto{\pgfqpoint{2.980341in}{1.172497in}}%
\pgfpathlineto{\pgfqpoint{2.982885in}{1.171581in}}%
\pgfpathlineto{\pgfqpoint{2.985666in}{1.167956in}}%
\pgfpathlineto{\pgfqpoint{2.988238in}{1.170944in}}%
\pgfpathlineto{\pgfqpoint{2.990978in}{1.169951in}}%
\pgfpathlineto{\pgfqpoint{2.993595in}{1.171364in}}%
\pgfpathlineto{\pgfqpoint{2.996300in}{1.171966in}}%
\pgfpathlineto{\pgfqpoint{2.999103in}{1.180012in}}%
\pgfpathlineto{\pgfqpoint{3.001635in}{1.167189in}}%
\pgfpathlineto{\pgfqpoint{3.004419in}{1.164129in}}%
\pgfpathlineto{\pgfqpoint{3.006993in}{1.168685in}}%
\pgfpathlineto{\pgfqpoint{3.009784in}{1.166038in}}%
\pgfpathlineto{\pgfqpoint{3.012351in}{1.163001in}}%
\pgfpathlineto{\pgfqpoint{3.015097in}{1.159401in}}%
\pgfpathlineto{\pgfqpoint{3.017707in}{1.167693in}}%
\pgfpathlineto{\pgfqpoint{3.020382in}{1.165481in}}%
\pgfpathlineto{\pgfqpoint{3.023058in}{1.168693in}}%
\pgfpathlineto{\pgfqpoint{3.025803in}{1.166727in}}%
\pgfpathlineto{\pgfqpoint{3.028412in}{1.173828in}}%
\pgfpathlineto{\pgfqpoint{3.031091in}{1.172337in}}%
\pgfpathlineto{\pgfqpoint{3.033921in}{1.169682in}}%
\pgfpathlineto{\pgfqpoint{3.036456in}{1.168227in}}%
\pgfpathlineto{\pgfqpoint{3.039262in}{1.173158in}}%
\pgfpathlineto{\pgfqpoint{3.041813in}{1.169727in}}%
\pgfpathlineto{\pgfqpoint{3.044568in}{1.168738in}}%
\pgfpathlineto{\pgfqpoint{3.047157in}{1.170270in}}%
\pgfpathlineto{\pgfqpoint{3.049988in}{1.189479in}}%
\pgfpathlineto{\pgfqpoint{3.052526in}{1.179923in}}%
\pgfpathlineto{\pgfqpoint{3.055202in}{1.176330in}}%
\pgfpathlineto{\pgfqpoint{3.057884in}{1.170376in}}%
\pgfpathlineto{\pgfqpoint{3.060561in}{1.168579in}}%
\pgfpathlineto{\pgfqpoint{3.063230in}{1.165492in}}%
\pgfpathlineto{\pgfqpoint{3.065916in}{1.166560in}}%
\pgfpathlineto{\pgfqpoint{3.068709in}{1.169793in}}%
\pgfpathlineto{\pgfqpoint{3.071266in}{1.170245in}}%
\pgfpathlineto{\pgfqpoint{3.074056in}{1.169600in}}%
\pgfpathlineto{\pgfqpoint{3.076631in}{1.169265in}}%
\pgfpathlineto{\pgfqpoint{3.079381in}{1.164671in}}%
\pgfpathlineto{\pgfqpoint{3.081990in}{1.168706in}}%
\pgfpathlineto{\pgfqpoint{3.084671in}{1.168893in}}%
\pgfpathlineto{\pgfqpoint{3.087343in}{1.172764in}}%
\pgfpathlineto{\pgfqpoint{3.090023in}{1.164915in}}%
\pgfpathlineto{\pgfqpoint{3.092699in}{1.162122in}}%
\pgfpathlineto{\pgfqpoint{3.095388in}{1.167593in}}%
\pgfpathlineto{\pgfqpoint{3.098163in}{1.160686in}}%
\pgfpathlineto{\pgfqpoint{3.100737in}{1.157428in}}%
\pgfpathlineto{\pgfqpoint{3.103508in}{1.156232in}}%
\pgfpathlineto{\pgfqpoint{3.106094in}{1.160739in}}%
\pgfpathlineto{\pgfqpoint{3.108896in}{1.161949in}}%
\pgfpathlineto{\pgfqpoint{3.111451in}{1.160804in}}%
\pgfpathlineto{\pgfqpoint{3.114242in}{1.162391in}}%
\pgfpathlineto{\pgfqpoint{3.116807in}{1.168327in}}%
\pgfpathlineto{\pgfqpoint{3.119487in}{1.164861in}}%
\pgfpathlineto{\pgfqpoint{3.122163in}{1.162350in}}%
\pgfpathlineto{\pgfqpoint{3.124842in}{1.164140in}}%
\pgfpathlineto{\pgfqpoint{3.127512in}{1.167278in}}%
\pgfpathlineto{\pgfqpoint{3.130199in}{1.169306in}}%
\pgfpathlineto{\pgfqpoint{3.132946in}{1.169899in}}%
\pgfpathlineto{\pgfqpoint{3.135550in}{1.169377in}}%
\pgfpathlineto{\pgfqpoint{3.138375in}{1.166132in}}%
\pgfpathlineto{\pgfqpoint{3.140913in}{1.155768in}}%
\pgfpathlineto{\pgfqpoint{3.143740in}{1.150485in}}%
\pgfpathlineto{\pgfqpoint{3.146271in}{1.154565in}}%
\pgfpathlineto{\pgfqpoint{3.149057in}{1.154982in}}%
\pgfpathlineto{\pgfqpoint{3.151612in}{1.150485in}}%
\pgfpathlineto{\pgfqpoint{3.154327in}{1.159622in}}%
\pgfpathlineto{\pgfqpoint{3.156981in}{1.157472in}}%
\pgfpathlineto{\pgfqpoint{3.159675in}{1.159781in}}%
\pgfpathlineto{\pgfqpoint{3.162474in}{1.166870in}}%
\pgfpathlineto{\pgfqpoint{3.165019in}{1.161042in}}%
\pgfpathlineto{\pgfqpoint{3.167776in}{1.156768in}}%
\pgfpathlineto{\pgfqpoint{3.170375in}{1.153066in}}%
\pgfpathlineto{\pgfqpoint{3.173142in}{1.153316in}}%
\pgfpathlineto{\pgfqpoint{3.175724in}{1.154859in}}%
\pgfpathlineto{\pgfqpoint{3.178525in}{1.157271in}}%
\pgfpathlineto{\pgfqpoint{3.181089in}{1.162035in}}%
\pgfpathlineto{\pgfqpoint{3.183760in}{1.161683in}}%
\pgfpathlineto{\pgfqpoint{3.186440in}{1.167648in}}%
\pgfpathlineto{\pgfqpoint{3.189117in}{1.152859in}}%
\pgfpathlineto{\pgfqpoint{3.191796in}{1.152684in}}%
\pgfpathlineto{\pgfqpoint{3.194508in}{1.155449in}}%
\pgfpathlineto{\pgfqpoint{3.197226in}{1.153502in}}%
\pgfpathlineto{\pgfqpoint{3.199823in}{1.150674in}}%
\pgfpathlineto{\pgfqpoint{3.202562in}{1.150632in}}%
\pgfpathlineto{\pgfqpoint{3.205195in}{1.153694in}}%
\pgfpathlineto{\pgfqpoint{3.207984in}{1.158814in}}%
\pgfpathlineto{\pgfqpoint{3.210545in}{1.155945in}}%
\pgfpathlineto{\pgfqpoint{3.213342in}{1.155000in}}%
\pgfpathlineto{\pgfqpoint{3.215908in}{1.152856in}}%
\pgfpathlineto{\pgfqpoint{3.218586in}{1.158801in}}%
\pgfpathlineto{\pgfqpoint{3.221255in}{1.154923in}}%
\pgfpathlineto{\pgfqpoint{3.223942in}{1.162530in}}%
\pgfpathlineto{\pgfqpoint{3.226609in}{1.171145in}}%
\pgfpathlineto{\pgfqpoint{3.229310in}{1.166217in}}%
\pgfpathlineto{\pgfqpoint{3.232069in}{1.155332in}}%
\pgfpathlineto{\pgfqpoint{3.234658in}{1.160133in}}%
\pgfpathlineto{\pgfqpoint{3.237411in}{1.162885in}}%
\pgfpathlineto{\pgfqpoint{3.240010in}{1.165266in}}%
\pgfpathlineto{\pgfqpoint{3.242807in}{1.165850in}}%
\pgfpathlineto{\pgfqpoint{3.245363in}{1.164398in}}%
\pgfpathlineto{\pgfqpoint{3.248049in}{1.163927in}}%
\pgfpathlineto{\pgfqpoint{3.250716in}{1.164104in}}%
\pgfpathlineto{\pgfqpoint{3.253404in}{1.164296in}}%
\pgfpathlineto{\pgfqpoint{3.256083in}{1.166150in}}%
\pgfpathlineto{\pgfqpoint{3.258784in}{1.164585in}}%
\pgfpathlineto{\pgfqpoint{3.261594in}{1.166685in}}%
\pgfpathlineto{\pgfqpoint{3.264119in}{1.169913in}}%
\pgfpathlineto{\pgfqpoint{3.266849in}{1.171550in}}%
\pgfpathlineto{\pgfqpoint{3.269478in}{1.168362in}}%
\pgfpathlineto{\pgfqpoint{3.272254in}{1.159433in}}%
\pgfpathlineto{\pgfqpoint{3.274831in}{1.165820in}}%
\pgfpathlineto{\pgfqpoint{3.277603in}{1.167237in}}%
\pgfpathlineto{\pgfqpoint{3.280189in}{1.169544in}}%
\pgfpathlineto{\pgfqpoint{3.282870in}{1.164133in}}%
\pgfpathlineto{\pgfqpoint{3.285534in}{1.164326in}}%
\pgfpathlineto{\pgfqpoint{3.288225in}{1.169697in}}%
\pgfpathlineto{\pgfqpoint{3.290890in}{1.169328in}}%
\pgfpathlineto{\pgfqpoint{3.293574in}{1.169638in}}%
\pgfpathlineto{\pgfqpoint{3.296376in}{1.165918in}}%
\pgfpathlineto{\pgfqpoint{3.298937in}{1.161210in}}%
\pgfpathlineto{\pgfqpoint{3.301719in}{1.169500in}}%
\pgfpathlineto{\pgfqpoint{3.304295in}{1.170471in}}%
\pgfpathlineto{\pgfqpoint{3.307104in}{1.169379in}}%
\pgfpathlineto{\pgfqpoint{3.309652in}{1.171912in}}%
\pgfpathlineto{\pgfqpoint{3.312480in}{1.172720in}}%
\pgfpathlineto{\pgfqpoint{3.315008in}{1.168912in}}%
\pgfpathlineto{\pgfqpoint{3.317688in}{1.171965in}}%
\pgfpathlineto{\pgfqpoint{3.320366in}{1.170272in}}%
\pgfpathlineto{\pgfqpoint{3.323049in}{1.168811in}}%
\pgfpathlineto{\pgfqpoint{3.325860in}{1.167292in}}%
\pgfpathlineto{\pgfqpoint{3.328401in}{1.171871in}}%
\pgfpathlineto{\pgfqpoint{3.331183in}{1.169810in}}%
\pgfpathlineto{\pgfqpoint{3.333758in}{1.169977in}}%
\pgfpathlineto{\pgfqpoint{3.336541in}{1.171731in}}%
\pgfpathlineto{\pgfqpoint{3.339101in}{1.168612in}}%
\pgfpathlineto{\pgfqpoint{3.341893in}{1.170329in}}%
\pgfpathlineto{\pgfqpoint{3.344468in}{1.167432in}}%
\pgfpathlineto{\pgfqpoint{3.347139in}{1.167050in}}%
\pgfpathlineto{\pgfqpoint{3.349828in}{1.164204in}}%
\pgfpathlineto{\pgfqpoint{3.352505in}{1.171332in}}%
\pgfpathlineto{\pgfqpoint{3.355177in}{1.170354in}}%
\pgfpathlineto{\pgfqpoint{3.357862in}{1.165245in}}%
\pgfpathlineto{\pgfqpoint{3.360620in}{1.170162in}}%
\pgfpathlineto{\pgfqpoint{3.363221in}{1.169465in}}%
\pgfpathlineto{\pgfqpoint{3.365996in}{1.163258in}}%
\pgfpathlineto{\pgfqpoint{3.368577in}{1.160245in}}%
\pgfpathlineto{\pgfqpoint{3.371357in}{1.166544in}}%
\pgfpathlineto{\pgfqpoint{3.373921in}{1.162473in}}%
\pgfpathlineto{\pgfqpoint{3.376735in}{1.163172in}}%
\pgfpathlineto{\pgfqpoint{3.379290in}{1.158800in}}%
\pgfpathlineto{\pgfqpoint{3.381959in}{1.163402in}}%
\pgfpathlineto{\pgfqpoint{3.384647in}{1.164225in}}%
\pgfpathlineto{\pgfqpoint{3.387309in}{1.170125in}}%
\pgfpathlineto{\pgfqpoint{3.390102in}{1.167197in}}%
\pgfpathlineto{\pgfqpoint{3.392681in}{1.166951in}}%
\pgfpathlineto{\pgfqpoint{3.395461in}{1.165642in}}%
\pgfpathlineto{\pgfqpoint{3.398037in}{1.167275in}}%
\pgfpathlineto{\pgfqpoint{3.400783in}{1.170943in}}%
\pgfpathlineto{\pgfqpoint{3.403394in}{1.168709in}}%
\pgfpathlineto{\pgfqpoint{3.406202in}{1.169873in}}%
\pgfpathlineto{\pgfqpoint{3.408752in}{1.167413in}}%
\pgfpathlineto{\pgfqpoint{3.411431in}{1.174414in}}%
\pgfpathlineto{\pgfqpoint{3.414109in}{1.173936in}}%
\pgfpathlineto{\pgfqpoint{3.416780in}{1.173645in}}%
\pgfpathlineto{\pgfqpoint{3.419455in}{1.174628in}}%
\pgfpathlineto{\pgfqpoint{3.422142in}{1.173198in}}%
\pgfpathlineto{\pgfqpoint{3.424887in}{1.175852in}}%
\pgfpathlineto{\pgfqpoint{3.427501in}{1.176866in}}%
\pgfpathlineto{\pgfqpoint{3.430313in}{1.171783in}}%
\pgfpathlineto{\pgfqpoint{3.432851in}{1.174892in}}%
\pgfpathlineto{\pgfqpoint{3.435635in}{1.171937in}}%
\pgfpathlineto{\pgfqpoint{3.438210in}{1.175684in}}%
\pgfpathlineto{\pgfqpoint{3.440996in}{1.175264in}}%
\pgfpathlineto{\pgfqpoint{3.443574in}{1.177986in}}%
\pgfpathlineto{\pgfqpoint{3.446257in}{1.179052in}}%
\pgfpathlineto{\pgfqpoint{3.448926in}{1.179824in}}%
\pgfpathlineto{\pgfqpoint{3.451597in}{1.176858in}}%
\pgfpathlineto{\pgfqpoint{3.454285in}{1.170604in}}%
\pgfpathlineto{\pgfqpoint{3.456960in}{1.170736in}}%
\pgfpathlineto{\pgfqpoint{3.459695in}{1.168373in}}%
\pgfpathlineto{\pgfqpoint{3.462321in}{1.164839in}}%
\pgfpathlineto{\pgfqpoint{3.465072in}{1.166341in}}%
\pgfpathlineto{\pgfqpoint{3.467678in}{1.167746in}}%
\pgfpathlineto{\pgfqpoint{3.470466in}{1.170701in}}%
\pgfpathlineto{\pgfqpoint{3.473021in}{1.173091in}}%
\pgfpathlineto{\pgfqpoint{3.475821in}{1.164064in}}%
\pgfpathlineto{\pgfqpoint{3.478378in}{1.164930in}}%
\pgfpathlineto{\pgfqpoint{3.481072in}{1.162952in}}%
\pgfpathlineto{\pgfqpoint{3.483744in}{1.172317in}}%
\pgfpathlineto{\pgfqpoint{3.486442in}{1.168204in}}%
\pgfpathlineto{\pgfqpoint{3.489223in}{1.166695in}}%
\pgfpathlineto{\pgfqpoint{3.491783in}{1.167352in}}%
\pgfpathlineto{\pgfqpoint{3.494581in}{1.164747in}}%
\pgfpathlineto{\pgfqpoint{3.497139in}{1.167154in}}%
\pgfpathlineto{\pgfqpoint{3.499909in}{1.167115in}}%
\pgfpathlineto{\pgfqpoint{3.502488in}{1.171099in}}%
\pgfpathlineto{\pgfqpoint{3.505262in}{1.165940in}}%
\pgfpathlineto{\pgfqpoint{3.507840in}{1.171360in}}%
\pgfpathlineto{\pgfqpoint{3.510533in}{1.174574in}}%
\pgfpathlineto{\pgfqpoint{3.513209in}{1.172028in}}%
\pgfpathlineto{\pgfqpoint{3.515884in}{1.173530in}}%
\pgfpathlineto{\pgfqpoint{3.518565in}{1.168647in}}%
\pgfpathlineto{\pgfqpoint{3.521244in}{1.168500in}}%
\pgfpathlineto{\pgfqpoint{3.524041in}{1.169052in}}%
\pgfpathlineto{\pgfqpoint{3.526601in}{1.172946in}}%
\pgfpathlineto{\pgfqpoint{3.529327in}{1.176261in}}%
\pgfpathlineto{\pgfqpoint{3.531955in}{1.175440in}}%
\pgfpathlineto{\pgfqpoint{3.534783in}{1.166697in}}%
\pgfpathlineto{\pgfqpoint{3.537309in}{1.172152in}}%
\pgfpathlineto{\pgfqpoint{3.540093in}{1.172030in}}%
\pgfpathlineto{\pgfqpoint{3.542656in}{1.174009in}}%
\pgfpathlineto{\pgfqpoint{3.545349in}{1.183469in}}%
\pgfpathlineto{\pgfqpoint{3.548029in}{1.174928in}}%
\pgfpathlineto{\pgfqpoint{3.550713in}{1.169245in}}%
\pgfpathlineto{\pgfqpoint{3.553498in}{1.170878in}}%
\pgfpathlineto{\pgfqpoint{3.556061in}{1.168579in}}%
\pgfpathlineto{\pgfqpoint{3.558853in}{1.170973in}}%
\pgfpathlineto{\pgfqpoint{3.561420in}{1.172455in}}%
\pgfpathlineto{\pgfqpoint{3.564188in}{1.173977in}}%
\pgfpathlineto{\pgfqpoint{3.566774in}{1.177161in}}%
\pgfpathlineto{\pgfqpoint{3.569584in}{1.175986in}}%
\pgfpathlineto{\pgfqpoint{3.572126in}{1.174301in}}%
\pgfpathlineto{\pgfqpoint{3.574814in}{1.177369in}}%
\pgfpathlineto{\pgfqpoint{3.577487in}{1.172775in}}%
\pgfpathlineto{\pgfqpoint{3.580191in}{1.168491in}}%
\pgfpathlineto{\pgfqpoint{3.582851in}{1.175381in}}%
\pgfpathlineto{\pgfqpoint{3.585532in}{1.172884in}}%
\pgfpathlineto{\pgfqpoint{3.588258in}{1.175510in}}%
\pgfpathlineto{\pgfqpoint{3.590883in}{1.177847in}}%
\pgfpathlineto{\pgfqpoint{3.593620in}{1.173017in}}%
\pgfpathlineto{\pgfqpoint{3.596240in}{1.175311in}}%
\pgfpathlineto{\pgfqpoint{3.598998in}{1.175946in}}%
\pgfpathlineto{\pgfqpoint{3.601590in}{1.175152in}}%
\pgfpathlineto{\pgfqpoint{3.604387in}{1.175853in}}%
\pgfpathlineto{\pgfqpoint{3.606951in}{1.176426in}}%
\pgfpathlineto{\pgfqpoint{3.609632in}{1.179526in}}%
\pgfpathlineto{\pgfqpoint{3.612311in}{1.172442in}}%
\pgfpathlineto{\pgfqpoint{3.614982in}{1.172710in}}%
\pgfpathlineto{\pgfqpoint{3.617667in}{1.172709in}}%
\pgfpathlineto{\pgfqpoint{3.620345in}{1.175723in}}%
\pgfpathlineto{\pgfqpoint{3.623165in}{1.174274in}}%
\pgfpathlineto{\pgfqpoint{3.625689in}{1.175737in}}%
\pgfpathlineto{\pgfqpoint{3.628460in}{1.170583in}}%
\pgfpathlineto{\pgfqpoint{3.631058in}{1.174454in}}%
\pgfpathlineto{\pgfqpoint{3.633858in}{1.173844in}}%
\pgfpathlineto{\pgfqpoint{3.636413in}{1.174586in}}%
\pgfpathlineto{\pgfqpoint{3.639207in}{1.170651in}}%
\pgfpathlineto{\pgfqpoint{3.641773in}{1.171425in}}%
\pgfpathlineto{\pgfqpoint{3.644452in}{1.157027in}}%
\pgfpathlineto{\pgfqpoint{3.647130in}{1.153540in}}%
\pgfpathlineto{\pgfqpoint{3.649837in}{1.158604in}}%
\pgfpathlineto{\pgfqpoint{3.652628in}{1.160396in}}%
\pgfpathlineto{\pgfqpoint{3.655165in}{1.162514in}}%
\pgfpathlineto{\pgfqpoint{3.657917in}{1.182340in}}%
\pgfpathlineto{\pgfqpoint{3.660515in}{1.195117in}}%
\pgfpathlineto{\pgfqpoint{3.663276in}{1.201219in}}%
\pgfpathlineto{\pgfqpoint{3.665864in}{1.184126in}}%
\pgfpathlineto{\pgfqpoint{3.668665in}{1.178612in}}%
\pgfpathlineto{\pgfqpoint{3.671232in}{1.172310in}}%
\pgfpathlineto{\pgfqpoint{3.673911in}{1.164844in}}%
\pgfpathlineto{\pgfqpoint{3.676591in}{1.174424in}}%
\pgfpathlineto{\pgfqpoint{3.679273in}{1.184106in}}%
\pgfpathlineto{\pgfqpoint{3.681948in}{1.181295in}}%
\pgfpathlineto{\pgfqpoint{3.684620in}{1.168605in}}%
\pgfpathlineto{\pgfqpoint{3.687442in}{1.165502in}}%
\pgfpathlineto{\pgfqpoint{3.689983in}{1.166405in}}%
\pgfpathlineto{\pgfqpoint{3.692765in}{1.164923in}}%
\pgfpathlineto{\pgfqpoint{3.695331in}{1.167753in}}%
\pgfpathlineto{\pgfqpoint{3.698125in}{1.170040in}}%
\pgfpathlineto{\pgfqpoint{3.700684in}{1.165776in}}%
\pgfpathlineto{\pgfqpoint{3.703460in}{1.162149in}}%
\pgfpathlineto{\pgfqpoint{3.706053in}{1.169940in}}%
\pgfpathlineto{\pgfqpoint{3.708729in}{1.165523in}}%
\pgfpathlineto{\pgfqpoint{3.711410in}{1.164565in}}%
\pgfpathlineto{\pgfqpoint{3.714086in}{1.164250in}}%
\pgfpathlineto{\pgfqpoint{3.716875in}{1.163703in}}%
\pgfpathlineto{\pgfqpoint{3.719446in}{1.163469in}}%
\pgfpathlineto{\pgfqpoint{3.722228in}{1.165430in}}%
\pgfpathlineto{\pgfqpoint{3.724804in}{1.164619in}}%
\pgfpathlineto{\pgfqpoint{3.727581in}{1.162636in}}%
\pgfpathlineto{\pgfqpoint{3.730158in}{1.162485in}}%
\pgfpathlineto{\pgfqpoint{3.732950in}{1.164027in}}%
\pgfpathlineto{\pgfqpoint{3.735509in}{1.164892in}}%
\pgfpathlineto{\pgfqpoint{3.738194in}{1.164058in}}%
\pgfpathlineto{\pgfqpoint{3.740874in}{1.169547in}}%
\pgfpathlineto{\pgfqpoint{3.743548in}{1.165007in}}%
\pgfpathlineto{\pgfqpoint{3.746229in}{1.170293in}}%
\pgfpathlineto{\pgfqpoint{3.748903in}{1.166720in}}%
\pgfpathlineto{\pgfqpoint{3.751728in}{1.167668in}}%
\pgfpathlineto{\pgfqpoint{3.754265in}{1.171288in}}%
\pgfpathlineto{\pgfqpoint{3.757065in}{1.168968in}}%
\pgfpathlineto{\pgfqpoint{3.759622in}{1.169123in}}%
\pgfpathlineto{\pgfqpoint{3.762389in}{1.169288in}}%
\pgfpathlineto{\pgfqpoint{3.764966in}{1.159883in}}%
\pgfpathlineto{\pgfqpoint{3.767782in}{1.160539in}}%
\pgfpathlineto{\pgfqpoint{3.770323in}{1.170437in}}%
\pgfpathlineto{\pgfqpoint{3.773014in}{1.175523in}}%
\pgfpathlineto{\pgfqpoint{3.775691in}{1.171954in}}%
\pgfpathlineto{\pgfqpoint{3.778370in}{1.206496in}}%
\pgfpathlineto{\pgfqpoint{3.781046in}{1.250656in}}%
\pgfpathlineto{\pgfqpoint{3.783725in}{1.216277in}}%
\pgfpathlineto{\pgfqpoint{3.786504in}{1.198656in}}%
\pgfpathlineto{\pgfqpoint{3.789084in}{1.185618in}}%
\pgfpathlineto{\pgfqpoint{3.791897in}{1.180115in}}%
\pgfpathlineto{\pgfqpoint{3.794435in}{1.175438in}}%
\pgfpathlineto{\pgfqpoint{3.797265in}{1.167866in}}%
\pgfpathlineto{\pgfqpoint{3.799797in}{1.170497in}}%
\pgfpathlineto{\pgfqpoint{3.802569in}{1.170493in}}%
\pgfpathlineto{\pgfqpoint{3.805145in}{1.165814in}}%
\pgfpathlineto{\pgfqpoint{3.807832in}{1.164163in}}%
\pgfpathlineto{\pgfqpoint{3.810510in}{1.169645in}}%
\pgfpathlineto{\pgfqpoint{3.813172in}{1.165996in}}%
\pgfpathlineto{\pgfqpoint{3.815983in}{1.171501in}}%
\pgfpathlineto{\pgfqpoint{3.818546in}{1.166252in}}%
\pgfpathlineto{\pgfqpoint{3.821315in}{1.165114in}}%
\pgfpathlineto{\pgfqpoint{3.823903in}{1.171227in}}%
\pgfpathlineto{\pgfqpoint{3.826679in}{1.165982in}}%
\pgfpathlineto{\pgfqpoint{3.829252in}{1.169967in}}%
\pgfpathlineto{\pgfqpoint{3.832053in}{1.172016in}}%
\pgfpathlineto{\pgfqpoint{3.834616in}{1.172440in}}%
\pgfpathlineto{\pgfqpoint{3.837286in}{1.169541in}}%
\pgfpathlineto{\pgfqpoint{3.839960in}{1.173844in}}%
\pgfpathlineto{\pgfqpoint{3.842641in}{1.172048in}}%
\pgfpathlineto{\pgfqpoint{3.845329in}{1.177440in}}%
\pgfpathlineto{\pgfqpoint{3.848005in}{1.172122in}}%
\pgfpathlineto{\pgfqpoint{3.850814in}{1.171189in}}%
\pgfpathlineto{\pgfqpoint{3.853358in}{1.163417in}}%
\pgfpathlineto{\pgfqpoint{3.856100in}{1.161908in}}%
\pgfpathlineto{\pgfqpoint{3.858720in}{1.162194in}}%
\pgfpathlineto{\pgfqpoint{3.861561in}{1.158481in}}%
\pgfpathlineto{\pgfqpoint{3.864073in}{1.162320in}}%
\pgfpathlineto{\pgfqpoint{3.866815in}{1.161628in}}%
\pgfpathlineto{\pgfqpoint{3.869435in}{1.166569in}}%
\pgfpathlineto{\pgfqpoint{3.872114in}{1.166592in}}%
\pgfpathlineto{\pgfqpoint{3.874790in}{1.165037in}}%
\pgfpathlineto{\pgfqpoint{3.877466in}{1.169748in}}%
\pgfpathlineto{\pgfqpoint{3.880237in}{1.165970in}}%
\pgfpathlineto{\pgfqpoint{3.882850in}{1.169347in}}%
\pgfpathlineto{\pgfqpoint{3.885621in}{1.165843in}}%
\pgfpathlineto{\pgfqpoint{3.888188in}{1.164699in}}%
\pgfpathlineto{\pgfqpoint{3.890926in}{1.161817in}}%
\pgfpathlineto{\pgfqpoint{3.893541in}{1.175434in}}%
\pgfpathlineto{\pgfqpoint{3.896345in}{1.178757in}}%
\pgfpathlineto{\pgfqpoint{3.898891in}{1.174495in}}%
\pgfpathlineto{\pgfqpoint{3.901573in}{1.173982in}}%
\pgfpathlineto{\pgfqpoint{3.904252in}{1.172858in}}%
\pgfpathlineto{\pgfqpoint{3.906918in}{1.173842in}}%
\pgfpathlineto{\pgfqpoint{3.909602in}{1.171619in}}%
\pgfpathlineto{\pgfqpoint{3.912296in}{1.172473in}}%
\pgfpathlineto{\pgfqpoint{3.915107in}{1.171412in}}%
\pgfpathlineto{\pgfqpoint{3.917646in}{1.168907in}}%
\pgfpathlineto{\pgfqpoint{3.920412in}{1.167140in}}%
\pgfpathlineto{\pgfqpoint{3.923005in}{1.168822in}}%
\pgfpathlineto{\pgfqpoint{3.925778in}{1.173622in}}%
\pgfpathlineto{\pgfqpoint{3.928347in}{1.171167in}}%
\pgfpathlineto{\pgfqpoint{3.931202in}{1.172907in}}%
\pgfpathlineto{\pgfqpoint{3.933714in}{1.171235in}}%
\pgfpathlineto{\pgfqpoint{3.936395in}{1.168728in}}%
\pgfpathlineto{\pgfqpoint{3.939075in}{1.201545in}}%
\pgfpathlineto{\pgfqpoint{3.941778in}{1.255389in}}%
\pgfpathlineto{\pgfqpoint{3.944431in}{1.254262in}}%
\pgfpathlineto{\pgfqpoint{3.947101in}{1.233095in}}%
\pgfpathlineto{\pgfqpoint{3.949894in}{1.221224in}}%
\pgfpathlineto{\pgfqpoint{3.952464in}{1.204468in}}%
\pgfpathlineto{\pgfqpoint{3.955211in}{1.213967in}}%
\pgfpathlineto{\pgfqpoint{3.957823in}{1.210332in}}%
\pgfpathlineto{\pgfqpoint{3.960635in}{1.209170in}}%
\pgfpathlineto{\pgfqpoint{3.963176in}{1.206195in}}%
\pgfpathlineto{\pgfqpoint{3.966013in}{1.193259in}}%
\pgfpathlineto{\pgfqpoint{3.968523in}{1.189504in}}%
\pgfpathlineto{\pgfqpoint{3.971250in}{1.183299in}}%
\pgfpathlineto{\pgfqpoint{3.973885in}{1.174813in}}%
\pgfpathlineto{\pgfqpoint{3.976563in}{1.175813in}}%
\pgfpathlineto{\pgfqpoint{3.979389in}{1.176602in}}%
\pgfpathlineto{\pgfqpoint{3.981929in}{1.177345in}}%
\pgfpathlineto{\pgfqpoint{3.984714in}{1.175991in}}%
\pgfpathlineto{\pgfqpoint{3.987270in}{1.174342in}}%
\pgfpathlineto{\pgfqpoint{3.990055in}{1.168606in}}%
\pgfpathlineto{\pgfqpoint{3.992642in}{1.172535in}}%
\pgfpathlineto{\pgfqpoint{3.995417in}{1.169723in}}%
\pgfpathlineto{\pgfqpoint{3.997990in}{1.168342in}}%
\pgfpathlineto{\pgfqpoint{4.000674in}{1.169635in}}%
\pgfpathlineto{\pgfqpoint{4.003348in}{1.172598in}}%
\pgfpathlineto{\pgfqpoint{4.006034in}{1.166293in}}%
\pgfpathlineto{\pgfqpoint{4.008699in}{1.168594in}}%
\pgfpathlineto{\pgfqpoint{4.011394in}{1.171726in}}%
\pgfpathlineto{\pgfqpoint{4.014186in}{1.167583in}}%
\pgfpathlineto{\pgfqpoint{4.016744in}{1.167157in}}%
\pgfpathlineto{\pgfqpoint{4.019518in}{1.165306in}}%
\pgfpathlineto{\pgfqpoint{4.022097in}{1.167621in}}%
\pgfpathlineto{\pgfqpoint{4.024868in}{1.165321in}}%
\pgfpathlineto{\pgfqpoint{4.027447in}{1.166468in}}%
\pgfpathlineto{\pgfqpoint{4.030229in}{1.166165in}}%
\pgfpathlineto{\pgfqpoint{4.032817in}{1.169087in}}%
\pgfpathlineto{\pgfqpoint{4.035492in}{1.167792in}}%
\pgfpathlineto{\pgfqpoint{4.038174in}{1.167872in}}%
\pgfpathlineto{\pgfqpoint{4.040852in}{1.168702in}}%
\pgfpathlineto{\pgfqpoint{4.043667in}{1.169429in}}%
\pgfpathlineto{\pgfqpoint{4.046210in}{1.177018in}}%
\pgfpathlineto{\pgfqpoint{4.049006in}{1.172697in}}%
\pgfpathlineto{\pgfqpoint{4.051557in}{1.172919in}}%
\pgfpathlineto{\pgfqpoint{4.054326in}{1.170173in}}%
\pgfpathlineto{\pgfqpoint{4.056911in}{1.173151in}}%
\pgfpathlineto{\pgfqpoint{4.059702in}{1.172026in}}%
\pgfpathlineto{\pgfqpoint{4.062266in}{1.174764in}}%
\pgfpathlineto{\pgfqpoint{4.064957in}{1.175529in}}%
\pgfpathlineto{\pgfqpoint{4.067636in}{1.173930in}}%
\pgfpathlineto{\pgfqpoint{4.070313in}{1.172656in}}%
\pgfpathlineto{\pgfqpoint{4.072985in}{1.172025in}}%
\pgfpathlineto{\pgfqpoint{4.075705in}{1.170954in}}%
\pgfpathlineto{\pgfqpoint{4.078471in}{1.162516in}}%
\pgfpathlineto{\pgfqpoint{4.081018in}{1.160905in}}%
\pgfpathlineto{\pgfqpoint{4.083870in}{1.162440in}}%
\pgfpathlineto{\pgfqpoint{4.086385in}{1.163113in}}%
\pgfpathlineto{\pgfqpoint{4.089159in}{1.162933in}}%
\pgfpathlineto{\pgfqpoint{4.091729in}{1.169149in}}%
\pgfpathlineto{\pgfqpoint{4.094527in}{1.167961in}}%
\pgfpathlineto{\pgfqpoint{4.097092in}{1.167912in}}%
\pgfpathlineto{\pgfqpoint{4.099777in}{1.168822in}}%
\pgfpathlineto{\pgfqpoint{4.102456in}{1.184766in}}%
\pgfpathlineto{\pgfqpoint{4.105185in}{1.256095in}}%
\pgfpathlineto{\pgfqpoint{4.107814in}{1.255706in}}%
\pgfpathlineto{\pgfqpoint{4.110488in}{1.234603in}}%
\pgfpathlineto{\pgfqpoint{4.113252in}{1.216491in}}%
\pgfpathlineto{\pgfqpoint{4.115844in}{1.212184in}}%
\pgfpathlineto{\pgfqpoint{4.118554in}{1.214000in}}%
\pgfpathlineto{\pgfqpoint{4.121205in}{1.235111in}}%
\pgfpathlineto{\pgfqpoint{4.124019in}{1.290113in}}%
\pgfpathlineto{\pgfqpoint{4.126553in}{1.355137in}}%
\pgfpathlineto{\pgfqpoint{4.129349in}{1.341804in}}%
\pgfpathlineto{\pgfqpoint{4.131920in}{1.317827in}}%
\pgfpathlineto{\pgfqpoint{4.134615in}{1.293253in}}%
\pgfpathlineto{\pgfqpoint{4.137272in}{1.271912in}}%
\pgfpathlineto{\pgfqpoint{4.139963in}{1.253752in}}%
\pgfpathlineto{\pgfqpoint{4.142713in}{1.237794in}}%
\pgfpathlineto{\pgfqpoint{4.145310in}{1.226673in}}%
\pgfpathlineto{\pgfqpoint{4.148082in}{1.219198in}}%
\pgfpathlineto{\pgfqpoint{4.150665in}{1.201431in}}%
\pgfpathlineto{\pgfqpoint{4.153423in}{1.199317in}}%
\pgfpathlineto{\pgfqpoint{4.156016in}{1.189554in}}%
\pgfpathlineto{\pgfqpoint{4.158806in}{1.183553in}}%
\pgfpathlineto{\pgfqpoint{4.161380in}{1.192229in}}%
\pgfpathlineto{\pgfqpoint{4.164059in}{1.240411in}}%
\pgfpathlineto{\pgfqpoint{4.166737in}{1.216220in}}%
\pgfpathlineto{\pgfqpoint{4.169415in}{1.197858in}}%
\pgfpathlineto{\pgfqpoint{4.172093in}{1.188935in}}%
\pgfpathlineto{\pgfqpoint{4.174770in}{1.186724in}}%
\pgfpathlineto{\pgfqpoint{4.177593in}{1.181893in}}%
\pgfpathlineto{\pgfqpoint{4.180129in}{1.213377in}}%
\pgfpathlineto{\pgfqpoint{4.182899in}{1.309829in}}%
\pgfpathlineto{\pgfqpoint{4.185481in}{1.353091in}}%
\pgfpathlineto{\pgfqpoint{4.188318in}{1.344318in}}%
\pgfpathlineto{\pgfqpoint{4.190842in}{1.321668in}}%
\pgfpathlineto{\pgfqpoint{4.193638in}{1.301880in}}%
\pgfpathlineto{\pgfqpoint{4.196186in}{1.286915in}}%
\pgfpathlineto{\pgfqpoint{4.198878in}{1.272736in}}%
\pgfpathlineto{\pgfqpoint{4.201542in}{1.274648in}}%
\pgfpathlineto{\pgfqpoint{4.204240in}{1.260989in}}%
\pgfpathlineto{\pgfqpoint{4.207076in}{1.240537in}}%
\pgfpathlineto{\pgfqpoint{4.209597in}{1.218241in}}%
\pgfpathlineto{\pgfqpoint{4.212383in}{1.211659in}}%
\pgfpathlineto{\pgfqpoint{4.214948in}{1.205891in}}%
\pgfpathlineto{\pgfqpoint{4.217694in}{1.196666in}}%
\pgfpathlineto{\pgfqpoint{4.220304in}{1.190473in}}%
\pgfpathlineto{\pgfqpoint{4.223082in}{1.187440in}}%
\pgfpathlineto{\pgfqpoint{4.225654in}{1.184396in}}%
\pgfpathlineto{\pgfqpoint{4.228331in}{1.179610in}}%
\pgfpathlineto{\pgfqpoint{4.231013in}{1.182518in}}%
\pgfpathlineto{\pgfqpoint{4.233691in}{1.178516in}}%
\pgfpathlineto{\pgfqpoint{4.236375in}{1.179253in}}%
\pgfpathlineto{\pgfqpoint{4.239084in}{1.175493in}}%
\pgfpathlineto{\pgfqpoint{4.241900in}{1.177071in}}%
\pgfpathlineto{\pgfqpoint{4.244394in}{1.176460in}}%
\pgfpathlineto{\pgfqpoint{4.247225in}{1.175324in}}%
\pgfpathlineto{\pgfqpoint{4.249767in}{1.177868in}}%
\pgfpathlineto{\pgfqpoint{4.252581in}{1.168672in}}%
\pgfpathlineto{\pgfqpoint{4.255120in}{1.168587in}}%
\pgfpathlineto{\pgfqpoint{4.257958in}{1.172992in}}%
\pgfpathlineto{\pgfqpoint{4.260477in}{1.174948in}}%
\pgfpathlineto{\pgfqpoint{4.263157in}{1.169934in}}%
\pgfpathlineto{\pgfqpoint{4.265824in}{1.172013in}}%
\pgfpathlineto{\pgfqpoint{4.268590in}{1.171952in}}%
\pgfpathlineto{\pgfqpoint{4.271187in}{1.175462in}}%
\pgfpathlineto{\pgfqpoint{4.273874in}{1.170448in}}%
\pgfpathlineto{\pgfqpoint{4.276635in}{1.169961in}}%
\pgfpathlineto{\pgfqpoint{4.279212in}{1.165354in}}%
\pgfpathlineto{\pgfqpoint{4.282000in}{1.169883in}}%
\pgfpathlineto{\pgfqpoint{4.284586in}{1.171971in}}%
\pgfpathlineto{\pgfqpoint{4.287399in}{1.161501in}}%
\pgfpathlineto{\pgfqpoint{4.289936in}{1.168414in}}%
\pgfpathlineto{\pgfqpoint{4.292786in}{1.164768in}}%
\pgfpathlineto{\pgfqpoint{4.295299in}{1.163191in}}%
\pgfpathlineto{\pgfqpoint{4.297977in}{1.168414in}}%
\pgfpathlineto{\pgfqpoint{4.300656in}{1.168707in}}%
\pgfpathlineto{\pgfqpoint{4.303357in}{1.169529in}}%
\pgfpathlineto{\pgfqpoint{4.306118in}{1.168022in}}%
\pgfpathlineto{\pgfqpoint{4.308691in}{1.164623in}}%
\pgfpathlineto{\pgfqpoint{4.311494in}{1.166753in}}%
\pgfpathlineto{\pgfqpoint{4.314032in}{1.166907in}}%
\pgfpathlineto{\pgfqpoint{4.316856in}{1.163015in}}%
\pgfpathlineto{\pgfqpoint{4.319405in}{1.165950in}}%
\pgfpathlineto{\pgfqpoint{4.322181in}{1.163750in}}%
\pgfpathlineto{\pgfqpoint{4.324760in}{1.169476in}}%
\pgfpathlineto{\pgfqpoint{4.327440in}{1.167653in}}%
\pgfpathlineto{\pgfqpoint{4.330118in}{1.166257in}}%
\pgfpathlineto{\pgfqpoint{4.332796in}{1.165945in}}%
\pgfpathlineto{\pgfqpoint{4.335463in}{1.166754in}}%
\pgfpathlineto{\pgfqpoint{4.338154in}{1.165371in}}%
\pgfpathlineto{\pgfqpoint{4.340976in}{1.166067in}}%
\pgfpathlineto{\pgfqpoint{4.343510in}{1.165709in}}%
\pgfpathlineto{\pgfqpoint{4.346263in}{1.164040in}}%
\pgfpathlineto{\pgfqpoint{4.348868in}{1.162603in}}%
\pgfpathlineto{\pgfqpoint{4.351645in}{1.167431in}}%
\pgfpathlineto{\pgfqpoint{4.354224in}{1.162830in}}%
\pgfpathlineto{\pgfqpoint{4.357014in}{1.170287in}}%
\pgfpathlineto{\pgfqpoint{4.359582in}{1.165922in}}%
\pgfpathlineto{\pgfqpoint{4.362270in}{1.166914in}}%
\pgfpathlineto{\pgfqpoint{4.364936in}{1.165973in}}%
\pgfpathlineto{\pgfqpoint{4.367646in}{1.163577in}}%
\pgfpathlineto{\pgfqpoint{4.370437in}{1.170470in}}%
\pgfpathlineto{\pgfqpoint{4.372976in}{1.174586in}}%
\pgfpathlineto{\pgfqpoint{4.375761in}{1.171516in}}%
\pgfpathlineto{\pgfqpoint{4.378329in}{1.171470in}}%
\pgfpathlineto{\pgfqpoint{4.381097in}{1.170424in}}%
\pgfpathlineto{\pgfqpoint{4.383674in}{1.168008in}}%
\pgfpathlineto{\pgfqpoint{4.386431in}{1.164457in}}%
\pgfpathlineto{\pgfqpoint{4.389044in}{1.167472in}}%
\pgfpathlineto{\pgfqpoint{4.391721in}{1.167281in}}%
\pgfpathlineto{\pgfqpoint{4.394400in}{1.162764in}}%
\pgfpathlineto{\pgfqpoint{4.397076in}{1.158159in}}%
\pgfpathlineto{\pgfqpoint{4.399745in}{1.157756in}}%
\pgfpathlineto{\pgfqpoint{4.402468in}{1.161871in}}%
\pgfpathlineto{\pgfqpoint{4.405234in}{1.165618in}}%
\pgfpathlineto{\pgfqpoint{4.407788in}{1.163089in}}%
\pgfpathlineto{\pgfqpoint{4.410587in}{1.171206in}}%
\pgfpathlineto{\pgfqpoint{4.413149in}{1.163739in}}%
\pgfpathlineto{\pgfqpoint{4.415932in}{1.169250in}}%
\pgfpathlineto{\pgfqpoint{4.418506in}{1.170586in}}%
\pgfpathlineto{\pgfqpoint{4.421292in}{1.170875in}}%
\pgfpathlineto{\pgfqpoint{4.423863in}{1.173015in}}%
\pgfpathlineto{\pgfqpoint{4.426534in}{1.172683in}}%
\pgfpathlineto{\pgfqpoint{4.429220in}{1.169554in}}%
\pgfpathlineto{\pgfqpoint{4.431901in}{1.175562in}}%
\pgfpathlineto{\pgfqpoint{4.434569in}{1.172237in}}%
\pgfpathlineto{\pgfqpoint{4.437253in}{1.181657in}}%
\pgfpathlineto{\pgfqpoint{4.440041in}{1.183511in}}%
\pgfpathlineto{\pgfqpoint{4.442611in}{1.182360in}}%
\pgfpathlineto{\pgfqpoint{4.445423in}{1.177551in}}%
\pgfpathlineto{\pgfqpoint{4.447965in}{1.174754in}}%
\pgfpathlineto{\pgfqpoint{4.450767in}{1.170999in}}%
\pgfpathlineto{\pgfqpoint{4.453312in}{1.165566in}}%
\pgfpathlineto{\pgfqpoint{4.456138in}{1.163256in}}%
\pgfpathlineto{\pgfqpoint{4.458681in}{1.171363in}}%
\pgfpathlineto{\pgfqpoint{4.461367in}{1.174262in}}%
\pgfpathlineto{\pgfqpoint{4.464029in}{1.170123in}}%
\pgfpathlineto{\pgfqpoint{4.466717in}{1.173872in}}%
\pgfpathlineto{\pgfqpoint{4.469492in}{1.172107in}}%
\pgfpathlineto{\pgfqpoint{4.472059in}{1.165920in}}%
\pgfpathlineto{\pgfqpoint{4.474861in}{1.165286in}}%
\pgfpathlineto{\pgfqpoint{4.477430in}{1.163269in}}%
\pgfpathlineto{\pgfqpoint{4.480201in}{1.163946in}}%
\pgfpathlineto{\pgfqpoint{4.482778in}{1.165669in}}%
\pgfpathlineto{\pgfqpoint{4.485581in}{1.163009in}}%
\pgfpathlineto{\pgfqpoint{4.488130in}{1.158481in}}%
\pgfpathlineto{\pgfqpoint{4.490822in}{1.161242in}}%
\pgfpathlineto{\pgfqpoint{4.493492in}{1.161063in}}%
\pgfpathlineto{\pgfqpoint{4.496167in}{1.165369in}}%
\pgfpathlineto{\pgfqpoint{4.498850in}{1.161087in}}%
\pgfpathlineto{\pgfqpoint{4.501529in}{1.163770in}}%
\pgfpathlineto{\pgfqpoint{4.504305in}{1.165775in}}%
\pgfpathlineto{\pgfqpoint{4.506893in}{1.169167in}}%
\pgfpathlineto{\pgfqpoint{4.509643in}{1.167576in}}%
\pgfpathlineto{\pgfqpoint{4.512246in}{1.170688in}}%
\pgfpathlineto{\pgfqpoint{4.515080in}{1.169149in}}%
\pgfpathlineto{\pgfqpoint{4.517598in}{1.166463in}}%
\pgfpathlineto{\pgfqpoint{4.520345in}{1.193984in}}%
\pgfpathlineto{\pgfqpoint{4.522962in}{1.236745in}}%
\pgfpathlineto{\pgfqpoint{4.525640in}{1.215511in}}%
\pgfpathlineto{\pgfqpoint{4.528307in}{1.198267in}}%
\pgfpathlineto{\pgfqpoint{4.530990in}{1.188670in}}%
\pgfpathlineto{\pgfqpoint{4.533764in}{1.185270in}}%
\pgfpathlineto{\pgfqpoint{4.536400in}{1.189343in}}%
\pgfpathlineto{\pgfqpoint{4.539144in}{1.180684in}}%
\pgfpathlineto{\pgfqpoint{4.541711in}{1.174750in}}%
\pgfpathlineto{\pgfqpoint{4.544464in}{1.174794in}}%
\pgfpathlineto{\pgfqpoint{4.547064in}{1.171757in}}%
\pgfpathlineto{\pgfqpoint{4.549822in}{1.169491in}}%
\pgfpathlineto{\pgfqpoint{4.552425in}{1.171106in}}%
\pgfpathlineto{\pgfqpoint{4.555106in}{1.167793in}}%
\pgfpathlineto{\pgfqpoint{4.557777in}{1.169529in}}%
\pgfpathlineto{\pgfqpoint{4.560448in}{1.175890in}}%
\pgfpathlineto{\pgfqpoint{4.563125in}{1.170218in}}%
\pgfpathlineto{\pgfqpoint{4.565820in}{1.173729in}}%
\pgfpathlineto{\pgfqpoint{4.568612in}{1.170559in}}%
\pgfpathlineto{\pgfqpoint{4.571171in}{1.172029in}}%
\pgfpathlineto{\pgfqpoint{4.573947in}{1.161920in}}%
\pgfpathlineto{\pgfqpoint{4.576531in}{1.168619in}}%
\pgfpathlineto{\pgfqpoint{4.579305in}{1.167190in}}%
\pgfpathlineto{\pgfqpoint{4.581888in}{1.166161in}}%
\pgfpathlineto{\pgfqpoint{4.584672in}{1.165475in}}%
\pgfpathlineto{\pgfqpoint{4.587244in}{1.169009in}}%
\pgfpathlineto{\pgfqpoint{4.589920in}{1.159709in}}%
\pgfpathlineto{\pgfqpoint{4.592589in}{1.159596in}}%
\pgfpathlineto{\pgfqpoint{4.595281in}{1.156710in}}%
\pgfpathlineto{\pgfqpoint{4.597951in}{1.161941in}}%
\pgfpathlineto{\pgfqpoint{4.600633in}{1.171234in}}%
\pgfpathlineto{\pgfqpoint{4.603430in}{1.161571in}}%
\pgfpathlineto{\pgfqpoint{4.605990in}{1.162584in}}%
\pgfpathlineto{\pgfqpoint{4.608808in}{1.162539in}}%
\pgfpathlineto{\pgfqpoint{4.611350in}{1.165473in}}%
\pgfpathlineto{\pgfqpoint{4.614134in}{1.163740in}}%
\pgfpathlineto{\pgfqpoint{4.616702in}{1.159167in}}%
\pgfpathlineto{\pgfqpoint{4.619529in}{1.167904in}}%
\pgfpathlineto{\pgfqpoint{4.622056in}{1.165307in}}%
\pgfpathlineto{\pgfqpoint{4.624741in}{1.165554in}}%
\pgfpathlineto{\pgfqpoint{4.627411in}{1.163457in}}%
\pgfpathlineto{\pgfqpoint{4.630096in}{1.162091in}}%
\pgfpathlineto{\pgfqpoint{4.632902in}{1.161901in}}%
\pgfpathlineto{\pgfqpoint{4.635445in}{1.165362in}}%
\pgfpathlineto{\pgfqpoint{4.638204in}{1.167440in}}%
\pgfpathlineto{\pgfqpoint{4.640809in}{1.164456in}}%
\pgfpathlineto{\pgfqpoint{4.643628in}{1.165998in}}%
\pgfpathlineto{\pgfqpoint{4.646169in}{1.169451in}}%
\pgfpathlineto{\pgfqpoint{4.648922in}{1.168772in}}%
\pgfpathlineto{\pgfqpoint{4.651524in}{1.166340in}}%
\pgfpathlineto{\pgfqpoint{4.654203in}{1.164919in}}%
\pgfpathlineto{\pgfqpoint{4.656873in}{1.170463in}}%
\pgfpathlineto{\pgfqpoint{4.659590in}{1.167648in}}%
\pgfpathlineto{\pgfqpoint{4.662237in}{1.163655in}}%
\pgfpathlineto{\pgfqpoint{4.664923in}{1.165899in}}%
\pgfpathlineto{\pgfqpoint{4.667764in}{1.161687in}}%
\pgfpathlineto{\pgfqpoint{4.670261in}{1.169304in}}%
\pgfpathlineto{\pgfqpoint{4.673068in}{1.164571in}}%
\pgfpathlineto{\pgfqpoint{4.675619in}{1.170694in}}%
\pgfpathlineto{\pgfqpoint{4.678448in}{1.168559in}}%
\pgfpathlineto{\pgfqpoint{4.680988in}{1.167697in}}%
\pgfpathlineto{\pgfqpoint{4.683799in}{1.168102in}}%
\pgfpathlineto{\pgfqpoint{4.686337in}{1.163761in}}%
\pgfpathlineto{\pgfqpoint{4.689051in}{1.168069in}}%
\pgfpathlineto{\pgfqpoint{4.691694in}{1.164427in}}%
\pgfpathlineto{\pgfqpoint{4.694381in}{1.167895in}}%
\pgfpathlineto{\pgfqpoint{4.697170in}{1.166914in}}%
\pgfpathlineto{\pgfqpoint{4.699734in}{1.169970in}}%
\pgfpathlineto{\pgfqpoint{4.702517in}{1.166829in}}%
\pgfpathlineto{\pgfqpoint{4.705094in}{1.166214in}}%
\pgfpathlineto{\pgfqpoint{4.707824in}{1.167179in}}%
\pgfpathlineto{\pgfqpoint{4.710437in}{1.164187in}}%
\pgfpathlineto{\pgfqpoint{4.713275in}{1.161733in}}%
\pgfpathlineto{\pgfqpoint{4.715806in}{1.165927in}}%
\pgfpathlineto{\pgfqpoint{4.718486in}{1.162879in}}%
\pgfpathlineto{\pgfqpoint{4.721160in}{1.166670in}}%
\pgfpathlineto{\pgfqpoint{4.723873in}{1.165528in}}%
\pgfpathlineto{\pgfqpoint{4.726508in}{1.165151in}}%
\pgfpathlineto{\pgfqpoint{4.729233in}{1.169177in}}%
\pgfpathlineto{\pgfqpoint{4.731901in}{1.163858in}}%
\pgfpathlineto{\pgfqpoint{4.734552in}{1.161574in}}%
\pgfpathlineto{\pgfqpoint{4.737348in}{1.167112in}}%
\pgfpathlineto{\pgfqpoint{4.739912in}{1.167288in}}%
\pgfpathlineto{\pgfqpoint{4.742696in}{1.168616in}}%
\pgfpathlineto{\pgfqpoint{4.745256in}{1.168829in}}%
\pgfpathlineto{\pgfqpoint{4.748081in}{1.161993in}}%
\pgfpathlineto{\pgfqpoint{4.750627in}{1.166736in}}%
\pgfpathlineto{\pgfqpoint{4.753298in}{1.164681in}}%
\pgfpathlineto{\pgfqpoint{4.755983in}{1.161581in}}%
\pgfpathlineto{\pgfqpoint{4.758653in}{1.168704in}}%
\pgfpathlineto{\pgfqpoint{4.761337in}{1.169313in}}%
\pgfpathlineto{\pgfqpoint{4.764018in}{1.167015in}}%
\pgfpathlineto{\pgfqpoint{4.766783in}{1.166092in}}%
\pgfpathlineto{\pgfqpoint{4.769367in}{1.165646in}}%
\pgfpathlineto{\pgfqpoint{4.772198in}{1.163382in}}%
\pgfpathlineto{\pgfqpoint{4.774732in}{1.162226in}}%
\pgfpathlineto{\pgfqpoint{4.777535in}{1.162894in}}%
\pgfpathlineto{\pgfqpoint{4.780083in}{1.164913in}}%
\pgfpathlineto{\pgfqpoint{4.782872in}{1.158167in}}%
\pgfpathlineto{\pgfqpoint{4.785445in}{1.162239in}}%
\pgfpathlineto{\pgfqpoint{4.788116in}{1.162699in}}%
\pgfpathlineto{\pgfqpoint{4.790798in}{1.165623in}}%
\pgfpathlineto{\pgfqpoint{4.793512in}{1.166345in}}%
\pgfpathlineto{\pgfqpoint{4.796274in}{1.164827in}}%
\pgfpathlineto{\pgfqpoint{4.798830in}{1.165190in}}%
\pgfpathlineto{\pgfqpoint{4.801586in}{1.165757in}}%
\pgfpathlineto{\pgfqpoint{4.804193in}{1.162071in}}%
\pgfpathlineto{\pgfqpoint{4.807017in}{1.162121in}}%
\pgfpathlineto{\pgfqpoint{4.809538in}{1.160240in}}%
\pgfpathlineto{\pgfqpoint{4.812377in}{1.158625in}}%
\pgfpathlineto{\pgfqpoint{4.814907in}{1.152088in}}%
\pgfpathlineto{\pgfqpoint{4.817587in}{1.150485in}}%
\pgfpathlineto{\pgfqpoint{4.820265in}{1.150485in}}%
\pgfpathlineto{\pgfqpoint{4.822945in}{1.150485in}}%
\pgfpathlineto{\pgfqpoint{4.825619in}{1.152704in}}%
\pgfpathlineto{\pgfqpoint{4.828291in}{1.150857in}}%
\pgfpathlineto{\pgfqpoint{4.831045in}{1.157615in}}%
\pgfpathlineto{\pgfqpoint{4.833657in}{1.161194in}}%
\pgfpathlineto{\pgfqpoint{4.837992in}{1.160863in}}%
\pgfpathlineto{\pgfqpoint{4.839922in}{1.160551in}}%
\pgfpathlineto{\pgfqpoint{4.842380in}{1.160221in}}%
\pgfpathlineto{\pgfqpoint{4.844361in}{1.161732in}}%
\pgfpathlineto{\pgfqpoint{4.847127in}{1.162700in}}%
\pgfpathlineto{\pgfqpoint{4.849715in}{1.174735in}}%
\pgfpathlineto{\pgfqpoint{4.852404in}{1.177181in}}%
\pgfpathlineto{\pgfqpoint{4.855070in}{1.172677in}}%
\pgfpathlineto{\pgfqpoint{4.857807in}{1.167306in}}%
\pgfpathlineto{\pgfqpoint{4.860544in}{1.169786in}}%
\pgfpathlineto{\pgfqpoint{4.863116in}{1.165331in}}%
\pgfpathlineto{\pgfqpoint{4.865910in}{1.167131in}}%
\pgfpathlineto{\pgfqpoint{4.868474in}{1.168592in}}%
\pgfpathlineto{\pgfqpoint{4.871209in}{1.169417in}}%
\pgfpathlineto{\pgfqpoint{4.873832in}{1.172608in}}%
\pgfpathlineto{\pgfqpoint{4.876636in}{1.172778in}}%
\pgfpathlineto{\pgfqpoint{4.879180in}{1.170888in}}%
\pgfpathlineto{\pgfqpoint{4.881864in}{1.170599in}}%
\pgfpathlineto{\pgfqpoint{4.884540in}{1.165337in}}%
\pgfpathlineto{\pgfqpoint{4.887211in}{1.167730in}}%
\pgfpathlineto{\pgfqpoint{4.889902in}{1.166841in}}%
\pgfpathlineto{\pgfqpoint{4.892611in}{1.169675in}}%
\pgfpathlineto{\pgfqpoint{4.895399in}{1.169081in}}%
\pgfpathlineto{\pgfqpoint{4.897938in}{1.170519in}}%
\pgfpathlineto{\pgfqpoint{4.900712in}{1.160788in}}%
\pgfpathlineto{\pgfqpoint{4.903295in}{1.159988in}}%
\pgfpathlineto{\pgfqpoint{4.906096in}{1.181266in}}%
\pgfpathlineto{\pgfqpoint{4.908648in}{1.213542in}}%
\pgfpathlineto{\pgfqpoint{4.911435in}{1.217444in}}%
\pgfpathlineto{\pgfqpoint{4.914009in}{1.200276in}}%
\pgfpathlineto{\pgfqpoint{4.916681in}{1.193263in}}%
\pgfpathlineto{\pgfqpoint{4.919352in}{1.186996in}}%
\pgfpathlineto{\pgfqpoint{4.922041in}{1.182548in}}%
\pgfpathlineto{\pgfqpoint{4.924708in}{1.172484in}}%
\pgfpathlineto{\pgfqpoint{4.927400in}{1.178036in}}%
\pgfpathlineto{\pgfqpoint{4.930170in}{1.175148in}}%
\pgfpathlineto{\pgfqpoint{4.932742in}{1.173955in}}%
\pgfpathlineto{\pgfqpoint{4.935515in}{1.169689in}}%
\pgfpathlineto{\pgfqpoint{4.938112in}{1.172160in}}%
\pgfpathlineto{\pgfqpoint{4.940881in}{1.172484in}}%
\pgfpathlineto{\pgfqpoint{4.943466in}{1.170118in}}%
\pgfpathlineto{\pgfqpoint{4.946151in}{1.170815in}}%
\pgfpathlineto{\pgfqpoint{4.948827in}{1.170003in}}%
\pgfpathlineto{\pgfqpoint{4.951504in}{1.167800in}}%
\pgfpathlineto{\pgfqpoint{4.954182in}{1.164920in}}%
\pgfpathlineto{\pgfqpoint{4.956862in}{1.171941in}}%
\pgfpathlineto{\pgfqpoint{4.959689in}{1.168701in}}%
\pgfpathlineto{\pgfqpoint{4.962219in}{1.168243in}}%
\pgfpathlineto{\pgfqpoint{4.965002in}{1.162925in}}%
\pgfpathlineto{\pgfqpoint{4.967575in}{1.167444in}}%
\pgfpathlineto{\pgfqpoint{4.970314in}{1.170231in}}%
\pgfpathlineto{\pgfqpoint{4.972933in}{1.168458in}}%
\pgfpathlineto{\pgfqpoint{4.975703in}{1.169319in}}%
\pgfpathlineto{\pgfqpoint{4.978287in}{1.169833in}}%
\pgfpathlineto{\pgfqpoint{4.980967in}{1.166827in}}%
\pgfpathlineto{\pgfqpoint{4.983637in}{1.169542in}}%
\pgfpathlineto{\pgfqpoint{4.986325in}{1.171021in}}%
\pgfpathlineto{\pgfqpoint{4.989001in}{1.167761in}}%
\pgfpathlineto{\pgfqpoint{4.991683in}{1.164282in}}%
\pgfpathlineto{\pgfqpoint{4.994390in}{1.166791in}}%
\pgfpathlineto{\pgfqpoint{4.997028in}{1.171228in}}%
\pgfpathlineto{\pgfqpoint{4.999780in}{1.167957in}}%
\pgfpathlineto{\pgfqpoint{5.002384in}{1.161105in}}%
\pgfpathlineto{\pgfqpoint{5.005178in}{1.167454in}}%
\pgfpathlineto{\pgfqpoint{5.007751in}{1.166621in}}%
\pgfpathlineto{\pgfqpoint{5.010562in}{1.162787in}}%
\pgfpathlineto{\pgfqpoint{5.013104in}{1.169048in}}%
\pgfpathlineto{\pgfqpoint{5.015820in}{1.180969in}}%
\pgfpathlineto{\pgfqpoint{5.018466in}{1.192887in}}%
\pgfpathlineto{\pgfqpoint{5.021147in}{1.185104in}}%
\pgfpathlineto{\pgfqpoint{5.023927in}{1.176287in}}%
\pgfpathlineto{\pgfqpoint{5.026501in}{1.170959in}}%
\pgfpathlineto{\pgfqpoint{5.029275in}{1.169916in}}%
\pgfpathlineto{\pgfqpoint{5.031849in}{1.177801in}}%
\pgfpathlineto{\pgfqpoint{5.034649in}{1.180422in}}%
\pgfpathlineto{\pgfqpoint{5.037214in}{1.187459in}}%
\pgfpathlineto{\pgfqpoint{5.039962in}{1.184395in}}%
\pgfpathlineto{\pgfqpoint{5.042572in}{1.175511in}}%
\pgfpathlineto{\pgfqpoint{5.045249in}{1.175249in}}%
\pgfpathlineto{\pgfqpoint{5.047924in}{1.179971in}}%
\pgfpathlineto{\pgfqpoint{5.050606in}{1.172297in}}%
\pgfpathlineto{\pgfqpoint{5.053284in}{1.172678in}}%
\pgfpathlineto{\pgfqpoint{5.055952in}{1.166702in}}%
\pgfpathlineto{\pgfqpoint{5.058711in}{1.170633in}}%
\pgfpathlineto{\pgfqpoint{5.061315in}{1.165896in}}%
\pgfpathlineto{\pgfqpoint{5.064144in}{1.175191in}}%
\pgfpathlineto{\pgfqpoint{5.066677in}{1.170199in}}%
\pgfpathlineto{\pgfqpoint{5.069463in}{1.165112in}}%
\pgfpathlineto{\pgfqpoint{5.072030in}{1.167495in}}%
\pgfpathlineto{\pgfqpoint{5.074851in}{1.164100in}}%
\pgfpathlineto{\pgfqpoint{5.077390in}{1.164905in}}%
\pgfpathlineto{\pgfqpoint{5.080067in}{1.159894in}}%
\pgfpathlineto{\pgfqpoint{5.082746in}{1.165601in}}%
\pgfpathlineto{\pgfqpoint{5.085426in}{1.167952in}}%
\pgfpathlineto{\pgfqpoint{5.088103in}{1.169263in}}%
\pgfpathlineto{\pgfqpoint{5.090788in}{1.164395in}}%
\pgfpathlineto{\pgfqpoint{5.093579in}{1.168472in}}%
\pgfpathlineto{\pgfqpoint{5.096142in}{1.166008in}}%
\pgfpathlineto{\pgfqpoint{5.098948in}{1.169830in}}%
\pgfpathlineto{\pgfqpoint{5.101496in}{1.167356in}}%
\pgfpathlineto{\pgfqpoint{5.104312in}{1.169496in}}%
\pgfpathlineto{\pgfqpoint{5.106842in}{1.165505in}}%
\pgfpathlineto{\pgfqpoint{5.109530in}{1.169181in}}%
\pgfpathlineto{\pgfqpoint{5.112209in}{1.170235in}}%
\pgfpathlineto{\pgfqpoint{5.114887in}{1.166668in}}%
\pgfpathlineto{\pgfqpoint{5.117550in}{1.167852in}}%
\pgfpathlineto{\pgfqpoint{5.120243in}{1.168112in}}%
\pgfpathlineto{\pgfqpoint{5.123042in}{1.166441in}}%
\pgfpathlineto{\pgfqpoint{5.125599in}{1.166859in}}%
\pgfpathlineto{\pgfqpoint{5.128421in}{1.166335in}}%
\pgfpathlineto{\pgfqpoint{5.130953in}{1.168301in}}%
\pgfpathlineto{\pgfqpoint{5.133716in}{1.166124in}}%
\pgfpathlineto{\pgfqpoint{5.136311in}{1.166929in}}%
\pgfpathlineto{\pgfqpoint{5.139072in}{1.168143in}}%
\pgfpathlineto{\pgfqpoint{5.141660in}{1.167192in}}%
\pgfpathlineto{\pgfqpoint{5.144349in}{1.165511in}}%
\pgfpathlineto{\pgfqpoint{5.147029in}{1.167421in}}%
\pgfpathlineto{\pgfqpoint{5.149734in}{1.161731in}}%
\pgfpathlineto{\pgfqpoint{5.152382in}{1.156403in}}%
\pgfpathlineto{\pgfqpoint{5.155059in}{1.154645in}}%
\pgfpathlineto{\pgfqpoint{5.157815in}{1.157757in}}%
\pgfpathlineto{\pgfqpoint{5.160420in}{1.155685in}}%
\pgfpathlineto{\pgfqpoint{5.163243in}{1.158152in}}%
\pgfpathlineto{\pgfqpoint{5.165775in}{1.161747in}}%
\pgfpathlineto{\pgfqpoint{5.168591in}{1.159028in}}%
\pgfpathlineto{\pgfqpoint{5.171133in}{1.161658in}}%
\pgfpathlineto{\pgfqpoint{5.173925in}{1.168781in}}%
\pgfpathlineto{\pgfqpoint{5.176477in}{1.166862in}}%
\pgfpathlineto{\pgfqpoint{5.179188in}{1.168561in}}%
\pgfpathlineto{\pgfqpoint{5.181848in}{1.167272in}}%
\pgfpathlineto{\pgfqpoint{5.184522in}{1.162880in}}%
\pgfpathlineto{\pgfqpoint{5.187294in}{1.162170in}}%
\pgfpathlineto{\pgfqpoint{5.189880in}{1.166992in}}%
\pgfpathlineto{\pgfqpoint{5.192680in}{1.162131in}}%
\pgfpathlineto{\pgfqpoint{5.195239in}{1.164976in}}%
\pgfpathlineto{\pgfqpoint{5.198008in}{1.165070in}}%
\pgfpathlineto{\pgfqpoint{5.200594in}{1.164158in}}%
\pgfpathlineto{\pgfqpoint{5.203388in}{1.155433in}}%
\pgfpathlineto{\pgfqpoint{5.205952in}{1.158914in}}%
\pgfpathlineto{\pgfqpoint{5.208630in}{1.155718in}}%
\pgfpathlineto{\pgfqpoint{5.211299in}{1.162612in}}%
\pgfpathlineto{\pgfqpoint{5.214027in}{1.163820in}}%
\pgfpathlineto{\pgfqpoint{5.216667in}{1.162005in}}%
\pgfpathlineto{\pgfqpoint{5.219345in}{1.163402in}}%
\pgfpathlineto{\pgfqpoint{5.222151in}{1.164919in}}%
\pgfpathlineto{\pgfqpoint{5.224695in}{1.186224in}}%
\pgfpathlineto{\pgfqpoint{5.227470in}{1.180279in}}%
\pgfpathlineto{\pgfqpoint{5.230059in}{1.172910in}}%
\pgfpathlineto{\pgfqpoint{5.232855in}{1.177252in}}%
\pgfpathlineto{\pgfqpoint{5.235409in}{1.174914in}}%
\pgfpathlineto{\pgfqpoint{5.238173in}{1.176542in}}%
\pgfpathlineto{\pgfqpoint{5.240777in}{1.180170in}}%
\pgfpathlineto{\pgfqpoint{5.243445in}{1.169856in}}%
\pgfpathlineto{\pgfqpoint{5.246130in}{1.172393in}}%
\pgfpathlineto{\pgfqpoint{5.248816in}{1.174045in}}%
\pgfpathlineto{\pgfqpoint{5.251590in}{1.169850in}}%
\pgfpathlineto{\pgfqpoint{5.254236in}{1.169570in}}%
\pgfpathlineto{\pgfqpoint{5.256973in}{1.169737in}}%
\pgfpathlineto{\pgfqpoint{5.259511in}{1.170134in}}%
\pgfpathlineto{\pgfqpoint{5.262264in}{1.164806in}}%
\pgfpathlineto{\pgfqpoint{5.264876in}{1.159297in}}%
\pgfpathlineto{\pgfqpoint{5.267691in}{1.163883in}}%
\pgfpathlineto{\pgfqpoint{5.270238in}{1.163560in}}%
\pgfpathlineto{\pgfqpoint{5.272913in}{1.163641in}}%
\pgfpathlineto{\pgfqpoint{5.275589in}{1.159189in}}%
\pgfpathlineto{\pgfqpoint{5.278322in}{1.162754in}}%
\pgfpathlineto{\pgfqpoint{5.280947in}{1.157241in}}%
\pgfpathlineto{\pgfqpoint{5.283631in}{1.161489in}}%
\pgfpathlineto{\pgfqpoint{5.286436in}{1.175053in}}%
\pgfpathlineto{\pgfqpoint{5.288984in}{1.167117in}}%
\pgfpathlineto{\pgfqpoint{5.291794in}{1.167888in}}%
\pgfpathlineto{\pgfqpoint{5.294339in}{1.169852in}}%
\pgfpathlineto{\pgfqpoint{5.297140in}{1.171430in}}%
\pgfpathlineto{\pgfqpoint{5.299696in}{1.171308in}}%
\pgfpathlineto{\pgfqpoint{5.302443in}{1.173406in}}%
\pgfpathlineto{\pgfqpoint{5.305054in}{1.172656in}}%
\pgfpathlineto{\pgfqpoint{5.307731in}{1.167505in}}%
\pgfpathlineto{\pgfqpoint{5.310411in}{1.170199in}}%
\pgfpathlineto{\pgfqpoint{5.313089in}{1.172062in}}%
\pgfpathlineto{\pgfqpoint{5.315754in}{1.172036in}}%
\pgfpathlineto{\pgfqpoint{5.318430in}{1.168223in}}%
\pgfpathlineto{\pgfqpoint{5.321256in}{1.170181in}}%
\pgfpathlineto{\pgfqpoint{5.323802in}{1.156798in}}%
\pgfpathlineto{\pgfqpoint{5.326564in}{1.155910in}}%
\pgfpathlineto{\pgfqpoint{5.329159in}{1.152123in}}%
\pgfpathlineto{\pgfqpoint{5.331973in}{1.151993in}}%
\pgfpathlineto{\pgfqpoint{5.334510in}{1.158662in}}%
\pgfpathlineto{\pgfqpoint{5.337353in}{1.157986in}}%
\pgfpathlineto{\pgfqpoint{5.339872in}{1.158663in}}%
\pgfpathlineto{\pgfqpoint{5.342549in}{1.159512in}}%
\pgfpathlineto{\pgfqpoint{5.345224in}{1.172625in}}%
\pgfpathlineto{\pgfqpoint{5.347905in}{1.182151in}}%
\pgfpathlineto{\pgfqpoint{5.350723in}{1.169045in}}%
\pgfpathlineto{\pgfqpoint{5.353262in}{1.173995in}}%
\pgfpathlineto{\pgfqpoint{5.356056in}{1.170554in}}%
\pgfpathlineto{\pgfqpoint{5.358612in}{1.161591in}}%
\pgfpathlineto{\pgfqpoint{5.361370in}{1.162341in}}%
\pgfpathlineto{\pgfqpoint{5.363966in}{1.162103in}}%
\pgfpathlineto{\pgfqpoint{5.366727in}{1.158392in}}%
\pgfpathlineto{\pgfqpoint{5.369335in}{1.161954in}}%
\pgfpathlineto{\pgfqpoint{5.372013in}{1.159347in}}%
\pgfpathlineto{\pgfqpoint{5.374692in}{1.157948in}}%
\pgfpathlineto{\pgfqpoint{5.377370in}{1.164637in}}%
\pgfpathlineto{\pgfqpoint{5.380048in}{1.166663in}}%
\pgfpathlineto{\pgfqpoint{5.382725in}{1.166329in}}%
\pgfpathlineto{\pgfqpoint{5.385550in}{1.168239in}}%
\pgfpathlineto{\pgfqpoint{5.388083in}{1.169276in}}%
\pgfpathlineto{\pgfqpoint{5.390900in}{1.170623in}}%
\pgfpathlineto{\pgfqpoint{5.393441in}{1.163865in}}%
\pgfpathlineto{\pgfqpoint{5.396219in}{1.168492in}}%
\pgfpathlineto{\pgfqpoint{5.398784in}{1.175058in}}%
\pgfpathlineto{\pgfqpoint{5.401576in}{1.180121in}}%
\pgfpathlineto{\pgfqpoint{5.404154in}{1.174465in}}%
\pgfpathlineto{\pgfqpoint{5.406832in}{1.170588in}}%
\pgfpathlineto{\pgfqpoint{5.409507in}{1.170075in}}%
\pgfpathlineto{\pgfqpoint{5.412190in}{1.168218in}}%
\pgfpathlineto{\pgfqpoint{5.414954in}{1.174772in}}%
\pgfpathlineto{\pgfqpoint{5.417547in}{1.174640in}}%
\pgfpathlineto{\pgfqpoint{5.420304in}{1.180438in}}%
\pgfpathlineto{\pgfqpoint{5.422897in}{1.174999in}}%
\pgfpathlineto{\pgfqpoint{5.425661in}{1.174978in}}%
\pgfpathlineto{\pgfqpoint{5.428259in}{1.171725in}}%
\pgfpathlineto{\pgfqpoint{5.431015in}{1.173641in}}%
\pgfpathlineto{\pgfqpoint{5.433616in}{1.171165in}}%
\pgfpathlineto{\pgfqpoint{5.436295in}{1.172687in}}%
\pgfpathlineto{\pgfqpoint{5.438974in}{1.175557in}}%
\pgfpathlineto{\pgfqpoint{5.441698in}{1.175809in}}%
\pgfpathlineto{\pgfqpoint{5.444328in}{1.188943in}}%
\pgfpathlineto{\pgfqpoint{5.447021in}{1.178738in}}%
\pgfpathlineto{\pgfqpoint{5.449769in}{1.175545in}}%
\pgfpathlineto{\pgfqpoint{5.452365in}{1.168258in}}%
\pgfpathlineto{\pgfqpoint{5.455168in}{1.170455in}}%
\pgfpathlineto{\pgfqpoint{5.457721in}{1.171708in}}%
\pgfpathlineto{\pgfqpoint{5.460489in}{1.168799in}}%
\pgfpathlineto{\pgfqpoint{5.463079in}{1.170108in}}%
\pgfpathlineto{\pgfqpoint{5.465888in}{1.179315in}}%
\pgfpathlineto{\pgfqpoint{5.468425in}{1.170536in}}%
\pgfpathlineto{\pgfqpoint{5.471113in}{1.169570in}}%
\pgfpathlineto{\pgfqpoint{5.473792in}{1.165508in}}%
\pgfpathlineto{\pgfqpoint{5.476458in}{1.169537in}}%
\pgfpathlineto{\pgfqpoint{5.479152in}{1.168393in}}%
\pgfpathlineto{\pgfqpoint{5.481825in}{1.164109in}}%
\pgfpathlineto{\pgfqpoint{5.484641in}{1.168125in}}%
\pgfpathlineto{\pgfqpoint{5.487176in}{1.166744in}}%
\pgfpathlineto{\pgfqpoint{5.490000in}{1.168546in}}%
\pgfpathlineto{\pgfqpoint{5.492541in}{1.168734in}}%
\pgfpathlineto{\pgfqpoint{5.495346in}{1.169731in}}%
\pgfpathlineto{\pgfqpoint{5.497898in}{1.170427in}}%
\pgfpathlineto{\pgfqpoint{5.500687in}{1.168819in}}%
\pgfpathlineto{\pgfqpoint{5.503255in}{1.171632in}}%
\pgfpathlineto{\pgfqpoint{5.505933in}{1.168671in}}%
\pgfpathlineto{\pgfqpoint{5.508612in}{1.168732in}}%
\pgfpathlineto{\pgfqpoint{5.511290in}{1.167195in}}%
\pgfpathlineto{\pgfqpoint{5.514080in}{1.166954in}}%
\pgfpathlineto{\pgfqpoint{5.516646in}{1.163855in}}%
\pgfpathlineto{\pgfqpoint{5.519433in}{1.162536in}}%
\pgfpathlineto{\pgfqpoint{5.522003in}{1.167615in}}%
\pgfpathlineto{\pgfqpoint{5.524756in}{1.163733in}}%
\pgfpathlineto{\pgfqpoint{5.527360in}{1.164335in}}%
\pgfpathlineto{\pgfqpoint{5.530148in}{1.166157in}}%
\pgfpathlineto{\pgfqpoint{5.532717in}{1.169122in}}%
\pgfpathlineto{\pgfqpoint{5.535395in}{1.171840in}}%
\pgfpathlineto{\pgfqpoint{5.538074in}{1.167242in}}%
\pgfpathlineto{\pgfqpoint{5.540750in}{1.170115in}}%
\pgfpathlineto{\pgfqpoint{5.543421in}{1.171864in}}%
\pgfpathlineto{\pgfqpoint{5.546110in}{1.168105in}}%
\pgfpathlineto{\pgfqpoint{5.548921in}{1.167768in}}%
\pgfpathlineto{\pgfqpoint{5.551457in}{1.168246in}}%
\pgfpathlineto{\pgfqpoint{5.554198in}{1.174578in}}%
\pgfpathlineto{\pgfqpoint{5.556822in}{1.181391in}}%
\pgfpathlineto{\pgfqpoint{5.559612in}{1.181094in}}%
\pgfpathlineto{\pgfqpoint{5.562180in}{1.181934in}}%
\pgfpathlineto{\pgfqpoint{5.564940in}{1.184825in}}%
\pgfpathlineto{\pgfqpoint{5.567536in}{1.177970in}}%
\pgfpathlineto{\pgfqpoint{5.570215in}{1.174044in}}%
\pgfpathlineto{\pgfqpoint{5.572893in}{1.171733in}}%
\pgfpathlineto{\pgfqpoint{5.575596in}{1.172055in}}%
\pgfpathlineto{\pgfqpoint{5.578342in}{1.175852in}}%
\pgfpathlineto{\pgfqpoint{5.580914in}{1.172052in}}%
\pgfpathlineto{\pgfqpoint{5.583709in}{1.171558in}}%
\pgfpathlineto{\pgfqpoint{5.586269in}{1.172643in}}%
\pgfpathlineto{\pgfqpoint{5.589040in}{1.172570in}}%
\pgfpathlineto{\pgfqpoint{5.591641in}{1.173588in}}%
\pgfpathlineto{\pgfqpoint{5.594368in}{1.170217in}}%
\pgfpathlineto{\pgfqpoint{5.596999in}{1.171913in}}%
\pgfpathlineto{\pgfqpoint{5.599674in}{1.172447in}}%
\pgfpathlineto{\pgfqpoint{5.602352in}{1.174015in}}%
\pgfpathlineto{\pgfqpoint{5.605073in}{1.176271in}}%
\pgfpathlineto{\pgfqpoint{5.607698in}{1.172450in}}%
\pgfpathlineto{\pgfqpoint{5.610389in}{1.174497in}}%
\pgfpathlineto{\pgfqpoint{5.613235in}{1.167491in}}%
\pgfpathlineto{\pgfqpoint{5.615743in}{1.165888in}}%
\pgfpathlineto{\pgfqpoint{5.618526in}{1.170040in}}%
\pgfpathlineto{\pgfqpoint{5.621102in}{1.165178in}}%
\pgfpathlineto{\pgfqpoint{5.623868in}{1.167330in}}%
\pgfpathlineto{\pgfqpoint{5.626460in}{1.173370in}}%
\pgfpathlineto{\pgfqpoint{5.629232in}{1.171121in}}%
\pgfpathlineto{\pgfqpoint{5.631815in}{1.173171in}}%
\pgfpathlineto{\pgfqpoint{5.634496in}{1.177721in}}%
\pgfpathlineto{\pgfqpoint{5.637172in}{1.179457in}}%
\pgfpathlineto{\pgfqpoint{5.639852in}{1.178999in}}%
\pgfpathlineto{\pgfqpoint{5.642518in}{1.176794in}}%
\pgfpathlineto{\pgfqpoint{5.645243in}{1.179618in}}%
\pgfpathlineto{\pgfqpoint{5.648008in}{1.190335in}}%
\pgfpathlineto{\pgfqpoint{5.650563in}{1.184511in}}%
\pgfpathlineto{\pgfqpoint{5.653376in}{1.172386in}}%
\pgfpathlineto{\pgfqpoint{5.655919in}{1.170287in}}%
\pgfpathlineto{\pgfqpoint{5.658723in}{1.168117in}}%
\pgfpathlineto{\pgfqpoint{5.661273in}{1.171566in}}%
\pgfpathlineto{\pgfqpoint{5.664099in}{1.169215in}}%
\pgfpathlineto{\pgfqpoint{5.666632in}{1.171045in}}%
\pgfpathlineto{\pgfqpoint{5.669313in}{1.170600in}}%
\pgfpathlineto{\pgfqpoint{5.671991in}{1.172827in}}%
\pgfpathlineto{\pgfqpoint{5.674667in}{1.164541in}}%
\pgfpathlineto{\pgfqpoint{5.677486in}{1.163790in}}%
\pgfpathlineto{\pgfqpoint{5.680027in}{1.165977in}}%
\pgfpathlineto{\pgfqpoint{5.682836in}{1.169445in}}%
\pgfpathlineto{\pgfqpoint{5.685385in}{1.167659in}}%
\pgfpathlineto{\pgfqpoint{5.688159in}{1.167422in}}%
\pgfpathlineto{\pgfqpoint{5.690730in}{1.171300in}}%
\pgfpathlineto{\pgfqpoint{5.693473in}{1.176626in}}%
\pgfpathlineto{\pgfqpoint{5.696101in}{1.174260in}}%
\pgfpathlineto{\pgfqpoint{5.698775in}{1.170360in}}%
\pgfpathlineto{\pgfqpoint{5.701453in}{1.173150in}}%
\pgfpathlineto{\pgfqpoint{5.704130in}{1.175247in}}%
\pgfpathlineto{\pgfqpoint{5.706800in}{1.172033in}}%
\pgfpathlineto{\pgfqpoint{5.709490in}{1.177160in}}%
\pgfpathlineto{\pgfqpoint{5.712291in}{1.179454in}}%
\pgfpathlineto{\pgfqpoint{5.714834in}{1.174600in}}%
\pgfpathlineto{\pgfqpoint{5.717671in}{1.169696in}}%
\pgfpathlineto{\pgfqpoint{5.720201in}{1.165856in}}%
\pgfpathlineto{\pgfqpoint{5.722950in}{1.162621in}}%
\pgfpathlineto{\pgfqpoint{5.725548in}{1.168501in}}%
\pgfpathlineto{\pgfqpoint{5.728339in}{1.173332in}}%
\pgfpathlineto{\pgfqpoint{5.730919in}{1.172560in}}%
\pgfpathlineto{\pgfqpoint{5.733594in}{1.166647in}}%
\pgfpathlineto{\pgfqpoint{5.736276in}{1.168131in}}%
\pgfpathlineto{\pgfqpoint{5.738974in}{1.170584in}}%
\pgfpathlineto{\pgfqpoint{5.741745in}{1.164319in}}%
\pgfpathlineto{\pgfqpoint{5.744310in}{1.169278in}}%
\pgfpathlineto{\pgfqpoint{5.744310in}{0.413320in}}%
\pgfpathlineto{\pgfqpoint{5.744310in}{0.413320in}}%
\pgfpathlineto{\pgfqpoint{5.741745in}{0.413320in}}%
\pgfpathlineto{\pgfqpoint{5.738974in}{0.413320in}}%
\pgfpathlineto{\pgfqpoint{5.736276in}{0.413320in}}%
\pgfpathlineto{\pgfqpoint{5.733594in}{0.413320in}}%
\pgfpathlineto{\pgfqpoint{5.730919in}{0.413320in}}%
\pgfpathlineto{\pgfqpoint{5.728339in}{0.413320in}}%
\pgfpathlineto{\pgfqpoint{5.725548in}{0.413320in}}%
\pgfpathlineto{\pgfqpoint{5.722950in}{0.413320in}}%
\pgfpathlineto{\pgfqpoint{5.720201in}{0.413320in}}%
\pgfpathlineto{\pgfqpoint{5.717671in}{0.413320in}}%
\pgfpathlineto{\pgfqpoint{5.714834in}{0.413320in}}%
\pgfpathlineto{\pgfqpoint{5.712291in}{0.413320in}}%
\pgfpathlineto{\pgfqpoint{5.709490in}{0.413320in}}%
\pgfpathlineto{\pgfqpoint{5.706800in}{0.413320in}}%
\pgfpathlineto{\pgfqpoint{5.704130in}{0.413320in}}%
\pgfpathlineto{\pgfqpoint{5.701453in}{0.413320in}}%
\pgfpathlineto{\pgfqpoint{5.698775in}{0.413320in}}%
\pgfpathlineto{\pgfqpoint{5.696101in}{0.413320in}}%
\pgfpathlineto{\pgfqpoint{5.693473in}{0.413320in}}%
\pgfpathlineto{\pgfqpoint{5.690730in}{0.413320in}}%
\pgfpathlineto{\pgfqpoint{5.688159in}{0.413320in}}%
\pgfpathlineto{\pgfqpoint{5.685385in}{0.413320in}}%
\pgfpathlineto{\pgfqpoint{5.682836in}{0.413320in}}%
\pgfpathlineto{\pgfqpoint{5.680027in}{0.413320in}}%
\pgfpathlineto{\pgfqpoint{5.677486in}{0.413320in}}%
\pgfpathlineto{\pgfqpoint{5.674667in}{0.413320in}}%
\pgfpathlineto{\pgfqpoint{5.671991in}{0.413320in}}%
\pgfpathlineto{\pgfqpoint{5.669313in}{0.413320in}}%
\pgfpathlineto{\pgfqpoint{5.666632in}{0.413320in}}%
\pgfpathlineto{\pgfqpoint{5.664099in}{0.413320in}}%
\pgfpathlineto{\pgfqpoint{5.661273in}{0.413320in}}%
\pgfpathlineto{\pgfqpoint{5.658723in}{0.413320in}}%
\pgfpathlineto{\pgfqpoint{5.655919in}{0.413320in}}%
\pgfpathlineto{\pgfqpoint{5.653376in}{0.413320in}}%
\pgfpathlineto{\pgfqpoint{5.650563in}{0.413320in}}%
\pgfpathlineto{\pgfqpoint{5.648008in}{0.413320in}}%
\pgfpathlineto{\pgfqpoint{5.645243in}{0.413320in}}%
\pgfpathlineto{\pgfqpoint{5.642518in}{0.413320in}}%
\pgfpathlineto{\pgfqpoint{5.639852in}{0.413320in}}%
\pgfpathlineto{\pgfqpoint{5.637172in}{0.413320in}}%
\pgfpathlineto{\pgfqpoint{5.634496in}{0.413320in}}%
\pgfpathlineto{\pgfqpoint{5.631815in}{0.413320in}}%
\pgfpathlineto{\pgfqpoint{5.629232in}{0.413320in}}%
\pgfpathlineto{\pgfqpoint{5.626460in}{0.413320in}}%
\pgfpathlineto{\pgfqpoint{5.623868in}{0.413320in}}%
\pgfpathlineto{\pgfqpoint{5.621102in}{0.413320in}}%
\pgfpathlineto{\pgfqpoint{5.618526in}{0.413320in}}%
\pgfpathlineto{\pgfqpoint{5.615743in}{0.413320in}}%
\pgfpathlineto{\pgfqpoint{5.613235in}{0.413320in}}%
\pgfpathlineto{\pgfqpoint{5.610389in}{0.413320in}}%
\pgfpathlineto{\pgfqpoint{5.607698in}{0.413320in}}%
\pgfpathlineto{\pgfqpoint{5.605073in}{0.413320in}}%
\pgfpathlineto{\pgfqpoint{5.602352in}{0.413320in}}%
\pgfpathlineto{\pgfqpoint{5.599674in}{0.413320in}}%
\pgfpathlineto{\pgfqpoint{5.596999in}{0.413320in}}%
\pgfpathlineto{\pgfqpoint{5.594368in}{0.413320in}}%
\pgfpathlineto{\pgfqpoint{5.591641in}{0.413320in}}%
\pgfpathlineto{\pgfqpoint{5.589040in}{0.413320in}}%
\pgfpathlineto{\pgfqpoint{5.586269in}{0.413320in}}%
\pgfpathlineto{\pgfqpoint{5.583709in}{0.413320in}}%
\pgfpathlineto{\pgfqpoint{5.580914in}{0.413320in}}%
\pgfpathlineto{\pgfqpoint{5.578342in}{0.413320in}}%
\pgfpathlineto{\pgfqpoint{5.575596in}{0.413320in}}%
\pgfpathlineto{\pgfqpoint{5.572893in}{0.413320in}}%
\pgfpathlineto{\pgfqpoint{5.570215in}{0.413320in}}%
\pgfpathlineto{\pgfqpoint{5.567536in}{0.413320in}}%
\pgfpathlineto{\pgfqpoint{5.564940in}{0.413320in}}%
\pgfpathlineto{\pgfqpoint{5.562180in}{0.413320in}}%
\pgfpathlineto{\pgfqpoint{5.559612in}{0.413320in}}%
\pgfpathlineto{\pgfqpoint{5.556822in}{0.413320in}}%
\pgfpathlineto{\pgfqpoint{5.554198in}{0.413320in}}%
\pgfpathlineto{\pgfqpoint{5.551457in}{0.413320in}}%
\pgfpathlineto{\pgfqpoint{5.548921in}{0.413320in}}%
\pgfpathlineto{\pgfqpoint{5.546110in}{0.413320in}}%
\pgfpathlineto{\pgfqpoint{5.543421in}{0.413320in}}%
\pgfpathlineto{\pgfqpoint{5.540750in}{0.413320in}}%
\pgfpathlineto{\pgfqpoint{5.538074in}{0.413320in}}%
\pgfpathlineto{\pgfqpoint{5.535395in}{0.413320in}}%
\pgfpathlineto{\pgfqpoint{5.532717in}{0.413320in}}%
\pgfpathlineto{\pgfqpoint{5.530148in}{0.413320in}}%
\pgfpathlineto{\pgfqpoint{5.527360in}{0.413320in}}%
\pgfpathlineto{\pgfqpoint{5.524756in}{0.413320in}}%
\pgfpathlineto{\pgfqpoint{5.522003in}{0.413320in}}%
\pgfpathlineto{\pgfqpoint{5.519433in}{0.413320in}}%
\pgfpathlineto{\pgfqpoint{5.516646in}{0.413320in}}%
\pgfpathlineto{\pgfqpoint{5.514080in}{0.413320in}}%
\pgfpathlineto{\pgfqpoint{5.511290in}{0.413320in}}%
\pgfpathlineto{\pgfqpoint{5.508612in}{0.413320in}}%
\pgfpathlineto{\pgfqpoint{5.505933in}{0.413320in}}%
\pgfpathlineto{\pgfqpoint{5.503255in}{0.413320in}}%
\pgfpathlineto{\pgfqpoint{5.500687in}{0.413320in}}%
\pgfpathlineto{\pgfqpoint{5.497898in}{0.413320in}}%
\pgfpathlineto{\pgfqpoint{5.495346in}{0.413320in}}%
\pgfpathlineto{\pgfqpoint{5.492541in}{0.413320in}}%
\pgfpathlineto{\pgfqpoint{5.490000in}{0.413320in}}%
\pgfpathlineto{\pgfqpoint{5.487176in}{0.413320in}}%
\pgfpathlineto{\pgfqpoint{5.484641in}{0.413320in}}%
\pgfpathlineto{\pgfqpoint{5.481825in}{0.413320in}}%
\pgfpathlineto{\pgfqpoint{5.479152in}{0.413320in}}%
\pgfpathlineto{\pgfqpoint{5.476458in}{0.413320in}}%
\pgfpathlineto{\pgfqpoint{5.473792in}{0.413320in}}%
\pgfpathlineto{\pgfqpoint{5.471113in}{0.413320in}}%
\pgfpathlineto{\pgfqpoint{5.468425in}{0.413320in}}%
\pgfpathlineto{\pgfqpoint{5.465888in}{0.413320in}}%
\pgfpathlineto{\pgfqpoint{5.463079in}{0.413320in}}%
\pgfpathlineto{\pgfqpoint{5.460489in}{0.413320in}}%
\pgfpathlineto{\pgfqpoint{5.457721in}{0.413320in}}%
\pgfpathlineto{\pgfqpoint{5.455168in}{0.413320in}}%
\pgfpathlineto{\pgfqpoint{5.452365in}{0.413320in}}%
\pgfpathlineto{\pgfqpoint{5.449769in}{0.413320in}}%
\pgfpathlineto{\pgfqpoint{5.447021in}{0.413320in}}%
\pgfpathlineto{\pgfqpoint{5.444328in}{0.413320in}}%
\pgfpathlineto{\pgfqpoint{5.441698in}{0.413320in}}%
\pgfpathlineto{\pgfqpoint{5.438974in}{0.413320in}}%
\pgfpathlineto{\pgfqpoint{5.436295in}{0.413320in}}%
\pgfpathlineto{\pgfqpoint{5.433616in}{0.413320in}}%
\pgfpathlineto{\pgfqpoint{5.431015in}{0.413320in}}%
\pgfpathlineto{\pgfqpoint{5.428259in}{0.413320in}}%
\pgfpathlineto{\pgfqpoint{5.425661in}{0.413320in}}%
\pgfpathlineto{\pgfqpoint{5.422897in}{0.413320in}}%
\pgfpathlineto{\pgfqpoint{5.420304in}{0.413320in}}%
\pgfpathlineto{\pgfqpoint{5.417547in}{0.413320in}}%
\pgfpathlineto{\pgfqpoint{5.414954in}{0.413320in}}%
\pgfpathlineto{\pgfqpoint{5.412190in}{0.413320in}}%
\pgfpathlineto{\pgfqpoint{5.409507in}{0.413320in}}%
\pgfpathlineto{\pgfqpoint{5.406832in}{0.413320in}}%
\pgfpathlineto{\pgfqpoint{5.404154in}{0.413320in}}%
\pgfpathlineto{\pgfqpoint{5.401576in}{0.413320in}}%
\pgfpathlineto{\pgfqpoint{5.398784in}{0.413320in}}%
\pgfpathlineto{\pgfqpoint{5.396219in}{0.413320in}}%
\pgfpathlineto{\pgfqpoint{5.393441in}{0.413320in}}%
\pgfpathlineto{\pgfqpoint{5.390900in}{0.413320in}}%
\pgfpathlineto{\pgfqpoint{5.388083in}{0.413320in}}%
\pgfpathlineto{\pgfqpoint{5.385550in}{0.413320in}}%
\pgfpathlineto{\pgfqpoint{5.382725in}{0.413320in}}%
\pgfpathlineto{\pgfqpoint{5.380048in}{0.413320in}}%
\pgfpathlineto{\pgfqpoint{5.377370in}{0.413320in}}%
\pgfpathlineto{\pgfqpoint{5.374692in}{0.413320in}}%
\pgfpathlineto{\pgfqpoint{5.372013in}{0.413320in}}%
\pgfpathlineto{\pgfqpoint{5.369335in}{0.413320in}}%
\pgfpathlineto{\pgfqpoint{5.366727in}{0.413320in}}%
\pgfpathlineto{\pgfqpoint{5.363966in}{0.413320in}}%
\pgfpathlineto{\pgfqpoint{5.361370in}{0.413320in}}%
\pgfpathlineto{\pgfqpoint{5.358612in}{0.413320in}}%
\pgfpathlineto{\pgfqpoint{5.356056in}{0.413320in}}%
\pgfpathlineto{\pgfqpoint{5.353262in}{0.413320in}}%
\pgfpathlineto{\pgfqpoint{5.350723in}{0.413320in}}%
\pgfpathlineto{\pgfqpoint{5.347905in}{0.413320in}}%
\pgfpathlineto{\pgfqpoint{5.345224in}{0.413320in}}%
\pgfpathlineto{\pgfqpoint{5.342549in}{0.413320in}}%
\pgfpathlineto{\pgfqpoint{5.339872in}{0.413320in}}%
\pgfpathlineto{\pgfqpoint{5.337353in}{0.413320in}}%
\pgfpathlineto{\pgfqpoint{5.334510in}{0.413320in}}%
\pgfpathlineto{\pgfqpoint{5.331973in}{0.413320in}}%
\pgfpathlineto{\pgfqpoint{5.329159in}{0.413320in}}%
\pgfpathlineto{\pgfqpoint{5.326564in}{0.413320in}}%
\pgfpathlineto{\pgfqpoint{5.323802in}{0.413320in}}%
\pgfpathlineto{\pgfqpoint{5.321256in}{0.413320in}}%
\pgfpathlineto{\pgfqpoint{5.318430in}{0.413320in}}%
\pgfpathlineto{\pgfqpoint{5.315754in}{0.413320in}}%
\pgfpathlineto{\pgfqpoint{5.313089in}{0.413320in}}%
\pgfpathlineto{\pgfqpoint{5.310411in}{0.413320in}}%
\pgfpathlineto{\pgfqpoint{5.307731in}{0.413320in}}%
\pgfpathlineto{\pgfqpoint{5.305054in}{0.413320in}}%
\pgfpathlineto{\pgfqpoint{5.302443in}{0.413320in}}%
\pgfpathlineto{\pgfqpoint{5.299696in}{0.413320in}}%
\pgfpathlineto{\pgfqpoint{5.297140in}{0.413320in}}%
\pgfpathlineto{\pgfqpoint{5.294339in}{0.413320in}}%
\pgfpathlineto{\pgfqpoint{5.291794in}{0.413320in}}%
\pgfpathlineto{\pgfqpoint{5.288984in}{0.413320in}}%
\pgfpathlineto{\pgfqpoint{5.286436in}{0.413320in}}%
\pgfpathlineto{\pgfqpoint{5.283631in}{0.413320in}}%
\pgfpathlineto{\pgfqpoint{5.280947in}{0.413320in}}%
\pgfpathlineto{\pgfqpoint{5.278322in}{0.413320in}}%
\pgfpathlineto{\pgfqpoint{5.275589in}{0.413320in}}%
\pgfpathlineto{\pgfqpoint{5.272913in}{0.413320in}}%
\pgfpathlineto{\pgfqpoint{5.270238in}{0.413320in}}%
\pgfpathlineto{\pgfqpoint{5.267691in}{0.413320in}}%
\pgfpathlineto{\pgfqpoint{5.264876in}{0.413320in}}%
\pgfpathlineto{\pgfqpoint{5.262264in}{0.413320in}}%
\pgfpathlineto{\pgfqpoint{5.259511in}{0.413320in}}%
\pgfpathlineto{\pgfqpoint{5.256973in}{0.413320in}}%
\pgfpathlineto{\pgfqpoint{5.254236in}{0.413320in}}%
\pgfpathlineto{\pgfqpoint{5.251590in}{0.413320in}}%
\pgfpathlineto{\pgfqpoint{5.248816in}{0.413320in}}%
\pgfpathlineto{\pgfqpoint{5.246130in}{0.413320in}}%
\pgfpathlineto{\pgfqpoint{5.243445in}{0.413320in}}%
\pgfpathlineto{\pgfqpoint{5.240777in}{0.413320in}}%
\pgfpathlineto{\pgfqpoint{5.238173in}{0.413320in}}%
\pgfpathlineto{\pgfqpoint{5.235409in}{0.413320in}}%
\pgfpathlineto{\pgfqpoint{5.232855in}{0.413320in}}%
\pgfpathlineto{\pgfqpoint{5.230059in}{0.413320in}}%
\pgfpathlineto{\pgfqpoint{5.227470in}{0.413320in}}%
\pgfpathlineto{\pgfqpoint{5.224695in}{0.413320in}}%
\pgfpathlineto{\pgfqpoint{5.222151in}{0.413320in}}%
\pgfpathlineto{\pgfqpoint{5.219345in}{0.413320in}}%
\pgfpathlineto{\pgfqpoint{5.216667in}{0.413320in}}%
\pgfpathlineto{\pgfqpoint{5.214027in}{0.413320in}}%
\pgfpathlineto{\pgfqpoint{5.211299in}{0.413320in}}%
\pgfpathlineto{\pgfqpoint{5.208630in}{0.413320in}}%
\pgfpathlineto{\pgfqpoint{5.205952in}{0.413320in}}%
\pgfpathlineto{\pgfqpoint{5.203388in}{0.413320in}}%
\pgfpathlineto{\pgfqpoint{5.200594in}{0.413320in}}%
\pgfpathlineto{\pgfqpoint{5.198008in}{0.413320in}}%
\pgfpathlineto{\pgfqpoint{5.195239in}{0.413320in}}%
\pgfpathlineto{\pgfqpoint{5.192680in}{0.413320in}}%
\pgfpathlineto{\pgfqpoint{5.189880in}{0.413320in}}%
\pgfpathlineto{\pgfqpoint{5.187294in}{0.413320in}}%
\pgfpathlineto{\pgfqpoint{5.184522in}{0.413320in}}%
\pgfpathlineto{\pgfqpoint{5.181848in}{0.413320in}}%
\pgfpathlineto{\pgfqpoint{5.179188in}{0.413320in}}%
\pgfpathlineto{\pgfqpoint{5.176477in}{0.413320in}}%
\pgfpathlineto{\pgfqpoint{5.173925in}{0.413320in}}%
\pgfpathlineto{\pgfqpoint{5.171133in}{0.413320in}}%
\pgfpathlineto{\pgfqpoint{5.168591in}{0.413320in}}%
\pgfpathlineto{\pgfqpoint{5.165775in}{0.413320in}}%
\pgfpathlineto{\pgfqpoint{5.163243in}{0.413320in}}%
\pgfpathlineto{\pgfqpoint{5.160420in}{0.413320in}}%
\pgfpathlineto{\pgfqpoint{5.157815in}{0.413320in}}%
\pgfpathlineto{\pgfqpoint{5.155059in}{0.413320in}}%
\pgfpathlineto{\pgfqpoint{5.152382in}{0.413320in}}%
\pgfpathlineto{\pgfqpoint{5.149734in}{0.413320in}}%
\pgfpathlineto{\pgfqpoint{5.147029in}{0.413320in}}%
\pgfpathlineto{\pgfqpoint{5.144349in}{0.413320in}}%
\pgfpathlineto{\pgfqpoint{5.141660in}{0.413320in}}%
\pgfpathlineto{\pgfqpoint{5.139072in}{0.413320in}}%
\pgfpathlineto{\pgfqpoint{5.136311in}{0.413320in}}%
\pgfpathlineto{\pgfqpoint{5.133716in}{0.413320in}}%
\pgfpathlineto{\pgfqpoint{5.130953in}{0.413320in}}%
\pgfpathlineto{\pgfqpoint{5.128421in}{0.413320in}}%
\pgfpathlineto{\pgfqpoint{5.125599in}{0.413320in}}%
\pgfpathlineto{\pgfqpoint{5.123042in}{0.413320in}}%
\pgfpathlineto{\pgfqpoint{5.120243in}{0.413320in}}%
\pgfpathlineto{\pgfqpoint{5.117550in}{0.413320in}}%
\pgfpathlineto{\pgfqpoint{5.114887in}{0.413320in}}%
\pgfpathlineto{\pgfqpoint{5.112209in}{0.413320in}}%
\pgfpathlineto{\pgfqpoint{5.109530in}{0.413320in}}%
\pgfpathlineto{\pgfqpoint{5.106842in}{0.413320in}}%
\pgfpathlineto{\pgfqpoint{5.104312in}{0.413320in}}%
\pgfpathlineto{\pgfqpoint{5.101496in}{0.413320in}}%
\pgfpathlineto{\pgfqpoint{5.098948in}{0.413320in}}%
\pgfpathlineto{\pgfqpoint{5.096142in}{0.413320in}}%
\pgfpathlineto{\pgfqpoint{5.093579in}{0.413320in}}%
\pgfpathlineto{\pgfqpoint{5.090788in}{0.413320in}}%
\pgfpathlineto{\pgfqpoint{5.088103in}{0.413320in}}%
\pgfpathlineto{\pgfqpoint{5.085426in}{0.413320in}}%
\pgfpathlineto{\pgfqpoint{5.082746in}{0.413320in}}%
\pgfpathlineto{\pgfqpoint{5.080067in}{0.413320in}}%
\pgfpathlineto{\pgfqpoint{5.077390in}{0.413320in}}%
\pgfpathlineto{\pgfqpoint{5.074851in}{0.413320in}}%
\pgfpathlineto{\pgfqpoint{5.072030in}{0.413320in}}%
\pgfpathlineto{\pgfqpoint{5.069463in}{0.413320in}}%
\pgfpathlineto{\pgfqpoint{5.066677in}{0.413320in}}%
\pgfpathlineto{\pgfqpoint{5.064144in}{0.413320in}}%
\pgfpathlineto{\pgfqpoint{5.061315in}{0.413320in}}%
\pgfpathlineto{\pgfqpoint{5.058711in}{0.413320in}}%
\pgfpathlineto{\pgfqpoint{5.055952in}{0.413320in}}%
\pgfpathlineto{\pgfqpoint{5.053284in}{0.413320in}}%
\pgfpathlineto{\pgfqpoint{5.050606in}{0.413320in}}%
\pgfpathlineto{\pgfqpoint{5.047924in}{0.413320in}}%
\pgfpathlineto{\pgfqpoint{5.045249in}{0.413320in}}%
\pgfpathlineto{\pgfqpoint{5.042572in}{0.413320in}}%
\pgfpathlineto{\pgfqpoint{5.039962in}{0.413320in}}%
\pgfpathlineto{\pgfqpoint{5.037214in}{0.413320in}}%
\pgfpathlineto{\pgfqpoint{5.034649in}{0.413320in}}%
\pgfpathlineto{\pgfqpoint{5.031849in}{0.413320in}}%
\pgfpathlineto{\pgfqpoint{5.029275in}{0.413320in}}%
\pgfpathlineto{\pgfqpoint{5.026501in}{0.413320in}}%
\pgfpathlineto{\pgfqpoint{5.023927in}{0.413320in}}%
\pgfpathlineto{\pgfqpoint{5.021147in}{0.413320in}}%
\pgfpathlineto{\pgfqpoint{5.018466in}{0.413320in}}%
\pgfpathlineto{\pgfqpoint{5.015820in}{0.413320in}}%
\pgfpathlineto{\pgfqpoint{5.013104in}{0.413320in}}%
\pgfpathlineto{\pgfqpoint{5.010562in}{0.413320in}}%
\pgfpathlineto{\pgfqpoint{5.007751in}{0.413320in}}%
\pgfpathlineto{\pgfqpoint{5.005178in}{0.413320in}}%
\pgfpathlineto{\pgfqpoint{5.002384in}{0.413320in}}%
\pgfpathlineto{\pgfqpoint{4.999780in}{0.413320in}}%
\pgfpathlineto{\pgfqpoint{4.997028in}{0.413320in}}%
\pgfpathlineto{\pgfqpoint{4.994390in}{0.413320in}}%
\pgfpathlineto{\pgfqpoint{4.991683in}{0.413320in}}%
\pgfpathlineto{\pgfqpoint{4.989001in}{0.413320in}}%
\pgfpathlineto{\pgfqpoint{4.986325in}{0.413320in}}%
\pgfpathlineto{\pgfqpoint{4.983637in}{0.413320in}}%
\pgfpathlineto{\pgfqpoint{4.980967in}{0.413320in}}%
\pgfpathlineto{\pgfqpoint{4.978287in}{0.413320in}}%
\pgfpathlineto{\pgfqpoint{4.975703in}{0.413320in}}%
\pgfpathlineto{\pgfqpoint{4.972933in}{0.413320in}}%
\pgfpathlineto{\pgfqpoint{4.970314in}{0.413320in}}%
\pgfpathlineto{\pgfqpoint{4.967575in}{0.413320in}}%
\pgfpathlineto{\pgfqpoint{4.965002in}{0.413320in}}%
\pgfpathlineto{\pgfqpoint{4.962219in}{0.413320in}}%
\pgfpathlineto{\pgfqpoint{4.959689in}{0.413320in}}%
\pgfpathlineto{\pgfqpoint{4.956862in}{0.413320in}}%
\pgfpathlineto{\pgfqpoint{4.954182in}{0.413320in}}%
\pgfpathlineto{\pgfqpoint{4.951504in}{0.413320in}}%
\pgfpathlineto{\pgfqpoint{4.948827in}{0.413320in}}%
\pgfpathlineto{\pgfqpoint{4.946151in}{0.413320in}}%
\pgfpathlineto{\pgfqpoint{4.943466in}{0.413320in}}%
\pgfpathlineto{\pgfqpoint{4.940881in}{0.413320in}}%
\pgfpathlineto{\pgfqpoint{4.938112in}{0.413320in}}%
\pgfpathlineto{\pgfqpoint{4.935515in}{0.413320in}}%
\pgfpathlineto{\pgfqpoint{4.932742in}{0.413320in}}%
\pgfpathlineto{\pgfqpoint{4.930170in}{0.413320in}}%
\pgfpathlineto{\pgfqpoint{4.927400in}{0.413320in}}%
\pgfpathlineto{\pgfqpoint{4.924708in}{0.413320in}}%
\pgfpathlineto{\pgfqpoint{4.922041in}{0.413320in}}%
\pgfpathlineto{\pgfqpoint{4.919352in}{0.413320in}}%
\pgfpathlineto{\pgfqpoint{4.916681in}{0.413320in}}%
\pgfpathlineto{\pgfqpoint{4.914009in}{0.413320in}}%
\pgfpathlineto{\pgfqpoint{4.911435in}{0.413320in}}%
\pgfpathlineto{\pgfqpoint{4.908648in}{0.413320in}}%
\pgfpathlineto{\pgfqpoint{4.906096in}{0.413320in}}%
\pgfpathlineto{\pgfqpoint{4.903295in}{0.413320in}}%
\pgfpathlineto{\pgfqpoint{4.900712in}{0.413320in}}%
\pgfpathlineto{\pgfqpoint{4.897938in}{0.413320in}}%
\pgfpathlineto{\pgfqpoint{4.895399in}{0.413320in}}%
\pgfpathlineto{\pgfqpoint{4.892611in}{0.413320in}}%
\pgfpathlineto{\pgfqpoint{4.889902in}{0.413320in}}%
\pgfpathlineto{\pgfqpoint{4.887211in}{0.413320in}}%
\pgfpathlineto{\pgfqpoint{4.884540in}{0.413320in}}%
\pgfpathlineto{\pgfqpoint{4.881864in}{0.413320in}}%
\pgfpathlineto{\pgfqpoint{4.879180in}{0.413320in}}%
\pgfpathlineto{\pgfqpoint{4.876636in}{0.413320in}}%
\pgfpathlineto{\pgfqpoint{4.873832in}{0.413320in}}%
\pgfpathlineto{\pgfqpoint{4.871209in}{0.413320in}}%
\pgfpathlineto{\pgfqpoint{4.868474in}{0.413320in}}%
\pgfpathlineto{\pgfqpoint{4.865910in}{0.413320in}}%
\pgfpathlineto{\pgfqpoint{4.863116in}{0.413320in}}%
\pgfpathlineto{\pgfqpoint{4.860544in}{0.413320in}}%
\pgfpathlineto{\pgfqpoint{4.857807in}{0.413320in}}%
\pgfpathlineto{\pgfqpoint{4.855070in}{0.413320in}}%
\pgfpathlineto{\pgfqpoint{4.852404in}{0.413320in}}%
\pgfpathlineto{\pgfqpoint{4.849715in}{0.413320in}}%
\pgfpathlineto{\pgfqpoint{4.847127in}{0.413320in}}%
\pgfpathlineto{\pgfqpoint{4.844361in}{0.413320in}}%
\pgfpathlineto{\pgfqpoint{4.842380in}{0.413320in}}%
\pgfpathlineto{\pgfqpoint{4.839922in}{0.413320in}}%
\pgfpathlineto{\pgfqpoint{4.837992in}{0.413320in}}%
\pgfpathlineto{\pgfqpoint{4.833657in}{0.413320in}}%
\pgfpathlineto{\pgfqpoint{4.831045in}{0.413320in}}%
\pgfpathlineto{\pgfqpoint{4.828291in}{0.413320in}}%
\pgfpathlineto{\pgfqpoint{4.825619in}{0.413320in}}%
\pgfpathlineto{\pgfqpoint{4.822945in}{0.413320in}}%
\pgfpathlineto{\pgfqpoint{4.820265in}{0.413320in}}%
\pgfpathlineto{\pgfqpoint{4.817587in}{0.413320in}}%
\pgfpathlineto{\pgfqpoint{4.814907in}{0.413320in}}%
\pgfpathlineto{\pgfqpoint{4.812377in}{0.413320in}}%
\pgfpathlineto{\pgfqpoint{4.809538in}{0.413320in}}%
\pgfpathlineto{\pgfqpoint{4.807017in}{0.413320in}}%
\pgfpathlineto{\pgfqpoint{4.804193in}{0.413320in}}%
\pgfpathlineto{\pgfqpoint{4.801586in}{0.413320in}}%
\pgfpathlineto{\pgfqpoint{4.798830in}{0.413320in}}%
\pgfpathlineto{\pgfqpoint{4.796274in}{0.413320in}}%
\pgfpathlineto{\pgfqpoint{4.793512in}{0.413320in}}%
\pgfpathlineto{\pgfqpoint{4.790798in}{0.413320in}}%
\pgfpathlineto{\pgfqpoint{4.788116in}{0.413320in}}%
\pgfpathlineto{\pgfqpoint{4.785445in}{0.413320in}}%
\pgfpathlineto{\pgfqpoint{4.782872in}{0.413320in}}%
\pgfpathlineto{\pgfqpoint{4.780083in}{0.413320in}}%
\pgfpathlineto{\pgfqpoint{4.777535in}{0.413320in}}%
\pgfpathlineto{\pgfqpoint{4.774732in}{0.413320in}}%
\pgfpathlineto{\pgfqpoint{4.772198in}{0.413320in}}%
\pgfpathlineto{\pgfqpoint{4.769367in}{0.413320in}}%
\pgfpathlineto{\pgfqpoint{4.766783in}{0.413320in}}%
\pgfpathlineto{\pgfqpoint{4.764018in}{0.413320in}}%
\pgfpathlineto{\pgfqpoint{4.761337in}{0.413320in}}%
\pgfpathlineto{\pgfqpoint{4.758653in}{0.413320in}}%
\pgfpathlineto{\pgfqpoint{4.755983in}{0.413320in}}%
\pgfpathlineto{\pgfqpoint{4.753298in}{0.413320in}}%
\pgfpathlineto{\pgfqpoint{4.750627in}{0.413320in}}%
\pgfpathlineto{\pgfqpoint{4.748081in}{0.413320in}}%
\pgfpathlineto{\pgfqpoint{4.745256in}{0.413320in}}%
\pgfpathlineto{\pgfqpoint{4.742696in}{0.413320in}}%
\pgfpathlineto{\pgfqpoint{4.739912in}{0.413320in}}%
\pgfpathlineto{\pgfqpoint{4.737348in}{0.413320in}}%
\pgfpathlineto{\pgfqpoint{4.734552in}{0.413320in}}%
\pgfpathlineto{\pgfqpoint{4.731901in}{0.413320in}}%
\pgfpathlineto{\pgfqpoint{4.729233in}{0.413320in}}%
\pgfpathlineto{\pgfqpoint{4.726508in}{0.413320in}}%
\pgfpathlineto{\pgfqpoint{4.723873in}{0.413320in}}%
\pgfpathlineto{\pgfqpoint{4.721160in}{0.413320in}}%
\pgfpathlineto{\pgfqpoint{4.718486in}{0.413320in}}%
\pgfpathlineto{\pgfqpoint{4.715806in}{0.413320in}}%
\pgfpathlineto{\pgfqpoint{4.713275in}{0.413320in}}%
\pgfpathlineto{\pgfqpoint{4.710437in}{0.413320in}}%
\pgfpathlineto{\pgfqpoint{4.707824in}{0.413320in}}%
\pgfpathlineto{\pgfqpoint{4.705094in}{0.413320in}}%
\pgfpathlineto{\pgfqpoint{4.702517in}{0.413320in}}%
\pgfpathlineto{\pgfqpoint{4.699734in}{0.413320in}}%
\pgfpathlineto{\pgfqpoint{4.697170in}{0.413320in}}%
\pgfpathlineto{\pgfqpoint{4.694381in}{0.413320in}}%
\pgfpathlineto{\pgfqpoint{4.691694in}{0.413320in}}%
\pgfpathlineto{\pgfqpoint{4.689051in}{0.413320in}}%
\pgfpathlineto{\pgfqpoint{4.686337in}{0.413320in}}%
\pgfpathlineto{\pgfqpoint{4.683799in}{0.413320in}}%
\pgfpathlineto{\pgfqpoint{4.680988in}{0.413320in}}%
\pgfpathlineto{\pgfqpoint{4.678448in}{0.413320in}}%
\pgfpathlineto{\pgfqpoint{4.675619in}{0.413320in}}%
\pgfpathlineto{\pgfqpoint{4.673068in}{0.413320in}}%
\pgfpathlineto{\pgfqpoint{4.670261in}{0.413320in}}%
\pgfpathlineto{\pgfqpoint{4.667764in}{0.413320in}}%
\pgfpathlineto{\pgfqpoint{4.664923in}{0.413320in}}%
\pgfpathlineto{\pgfqpoint{4.662237in}{0.413320in}}%
\pgfpathlineto{\pgfqpoint{4.659590in}{0.413320in}}%
\pgfpathlineto{\pgfqpoint{4.656873in}{0.413320in}}%
\pgfpathlineto{\pgfqpoint{4.654203in}{0.413320in}}%
\pgfpathlineto{\pgfqpoint{4.651524in}{0.413320in}}%
\pgfpathlineto{\pgfqpoint{4.648922in}{0.413320in}}%
\pgfpathlineto{\pgfqpoint{4.646169in}{0.413320in}}%
\pgfpathlineto{\pgfqpoint{4.643628in}{0.413320in}}%
\pgfpathlineto{\pgfqpoint{4.640809in}{0.413320in}}%
\pgfpathlineto{\pgfqpoint{4.638204in}{0.413320in}}%
\pgfpathlineto{\pgfqpoint{4.635445in}{0.413320in}}%
\pgfpathlineto{\pgfqpoint{4.632902in}{0.413320in}}%
\pgfpathlineto{\pgfqpoint{4.630096in}{0.413320in}}%
\pgfpathlineto{\pgfqpoint{4.627411in}{0.413320in}}%
\pgfpathlineto{\pgfqpoint{4.624741in}{0.413320in}}%
\pgfpathlineto{\pgfqpoint{4.622056in}{0.413320in}}%
\pgfpathlineto{\pgfqpoint{4.619529in}{0.413320in}}%
\pgfpathlineto{\pgfqpoint{4.616702in}{0.413320in}}%
\pgfpathlineto{\pgfqpoint{4.614134in}{0.413320in}}%
\pgfpathlineto{\pgfqpoint{4.611350in}{0.413320in}}%
\pgfpathlineto{\pgfqpoint{4.608808in}{0.413320in}}%
\pgfpathlineto{\pgfqpoint{4.605990in}{0.413320in}}%
\pgfpathlineto{\pgfqpoint{4.603430in}{0.413320in}}%
\pgfpathlineto{\pgfqpoint{4.600633in}{0.413320in}}%
\pgfpathlineto{\pgfqpoint{4.597951in}{0.413320in}}%
\pgfpathlineto{\pgfqpoint{4.595281in}{0.413320in}}%
\pgfpathlineto{\pgfqpoint{4.592589in}{0.413320in}}%
\pgfpathlineto{\pgfqpoint{4.589920in}{0.413320in}}%
\pgfpathlineto{\pgfqpoint{4.587244in}{0.413320in}}%
\pgfpathlineto{\pgfqpoint{4.584672in}{0.413320in}}%
\pgfpathlineto{\pgfqpoint{4.581888in}{0.413320in}}%
\pgfpathlineto{\pgfqpoint{4.579305in}{0.413320in}}%
\pgfpathlineto{\pgfqpoint{4.576531in}{0.413320in}}%
\pgfpathlineto{\pgfqpoint{4.573947in}{0.413320in}}%
\pgfpathlineto{\pgfqpoint{4.571171in}{0.413320in}}%
\pgfpathlineto{\pgfqpoint{4.568612in}{0.413320in}}%
\pgfpathlineto{\pgfqpoint{4.565820in}{0.413320in}}%
\pgfpathlineto{\pgfqpoint{4.563125in}{0.413320in}}%
\pgfpathlineto{\pgfqpoint{4.560448in}{0.413320in}}%
\pgfpathlineto{\pgfqpoint{4.557777in}{0.413320in}}%
\pgfpathlineto{\pgfqpoint{4.555106in}{0.413320in}}%
\pgfpathlineto{\pgfqpoint{4.552425in}{0.413320in}}%
\pgfpathlineto{\pgfqpoint{4.549822in}{0.413320in}}%
\pgfpathlineto{\pgfqpoint{4.547064in}{0.413320in}}%
\pgfpathlineto{\pgfqpoint{4.544464in}{0.413320in}}%
\pgfpathlineto{\pgfqpoint{4.541711in}{0.413320in}}%
\pgfpathlineto{\pgfqpoint{4.539144in}{0.413320in}}%
\pgfpathlineto{\pgfqpoint{4.536400in}{0.413320in}}%
\pgfpathlineto{\pgfqpoint{4.533764in}{0.413320in}}%
\pgfpathlineto{\pgfqpoint{4.530990in}{0.413320in}}%
\pgfpathlineto{\pgfqpoint{4.528307in}{0.413320in}}%
\pgfpathlineto{\pgfqpoint{4.525640in}{0.413320in}}%
\pgfpathlineto{\pgfqpoint{4.522962in}{0.413320in}}%
\pgfpathlineto{\pgfqpoint{4.520345in}{0.413320in}}%
\pgfpathlineto{\pgfqpoint{4.517598in}{0.413320in}}%
\pgfpathlineto{\pgfqpoint{4.515080in}{0.413320in}}%
\pgfpathlineto{\pgfqpoint{4.512246in}{0.413320in}}%
\pgfpathlineto{\pgfqpoint{4.509643in}{0.413320in}}%
\pgfpathlineto{\pgfqpoint{4.506893in}{0.413320in}}%
\pgfpathlineto{\pgfqpoint{4.504305in}{0.413320in}}%
\pgfpathlineto{\pgfqpoint{4.501529in}{0.413320in}}%
\pgfpathlineto{\pgfqpoint{4.498850in}{0.413320in}}%
\pgfpathlineto{\pgfqpoint{4.496167in}{0.413320in}}%
\pgfpathlineto{\pgfqpoint{4.493492in}{0.413320in}}%
\pgfpathlineto{\pgfqpoint{4.490822in}{0.413320in}}%
\pgfpathlineto{\pgfqpoint{4.488130in}{0.413320in}}%
\pgfpathlineto{\pgfqpoint{4.485581in}{0.413320in}}%
\pgfpathlineto{\pgfqpoint{4.482778in}{0.413320in}}%
\pgfpathlineto{\pgfqpoint{4.480201in}{0.413320in}}%
\pgfpathlineto{\pgfqpoint{4.477430in}{0.413320in}}%
\pgfpathlineto{\pgfqpoint{4.474861in}{0.413320in}}%
\pgfpathlineto{\pgfqpoint{4.472059in}{0.413320in}}%
\pgfpathlineto{\pgfqpoint{4.469492in}{0.413320in}}%
\pgfpathlineto{\pgfqpoint{4.466717in}{0.413320in}}%
\pgfpathlineto{\pgfqpoint{4.464029in}{0.413320in}}%
\pgfpathlineto{\pgfqpoint{4.461367in}{0.413320in}}%
\pgfpathlineto{\pgfqpoint{4.458681in}{0.413320in}}%
\pgfpathlineto{\pgfqpoint{4.456138in}{0.413320in}}%
\pgfpathlineto{\pgfqpoint{4.453312in}{0.413320in}}%
\pgfpathlineto{\pgfqpoint{4.450767in}{0.413320in}}%
\pgfpathlineto{\pgfqpoint{4.447965in}{0.413320in}}%
\pgfpathlineto{\pgfqpoint{4.445423in}{0.413320in}}%
\pgfpathlineto{\pgfqpoint{4.442611in}{0.413320in}}%
\pgfpathlineto{\pgfqpoint{4.440041in}{0.413320in}}%
\pgfpathlineto{\pgfqpoint{4.437253in}{0.413320in}}%
\pgfpathlineto{\pgfqpoint{4.434569in}{0.413320in}}%
\pgfpathlineto{\pgfqpoint{4.431901in}{0.413320in}}%
\pgfpathlineto{\pgfqpoint{4.429220in}{0.413320in}}%
\pgfpathlineto{\pgfqpoint{4.426534in}{0.413320in}}%
\pgfpathlineto{\pgfqpoint{4.423863in}{0.413320in}}%
\pgfpathlineto{\pgfqpoint{4.421292in}{0.413320in}}%
\pgfpathlineto{\pgfqpoint{4.418506in}{0.413320in}}%
\pgfpathlineto{\pgfqpoint{4.415932in}{0.413320in}}%
\pgfpathlineto{\pgfqpoint{4.413149in}{0.413320in}}%
\pgfpathlineto{\pgfqpoint{4.410587in}{0.413320in}}%
\pgfpathlineto{\pgfqpoint{4.407788in}{0.413320in}}%
\pgfpathlineto{\pgfqpoint{4.405234in}{0.413320in}}%
\pgfpathlineto{\pgfqpoint{4.402468in}{0.413320in}}%
\pgfpathlineto{\pgfqpoint{4.399745in}{0.413320in}}%
\pgfpathlineto{\pgfqpoint{4.397076in}{0.413320in}}%
\pgfpathlineto{\pgfqpoint{4.394400in}{0.413320in}}%
\pgfpathlineto{\pgfqpoint{4.391721in}{0.413320in}}%
\pgfpathlineto{\pgfqpoint{4.389044in}{0.413320in}}%
\pgfpathlineto{\pgfqpoint{4.386431in}{0.413320in}}%
\pgfpathlineto{\pgfqpoint{4.383674in}{0.413320in}}%
\pgfpathlineto{\pgfqpoint{4.381097in}{0.413320in}}%
\pgfpathlineto{\pgfqpoint{4.378329in}{0.413320in}}%
\pgfpathlineto{\pgfqpoint{4.375761in}{0.413320in}}%
\pgfpathlineto{\pgfqpoint{4.372976in}{0.413320in}}%
\pgfpathlineto{\pgfqpoint{4.370437in}{0.413320in}}%
\pgfpathlineto{\pgfqpoint{4.367646in}{0.413320in}}%
\pgfpathlineto{\pgfqpoint{4.364936in}{0.413320in}}%
\pgfpathlineto{\pgfqpoint{4.362270in}{0.413320in}}%
\pgfpathlineto{\pgfqpoint{4.359582in}{0.413320in}}%
\pgfpathlineto{\pgfqpoint{4.357014in}{0.413320in}}%
\pgfpathlineto{\pgfqpoint{4.354224in}{0.413320in}}%
\pgfpathlineto{\pgfqpoint{4.351645in}{0.413320in}}%
\pgfpathlineto{\pgfqpoint{4.348868in}{0.413320in}}%
\pgfpathlineto{\pgfqpoint{4.346263in}{0.413320in}}%
\pgfpathlineto{\pgfqpoint{4.343510in}{0.413320in}}%
\pgfpathlineto{\pgfqpoint{4.340976in}{0.413320in}}%
\pgfpathlineto{\pgfqpoint{4.338154in}{0.413320in}}%
\pgfpathlineto{\pgfqpoint{4.335463in}{0.413320in}}%
\pgfpathlineto{\pgfqpoint{4.332796in}{0.413320in}}%
\pgfpathlineto{\pgfqpoint{4.330118in}{0.413320in}}%
\pgfpathlineto{\pgfqpoint{4.327440in}{0.413320in}}%
\pgfpathlineto{\pgfqpoint{4.324760in}{0.413320in}}%
\pgfpathlineto{\pgfqpoint{4.322181in}{0.413320in}}%
\pgfpathlineto{\pgfqpoint{4.319405in}{0.413320in}}%
\pgfpathlineto{\pgfqpoint{4.316856in}{0.413320in}}%
\pgfpathlineto{\pgfqpoint{4.314032in}{0.413320in}}%
\pgfpathlineto{\pgfqpoint{4.311494in}{0.413320in}}%
\pgfpathlineto{\pgfqpoint{4.308691in}{0.413320in}}%
\pgfpathlineto{\pgfqpoint{4.306118in}{0.413320in}}%
\pgfpathlineto{\pgfqpoint{4.303357in}{0.413320in}}%
\pgfpathlineto{\pgfqpoint{4.300656in}{0.413320in}}%
\pgfpathlineto{\pgfqpoint{4.297977in}{0.413320in}}%
\pgfpathlineto{\pgfqpoint{4.295299in}{0.413320in}}%
\pgfpathlineto{\pgfqpoint{4.292786in}{0.413320in}}%
\pgfpathlineto{\pgfqpoint{4.289936in}{0.413320in}}%
\pgfpathlineto{\pgfqpoint{4.287399in}{0.413320in}}%
\pgfpathlineto{\pgfqpoint{4.284586in}{0.413320in}}%
\pgfpathlineto{\pgfqpoint{4.282000in}{0.413320in}}%
\pgfpathlineto{\pgfqpoint{4.279212in}{0.413320in}}%
\pgfpathlineto{\pgfqpoint{4.276635in}{0.413320in}}%
\pgfpathlineto{\pgfqpoint{4.273874in}{0.413320in}}%
\pgfpathlineto{\pgfqpoint{4.271187in}{0.413320in}}%
\pgfpathlineto{\pgfqpoint{4.268590in}{0.413320in}}%
\pgfpathlineto{\pgfqpoint{4.265824in}{0.413320in}}%
\pgfpathlineto{\pgfqpoint{4.263157in}{0.413320in}}%
\pgfpathlineto{\pgfqpoint{4.260477in}{0.413320in}}%
\pgfpathlineto{\pgfqpoint{4.257958in}{0.413320in}}%
\pgfpathlineto{\pgfqpoint{4.255120in}{0.413320in}}%
\pgfpathlineto{\pgfqpoint{4.252581in}{0.413320in}}%
\pgfpathlineto{\pgfqpoint{4.249767in}{0.413320in}}%
\pgfpathlineto{\pgfqpoint{4.247225in}{0.413320in}}%
\pgfpathlineto{\pgfqpoint{4.244394in}{0.413320in}}%
\pgfpathlineto{\pgfqpoint{4.241900in}{0.413320in}}%
\pgfpathlineto{\pgfqpoint{4.239084in}{0.413320in}}%
\pgfpathlineto{\pgfqpoint{4.236375in}{0.413320in}}%
\pgfpathlineto{\pgfqpoint{4.233691in}{0.413320in}}%
\pgfpathlineto{\pgfqpoint{4.231013in}{0.413320in}}%
\pgfpathlineto{\pgfqpoint{4.228331in}{0.413320in}}%
\pgfpathlineto{\pgfqpoint{4.225654in}{0.413320in}}%
\pgfpathlineto{\pgfqpoint{4.223082in}{0.413320in}}%
\pgfpathlineto{\pgfqpoint{4.220304in}{0.413320in}}%
\pgfpathlineto{\pgfqpoint{4.217694in}{0.413320in}}%
\pgfpathlineto{\pgfqpoint{4.214948in}{0.413320in}}%
\pgfpathlineto{\pgfqpoint{4.212383in}{0.413320in}}%
\pgfpathlineto{\pgfqpoint{4.209597in}{0.413320in}}%
\pgfpathlineto{\pgfqpoint{4.207076in}{0.413320in}}%
\pgfpathlineto{\pgfqpoint{4.204240in}{0.413320in}}%
\pgfpathlineto{\pgfqpoint{4.201542in}{0.413320in}}%
\pgfpathlineto{\pgfqpoint{4.198878in}{0.413320in}}%
\pgfpathlineto{\pgfqpoint{4.196186in}{0.413320in}}%
\pgfpathlineto{\pgfqpoint{4.193638in}{0.413320in}}%
\pgfpathlineto{\pgfqpoint{4.190842in}{0.413320in}}%
\pgfpathlineto{\pgfqpoint{4.188318in}{0.413320in}}%
\pgfpathlineto{\pgfqpoint{4.185481in}{0.413320in}}%
\pgfpathlineto{\pgfqpoint{4.182899in}{0.413320in}}%
\pgfpathlineto{\pgfqpoint{4.180129in}{0.413320in}}%
\pgfpathlineto{\pgfqpoint{4.177593in}{0.413320in}}%
\pgfpathlineto{\pgfqpoint{4.174770in}{0.413320in}}%
\pgfpathlineto{\pgfqpoint{4.172093in}{0.413320in}}%
\pgfpathlineto{\pgfqpoint{4.169415in}{0.413320in}}%
\pgfpathlineto{\pgfqpoint{4.166737in}{0.413320in}}%
\pgfpathlineto{\pgfqpoint{4.164059in}{0.413320in}}%
\pgfpathlineto{\pgfqpoint{4.161380in}{0.413320in}}%
\pgfpathlineto{\pgfqpoint{4.158806in}{0.413320in}}%
\pgfpathlineto{\pgfqpoint{4.156016in}{0.413320in}}%
\pgfpathlineto{\pgfqpoint{4.153423in}{0.413320in}}%
\pgfpathlineto{\pgfqpoint{4.150665in}{0.413320in}}%
\pgfpathlineto{\pgfqpoint{4.148082in}{0.413320in}}%
\pgfpathlineto{\pgfqpoint{4.145310in}{0.413320in}}%
\pgfpathlineto{\pgfqpoint{4.142713in}{0.413320in}}%
\pgfpathlineto{\pgfqpoint{4.139963in}{0.413320in}}%
\pgfpathlineto{\pgfqpoint{4.137272in}{0.413320in}}%
\pgfpathlineto{\pgfqpoint{4.134615in}{0.413320in}}%
\pgfpathlineto{\pgfqpoint{4.131920in}{0.413320in}}%
\pgfpathlineto{\pgfqpoint{4.129349in}{0.413320in}}%
\pgfpathlineto{\pgfqpoint{4.126553in}{0.413320in}}%
\pgfpathlineto{\pgfqpoint{4.124019in}{0.413320in}}%
\pgfpathlineto{\pgfqpoint{4.121205in}{0.413320in}}%
\pgfpathlineto{\pgfqpoint{4.118554in}{0.413320in}}%
\pgfpathlineto{\pgfqpoint{4.115844in}{0.413320in}}%
\pgfpathlineto{\pgfqpoint{4.113252in}{0.413320in}}%
\pgfpathlineto{\pgfqpoint{4.110488in}{0.413320in}}%
\pgfpathlineto{\pgfqpoint{4.107814in}{0.413320in}}%
\pgfpathlineto{\pgfqpoint{4.105185in}{0.413320in}}%
\pgfpathlineto{\pgfqpoint{4.102456in}{0.413320in}}%
\pgfpathlineto{\pgfqpoint{4.099777in}{0.413320in}}%
\pgfpathlineto{\pgfqpoint{4.097092in}{0.413320in}}%
\pgfpathlineto{\pgfqpoint{4.094527in}{0.413320in}}%
\pgfpathlineto{\pgfqpoint{4.091729in}{0.413320in}}%
\pgfpathlineto{\pgfqpoint{4.089159in}{0.413320in}}%
\pgfpathlineto{\pgfqpoint{4.086385in}{0.413320in}}%
\pgfpathlineto{\pgfqpoint{4.083870in}{0.413320in}}%
\pgfpathlineto{\pgfqpoint{4.081018in}{0.413320in}}%
\pgfpathlineto{\pgfqpoint{4.078471in}{0.413320in}}%
\pgfpathlineto{\pgfqpoint{4.075705in}{0.413320in}}%
\pgfpathlineto{\pgfqpoint{4.072985in}{0.413320in}}%
\pgfpathlineto{\pgfqpoint{4.070313in}{0.413320in}}%
\pgfpathlineto{\pgfqpoint{4.067636in}{0.413320in}}%
\pgfpathlineto{\pgfqpoint{4.064957in}{0.413320in}}%
\pgfpathlineto{\pgfqpoint{4.062266in}{0.413320in}}%
\pgfpathlineto{\pgfqpoint{4.059702in}{0.413320in}}%
\pgfpathlineto{\pgfqpoint{4.056911in}{0.413320in}}%
\pgfpathlineto{\pgfqpoint{4.054326in}{0.413320in}}%
\pgfpathlineto{\pgfqpoint{4.051557in}{0.413320in}}%
\pgfpathlineto{\pgfqpoint{4.049006in}{0.413320in}}%
\pgfpathlineto{\pgfqpoint{4.046210in}{0.413320in}}%
\pgfpathlineto{\pgfqpoint{4.043667in}{0.413320in}}%
\pgfpathlineto{\pgfqpoint{4.040852in}{0.413320in}}%
\pgfpathlineto{\pgfqpoint{4.038174in}{0.413320in}}%
\pgfpathlineto{\pgfqpoint{4.035492in}{0.413320in}}%
\pgfpathlineto{\pgfqpoint{4.032817in}{0.413320in}}%
\pgfpathlineto{\pgfqpoint{4.030229in}{0.413320in}}%
\pgfpathlineto{\pgfqpoint{4.027447in}{0.413320in}}%
\pgfpathlineto{\pgfqpoint{4.024868in}{0.413320in}}%
\pgfpathlineto{\pgfqpoint{4.022097in}{0.413320in}}%
\pgfpathlineto{\pgfqpoint{4.019518in}{0.413320in}}%
\pgfpathlineto{\pgfqpoint{4.016744in}{0.413320in}}%
\pgfpathlineto{\pgfqpoint{4.014186in}{0.413320in}}%
\pgfpathlineto{\pgfqpoint{4.011394in}{0.413320in}}%
\pgfpathlineto{\pgfqpoint{4.008699in}{0.413320in}}%
\pgfpathlineto{\pgfqpoint{4.006034in}{0.413320in}}%
\pgfpathlineto{\pgfqpoint{4.003348in}{0.413320in}}%
\pgfpathlineto{\pgfqpoint{4.000674in}{0.413320in}}%
\pgfpathlineto{\pgfqpoint{3.997990in}{0.413320in}}%
\pgfpathlineto{\pgfqpoint{3.995417in}{0.413320in}}%
\pgfpathlineto{\pgfqpoint{3.992642in}{0.413320in}}%
\pgfpathlineto{\pgfqpoint{3.990055in}{0.413320in}}%
\pgfpathlineto{\pgfqpoint{3.987270in}{0.413320in}}%
\pgfpathlineto{\pgfqpoint{3.984714in}{0.413320in}}%
\pgfpathlineto{\pgfqpoint{3.981929in}{0.413320in}}%
\pgfpathlineto{\pgfqpoint{3.979389in}{0.413320in}}%
\pgfpathlineto{\pgfqpoint{3.976563in}{0.413320in}}%
\pgfpathlineto{\pgfqpoint{3.973885in}{0.413320in}}%
\pgfpathlineto{\pgfqpoint{3.971250in}{0.413320in}}%
\pgfpathlineto{\pgfqpoint{3.968523in}{0.413320in}}%
\pgfpathlineto{\pgfqpoint{3.966013in}{0.413320in}}%
\pgfpathlineto{\pgfqpoint{3.963176in}{0.413320in}}%
\pgfpathlineto{\pgfqpoint{3.960635in}{0.413320in}}%
\pgfpathlineto{\pgfqpoint{3.957823in}{0.413320in}}%
\pgfpathlineto{\pgfqpoint{3.955211in}{0.413320in}}%
\pgfpathlineto{\pgfqpoint{3.952464in}{0.413320in}}%
\pgfpathlineto{\pgfqpoint{3.949894in}{0.413320in}}%
\pgfpathlineto{\pgfqpoint{3.947101in}{0.413320in}}%
\pgfpathlineto{\pgfqpoint{3.944431in}{0.413320in}}%
\pgfpathlineto{\pgfqpoint{3.941778in}{0.413320in}}%
\pgfpathlineto{\pgfqpoint{3.939075in}{0.413320in}}%
\pgfpathlineto{\pgfqpoint{3.936395in}{0.413320in}}%
\pgfpathlineto{\pgfqpoint{3.933714in}{0.413320in}}%
\pgfpathlineto{\pgfqpoint{3.931202in}{0.413320in}}%
\pgfpathlineto{\pgfqpoint{3.928347in}{0.413320in}}%
\pgfpathlineto{\pgfqpoint{3.925778in}{0.413320in}}%
\pgfpathlineto{\pgfqpoint{3.923005in}{0.413320in}}%
\pgfpathlineto{\pgfqpoint{3.920412in}{0.413320in}}%
\pgfpathlineto{\pgfqpoint{3.917646in}{0.413320in}}%
\pgfpathlineto{\pgfqpoint{3.915107in}{0.413320in}}%
\pgfpathlineto{\pgfqpoint{3.912296in}{0.413320in}}%
\pgfpathlineto{\pgfqpoint{3.909602in}{0.413320in}}%
\pgfpathlineto{\pgfqpoint{3.906918in}{0.413320in}}%
\pgfpathlineto{\pgfqpoint{3.904252in}{0.413320in}}%
\pgfpathlineto{\pgfqpoint{3.901573in}{0.413320in}}%
\pgfpathlineto{\pgfqpoint{3.898891in}{0.413320in}}%
\pgfpathlineto{\pgfqpoint{3.896345in}{0.413320in}}%
\pgfpathlineto{\pgfqpoint{3.893541in}{0.413320in}}%
\pgfpathlineto{\pgfqpoint{3.890926in}{0.413320in}}%
\pgfpathlineto{\pgfqpoint{3.888188in}{0.413320in}}%
\pgfpathlineto{\pgfqpoint{3.885621in}{0.413320in}}%
\pgfpathlineto{\pgfqpoint{3.882850in}{0.413320in}}%
\pgfpathlineto{\pgfqpoint{3.880237in}{0.413320in}}%
\pgfpathlineto{\pgfqpoint{3.877466in}{0.413320in}}%
\pgfpathlineto{\pgfqpoint{3.874790in}{0.413320in}}%
\pgfpathlineto{\pgfqpoint{3.872114in}{0.413320in}}%
\pgfpathlineto{\pgfqpoint{3.869435in}{0.413320in}}%
\pgfpathlineto{\pgfqpoint{3.866815in}{0.413320in}}%
\pgfpathlineto{\pgfqpoint{3.864073in}{0.413320in}}%
\pgfpathlineto{\pgfqpoint{3.861561in}{0.413320in}}%
\pgfpathlineto{\pgfqpoint{3.858720in}{0.413320in}}%
\pgfpathlineto{\pgfqpoint{3.856100in}{0.413320in}}%
\pgfpathlineto{\pgfqpoint{3.853358in}{0.413320in}}%
\pgfpathlineto{\pgfqpoint{3.850814in}{0.413320in}}%
\pgfpathlineto{\pgfqpoint{3.848005in}{0.413320in}}%
\pgfpathlineto{\pgfqpoint{3.845329in}{0.413320in}}%
\pgfpathlineto{\pgfqpoint{3.842641in}{0.413320in}}%
\pgfpathlineto{\pgfqpoint{3.839960in}{0.413320in}}%
\pgfpathlineto{\pgfqpoint{3.837286in}{0.413320in}}%
\pgfpathlineto{\pgfqpoint{3.834616in}{0.413320in}}%
\pgfpathlineto{\pgfqpoint{3.832053in}{0.413320in}}%
\pgfpathlineto{\pgfqpoint{3.829252in}{0.413320in}}%
\pgfpathlineto{\pgfqpoint{3.826679in}{0.413320in}}%
\pgfpathlineto{\pgfqpoint{3.823903in}{0.413320in}}%
\pgfpathlineto{\pgfqpoint{3.821315in}{0.413320in}}%
\pgfpathlineto{\pgfqpoint{3.818546in}{0.413320in}}%
\pgfpathlineto{\pgfqpoint{3.815983in}{0.413320in}}%
\pgfpathlineto{\pgfqpoint{3.813172in}{0.413320in}}%
\pgfpathlineto{\pgfqpoint{3.810510in}{0.413320in}}%
\pgfpathlineto{\pgfqpoint{3.807832in}{0.413320in}}%
\pgfpathlineto{\pgfqpoint{3.805145in}{0.413320in}}%
\pgfpathlineto{\pgfqpoint{3.802569in}{0.413320in}}%
\pgfpathlineto{\pgfqpoint{3.799797in}{0.413320in}}%
\pgfpathlineto{\pgfqpoint{3.797265in}{0.413320in}}%
\pgfpathlineto{\pgfqpoint{3.794435in}{0.413320in}}%
\pgfpathlineto{\pgfqpoint{3.791897in}{0.413320in}}%
\pgfpathlineto{\pgfqpoint{3.789084in}{0.413320in}}%
\pgfpathlineto{\pgfqpoint{3.786504in}{0.413320in}}%
\pgfpathlineto{\pgfqpoint{3.783725in}{0.413320in}}%
\pgfpathlineto{\pgfqpoint{3.781046in}{0.413320in}}%
\pgfpathlineto{\pgfqpoint{3.778370in}{0.413320in}}%
\pgfpathlineto{\pgfqpoint{3.775691in}{0.413320in}}%
\pgfpathlineto{\pgfqpoint{3.773014in}{0.413320in}}%
\pgfpathlineto{\pgfqpoint{3.770323in}{0.413320in}}%
\pgfpathlineto{\pgfqpoint{3.767782in}{0.413320in}}%
\pgfpathlineto{\pgfqpoint{3.764966in}{0.413320in}}%
\pgfpathlineto{\pgfqpoint{3.762389in}{0.413320in}}%
\pgfpathlineto{\pgfqpoint{3.759622in}{0.413320in}}%
\pgfpathlineto{\pgfqpoint{3.757065in}{0.413320in}}%
\pgfpathlineto{\pgfqpoint{3.754265in}{0.413320in}}%
\pgfpathlineto{\pgfqpoint{3.751728in}{0.413320in}}%
\pgfpathlineto{\pgfqpoint{3.748903in}{0.413320in}}%
\pgfpathlineto{\pgfqpoint{3.746229in}{0.413320in}}%
\pgfpathlineto{\pgfqpoint{3.743548in}{0.413320in}}%
\pgfpathlineto{\pgfqpoint{3.740874in}{0.413320in}}%
\pgfpathlineto{\pgfqpoint{3.738194in}{0.413320in}}%
\pgfpathlineto{\pgfqpoint{3.735509in}{0.413320in}}%
\pgfpathlineto{\pgfqpoint{3.732950in}{0.413320in}}%
\pgfpathlineto{\pgfqpoint{3.730158in}{0.413320in}}%
\pgfpathlineto{\pgfqpoint{3.727581in}{0.413320in}}%
\pgfpathlineto{\pgfqpoint{3.724804in}{0.413320in}}%
\pgfpathlineto{\pgfqpoint{3.722228in}{0.413320in}}%
\pgfpathlineto{\pgfqpoint{3.719446in}{0.413320in}}%
\pgfpathlineto{\pgfqpoint{3.716875in}{0.413320in}}%
\pgfpathlineto{\pgfqpoint{3.714086in}{0.413320in}}%
\pgfpathlineto{\pgfqpoint{3.711410in}{0.413320in}}%
\pgfpathlineto{\pgfqpoint{3.708729in}{0.413320in}}%
\pgfpathlineto{\pgfqpoint{3.706053in}{0.413320in}}%
\pgfpathlineto{\pgfqpoint{3.703460in}{0.413320in}}%
\pgfpathlineto{\pgfqpoint{3.700684in}{0.413320in}}%
\pgfpathlineto{\pgfqpoint{3.698125in}{0.413320in}}%
\pgfpathlineto{\pgfqpoint{3.695331in}{0.413320in}}%
\pgfpathlineto{\pgfqpoint{3.692765in}{0.413320in}}%
\pgfpathlineto{\pgfqpoint{3.689983in}{0.413320in}}%
\pgfpathlineto{\pgfqpoint{3.687442in}{0.413320in}}%
\pgfpathlineto{\pgfqpoint{3.684620in}{0.413320in}}%
\pgfpathlineto{\pgfqpoint{3.681948in}{0.413320in}}%
\pgfpathlineto{\pgfqpoint{3.679273in}{0.413320in}}%
\pgfpathlineto{\pgfqpoint{3.676591in}{0.413320in}}%
\pgfpathlineto{\pgfqpoint{3.673911in}{0.413320in}}%
\pgfpathlineto{\pgfqpoint{3.671232in}{0.413320in}}%
\pgfpathlineto{\pgfqpoint{3.668665in}{0.413320in}}%
\pgfpathlineto{\pgfqpoint{3.665864in}{0.413320in}}%
\pgfpathlineto{\pgfqpoint{3.663276in}{0.413320in}}%
\pgfpathlineto{\pgfqpoint{3.660515in}{0.413320in}}%
\pgfpathlineto{\pgfqpoint{3.657917in}{0.413320in}}%
\pgfpathlineto{\pgfqpoint{3.655165in}{0.413320in}}%
\pgfpathlineto{\pgfqpoint{3.652628in}{0.413320in}}%
\pgfpathlineto{\pgfqpoint{3.649837in}{0.413320in}}%
\pgfpathlineto{\pgfqpoint{3.647130in}{0.413320in}}%
\pgfpathlineto{\pgfqpoint{3.644452in}{0.413320in}}%
\pgfpathlineto{\pgfqpoint{3.641773in}{0.413320in}}%
\pgfpathlineto{\pgfqpoint{3.639207in}{0.413320in}}%
\pgfpathlineto{\pgfqpoint{3.636413in}{0.413320in}}%
\pgfpathlineto{\pgfqpoint{3.633858in}{0.413320in}}%
\pgfpathlineto{\pgfqpoint{3.631058in}{0.413320in}}%
\pgfpathlineto{\pgfqpoint{3.628460in}{0.413320in}}%
\pgfpathlineto{\pgfqpoint{3.625689in}{0.413320in}}%
\pgfpathlineto{\pgfqpoint{3.623165in}{0.413320in}}%
\pgfpathlineto{\pgfqpoint{3.620345in}{0.413320in}}%
\pgfpathlineto{\pgfqpoint{3.617667in}{0.413320in}}%
\pgfpathlineto{\pgfqpoint{3.614982in}{0.413320in}}%
\pgfpathlineto{\pgfqpoint{3.612311in}{0.413320in}}%
\pgfpathlineto{\pgfqpoint{3.609632in}{0.413320in}}%
\pgfpathlineto{\pgfqpoint{3.606951in}{0.413320in}}%
\pgfpathlineto{\pgfqpoint{3.604387in}{0.413320in}}%
\pgfpathlineto{\pgfqpoint{3.601590in}{0.413320in}}%
\pgfpathlineto{\pgfqpoint{3.598998in}{0.413320in}}%
\pgfpathlineto{\pgfqpoint{3.596240in}{0.413320in}}%
\pgfpathlineto{\pgfqpoint{3.593620in}{0.413320in}}%
\pgfpathlineto{\pgfqpoint{3.590883in}{0.413320in}}%
\pgfpathlineto{\pgfqpoint{3.588258in}{0.413320in}}%
\pgfpathlineto{\pgfqpoint{3.585532in}{0.413320in}}%
\pgfpathlineto{\pgfqpoint{3.582851in}{0.413320in}}%
\pgfpathlineto{\pgfqpoint{3.580191in}{0.413320in}}%
\pgfpathlineto{\pgfqpoint{3.577487in}{0.413320in}}%
\pgfpathlineto{\pgfqpoint{3.574814in}{0.413320in}}%
\pgfpathlineto{\pgfqpoint{3.572126in}{0.413320in}}%
\pgfpathlineto{\pgfqpoint{3.569584in}{0.413320in}}%
\pgfpathlineto{\pgfqpoint{3.566774in}{0.413320in}}%
\pgfpathlineto{\pgfqpoint{3.564188in}{0.413320in}}%
\pgfpathlineto{\pgfqpoint{3.561420in}{0.413320in}}%
\pgfpathlineto{\pgfqpoint{3.558853in}{0.413320in}}%
\pgfpathlineto{\pgfqpoint{3.556061in}{0.413320in}}%
\pgfpathlineto{\pgfqpoint{3.553498in}{0.413320in}}%
\pgfpathlineto{\pgfqpoint{3.550713in}{0.413320in}}%
\pgfpathlineto{\pgfqpoint{3.548029in}{0.413320in}}%
\pgfpathlineto{\pgfqpoint{3.545349in}{0.413320in}}%
\pgfpathlineto{\pgfqpoint{3.542656in}{0.413320in}}%
\pgfpathlineto{\pgfqpoint{3.540093in}{0.413320in}}%
\pgfpathlineto{\pgfqpoint{3.537309in}{0.413320in}}%
\pgfpathlineto{\pgfqpoint{3.534783in}{0.413320in}}%
\pgfpathlineto{\pgfqpoint{3.531955in}{0.413320in}}%
\pgfpathlineto{\pgfqpoint{3.529327in}{0.413320in}}%
\pgfpathlineto{\pgfqpoint{3.526601in}{0.413320in}}%
\pgfpathlineto{\pgfqpoint{3.524041in}{0.413320in}}%
\pgfpathlineto{\pgfqpoint{3.521244in}{0.413320in}}%
\pgfpathlineto{\pgfqpoint{3.518565in}{0.413320in}}%
\pgfpathlineto{\pgfqpoint{3.515884in}{0.413320in}}%
\pgfpathlineto{\pgfqpoint{3.513209in}{0.413320in}}%
\pgfpathlineto{\pgfqpoint{3.510533in}{0.413320in}}%
\pgfpathlineto{\pgfqpoint{3.507840in}{0.413320in}}%
\pgfpathlineto{\pgfqpoint{3.505262in}{0.413320in}}%
\pgfpathlineto{\pgfqpoint{3.502488in}{0.413320in}}%
\pgfpathlineto{\pgfqpoint{3.499909in}{0.413320in}}%
\pgfpathlineto{\pgfqpoint{3.497139in}{0.413320in}}%
\pgfpathlineto{\pgfqpoint{3.494581in}{0.413320in}}%
\pgfpathlineto{\pgfqpoint{3.491783in}{0.413320in}}%
\pgfpathlineto{\pgfqpoint{3.489223in}{0.413320in}}%
\pgfpathlineto{\pgfqpoint{3.486442in}{0.413320in}}%
\pgfpathlineto{\pgfqpoint{3.483744in}{0.413320in}}%
\pgfpathlineto{\pgfqpoint{3.481072in}{0.413320in}}%
\pgfpathlineto{\pgfqpoint{3.478378in}{0.413320in}}%
\pgfpathlineto{\pgfqpoint{3.475821in}{0.413320in}}%
\pgfpathlineto{\pgfqpoint{3.473021in}{0.413320in}}%
\pgfpathlineto{\pgfqpoint{3.470466in}{0.413320in}}%
\pgfpathlineto{\pgfqpoint{3.467678in}{0.413320in}}%
\pgfpathlineto{\pgfqpoint{3.465072in}{0.413320in}}%
\pgfpathlineto{\pgfqpoint{3.462321in}{0.413320in}}%
\pgfpathlineto{\pgfqpoint{3.459695in}{0.413320in}}%
\pgfpathlineto{\pgfqpoint{3.456960in}{0.413320in}}%
\pgfpathlineto{\pgfqpoint{3.454285in}{0.413320in}}%
\pgfpathlineto{\pgfqpoint{3.451597in}{0.413320in}}%
\pgfpathlineto{\pgfqpoint{3.448926in}{0.413320in}}%
\pgfpathlineto{\pgfqpoint{3.446257in}{0.413320in}}%
\pgfpathlineto{\pgfqpoint{3.443574in}{0.413320in}}%
\pgfpathlineto{\pgfqpoint{3.440996in}{0.413320in}}%
\pgfpathlineto{\pgfqpoint{3.438210in}{0.413320in}}%
\pgfpathlineto{\pgfqpoint{3.435635in}{0.413320in}}%
\pgfpathlineto{\pgfqpoint{3.432851in}{0.413320in}}%
\pgfpathlineto{\pgfqpoint{3.430313in}{0.413320in}}%
\pgfpathlineto{\pgfqpoint{3.427501in}{0.413320in}}%
\pgfpathlineto{\pgfqpoint{3.424887in}{0.413320in}}%
\pgfpathlineto{\pgfqpoint{3.422142in}{0.413320in}}%
\pgfpathlineto{\pgfqpoint{3.419455in}{0.413320in}}%
\pgfpathlineto{\pgfqpoint{3.416780in}{0.413320in}}%
\pgfpathlineto{\pgfqpoint{3.414109in}{0.413320in}}%
\pgfpathlineto{\pgfqpoint{3.411431in}{0.413320in}}%
\pgfpathlineto{\pgfqpoint{3.408752in}{0.413320in}}%
\pgfpathlineto{\pgfqpoint{3.406202in}{0.413320in}}%
\pgfpathlineto{\pgfqpoint{3.403394in}{0.413320in}}%
\pgfpathlineto{\pgfqpoint{3.400783in}{0.413320in}}%
\pgfpathlineto{\pgfqpoint{3.398037in}{0.413320in}}%
\pgfpathlineto{\pgfqpoint{3.395461in}{0.413320in}}%
\pgfpathlineto{\pgfqpoint{3.392681in}{0.413320in}}%
\pgfpathlineto{\pgfqpoint{3.390102in}{0.413320in}}%
\pgfpathlineto{\pgfqpoint{3.387309in}{0.413320in}}%
\pgfpathlineto{\pgfqpoint{3.384647in}{0.413320in}}%
\pgfpathlineto{\pgfqpoint{3.381959in}{0.413320in}}%
\pgfpathlineto{\pgfqpoint{3.379290in}{0.413320in}}%
\pgfpathlineto{\pgfqpoint{3.376735in}{0.413320in}}%
\pgfpathlineto{\pgfqpoint{3.373921in}{0.413320in}}%
\pgfpathlineto{\pgfqpoint{3.371357in}{0.413320in}}%
\pgfpathlineto{\pgfqpoint{3.368577in}{0.413320in}}%
\pgfpathlineto{\pgfqpoint{3.365996in}{0.413320in}}%
\pgfpathlineto{\pgfqpoint{3.363221in}{0.413320in}}%
\pgfpathlineto{\pgfqpoint{3.360620in}{0.413320in}}%
\pgfpathlineto{\pgfqpoint{3.357862in}{0.413320in}}%
\pgfpathlineto{\pgfqpoint{3.355177in}{0.413320in}}%
\pgfpathlineto{\pgfqpoint{3.352505in}{0.413320in}}%
\pgfpathlineto{\pgfqpoint{3.349828in}{0.413320in}}%
\pgfpathlineto{\pgfqpoint{3.347139in}{0.413320in}}%
\pgfpathlineto{\pgfqpoint{3.344468in}{0.413320in}}%
\pgfpathlineto{\pgfqpoint{3.341893in}{0.413320in}}%
\pgfpathlineto{\pgfqpoint{3.339101in}{0.413320in}}%
\pgfpathlineto{\pgfqpoint{3.336541in}{0.413320in}}%
\pgfpathlineto{\pgfqpoint{3.333758in}{0.413320in}}%
\pgfpathlineto{\pgfqpoint{3.331183in}{0.413320in}}%
\pgfpathlineto{\pgfqpoint{3.328401in}{0.413320in}}%
\pgfpathlineto{\pgfqpoint{3.325860in}{0.413320in}}%
\pgfpathlineto{\pgfqpoint{3.323049in}{0.413320in}}%
\pgfpathlineto{\pgfqpoint{3.320366in}{0.413320in}}%
\pgfpathlineto{\pgfqpoint{3.317688in}{0.413320in}}%
\pgfpathlineto{\pgfqpoint{3.315008in}{0.413320in}}%
\pgfpathlineto{\pgfqpoint{3.312480in}{0.413320in}}%
\pgfpathlineto{\pgfqpoint{3.309652in}{0.413320in}}%
\pgfpathlineto{\pgfqpoint{3.307104in}{0.413320in}}%
\pgfpathlineto{\pgfqpoint{3.304295in}{0.413320in}}%
\pgfpathlineto{\pgfqpoint{3.301719in}{0.413320in}}%
\pgfpathlineto{\pgfqpoint{3.298937in}{0.413320in}}%
\pgfpathlineto{\pgfqpoint{3.296376in}{0.413320in}}%
\pgfpathlineto{\pgfqpoint{3.293574in}{0.413320in}}%
\pgfpathlineto{\pgfqpoint{3.290890in}{0.413320in}}%
\pgfpathlineto{\pgfqpoint{3.288225in}{0.413320in}}%
\pgfpathlineto{\pgfqpoint{3.285534in}{0.413320in}}%
\pgfpathlineto{\pgfqpoint{3.282870in}{0.413320in}}%
\pgfpathlineto{\pgfqpoint{3.280189in}{0.413320in}}%
\pgfpathlineto{\pgfqpoint{3.277603in}{0.413320in}}%
\pgfpathlineto{\pgfqpoint{3.274831in}{0.413320in}}%
\pgfpathlineto{\pgfqpoint{3.272254in}{0.413320in}}%
\pgfpathlineto{\pgfqpoint{3.269478in}{0.413320in}}%
\pgfpathlineto{\pgfqpoint{3.266849in}{0.413320in}}%
\pgfpathlineto{\pgfqpoint{3.264119in}{0.413320in}}%
\pgfpathlineto{\pgfqpoint{3.261594in}{0.413320in}}%
\pgfpathlineto{\pgfqpoint{3.258784in}{0.413320in}}%
\pgfpathlineto{\pgfqpoint{3.256083in}{0.413320in}}%
\pgfpathlineto{\pgfqpoint{3.253404in}{0.413320in}}%
\pgfpathlineto{\pgfqpoint{3.250716in}{0.413320in}}%
\pgfpathlineto{\pgfqpoint{3.248049in}{0.413320in}}%
\pgfpathlineto{\pgfqpoint{3.245363in}{0.413320in}}%
\pgfpathlineto{\pgfqpoint{3.242807in}{0.413320in}}%
\pgfpathlineto{\pgfqpoint{3.240010in}{0.413320in}}%
\pgfpathlineto{\pgfqpoint{3.237411in}{0.413320in}}%
\pgfpathlineto{\pgfqpoint{3.234658in}{0.413320in}}%
\pgfpathlineto{\pgfqpoint{3.232069in}{0.413320in}}%
\pgfpathlineto{\pgfqpoint{3.229310in}{0.413320in}}%
\pgfpathlineto{\pgfqpoint{3.226609in}{0.413320in}}%
\pgfpathlineto{\pgfqpoint{3.223942in}{0.413320in}}%
\pgfpathlineto{\pgfqpoint{3.221255in}{0.413320in}}%
\pgfpathlineto{\pgfqpoint{3.218586in}{0.413320in}}%
\pgfpathlineto{\pgfqpoint{3.215908in}{0.413320in}}%
\pgfpathlineto{\pgfqpoint{3.213342in}{0.413320in}}%
\pgfpathlineto{\pgfqpoint{3.210545in}{0.413320in}}%
\pgfpathlineto{\pgfqpoint{3.207984in}{0.413320in}}%
\pgfpathlineto{\pgfqpoint{3.205195in}{0.413320in}}%
\pgfpathlineto{\pgfqpoint{3.202562in}{0.413320in}}%
\pgfpathlineto{\pgfqpoint{3.199823in}{0.413320in}}%
\pgfpathlineto{\pgfqpoint{3.197226in}{0.413320in}}%
\pgfpathlineto{\pgfqpoint{3.194508in}{0.413320in}}%
\pgfpathlineto{\pgfqpoint{3.191796in}{0.413320in}}%
\pgfpathlineto{\pgfqpoint{3.189117in}{0.413320in}}%
\pgfpathlineto{\pgfqpoint{3.186440in}{0.413320in}}%
\pgfpathlineto{\pgfqpoint{3.183760in}{0.413320in}}%
\pgfpathlineto{\pgfqpoint{3.181089in}{0.413320in}}%
\pgfpathlineto{\pgfqpoint{3.178525in}{0.413320in}}%
\pgfpathlineto{\pgfqpoint{3.175724in}{0.413320in}}%
\pgfpathlineto{\pgfqpoint{3.173142in}{0.413320in}}%
\pgfpathlineto{\pgfqpoint{3.170375in}{0.413320in}}%
\pgfpathlineto{\pgfqpoint{3.167776in}{0.413320in}}%
\pgfpathlineto{\pgfqpoint{3.165019in}{0.413320in}}%
\pgfpathlineto{\pgfqpoint{3.162474in}{0.413320in}}%
\pgfpathlineto{\pgfqpoint{3.159675in}{0.413320in}}%
\pgfpathlineto{\pgfqpoint{3.156981in}{0.413320in}}%
\pgfpathlineto{\pgfqpoint{3.154327in}{0.413320in}}%
\pgfpathlineto{\pgfqpoint{3.151612in}{0.413320in}}%
\pgfpathlineto{\pgfqpoint{3.149057in}{0.413320in}}%
\pgfpathlineto{\pgfqpoint{3.146271in}{0.413320in}}%
\pgfpathlineto{\pgfqpoint{3.143740in}{0.413320in}}%
\pgfpathlineto{\pgfqpoint{3.140913in}{0.413320in}}%
\pgfpathlineto{\pgfqpoint{3.138375in}{0.413320in}}%
\pgfpathlineto{\pgfqpoint{3.135550in}{0.413320in}}%
\pgfpathlineto{\pgfqpoint{3.132946in}{0.413320in}}%
\pgfpathlineto{\pgfqpoint{3.130199in}{0.413320in}}%
\pgfpathlineto{\pgfqpoint{3.127512in}{0.413320in}}%
\pgfpathlineto{\pgfqpoint{3.124842in}{0.413320in}}%
\pgfpathlineto{\pgfqpoint{3.122163in}{0.413320in}}%
\pgfpathlineto{\pgfqpoint{3.119487in}{0.413320in}}%
\pgfpathlineto{\pgfqpoint{3.116807in}{0.413320in}}%
\pgfpathlineto{\pgfqpoint{3.114242in}{0.413320in}}%
\pgfpathlineto{\pgfqpoint{3.111451in}{0.413320in}}%
\pgfpathlineto{\pgfqpoint{3.108896in}{0.413320in}}%
\pgfpathlineto{\pgfqpoint{3.106094in}{0.413320in}}%
\pgfpathlineto{\pgfqpoint{3.103508in}{0.413320in}}%
\pgfpathlineto{\pgfqpoint{3.100737in}{0.413320in}}%
\pgfpathlineto{\pgfqpoint{3.098163in}{0.413320in}}%
\pgfpathlineto{\pgfqpoint{3.095388in}{0.413320in}}%
\pgfpathlineto{\pgfqpoint{3.092699in}{0.413320in}}%
\pgfpathlineto{\pgfqpoint{3.090023in}{0.413320in}}%
\pgfpathlineto{\pgfqpoint{3.087343in}{0.413320in}}%
\pgfpathlineto{\pgfqpoint{3.084671in}{0.413320in}}%
\pgfpathlineto{\pgfqpoint{3.081990in}{0.413320in}}%
\pgfpathlineto{\pgfqpoint{3.079381in}{0.413320in}}%
\pgfpathlineto{\pgfqpoint{3.076631in}{0.413320in}}%
\pgfpathlineto{\pgfqpoint{3.074056in}{0.413320in}}%
\pgfpathlineto{\pgfqpoint{3.071266in}{0.413320in}}%
\pgfpathlineto{\pgfqpoint{3.068709in}{0.413320in}}%
\pgfpathlineto{\pgfqpoint{3.065916in}{0.413320in}}%
\pgfpathlineto{\pgfqpoint{3.063230in}{0.413320in}}%
\pgfpathlineto{\pgfqpoint{3.060561in}{0.413320in}}%
\pgfpathlineto{\pgfqpoint{3.057884in}{0.413320in}}%
\pgfpathlineto{\pgfqpoint{3.055202in}{0.413320in}}%
\pgfpathlineto{\pgfqpoint{3.052526in}{0.413320in}}%
\pgfpathlineto{\pgfqpoint{3.049988in}{0.413320in}}%
\pgfpathlineto{\pgfqpoint{3.047157in}{0.413320in}}%
\pgfpathlineto{\pgfqpoint{3.044568in}{0.413320in}}%
\pgfpathlineto{\pgfqpoint{3.041813in}{0.413320in}}%
\pgfpathlineto{\pgfqpoint{3.039262in}{0.413320in}}%
\pgfpathlineto{\pgfqpoint{3.036456in}{0.413320in}}%
\pgfpathlineto{\pgfqpoint{3.033921in}{0.413320in}}%
\pgfpathlineto{\pgfqpoint{3.031091in}{0.413320in}}%
\pgfpathlineto{\pgfqpoint{3.028412in}{0.413320in}}%
\pgfpathlineto{\pgfqpoint{3.025803in}{0.413320in}}%
\pgfpathlineto{\pgfqpoint{3.023058in}{0.413320in}}%
\pgfpathlineto{\pgfqpoint{3.020382in}{0.413320in}}%
\pgfpathlineto{\pgfqpoint{3.017707in}{0.413320in}}%
\pgfpathlineto{\pgfqpoint{3.015097in}{0.413320in}}%
\pgfpathlineto{\pgfqpoint{3.012351in}{0.413320in}}%
\pgfpathlineto{\pgfqpoint{3.009784in}{0.413320in}}%
\pgfpathlineto{\pgfqpoint{3.006993in}{0.413320in}}%
\pgfpathlineto{\pgfqpoint{3.004419in}{0.413320in}}%
\pgfpathlineto{\pgfqpoint{3.001635in}{0.413320in}}%
\pgfpathlineto{\pgfqpoint{2.999103in}{0.413320in}}%
\pgfpathlineto{\pgfqpoint{2.996300in}{0.413320in}}%
\pgfpathlineto{\pgfqpoint{2.993595in}{0.413320in}}%
\pgfpathlineto{\pgfqpoint{2.990978in}{0.413320in}}%
\pgfpathlineto{\pgfqpoint{2.988238in}{0.413320in}}%
\pgfpathlineto{\pgfqpoint{2.985666in}{0.413320in}}%
\pgfpathlineto{\pgfqpoint{2.982885in}{0.413320in}}%
\pgfpathlineto{\pgfqpoint{2.980341in}{0.413320in}}%
\pgfpathlineto{\pgfqpoint{2.977517in}{0.413320in}}%
\pgfpathlineto{\pgfqpoint{2.974972in}{0.413320in}}%
\pgfpathlineto{\pgfqpoint{2.972177in}{0.413320in}}%
\pgfpathlineto{\pgfqpoint{2.969599in}{0.413320in}}%
\pgfpathlineto{\pgfqpoint{2.966812in}{0.413320in}}%
\pgfpathlineto{\pgfqpoint{2.964127in}{0.413320in}}%
\pgfpathlineto{\pgfqpoint{2.961460in}{0.413320in}}%
\pgfpathlineto{\pgfqpoint{2.958782in}{0.413320in}}%
\pgfpathlineto{\pgfqpoint{2.956103in}{0.413320in}}%
\pgfpathlineto{\pgfqpoint{2.953422in}{0.413320in}}%
\pgfpathlineto{\pgfqpoint{2.950884in}{0.413320in}}%
\pgfpathlineto{\pgfqpoint{2.948068in}{0.413320in}}%
\pgfpathlineto{\pgfqpoint{2.945461in}{0.413320in}}%
\pgfpathlineto{\pgfqpoint{2.942711in}{0.413320in}}%
\pgfpathlineto{\pgfqpoint{2.940120in}{0.413320in}}%
\pgfpathlineto{\pgfqpoint{2.937352in}{0.413320in}}%
\pgfpathlineto{\pgfqpoint{2.934759in}{0.413320in}}%
\pgfpathlineto{\pgfqpoint{2.932033in}{0.413320in}}%
\pgfpathlineto{\pgfqpoint{2.929321in}{0.413320in}}%
\pgfpathlineto{\pgfqpoint{2.926655in}{0.413320in}}%
\pgfpathlineto{\pgfqpoint{2.923963in}{0.413320in}}%
\pgfpathlineto{\pgfqpoint{2.921363in}{0.413320in}}%
\pgfpathlineto{\pgfqpoint{2.918606in}{0.413320in}}%
\pgfpathlineto{\pgfqpoint{2.916061in}{0.413320in}}%
\pgfpathlineto{\pgfqpoint{2.913243in}{0.413320in}}%
\pgfpathlineto{\pgfqpoint{2.910631in}{0.413320in}}%
\pgfpathlineto{\pgfqpoint{2.907882in}{0.413320in}}%
\pgfpathlineto{\pgfqpoint{2.905341in}{0.413320in}}%
\pgfpathlineto{\pgfqpoint{2.902535in}{0.413320in}}%
\pgfpathlineto{\pgfqpoint{2.899858in}{0.413320in}}%
\pgfpathlineto{\pgfqpoint{2.897179in}{0.413320in}}%
\pgfpathlineto{\pgfqpoint{2.894487in}{0.413320in}}%
\pgfpathlineto{\pgfqpoint{2.891809in}{0.413320in}}%
\pgfpathlineto{\pgfqpoint{2.889145in}{0.413320in}}%
\pgfpathlineto{\pgfqpoint{2.886578in}{0.413320in}}%
\pgfpathlineto{\pgfqpoint{2.883780in}{0.413320in}}%
\pgfpathlineto{\pgfqpoint{2.881254in}{0.413320in}}%
\pgfpathlineto{\pgfqpoint{2.878431in}{0.413320in}}%
\pgfpathlineto{\pgfqpoint{2.875882in}{0.413320in}}%
\pgfpathlineto{\pgfqpoint{2.873074in}{0.413320in}}%
\pgfpathlineto{\pgfqpoint{2.870475in}{0.413320in}}%
\pgfpathlineto{\pgfqpoint{2.867713in}{0.413320in}}%
\pgfpathlineto{\pgfqpoint{2.865031in}{0.413320in}}%
\pgfpathlineto{\pgfqpoint{2.862402in}{0.413320in}}%
\pgfpathlineto{\pgfqpoint{2.859668in}{0.413320in}}%
\pgfpathlineto{\pgfqpoint{2.857003in}{0.413320in}}%
\pgfpathlineto{\pgfqpoint{2.854325in}{0.413320in}}%
\pgfpathlineto{\pgfqpoint{2.851793in}{0.413320in}}%
\pgfpathlineto{\pgfqpoint{2.848960in}{0.413320in}}%
\pgfpathlineto{\pgfqpoint{2.846408in}{0.413320in}}%
\pgfpathlineto{\pgfqpoint{2.843611in}{0.413320in}}%
\pgfpathlineto{\pgfqpoint{2.841055in}{0.413320in}}%
\pgfpathlineto{\pgfqpoint{2.838254in}{0.413320in}}%
\pgfpathlineto{\pgfqpoint{2.835698in}{0.413320in}}%
\pgfpathlineto{\pgfqpoint{2.832894in}{0.413320in}}%
\pgfpathlineto{\pgfqpoint{2.830219in}{0.413320in}}%
\pgfpathlineto{\pgfqpoint{2.827567in}{0.413320in}}%
\pgfpathlineto{\pgfqpoint{2.824851in}{0.413320in}}%
\pgfpathlineto{\pgfqpoint{2.822303in}{0.413320in}}%
\pgfpathlineto{\pgfqpoint{2.819506in}{0.413320in}}%
\pgfpathlineto{\pgfqpoint{2.816867in}{0.413320in}}%
\pgfpathlineto{\pgfqpoint{2.814141in}{0.413320in}}%
\pgfpathlineto{\pgfqpoint{2.811597in}{0.413320in}}%
\pgfpathlineto{\pgfqpoint{2.808792in}{0.413320in}}%
\pgfpathlineto{\pgfqpoint{2.806175in}{0.413320in}}%
\pgfpathlineto{\pgfqpoint{2.803435in}{0.413320in}}%
\pgfpathlineto{\pgfqpoint{2.800756in}{0.413320in}}%
\pgfpathlineto{\pgfqpoint{2.798070in}{0.413320in}}%
\pgfpathlineto{\pgfqpoint{2.795398in}{0.413320in}}%
\pgfpathlineto{\pgfqpoint{2.792721in}{0.413320in}}%
\pgfpathlineto{\pgfqpoint{2.790044in}{0.413320in}}%
\pgfpathlineto{\pgfqpoint{2.787468in}{0.413320in}}%
\pgfpathlineto{\pgfqpoint{2.784687in}{0.413320in}}%
\pgfpathlineto{\pgfqpoint{2.782113in}{0.413320in}}%
\pgfpathlineto{\pgfqpoint{2.779330in}{0.413320in}}%
\pgfpathlineto{\pgfqpoint{2.776767in}{0.413320in}}%
\pgfpathlineto{\pgfqpoint{2.773972in}{0.413320in}}%
\pgfpathlineto{\pgfqpoint{2.771373in}{0.413320in}}%
\pgfpathlineto{\pgfqpoint{2.768617in}{0.413320in}}%
\pgfpathlineto{\pgfqpoint{2.765935in}{0.413320in}}%
\pgfpathlineto{\pgfqpoint{2.763253in}{0.413320in}}%
\pgfpathlineto{\pgfqpoint{2.760581in}{0.413320in}}%
\pgfpathlineto{\pgfqpoint{2.758028in}{0.413320in}}%
\pgfpathlineto{\pgfqpoint{2.755224in}{0.413320in}}%
\pgfpathlineto{\pgfqpoint{2.752614in}{0.413320in}}%
\pgfpathlineto{\pgfqpoint{2.749868in}{0.413320in}}%
\pgfpathlineto{\pgfqpoint{2.747260in}{0.413320in}}%
\pgfpathlineto{\pgfqpoint{2.744510in}{0.413320in}}%
\pgfpathlineto{\pgfqpoint{2.741928in}{0.413320in}}%
\pgfpathlineto{\pgfqpoint{2.739155in}{0.413320in}}%
\pgfpathlineto{\pgfqpoint{2.736476in}{0.413320in}}%
\pgfpathlineto{\pgfqpoint{2.733798in}{0.413320in}}%
\pgfpathlineto{\pgfqpoint{2.731119in}{0.413320in}}%
\pgfpathlineto{\pgfqpoint{2.728439in}{0.413320in}}%
\pgfpathlineto{\pgfqpoint{2.725760in}{0.413320in}}%
\pgfpathlineto{\pgfqpoint{2.723211in}{0.413320in}}%
\pgfpathlineto{\pgfqpoint{2.720404in}{0.413320in}}%
\pgfpathlineto{\pgfqpoint{2.717773in}{0.413320in}}%
\pgfpathlineto{\pgfqpoint{2.715036in}{0.413320in}}%
\pgfpathlineto{\pgfqpoint{2.712477in}{0.413320in}}%
\pgfpathlineto{\pgfqpoint{2.709683in}{0.413320in}}%
\pgfpathlineto{\pgfqpoint{2.707125in}{0.413320in}}%
\pgfpathlineto{\pgfqpoint{2.704326in}{0.413320in}}%
\pgfpathlineto{\pgfqpoint{2.701657in}{0.413320in}}%
\pgfpathlineto{\pgfqpoint{2.698968in}{0.413320in}}%
\pgfpathlineto{\pgfqpoint{2.696293in}{0.413320in}}%
\pgfpathlineto{\pgfqpoint{2.693611in}{0.413320in}}%
\pgfpathlineto{\pgfqpoint{2.690940in}{0.413320in}}%
\pgfpathlineto{\pgfqpoint{2.688328in}{0.413320in}}%
\pgfpathlineto{\pgfqpoint{2.685586in}{0.413320in}}%
\pgfpathlineto{\pgfqpoint{2.683009in}{0.413320in}}%
\pgfpathlineto{\pgfqpoint{2.680224in}{0.413320in}}%
\pgfpathlineto{\pgfqpoint{2.677650in}{0.413320in}}%
\pgfpathlineto{\pgfqpoint{2.674873in}{0.413320in}}%
\pgfpathlineto{\pgfqpoint{2.672301in}{0.413320in}}%
\pgfpathlineto{\pgfqpoint{2.669506in}{0.413320in}}%
\pgfpathlineto{\pgfqpoint{2.666836in}{0.413320in}}%
\pgfpathlineto{\pgfqpoint{2.664151in}{0.413320in}}%
\pgfpathlineto{\pgfqpoint{2.661481in}{0.413320in}}%
\pgfpathlineto{\pgfqpoint{2.658942in}{0.413320in}}%
\pgfpathlineto{\pgfqpoint{2.656124in}{0.413320in}}%
\pgfpathlineto{\pgfqpoint{2.653567in}{0.413320in}}%
\pgfpathlineto{\pgfqpoint{2.650767in}{0.413320in}}%
\pgfpathlineto{\pgfqpoint{2.648196in}{0.413320in}}%
\pgfpathlineto{\pgfqpoint{2.645408in}{0.413320in}}%
\pgfpathlineto{\pgfqpoint{2.642827in}{0.413320in}}%
\pgfpathlineto{\pgfqpoint{2.640053in}{0.413320in}}%
\pgfpathlineto{\pgfqpoint{2.637369in}{0.413320in}}%
\pgfpathlineto{\pgfqpoint{2.634700in}{0.413320in}}%
\pgfpathlineto{\pgfqpoint{2.632018in}{0.413320in}}%
\pgfpathlineto{\pgfqpoint{2.629340in}{0.413320in}}%
\pgfpathlineto{\pgfqpoint{2.626653in}{0.413320in}}%
\pgfpathlineto{\pgfqpoint{2.624077in}{0.413320in}}%
\pgfpathlineto{\pgfqpoint{2.621304in}{0.413320in}}%
\pgfpathlineto{\pgfqpoint{2.618773in}{0.413320in}}%
\pgfpathlineto{\pgfqpoint{2.615934in}{0.413320in}}%
\pgfpathlineto{\pgfqpoint{2.613393in}{0.413320in}}%
\pgfpathlineto{\pgfqpoint{2.610588in}{0.413320in}}%
\pgfpathlineto{\pgfqpoint{2.608004in}{0.413320in}}%
\pgfpathlineto{\pgfqpoint{2.605232in}{0.413320in}}%
\pgfpathlineto{\pgfqpoint{2.602557in}{0.413320in}}%
\pgfpathlineto{\pgfqpoint{2.599920in}{0.413320in}}%
\pgfpathlineto{\pgfqpoint{2.597196in}{0.413320in}}%
\pgfpathlineto{\pgfqpoint{2.594630in}{0.413320in}}%
\pgfpathlineto{\pgfqpoint{2.591842in}{0.413320in}}%
\pgfpathlineto{\pgfqpoint{2.589248in}{0.413320in}}%
\pgfpathlineto{\pgfqpoint{2.586484in}{0.413320in}}%
\pgfpathlineto{\pgfqpoint{2.583913in}{0.413320in}}%
\pgfpathlineto{\pgfqpoint{2.581129in}{0.413320in}}%
\pgfpathlineto{\pgfqpoint{2.578567in}{0.413320in}}%
\pgfpathlineto{\pgfqpoint{2.575779in}{0.413320in}}%
\pgfpathlineto{\pgfqpoint{2.573082in}{0.413320in}}%
\pgfpathlineto{\pgfqpoint{2.570411in}{0.413320in}}%
\pgfpathlineto{\pgfqpoint{2.567730in}{0.413320in}}%
\pgfpathlineto{\pgfqpoint{2.565045in}{0.413320in}}%
\pgfpathlineto{\pgfqpoint{2.562375in}{0.413320in}}%
\pgfpathlineto{\pgfqpoint{2.559790in}{0.413320in}}%
\pgfpathlineto{\pgfqpoint{2.557009in}{0.413320in}}%
\pgfpathlineto{\pgfqpoint{2.554493in}{0.413320in}}%
\pgfpathlineto{\pgfqpoint{2.551664in}{0.413320in}}%
\pgfpathlineto{\pgfqpoint{2.549114in}{0.413320in}}%
\pgfpathlineto{\pgfqpoint{2.546310in}{0.413320in}}%
\pgfpathlineto{\pgfqpoint{2.543765in}{0.413320in}}%
\pgfpathlineto{\pgfqpoint{2.540949in}{0.413320in}}%
\pgfpathlineto{\pgfqpoint{2.538274in}{0.413320in}}%
\pgfpathlineto{\pgfqpoint{2.535624in}{0.413320in}}%
\pgfpathlineto{\pgfqpoint{2.532917in}{0.413320in}}%
\pgfpathlineto{\pgfqpoint{2.530234in}{0.413320in}}%
\pgfpathlineto{\pgfqpoint{2.527560in}{0.413320in}}%
\pgfpathlineto{\pgfqpoint{2.524988in}{0.413320in}}%
\pgfpathlineto{\pgfqpoint{2.522197in}{0.413320in}}%
\pgfpathlineto{\pgfqpoint{2.519607in}{0.413320in}}%
\pgfpathlineto{\pgfqpoint{2.516845in}{0.413320in}}%
\pgfpathlineto{\pgfqpoint{2.514268in}{0.413320in}}%
\pgfpathlineto{\pgfqpoint{2.511478in}{0.413320in}}%
\pgfpathlineto{\pgfqpoint{2.508917in}{0.413320in}}%
\pgfpathlineto{\pgfqpoint{2.506163in}{0.413320in}}%
\pgfpathlineto{\pgfqpoint{2.503454in}{0.413320in}}%
\pgfpathlineto{\pgfqpoint{2.500801in}{0.413320in}}%
\pgfpathlineto{\pgfqpoint{2.498085in}{0.413320in}}%
\pgfpathlineto{\pgfqpoint{2.495542in}{0.413320in}}%
\pgfpathlineto{\pgfqpoint{2.492729in}{0.413320in}}%
\pgfpathlineto{\pgfqpoint{2.490183in}{0.413320in}}%
\pgfpathlineto{\pgfqpoint{2.487384in}{0.413320in}}%
\pgfpathlineto{\pgfqpoint{2.484870in}{0.413320in}}%
\pgfpathlineto{\pgfqpoint{2.482026in}{0.413320in}}%
\pgfpathlineto{\pgfqpoint{2.479420in}{0.413320in}}%
\pgfpathlineto{\pgfqpoint{2.476671in}{0.413320in}}%
\pgfpathlineto{\pgfqpoint{2.473989in}{0.413320in}}%
\pgfpathlineto{\pgfqpoint{2.471311in}{0.413320in}}%
\pgfpathlineto{\pgfqpoint{2.468635in}{0.413320in}}%
\pgfpathlineto{\pgfqpoint{2.465957in}{0.413320in}}%
\pgfpathlineto{\pgfqpoint{2.463280in}{0.413320in}}%
\pgfpathlineto{\pgfqpoint{2.460711in}{0.413320in}}%
\pgfpathlineto{\pgfqpoint{2.457917in}{0.413320in}}%
\pgfpathlineto{\pgfqpoint{2.455353in}{0.413320in}}%
\pgfpathlineto{\pgfqpoint{2.452562in}{0.413320in}}%
\pgfpathlineto{\pgfqpoint{2.450032in}{0.413320in}}%
\pgfpathlineto{\pgfqpoint{2.447209in}{0.413320in}}%
\pgfpathlineto{\pgfqpoint{2.444677in}{0.413320in}}%
\pgfpathlineto{\pgfqpoint{2.441876in}{0.413320in}}%
\pgfpathlineto{\pgfqpoint{2.439167in}{0.413320in}}%
\pgfpathlineto{\pgfqpoint{2.436518in}{0.413320in}}%
\pgfpathlineto{\pgfqpoint{2.433815in}{0.413320in}}%
\pgfpathlineto{\pgfqpoint{2.431251in}{0.413320in}}%
\pgfpathlineto{\pgfqpoint{2.428453in}{0.413320in}}%
\pgfpathlineto{\pgfqpoint{2.425878in}{0.413320in}}%
\pgfpathlineto{\pgfqpoint{2.423098in}{0.413320in}}%
\pgfpathlineto{\pgfqpoint{2.420528in}{0.413320in}}%
\pgfpathlineto{\pgfqpoint{2.417747in}{0.413320in}}%
\pgfpathlineto{\pgfqpoint{2.415184in}{0.413320in}}%
\pgfpathlineto{\pgfqpoint{2.412389in}{0.413320in}}%
\pgfpathlineto{\pgfqpoint{2.409699in}{0.413320in}}%
\pgfpathlineto{\pgfqpoint{2.407024in}{0.413320in}}%
\pgfpathlineto{\pgfqpoint{2.404352in}{0.413320in}}%
\pgfpathlineto{\pgfqpoint{2.401675in}{0.413320in}}%
\pgfpathlineto{\pgfqpoint{2.398995in}{0.413320in}}%
\pgfpathclose%
\pgfusepath{stroke,fill}%
\end{pgfscope}%
\begin{pgfscope}%
\pgfpathrectangle{\pgfqpoint{2.398995in}{0.319877in}}{\pgfqpoint{3.986877in}{1.993438in}} %
\pgfusepath{clip}%
\pgfsetbuttcap%
\pgfsetroundjoin%
\definecolor{currentfill}{rgb}{1.000000,1.000000,1.000000}%
\pgfsetfillcolor{currentfill}%
\pgfsetlinewidth{1.003750pt}%
\definecolor{currentstroke}{rgb}{0.958705,0.366226,0.923147}%
\pgfsetstrokecolor{currentstroke}%
\pgfsetdash{}{0pt}%
\pgfpathmoveto{\pgfqpoint{2.398995in}{0.413320in}}%
\pgfpathlineto{\pgfqpoint{2.398995in}{1.105711in}}%
\pgfpathlineto{\pgfqpoint{2.401675in}{1.108998in}}%
\pgfpathlineto{\pgfqpoint{2.404352in}{1.110056in}}%
\pgfpathlineto{\pgfqpoint{2.407024in}{1.101664in}}%
\pgfpathlineto{\pgfqpoint{2.409699in}{1.098436in}}%
\pgfpathlineto{\pgfqpoint{2.412389in}{1.106048in}}%
\pgfpathlineto{\pgfqpoint{2.415184in}{1.115919in}}%
\pgfpathlineto{\pgfqpoint{2.417747in}{1.121043in}}%
\pgfpathlineto{\pgfqpoint{2.420528in}{1.113553in}}%
\pgfpathlineto{\pgfqpoint{2.423098in}{1.115512in}}%
\pgfpathlineto{\pgfqpoint{2.425878in}{1.115141in}}%
\pgfpathlineto{\pgfqpoint{2.428453in}{1.110491in}}%
\pgfpathlineto{\pgfqpoint{2.431251in}{1.121320in}}%
\pgfpathlineto{\pgfqpoint{2.433815in}{1.152966in}}%
\pgfpathlineto{\pgfqpoint{2.436518in}{1.159953in}}%
\pgfpathlineto{\pgfqpoint{2.439167in}{1.145369in}}%
\pgfpathlineto{\pgfqpoint{2.441876in}{1.145798in}}%
\pgfpathlineto{\pgfqpoint{2.444677in}{1.146876in}}%
\pgfpathlineto{\pgfqpoint{2.447209in}{1.135945in}}%
\pgfpathlineto{\pgfqpoint{2.450032in}{1.128263in}}%
\pgfpathlineto{\pgfqpoint{2.452562in}{1.124352in}}%
\pgfpathlineto{\pgfqpoint{2.455353in}{1.128048in}}%
\pgfpathlineto{\pgfqpoint{2.457917in}{1.121716in}}%
\pgfpathlineto{\pgfqpoint{2.460711in}{1.119031in}}%
\pgfpathlineto{\pgfqpoint{2.463280in}{1.117713in}}%
\pgfpathlineto{\pgfqpoint{2.465957in}{1.118828in}}%
\pgfpathlineto{\pgfqpoint{2.468635in}{1.123032in}}%
\pgfpathlineto{\pgfqpoint{2.471311in}{1.122515in}}%
\pgfpathlineto{\pgfqpoint{2.473989in}{1.119638in}}%
\pgfpathlineto{\pgfqpoint{2.476671in}{1.118641in}}%
\pgfpathlineto{\pgfqpoint{2.479420in}{1.114854in}}%
\pgfpathlineto{\pgfqpoint{2.482026in}{1.111782in}}%
\pgfpathlineto{\pgfqpoint{2.484870in}{1.112429in}}%
\pgfpathlineto{\pgfqpoint{2.487384in}{1.111268in}}%
\pgfpathlineto{\pgfqpoint{2.490183in}{1.111557in}}%
\pgfpathlineto{\pgfqpoint{2.492729in}{1.111573in}}%
\pgfpathlineto{\pgfqpoint{2.495542in}{1.108609in}}%
\pgfpathlineto{\pgfqpoint{2.498085in}{1.109500in}}%
\pgfpathlineto{\pgfqpoint{2.500801in}{1.112296in}}%
\pgfpathlineto{\pgfqpoint{2.503454in}{1.114794in}}%
\pgfpathlineto{\pgfqpoint{2.506163in}{1.113031in}}%
\pgfpathlineto{\pgfqpoint{2.508917in}{1.109761in}}%
\pgfpathlineto{\pgfqpoint{2.511478in}{1.107586in}}%
\pgfpathlineto{\pgfqpoint{2.514268in}{1.109357in}}%
\pgfpathlineto{\pgfqpoint{2.516845in}{1.111954in}}%
\pgfpathlineto{\pgfqpoint{2.519607in}{1.111953in}}%
\pgfpathlineto{\pgfqpoint{2.522197in}{1.106816in}}%
\pgfpathlineto{\pgfqpoint{2.524988in}{1.106335in}}%
\pgfpathlineto{\pgfqpoint{2.527560in}{1.103415in}}%
\pgfpathlineto{\pgfqpoint{2.530234in}{1.100209in}}%
\pgfpathlineto{\pgfqpoint{2.532917in}{1.107331in}}%
\pgfpathlineto{\pgfqpoint{2.535624in}{1.107134in}}%
\pgfpathlineto{\pgfqpoint{2.538274in}{1.111593in}}%
\pgfpathlineto{\pgfqpoint{2.540949in}{1.108284in}}%
\pgfpathlineto{\pgfqpoint{2.543765in}{1.111067in}}%
\pgfpathlineto{\pgfqpoint{2.546310in}{1.106038in}}%
\pgfpathlineto{\pgfqpoint{2.549114in}{1.111465in}}%
\pgfpathlineto{\pgfqpoint{2.551664in}{1.104525in}}%
\pgfpathlineto{\pgfqpoint{2.554493in}{1.106352in}}%
\pgfpathlineto{\pgfqpoint{2.557009in}{1.107618in}}%
\pgfpathlineto{\pgfqpoint{2.559790in}{1.101309in}}%
\pgfpathlineto{\pgfqpoint{2.562375in}{1.098674in}}%
\pgfpathlineto{\pgfqpoint{2.565045in}{1.105348in}}%
\pgfpathlineto{\pgfqpoint{2.567730in}{1.103494in}}%
\pgfpathlineto{\pgfqpoint{2.570411in}{1.106762in}}%
\pgfpathlineto{\pgfqpoint{2.573082in}{1.117156in}}%
\pgfpathlineto{\pgfqpoint{2.575779in}{1.126028in}}%
\pgfpathlineto{\pgfqpoint{2.578567in}{1.116885in}}%
\pgfpathlineto{\pgfqpoint{2.581129in}{1.113169in}}%
\pgfpathlineto{\pgfqpoint{2.583913in}{1.103553in}}%
\pgfpathlineto{\pgfqpoint{2.586484in}{1.112729in}}%
\pgfpathlineto{\pgfqpoint{2.589248in}{1.110248in}}%
\pgfpathlineto{\pgfqpoint{2.591842in}{1.108032in}}%
\pgfpathlineto{\pgfqpoint{2.594630in}{1.106716in}}%
\pgfpathlineto{\pgfqpoint{2.597196in}{1.107248in}}%
\pgfpathlineto{\pgfqpoint{2.599920in}{1.101949in}}%
\pgfpathlineto{\pgfqpoint{2.602557in}{1.097692in}}%
\pgfpathlineto{\pgfqpoint{2.605232in}{1.106603in}}%
\pgfpathlineto{\pgfqpoint{2.608004in}{1.105217in}}%
\pgfpathlineto{\pgfqpoint{2.610588in}{1.103762in}}%
\pgfpathlineto{\pgfqpoint{2.613393in}{1.101443in}}%
\pgfpathlineto{\pgfqpoint{2.615934in}{1.101837in}}%
\pgfpathlineto{\pgfqpoint{2.618773in}{1.102101in}}%
\pgfpathlineto{\pgfqpoint{2.621304in}{1.102451in}}%
\pgfpathlineto{\pgfqpoint{2.624077in}{1.101739in}}%
\pgfpathlineto{\pgfqpoint{2.626653in}{1.104953in}}%
\pgfpathlineto{\pgfqpoint{2.629340in}{1.104484in}}%
\pgfpathlineto{\pgfqpoint{2.632018in}{1.106614in}}%
\pgfpathlineto{\pgfqpoint{2.634700in}{1.109294in}}%
\pgfpathlineto{\pgfqpoint{2.637369in}{1.105873in}}%
\pgfpathlineto{\pgfqpoint{2.640053in}{1.111431in}}%
\pgfpathlineto{\pgfqpoint{2.642827in}{1.112951in}}%
\pgfpathlineto{\pgfqpoint{2.645408in}{1.112659in}}%
\pgfpathlineto{\pgfqpoint{2.648196in}{1.106070in}}%
\pgfpathlineto{\pgfqpoint{2.650767in}{1.103038in}}%
\pgfpathlineto{\pgfqpoint{2.653567in}{1.102660in}}%
\pgfpathlineto{\pgfqpoint{2.656124in}{1.103444in}}%
\pgfpathlineto{\pgfqpoint{2.658942in}{1.101159in}}%
\pgfpathlineto{\pgfqpoint{2.661481in}{1.098070in}}%
\pgfpathlineto{\pgfqpoint{2.664151in}{1.099305in}}%
\pgfpathlineto{\pgfqpoint{2.666836in}{1.100189in}}%
\pgfpathlineto{\pgfqpoint{2.669506in}{1.098513in}}%
\pgfpathlineto{\pgfqpoint{2.672301in}{1.104873in}}%
\pgfpathlineto{\pgfqpoint{2.674873in}{1.104534in}}%
\pgfpathlineto{\pgfqpoint{2.677650in}{1.105515in}}%
\pgfpathlineto{\pgfqpoint{2.680224in}{1.105179in}}%
\pgfpathlineto{\pgfqpoint{2.683009in}{1.101771in}}%
\pgfpathlineto{\pgfqpoint{2.685586in}{1.109121in}}%
\pgfpathlineto{\pgfqpoint{2.688328in}{1.111848in}}%
\pgfpathlineto{\pgfqpoint{2.690940in}{1.110001in}}%
\pgfpathlineto{\pgfqpoint{2.693611in}{1.107787in}}%
\pgfpathlineto{\pgfqpoint{2.696293in}{1.106457in}}%
\pgfpathlineto{\pgfqpoint{2.698968in}{1.102820in}}%
\pgfpathlineto{\pgfqpoint{2.701657in}{1.103648in}}%
\pgfpathlineto{\pgfqpoint{2.704326in}{1.104042in}}%
\pgfpathlineto{\pgfqpoint{2.707125in}{1.111486in}}%
\pgfpathlineto{\pgfqpoint{2.709683in}{1.106378in}}%
\pgfpathlineto{\pgfqpoint{2.712477in}{1.111049in}}%
\pgfpathlineto{\pgfqpoint{2.715036in}{1.110765in}}%
\pgfpathlineto{\pgfqpoint{2.717773in}{1.104131in}}%
\pgfpathlineto{\pgfqpoint{2.720404in}{1.105513in}}%
\pgfpathlineto{\pgfqpoint{2.723211in}{1.103022in}}%
\pgfpathlineto{\pgfqpoint{2.725760in}{1.106663in}}%
\pgfpathlineto{\pgfqpoint{2.728439in}{1.103162in}}%
\pgfpathlineto{\pgfqpoint{2.731119in}{1.093509in}}%
\pgfpathlineto{\pgfqpoint{2.733798in}{1.096062in}}%
\pgfpathlineto{\pgfqpoint{2.736476in}{1.086963in}}%
\pgfpathlineto{\pgfqpoint{2.739155in}{1.088825in}}%
\pgfpathlineto{\pgfqpoint{2.741928in}{1.093675in}}%
\pgfpathlineto{\pgfqpoint{2.744510in}{1.093792in}}%
\pgfpathlineto{\pgfqpoint{2.747260in}{1.099695in}}%
\pgfpathlineto{\pgfqpoint{2.749868in}{1.098868in}}%
\pgfpathlineto{\pgfqpoint{2.752614in}{1.098124in}}%
\pgfpathlineto{\pgfqpoint{2.755224in}{1.103682in}}%
\pgfpathlineto{\pgfqpoint{2.758028in}{1.100120in}}%
\pgfpathlineto{\pgfqpoint{2.760581in}{1.096894in}}%
\pgfpathlineto{\pgfqpoint{2.763253in}{1.100824in}}%
\pgfpathlineto{\pgfqpoint{2.765935in}{1.087928in}}%
\pgfpathlineto{\pgfqpoint{2.768617in}{1.092725in}}%
\pgfpathlineto{\pgfqpoint{2.771373in}{1.090223in}}%
\pgfpathlineto{\pgfqpoint{2.773972in}{1.090730in}}%
\pgfpathlineto{\pgfqpoint{2.776767in}{1.095138in}}%
\pgfpathlineto{\pgfqpoint{2.779330in}{1.096313in}}%
\pgfpathlineto{\pgfqpoint{2.782113in}{1.096224in}}%
\pgfpathlineto{\pgfqpoint{2.784687in}{1.094198in}}%
\pgfpathlineto{\pgfqpoint{2.787468in}{1.106877in}}%
\pgfpathlineto{\pgfqpoint{2.790044in}{1.105651in}}%
\pgfpathlineto{\pgfqpoint{2.792721in}{1.102547in}}%
\pgfpathlineto{\pgfqpoint{2.795398in}{1.106751in}}%
\pgfpathlineto{\pgfqpoint{2.798070in}{1.103015in}}%
\pgfpathlineto{\pgfqpoint{2.800756in}{1.110645in}}%
\pgfpathlineto{\pgfqpoint{2.803435in}{1.111840in}}%
\pgfpathlineto{\pgfqpoint{2.806175in}{1.107508in}}%
\pgfpathlineto{\pgfqpoint{2.808792in}{1.101736in}}%
\pgfpathlineto{\pgfqpoint{2.811597in}{1.106773in}}%
\pgfpathlineto{\pgfqpoint{2.814141in}{1.105699in}}%
\pgfpathlineto{\pgfqpoint{2.816867in}{1.104166in}}%
\pgfpathlineto{\pgfqpoint{2.819506in}{1.108515in}}%
\pgfpathlineto{\pgfqpoint{2.822303in}{1.102231in}}%
\pgfpathlineto{\pgfqpoint{2.824851in}{1.105492in}}%
\pgfpathlineto{\pgfqpoint{2.827567in}{1.108707in}}%
\pgfpathlineto{\pgfqpoint{2.830219in}{1.112647in}}%
\pgfpathlineto{\pgfqpoint{2.832894in}{1.110935in}}%
\pgfpathlineto{\pgfqpoint{2.835698in}{1.112032in}}%
\pgfpathlineto{\pgfqpoint{2.838254in}{1.109164in}}%
\pgfpathlineto{\pgfqpoint{2.841055in}{1.108390in}}%
\pgfpathlineto{\pgfqpoint{2.843611in}{1.107393in}}%
\pgfpathlineto{\pgfqpoint{2.846408in}{1.104902in}}%
\pgfpathlineto{\pgfqpoint{2.848960in}{1.105247in}}%
\pgfpathlineto{\pgfqpoint{2.851793in}{1.103995in}}%
\pgfpathlineto{\pgfqpoint{2.854325in}{1.106546in}}%
\pgfpathlineto{\pgfqpoint{2.857003in}{1.106058in}}%
\pgfpathlineto{\pgfqpoint{2.859668in}{1.107035in}}%
\pgfpathlineto{\pgfqpoint{2.862402in}{1.107684in}}%
\pgfpathlineto{\pgfqpoint{2.865031in}{1.107040in}}%
\pgfpathlineto{\pgfqpoint{2.867713in}{1.106420in}}%
\pgfpathlineto{\pgfqpoint{2.870475in}{1.107189in}}%
\pgfpathlineto{\pgfqpoint{2.873074in}{1.109697in}}%
\pgfpathlineto{\pgfqpoint{2.875882in}{1.106322in}}%
\pgfpathlineto{\pgfqpoint{2.878431in}{1.107105in}}%
\pgfpathlineto{\pgfqpoint{2.881254in}{1.115567in}}%
\pgfpathlineto{\pgfqpoint{2.883780in}{1.113469in}}%
\pgfpathlineto{\pgfqpoint{2.886578in}{1.109756in}}%
\pgfpathlineto{\pgfqpoint{2.889145in}{1.110982in}}%
\pgfpathlineto{\pgfqpoint{2.891809in}{1.104715in}}%
\pgfpathlineto{\pgfqpoint{2.894487in}{1.105252in}}%
\pgfpathlineto{\pgfqpoint{2.897179in}{1.114280in}}%
\pgfpathlineto{\pgfqpoint{2.899858in}{1.112837in}}%
\pgfpathlineto{\pgfqpoint{2.902535in}{1.109053in}}%
\pgfpathlineto{\pgfqpoint{2.905341in}{1.108091in}}%
\pgfpathlineto{\pgfqpoint{2.907882in}{1.107025in}}%
\pgfpathlineto{\pgfqpoint{2.910631in}{1.110712in}}%
\pgfpathlineto{\pgfqpoint{2.913243in}{1.120303in}}%
\pgfpathlineto{\pgfqpoint{2.916061in}{1.108765in}}%
\pgfpathlineto{\pgfqpoint{2.918606in}{1.108350in}}%
\pgfpathlineto{\pgfqpoint{2.921363in}{1.110068in}}%
\pgfpathlineto{\pgfqpoint{2.923963in}{1.111134in}}%
\pgfpathlineto{\pgfqpoint{2.926655in}{1.113166in}}%
\pgfpathlineto{\pgfqpoint{2.929321in}{1.114303in}}%
\pgfpathlineto{\pgfqpoint{2.932033in}{1.113281in}}%
\pgfpathlineto{\pgfqpoint{2.934759in}{1.109676in}}%
\pgfpathlineto{\pgfqpoint{2.937352in}{1.108885in}}%
\pgfpathlineto{\pgfqpoint{2.940120in}{1.108221in}}%
\pgfpathlineto{\pgfqpoint{2.942711in}{1.106504in}}%
\pgfpathlineto{\pgfqpoint{2.945461in}{1.104475in}}%
\pgfpathlineto{\pgfqpoint{2.948068in}{1.114432in}}%
\pgfpathlineto{\pgfqpoint{2.950884in}{1.111717in}}%
\pgfpathlineto{\pgfqpoint{2.953422in}{1.111137in}}%
\pgfpathlineto{\pgfqpoint{2.956103in}{1.109658in}}%
\pgfpathlineto{\pgfqpoint{2.958782in}{1.106595in}}%
\pgfpathlineto{\pgfqpoint{2.961460in}{1.107364in}}%
\pgfpathlineto{\pgfqpoint{2.964127in}{1.110177in}}%
\pgfpathlineto{\pgfqpoint{2.966812in}{1.113715in}}%
\pgfpathlineto{\pgfqpoint{2.969599in}{1.110017in}}%
\pgfpathlineto{\pgfqpoint{2.972177in}{1.110466in}}%
\pgfpathlineto{\pgfqpoint{2.974972in}{1.109951in}}%
\pgfpathlineto{\pgfqpoint{2.977517in}{1.108515in}}%
\pgfpathlineto{\pgfqpoint{2.980341in}{1.112912in}}%
\pgfpathlineto{\pgfqpoint{2.982885in}{1.111824in}}%
\pgfpathlineto{\pgfqpoint{2.985666in}{1.111001in}}%
\pgfpathlineto{\pgfqpoint{2.988238in}{1.109762in}}%
\pgfpathlineto{\pgfqpoint{2.990978in}{1.107586in}}%
\pgfpathlineto{\pgfqpoint{2.993595in}{1.109092in}}%
\pgfpathlineto{\pgfqpoint{2.996300in}{1.109373in}}%
\pgfpathlineto{\pgfqpoint{2.999103in}{1.131020in}}%
\pgfpathlineto{\pgfqpoint{3.001635in}{1.125036in}}%
\pgfpathlineto{\pgfqpoint{3.004419in}{1.121021in}}%
\pgfpathlineto{\pgfqpoint{3.006993in}{1.111750in}}%
\pgfpathlineto{\pgfqpoint{3.009784in}{1.110896in}}%
\pgfpathlineto{\pgfqpoint{3.012351in}{1.106521in}}%
\pgfpathlineto{\pgfqpoint{3.015097in}{1.109470in}}%
\pgfpathlineto{\pgfqpoint{3.017707in}{1.108174in}}%
\pgfpathlineto{\pgfqpoint{3.020382in}{1.103773in}}%
\pgfpathlineto{\pgfqpoint{3.023058in}{1.102276in}}%
\pgfpathlineto{\pgfqpoint{3.025803in}{1.108357in}}%
\pgfpathlineto{\pgfqpoint{3.028412in}{1.109909in}}%
\pgfpathlineto{\pgfqpoint{3.031091in}{1.113230in}}%
\pgfpathlineto{\pgfqpoint{3.033921in}{1.113758in}}%
\pgfpathlineto{\pgfqpoint{3.036456in}{1.115650in}}%
\pgfpathlineto{\pgfqpoint{3.039262in}{1.112602in}}%
\pgfpathlineto{\pgfqpoint{3.041813in}{1.113293in}}%
\pgfpathlineto{\pgfqpoint{3.044568in}{1.120798in}}%
\pgfpathlineto{\pgfqpoint{3.047157in}{1.131598in}}%
\pgfpathlineto{\pgfqpoint{3.049988in}{1.137104in}}%
\pgfpathlineto{\pgfqpoint{3.052526in}{1.119641in}}%
\pgfpathlineto{\pgfqpoint{3.055202in}{1.113805in}}%
\pgfpathlineto{\pgfqpoint{3.057884in}{1.111423in}}%
\pgfpathlineto{\pgfqpoint{3.060561in}{1.106947in}}%
\pgfpathlineto{\pgfqpoint{3.063230in}{1.120527in}}%
\pgfpathlineto{\pgfqpoint{3.065916in}{1.117354in}}%
\pgfpathlineto{\pgfqpoint{3.068709in}{1.117895in}}%
\pgfpathlineto{\pgfqpoint{3.071266in}{1.113371in}}%
\pgfpathlineto{\pgfqpoint{3.074056in}{1.113173in}}%
\pgfpathlineto{\pgfqpoint{3.076631in}{1.111058in}}%
\pgfpathlineto{\pgfqpoint{3.079381in}{1.114109in}}%
\pgfpathlineto{\pgfqpoint{3.081990in}{1.123923in}}%
\pgfpathlineto{\pgfqpoint{3.084671in}{1.124415in}}%
\pgfpathlineto{\pgfqpoint{3.087343in}{1.121495in}}%
\pgfpathlineto{\pgfqpoint{3.090023in}{1.121220in}}%
\pgfpathlineto{\pgfqpoint{3.092699in}{1.120751in}}%
\pgfpathlineto{\pgfqpoint{3.095388in}{1.115104in}}%
\pgfpathlineto{\pgfqpoint{3.098163in}{1.117489in}}%
\pgfpathlineto{\pgfqpoint{3.100737in}{1.118727in}}%
\pgfpathlineto{\pgfqpoint{3.103508in}{1.115344in}}%
\pgfpathlineto{\pgfqpoint{3.106094in}{1.113541in}}%
\pgfpathlineto{\pgfqpoint{3.108896in}{1.110456in}}%
\pgfpathlineto{\pgfqpoint{3.111451in}{1.103871in}}%
\pgfpathlineto{\pgfqpoint{3.114242in}{1.104182in}}%
\pgfpathlineto{\pgfqpoint{3.116807in}{1.110114in}}%
\pgfpathlineto{\pgfqpoint{3.119487in}{1.114363in}}%
\pgfpathlineto{\pgfqpoint{3.122163in}{1.097968in}}%
\pgfpathlineto{\pgfqpoint{3.124842in}{1.097408in}}%
\pgfpathlineto{\pgfqpoint{3.127512in}{1.098193in}}%
\pgfpathlineto{\pgfqpoint{3.130199in}{1.102255in}}%
\pgfpathlineto{\pgfqpoint{3.132946in}{1.105115in}}%
\pgfpathlineto{\pgfqpoint{3.135550in}{1.101233in}}%
\pgfpathlineto{\pgfqpoint{3.138375in}{1.095097in}}%
\pgfpathlineto{\pgfqpoint{3.140913in}{1.109389in}}%
\pgfpathlineto{\pgfqpoint{3.143740in}{1.123638in}}%
\pgfpathlineto{\pgfqpoint{3.146271in}{1.114270in}}%
\pgfpathlineto{\pgfqpoint{3.149057in}{1.116053in}}%
\pgfpathlineto{\pgfqpoint{3.151612in}{1.126958in}}%
\pgfpathlineto{\pgfqpoint{3.154327in}{1.092681in}}%
\pgfpathlineto{\pgfqpoint{3.156981in}{1.085182in}}%
\pgfpathlineto{\pgfqpoint{3.159675in}{1.090840in}}%
\pgfpathlineto{\pgfqpoint{3.162474in}{1.093757in}}%
\pgfpathlineto{\pgfqpoint{3.165019in}{1.093717in}}%
\pgfpathlineto{\pgfqpoint{3.167776in}{1.109779in}}%
\pgfpathlineto{\pgfqpoint{3.170375in}{1.105814in}}%
\pgfpathlineto{\pgfqpoint{3.173142in}{1.103712in}}%
\pgfpathlineto{\pgfqpoint{3.175724in}{1.086708in}}%
\pgfpathlineto{\pgfqpoint{3.178525in}{1.085182in}}%
\pgfpathlineto{\pgfqpoint{3.181089in}{1.088669in}}%
\pgfpathlineto{\pgfqpoint{3.183760in}{1.095710in}}%
\pgfpathlineto{\pgfqpoint{3.186440in}{1.100522in}}%
\pgfpathlineto{\pgfqpoint{3.189117in}{1.095821in}}%
\pgfpathlineto{\pgfqpoint{3.191796in}{1.098589in}}%
\pgfpathlineto{\pgfqpoint{3.194508in}{1.101805in}}%
\pgfpathlineto{\pgfqpoint{3.197226in}{1.102109in}}%
\pgfpathlineto{\pgfqpoint{3.199823in}{1.111524in}}%
\pgfpathlineto{\pgfqpoint{3.202562in}{1.108011in}}%
\pgfpathlineto{\pgfqpoint{3.205195in}{1.098266in}}%
\pgfpathlineto{\pgfqpoint{3.207984in}{1.090112in}}%
\pgfpathlineto{\pgfqpoint{3.210545in}{1.094163in}}%
\pgfpathlineto{\pgfqpoint{3.213342in}{1.096137in}}%
\pgfpathlineto{\pgfqpoint{3.215908in}{1.092303in}}%
\pgfpathlineto{\pgfqpoint{3.218586in}{1.099805in}}%
\pgfpathlineto{\pgfqpoint{3.221255in}{1.101387in}}%
\pgfpathlineto{\pgfqpoint{3.223942in}{1.096734in}}%
\pgfpathlineto{\pgfqpoint{3.226609in}{1.097818in}}%
\pgfpathlineto{\pgfqpoint{3.229310in}{1.096881in}}%
\pgfpathlineto{\pgfqpoint{3.232069in}{1.092486in}}%
\pgfpathlineto{\pgfqpoint{3.234658in}{1.096792in}}%
\pgfpathlineto{\pgfqpoint{3.237411in}{1.103176in}}%
\pgfpathlineto{\pgfqpoint{3.240010in}{1.100708in}}%
\pgfpathlineto{\pgfqpoint{3.242807in}{1.101332in}}%
\pgfpathlineto{\pgfqpoint{3.245363in}{1.102794in}}%
\pgfpathlineto{\pgfqpoint{3.248049in}{1.097902in}}%
\pgfpathlineto{\pgfqpoint{3.250716in}{1.098187in}}%
\pgfpathlineto{\pgfqpoint{3.253404in}{1.104547in}}%
\pgfpathlineto{\pgfqpoint{3.256083in}{1.105235in}}%
\pgfpathlineto{\pgfqpoint{3.258784in}{1.106984in}}%
\pgfpathlineto{\pgfqpoint{3.261594in}{1.104420in}}%
\pgfpathlineto{\pgfqpoint{3.264119in}{1.101728in}}%
\pgfpathlineto{\pgfqpoint{3.266849in}{1.099392in}}%
\pgfpathlineto{\pgfqpoint{3.269478in}{1.097340in}}%
\pgfpathlineto{\pgfqpoint{3.272254in}{1.090216in}}%
\pgfpathlineto{\pgfqpoint{3.274831in}{1.093741in}}%
\pgfpathlineto{\pgfqpoint{3.277603in}{1.100816in}}%
\pgfpathlineto{\pgfqpoint{3.280189in}{1.103879in}}%
\pgfpathlineto{\pgfqpoint{3.282870in}{1.100153in}}%
\pgfpathlineto{\pgfqpoint{3.285534in}{1.105725in}}%
\pgfpathlineto{\pgfqpoint{3.288225in}{1.103649in}}%
\pgfpathlineto{\pgfqpoint{3.290890in}{1.110823in}}%
\pgfpathlineto{\pgfqpoint{3.293574in}{1.103150in}}%
\pgfpathlineto{\pgfqpoint{3.296376in}{1.100162in}}%
\pgfpathlineto{\pgfqpoint{3.298937in}{1.099734in}}%
\pgfpathlineto{\pgfqpoint{3.301719in}{1.100227in}}%
\pgfpathlineto{\pgfqpoint{3.304295in}{1.099189in}}%
\pgfpathlineto{\pgfqpoint{3.307104in}{1.100978in}}%
\pgfpathlineto{\pgfqpoint{3.309652in}{1.104375in}}%
\pgfpathlineto{\pgfqpoint{3.312480in}{1.107148in}}%
\pgfpathlineto{\pgfqpoint{3.315008in}{1.103549in}}%
\pgfpathlineto{\pgfqpoint{3.317688in}{1.107378in}}%
\pgfpathlineto{\pgfqpoint{3.320366in}{1.106064in}}%
\pgfpathlineto{\pgfqpoint{3.323049in}{1.106022in}}%
\pgfpathlineto{\pgfqpoint{3.325860in}{1.106595in}}%
\pgfpathlineto{\pgfqpoint{3.328401in}{1.110545in}}%
\pgfpathlineto{\pgfqpoint{3.331183in}{1.106416in}}%
\pgfpathlineto{\pgfqpoint{3.333758in}{1.106476in}}%
\pgfpathlineto{\pgfqpoint{3.336541in}{1.100908in}}%
\pgfpathlineto{\pgfqpoint{3.339101in}{1.104864in}}%
\pgfpathlineto{\pgfqpoint{3.341893in}{1.107371in}}%
\pgfpathlineto{\pgfqpoint{3.344468in}{1.106328in}}%
\pgfpathlineto{\pgfqpoint{3.347139in}{1.107829in}}%
\pgfpathlineto{\pgfqpoint{3.349828in}{1.100295in}}%
\pgfpathlineto{\pgfqpoint{3.352505in}{1.103816in}}%
\pgfpathlineto{\pgfqpoint{3.355177in}{1.109556in}}%
\pgfpathlineto{\pgfqpoint{3.357862in}{1.107654in}}%
\pgfpathlineto{\pgfqpoint{3.360620in}{1.111151in}}%
\pgfpathlineto{\pgfqpoint{3.363221in}{1.103609in}}%
\pgfpathlineto{\pgfqpoint{3.365996in}{1.106450in}}%
\pgfpathlineto{\pgfqpoint{3.368577in}{1.107407in}}%
\pgfpathlineto{\pgfqpoint{3.371357in}{1.106507in}}%
\pgfpathlineto{\pgfqpoint{3.373921in}{1.108318in}}%
\pgfpathlineto{\pgfqpoint{3.376735in}{1.107959in}}%
\pgfpathlineto{\pgfqpoint{3.379290in}{1.108135in}}%
\pgfpathlineto{\pgfqpoint{3.381959in}{1.109261in}}%
\pgfpathlineto{\pgfqpoint{3.384647in}{1.107341in}}%
\pgfpathlineto{\pgfqpoint{3.387309in}{1.106810in}}%
\pgfpathlineto{\pgfqpoint{3.390102in}{1.103687in}}%
\pgfpathlineto{\pgfqpoint{3.392681in}{1.104691in}}%
\pgfpathlineto{\pgfqpoint{3.395461in}{1.107968in}}%
\pgfpathlineto{\pgfqpoint{3.398037in}{1.109612in}}%
\pgfpathlineto{\pgfqpoint{3.400783in}{1.111570in}}%
\pgfpathlineto{\pgfqpoint{3.403394in}{1.106609in}}%
\pgfpathlineto{\pgfqpoint{3.406202in}{1.107869in}}%
\pgfpathlineto{\pgfqpoint{3.408752in}{1.105939in}}%
\pgfpathlineto{\pgfqpoint{3.411431in}{1.106299in}}%
\pgfpathlineto{\pgfqpoint{3.414109in}{1.105483in}}%
\pgfpathlineto{\pgfqpoint{3.416780in}{1.108282in}}%
\pgfpathlineto{\pgfqpoint{3.419455in}{1.105714in}}%
\pgfpathlineto{\pgfqpoint{3.422142in}{1.112606in}}%
\pgfpathlineto{\pgfqpoint{3.424887in}{1.109569in}}%
\pgfpathlineto{\pgfqpoint{3.427501in}{1.107864in}}%
\pgfpathlineto{\pgfqpoint{3.430313in}{1.110631in}}%
\pgfpathlineto{\pgfqpoint{3.432851in}{1.108995in}}%
\pgfpathlineto{\pgfqpoint{3.435635in}{1.107304in}}%
\pgfpathlineto{\pgfqpoint{3.438210in}{1.110207in}}%
\pgfpathlineto{\pgfqpoint{3.440996in}{1.112517in}}%
\pgfpathlineto{\pgfqpoint{3.443574in}{1.112135in}}%
\pgfpathlineto{\pgfqpoint{3.446257in}{1.111550in}}%
\pgfpathlineto{\pgfqpoint{3.448926in}{1.111325in}}%
\pgfpathlineto{\pgfqpoint{3.451597in}{1.107035in}}%
\pgfpathlineto{\pgfqpoint{3.454285in}{1.109872in}}%
\pgfpathlineto{\pgfqpoint{3.456960in}{1.108022in}}%
\pgfpathlineto{\pgfqpoint{3.459695in}{1.107053in}}%
\pgfpathlineto{\pgfqpoint{3.462321in}{1.104647in}}%
\pgfpathlineto{\pgfqpoint{3.465072in}{1.106154in}}%
\pgfpathlineto{\pgfqpoint{3.467678in}{1.104504in}}%
\pgfpathlineto{\pgfqpoint{3.470466in}{1.104762in}}%
\pgfpathlineto{\pgfqpoint{3.473021in}{1.104676in}}%
\pgfpathlineto{\pgfqpoint{3.475821in}{1.107480in}}%
\pgfpathlineto{\pgfqpoint{3.478378in}{1.108913in}}%
\pgfpathlineto{\pgfqpoint{3.481072in}{1.107693in}}%
\pgfpathlineto{\pgfqpoint{3.483744in}{1.112207in}}%
\pgfpathlineto{\pgfqpoint{3.486442in}{1.110648in}}%
\pgfpathlineto{\pgfqpoint{3.489223in}{1.113475in}}%
\pgfpathlineto{\pgfqpoint{3.491783in}{1.109061in}}%
\pgfpathlineto{\pgfqpoint{3.494581in}{1.105152in}}%
\pgfpathlineto{\pgfqpoint{3.497139in}{1.114115in}}%
\pgfpathlineto{\pgfqpoint{3.499909in}{1.113263in}}%
\pgfpathlineto{\pgfqpoint{3.502488in}{1.113344in}}%
\pgfpathlineto{\pgfqpoint{3.505262in}{1.116233in}}%
\pgfpathlineto{\pgfqpoint{3.507840in}{1.110212in}}%
\pgfpathlineto{\pgfqpoint{3.510533in}{1.105380in}}%
\pgfpathlineto{\pgfqpoint{3.513209in}{1.108731in}}%
\pgfpathlineto{\pgfqpoint{3.515884in}{1.108810in}}%
\pgfpathlineto{\pgfqpoint{3.518565in}{1.112860in}}%
\pgfpathlineto{\pgfqpoint{3.521244in}{1.110363in}}%
\pgfpathlineto{\pgfqpoint{3.524041in}{1.111529in}}%
\pgfpathlineto{\pgfqpoint{3.526601in}{1.109983in}}%
\pgfpathlineto{\pgfqpoint{3.529327in}{1.110207in}}%
\pgfpathlineto{\pgfqpoint{3.531955in}{1.111148in}}%
\pgfpathlineto{\pgfqpoint{3.534783in}{1.109896in}}%
\pgfpathlineto{\pgfqpoint{3.537309in}{1.110871in}}%
\pgfpathlineto{\pgfqpoint{3.540093in}{1.108791in}}%
\pgfpathlineto{\pgfqpoint{3.542656in}{1.109705in}}%
\pgfpathlineto{\pgfqpoint{3.545349in}{1.108386in}}%
\pgfpathlineto{\pgfqpoint{3.548029in}{1.105684in}}%
\pgfpathlineto{\pgfqpoint{3.550713in}{1.102469in}}%
\pgfpathlineto{\pgfqpoint{3.553498in}{1.106486in}}%
\pgfpathlineto{\pgfqpoint{3.556061in}{1.113256in}}%
\pgfpathlineto{\pgfqpoint{3.558853in}{1.113086in}}%
\pgfpathlineto{\pgfqpoint{3.561420in}{1.107824in}}%
\pgfpathlineto{\pgfqpoint{3.564188in}{1.111435in}}%
\pgfpathlineto{\pgfqpoint{3.566774in}{1.108623in}}%
\pgfpathlineto{\pgfqpoint{3.569584in}{1.112148in}}%
\pgfpathlineto{\pgfqpoint{3.572126in}{1.108296in}}%
\pgfpathlineto{\pgfqpoint{3.574814in}{1.105997in}}%
\pgfpathlineto{\pgfqpoint{3.577487in}{1.105168in}}%
\pgfpathlineto{\pgfqpoint{3.580191in}{1.108822in}}%
\pgfpathlineto{\pgfqpoint{3.582851in}{1.107694in}}%
\pgfpathlineto{\pgfqpoint{3.585532in}{1.106154in}}%
\pgfpathlineto{\pgfqpoint{3.588258in}{1.107805in}}%
\pgfpathlineto{\pgfqpoint{3.590883in}{1.107816in}}%
\pgfpathlineto{\pgfqpoint{3.593620in}{1.102309in}}%
\pgfpathlineto{\pgfqpoint{3.596240in}{1.108521in}}%
\pgfpathlineto{\pgfqpoint{3.598998in}{1.108746in}}%
\pgfpathlineto{\pgfqpoint{3.601590in}{1.107868in}}%
\pgfpathlineto{\pgfqpoint{3.604387in}{1.110548in}}%
\pgfpathlineto{\pgfqpoint{3.606951in}{1.105017in}}%
\pgfpathlineto{\pgfqpoint{3.609632in}{1.109785in}}%
\pgfpathlineto{\pgfqpoint{3.612311in}{1.110554in}}%
\pgfpathlineto{\pgfqpoint{3.614982in}{1.108995in}}%
\pgfpathlineto{\pgfqpoint{3.617667in}{1.110565in}}%
\pgfpathlineto{\pgfqpoint{3.620345in}{1.113581in}}%
\pgfpathlineto{\pgfqpoint{3.623165in}{1.112044in}}%
\pgfpathlineto{\pgfqpoint{3.625689in}{1.110556in}}%
\pgfpathlineto{\pgfqpoint{3.628460in}{1.111412in}}%
\pgfpathlineto{\pgfqpoint{3.631058in}{1.106978in}}%
\pgfpathlineto{\pgfqpoint{3.633858in}{1.111723in}}%
\pgfpathlineto{\pgfqpoint{3.636413in}{1.112201in}}%
\pgfpathlineto{\pgfqpoint{3.639207in}{1.109081in}}%
\pgfpathlineto{\pgfqpoint{3.641773in}{1.109234in}}%
\pgfpathlineto{\pgfqpoint{3.644452in}{1.100569in}}%
\pgfpathlineto{\pgfqpoint{3.647130in}{1.090234in}}%
\pgfpathlineto{\pgfqpoint{3.649837in}{1.096045in}}%
\pgfpathlineto{\pgfqpoint{3.652628in}{1.101923in}}%
\pgfpathlineto{\pgfqpoint{3.655165in}{1.104427in}}%
\pgfpathlineto{\pgfqpoint{3.657917in}{1.121886in}}%
\pgfpathlineto{\pgfqpoint{3.660515in}{1.140362in}}%
\pgfpathlineto{\pgfqpoint{3.663276in}{1.122108in}}%
\pgfpathlineto{\pgfqpoint{3.665864in}{1.103038in}}%
\pgfpathlineto{\pgfqpoint{3.668665in}{1.100867in}}%
\pgfpathlineto{\pgfqpoint{3.671232in}{1.102903in}}%
\pgfpathlineto{\pgfqpoint{3.673911in}{1.103005in}}%
\pgfpathlineto{\pgfqpoint{3.676591in}{1.106830in}}%
\pgfpathlineto{\pgfqpoint{3.679273in}{1.115884in}}%
\pgfpathlineto{\pgfqpoint{3.681948in}{1.115645in}}%
\pgfpathlineto{\pgfqpoint{3.684620in}{1.114773in}}%
\pgfpathlineto{\pgfqpoint{3.687442in}{1.112244in}}%
\pgfpathlineto{\pgfqpoint{3.689983in}{1.110340in}}%
\pgfpathlineto{\pgfqpoint{3.692765in}{1.104860in}}%
\pgfpathlineto{\pgfqpoint{3.695331in}{1.103902in}}%
\pgfpathlineto{\pgfqpoint{3.698125in}{1.105942in}}%
\pgfpathlineto{\pgfqpoint{3.700684in}{1.106863in}}%
\pgfpathlineto{\pgfqpoint{3.703460in}{1.107265in}}%
\pgfpathlineto{\pgfqpoint{3.706053in}{1.107904in}}%
\pgfpathlineto{\pgfqpoint{3.708729in}{1.107465in}}%
\pgfpathlineto{\pgfqpoint{3.711410in}{1.110174in}}%
\pgfpathlineto{\pgfqpoint{3.714086in}{1.111425in}}%
\pgfpathlineto{\pgfqpoint{3.716875in}{1.107696in}}%
\pgfpathlineto{\pgfqpoint{3.719446in}{1.122268in}}%
\pgfpathlineto{\pgfqpoint{3.722228in}{1.159568in}}%
\pgfpathlineto{\pgfqpoint{3.724804in}{1.139803in}}%
\pgfpathlineto{\pgfqpoint{3.727581in}{1.139397in}}%
\pgfpathlineto{\pgfqpoint{3.730158in}{1.132167in}}%
\pgfpathlineto{\pgfqpoint{3.732950in}{1.130759in}}%
\pgfpathlineto{\pgfqpoint{3.735509in}{1.123036in}}%
\pgfpathlineto{\pgfqpoint{3.738194in}{1.121948in}}%
\pgfpathlineto{\pgfqpoint{3.740874in}{1.120001in}}%
\pgfpathlineto{\pgfqpoint{3.743548in}{1.120140in}}%
\pgfpathlineto{\pgfqpoint{3.746229in}{1.112956in}}%
\pgfpathlineto{\pgfqpoint{3.748903in}{1.120274in}}%
\pgfpathlineto{\pgfqpoint{3.751728in}{1.121014in}}%
\pgfpathlineto{\pgfqpoint{3.754265in}{1.123140in}}%
\pgfpathlineto{\pgfqpoint{3.757065in}{1.124946in}}%
\pgfpathlineto{\pgfqpoint{3.759622in}{1.123881in}}%
\pgfpathlineto{\pgfqpoint{3.762389in}{1.117625in}}%
\pgfpathlineto{\pgfqpoint{3.764966in}{1.124944in}}%
\pgfpathlineto{\pgfqpoint{3.767782in}{1.118397in}}%
\pgfpathlineto{\pgfqpoint{3.770323in}{1.126528in}}%
\pgfpathlineto{\pgfqpoint{3.773014in}{1.113396in}}%
\pgfpathlineto{\pgfqpoint{3.775691in}{1.112196in}}%
\pgfpathlineto{\pgfqpoint{3.778370in}{1.096740in}}%
\pgfpathlineto{\pgfqpoint{3.781046in}{1.085182in}}%
\pgfpathlineto{\pgfqpoint{3.783725in}{1.091082in}}%
\pgfpathlineto{\pgfqpoint{3.786504in}{1.095348in}}%
\pgfpathlineto{\pgfqpoint{3.789084in}{1.100167in}}%
\pgfpathlineto{\pgfqpoint{3.791897in}{1.103095in}}%
\pgfpathlineto{\pgfqpoint{3.794435in}{1.101204in}}%
\pgfpathlineto{\pgfqpoint{3.797265in}{1.098207in}}%
\pgfpathlineto{\pgfqpoint{3.799797in}{1.096926in}}%
\pgfpathlineto{\pgfqpoint{3.802569in}{1.101158in}}%
\pgfpathlineto{\pgfqpoint{3.805145in}{1.105397in}}%
\pgfpathlineto{\pgfqpoint{3.807832in}{1.098776in}}%
\pgfpathlineto{\pgfqpoint{3.810510in}{1.098326in}}%
\pgfpathlineto{\pgfqpoint{3.813172in}{1.098557in}}%
\pgfpathlineto{\pgfqpoint{3.815983in}{1.101367in}}%
\pgfpathlineto{\pgfqpoint{3.818546in}{1.097422in}}%
\pgfpathlineto{\pgfqpoint{3.821315in}{1.099701in}}%
\pgfpathlineto{\pgfqpoint{3.823903in}{1.098762in}}%
\pgfpathlineto{\pgfqpoint{3.826679in}{1.108872in}}%
\pgfpathlineto{\pgfqpoint{3.829252in}{1.106632in}}%
\pgfpathlineto{\pgfqpoint{3.832053in}{1.107486in}}%
\pgfpathlineto{\pgfqpoint{3.834616in}{1.107083in}}%
\pgfpathlineto{\pgfqpoint{3.837286in}{1.107413in}}%
\pgfpathlineto{\pgfqpoint{3.839960in}{1.110470in}}%
\pgfpathlineto{\pgfqpoint{3.842641in}{1.109298in}}%
\pgfpathlineto{\pgfqpoint{3.845329in}{1.108935in}}%
\pgfpathlineto{\pgfqpoint{3.848005in}{1.109478in}}%
\pgfpathlineto{\pgfqpoint{3.850814in}{1.101292in}}%
\pgfpathlineto{\pgfqpoint{3.853358in}{1.107753in}}%
\pgfpathlineto{\pgfqpoint{3.856100in}{1.101993in}}%
\pgfpathlineto{\pgfqpoint{3.858720in}{1.103544in}}%
\pgfpathlineto{\pgfqpoint{3.861561in}{1.098977in}}%
\pgfpathlineto{\pgfqpoint{3.864073in}{1.107427in}}%
\pgfpathlineto{\pgfqpoint{3.866815in}{1.105778in}}%
\pgfpathlineto{\pgfqpoint{3.869435in}{1.103715in}}%
\pgfpathlineto{\pgfqpoint{3.872114in}{1.103785in}}%
\pgfpathlineto{\pgfqpoint{3.874790in}{1.104211in}}%
\pgfpathlineto{\pgfqpoint{3.877466in}{1.104407in}}%
\pgfpathlineto{\pgfqpoint{3.880237in}{1.104670in}}%
\pgfpathlineto{\pgfqpoint{3.882850in}{1.102828in}}%
\pgfpathlineto{\pgfqpoint{3.885621in}{1.102494in}}%
\pgfpathlineto{\pgfqpoint{3.888188in}{1.103179in}}%
\pgfpathlineto{\pgfqpoint{3.890926in}{1.107340in}}%
\pgfpathlineto{\pgfqpoint{3.893541in}{1.110170in}}%
\pgfpathlineto{\pgfqpoint{3.896345in}{1.101919in}}%
\pgfpathlineto{\pgfqpoint{3.898891in}{1.099093in}}%
\pgfpathlineto{\pgfqpoint{3.901573in}{1.100957in}}%
\pgfpathlineto{\pgfqpoint{3.904252in}{1.101428in}}%
\pgfpathlineto{\pgfqpoint{3.906918in}{1.102237in}}%
\pgfpathlineto{\pgfqpoint{3.909602in}{1.100276in}}%
\pgfpathlineto{\pgfqpoint{3.912296in}{1.106927in}}%
\pgfpathlineto{\pgfqpoint{3.915107in}{1.101277in}}%
\pgfpathlineto{\pgfqpoint{3.917646in}{1.097610in}}%
\pgfpathlineto{\pgfqpoint{3.920412in}{1.098467in}}%
\pgfpathlineto{\pgfqpoint{3.923005in}{1.103386in}}%
\pgfpathlineto{\pgfqpoint{3.925778in}{1.105776in}}%
\pgfpathlineto{\pgfqpoint{3.928347in}{1.104282in}}%
\pgfpathlineto{\pgfqpoint{3.931202in}{1.107382in}}%
\pgfpathlineto{\pgfqpoint{3.933714in}{1.105538in}}%
\pgfpathlineto{\pgfqpoint{3.936395in}{1.104144in}}%
\pgfpathlineto{\pgfqpoint{3.939075in}{1.100954in}}%
\pgfpathlineto{\pgfqpoint{3.941778in}{1.086695in}}%
\pgfpathlineto{\pgfqpoint{3.944431in}{1.093684in}}%
\pgfpathlineto{\pgfqpoint{3.947101in}{1.094055in}}%
\pgfpathlineto{\pgfqpoint{3.949894in}{1.097567in}}%
\pgfpathlineto{\pgfqpoint{3.952464in}{1.095518in}}%
\pgfpathlineto{\pgfqpoint{3.955211in}{1.099665in}}%
\pgfpathlineto{\pgfqpoint{3.957823in}{1.100880in}}%
\pgfpathlineto{\pgfqpoint{3.960635in}{1.102200in}}%
\pgfpathlineto{\pgfqpoint{3.963176in}{1.108569in}}%
\pgfpathlineto{\pgfqpoint{3.966013in}{1.108691in}}%
\pgfpathlineto{\pgfqpoint{3.968523in}{1.107046in}}%
\pgfpathlineto{\pgfqpoint{3.971250in}{1.110916in}}%
\pgfpathlineto{\pgfqpoint{3.973885in}{1.101732in}}%
\pgfpathlineto{\pgfqpoint{3.976563in}{1.098129in}}%
\pgfpathlineto{\pgfqpoint{3.979389in}{1.104572in}}%
\pgfpathlineto{\pgfqpoint{3.981929in}{1.106161in}}%
\pgfpathlineto{\pgfqpoint{3.984714in}{1.099159in}}%
\pgfpathlineto{\pgfqpoint{3.987270in}{1.101758in}}%
\pgfpathlineto{\pgfqpoint{3.990055in}{1.105789in}}%
\pgfpathlineto{\pgfqpoint{3.992642in}{1.099267in}}%
\pgfpathlineto{\pgfqpoint{3.995417in}{1.098993in}}%
\pgfpathlineto{\pgfqpoint{3.997990in}{1.105391in}}%
\pgfpathlineto{\pgfqpoint{4.000674in}{1.106604in}}%
\pgfpathlineto{\pgfqpoint{4.003348in}{1.104610in}}%
\pgfpathlineto{\pgfqpoint{4.006034in}{1.100092in}}%
\pgfpathlineto{\pgfqpoint{4.008699in}{1.099234in}}%
\pgfpathlineto{\pgfqpoint{4.011394in}{1.103764in}}%
\pgfpathlineto{\pgfqpoint{4.014186in}{1.105344in}}%
\pgfpathlineto{\pgfqpoint{4.016744in}{1.106256in}}%
\pgfpathlineto{\pgfqpoint{4.019518in}{1.100779in}}%
\pgfpathlineto{\pgfqpoint{4.022097in}{1.107150in}}%
\pgfpathlineto{\pgfqpoint{4.024868in}{1.099529in}}%
\pgfpathlineto{\pgfqpoint{4.027447in}{1.105954in}}%
\pgfpathlineto{\pgfqpoint{4.030229in}{1.103686in}}%
\pgfpathlineto{\pgfqpoint{4.032817in}{1.110285in}}%
\pgfpathlineto{\pgfqpoint{4.035492in}{1.108407in}}%
\pgfpathlineto{\pgfqpoint{4.038174in}{1.108224in}}%
\pgfpathlineto{\pgfqpoint{4.040852in}{1.108099in}}%
\pgfpathlineto{\pgfqpoint{4.043667in}{1.110283in}}%
\pgfpathlineto{\pgfqpoint{4.046210in}{1.108494in}}%
\pgfpathlineto{\pgfqpoint{4.049006in}{1.111782in}}%
\pgfpathlineto{\pgfqpoint{4.051557in}{1.103363in}}%
\pgfpathlineto{\pgfqpoint{4.054326in}{1.103748in}}%
\pgfpathlineto{\pgfqpoint{4.056911in}{1.106137in}}%
\pgfpathlineto{\pgfqpoint{4.059702in}{1.111556in}}%
\pgfpathlineto{\pgfqpoint{4.062266in}{1.107248in}}%
\pgfpathlineto{\pgfqpoint{4.064957in}{1.109286in}}%
\pgfpathlineto{\pgfqpoint{4.067636in}{1.107719in}}%
\pgfpathlineto{\pgfqpoint{4.070313in}{1.108792in}}%
\pgfpathlineto{\pgfqpoint{4.072985in}{1.107827in}}%
\pgfpathlineto{\pgfqpoint{4.075705in}{1.110104in}}%
\pgfpathlineto{\pgfqpoint{4.078471in}{1.100823in}}%
\pgfpathlineto{\pgfqpoint{4.081018in}{1.105746in}}%
\pgfpathlineto{\pgfqpoint{4.083870in}{1.107214in}}%
\pgfpathlineto{\pgfqpoint{4.086385in}{1.103792in}}%
\pgfpathlineto{\pgfqpoint{4.089159in}{1.100225in}}%
\pgfpathlineto{\pgfqpoint{4.091729in}{1.106548in}}%
\pgfpathlineto{\pgfqpoint{4.094527in}{1.100495in}}%
\pgfpathlineto{\pgfqpoint{4.097092in}{1.098872in}}%
\pgfpathlineto{\pgfqpoint{4.099777in}{1.101088in}}%
\pgfpathlineto{\pgfqpoint{4.102456in}{1.102605in}}%
\pgfpathlineto{\pgfqpoint{4.105185in}{1.100883in}}%
\pgfpathlineto{\pgfqpoint{4.107814in}{1.095542in}}%
\pgfpathlineto{\pgfqpoint{4.110488in}{1.101751in}}%
\pgfpathlineto{\pgfqpoint{4.113252in}{1.099502in}}%
\pgfpathlineto{\pgfqpoint{4.115844in}{1.105701in}}%
\pgfpathlineto{\pgfqpoint{4.118554in}{1.106138in}}%
\pgfpathlineto{\pgfqpoint{4.121205in}{1.104517in}}%
\pgfpathlineto{\pgfqpoint{4.124019in}{1.100967in}}%
\pgfpathlineto{\pgfqpoint{4.126553in}{1.098656in}}%
\pgfpathlineto{\pgfqpoint{4.129349in}{1.100011in}}%
\pgfpathlineto{\pgfqpoint{4.131920in}{1.096525in}}%
\pgfpathlineto{\pgfqpoint{4.134615in}{1.102588in}}%
\pgfpathlineto{\pgfqpoint{4.137272in}{1.103068in}}%
\pgfpathlineto{\pgfqpoint{4.139963in}{1.103025in}}%
\pgfpathlineto{\pgfqpoint{4.142713in}{1.099517in}}%
\pgfpathlineto{\pgfqpoint{4.145310in}{1.100027in}}%
\pgfpathlineto{\pgfqpoint{4.148082in}{1.100447in}}%
\pgfpathlineto{\pgfqpoint{4.150665in}{1.103226in}}%
\pgfpathlineto{\pgfqpoint{4.153423in}{1.106674in}}%
\pgfpathlineto{\pgfqpoint{4.156016in}{1.103126in}}%
\pgfpathlineto{\pgfqpoint{4.158806in}{1.103220in}}%
\pgfpathlineto{\pgfqpoint{4.161380in}{1.106673in}}%
\pgfpathlineto{\pgfqpoint{4.164059in}{1.105610in}}%
\pgfpathlineto{\pgfqpoint{4.166737in}{1.100280in}}%
\pgfpathlineto{\pgfqpoint{4.169415in}{1.101300in}}%
\pgfpathlineto{\pgfqpoint{4.172093in}{1.101090in}}%
\pgfpathlineto{\pgfqpoint{4.174770in}{1.101885in}}%
\pgfpathlineto{\pgfqpoint{4.177593in}{1.101098in}}%
\pgfpathlineto{\pgfqpoint{4.180129in}{1.096071in}}%
\pgfpathlineto{\pgfqpoint{4.182899in}{1.094568in}}%
\pgfpathlineto{\pgfqpoint{4.185481in}{1.094939in}}%
\pgfpathlineto{\pgfqpoint{4.188318in}{1.093219in}}%
\pgfpathlineto{\pgfqpoint{4.190842in}{1.094521in}}%
\pgfpathlineto{\pgfqpoint{4.193638in}{1.095570in}}%
\pgfpathlineto{\pgfqpoint{4.196186in}{1.097827in}}%
\pgfpathlineto{\pgfqpoint{4.198878in}{1.095277in}}%
\pgfpathlineto{\pgfqpoint{4.201542in}{1.099684in}}%
\pgfpathlineto{\pgfqpoint{4.204240in}{1.097752in}}%
\pgfpathlineto{\pgfqpoint{4.207076in}{1.098211in}}%
\pgfpathlineto{\pgfqpoint{4.209597in}{1.096610in}}%
\pgfpathlineto{\pgfqpoint{4.212383in}{1.099920in}}%
\pgfpathlineto{\pgfqpoint{4.214948in}{1.097612in}}%
\pgfpathlineto{\pgfqpoint{4.217694in}{1.098824in}}%
\pgfpathlineto{\pgfqpoint{4.220304in}{1.099787in}}%
\pgfpathlineto{\pgfqpoint{4.223082in}{1.098134in}}%
\pgfpathlineto{\pgfqpoint{4.225654in}{1.103837in}}%
\pgfpathlineto{\pgfqpoint{4.228331in}{1.105784in}}%
\pgfpathlineto{\pgfqpoint{4.231013in}{1.106733in}}%
\pgfpathlineto{\pgfqpoint{4.233691in}{1.105536in}}%
\pgfpathlineto{\pgfqpoint{4.236375in}{1.102618in}}%
\pgfpathlineto{\pgfqpoint{4.239084in}{1.101281in}}%
\pgfpathlineto{\pgfqpoint{4.241900in}{1.106470in}}%
\pgfpathlineto{\pgfqpoint{4.244394in}{1.103362in}}%
\pgfpathlineto{\pgfqpoint{4.247225in}{1.106559in}}%
\pgfpathlineto{\pgfqpoint{4.249767in}{1.108320in}}%
\pgfpathlineto{\pgfqpoint{4.252581in}{1.109618in}}%
\pgfpathlineto{\pgfqpoint{4.255120in}{1.107262in}}%
\pgfpathlineto{\pgfqpoint{4.257958in}{1.106411in}}%
\pgfpathlineto{\pgfqpoint{4.260477in}{1.106189in}}%
\pgfpathlineto{\pgfqpoint{4.263157in}{1.110092in}}%
\pgfpathlineto{\pgfqpoint{4.265824in}{1.107998in}}%
\pgfpathlineto{\pgfqpoint{4.268590in}{1.108599in}}%
\pgfpathlineto{\pgfqpoint{4.271187in}{1.103050in}}%
\pgfpathlineto{\pgfqpoint{4.273874in}{1.103068in}}%
\pgfpathlineto{\pgfqpoint{4.276635in}{1.103852in}}%
\pgfpathlineto{\pgfqpoint{4.279212in}{1.101663in}}%
\pgfpathlineto{\pgfqpoint{4.282000in}{1.113739in}}%
\pgfpathlineto{\pgfqpoint{4.284586in}{1.114695in}}%
\pgfpathlineto{\pgfqpoint{4.287399in}{1.102760in}}%
\pgfpathlineto{\pgfqpoint{4.289936in}{1.105916in}}%
\pgfpathlineto{\pgfqpoint{4.292786in}{1.105599in}}%
\pgfpathlineto{\pgfqpoint{4.295299in}{1.102497in}}%
\pgfpathlineto{\pgfqpoint{4.297977in}{1.111761in}}%
\pgfpathlineto{\pgfqpoint{4.300656in}{1.110279in}}%
\pgfpathlineto{\pgfqpoint{4.303357in}{1.112485in}}%
\pgfpathlineto{\pgfqpoint{4.306118in}{1.108646in}}%
\pgfpathlineto{\pgfqpoint{4.308691in}{1.111371in}}%
\pgfpathlineto{\pgfqpoint{4.311494in}{1.112149in}}%
\pgfpathlineto{\pgfqpoint{4.314032in}{1.110571in}}%
\pgfpathlineto{\pgfqpoint{4.316856in}{1.110279in}}%
\pgfpathlineto{\pgfqpoint{4.319405in}{1.113727in}}%
\pgfpathlineto{\pgfqpoint{4.322181in}{1.110658in}}%
\pgfpathlineto{\pgfqpoint{4.324760in}{1.113264in}}%
\pgfpathlineto{\pgfqpoint{4.327440in}{1.112869in}}%
\pgfpathlineto{\pgfqpoint{4.330118in}{1.112828in}}%
\pgfpathlineto{\pgfqpoint{4.332796in}{1.105838in}}%
\pgfpathlineto{\pgfqpoint{4.335463in}{1.105120in}}%
\pgfpathlineto{\pgfqpoint{4.338154in}{1.109460in}}%
\pgfpathlineto{\pgfqpoint{4.340976in}{1.110201in}}%
\pgfpathlineto{\pgfqpoint{4.343510in}{1.110519in}}%
\pgfpathlineto{\pgfqpoint{4.346263in}{1.109697in}}%
\pgfpathlineto{\pgfqpoint{4.348868in}{1.110773in}}%
\pgfpathlineto{\pgfqpoint{4.351645in}{1.111173in}}%
\pgfpathlineto{\pgfqpoint{4.354224in}{1.112216in}}%
\pgfpathlineto{\pgfqpoint{4.357014in}{1.106800in}}%
\pgfpathlineto{\pgfqpoint{4.359582in}{1.111209in}}%
\pgfpathlineto{\pgfqpoint{4.362270in}{1.107953in}}%
\pgfpathlineto{\pgfqpoint{4.364936in}{1.111435in}}%
\pgfpathlineto{\pgfqpoint{4.367646in}{1.109242in}}%
\pgfpathlineto{\pgfqpoint{4.370437in}{1.114454in}}%
\pgfpathlineto{\pgfqpoint{4.372976in}{1.113936in}}%
\pgfpathlineto{\pgfqpoint{4.375761in}{1.109522in}}%
\pgfpathlineto{\pgfqpoint{4.378329in}{1.110717in}}%
\pgfpathlineto{\pgfqpoint{4.381097in}{1.109824in}}%
\pgfpathlineto{\pgfqpoint{4.383674in}{1.110343in}}%
\pgfpathlineto{\pgfqpoint{4.386431in}{1.111248in}}%
\pgfpathlineto{\pgfqpoint{4.389044in}{1.108603in}}%
\pgfpathlineto{\pgfqpoint{4.391721in}{1.110577in}}%
\pgfpathlineto{\pgfqpoint{4.394400in}{1.104471in}}%
\pgfpathlineto{\pgfqpoint{4.397076in}{1.098182in}}%
\pgfpathlineto{\pgfqpoint{4.399745in}{1.102870in}}%
\pgfpathlineto{\pgfqpoint{4.402468in}{1.103355in}}%
\pgfpathlineto{\pgfqpoint{4.405234in}{1.107595in}}%
\pgfpathlineto{\pgfqpoint{4.407788in}{1.107215in}}%
\pgfpathlineto{\pgfqpoint{4.410587in}{1.110084in}}%
\pgfpathlineto{\pgfqpoint{4.413149in}{1.108185in}}%
\pgfpathlineto{\pgfqpoint{4.415932in}{1.111828in}}%
\pgfpathlineto{\pgfqpoint{4.418506in}{1.111080in}}%
\pgfpathlineto{\pgfqpoint{4.421292in}{1.108491in}}%
\pgfpathlineto{\pgfqpoint{4.423863in}{1.110619in}}%
\pgfpathlineto{\pgfqpoint{4.426534in}{1.108238in}}%
\pgfpathlineto{\pgfqpoint{4.429220in}{1.110553in}}%
\pgfpathlineto{\pgfqpoint{4.431901in}{1.108021in}}%
\pgfpathlineto{\pgfqpoint{4.434569in}{1.110725in}}%
\pgfpathlineto{\pgfqpoint{4.437253in}{1.112971in}}%
\pgfpathlineto{\pgfqpoint{4.440041in}{1.116199in}}%
\pgfpathlineto{\pgfqpoint{4.442611in}{1.111740in}}%
\pgfpathlineto{\pgfqpoint{4.445423in}{1.107543in}}%
\pgfpathlineto{\pgfqpoint{4.447965in}{1.106802in}}%
\pgfpathlineto{\pgfqpoint{4.450767in}{1.105301in}}%
\pgfpathlineto{\pgfqpoint{4.453312in}{1.107845in}}%
\pgfpathlineto{\pgfqpoint{4.456138in}{1.111233in}}%
\pgfpathlineto{\pgfqpoint{4.458681in}{1.127963in}}%
\pgfpathlineto{\pgfqpoint{4.461367in}{1.120344in}}%
\pgfpathlineto{\pgfqpoint{4.464029in}{1.114280in}}%
\pgfpathlineto{\pgfqpoint{4.466717in}{1.111045in}}%
\pgfpathlineto{\pgfqpoint{4.469492in}{1.108507in}}%
\pgfpathlineto{\pgfqpoint{4.472059in}{1.108065in}}%
\pgfpathlineto{\pgfqpoint{4.474861in}{1.106418in}}%
\pgfpathlineto{\pgfqpoint{4.477430in}{1.106067in}}%
\pgfpathlineto{\pgfqpoint{4.480201in}{1.105568in}}%
\pgfpathlineto{\pgfqpoint{4.482778in}{1.107718in}}%
\pgfpathlineto{\pgfqpoint{4.485581in}{1.108325in}}%
\pgfpathlineto{\pgfqpoint{4.488130in}{1.113053in}}%
\pgfpathlineto{\pgfqpoint{4.490822in}{1.108897in}}%
\pgfpathlineto{\pgfqpoint{4.493492in}{1.106710in}}%
\pgfpathlineto{\pgfqpoint{4.496167in}{1.107775in}}%
\pgfpathlineto{\pgfqpoint{4.498850in}{1.105063in}}%
\pgfpathlineto{\pgfqpoint{4.501529in}{1.111903in}}%
\pgfpathlineto{\pgfqpoint{4.504305in}{1.103416in}}%
\pgfpathlineto{\pgfqpoint{4.506893in}{1.109321in}}%
\pgfpathlineto{\pgfqpoint{4.509643in}{1.110615in}}%
\pgfpathlineto{\pgfqpoint{4.512246in}{1.111633in}}%
\pgfpathlineto{\pgfqpoint{4.515080in}{1.108789in}}%
\pgfpathlineto{\pgfqpoint{4.517598in}{1.107357in}}%
\pgfpathlineto{\pgfqpoint{4.520345in}{1.097804in}}%
\pgfpathlineto{\pgfqpoint{4.522962in}{1.098724in}}%
\pgfpathlineto{\pgfqpoint{4.525640in}{1.109746in}}%
\pgfpathlineto{\pgfqpoint{4.528307in}{1.106970in}}%
\pgfpathlineto{\pgfqpoint{4.530990in}{1.107582in}}%
\pgfpathlineto{\pgfqpoint{4.533764in}{1.106285in}}%
\pgfpathlineto{\pgfqpoint{4.536400in}{1.115679in}}%
\pgfpathlineto{\pgfqpoint{4.539144in}{1.107933in}}%
\pgfpathlineto{\pgfqpoint{4.541711in}{1.108339in}}%
\pgfpathlineto{\pgfqpoint{4.544464in}{1.113362in}}%
\pgfpathlineto{\pgfqpoint{4.547064in}{1.104289in}}%
\pgfpathlineto{\pgfqpoint{4.549822in}{1.109563in}}%
\pgfpathlineto{\pgfqpoint{4.552425in}{1.107248in}}%
\pgfpathlineto{\pgfqpoint{4.555106in}{1.112614in}}%
\pgfpathlineto{\pgfqpoint{4.557777in}{1.108707in}}%
\pgfpathlineto{\pgfqpoint{4.560448in}{1.111622in}}%
\pgfpathlineto{\pgfqpoint{4.563125in}{1.112535in}}%
\pgfpathlineto{\pgfqpoint{4.565820in}{1.114072in}}%
\pgfpathlineto{\pgfqpoint{4.568612in}{1.114498in}}%
\pgfpathlineto{\pgfqpoint{4.571171in}{1.111884in}}%
\pgfpathlineto{\pgfqpoint{4.573947in}{1.103559in}}%
\pgfpathlineto{\pgfqpoint{4.576531in}{1.110567in}}%
\pgfpathlineto{\pgfqpoint{4.579305in}{1.113407in}}%
\pgfpathlineto{\pgfqpoint{4.581888in}{1.118863in}}%
\pgfpathlineto{\pgfqpoint{4.584672in}{1.114695in}}%
\pgfpathlineto{\pgfqpoint{4.587244in}{1.109568in}}%
\pgfpathlineto{\pgfqpoint{4.589920in}{1.110462in}}%
\pgfpathlineto{\pgfqpoint{4.592589in}{1.104522in}}%
\pgfpathlineto{\pgfqpoint{4.595281in}{1.106962in}}%
\pgfpathlineto{\pgfqpoint{4.597951in}{1.111940in}}%
\pgfpathlineto{\pgfqpoint{4.600633in}{1.109869in}}%
\pgfpathlineto{\pgfqpoint{4.603430in}{1.103362in}}%
\pgfpathlineto{\pgfqpoint{4.605990in}{1.109281in}}%
\pgfpathlineto{\pgfqpoint{4.608808in}{1.109852in}}%
\pgfpathlineto{\pgfqpoint{4.611350in}{1.111207in}}%
\pgfpathlineto{\pgfqpoint{4.614134in}{1.114242in}}%
\pgfpathlineto{\pgfqpoint{4.616702in}{1.109465in}}%
\pgfpathlineto{\pgfqpoint{4.619529in}{1.106998in}}%
\pgfpathlineto{\pgfqpoint{4.622056in}{1.109356in}}%
\pgfpathlineto{\pgfqpoint{4.624741in}{1.108972in}}%
\pgfpathlineto{\pgfqpoint{4.627411in}{1.106052in}}%
\pgfpathlineto{\pgfqpoint{4.630096in}{1.105578in}}%
\pgfpathlineto{\pgfqpoint{4.632902in}{1.111952in}}%
\pgfpathlineto{\pgfqpoint{4.635445in}{1.110167in}}%
\pgfpathlineto{\pgfqpoint{4.638204in}{1.109725in}}%
\pgfpathlineto{\pgfqpoint{4.640809in}{1.105372in}}%
\pgfpathlineto{\pgfqpoint{4.643628in}{1.105324in}}%
\pgfpathlineto{\pgfqpoint{4.646169in}{1.107471in}}%
\pgfpathlineto{\pgfqpoint{4.648922in}{1.108180in}}%
\pgfpathlineto{\pgfqpoint{4.651524in}{1.109952in}}%
\pgfpathlineto{\pgfqpoint{4.654203in}{1.109409in}}%
\pgfpathlineto{\pgfqpoint{4.656873in}{1.111839in}}%
\pgfpathlineto{\pgfqpoint{4.659590in}{1.105115in}}%
\pgfpathlineto{\pgfqpoint{4.662237in}{1.105350in}}%
\pgfpathlineto{\pgfqpoint{4.664923in}{1.111739in}}%
\pgfpathlineto{\pgfqpoint{4.667764in}{1.110181in}}%
\pgfpathlineto{\pgfqpoint{4.670261in}{1.105183in}}%
\pgfpathlineto{\pgfqpoint{4.673068in}{1.109653in}}%
\pgfpathlineto{\pgfqpoint{4.675619in}{1.112046in}}%
\pgfpathlineto{\pgfqpoint{4.678448in}{1.107741in}}%
\pgfpathlineto{\pgfqpoint{4.680988in}{1.112399in}}%
\pgfpathlineto{\pgfqpoint{4.683799in}{1.112035in}}%
\pgfpathlineto{\pgfqpoint{4.686337in}{1.114440in}}%
\pgfpathlineto{\pgfqpoint{4.689051in}{1.110135in}}%
\pgfpathlineto{\pgfqpoint{4.691694in}{1.104930in}}%
\pgfpathlineto{\pgfqpoint{4.694381in}{1.106387in}}%
\pgfpathlineto{\pgfqpoint{4.697170in}{1.108861in}}%
\pgfpathlineto{\pgfqpoint{4.699734in}{1.109643in}}%
\pgfpathlineto{\pgfqpoint{4.702517in}{1.109242in}}%
\pgfpathlineto{\pgfqpoint{4.705094in}{1.107880in}}%
\pgfpathlineto{\pgfqpoint{4.707824in}{1.110848in}}%
\pgfpathlineto{\pgfqpoint{4.710437in}{1.115751in}}%
\pgfpathlineto{\pgfqpoint{4.713275in}{1.116648in}}%
\pgfpathlineto{\pgfqpoint{4.715806in}{1.111200in}}%
\pgfpathlineto{\pgfqpoint{4.718486in}{1.112126in}}%
\pgfpathlineto{\pgfqpoint{4.721160in}{1.112098in}}%
\pgfpathlineto{\pgfqpoint{4.723873in}{1.108887in}}%
\pgfpathlineto{\pgfqpoint{4.726508in}{1.108017in}}%
\pgfpathlineto{\pgfqpoint{4.729233in}{1.111862in}}%
\pgfpathlineto{\pgfqpoint{4.731901in}{1.105769in}}%
\pgfpathlineto{\pgfqpoint{4.734552in}{1.106776in}}%
\pgfpathlineto{\pgfqpoint{4.737348in}{1.108767in}}%
\pgfpathlineto{\pgfqpoint{4.739912in}{1.101447in}}%
\pgfpathlineto{\pgfqpoint{4.742696in}{1.103935in}}%
\pgfpathlineto{\pgfqpoint{4.745256in}{1.104642in}}%
\pgfpathlineto{\pgfqpoint{4.748081in}{1.102497in}}%
\pgfpathlineto{\pgfqpoint{4.750627in}{1.110998in}}%
\pgfpathlineto{\pgfqpoint{4.753298in}{1.105617in}}%
\pgfpathlineto{\pgfqpoint{4.755983in}{1.103526in}}%
\pgfpathlineto{\pgfqpoint{4.758653in}{1.110347in}}%
\pgfpathlineto{\pgfqpoint{4.761337in}{1.114955in}}%
\pgfpathlineto{\pgfqpoint{4.764018in}{1.112010in}}%
\pgfpathlineto{\pgfqpoint{4.766783in}{1.119412in}}%
\pgfpathlineto{\pgfqpoint{4.769367in}{1.110205in}}%
\pgfpathlineto{\pgfqpoint{4.772198in}{1.100384in}}%
\pgfpathlineto{\pgfqpoint{4.774732in}{1.102425in}}%
\pgfpathlineto{\pgfqpoint{4.777535in}{1.097841in}}%
\pgfpathlineto{\pgfqpoint{4.780083in}{1.104714in}}%
\pgfpathlineto{\pgfqpoint{4.782872in}{1.106025in}}%
\pgfpathlineto{\pgfqpoint{4.785445in}{1.102750in}}%
\pgfpathlineto{\pgfqpoint{4.788116in}{1.102151in}}%
\pgfpathlineto{\pgfqpoint{4.790798in}{1.099245in}}%
\pgfpathlineto{\pgfqpoint{4.793512in}{1.105080in}}%
\pgfpathlineto{\pgfqpoint{4.796274in}{1.106027in}}%
\pgfpathlineto{\pgfqpoint{4.798830in}{1.101591in}}%
\pgfpathlineto{\pgfqpoint{4.801586in}{1.109179in}}%
\pgfpathlineto{\pgfqpoint{4.804193in}{1.116166in}}%
\pgfpathlineto{\pgfqpoint{4.807017in}{1.121988in}}%
\pgfpathlineto{\pgfqpoint{4.809538in}{1.164187in}}%
\pgfpathlineto{\pgfqpoint{4.812377in}{1.209042in}}%
\pgfpathlineto{\pgfqpoint{4.814907in}{1.258487in}}%
\pgfpathlineto{\pgfqpoint{4.817587in}{1.259073in}}%
\pgfpathlineto{\pgfqpoint{4.820265in}{1.301473in}}%
\pgfpathlineto{\pgfqpoint{4.822945in}{1.329116in}}%
\pgfpathlineto{\pgfqpoint{4.825619in}{1.292186in}}%
\pgfpathlineto{\pgfqpoint{4.828291in}{1.260885in}}%
\pgfpathlineto{\pgfqpoint{4.831045in}{1.232264in}}%
\pgfpathlineto{\pgfqpoint{4.833657in}{1.209728in}}%
\pgfpathlineto{\pgfqpoint{4.837992in}{1.199842in}}%
\pgfpathlineto{\pgfqpoint{4.839922in}{1.266593in}}%
\pgfpathlineto{\pgfqpoint{4.842380in}{1.309398in}}%
\pgfpathlineto{\pgfqpoint{4.844361in}{1.328567in}}%
\pgfpathlineto{\pgfqpoint{4.847127in}{1.341560in}}%
\pgfpathlineto{\pgfqpoint{4.849715in}{1.347240in}}%
\pgfpathlineto{\pgfqpoint{4.852404in}{1.361590in}}%
\pgfpathlineto{\pgfqpoint{4.855070in}{1.358153in}}%
\pgfpathlineto{\pgfqpoint{4.857807in}{1.342018in}}%
\pgfpathlineto{\pgfqpoint{4.860544in}{1.332791in}}%
\pgfpathlineto{\pgfqpoint{4.863116in}{1.311637in}}%
\pgfpathlineto{\pgfqpoint{4.865910in}{1.280303in}}%
\pgfpathlineto{\pgfqpoint{4.868474in}{1.261101in}}%
\pgfpathlineto{\pgfqpoint{4.871209in}{1.237791in}}%
\pgfpathlineto{\pgfqpoint{4.873832in}{1.216565in}}%
\pgfpathlineto{\pgfqpoint{4.876636in}{1.201183in}}%
\pgfpathlineto{\pgfqpoint{4.879180in}{1.180986in}}%
\pgfpathlineto{\pgfqpoint{4.881864in}{1.165505in}}%
\pgfpathlineto{\pgfqpoint{4.884540in}{1.158744in}}%
\pgfpathlineto{\pgfqpoint{4.887211in}{1.154133in}}%
\pgfpathlineto{\pgfqpoint{4.889902in}{1.144927in}}%
\pgfpathlineto{\pgfqpoint{4.892611in}{1.147920in}}%
\pgfpathlineto{\pgfqpoint{4.895399in}{1.137948in}}%
\pgfpathlineto{\pgfqpoint{4.897938in}{1.132157in}}%
\pgfpathlineto{\pgfqpoint{4.900712in}{1.127158in}}%
\pgfpathlineto{\pgfqpoint{4.903295in}{1.129269in}}%
\pgfpathlineto{\pgfqpoint{4.906096in}{1.143672in}}%
\pgfpathlineto{\pgfqpoint{4.908648in}{1.153326in}}%
\pgfpathlineto{\pgfqpoint{4.911435in}{1.151221in}}%
\pgfpathlineto{\pgfqpoint{4.914009in}{1.141149in}}%
\pgfpathlineto{\pgfqpoint{4.916681in}{1.126023in}}%
\pgfpathlineto{\pgfqpoint{4.919352in}{1.121723in}}%
\pgfpathlineto{\pgfqpoint{4.922041in}{1.118913in}}%
\pgfpathlineto{\pgfqpoint{4.924708in}{1.112591in}}%
\pgfpathlineto{\pgfqpoint{4.927400in}{1.113731in}}%
\pgfpathlineto{\pgfqpoint{4.930170in}{1.107268in}}%
\pgfpathlineto{\pgfqpoint{4.932742in}{1.104530in}}%
\pgfpathlineto{\pgfqpoint{4.935515in}{1.114257in}}%
\pgfpathlineto{\pgfqpoint{4.938112in}{1.132739in}}%
\pgfpathlineto{\pgfqpoint{4.940881in}{1.133070in}}%
\pgfpathlineto{\pgfqpoint{4.943466in}{1.135581in}}%
\pgfpathlineto{\pgfqpoint{4.946151in}{1.127996in}}%
\pgfpathlineto{\pgfqpoint{4.948827in}{1.127451in}}%
\pgfpathlineto{\pgfqpoint{4.951504in}{1.116798in}}%
\pgfpathlineto{\pgfqpoint{4.954182in}{1.113449in}}%
\pgfpathlineto{\pgfqpoint{4.956862in}{1.101145in}}%
\pgfpathlineto{\pgfqpoint{4.959689in}{1.099324in}}%
\pgfpathlineto{\pgfqpoint{4.962219in}{1.100000in}}%
\pgfpathlineto{\pgfqpoint{4.965002in}{1.103866in}}%
\pgfpathlineto{\pgfqpoint{4.967575in}{1.108297in}}%
\pgfpathlineto{\pgfqpoint{4.970314in}{1.115914in}}%
\pgfpathlineto{\pgfqpoint{4.972933in}{1.106478in}}%
\pgfpathlineto{\pgfqpoint{4.975703in}{1.113910in}}%
\pgfpathlineto{\pgfqpoint{4.978287in}{1.112708in}}%
\pgfpathlineto{\pgfqpoint{4.980967in}{1.114615in}}%
\pgfpathlineto{\pgfqpoint{4.983637in}{1.111994in}}%
\pgfpathlineto{\pgfqpoint{4.986325in}{1.114843in}}%
\pgfpathlineto{\pgfqpoint{4.989001in}{1.112104in}}%
\pgfpathlineto{\pgfqpoint{4.991683in}{1.103246in}}%
\pgfpathlineto{\pgfqpoint{4.994390in}{1.106138in}}%
\pgfpathlineto{\pgfqpoint{4.997028in}{1.107826in}}%
\pgfpathlineto{\pgfqpoint{4.999780in}{1.111861in}}%
\pgfpathlineto{\pgfqpoint{5.002384in}{1.098972in}}%
\pgfpathlineto{\pgfqpoint{5.005178in}{1.104878in}}%
\pgfpathlineto{\pgfqpoint{5.007751in}{1.107079in}}%
\pgfpathlineto{\pgfqpoint{5.010562in}{1.103711in}}%
\pgfpathlineto{\pgfqpoint{5.013104in}{1.106737in}}%
\pgfpathlineto{\pgfqpoint{5.015820in}{1.109135in}}%
\pgfpathlineto{\pgfqpoint{5.018466in}{1.110140in}}%
\pgfpathlineto{\pgfqpoint{5.021147in}{1.111190in}}%
\pgfpathlineto{\pgfqpoint{5.023927in}{1.104765in}}%
\pgfpathlineto{\pgfqpoint{5.026501in}{1.103660in}}%
\pgfpathlineto{\pgfqpoint{5.029275in}{1.104545in}}%
\pgfpathlineto{\pgfqpoint{5.031849in}{1.108840in}}%
\pgfpathlineto{\pgfqpoint{5.034649in}{1.096460in}}%
\pgfpathlineto{\pgfqpoint{5.037214in}{1.101472in}}%
\pgfpathlineto{\pgfqpoint{5.039962in}{1.097714in}}%
\pgfpathlineto{\pgfqpoint{5.042572in}{1.100239in}}%
\pgfpathlineto{\pgfqpoint{5.045249in}{1.101532in}}%
\pgfpathlineto{\pgfqpoint{5.047924in}{1.108519in}}%
\pgfpathlineto{\pgfqpoint{5.050606in}{1.103722in}}%
\pgfpathlineto{\pgfqpoint{5.053284in}{1.104323in}}%
\pgfpathlineto{\pgfqpoint{5.055952in}{1.106462in}}%
\pgfpathlineto{\pgfqpoint{5.058711in}{1.102866in}}%
\pgfpathlineto{\pgfqpoint{5.061315in}{1.101379in}}%
\pgfpathlineto{\pgfqpoint{5.064144in}{1.104361in}}%
\pgfpathlineto{\pgfqpoint{5.066677in}{1.104730in}}%
\pgfpathlineto{\pgfqpoint{5.069463in}{1.106366in}}%
\pgfpathlineto{\pgfqpoint{5.072030in}{1.101933in}}%
\pgfpathlineto{\pgfqpoint{5.074851in}{1.100442in}}%
\pgfpathlineto{\pgfqpoint{5.077390in}{1.101679in}}%
\pgfpathlineto{\pgfqpoint{5.080067in}{1.097315in}}%
\pgfpathlineto{\pgfqpoint{5.082746in}{1.097661in}}%
\pgfpathlineto{\pgfqpoint{5.085426in}{1.103065in}}%
\pgfpathlineto{\pgfqpoint{5.088103in}{1.102317in}}%
\pgfpathlineto{\pgfqpoint{5.090788in}{1.105308in}}%
\pgfpathlineto{\pgfqpoint{5.093579in}{1.107043in}}%
\pgfpathlineto{\pgfqpoint{5.096142in}{1.108706in}}%
\pgfpathlineto{\pgfqpoint{5.098948in}{1.100641in}}%
\pgfpathlineto{\pgfqpoint{5.101496in}{1.102197in}}%
\pgfpathlineto{\pgfqpoint{5.104312in}{1.104966in}}%
\pgfpathlineto{\pgfqpoint{5.106842in}{1.106917in}}%
\pgfpathlineto{\pgfqpoint{5.109530in}{1.107381in}}%
\pgfpathlineto{\pgfqpoint{5.112209in}{1.108038in}}%
\pgfpathlineto{\pgfqpoint{5.114887in}{1.107717in}}%
\pgfpathlineto{\pgfqpoint{5.117550in}{1.106775in}}%
\pgfpathlineto{\pgfqpoint{5.120243in}{1.112192in}}%
\pgfpathlineto{\pgfqpoint{5.123042in}{1.107131in}}%
\pgfpathlineto{\pgfqpoint{5.125599in}{1.106691in}}%
\pgfpathlineto{\pgfqpoint{5.128421in}{1.110497in}}%
\pgfpathlineto{\pgfqpoint{5.130953in}{1.109999in}}%
\pgfpathlineto{\pgfqpoint{5.133716in}{1.110898in}}%
\pgfpathlineto{\pgfqpoint{5.136311in}{1.101969in}}%
\pgfpathlineto{\pgfqpoint{5.139072in}{1.101407in}}%
\pgfpathlineto{\pgfqpoint{5.141660in}{1.106254in}}%
\pgfpathlineto{\pgfqpoint{5.144349in}{1.102515in}}%
\pgfpathlineto{\pgfqpoint{5.147029in}{1.098455in}}%
\pgfpathlineto{\pgfqpoint{5.149734in}{1.098290in}}%
\pgfpathlineto{\pgfqpoint{5.152382in}{1.089508in}}%
\pgfpathlineto{\pgfqpoint{5.155059in}{1.095207in}}%
\pgfpathlineto{\pgfqpoint{5.157815in}{1.099083in}}%
\pgfpathlineto{\pgfqpoint{5.160420in}{1.095295in}}%
\pgfpathlineto{\pgfqpoint{5.163243in}{1.102514in}}%
\pgfpathlineto{\pgfqpoint{5.165775in}{1.101662in}}%
\pgfpathlineto{\pgfqpoint{5.168591in}{1.104080in}}%
\pgfpathlineto{\pgfqpoint{5.171133in}{1.103777in}}%
\pgfpathlineto{\pgfqpoint{5.173925in}{1.104185in}}%
\pgfpathlineto{\pgfqpoint{5.176477in}{1.100815in}}%
\pgfpathlineto{\pgfqpoint{5.179188in}{1.107872in}}%
\pgfpathlineto{\pgfqpoint{5.181848in}{1.102362in}}%
\pgfpathlineto{\pgfqpoint{5.184522in}{1.101250in}}%
\pgfpathlineto{\pgfqpoint{5.187294in}{1.103387in}}%
\pgfpathlineto{\pgfqpoint{5.189880in}{1.099486in}}%
\pgfpathlineto{\pgfqpoint{5.192680in}{1.103095in}}%
\pgfpathlineto{\pgfqpoint{5.195239in}{1.101657in}}%
\pgfpathlineto{\pgfqpoint{5.198008in}{1.101459in}}%
\pgfpathlineto{\pgfqpoint{5.200594in}{1.107396in}}%
\pgfpathlineto{\pgfqpoint{5.203388in}{1.099776in}}%
\pgfpathlineto{\pgfqpoint{5.205952in}{1.099971in}}%
\pgfpathlineto{\pgfqpoint{5.208630in}{1.103802in}}%
\pgfpathlineto{\pgfqpoint{5.211299in}{1.105131in}}%
\pgfpathlineto{\pgfqpoint{5.214027in}{1.103893in}}%
\pgfpathlineto{\pgfqpoint{5.216667in}{1.105595in}}%
\pgfpathlineto{\pgfqpoint{5.219345in}{1.097383in}}%
\pgfpathlineto{\pgfqpoint{5.222151in}{1.101147in}}%
\pgfpathlineto{\pgfqpoint{5.224695in}{1.097589in}}%
\pgfpathlineto{\pgfqpoint{5.227470in}{1.095317in}}%
\pgfpathlineto{\pgfqpoint{5.230059in}{1.102562in}}%
\pgfpathlineto{\pgfqpoint{5.232855in}{1.101502in}}%
\pgfpathlineto{\pgfqpoint{5.235409in}{1.102115in}}%
\pgfpathlineto{\pgfqpoint{5.238173in}{1.106415in}}%
\pgfpathlineto{\pgfqpoint{5.240777in}{1.099618in}}%
\pgfpathlineto{\pgfqpoint{5.243445in}{1.099900in}}%
\pgfpathlineto{\pgfqpoint{5.246130in}{1.102193in}}%
\pgfpathlineto{\pgfqpoint{5.248816in}{1.099619in}}%
\pgfpathlineto{\pgfqpoint{5.251590in}{1.105310in}}%
\pgfpathlineto{\pgfqpoint{5.254236in}{1.100551in}}%
\pgfpathlineto{\pgfqpoint{5.256973in}{1.104202in}}%
\pgfpathlineto{\pgfqpoint{5.259511in}{1.101532in}}%
\pgfpathlineto{\pgfqpoint{5.262264in}{1.100250in}}%
\pgfpathlineto{\pgfqpoint{5.264876in}{1.092115in}}%
\pgfpathlineto{\pgfqpoint{5.267691in}{1.094615in}}%
\pgfpathlineto{\pgfqpoint{5.270238in}{1.103923in}}%
\pgfpathlineto{\pgfqpoint{5.272913in}{1.107297in}}%
\pgfpathlineto{\pgfqpoint{5.275589in}{1.098338in}}%
\pgfpathlineto{\pgfqpoint{5.278322in}{1.105585in}}%
\pgfpathlineto{\pgfqpoint{5.280947in}{1.101858in}}%
\pgfpathlineto{\pgfqpoint{5.283631in}{1.106390in}}%
\pgfpathlineto{\pgfqpoint{5.286436in}{1.106077in}}%
\pgfpathlineto{\pgfqpoint{5.288984in}{1.106628in}}%
\pgfpathlineto{\pgfqpoint{5.291794in}{1.107314in}}%
\pgfpathlineto{\pgfqpoint{5.294339in}{1.105143in}}%
\pgfpathlineto{\pgfqpoint{5.297140in}{1.110058in}}%
\pgfpathlineto{\pgfqpoint{5.299696in}{1.112426in}}%
\pgfpathlineto{\pgfqpoint{5.302443in}{1.108965in}}%
\pgfpathlineto{\pgfqpoint{5.305054in}{1.109331in}}%
\pgfpathlineto{\pgfqpoint{5.307731in}{1.106473in}}%
\pgfpathlineto{\pgfqpoint{5.310411in}{1.107455in}}%
\pgfpathlineto{\pgfqpoint{5.313089in}{1.108226in}}%
\pgfpathlineto{\pgfqpoint{5.315754in}{1.107791in}}%
\pgfpathlineto{\pgfqpoint{5.318430in}{1.110091in}}%
\pgfpathlineto{\pgfqpoint{5.321256in}{1.105536in}}%
\pgfpathlineto{\pgfqpoint{5.323802in}{1.099331in}}%
\pgfpathlineto{\pgfqpoint{5.326564in}{1.100113in}}%
\pgfpathlineto{\pgfqpoint{5.329159in}{1.089877in}}%
\pgfpathlineto{\pgfqpoint{5.331973in}{1.093093in}}%
\pgfpathlineto{\pgfqpoint{5.334510in}{1.088367in}}%
\pgfpathlineto{\pgfqpoint{5.337353in}{1.093508in}}%
\pgfpathlineto{\pgfqpoint{5.339872in}{1.095898in}}%
\pgfpathlineto{\pgfqpoint{5.342549in}{1.097809in}}%
\pgfpathlineto{\pgfqpoint{5.345224in}{1.092962in}}%
\pgfpathlineto{\pgfqpoint{5.347905in}{1.092657in}}%
\pgfpathlineto{\pgfqpoint{5.350723in}{1.089157in}}%
\pgfpathlineto{\pgfqpoint{5.353262in}{1.092942in}}%
\pgfpathlineto{\pgfqpoint{5.356056in}{1.089730in}}%
\pgfpathlineto{\pgfqpoint{5.358612in}{1.089045in}}%
\pgfpathlineto{\pgfqpoint{5.361370in}{1.096199in}}%
\pgfpathlineto{\pgfqpoint{5.363966in}{1.095736in}}%
\pgfpathlineto{\pgfqpoint{5.366727in}{1.095947in}}%
\pgfpathlineto{\pgfqpoint{5.369335in}{1.095401in}}%
\pgfpathlineto{\pgfqpoint{5.372013in}{1.101217in}}%
\pgfpathlineto{\pgfqpoint{5.374692in}{1.101556in}}%
\pgfpathlineto{\pgfqpoint{5.377370in}{1.101552in}}%
\pgfpathlineto{\pgfqpoint{5.380048in}{1.103925in}}%
\pgfpathlineto{\pgfqpoint{5.382725in}{1.105903in}}%
\pgfpathlineto{\pgfqpoint{5.385550in}{1.103225in}}%
\pgfpathlineto{\pgfqpoint{5.388083in}{1.098882in}}%
\pgfpathlineto{\pgfqpoint{5.390900in}{1.100645in}}%
\pgfpathlineto{\pgfqpoint{5.393441in}{1.087123in}}%
\pgfpathlineto{\pgfqpoint{5.396219in}{1.096351in}}%
\pgfpathlineto{\pgfqpoint{5.398784in}{1.105830in}}%
\pgfpathlineto{\pgfqpoint{5.401576in}{1.105187in}}%
\pgfpathlineto{\pgfqpoint{5.404154in}{1.101676in}}%
\pgfpathlineto{\pgfqpoint{5.406832in}{1.102921in}}%
\pgfpathlineto{\pgfqpoint{5.409507in}{1.101057in}}%
\pgfpathlineto{\pgfqpoint{5.412190in}{1.105521in}}%
\pgfpathlineto{\pgfqpoint{5.414954in}{1.109305in}}%
\pgfpathlineto{\pgfqpoint{5.417547in}{1.105943in}}%
\pgfpathlineto{\pgfqpoint{5.420304in}{1.110457in}}%
\pgfpathlineto{\pgfqpoint{5.422897in}{1.106680in}}%
\pgfpathlineto{\pgfqpoint{5.425661in}{1.106400in}}%
\pgfpathlineto{\pgfqpoint{5.428259in}{1.104713in}}%
\pgfpathlineto{\pgfqpoint{5.431015in}{1.105434in}}%
\pgfpathlineto{\pgfqpoint{5.433616in}{1.110672in}}%
\pgfpathlineto{\pgfqpoint{5.436295in}{1.110301in}}%
\pgfpathlineto{\pgfqpoint{5.438974in}{1.110850in}}%
\pgfpathlineto{\pgfqpoint{5.441698in}{1.113684in}}%
\pgfpathlineto{\pgfqpoint{5.444328in}{1.108133in}}%
\pgfpathlineto{\pgfqpoint{5.447021in}{1.110997in}}%
\pgfpathlineto{\pgfqpoint{5.449769in}{1.110049in}}%
\pgfpathlineto{\pgfqpoint{5.452365in}{1.110960in}}%
\pgfpathlineto{\pgfqpoint{5.455168in}{1.110576in}}%
\pgfpathlineto{\pgfqpoint{5.457721in}{1.117858in}}%
\pgfpathlineto{\pgfqpoint{5.460489in}{1.112114in}}%
\pgfpathlineto{\pgfqpoint{5.463079in}{1.111430in}}%
\pgfpathlineto{\pgfqpoint{5.465888in}{1.109249in}}%
\pgfpathlineto{\pgfqpoint{5.468425in}{1.111937in}}%
\pgfpathlineto{\pgfqpoint{5.471113in}{1.107560in}}%
\pgfpathlineto{\pgfqpoint{5.473792in}{1.107145in}}%
\pgfpathlineto{\pgfqpoint{5.476458in}{1.108393in}}%
\pgfpathlineto{\pgfqpoint{5.479152in}{1.109221in}}%
\pgfpathlineto{\pgfqpoint{5.481825in}{1.112190in}}%
\pgfpathlineto{\pgfqpoint{5.484641in}{1.108690in}}%
\pgfpathlineto{\pgfqpoint{5.487176in}{1.109956in}}%
\pgfpathlineto{\pgfqpoint{5.490000in}{1.117017in}}%
\pgfpathlineto{\pgfqpoint{5.492541in}{1.115186in}}%
\pgfpathlineto{\pgfqpoint{5.495346in}{1.110500in}}%
\pgfpathlineto{\pgfqpoint{5.497898in}{1.109847in}}%
\pgfpathlineto{\pgfqpoint{5.500687in}{1.103793in}}%
\pgfpathlineto{\pgfqpoint{5.503255in}{1.108769in}}%
\pgfpathlineto{\pgfqpoint{5.505933in}{1.107037in}}%
\pgfpathlineto{\pgfqpoint{5.508612in}{1.109072in}}%
\pgfpathlineto{\pgfqpoint{5.511290in}{1.109330in}}%
\pgfpathlineto{\pgfqpoint{5.514080in}{1.108866in}}%
\pgfpathlineto{\pgfqpoint{5.516646in}{1.112617in}}%
\pgfpathlineto{\pgfqpoint{5.519433in}{1.113325in}}%
\pgfpathlineto{\pgfqpoint{5.522003in}{1.106201in}}%
\pgfpathlineto{\pgfqpoint{5.524756in}{1.109531in}}%
\pgfpathlineto{\pgfqpoint{5.527360in}{1.109816in}}%
\pgfpathlineto{\pgfqpoint{5.530148in}{1.109917in}}%
\pgfpathlineto{\pgfqpoint{5.532717in}{1.112592in}}%
\pgfpathlineto{\pgfqpoint{5.535395in}{1.109372in}}%
\pgfpathlineto{\pgfqpoint{5.538074in}{1.111464in}}%
\pgfpathlineto{\pgfqpoint{5.540750in}{1.109176in}}%
\pgfpathlineto{\pgfqpoint{5.543421in}{1.107495in}}%
\pgfpathlineto{\pgfqpoint{5.546110in}{1.110929in}}%
\pgfpathlineto{\pgfqpoint{5.548921in}{1.111393in}}%
\pgfpathlineto{\pgfqpoint{5.551457in}{1.116490in}}%
\pgfpathlineto{\pgfqpoint{5.554198in}{1.114469in}}%
\pgfpathlineto{\pgfqpoint{5.556822in}{1.114004in}}%
\pgfpathlineto{\pgfqpoint{5.559612in}{1.110606in}}%
\pgfpathlineto{\pgfqpoint{5.562180in}{1.113193in}}%
\pgfpathlineto{\pgfqpoint{5.564940in}{1.110564in}}%
\pgfpathlineto{\pgfqpoint{5.567536in}{1.103377in}}%
\pgfpathlineto{\pgfqpoint{5.570215in}{1.106959in}}%
\pgfpathlineto{\pgfqpoint{5.572893in}{1.115785in}}%
\pgfpathlineto{\pgfqpoint{5.575596in}{1.110803in}}%
\pgfpathlineto{\pgfqpoint{5.578342in}{1.112311in}}%
\pgfpathlineto{\pgfqpoint{5.580914in}{1.108720in}}%
\pgfpathlineto{\pgfqpoint{5.583709in}{1.105950in}}%
\pgfpathlineto{\pgfqpoint{5.586269in}{1.106544in}}%
\pgfpathlineto{\pgfqpoint{5.589040in}{1.113399in}}%
\pgfpathlineto{\pgfqpoint{5.591641in}{1.109348in}}%
\pgfpathlineto{\pgfqpoint{5.594368in}{1.110685in}}%
\pgfpathlineto{\pgfqpoint{5.596999in}{1.115494in}}%
\pgfpathlineto{\pgfqpoint{5.599674in}{1.113148in}}%
\pgfpathlineto{\pgfqpoint{5.602352in}{1.112991in}}%
\pgfpathlineto{\pgfqpoint{5.605073in}{1.113219in}}%
\pgfpathlineto{\pgfqpoint{5.607698in}{1.108837in}}%
\pgfpathlineto{\pgfqpoint{5.610389in}{1.107225in}}%
\pgfpathlineto{\pgfqpoint{5.613235in}{1.107471in}}%
\pgfpathlineto{\pgfqpoint{5.615743in}{1.104512in}}%
\pgfpathlineto{\pgfqpoint{5.618526in}{1.102311in}}%
\pgfpathlineto{\pgfqpoint{5.621102in}{1.099552in}}%
\pgfpathlineto{\pgfqpoint{5.623868in}{1.100864in}}%
\pgfpathlineto{\pgfqpoint{5.626460in}{1.104134in}}%
\pgfpathlineto{\pgfqpoint{5.629232in}{1.105516in}}%
\pgfpathlineto{\pgfqpoint{5.631815in}{1.102588in}}%
\pgfpathlineto{\pgfqpoint{5.634496in}{1.104153in}}%
\pgfpathlineto{\pgfqpoint{5.637172in}{1.108800in}}%
\pgfpathlineto{\pgfqpoint{5.639852in}{1.104197in}}%
\pgfpathlineto{\pgfqpoint{5.642518in}{1.104085in}}%
\pgfpathlineto{\pgfqpoint{5.645243in}{1.104576in}}%
\pgfpathlineto{\pgfqpoint{5.648008in}{1.119250in}}%
\pgfpathlineto{\pgfqpoint{5.650563in}{1.109847in}}%
\pgfpathlineto{\pgfqpoint{5.653376in}{1.101719in}}%
\pgfpathlineto{\pgfqpoint{5.655919in}{1.098565in}}%
\pgfpathlineto{\pgfqpoint{5.658723in}{1.097155in}}%
\pgfpathlineto{\pgfqpoint{5.661273in}{1.104228in}}%
\pgfpathlineto{\pgfqpoint{5.664099in}{1.102670in}}%
\pgfpathlineto{\pgfqpoint{5.666632in}{1.104080in}}%
\pgfpathlineto{\pgfqpoint{5.669313in}{1.107592in}}%
\pgfpathlineto{\pgfqpoint{5.671991in}{1.102915in}}%
\pgfpathlineto{\pgfqpoint{5.674667in}{1.101920in}}%
\pgfpathlineto{\pgfqpoint{5.677486in}{1.104978in}}%
\pgfpathlineto{\pgfqpoint{5.680027in}{1.101111in}}%
\pgfpathlineto{\pgfqpoint{5.682836in}{1.105782in}}%
\pgfpathlineto{\pgfqpoint{5.685385in}{1.102000in}}%
\pgfpathlineto{\pgfqpoint{5.688159in}{1.107982in}}%
\pgfpathlineto{\pgfqpoint{5.690730in}{1.106324in}}%
\pgfpathlineto{\pgfqpoint{5.693473in}{1.110027in}}%
\pgfpathlineto{\pgfqpoint{5.696101in}{1.107330in}}%
\pgfpathlineto{\pgfqpoint{5.698775in}{1.110631in}}%
\pgfpathlineto{\pgfqpoint{5.701453in}{1.106744in}}%
\pgfpathlineto{\pgfqpoint{5.704130in}{1.102340in}}%
\pgfpathlineto{\pgfqpoint{5.706800in}{1.093722in}}%
\pgfpathlineto{\pgfqpoint{5.709490in}{1.089073in}}%
\pgfpathlineto{\pgfqpoint{5.712291in}{1.090168in}}%
\pgfpathlineto{\pgfqpoint{5.714834in}{1.093335in}}%
\pgfpathlineto{\pgfqpoint{5.717671in}{1.096871in}}%
\pgfpathlineto{\pgfqpoint{5.720201in}{1.099780in}}%
\pgfpathlineto{\pgfqpoint{5.722950in}{1.100587in}}%
\pgfpathlineto{\pgfqpoint{5.725548in}{1.102674in}}%
\pgfpathlineto{\pgfqpoint{5.728339in}{1.104676in}}%
\pgfpathlineto{\pgfqpoint{5.730919in}{1.104747in}}%
\pgfpathlineto{\pgfqpoint{5.733594in}{1.104509in}}%
\pgfpathlineto{\pgfqpoint{5.736276in}{1.104266in}}%
\pgfpathlineto{\pgfqpoint{5.738974in}{1.106469in}}%
\pgfpathlineto{\pgfqpoint{5.741745in}{1.101476in}}%
\pgfpathlineto{\pgfqpoint{5.744310in}{1.102908in}}%
\pgfpathlineto{\pgfqpoint{5.744310in}{0.413320in}}%
\pgfpathlineto{\pgfqpoint{5.744310in}{0.413320in}}%
\pgfpathlineto{\pgfqpoint{5.741745in}{0.413320in}}%
\pgfpathlineto{\pgfqpoint{5.738974in}{0.413320in}}%
\pgfpathlineto{\pgfqpoint{5.736276in}{0.413320in}}%
\pgfpathlineto{\pgfqpoint{5.733594in}{0.413320in}}%
\pgfpathlineto{\pgfqpoint{5.730919in}{0.413320in}}%
\pgfpathlineto{\pgfqpoint{5.728339in}{0.413320in}}%
\pgfpathlineto{\pgfqpoint{5.725548in}{0.413320in}}%
\pgfpathlineto{\pgfqpoint{5.722950in}{0.413320in}}%
\pgfpathlineto{\pgfqpoint{5.720201in}{0.413320in}}%
\pgfpathlineto{\pgfqpoint{5.717671in}{0.413320in}}%
\pgfpathlineto{\pgfqpoint{5.714834in}{0.413320in}}%
\pgfpathlineto{\pgfqpoint{5.712291in}{0.413320in}}%
\pgfpathlineto{\pgfqpoint{5.709490in}{0.413320in}}%
\pgfpathlineto{\pgfqpoint{5.706800in}{0.413320in}}%
\pgfpathlineto{\pgfqpoint{5.704130in}{0.413320in}}%
\pgfpathlineto{\pgfqpoint{5.701453in}{0.413320in}}%
\pgfpathlineto{\pgfqpoint{5.698775in}{0.413320in}}%
\pgfpathlineto{\pgfqpoint{5.696101in}{0.413320in}}%
\pgfpathlineto{\pgfqpoint{5.693473in}{0.413320in}}%
\pgfpathlineto{\pgfqpoint{5.690730in}{0.413320in}}%
\pgfpathlineto{\pgfqpoint{5.688159in}{0.413320in}}%
\pgfpathlineto{\pgfqpoint{5.685385in}{0.413320in}}%
\pgfpathlineto{\pgfqpoint{5.682836in}{0.413320in}}%
\pgfpathlineto{\pgfqpoint{5.680027in}{0.413320in}}%
\pgfpathlineto{\pgfqpoint{5.677486in}{0.413320in}}%
\pgfpathlineto{\pgfqpoint{5.674667in}{0.413320in}}%
\pgfpathlineto{\pgfqpoint{5.671991in}{0.413320in}}%
\pgfpathlineto{\pgfqpoint{5.669313in}{0.413320in}}%
\pgfpathlineto{\pgfqpoint{5.666632in}{0.413320in}}%
\pgfpathlineto{\pgfqpoint{5.664099in}{0.413320in}}%
\pgfpathlineto{\pgfqpoint{5.661273in}{0.413320in}}%
\pgfpathlineto{\pgfqpoint{5.658723in}{0.413320in}}%
\pgfpathlineto{\pgfqpoint{5.655919in}{0.413320in}}%
\pgfpathlineto{\pgfqpoint{5.653376in}{0.413320in}}%
\pgfpathlineto{\pgfqpoint{5.650563in}{0.413320in}}%
\pgfpathlineto{\pgfqpoint{5.648008in}{0.413320in}}%
\pgfpathlineto{\pgfqpoint{5.645243in}{0.413320in}}%
\pgfpathlineto{\pgfqpoint{5.642518in}{0.413320in}}%
\pgfpathlineto{\pgfqpoint{5.639852in}{0.413320in}}%
\pgfpathlineto{\pgfqpoint{5.637172in}{0.413320in}}%
\pgfpathlineto{\pgfqpoint{5.634496in}{0.413320in}}%
\pgfpathlineto{\pgfqpoint{5.631815in}{0.413320in}}%
\pgfpathlineto{\pgfqpoint{5.629232in}{0.413320in}}%
\pgfpathlineto{\pgfqpoint{5.626460in}{0.413320in}}%
\pgfpathlineto{\pgfqpoint{5.623868in}{0.413320in}}%
\pgfpathlineto{\pgfqpoint{5.621102in}{0.413320in}}%
\pgfpathlineto{\pgfqpoint{5.618526in}{0.413320in}}%
\pgfpathlineto{\pgfqpoint{5.615743in}{0.413320in}}%
\pgfpathlineto{\pgfqpoint{5.613235in}{0.413320in}}%
\pgfpathlineto{\pgfqpoint{5.610389in}{0.413320in}}%
\pgfpathlineto{\pgfqpoint{5.607698in}{0.413320in}}%
\pgfpathlineto{\pgfqpoint{5.605073in}{0.413320in}}%
\pgfpathlineto{\pgfqpoint{5.602352in}{0.413320in}}%
\pgfpathlineto{\pgfqpoint{5.599674in}{0.413320in}}%
\pgfpathlineto{\pgfqpoint{5.596999in}{0.413320in}}%
\pgfpathlineto{\pgfqpoint{5.594368in}{0.413320in}}%
\pgfpathlineto{\pgfqpoint{5.591641in}{0.413320in}}%
\pgfpathlineto{\pgfqpoint{5.589040in}{0.413320in}}%
\pgfpathlineto{\pgfqpoint{5.586269in}{0.413320in}}%
\pgfpathlineto{\pgfqpoint{5.583709in}{0.413320in}}%
\pgfpathlineto{\pgfqpoint{5.580914in}{0.413320in}}%
\pgfpathlineto{\pgfqpoint{5.578342in}{0.413320in}}%
\pgfpathlineto{\pgfqpoint{5.575596in}{0.413320in}}%
\pgfpathlineto{\pgfqpoint{5.572893in}{0.413320in}}%
\pgfpathlineto{\pgfqpoint{5.570215in}{0.413320in}}%
\pgfpathlineto{\pgfqpoint{5.567536in}{0.413320in}}%
\pgfpathlineto{\pgfqpoint{5.564940in}{0.413320in}}%
\pgfpathlineto{\pgfqpoint{5.562180in}{0.413320in}}%
\pgfpathlineto{\pgfqpoint{5.559612in}{0.413320in}}%
\pgfpathlineto{\pgfqpoint{5.556822in}{0.413320in}}%
\pgfpathlineto{\pgfqpoint{5.554198in}{0.413320in}}%
\pgfpathlineto{\pgfqpoint{5.551457in}{0.413320in}}%
\pgfpathlineto{\pgfqpoint{5.548921in}{0.413320in}}%
\pgfpathlineto{\pgfqpoint{5.546110in}{0.413320in}}%
\pgfpathlineto{\pgfqpoint{5.543421in}{0.413320in}}%
\pgfpathlineto{\pgfqpoint{5.540750in}{0.413320in}}%
\pgfpathlineto{\pgfqpoint{5.538074in}{0.413320in}}%
\pgfpathlineto{\pgfqpoint{5.535395in}{0.413320in}}%
\pgfpathlineto{\pgfqpoint{5.532717in}{0.413320in}}%
\pgfpathlineto{\pgfqpoint{5.530148in}{0.413320in}}%
\pgfpathlineto{\pgfqpoint{5.527360in}{0.413320in}}%
\pgfpathlineto{\pgfqpoint{5.524756in}{0.413320in}}%
\pgfpathlineto{\pgfqpoint{5.522003in}{0.413320in}}%
\pgfpathlineto{\pgfqpoint{5.519433in}{0.413320in}}%
\pgfpathlineto{\pgfqpoint{5.516646in}{0.413320in}}%
\pgfpathlineto{\pgfqpoint{5.514080in}{0.413320in}}%
\pgfpathlineto{\pgfqpoint{5.511290in}{0.413320in}}%
\pgfpathlineto{\pgfqpoint{5.508612in}{0.413320in}}%
\pgfpathlineto{\pgfqpoint{5.505933in}{0.413320in}}%
\pgfpathlineto{\pgfqpoint{5.503255in}{0.413320in}}%
\pgfpathlineto{\pgfqpoint{5.500687in}{0.413320in}}%
\pgfpathlineto{\pgfqpoint{5.497898in}{0.413320in}}%
\pgfpathlineto{\pgfqpoint{5.495346in}{0.413320in}}%
\pgfpathlineto{\pgfqpoint{5.492541in}{0.413320in}}%
\pgfpathlineto{\pgfqpoint{5.490000in}{0.413320in}}%
\pgfpathlineto{\pgfqpoint{5.487176in}{0.413320in}}%
\pgfpathlineto{\pgfqpoint{5.484641in}{0.413320in}}%
\pgfpathlineto{\pgfqpoint{5.481825in}{0.413320in}}%
\pgfpathlineto{\pgfqpoint{5.479152in}{0.413320in}}%
\pgfpathlineto{\pgfqpoint{5.476458in}{0.413320in}}%
\pgfpathlineto{\pgfqpoint{5.473792in}{0.413320in}}%
\pgfpathlineto{\pgfqpoint{5.471113in}{0.413320in}}%
\pgfpathlineto{\pgfqpoint{5.468425in}{0.413320in}}%
\pgfpathlineto{\pgfqpoint{5.465888in}{0.413320in}}%
\pgfpathlineto{\pgfqpoint{5.463079in}{0.413320in}}%
\pgfpathlineto{\pgfqpoint{5.460489in}{0.413320in}}%
\pgfpathlineto{\pgfqpoint{5.457721in}{0.413320in}}%
\pgfpathlineto{\pgfqpoint{5.455168in}{0.413320in}}%
\pgfpathlineto{\pgfqpoint{5.452365in}{0.413320in}}%
\pgfpathlineto{\pgfqpoint{5.449769in}{0.413320in}}%
\pgfpathlineto{\pgfqpoint{5.447021in}{0.413320in}}%
\pgfpathlineto{\pgfqpoint{5.444328in}{0.413320in}}%
\pgfpathlineto{\pgfqpoint{5.441698in}{0.413320in}}%
\pgfpathlineto{\pgfqpoint{5.438974in}{0.413320in}}%
\pgfpathlineto{\pgfqpoint{5.436295in}{0.413320in}}%
\pgfpathlineto{\pgfqpoint{5.433616in}{0.413320in}}%
\pgfpathlineto{\pgfqpoint{5.431015in}{0.413320in}}%
\pgfpathlineto{\pgfqpoint{5.428259in}{0.413320in}}%
\pgfpathlineto{\pgfqpoint{5.425661in}{0.413320in}}%
\pgfpathlineto{\pgfqpoint{5.422897in}{0.413320in}}%
\pgfpathlineto{\pgfqpoint{5.420304in}{0.413320in}}%
\pgfpathlineto{\pgfqpoint{5.417547in}{0.413320in}}%
\pgfpathlineto{\pgfqpoint{5.414954in}{0.413320in}}%
\pgfpathlineto{\pgfqpoint{5.412190in}{0.413320in}}%
\pgfpathlineto{\pgfqpoint{5.409507in}{0.413320in}}%
\pgfpathlineto{\pgfqpoint{5.406832in}{0.413320in}}%
\pgfpathlineto{\pgfqpoint{5.404154in}{0.413320in}}%
\pgfpathlineto{\pgfqpoint{5.401576in}{0.413320in}}%
\pgfpathlineto{\pgfqpoint{5.398784in}{0.413320in}}%
\pgfpathlineto{\pgfqpoint{5.396219in}{0.413320in}}%
\pgfpathlineto{\pgfqpoint{5.393441in}{0.413320in}}%
\pgfpathlineto{\pgfqpoint{5.390900in}{0.413320in}}%
\pgfpathlineto{\pgfqpoint{5.388083in}{0.413320in}}%
\pgfpathlineto{\pgfqpoint{5.385550in}{0.413320in}}%
\pgfpathlineto{\pgfqpoint{5.382725in}{0.413320in}}%
\pgfpathlineto{\pgfqpoint{5.380048in}{0.413320in}}%
\pgfpathlineto{\pgfqpoint{5.377370in}{0.413320in}}%
\pgfpathlineto{\pgfqpoint{5.374692in}{0.413320in}}%
\pgfpathlineto{\pgfqpoint{5.372013in}{0.413320in}}%
\pgfpathlineto{\pgfqpoint{5.369335in}{0.413320in}}%
\pgfpathlineto{\pgfqpoint{5.366727in}{0.413320in}}%
\pgfpathlineto{\pgfqpoint{5.363966in}{0.413320in}}%
\pgfpathlineto{\pgfqpoint{5.361370in}{0.413320in}}%
\pgfpathlineto{\pgfqpoint{5.358612in}{0.413320in}}%
\pgfpathlineto{\pgfqpoint{5.356056in}{0.413320in}}%
\pgfpathlineto{\pgfqpoint{5.353262in}{0.413320in}}%
\pgfpathlineto{\pgfqpoint{5.350723in}{0.413320in}}%
\pgfpathlineto{\pgfqpoint{5.347905in}{0.413320in}}%
\pgfpathlineto{\pgfqpoint{5.345224in}{0.413320in}}%
\pgfpathlineto{\pgfqpoint{5.342549in}{0.413320in}}%
\pgfpathlineto{\pgfqpoint{5.339872in}{0.413320in}}%
\pgfpathlineto{\pgfqpoint{5.337353in}{0.413320in}}%
\pgfpathlineto{\pgfqpoint{5.334510in}{0.413320in}}%
\pgfpathlineto{\pgfqpoint{5.331973in}{0.413320in}}%
\pgfpathlineto{\pgfqpoint{5.329159in}{0.413320in}}%
\pgfpathlineto{\pgfqpoint{5.326564in}{0.413320in}}%
\pgfpathlineto{\pgfqpoint{5.323802in}{0.413320in}}%
\pgfpathlineto{\pgfqpoint{5.321256in}{0.413320in}}%
\pgfpathlineto{\pgfqpoint{5.318430in}{0.413320in}}%
\pgfpathlineto{\pgfqpoint{5.315754in}{0.413320in}}%
\pgfpathlineto{\pgfqpoint{5.313089in}{0.413320in}}%
\pgfpathlineto{\pgfqpoint{5.310411in}{0.413320in}}%
\pgfpathlineto{\pgfqpoint{5.307731in}{0.413320in}}%
\pgfpathlineto{\pgfqpoint{5.305054in}{0.413320in}}%
\pgfpathlineto{\pgfqpoint{5.302443in}{0.413320in}}%
\pgfpathlineto{\pgfqpoint{5.299696in}{0.413320in}}%
\pgfpathlineto{\pgfqpoint{5.297140in}{0.413320in}}%
\pgfpathlineto{\pgfqpoint{5.294339in}{0.413320in}}%
\pgfpathlineto{\pgfqpoint{5.291794in}{0.413320in}}%
\pgfpathlineto{\pgfqpoint{5.288984in}{0.413320in}}%
\pgfpathlineto{\pgfqpoint{5.286436in}{0.413320in}}%
\pgfpathlineto{\pgfqpoint{5.283631in}{0.413320in}}%
\pgfpathlineto{\pgfqpoint{5.280947in}{0.413320in}}%
\pgfpathlineto{\pgfqpoint{5.278322in}{0.413320in}}%
\pgfpathlineto{\pgfqpoint{5.275589in}{0.413320in}}%
\pgfpathlineto{\pgfqpoint{5.272913in}{0.413320in}}%
\pgfpathlineto{\pgfqpoint{5.270238in}{0.413320in}}%
\pgfpathlineto{\pgfqpoint{5.267691in}{0.413320in}}%
\pgfpathlineto{\pgfqpoint{5.264876in}{0.413320in}}%
\pgfpathlineto{\pgfqpoint{5.262264in}{0.413320in}}%
\pgfpathlineto{\pgfqpoint{5.259511in}{0.413320in}}%
\pgfpathlineto{\pgfqpoint{5.256973in}{0.413320in}}%
\pgfpathlineto{\pgfqpoint{5.254236in}{0.413320in}}%
\pgfpathlineto{\pgfqpoint{5.251590in}{0.413320in}}%
\pgfpathlineto{\pgfqpoint{5.248816in}{0.413320in}}%
\pgfpathlineto{\pgfqpoint{5.246130in}{0.413320in}}%
\pgfpathlineto{\pgfqpoint{5.243445in}{0.413320in}}%
\pgfpathlineto{\pgfqpoint{5.240777in}{0.413320in}}%
\pgfpathlineto{\pgfqpoint{5.238173in}{0.413320in}}%
\pgfpathlineto{\pgfqpoint{5.235409in}{0.413320in}}%
\pgfpathlineto{\pgfqpoint{5.232855in}{0.413320in}}%
\pgfpathlineto{\pgfqpoint{5.230059in}{0.413320in}}%
\pgfpathlineto{\pgfqpoint{5.227470in}{0.413320in}}%
\pgfpathlineto{\pgfqpoint{5.224695in}{0.413320in}}%
\pgfpathlineto{\pgfqpoint{5.222151in}{0.413320in}}%
\pgfpathlineto{\pgfqpoint{5.219345in}{0.413320in}}%
\pgfpathlineto{\pgfqpoint{5.216667in}{0.413320in}}%
\pgfpathlineto{\pgfqpoint{5.214027in}{0.413320in}}%
\pgfpathlineto{\pgfqpoint{5.211299in}{0.413320in}}%
\pgfpathlineto{\pgfqpoint{5.208630in}{0.413320in}}%
\pgfpathlineto{\pgfqpoint{5.205952in}{0.413320in}}%
\pgfpathlineto{\pgfqpoint{5.203388in}{0.413320in}}%
\pgfpathlineto{\pgfqpoint{5.200594in}{0.413320in}}%
\pgfpathlineto{\pgfqpoint{5.198008in}{0.413320in}}%
\pgfpathlineto{\pgfqpoint{5.195239in}{0.413320in}}%
\pgfpathlineto{\pgfqpoint{5.192680in}{0.413320in}}%
\pgfpathlineto{\pgfqpoint{5.189880in}{0.413320in}}%
\pgfpathlineto{\pgfqpoint{5.187294in}{0.413320in}}%
\pgfpathlineto{\pgfqpoint{5.184522in}{0.413320in}}%
\pgfpathlineto{\pgfqpoint{5.181848in}{0.413320in}}%
\pgfpathlineto{\pgfqpoint{5.179188in}{0.413320in}}%
\pgfpathlineto{\pgfqpoint{5.176477in}{0.413320in}}%
\pgfpathlineto{\pgfqpoint{5.173925in}{0.413320in}}%
\pgfpathlineto{\pgfqpoint{5.171133in}{0.413320in}}%
\pgfpathlineto{\pgfqpoint{5.168591in}{0.413320in}}%
\pgfpathlineto{\pgfqpoint{5.165775in}{0.413320in}}%
\pgfpathlineto{\pgfqpoint{5.163243in}{0.413320in}}%
\pgfpathlineto{\pgfqpoint{5.160420in}{0.413320in}}%
\pgfpathlineto{\pgfqpoint{5.157815in}{0.413320in}}%
\pgfpathlineto{\pgfqpoint{5.155059in}{0.413320in}}%
\pgfpathlineto{\pgfqpoint{5.152382in}{0.413320in}}%
\pgfpathlineto{\pgfqpoint{5.149734in}{0.413320in}}%
\pgfpathlineto{\pgfqpoint{5.147029in}{0.413320in}}%
\pgfpathlineto{\pgfqpoint{5.144349in}{0.413320in}}%
\pgfpathlineto{\pgfqpoint{5.141660in}{0.413320in}}%
\pgfpathlineto{\pgfqpoint{5.139072in}{0.413320in}}%
\pgfpathlineto{\pgfqpoint{5.136311in}{0.413320in}}%
\pgfpathlineto{\pgfqpoint{5.133716in}{0.413320in}}%
\pgfpathlineto{\pgfqpoint{5.130953in}{0.413320in}}%
\pgfpathlineto{\pgfqpoint{5.128421in}{0.413320in}}%
\pgfpathlineto{\pgfqpoint{5.125599in}{0.413320in}}%
\pgfpathlineto{\pgfqpoint{5.123042in}{0.413320in}}%
\pgfpathlineto{\pgfqpoint{5.120243in}{0.413320in}}%
\pgfpathlineto{\pgfqpoint{5.117550in}{0.413320in}}%
\pgfpathlineto{\pgfqpoint{5.114887in}{0.413320in}}%
\pgfpathlineto{\pgfqpoint{5.112209in}{0.413320in}}%
\pgfpathlineto{\pgfqpoint{5.109530in}{0.413320in}}%
\pgfpathlineto{\pgfqpoint{5.106842in}{0.413320in}}%
\pgfpathlineto{\pgfqpoint{5.104312in}{0.413320in}}%
\pgfpathlineto{\pgfqpoint{5.101496in}{0.413320in}}%
\pgfpathlineto{\pgfqpoint{5.098948in}{0.413320in}}%
\pgfpathlineto{\pgfqpoint{5.096142in}{0.413320in}}%
\pgfpathlineto{\pgfqpoint{5.093579in}{0.413320in}}%
\pgfpathlineto{\pgfqpoint{5.090788in}{0.413320in}}%
\pgfpathlineto{\pgfqpoint{5.088103in}{0.413320in}}%
\pgfpathlineto{\pgfqpoint{5.085426in}{0.413320in}}%
\pgfpathlineto{\pgfqpoint{5.082746in}{0.413320in}}%
\pgfpathlineto{\pgfqpoint{5.080067in}{0.413320in}}%
\pgfpathlineto{\pgfqpoint{5.077390in}{0.413320in}}%
\pgfpathlineto{\pgfqpoint{5.074851in}{0.413320in}}%
\pgfpathlineto{\pgfqpoint{5.072030in}{0.413320in}}%
\pgfpathlineto{\pgfqpoint{5.069463in}{0.413320in}}%
\pgfpathlineto{\pgfqpoint{5.066677in}{0.413320in}}%
\pgfpathlineto{\pgfqpoint{5.064144in}{0.413320in}}%
\pgfpathlineto{\pgfqpoint{5.061315in}{0.413320in}}%
\pgfpathlineto{\pgfqpoint{5.058711in}{0.413320in}}%
\pgfpathlineto{\pgfqpoint{5.055952in}{0.413320in}}%
\pgfpathlineto{\pgfqpoint{5.053284in}{0.413320in}}%
\pgfpathlineto{\pgfqpoint{5.050606in}{0.413320in}}%
\pgfpathlineto{\pgfqpoint{5.047924in}{0.413320in}}%
\pgfpathlineto{\pgfqpoint{5.045249in}{0.413320in}}%
\pgfpathlineto{\pgfqpoint{5.042572in}{0.413320in}}%
\pgfpathlineto{\pgfqpoint{5.039962in}{0.413320in}}%
\pgfpathlineto{\pgfqpoint{5.037214in}{0.413320in}}%
\pgfpathlineto{\pgfqpoint{5.034649in}{0.413320in}}%
\pgfpathlineto{\pgfqpoint{5.031849in}{0.413320in}}%
\pgfpathlineto{\pgfqpoint{5.029275in}{0.413320in}}%
\pgfpathlineto{\pgfqpoint{5.026501in}{0.413320in}}%
\pgfpathlineto{\pgfqpoint{5.023927in}{0.413320in}}%
\pgfpathlineto{\pgfqpoint{5.021147in}{0.413320in}}%
\pgfpathlineto{\pgfqpoint{5.018466in}{0.413320in}}%
\pgfpathlineto{\pgfqpoint{5.015820in}{0.413320in}}%
\pgfpathlineto{\pgfqpoint{5.013104in}{0.413320in}}%
\pgfpathlineto{\pgfqpoint{5.010562in}{0.413320in}}%
\pgfpathlineto{\pgfqpoint{5.007751in}{0.413320in}}%
\pgfpathlineto{\pgfqpoint{5.005178in}{0.413320in}}%
\pgfpathlineto{\pgfqpoint{5.002384in}{0.413320in}}%
\pgfpathlineto{\pgfqpoint{4.999780in}{0.413320in}}%
\pgfpathlineto{\pgfqpoint{4.997028in}{0.413320in}}%
\pgfpathlineto{\pgfqpoint{4.994390in}{0.413320in}}%
\pgfpathlineto{\pgfqpoint{4.991683in}{0.413320in}}%
\pgfpathlineto{\pgfqpoint{4.989001in}{0.413320in}}%
\pgfpathlineto{\pgfqpoint{4.986325in}{0.413320in}}%
\pgfpathlineto{\pgfqpoint{4.983637in}{0.413320in}}%
\pgfpathlineto{\pgfqpoint{4.980967in}{0.413320in}}%
\pgfpathlineto{\pgfqpoint{4.978287in}{0.413320in}}%
\pgfpathlineto{\pgfqpoint{4.975703in}{0.413320in}}%
\pgfpathlineto{\pgfqpoint{4.972933in}{0.413320in}}%
\pgfpathlineto{\pgfqpoint{4.970314in}{0.413320in}}%
\pgfpathlineto{\pgfqpoint{4.967575in}{0.413320in}}%
\pgfpathlineto{\pgfqpoint{4.965002in}{0.413320in}}%
\pgfpathlineto{\pgfqpoint{4.962219in}{0.413320in}}%
\pgfpathlineto{\pgfqpoint{4.959689in}{0.413320in}}%
\pgfpathlineto{\pgfqpoint{4.956862in}{0.413320in}}%
\pgfpathlineto{\pgfqpoint{4.954182in}{0.413320in}}%
\pgfpathlineto{\pgfqpoint{4.951504in}{0.413320in}}%
\pgfpathlineto{\pgfqpoint{4.948827in}{0.413320in}}%
\pgfpathlineto{\pgfqpoint{4.946151in}{0.413320in}}%
\pgfpathlineto{\pgfqpoint{4.943466in}{0.413320in}}%
\pgfpathlineto{\pgfqpoint{4.940881in}{0.413320in}}%
\pgfpathlineto{\pgfqpoint{4.938112in}{0.413320in}}%
\pgfpathlineto{\pgfqpoint{4.935515in}{0.413320in}}%
\pgfpathlineto{\pgfqpoint{4.932742in}{0.413320in}}%
\pgfpathlineto{\pgfqpoint{4.930170in}{0.413320in}}%
\pgfpathlineto{\pgfqpoint{4.927400in}{0.413320in}}%
\pgfpathlineto{\pgfqpoint{4.924708in}{0.413320in}}%
\pgfpathlineto{\pgfqpoint{4.922041in}{0.413320in}}%
\pgfpathlineto{\pgfqpoint{4.919352in}{0.413320in}}%
\pgfpathlineto{\pgfqpoint{4.916681in}{0.413320in}}%
\pgfpathlineto{\pgfqpoint{4.914009in}{0.413320in}}%
\pgfpathlineto{\pgfqpoint{4.911435in}{0.413320in}}%
\pgfpathlineto{\pgfqpoint{4.908648in}{0.413320in}}%
\pgfpathlineto{\pgfqpoint{4.906096in}{0.413320in}}%
\pgfpathlineto{\pgfqpoint{4.903295in}{0.413320in}}%
\pgfpathlineto{\pgfqpoint{4.900712in}{0.413320in}}%
\pgfpathlineto{\pgfqpoint{4.897938in}{0.413320in}}%
\pgfpathlineto{\pgfqpoint{4.895399in}{0.413320in}}%
\pgfpathlineto{\pgfqpoint{4.892611in}{0.413320in}}%
\pgfpathlineto{\pgfqpoint{4.889902in}{0.413320in}}%
\pgfpathlineto{\pgfqpoint{4.887211in}{0.413320in}}%
\pgfpathlineto{\pgfqpoint{4.884540in}{0.413320in}}%
\pgfpathlineto{\pgfqpoint{4.881864in}{0.413320in}}%
\pgfpathlineto{\pgfqpoint{4.879180in}{0.413320in}}%
\pgfpathlineto{\pgfqpoint{4.876636in}{0.413320in}}%
\pgfpathlineto{\pgfqpoint{4.873832in}{0.413320in}}%
\pgfpathlineto{\pgfqpoint{4.871209in}{0.413320in}}%
\pgfpathlineto{\pgfqpoint{4.868474in}{0.413320in}}%
\pgfpathlineto{\pgfqpoint{4.865910in}{0.413320in}}%
\pgfpathlineto{\pgfqpoint{4.863116in}{0.413320in}}%
\pgfpathlineto{\pgfqpoint{4.860544in}{0.413320in}}%
\pgfpathlineto{\pgfqpoint{4.857807in}{0.413320in}}%
\pgfpathlineto{\pgfqpoint{4.855070in}{0.413320in}}%
\pgfpathlineto{\pgfqpoint{4.852404in}{0.413320in}}%
\pgfpathlineto{\pgfqpoint{4.849715in}{0.413320in}}%
\pgfpathlineto{\pgfqpoint{4.847127in}{0.413320in}}%
\pgfpathlineto{\pgfqpoint{4.844361in}{0.413320in}}%
\pgfpathlineto{\pgfqpoint{4.842380in}{0.413320in}}%
\pgfpathlineto{\pgfqpoint{4.839922in}{0.413320in}}%
\pgfpathlineto{\pgfqpoint{4.837992in}{0.413320in}}%
\pgfpathlineto{\pgfqpoint{4.833657in}{0.413320in}}%
\pgfpathlineto{\pgfqpoint{4.831045in}{0.413320in}}%
\pgfpathlineto{\pgfqpoint{4.828291in}{0.413320in}}%
\pgfpathlineto{\pgfqpoint{4.825619in}{0.413320in}}%
\pgfpathlineto{\pgfqpoint{4.822945in}{0.413320in}}%
\pgfpathlineto{\pgfqpoint{4.820265in}{0.413320in}}%
\pgfpathlineto{\pgfqpoint{4.817587in}{0.413320in}}%
\pgfpathlineto{\pgfqpoint{4.814907in}{0.413320in}}%
\pgfpathlineto{\pgfqpoint{4.812377in}{0.413320in}}%
\pgfpathlineto{\pgfqpoint{4.809538in}{0.413320in}}%
\pgfpathlineto{\pgfqpoint{4.807017in}{0.413320in}}%
\pgfpathlineto{\pgfqpoint{4.804193in}{0.413320in}}%
\pgfpathlineto{\pgfqpoint{4.801586in}{0.413320in}}%
\pgfpathlineto{\pgfqpoint{4.798830in}{0.413320in}}%
\pgfpathlineto{\pgfqpoint{4.796274in}{0.413320in}}%
\pgfpathlineto{\pgfqpoint{4.793512in}{0.413320in}}%
\pgfpathlineto{\pgfqpoint{4.790798in}{0.413320in}}%
\pgfpathlineto{\pgfqpoint{4.788116in}{0.413320in}}%
\pgfpathlineto{\pgfqpoint{4.785445in}{0.413320in}}%
\pgfpathlineto{\pgfqpoint{4.782872in}{0.413320in}}%
\pgfpathlineto{\pgfqpoint{4.780083in}{0.413320in}}%
\pgfpathlineto{\pgfqpoint{4.777535in}{0.413320in}}%
\pgfpathlineto{\pgfqpoint{4.774732in}{0.413320in}}%
\pgfpathlineto{\pgfqpoint{4.772198in}{0.413320in}}%
\pgfpathlineto{\pgfqpoint{4.769367in}{0.413320in}}%
\pgfpathlineto{\pgfqpoint{4.766783in}{0.413320in}}%
\pgfpathlineto{\pgfqpoint{4.764018in}{0.413320in}}%
\pgfpathlineto{\pgfqpoint{4.761337in}{0.413320in}}%
\pgfpathlineto{\pgfqpoint{4.758653in}{0.413320in}}%
\pgfpathlineto{\pgfqpoint{4.755983in}{0.413320in}}%
\pgfpathlineto{\pgfqpoint{4.753298in}{0.413320in}}%
\pgfpathlineto{\pgfqpoint{4.750627in}{0.413320in}}%
\pgfpathlineto{\pgfqpoint{4.748081in}{0.413320in}}%
\pgfpathlineto{\pgfqpoint{4.745256in}{0.413320in}}%
\pgfpathlineto{\pgfqpoint{4.742696in}{0.413320in}}%
\pgfpathlineto{\pgfqpoint{4.739912in}{0.413320in}}%
\pgfpathlineto{\pgfqpoint{4.737348in}{0.413320in}}%
\pgfpathlineto{\pgfqpoint{4.734552in}{0.413320in}}%
\pgfpathlineto{\pgfqpoint{4.731901in}{0.413320in}}%
\pgfpathlineto{\pgfqpoint{4.729233in}{0.413320in}}%
\pgfpathlineto{\pgfqpoint{4.726508in}{0.413320in}}%
\pgfpathlineto{\pgfqpoint{4.723873in}{0.413320in}}%
\pgfpathlineto{\pgfqpoint{4.721160in}{0.413320in}}%
\pgfpathlineto{\pgfqpoint{4.718486in}{0.413320in}}%
\pgfpathlineto{\pgfqpoint{4.715806in}{0.413320in}}%
\pgfpathlineto{\pgfqpoint{4.713275in}{0.413320in}}%
\pgfpathlineto{\pgfqpoint{4.710437in}{0.413320in}}%
\pgfpathlineto{\pgfqpoint{4.707824in}{0.413320in}}%
\pgfpathlineto{\pgfqpoint{4.705094in}{0.413320in}}%
\pgfpathlineto{\pgfqpoint{4.702517in}{0.413320in}}%
\pgfpathlineto{\pgfqpoint{4.699734in}{0.413320in}}%
\pgfpathlineto{\pgfqpoint{4.697170in}{0.413320in}}%
\pgfpathlineto{\pgfqpoint{4.694381in}{0.413320in}}%
\pgfpathlineto{\pgfqpoint{4.691694in}{0.413320in}}%
\pgfpathlineto{\pgfqpoint{4.689051in}{0.413320in}}%
\pgfpathlineto{\pgfqpoint{4.686337in}{0.413320in}}%
\pgfpathlineto{\pgfqpoint{4.683799in}{0.413320in}}%
\pgfpathlineto{\pgfqpoint{4.680988in}{0.413320in}}%
\pgfpathlineto{\pgfqpoint{4.678448in}{0.413320in}}%
\pgfpathlineto{\pgfqpoint{4.675619in}{0.413320in}}%
\pgfpathlineto{\pgfqpoint{4.673068in}{0.413320in}}%
\pgfpathlineto{\pgfqpoint{4.670261in}{0.413320in}}%
\pgfpathlineto{\pgfqpoint{4.667764in}{0.413320in}}%
\pgfpathlineto{\pgfqpoint{4.664923in}{0.413320in}}%
\pgfpathlineto{\pgfqpoint{4.662237in}{0.413320in}}%
\pgfpathlineto{\pgfqpoint{4.659590in}{0.413320in}}%
\pgfpathlineto{\pgfqpoint{4.656873in}{0.413320in}}%
\pgfpathlineto{\pgfqpoint{4.654203in}{0.413320in}}%
\pgfpathlineto{\pgfqpoint{4.651524in}{0.413320in}}%
\pgfpathlineto{\pgfqpoint{4.648922in}{0.413320in}}%
\pgfpathlineto{\pgfqpoint{4.646169in}{0.413320in}}%
\pgfpathlineto{\pgfqpoint{4.643628in}{0.413320in}}%
\pgfpathlineto{\pgfqpoint{4.640809in}{0.413320in}}%
\pgfpathlineto{\pgfqpoint{4.638204in}{0.413320in}}%
\pgfpathlineto{\pgfqpoint{4.635445in}{0.413320in}}%
\pgfpathlineto{\pgfqpoint{4.632902in}{0.413320in}}%
\pgfpathlineto{\pgfqpoint{4.630096in}{0.413320in}}%
\pgfpathlineto{\pgfqpoint{4.627411in}{0.413320in}}%
\pgfpathlineto{\pgfqpoint{4.624741in}{0.413320in}}%
\pgfpathlineto{\pgfqpoint{4.622056in}{0.413320in}}%
\pgfpathlineto{\pgfqpoint{4.619529in}{0.413320in}}%
\pgfpathlineto{\pgfqpoint{4.616702in}{0.413320in}}%
\pgfpathlineto{\pgfqpoint{4.614134in}{0.413320in}}%
\pgfpathlineto{\pgfqpoint{4.611350in}{0.413320in}}%
\pgfpathlineto{\pgfqpoint{4.608808in}{0.413320in}}%
\pgfpathlineto{\pgfqpoint{4.605990in}{0.413320in}}%
\pgfpathlineto{\pgfqpoint{4.603430in}{0.413320in}}%
\pgfpathlineto{\pgfqpoint{4.600633in}{0.413320in}}%
\pgfpathlineto{\pgfqpoint{4.597951in}{0.413320in}}%
\pgfpathlineto{\pgfqpoint{4.595281in}{0.413320in}}%
\pgfpathlineto{\pgfqpoint{4.592589in}{0.413320in}}%
\pgfpathlineto{\pgfqpoint{4.589920in}{0.413320in}}%
\pgfpathlineto{\pgfqpoint{4.587244in}{0.413320in}}%
\pgfpathlineto{\pgfqpoint{4.584672in}{0.413320in}}%
\pgfpathlineto{\pgfqpoint{4.581888in}{0.413320in}}%
\pgfpathlineto{\pgfqpoint{4.579305in}{0.413320in}}%
\pgfpathlineto{\pgfqpoint{4.576531in}{0.413320in}}%
\pgfpathlineto{\pgfqpoint{4.573947in}{0.413320in}}%
\pgfpathlineto{\pgfqpoint{4.571171in}{0.413320in}}%
\pgfpathlineto{\pgfqpoint{4.568612in}{0.413320in}}%
\pgfpathlineto{\pgfqpoint{4.565820in}{0.413320in}}%
\pgfpathlineto{\pgfqpoint{4.563125in}{0.413320in}}%
\pgfpathlineto{\pgfqpoint{4.560448in}{0.413320in}}%
\pgfpathlineto{\pgfqpoint{4.557777in}{0.413320in}}%
\pgfpathlineto{\pgfqpoint{4.555106in}{0.413320in}}%
\pgfpathlineto{\pgfqpoint{4.552425in}{0.413320in}}%
\pgfpathlineto{\pgfqpoint{4.549822in}{0.413320in}}%
\pgfpathlineto{\pgfqpoint{4.547064in}{0.413320in}}%
\pgfpathlineto{\pgfqpoint{4.544464in}{0.413320in}}%
\pgfpathlineto{\pgfqpoint{4.541711in}{0.413320in}}%
\pgfpathlineto{\pgfqpoint{4.539144in}{0.413320in}}%
\pgfpathlineto{\pgfqpoint{4.536400in}{0.413320in}}%
\pgfpathlineto{\pgfqpoint{4.533764in}{0.413320in}}%
\pgfpathlineto{\pgfqpoint{4.530990in}{0.413320in}}%
\pgfpathlineto{\pgfqpoint{4.528307in}{0.413320in}}%
\pgfpathlineto{\pgfqpoint{4.525640in}{0.413320in}}%
\pgfpathlineto{\pgfqpoint{4.522962in}{0.413320in}}%
\pgfpathlineto{\pgfqpoint{4.520345in}{0.413320in}}%
\pgfpathlineto{\pgfqpoint{4.517598in}{0.413320in}}%
\pgfpathlineto{\pgfqpoint{4.515080in}{0.413320in}}%
\pgfpathlineto{\pgfqpoint{4.512246in}{0.413320in}}%
\pgfpathlineto{\pgfqpoint{4.509643in}{0.413320in}}%
\pgfpathlineto{\pgfqpoint{4.506893in}{0.413320in}}%
\pgfpathlineto{\pgfqpoint{4.504305in}{0.413320in}}%
\pgfpathlineto{\pgfqpoint{4.501529in}{0.413320in}}%
\pgfpathlineto{\pgfqpoint{4.498850in}{0.413320in}}%
\pgfpathlineto{\pgfqpoint{4.496167in}{0.413320in}}%
\pgfpathlineto{\pgfqpoint{4.493492in}{0.413320in}}%
\pgfpathlineto{\pgfqpoint{4.490822in}{0.413320in}}%
\pgfpathlineto{\pgfqpoint{4.488130in}{0.413320in}}%
\pgfpathlineto{\pgfqpoint{4.485581in}{0.413320in}}%
\pgfpathlineto{\pgfqpoint{4.482778in}{0.413320in}}%
\pgfpathlineto{\pgfqpoint{4.480201in}{0.413320in}}%
\pgfpathlineto{\pgfqpoint{4.477430in}{0.413320in}}%
\pgfpathlineto{\pgfqpoint{4.474861in}{0.413320in}}%
\pgfpathlineto{\pgfqpoint{4.472059in}{0.413320in}}%
\pgfpathlineto{\pgfqpoint{4.469492in}{0.413320in}}%
\pgfpathlineto{\pgfqpoint{4.466717in}{0.413320in}}%
\pgfpathlineto{\pgfqpoint{4.464029in}{0.413320in}}%
\pgfpathlineto{\pgfqpoint{4.461367in}{0.413320in}}%
\pgfpathlineto{\pgfqpoint{4.458681in}{0.413320in}}%
\pgfpathlineto{\pgfqpoint{4.456138in}{0.413320in}}%
\pgfpathlineto{\pgfqpoint{4.453312in}{0.413320in}}%
\pgfpathlineto{\pgfqpoint{4.450767in}{0.413320in}}%
\pgfpathlineto{\pgfqpoint{4.447965in}{0.413320in}}%
\pgfpathlineto{\pgfqpoint{4.445423in}{0.413320in}}%
\pgfpathlineto{\pgfqpoint{4.442611in}{0.413320in}}%
\pgfpathlineto{\pgfqpoint{4.440041in}{0.413320in}}%
\pgfpathlineto{\pgfqpoint{4.437253in}{0.413320in}}%
\pgfpathlineto{\pgfqpoint{4.434569in}{0.413320in}}%
\pgfpathlineto{\pgfqpoint{4.431901in}{0.413320in}}%
\pgfpathlineto{\pgfqpoint{4.429220in}{0.413320in}}%
\pgfpathlineto{\pgfqpoint{4.426534in}{0.413320in}}%
\pgfpathlineto{\pgfqpoint{4.423863in}{0.413320in}}%
\pgfpathlineto{\pgfqpoint{4.421292in}{0.413320in}}%
\pgfpathlineto{\pgfqpoint{4.418506in}{0.413320in}}%
\pgfpathlineto{\pgfqpoint{4.415932in}{0.413320in}}%
\pgfpathlineto{\pgfqpoint{4.413149in}{0.413320in}}%
\pgfpathlineto{\pgfqpoint{4.410587in}{0.413320in}}%
\pgfpathlineto{\pgfqpoint{4.407788in}{0.413320in}}%
\pgfpathlineto{\pgfqpoint{4.405234in}{0.413320in}}%
\pgfpathlineto{\pgfqpoint{4.402468in}{0.413320in}}%
\pgfpathlineto{\pgfqpoint{4.399745in}{0.413320in}}%
\pgfpathlineto{\pgfqpoint{4.397076in}{0.413320in}}%
\pgfpathlineto{\pgfqpoint{4.394400in}{0.413320in}}%
\pgfpathlineto{\pgfqpoint{4.391721in}{0.413320in}}%
\pgfpathlineto{\pgfqpoint{4.389044in}{0.413320in}}%
\pgfpathlineto{\pgfqpoint{4.386431in}{0.413320in}}%
\pgfpathlineto{\pgfqpoint{4.383674in}{0.413320in}}%
\pgfpathlineto{\pgfqpoint{4.381097in}{0.413320in}}%
\pgfpathlineto{\pgfqpoint{4.378329in}{0.413320in}}%
\pgfpathlineto{\pgfqpoint{4.375761in}{0.413320in}}%
\pgfpathlineto{\pgfqpoint{4.372976in}{0.413320in}}%
\pgfpathlineto{\pgfqpoint{4.370437in}{0.413320in}}%
\pgfpathlineto{\pgfqpoint{4.367646in}{0.413320in}}%
\pgfpathlineto{\pgfqpoint{4.364936in}{0.413320in}}%
\pgfpathlineto{\pgfqpoint{4.362270in}{0.413320in}}%
\pgfpathlineto{\pgfqpoint{4.359582in}{0.413320in}}%
\pgfpathlineto{\pgfqpoint{4.357014in}{0.413320in}}%
\pgfpathlineto{\pgfqpoint{4.354224in}{0.413320in}}%
\pgfpathlineto{\pgfqpoint{4.351645in}{0.413320in}}%
\pgfpathlineto{\pgfqpoint{4.348868in}{0.413320in}}%
\pgfpathlineto{\pgfqpoint{4.346263in}{0.413320in}}%
\pgfpathlineto{\pgfqpoint{4.343510in}{0.413320in}}%
\pgfpathlineto{\pgfqpoint{4.340976in}{0.413320in}}%
\pgfpathlineto{\pgfqpoint{4.338154in}{0.413320in}}%
\pgfpathlineto{\pgfqpoint{4.335463in}{0.413320in}}%
\pgfpathlineto{\pgfqpoint{4.332796in}{0.413320in}}%
\pgfpathlineto{\pgfqpoint{4.330118in}{0.413320in}}%
\pgfpathlineto{\pgfqpoint{4.327440in}{0.413320in}}%
\pgfpathlineto{\pgfqpoint{4.324760in}{0.413320in}}%
\pgfpathlineto{\pgfqpoint{4.322181in}{0.413320in}}%
\pgfpathlineto{\pgfqpoint{4.319405in}{0.413320in}}%
\pgfpathlineto{\pgfqpoint{4.316856in}{0.413320in}}%
\pgfpathlineto{\pgfqpoint{4.314032in}{0.413320in}}%
\pgfpathlineto{\pgfqpoint{4.311494in}{0.413320in}}%
\pgfpathlineto{\pgfqpoint{4.308691in}{0.413320in}}%
\pgfpathlineto{\pgfqpoint{4.306118in}{0.413320in}}%
\pgfpathlineto{\pgfqpoint{4.303357in}{0.413320in}}%
\pgfpathlineto{\pgfqpoint{4.300656in}{0.413320in}}%
\pgfpathlineto{\pgfqpoint{4.297977in}{0.413320in}}%
\pgfpathlineto{\pgfqpoint{4.295299in}{0.413320in}}%
\pgfpathlineto{\pgfqpoint{4.292786in}{0.413320in}}%
\pgfpathlineto{\pgfqpoint{4.289936in}{0.413320in}}%
\pgfpathlineto{\pgfqpoint{4.287399in}{0.413320in}}%
\pgfpathlineto{\pgfqpoint{4.284586in}{0.413320in}}%
\pgfpathlineto{\pgfqpoint{4.282000in}{0.413320in}}%
\pgfpathlineto{\pgfqpoint{4.279212in}{0.413320in}}%
\pgfpathlineto{\pgfqpoint{4.276635in}{0.413320in}}%
\pgfpathlineto{\pgfqpoint{4.273874in}{0.413320in}}%
\pgfpathlineto{\pgfqpoint{4.271187in}{0.413320in}}%
\pgfpathlineto{\pgfqpoint{4.268590in}{0.413320in}}%
\pgfpathlineto{\pgfqpoint{4.265824in}{0.413320in}}%
\pgfpathlineto{\pgfqpoint{4.263157in}{0.413320in}}%
\pgfpathlineto{\pgfqpoint{4.260477in}{0.413320in}}%
\pgfpathlineto{\pgfqpoint{4.257958in}{0.413320in}}%
\pgfpathlineto{\pgfqpoint{4.255120in}{0.413320in}}%
\pgfpathlineto{\pgfqpoint{4.252581in}{0.413320in}}%
\pgfpathlineto{\pgfqpoint{4.249767in}{0.413320in}}%
\pgfpathlineto{\pgfqpoint{4.247225in}{0.413320in}}%
\pgfpathlineto{\pgfqpoint{4.244394in}{0.413320in}}%
\pgfpathlineto{\pgfqpoint{4.241900in}{0.413320in}}%
\pgfpathlineto{\pgfqpoint{4.239084in}{0.413320in}}%
\pgfpathlineto{\pgfqpoint{4.236375in}{0.413320in}}%
\pgfpathlineto{\pgfqpoint{4.233691in}{0.413320in}}%
\pgfpathlineto{\pgfqpoint{4.231013in}{0.413320in}}%
\pgfpathlineto{\pgfqpoint{4.228331in}{0.413320in}}%
\pgfpathlineto{\pgfqpoint{4.225654in}{0.413320in}}%
\pgfpathlineto{\pgfqpoint{4.223082in}{0.413320in}}%
\pgfpathlineto{\pgfqpoint{4.220304in}{0.413320in}}%
\pgfpathlineto{\pgfqpoint{4.217694in}{0.413320in}}%
\pgfpathlineto{\pgfqpoint{4.214948in}{0.413320in}}%
\pgfpathlineto{\pgfqpoint{4.212383in}{0.413320in}}%
\pgfpathlineto{\pgfqpoint{4.209597in}{0.413320in}}%
\pgfpathlineto{\pgfqpoint{4.207076in}{0.413320in}}%
\pgfpathlineto{\pgfqpoint{4.204240in}{0.413320in}}%
\pgfpathlineto{\pgfqpoint{4.201542in}{0.413320in}}%
\pgfpathlineto{\pgfqpoint{4.198878in}{0.413320in}}%
\pgfpathlineto{\pgfqpoint{4.196186in}{0.413320in}}%
\pgfpathlineto{\pgfqpoint{4.193638in}{0.413320in}}%
\pgfpathlineto{\pgfqpoint{4.190842in}{0.413320in}}%
\pgfpathlineto{\pgfqpoint{4.188318in}{0.413320in}}%
\pgfpathlineto{\pgfqpoint{4.185481in}{0.413320in}}%
\pgfpathlineto{\pgfqpoint{4.182899in}{0.413320in}}%
\pgfpathlineto{\pgfqpoint{4.180129in}{0.413320in}}%
\pgfpathlineto{\pgfqpoint{4.177593in}{0.413320in}}%
\pgfpathlineto{\pgfqpoint{4.174770in}{0.413320in}}%
\pgfpathlineto{\pgfqpoint{4.172093in}{0.413320in}}%
\pgfpathlineto{\pgfqpoint{4.169415in}{0.413320in}}%
\pgfpathlineto{\pgfqpoint{4.166737in}{0.413320in}}%
\pgfpathlineto{\pgfqpoint{4.164059in}{0.413320in}}%
\pgfpathlineto{\pgfqpoint{4.161380in}{0.413320in}}%
\pgfpathlineto{\pgfqpoint{4.158806in}{0.413320in}}%
\pgfpathlineto{\pgfqpoint{4.156016in}{0.413320in}}%
\pgfpathlineto{\pgfqpoint{4.153423in}{0.413320in}}%
\pgfpathlineto{\pgfqpoint{4.150665in}{0.413320in}}%
\pgfpathlineto{\pgfqpoint{4.148082in}{0.413320in}}%
\pgfpathlineto{\pgfqpoint{4.145310in}{0.413320in}}%
\pgfpathlineto{\pgfqpoint{4.142713in}{0.413320in}}%
\pgfpathlineto{\pgfqpoint{4.139963in}{0.413320in}}%
\pgfpathlineto{\pgfqpoint{4.137272in}{0.413320in}}%
\pgfpathlineto{\pgfqpoint{4.134615in}{0.413320in}}%
\pgfpathlineto{\pgfqpoint{4.131920in}{0.413320in}}%
\pgfpathlineto{\pgfqpoint{4.129349in}{0.413320in}}%
\pgfpathlineto{\pgfqpoint{4.126553in}{0.413320in}}%
\pgfpathlineto{\pgfqpoint{4.124019in}{0.413320in}}%
\pgfpathlineto{\pgfqpoint{4.121205in}{0.413320in}}%
\pgfpathlineto{\pgfqpoint{4.118554in}{0.413320in}}%
\pgfpathlineto{\pgfqpoint{4.115844in}{0.413320in}}%
\pgfpathlineto{\pgfqpoint{4.113252in}{0.413320in}}%
\pgfpathlineto{\pgfqpoint{4.110488in}{0.413320in}}%
\pgfpathlineto{\pgfqpoint{4.107814in}{0.413320in}}%
\pgfpathlineto{\pgfqpoint{4.105185in}{0.413320in}}%
\pgfpathlineto{\pgfqpoint{4.102456in}{0.413320in}}%
\pgfpathlineto{\pgfqpoint{4.099777in}{0.413320in}}%
\pgfpathlineto{\pgfqpoint{4.097092in}{0.413320in}}%
\pgfpathlineto{\pgfqpoint{4.094527in}{0.413320in}}%
\pgfpathlineto{\pgfqpoint{4.091729in}{0.413320in}}%
\pgfpathlineto{\pgfqpoint{4.089159in}{0.413320in}}%
\pgfpathlineto{\pgfqpoint{4.086385in}{0.413320in}}%
\pgfpathlineto{\pgfqpoint{4.083870in}{0.413320in}}%
\pgfpathlineto{\pgfqpoint{4.081018in}{0.413320in}}%
\pgfpathlineto{\pgfqpoint{4.078471in}{0.413320in}}%
\pgfpathlineto{\pgfqpoint{4.075705in}{0.413320in}}%
\pgfpathlineto{\pgfqpoint{4.072985in}{0.413320in}}%
\pgfpathlineto{\pgfqpoint{4.070313in}{0.413320in}}%
\pgfpathlineto{\pgfqpoint{4.067636in}{0.413320in}}%
\pgfpathlineto{\pgfqpoint{4.064957in}{0.413320in}}%
\pgfpathlineto{\pgfqpoint{4.062266in}{0.413320in}}%
\pgfpathlineto{\pgfqpoint{4.059702in}{0.413320in}}%
\pgfpathlineto{\pgfqpoint{4.056911in}{0.413320in}}%
\pgfpathlineto{\pgfqpoint{4.054326in}{0.413320in}}%
\pgfpathlineto{\pgfqpoint{4.051557in}{0.413320in}}%
\pgfpathlineto{\pgfqpoint{4.049006in}{0.413320in}}%
\pgfpathlineto{\pgfqpoint{4.046210in}{0.413320in}}%
\pgfpathlineto{\pgfqpoint{4.043667in}{0.413320in}}%
\pgfpathlineto{\pgfqpoint{4.040852in}{0.413320in}}%
\pgfpathlineto{\pgfqpoint{4.038174in}{0.413320in}}%
\pgfpathlineto{\pgfqpoint{4.035492in}{0.413320in}}%
\pgfpathlineto{\pgfqpoint{4.032817in}{0.413320in}}%
\pgfpathlineto{\pgfqpoint{4.030229in}{0.413320in}}%
\pgfpathlineto{\pgfqpoint{4.027447in}{0.413320in}}%
\pgfpathlineto{\pgfqpoint{4.024868in}{0.413320in}}%
\pgfpathlineto{\pgfqpoint{4.022097in}{0.413320in}}%
\pgfpathlineto{\pgfqpoint{4.019518in}{0.413320in}}%
\pgfpathlineto{\pgfqpoint{4.016744in}{0.413320in}}%
\pgfpathlineto{\pgfqpoint{4.014186in}{0.413320in}}%
\pgfpathlineto{\pgfqpoint{4.011394in}{0.413320in}}%
\pgfpathlineto{\pgfqpoint{4.008699in}{0.413320in}}%
\pgfpathlineto{\pgfqpoint{4.006034in}{0.413320in}}%
\pgfpathlineto{\pgfqpoint{4.003348in}{0.413320in}}%
\pgfpathlineto{\pgfqpoint{4.000674in}{0.413320in}}%
\pgfpathlineto{\pgfqpoint{3.997990in}{0.413320in}}%
\pgfpathlineto{\pgfqpoint{3.995417in}{0.413320in}}%
\pgfpathlineto{\pgfqpoint{3.992642in}{0.413320in}}%
\pgfpathlineto{\pgfqpoint{3.990055in}{0.413320in}}%
\pgfpathlineto{\pgfqpoint{3.987270in}{0.413320in}}%
\pgfpathlineto{\pgfqpoint{3.984714in}{0.413320in}}%
\pgfpathlineto{\pgfqpoint{3.981929in}{0.413320in}}%
\pgfpathlineto{\pgfqpoint{3.979389in}{0.413320in}}%
\pgfpathlineto{\pgfqpoint{3.976563in}{0.413320in}}%
\pgfpathlineto{\pgfqpoint{3.973885in}{0.413320in}}%
\pgfpathlineto{\pgfqpoint{3.971250in}{0.413320in}}%
\pgfpathlineto{\pgfqpoint{3.968523in}{0.413320in}}%
\pgfpathlineto{\pgfqpoint{3.966013in}{0.413320in}}%
\pgfpathlineto{\pgfqpoint{3.963176in}{0.413320in}}%
\pgfpathlineto{\pgfqpoint{3.960635in}{0.413320in}}%
\pgfpathlineto{\pgfqpoint{3.957823in}{0.413320in}}%
\pgfpathlineto{\pgfqpoint{3.955211in}{0.413320in}}%
\pgfpathlineto{\pgfqpoint{3.952464in}{0.413320in}}%
\pgfpathlineto{\pgfqpoint{3.949894in}{0.413320in}}%
\pgfpathlineto{\pgfqpoint{3.947101in}{0.413320in}}%
\pgfpathlineto{\pgfqpoint{3.944431in}{0.413320in}}%
\pgfpathlineto{\pgfqpoint{3.941778in}{0.413320in}}%
\pgfpathlineto{\pgfqpoint{3.939075in}{0.413320in}}%
\pgfpathlineto{\pgfqpoint{3.936395in}{0.413320in}}%
\pgfpathlineto{\pgfqpoint{3.933714in}{0.413320in}}%
\pgfpathlineto{\pgfqpoint{3.931202in}{0.413320in}}%
\pgfpathlineto{\pgfqpoint{3.928347in}{0.413320in}}%
\pgfpathlineto{\pgfqpoint{3.925778in}{0.413320in}}%
\pgfpathlineto{\pgfqpoint{3.923005in}{0.413320in}}%
\pgfpathlineto{\pgfqpoint{3.920412in}{0.413320in}}%
\pgfpathlineto{\pgfqpoint{3.917646in}{0.413320in}}%
\pgfpathlineto{\pgfqpoint{3.915107in}{0.413320in}}%
\pgfpathlineto{\pgfqpoint{3.912296in}{0.413320in}}%
\pgfpathlineto{\pgfqpoint{3.909602in}{0.413320in}}%
\pgfpathlineto{\pgfqpoint{3.906918in}{0.413320in}}%
\pgfpathlineto{\pgfqpoint{3.904252in}{0.413320in}}%
\pgfpathlineto{\pgfqpoint{3.901573in}{0.413320in}}%
\pgfpathlineto{\pgfqpoint{3.898891in}{0.413320in}}%
\pgfpathlineto{\pgfqpoint{3.896345in}{0.413320in}}%
\pgfpathlineto{\pgfqpoint{3.893541in}{0.413320in}}%
\pgfpathlineto{\pgfqpoint{3.890926in}{0.413320in}}%
\pgfpathlineto{\pgfqpoint{3.888188in}{0.413320in}}%
\pgfpathlineto{\pgfqpoint{3.885621in}{0.413320in}}%
\pgfpathlineto{\pgfqpoint{3.882850in}{0.413320in}}%
\pgfpathlineto{\pgfqpoint{3.880237in}{0.413320in}}%
\pgfpathlineto{\pgfqpoint{3.877466in}{0.413320in}}%
\pgfpathlineto{\pgfqpoint{3.874790in}{0.413320in}}%
\pgfpathlineto{\pgfqpoint{3.872114in}{0.413320in}}%
\pgfpathlineto{\pgfqpoint{3.869435in}{0.413320in}}%
\pgfpathlineto{\pgfqpoint{3.866815in}{0.413320in}}%
\pgfpathlineto{\pgfqpoint{3.864073in}{0.413320in}}%
\pgfpathlineto{\pgfqpoint{3.861561in}{0.413320in}}%
\pgfpathlineto{\pgfqpoint{3.858720in}{0.413320in}}%
\pgfpathlineto{\pgfqpoint{3.856100in}{0.413320in}}%
\pgfpathlineto{\pgfqpoint{3.853358in}{0.413320in}}%
\pgfpathlineto{\pgfqpoint{3.850814in}{0.413320in}}%
\pgfpathlineto{\pgfqpoint{3.848005in}{0.413320in}}%
\pgfpathlineto{\pgfqpoint{3.845329in}{0.413320in}}%
\pgfpathlineto{\pgfqpoint{3.842641in}{0.413320in}}%
\pgfpathlineto{\pgfqpoint{3.839960in}{0.413320in}}%
\pgfpathlineto{\pgfqpoint{3.837286in}{0.413320in}}%
\pgfpathlineto{\pgfqpoint{3.834616in}{0.413320in}}%
\pgfpathlineto{\pgfqpoint{3.832053in}{0.413320in}}%
\pgfpathlineto{\pgfqpoint{3.829252in}{0.413320in}}%
\pgfpathlineto{\pgfqpoint{3.826679in}{0.413320in}}%
\pgfpathlineto{\pgfqpoint{3.823903in}{0.413320in}}%
\pgfpathlineto{\pgfqpoint{3.821315in}{0.413320in}}%
\pgfpathlineto{\pgfqpoint{3.818546in}{0.413320in}}%
\pgfpathlineto{\pgfqpoint{3.815983in}{0.413320in}}%
\pgfpathlineto{\pgfqpoint{3.813172in}{0.413320in}}%
\pgfpathlineto{\pgfqpoint{3.810510in}{0.413320in}}%
\pgfpathlineto{\pgfqpoint{3.807832in}{0.413320in}}%
\pgfpathlineto{\pgfqpoint{3.805145in}{0.413320in}}%
\pgfpathlineto{\pgfqpoint{3.802569in}{0.413320in}}%
\pgfpathlineto{\pgfqpoint{3.799797in}{0.413320in}}%
\pgfpathlineto{\pgfqpoint{3.797265in}{0.413320in}}%
\pgfpathlineto{\pgfqpoint{3.794435in}{0.413320in}}%
\pgfpathlineto{\pgfqpoint{3.791897in}{0.413320in}}%
\pgfpathlineto{\pgfqpoint{3.789084in}{0.413320in}}%
\pgfpathlineto{\pgfqpoint{3.786504in}{0.413320in}}%
\pgfpathlineto{\pgfqpoint{3.783725in}{0.413320in}}%
\pgfpathlineto{\pgfqpoint{3.781046in}{0.413320in}}%
\pgfpathlineto{\pgfqpoint{3.778370in}{0.413320in}}%
\pgfpathlineto{\pgfqpoint{3.775691in}{0.413320in}}%
\pgfpathlineto{\pgfqpoint{3.773014in}{0.413320in}}%
\pgfpathlineto{\pgfqpoint{3.770323in}{0.413320in}}%
\pgfpathlineto{\pgfqpoint{3.767782in}{0.413320in}}%
\pgfpathlineto{\pgfqpoint{3.764966in}{0.413320in}}%
\pgfpathlineto{\pgfqpoint{3.762389in}{0.413320in}}%
\pgfpathlineto{\pgfqpoint{3.759622in}{0.413320in}}%
\pgfpathlineto{\pgfqpoint{3.757065in}{0.413320in}}%
\pgfpathlineto{\pgfqpoint{3.754265in}{0.413320in}}%
\pgfpathlineto{\pgfqpoint{3.751728in}{0.413320in}}%
\pgfpathlineto{\pgfqpoint{3.748903in}{0.413320in}}%
\pgfpathlineto{\pgfqpoint{3.746229in}{0.413320in}}%
\pgfpathlineto{\pgfqpoint{3.743548in}{0.413320in}}%
\pgfpathlineto{\pgfqpoint{3.740874in}{0.413320in}}%
\pgfpathlineto{\pgfqpoint{3.738194in}{0.413320in}}%
\pgfpathlineto{\pgfqpoint{3.735509in}{0.413320in}}%
\pgfpathlineto{\pgfqpoint{3.732950in}{0.413320in}}%
\pgfpathlineto{\pgfqpoint{3.730158in}{0.413320in}}%
\pgfpathlineto{\pgfqpoint{3.727581in}{0.413320in}}%
\pgfpathlineto{\pgfqpoint{3.724804in}{0.413320in}}%
\pgfpathlineto{\pgfqpoint{3.722228in}{0.413320in}}%
\pgfpathlineto{\pgfqpoint{3.719446in}{0.413320in}}%
\pgfpathlineto{\pgfqpoint{3.716875in}{0.413320in}}%
\pgfpathlineto{\pgfqpoint{3.714086in}{0.413320in}}%
\pgfpathlineto{\pgfqpoint{3.711410in}{0.413320in}}%
\pgfpathlineto{\pgfqpoint{3.708729in}{0.413320in}}%
\pgfpathlineto{\pgfqpoint{3.706053in}{0.413320in}}%
\pgfpathlineto{\pgfqpoint{3.703460in}{0.413320in}}%
\pgfpathlineto{\pgfqpoint{3.700684in}{0.413320in}}%
\pgfpathlineto{\pgfqpoint{3.698125in}{0.413320in}}%
\pgfpathlineto{\pgfqpoint{3.695331in}{0.413320in}}%
\pgfpathlineto{\pgfqpoint{3.692765in}{0.413320in}}%
\pgfpathlineto{\pgfqpoint{3.689983in}{0.413320in}}%
\pgfpathlineto{\pgfqpoint{3.687442in}{0.413320in}}%
\pgfpathlineto{\pgfqpoint{3.684620in}{0.413320in}}%
\pgfpathlineto{\pgfqpoint{3.681948in}{0.413320in}}%
\pgfpathlineto{\pgfqpoint{3.679273in}{0.413320in}}%
\pgfpathlineto{\pgfqpoint{3.676591in}{0.413320in}}%
\pgfpathlineto{\pgfqpoint{3.673911in}{0.413320in}}%
\pgfpathlineto{\pgfqpoint{3.671232in}{0.413320in}}%
\pgfpathlineto{\pgfqpoint{3.668665in}{0.413320in}}%
\pgfpathlineto{\pgfqpoint{3.665864in}{0.413320in}}%
\pgfpathlineto{\pgfqpoint{3.663276in}{0.413320in}}%
\pgfpathlineto{\pgfqpoint{3.660515in}{0.413320in}}%
\pgfpathlineto{\pgfqpoint{3.657917in}{0.413320in}}%
\pgfpathlineto{\pgfqpoint{3.655165in}{0.413320in}}%
\pgfpathlineto{\pgfqpoint{3.652628in}{0.413320in}}%
\pgfpathlineto{\pgfqpoint{3.649837in}{0.413320in}}%
\pgfpathlineto{\pgfqpoint{3.647130in}{0.413320in}}%
\pgfpathlineto{\pgfqpoint{3.644452in}{0.413320in}}%
\pgfpathlineto{\pgfqpoint{3.641773in}{0.413320in}}%
\pgfpathlineto{\pgfqpoint{3.639207in}{0.413320in}}%
\pgfpathlineto{\pgfqpoint{3.636413in}{0.413320in}}%
\pgfpathlineto{\pgfqpoint{3.633858in}{0.413320in}}%
\pgfpathlineto{\pgfqpoint{3.631058in}{0.413320in}}%
\pgfpathlineto{\pgfqpoint{3.628460in}{0.413320in}}%
\pgfpathlineto{\pgfqpoint{3.625689in}{0.413320in}}%
\pgfpathlineto{\pgfqpoint{3.623165in}{0.413320in}}%
\pgfpathlineto{\pgfqpoint{3.620345in}{0.413320in}}%
\pgfpathlineto{\pgfqpoint{3.617667in}{0.413320in}}%
\pgfpathlineto{\pgfqpoint{3.614982in}{0.413320in}}%
\pgfpathlineto{\pgfqpoint{3.612311in}{0.413320in}}%
\pgfpathlineto{\pgfqpoint{3.609632in}{0.413320in}}%
\pgfpathlineto{\pgfqpoint{3.606951in}{0.413320in}}%
\pgfpathlineto{\pgfqpoint{3.604387in}{0.413320in}}%
\pgfpathlineto{\pgfqpoint{3.601590in}{0.413320in}}%
\pgfpathlineto{\pgfqpoint{3.598998in}{0.413320in}}%
\pgfpathlineto{\pgfqpoint{3.596240in}{0.413320in}}%
\pgfpathlineto{\pgfqpoint{3.593620in}{0.413320in}}%
\pgfpathlineto{\pgfqpoint{3.590883in}{0.413320in}}%
\pgfpathlineto{\pgfqpoint{3.588258in}{0.413320in}}%
\pgfpathlineto{\pgfqpoint{3.585532in}{0.413320in}}%
\pgfpathlineto{\pgfqpoint{3.582851in}{0.413320in}}%
\pgfpathlineto{\pgfqpoint{3.580191in}{0.413320in}}%
\pgfpathlineto{\pgfqpoint{3.577487in}{0.413320in}}%
\pgfpathlineto{\pgfqpoint{3.574814in}{0.413320in}}%
\pgfpathlineto{\pgfqpoint{3.572126in}{0.413320in}}%
\pgfpathlineto{\pgfqpoint{3.569584in}{0.413320in}}%
\pgfpathlineto{\pgfqpoint{3.566774in}{0.413320in}}%
\pgfpathlineto{\pgfqpoint{3.564188in}{0.413320in}}%
\pgfpathlineto{\pgfqpoint{3.561420in}{0.413320in}}%
\pgfpathlineto{\pgfqpoint{3.558853in}{0.413320in}}%
\pgfpathlineto{\pgfqpoint{3.556061in}{0.413320in}}%
\pgfpathlineto{\pgfqpoint{3.553498in}{0.413320in}}%
\pgfpathlineto{\pgfqpoint{3.550713in}{0.413320in}}%
\pgfpathlineto{\pgfqpoint{3.548029in}{0.413320in}}%
\pgfpathlineto{\pgfqpoint{3.545349in}{0.413320in}}%
\pgfpathlineto{\pgfqpoint{3.542656in}{0.413320in}}%
\pgfpathlineto{\pgfqpoint{3.540093in}{0.413320in}}%
\pgfpathlineto{\pgfqpoint{3.537309in}{0.413320in}}%
\pgfpathlineto{\pgfqpoint{3.534783in}{0.413320in}}%
\pgfpathlineto{\pgfqpoint{3.531955in}{0.413320in}}%
\pgfpathlineto{\pgfqpoint{3.529327in}{0.413320in}}%
\pgfpathlineto{\pgfqpoint{3.526601in}{0.413320in}}%
\pgfpathlineto{\pgfqpoint{3.524041in}{0.413320in}}%
\pgfpathlineto{\pgfqpoint{3.521244in}{0.413320in}}%
\pgfpathlineto{\pgfqpoint{3.518565in}{0.413320in}}%
\pgfpathlineto{\pgfqpoint{3.515884in}{0.413320in}}%
\pgfpathlineto{\pgfqpoint{3.513209in}{0.413320in}}%
\pgfpathlineto{\pgfqpoint{3.510533in}{0.413320in}}%
\pgfpathlineto{\pgfqpoint{3.507840in}{0.413320in}}%
\pgfpathlineto{\pgfqpoint{3.505262in}{0.413320in}}%
\pgfpathlineto{\pgfqpoint{3.502488in}{0.413320in}}%
\pgfpathlineto{\pgfqpoint{3.499909in}{0.413320in}}%
\pgfpathlineto{\pgfqpoint{3.497139in}{0.413320in}}%
\pgfpathlineto{\pgfqpoint{3.494581in}{0.413320in}}%
\pgfpathlineto{\pgfqpoint{3.491783in}{0.413320in}}%
\pgfpathlineto{\pgfqpoint{3.489223in}{0.413320in}}%
\pgfpathlineto{\pgfqpoint{3.486442in}{0.413320in}}%
\pgfpathlineto{\pgfqpoint{3.483744in}{0.413320in}}%
\pgfpathlineto{\pgfqpoint{3.481072in}{0.413320in}}%
\pgfpathlineto{\pgfqpoint{3.478378in}{0.413320in}}%
\pgfpathlineto{\pgfqpoint{3.475821in}{0.413320in}}%
\pgfpathlineto{\pgfqpoint{3.473021in}{0.413320in}}%
\pgfpathlineto{\pgfqpoint{3.470466in}{0.413320in}}%
\pgfpathlineto{\pgfqpoint{3.467678in}{0.413320in}}%
\pgfpathlineto{\pgfqpoint{3.465072in}{0.413320in}}%
\pgfpathlineto{\pgfqpoint{3.462321in}{0.413320in}}%
\pgfpathlineto{\pgfqpoint{3.459695in}{0.413320in}}%
\pgfpathlineto{\pgfqpoint{3.456960in}{0.413320in}}%
\pgfpathlineto{\pgfqpoint{3.454285in}{0.413320in}}%
\pgfpathlineto{\pgfqpoint{3.451597in}{0.413320in}}%
\pgfpathlineto{\pgfqpoint{3.448926in}{0.413320in}}%
\pgfpathlineto{\pgfqpoint{3.446257in}{0.413320in}}%
\pgfpathlineto{\pgfqpoint{3.443574in}{0.413320in}}%
\pgfpathlineto{\pgfqpoint{3.440996in}{0.413320in}}%
\pgfpathlineto{\pgfqpoint{3.438210in}{0.413320in}}%
\pgfpathlineto{\pgfqpoint{3.435635in}{0.413320in}}%
\pgfpathlineto{\pgfqpoint{3.432851in}{0.413320in}}%
\pgfpathlineto{\pgfqpoint{3.430313in}{0.413320in}}%
\pgfpathlineto{\pgfqpoint{3.427501in}{0.413320in}}%
\pgfpathlineto{\pgfqpoint{3.424887in}{0.413320in}}%
\pgfpathlineto{\pgfqpoint{3.422142in}{0.413320in}}%
\pgfpathlineto{\pgfqpoint{3.419455in}{0.413320in}}%
\pgfpathlineto{\pgfqpoint{3.416780in}{0.413320in}}%
\pgfpathlineto{\pgfqpoint{3.414109in}{0.413320in}}%
\pgfpathlineto{\pgfqpoint{3.411431in}{0.413320in}}%
\pgfpathlineto{\pgfqpoint{3.408752in}{0.413320in}}%
\pgfpathlineto{\pgfqpoint{3.406202in}{0.413320in}}%
\pgfpathlineto{\pgfqpoint{3.403394in}{0.413320in}}%
\pgfpathlineto{\pgfqpoint{3.400783in}{0.413320in}}%
\pgfpathlineto{\pgfqpoint{3.398037in}{0.413320in}}%
\pgfpathlineto{\pgfqpoint{3.395461in}{0.413320in}}%
\pgfpathlineto{\pgfqpoint{3.392681in}{0.413320in}}%
\pgfpathlineto{\pgfqpoint{3.390102in}{0.413320in}}%
\pgfpathlineto{\pgfqpoint{3.387309in}{0.413320in}}%
\pgfpathlineto{\pgfqpoint{3.384647in}{0.413320in}}%
\pgfpathlineto{\pgfqpoint{3.381959in}{0.413320in}}%
\pgfpathlineto{\pgfqpoint{3.379290in}{0.413320in}}%
\pgfpathlineto{\pgfqpoint{3.376735in}{0.413320in}}%
\pgfpathlineto{\pgfqpoint{3.373921in}{0.413320in}}%
\pgfpathlineto{\pgfqpoint{3.371357in}{0.413320in}}%
\pgfpathlineto{\pgfqpoint{3.368577in}{0.413320in}}%
\pgfpathlineto{\pgfqpoint{3.365996in}{0.413320in}}%
\pgfpathlineto{\pgfqpoint{3.363221in}{0.413320in}}%
\pgfpathlineto{\pgfqpoint{3.360620in}{0.413320in}}%
\pgfpathlineto{\pgfqpoint{3.357862in}{0.413320in}}%
\pgfpathlineto{\pgfqpoint{3.355177in}{0.413320in}}%
\pgfpathlineto{\pgfqpoint{3.352505in}{0.413320in}}%
\pgfpathlineto{\pgfqpoint{3.349828in}{0.413320in}}%
\pgfpathlineto{\pgfqpoint{3.347139in}{0.413320in}}%
\pgfpathlineto{\pgfqpoint{3.344468in}{0.413320in}}%
\pgfpathlineto{\pgfqpoint{3.341893in}{0.413320in}}%
\pgfpathlineto{\pgfqpoint{3.339101in}{0.413320in}}%
\pgfpathlineto{\pgfqpoint{3.336541in}{0.413320in}}%
\pgfpathlineto{\pgfqpoint{3.333758in}{0.413320in}}%
\pgfpathlineto{\pgfqpoint{3.331183in}{0.413320in}}%
\pgfpathlineto{\pgfqpoint{3.328401in}{0.413320in}}%
\pgfpathlineto{\pgfqpoint{3.325860in}{0.413320in}}%
\pgfpathlineto{\pgfqpoint{3.323049in}{0.413320in}}%
\pgfpathlineto{\pgfqpoint{3.320366in}{0.413320in}}%
\pgfpathlineto{\pgfqpoint{3.317688in}{0.413320in}}%
\pgfpathlineto{\pgfqpoint{3.315008in}{0.413320in}}%
\pgfpathlineto{\pgfqpoint{3.312480in}{0.413320in}}%
\pgfpathlineto{\pgfqpoint{3.309652in}{0.413320in}}%
\pgfpathlineto{\pgfqpoint{3.307104in}{0.413320in}}%
\pgfpathlineto{\pgfqpoint{3.304295in}{0.413320in}}%
\pgfpathlineto{\pgfqpoint{3.301719in}{0.413320in}}%
\pgfpathlineto{\pgfqpoint{3.298937in}{0.413320in}}%
\pgfpathlineto{\pgfqpoint{3.296376in}{0.413320in}}%
\pgfpathlineto{\pgfqpoint{3.293574in}{0.413320in}}%
\pgfpathlineto{\pgfqpoint{3.290890in}{0.413320in}}%
\pgfpathlineto{\pgfqpoint{3.288225in}{0.413320in}}%
\pgfpathlineto{\pgfqpoint{3.285534in}{0.413320in}}%
\pgfpathlineto{\pgfqpoint{3.282870in}{0.413320in}}%
\pgfpathlineto{\pgfqpoint{3.280189in}{0.413320in}}%
\pgfpathlineto{\pgfqpoint{3.277603in}{0.413320in}}%
\pgfpathlineto{\pgfqpoint{3.274831in}{0.413320in}}%
\pgfpathlineto{\pgfqpoint{3.272254in}{0.413320in}}%
\pgfpathlineto{\pgfqpoint{3.269478in}{0.413320in}}%
\pgfpathlineto{\pgfqpoint{3.266849in}{0.413320in}}%
\pgfpathlineto{\pgfqpoint{3.264119in}{0.413320in}}%
\pgfpathlineto{\pgfqpoint{3.261594in}{0.413320in}}%
\pgfpathlineto{\pgfqpoint{3.258784in}{0.413320in}}%
\pgfpathlineto{\pgfqpoint{3.256083in}{0.413320in}}%
\pgfpathlineto{\pgfqpoint{3.253404in}{0.413320in}}%
\pgfpathlineto{\pgfqpoint{3.250716in}{0.413320in}}%
\pgfpathlineto{\pgfqpoint{3.248049in}{0.413320in}}%
\pgfpathlineto{\pgfqpoint{3.245363in}{0.413320in}}%
\pgfpathlineto{\pgfqpoint{3.242807in}{0.413320in}}%
\pgfpathlineto{\pgfqpoint{3.240010in}{0.413320in}}%
\pgfpathlineto{\pgfqpoint{3.237411in}{0.413320in}}%
\pgfpathlineto{\pgfqpoint{3.234658in}{0.413320in}}%
\pgfpathlineto{\pgfqpoint{3.232069in}{0.413320in}}%
\pgfpathlineto{\pgfqpoint{3.229310in}{0.413320in}}%
\pgfpathlineto{\pgfqpoint{3.226609in}{0.413320in}}%
\pgfpathlineto{\pgfqpoint{3.223942in}{0.413320in}}%
\pgfpathlineto{\pgfqpoint{3.221255in}{0.413320in}}%
\pgfpathlineto{\pgfqpoint{3.218586in}{0.413320in}}%
\pgfpathlineto{\pgfqpoint{3.215908in}{0.413320in}}%
\pgfpathlineto{\pgfqpoint{3.213342in}{0.413320in}}%
\pgfpathlineto{\pgfqpoint{3.210545in}{0.413320in}}%
\pgfpathlineto{\pgfqpoint{3.207984in}{0.413320in}}%
\pgfpathlineto{\pgfqpoint{3.205195in}{0.413320in}}%
\pgfpathlineto{\pgfqpoint{3.202562in}{0.413320in}}%
\pgfpathlineto{\pgfqpoint{3.199823in}{0.413320in}}%
\pgfpathlineto{\pgfqpoint{3.197226in}{0.413320in}}%
\pgfpathlineto{\pgfqpoint{3.194508in}{0.413320in}}%
\pgfpathlineto{\pgfqpoint{3.191796in}{0.413320in}}%
\pgfpathlineto{\pgfqpoint{3.189117in}{0.413320in}}%
\pgfpathlineto{\pgfqpoint{3.186440in}{0.413320in}}%
\pgfpathlineto{\pgfqpoint{3.183760in}{0.413320in}}%
\pgfpathlineto{\pgfqpoint{3.181089in}{0.413320in}}%
\pgfpathlineto{\pgfqpoint{3.178525in}{0.413320in}}%
\pgfpathlineto{\pgfqpoint{3.175724in}{0.413320in}}%
\pgfpathlineto{\pgfqpoint{3.173142in}{0.413320in}}%
\pgfpathlineto{\pgfqpoint{3.170375in}{0.413320in}}%
\pgfpathlineto{\pgfqpoint{3.167776in}{0.413320in}}%
\pgfpathlineto{\pgfqpoint{3.165019in}{0.413320in}}%
\pgfpathlineto{\pgfqpoint{3.162474in}{0.413320in}}%
\pgfpathlineto{\pgfqpoint{3.159675in}{0.413320in}}%
\pgfpathlineto{\pgfqpoint{3.156981in}{0.413320in}}%
\pgfpathlineto{\pgfqpoint{3.154327in}{0.413320in}}%
\pgfpathlineto{\pgfqpoint{3.151612in}{0.413320in}}%
\pgfpathlineto{\pgfqpoint{3.149057in}{0.413320in}}%
\pgfpathlineto{\pgfqpoint{3.146271in}{0.413320in}}%
\pgfpathlineto{\pgfqpoint{3.143740in}{0.413320in}}%
\pgfpathlineto{\pgfqpoint{3.140913in}{0.413320in}}%
\pgfpathlineto{\pgfqpoint{3.138375in}{0.413320in}}%
\pgfpathlineto{\pgfqpoint{3.135550in}{0.413320in}}%
\pgfpathlineto{\pgfqpoint{3.132946in}{0.413320in}}%
\pgfpathlineto{\pgfqpoint{3.130199in}{0.413320in}}%
\pgfpathlineto{\pgfqpoint{3.127512in}{0.413320in}}%
\pgfpathlineto{\pgfqpoint{3.124842in}{0.413320in}}%
\pgfpathlineto{\pgfqpoint{3.122163in}{0.413320in}}%
\pgfpathlineto{\pgfqpoint{3.119487in}{0.413320in}}%
\pgfpathlineto{\pgfqpoint{3.116807in}{0.413320in}}%
\pgfpathlineto{\pgfqpoint{3.114242in}{0.413320in}}%
\pgfpathlineto{\pgfqpoint{3.111451in}{0.413320in}}%
\pgfpathlineto{\pgfqpoint{3.108896in}{0.413320in}}%
\pgfpathlineto{\pgfqpoint{3.106094in}{0.413320in}}%
\pgfpathlineto{\pgfqpoint{3.103508in}{0.413320in}}%
\pgfpathlineto{\pgfqpoint{3.100737in}{0.413320in}}%
\pgfpathlineto{\pgfqpoint{3.098163in}{0.413320in}}%
\pgfpathlineto{\pgfqpoint{3.095388in}{0.413320in}}%
\pgfpathlineto{\pgfqpoint{3.092699in}{0.413320in}}%
\pgfpathlineto{\pgfqpoint{3.090023in}{0.413320in}}%
\pgfpathlineto{\pgfqpoint{3.087343in}{0.413320in}}%
\pgfpathlineto{\pgfqpoint{3.084671in}{0.413320in}}%
\pgfpathlineto{\pgfqpoint{3.081990in}{0.413320in}}%
\pgfpathlineto{\pgfqpoint{3.079381in}{0.413320in}}%
\pgfpathlineto{\pgfqpoint{3.076631in}{0.413320in}}%
\pgfpathlineto{\pgfqpoint{3.074056in}{0.413320in}}%
\pgfpathlineto{\pgfqpoint{3.071266in}{0.413320in}}%
\pgfpathlineto{\pgfqpoint{3.068709in}{0.413320in}}%
\pgfpathlineto{\pgfqpoint{3.065916in}{0.413320in}}%
\pgfpathlineto{\pgfqpoint{3.063230in}{0.413320in}}%
\pgfpathlineto{\pgfqpoint{3.060561in}{0.413320in}}%
\pgfpathlineto{\pgfqpoint{3.057884in}{0.413320in}}%
\pgfpathlineto{\pgfqpoint{3.055202in}{0.413320in}}%
\pgfpathlineto{\pgfqpoint{3.052526in}{0.413320in}}%
\pgfpathlineto{\pgfqpoint{3.049988in}{0.413320in}}%
\pgfpathlineto{\pgfqpoint{3.047157in}{0.413320in}}%
\pgfpathlineto{\pgfqpoint{3.044568in}{0.413320in}}%
\pgfpathlineto{\pgfqpoint{3.041813in}{0.413320in}}%
\pgfpathlineto{\pgfqpoint{3.039262in}{0.413320in}}%
\pgfpathlineto{\pgfqpoint{3.036456in}{0.413320in}}%
\pgfpathlineto{\pgfqpoint{3.033921in}{0.413320in}}%
\pgfpathlineto{\pgfqpoint{3.031091in}{0.413320in}}%
\pgfpathlineto{\pgfqpoint{3.028412in}{0.413320in}}%
\pgfpathlineto{\pgfqpoint{3.025803in}{0.413320in}}%
\pgfpathlineto{\pgfqpoint{3.023058in}{0.413320in}}%
\pgfpathlineto{\pgfqpoint{3.020382in}{0.413320in}}%
\pgfpathlineto{\pgfqpoint{3.017707in}{0.413320in}}%
\pgfpathlineto{\pgfqpoint{3.015097in}{0.413320in}}%
\pgfpathlineto{\pgfqpoint{3.012351in}{0.413320in}}%
\pgfpathlineto{\pgfqpoint{3.009784in}{0.413320in}}%
\pgfpathlineto{\pgfqpoint{3.006993in}{0.413320in}}%
\pgfpathlineto{\pgfqpoint{3.004419in}{0.413320in}}%
\pgfpathlineto{\pgfqpoint{3.001635in}{0.413320in}}%
\pgfpathlineto{\pgfqpoint{2.999103in}{0.413320in}}%
\pgfpathlineto{\pgfqpoint{2.996300in}{0.413320in}}%
\pgfpathlineto{\pgfqpoint{2.993595in}{0.413320in}}%
\pgfpathlineto{\pgfqpoint{2.990978in}{0.413320in}}%
\pgfpathlineto{\pgfqpoint{2.988238in}{0.413320in}}%
\pgfpathlineto{\pgfqpoint{2.985666in}{0.413320in}}%
\pgfpathlineto{\pgfqpoint{2.982885in}{0.413320in}}%
\pgfpathlineto{\pgfqpoint{2.980341in}{0.413320in}}%
\pgfpathlineto{\pgfqpoint{2.977517in}{0.413320in}}%
\pgfpathlineto{\pgfqpoint{2.974972in}{0.413320in}}%
\pgfpathlineto{\pgfqpoint{2.972177in}{0.413320in}}%
\pgfpathlineto{\pgfqpoint{2.969599in}{0.413320in}}%
\pgfpathlineto{\pgfqpoint{2.966812in}{0.413320in}}%
\pgfpathlineto{\pgfqpoint{2.964127in}{0.413320in}}%
\pgfpathlineto{\pgfqpoint{2.961460in}{0.413320in}}%
\pgfpathlineto{\pgfqpoint{2.958782in}{0.413320in}}%
\pgfpathlineto{\pgfqpoint{2.956103in}{0.413320in}}%
\pgfpathlineto{\pgfqpoint{2.953422in}{0.413320in}}%
\pgfpathlineto{\pgfqpoint{2.950884in}{0.413320in}}%
\pgfpathlineto{\pgfqpoint{2.948068in}{0.413320in}}%
\pgfpathlineto{\pgfqpoint{2.945461in}{0.413320in}}%
\pgfpathlineto{\pgfqpoint{2.942711in}{0.413320in}}%
\pgfpathlineto{\pgfqpoint{2.940120in}{0.413320in}}%
\pgfpathlineto{\pgfqpoint{2.937352in}{0.413320in}}%
\pgfpathlineto{\pgfqpoint{2.934759in}{0.413320in}}%
\pgfpathlineto{\pgfqpoint{2.932033in}{0.413320in}}%
\pgfpathlineto{\pgfqpoint{2.929321in}{0.413320in}}%
\pgfpathlineto{\pgfqpoint{2.926655in}{0.413320in}}%
\pgfpathlineto{\pgfqpoint{2.923963in}{0.413320in}}%
\pgfpathlineto{\pgfqpoint{2.921363in}{0.413320in}}%
\pgfpathlineto{\pgfqpoint{2.918606in}{0.413320in}}%
\pgfpathlineto{\pgfqpoint{2.916061in}{0.413320in}}%
\pgfpathlineto{\pgfqpoint{2.913243in}{0.413320in}}%
\pgfpathlineto{\pgfqpoint{2.910631in}{0.413320in}}%
\pgfpathlineto{\pgfqpoint{2.907882in}{0.413320in}}%
\pgfpathlineto{\pgfqpoint{2.905341in}{0.413320in}}%
\pgfpathlineto{\pgfqpoint{2.902535in}{0.413320in}}%
\pgfpathlineto{\pgfqpoint{2.899858in}{0.413320in}}%
\pgfpathlineto{\pgfqpoint{2.897179in}{0.413320in}}%
\pgfpathlineto{\pgfqpoint{2.894487in}{0.413320in}}%
\pgfpathlineto{\pgfqpoint{2.891809in}{0.413320in}}%
\pgfpathlineto{\pgfqpoint{2.889145in}{0.413320in}}%
\pgfpathlineto{\pgfqpoint{2.886578in}{0.413320in}}%
\pgfpathlineto{\pgfqpoint{2.883780in}{0.413320in}}%
\pgfpathlineto{\pgfqpoint{2.881254in}{0.413320in}}%
\pgfpathlineto{\pgfqpoint{2.878431in}{0.413320in}}%
\pgfpathlineto{\pgfqpoint{2.875882in}{0.413320in}}%
\pgfpathlineto{\pgfqpoint{2.873074in}{0.413320in}}%
\pgfpathlineto{\pgfqpoint{2.870475in}{0.413320in}}%
\pgfpathlineto{\pgfqpoint{2.867713in}{0.413320in}}%
\pgfpathlineto{\pgfqpoint{2.865031in}{0.413320in}}%
\pgfpathlineto{\pgfqpoint{2.862402in}{0.413320in}}%
\pgfpathlineto{\pgfqpoint{2.859668in}{0.413320in}}%
\pgfpathlineto{\pgfqpoint{2.857003in}{0.413320in}}%
\pgfpathlineto{\pgfqpoint{2.854325in}{0.413320in}}%
\pgfpathlineto{\pgfqpoint{2.851793in}{0.413320in}}%
\pgfpathlineto{\pgfqpoint{2.848960in}{0.413320in}}%
\pgfpathlineto{\pgfqpoint{2.846408in}{0.413320in}}%
\pgfpathlineto{\pgfqpoint{2.843611in}{0.413320in}}%
\pgfpathlineto{\pgfqpoint{2.841055in}{0.413320in}}%
\pgfpathlineto{\pgfqpoint{2.838254in}{0.413320in}}%
\pgfpathlineto{\pgfqpoint{2.835698in}{0.413320in}}%
\pgfpathlineto{\pgfqpoint{2.832894in}{0.413320in}}%
\pgfpathlineto{\pgfqpoint{2.830219in}{0.413320in}}%
\pgfpathlineto{\pgfqpoint{2.827567in}{0.413320in}}%
\pgfpathlineto{\pgfqpoint{2.824851in}{0.413320in}}%
\pgfpathlineto{\pgfqpoint{2.822303in}{0.413320in}}%
\pgfpathlineto{\pgfqpoint{2.819506in}{0.413320in}}%
\pgfpathlineto{\pgfqpoint{2.816867in}{0.413320in}}%
\pgfpathlineto{\pgfqpoint{2.814141in}{0.413320in}}%
\pgfpathlineto{\pgfqpoint{2.811597in}{0.413320in}}%
\pgfpathlineto{\pgfqpoint{2.808792in}{0.413320in}}%
\pgfpathlineto{\pgfqpoint{2.806175in}{0.413320in}}%
\pgfpathlineto{\pgfqpoint{2.803435in}{0.413320in}}%
\pgfpathlineto{\pgfqpoint{2.800756in}{0.413320in}}%
\pgfpathlineto{\pgfqpoint{2.798070in}{0.413320in}}%
\pgfpathlineto{\pgfqpoint{2.795398in}{0.413320in}}%
\pgfpathlineto{\pgfqpoint{2.792721in}{0.413320in}}%
\pgfpathlineto{\pgfqpoint{2.790044in}{0.413320in}}%
\pgfpathlineto{\pgfqpoint{2.787468in}{0.413320in}}%
\pgfpathlineto{\pgfqpoint{2.784687in}{0.413320in}}%
\pgfpathlineto{\pgfqpoint{2.782113in}{0.413320in}}%
\pgfpathlineto{\pgfqpoint{2.779330in}{0.413320in}}%
\pgfpathlineto{\pgfqpoint{2.776767in}{0.413320in}}%
\pgfpathlineto{\pgfqpoint{2.773972in}{0.413320in}}%
\pgfpathlineto{\pgfqpoint{2.771373in}{0.413320in}}%
\pgfpathlineto{\pgfqpoint{2.768617in}{0.413320in}}%
\pgfpathlineto{\pgfqpoint{2.765935in}{0.413320in}}%
\pgfpathlineto{\pgfqpoint{2.763253in}{0.413320in}}%
\pgfpathlineto{\pgfqpoint{2.760581in}{0.413320in}}%
\pgfpathlineto{\pgfqpoint{2.758028in}{0.413320in}}%
\pgfpathlineto{\pgfqpoint{2.755224in}{0.413320in}}%
\pgfpathlineto{\pgfqpoint{2.752614in}{0.413320in}}%
\pgfpathlineto{\pgfqpoint{2.749868in}{0.413320in}}%
\pgfpathlineto{\pgfqpoint{2.747260in}{0.413320in}}%
\pgfpathlineto{\pgfqpoint{2.744510in}{0.413320in}}%
\pgfpathlineto{\pgfqpoint{2.741928in}{0.413320in}}%
\pgfpathlineto{\pgfqpoint{2.739155in}{0.413320in}}%
\pgfpathlineto{\pgfqpoint{2.736476in}{0.413320in}}%
\pgfpathlineto{\pgfqpoint{2.733798in}{0.413320in}}%
\pgfpathlineto{\pgfqpoint{2.731119in}{0.413320in}}%
\pgfpathlineto{\pgfqpoint{2.728439in}{0.413320in}}%
\pgfpathlineto{\pgfqpoint{2.725760in}{0.413320in}}%
\pgfpathlineto{\pgfqpoint{2.723211in}{0.413320in}}%
\pgfpathlineto{\pgfqpoint{2.720404in}{0.413320in}}%
\pgfpathlineto{\pgfqpoint{2.717773in}{0.413320in}}%
\pgfpathlineto{\pgfqpoint{2.715036in}{0.413320in}}%
\pgfpathlineto{\pgfqpoint{2.712477in}{0.413320in}}%
\pgfpathlineto{\pgfqpoint{2.709683in}{0.413320in}}%
\pgfpathlineto{\pgfqpoint{2.707125in}{0.413320in}}%
\pgfpathlineto{\pgfqpoint{2.704326in}{0.413320in}}%
\pgfpathlineto{\pgfqpoint{2.701657in}{0.413320in}}%
\pgfpathlineto{\pgfqpoint{2.698968in}{0.413320in}}%
\pgfpathlineto{\pgfqpoint{2.696293in}{0.413320in}}%
\pgfpathlineto{\pgfqpoint{2.693611in}{0.413320in}}%
\pgfpathlineto{\pgfqpoint{2.690940in}{0.413320in}}%
\pgfpathlineto{\pgfqpoint{2.688328in}{0.413320in}}%
\pgfpathlineto{\pgfqpoint{2.685586in}{0.413320in}}%
\pgfpathlineto{\pgfqpoint{2.683009in}{0.413320in}}%
\pgfpathlineto{\pgfqpoint{2.680224in}{0.413320in}}%
\pgfpathlineto{\pgfqpoint{2.677650in}{0.413320in}}%
\pgfpathlineto{\pgfqpoint{2.674873in}{0.413320in}}%
\pgfpathlineto{\pgfqpoint{2.672301in}{0.413320in}}%
\pgfpathlineto{\pgfqpoint{2.669506in}{0.413320in}}%
\pgfpathlineto{\pgfqpoint{2.666836in}{0.413320in}}%
\pgfpathlineto{\pgfqpoint{2.664151in}{0.413320in}}%
\pgfpathlineto{\pgfqpoint{2.661481in}{0.413320in}}%
\pgfpathlineto{\pgfqpoint{2.658942in}{0.413320in}}%
\pgfpathlineto{\pgfqpoint{2.656124in}{0.413320in}}%
\pgfpathlineto{\pgfqpoint{2.653567in}{0.413320in}}%
\pgfpathlineto{\pgfqpoint{2.650767in}{0.413320in}}%
\pgfpathlineto{\pgfqpoint{2.648196in}{0.413320in}}%
\pgfpathlineto{\pgfqpoint{2.645408in}{0.413320in}}%
\pgfpathlineto{\pgfqpoint{2.642827in}{0.413320in}}%
\pgfpathlineto{\pgfqpoint{2.640053in}{0.413320in}}%
\pgfpathlineto{\pgfqpoint{2.637369in}{0.413320in}}%
\pgfpathlineto{\pgfqpoint{2.634700in}{0.413320in}}%
\pgfpathlineto{\pgfqpoint{2.632018in}{0.413320in}}%
\pgfpathlineto{\pgfqpoint{2.629340in}{0.413320in}}%
\pgfpathlineto{\pgfqpoint{2.626653in}{0.413320in}}%
\pgfpathlineto{\pgfqpoint{2.624077in}{0.413320in}}%
\pgfpathlineto{\pgfqpoint{2.621304in}{0.413320in}}%
\pgfpathlineto{\pgfqpoint{2.618773in}{0.413320in}}%
\pgfpathlineto{\pgfqpoint{2.615934in}{0.413320in}}%
\pgfpathlineto{\pgfqpoint{2.613393in}{0.413320in}}%
\pgfpathlineto{\pgfqpoint{2.610588in}{0.413320in}}%
\pgfpathlineto{\pgfqpoint{2.608004in}{0.413320in}}%
\pgfpathlineto{\pgfqpoint{2.605232in}{0.413320in}}%
\pgfpathlineto{\pgfqpoint{2.602557in}{0.413320in}}%
\pgfpathlineto{\pgfqpoint{2.599920in}{0.413320in}}%
\pgfpathlineto{\pgfqpoint{2.597196in}{0.413320in}}%
\pgfpathlineto{\pgfqpoint{2.594630in}{0.413320in}}%
\pgfpathlineto{\pgfqpoint{2.591842in}{0.413320in}}%
\pgfpathlineto{\pgfqpoint{2.589248in}{0.413320in}}%
\pgfpathlineto{\pgfqpoint{2.586484in}{0.413320in}}%
\pgfpathlineto{\pgfqpoint{2.583913in}{0.413320in}}%
\pgfpathlineto{\pgfqpoint{2.581129in}{0.413320in}}%
\pgfpathlineto{\pgfqpoint{2.578567in}{0.413320in}}%
\pgfpathlineto{\pgfqpoint{2.575779in}{0.413320in}}%
\pgfpathlineto{\pgfqpoint{2.573082in}{0.413320in}}%
\pgfpathlineto{\pgfqpoint{2.570411in}{0.413320in}}%
\pgfpathlineto{\pgfqpoint{2.567730in}{0.413320in}}%
\pgfpathlineto{\pgfqpoint{2.565045in}{0.413320in}}%
\pgfpathlineto{\pgfqpoint{2.562375in}{0.413320in}}%
\pgfpathlineto{\pgfqpoint{2.559790in}{0.413320in}}%
\pgfpathlineto{\pgfqpoint{2.557009in}{0.413320in}}%
\pgfpathlineto{\pgfqpoint{2.554493in}{0.413320in}}%
\pgfpathlineto{\pgfqpoint{2.551664in}{0.413320in}}%
\pgfpathlineto{\pgfqpoint{2.549114in}{0.413320in}}%
\pgfpathlineto{\pgfqpoint{2.546310in}{0.413320in}}%
\pgfpathlineto{\pgfqpoint{2.543765in}{0.413320in}}%
\pgfpathlineto{\pgfqpoint{2.540949in}{0.413320in}}%
\pgfpathlineto{\pgfqpoint{2.538274in}{0.413320in}}%
\pgfpathlineto{\pgfqpoint{2.535624in}{0.413320in}}%
\pgfpathlineto{\pgfqpoint{2.532917in}{0.413320in}}%
\pgfpathlineto{\pgfqpoint{2.530234in}{0.413320in}}%
\pgfpathlineto{\pgfqpoint{2.527560in}{0.413320in}}%
\pgfpathlineto{\pgfqpoint{2.524988in}{0.413320in}}%
\pgfpathlineto{\pgfqpoint{2.522197in}{0.413320in}}%
\pgfpathlineto{\pgfqpoint{2.519607in}{0.413320in}}%
\pgfpathlineto{\pgfqpoint{2.516845in}{0.413320in}}%
\pgfpathlineto{\pgfqpoint{2.514268in}{0.413320in}}%
\pgfpathlineto{\pgfqpoint{2.511478in}{0.413320in}}%
\pgfpathlineto{\pgfqpoint{2.508917in}{0.413320in}}%
\pgfpathlineto{\pgfqpoint{2.506163in}{0.413320in}}%
\pgfpathlineto{\pgfqpoint{2.503454in}{0.413320in}}%
\pgfpathlineto{\pgfqpoint{2.500801in}{0.413320in}}%
\pgfpathlineto{\pgfqpoint{2.498085in}{0.413320in}}%
\pgfpathlineto{\pgfqpoint{2.495542in}{0.413320in}}%
\pgfpathlineto{\pgfqpoint{2.492729in}{0.413320in}}%
\pgfpathlineto{\pgfqpoint{2.490183in}{0.413320in}}%
\pgfpathlineto{\pgfqpoint{2.487384in}{0.413320in}}%
\pgfpathlineto{\pgfqpoint{2.484870in}{0.413320in}}%
\pgfpathlineto{\pgfqpoint{2.482026in}{0.413320in}}%
\pgfpathlineto{\pgfqpoint{2.479420in}{0.413320in}}%
\pgfpathlineto{\pgfqpoint{2.476671in}{0.413320in}}%
\pgfpathlineto{\pgfqpoint{2.473989in}{0.413320in}}%
\pgfpathlineto{\pgfqpoint{2.471311in}{0.413320in}}%
\pgfpathlineto{\pgfqpoint{2.468635in}{0.413320in}}%
\pgfpathlineto{\pgfqpoint{2.465957in}{0.413320in}}%
\pgfpathlineto{\pgfqpoint{2.463280in}{0.413320in}}%
\pgfpathlineto{\pgfqpoint{2.460711in}{0.413320in}}%
\pgfpathlineto{\pgfqpoint{2.457917in}{0.413320in}}%
\pgfpathlineto{\pgfqpoint{2.455353in}{0.413320in}}%
\pgfpathlineto{\pgfqpoint{2.452562in}{0.413320in}}%
\pgfpathlineto{\pgfqpoint{2.450032in}{0.413320in}}%
\pgfpathlineto{\pgfqpoint{2.447209in}{0.413320in}}%
\pgfpathlineto{\pgfqpoint{2.444677in}{0.413320in}}%
\pgfpathlineto{\pgfqpoint{2.441876in}{0.413320in}}%
\pgfpathlineto{\pgfqpoint{2.439167in}{0.413320in}}%
\pgfpathlineto{\pgfqpoint{2.436518in}{0.413320in}}%
\pgfpathlineto{\pgfqpoint{2.433815in}{0.413320in}}%
\pgfpathlineto{\pgfqpoint{2.431251in}{0.413320in}}%
\pgfpathlineto{\pgfqpoint{2.428453in}{0.413320in}}%
\pgfpathlineto{\pgfqpoint{2.425878in}{0.413320in}}%
\pgfpathlineto{\pgfqpoint{2.423098in}{0.413320in}}%
\pgfpathlineto{\pgfqpoint{2.420528in}{0.413320in}}%
\pgfpathlineto{\pgfqpoint{2.417747in}{0.413320in}}%
\pgfpathlineto{\pgfqpoint{2.415184in}{0.413320in}}%
\pgfpathlineto{\pgfqpoint{2.412389in}{0.413320in}}%
\pgfpathlineto{\pgfqpoint{2.409699in}{0.413320in}}%
\pgfpathlineto{\pgfqpoint{2.407024in}{0.413320in}}%
\pgfpathlineto{\pgfqpoint{2.404352in}{0.413320in}}%
\pgfpathlineto{\pgfqpoint{2.401675in}{0.413320in}}%
\pgfpathlineto{\pgfqpoint{2.398995in}{0.413320in}}%
\pgfpathclose%
\pgfusepath{stroke,fill}%
\end{pgfscope}%
\begin{pgfscope}%
\pgfpathrectangle{\pgfqpoint{2.398995in}{0.319877in}}{\pgfqpoint{3.986877in}{1.993438in}} %
\pgfusepath{clip}%
\pgfsetbuttcap%
\pgfsetroundjoin%
\definecolor{currentfill}{rgb}{1.000000,1.000000,1.000000}%
\pgfsetfillcolor{currentfill}%
\pgfsetlinewidth{1.003750pt}%
\definecolor{currentstroke}{rgb}{0.213369,0.672810,0.686696}%
\pgfsetstrokecolor{currentstroke}%
\pgfsetdash{}{0pt}%
\pgfpathmoveto{\pgfqpoint{2.398995in}{0.413320in}}%
\pgfpathlineto{\pgfqpoint{2.398995in}{1.048524in}}%
\pgfpathlineto{\pgfqpoint{2.401675in}{1.045903in}}%
\pgfpathlineto{\pgfqpoint{2.404352in}{1.047319in}}%
\pgfpathlineto{\pgfqpoint{2.407024in}{1.039002in}}%
\pgfpathlineto{\pgfqpoint{2.409699in}{1.035174in}}%
\pgfpathlineto{\pgfqpoint{2.412389in}{1.039852in}}%
\pgfpathlineto{\pgfqpoint{2.415184in}{1.041543in}}%
\pgfpathlineto{\pgfqpoint{2.417747in}{1.038407in}}%
\pgfpathlineto{\pgfqpoint{2.420528in}{1.040195in}}%
\pgfpathlineto{\pgfqpoint{2.423098in}{1.037158in}}%
\pgfpathlineto{\pgfqpoint{2.425878in}{1.040942in}}%
\pgfpathlineto{\pgfqpoint{2.428453in}{1.044231in}}%
\pgfpathlineto{\pgfqpoint{2.431251in}{1.042115in}}%
\pgfpathlineto{\pgfqpoint{2.433815in}{1.050969in}}%
\pgfpathlineto{\pgfqpoint{2.436518in}{1.059808in}}%
\pgfpathlineto{\pgfqpoint{2.439167in}{1.055422in}}%
\pgfpathlineto{\pgfqpoint{2.441876in}{1.048064in}}%
\pgfpathlineto{\pgfqpoint{2.444677in}{1.042241in}}%
\pgfpathlineto{\pgfqpoint{2.447209in}{1.041409in}}%
\pgfpathlineto{\pgfqpoint{2.450032in}{1.037614in}}%
\pgfpathlineto{\pgfqpoint{2.452562in}{1.042503in}}%
\pgfpathlineto{\pgfqpoint{2.455353in}{1.035410in}}%
\pgfpathlineto{\pgfqpoint{2.457917in}{1.043856in}}%
\pgfpathlineto{\pgfqpoint{2.460711in}{1.044020in}}%
\pgfpathlineto{\pgfqpoint{2.463280in}{1.041142in}}%
\pgfpathlineto{\pgfqpoint{2.465957in}{1.042433in}}%
\pgfpathlineto{\pgfqpoint{2.468635in}{1.040816in}}%
\pgfpathlineto{\pgfqpoint{2.471311in}{1.042205in}}%
\pgfpathlineto{\pgfqpoint{2.473989in}{1.037054in}}%
\pgfpathlineto{\pgfqpoint{2.476671in}{1.042443in}}%
\pgfpathlineto{\pgfqpoint{2.479420in}{1.039701in}}%
\pgfpathlineto{\pgfqpoint{2.482026in}{1.038698in}}%
\pgfpathlineto{\pgfqpoint{2.484870in}{1.030080in}}%
\pgfpathlineto{\pgfqpoint{2.487384in}{1.027986in}}%
\pgfpathlineto{\pgfqpoint{2.490183in}{1.033771in}}%
\pgfpathlineto{\pgfqpoint{2.492729in}{1.030361in}}%
\pgfpathlineto{\pgfqpoint{2.495542in}{1.035309in}}%
\pgfpathlineto{\pgfqpoint{2.498085in}{1.039482in}}%
\pgfpathlineto{\pgfqpoint{2.500801in}{1.035650in}}%
\pgfpathlineto{\pgfqpoint{2.503454in}{1.032128in}}%
\pgfpathlineto{\pgfqpoint{2.506163in}{1.037973in}}%
\pgfpathlineto{\pgfqpoint{2.508917in}{1.036312in}}%
\pgfpathlineto{\pgfqpoint{2.511478in}{1.038656in}}%
\pgfpathlineto{\pgfqpoint{2.514268in}{1.037636in}}%
\pgfpathlineto{\pgfqpoint{2.516845in}{1.040850in}}%
\pgfpathlineto{\pgfqpoint{2.519607in}{1.055894in}}%
\pgfpathlineto{\pgfqpoint{2.522197in}{1.056423in}}%
\pgfpathlineto{\pgfqpoint{2.524988in}{1.048423in}}%
\pgfpathlineto{\pgfqpoint{2.527560in}{1.043745in}}%
\pgfpathlineto{\pgfqpoint{2.530234in}{1.040284in}}%
\pgfpathlineto{\pgfqpoint{2.532917in}{1.038317in}}%
\pgfpathlineto{\pgfqpoint{2.535624in}{1.034832in}}%
\pgfpathlineto{\pgfqpoint{2.538274in}{1.032604in}}%
\pgfpathlineto{\pgfqpoint{2.540949in}{1.033139in}}%
\pgfpathlineto{\pgfqpoint{2.543765in}{1.039832in}}%
\pgfpathlineto{\pgfqpoint{2.546310in}{1.036775in}}%
\pgfpathlineto{\pgfqpoint{2.549114in}{1.042507in}}%
\pgfpathlineto{\pgfqpoint{2.551664in}{1.037731in}}%
\pgfpathlineto{\pgfqpoint{2.554493in}{1.035582in}}%
\pgfpathlineto{\pgfqpoint{2.557009in}{1.038087in}}%
\pgfpathlineto{\pgfqpoint{2.559790in}{1.037952in}}%
\pgfpathlineto{\pgfqpoint{2.562375in}{1.038831in}}%
\pgfpathlineto{\pgfqpoint{2.565045in}{1.034681in}}%
\pgfpathlineto{\pgfqpoint{2.567730in}{1.035103in}}%
\pgfpathlineto{\pgfqpoint{2.570411in}{1.034470in}}%
\pgfpathlineto{\pgfqpoint{2.573082in}{1.056628in}}%
\pgfpathlineto{\pgfqpoint{2.575779in}{1.074339in}}%
\pgfpathlineto{\pgfqpoint{2.578567in}{1.067394in}}%
\pgfpathlineto{\pgfqpoint{2.581129in}{1.057971in}}%
\pgfpathlineto{\pgfqpoint{2.583913in}{1.060139in}}%
\pgfpathlineto{\pgfqpoint{2.586484in}{1.051472in}}%
\pgfpathlineto{\pgfqpoint{2.589248in}{1.044389in}}%
\pgfpathlineto{\pgfqpoint{2.591842in}{1.043722in}}%
\pgfpathlineto{\pgfqpoint{2.594630in}{1.046891in}}%
\pgfpathlineto{\pgfqpoint{2.597196in}{1.043085in}}%
\pgfpathlineto{\pgfqpoint{2.599920in}{1.037196in}}%
\pgfpathlineto{\pgfqpoint{2.602557in}{1.040905in}}%
\pgfpathlineto{\pgfqpoint{2.605232in}{1.039459in}}%
\pgfpathlineto{\pgfqpoint{2.608004in}{1.042013in}}%
\pgfpathlineto{\pgfqpoint{2.610588in}{1.047575in}}%
\pgfpathlineto{\pgfqpoint{2.613393in}{1.050665in}}%
\pgfpathlineto{\pgfqpoint{2.615934in}{1.042023in}}%
\pgfpathlineto{\pgfqpoint{2.618773in}{1.046716in}}%
\pgfpathlineto{\pgfqpoint{2.621304in}{1.040201in}}%
\pgfpathlineto{\pgfqpoint{2.624077in}{1.038852in}}%
\pgfpathlineto{\pgfqpoint{2.626653in}{1.044491in}}%
\pgfpathlineto{\pgfqpoint{2.629340in}{1.042009in}}%
\pgfpathlineto{\pgfqpoint{2.632018in}{1.044924in}}%
\pgfpathlineto{\pgfqpoint{2.634700in}{1.045577in}}%
\pgfpathlineto{\pgfqpoint{2.637369in}{1.044166in}}%
\pgfpathlineto{\pgfqpoint{2.640053in}{1.042800in}}%
\pgfpathlineto{\pgfqpoint{2.642827in}{1.043178in}}%
\pgfpathlineto{\pgfqpoint{2.645408in}{1.044771in}}%
\pgfpathlineto{\pgfqpoint{2.648196in}{1.054596in}}%
\pgfpathlineto{\pgfqpoint{2.650767in}{1.052563in}}%
\pgfpathlineto{\pgfqpoint{2.653567in}{1.052652in}}%
\pgfpathlineto{\pgfqpoint{2.656124in}{1.051146in}}%
\pgfpathlineto{\pgfqpoint{2.658942in}{1.051237in}}%
\pgfpathlineto{\pgfqpoint{2.661481in}{1.050118in}}%
\pgfpathlineto{\pgfqpoint{2.664151in}{1.045664in}}%
\pgfpathlineto{\pgfqpoint{2.666836in}{1.052792in}}%
\pgfpathlineto{\pgfqpoint{2.669506in}{1.050278in}}%
\pgfpathlineto{\pgfqpoint{2.672301in}{1.045126in}}%
\pgfpathlineto{\pgfqpoint{2.674873in}{1.040737in}}%
\pgfpathlineto{\pgfqpoint{2.677650in}{1.041935in}}%
\pgfpathlineto{\pgfqpoint{2.680224in}{1.049139in}}%
\pgfpathlineto{\pgfqpoint{2.683009in}{1.047095in}}%
\pgfpathlineto{\pgfqpoint{2.685586in}{1.041199in}}%
\pgfpathlineto{\pgfqpoint{2.688328in}{1.037377in}}%
\pgfpathlineto{\pgfqpoint{2.690940in}{1.043409in}}%
\pgfpathlineto{\pgfqpoint{2.693611in}{1.038150in}}%
\pgfpathlineto{\pgfqpoint{2.696293in}{1.042657in}}%
\pgfpathlineto{\pgfqpoint{2.698968in}{1.035713in}}%
\pgfpathlineto{\pgfqpoint{2.701657in}{1.037170in}}%
\pgfpathlineto{\pgfqpoint{2.704326in}{1.041560in}}%
\pgfpathlineto{\pgfqpoint{2.707125in}{1.038608in}}%
\pgfpathlineto{\pgfqpoint{2.709683in}{1.045077in}}%
\pgfpathlineto{\pgfqpoint{2.712477in}{1.041247in}}%
\pgfpathlineto{\pgfqpoint{2.715036in}{1.049340in}}%
\pgfpathlineto{\pgfqpoint{2.717773in}{1.050876in}}%
\pgfpathlineto{\pgfqpoint{2.720404in}{1.046853in}}%
\pgfpathlineto{\pgfqpoint{2.723211in}{1.041367in}}%
\pgfpathlineto{\pgfqpoint{2.725760in}{1.036579in}}%
\pgfpathlineto{\pgfqpoint{2.728439in}{1.040698in}}%
\pgfpathlineto{\pgfqpoint{2.731119in}{1.032882in}}%
\pgfpathlineto{\pgfqpoint{2.733798in}{1.030200in}}%
\pgfpathlineto{\pgfqpoint{2.736476in}{1.028505in}}%
\pgfpathlineto{\pgfqpoint{2.739155in}{1.027986in}}%
\pgfpathlineto{\pgfqpoint{2.741928in}{1.032177in}}%
\pgfpathlineto{\pgfqpoint{2.744510in}{1.040652in}}%
\pgfpathlineto{\pgfqpoint{2.747260in}{1.040581in}}%
\pgfpathlineto{\pgfqpoint{2.749868in}{1.038855in}}%
\pgfpathlineto{\pgfqpoint{2.752614in}{1.038116in}}%
\pgfpathlineto{\pgfqpoint{2.755224in}{1.038228in}}%
\pgfpathlineto{\pgfqpoint{2.758028in}{1.041964in}}%
\pgfpathlineto{\pgfqpoint{2.760581in}{1.041162in}}%
\pgfpathlineto{\pgfqpoint{2.763253in}{1.043681in}}%
\pgfpathlineto{\pgfqpoint{2.765935in}{1.042229in}}%
\pgfpathlineto{\pgfqpoint{2.768617in}{1.038762in}}%
\pgfpathlineto{\pgfqpoint{2.771373in}{1.038182in}}%
\pgfpathlineto{\pgfqpoint{2.773972in}{1.041468in}}%
\pgfpathlineto{\pgfqpoint{2.776767in}{1.030511in}}%
\pgfpathlineto{\pgfqpoint{2.779330in}{1.034916in}}%
\pgfpathlineto{\pgfqpoint{2.782113in}{1.035802in}}%
\pgfpathlineto{\pgfqpoint{2.784687in}{1.035350in}}%
\pgfpathlineto{\pgfqpoint{2.787468in}{1.039461in}}%
\pgfpathlineto{\pgfqpoint{2.790044in}{1.036162in}}%
\pgfpathlineto{\pgfqpoint{2.792721in}{1.040232in}}%
\pgfpathlineto{\pgfqpoint{2.795398in}{1.034457in}}%
\pgfpathlineto{\pgfqpoint{2.798070in}{1.033720in}}%
\pgfpathlineto{\pgfqpoint{2.800756in}{1.035808in}}%
\pgfpathlineto{\pgfqpoint{2.803435in}{1.030919in}}%
\pgfpathlineto{\pgfqpoint{2.806175in}{1.036661in}}%
\pgfpathlineto{\pgfqpoint{2.808792in}{1.037874in}}%
\pgfpathlineto{\pgfqpoint{2.811597in}{1.039333in}}%
\pgfpathlineto{\pgfqpoint{2.814141in}{1.039284in}}%
\pgfpathlineto{\pgfqpoint{2.816867in}{1.044572in}}%
\pgfpathlineto{\pgfqpoint{2.819506in}{1.046394in}}%
\pgfpathlineto{\pgfqpoint{2.822303in}{1.042498in}}%
\pgfpathlineto{\pgfqpoint{2.824851in}{1.045038in}}%
\pgfpathlineto{\pgfqpoint{2.827567in}{1.043467in}}%
\pgfpathlineto{\pgfqpoint{2.830219in}{1.046103in}}%
\pgfpathlineto{\pgfqpoint{2.832894in}{1.040603in}}%
\pgfpathlineto{\pgfqpoint{2.835698in}{1.037159in}}%
\pgfpathlineto{\pgfqpoint{2.838254in}{1.036564in}}%
\pgfpathlineto{\pgfqpoint{2.841055in}{1.032344in}}%
\pgfpathlineto{\pgfqpoint{2.843611in}{1.030672in}}%
\pgfpathlineto{\pgfqpoint{2.846408in}{1.039958in}}%
\pgfpathlineto{\pgfqpoint{2.848960in}{1.042257in}}%
\pgfpathlineto{\pgfqpoint{2.851793in}{1.038316in}}%
\pgfpathlineto{\pgfqpoint{2.854325in}{1.038716in}}%
\pgfpathlineto{\pgfqpoint{2.857003in}{1.035941in}}%
\pgfpathlineto{\pgfqpoint{2.859668in}{1.039333in}}%
\pgfpathlineto{\pgfqpoint{2.862402in}{1.040609in}}%
\pgfpathlineto{\pgfqpoint{2.865031in}{1.037730in}}%
\pgfpathlineto{\pgfqpoint{2.867713in}{1.044065in}}%
\pgfpathlineto{\pgfqpoint{2.870475in}{1.041642in}}%
\pgfpathlineto{\pgfqpoint{2.873074in}{1.033301in}}%
\pgfpathlineto{\pgfqpoint{2.875882in}{1.037333in}}%
\pgfpathlineto{\pgfqpoint{2.878431in}{1.040389in}}%
\pgfpathlineto{\pgfqpoint{2.881254in}{1.041726in}}%
\pgfpathlineto{\pgfqpoint{2.883780in}{1.041358in}}%
\pgfpathlineto{\pgfqpoint{2.886578in}{1.039553in}}%
\pgfpathlineto{\pgfqpoint{2.889145in}{1.036711in}}%
\pgfpathlineto{\pgfqpoint{2.891809in}{1.040158in}}%
\pgfpathlineto{\pgfqpoint{2.894487in}{1.040875in}}%
\pgfpathlineto{\pgfqpoint{2.897179in}{1.044793in}}%
\pgfpathlineto{\pgfqpoint{2.899858in}{1.044150in}}%
\pgfpathlineto{\pgfqpoint{2.902535in}{1.041237in}}%
\pgfpathlineto{\pgfqpoint{2.905341in}{1.036979in}}%
\pgfpathlineto{\pgfqpoint{2.907882in}{1.041477in}}%
\pgfpathlineto{\pgfqpoint{2.910631in}{1.038483in}}%
\pgfpathlineto{\pgfqpoint{2.913243in}{1.038878in}}%
\pgfpathlineto{\pgfqpoint{2.916061in}{1.044067in}}%
\pgfpathlineto{\pgfqpoint{2.918606in}{1.042590in}}%
\pgfpathlineto{\pgfqpoint{2.921363in}{1.042478in}}%
\pgfpathlineto{\pgfqpoint{2.923963in}{1.048932in}}%
\pgfpathlineto{\pgfqpoint{2.926655in}{1.043772in}}%
\pgfpathlineto{\pgfqpoint{2.929321in}{1.044836in}}%
\pgfpathlineto{\pgfqpoint{2.932033in}{1.045818in}}%
\pgfpathlineto{\pgfqpoint{2.934759in}{1.041978in}}%
\pgfpathlineto{\pgfqpoint{2.937352in}{1.046746in}}%
\pgfpathlineto{\pgfqpoint{2.940120in}{1.045527in}}%
\pgfpathlineto{\pgfqpoint{2.942711in}{1.044141in}}%
\pgfpathlineto{\pgfqpoint{2.945461in}{1.040282in}}%
\pgfpathlineto{\pgfqpoint{2.948068in}{1.043061in}}%
\pgfpathlineto{\pgfqpoint{2.950884in}{1.041767in}}%
\pgfpathlineto{\pgfqpoint{2.953422in}{1.042310in}}%
\pgfpathlineto{\pgfqpoint{2.956103in}{1.045649in}}%
\pgfpathlineto{\pgfqpoint{2.958782in}{1.041836in}}%
\pgfpathlineto{\pgfqpoint{2.961460in}{1.047248in}}%
\pgfpathlineto{\pgfqpoint{2.964127in}{1.041845in}}%
\pgfpathlineto{\pgfqpoint{2.966812in}{1.046583in}}%
\pgfpathlineto{\pgfqpoint{2.969599in}{1.045838in}}%
\pgfpathlineto{\pgfqpoint{2.972177in}{1.042630in}}%
\pgfpathlineto{\pgfqpoint{2.974972in}{1.041822in}}%
\pgfpathlineto{\pgfqpoint{2.977517in}{1.041391in}}%
\pgfpathlineto{\pgfqpoint{2.980341in}{1.044393in}}%
\pgfpathlineto{\pgfqpoint{2.982885in}{1.042226in}}%
\pgfpathlineto{\pgfqpoint{2.985666in}{1.044818in}}%
\pgfpathlineto{\pgfqpoint{2.988238in}{1.044522in}}%
\pgfpathlineto{\pgfqpoint{2.990978in}{1.050838in}}%
\pgfpathlineto{\pgfqpoint{2.993595in}{1.053983in}}%
\pgfpathlineto{\pgfqpoint{2.996300in}{1.044500in}}%
\pgfpathlineto{\pgfqpoint{2.999103in}{1.040073in}}%
\pgfpathlineto{\pgfqpoint{3.001635in}{1.035616in}}%
\pgfpathlineto{\pgfqpoint{3.004419in}{1.034659in}}%
\pgfpathlineto{\pgfqpoint{3.006993in}{1.044059in}}%
\pgfpathlineto{\pgfqpoint{3.009784in}{1.044786in}}%
\pgfpathlineto{\pgfqpoint{3.012351in}{1.041015in}}%
\pgfpathlineto{\pgfqpoint{3.015097in}{1.043339in}}%
\pgfpathlineto{\pgfqpoint{3.017707in}{1.045976in}}%
\pgfpathlineto{\pgfqpoint{3.020382in}{1.047251in}}%
\pgfpathlineto{\pgfqpoint{3.023058in}{1.049190in}}%
\pgfpathlineto{\pgfqpoint{3.025803in}{1.044543in}}%
\pgfpathlineto{\pgfqpoint{3.028412in}{1.046298in}}%
\pgfpathlineto{\pgfqpoint{3.031091in}{1.041855in}}%
\pgfpathlineto{\pgfqpoint{3.033921in}{1.053318in}}%
\pgfpathlineto{\pgfqpoint{3.036456in}{1.052194in}}%
\pgfpathlineto{\pgfqpoint{3.039262in}{1.047988in}}%
\pgfpathlineto{\pgfqpoint{3.041813in}{1.053101in}}%
\pgfpathlineto{\pgfqpoint{3.044568in}{1.043337in}}%
\pgfpathlineto{\pgfqpoint{3.047157in}{1.043400in}}%
\pgfpathlineto{\pgfqpoint{3.049988in}{1.040012in}}%
\pgfpathlineto{\pgfqpoint{3.052526in}{1.043682in}}%
\pgfpathlineto{\pgfqpoint{3.055202in}{1.041888in}}%
\pgfpathlineto{\pgfqpoint{3.057884in}{1.040755in}}%
\pgfpathlineto{\pgfqpoint{3.060561in}{1.045172in}}%
\pgfpathlineto{\pgfqpoint{3.063230in}{1.039990in}}%
\pgfpathlineto{\pgfqpoint{3.065916in}{1.041495in}}%
\pgfpathlineto{\pgfqpoint{3.068709in}{1.039334in}}%
\pgfpathlineto{\pgfqpoint{3.071266in}{1.044880in}}%
\pgfpathlineto{\pgfqpoint{3.074056in}{1.048294in}}%
\pgfpathlineto{\pgfqpoint{3.076631in}{1.046726in}}%
\pgfpathlineto{\pgfqpoint{3.079381in}{1.045162in}}%
\pgfpathlineto{\pgfqpoint{3.081990in}{1.045381in}}%
\pgfpathlineto{\pgfqpoint{3.084671in}{1.047458in}}%
\pgfpathlineto{\pgfqpoint{3.087343in}{1.047602in}}%
\pgfpathlineto{\pgfqpoint{3.090023in}{1.042184in}}%
\pgfpathlineto{\pgfqpoint{3.092699in}{1.036204in}}%
\pgfpathlineto{\pgfqpoint{3.095388in}{1.036946in}}%
\pgfpathlineto{\pgfqpoint{3.098163in}{1.037367in}}%
\pgfpathlineto{\pgfqpoint{3.100737in}{1.027986in}}%
\pgfpathlineto{\pgfqpoint{3.103508in}{1.027986in}}%
\pgfpathlineto{\pgfqpoint{3.106094in}{1.032291in}}%
\pgfpathlineto{\pgfqpoint{3.108896in}{1.035332in}}%
\pgfpathlineto{\pgfqpoint{3.111451in}{1.042289in}}%
\pgfpathlineto{\pgfqpoint{3.114242in}{1.042401in}}%
\pgfpathlineto{\pgfqpoint{3.116807in}{1.038924in}}%
\pgfpathlineto{\pgfqpoint{3.119487in}{1.036565in}}%
\pgfpathlineto{\pgfqpoint{3.122163in}{1.032011in}}%
\pgfpathlineto{\pgfqpoint{3.124842in}{1.039192in}}%
\pgfpathlineto{\pgfqpoint{3.127512in}{1.040865in}}%
\pgfpathlineto{\pgfqpoint{3.130199in}{1.035816in}}%
\pgfpathlineto{\pgfqpoint{3.132946in}{1.037593in}}%
\pgfpathlineto{\pgfqpoint{3.135550in}{1.045644in}}%
\pgfpathlineto{\pgfqpoint{3.138375in}{1.039787in}}%
\pgfpathlineto{\pgfqpoint{3.140913in}{1.027986in}}%
\pgfpathlineto{\pgfqpoint{3.143740in}{1.027986in}}%
\pgfpathlineto{\pgfqpoint{3.146271in}{1.027986in}}%
\pgfpathlineto{\pgfqpoint{3.149057in}{1.027986in}}%
\pgfpathlineto{\pgfqpoint{3.151612in}{1.027986in}}%
\pgfpathlineto{\pgfqpoint{3.154327in}{1.027986in}}%
\pgfpathlineto{\pgfqpoint{3.156981in}{1.031500in}}%
\pgfpathlineto{\pgfqpoint{3.159675in}{1.027986in}}%
\pgfpathlineto{\pgfqpoint{3.162474in}{1.027986in}}%
\pgfpathlineto{\pgfqpoint{3.165019in}{1.027986in}}%
\pgfpathlineto{\pgfqpoint{3.167776in}{1.027986in}}%
\pgfpathlineto{\pgfqpoint{3.170375in}{1.027986in}}%
\pgfpathlineto{\pgfqpoint{3.173142in}{1.027986in}}%
\pgfpathlineto{\pgfqpoint{3.175724in}{1.027986in}}%
\pgfpathlineto{\pgfqpoint{3.178525in}{1.027986in}}%
\pgfpathlineto{\pgfqpoint{3.181089in}{1.027986in}}%
\pgfpathlineto{\pgfqpoint{3.183760in}{1.027986in}}%
\pgfpathlineto{\pgfqpoint{3.186440in}{1.027986in}}%
\pgfpathlineto{\pgfqpoint{3.189117in}{1.027986in}}%
\pgfpathlineto{\pgfqpoint{3.191796in}{1.027986in}}%
\pgfpathlineto{\pgfqpoint{3.194508in}{1.031603in}}%
\pgfpathlineto{\pgfqpoint{3.197226in}{1.027986in}}%
\pgfpathlineto{\pgfqpoint{3.199823in}{1.027986in}}%
\pgfpathlineto{\pgfqpoint{3.202562in}{1.027986in}}%
\pgfpathlineto{\pgfqpoint{3.205195in}{1.028310in}}%
\pgfpathlineto{\pgfqpoint{3.207984in}{1.029332in}}%
\pgfpathlineto{\pgfqpoint{3.210545in}{1.034830in}}%
\pgfpathlineto{\pgfqpoint{3.213342in}{1.036579in}}%
\pgfpathlineto{\pgfqpoint{3.215908in}{1.038144in}}%
\pgfpathlineto{\pgfqpoint{3.218586in}{1.040580in}}%
\pgfpathlineto{\pgfqpoint{3.221255in}{1.041356in}}%
\pgfpathlineto{\pgfqpoint{3.223942in}{1.042588in}}%
\pgfpathlineto{\pgfqpoint{3.226609in}{1.035436in}}%
\pgfpathlineto{\pgfqpoint{3.229310in}{1.032492in}}%
\pgfpathlineto{\pgfqpoint{3.232069in}{1.037553in}}%
\pgfpathlineto{\pgfqpoint{3.234658in}{1.037717in}}%
\pgfpathlineto{\pgfqpoint{3.237411in}{1.039462in}}%
\pgfpathlineto{\pgfqpoint{3.240010in}{1.038820in}}%
\pgfpathlineto{\pgfqpoint{3.242807in}{1.035170in}}%
\pgfpathlineto{\pgfqpoint{3.245363in}{1.039396in}}%
\pgfpathlineto{\pgfqpoint{3.248049in}{1.038154in}}%
\pgfpathlineto{\pgfqpoint{3.250716in}{1.041577in}}%
\pgfpathlineto{\pgfqpoint{3.253404in}{1.036277in}}%
\pgfpathlineto{\pgfqpoint{3.256083in}{1.044627in}}%
\pgfpathlineto{\pgfqpoint{3.258784in}{1.041006in}}%
\pgfpathlineto{\pgfqpoint{3.261594in}{1.045852in}}%
\pgfpathlineto{\pgfqpoint{3.264119in}{1.046740in}}%
\pgfpathlineto{\pgfqpoint{3.266849in}{1.047535in}}%
\pgfpathlineto{\pgfqpoint{3.269478in}{1.051955in}}%
\pgfpathlineto{\pgfqpoint{3.272254in}{1.052734in}}%
\pgfpathlineto{\pgfqpoint{3.274831in}{1.053971in}}%
\pgfpathlineto{\pgfqpoint{3.277603in}{1.050968in}}%
\pgfpathlineto{\pgfqpoint{3.280189in}{1.048682in}}%
\pgfpathlineto{\pgfqpoint{3.282870in}{1.053032in}}%
\pgfpathlineto{\pgfqpoint{3.285534in}{1.050994in}}%
\pgfpathlineto{\pgfqpoint{3.288225in}{1.047952in}}%
\pgfpathlineto{\pgfqpoint{3.290890in}{1.046041in}}%
\pgfpathlineto{\pgfqpoint{3.293574in}{1.050113in}}%
\pgfpathlineto{\pgfqpoint{3.296376in}{1.047228in}}%
\pgfpathlineto{\pgfqpoint{3.298937in}{1.050384in}}%
\pgfpathlineto{\pgfqpoint{3.301719in}{1.045957in}}%
\pgfpathlineto{\pgfqpoint{3.304295in}{1.049837in}}%
\pgfpathlineto{\pgfqpoint{3.307104in}{1.049531in}}%
\pgfpathlineto{\pgfqpoint{3.309652in}{1.049211in}}%
\pgfpathlineto{\pgfqpoint{3.312480in}{1.049168in}}%
\pgfpathlineto{\pgfqpoint{3.315008in}{1.050035in}}%
\pgfpathlineto{\pgfqpoint{3.317688in}{1.053367in}}%
\pgfpathlineto{\pgfqpoint{3.320366in}{1.048113in}}%
\pgfpathlineto{\pgfqpoint{3.323049in}{1.051922in}}%
\pgfpathlineto{\pgfqpoint{3.325860in}{1.046730in}}%
\pgfpathlineto{\pgfqpoint{3.328401in}{1.044130in}}%
\pgfpathlineto{\pgfqpoint{3.331183in}{1.045915in}}%
\pgfpathlineto{\pgfqpoint{3.333758in}{1.051506in}}%
\pgfpathlineto{\pgfqpoint{3.336541in}{1.050049in}}%
\pgfpathlineto{\pgfqpoint{3.339101in}{1.053491in}}%
\pgfpathlineto{\pgfqpoint{3.341893in}{1.051046in}}%
\pgfpathlineto{\pgfqpoint{3.344468in}{1.042776in}}%
\pgfpathlineto{\pgfqpoint{3.347139in}{1.048879in}}%
\pgfpathlineto{\pgfqpoint{3.349828in}{1.057352in}}%
\pgfpathlineto{\pgfqpoint{3.352505in}{1.046916in}}%
\pgfpathlineto{\pgfqpoint{3.355177in}{1.051650in}}%
\pgfpathlineto{\pgfqpoint{3.357862in}{1.044539in}}%
\pgfpathlineto{\pgfqpoint{3.360620in}{1.049146in}}%
\pgfpathlineto{\pgfqpoint{3.363221in}{1.046835in}}%
\pgfpathlineto{\pgfqpoint{3.365996in}{1.049538in}}%
\pgfpathlineto{\pgfqpoint{3.368577in}{1.046572in}}%
\pgfpathlineto{\pgfqpoint{3.371357in}{1.044346in}}%
\pgfpathlineto{\pgfqpoint{3.373921in}{1.045834in}}%
\pgfpathlineto{\pgfqpoint{3.376735in}{1.049208in}}%
\pgfpathlineto{\pgfqpoint{3.379290in}{1.048802in}}%
\pgfpathlineto{\pgfqpoint{3.381959in}{1.050340in}}%
\pgfpathlineto{\pgfqpoint{3.384647in}{1.045454in}}%
\pgfpathlineto{\pgfqpoint{3.387309in}{1.046523in}}%
\pgfpathlineto{\pgfqpoint{3.390102in}{1.047274in}}%
\pgfpathlineto{\pgfqpoint{3.392681in}{1.047202in}}%
\pgfpathlineto{\pgfqpoint{3.395461in}{1.039668in}}%
\pgfpathlineto{\pgfqpoint{3.398037in}{1.043214in}}%
\pgfpathlineto{\pgfqpoint{3.400783in}{1.046606in}}%
\pgfpathlineto{\pgfqpoint{3.403394in}{1.042942in}}%
\pgfpathlineto{\pgfqpoint{3.406202in}{1.043381in}}%
\pgfpathlineto{\pgfqpoint{3.408752in}{1.046337in}}%
\pgfpathlineto{\pgfqpoint{3.411431in}{1.047235in}}%
\pgfpathlineto{\pgfqpoint{3.414109in}{1.042029in}}%
\pgfpathlineto{\pgfqpoint{3.416780in}{1.045152in}}%
\pgfpathlineto{\pgfqpoint{3.419455in}{1.045760in}}%
\pgfpathlineto{\pgfqpoint{3.422142in}{1.045140in}}%
\pgfpathlineto{\pgfqpoint{3.424887in}{1.045792in}}%
\pgfpathlineto{\pgfqpoint{3.427501in}{1.046368in}}%
\pgfpathlineto{\pgfqpoint{3.430313in}{1.048780in}}%
\pgfpathlineto{\pgfqpoint{3.432851in}{1.052623in}}%
\pgfpathlineto{\pgfqpoint{3.435635in}{1.048617in}}%
\pgfpathlineto{\pgfqpoint{3.438210in}{1.044982in}}%
\pgfpathlineto{\pgfqpoint{3.440996in}{1.057454in}}%
\pgfpathlineto{\pgfqpoint{3.443574in}{1.085728in}}%
\pgfpathlineto{\pgfqpoint{3.446257in}{1.092382in}}%
\pgfpathlineto{\pgfqpoint{3.448926in}{1.073696in}}%
\pgfpathlineto{\pgfqpoint{3.451597in}{1.057852in}}%
\pgfpathlineto{\pgfqpoint{3.454285in}{1.055642in}}%
\pgfpathlineto{\pgfqpoint{3.456960in}{1.060761in}}%
\pgfpathlineto{\pgfqpoint{3.459695in}{1.052986in}}%
\pgfpathlineto{\pgfqpoint{3.462321in}{1.058262in}}%
\pgfpathlineto{\pgfqpoint{3.465072in}{1.060010in}}%
\pgfpathlineto{\pgfqpoint{3.467678in}{1.055382in}}%
\pgfpathlineto{\pgfqpoint{3.470466in}{1.053112in}}%
\pgfpathlineto{\pgfqpoint{3.473021in}{1.050898in}}%
\pgfpathlineto{\pgfqpoint{3.475821in}{1.053682in}}%
\pgfpathlineto{\pgfqpoint{3.478378in}{1.049165in}}%
\pgfpathlineto{\pgfqpoint{3.481072in}{1.050829in}}%
\pgfpathlineto{\pgfqpoint{3.483744in}{1.052284in}}%
\pgfpathlineto{\pgfqpoint{3.486442in}{1.049076in}}%
\pgfpathlineto{\pgfqpoint{3.489223in}{1.050609in}}%
\pgfpathlineto{\pgfqpoint{3.491783in}{1.050230in}}%
\pgfpathlineto{\pgfqpoint{3.494581in}{1.047956in}}%
\pgfpathlineto{\pgfqpoint{3.497139in}{1.049457in}}%
\pgfpathlineto{\pgfqpoint{3.499909in}{1.048382in}}%
\pgfpathlineto{\pgfqpoint{3.502488in}{1.048836in}}%
\pgfpathlineto{\pgfqpoint{3.505262in}{1.050780in}}%
\pgfpathlineto{\pgfqpoint{3.507840in}{1.052571in}}%
\pgfpathlineto{\pgfqpoint{3.510533in}{1.075711in}}%
\pgfpathlineto{\pgfqpoint{3.513209in}{1.069034in}}%
\pgfpathlineto{\pgfqpoint{3.515884in}{1.062717in}}%
\pgfpathlineto{\pgfqpoint{3.518565in}{1.050652in}}%
\pgfpathlineto{\pgfqpoint{3.521244in}{1.051659in}}%
\pgfpathlineto{\pgfqpoint{3.524041in}{1.050809in}}%
\pgfpathlineto{\pgfqpoint{3.526601in}{1.054141in}}%
\pgfpathlineto{\pgfqpoint{3.529327in}{1.051950in}}%
\pgfpathlineto{\pgfqpoint{3.531955in}{1.044389in}}%
\pgfpathlineto{\pgfqpoint{3.534783in}{1.046694in}}%
\pgfpathlineto{\pgfqpoint{3.537309in}{1.046922in}}%
\pgfpathlineto{\pgfqpoint{3.540093in}{1.047468in}}%
\pgfpathlineto{\pgfqpoint{3.542656in}{1.041858in}}%
\pgfpathlineto{\pgfqpoint{3.545349in}{1.050444in}}%
\pgfpathlineto{\pgfqpoint{3.548029in}{1.053389in}}%
\pgfpathlineto{\pgfqpoint{3.550713in}{1.053483in}}%
\pgfpathlineto{\pgfqpoint{3.553498in}{1.052742in}}%
\pgfpathlineto{\pgfqpoint{3.556061in}{1.046941in}}%
\pgfpathlineto{\pgfqpoint{3.558853in}{1.045623in}}%
\pgfpathlineto{\pgfqpoint{3.561420in}{1.045612in}}%
\pgfpathlineto{\pgfqpoint{3.564188in}{1.047223in}}%
\pgfpathlineto{\pgfqpoint{3.566774in}{1.040761in}}%
\pgfpathlineto{\pgfqpoint{3.569584in}{1.043052in}}%
\pgfpathlineto{\pgfqpoint{3.572126in}{1.049568in}}%
\pgfpathlineto{\pgfqpoint{3.574814in}{1.040980in}}%
\pgfpathlineto{\pgfqpoint{3.577487in}{1.045627in}}%
\pgfpathlineto{\pgfqpoint{3.580191in}{1.050267in}}%
\pgfpathlineto{\pgfqpoint{3.582851in}{1.049555in}}%
\pgfpathlineto{\pgfqpoint{3.585532in}{1.054013in}}%
\pgfpathlineto{\pgfqpoint{3.588258in}{1.048511in}}%
\pgfpathlineto{\pgfqpoint{3.590883in}{1.047960in}}%
\pgfpathlineto{\pgfqpoint{3.593620in}{1.057580in}}%
\pgfpathlineto{\pgfqpoint{3.596240in}{1.049372in}}%
\pgfpathlineto{\pgfqpoint{3.598998in}{1.048510in}}%
\pgfpathlineto{\pgfqpoint{3.601590in}{1.042388in}}%
\pgfpathlineto{\pgfqpoint{3.604387in}{1.041841in}}%
\pgfpathlineto{\pgfqpoint{3.606951in}{1.050437in}}%
\pgfpathlineto{\pgfqpoint{3.609632in}{1.046307in}}%
\pgfpathlineto{\pgfqpoint{3.612311in}{1.033771in}}%
\pgfpathlineto{\pgfqpoint{3.614982in}{1.039610in}}%
\pgfpathlineto{\pgfqpoint{3.617667in}{1.043775in}}%
\pgfpathlineto{\pgfqpoint{3.620345in}{1.047691in}}%
\pgfpathlineto{\pgfqpoint{3.623165in}{1.044644in}}%
\pgfpathlineto{\pgfqpoint{3.625689in}{1.044803in}}%
\pgfpathlineto{\pgfqpoint{3.628460in}{1.046771in}}%
\pgfpathlineto{\pgfqpoint{3.631058in}{1.046009in}}%
\pgfpathlineto{\pgfqpoint{3.633858in}{1.047447in}}%
\pgfpathlineto{\pgfqpoint{3.636413in}{1.045931in}}%
\pgfpathlineto{\pgfqpoint{3.639207in}{1.051669in}}%
\pgfpathlineto{\pgfqpoint{3.641773in}{1.053085in}}%
\pgfpathlineto{\pgfqpoint{3.644452in}{1.042538in}}%
\pgfpathlineto{\pgfqpoint{3.647130in}{1.038016in}}%
\pgfpathlineto{\pgfqpoint{3.649837in}{1.051407in}}%
\pgfpathlineto{\pgfqpoint{3.652628in}{1.045799in}}%
\pgfpathlineto{\pgfqpoint{3.655165in}{1.053128in}}%
\pgfpathlineto{\pgfqpoint{3.657917in}{1.052142in}}%
\pgfpathlineto{\pgfqpoint{3.660515in}{1.049132in}}%
\pgfpathlineto{\pgfqpoint{3.663276in}{1.046570in}}%
\pgfpathlineto{\pgfqpoint{3.665864in}{1.052266in}}%
\pgfpathlineto{\pgfqpoint{3.668665in}{1.048618in}}%
\pgfpathlineto{\pgfqpoint{3.671232in}{1.054467in}}%
\pgfpathlineto{\pgfqpoint{3.673911in}{1.051447in}}%
\pgfpathlineto{\pgfqpoint{3.676591in}{1.052372in}}%
\pgfpathlineto{\pgfqpoint{3.679273in}{1.052375in}}%
\pgfpathlineto{\pgfqpoint{3.681948in}{1.074067in}}%
\pgfpathlineto{\pgfqpoint{3.684620in}{1.062957in}}%
\pgfpathlineto{\pgfqpoint{3.687442in}{1.057408in}}%
\pgfpathlineto{\pgfqpoint{3.689983in}{1.054115in}}%
\pgfpathlineto{\pgfqpoint{3.692765in}{1.054909in}}%
\pgfpathlineto{\pgfqpoint{3.695331in}{1.052083in}}%
\pgfpathlineto{\pgfqpoint{3.698125in}{1.052249in}}%
\pgfpathlineto{\pgfqpoint{3.700684in}{1.049371in}}%
\pgfpathlineto{\pgfqpoint{3.703460in}{1.050580in}}%
\pgfpathlineto{\pgfqpoint{3.706053in}{1.053245in}}%
\pgfpathlineto{\pgfqpoint{3.708729in}{1.049865in}}%
\pgfpathlineto{\pgfqpoint{3.711410in}{1.048680in}}%
\pgfpathlineto{\pgfqpoint{3.714086in}{1.051132in}}%
\pgfpathlineto{\pgfqpoint{3.716875in}{1.050436in}}%
\pgfpathlineto{\pgfqpoint{3.719446in}{1.049939in}}%
\pgfpathlineto{\pgfqpoint{3.722228in}{1.038522in}}%
\pgfpathlineto{\pgfqpoint{3.724804in}{1.045629in}}%
\pgfpathlineto{\pgfqpoint{3.727581in}{1.048190in}}%
\pgfpathlineto{\pgfqpoint{3.730158in}{1.045043in}}%
\pgfpathlineto{\pgfqpoint{3.732950in}{1.046571in}}%
\pgfpathlineto{\pgfqpoint{3.735509in}{1.039776in}}%
\pgfpathlineto{\pgfqpoint{3.738194in}{1.047448in}}%
\pgfpathlineto{\pgfqpoint{3.740874in}{1.047676in}}%
\pgfpathlineto{\pgfqpoint{3.743548in}{1.048524in}}%
\pgfpathlineto{\pgfqpoint{3.746229in}{1.048997in}}%
\pgfpathlineto{\pgfqpoint{3.748903in}{1.046703in}}%
\pgfpathlineto{\pgfqpoint{3.751728in}{1.049052in}}%
\pgfpathlineto{\pgfqpoint{3.754265in}{1.049637in}}%
\pgfpathlineto{\pgfqpoint{3.757065in}{1.049155in}}%
\pgfpathlineto{\pgfqpoint{3.759622in}{1.047166in}}%
\pgfpathlineto{\pgfqpoint{3.762389in}{1.045579in}}%
\pgfpathlineto{\pgfqpoint{3.764966in}{1.056811in}}%
\pgfpathlineto{\pgfqpoint{3.767782in}{1.049099in}}%
\pgfpathlineto{\pgfqpoint{3.770323in}{1.041060in}}%
\pgfpathlineto{\pgfqpoint{3.773014in}{1.033133in}}%
\pgfpathlineto{\pgfqpoint{3.775691in}{1.027986in}}%
\pgfpathlineto{\pgfqpoint{3.778370in}{1.027986in}}%
\pgfpathlineto{\pgfqpoint{3.781046in}{1.027986in}}%
\pgfpathlineto{\pgfqpoint{3.783725in}{1.027986in}}%
\pgfpathlineto{\pgfqpoint{3.786504in}{1.027986in}}%
\pgfpathlineto{\pgfqpoint{3.789084in}{1.027986in}}%
\pgfpathlineto{\pgfqpoint{3.791897in}{1.027986in}}%
\pgfpathlineto{\pgfqpoint{3.794435in}{1.027986in}}%
\pgfpathlineto{\pgfqpoint{3.797265in}{1.032884in}}%
\pgfpathlineto{\pgfqpoint{3.799797in}{1.030100in}}%
\pgfpathlineto{\pgfqpoint{3.802569in}{1.034357in}}%
\pgfpathlineto{\pgfqpoint{3.805145in}{1.035350in}}%
\pgfpathlineto{\pgfqpoint{3.807832in}{1.033909in}}%
\pgfpathlineto{\pgfqpoint{3.810510in}{1.039207in}}%
\pgfpathlineto{\pgfqpoint{3.813172in}{1.035043in}}%
\pgfpathlineto{\pgfqpoint{3.815983in}{1.038491in}}%
\pgfpathlineto{\pgfqpoint{3.818546in}{1.040467in}}%
\pgfpathlineto{\pgfqpoint{3.821315in}{1.036002in}}%
\pgfpathlineto{\pgfqpoint{3.823903in}{1.041307in}}%
\pgfpathlineto{\pgfqpoint{3.826679in}{1.038789in}}%
\pgfpathlineto{\pgfqpoint{3.829252in}{1.046070in}}%
\pgfpathlineto{\pgfqpoint{3.832053in}{1.045425in}}%
\pgfpathlineto{\pgfqpoint{3.834616in}{1.056646in}}%
\pgfpathlineto{\pgfqpoint{3.837286in}{1.054451in}}%
\pgfpathlineto{\pgfqpoint{3.839960in}{1.049180in}}%
\pgfpathlineto{\pgfqpoint{3.842641in}{1.050240in}}%
\pgfpathlineto{\pgfqpoint{3.845329in}{1.052447in}}%
\pgfpathlineto{\pgfqpoint{3.848005in}{1.052835in}}%
\pgfpathlineto{\pgfqpoint{3.850814in}{1.049105in}}%
\pgfpathlineto{\pgfqpoint{3.853358in}{1.049116in}}%
\pgfpathlineto{\pgfqpoint{3.856100in}{1.043022in}}%
\pgfpathlineto{\pgfqpoint{3.858720in}{1.047335in}}%
\pgfpathlineto{\pgfqpoint{3.861561in}{1.047512in}}%
\pgfpathlineto{\pgfqpoint{3.864073in}{1.046080in}}%
\pgfpathlineto{\pgfqpoint{3.866815in}{1.049353in}}%
\pgfpathlineto{\pgfqpoint{3.869435in}{1.051805in}}%
\pgfpathlineto{\pgfqpoint{3.872114in}{1.048260in}}%
\pgfpathlineto{\pgfqpoint{3.874790in}{1.049662in}}%
\pgfpathlineto{\pgfqpoint{3.877466in}{1.051354in}}%
\pgfpathlineto{\pgfqpoint{3.880237in}{1.047964in}}%
\pgfpathlineto{\pgfqpoint{3.882850in}{1.046328in}}%
\pgfpathlineto{\pgfqpoint{3.885621in}{1.051307in}}%
\pgfpathlineto{\pgfqpoint{3.888188in}{1.054418in}}%
\pgfpathlineto{\pgfqpoint{3.890926in}{1.054782in}}%
\pgfpathlineto{\pgfqpoint{3.893541in}{1.055481in}}%
\pgfpathlineto{\pgfqpoint{3.896345in}{1.054325in}}%
\pgfpathlineto{\pgfqpoint{3.898891in}{1.049050in}}%
\pgfpathlineto{\pgfqpoint{3.901573in}{1.047799in}}%
\pgfpathlineto{\pgfqpoint{3.904252in}{1.041759in}}%
\pgfpathlineto{\pgfqpoint{3.906918in}{1.043133in}}%
\pgfpathlineto{\pgfqpoint{3.909602in}{1.049320in}}%
\pgfpathlineto{\pgfqpoint{3.912296in}{1.046758in}}%
\pgfpathlineto{\pgfqpoint{3.915107in}{1.049646in}}%
\pgfpathlineto{\pgfqpoint{3.917646in}{1.050495in}}%
\pgfpathlineto{\pgfqpoint{3.920412in}{1.048009in}}%
\pgfpathlineto{\pgfqpoint{3.923005in}{1.050993in}}%
\pgfpathlineto{\pgfqpoint{3.925778in}{1.047409in}}%
\pgfpathlineto{\pgfqpoint{3.928347in}{1.052296in}}%
\pgfpathlineto{\pgfqpoint{3.931202in}{1.052706in}}%
\pgfpathlineto{\pgfqpoint{3.933714in}{1.050020in}}%
\pgfpathlineto{\pgfqpoint{3.936395in}{1.053056in}}%
\pgfpathlineto{\pgfqpoint{3.939075in}{1.047741in}}%
\pgfpathlineto{\pgfqpoint{3.941778in}{1.037949in}}%
\pgfpathlineto{\pgfqpoint{3.944431in}{1.040864in}}%
\pgfpathlineto{\pgfqpoint{3.947101in}{1.048961in}}%
\pgfpathlineto{\pgfqpoint{3.949894in}{1.039456in}}%
\pgfpathlineto{\pgfqpoint{3.952464in}{1.031322in}}%
\pgfpathlineto{\pgfqpoint{3.955211in}{1.038778in}}%
\pgfpathlineto{\pgfqpoint{3.957823in}{1.042094in}}%
\pgfpathlineto{\pgfqpoint{3.960635in}{1.050432in}}%
\pgfpathlineto{\pgfqpoint{3.963176in}{1.047153in}}%
\pgfpathlineto{\pgfqpoint{3.966013in}{1.051169in}}%
\pgfpathlineto{\pgfqpoint{3.968523in}{1.045497in}}%
\pgfpathlineto{\pgfqpoint{3.971250in}{1.049720in}}%
\pgfpathlineto{\pgfqpoint{3.973885in}{1.043382in}}%
\pgfpathlineto{\pgfqpoint{3.976563in}{1.043780in}}%
\pgfpathlineto{\pgfqpoint{3.979389in}{1.045944in}}%
\pgfpathlineto{\pgfqpoint{3.981929in}{1.044511in}}%
\pgfpathlineto{\pgfqpoint{3.984714in}{1.047003in}}%
\pgfpathlineto{\pgfqpoint{3.987270in}{1.046079in}}%
\pgfpathlineto{\pgfqpoint{3.990055in}{1.047402in}}%
\pgfpathlineto{\pgfqpoint{3.992642in}{1.050156in}}%
\pgfpathlineto{\pgfqpoint{3.995417in}{1.049756in}}%
\pgfpathlineto{\pgfqpoint{3.997990in}{1.049123in}}%
\pgfpathlineto{\pgfqpoint{4.000674in}{1.049991in}}%
\pgfpathlineto{\pgfqpoint{4.003348in}{1.049710in}}%
\pgfpathlineto{\pgfqpoint{4.006034in}{1.048100in}}%
\pgfpathlineto{\pgfqpoint{4.008699in}{1.054839in}}%
\pgfpathlineto{\pgfqpoint{4.011394in}{1.053318in}}%
\pgfpathlineto{\pgfqpoint{4.014186in}{1.050353in}}%
\pgfpathlineto{\pgfqpoint{4.016744in}{1.049805in}}%
\pgfpathlineto{\pgfqpoint{4.019518in}{1.052076in}}%
\pgfpathlineto{\pgfqpoint{4.022097in}{1.054813in}}%
\pgfpathlineto{\pgfqpoint{4.024868in}{1.050239in}}%
\pgfpathlineto{\pgfqpoint{4.027447in}{1.050536in}}%
\pgfpathlineto{\pgfqpoint{4.030229in}{1.048308in}}%
\pgfpathlineto{\pgfqpoint{4.032817in}{1.053403in}}%
\pgfpathlineto{\pgfqpoint{4.035492in}{1.049560in}}%
\pgfpathlineto{\pgfqpoint{4.038174in}{1.053304in}}%
\pgfpathlineto{\pgfqpoint{4.040852in}{1.057499in}}%
\pgfpathlineto{\pgfqpoint{4.043667in}{1.050270in}}%
\pgfpathlineto{\pgfqpoint{4.046210in}{1.046013in}}%
\pgfpathlineto{\pgfqpoint{4.049006in}{1.046494in}}%
\pgfpathlineto{\pgfqpoint{4.051557in}{1.046126in}}%
\pgfpathlineto{\pgfqpoint{4.054326in}{1.046744in}}%
\pgfpathlineto{\pgfqpoint{4.056911in}{1.042442in}}%
\pgfpathlineto{\pgfqpoint{4.059702in}{1.044742in}}%
\pgfpathlineto{\pgfqpoint{4.062266in}{1.045905in}}%
\pgfpathlineto{\pgfqpoint{4.064957in}{1.050123in}}%
\pgfpathlineto{\pgfqpoint{4.067636in}{1.048089in}}%
\pgfpathlineto{\pgfqpoint{4.070313in}{1.054275in}}%
\pgfpathlineto{\pgfqpoint{4.072985in}{1.052155in}}%
\pgfpathlineto{\pgfqpoint{4.075705in}{1.050589in}}%
\pgfpathlineto{\pgfqpoint{4.078471in}{1.049391in}}%
\pgfpathlineto{\pgfqpoint{4.081018in}{1.050677in}}%
\pgfpathlineto{\pgfqpoint{4.083870in}{1.046586in}}%
\pgfpathlineto{\pgfqpoint{4.086385in}{1.045517in}}%
\pgfpathlineto{\pgfqpoint{4.089159in}{1.052043in}}%
\pgfpathlineto{\pgfqpoint{4.091729in}{1.047670in}}%
\pgfpathlineto{\pgfqpoint{4.094527in}{1.052507in}}%
\pgfpathlineto{\pgfqpoint{4.097092in}{1.047395in}}%
\pgfpathlineto{\pgfqpoint{4.099777in}{1.052759in}}%
\pgfpathlineto{\pgfqpoint{4.102456in}{1.049358in}}%
\pgfpathlineto{\pgfqpoint{4.105185in}{1.053929in}}%
\pgfpathlineto{\pgfqpoint{4.107814in}{1.053873in}}%
\pgfpathlineto{\pgfqpoint{4.110488in}{1.051649in}}%
\pgfpathlineto{\pgfqpoint{4.113252in}{1.049687in}}%
\pgfpathlineto{\pgfqpoint{4.115844in}{1.052700in}}%
\pgfpathlineto{\pgfqpoint{4.118554in}{1.049062in}}%
\pgfpathlineto{\pgfqpoint{4.121205in}{1.054348in}}%
\pgfpathlineto{\pgfqpoint{4.124019in}{1.049667in}}%
\pgfpathlineto{\pgfqpoint{4.126553in}{1.049729in}}%
\pgfpathlineto{\pgfqpoint{4.129349in}{1.047402in}}%
\pgfpathlineto{\pgfqpoint{4.131920in}{1.042608in}}%
\pgfpathlineto{\pgfqpoint{4.134615in}{1.047327in}}%
\pgfpathlineto{\pgfqpoint{4.137272in}{1.052301in}}%
\pgfpathlineto{\pgfqpoint{4.139963in}{1.052982in}}%
\pgfpathlineto{\pgfqpoint{4.142713in}{1.052349in}}%
\pgfpathlineto{\pgfqpoint{4.145310in}{1.050378in}}%
\pgfpathlineto{\pgfqpoint{4.148082in}{1.047658in}}%
\pgfpathlineto{\pgfqpoint{4.150665in}{1.046306in}}%
\pgfpathlineto{\pgfqpoint{4.153423in}{1.049978in}}%
\pgfpathlineto{\pgfqpoint{4.156016in}{1.046529in}}%
\pgfpathlineto{\pgfqpoint{4.158806in}{1.050957in}}%
\pgfpathlineto{\pgfqpoint{4.161380in}{1.054279in}}%
\pgfpathlineto{\pgfqpoint{4.164059in}{1.052025in}}%
\pgfpathlineto{\pgfqpoint{4.166737in}{1.046297in}}%
\pgfpathlineto{\pgfqpoint{4.169415in}{1.049093in}}%
\pgfpathlineto{\pgfqpoint{4.172093in}{1.053290in}}%
\pgfpathlineto{\pgfqpoint{4.174770in}{1.050272in}}%
\pgfpathlineto{\pgfqpoint{4.177593in}{1.051155in}}%
\pgfpathlineto{\pgfqpoint{4.180129in}{1.046728in}}%
\pgfpathlineto{\pgfqpoint{4.182899in}{1.043129in}}%
\pgfpathlineto{\pgfqpoint{4.185481in}{1.047978in}}%
\pgfpathlineto{\pgfqpoint{4.188318in}{1.042655in}}%
\pgfpathlineto{\pgfqpoint{4.190842in}{1.046315in}}%
\pgfpathlineto{\pgfqpoint{4.193638in}{1.043514in}}%
\pgfpathlineto{\pgfqpoint{4.196186in}{1.043506in}}%
\pgfpathlineto{\pgfqpoint{4.198878in}{1.043304in}}%
\pgfpathlineto{\pgfqpoint{4.201542in}{1.036114in}}%
\pgfpathlineto{\pgfqpoint{4.204240in}{1.042005in}}%
\pgfpathlineto{\pgfqpoint{4.207076in}{1.033295in}}%
\pgfpathlineto{\pgfqpoint{4.209597in}{1.030366in}}%
\pgfpathlineto{\pgfqpoint{4.212383in}{1.030402in}}%
\pgfpathlineto{\pgfqpoint{4.214948in}{1.035233in}}%
\pgfpathlineto{\pgfqpoint{4.217694in}{1.037141in}}%
\pgfpathlineto{\pgfqpoint{4.220304in}{1.042438in}}%
\pgfpathlineto{\pgfqpoint{4.223082in}{1.049034in}}%
\pgfpathlineto{\pgfqpoint{4.225654in}{1.046667in}}%
\pgfpathlineto{\pgfqpoint{4.228331in}{1.058040in}}%
\pgfpathlineto{\pgfqpoint{4.231013in}{1.049318in}}%
\pgfpathlineto{\pgfqpoint{4.233691in}{1.046963in}}%
\pgfpathlineto{\pgfqpoint{4.236375in}{1.049907in}}%
\pgfpathlineto{\pgfqpoint{4.239084in}{1.044098in}}%
\pgfpathlineto{\pgfqpoint{4.241900in}{1.048978in}}%
\pgfpathlineto{\pgfqpoint{4.244394in}{1.044257in}}%
\pgfpathlineto{\pgfqpoint{4.247225in}{1.047827in}}%
\pgfpathlineto{\pgfqpoint{4.249767in}{1.051160in}}%
\pgfpathlineto{\pgfqpoint{4.252581in}{1.049440in}}%
\pgfpathlineto{\pgfqpoint{4.255120in}{1.050936in}}%
\pgfpathlineto{\pgfqpoint{4.257958in}{1.051276in}}%
\pgfpathlineto{\pgfqpoint{4.260477in}{1.051769in}}%
\pgfpathlineto{\pgfqpoint{4.263157in}{1.059845in}}%
\pgfpathlineto{\pgfqpoint{4.265824in}{1.053808in}}%
\pgfpathlineto{\pgfqpoint{4.268590in}{1.057197in}}%
\pgfpathlineto{\pgfqpoint{4.271187in}{1.058628in}}%
\pgfpathlineto{\pgfqpoint{4.273874in}{1.050289in}}%
\pgfpathlineto{\pgfqpoint{4.276635in}{1.051468in}}%
\pgfpathlineto{\pgfqpoint{4.279212in}{1.048468in}}%
\pgfpathlineto{\pgfqpoint{4.282000in}{1.039995in}}%
\pgfpathlineto{\pgfqpoint{4.284586in}{1.038838in}}%
\pgfpathlineto{\pgfqpoint{4.287399in}{1.040534in}}%
\pgfpathlineto{\pgfqpoint{4.289936in}{1.041351in}}%
\pgfpathlineto{\pgfqpoint{4.292786in}{1.045494in}}%
\pgfpathlineto{\pgfqpoint{4.295299in}{1.045243in}}%
\pgfpathlineto{\pgfqpoint{4.297977in}{1.039634in}}%
\pgfpathlineto{\pgfqpoint{4.300656in}{1.047171in}}%
\pgfpathlineto{\pgfqpoint{4.303357in}{1.040947in}}%
\pgfpathlineto{\pgfqpoint{4.306118in}{1.041215in}}%
\pgfpathlineto{\pgfqpoint{4.308691in}{1.038893in}}%
\pgfpathlineto{\pgfqpoint{4.311494in}{1.042507in}}%
\pgfpathlineto{\pgfqpoint{4.314032in}{1.035928in}}%
\pgfpathlineto{\pgfqpoint{4.316856in}{1.040077in}}%
\pgfpathlineto{\pgfqpoint{4.319405in}{1.034255in}}%
\pgfpathlineto{\pgfqpoint{4.322181in}{1.043221in}}%
\pgfpathlineto{\pgfqpoint{4.324760in}{1.040369in}}%
\pgfpathlineto{\pgfqpoint{4.327440in}{1.040382in}}%
\pgfpathlineto{\pgfqpoint{4.330118in}{1.039328in}}%
\pgfpathlineto{\pgfqpoint{4.332796in}{1.041608in}}%
\pgfpathlineto{\pgfqpoint{4.335463in}{1.047188in}}%
\pgfpathlineto{\pgfqpoint{4.338154in}{1.054898in}}%
\pgfpathlineto{\pgfqpoint{4.340976in}{1.049148in}}%
\pgfpathlineto{\pgfqpoint{4.343510in}{1.045125in}}%
\pgfpathlineto{\pgfqpoint{4.346263in}{1.041572in}}%
\pgfpathlineto{\pgfqpoint{4.348868in}{1.044592in}}%
\pgfpathlineto{\pgfqpoint{4.351645in}{1.043588in}}%
\pgfpathlineto{\pgfqpoint{4.354224in}{1.042176in}}%
\pgfpathlineto{\pgfqpoint{4.357014in}{1.055365in}}%
\pgfpathlineto{\pgfqpoint{4.359582in}{1.046218in}}%
\pgfpathlineto{\pgfqpoint{4.362270in}{1.047572in}}%
\pgfpathlineto{\pgfqpoint{4.364936in}{1.046350in}}%
\pgfpathlineto{\pgfqpoint{4.367646in}{1.046761in}}%
\pgfpathlineto{\pgfqpoint{4.370437in}{1.040291in}}%
\pgfpathlineto{\pgfqpoint{4.372976in}{1.041690in}}%
\pgfpathlineto{\pgfqpoint{4.375761in}{1.046827in}}%
\pgfpathlineto{\pgfqpoint{4.378329in}{1.047147in}}%
\pgfpathlineto{\pgfqpoint{4.381097in}{1.050040in}}%
\pgfpathlineto{\pgfqpoint{4.383674in}{1.042129in}}%
\pgfpathlineto{\pgfqpoint{4.386431in}{1.039406in}}%
\pgfpathlineto{\pgfqpoint{4.389044in}{1.037567in}}%
\pgfpathlineto{\pgfqpoint{4.391721in}{1.039752in}}%
\pgfpathlineto{\pgfqpoint{4.394400in}{1.037431in}}%
\pgfpathlineto{\pgfqpoint{4.397076in}{1.027986in}}%
\pgfpathlineto{\pgfqpoint{4.399745in}{1.034257in}}%
\pgfpathlineto{\pgfqpoint{4.402468in}{1.041003in}}%
\pgfpathlineto{\pgfqpoint{4.405234in}{1.037754in}}%
\pgfpathlineto{\pgfqpoint{4.407788in}{1.040320in}}%
\pgfpathlineto{\pgfqpoint{4.410587in}{1.038122in}}%
\pgfpathlineto{\pgfqpoint{4.413149in}{1.037120in}}%
\pgfpathlineto{\pgfqpoint{4.415932in}{1.042413in}}%
\pgfpathlineto{\pgfqpoint{4.418506in}{1.039821in}}%
\pgfpathlineto{\pgfqpoint{4.421292in}{1.038385in}}%
\pgfpathlineto{\pgfqpoint{4.423863in}{1.042861in}}%
\pgfpathlineto{\pgfqpoint{4.426534in}{1.043438in}}%
\pgfpathlineto{\pgfqpoint{4.429220in}{1.042558in}}%
\pgfpathlineto{\pgfqpoint{4.431901in}{1.042026in}}%
\pgfpathlineto{\pgfqpoint{4.434569in}{1.047733in}}%
\pgfpathlineto{\pgfqpoint{4.437253in}{1.062529in}}%
\pgfpathlineto{\pgfqpoint{4.440041in}{1.110212in}}%
\pgfpathlineto{\pgfqpoint{4.442611in}{1.108239in}}%
\pgfpathlineto{\pgfqpoint{4.445423in}{1.090495in}}%
\pgfpathlineto{\pgfqpoint{4.447965in}{1.078096in}}%
\pgfpathlineto{\pgfqpoint{4.450767in}{1.074306in}}%
\pgfpathlineto{\pgfqpoint{4.453312in}{1.073304in}}%
\pgfpathlineto{\pgfqpoint{4.456138in}{1.069047in}}%
\pgfpathlineto{\pgfqpoint{4.458681in}{1.060497in}}%
\pgfpathlineto{\pgfqpoint{4.461367in}{1.056289in}}%
\pgfpathlineto{\pgfqpoint{4.464029in}{1.062562in}}%
\pgfpathlineto{\pgfqpoint{4.466717in}{1.066762in}}%
\pgfpathlineto{\pgfqpoint{4.469492in}{1.065077in}}%
\pgfpathlineto{\pgfqpoint{4.472059in}{1.063288in}}%
\pgfpathlineto{\pgfqpoint{4.474861in}{1.063964in}}%
\pgfpathlineto{\pgfqpoint{4.477430in}{1.061669in}}%
\pgfpathlineto{\pgfqpoint{4.480201in}{1.058264in}}%
\pgfpathlineto{\pgfqpoint{4.482778in}{1.058771in}}%
\pgfpathlineto{\pgfqpoint{4.485581in}{1.052394in}}%
\pgfpathlineto{\pgfqpoint{4.488130in}{1.052768in}}%
\pgfpathlineto{\pgfqpoint{4.490822in}{1.052006in}}%
\pgfpathlineto{\pgfqpoint{4.493492in}{1.048405in}}%
\pgfpathlineto{\pgfqpoint{4.496167in}{1.042210in}}%
\pgfpathlineto{\pgfqpoint{4.498850in}{1.044026in}}%
\pgfpathlineto{\pgfqpoint{4.501529in}{1.044217in}}%
\pgfpathlineto{\pgfqpoint{4.504305in}{1.045604in}}%
\pgfpathlineto{\pgfqpoint{4.506893in}{1.048128in}}%
\pgfpathlineto{\pgfqpoint{4.509643in}{1.047755in}}%
\pgfpathlineto{\pgfqpoint{4.512246in}{1.045446in}}%
\pgfpathlineto{\pgfqpoint{4.515080in}{1.047879in}}%
\pgfpathlineto{\pgfqpoint{4.517598in}{1.047323in}}%
\pgfpathlineto{\pgfqpoint{4.520345in}{1.040906in}}%
\pgfpathlineto{\pgfqpoint{4.522962in}{1.040236in}}%
\pgfpathlineto{\pgfqpoint{4.525640in}{1.042332in}}%
\pgfpathlineto{\pgfqpoint{4.528307in}{1.042405in}}%
\pgfpathlineto{\pgfqpoint{4.530990in}{1.042902in}}%
\pgfpathlineto{\pgfqpoint{4.533764in}{1.049407in}}%
\pgfpathlineto{\pgfqpoint{4.536400in}{1.068628in}}%
\pgfpathlineto{\pgfqpoint{4.539144in}{1.054394in}}%
\pgfpathlineto{\pgfqpoint{4.541711in}{1.048317in}}%
\pgfpathlineto{\pgfqpoint{4.544464in}{1.041097in}}%
\pgfpathlineto{\pgfqpoint{4.547064in}{1.041670in}}%
\pgfpathlineto{\pgfqpoint{4.549822in}{1.043968in}}%
\pgfpathlineto{\pgfqpoint{4.552425in}{1.042612in}}%
\pgfpathlineto{\pgfqpoint{4.555106in}{1.042827in}}%
\pgfpathlineto{\pgfqpoint{4.557777in}{1.047931in}}%
\pgfpathlineto{\pgfqpoint{4.560448in}{1.049655in}}%
\pgfpathlineto{\pgfqpoint{4.563125in}{1.048112in}}%
\pgfpathlineto{\pgfqpoint{4.565820in}{1.053549in}}%
\pgfpathlineto{\pgfqpoint{4.568612in}{1.043542in}}%
\pgfpathlineto{\pgfqpoint{4.571171in}{1.043570in}}%
\pgfpathlineto{\pgfqpoint{4.573947in}{1.041810in}}%
\pgfpathlineto{\pgfqpoint{4.576531in}{1.043168in}}%
\pgfpathlineto{\pgfqpoint{4.579305in}{1.042970in}}%
\pgfpathlineto{\pgfqpoint{4.581888in}{1.049977in}}%
\pgfpathlineto{\pgfqpoint{4.584672in}{1.044847in}}%
\pgfpathlineto{\pgfqpoint{4.587244in}{1.037096in}}%
\pgfpathlineto{\pgfqpoint{4.589920in}{1.037740in}}%
\pgfpathlineto{\pgfqpoint{4.592589in}{1.031968in}}%
\pgfpathlineto{\pgfqpoint{4.595281in}{1.037754in}}%
\pgfpathlineto{\pgfqpoint{4.597951in}{1.037999in}}%
\pgfpathlineto{\pgfqpoint{4.600633in}{1.043380in}}%
\pgfpathlineto{\pgfqpoint{4.603430in}{1.042186in}}%
\pgfpathlineto{\pgfqpoint{4.605990in}{1.042543in}}%
\pgfpathlineto{\pgfqpoint{4.608808in}{1.041029in}}%
\pgfpathlineto{\pgfqpoint{4.611350in}{1.039580in}}%
\pgfpathlineto{\pgfqpoint{4.614134in}{1.042985in}}%
\pgfpathlineto{\pgfqpoint{4.616702in}{1.045134in}}%
\pgfpathlineto{\pgfqpoint{4.619529in}{1.043470in}}%
\pgfpathlineto{\pgfqpoint{4.622056in}{1.044126in}}%
\pgfpathlineto{\pgfqpoint{4.624741in}{1.040904in}}%
\pgfpathlineto{\pgfqpoint{4.627411in}{1.048848in}}%
\pgfpathlineto{\pgfqpoint{4.630096in}{1.059242in}}%
\pgfpathlineto{\pgfqpoint{4.632902in}{1.059054in}}%
\pgfpathlineto{\pgfqpoint{4.635445in}{1.049579in}}%
\pgfpathlineto{\pgfqpoint{4.638204in}{1.048478in}}%
\pgfpathlineto{\pgfqpoint{4.640809in}{1.045471in}}%
\pgfpathlineto{\pgfqpoint{4.643628in}{1.043900in}}%
\pgfpathlineto{\pgfqpoint{4.646169in}{1.045950in}}%
\pgfpathlineto{\pgfqpoint{4.648922in}{1.043432in}}%
\pgfpathlineto{\pgfqpoint{4.651524in}{1.045528in}}%
\pgfpathlineto{\pgfqpoint{4.654203in}{1.048267in}}%
\pgfpathlineto{\pgfqpoint{4.656873in}{1.050108in}}%
\pgfpathlineto{\pgfqpoint{4.659590in}{1.047691in}}%
\pgfpathlineto{\pgfqpoint{4.662237in}{1.054184in}}%
\pgfpathlineto{\pgfqpoint{4.664923in}{1.048922in}}%
\pgfpathlineto{\pgfqpoint{4.667764in}{1.044061in}}%
\pgfpathlineto{\pgfqpoint{4.670261in}{1.046971in}}%
\pgfpathlineto{\pgfqpoint{4.673068in}{1.044768in}}%
\pgfpathlineto{\pgfqpoint{4.675619in}{1.045157in}}%
\pgfpathlineto{\pgfqpoint{4.678448in}{1.045423in}}%
\pgfpathlineto{\pgfqpoint{4.680988in}{1.050594in}}%
\pgfpathlineto{\pgfqpoint{4.683799in}{1.045089in}}%
\pgfpathlineto{\pgfqpoint{4.686337in}{1.037994in}}%
\pgfpathlineto{\pgfqpoint{4.689051in}{1.036791in}}%
\pgfpathlineto{\pgfqpoint{4.691694in}{1.040546in}}%
\pgfpathlineto{\pgfqpoint{4.694381in}{1.043895in}}%
\pgfpathlineto{\pgfqpoint{4.697170in}{1.042307in}}%
\pgfpathlineto{\pgfqpoint{4.699734in}{1.044398in}}%
\pgfpathlineto{\pgfqpoint{4.702517in}{1.047485in}}%
\pgfpathlineto{\pgfqpoint{4.705094in}{1.048638in}}%
\pgfpathlineto{\pgfqpoint{4.707824in}{1.044652in}}%
\pgfpathlineto{\pgfqpoint{4.710437in}{1.039979in}}%
\pgfpathlineto{\pgfqpoint{4.713275in}{1.041430in}}%
\pgfpathlineto{\pgfqpoint{4.715806in}{1.042947in}}%
\pgfpathlineto{\pgfqpoint{4.718486in}{1.045768in}}%
\pgfpathlineto{\pgfqpoint{4.721160in}{1.045628in}}%
\pgfpathlineto{\pgfqpoint{4.723873in}{1.040942in}}%
\pgfpathlineto{\pgfqpoint{4.726508in}{1.048052in}}%
\pgfpathlineto{\pgfqpoint{4.729233in}{1.046879in}}%
\pgfpathlineto{\pgfqpoint{4.731901in}{1.039655in}}%
\pgfpathlineto{\pgfqpoint{4.734552in}{1.048049in}}%
\pgfpathlineto{\pgfqpoint{4.737348in}{1.050654in}}%
\pgfpathlineto{\pgfqpoint{4.739912in}{1.044398in}}%
\pgfpathlineto{\pgfqpoint{4.742696in}{1.050701in}}%
\pgfpathlineto{\pgfqpoint{4.745256in}{1.049170in}}%
\pgfpathlineto{\pgfqpoint{4.748081in}{1.049856in}}%
\pgfpathlineto{\pgfqpoint{4.750627in}{1.054607in}}%
\pgfpathlineto{\pgfqpoint{4.753298in}{1.052107in}}%
\pgfpathlineto{\pgfqpoint{4.755983in}{1.055566in}}%
\pgfpathlineto{\pgfqpoint{4.758653in}{1.048383in}}%
\pgfpathlineto{\pgfqpoint{4.761337in}{1.052859in}}%
\pgfpathlineto{\pgfqpoint{4.764018in}{1.048826in}}%
\pgfpathlineto{\pgfqpoint{4.766783in}{1.050128in}}%
\pgfpathlineto{\pgfqpoint{4.769367in}{1.048093in}}%
\pgfpathlineto{\pgfqpoint{4.772198in}{1.048934in}}%
\pgfpathlineto{\pgfqpoint{4.774732in}{1.053257in}}%
\pgfpathlineto{\pgfqpoint{4.777535in}{1.050876in}}%
\pgfpathlineto{\pgfqpoint{4.780083in}{1.047739in}}%
\pgfpathlineto{\pgfqpoint{4.782872in}{1.049699in}}%
\pgfpathlineto{\pgfqpoint{4.785445in}{1.049311in}}%
\pgfpathlineto{\pgfqpoint{4.788116in}{1.050588in}}%
\pgfpathlineto{\pgfqpoint{4.790798in}{1.045687in}}%
\pgfpathlineto{\pgfqpoint{4.793512in}{1.046534in}}%
\pgfpathlineto{\pgfqpoint{4.796274in}{1.045431in}}%
\pgfpathlineto{\pgfqpoint{4.798830in}{1.043881in}}%
\pgfpathlineto{\pgfqpoint{4.801586in}{1.041865in}}%
\pgfpathlineto{\pgfqpoint{4.804193in}{1.048099in}}%
\pgfpathlineto{\pgfqpoint{4.807017in}{1.042629in}}%
\pgfpathlineto{\pgfqpoint{4.809538in}{1.037937in}}%
\pgfpathlineto{\pgfqpoint{4.812377in}{1.027986in}}%
\pgfpathlineto{\pgfqpoint{4.814907in}{1.031328in}}%
\pgfpathlineto{\pgfqpoint{4.817587in}{1.027986in}}%
\pgfpathlineto{\pgfqpoint{4.820265in}{1.027986in}}%
\pgfpathlineto{\pgfqpoint{4.822945in}{1.027986in}}%
\pgfpathlineto{\pgfqpoint{4.825619in}{1.034007in}}%
\pgfpathlineto{\pgfqpoint{4.828291in}{1.036956in}}%
\pgfpathlineto{\pgfqpoint{4.831045in}{1.031819in}}%
\pgfpathlineto{\pgfqpoint{4.833657in}{1.030689in}}%
\pgfpathlineto{\pgfqpoint{4.837992in}{1.028564in}}%
\pgfpathlineto{\pgfqpoint{4.839922in}{1.034235in}}%
\pgfpathlineto{\pgfqpoint{4.842380in}{1.030986in}}%
\pgfpathlineto{\pgfqpoint{4.844361in}{1.036197in}}%
\pgfpathlineto{\pgfqpoint{4.847127in}{1.033119in}}%
\pgfpathlineto{\pgfqpoint{4.849715in}{1.033604in}}%
\pgfpathlineto{\pgfqpoint{4.852404in}{1.027986in}}%
\pgfpathlineto{\pgfqpoint{4.855070in}{1.032002in}}%
\pgfpathlineto{\pgfqpoint{4.857807in}{1.029610in}}%
\pgfpathlineto{\pgfqpoint{4.860544in}{1.034323in}}%
\pgfpathlineto{\pgfqpoint{4.863116in}{1.043854in}}%
\pgfpathlineto{\pgfqpoint{4.865910in}{1.037845in}}%
\pgfpathlineto{\pgfqpoint{4.868474in}{1.044156in}}%
\pgfpathlineto{\pgfqpoint{4.871209in}{1.047048in}}%
\pgfpathlineto{\pgfqpoint{4.873832in}{1.047186in}}%
\pgfpathlineto{\pgfqpoint{4.876636in}{1.043261in}}%
\pgfpathlineto{\pgfqpoint{4.879180in}{1.046204in}}%
\pgfpathlineto{\pgfqpoint{4.881864in}{1.046024in}}%
\pgfpathlineto{\pgfqpoint{4.884540in}{1.043978in}}%
\pgfpathlineto{\pgfqpoint{4.887211in}{1.039686in}}%
\pgfpathlineto{\pgfqpoint{4.889902in}{1.050356in}}%
\pgfpathlineto{\pgfqpoint{4.892611in}{1.051518in}}%
\pgfpathlineto{\pgfqpoint{4.895399in}{1.058670in}}%
\pgfpathlineto{\pgfqpoint{4.897938in}{1.059068in}}%
\pgfpathlineto{\pgfqpoint{4.900712in}{1.052933in}}%
\pgfpathlineto{\pgfqpoint{4.903295in}{1.052998in}}%
\pgfpathlineto{\pgfqpoint{4.906096in}{1.064594in}}%
\pgfpathlineto{\pgfqpoint{4.908648in}{1.056168in}}%
\pgfpathlineto{\pgfqpoint{4.911435in}{1.055986in}}%
\pgfpathlineto{\pgfqpoint{4.914009in}{1.060069in}}%
\pgfpathlineto{\pgfqpoint{4.916681in}{1.060143in}}%
\pgfpathlineto{\pgfqpoint{4.919352in}{1.058662in}}%
\pgfpathlineto{\pgfqpoint{4.922041in}{1.061760in}}%
\pgfpathlineto{\pgfqpoint{4.924708in}{1.062006in}}%
\pgfpathlineto{\pgfqpoint{4.927400in}{1.060425in}}%
\pgfpathlineto{\pgfqpoint{4.930170in}{1.063405in}}%
\pgfpathlineto{\pgfqpoint{4.932742in}{1.064888in}}%
\pgfpathlineto{\pgfqpoint{4.935515in}{1.060343in}}%
\pgfpathlineto{\pgfqpoint{4.938112in}{1.055671in}}%
\pgfpathlineto{\pgfqpoint{4.940881in}{1.053021in}}%
\pgfpathlineto{\pgfqpoint{4.943466in}{1.049273in}}%
\pgfpathlineto{\pgfqpoint{4.946151in}{1.051443in}}%
\pgfpathlineto{\pgfqpoint{4.948827in}{1.052372in}}%
\pgfpathlineto{\pgfqpoint{4.951504in}{1.054669in}}%
\pgfpathlineto{\pgfqpoint{4.954182in}{1.054041in}}%
\pgfpathlineto{\pgfqpoint{4.956862in}{1.043422in}}%
\pgfpathlineto{\pgfqpoint{4.959689in}{1.042930in}}%
\pgfpathlineto{\pgfqpoint{4.962219in}{1.044025in}}%
\pgfpathlineto{\pgfqpoint{4.965002in}{1.041696in}}%
\pgfpathlineto{\pgfqpoint{4.967575in}{1.043954in}}%
\pgfpathlineto{\pgfqpoint{4.970314in}{1.038584in}}%
\pgfpathlineto{\pgfqpoint{4.972933in}{1.047079in}}%
\pgfpathlineto{\pgfqpoint{4.975703in}{1.045049in}}%
\pgfpathlineto{\pgfqpoint{4.978287in}{1.043026in}}%
\pgfpathlineto{\pgfqpoint{4.980967in}{1.041208in}}%
\pgfpathlineto{\pgfqpoint{4.983637in}{1.037113in}}%
\pgfpathlineto{\pgfqpoint{4.986325in}{1.038295in}}%
\pgfpathlineto{\pgfqpoint{4.989001in}{1.040508in}}%
\pgfpathlineto{\pgfqpoint{4.991683in}{1.044528in}}%
\pgfpathlineto{\pgfqpoint{4.994390in}{1.042496in}}%
\pgfpathlineto{\pgfqpoint{4.997028in}{1.041421in}}%
\pgfpathlineto{\pgfqpoint{4.999780in}{1.038475in}}%
\pgfpathlineto{\pgfqpoint{5.002384in}{1.042000in}}%
\pgfpathlineto{\pgfqpoint{5.005178in}{1.041945in}}%
\pgfpathlineto{\pgfqpoint{5.007751in}{1.043281in}}%
\pgfpathlineto{\pgfqpoint{5.010562in}{1.050858in}}%
\pgfpathlineto{\pgfqpoint{5.013104in}{1.043842in}}%
\pgfpathlineto{\pgfqpoint{5.015820in}{1.057059in}}%
\pgfpathlineto{\pgfqpoint{5.018466in}{1.058692in}}%
\pgfpathlineto{\pgfqpoint{5.021147in}{1.053108in}}%
\pgfpathlineto{\pgfqpoint{5.023927in}{1.052834in}}%
\pgfpathlineto{\pgfqpoint{5.026501in}{1.049788in}}%
\pgfpathlineto{\pgfqpoint{5.029275in}{1.048387in}}%
\pgfpathlineto{\pgfqpoint{5.031849in}{1.049552in}}%
\pgfpathlineto{\pgfqpoint{5.034649in}{1.048417in}}%
\pgfpathlineto{\pgfqpoint{5.037214in}{1.050897in}}%
\pgfpathlineto{\pgfqpoint{5.039962in}{1.046238in}}%
\pgfpathlineto{\pgfqpoint{5.042572in}{1.041849in}}%
\pgfpathlineto{\pgfqpoint{5.045249in}{1.045722in}}%
\pgfpathlineto{\pgfqpoint{5.047924in}{1.040902in}}%
\pgfpathlineto{\pgfqpoint{5.050606in}{1.042556in}}%
\pgfpathlineto{\pgfqpoint{5.053284in}{1.047791in}}%
\pgfpathlineto{\pgfqpoint{5.055952in}{1.049060in}}%
\pgfpathlineto{\pgfqpoint{5.058711in}{1.046559in}}%
\pgfpathlineto{\pgfqpoint{5.061315in}{1.046919in}}%
\pgfpathlineto{\pgfqpoint{5.064144in}{1.046877in}}%
\pgfpathlineto{\pgfqpoint{5.066677in}{1.050226in}}%
\pgfpathlineto{\pgfqpoint{5.069463in}{1.054031in}}%
\pgfpathlineto{\pgfqpoint{5.072030in}{1.048793in}}%
\pgfpathlineto{\pgfqpoint{5.074851in}{1.049777in}}%
\pgfpathlineto{\pgfqpoint{5.077390in}{1.038630in}}%
\pgfpathlineto{\pgfqpoint{5.080067in}{1.041892in}}%
\pgfpathlineto{\pgfqpoint{5.082746in}{1.045773in}}%
\pgfpathlineto{\pgfqpoint{5.085426in}{1.042566in}}%
\pgfpathlineto{\pgfqpoint{5.088103in}{1.043210in}}%
\pgfpathlineto{\pgfqpoint{5.090788in}{1.045523in}}%
\pgfpathlineto{\pgfqpoint{5.093579in}{1.043389in}}%
\pgfpathlineto{\pgfqpoint{5.096142in}{1.044823in}}%
\pgfpathlineto{\pgfqpoint{5.098948in}{1.046545in}}%
\pgfpathlineto{\pgfqpoint{5.101496in}{1.048014in}}%
\pgfpathlineto{\pgfqpoint{5.104312in}{1.050779in}}%
\pgfpathlineto{\pgfqpoint{5.106842in}{1.052787in}}%
\pgfpathlineto{\pgfqpoint{5.109530in}{1.048666in}}%
\pgfpathlineto{\pgfqpoint{5.112209in}{1.048213in}}%
\pgfpathlineto{\pgfqpoint{5.114887in}{1.049034in}}%
\pgfpathlineto{\pgfqpoint{5.117550in}{1.051423in}}%
\pgfpathlineto{\pgfqpoint{5.120243in}{1.047848in}}%
\pgfpathlineto{\pgfqpoint{5.123042in}{1.051998in}}%
\pgfpathlineto{\pgfqpoint{5.125599in}{1.050530in}}%
\pgfpathlineto{\pgfqpoint{5.128421in}{1.049136in}}%
\pgfpathlineto{\pgfqpoint{5.130953in}{1.046942in}}%
\pgfpathlineto{\pgfqpoint{5.133716in}{1.045595in}}%
\pgfpathlineto{\pgfqpoint{5.136311in}{1.050058in}}%
\pgfpathlineto{\pgfqpoint{5.139072in}{1.045676in}}%
\pgfpathlineto{\pgfqpoint{5.141660in}{1.051077in}}%
\pgfpathlineto{\pgfqpoint{5.144349in}{1.043621in}}%
\pgfpathlineto{\pgfqpoint{5.147029in}{1.039946in}}%
\pgfpathlineto{\pgfqpoint{5.149734in}{1.045970in}}%
\pgfpathlineto{\pgfqpoint{5.152382in}{1.035594in}}%
\pgfpathlineto{\pgfqpoint{5.155059in}{1.030476in}}%
\pgfpathlineto{\pgfqpoint{5.157815in}{1.037050in}}%
\pgfpathlineto{\pgfqpoint{5.160420in}{1.040758in}}%
\pgfpathlineto{\pgfqpoint{5.163243in}{1.041953in}}%
\pgfpathlineto{\pgfqpoint{5.165775in}{1.042933in}}%
\pgfpathlineto{\pgfqpoint{5.168591in}{1.043339in}}%
\pgfpathlineto{\pgfqpoint{5.171133in}{1.046069in}}%
\pgfpathlineto{\pgfqpoint{5.173925in}{1.047345in}}%
\pgfpathlineto{\pgfqpoint{5.176477in}{1.045365in}}%
\pgfpathlineto{\pgfqpoint{5.179188in}{1.050158in}}%
\pgfpathlineto{\pgfqpoint{5.181848in}{1.044454in}}%
\pgfpathlineto{\pgfqpoint{5.184522in}{1.048652in}}%
\pgfpathlineto{\pgfqpoint{5.187294in}{1.049832in}}%
\pgfpathlineto{\pgfqpoint{5.189880in}{1.046809in}}%
\pgfpathlineto{\pgfqpoint{5.192680in}{1.053111in}}%
\pgfpathlineto{\pgfqpoint{5.195239in}{1.054191in}}%
\pgfpathlineto{\pgfqpoint{5.198008in}{1.055255in}}%
\pgfpathlineto{\pgfqpoint{5.200594in}{1.045040in}}%
\pgfpathlineto{\pgfqpoint{5.203388in}{1.051239in}}%
\pgfpathlineto{\pgfqpoint{5.205952in}{1.055113in}}%
\pgfpathlineto{\pgfqpoint{5.208630in}{1.055006in}}%
\pgfpathlineto{\pgfqpoint{5.211299in}{1.051503in}}%
\pgfpathlineto{\pgfqpoint{5.214027in}{1.051814in}}%
\pgfpathlineto{\pgfqpoint{5.216667in}{1.057365in}}%
\pgfpathlineto{\pgfqpoint{5.219345in}{1.050968in}}%
\pgfpathlineto{\pgfqpoint{5.222151in}{1.051565in}}%
\pgfpathlineto{\pgfqpoint{5.224695in}{1.054114in}}%
\pgfpathlineto{\pgfqpoint{5.227470in}{1.045798in}}%
\pgfpathlineto{\pgfqpoint{5.230059in}{1.044687in}}%
\pgfpathlineto{\pgfqpoint{5.232855in}{1.045515in}}%
\pgfpathlineto{\pgfqpoint{5.235409in}{1.050054in}}%
\pgfpathlineto{\pgfqpoint{5.238173in}{1.051769in}}%
\pgfpathlineto{\pgfqpoint{5.240777in}{1.049271in}}%
\pgfpathlineto{\pgfqpoint{5.243445in}{1.046166in}}%
\pgfpathlineto{\pgfqpoint{5.246130in}{1.043351in}}%
\pgfpathlineto{\pgfqpoint{5.248816in}{1.048623in}}%
\pgfpathlineto{\pgfqpoint{5.251590in}{1.051038in}}%
\pgfpathlineto{\pgfqpoint{5.254236in}{1.047911in}}%
\pgfpathlineto{\pgfqpoint{5.256973in}{1.043963in}}%
\pgfpathlineto{\pgfqpoint{5.259511in}{1.040472in}}%
\pgfpathlineto{\pgfqpoint{5.262264in}{1.047243in}}%
\pgfpathlineto{\pgfqpoint{5.264876in}{1.046039in}}%
\pgfpathlineto{\pgfqpoint{5.267691in}{1.047627in}}%
\pgfpathlineto{\pgfqpoint{5.270238in}{1.051458in}}%
\pgfpathlineto{\pgfqpoint{5.272913in}{1.055284in}}%
\pgfpathlineto{\pgfqpoint{5.275589in}{1.059440in}}%
\pgfpathlineto{\pgfqpoint{5.278322in}{1.050669in}}%
\pgfpathlineto{\pgfqpoint{5.280947in}{1.041143in}}%
\pgfpathlineto{\pgfqpoint{5.283631in}{1.048692in}}%
\pgfpathlineto{\pgfqpoint{5.286436in}{1.047786in}}%
\pgfpathlineto{\pgfqpoint{5.288984in}{1.049132in}}%
\pgfpathlineto{\pgfqpoint{5.291794in}{1.053785in}}%
\pgfpathlineto{\pgfqpoint{5.294339in}{1.054009in}}%
\pgfpathlineto{\pgfqpoint{5.297140in}{1.046954in}}%
\pgfpathlineto{\pgfqpoint{5.299696in}{1.046381in}}%
\pgfpathlineto{\pgfqpoint{5.302443in}{1.049251in}}%
\pgfpathlineto{\pgfqpoint{5.305054in}{1.049732in}}%
\pgfpathlineto{\pgfqpoint{5.307731in}{1.052673in}}%
\pgfpathlineto{\pgfqpoint{5.310411in}{1.054797in}}%
\pgfpathlineto{\pgfqpoint{5.313089in}{1.049505in}}%
\pgfpathlineto{\pgfqpoint{5.315754in}{1.053799in}}%
\pgfpathlineto{\pgfqpoint{5.318430in}{1.069618in}}%
\pgfpathlineto{\pgfqpoint{5.321256in}{1.072333in}}%
\pgfpathlineto{\pgfqpoint{5.323802in}{1.070544in}}%
\pgfpathlineto{\pgfqpoint{5.326564in}{1.060436in}}%
\pgfpathlineto{\pgfqpoint{5.329159in}{1.058407in}}%
\pgfpathlineto{\pgfqpoint{5.331973in}{1.046094in}}%
\pgfpathlineto{\pgfqpoint{5.334510in}{1.046122in}}%
\pgfpathlineto{\pgfqpoint{5.337353in}{1.044049in}}%
\pgfpathlineto{\pgfqpoint{5.339872in}{1.049949in}}%
\pgfpathlineto{\pgfqpoint{5.342549in}{1.050762in}}%
\pgfpathlineto{\pgfqpoint{5.345224in}{1.047508in}}%
\pgfpathlineto{\pgfqpoint{5.347905in}{1.044692in}}%
\pgfpathlineto{\pgfqpoint{5.350723in}{1.046926in}}%
\pgfpathlineto{\pgfqpoint{5.353262in}{1.045832in}}%
\pgfpathlineto{\pgfqpoint{5.356056in}{1.047154in}}%
\pgfpathlineto{\pgfqpoint{5.358612in}{1.043634in}}%
\pgfpathlineto{\pgfqpoint{5.361370in}{1.044503in}}%
\pgfpathlineto{\pgfqpoint{5.363966in}{1.040593in}}%
\pgfpathlineto{\pgfqpoint{5.366727in}{1.048748in}}%
\pgfpathlineto{\pgfqpoint{5.369335in}{1.048722in}}%
\pgfpathlineto{\pgfqpoint{5.372013in}{1.054780in}}%
\pgfpathlineto{\pgfqpoint{5.374692in}{1.048834in}}%
\pgfpathlineto{\pgfqpoint{5.377370in}{1.044408in}}%
\pgfpathlineto{\pgfqpoint{5.380048in}{1.042082in}}%
\pgfpathlineto{\pgfqpoint{5.382725in}{1.046397in}}%
\pgfpathlineto{\pgfqpoint{5.385550in}{1.046471in}}%
\pgfpathlineto{\pgfqpoint{5.388083in}{1.043326in}}%
\pgfpathlineto{\pgfqpoint{5.390900in}{1.034498in}}%
\pgfpathlineto{\pgfqpoint{5.393441in}{1.027986in}}%
\pgfpathlineto{\pgfqpoint{5.396219in}{1.041157in}}%
\pgfpathlineto{\pgfqpoint{5.398784in}{1.036604in}}%
\pgfpathlineto{\pgfqpoint{5.401576in}{1.034529in}}%
\pgfpathlineto{\pgfqpoint{5.404154in}{1.039729in}}%
\pgfpathlineto{\pgfqpoint{5.406832in}{1.044049in}}%
\pgfpathlineto{\pgfqpoint{5.409507in}{1.046352in}}%
\pgfpathlineto{\pgfqpoint{5.412190in}{1.040838in}}%
\pgfpathlineto{\pgfqpoint{5.414954in}{1.043782in}}%
\pgfpathlineto{\pgfqpoint{5.417547in}{1.043871in}}%
\pgfpathlineto{\pgfqpoint{5.420304in}{1.047364in}}%
\pgfpathlineto{\pgfqpoint{5.422897in}{1.048556in}}%
\pgfpathlineto{\pgfqpoint{5.425661in}{1.045388in}}%
\pgfpathlineto{\pgfqpoint{5.428259in}{1.050032in}}%
\pgfpathlineto{\pgfqpoint{5.431015in}{1.050348in}}%
\pgfpathlineto{\pgfqpoint{5.433616in}{1.049727in}}%
\pgfpathlineto{\pgfqpoint{5.436295in}{1.048358in}}%
\pgfpathlineto{\pgfqpoint{5.438974in}{1.050536in}}%
\pgfpathlineto{\pgfqpoint{5.441698in}{1.050900in}}%
\pgfpathlineto{\pgfqpoint{5.444328in}{1.055335in}}%
\pgfpathlineto{\pgfqpoint{5.447021in}{1.054117in}}%
\pgfpathlineto{\pgfqpoint{5.449769in}{1.048362in}}%
\pgfpathlineto{\pgfqpoint{5.452365in}{1.045031in}}%
\pgfpathlineto{\pgfqpoint{5.455168in}{1.054415in}}%
\pgfpathlineto{\pgfqpoint{5.457721in}{1.051557in}}%
\pgfpathlineto{\pgfqpoint{5.460489in}{1.054723in}}%
\pgfpathlineto{\pgfqpoint{5.463079in}{1.054327in}}%
\pgfpathlineto{\pgfqpoint{5.465888in}{1.055534in}}%
\pgfpathlineto{\pgfqpoint{5.468425in}{1.055209in}}%
\pgfpathlineto{\pgfqpoint{5.471113in}{1.051130in}}%
\pgfpathlineto{\pgfqpoint{5.473792in}{1.039938in}}%
\pgfpathlineto{\pgfqpoint{5.476458in}{1.043847in}}%
\pgfpathlineto{\pgfqpoint{5.479152in}{1.045339in}}%
\pgfpathlineto{\pgfqpoint{5.481825in}{1.043551in}}%
\pgfpathlineto{\pgfqpoint{5.484641in}{1.043486in}}%
\pgfpathlineto{\pgfqpoint{5.487176in}{1.045116in}}%
\pgfpathlineto{\pgfqpoint{5.490000in}{1.049997in}}%
\pgfpathlineto{\pgfqpoint{5.492541in}{1.044837in}}%
\pgfpathlineto{\pgfqpoint{5.495346in}{1.046520in}}%
\pgfpathlineto{\pgfqpoint{5.497898in}{1.048667in}}%
\pgfpathlineto{\pgfqpoint{5.500687in}{1.042347in}}%
\pgfpathlineto{\pgfqpoint{5.503255in}{1.046793in}}%
\pgfpathlineto{\pgfqpoint{5.505933in}{1.047694in}}%
\pgfpathlineto{\pgfqpoint{5.508612in}{1.062314in}}%
\pgfpathlineto{\pgfqpoint{5.511290in}{1.061744in}}%
\pgfpathlineto{\pgfqpoint{5.514080in}{1.062432in}}%
\pgfpathlineto{\pgfqpoint{5.516646in}{1.055226in}}%
\pgfpathlineto{\pgfqpoint{5.519433in}{1.050348in}}%
\pgfpathlineto{\pgfqpoint{5.522003in}{1.046710in}}%
\pgfpathlineto{\pgfqpoint{5.524756in}{1.058821in}}%
\pgfpathlineto{\pgfqpoint{5.527360in}{1.046001in}}%
\pgfpathlineto{\pgfqpoint{5.530148in}{1.045493in}}%
\pgfpathlineto{\pgfqpoint{5.532717in}{1.051756in}}%
\pgfpathlineto{\pgfqpoint{5.535395in}{1.054104in}}%
\pgfpathlineto{\pgfqpoint{5.538074in}{1.049714in}}%
\pgfpathlineto{\pgfqpoint{5.540750in}{1.039809in}}%
\pgfpathlineto{\pgfqpoint{5.543421in}{1.051384in}}%
\pgfpathlineto{\pgfqpoint{5.546110in}{1.052074in}}%
\pgfpathlineto{\pgfqpoint{5.548921in}{1.046475in}}%
\pgfpathlineto{\pgfqpoint{5.551457in}{1.044824in}}%
\pgfpathlineto{\pgfqpoint{5.554198in}{1.046923in}}%
\pgfpathlineto{\pgfqpoint{5.556822in}{1.048301in}}%
\pgfpathlineto{\pgfqpoint{5.559612in}{1.049118in}}%
\pgfpathlineto{\pgfqpoint{5.562180in}{1.045268in}}%
\pgfpathlineto{\pgfqpoint{5.564940in}{1.045123in}}%
\pgfpathlineto{\pgfqpoint{5.567536in}{1.050870in}}%
\pgfpathlineto{\pgfqpoint{5.570215in}{1.049053in}}%
\pgfpathlineto{\pgfqpoint{5.572893in}{1.047133in}}%
\pgfpathlineto{\pgfqpoint{5.575596in}{1.049125in}}%
\pgfpathlineto{\pgfqpoint{5.578342in}{1.052918in}}%
\pgfpathlineto{\pgfqpoint{5.580914in}{1.051329in}}%
\pgfpathlineto{\pgfqpoint{5.583709in}{1.053502in}}%
\pgfpathlineto{\pgfqpoint{5.586269in}{1.046290in}}%
\pgfpathlineto{\pgfqpoint{5.589040in}{1.048898in}}%
\pgfpathlineto{\pgfqpoint{5.591641in}{1.048736in}}%
\pgfpathlineto{\pgfqpoint{5.594368in}{1.045114in}}%
\pgfpathlineto{\pgfqpoint{5.596999in}{1.046752in}}%
\pgfpathlineto{\pgfqpoint{5.599674in}{1.050593in}}%
\pgfpathlineto{\pgfqpoint{5.602352in}{1.049552in}}%
\pgfpathlineto{\pgfqpoint{5.605073in}{1.055555in}}%
\pgfpathlineto{\pgfqpoint{5.607698in}{1.055772in}}%
\pgfpathlineto{\pgfqpoint{5.610389in}{1.059252in}}%
\pgfpathlineto{\pgfqpoint{5.613235in}{1.048865in}}%
\pgfpathlineto{\pgfqpoint{5.615743in}{1.052815in}}%
\pgfpathlineto{\pgfqpoint{5.618526in}{1.055140in}}%
\pgfpathlineto{\pgfqpoint{5.621102in}{1.053740in}}%
\pgfpathlineto{\pgfqpoint{5.623868in}{1.055010in}}%
\pgfpathlineto{\pgfqpoint{5.626460in}{1.049315in}}%
\pgfpathlineto{\pgfqpoint{5.629232in}{1.046140in}}%
\pgfpathlineto{\pgfqpoint{5.631815in}{1.050240in}}%
\pgfpathlineto{\pgfqpoint{5.634496in}{1.076268in}}%
\pgfpathlineto{\pgfqpoint{5.637172in}{1.087457in}}%
\pgfpathlineto{\pgfqpoint{5.639852in}{1.085695in}}%
\pgfpathlineto{\pgfqpoint{5.642518in}{1.092984in}}%
\pgfpathlineto{\pgfqpoint{5.645243in}{1.086365in}}%
\pgfpathlineto{\pgfqpoint{5.648008in}{1.087755in}}%
\pgfpathlineto{\pgfqpoint{5.650563in}{1.074682in}}%
\pgfpathlineto{\pgfqpoint{5.653376in}{1.075299in}}%
\pgfpathlineto{\pgfqpoint{5.655919in}{1.066040in}}%
\pgfpathlineto{\pgfqpoint{5.658723in}{1.052304in}}%
\pgfpathlineto{\pgfqpoint{5.661273in}{1.051968in}}%
\pgfpathlineto{\pgfqpoint{5.664099in}{1.054497in}}%
\pgfpathlineto{\pgfqpoint{5.666632in}{1.054207in}}%
\pgfpathlineto{\pgfqpoint{5.669313in}{1.062970in}}%
\pgfpathlineto{\pgfqpoint{5.671991in}{1.059726in}}%
\pgfpathlineto{\pgfqpoint{5.674667in}{1.055974in}}%
\pgfpathlineto{\pgfqpoint{5.677486in}{1.055841in}}%
\pgfpathlineto{\pgfqpoint{5.680027in}{1.058377in}}%
\pgfpathlineto{\pgfqpoint{5.682836in}{1.053191in}}%
\pgfpathlineto{\pgfqpoint{5.685385in}{1.041999in}}%
\pgfpathlineto{\pgfqpoint{5.688159in}{1.041403in}}%
\pgfpathlineto{\pgfqpoint{5.690730in}{1.047086in}}%
\pgfpathlineto{\pgfqpoint{5.693473in}{1.048069in}}%
\pgfpathlineto{\pgfqpoint{5.696101in}{1.054329in}}%
\pgfpathlineto{\pgfqpoint{5.698775in}{1.055506in}}%
\pgfpathlineto{\pgfqpoint{5.701453in}{1.053206in}}%
\pgfpathlineto{\pgfqpoint{5.704130in}{1.054285in}}%
\pgfpathlineto{\pgfqpoint{5.706800in}{1.054388in}}%
\pgfpathlineto{\pgfqpoint{5.709490in}{1.052843in}}%
\pgfpathlineto{\pgfqpoint{5.712291in}{1.046805in}}%
\pgfpathlineto{\pgfqpoint{5.714834in}{1.049921in}}%
\pgfpathlineto{\pgfqpoint{5.717671in}{1.047168in}}%
\pgfpathlineto{\pgfqpoint{5.720201in}{1.044367in}}%
\pgfpathlineto{\pgfqpoint{5.722950in}{1.047705in}}%
\pgfpathlineto{\pgfqpoint{5.725548in}{1.049167in}}%
\pgfpathlineto{\pgfqpoint{5.728339in}{1.051404in}}%
\pgfpathlineto{\pgfqpoint{5.730919in}{1.052454in}}%
\pgfpathlineto{\pgfqpoint{5.733594in}{1.043955in}}%
\pgfpathlineto{\pgfqpoint{5.736276in}{1.048589in}}%
\pgfpathlineto{\pgfqpoint{5.738974in}{1.047807in}}%
\pgfpathlineto{\pgfqpoint{5.741745in}{1.052036in}}%
\pgfpathlineto{\pgfqpoint{5.744310in}{1.050344in}}%
\pgfpathlineto{\pgfqpoint{5.744310in}{0.413320in}}%
\pgfpathlineto{\pgfqpoint{5.744310in}{0.413320in}}%
\pgfpathlineto{\pgfqpoint{5.741745in}{0.413320in}}%
\pgfpathlineto{\pgfqpoint{5.738974in}{0.413320in}}%
\pgfpathlineto{\pgfqpoint{5.736276in}{0.413320in}}%
\pgfpathlineto{\pgfqpoint{5.733594in}{0.413320in}}%
\pgfpathlineto{\pgfqpoint{5.730919in}{0.413320in}}%
\pgfpathlineto{\pgfqpoint{5.728339in}{0.413320in}}%
\pgfpathlineto{\pgfqpoint{5.725548in}{0.413320in}}%
\pgfpathlineto{\pgfqpoint{5.722950in}{0.413320in}}%
\pgfpathlineto{\pgfqpoint{5.720201in}{0.413320in}}%
\pgfpathlineto{\pgfqpoint{5.717671in}{0.413320in}}%
\pgfpathlineto{\pgfqpoint{5.714834in}{0.413320in}}%
\pgfpathlineto{\pgfqpoint{5.712291in}{0.413320in}}%
\pgfpathlineto{\pgfqpoint{5.709490in}{0.413320in}}%
\pgfpathlineto{\pgfqpoint{5.706800in}{0.413320in}}%
\pgfpathlineto{\pgfqpoint{5.704130in}{0.413320in}}%
\pgfpathlineto{\pgfqpoint{5.701453in}{0.413320in}}%
\pgfpathlineto{\pgfqpoint{5.698775in}{0.413320in}}%
\pgfpathlineto{\pgfqpoint{5.696101in}{0.413320in}}%
\pgfpathlineto{\pgfqpoint{5.693473in}{0.413320in}}%
\pgfpathlineto{\pgfqpoint{5.690730in}{0.413320in}}%
\pgfpathlineto{\pgfqpoint{5.688159in}{0.413320in}}%
\pgfpathlineto{\pgfqpoint{5.685385in}{0.413320in}}%
\pgfpathlineto{\pgfqpoint{5.682836in}{0.413320in}}%
\pgfpathlineto{\pgfqpoint{5.680027in}{0.413320in}}%
\pgfpathlineto{\pgfqpoint{5.677486in}{0.413320in}}%
\pgfpathlineto{\pgfqpoint{5.674667in}{0.413320in}}%
\pgfpathlineto{\pgfqpoint{5.671991in}{0.413320in}}%
\pgfpathlineto{\pgfqpoint{5.669313in}{0.413320in}}%
\pgfpathlineto{\pgfqpoint{5.666632in}{0.413320in}}%
\pgfpathlineto{\pgfqpoint{5.664099in}{0.413320in}}%
\pgfpathlineto{\pgfqpoint{5.661273in}{0.413320in}}%
\pgfpathlineto{\pgfqpoint{5.658723in}{0.413320in}}%
\pgfpathlineto{\pgfqpoint{5.655919in}{0.413320in}}%
\pgfpathlineto{\pgfqpoint{5.653376in}{0.413320in}}%
\pgfpathlineto{\pgfqpoint{5.650563in}{0.413320in}}%
\pgfpathlineto{\pgfqpoint{5.648008in}{0.413320in}}%
\pgfpathlineto{\pgfqpoint{5.645243in}{0.413320in}}%
\pgfpathlineto{\pgfqpoint{5.642518in}{0.413320in}}%
\pgfpathlineto{\pgfqpoint{5.639852in}{0.413320in}}%
\pgfpathlineto{\pgfqpoint{5.637172in}{0.413320in}}%
\pgfpathlineto{\pgfqpoint{5.634496in}{0.413320in}}%
\pgfpathlineto{\pgfqpoint{5.631815in}{0.413320in}}%
\pgfpathlineto{\pgfqpoint{5.629232in}{0.413320in}}%
\pgfpathlineto{\pgfqpoint{5.626460in}{0.413320in}}%
\pgfpathlineto{\pgfqpoint{5.623868in}{0.413320in}}%
\pgfpathlineto{\pgfqpoint{5.621102in}{0.413320in}}%
\pgfpathlineto{\pgfqpoint{5.618526in}{0.413320in}}%
\pgfpathlineto{\pgfqpoint{5.615743in}{0.413320in}}%
\pgfpathlineto{\pgfqpoint{5.613235in}{0.413320in}}%
\pgfpathlineto{\pgfqpoint{5.610389in}{0.413320in}}%
\pgfpathlineto{\pgfqpoint{5.607698in}{0.413320in}}%
\pgfpathlineto{\pgfqpoint{5.605073in}{0.413320in}}%
\pgfpathlineto{\pgfqpoint{5.602352in}{0.413320in}}%
\pgfpathlineto{\pgfqpoint{5.599674in}{0.413320in}}%
\pgfpathlineto{\pgfqpoint{5.596999in}{0.413320in}}%
\pgfpathlineto{\pgfqpoint{5.594368in}{0.413320in}}%
\pgfpathlineto{\pgfqpoint{5.591641in}{0.413320in}}%
\pgfpathlineto{\pgfqpoint{5.589040in}{0.413320in}}%
\pgfpathlineto{\pgfqpoint{5.586269in}{0.413320in}}%
\pgfpathlineto{\pgfqpoint{5.583709in}{0.413320in}}%
\pgfpathlineto{\pgfqpoint{5.580914in}{0.413320in}}%
\pgfpathlineto{\pgfqpoint{5.578342in}{0.413320in}}%
\pgfpathlineto{\pgfqpoint{5.575596in}{0.413320in}}%
\pgfpathlineto{\pgfqpoint{5.572893in}{0.413320in}}%
\pgfpathlineto{\pgfqpoint{5.570215in}{0.413320in}}%
\pgfpathlineto{\pgfqpoint{5.567536in}{0.413320in}}%
\pgfpathlineto{\pgfqpoint{5.564940in}{0.413320in}}%
\pgfpathlineto{\pgfqpoint{5.562180in}{0.413320in}}%
\pgfpathlineto{\pgfqpoint{5.559612in}{0.413320in}}%
\pgfpathlineto{\pgfqpoint{5.556822in}{0.413320in}}%
\pgfpathlineto{\pgfqpoint{5.554198in}{0.413320in}}%
\pgfpathlineto{\pgfqpoint{5.551457in}{0.413320in}}%
\pgfpathlineto{\pgfqpoint{5.548921in}{0.413320in}}%
\pgfpathlineto{\pgfqpoint{5.546110in}{0.413320in}}%
\pgfpathlineto{\pgfqpoint{5.543421in}{0.413320in}}%
\pgfpathlineto{\pgfqpoint{5.540750in}{0.413320in}}%
\pgfpathlineto{\pgfqpoint{5.538074in}{0.413320in}}%
\pgfpathlineto{\pgfqpoint{5.535395in}{0.413320in}}%
\pgfpathlineto{\pgfqpoint{5.532717in}{0.413320in}}%
\pgfpathlineto{\pgfqpoint{5.530148in}{0.413320in}}%
\pgfpathlineto{\pgfqpoint{5.527360in}{0.413320in}}%
\pgfpathlineto{\pgfqpoint{5.524756in}{0.413320in}}%
\pgfpathlineto{\pgfqpoint{5.522003in}{0.413320in}}%
\pgfpathlineto{\pgfqpoint{5.519433in}{0.413320in}}%
\pgfpathlineto{\pgfqpoint{5.516646in}{0.413320in}}%
\pgfpathlineto{\pgfqpoint{5.514080in}{0.413320in}}%
\pgfpathlineto{\pgfqpoint{5.511290in}{0.413320in}}%
\pgfpathlineto{\pgfqpoint{5.508612in}{0.413320in}}%
\pgfpathlineto{\pgfqpoint{5.505933in}{0.413320in}}%
\pgfpathlineto{\pgfqpoint{5.503255in}{0.413320in}}%
\pgfpathlineto{\pgfqpoint{5.500687in}{0.413320in}}%
\pgfpathlineto{\pgfqpoint{5.497898in}{0.413320in}}%
\pgfpathlineto{\pgfqpoint{5.495346in}{0.413320in}}%
\pgfpathlineto{\pgfqpoint{5.492541in}{0.413320in}}%
\pgfpathlineto{\pgfqpoint{5.490000in}{0.413320in}}%
\pgfpathlineto{\pgfqpoint{5.487176in}{0.413320in}}%
\pgfpathlineto{\pgfqpoint{5.484641in}{0.413320in}}%
\pgfpathlineto{\pgfqpoint{5.481825in}{0.413320in}}%
\pgfpathlineto{\pgfqpoint{5.479152in}{0.413320in}}%
\pgfpathlineto{\pgfqpoint{5.476458in}{0.413320in}}%
\pgfpathlineto{\pgfqpoint{5.473792in}{0.413320in}}%
\pgfpathlineto{\pgfqpoint{5.471113in}{0.413320in}}%
\pgfpathlineto{\pgfqpoint{5.468425in}{0.413320in}}%
\pgfpathlineto{\pgfqpoint{5.465888in}{0.413320in}}%
\pgfpathlineto{\pgfqpoint{5.463079in}{0.413320in}}%
\pgfpathlineto{\pgfqpoint{5.460489in}{0.413320in}}%
\pgfpathlineto{\pgfqpoint{5.457721in}{0.413320in}}%
\pgfpathlineto{\pgfqpoint{5.455168in}{0.413320in}}%
\pgfpathlineto{\pgfqpoint{5.452365in}{0.413320in}}%
\pgfpathlineto{\pgfqpoint{5.449769in}{0.413320in}}%
\pgfpathlineto{\pgfqpoint{5.447021in}{0.413320in}}%
\pgfpathlineto{\pgfqpoint{5.444328in}{0.413320in}}%
\pgfpathlineto{\pgfqpoint{5.441698in}{0.413320in}}%
\pgfpathlineto{\pgfqpoint{5.438974in}{0.413320in}}%
\pgfpathlineto{\pgfqpoint{5.436295in}{0.413320in}}%
\pgfpathlineto{\pgfqpoint{5.433616in}{0.413320in}}%
\pgfpathlineto{\pgfqpoint{5.431015in}{0.413320in}}%
\pgfpathlineto{\pgfqpoint{5.428259in}{0.413320in}}%
\pgfpathlineto{\pgfqpoint{5.425661in}{0.413320in}}%
\pgfpathlineto{\pgfqpoint{5.422897in}{0.413320in}}%
\pgfpathlineto{\pgfqpoint{5.420304in}{0.413320in}}%
\pgfpathlineto{\pgfqpoint{5.417547in}{0.413320in}}%
\pgfpathlineto{\pgfqpoint{5.414954in}{0.413320in}}%
\pgfpathlineto{\pgfqpoint{5.412190in}{0.413320in}}%
\pgfpathlineto{\pgfqpoint{5.409507in}{0.413320in}}%
\pgfpathlineto{\pgfqpoint{5.406832in}{0.413320in}}%
\pgfpathlineto{\pgfqpoint{5.404154in}{0.413320in}}%
\pgfpathlineto{\pgfqpoint{5.401576in}{0.413320in}}%
\pgfpathlineto{\pgfqpoint{5.398784in}{0.413320in}}%
\pgfpathlineto{\pgfqpoint{5.396219in}{0.413320in}}%
\pgfpathlineto{\pgfqpoint{5.393441in}{0.413320in}}%
\pgfpathlineto{\pgfqpoint{5.390900in}{0.413320in}}%
\pgfpathlineto{\pgfqpoint{5.388083in}{0.413320in}}%
\pgfpathlineto{\pgfqpoint{5.385550in}{0.413320in}}%
\pgfpathlineto{\pgfqpoint{5.382725in}{0.413320in}}%
\pgfpathlineto{\pgfqpoint{5.380048in}{0.413320in}}%
\pgfpathlineto{\pgfqpoint{5.377370in}{0.413320in}}%
\pgfpathlineto{\pgfqpoint{5.374692in}{0.413320in}}%
\pgfpathlineto{\pgfqpoint{5.372013in}{0.413320in}}%
\pgfpathlineto{\pgfqpoint{5.369335in}{0.413320in}}%
\pgfpathlineto{\pgfqpoint{5.366727in}{0.413320in}}%
\pgfpathlineto{\pgfqpoint{5.363966in}{0.413320in}}%
\pgfpathlineto{\pgfqpoint{5.361370in}{0.413320in}}%
\pgfpathlineto{\pgfqpoint{5.358612in}{0.413320in}}%
\pgfpathlineto{\pgfqpoint{5.356056in}{0.413320in}}%
\pgfpathlineto{\pgfqpoint{5.353262in}{0.413320in}}%
\pgfpathlineto{\pgfqpoint{5.350723in}{0.413320in}}%
\pgfpathlineto{\pgfqpoint{5.347905in}{0.413320in}}%
\pgfpathlineto{\pgfqpoint{5.345224in}{0.413320in}}%
\pgfpathlineto{\pgfqpoint{5.342549in}{0.413320in}}%
\pgfpathlineto{\pgfqpoint{5.339872in}{0.413320in}}%
\pgfpathlineto{\pgfqpoint{5.337353in}{0.413320in}}%
\pgfpathlineto{\pgfqpoint{5.334510in}{0.413320in}}%
\pgfpathlineto{\pgfqpoint{5.331973in}{0.413320in}}%
\pgfpathlineto{\pgfqpoint{5.329159in}{0.413320in}}%
\pgfpathlineto{\pgfqpoint{5.326564in}{0.413320in}}%
\pgfpathlineto{\pgfqpoint{5.323802in}{0.413320in}}%
\pgfpathlineto{\pgfqpoint{5.321256in}{0.413320in}}%
\pgfpathlineto{\pgfqpoint{5.318430in}{0.413320in}}%
\pgfpathlineto{\pgfqpoint{5.315754in}{0.413320in}}%
\pgfpathlineto{\pgfqpoint{5.313089in}{0.413320in}}%
\pgfpathlineto{\pgfqpoint{5.310411in}{0.413320in}}%
\pgfpathlineto{\pgfqpoint{5.307731in}{0.413320in}}%
\pgfpathlineto{\pgfqpoint{5.305054in}{0.413320in}}%
\pgfpathlineto{\pgfqpoint{5.302443in}{0.413320in}}%
\pgfpathlineto{\pgfqpoint{5.299696in}{0.413320in}}%
\pgfpathlineto{\pgfqpoint{5.297140in}{0.413320in}}%
\pgfpathlineto{\pgfqpoint{5.294339in}{0.413320in}}%
\pgfpathlineto{\pgfqpoint{5.291794in}{0.413320in}}%
\pgfpathlineto{\pgfqpoint{5.288984in}{0.413320in}}%
\pgfpathlineto{\pgfqpoint{5.286436in}{0.413320in}}%
\pgfpathlineto{\pgfqpoint{5.283631in}{0.413320in}}%
\pgfpathlineto{\pgfqpoint{5.280947in}{0.413320in}}%
\pgfpathlineto{\pgfqpoint{5.278322in}{0.413320in}}%
\pgfpathlineto{\pgfqpoint{5.275589in}{0.413320in}}%
\pgfpathlineto{\pgfqpoint{5.272913in}{0.413320in}}%
\pgfpathlineto{\pgfqpoint{5.270238in}{0.413320in}}%
\pgfpathlineto{\pgfqpoint{5.267691in}{0.413320in}}%
\pgfpathlineto{\pgfqpoint{5.264876in}{0.413320in}}%
\pgfpathlineto{\pgfqpoint{5.262264in}{0.413320in}}%
\pgfpathlineto{\pgfqpoint{5.259511in}{0.413320in}}%
\pgfpathlineto{\pgfqpoint{5.256973in}{0.413320in}}%
\pgfpathlineto{\pgfqpoint{5.254236in}{0.413320in}}%
\pgfpathlineto{\pgfqpoint{5.251590in}{0.413320in}}%
\pgfpathlineto{\pgfqpoint{5.248816in}{0.413320in}}%
\pgfpathlineto{\pgfqpoint{5.246130in}{0.413320in}}%
\pgfpathlineto{\pgfqpoint{5.243445in}{0.413320in}}%
\pgfpathlineto{\pgfqpoint{5.240777in}{0.413320in}}%
\pgfpathlineto{\pgfqpoint{5.238173in}{0.413320in}}%
\pgfpathlineto{\pgfqpoint{5.235409in}{0.413320in}}%
\pgfpathlineto{\pgfqpoint{5.232855in}{0.413320in}}%
\pgfpathlineto{\pgfqpoint{5.230059in}{0.413320in}}%
\pgfpathlineto{\pgfqpoint{5.227470in}{0.413320in}}%
\pgfpathlineto{\pgfqpoint{5.224695in}{0.413320in}}%
\pgfpathlineto{\pgfqpoint{5.222151in}{0.413320in}}%
\pgfpathlineto{\pgfqpoint{5.219345in}{0.413320in}}%
\pgfpathlineto{\pgfqpoint{5.216667in}{0.413320in}}%
\pgfpathlineto{\pgfqpoint{5.214027in}{0.413320in}}%
\pgfpathlineto{\pgfqpoint{5.211299in}{0.413320in}}%
\pgfpathlineto{\pgfqpoint{5.208630in}{0.413320in}}%
\pgfpathlineto{\pgfqpoint{5.205952in}{0.413320in}}%
\pgfpathlineto{\pgfqpoint{5.203388in}{0.413320in}}%
\pgfpathlineto{\pgfqpoint{5.200594in}{0.413320in}}%
\pgfpathlineto{\pgfqpoint{5.198008in}{0.413320in}}%
\pgfpathlineto{\pgfqpoint{5.195239in}{0.413320in}}%
\pgfpathlineto{\pgfqpoint{5.192680in}{0.413320in}}%
\pgfpathlineto{\pgfqpoint{5.189880in}{0.413320in}}%
\pgfpathlineto{\pgfqpoint{5.187294in}{0.413320in}}%
\pgfpathlineto{\pgfqpoint{5.184522in}{0.413320in}}%
\pgfpathlineto{\pgfqpoint{5.181848in}{0.413320in}}%
\pgfpathlineto{\pgfqpoint{5.179188in}{0.413320in}}%
\pgfpathlineto{\pgfqpoint{5.176477in}{0.413320in}}%
\pgfpathlineto{\pgfqpoint{5.173925in}{0.413320in}}%
\pgfpathlineto{\pgfqpoint{5.171133in}{0.413320in}}%
\pgfpathlineto{\pgfqpoint{5.168591in}{0.413320in}}%
\pgfpathlineto{\pgfqpoint{5.165775in}{0.413320in}}%
\pgfpathlineto{\pgfqpoint{5.163243in}{0.413320in}}%
\pgfpathlineto{\pgfqpoint{5.160420in}{0.413320in}}%
\pgfpathlineto{\pgfqpoint{5.157815in}{0.413320in}}%
\pgfpathlineto{\pgfqpoint{5.155059in}{0.413320in}}%
\pgfpathlineto{\pgfqpoint{5.152382in}{0.413320in}}%
\pgfpathlineto{\pgfqpoint{5.149734in}{0.413320in}}%
\pgfpathlineto{\pgfqpoint{5.147029in}{0.413320in}}%
\pgfpathlineto{\pgfqpoint{5.144349in}{0.413320in}}%
\pgfpathlineto{\pgfqpoint{5.141660in}{0.413320in}}%
\pgfpathlineto{\pgfqpoint{5.139072in}{0.413320in}}%
\pgfpathlineto{\pgfqpoint{5.136311in}{0.413320in}}%
\pgfpathlineto{\pgfqpoint{5.133716in}{0.413320in}}%
\pgfpathlineto{\pgfqpoint{5.130953in}{0.413320in}}%
\pgfpathlineto{\pgfqpoint{5.128421in}{0.413320in}}%
\pgfpathlineto{\pgfqpoint{5.125599in}{0.413320in}}%
\pgfpathlineto{\pgfqpoint{5.123042in}{0.413320in}}%
\pgfpathlineto{\pgfqpoint{5.120243in}{0.413320in}}%
\pgfpathlineto{\pgfqpoint{5.117550in}{0.413320in}}%
\pgfpathlineto{\pgfqpoint{5.114887in}{0.413320in}}%
\pgfpathlineto{\pgfqpoint{5.112209in}{0.413320in}}%
\pgfpathlineto{\pgfqpoint{5.109530in}{0.413320in}}%
\pgfpathlineto{\pgfqpoint{5.106842in}{0.413320in}}%
\pgfpathlineto{\pgfqpoint{5.104312in}{0.413320in}}%
\pgfpathlineto{\pgfqpoint{5.101496in}{0.413320in}}%
\pgfpathlineto{\pgfqpoint{5.098948in}{0.413320in}}%
\pgfpathlineto{\pgfqpoint{5.096142in}{0.413320in}}%
\pgfpathlineto{\pgfqpoint{5.093579in}{0.413320in}}%
\pgfpathlineto{\pgfqpoint{5.090788in}{0.413320in}}%
\pgfpathlineto{\pgfqpoint{5.088103in}{0.413320in}}%
\pgfpathlineto{\pgfqpoint{5.085426in}{0.413320in}}%
\pgfpathlineto{\pgfqpoint{5.082746in}{0.413320in}}%
\pgfpathlineto{\pgfqpoint{5.080067in}{0.413320in}}%
\pgfpathlineto{\pgfqpoint{5.077390in}{0.413320in}}%
\pgfpathlineto{\pgfqpoint{5.074851in}{0.413320in}}%
\pgfpathlineto{\pgfqpoint{5.072030in}{0.413320in}}%
\pgfpathlineto{\pgfqpoint{5.069463in}{0.413320in}}%
\pgfpathlineto{\pgfqpoint{5.066677in}{0.413320in}}%
\pgfpathlineto{\pgfqpoint{5.064144in}{0.413320in}}%
\pgfpathlineto{\pgfqpoint{5.061315in}{0.413320in}}%
\pgfpathlineto{\pgfqpoint{5.058711in}{0.413320in}}%
\pgfpathlineto{\pgfqpoint{5.055952in}{0.413320in}}%
\pgfpathlineto{\pgfqpoint{5.053284in}{0.413320in}}%
\pgfpathlineto{\pgfqpoint{5.050606in}{0.413320in}}%
\pgfpathlineto{\pgfqpoint{5.047924in}{0.413320in}}%
\pgfpathlineto{\pgfqpoint{5.045249in}{0.413320in}}%
\pgfpathlineto{\pgfqpoint{5.042572in}{0.413320in}}%
\pgfpathlineto{\pgfqpoint{5.039962in}{0.413320in}}%
\pgfpathlineto{\pgfqpoint{5.037214in}{0.413320in}}%
\pgfpathlineto{\pgfqpoint{5.034649in}{0.413320in}}%
\pgfpathlineto{\pgfqpoint{5.031849in}{0.413320in}}%
\pgfpathlineto{\pgfqpoint{5.029275in}{0.413320in}}%
\pgfpathlineto{\pgfqpoint{5.026501in}{0.413320in}}%
\pgfpathlineto{\pgfqpoint{5.023927in}{0.413320in}}%
\pgfpathlineto{\pgfqpoint{5.021147in}{0.413320in}}%
\pgfpathlineto{\pgfqpoint{5.018466in}{0.413320in}}%
\pgfpathlineto{\pgfqpoint{5.015820in}{0.413320in}}%
\pgfpathlineto{\pgfqpoint{5.013104in}{0.413320in}}%
\pgfpathlineto{\pgfqpoint{5.010562in}{0.413320in}}%
\pgfpathlineto{\pgfqpoint{5.007751in}{0.413320in}}%
\pgfpathlineto{\pgfqpoint{5.005178in}{0.413320in}}%
\pgfpathlineto{\pgfqpoint{5.002384in}{0.413320in}}%
\pgfpathlineto{\pgfqpoint{4.999780in}{0.413320in}}%
\pgfpathlineto{\pgfqpoint{4.997028in}{0.413320in}}%
\pgfpathlineto{\pgfqpoint{4.994390in}{0.413320in}}%
\pgfpathlineto{\pgfqpoint{4.991683in}{0.413320in}}%
\pgfpathlineto{\pgfqpoint{4.989001in}{0.413320in}}%
\pgfpathlineto{\pgfqpoint{4.986325in}{0.413320in}}%
\pgfpathlineto{\pgfqpoint{4.983637in}{0.413320in}}%
\pgfpathlineto{\pgfqpoint{4.980967in}{0.413320in}}%
\pgfpathlineto{\pgfqpoint{4.978287in}{0.413320in}}%
\pgfpathlineto{\pgfqpoint{4.975703in}{0.413320in}}%
\pgfpathlineto{\pgfqpoint{4.972933in}{0.413320in}}%
\pgfpathlineto{\pgfqpoint{4.970314in}{0.413320in}}%
\pgfpathlineto{\pgfqpoint{4.967575in}{0.413320in}}%
\pgfpathlineto{\pgfqpoint{4.965002in}{0.413320in}}%
\pgfpathlineto{\pgfqpoint{4.962219in}{0.413320in}}%
\pgfpathlineto{\pgfqpoint{4.959689in}{0.413320in}}%
\pgfpathlineto{\pgfqpoint{4.956862in}{0.413320in}}%
\pgfpathlineto{\pgfqpoint{4.954182in}{0.413320in}}%
\pgfpathlineto{\pgfqpoint{4.951504in}{0.413320in}}%
\pgfpathlineto{\pgfqpoint{4.948827in}{0.413320in}}%
\pgfpathlineto{\pgfqpoint{4.946151in}{0.413320in}}%
\pgfpathlineto{\pgfqpoint{4.943466in}{0.413320in}}%
\pgfpathlineto{\pgfqpoint{4.940881in}{0.413320in}}%
\pgfpathlineto{\pgfqpoint{4.938112in}{0.413320in}}%
\pgfpathlineto{\pgfqpoint{4.935515in}{0.413320in}}%
\pgfpathlineto{\pgfqpoint{4.932742in}{0.413320in}}%
\pgfpathlineto{\pgfqpoint{4.930170in}{0.413320in}}%
\pgfpathlineto{\pgfqpoint{4.927400in}{0.413320in}}%
\pgfpathlineto{\pgfqpoint{4.924708in}{0.413320in}}%
\pgfpathlineto{\pgfqpoint{4.922041in}{0.413320in}}%
\pgfpathlineto{\pgfqpoint{4.919352in}{0.413320in}}%
\pgfpathlineto{\pgfqpoint{4.916681in}{0.413320in}}%
\pgfpathlineto{\pgfqpoint{4.914009in}{0.413320in}}%
\pgfpathlineto{\pgfqpoint{4.911435in}{0.413320in}}%
\pgfpathlineto{\pgfqpoint{4.908648in}{0.413320in}}%
\pgfpathlineto{\pgfqpoint{4.906096in}{0.413320in}}%
\pgfpathlineto{\pgfqpoint{4.903295in}{0.413320in}}%
\pgfpathlineto{\pgfqpoint{4.900712in}{0.413320in}}%
\pgfpathlineto{\pgfqpoint{4.897938in}{0.413320in}}%
\pgfpathlineto{\pgfqpoint{4.895399in}{0.413320in}}%
\pgfpathlineto{\pgfqpoint{4.892611in}{0.413320in}}%
\pgfpathlineto{\pgfqpoint{4.889902in}{0.413320in}}%
\pgfpathlineto{\pgfqpoint{4.887211in}{0.413320in}}%
\pgfpathlineto{\pgfqpoint{4.884540in}{0.413320in}}%
\pgfpathlineto{\pgfqpoint{4.881864in}{0.413320in}}%
\pgfpathlineto{\pgfqpoint{4.879180in}{0.413320in}}%
\pgfpathlineto{\pgfqpoint{4.876636in}{0.413320in}}%
\pgfpathlineto{\pgfqpoint{4.873832in}{0.413320in}}%
\pgfpathlineto{\pgfqpoint{4.871209in}{0.413320in}}%
\pgfpathlineto{\pgfqpoint{4.868474in}{0.413320in}}%
\pgfpathlineto{\pgfqpoint{4.865910in}{0.413320in}}%
\pgfpathlineto{\pgfqpoint{4.863116in}{0.413320in}}%
\pgfpathlineto{\pgfqpoint{4.860544in}{0.413320in}}%
\pgfpathlineto{\pgfqpoint{4.857807in}{0.413320in}}%
\pgfpathlineto{\pgfqpoint{4.855070in}{0.413320in}}%
\pgfpathlineto{\pgfqpoint{4.852404in}{0.413320in}}%
\pgfpathlineto{\pgfqpoint{4.849715in}{0.413320in}}%
\pgfpathlineto{\pgfqpoint{4.847127in}{0.413320in}}%
\pgfpathlineto{\pgfqpoint{4.844361in}{0.413320in}}%
\pgfpathlineto{\pgfqpoint{4.842380in}{0.413320in}}%
\pgfpathlineto{\pgfqpoint{4.839922in}{0.413320in}}%
\pgfpathlineto{\pgfqpoint{4.837992in}{0.413320in}}%
\pgfpathlineto{\pgfqpoint{4.833657in}{0.413320in}}%
\pgfpathlineto{\pgfqpoint{4.831045in}{0.413320in}}%
\pgfpathlineto{\pgfqpoint{4.828291in}{0.413320in}}%
\pgfpathlineto{\pgfqpoint{4.825619in}{0.413320in}}%
\pgfpathlineto{\pgfqpoint{4.822945in}{0.413320in}}%
\pgfpathlineto{\pgfqpoint{4.820265in}{0.413320in}}%
\pgfpathlineto{\pgfqpoint{4.817587in}{0.413320in}}%
\pgfpathlineto{\pgfqpoint{4.814907in}{0.413320in}}%
\pgfpathlineto{\pgfqpoint{4.812377in}{0.413320in}}%
\pgfpathlineto{\pgfqpoint{4.809538in}{0.413320in}}%
\pgfpathlineto{\pgfqpoint{4.807017in}{0.413320in}}%
\pgfpathlineto{\pgfqpoint{4.804193in}{0.413320in}}%
\pgfpathlineto{\pgfqpoint{4.801586in}{0.413320in}}%
\pgfpathlineto{\pgfqpoint{4.798830in}{0.413320in}}%
\pgfpathlineto{\pgfqpoint{4.796274in}{0.413320in}}%
\pgfpathlineto{\pgfqpoint{4.793512in}{0.413320in}}%
\pgfpathlineto{\pgfqpoint{4.790798in}{0.413320in}}%
\pgfpathlineto{\pgfqpoint{4.788116in}{0.413320in}}%
\pgfpathlineto{\pgfqpoint{4.785445in}{0.413320in}}%
\pgfpathlineto{\pgfqpoint{4.782872in}{0.413320in}}%
\pgfpathlineto{\pgfqpoint{4.780083in}{0.413320in}}%
\pgfpathlineto{\pgfqpoint{4.777535in}{0.413320in}}%
\pgfpathlineto{\pgfqpoint{4.774732in}{0.413320in}}%
\pgfpathlineto{\pgfqpoint{4.772198in}{0.413320in}}%
\pgfpathlineto{\pgfqpoint{4.769367in}{0.413320in}}%
\pgfpathlineto{\pgfqpoint{4.766783in}{0.413320in}}%
\pgfpathlineto{\pgfqpoint{4.764018in}{0.413320in}}%
\pgfpathlineto{\pgfqpoint{4.761337in}{0.413320in}}%
\pgfpathlineto{\pgfqpoint{4.758653in}{0.413320in}}%
\pgfpathlineto{\pgfqpoint{4.755983in}{0.413320in}}%
\pgfpathlineto{\pgfqpoint{4.753298in}{0.413320in}}%
\pgfpathlineto{\pgfqpoint{4.750627in}{0.413320in}}%
\pgfpathlineto{\pgfqpoint{4.748081in}{0.413320in}}%
\pgfpathlineto{\pgfqpoint{4.745256in}{0.413320in}}%
\pgfpathlineto{\pgfqpoint{4.742696in}{0.413320in}}%
\pgfpathlineto{\pgfqpoint{4.739912in}{0.413320in}}%
\pgfpathlineto{\pgfqpoint{4.737348in}{0.413320in}}%
\pgfpathlineto{\pgfqpoint{4.734552in}{0.413320in}}%
\pgfpathlineto{\pgfqpoint{4.731901in}{0.413320in}}%
\pgfpathlineto{\pgfqpoint{4.729233in}{0.413320in}}%
\pgfpathlineto{\pgfqpoint{4.726508in}{0.413320in}}%
\pgfpathlineto{\pgfqpoint{4.723873in}{0.413320in}}%
\pgfpathlineto{\pgfqpoint{4.721160in}{0.413320in}}%
\pgfpathlineto{\pgfqpoint{4.718486in}{0.413320in}}%
\pgfpathlineto{\pgfqpoint{4.715806in}{0.413320in}}%
\pgfpathlineto{\pgfqpoint{4.713275in}{0.413320in}}%
\pgfpathlineto{\pgfqpoint{4.710437in}{0.413320in}}%
\pgfpathlineto{\pgfqpoint{4.707824in}{0.413320in}}%
\pgfpathlineto{\pgfqpoint{4.705094in}{0.413320in}}%
\pgfpathlineto{\pgfqpoint{4.702517in}{0.413320in}}%
\pgfpathlineto{\pgfqpoint{4.699734in}{0.413320in}}%
\pgfpathlineto{\pgfqpoint{4.697170in}{0.413320in}}%
\pgfpathlineto{\pgfqpoint{4.694381in}{0.413320in}}%
\pgfpathlineto{\pgfqpoint{4.691694in}{0.413320in}}%
\pgfpathlineto{\pgfqpoint{4.689051in}{0.413320in}}%
\pgfpathlineto{\pgfqpoint{4.686337in}{0.413320in}}%
\pgfpathlineto{\pgfqpoint{4.683799in}{0.413320in}}%
\pgfpathlineto{\pgfqpoint{4.680988in}{0.413320in}}%
\pgfpathlineto{\pgfqpoint{4.678448in}{0.413320in}}%
\pgfpathlineto{\pgfqpoint{4.675619in}{0.413320in}}%
\pgfpathlineto{\pgfqpoint{4.673068in}{0.413320in}}%
\pgfpathlineto{\pgfqpoint{4.670261in}{0.413320in}}%
\pgfpathlineto{\pgfqpoint{4.667764in}{0.413320in}}%
\pgfpathlineto{\pgfqpoint{4.664923in}{0.413320in}}%
\pgfpathlineto{\pgfqpoint{4.662237in}{0.413320in}}%
\pgfpathlineto{\pgfqpoint{4.659590in}{0.413320in}}%
\pgfpathlineto{\pgfqpoint{4.656873in}{0.413320in}}%
\pgfpathlineto{\pgfqpoint{4.654203in}{0.413320in}}%
\pgfpathlineto{\pgfqpoint{4.651524in}{0.413320in}}%
\pgfpathlineto{\pgfqpoint{4.648922in}{0.413320in}}%
\pgfpathlineto{\pgfqpoint{4.646169in}{0.413320in}}%
\pgfpathlineto{\pgfqpoint{4.643628in}{0.413320in}}%
\pgfpathlineto{\pgfqpoint{4.640809in}{0.413320in}}%
\pgfpathlineto{\pgfqpoint{4.638204in}{0.413320in}}%
\pgfpathlineto{\pgfqpoint{4.635445in}{0.413320in}}%
\pgfpathlineto{\pgfqpoint{4.632902in}{0.413320in}}%
\pgfpathlineto{\pgfqpoint{4.630096in}{0.413320in}}%
\pgfpathlineto{\pgfqpoint{4.627411in}{0.413320in}}%
\pgfpathlineto{\pgfqpoint{4.624741in}{0.413320in}}%
\pgfpathlineto{\pgfqpoint{4.622056in}{0.413320in}}%
\pgfpathlineto{\pgfqpoint{4.619529in}{0.413320in}}%
\pgfpathlineto{\pgfqpoint{4.616702in}{0.413320in}}%
\pgfpathlineto{\pgfqpoint{4.614134in}{0.413320in}}%
\pgfpathlineto{\pgfqpoint{4.611350in}{0.413320in}}%
\pgfpathlineto{\pgfqpoint{4.608808in}{0.413320in}}%
\pgfpathlineto{\pgfqpoint{4.605990in}{0.413320in}}%
\pgfpathlineto{\pgfqpoint{4.603430in}{0.413320in}}%
\pgfpathlineto{\pgfqpoint{4.600633in}{0.413320in}}%
\pgfpathlineto{\pgfqpoint{4.597951in}{0.413320in}}%
\pgfpathlineto{\pgfqpoint{4.595281in}{0.413320in}}%
\pgfpathlineto{\pgfqpoint{4.592589in}{0.413320in}}%
\pgfpathlineto{\pgfqpoint{4.589920in}{0.413320in}}%
\pgfpathlineto{\pgfqpoint{4.587244in}{0.413320in}}%
\pgfpathlineto{\pgfqpoint{4.584672in}{0.413320in}}%
\pgfpathlineto{\pgfqpoint{4.581888in}{0.413320in}}%
\pgfpathlineto{\pgfqpoint{4.579305in}{0.413320in}}%
\pgfpathlineto{\pgfqpoint{4.576531in}{0.413320in}}%
\pgfpathlineto{\pgfqpoint{4.573947in}{0.413320in}}%
\pgfpathlineto{\pgfqpoint{4.571171in}{0.413320in}}%
\pgfpathlineto{\pgfqpoint{4.568612in}{0.413320in}}%
\pgfpathlineto{\pgfqpoint{4.565820in}{0.413320in}}%
\pgfpathlineto{\pgfqpoint{4.563125in}{0.413320in}}%
\pgfpathlineto{\pgfqpoint{4.560448in}{0.413320in}}%
\pgfpathlineto{\pgfqpoint{4.557777in}{0.413320in}}%
\pgfpathlineto{\pgfqpoint{4.555106in}{0.413320in}}%
\pgfpathlineto{\pgfqpoint{4.552425in}{0.413320in}}%
\pgfpathlineto{\pgfqpoint{4.549822in}{0.413320in}}%
\pgfpathlineto{\pgfqpoint{4.547064in}{0.413320in}}%
\pgfpathlineto{\pgfqpoint{4.544464in}{0.413320in}}%
\pgfpathlineto{\pgfqpoint{4.541711in}{0.413320in}}%
\pgfpathlineto{\pgfqpoint{4.539144in}{0.413320in}}%
\pgfpathlineto{\pgfqpoint{4.536400in}{0.413320in}}%
\pgfpathlineto{\pgfqpoint{4.533764in}{0.413320in}}%
\pgfpathlineto{\pgfqpoint{4.530990in}{0.413320in}}%
\pgfpathlineto{\pgfqpoint{4.528307in}{0.413320in}}%
\pgfpathlineto{\pgfqpoint{4.525640in}{0.413320in}}%
\pgfpathlineto{\pgfqpoint{4.522962in}{0.413320in}}%
\pgfpathlineto{\pgfqpoint{4.520345in}{0.413320in}}%
\pgfpathlineto{\pgfqpoint{4.517598in}{0.413320in}}%
\pgfpathlineto{\pgfqpoint{4.515080in}{0.413320in}}%
\pgfpathlineto{\pgfqpoint{4.512246in}{0.413320in}}%
\pgfpathlineto{\pgfqpoint{4.509643in}{0.413320in}}%
\pgfpathlineto{\pgfqpoint{4.506893in}{0.413320in}}%
\pgfpathlineto{\pgfqpoint{4.504305in}{0.413320in}}%
\pgfpathlineto{\pgfqpoint{4.501529in}{0.413320in}}%
\pgfpathlineto{\pgfqpoint{4.498850in}{0.413320in}}%
\pgfpathlineto{\pgfqpoint{4.496167in}{0.413320in}}%
\pgfpathlineto{\pgfqpoint{4.493492in}{0.413320in}}%
\pgfpathlineto{\pgfqpoint{4.490822in}{0.413320in}}%
\pgfpathlineto{\pgfqpoint{4.488130in}{0.413320in}}%
\pgfpathlineto{\pgfqpoint{4.485581in}{0.413320in}}%
\pgfpathlineto{\pgfqpoint{4.482778in}{0.413320in}}%
\pgfpathlineto{\pgfqpoint{4.480201in}{0.413320in}}%
\pgfpathlineto{\pgfqpoint{4.477430in}{0.413320in}}%
\pgfpathlineto{\pgfqpoint{4.474861in}{0.413320in}}%
\pgfpathlineto{\pgfqpoint{4.472059in}{0.413320in}}%
\pgfpathlineto{\pgfqpoint{4.469492in}{0.413320in}}%
\pgfpathlineto{\pgfqpoint{4.466717in}{0.413320in}}%
\pgfpathlineto{\pgfqpoint{4.464029in}{0.413320in}}%
\pgfpathlineto{\pgfqpoint{4.461367in}{0.413320in}}%
\pgfpathlineto{\pgfqpoint{4.458681in}{0.413320in}}%
\pgfpathlineto{\pgfqpoint{4.456138in}{0.413320in}}%
\pgfpathlineto{\pgfqpoint{4.453312in}{0.413320in}}%
\pgfpathlineto{\pgfqpoint{4.450767in}{0.413320in}}%
\pgfpathlineto{\pgfqpoint{4.447965in}{0.413320in}}%
\pgfpathlineto{\pgfqpoint{4.445423in}{0.413320in}}%
\pgfpathlineto{\pgfqpoint{4.442611in}{0.413320in}}%
\pgfpathlineto{\pgfqpoint{4.440041in}{0.413320in}}%
\pgfpathlineto{\pgfqpoint{4.437253in}{0.413320in}}%
\pgfpathlineto{\pgfqpoint{4.434569in}{0.413320in}}%
\pgfpathlineto{\pgfqpoint{4.431901in}{0.413320in}}%
\pgfpathlineto{\pgfqpoint{4.429220in}{0.413320in}}%
\pgfpathlineto{\pgfqpoint{4.426534in}{0.413320in}}%
\pgfpathlineto{\pgfqpoint{4.423863in}{0.413320in}}%
\pgfpathlineto{\pgfqpoint{4.421292in}{0.413320in}}%
\pgfpathlineto{\pgfqpoint{4.418506in}{0.413320in}}%
\pgfpathlineto{\pgfqpoint{4.415932in}{0.413320in}}%
\pgfpathlineto{\pgfqpoint{4.413149in}{0.413320in}}%
\pgfpathlineto{\pgfqpoint{4.410587in}{0.413320in}}%
\pgfpathlineto{\pgfqpoint{4.407788in}{0.413320in}}%
\pgfpathlineto{\pgfqpoint{4.405234in}{0.413320in}}%
\pgfpathlineto{\pgfqpoint{4.402468in}{0.413320in}}%
\pgfpathlineto{\pgfqpoint{4.399745in}{0.413320in}}%
\pgfpathlineto{\pgfqpoint{4.397076in}{0.413320in}}%
\pgfpathlineto{\pgfqpoint{4.394400in}{0.413320in}}%
\pgfpathlineto{\pgfqpoint{4.391721in}{0.413320in}}%
\pgfpathlineto{\pgfqpoint{4.389044in}{0.413320in}}%
\pgfpathlineto{\pgfqpoint{4.386431in}{0.413320in}}%
\pgfpathlineto{\pgfqpoint{4.383674in}{0.413320in}}%
\pgfpathlineto{\pgfqpoint{4.381097in}{0.413320in}}%
\pgfpathlineto{\pgfqpoint{4.378329in}{0.413320in}}%
\pgfpathlineto{\pgfqpoint{4.375761in}{0.413320in}}%
\pgfpathlineto{\pgfqpoint{4.372976in}{0.413320in}}%
\pgfpathlineto{\pgfqpoint{4.370437in}{0.413320in}}%
\pgfpathlineto{\pgfqpoint{4.367646in}{0.413320in}}%
\pgfpathlineto{\pgfqpoint{4.364936in}{0.413320in}}%
\pgfpathlineto{\pgfqpoint{4.362270in}{0.413320in}}%
\pgfpathlineto{\pgfqpoint{4.359582in}{0.413320in}}%
\pgfpathlineto{\pgfqpoint{4.357014in}{0.413320in}}%
\pgfpathlineto{\pgfqpoint{4.354224in}{0.413320in}}%
\pgfpathlineto{\pgfqpoint{4.351645in}{0.413320in}}%
\pgfpathlineto{\pgfqpoint{4.348868in}{0.413320in}}%
\pgfpathlineto{\pgfqpoint{4.346263in}{0.413320in}}%
\pgfpathlineto{\pgfqpoint{4.343510in}{0.413320in}}%
\pgfpathlineto{\pgfqpoint{4.340976in}{0.413320in}}%
\pgfpathlineto{\pgfqpoint{4.338154in}{0.413320in}}%
\pgfpathlineto{\pgfqpoint{4.335463in}{0.413320in}}%
\pgfpathlineto{\pgfqpoint{4.332796in}{0.413320in}}%
\pgfpathlineto{\pgfqpoint{4.330118in}{0.413320in}}%
\pgfpathlineto{\pgfqpoint{4.327440in}{0.413320in}}%
\pgfpathlineto{\pgfqpoint{4.324760in}{0.413320in}}%
\pgfpathlineto{\pgfqpoint{4.322181in}{0.413320in}}%
\pgfpathlineto{\pgfqpoint{4.319405in}{0.413320in}}%
\pgfpathlineto{\pgfqpoint{4.316856in}{0.413320in}}%
\pgfpathlineto{\pgfqpoint{4.314032in}{0.413320in}}%
\pgfpathlineto{\pgfqpoint{4.311494in}{0.413320in}}%
\pgfpathlineto{\pgfqpoint{4.308691in}{0.413320in}}%
\pgfpathlineto{\pgfqpoint{4.306118in}{0.413320in}}%
\pgfpathlineto{\pgfqpoint{4.303357in}{0.413320in}}%
\pgfpathlineto{\pgfqpoint{4.300656in}{0.413320in}}%
\pgfpathlineto{\pgfqpoint{4.297977in}{0.413320in}}%
\pgfpathlineto{\pgfqpoint{4.295299in}{0.413320in}}%
\pgfpathlineto{\pgfqpoint{4.292786in}{0.413320in}}%
\pgfpathlineto{\pgfqpoint{4.289936in}{0.413320in}}%
\pgfpathlineto{\pgfqpoint{4.287399in}{0.413320in}}%
\pgfpathlineto{\pgfqpoint{4.284586in}{0.413320in}}%
\pgfpathlineto{\pgfqpoint{4.282000in}{0.413320in}}%
\pgfpathlineto{\pgfqpoint{4.279212in}{0.413320in}}%
\pgfpathlineto{\pgfqpoint{4.276635in}{0.413320in}}%
\pgfpathlineto{\pgfqpoint{4.273874in}{0.413320in}}%
\pgfpathlineto{\pgfqpoint{4.271187in}{0.413320in}}%
\pgfpathlineto{\pgfqpoint{4.268590in}{0.413320in}}%
\pgfpathlineto{\pgfqpoint{4.265824in}{0.413320in}}%
\pgfpathlineto{\pgfqpoint{4.263157in}{0.413320in}}%
\pgfpathlineto{\pgfqpoint{4.260477in}{0.413320in}}%
\pgfpathlineto{\pgfqpoint{4.257958in}{0.413320in}}%
\pgfpathlineto{\pgfqpoint{4.255120in}{0.413320in}}%
\pgfpathlineto{\pgfqpoint{4.252581in}{0.413320in}}%
\pgfpathlineto{\pgfqpoint{4.249767in}{0.413320in}}%
\pgfpathlineto{\pgfqpoint{4.247225in}{0.413320in}}%
\pgfpathlineto{\pgfqpoint{4.244394in}{0.413320in}}%
\pgfpathlineto{\pgfqpoint{4.241900in}{0.413320in}}%
\pgfpathlineto{\pgfqpoint{4.239084in}{0.413320in}}%
\pgfpathlineto{\pgfqpoint{4.236375in}{0.413320in}}%
\pgfpathlineto{\pgfqpoint{4.233691in}{0.413320in}}%
\pgfpathlineto{\pgfqpoint{4.231013in}{0.413320in}}%
\pgfpathlineto{\pgfqpoint{4.228331in}{0.413320in}}%
\pgfpathlineto{\pgfqpoint{4.225654in}{0.413320in}}%
\pgfpathlineto{\pgfqpoint{4.223082in}{0.413320in}}%
\pgfpathlineto{\pgfqpoint{4.220304in}{0.413320in}}%
\pgfpathlineto{\pgfqpoint{4.217694in}{0.413320in}}%
\pgfpathlineto{\pgfqpoint{4.214948in}{0.413320in}}%
\pgfpathlineto{\pgfqpoint{4.212383in}{0.413320in}}%
\pgfpathlineto{\pgfqpoint{4.209597in}{0.413320in}}%
\pgfpathlineto{\pgfqpoint{4.207076in}{0.413320in}}%
\pgfpathlineto{\pgfqpoint{4.204240in}{0.413320in}}%
\pgfpathlineto{\pgfqpoint{4.201542in}{0.413320in}}%
\pgfpathlineto{\pgfqpoint{4.198878in}{0.413320in}}%
\pgfpathlineto{\pgfqpoint{4.196186in}{0.413320in}}%
\pgfpathlineto{\pgfqpoint{4.193638in}{0.413320in}}%
\pgfpathlineto{\pgfqpoint{4.190842in}{0.413320in}}%
\pgfpathlineto{\pgfqpoint{4.188318in}{0.413320in}}%
\pgfpathlineto{\pgfqpoint{4.185481in}{0.413320in}}%
\pgfpathlineto{\pgfqpoint{4.182899in}{0.413320in}}%
\pgfpathlineto{\pgfqpoint{4.180129in}{0.413320in}}%
\pgfpathlineto{\pgfqpoint{4.177593in}{0.413320in}}%
\pgfpathlineto{\pgfqpoint{4.174770in}{0.413320in}}%
\pgfpathlineto{\pgfqpoint{4.172093in}{0.413320in}}%
\pgfpathlineto{\pgfqpoint{4.169415in}{0.413320in}}%
\pgfpathlineto{\pgfqpoint{4.166737in}{0.413320in}}%
\pgfpathlineto{\pgfqpoint{4.164059in}{0.413320in}}%
\pgfpathlineto{\pgfqpoint{4.161380in}{0.413320in}}%
\pgfpathlineto{\pgfqpoint{4.158806in}{0.413320in}}%
\pgfpathlineto{\pgfqpoint{4.156016in}{0.413320in}}%
\pgfpathlineto{\pgfqpoint{4.153423in}{0.413320in}}%
\pgfpathlineto{\pgfqpoint{4.150665in}{0.413320in}}%
\pgfpathlineto{\pgfqpoint{4.148082in}{0.413320in}}%
\pgfpathlineto{\pgfqpoint{4.145310in}{0.413320in}}%
\pgfpathlineto{\pgfqpoint{4.142713in}{0.413320in}}%
\pgfpathlineto{\pgfqpoint{4.139963in}{0.413320in}}%
\pgfpathlineto{\pgfqpoint{4.137272in}{0.413320in}}%
\pgfpathlineto{\pgfqpoint{4.134615in}{0.413320in}}%
\pgfpathlineto{\pgfqpoint{4.131920in}{0.413320in}}%
\pgfpathlineto{\pgfqpoint{4.129349in}{0.413320in}}%
\pgfpathlineto{\pgfqpoint{4.126553in}{0.413320in}}%
\pgfpathlineto{\pgfqpoint{4.124019in}{0.413320in}}%
\pgfpathlineto{\pgfqpoint{4.121205in}{0.413320in}}%
\pgfpathlineto{\pgfqpoint{4.118554in}{0.413320in}}%
\pgfpathlineto{\pgfqpoint{4.115844in}{0.413320in}}%
\pgfpathlineto{\pgfqpoint{4.113252in}{0.413320in}}%
\pgfpathlineto{\pgfqpoint{4.110488in}{0.413320in}}%
\pgfpathlineto{\pgfqpoint{4.107814in}{0.413320in}}%
\pgfpathlineto{\pgfqpoint{4.105185in}{0.413320in}}%
\pgfpathlineto{\pgfqpoint{4.102456in}{0.413320in}}%
\pgfpathlineto{\pgfqpoint{4.099777in}{0.413320in}}%
\pgfpathlineto{\pgfqpoint{4.097092in}{0.413320in}}%
\pgfpathlineto{\pgfqpoint{4.094527in}{0.413320in}}%
\pgfpathlineto{\pgfqpoint{4.091729in}{0.413320in}}%
\pgfpathlineto{\pgfqpoint{4.089159in}{0.413320in}}%
\pgfpathlineto{\pgfqpoint{4.086385in}{0.413320in}}%
\pgfpathlineto{\pgfqpoint{4.083870in}{0.413320in}}%
\pgfpathlineto{\pgfqpoint{4.081018in}{0.413320in}}%
\pgfpathlineto{\pgfqpoint{4.078471in}{0.413320in}}%
\pgfpathlineto{\pgfqpoint{4.075705in}{0.413320in}}%
\pgfpathlineto{\pgfqpoint{4.072985in}{0.413320in}}%
\pgfpathlineto{\pgfqpoint{4.070313in}{0.413320in}}%
\pgfpathlineto{\pgfqpoint{4.067636in}{0.413320in}}%
\pgfpathlineto{\pgfqpoint{4.064957in}{0.413320in}}%
\pgfpathlineto{\pgfqpoint{4.062266in}{0.413320in}}%
\pgfpathlineto{\pgfqpoint{4.059702in}{0.413320in}}%
\pgfpathlineto{\pgfqpoint{4.056911in}{0.413320in}}%
\pgfpathlineto{\pgfqpoint{4.054326in}{0.413320in}}%
\pgfpathlineto{\pgfqpoint{4.051557in}{0.413320in}}%
\pgfpathlineto{\pgfqpoint{4.049006in}{0.413320in}}%
\pgfpathlineto{\pgfqpoint{4.046210in}{0.413320in}}%
\pgfpathlineto{\pgfqpoint{4.043667in}{0.413320in}}%
\pgfpathlineto{\pgfqpoint{4.040852in}{0.413320in}}%
\pgfpathlineto{\pgfqpoint{4.038174in}{0.413320in}}%
\pgfpathlineto{\pgfqpoint{4.035492in}{0.413320in}}%
\pgfpathlineto{\pgfqpoint{4.032817in}{0.413320in}}%
\pgfpathlineto{\pgfqpoint{4.030229in}{0.413320in}}%
\pgfpathlineto{\pgfqpoint{4.027447in}{0.413320in}}%
\pgfpathlineto{\pgfqpoint{4.024868in}{0.413320in}}%
\pgfpathlineto{\pgfqpoint{4.022097in}{0.413320in}}%
\pgfpathlineto{\pgfqpoint{4.019518in}{0.413320in}}%
\pgfpathlineto{\pgfqpoint{4.016744in}{0.413320in}}%
\pgfpathlineto{\pgfqpoint{4.014186in}{0.413320in}}%
\pgfpathlineto{\pgfqpoint{4.011394in}{0.413320in}}%
\pgfpathlineto{\pgfqpoint{4.008699in}{0.413320in}}%
\pgfpathlineto{\pgfqpoint{4.006034in}{0.413320in}}%
\pgfpathlineto{\pgfqpoint{4.003348in}{0.413320in}}%
\pgfpathlineto{\pgfqpoint{4.000674in}{0.413320in}}%
\pgfpathlineto{\pgfqpoint{3.997990in}{0.413320in}}%
\pgfpathlineto{\pgfqpoint{3.995417in}{0.413320in}}%
\pgfpathlineto{\pgfqpoint{3.992642in}{0.413320in}}%
\pgfpathlineto{\pgfqpoint{3.990055in}{0.413320in}}%
\pgfpathlineto{\pgfqpoint{3.987270in}{0.413320in}}%
\pgfpathlineto{\pgfqpoint{3.984714in}{0.413320in}}%
\pgfpathlineto{\pgfqpoint{3.981929in}{0.413320in}}%
\pgfpathlineto{\pgfqpoint{3.979389in}{0.413320in}}%
\pgfpathlineto{\pgfqpoint{3.976563in}{0.413320in}}%
\pgfpathlineto{\pgfqpoint{3.973885in}{0.413320in}}%
\pgfpathlineto{\pgfqpoint{3.971250in}{0.413320in}}%
\pgfpathlineto{\pgfqpoint{3.968523in}{0.413320in}}%
\pgfpathlineto{\pgfqpoint{3.966013in}{0.413320in}}%
\pgfpathlineto{\pgfqpoint{3.963176in}{0.413320in}}%
\pgfpathlineto{\pgfqpoint{3.960635in}{0.413320in}}%
\pgfpathlineto{\pgfqpoint{3.957823in}{0.413320in}}%
\pgfpathlineto{\pgfqpoint{3.955211in}{0.413320in}}%
\pgfpathlineto{\pgfqpoint{3.952464in}{0.413320in}}%
\pgfpathlineto{\pgfqpoint{3.949894in}{0.413320in}}%
\pgfpathlineto{\pgfqpoint{3.947101in}{0.413320in}}%
\pgfpathlineto{\pgfqpoint{3.944431in}{0.413320in}}%
\pgfpathlineto{\pgfqpoint{3.941778in}{0.413320in}}%
\pgfpathlineto{\pgfqpoint{3.939075in}{0.413320in}}%
\pgfpathlineto{\pgfqpoint{3.936395in}{0.413320in}}%
\pgfpathlineto{\pgfqpoint{3.933714in}{0.413320in}}%
\pgfpathlineto{\pgfqpoint{3.931202in}{0.413320in}}%
\pgfpathlineto{\pgfqpoint{3.928347in}{0.413320in}}%
\pgfpathlineto{\pgfqpoint{3.925778in}{0.413320in}}%
\pgfpathlineto{\pgfqpoint{3.923005in}{0.413320in}}%
\pgfpathlineto{\pgfqpoint{3.920412in}{0.413320in}}%
\pgfpathlineto{\pgfqpoint{3.917646in}{0.413320in}}%
\pgfpathlineto{\pgfqpoint{3.915107in}{0.413320in}}%
\pgfpathlineto{\pgfqpoint{3.912296in}{0.413320in}}%
\pgfpathlineto{\pgfqpoint{3.909602in}{0.413320in}}%
\pgfpathlineto{\pgfqpoint{3.906918in}{0.413320in}}%
\pgfpathlineto{\pgfqpoint{3.904252in}{0.413320in}}%
\pgfpathlineto{\pgfqpoint{3.901573in}{0.413320in}}%
\pgfpathlineto{\pgfqpoint{3.898891in}{0.413320in}}%
\pgfpathlineto{\pgfqpoint{3.896345in}{0.413320in}}%
\pgfpathlineto{\pgfqpoint{3.893541in}{0.413320in}}%
\pgfpathlineto{\pgfqpoint{3.890926in}{0.413320in}}%
\pgfpathlineto{\pgfqpoint{3.888188in}{0.413320in}}%
\pgfpathlineto{\pgfqpoint{3.885621in}{0.413320in}}%
\pgfpathlineto{\pgfqpoint{3.882850in}{0.413320in}}%
\pgfpathlineto{\pgfqpoint{3.880237in}{0.413320in}}%
\pgfpathlineto{\pgfqpoint{3.877466in}{0.413320in}}%
\pgfpathlineto{\pgfqpoint{3.874790in}{0.413320in}}%
\pgfpathlineto{\pgfqpoint{3.872114in}{0.413320in}}%
\pgfpathlineto{\pgfqpoint{3.869435in}{0.413320in}}%
\pgfpathlineto{\pgfqpoint{3.866815in}{0.413320in}}%
\pgfpathlineto{\pgfqpoint{3.864073in}{0.413320in}}%
\pgfpathlineto{\pgfqpoint{3.861561in}{0.413320in}}%
\pgfpathlineto{\pgfqpoint{3.858720in}{0.413320in}}%
\pgfpathlineto{\pgfqpoint{3.856100in}{0.413320in}}%
\pgfpathlineto{\pgfqpoint{3.853358in}{0.413320in}}%
\pgfpathlineto{\pgfqpoint{3.850814in}{0.413320in}}%
\pgfpathlineto{\pgfqpoint{3.848005in}{0.413320in}}%
\pgfpathlineto{\pgfqpoint{3.845329in}{0.413320in}}%
\pgfpathlineto{\pgfqpoint{3.842641in}{0.413320in}}%
\pgfpathlineto{\pgfqpoint{3.839960in}{0.413320in}}%
\pgfpathlineto{\pgfqpoint{3.837286in}{0.413320in}}%
\pgfpathlineto{\pgfqpoint{3.834616in}{0.413320in}}%
\pgfpathlineto{\pgfqpoint{3.832053in}{0.413320in}}%
\pgfpathlineto{\pgfqpoint{3.829252in}{0.413320in}}%
\pgfpathlineto{\pgfqpoint{3.826679in}{0.413320in}}%
\pgfpathlineto{\pgfqpoint{3.823903in}{0.413320in}}%
\pgfpathlineto{\pgfqpoint{3.821315in}{0.413320in}}%
\pgfpathlineto{\pgfqpoint{3.818546in}{0.413320in}}%
\pgfpathlineto{\pgfqpoint{3.815983in}{0.413320in}}%
\pgfpathlineto{\pgfqpoint{3.813172in}{0.413320in}}%
\pgfpathlineto{\pgfqpoint{3.810510in}{0.413320in}}%
\pgfpathlineto{\pgfqpoint{3.807832in}{0.413320in}}%
\pgfpathlineto{\pgfqpoint{3.805145in}{0.413320in}}%
\pgfpathlineto{\pgfqpoint{3.802569in}{0.413320in}}%
\pgfpathlineto{\pgfqpoint{3.799797in}{0.413320in}}%
\pgfpathlineto{\pgfqpoint{3.797265in}{0.413320in}}%
\pgfpathlineto{\pgfqpoint{3.794435in}{0.413320in}}%
\pgfpathlineto{\pgfqpoint{3.791897in}{0.413320in}}%
\pgfpathlineto{\pgfqpoint{3.789084in}{0.413320in}}%
\pgfpathlineto{\pgfqpoint{3.786504in}{0.413320in}}%
\pgfpathlineto{\pgfqpoint{3.783725in}{0.413320in}}%
\pgfpathlineto{\pgfqpoint{3.781046in}{0.413320in}}%
\pgfpathlineto{\pgfqpoint{3.778370in}{0.413320in}}%
\pgfpathlineto{\pgfqpoint{3.775691in}{0.413320in}}%
\pgfpathlineto{\pgfqpoint{3.773014in}{0.413320in}}%
\pgfpathlineto{\pgfqpoint{3.770323in}{0.413320in}}%
\pgfpathlineto{\pgfqpoint{3.767782in}{0.413320in}}%
\pgfpathlineto{\pgfqpoint{3.764966in}{0.413320in}}%
\pgfpathlineto{\pgfqpoint{3.762389in}{0.413320in}}%
\pgfpathlineto{\pgfqpoint{3.759622in}{0.413320in}}%
\pgfpathlineto{\pgfqpoint{3.757065in}{0.413320in}}%
\pgfpathlineto{\pgfqpoint{3.754265in}{0.413320in}}%
\pgfpathlineto{\pgfqpoint{3.751728in}{0.413320in}}%
\pgfpathlineto{\pgfqpoint{3.748903in}{0.413320in}}%
\pgfpathlineto{\pgfqpoint{3.746229in}{0.413320in}}%
\pgfpathlineto{\pgfqpoint{3.743548in}{0.413320in}}%
\pgfpathlineto{\pgfqpoint{3.740874in}{0.413320in}}%
\pgfpathlineto{\pgfqpoint{3.738194in}{0.413320in}}%
\pgfpathlineto{\pgfqpoint{3.735509in}{0.413320in}}%
\pgfpathlineto{\pgfqpoint{3.732950in}{0.413320in}}%
\pgfpathlineto{\pgfqpoint{3.730158in}{0.413320in}}%
\pgfpathlineto{\pgfqpoint{3.727581in}{0.413320in}}%
\pgfpathlineto{\pgfqpoint{3.724804in}{0.413320in}}%
\pgfpathlineto{\pgfqpoint{3.722228in}{0.413320in}}%
\pgfpathlineto{\pgfqpoint{3.719446in}{0.413320in}}%
\pgfpathlineto{\pgfqpoint{3.716875in}{0.413320in}}%
\pgfpathlineto{\pgfqpoint{3.714086in}{0.413320in}}%
\pgfpathlineto{\pgfqpoint{3.711410in}{0.413320in}}%
\pgfpathlineto{\pgfqpoint{3.708729in}{0.413320in}}%
\pgfpathlineto{\pgfqpoint{3.706053in}{0.413320in}}%
\pgfpathlineto{\pgfqpoint{3.703460in}{0.413320in}}%
\pgfpathlineto{\pgfqpoint{3.700684in}{0.413320in}}%
\pgfpathlineto{\pgfqpoint{3.698125in}{0.413320in}}%
\pgfpathlineto{\pgfqpoint{3.695331in}{0.413320in}}%
\pgfpathlineto{\pgfqpoint{3.692765in}{0.413320in}}%
\pgfpathlineto{\pgfqpoint{3.689983in}{0.413320in}}%
\pgfpathlineto{\pgfqpoint{3.687442in}{0.413320in}}%
\pgfpathlineto{\pgfqpoint{3.684620in}{0.413320in}}%
\pgfpathlineto{\pgfqpoint{3.681948in}{0.413320in}}%
\pgfpathlineto{\pgfqpoint{3.679273in}{0.413320in}}%
\pgfpathlineto{\pgfqpoint{3.676591in}{0.413320in}}%
\pgfpathlineto{\pgfqpoint{3.673911in}{0.413320in}}%
\pgfpathlineto{\pgfqpoint{3.671232in}{0.413320in}}%
\pgfpathlineto{\pgfqpoint{3.668665in}{0.413320in}}%
\pgfpathlineto{\pgfqpoint{3.665864in}{0.413320in}}%
\pgfpathlineto{\pgfqpoint{3.663276in}{0.413320in}}%
\pgfpathlineto{\pgfqpoint{3.660515in}{0.413320in}}%
\pgfpathlineto{\pgfqpoint{3.657917in}{0.413320in}}%
\pgfpathlineto{\pgfqpoint{3.655165in}{0.413320in}}%
\pgfpathlineto{\pgfqpoint{3.652628in}{0.413320in}}%
\pgfpathlineto{\pgfqpoint{3.649837in}{0.413320in}}%
\pgfpathlineto{\pgfqpoint{3.647130in}{0.413320in}}%
\pgfpathlineto{\pgfqpoint{3.644452in}{0.413320in}}%
\pgfpathlineto{\pgfqpoint{3.641773in}{0.413320in}}%
\pgfpathlineto{\pgfqpoint{3.639207in}{0.413320in}}%
\pgfpathlineto{\pgfqpoint{3.636413in}{0.413320in}}%
\pgfpathlineto{\pgfqpoint{3.633858in}{0.413320in}}%
\pgfpathlineto{\pgfqpoint{3.631058in}{0.413320in}}%
\pgfpathlineto{\pgfqpoint{3.628460in}{0.413320in}}%
\pgfpathlineto{\pgfqpoint{3.625689in}{0.413320in}}%
\pgfpathlineto{\pgfqpoint{3.623165in}{0.413320in}}%
\pgfpathlineto{\pgfqpoint{3.620345in}{0.413320in}}%
\pgfpathlineto{\pgfqpoint{3.617667in}{0.413320in}}%
\pgfpathlineto{\pgfqpoint{3.614982in}{0.413320in}}%
\pgfpathlineto{\pgfqpoint{3.612311in}{0.413320in}}%
\pgfpathlineto{\pgfqpoint{3.609632in}{0.413320in}}%
\pgfpathlineto{\pgfqpoint{3.606951in}{0.413320in}}%
\pgfpathlineto{\pgfqpoint{3.604387in}{0.413320in}}%
\pgfpathlineto{\pgfqpoint{3.601590in}{0.413320in}}%
\pgfpathlineto{\pgfqpoint{3.598998in}{0.413320in}}%
\pgfpathlineto{\pgfqpoint{3.596240in}{0.413320in}}%
\pgfpathlineto{\pgfqpoint{3.593620in}{0.413320in}}%
\pgfpathlineto{\pgfqpoint{3.590883in}{0.413320in}}%
\pgfpathlineto{\pgfqpoint{3.588258in}{0.413320in}}%
\pgfpathlineto{\pgfqpoint{3.585532in}{0.413320in}}%
\pgfpathlineto{\pgfqpoint{3.582851in}{0.413320in}}%
\pgfpathlineto{\pgfqpoint{3.580191in}{0.413320in}}%
\pgfpathlineto{\pgfqpoint{3.577487in}{0.413320in}}%
\pgfpathlineto{\pgfqpoint{3.574814in}{0.413320in}}%
\pgfpathlineto{\pgfqpoint{3.572126in}{0.413320in}}%
\pgfpathlineto{\pgfqpoint{3.569584in}{0.413320in}}%
\pgfpathlineto{\pgfqpoint{3.566774in}{0.413320in}}%
\pgfpathlineto{\pgfqpoint{3.564188in}{0.413320in}}%
\pgfpathlineto{\pgfqpoint{3.561420in}{0.413320in}}%
\pgfpathlineto{\pgfqpoint{3.558853in}{0.413320in}}%
\pgfpathlineto{\pgfqpoint{3.556061in}{0.413320in}}%
\pgfpathlineto{\pgfqpoint{3.553498in}{0.413320in}}%
\pgfpathlineto{\pgfqpoint{3.550713in}{0.413320in}}%
\pgfpathlineto{\pgfqpoint{3.548029in}{0.413320in}}%
\pgfpathlineto{\pgfqpoint{3.545349in}{0.413320in}}%
\pgfpathlineto{\pgfqpoint{3.542656in}{0.413320in}}%
\pgfpathlineto{\pgfqpoint{3.540093in}{0.413320in}}%
\pgfpathlineto{\pgfqpoint{3.537309in}{0.413320in}}%
\pgfpathlineto{\pgfqpoint{3.534783in}{0.413320in}}%
\pgfpathlineto{\pgfqpoint{3.531955in}{0.413320in}}%
\pgfpathlineto{\pgfqpoint{3.529327in}{0.413320in}}%
\pgfpathlineto{\pgfqpoint{3.526601in}{0.413320in}}%
\pgfpathlineto{\pgfqpoint{3.524041in}{0.413320in}}%
\pgfpathlineto{\pgfqpoint{3.521244in}{0.413320in}}%
\pgfpathlineto{\pgfqpoint{3.518565in}{0.413320in}}%
\pgfpathlineto{\pgfqpoint{3.515884in}{0.413320in}}%
\pgfpathlineto{\pgfqpoint{3.513209in}{0.413320in}}%
\pgfpathlineto{\pgfqpoint{3.510533in}{0.413320in}}%
\pgfpathlineto{\pgfqpoint{3.507840in}{0.413320in}}%
\pgfpathlineto{\pgfqpoint{3.505262in}{0.413320in}}%
\pgfpathlineto{\pgfqpoint{3.502488in}{0.413320in}}%
\pgfpathlineto{\pgfqpoint{3.499909in}{0.413320in}}%
\pgfpathlineto{\pgfqpoint{3.497139in}{0.413320in}}%
\pgfpathlineto{\pgfqpoint{3.494581in}{0.413320in}}%
\pgfpathlineto{\pgfqpoint{3.491783in}{0.413320in}}%
\pgfpathlineto{\pgfqpoint{3.489223in}{0.413320in}}%
\pgfpathlineto{\pgfqpoint{3.486442in}{0.413320in}}%
\pgfpathlineto{\pgfqpoint{3.483744in}{0.413320in}}%
\pgfpathlineto{\pgfqpoint{3.481072in}{0.413320in}}%
\pgfpathlineto{\pgfqpoint{3.478378in}{0.413320in}}%
\pgfpathlineto{\pgfqpoint{3.475821in}{0.413320in}}%
\pgfpathlineto{\pgfqpoint{3.473021in}{0.413320in}}%
\pgfpathlineto{\pgfqpoint{3.470466in}{0.413320in}}%
\pgfpathlineto{\pgfqpoint{3.467678in}{0.413320in}}%
\pgfpathlineto{\pgfqpoint{3.465072in}{0.413320in}}%
\pgfpathlineto{\pgfqpoint{3.462321in}{0.413320in}}%
\pgfpathlineto{\pgfqpoint{3.459695in}{0.413320in}}%
\pgfpathlineto{\pgfqpoint{3.456960in}{0.413320in}}%
\pgfpathlineto{\pgfqpoint{3.454285in}{0.413320in}}%
\pgfpathlineto{\pgfqpoint{3.451597in}{0.413320in}}%
\pgfpathlineto{\pgfqpoint{3.448926in}{0.413320in}}%
\pgfpathlineto{\pgfqpoint{3.446257in}{0.413320in}}%
\pgfpathlineto{\pgfqpoint{3.443574in}{0.413320in}}%
\pgfpathlineto{\pgfqpoint{3.440996in}{0.413320in}}%
\pgfpathlineto{\pgfqpoint{3.438210in}{0.413320in}}%
\pgfpathlineto{\pgfqpoint{3.435635in}{0.413320in}}%
\pgfpathlineto{\pgfqpoint{3.432851in}{0.413320in}}%
\pgfpathlineto{\pgfqpoint{3.430313in}{0.413320in}}%
\pgfpathlineto{\pgfqpoint{3.427501in}{0.413320in}}%
\pgfpathlineto{\pgfqpoint{3.424887in}{0.413320in}}%
\pgfpathlineto{\pgfqpoint{3.422142in}{0.413320in}}%
\pgfpathlineto{\pgfqpoint{3.419455in}{0.413320in}}%
\pgfpathlineto{\pgfqpoint{3.416780in}{0.413320in}}%
\pgfpathlineto{\pgfqpoint{3.414109in}{0.413320in}}%
\pgfpathlineto{\pgfqpoint{3.411431in}{0.413320in}}%
\pgfpathlineto{\pgfqpoint{3.408752in}{0.413320in}}%
\pgfpathlineto{\pgfqpoint{3.406202in}{0.413320in}}%
\pgfpathlineto{\pgfqpoint{3.403394in}{0.413320in}}%
\pgfpathlineto{\pgfqpoint{3.400783in}{0.413320in}}%
\pgfpathlineto{\pgfqpoint{3.398037in}{0.413320in}}%
\pgfpathlineto{\pgfqpoint{3.395461in}{0.413320in}}%
\pgfpathlineto{\pgfqpoint{3.392681in}{0.413320in}}%
\pgfpathlineto{\pgfqpoint{3.390102in}{0.413320in}}%
\pgfpathlineto{\pgfqpoint{3.387309in}{0.413320in}}%
\pgfpathlineto{\pgfqpoint{3.384647in}{0.413320in}}%
\pgfpathlineto{\pgfqpoint{3.381959in}{0.413320in}}%
\pgfpathlineto{\pgfqpoint{3.379290in}{0.413320in}}%
\pgfpathlineto{\pgfqpoint{3.376735in}{0.413320in}}%
\pgfpathlineto{\pgfqpoint{3.373921in}{0.413320in}}%
\pgfpathlineto{\pgfqpoint{3.371357in}{0.413320in}}%
\pgfpathlineto{\pgfqpoint{3.368577in}{0.413320in}}%
\pgfpathlineto{\pgfqpoint{3.365996in}{0.413320in}}%
\pgfpathlineto{\pgfqpoint{3.363221in}{0.413320in}}%
\pgfpathlineto{\pgfqpoint{3.360620in}{0.413320in}}%
\pgfpathlineto{\pgfqpoint{3.357862in}{0.413320in}}%
\pgfpathlineto{\pgfqpoint{3.355177in}{0.413320in}}%
\pgfpathlineto{\pgfqpoint{3.352505in}{0.413320in}}%
\pgfpathlineto{\pgfqpoint{3.349828in}{0.413320in}}%
\pgfpathlineto{\pgfqpoint{3.347139in}{0.413320in}}%
\pgfpathlineto{\pgfqpoint{3.344468in}{0.413320in}}%
\pgfpathlineto{\pgfqpoint{3.341893in}{0.413320in}}%
\pgfpathlineto{\pgfqpoint{3.339101in}{0.413320in}}%
\pgfpathlineto{\pgfqpoint{3.336541in}{0.413320in}}%
\pgfpathlineto{\pgfqpoint{3.333758in}{0.413320in}}%
\pgfpathlineto{\pgfqpoint{3.331183in}{0.413320in}}%
\pgfpathlineto{\pgfqpoint{3.328401in}{0.413320in}}%
\pgfpathlineto{\pgfqpoint{3.325860in}{0.413320in}}%
\pgfpathlineto{\pgfqpoint{3.323049in}{0.413320in}}%
\pgfpathlineto{\pgfqpoint{3.320366in}{0.413320in}}%
\pgfpathlineto{\pgfqpoint{3.317688in}{0.413320in}}%
\pgfpathlineto{\pgfqpoint{3.315008in}{0.413320in}}%
\pgfpathlineto{\pgfqpoint{3.312480in}{0.413320in}}%
\pgfpathlineto{\pgfqpoint{3.309652in}{0.413320in}}%
\pgfpathlineto{\pgfqpoint{3.307104in}{0.413320in}}%
\pgfpathlineto{\pgfqpoint{3.304295in}{0.413320in}}%
\pgfpathlineto{\pgfqpoint{3.301719in}{0.413320in}}%
\pgfpathlineto{\pgfqpoint{3.298937in}{0.413320in}}%
\pgfpathlineto{\pgfqpoint{3.296376in}{0.413320in}}%
\pgfpathlineto{\pgfqpoint{3.293574in}{0.413320in}}%
\pgfpathlineto{\pgfqpoint{3.290890in}{0.413320in}}%
\pgfpathlineto{\pgfqpoint{3.288225in}{0.413320in}}%
\pgfpathlineto{\pgfqpoint{3.285534in}{0.413320in}}%
\pgfpathlineto{\pgfqpoint{3.282870in}{0.413320in}}%
\pgfpathlineto{\pgfqpoint{3.280189in}{0.413320in}}%
\pgfpathlineto{\pgfqpoint{3.277603in}{0.413320in}}%
\pgfpathlineto{\pgfqpoint{3.274831in}{0.413320in}}%
\pgfpathlineto{\pgfqpoint{3.272254in}{0.413320in}}%
\pgfpathlineto{\pgfqpoint{3.269478in}{0.413320in}}%
\pgfpathlineto{\pgfqpoint{3.266849in}{0.413320in}}%
\pgfpathlineto{\pgfqpoint{3.264119in}{0.413320in}}%
\pgfpathlineto{\pgfqpoint{3.261594in}{0.413320in}}%
\pgfpathlineto{\pgfqpoint{3.258784in}{0.413320in}}%
\pgfpathlineto{\pgfqpoint{3.256083in}{0.413320in}}%
\pgfpathlineto{\pgfqpoint{3.253404in}{0.413320in}}%
\pgfpathlineto{\pgfqpoint{3.250716in}{0.413320in}}%
\pgfpathlineto{\pgfqpoint{3.248049in}{0.413320in}}%
\pgfpathlineto{\pgfqpoint{3.245363in}{0.413320in}}%
\pgfpathlineto{\pgfqpoint{3.242807in}{0.413320in}}%
\pgfpathlineto{\pgfqpoint{3.240010in}{0.413320in}}%
\pgfpathlineto{\pgfqpoint{3.237411in}{0.413320in}}%
\pgfpathlineto{\pgfqpoint{3.234658in}{0.413320in}}%
\pgfpathlineto{\pgfqpoint{3.232069in}{0.413320in}}%
\pgfpathlineto{\pgfqpoint{3.229310in}{0.413320in}}%
\pgfpathlineto{\pgfqpoint{3.226609in}{0.413320in}}%
\pgfpathlineto{\pgfqpoint{3.223942in}{0.413320in}}%
\pgfpathlineto{\pgfqpoint{3.221255in}{0.413320in}}%
\pgfpathlineto{\pgfqpoint{3.218586in}{0.413320in}}%
\pgfpathlineto{\pgfqpoint{3.215908in}{0.413320in}}%
\pgfpathlineto{\pgfqpoint{3.213342in}{0.413320in}}%
\pgfpathlineto{\pgfqpoint{3.210545in}{0.413320in}}%
\pgfpathlineto{\pgfqpoint{3.207984in}{0.413320in}}%
\pgfpathlineto{\pgfqpoint{3.205195in}{0.413320in}}%
\pgfpathlineto{\pgfqpoint{3.202562in}{0.413320in}}%
\pgfpathlineto{\pgfqpoint{3.199823in}{0.413320in}}%
\pgfpathlineto{\pgfqpoint{3.197226in}{0.413320in}}%
\pgfpathlineto{\pgfqpoint{3.194508in}{0.413320in}}%
\pgfpathlineto{\pgfqpoint{3.191796in}{0.413320in}}%
\pgfpathlineto{\pgfqpoint{3.189117in}{0.413320in}}%
\pgfpathlineto{\pgfqpoint{3.186440in}{0.413320in}}%
\pgfpathlineto{\pgfqpoint{3.183760in}{0.413320in}}%
\pgfpathlineto{\pgfqpoint{3.181089in}{0.413320in}}%
\pgfpathlineto{\pgfqpoint{3.178525in}{0.413320in}}%
\pgfpathlineto{\pgfqpoint{3.175724in}{0.413320in}}%
\pgfpathlineto{\pgfqpoint{3.173142in}{0.413320in}}%
\pgfpathlineto{\pgfqpoint{3.170375in}{0.413320in}}%
\pgfpathlineto{\pgfqpoint{3.167776in}{0.413320in}}%
\pgfpathlineto{\pgfqpoint{3.165019in}{0.413320in}}%
\pgfpathlineto{\pgfqpoint{3.162474in}{0.413320in}}%
\pgfpathlineto{\pgfqpoint{3.159675in}{0.413320in}}%
\pgfpathlineto{\pgfqpoint{3.156981in}{0.413320in}}%
\pgfpathlineto{\pgfqpoint{3.154327in}{0.413320in}}%
\pgfpathlineto{\pgfqpoint{3.151612in}{0.413320in}}%
\pgfpathlineto{\pgfqpoint{3.149057in}{0.413320in}}%
\pgfpathlineto{\pgfqpoint{3.146271in}{0.413320in}}%
\pgfpathlineto{\pgfqpoint{3.143740in}{0.413320in}}%
\pgfpathlineto{\pgfqpoint{3.140913in}{0.413320in}}%
\pgfpathlineto{\pgfqpoint{3.138375in}{0.413320in}}%
\pgfpathlineto{\pgfqpoint{3.135550in}{0.413320in}}%
\pgfpathlineto{\pgfqpoint{3.132946in}{0.413320in}}%
\pgfpathlineto{\pgfqpoint{3.130199in}{0.413320in}}%
\pgfpathlineto{\pgfqpoint{3.127512in}{0.413320in}}%
\pgfpathlineto{\pgfqpoint{3.124842in}{0.413320in}}%
\pgfpathlineto{\pgfqpoint{3.122163in}{0.413320in}}%
\pgfpathlineto{\pgfqpoint{3.119487in}{0.413320in}}%
\pgfpathlineto{\pgfqpoint{3.116807in}{0.413320in}}%
\pgfpathlineto{\pgfqpoint{3.114242in}{0.413320in}}%
\pgfpathlineto{\pgfqpoint{3.111451in}{0.413320in}}%
\pgfpathlineto{\pgfqpoint{3.108896in}{0.413320in}}%
\pgfpathlineto{\pgfqpoint{3.106094in}{0.413320in}}%
\pgfpathlineto{\pgfqpoint{3.103508in}{0.413320in}}%
\pgfpathlineto{\pgfqpoint{3.100737in}{0.413320in}}%
\pgfpathlineto{\pgfqpoint{3.098163in}{0.413320in}}%
\pgfpathlineto{\pgfqpoint{3.095388in}{0.413320in}}%
\pgfpathlineto{\pgfqpoint{3.092699in}{0.413320in}}%
\pgfpathlineto{\pgfqpoint{3.090023in}{0.413320in}}%
\pgfpathlineto{\pgfqpoint{3.087343in}{0.413320in}}%
\pgfpathlineto{\pgfqpoint{3.084671in}{0.413320in}}%
\pgfpathlineto{\pgfqpoint{3.081990in}{0.413320in}}%
\pgfpathlineto{\pgfqpoint{3.079381in}{0.413320in}}%
\pgfpathlineto{\pgfqpoint{3.076631in}{0.413320in}}%
\pgfpathlineto{\pgfqpoint{3.074056in}{0.413320in}}%
\pgfpathlineto{\pgfqpoint{3.071266in}{0.413320in}}%
\pgfpathlineto{\pgfqpoint{3.068709in}{0.413320in}}%
\pgfpathlineto{\pgfqpoint{3.065916in}{0.413320in}}%
\pgfpathlineto{\pgfqpoint{3.063230in}{0.413320in}}%
\pgfpathlineto{\pgfqpoint{3.060561in}{0.413320in}}%
\pgfpathlineto{\pgfqpoint{3.057884in}{0.413320in}}%
\pgfpathlineto{\pgfqpoint{3.055202in}{0.413320in}}%
\pgfpathlineto{\pgfqpoint{3.052526in}{0.413320in}}%
\pgfpathlineto{\pgfqpoint{3.049988in}{0.413320in}}%
\pgfpathlineto{\pgfqpoint{3.047157in}{0.413320in}}%
\pgfpathlineto{\pgfqpoint{3.044568in}{0.413320in}}%
\pgfpathlineto{\pgfqpoint{3.041813in}{0.413320in}}%
\pgfpathlineto{\pgfqpoint{3.039262in}{0.413320in}}%
\pgfpathlineto{\pgfqpoint{3.036456in}{0.413320in}}%
\pgfpathlineto{\pgfqpoint{3.033921in}{0.413320in}}%
\pgfpathlineto{\pgfqpoint{3.031091in}{0.413320in}}%
\pgfpathlineto{\pgfqpoint{3.028412in}{0.413320in}}%
\pgfpathlineto{\pgfqpoint{3.025803in}{0.413320in}}%
\pgfpathlineto{\pgfqpoint{3.023058in}{0.413320in}}%
\pgfpathlineto{\pgfqpoint{3.020382in}{0.413320in}}%
\pgfpathlineto{\pgfqpoint{3.017707in}{0.413320in}}%
\pgfpathlineto{\pgfqpoint{3.015097in}{0.413320in}}%
\pgfpathlineto{\pgfqpoint{3.012351in}{0.413320in}}%
\pgfpathlineto{\pgfqpoint{3.009784in}{0.413320in}}%
\pgfpathlineto{\pgfqpoint{3.006993in}{0.413320in}}%
\pgfpathlineto{\pgfqpoint{3.004419in}{0.413320in}}%
\pgfpathlineto{\pgfqpoint{3.001635in}{0.413320in}}%
\pgfpathlineto{\pgfqpoint{2.999103in}{0.413320in}}%
\pgfpathlineto{\pgfqpoint{2.996300in}{0.413320in}}%
\pgfpathlineto{\pgfqpoint{2.993595in}{0.413320in}}%
\pgfpathlineto{\pgfqpoint{2.990978in}{0.413320in}}%
\pgfpathlineto{\pgfqpoint{2.988238in}{0.413320in}}%
\pgfpathlineto{\pgfqpoint{2.985666in}{0.413320in}}%
\pgfpathlineto{\pgfqpoint{2.982885in}{0.413320in}}%
\pgfpathlineto{\pgfqpoint{2.980341in}{0.413320in}}%
\pgfpathlineto{\pgfqpoint{2.977517in}{0.413320in}}%
\pgfpathlineto{\pgfqpoint{2.974972in}{0.413320in}}%
\pgfpathlineto{\pgfqpoint{2.972177in}{0.413320in}}%
\pgfpathlineto{\pgfqpoint{2.969599in}{0.413320in}}%
\pgfpathlineto{\pgfqpoint{2.966812in}{0.413320in}}%
\pgfpathlineto{\pgfqpoint{2.964127in}{0.413320in}}%
\pgfpathlineto{\pgfqpoint{2.961460in}{0.413320in}}%
\pgfpathlineto{\pgfqpoint{2.958782in}{0.413320in}}%
\pgfpathlineto{\pgfqpoint{2.956103in}{0.413320in}}%
\pgfpathlineto{\pgfqpoint{2.953422in}{0.413320in}}%
\pgfpathlineto{\pgfqpoint{2.950884in}{0.413320in}}%
\pgfpathlineto{\pgfqpoint{2.948068in}{0.413320in}}%
\pgfpathlineto{\pgfqpoint{2.945461in}{0.413320in}}%
\pgfpathlineto{\pgfqpoint{2.942711in}{0.413320in}}%
\pgfpathlineto{\pgfqpoint{2.940120in}{0.413320in}}%
\pgfpathlineto{\pgfqpoint{2.937352in}{0.413320in}}%
\pgfpathlineto{\pgfqpoint{2.934759in}{0.413320in}}%
\pgfpathlineto{\pgfqpoint{2.932033in}{0.413320in}}%
\pgfpathlineto{\pgfqpoint{2.929321in}{0.413320in}}%
\pgfpathlineto{\pgfqpoint{2.926655in}{0.413320in}}%
\pgfpathlineto{\pgfqpoint{2.923963in}{0.413320in}}%
\pgfpathlineto{\pgfqpoint{2.921363in}{0.413320in}}%
\pgfpathlineto{\pgfqpoint{2.918606in}{0.413320in}}%
\pgfpathlineto{\pgfqpoint{2.916061in}{0.413320in}}%
\pgfpathlineto{\pgfqpoint{2.913243in}{0.413320in}}%
\pgfpathlineto{\pgfqpoint{2.910631in}{0.413320in}}%
\pgfpathlineto{\pgfqpoint{2.907882in}{0.413320in}}%
\pgfpathlineto{\pgfqpoint{2.905341in}{0.413320in}}%
\pgfpathlineto{\pgfqpoint{2.902535in}{0.413320in}}%
\pgfpathlineto{\pgfqpoint{2.899858in}{0.413320in}}%
\pgfpathlineto{\pgfqpoint{2.897179in}{0.413320in}}%
\pgfpathlineto{\pgfqpoint{2.894487in}{0.413320in}}%
\pgfpathlineto{\pgfqpoint{2.891809in}{0.413320in}}%
\pgfpathlineto{\pgfqpoint{2.889145in}{0.413320in}}%
\pgfpathlineto{\pgfqpoint{2.886578in}{0.413320in}}%
\pgfpathlineto{\pgfqpoint{2.883780in}{0.413320in}}%
\pgfpathlineto{\pgfqpoint{2.881254in}{0.413320in}}%
\pgfpathlineto{\pgfqpoint{2.878431in}{0.413320in}}%
\pgfpathlineto{\pgfqpoint{2.875882in}{0.413320in}}%
\pgfpathlineto{\pgfqpoint{2.873074in}{0.413320in}}%
\pgfpathlineto{\pgfqpoint{2.870475in}{0.413320in}}%
\pgfpathlineto{\pgfqpoint{2.867713in}{0.413320in}}%
\pgfpathlineto{\pgfqpoint{2.865031in}{0.413320in}}%
\pgfpathlineto{\pgfqpoint{2.862402in}{0.413320in}}%
\pgfpathlineto{\pgfqpoint{2.859668in}{0.413320in}}%
\pgfpathlineto{\pgfqpoint{2.857003in}{0.413320in}}%
\pgfpathlineto{\pgfqpoint{2.854325in}{0.413320in}}%
\pgfpathlineto{\pgfqpoint{2.851793in}{0.413320in}}%
\pgfpathlineto{\pgfqpoint{2.848960in}{0.413320in}}%
\pgfpathlineto{\pgfqpoint{2.846408in}{0.413320in}}%
\pgfpathlineto{\pgfqpoint{2.843611in}{0.413320in}}%
\pgfpathlineto{\pgfqpoint{2.841055in}{0.413320in}}%
\pgfpathlineto{\pgfqpoint{2.838254in}{0.413320in}}%
\pgfpathlineto{\pgfqpoint{2.835698in}{0.413320in}}%
\pgfpathlineto{\pgfqpoint{2.832894in}{0.413320in}}%
\pgfpathlineto{\pgfqpoint{2.830219in}{0.413320in}}%
\pgfpathlineto{\pgfqpoint{2.827567in}{0.413320in}}%
\pgfpathlineto{\pgfqpoint{2.824851in}{0.413320in}}%
\pgfpathlineto{\pgfqpoint{2.822303in}{0.413320in}}%
\pgfpathlineto{\pgfqpoint{2.819506in}{0.413320in}}%
\pgfpathlineto{\pgfqpoint{2.816867in}{0.413320in}}%
\pgfpathlineto{\pgfqpoint{2.814141in}{0.413320in}}%
\pgfpathlineto{\pgfqpoint{2.811597in}{0.413320in}}%
\pgfpathlineto{\pgfqpoint{2.808792in}{0.413320in}}%
\pgfpathlineto{\pgfqpoint{2.806175in}{0.413320in}}%
\pgfpathlineto{\pgfqpoint{2.803435in}{0.413320in}}%
\pgfpathlineto{\pgfqpoint{2.800756in}{0.413320in}}%
\pgfpathlineto{\pgfqpoint{2.798070in}{0.413320in}}%
\pgfpathlineto{\pgfqpoint{2.795398in}{0.413320in}}%
\pgfpathlineto{\pgfqpoint{2.792721in}{0.413320in}}%
\pgfpathlineto{\pgfqpoint{2.790044in}{0.413320in}}%
\pgfpathlineto{\pgfqpoint{2.787468in}{0.413320in}}%
\pgfpathlineto{\pgfqpoint{2.784687in}{0.413320in}}%
\pgfpathlineto{\pgfqpoint{2.782113in}{0.413320in}}%
\pgfpathlineto{\pgfqpoint{2.779330in}{0.413320in}}%
\pgfpathlineto{\pgfqpoint{2.776767in}{0.413320in}}%
\pgfpathlineto{\pgfqpoint{2.773972in}{0.413320in}}%
\pgfpathlineto{\pgfqpoint{2.771373in}{0.413320in}}%
\pgfpathlineto{\pgfqpoint{2.768617in}{0.413320in}}%
\pgfpathlineto{\pgfqpoint{2.765935in}{0.413320in}}%
\pgfpathlineto{\pgfqpoint{2.763253in}{0.413320in}}%
\pgfpathlineto{\pgfqpoint{2.760581in}{0.413320in}}%
\pgfpathlineto{\pgfqpoint{2.758028in}{0.413320in}}%
\pgfpathlineto{\pgfqpoint{2.755224in}{0.413320in}}%
\pgfpathlineto{\pgfqpoint{2.752614in}{0.413320in}}%
\pgfpathlineto{\pgfqpoint{2.749868in}{0.413320in}}%
\pgfpathlineto{\pgfqpoint{2.747260in}{0.413320in}}%
\pgfpathlineto{\pgfqpoint{2.744510in}{0.413320in}}%
\pgfpathlineto{\pgfqpoint{2.741928in}{0.413320in}}%
\pgfpathlineto{\pgfqpoint{2.739155in}{0.413320in}}%
\pgfpathlineto{\pgfqpoint{2.736476in}{0.413320in}}%
\pgfpathlineto{\pgfqpoint{2.733798in}{0.413320in}}%
\pgfpathlineto{\pgfqpoint{2.731119in}{0.413320in}}%
\pgfpathlineto{\pgfqpoint{2.728439in}{0.413320in}}%
\pgfpathlineto{\pgfqpoint{2.725760in}{0.413320in}}%
\pgfpathlineto{\pgfqpoint{2.723211in}{0.413320in}}%
\pgfpathlineto{\pgfqpoint{2.720404in}{0.413320in}}%
\pgfpathlineto{\pgfqpoint{2.717773in}{0.413320in}}%
\pgfpathlineto{\pgfqpoint{2.715036in}{0.413320in}}%
\pgfpathlineto{\pgfqpoint{2.712477in}{0.413320in}}%
\pgfpathlineto{\pgfqpoint{2.709683in}{0.413320in}}%
\pgfpathlineto{\pgfqpoint{2.707125in}{0.413320in}}%
\pgfpathlineto{\pgfqpoint{2.704326in}{0.413320in}}%
\pgfpathlineto{\pgfqpoint{2.701657in}{0.413320in}}%
\pgfpathlineto{\pgfqpoint{2.698968in}{0.413320in}}%
\pgfpathlineto{\pgfqpoint{2.696293in}{0.413320in}}%
\pgfpathlineto{\pgfqpoint{2.693611in}{0.413320in}}%
\pgfpathlineto{\pgfqpoint{2.690940in}{0.413320in}}%
\pgfpathlineto{\pgfqpoint{2.688328in}{0.413320in}}%
\pgfpathlineto{\pgfqpoint{2.685586in}{0.413320in}}%
\pgfpathlineto{\pgfqpoint{2.683009in}{0.413320in}}%
\pgfpathlineto{\pgfqpoint{2.680224in}{0.413320in}}%
\pgfpathlineto{\pgfqpoint{2.677650in}{0.413320in}}%
\pgfpathlineto{\pgfqpoint{2.674873in}{0.413320in}}%
\pgfpathlineto{\pgfqpoint{2.672301in}{0.413320in}}%
\pgfpathlineto{\pgfqpoint{2.669506in}{0.413320in}}%
\pgfpathlineto{\pgfqpoint{2.666836in}{0.413320in}}%
\pgfpathlineto{\pgfqpoint{2.664151in}{0.413320in}}%
\pgfpathlineto{\pgfqpoint{2.661481in}{0.413320in}}%
\pgfpathlineto{\pgfqpoint{2.658942in}{0.413320in}}%
\pgfpathlineto{\pgfqpoint{2.656124in}{0.413320in}}%
\pgfpathlineto{\pgfqpoint{2.653567in}{0.413320in}}%
\pgfpathlineto{\pgfqpoint{2.650767in}{0.413320in}}%
\pgfpathlineto{\pgfqpoint{2.648196in}{0.413320in}}%
\pgfpathlineto{\pgfqpoint{2.645408in}{0.413320in}}%
\pgfpathlineto{\pgfqpoint{2.642827in}{0.413320in}}%
\pgfpathlineto{\pgfqpoint{2.640053in}{0.413320in}}%
\pgfpathlineto{\pgfqpoint{2.637369in}{0.413320in}}%
\pgfpathlineto{\pgfqpoint{2.634700in}{0.413320in}}%
\pgfpathlineto{\pgfqpoint{2.632018in}{0.413320in}}%
\pgfpathlineto{\pgfqpoint{2.629340in}{0.413320in}}%
\pgfpathlineto{\pgfqpoint{2.626653in}{0.413320in}}%
\pgfpathlineto{\pgfqpoint{2.624077in}{0.413320in}}%
\pgfpathlineto{\pgfqpoint{2.621304in}{0.413320in}}%
\pgfpathlineto{\pgfqpoint{2.618773in}{0.413320in}}%
\pgfpathlineto{\pgfqpoint{2.615934in}{0.413320in}}%
\pgfpathlineto{\pgfqpoint{2.613393in}{0.413320in}}%
\pgfpathlineto{\pgfqpoint{2.610588in}{0.413320in}}%
\pgfpathlineto{\pgfqpoint{2.608004in}{0.413320in}}%
\pgfpathlineto{\pgfqpoint{2.605232in}{0.413320in}}%
\pgfpathlineto{\pgfqpoint{2.602557in}{0.413320in}}%
\pgfpathlineto{\pgfqpoint{2.599920in}{0.413320in}}%
\pgfpathlineto{\pgfqpoint{2.597196in}{0.413320in}}%
\pgfpathlineto{\pgfqpoint{2.594630in}{0.413320in}}%
\pgfpathlineto{\pgfqpoint{2.591842in}{0.413320in}}%
\pgfpathlineto{\pgfqpoint{2.589248in}{0.413320in}}%
\pgfpathlineto{\pgfqpoint{2.586484in}{0.413320in}}%
\pgfpathlineto{\pgfqpoint{2.583913in}{0.413320in}}%
\pgfpathlineto{\pgfqpoint{2.581129in}{0.413320in}}%
\pgfpathlineto{\pgfqpoint{2.578567in}{0.413320in}}%
\pgfpathlineto{\pgfqpoint{2.575779in}{0.413320in}}%
\pgfpathlineto{\pgfqpoint{2.573082in}{0.413320in}}%
\pgfpathlineto{\pgfqpoint{2.570411in}{0.413320in}}%
\pgfpathlineto{\pgfqpoint{2.567730in}{0.413320in}}%
\pgfpathlineto{\pgfqpoint{2.565045in}{0.413320in}}%
\pgfpathlineto{\pgfqpoint{2.562375in}{0.413320in}}%
\pgfpathlineto{\pgfqpoint{2.559790in}{0.413320in}}%
\pgfpathlineto{\pgfqpoint{2.557009in}{0.413320in}}%
\pgfpathlineto{\pgfqpoint{2.554493in}{0.413320in}}%
\pgfpathlineto{\pgfqpoint{2.551664in}{0.413320in}}%
\pgfpathlineto{\pgfqpoint{2.549114in}{0.413320in}}%
\pgfpathlineto{\pgfqpoint{2.546310in}{0.413320in}}%
\pgfpathlineto{\pgfqpoint{2.543765in}{0.413320in}}%
\pgfpathlineto{\pgfqpoint{2.540949in}{0.413320in}}%
\pgfpathlineto{\pgfqpoint{2.538274in}{0.413320in}}%
\pgfpathlineto{\pgfqpoint{2.535624in}{0.413320in}}%
\pgfpathlineto{\pgfqpoint{2.532917in}{0.413320in}}%
\pgfpathlineto{\pgfqpoint{2.530234in}{0.413320in}}%
\pgfpathlineto{\pgfqpoint{2.527560in}{0.413320in}}%
\pgfpathlineto{\pgfqpoint{2.524988in}{0.413320in}}%
\pgfpathlineto{\pgfqpoint{2.522197in}{0.413320in}}%
\pgfpathlineto{\pgfqpoint{2.519607in}{0.413320in}}%
\pgfpathlineto{\pgfqpoint{2.516845in}{0.413320in}}%
\pgfpathlineto{\pgfqpoint{2.514268in}{0.413320in}}%
\pgfpathlineto{\pgfqpoint{2.511478in}{0.413320in}}%
\pgfpathlineto{\pgfqpoint{2.508917in}{0.413320in}}%
\pgfpathlineto{\pgfqpoint{2.506163in}{0.413320in}}%
\pgfpathlineto{\pgfqpoint{2.503454in}{0.413320in}}%
\pgfpathlineto{\pgfqpoint{2.500801in}{0.413320in}}%
\pgfpathlineto{\pgfqpoint{2.498085in}{0.413320in}}%
\pgfpathlineto{\pgfqpoint{2.495542in}{0.413320in}}%
\pgfpathlineto{\pgfqpoint{2.492729in}{0.413320in}}%
\pgfpathlineto{\pgfqpoint{2.490183in}{0.413320in}}%
\pgfpathlineto{\pgfqpoint{2.487384in}{0.413320in}}%
\pgfpathlineto{\pgfqpoint{2.484870in}{0.413320in}}%
\pgfpathlineto{\pgfqpoint{2.482026in}{0.413320in}}%
\pgfpathlineto{\pgfqpoint{2.479420in}{0.413320in}}%
\pgfpathlineto{\pgfqpoint{2.476671in}{0.413320in}}%
\pgfpathlineto{\pgfqpoint{2.473989in}{0.413320in}}%
\pgfpathlineto{\pgfqpoint{2.471311in}{0.413320in}}%
\pgfpathlineto{\pgfqpoint{2.468635in}{0.413320in}}%
\pgfpathlineto{\pgfqpoint{2.465957in}{0.413320in}}%
\pgfpathlineto{\pgfqpoint{2.463280in}{0.413320in}}%
\pgfpathlineto{\pgfqpoint{2.460711in}{0.413320in}}%
\pgfpathlineto{\pgfqpoint{2.457917in}{0.413320in}}%
\pgfpathlineto{\pgfqpoint{2.455353in}{0.413320in}}%
\pgfpathlineto{\pgfqpoint{2.452562in}{0.413320in}}%
\pgfpathlineto{\pgfqpoint{2.450032in}{0.413320in}}%
\pgfpathlineto{\pgfqpoint{2.447209in}{0.413320in}}%
\pgfpathlineto{\pgfqpoint{2.444677in}{0.413320in}}%
\pgfpathlineto{\pgfqpoint{2.441876in}{0.413320in}}%
\pgfpathlineto{\pgfqpoint{2.439167in}{0.413320in}}%
\pgfpathlineto{\pgfqpoint{2.436518in}{0.413320in}}%
\pgfpathlineto{\pgfqpoint{2.433815in}{0.413320in}}%
\pgfpathlineto{\pgfqpoint{2.431251in}{0.413320in}}%
\pgfpathlineto{\pgfqpoint{2.428453in}{0.413320in}}%
\pgfpathlineto{\pgfqpoint{2.425878in}{0.413320in}}%
\pgfpathlineto{\pgfqpoint{2.423098in}{0.413320in}}%
\pgfpathlineto{\pgfqpoint{2.420528in}{0.413320in}}%
\pgfpathlineto{\pgfqpoint{2.417747in}{0.413320in}}%
\pgfpathlineto{\pgfqpoint{2.415184in}{0.413320in}}%
\pgfpathlineto{\pgfqpoint{2.412389in}{0.413320in}}%
\pgfpathlineto{\pgfqpoint{2.409699in}{0.413320in}}%
\pgfpathlineto{\pgfqpoint{2.407024in}{0.413320in}}%
\pgfpathlineto{\pgfqpoint{2.404352in}{0.413320in}}%
\pgfpathlineto{\pgfqpoint{2.401675in}{0.413320in}}%
\pgfpathlineto{\pgfqpoint{2.398995in}{0.413320in}}%
\pgfpathclose%
\pgfusepath{stroke,fill}%
\end{pgfscope}%
\begin{pgfscope}%
\pgfpathrectangle{\pgfqpoint{2.398995in}{0.319877in}}{\pgfqpoint{3.986877in}{1.993438in}} %
\pgfusepath{clip}%
\pgfsetbuttcap%
\pgfsetroundjoin%
\definecolor{currentfill}{rgb}{1.000000,1.000000,1.000000}%
\pgfsetfillcolor{currentfill}%
\pgfsetlinewidth{1.003750pt}%
\definecolor{currentstroke}{rgb}{0.965905,0.426914,0.643048}%
\pgfsetstrokecolor{currentstroke}%
\pgfsetdash{}{0pt}%
\pgfpathmoveto{\pgfqpoint{2.398995in}{0.413320in}}%
\pgfpathlineto{\pgfqpoint{2.398995in}{0.991923in}}%
\pgfpathlineto{\pgfqpoint{2.401675in}{0.981502in}}%
\pgfpathlineto{\pgfqpoint{2.404352in}{0.981339in}}%
\pgfpathlineto{\pgfqpoint{2.407024in}{0.984880in}}%
\pgfpathlineto{\pgfqpoint{2.409699in}{0.972515in}}%
\pgfpathlineto{\pgfqpoint{2.412389in}{0.970633in}}%
\pgfpathlineto{\pgfqpoint{2.415184in}{0.974484in}}%
\pgfpathlineto{\pgfqpoint{2.417747in}{0.978463in}}%
\pgfpathlineto{\pgfqpoint{2.420528in}{0.975743in}}%
\pgfpathlineto{\pgfqpoint{2.423098in}{0.976388in}}%
\pgfpathlineto{\pgfqpoint{2.425878in}{0.978980in}}%
\pgfpathlineto{\pgfqpoint{2.428453in}{0.980542in}}%
\pgfpathlineto{\pgfqpoint{2.431251in}{0.992191in}}%
\pgfpathlineto{\pgfqpoint{2.433815in}{1.031588in}}%
\pgfpathlineto{\pgfqpoint{2.436518in}{1.069711in}}%
\pgfpathlineto{\pgfqpoint{2.439167in}{1.048040in}}%
\pgfpathlineto{\pgfqpoint{2.441876in}{1.031555in}}%
\pgfpathlineto{\pgfqpoint{2.444677in}{1.022513in}}%
\pgfpathlineto{\pgfqpoint{2.447209in}{1.018487in}}%
\pgfpathlineto{\pgfqpoint{2.450032in}{1.010316in}}%
\pgfpathlineto{\pgfqpoint{2.452562in}{1.016195in}}%
\pgfpathlineto{\pgfqpoint{2.455353in}{1.000302in}}%
\pgfpathlineto{\pgfqpoint{2.457917in}{1.003086in}}%
\pgfpathlineto{\pgfqpoint{2.460711in}{1.000458in}}%
\pgfpathlineto{\pgfqpoint{2.463280in}{0.999049in}}%
\pgfpathlineto{\pgfqpoint{2.465957in}{0.997182in}}%
\pgfpathlineto{\pgfqpoint{2.468635in}{0.997003in}}%
\pgfpathlineto{\pgfqpoint{2.471311in}{0.991244in}}%
\pgfpathlineto{\pgfqpoint{2.473989in}{0.991489in}}%
\pgfpathlineto{\pgfqpoint{2.476671in}{0.990988in}}%
\pgfpathlineto{\pgfqpoint{2.479420in}{0.988582in}}%
\pgfpathlineto{\pgfqpoint{2.482026in}{0.999601in}}%
\pgfpathlineto{\pgfqpoint{2.484870in}{0.994826in}}%
\pgfpathlineto{\pgfqpoint{2.487384in}{0.988307in}}%
\pgfpathlineto{\pgfqpoint{2.490183in}{0.982675in}}%
\pgfpathlineto{\pgfqpoint{2.492729in}{0.982719in}}%
\pgfpathlineto{\pgfqpoint{2.495542in}{0.982532in}}%
\pgfpathlineto{\pgfqpoint{2.498085in}{0.982569in}}%
\pgfpathlineto{\pgfqpoint{2.500801in}{0.984753in}}%
\pgfpathlineto{\pgfqpoint{2.503454in}{0.980805in}}%
\pgfpathlineto{\pgfqpoint{2.506163in}{0.979588in}}%
\pgfpathlineto{\pgfqpoint{2.508917in}{0.990002in}}%
\pgfpathlineto{\pgfqpoint{2.511478in}{0.985225in}}%
\pgfpathlineto{\pgfqpoint{2.514268in}{0.983015in}}%
\pgfpathlineto{\pgfqpoint{2.516845in}{0.983200in}}%
\pgfpathlineto{\pgfqpoint{2.519607in}{0.986109in}}%
\pgfpathlineto{\pgfqpoint{2.522197in}{0.984903in}}%
\pgfpathlineto{\pgfqpoint{2.524988in}{0.988251in}}%
\pgfpathlineto{\pgfqpoint{2.527560in}{0.985632in}}%
\pgfpathlineto{\pgfqpoint{2.530234in}{0.985846in}}%
\pgfpathlineto{\pgfqpoint{2.532917in}{0.981076in}}%
\pgfpathlineto{\pgfqpoint{2.535624in}{0.979552in}}%
\pgfpathlineto{\pgfqpoint{2.538274in}{0.974530in}}%
\pgfpathlineto{\pgfqpoint{2.540949in}{0.977533in}}%
\pgfpathlineto{\pgfqpoint{2.543765in}{0.982242in}}%
\pgfpathlineto{\pgfqpoint{2.546310in}{0.980467in}}%
\pgfpathlineto{\pgfqpoint{2.549114in}{0.980368in}}%
\pgfpathlineto{\pgfqpoint{2.551664in}{0.993964in}}%
\pgfpathlineto{\pgfqpoint{2.554493in}{1.047208in}}%
\pgfpathlineto{\pgfqpoint{2.557009in}{1.075020in}}%
\pgfpathlineto{\pgfqpoint{2.559790in}{1.080963in}}%
\pgfpathlineto{\pgfqpoint{2.562375in}{1.069562in}}%
\pgfpathlineto{\pgfqpoint{2.565045in}{1.052960in}}%
\pgfpathlineto{\pgfqpoint{2.567730in}{1.034631in}}%
\pgfpathlineto{\pgfqpoint{2.570411in}{1.031482in}}%
\pgfpathlineto{\pgfqpoint{2.573082in}{1.040396in}}%
\pgfpathlineto{\pgfqpoint{2.575779in}{1.039017in}}%
\pgfpathlineto{\pgfqpoint{2.578567in}{1.035055in}}%
\pgfpathlineto{\pgfqpoint{2.581129in}{1.058586in}}%
\pgfpathlineto{\pgfqpoint{2.583913in}{1.046777in}}%
\pgfpathlineto{\pgfqpoint{2.586484in}{1.036000in}}%
\pgfpathlineto{\pgfqpoint{2.589248in}{1.034990in}}%
\pgfpathlineto{\pgfqpoint{2.591842in}{1.021671in}}%
\pgfpathlineto{\pgfqpoint{2.594630in}{1.012962in}}%
\pgfpathlineto{\pgfqpoint{2.597196in}{1.010845in}}%
\pgfpathlineto{\pgfqpoint{2.599920in}{1.003498in}}%
\pgfpathlineto{\pgfqpoint{2.602557in}{1.001512in}}%
\pgfpathlineto{\pgfqpoint{2.605232in}{0.994553in}}%
\pgfpathlineto{\pgfqpoint{2.608004in}{0.991206in}}%
\pgfpathlineto{\pgfqpoint{2.610588in}{0.990540in}}%
\pgfpathlineto{\pgfqpoint{2.613393in}{0.983704in}}%
\pgfpathlineto{\pgfqpoint{2.615934in}{0.988349in}}%
\pgfpathlineto{\pgfqpoint{2.618773in}{0.994091in}}%
\pgfpathlineto{\pgfqpoint{2.621304in}{0.997701in}}%
\pgfpathlineto{\pgfqpoint{2.624077in}{0.997249in}}%
\pgfpathlineto{\pgfqpoint{2.626653in}{0.992259in}}%
\pgfpathlineto{\pgfqpoint{2.629340in}{0.990892in}}%
\pgfpathlineto{\pgfqpoint{2.632018in}{0.987834in}}%
\pgfpathlineto{\pgfqpoint{2.634700in}{0.990024in}}%
\pgfpathlineto{\pgfqpoint{2.637369in}{0.985944in}}%
\pgfpathlineto{\pgfqpoint{2.640053in}{0.984615in}}%
\pgfpathlineto{\pgfqpoint{2.642827in}{0.988428in}}%
\pgfpathlineto{\pgfqpoint{2.645408in}{0.986777in}}%
\pgfpathlineto{\pgfqpoint{2.648196in}{0.990248in}}%
\pgfpathlineto{\pgfqpoint{2.650767in}{0.995304in}}%
\pgfpathlineto{\pgfqpoint{2.653567in}{0.993003in}}%
\pgfpathlineto{\pgfqpoint{2.656124in}{0.983852in}}%
\pgfpathlineto{\pgfqpoint{2.658942in}{0.979165in}}%
\pgfpathlineto{\pgfqpoint{2.661481in}{0.978220in}}%
\pgfpathlineto{\pgfqpoint{2.664151in}{0.979808in}}%
\pgfpathlineto{\pgfqpoint{2.666836in}{0.979246in}}%
\pgfpathlineto{\pgfqpoint{2.669506in}{0.981451in}}%
\pgfpathlineto{\pgfqpoint{2.672301in}{0.984065in}}%
\pgfpathlineto{\pgfqpoint{2.674873in}{0.987405in}}%
\pgfpathlineto{\pgfqpoint{2.677650in}{0.991194in}}%
\pgfpathlineto{\pgfqpoint{2.680224in}{0.985627in}}%
\pgfpathlineto{\pgfqpoint{2.683009in}{0.988852in}}%
\pgfpathlineto{\pgfqpoint{2.685586in}{0.981893in}}%
\pgfpathlineto{\pgfqpoint{2.688328in}{0.977999in}}%
\pgfpathlineto{\pgfqpoint{2.690940in}{0.977250in}}%
\pgfpathlineto{\pgfqpoint{2.693611in}{0.984858in}}%
\pgfpathlineto{\pgfqpoint{2.696293in}{0.981685in}}%
\pgfpathlineto{\pgfqpoint{2.698968in}{0.976890in}}%
\pgfpathlineto{\pgfqpoint{2.701657in}{0.980200in}}%
\pgfpathlineto{\pgfqpoint{2.704326in}{0.978644in}}%
\pgfpathlineto{\pgfqpoint{2.707125in}{0.978898in}}%
\pgfpathlineto{\pgfqpoint{2.709683in}{0.979436in}}%
\pgfpathlineto{\pgfqpoint{2.712477in}{0.975412in}}%
\pgfpathlineto{\pgfqpoint{2.715036in}{0.984964in}}%
\pgfpathlineto{\pgfqpoint{2.717773in}{0.978131in}}%
\pgfpathlineto{\pgfqpoint{2.720404in}{0.976086in}}%
\pgfpathlineto{\pgfqpoint{2.723211in}{0.978289in}}%
\pgfpathlineto{\pgfqpoint{2.725760in}{0.977735in}}%
\pgfpathlineto{\pgfqpoint{2.728439in}{0.979474in}}%
\pgfpathlineto{\pgfqpoint{2.731119in}{0.977896in}}%
\pgfpathlineto{\pgfqpoint{2.733798in}{0.972242in}}%
\pgfpathlineto{\pgfqpoint{2.736476in}{0.969594in}}%
\pgfpathlineto{\pgfqpoint{2.739155in}{0.969594in}}%
\pgfpathlineto{\pgfqpoint{2.741928in}{0.969594in}}%
\pgfpathlineto{\pgfqpoint{2.744510in}{0.973833in}}%
\pgfpathlineto{\pgfqpoint{2.747260in}{0.971430in}}%
\pgfpathlineto{\pgfqpoint{2.749868in}{0.976148in}}%
\pgfpathlineto{\pgfqpoint{2.752614in}{0.974610in}}%
\pgfpathlineto{\pgfqpoint{2.755224in}{0.974307in}}%
\pgfpathlineto{\pgfqpoint{2.758028in}{0.974960in}}%
\pgfpathlineto{\pgfqpoint{2.760581in}{0.973124in}}%
\pgfpathlineto{\pgfqpoint{2.763253in}{0.971161in}}%
\pgfpathlineto{\pgfqpoint{2.765935in}{0.969986in}}%
\pgfpathlineto{\pgfqpoint{2.768617in}{0.971232in}}%
\pgfpathlineto{\pgfqpoint{2.771373in}{0.970004in}}%
\pgfpathlineto{\pgfqpoint{2.773972in}{0.971763in}}%
\pgfpathlineto{\pgfqpoint{2.776767in}{0.969594in}}%
\pgfpathlineto{\pgfqpoint{2.779330in}{0.971598in}}%
\pgfpathlineto{\pgfqpoint{2.782113in}{0.974206in}}%
\pgfpathlineto{\pgfqpoint{2.784687in}{0.979305in}}%
\pgfpathlineto{\pgfqpoint{2.787468in}{0.971644in}}%
\pgfpathlineto{\pgfqpoint{2.790044in}{0.972558in}}%
\pgfpathlineto{\pgfqpoint{2.792721in}{0.974401in}}%
\pgfpathlineto{\pgfqpoint{2.795398in}{0.971208in}}%
\pgfpathlineto{\pgfqpoint{2.798070in}{0.978782in}}%
\pgfpathlineto{\pgfqpoint{2.800756in}{0.978472in}}%
\pgfpathlineto{\pgfqpoint{2.803435in}{0.976515in}}%
\pgfpathlineto{\pgfqpoint{2.806175in}{0.974636in}}%
\pgfpathlineto{\pgfqpoint{2.808792in}{0.984284in}}%
\pgfpathlineto{\pgfqpoint{2.811597in}{0.983189in}}%
\pgfpathlineto{\pgfqpoint{2.814141in}{0.983482in}}%
\pgfpathlineto{\pgfqpoint{2.816867in}{0.983779in}}%
\pgfpathlineto{\pgfqpoint{2.819506in}{0.981003in}}%
\pgfpathlineto{\pgfqpoint{2.822303in}{0.980911in}}%
\pgfpathlineto{\pgfqpoint{2.824851in}{0.978868in}}%
\pgfpathlineto{\pgfqpoint{2.827567in}{0.982568in}}%
\pgfpathlineto{\pgfqpoint{2.830219in}{0.979004in}}%
\pgfpathlineto{\pgfqpoint{2.832894in}{0.981741in}}%
\pgfpathlineto{\pgfqpoint{2.835698in}{0.977041in}}%
\pgfpathlineto{\pgfqpoint{2.838254in}{0.976376in}}%
\pgfpathlineto{\pgfqpoint{2.841055in}{0.977455in}}%
\pgfpathlineto{\pgfqpoint{2.843611in}{0.976978in}}%
\pgfpathlineto{\pgfqpoint{2.846408in}{0.971455in}}%
\pgfpathlineto{\pgfqpoint{2.848960in}{0.978213in}}%
\pgfpathlineto{\pgfqpoint{2.851793in}{0.985329in}}%
\pgfpathlineto{\pgfqpoint{2.854325in}{0.983805in}}%
\pgfpathlineto{\pgfqpoint{2.857003in}{0.982725in}}%
\pgfpathlineto{\pgfqpoint{2.859668in}{0.983156in}}%
\pgfpathlineto{\pgfqpoint{2.862402in}{0.981245in}}%
\pgfpathlineto{\pgfqpoint{2.865031in}{0.979659in}}%
\pgfpathlineto{\pgfqpoint{2.867713in}{0.980287in}}%
\pgfpathlineto{\pgfqpoint{2.870475in}{0.983348in}}%
\pgfpathlineto{\pgfqpoint{2.873074in}{0.981332in}}%
\pgfpathlineto{\pgfqpoint{2.875882in}{0.976801in}}%
\pgfpathlineto{\pgfqpoint{2.878431in}{0.974192in}}%
\pgfpathlineto{\pgfqpoint{2.881254in}{0.977866in}}%
\pgfpathlineto{\pgfqpoint{2.883780in}{0.989578in}}%
\pgfpathlineto{\pgfqpoint{2.886578in}{0.978911in}}%
\pgfpathlineto{\pgfqpoint{2.889145in}{0.973187in}}%
\pgfpathlineto{\pgfqpoint{2.891809in}{0.973064in}}%
\pgfpathlineto{\pgfqpoint{2.894487in}{0.973582in}}%
\pgfpathlineto{\pgfqpoint{2.897179in}{0.972016in}}%
\pgfpathlineto{\pgfqpoint{2.899858in}{0.973525in}}%
\pgfpathlineto{\pgfqpoint{2.902535in}{0.975811in}}%
\pgfpathlineto{\pgfqpoint{2.905341in}{0.980142in}}%
\pgfpathlineto{\pgfqpoint{2.907882in}{0.982335in}}%
\pgfpathlineto{\pgfqpoint{2.910631in}{0.982814in}}%
\pgfpathlineto{\pgfqpoint{2.913243in}{0.972638in}}%
\pgfpathlineto{\pgfqpoint{2.916061in}{0.978390in}}%
\pgfpathlineto{\pgfqpoint{2.918606in}{0.984181in}}%
\pgfpathlineto{\pgfqpoint{2.921363in}{0.981266in}}%
\pgfpathlineto{\pgfqpoint{2.923963in}{0.980071in}}%
\pgfpathlineto{\pgfqpoint{2.926655in}{0.984272in}}%
\pgfpathlineto{\pgfqpoint{2.929321in}{0.978338in}}%
\pgfpathlineto{\pgfqpoint{2.932033in}{0.977104in}}%
\pgfpathlineto{\pgfqpoint{2.934759in}{0.981743in}}%
\pgfpathlineto{\pgfqpoint{2.937352in}{0.982217in}}%
\pgfpathlineto{\pgfqpoint{2.940120in}{0.976547in}}%
\pgfpathlineto{\pgfqpoint{2.942711in}{0.981727in}}%
\pgfpathlineto{\pgfqpoint{2.945461in}{0.982397in}}%
\pgfpathlineto{\pgfqpoint{2.948068in}{0.976942in}}%
\pgfpathlineto{\pgfqpoint{2.950884in}{0.984747in}}%
\pgfpathlineto{\pgfqpoint{2.953422in}{0.976602in}}%
\pgfpathlineto{\pgfqpoint{2.956103in}{0.977837in}}%
\pgfpathlineto{\pgfqpoint{2.958782in}{0.976689in}}%
\pgfpathlineto{\pgfqpoint{2.961460in}{0.981234in}}%
\pgfpathlineto{\pgfqpoint{2.964127in}{0.976177in}}%
\pgfpathlineto{\pgfqpoint{2.966812in}{0.979306in}}%
\pgfpathlineto{\pgfqpoint{2.969599in}{0.977943in}}%
\pgfpathlineto{\pgfqpoint{2.972177in}{0.981082in}}%
\pgfpathlineto{\pgfqpoint{2.974972in}{0.981439in}}%
\pgfpathlineto{\pgfqpoint{2.977517in}{0.977565in}}%
\pgfpathlineto{\pgfqpoint{2.980341in}{0.975711in}}%
\pgfpathlineto{\pgfqpoint{2.982885in}{0.978870in}}%
\pgfpathlineto{\pgfqpoint{2.985666in}{0.975231in}}%
\pgfpathlineto{\pgfqpoint{2.988238in}{0.978079in}}%
\pgfpathlineto{\pgfqpoint{2.990978in}{0.979678in}}%
\pgfpathlineto{\pgfqpoint{2.993595in}{0.978918in}}%
\pgfpathlineto{\pgfqpoint{2.996300in}{0.980726in}}%
\pgfpathlineto{\pgfqpoint{2.999103in}{0.981011in}}%
\pgfpathlineto{\pgfqpoint{3.001635in}{0.974519in}}%
\pgfpathlineto{\pgfqpoint{3.004419in}{0.976465in}}%
\pgfpathlineto{\pgfqpoint{3.006993in}{0.974566in}}%
\pgfpathlineto{\pgfqpoint{3.009784in}{0.977159in}}%
\pgfpathlineto{\pgfqpoint{3.012351in}{0.977712in}}%
\pgfpathlineto{\pgfqpoint{3.015097in}{0.973319in}}%
\pgfpathlineto{\pgfqpoint{3.017707in}{0.978678in}}%
\pgfpathlineto{\pgfqpoint{3.020382in}{0.976033in}}%
\pgfpathlineto{\pgfqpoint{3.023058in}{0.980946in}}%
\pgfpathlineto{\pgfqpoint{3.025803in}{0.982714in}}%
\pgfpathlineto{\pgfqpoint{3.028412in}{0.978719in}}%
\pgfpathlineto{\pgfqpoint{3.031091in}{0.979605in}}%
\pgfpathlineto{\pgfqpoint{3.033921in}{0.980072in}}%
\pgfpathlineto{\pgfqpoint{3.036456in}{0.979887in}}%
\pgfpathlineto{\pgfqpoint{3.039262in}{0.982174in}}%
\pgfpathlineto{\pgfqpoint{3.041813in}{0.971977in}}%
\pgfpathlineto{\pgfqpoint{3.044568in}{0.972400in}}%
\pgfpathlineto{\pgfqpoint{3.047157in}{0.972343in}}%
\pgfpathlineto{\pgfqpoint{3.049988in}{0.977030in}}%
\pgfpathlineto{\pgfqpoint{3.052526in}{0.973462in}}%
\pgfpathlineto{\pgfqpoint{3.055202in}{0.981689in}}%
\pgfpathlineto{\pgfqpoint{3.057884in}{0.980445in}}%
\pgfpathlineto{\pgfqpoint{3.060561in}{0.982764in}}%
\pgfpathlineto{\pgfqpoint{3.063230in}{0.969700in}}%
\pgfpathlineto{\pgfqpoint{3.065916in}{0.975488in}}%
\pgfpathlineto{\pgfqpoint{3.068709in}{0.977037in}}%
\pgfpathlineto{\pgfqpoint{3.071266in}{0.969647in}}%
\pgfpathlineto{\pgfqpoint{3.074056in}{0.972428in}}%
\pgfpathlineto{\pgfqpoint{3.076631in}{0.973948in}}%
\pgfpathlineto{\pgfqpoint{3.079381in}{0.977724in}}%
\pgfpathlineto{\pgfqpoint{3.081990in}{0.979892in}}%
\pgfpathlineto{\pgfqpoint{3.084671in}{0.980302in}}%
\pgfpathlineto{\pgfqpoint{3.087343in}{0.975311in}}%
\pgfpathlineto{\pgfqpoint{3.090023in}{0.975625in}}%
\pgfpathlineto{\pgfqpoint{3.092699in}{0.977304in}}%
\pgfpathlineto{\pgfqpoint{3.095388in}{0.972454in}}%
\pgfpathlineto{\pgfqpoint{3.098163in}{0.969594in}}%
\pgfpathlineto{\pgfqpoint{3.100737in}{0.969594in}}%
\pgfpathlineto{\pgfqpoint{3.103508in}{0.969594in}}%
\pgfpathlineto{\pgfqpoint{3.106094in}{0.971197in}}%
\pgfpathlineto{\pgfqpoint{3.108896in}{0.974295in}}%
\pgfpathlineto{\pgfqpoint{3.111451in}{0.985499in}}%
\pgfpathlineto{\pgfqpoint{3.114242in}{0.981426in}}%
\pgfpathlineto{\pgfqpoint{3.116807in}{0.983492in}}%
\pgfpathlineto{\pgfqpoint{3.119487in}{0.972579in}}%
\pgfpathlineto{\pgfqpoint{3.122163in}{0.980422in}}%
\pgfpathlineto{\pgfqpoint{3.124842in}{0.981457in}}%
\pgfpathlineto{\pgfqpoint{3.127512in}{0.978802in}}%
\pgfpathlineto{\pgfqpoint{3.130199in}{0.975491in}}%
\pgfpathlineto{\pgfqpoint{3.132946in}{0.977304in}}%
\pgfpathlineto{\pgfqpoint{3.135550in}{0.977117in}}%
\pgfpathlineto{\pgfqpoint{3.138375in}{0.976382in}}%
\pgfpathlineto{\pgfqpoint{3.140913in}{0.975454in}}%
\pgfpathlineto{\pgfqpoint{3.143740in}{0.969594in}}%
\pgfpathlineto{\pgfqpoint{3.146271in}{0.969594in}}%
\pgfpathlineto{\pgfqpoint{3.149057in}{0.969594in}}%
\pgfpathlineto{\pgfqpoint{3.151612in}{0.969594in}}%
\pgfpathlineto{\pgfqpoint{3.154327in}{0.969594in}}%
\pgfpathlineto{\pgfqpoint{3.156981in}{0.971452in}}%
\pgfpathlineto{\pgfqpoint{3.159675in}{0.969594in}}%
\pgfpathlineto{\pgfqpoint{3.162474in}{0.969594in}}%
\pgfpathlineto{\pgfqpoint{3.165019in}{0.969594in}}%
\pgfpathlineto{\pgfqpoint{3.167776in}{0.969594in}}%
\pgfpathlineto{\pgfqpoint{3.170375in}{0.969594in}}%
\pgfpathlineto{\pgfqpoint{3.173142in}{0.969594in}}%
\pgfpathlineto{\pgfqpoint{3.175724in}{0.969594in}}%
\pgfpathlineto{\pgfqpoint{3.178525in}{0.969594in}}%
\pgfpathlineto{\pgfqpoint{3.181089in}{0.969594in}}%
\pgfpathlineto{\pgfqpoint{3.183760in}{0.969594in}}%
\pgfpathlineto{\pgfqpoint{3.186440in}{0.969594in}}%
\pgfpathlineto{\pgfqpoint{3.189117in}{0.970833in}}%
\pgfpathlineto{\pgfqpoint{3.191796in}{0.971192in}}%
\pgfpathlineto{\pgfqpoint{3.194508in}{0.969594in}}%
\pgfpathlineto{\pgfqpoint{3.197226in}{0.971653in}}%
\pgfpathlineto{\pgfqpoint{3.199823in}{0.969594in}}%
\pgfpathlineto{\pgfqpoint{3.202562in}{0.969594in}}%
\pgfpathlineto{\pgfqpoint{3.205195in}{0.969594in}}%
\pgfpathlineto{\pgfqpoint{3.207984in}{0.969594in}}%
\pgfpathlineto{\pgfqpoint{3.210545in}{0.976998in}}%
\pgfpathlineto{\pgfqpoint{3.213342in}{0.973088in}}%
\pgfpathlineto{\pgfqpoint{3.215908in}{0.973608in}}%
\pgfpathlineto{\pgfqpoint{3.218586in}{0.973565in}}%
\pgfpathlineto{\pgfqpoint{3.221255in}{0.975400in}}%
\pgfpathlineto{\pgfqpoint{3.223942in}{0.978633in}}%
\pgfpathlineto{\pgfqpoint{3.226609in}{0.978768in}}%
\pgfpathlineto{\pgfqpoint{3.229310in}{0.979010in}}%
\pgfpathlineto{\pgfqpoint{3.232069in}{0.979423in}}%
\pgfpathlineto{\pgfqpoint{3.234658in}{0.976957in}}%
\pgfpathlineto{\pgfqpoint{3.237411in}{0.977991in}}%
\pgfpathlineto{\pgfqpoint{3.240010in}{0.975497in}}%
\pgfpathlineto{\pgfqpoint{3.242807in}{0.976383in}}%
\pgfpathlineto{\pgfqpoint{3.245363in}{0.980489in}}%
\pgfpathlineto{\pgfqpoint{3.248049in}{0.979826in}}%
\pgfpathlineto{\pgfqpoint{3.250716in}{0.980158in}}%
\pgfpathlineto{\pgfqpoint{3.253404in}{0.980246in}}%
\pgfpathlineto{\pgfqpoint{3.256083in}{0.985090in}}%
\pgfpathlineto{\pgfqpoint{3.258784in}{0.983202in}}%
\pgfpathlineto{\pgfqpoint{3.261594in}{0.985091in}}%
\pgfpathlineto{\pgfqpoint{3.264119in}{0.984100in}}%
\pgfpathlineto{\pgfqpoint{3.266849in}{0.981885in}}%
\pgfpathlineto{\pgfqpoint{3.269478in}{0.984638in}}%
\pgfpathlineto{\pgfqpoint{3.272254in}{0.985247in}}%
\pgfpathlineto{\pgfqpoint{3.274831in}{0.984609in}}%
\pgfpathlineto{\pgfqpoint{3.277603in}{0.984803in}}%
\pgfpathlineto{\pgfqpoint{3.280189in}{0.983975in}}%
\pgfpathlineto{\pgfqpoint{3.282870in}{0.979514in}}%
\pgfpathlineto{\pgfqpoint{3.285534in}{0.984924in}}%
\pgfpathlineto{\pgfqpoint{3.288225in}{0.982630in}}%
\pgfpathlineto{\pgfqpoint{3.290890in}{0.978481in}}%
\pgfpathlineto{\pgfqpoint{3.293574in}{0.986633in}}%
\pgfpathlineto{\pgfqpoint{3.296376in}{0.990458in}}%
\pgfpathlineto{\pgfqpoint{3.298937in}{0.990978in}}%
\pgfpathlineto{\pgfqpoint{3.301719in}{0.984253in}}%
\pgfpathlineto{\pgfqpoint{3.304295in}{0.990560in}}%
\pgfpathlineto{\pgfqpoint{3.307104in}{0.987735in}}%
\pgfpathlineto{\pgfqpoint{3.309652in}{0.986980in}}%
\pgfpathlineto{\pgfqpoint{3.312480in}{0.985252in}}%
\pgfpathlineto{\pgfqpoint{3.315008in}{0.988767in}}%
\pgfpathlineto{\pgfqpoint{3.317688in}{0.980934in}}%
\pgfpathlineto{\pgfqpoint{3.320366in}{0.982268in}}%
\pgfpathlineto{\pgfqpoint{3.323049in}{0.981161in}}%
\pgfpathlineto{\pgfqpoint{3.325860in}{0.983959in}}%
\pgfpathlineto{\pgfqpoint{3.328401in}{0.980947in}}%
\pgfpathlineto{\pgfqpoint{3.331183in}{0.983221in}}%
\pgfpathlineto{\pgfqpoint{3.333758in}{0.979722in}}%
\pgfpathlineto{\pgfqpoint{3.336541in}{0.981985in}}%
\pgfpathlineto{\pgfqpoint{3.339101in}{0.983504in}}%
\pgfpathlineto{\pgfqpoint{3.341893in}{0.984907in}}%
\pgfpathlineto{\pgfqpoint{3.344468in}{0.985899in}}%
\pgfpathlineto{\pgfqpoint{3.347139in}{0.982823in}}%
\pgfpathlineto{\pgfqpoint{3.349828in}{0.991357in}}%
\pgfpathlineto{\pgfqpoint{3.352505in}{0.985160in}}%
\pgfpathlineto{\pgfqpoint{3.355177in}{0.980283in}}%
\pgfpathlineto{\pgfqpoint{3.357862in}{0.979079in}}%
\pgfpathlineto{\pgfqpoint{3.360620in}{0.979442in}}%
\pgfpathlineto{\pgfqpoint{3.363221in}{0.987873in}}%
\pgfpathlineto{\pgfqpoint{3.365996in}{0.979690in}}%
\pgfpathlineto{\pgfqpoint{3.368577in}{0.985741in}}%
\pgfpathlineto{\pgfqpoint{3.371357in}{0.981756in}}%
\pgfpathlineto{\pgfqpoint{3.373921in}{0.984533in}}%
\pgfpathlineto{\pgfqpoint{3.376735in}{0.983328in}}%
\pgfpathlineto{\pgfqpoint{3.379290in}{0.984204in}}%
\pgfpathlineto{\pgfqpoint{3.381959in}{0.987411in}}%
\pgfpathlineto{\pgfqpoint{3.384647in}{0.988740in}}%
\pgfpathlineto{\pgfqpoint{3.387309in}{0.987707in}}%
\pgfpathlineto{\pgfqpoint{3.390102in}{0.986233in}}%
\pgfpathlineto{\pgfqpoint{3.392681in}{0.985538in}}%
\pgfpathlineto{\pgfqpoint{3.395461in}{0.985367in}}%
\pgfpathlineto{\pgfqpoint{3.398037in}{0.984378in}}%
\pgfpathlineto{\pgfqpoint{3.400783in}{0.982868in}}%
\pgfpathlineto{\pgfqpoint{3.403394in}{0.984679in}}%
\pgfpathlineto{\pgfqpoint{3.406202in}{0.982706in}}%
\pgfpathlineto{\pgfqpoint{3.408752in}{0.986514in}}%
\pgfpathlineto{\pgfqpoint{3.411431in}{0.981555in}}%
\pgfpathlineto{\pgfqpoint{3.414109in}{0.983807in}}%
\pgfpathlineto{\pgfqpoint{3.416780in}{0.986533in}}%
\pgfpathlineto{\pgfqpoint{3.419455in}{0.981899in}}%
\pgfpathlineto{\pgfqpoint{3.422142in}{0.985367in}}%
\pgfpathlineto{\pgfqpoint{3.424887in}{0.985228in}}%
\pgfpathlineto{\pgfqpoint{3.427501in}{0.983573in}}%
\pgfpathlineto{\pgfqpoint{3.430313in}{0.983241in}}%
\pgfpathlineto{\pgfqpoint{3.432851in}{0.979087in}}%
\pgfpathlineto{\pgfqpoint{3.435635in}{0.983059in}}%
\pgfpathlineto{\pgfqpoint{3.438210in}{0.981206in}}%
\pgfpathlineto{\pgfqpoint{3.440996in}{0.980158in}}%
\pgfpathlineto{\pgfqpoint{3.443574in}{0.980127in}}%
\pgfpathlineto{\pgfqpoint{3.446257in}{0.980893in}}%
\pgfpathlineto{\pgfqpoint{3.448926in}{0.979479in}}%
\pgfpathlineto{\pgfqpoint{3.451597in}{0.982649in}}%
\pgfpathlineto{\pgfqpoint{3.454285in}{0.982879in}}%
\pgfpathlineto{\pgfqpoint{3.456960in}{0.979679in}}%
\pgfpathlineto{\pgfqpoint{3.459695in}{0.979717in}}%
\pgfpathlineto{\pgfqpoint{3.462321in}{0.978062in}}%
\pgfpathlineto{\pgfqpoint{3.465072in}{0.978514in}}%
\pgfpathlineto{\pgfqpoint{3.467678in}{0.978577in}}%
\pgfpathlineto{\pgfqpoint{3.470466in}{0.975107in}}%
\pgfpathlineto{\pgfqpoint{3.473021in}{0.982158in}}%
\pgfpathlineto{\pgfqpoint{3.475821in}{0.986529in}}%
\pgfpathlineto{\pgfqpoint{3.478378in}{0.986346in}}%
\pgfpathlineto{\pgfqpoint{3.481072in}{0.993886in}}%
\pgfpathlineto{\pgfqpoint{3.483744in}{0.994472in}}%
\pgfpathlineto{\pgfqpoint{3.486442in}{0.991932in}}%
\pgfpathlineto{\pgfqpoint{3.489223in}{0.989153in}}%
\pgfpathlineto{\pgfqpoint{3.491783in}{0.988720in}}%
\pgfpathlineto{\pgfqpoint{3.494581in}{0.989411in}}%
\pgfpathlineto{\pgfqpoint{3.497139in}{0.985896in}}%
\pgfpathlineto{\pgfqpoint{3.499909in}{0.978005in}}%
\pgfpathlineto{\pgfqpoint{3.502488in}{0.977304in}}%
\pgfpathlineto{\pgfqpoint{3.505262in}{0.981889in}}%
\pgfpathlineto{\pgfqpoint{3.507840in}{0.979636in}}%
\pgfpathlineto{\pgfqpoint{3.510533in}{0.983133in}}%
\pgfpathlineto{\pgfqpoint{3.513209in}{0.981179in}}%
\pgfpathlineto{\pgfqpoint{3.515884in}{0.978225in}}%
\pgfpathlineto{\pgfqpoint{3.518565in}{0.981876in}}%
\pgfpathlineto{\pgfqpoint{3.521244in}{0.982451in}}%
\pgfpathlineto{\pgfqpoint{3.524041in}{0.981750in}}%
\pgfpathlineto{\pgfqpoint{3.526601in}{0.980809in}}%
\pgfpathlineto{\pgfqpoint{3.529327in}{0.975703in}}%
\pgfpathlineto{\pgfqpoint{3.531955in}{0.980190in}}%
\pgfpathlineto{\pgfqpoint{3.534783in}{0.976040in}}%
\pgfpathlineto{\pgfqpoint{3.537309in}{0.981831in}}%
\pgfpathlineto{\pgfqpoint{3.540093in}{0.980789in}}%
\pgfpathlineto{\pgfqpoint{3.542656in}{0.980690in}}%
\pgfpathlineto{\pgfqpoint{3.545349in}{0.982904in}}%
\pgfpathlineto{\pgfqpoint{3.548029in}{0.983292in}}%
\pgfpathlineto{\pgfqpoint{3.550713in}{0.989746in}}%
\pgfpathlineto{\pgfqpoint{3.553498in}{0.987454in}}%
\pgfpathlineto{\pgfqpoint{3.556061in}{0.984879in}}%
\pgfpathlineto{\pgfqpoint{3.558853in}{0.977852in}}%
\pgfpathlineto{\pgfqpoint{3.561420in}{0.980731in}}%
\pgfpathlineto{\pgfqpoint{3.564188in}{0.978826in}}%
\pgfpathlineto{\pgfqpoint{3.566774in}{0.979674in}}%
\pgfpathlineto{\pgfqpoint{3.569584in}{0.977928in}}%
\pgfpathlineto{\pgfqpoint{3.572126in}{0.981179in}}%
\pgfpathlineto{\pgfqpoint{3.574814in}{0.978186in}}%
\pgfpathlineto{\pgfqpoint{3.577487in}{0.978883in}}%
\pgfpathlineto{\pgfqpoint{3.580191in}{0.978951in}}%
\pgfpathlineto{\pgfqpoint{3.582851in}{0.976157in}}%
\pgfpathlineto{\pgfqpoint{3.585532in}{0.980260in}}%
\pgfpathlineto{\pgfqpoint{3.588258in}{0.976202in}}%
\pgfpathlineto{\pgfqpoint{3.590883in}{0.976965in}}%
\pgfpathlineto{\pgfqpoint{3.593620in}{0.979067in}}%
\pgfpathlineto{\pgfqpoint{3.596240in}{0.978771in}}%
\pgfpathlineto{\pgfqpoint{3.598998in}{0.979335in}}%
\pgfpathlineto{\pgfqpoint{3.601590in}{0.983873in}}%
\pgfpathlineto{\pgfqpoint{3.604387in}{0.978935in}}%
\pgfpathlineto{\pgfqpoint{3.606951in}{0.982647in}}%
\pgfpathlineto{\pgfqpoint{3.609632in}{0.975005in}}%
\pgfpathlineto{\pgfqpoint{3.612311in}{0.972114in}}%
\pgfpathlineto{\pgfqpoint{3.614982in}{0.970792in}}%
\pgfpathlineto{\pgfqpoint{3.617667in}{0.976514in}}%
\pgfpathlineto{\pgfqpoint{3.620345in}{0.977868in}}%
\pgfpathlineto{\pgfqpoint{3.623165in}{0.976799in}}%
\pgfpathlineto{\pgfqpoint{3.625689in}{0.976626in}}%
\pgfpathlineto{\pgfqpoint{3.628460in}{0.979528in}}%
\pgfpathlineto{\pgfqpoint{3.631058in}{0.975632in}}%
\pgfpathlineto{\pgfqpoint{3.633858in}{0.979930in}}%
\pgfpathlineto{\pgfqpoint{3.636413in}{0.979224in}}%
\pgfpathlineto{\pgfqpoint{3.639207in}{0.976181in}}%
\pgfpathlineto{\pgfqpoint{3.641773in}{0.976872in}}%
\pgfpathlineto{\pgfqpoint{3.644452in}{0.969594in}}%
\pgfpathlineto{\pgfqpoint{3.647130in}{0.969594in}}%
\pgfpathlineto{\pgfqpoint{3.649837in}{0.969594in}}%
\pgfpathlineto{\pgfqpoint{3.652628in}{0.974533in}}%
\pgfpathlineto{\pgfqpoint{3.655165in}{0.985959in}}%
\pgfpathlineto{\pgfqpoint{3.657917in}{1.023230in}}%
\pgfpathlineto{\pgfqpoint{3.660515in}{1.059289in}}%
\pgfpathlineto{\pgfqpoint{3.663276in}{1.095364in}}%
\pgfpathlineto{\pgfqpoint{3.665864in}{1.092293in}}%
\pgfpathlineto{\pgfqpoint{3.668665in}{1.074635in}}%
\pgfpathlineto{\pgfqpoint{3.671232in}{1.064564in}}%
\pgfpathlineto{\pgfqpoint{3.673911in}{1.057755in}}%
\pgfpathlineto{\pgfqpoint{3.676591in}{1.036843in}}%
\pgfpathlineto{\pgfqpoint{3.679273in}{1.028304in}}%
\pgfpathlineto{\pgfqpoint{3.681948in}{1.024897in}}%
\pgfpathlineto{\pgfqpoint{3.684620in}{1.025047in}}%
\pgfpathlineto{\pgfqpoint{3.687442in}{1.022054in}}%
\pgfpathlineto{\pgfqpoint{3.689983in}{1.020975in}}%
\pgfpathlineto{\pgfqpoint{3.692765in}{1.015172in}}%
\pgfpathlineto{\pgfqpoint{3.695331in}{1.011038in}}%
\pgfpathlineto{\pgfqpoint{3.698125in}{1.008184in}}%
\pgfpathlineto{\pgfqpoint{3.700684in}{1.005263in}}%
\pgfpathlineto{\pgfqpoint{3.703460in}{1.002974in}}%
\pgfpathlineto{\pgfqpoint{3.706053in}{0.997664in}}%
\pgfpathlineto{\pgfqpoint{3.708729in}{0.997277in}}%
\pgfpathlineto{\pgfqpoint{3.711410in}{0.998537in}}%
\pgfpathlineto{\pgfqpoint{3.714086in}{0.993399in}}%
\pgfpathlineto{\pgfqpoint{3.716875in}{0.988950in}}%
\pgfpathlineto{\pgfqpoint{3.719446in}{0.988922in}}%
\pgfpathlineto{\pgfqpoint{3.722228in}{0.978356in}}%
\pgfpathlineto{\pgfqpoint{3.724804in}{0.982562in}}%
\pgfpathlineto{\pgfqpoint{3.727581in}{0.989642in}}%
\pgfpathlineto{\pgfqpoint{3.730158in}{0.988575in}}%
\pgfpathlineto{\pgfqpoint{3.732950in}{0.990548in}}%
\pgfpathlineto{\pgfqpoint{3.735509in}{0.984606in}}%
\pgfpathlineto{\pgfqpoint{3.738194in}{0.985948in}}%
\pgfpathlineto{\pgfqpoint{3.740874in}{0.989128in}}%
\pgfpathlineto{\pgfqpoint{3.743548in}{0.986635in}}%
\pgfpathlineto{\pgfqpoint{3.746229in}{0.990301in}}%
\pgfpathlineto{\pgfqpoint{3.748903in}{0.985666in}}%
\pgfpathlineto{\pgfqpoint{3.751728in}{0.987335in}}%
\pgfpathlineto{\pgfqpoint{3.754265in}{0.981888in}}%
\pgfpathlineto{\pgfqpoint{3.757065in}{0.981093in}}%
\pgfpathlineto{\pgfqpoint{3.759622in}{0.983576in}}%
\pgfpathlineto{\pgfqpoint{3.762389in}{0.984552in}}%
\pgfpathlineto{\pgfqpoint{3.764966in}{0.976474in}}%
\pgfpathlineto{\pgfqpoint{3.767782in}{0.976954in}}%
\pgfpathlineto{\pgfqpoint{3.770323in}{0.977241in}}%
\pgfpathlineto{\pgfqpoint{3.773014in}{1.016717in}}%
\pgfpathlineto{\pgfqpoint{3.775691in}{1.010572in}}%
\pgfpathlineto{\pgfqpoint{3.778370in}{0.996335in}}%
\pgfpathlineto{\pgfqpoint{3.781046in}{0.990858in}}%
\pgfpathlineto{\pgfqpoint{3.783725in}{0.985817in}}%
\pgfpathlineto{\pgfqpoint{3.786504in}{0.982393in}}%
\pgfpathlineto{\pgfqpoint{3.789084in}{0.984912in}}%
\pgfpathlineto{\pgfqpoint{3.791897in}{0.980594in}}%
\pgfpathlineto{\pgfqpoint{3.794435in}{0.981382in}}%
\pgfpathlineto{\pgfqpoint{3.797265in}{0.981201in}}%
\pgfpathlineto{\pgfqpoint{3.799797in}{0.976938in}}%
\pgfpathlineto{\pgfqpoint{3.802569in}{0.980964in}}%
\pgfpathlineto{\pgfqpoint{3.805145in}{0.983960in}}%
\pgfpathlineto{\pgfqpoint{3.807832in}{0.985290in}}%
\pgfpathlineto{\pgfqpoint{3.810510in}{0.981977in}}%
\pgfpathlineto{\pgfqpoint{3.813172in}{0.986696in}}%
\pgfpathlineto{\pgfqpoint{3.815983in}{0.986090in}}%
\pgfpathlineto{\pgfqpoint{3.818546in}{0.991156in}}%
\pgfpathlineto{\pgfqpoint{3.821315in}{0.983232in}}%
\pgfpathlineto{\pgfqpoint{3.823903in}{0.978332in}}%
\pgfpathlineto{\pgfqpoint{3.826679in}{0.977137in}}%
\pgfpathlineto{\pgfqpoint{3.829252in}{0.977414in}}%
\pgfpathlineto{\pgfqpoint{3.832053in}{0.976715in}}%
\pgfpathlineto{\pgfqpoint{3.834616in}{0.979009in}}%
\pgfpathlineto{\pgfqpoint{3.837286in}{0.979931in}}%
\pgfpathlineto{\pgfqpoint{3.839960in}{0.973366in}}%
\pgfpathlineto{\pgfqpoint{3.842641in}{0.973974in}}%
\pgfpathlineto{\pgfqpoint{3.845329in}{0.978967in}}%
\pgfpathlineto{\pgfqpoint{3.848005in}{0.980854in}}%
\pgfpathlineto{\pgfqpoint{3.850814in}{0.991355in}}%
\pgfpathlineto{\pgfqpoint{3.853358in}{0.983937in}}%
\pgfpathlineto{\pgfqpoint{3.856100in}{0.983706in}}%
\pgfpathlineto{\pgfqpoint{3.858720in}{0.986247in}}%
\pgfpathlineto{\pgfqpoint{3.861561in}{0.985723in}}%
\pgfpathlineto{\pgfqpoint{3.864073in}{0.982703in}}%
\pgfpathlineto{\pgfqpoint{3.866815in}{0.986191in}}%
\pgfpathlineto{\pgfqpoint{3.869435in}{0.990944in}}%
\pgfpathlineto{\pgfqpoint{3.872114in}{0.991226in}}%
\pgfpathlineto{\pgfqpoint{3.874790in}{0.987646in}}%
\pgfpathlineto{\pgfqpoint{3.877466in}{0.981786in}}%
\pgfpathlineto{\pgfqpoint{3.880237in}{0.986299in}}%
\pgfpathlineto{\pgfqpoint{3.882850in}{0.981706in}}%
\pgfpathlineto{\pgfqpoint{3.885621in}{0.985542in}}%
\pgfpathlineto{\pgfqpoint{3.888188in}{0.983764in}}%
\pgfpathlineto{\pgfqpoint{3.890926in}{1.011452in}}%
\pgfpathlineto{\pgfqpoint{3.893541in}{1.060835in}}%
\pgfpathlineto{\pgfqpoint{3.896345in}{1.092382in}}%
\pgfpathlineto{\pgfqpoint{3.898891in}{1.075995in}}%
\pgfpathlineto{\pgfqpoint{3.901573in}{1.068540in}}%
\pgfpathlineto{\pgfqpoint{3.904252in}{1.048714in}}%
\pgfpathlineto{\pgfqpoint{3.906918in}{1.039079in}}%
\pgfpathlineto{\pgfqpoint{3.909602in}{1.024953in}}%
\pgfpathlineto{\pgfqpoint{3.912296in}{1.021028in}}%
\pgfpathlineto{\pgfqpoint{3.915107in}{1.023465in}}%
\pgfpathlineto{\pgfqpoint{3.917646in}{1.024791in}}%
\pgfpathlineto{\pgfqpoint{3.920412in}{1.020792in}}%
\pgfpathlineto{\pgfqpoint{3.923005in}{1.016113in}}%
\pgfpathlineto{\pgfqpoint{3.925778in}{1.013935in}}%
\pgfpathlineto{\pgfqpoint{3.928347in}{1.005160in}}%
\pgfpathlineto{\pgfqpoint{3.931202in}{1.004591in}}%
\pgfpathlineto{\pgfqpoint{3.933714in}{0.996219in}}%
\pgfpathlineto{\pgfqpoint{3.936395in}{0.998432in}}%
\pgfpathlineto{\pgfqpoint{3.939075in}{0.995350in}}%
\pgfpathlineto{\pgfqpoint{3.941778in}{0.986801in}}%
\pgfpathlineto{\pgfqpoint{3.944431in}{0.986793in}}%
\pgfpathlineto{\pgfqpoint{3.947101in}{0.986584in}}%
\pgfpathlineto{\pgfqpoint{3.949894in}{0.988658in}}%
\pgfpathlineto{\pgfqpoint{3.952464in}{0.985823in}}%
\pgfpathlineto{\pgfqpoint{3.955211in}{0.987604in}}%
\pgfpathlineto{\pgfqpoint{3.957823in}{0.989950in}}%
\pgfpathlineto{\pgfqpoint{3.960635in}{0.990391in}}%
\pgfpathlineto{\pgfqpoint{3.963176in}{0.986911in}}%
\pgfpathlineto{\pgfqpoint{3.966013in}{0.987693in}}%
\pgfpathlineto{\pgfqpoint{3.968523in}{0.986642in}}%
\pgfpathlineto{\pgfqpoint{3.971250in}{0.983119in}}%
\pgfpathlineto{\pgfqpoint{3.973885in}{0.984859in}}%
\pgfpathlineto{\pgfqpoint{3.976563in}{0.984244in}}%
\pgfpathlineto{\pgfqpoint{3.979389in}{0.990034in}}%
\pgfpathlineto{\pgfqpoint{3.981929in}{0.985471in}}%
\pgfpathlineto{\pgfqpoint{3.984714in}{0.986073in}}%
\pgfpathlineto{\pgfqpoint{3.987270in}{0.986839in}}%
\pgfpathlineto{\pgfqpoint{3.990055in}{0.986733in}}%
\pgfpathlineto{\pgfqpoint{3.992642in}{0.990046in}}%
\pgfpathlineto{\pgfqpoint{3.995417in}{0.984348in}}%
\pgfpathlineto{\pgfqpoint{3.997990in}{0.984916in}}%
\pgfpathlineto{\pgfqpoint{4.000674in}{0.990820in}}%
\pgfpathlineto{\pgfqpoint{4.003348in}{0.989752in}}%
\pgfpathlineto{\pgfqpoint{4.006034in}{0.988544in}}%
\pgfpathlineto{\pgfqpoint{4.008699in}{0.991666in}}%
\pgfpathlineto{\pgfqpoint{4.011394in}{0.985019in}}%
\pgfpathlineto{\pgfqpoint{4.014186in}{0.987402in}}%
\pgfpathlineto{\pgfqpoint{4.016744in}{0.988025in}}%
\pgfpathlineto{\pgfqpoint{4.019518in}{0.991206in}}%
\pgfpathlineto{\pgfqpoint{4.022097in}{0.986749in}}%
\pgfpathlineto{\pgfqpoint{4.024868in}{0.993262in}}%
\pgfpathlineto{\pgfqpoint{4.027447in}{0.985910in}}%
\pgfpathlineto{\pgfqpoint{4.030229in}{0.990313in}}%
\pgfpathlineto{\pgfqpoint{4.032817in}{0.985705in}}%
\pgfpathlineto{\pgfqpoint{4.035492in}{0.981617in}}%
\pgfpathlineto{\pgfqpoint{4.038174in}{0.986164in}}%
\pgfpathlineto{\pgfqpoint{4.040852in}{0.985263in}}%
\pgfpathlineto{\pgfqpoint{4.043667in}{1.001230in}}%
\pgfpathlineto{\pgfqpoint{4.046210in}{1.015246in}}%
\pgfpathlineto{\pgfqpoint{4.049006in}{1.012822in}}%
\pgfpathlineto{\pgfqpoint{4.051557in}{1.007223in}}%
\pgfpathlineto{\pgfqpoint{4.054326in}{1.002058in}}%
\pgfpathlineto{\pgfqpoint{4.056911in}{0.991473in}}%
\pgfpathlineto{\pgfqpoint{4.059702in}{0.989497in}}%
\pgfpathlineto{\pgfqpoint{4.062266in}{0.990349in}}%
\pgfpathlineto{\pgfqpoint{4.064957in}{0.989099in}}%
\pgfpathlineto{\pgfqpoint{4.067636in}{0.997130in}}%
\pgfpathlineto{\pgfqpoint{4.070313in}{0.981703in}}%
\pgfpathlineto{\pgfqpoint{4.072985in}{0.988138in}}%
\pgfpathlineto{\pgfqpoint{4.075705in}{0.987268in}}%
\pgfpathlineto{\pgfqpoint{4.078471in}{0.984781in}}%
\pgfpathlineto{\pgfqpoint{4.081018in}{0.980849in}}%
\pgfpathlineto{\pgfqpoint{4.083870in}{0.981592in}}%
\pgfpathlineto{\pgfqpoint{4.086385in}{0.983779in}}%
\pgfpathlineto{\pgfqpoint{4.089159in}{0.986195in}}%
\pgfpathlineto{\pgfqpoint{4.091729in}{0.987531in}}%
\pgfpathlineto{\pgfqpoint{4.094527in}{0.987886in}}%
\pgfpathlineto{\pgfqpoint{4.097092in}{0.988849in}}%
\pgfpathlineto{\pgfqpoint{4.099777in}{0.990859in}}%
\pgfpathlineto{\pgfqpoint{4.102456in}{0.991149in}}%
\pgfpathlineto{\pgfqpoint{4.105185in}{0.990735in}}%
\pgfpathlineto{\pgfqpoint{4.107814in}{0.987894in}}%
\pgfpathlineto{\pgfqpoint{4.110488in}{0.989634in}}%
\pgfpathlineto{\pgfqpoint{4.113252in}{0.987578in}}%
\pgfpathlineto{\pgfqpoint{4.115844in}{0.991515in}}%
\pgfpathlineto{\pgfqpoint{4.118554in}{0.986320in}}%
\pgfpathlineto{\pgfqpoint{4.121205in}{0.985729in}}%
\pgfpathlineto{\pgfqpoint{4.124019in}{0.983283in}}%
\pgfpathlineto{\pgfqpoint{4.126553in}{0.980491in}}%
\pgfpathlineto{\pgfqpoint{4.129349in}{0.980825in}}%
\pgfpathlineto{\pgfqpoint{4.131920in}{0.985970in}}%
\pgfpathlineto{\pgfqpoint{4.134615in}{0.983905in}}%
\pgfpathlineto{\pgfqpoint{4.137272in}{0.983989in}}%
\pgfpathlineto{\pgfqpoint{4.139963in}{0.982505in}}%
\pgfpathlineto{\pgfqpoint{4.142713in}{0.989205in}}%
\pgfpathlineto{\pgfqpoint{4.145310in}{0.984324in}}%
\pgfpathlineto{\pgfqpoint{4.148082in}{0.986898in}}%
\pgfpathlineto{\pgfqpoint{4.150665in}{0.983407in}}%
\pgfpathlineto{\pgfqpoint{4.153423in}{0.984425in}}%
\pgfpathlineto{\pgfqpoint{4.156016in}{0.985369in}}%
\pgfpathlineto{\pgfqpoint{4.158806in}{0.983964in}}%
\pgfpathlineto{\pgfqpoint{4.161380in}{0.983793in}}%
\pgfpathlineto{\pgfqpoint{4.164059in}{0.978436in}}%
\pgfpathlineto{\pgfqpoint{4.166737in}{0.982260in}}%
\pgfpathlineto{\pgfqpoint{4.169415in}{0.979023in}}%
\pgfpathlineto{\pgfqpoint{4.172093in}{0.981365in}}%
\pgfpathlineto{\pgfqpoint{4.174770in}{0.983538in}}%
\pgfpathlineto{\pgfqpoint{4.177593in}{0.986683in}}%
\pgfpathlineto{\pgfqpoint{4.180129in}{0.987643in}}%
\pgfpathlineto{\pgfqpoint{4.182899in}{0.986504in}}%
\pgfpathlineto{\pgfqpoint{4.185481in}{0.984393in}}%
\pgfpathlineto{\pgfqpoint{4.188318in}{0.984478in}}%
\pgfpathlineto{\pgfqpoint{4.190842in}{0.979995in}}%
\pgfpathlineto{\pgfqpoint{4.193638in}{0.983080in}}%
\pgfpathlineto{\pgfqpoint{4.196186in}{0.984336in}}%
\pgfpathlineto{\pgfqpoint{4.198878in}{0.983667in}}%
\pgfpathlineto{\pgfqpoint{4.201542in}{0.984118in}}%
\pgfpathlineto{\pgfqpoint{4.204240in}{0.979745in}}%
\pgfpathlineto{\pgfqpoint{4.207076in}{0.973974in}}%
\pgfpathlineto{\pgfqpoint{4.209597in}{0.969815in}}%
\pgfpathlineto{\pgfqpoint{4.212383in}{0.974360in}}%
\pgfpathlineto{\pgfqpoint{4.214948in}{0.981160in}}%
\pgfpathlineto{\pgfqpoint{4.217694in}{0.982077in}}%
\pgfpathlineto{\pgfqpoint{4.220304in}{0.988756in}}%
\pgfpathlineto{\pgfqpoint{4.223082in}{1.000976in}}%
\pgfpathlineto{\pgfqpoint{4.225654in}{0.989225in}}%
\pgfpathlineto{\pgfqpoint{4.228331in}{0.994436in}}%
\pgfpathlineto{\pgfqpoint{4.231013in}{0.985092in}}%
\pgfpathlineto{\pgfqpoint{4.233691in}{0.984286in}}%
\pgfpathlineto{\pgfqpoint{4.236375in}{0.983747in}}%
\pgfpathlineto{\pgfqpoint{4.239084in}{0.981982in}}%
\pgfpathlineto{\pgfqpoint{4.241900in}{0.988191in}}%
\pgfpathlineto{\pgfqpoint{4.244394in}{0.983233in}}%
\pgfpathlineto{\pgfqpoint{4.247225in}{0.984144in}}%
\pgfpathlineto{\pgfqpoint{4.249767in}{0.981082in}}%
\pgfpathlineto{\pgfqpoint{4.252581in}{0.983539in}}%
\pgfpathlineto{\pgfqpoint{4.255120in}{0.987801in}}%
\pgfpathlineto{\pgfqpoint{4.257958in}{0.982063in}}%
\pgfpathlineto{\pgfqpoint{4.260477in}{0.981627in}}%
\pgfpathlineto{\pgfqpoint{4.263157in}{0.981484in}}%
\pgfpathlineto{\pgfqpoint{4.265824in}{0.978323in}}%
\pgfpathlineto{\pgfqpoint{4.268590in}{0.982895in}}%
\pgfpathlineto{\pgfqpoint{4.271187in}{0.981901in}}%
\pgfpathlineto{\pgfqpoint{4.273874in}{0.979352in}}%
\pgfpathlineto{\pgfqpoint{4.276635in}{0.982926in}}%
\pgfpathlineto{\pgfqpoint{4.279212in}{0.984028in}}%
\pgfpathlineto{\pgfqpoint{4.282000in}{0.982186in}}%
\pgfpathlineto{\pgfqpoint{4.284586in}{0.981265in}}%
\pgfpathlineto{\pgfqpoint{4.287399in}{0.980739in}}%
\pgfpathlineto{\pgfqpoint{4.289936in}{0.982971in}}%
\pgfpathlineto{\pgfqpoint{4.292786in}{0.982101in}}%
\pgfpathlineto{\pgfqpoint{4.295299in}{0.984107in}}%
\pgfpathlineto{\pgfqpoint{4.297977in}{0.975662in}}%
\pgfpathlineto{\pgfqpoint{4.300656in}{0.978851in}}%
\pgfpathlineto{\pgfqpoint{4.303357in}{0.976959in}}%
\pgfpathlineto{\pgfqpoint{4.306118in}{0.977451in}}%
\pgfpathlineto{\pgfqpoint{4.308691in}{0.985194in}}%
\pgfpathlineto{\pgfqpoint{4.311494in}{0.981728in}}%
\pgfpathlineto{\pgfqpoint{4.314032in}{0.978349in}}%
\pgfpathlineto{\pgfqpoint{4.316856in}{0.978229in}}%
\pgfpathlineto{\pgfqpoint{4.319405in}{0.979475in}}%
\pgfpathlineto{\pgfqpoint{4.322181in}{0.977108in}}%
\pgfpathlineto{\pgfqpoint{4.324760in}{0.972618in}}%
\pgfpathlineto{\pgfqpoint{4.327440in}{0.972241in}}%
\pgfpathlineto{\pgfqpoint{4.330118in}{0.975442in}}%
\pgfpathlineto{\pgfqpoint{4.332796in}{0.981427in}}%
\pgfpathlineto{\pgfqpoint{4.335463in}{0.972511in}}%
\pgfpathlineto{\pgfqpoint{4.338154in}{0.974485in}}%
\pgfpathlineto{\pgfqpoint{4.340976in}{0.973672in}}%
\pgfpathlineto{\pgfqpoint{4.343510in}{0.971760in}}%
\pgfpathlineto{\pgfqpoint{4.346263in}{0.973761in}}%
\pgfpathlineto{\pgfqpoint{4.348868in}{0.972897in}}%
\pgfpathlineto{\pgfqpoint{4.351645in}{0.974628in}}%
\pgfpathlineto{\pgfqpoint{4.354224in}{0.977587in}}%
\pgfpathlineto{\pgfqpoint{4.357014in}{0.976686in}}%
\pgfpathlineto{\pgfqpoint{4.359582in}{0.974276in}}%
\pgfpathlineto{\pgfqpoint{4.362270in}{0.978286in}}%
\pgfpathlineto{\pgfqpoint{4.364936in}{0.983388in}}%
\pgfpathlineto{\pgfqpoint{4.367646in}{0.980371in}}%
\pgfpathlineto{\pgfqpoint{4.370437in}{1.008417in}}%
\pgfpathlineto{\pgfqpoint{4.372976in}{1.016707in}}%
\pgfpathlineto{\pgfqpoint{4.375761in}{0.999427in}}%
\pgfpathlineto{\pgfqpoint{4.378329in}{0.994028in}}%
\pgfpathlineto{\pgfqpoint{4.381097in}{0.987362in}}%
\pgfpathlineto{\pgfqpoint{4.383674in}{0.986654in}}%
\pgfpathlineto{\pgfqpoint{4.386431in}{0.984879in}}%
\pgfpathlineto{\pgfqpoint{4.389044in}{0.989066in}}%
\pgfpathlineto{\pgfqpoint{4.391721in}{0.990737in}}%
\pgfpathlineto{\pgfqpoint{4.394400in}{0.994925in}}%
\pgfpathlineto{\pgfqpoint{4.397076in}{0.984914in}}%
\pgfpathlineto{\pgfqpoint{4.399745in}{0.979047in}}%
\pgfpathlineto{\pgfqpoint{4.402468in}{0.979603in}}%
\pgfpathlineto{\pgfqpoint{4.405234in}{0.978044in}}%
\pgfpathlineto{\pgfqpoint{4.407788in}{0.975925in}}%
\pgfpathlineto{\pgfqpoint{4.410587in}{0.979941in}}%
\pgfpathlineto{\pgfqpoint{4.413149in}{0.978392in}}%
\pgfpathlineto{\pgfqpoint{4.415932in}{0.976499in}}%
\pgfpathlineto{\pgfqpoint{4.418506in}{0.970223in}}%
\pgfpathlineto{\pgfqpoint{4.421292in}{0.973088in}}%
\pgfpathlineto{\pgfqpoint{4.423863in}{0.969594in}}%
\pgfpathlineto{\pgfqpoint{4.426534in}{0.975034in}}%
\pgfpathlineto{\pgfqpoint{4.429220in}{0.970938in}}%
\pgfpathlineto{\pgfqpoint{4.431901in}{0.976240in}}%
\pgfpathlineto{\pgfqpoint{4.434569in}{0.974338in}}%
\pgfpathlineto{\pgfqpoint{4.437253in}{0.978271in}}%
\pgfpathlineto{\pgfqpoint{4.440041in}{0.982905in}}%
\pgfpathlineto{\pgfqpoint{4.442611in}{0.983447in}}%
\pgfpathlineto{\pgfqpoint{4.445423in}{0.987089in}}%
\pgfpathlineto{\pgfqpoint{4.447965in}{0.982300in}}%
\pgfpathlineto{\pgfqpoint{4.450767in}{0.980935in}}%
\pgfpathlineto{\pgfqpoint{4.453312in}{0.983882in}}%
\pgfpathlineto{\pgfqpoint{4.456138in}{0.989159in}}%
\pgfpathlineto{\pgfqpoint{4.458681in}{1.013055in}}%
\pgfpathlineto{\pgfqpoint{4.461367in}{1.030705in}}%
\pgfpathlineto{\pgfqpoint{4.464029in}{1.028951in}}%
\pgfpathlineto{\pgfqpoint{4.466717in}{1.034104in}}%
\pgfpathlineto{\pgfqpoint{4.469492in}{1.025260in}}%
\pgfpathlineto{\pgfqpoint{4.472059in}{1.021541in}}%
\pgfpathlineto{\pgfqpoint{4.474861in}{1.017871in}}%
\pgfpathlineto{\pgfqpoint{4.477430in}{1.011830in}}%
\pgfpathlineto{\pgfqpoint{4.480201in}{1.009242in}}%
\pgfpathlineto{\pgfqpoint{4.482778in}{1.006894in}}%
\pgfpathlineto{\pgfqpoint{4.485581in}{1.010982in}}%
\pgfpathlineto{\pgfqpoint{4.488130in}{1.011165in}}%
\pgfpathlineto{\pgfqpoint{4.490822in}{1.010388in}}%
\pgfpathlineto{\pgfqpoint{4.493492in}{1.002480in}}%
\pgfpathlineto{\pgfqpoint{4.496167in}{0.999182in}}%
\pgfpathlineto{\pgfqpoint{4.498850in}{0.999123in}}%
\pgfpathlineto{\pgfqpoint{4.501529in}{0.996613in}}%
\pgfpathlineto{\pgfqpoint{4.504305in}{0.991485in}}%
\pgfpathlineto{\pgfqpoint{4.506893in}{0.988917in}}%
\pgfpathlineto{\pgfqpoint{4.509643in}{0.984555in}}%
\pgfpathlineto{\pgfqpoint{4.512246in}{0.985188in}}%
\pgfpathlineto{\pgfqpoint{4.515080in}{0.988479in}}%
\pgfpathlineto{\pgfqpoint{4.517598in}{0.986547in}}%
\pgfpathlineto{\pgfqpoint{4.520345in}{0.992123in}}%
\pgfpathlineto{\pgfqpoint{4.522962in}{0.997785in}}%
\pgfpathlineto{\pgfqpoint{4.525640in}{0.979251in}}%
\pgfpathlineto{\pgfqpoint{4.528307in}{0.982282in}}%
\pgfpathlineto{\pgfqpoint{4.530990in}{0.989439in}}%
\pgfpathlineto{\pgfqpoint{4.533764in}{0.986004in}}%
\pgfpathlineto{\pgfqpoint{4.536400in}{0.992914in}}%
\pgfpathlineto{\pgfqpoint{4.539144in}{0.984765in}}%
\pgfpathlineto{\pgfqpoint{4.541711in}{0.980974in}}%
\pgfpathlineto{\pgfqpoint{4.544464in}{0.974222in}}%
\pgfpathlineto{\pgfqpoint{4.547064in}{0.978196in}}%
\pgfpathlineto{\pgfqpoint{4.549822in}{0.978059in}}%
\pgfpathlineto{\pgfqpoint{4.552425in}{0.978258in}}%
\pgfpathlineto{\pgfqpoint{4.555106in}{0.978524in}}%
\pgfpathlineto{\pgfqpoint{4.557777in}{0.972704in}}%
\pgfpathlineto{\pgfqpoint{4.560448in}{0.975364in}}%
\pgfpathlineto{\pgfqpoint{4.563125in}{0.974504in}}%
\pgfpathlineto{\pgfqpoint{4.565820in}{0.973900in}}%
\pgfpathlineto{\pgfqpoint{4.568612in}{0.977560in}}%
\pgfpathlineto{\pgfqpoint{4.571171in}{0.969594in}}%
\pgfpathlineto{\pgfqpoint{4.573947in}{0.969594in}}%
\pgfpathlineto{\pgfqpoint{4.576531in}{0.972941in}}%
\pgfpathlineto{\pgfqpoint{4.579305in}{0.974164in}}%
\pgfpathlineto{\pgfqpoint{4.581888in}{0.978833in}}%
\pgfpathlineto{\pgfqpoint{4.584672in}{0.977379in}}%
\pgfpathlineto{\pgfqpoint{4.587244in}{0.973340in}}%
\pgfpathlineto{\pgfqpoint{4.589920in}{0.969594in}}%
\pgfpathlineto{\pgfqpoint{4.592589in}{0.969594in}}%
\pgfpathlineto{\pgfqpoint{4.595281in}{0.972665in}}%
\pgfpathlineto{\pgfqpoint{4.597951in}{0.975620in}}%
\pgfpathlineto{\pgfqpoint{4.600633in}{0.978017in}}%
\pgfpathlineto{\pgfqpoint{4.603430in}{0.979821in}}%
\pgfpathlineto{\pgfqpoint{4.605990in}{0.982902in}}%
\pgfpathlineto{\pgfqpoint{4.608808in}{0.985329in}}%
\pgfpathlineto{\pgfqpoint{4.611350in}{0.983268in}}%
\pgfpathlineto{\pgfqpoint{4.614134in}{0.978120in}}%
\pgfpathlineto{\pgfqpoint{4.616702in}{0.983140in}}%
\pgfpathlineto{\pgfqpoint{4.619529in}{0.983063in}}%
\pgfpathlineto{\pgfqpoint{4.622056in}{0.979677in}}%
\pgfpathlineto{\pgfqpoint{4.624741in}{0.986297in}}%
\pgfpathlineto{\pgfqpoint{4.627411in}{0.985352in}}%
\pgfpathlineto{\pgfqpoint{4.630096in}{0.988503in}}%
\pgfpathlineto{\pgfqpoint{4.632902in}{0.985892in}}%
\pgfpathlineto{\pgfqpoint{4.635445in}{0.983677in}}%
\pgfpathlineto{\pgfqpoint{4.638204in}{0.982830in}}%
\pgfpathlineto{\pgfqpoint{4.640809in}{0.986410in}}%
\pgfpathlineto{\pgfqpoint{4.643628in}{0.985201in}}%
\pgfpathlineto{\pgfqpoint{4.646169in}{0.980559in}}%
\pgfpathlineto{\pgfqpoint{4.648922in}{0.981595in}}%
\pgfpathlineto{\pgfqpoint{4.651524in}{0.982586in}}%
\pgfpathlineto{\pgfqpoint{4.654203in}{0.984734in}}%
\pgfpathlineto{\pgfqpoint{4.656873in}{0.983774in}}%
\pgfpathlineto{\pgfqpoint{4.659590in}{0.982577in}}%
\pgfpathlineto{\pgfqpoint{4.662237in}{0.987897in}}%
\pgfpathlineto{\pgfqpoint{4.664923in}{0.981675in}}%
\pgfpathlineto{\pgfqpoint{4.667764in}{0.980546in}}%
\pgfpathlineto{\pgfqpoint{4.670261in}{0.981612in}}%
\pgfpathlineto{\pgfqpoint{4.673068in}{0.977552in}}%
\pgfpathlineto{\pgfqpoint{4.675619in}{0.973725in}}%
\pgfpathlineto{\pgfqpoint{4.678448in}{0.979079in}}%
\pgfpathlineto{\pgfqpoint{4.680988in}{0.975642in}}%
\pgfpathlineto{\pgfqpoint{4.683799in}{0.978372in}}%
\pgfpathlineto{\pgfqpoint{4.686337in}{0.975752in}}%
\pgfpathlineto{\pgfqpoint{4.689051in}{0.974362in}}%
\pgfpathlineto{\pgfqpoint{4.691694in}{0.978714in}}%
\pgfpathlineto{\pgfqpoint{4.694381in}{0.981181in}}%
\pgfpathlineto{\pgfqpoint{4.697170in}{0.977498in}}%
\pgfpathlineto{\pgfqpoint{4.699734in}{0.978269in}}%
\pgfpathlineto{\pgfqpoint{4.702517in}{0.976581in}}%
\pgfpathlineto{\pgfqpoint{4.705094in}{0.977316in}}%
\pgfpathlineto{\pgfqpoint{4.707824in}{0.976404in}}%
\pgfpathlineto{\pgfqpoint{4.710437in}{0.979126in}}%
\pgfpathlineto{\pgfqpoint{4.713275in}{0.981605in}}%
\pgfpathlineto{\pgfqpoint{4.715806in}{0.984829in}}%
\pgfpathlineto{\pgfqpoint{4.718486in}{0.980549in}}%
\pgfpathlineto{\pgfqpoint{4.721160in}{0.979709in}}%
\pgfpathlineto{\pgfqpoint{4.723873in}{0.984158in}}%
\pgfpathlineto{\pgfqpoint{4.726508in}{0.979914in}}%
\pgfpathlineto{\pgfqpoint{4.729233in}{0.981257in}}%
\pgfpathlineto{\pgfqpoint{4.731901in}{0.986452in}}%
\pgfpathlineto{\pgfqpoint{4.734552in}{0.985219in}}%
\pgfpathlineto{\pgfqpoint{4.737348in}{0.983172in}}%
\pgfpathlineto{\pgfqpoint{4.739912in}{0.985229in}}%
\pgfpathlineto{\pgfqpoint{4.742696in}{0.982090in}}%
\pgfpathlineto{\pgfqpoint{4.745256in}{0.986014in}}%
\pgfpathlineto{\pgfqpoint{4.748081in}{0.988055in}}%
\pgfpathlineto{\pgfqpoint{4.750627in}{0.985317in}}%
\pgfpathlineto{\pgfqpoint{4.753298in}{0.988227in}}%
\pgfpathlineto{\pgfqpoint{4.755983in}{0.984981in}}%
\pgfpathlineto{\pgfqpoint{4.758653in}{0.980528in}}%
\pgfpathlineto{\pgfqpoint{4.761337in}{0.976898in}}%
\pgfpathlineto{\pgfqpoint{4.764018in}{0.974081in}}%
\pgfpathlineto{\pgfqpoint{4.766783in}{0.974814in}}%
\pgfpathlineto{\pgfqpoint{4.769367in}{0.980157in}}%
\pgfpathlineto{\pgfqpoint{4.772198in}{0.982040in}}%
\pgfpathlineto{\pgfqpoint{4.774732in}{0.983072in}}%
\pgfpathlineto{\pgfqpoint{4.777535in}{0.981981in}}%
\pgfpathlineto{\pgfqpoint{4.780083in}{0.981027in}}%
\pgfpathlineto{\pgfqpoint{4.782872in}{0.984846in}}%
\pgfpathlineto{\pgfqpoint{4.785445in}{0.983769in}}%
\pgfpathlineto{\pgfqpoint{4.788116in}{0.984297in}}%
\pgfpathlineto{\pgfqpoint{4.790798in}{0.989782in}}%
\pgfpathlineto{\pgfqpoint{4.793512in}{0.989303in}}%
\pgfpathlineto{\pgfqpoint{4.796274in}{0.986116in}}%
\pgfpathlineto{\pgfqpoint{4.798830in}{0.984779in}}%
\pgfpathlineto{\pgfqpoint{4.801586in}{0.980940in}}%
\pgfpathlineto{\pgfqpoint{4.804193in}{0.978635in}}%
\pgfpathlineto{\pgfqpoint{4.807017in}{0.973319in}}%
\pgfpathlineto{\pgfqpoint{4.809538in}{0.977082in}}%
\pgfpathlineto{\pgfqpoint{4.812377in}{0.978662in}}%
\pgfpathlineto{\pgfqpoint{4.814907in}{0.969594in}}%
\pgfpathlineto{\pgfqpoint{4.817587in}{0.969594in}}%
\pgfpathlineto{\pgfqpoint{4.820265in}{0.969594in}}%
\pgfpathlineto{\pgfqpoint{4.822945in}{0.969594in}}%
\pgfpathlineto{\pgfqpoint{4.825619in}{0.969737in}}%
\pgfpathlineto{\pgfqpoint{4.828291in}{0.969594in}}%
\pgfpathlineto{\pgfqpoint{4.831045in}{0.973951in}}%
\pgfpathlineto{\pgfqpoint{4.833657in}{0.981201in}}%
\pgfpathlineto{\pgfqpoint{4.837992in}{0.981847in}}%
\pgfpathlineto{\pgfqpoint{4.839922in}{0.975565in}}%
\pgfpathlineto{\pgfqpoint{4.842380in}{0.973768in}}%
\pgfpathlineto{\pgfqpoint{4.844361in}{0.978591in}}%
\pgfpathlineto{\pgfqpoint{4.847127in}{0.981139in}}%
\pgfpathlineto{\pgfqpoint{4.849715in}{0.989061in}}%
\pgfpathlineto{\pgfqpoint{4.852404in}{0.997587in}}%
\pgfpathlineto{\pgfqpoint{4.855070in}{0.992898in}}%
\pgfpathlineto{\pgfqpoint{4.857807in}{0.987009in}}%
\pgfpathlineto{\pgfqpoint{4.860544in}{0.986095in}}%
\pgfpathlineto{\pgfqpoint{4.863116in}{0.990931in}}%
\pgfpathlineto{\pgfqpoint{4.865910in}{0.987541in}}%
\pgfpathlineto{\pgfqpoint{4.868474in}{0.982508in}}%
\pgfpathlineto{\pgfqpoint{4.871209in}{0.984518in}}%
\pgfpathlineto{\pgfqpoint{4.873832in}{0.980795in}}%
\pgfpathlineto{\pgfqpoint{4.876636in}{0.976844in}}%
\pgfpathlineto{\pgfqpoint{4.879180in}{0.976859in}}%
\pgfpathlineto{\pgfqpoint{4.881864in}{0.981543in}}%
\pgfpathlineto{\pgfqpoint{4.884540in}{0.973703in}}%
\pgfpathlineto{\pgfqpoint{4.887211in}{0.971963in}}%
\pgfpathlineto{\pgfqpoint{4.889902in}{0.974882in}}%
\pgfpathlineto{\pgfqpoint{4.892611in}{0.973781in}}%
\pgfpathlineto{\pgfqpoint{4.895399in}{0.976024in}}%
\pgfpathlineto{\pgfqpoint{4.897938in}{0.973885in}}%
\pgfpathlineto{\pgfqpoint{4.900712in}{0.975273in}}%
\pgfpathlineto{\pgfqpoint{4.903295in}{0.988496in}}%
\pgfpathlineto{\pgfqpoint{4.906096in}{1.036697in}}%
\pgfpathlineto{\pgfqpoint{4.908648in}{1.072388in}}%
\pgfpathlineto{\pgfqpoint{4.911435in}{1.107554in}}%
\pgfpathlineto{\pgfqpoint{4.914009in}{1.115102in}}%
\pgfpathlineto{\pgfqpoint{4.916681in}{1.097378in}}%
\pgfpathlineto{\pgfqpoint{4.919352in}{1.084789in}}%
\pgfpathlineto{\pgfqpoint{4.922041in}{1.067088in}}%
\pgfpathlineto{\pgfqpoint{4.924708in}{1.051214in}}%
\pgfpathlineto{\pgfqpoint{4.927400in}{1.046266in}}%
\pgfpathlineto{\pgfqpoint{4.930170in}{1.028121in}}%
\pgfpathlineto{\pgfqpoint{4.932742in}{1.021954in}}%
\pgfpathlineto{\pgfqpoint{4.935515in}{1.026100in}}%
\pgfpathlineto{\pgfqpoint{4.938112in}{1.016689in}}%
\pgfpathlineto{\pgfqpoint{4.940881in}{1.013911in}}%
\pgfpathlineto{\pgfqpoint{4.943466in}{1.005998in}}%
\pgfpathlineto{\pgfqpoint{4.946151in}{1.003973in}}%
\pgfpathlineto{\pgfqpoint{4.948827in}{0.996715in}}%
\pgfpathlineto{\pgfqpoint{4.951504in}{0.995009in}}%
\pgfpathlineto{\pgfqpoint{4.954182in}{0.993906in}}%
\pgfpathlineto{\pgfqpoint{4.956862in}{0.988138in}}%
\pgfpathlineto{\pgfqpoint{4.959689in}{0.987336in}}%
\pgfpathlineto{\pgfqpoint{4.962219in}{0.988189in}}%
\pgfpathlineto{\pgfqpoint{4.965002in}{0.986980in}}%
\pgfpathlineto{\pgfqpoint{4.967575in}{0.988079in}}%
\pgfpathlineto{\pgfqpoint{4.970314in}{0.986570in}}%
\pgfpathlineto{\pgfqpoint{4.972933in}{0.991096in}}%
\pgfpathlineto{\pgfqpoint{4.975703in}{0.986901in}}%
\pgfpathlineto{\pgfqpoint{4.978287in}{0.983883in}}%
\pgfpathlineto{\pgfqpoint{4.980967in}{0.983715in}}%
\pgfpathlineto{\pgfqpoint{4.983637in}{0.989089in}}%
\pgfpathlineto{\pgfqpoint{4.986325in}{0.991466in}}%
\pgfpathlineto{\pgfqpoint{4.989001in}{0.986490in}}%
\pgfpathlineto{\pgfqpoint{4.991683in}{0.980126in}}%
\pgfpathlineto{\pgfqpoint{4.994390in}{0.979347in}}%
\pgfpathlineto{\pgfqpoint{4.997028in}{0.986109in}}%
\pgfpathlineto{\pgfqpoint{4.999780in}{0.983964in}}%
\pgfpathlineto{\pgfqpoint{5.002384in}{0.988445in}}%
\pgfpathlineto{\pgfqpoint{5.005178in}{0.985675in}}%
\pgfpathlineto{\pgfqpoint{5.007751in}{0.987717in}}%
\pgfpathlineto{\pgfqpoint{5.010562in}{0.981138in}}%
\pgfpathlineto{\pgfqpoint{5.013104in}{0.974295in}}%
\pgfpathlineto{\pgfqpoint{5.015820in}{1.015374in}}%
\pgfpathlineto{\pgfqpoint{5.018466in}{1.057931in}}%
\pgfpathlineto{\pgfqpoint{5.021147in}{1.049813in}}%
\pgfpathlineto{\pgfqpoint{5.023927in}{1.039312in}}%
\pgfpathlineto{\pgfqpoint{5.026501in}{1.033537in}}%
\pgfpathlineto{\pgfqpoint{5.029275in}{1.031600in}}%
\pgfpathlineto{\pgfqpoint{5.031849in}{1.088167in}}%
\pgfpathlineto{\pgfqpoint{5.034649in}{1.151953in}}%
\pgfpathlineto{\pgfqpoint{5.037214in}{1.151376in}}%
\pgfpathlineto{\pgfqpoint{5.039962in}{1.129349in}}%
\pgfpathlineto{\pgfqpoint{5.042572in}{1.114616in}}%
\pgfpathlineto{\pgfqpoint{5.045249in}{1.096779in}}%
\pgfpathlineto{\pgfqpoint{5.047924in}{1.077732in}}%
\pgfpathlineto{\pgfqpoint{5.050606in}{1.069557in}}%
\pgfpathlineto{\pgfqpoint{5.053284in}{1.066659in}}%
\pgfpathlineto{\pgfqpoint{5.055952in}{1.062890in}}%
\pgfpathlineto{\pgfqpoint{5.058711in}{1.051836in}}%
\pgfpathlineto{\pgfqpoint{5.061315in}{1.044464in}}%
\pgfpathlineto{\pgfqpoint{5.064144in}{1.041683in}}%
\pgfpathlineto{\pgfqpoint{5.066677in}{1.033026in}}%
\pgfpathlineto{\pgfqpoint{5.069463in}{1.026211in}}%
\pgfpathlineto{\pgfqpoint{5.072030in}{1.022915in}}%
\pgfpathlineto{\pgfqpoint{5.074851in}{1.012588in}}%
\pgfpathlineto{\pgfqpoint{5.077390in}{1.005202in}}%
\pgfpathlineto{\pgfqpoint{5.080067in}{1.007750in}}%
\pgfpathlineto{\pgfqpoint{5.082746in}{1.003726in}}%
\pgfpathlineto{\pgfqpoint{5.085426in}{1.001546in}}%
\pgfpathlineto{\pgfqpoint{5.088103in}{1.003752in}}%
\pgfpathlineto{\pgfqpoint{5.090788in}{0.998653in}}%
\pgfpathlineto{\pgfqpoint{5.093579in}{0.994282in}}%
\pgfpathlineto{\pgfqpoint{5.096142in}{0.992040in}}%
\pgfpathlineto{\pgfqpoint{5.098948in}{0.990558in}}%
\pgfpathlineto{\pgfqpoint{5.101496in}{0.989169in}}%
\pgfpathlineto{\pgfqpoint{5.104312in}{0.987802in}}%
\pgfpathlineto{\pgfqpoint{5.106842in}{0.985177in}}%
\pgfpathlineto{\pgfqpoint{5.109530in}{0.991534in}}%
\pgfpathlineto{\pgfqpoint{5.112209in}{0.989433in}}%
\pgfpathlineto{\pgfqpoint{5.114887in}{0.990342in}}%
\pgfpathlineto{\pgfqpoint{5.117550in}{0.985427in}}%
\pgfpathlineto{\pgfqpoint{5.120243in}{0.988685in}}%
\pgfpathlineto{\pgfqpoint{5.123042in}{0.987304in}}%
\pgfpathlineto{\pgfqpoint{5.125599in}{0.986373in}}%
\pgfpathlineto{\pgfqpoint{5.128421in}{0.987068in}}%
\pgfpathlineto{\pgfqpoint{5.130953in}{0.984423in}}%
\pgfpathlineto{\pgfqpoint{5.133716in}{0.982833in}}%
\pgfpathlineto{\pgfqpoint{5.136311in}{0.990922in}}%
\pgfpathlineto{\pgfqpoint{5.139072in}{0.991458in}}%
\pgfpathlineto{\pgfqpoint{5.141660in}{0.985781in}}%
\pgfpathlineto{\pgfqpoint{5.144349in}{0.989959in}}%
\pgfpathlineto{\pgfqpoint{5.147029in}{0.988629in}}%
\pgfpathlineto{\pgfqpoint{5.149734in}{0.990418in}}%
\pgfpathlineto{\pgfqpoint{5.152382in}{0.983344in}}%
\pgfpathlineto{\pgfqpoint{5.155059in}{0.980580in}}%
\pgfpathlineto{\pgfqpoint{5.157815in}{0.977818in}}%
\pgfpathlineto{\pgfqpoint{5.160420in}{0.982860in}}%
\pgfpathlineto{\pgfqpoint{5.163243in}{0.980741in}}%
\pgfpathlineto{\pgfqpoint{5.165775in}{0.981964in}}%
\pgfpathlineto{\pgfqpoint{5.168591in}{0.979128in}}%
\pgfpathlineto{\pgfqpoint{5.171133in}{0.981599in}}%
\pgfpathlineto{\pgfqpoint{5.173925in}{0.977502in}}%
\pgfpathlineto{\pgfqpoint{5.176477in}{0.978804in}}%
\pgfpathlineto{\pgfqpoint{5.179188in}{0.982481in}}%
\pgfpathlineto{\pgfqpoint{5.181848in}{0.978478in}}%
\pgfpathlineto{\pgfqpoint{5.184522in}{0.977021in}}%
\pgfpathlineto{\pgfqpoint{5.187294in}{0.981322in}}%
\pgfpathlineto{\pgfqpoint{5.189880in}{0.985234in}}%
\pgfpathlineto{\pgfqpoint{5.192680in}{0.980756in}}%
\pgfpathlineto{\pgfqpoint{5.195239in}{0.980213in}}%
\pgfpathlineto{\pgfqpoint{5.198008in}{0.984001in}}%
\pgfpathlineto{\pgfqpoint{5.200594in}{0.983965in}}%
\pgfpathlineto{\pgfqpoint{5.203388in}{0.981990in}}%
\pgfpathlineto{\pgfqpoint{5.205952in}{0.978852in}}%
\pgfpathlineto{\pgfqpoint{5.208630in}{0.974695in}}%
\pgfpathlineto{\pgfqpoint{5.211299in}{0.977561in}}%
\pgfpathlineto{\pgfqpoint{5.214027in}{0.979346in}}%
\pgfpathlineto{\pgfqpoint{5.216667in}{0.980417in}}%
\pgfpathlineto{\pgfqpoint{5.219345in}{0.985384in}}%
\pgfpathlineto{\pgfqpoint{5.222151in}{0.984853in}}%
\pgfpathlineto{\pgfqpoint{5.224695in}{0.981950in}}%
\pgfpathlineto{\pgfqpoint{5.227470in}{0.991404in}}%
\pgfpathlineto{\pgfqpoint{5.230059in}{0.986680in}}%
\pgfpathlineto{\pgfqpoint{5.232855in}{0.985355in}}%
\pgfpathlineto{\pgfqpoint{5.235409in}{0.985390in}}%
\pgfpathlineto{\pgfqpoint{5.238173in}{0.985355in}}%
\pgfpathlineto{\pgfqpoint{5.240777in}{0.989902in}}%
\pgfpathlineto{\pgfqpoint{5.243445in}{0.991098in}}%
\pgfpathlineto{\pgfqpoint{5.246130in}{0.988883in}}%
\pgfpathlineto{\pgfqpoint{5.248816in}{0.988119in}}%
\pgfpathlineto{\pgfqpoint{5.251590in}{0.982930in}}%
\pgfpathlineto{\pgfqpoint{5.254236in}{0.986236in}}%
\pgfpathlineto{\pgfqpoint{5.256973in}{0.987201in}}%
\pgfpathlineto{\pgfqpoint{5.259511in}{0.986857in}}%
\pgfpathlineto{\pgfqpoint{5.262264in}{0.982786in}}%
\pgfpathlineto{\pgfqpoint{5.264876in}{0.975076in}}%
\pgfpathlineto{\pgfqpoint{5.267691in}{0.984925in}}%
\pgfpathlineto{\pgfqpoint{5.270238in}{0.983981in}}%
\pgfpathlineto{\pgfqpoint{5.272913in}{0.988052in}}%
\pgfpathlineto{\pgfqpoint{5.275589in}{0.986812in}}%
\pgfpathlineto{\pgfqpoint{5.278322in}{0.978558in}}%
\pgfpathlineto{\pgfqpoint{5.280947in}{0.970287in}}%
\pgfpathlineto{\pgfqpoint{5.283631in}{0.972225in}}%
\pgfpathlineto{\pgfqpoint{5.286436in}{0.969594in}}%
\pgfpathlineto{\pgfqpoint{5.288984in}{0.972163in}}%
\pgfpathlineto{\pgfqpoint{5.291794in}{0.976896in}}%
\pgfpathlineto{\pgfqpoint{5.294339in}{0.975122in}}%
\pgfpathlineto{\pgfqpoint{5.297140in}{0.976905in}}%
\pgfpathlineto{\pgfqpoint{5.299696in}{0.972272in}}%
\pgfpathlineto{\pgfqpoint{5.302443in}{0.979000in}}%
\pgfpathlineto{\pgfqpoint{5.305054in}{0.974667in}}%
\pgfpathlineto{\pgfqpoint{5.307731in}{0.976363in}}%
\pgfpathlineto{\pgfqpoint{5.310411in}{0.975850in}}%
\pgfpathlineto{\pgfqpoint{5.313089in}{0.976756in}}%
\pgfpathlineto{\pgfqpoint{5.315754in}{0.978440in}}%
\pgfpathlineto{\pgfqpoint{5.318430in}{0.980590in}}%
\pgfpathlineto{\pgfqpoint{5.321256in}{0.978466in}}%
\pgfpathlineto{\pgfqpoint{5.323802in}{0.978809in}}%
\pgfpathlineto{\pgfqpoint{5.326564in}{0.975978in}}%
\pgfpathlineto{\pgfqpoint{5.329159in}{0.978970in}}%
\pgfpathlineto{\pgfqpoint{5.331973in}{0.979935in}}%
\pgfpathlineto{\pgfqpoint{5.334510in}{0.983696in}}%
\pgfpathlineto{\pgfqpoint{5.337353in}{0.981724in}}%
\pgfpathlineto{\pgfqpoint{5.339872in}{0.977597in}}%
\pgfpathlineto{\pgfqpoint{5.342549in}{0.977415in}}%
\pgfpathlineto{\pgfqpoint{5.345224in}{0.970759in}}%
\pgfpathlineto{\pgfqpoint{5.347905in}{0.969594in}}%
\pgfpathlineto{\pgfqpoint{5.350723in}{0.971420in}}%
\pgfpathlineto{\pgfqpoint{5.353262in}{0.977189in}}%
\pgfpathlineto{\pgfqpoint{5.356056in}{0.978842in}}%
\pgfpathlineto{\pgfqpoint{5.358612in}{0.979772in}}%
\pgfpathlineto{\pgfqpoint{5.361370in}{0.985867in}}%
\pgfpathlineto{\pgfqpoint{5.363966in}{0.975295in}}%
\pgfpathlineto{\pgfqpoint{5.366727in}{0.971588in}}%
\pgfpathlineto{\pgfqpoint{5.369335in}{0.974848in}}%
\pgfpathlineto{\pgfqpoint{5.372013in}{0.974247in}}%
\pgfpathlineto{\pgfqpoint{5.374692in}{0.973846in}}%
\pgfpathlineto{\pgfqpoint{5.377370in}{0.977719in}}%
\pgfpathlineto{\pgfqpoint{5.380048in}{0.974395in}}%
\pgfpathlineto{\pgfqpoint{5.382725in}{0.977819in}}%
\pgfpathlineto{\pgfqpoint{5.385550in}{0.976256in}}%
\pgfpathlineto{\pgfqpoint{5.388083in}{0.978899in}}%
\pgfpathlineto{\pgfqpoint{5.390900in}{0.986903in}}%
\pgfpathlineto{\pgfqpoint{5.393441in}{0.990677in}}%
\pgfpathlineto{\pgfqpoint{5.396219in}{0.984826in}}%
\pgfpathlineto{\pgfqpoint{5.398784in}{0.982546in}}%
\pgfpathlineto{\pgfqpoint{5.401576in}{0.979044in}}%
\pgfpathlineto{\pgfqpoint{5.404154in}{0.974665in}}%
\pgfpathlineto{\pgfqpoint{5.406832in}{0.974344in}}%
\pgfpathlineto{\pgfqpoint{5.409507in}{0.979431in}}%
\pgfpathlineto{\pgfqpoint{5.412190in}{0.977081in}}%
\pgfpathlineto{\pgfqpoint{5.414954in}{0.978204in}}%
\pgfpathlineto{\pgfqpoint{5.417547in}{0.985768in}}%
\pgfpathlineto{\pgfqpoint{5.420304in}{0.981645in}}%
\pgfpathlineto{\pgfqpoint{5.422897in}{0.979192in}}%
\pgfpathlineto{\pgfqpoint{5.425661in}{0.977518in}}%
\pgfpathlineto{\pgfqpoint{5.428259in}{0.976446in}}%
\pgfpathlineto{\pgfqpoint{5.431015in}{0.975158in}}%
\pgfpathlineto{\pgfqpoint{5.433616in}{0.974781in}}%
\pgfpathlineto{\pgfqpoint{5.436295in}{0.978093in}}%
\pgfpathlineto{\pgfqpoint{5.438974in}{0.978708in}}%
\pgfpathlineto{\pgfqpoint{5.441698in}{0.977757in}}%
\pgfpathlineto{\pgfqpoint{5.444328in}{0.980470in}}%
\pgfpathlineto{\pgfqpoint{5.447021in}{0.979677in}}%
\pgfpathlineto{\pgfqpoint{5.449769in}{0.974163in}}%
\pgfpathlineto{\pgfqpoint{5.452365in}{0.977640in}}%
\pgfpathlineto{\pgfqpoint{5.455168in}{0.974448in}}%
\pgfpathlineto{\pgfqpoint{5.457721in}{0.976405in}}%
\pgfpathlineto{\pgfqpoint{5.460489in}{0.979225in}}%
\pgfpathlineto{\pgfqpoint{5.463079in}{0.976362in}}%
\pgfpathlineto{\pgfqpoint{5.465888in}{0.984500in}}%
\pgfpathlineto{\pgfqpoint{5.468425in}{0.981560in}}%
\pgfpathlineto{\pgfqpoint{5.471113in}{0.978985in}}%
\pgfpathlineto{\pgfqpoint{5.473792in}{0.980214in}}%
\pgfpathlineto{\pgfqpoint{5.476458in}{0.981151in}}%
\pgfpathlineto{\pgfqpoint{5.479152in}{0.975276in}}%
\pgfpathlineto{\pgfqpoint{5.481825in}{0.979813in}}%
\pgfpathlineto{\pgfqpoint{5.484641in}{0.986918in}}%
\pgfpathlineto{\pgfqpoint{5.487176in}{0.981453in}}%
\pgfpathlineto{\pgfqpoint{5.490000in}{0.978425in}}%
\pgfpathlineto{\pgfqpoint{5.492541in}{0.982259in}}%
\pgfpathlineto{\pgfqpoint{5.495346in}{0.982849in}}%
\pgfpathlineto{\pgfqpoint{5.497898in}{0.980370in}}%
\pgfpathlineto{\pgfqpoint{5.500687in}{0.987995in}}%
\pgfpathlineto{\pgfqpoint{5.503255in}{0.985101in}}%
\pgfpathlineto{\pgfqpoint{5.505933in}{0.987598in}}%
\pgfpathlineto{\pgfqpoint{5.508612in}{0.990620in}}%
\pgfpathlineto{\pgfqpoint{5.511290in}{0.985884in}}%
\pgfpathlineto{\pgfqpoint{5.514080in}{0.991464in}}%
\pgfpathlineto{\pgfqpoint{5.516646in}{0.991675in}}%
\pgfpathlineto{\pgfqpoint{5.519433in}{0.986326in}}%
\pgfpathlineto{\pgfqpoint{5.522003in}{0.984619in}}%
\pgfpathlineto{\pgfqpoint{5.524756in}{0.990062in}}%
\pgfpathlineto{\pgfqpoint{5.527360in}{0.984816in}}%
\pgfpathlineto{\pgfqpoint{5.530148in}{0.982999in}}%
\pgfpathlineto{\pgfqpoint{5.532717in}{0.982556in}}%
\pgfpathlineto{\pgfqpoint{5.535395in}{0.989359in}}%
\pgfpathlineto{\pgfqpoint{5.538074in}{0.985071in}}%
\pgfpathlineto{\pgfqpoint{5.540750in}{0.984327in}}%
\pgfpathlineto{\pgfqpoint{5.543421in}{0.984298in}}%
\pgfpathlineto{\pgfqpoint{5.546110in}{0.980464in}}%
\pgfpathlineto{\pgfqpoint{5.548921in}{0.983341in}}%
\pgfpathlineto{\pgfqpoint{5.551457in}{0.988046in}}%
\pgfpathlineto{\pgfqpoint{5.554198in}{1.015187in}}%
\pgfpathlineto{\pgfqpoint{5.556822in}{1.025682in}}%
\pgfpathlineto{\pgfqpoint{5.559612in}{1.040762in}}%
\pgfpathlineto{\pgfqpoint{5.562180in}{1.072047in}}%
\pgfpathlineto{\pgfqpoint{5.564940in}{1.087470in}}%
\pgfpathlineto{\pgfqpoint{5.567536in}{1.068341in}}%
\pgfpathlineto{\pgfqpoint{5.570215in}{1.052501in}}%
\pgfpathlineto{\pgfqpoint{5.572893in}{1.035525in}}%
\pgfpathlineto{\pgfqpoint{5.575596in}{1.031045in}}%
\pgfpathlineto{\pgfqpoint{5.578342in}{1.020646in}}%
\pgfpathlineto{\pgfqpoint{5.580914in}{1.018170in}}%
\pgfpathlineto{\pgfqpoint{5.583709in}{1.020518in}}%
\pgfpathlineto{\pgfqpoint{5.586269in}{1.008442in}}%
\pgfpathlineto{\pgfqpoint{5.589040in}{1.001601in}}%
\pgfpathlineto{\pgfqpoint{5.591641in}{1.002045in}}%
\pgfpathlineto{\pgfqpoint{5.594368in}{0.999180in}}%
\pgfpathlineto{\pgfqpoint{5.596999in}{0.991854in}}%
\pgfpathlineto{\pgfqpoint{5.599674in}{0.990658in}}%
\pgfpathlineto{\pgfqpoint{5.602352in}{0.993298in}}%
\pgfpathlineto{\pgfqpoint{5.605073in}{0.986663in}}%
\pgfpathlineto{\pgfqpoint{5.607698in}{0.986802in}}%
\pgfpathlineto{\pgfqpoint{5.610389in}{0.983451in}}%
\pgfpathlineto{\pgfqpoint{5.613235in}{0.982971in}}%
\pgfpathlineto{\pgfqpoint{5.615743in}{0.983613in}}%
\pgfpathlineto{\pgfqpoint{5.618526in}{0.981592in}}%
\pgfpathlineto{\pgfqpoint{5.621102in}{0.985569in}}%
\pgfpathlineto{\pgfqpoint{5.623868in}{0.976454in}}%
\pgfpathlineto{\pgfqpoint{5.626460in}{0.982034in}}%
\pgfpathlineto{\pgfqpoint{5.629232in}{0.980847in}}%
\pgfpathlineto{\pgfqpoint{5.631815in}{0.977063in}}%
\pgfpathlineto{\pgfqpoint{5.634496in}{0.977081in}}%
\pgfpathlineto{\pgfqpoint{5.637172in}{0.978804in}}%
\pgfpathlineto{\pgfqpoint{5.639852in}{0.989688in}}%
\pgfpathlineto{\pgfqpoint{5.642518in}{1.004335in}}%
\pgfpathlineto{\pgfqpoint{5.645243in}{0.997875in}}%
\pgfpathlineto{\pgfqpoint{5.648008in}{1.003353in}}%
\pgfpathlineto{\pgfqpoint{5.650563in}{0.990828in}}%
\pgfpathlineto{\pgfqpoint{5.653376in}{0.971893in}}%
\pgfpathlineto{\pgfqpoint{5.655919in}{0.969594in}}%
\pgfpathlineto{\pgfqpoint{5.658723in}{0.971024in}}%
\pgfpathlineto{\pgfqpoint{5.661273in}{0.972665in}}%
\pgfpathlineto{\pgfqpoint{5.664099in}{0.977383in}}%
\pgfpathlineto{\pgfqpoint{5.666632in}{0.978755in}}%
\pgfpathlineto{\pgfqpoint{5.669313in}{0.983586in}}%
\pgfpathlineto{\pgfqpoint{5.671991in}{0.981959in}}%
\pgfpathlineto{\pgfqpoint{5.674667in}{0.979817in}}%
\pgfpathlineto{\pgfqpoint{5.677486in}{0.977733in}}%
\pgfpathlineto{\pgfqpoint{5.680027in}{0.980777in}}%
\pgfpathlineto{\pgfqpoint{5.682836in}{0.982467in}}%
\pgfpathlineto{\pgfqpoint{5.685385in}{0.984671in}}%
\pgfpathlineto{\pgfqpoint{5.688159in}{0.984523in}}%
\pgfpathlineto{\pgfqpoint{5.690730in}{0.986836in}}%
\pgfpathlineto{\pgfqpoint{5.693473in}{0.984462in}}%
\pgfpathlineto{\pgfqpoint{5.696101in}{0.987902in}}%
\pgfpathlineto{\pgfqpoint{5.698775in}{0.980930in}}%
\pgfpathlineto{\pgfqpoint{5.701453in}{0.980124in}}%
\pgfpathlineto{\pgfqpoint{5.704130in}{0.983703in}}%
\pgfpathlineto{\pgfqpoint{5.706800in}{0.991375in}}%
\pgfpathlineto{\pgfqpoint{5.709490in}{0.994247in}}%
\pgfpathlineto{\pgfqpoint{5.712291in}{0.993090in}}%
\pgfpathlineto{\pgfqpoint{5.714834in}{0.999369in}}%
\pgfpathlineto{\pgfqpoint{5.717671in}{1.004287in}}%
\pgfpathlineto{\pgfqpoint{5.720201in}{1.000043in}}%
\pgfpathlineto{\pgfqpoint{5.722950in}{0.996679in}}%
\pgfpathlineto{\pgfqpoint{5.725548in}{0.991536in}}%
\pgfpathlineto{\pgfqpoint{5.728339in}{0.996293in}}%
\pgfpathlineto{\pgfqpoint{5.730919in}{0.996385in}}%
\pgfpathlineto{\pgfqpoint{5.733594in}{0.983621in}}%
\pgfpathlineto{\pgfqpoint{5.736276in}{0.984524in}}%
\pgfpathlineto{\pgfqpoint{5.738974in}{0.988966in}}%
\pgfpathlineto{\pgfqpoint{5.741745in}{0.991694in}}%
\pgfpathlineto{\pgfqpoint{5.744310in}{0.989825in}}%
\pgfpathlineto{\pgfqpoint{5.744310in}{0.413320in}}%
\pgfpathlineto{\pgfqpoint{5.744310in}{0.413320in}}%
\pgfpathlineto{\pgfqpoint{5.741745in}{0.413320in}}%
\pgfpathlineto{\pgfqpoint{5.738974in}{0.413320in}}%
\pgfpathlineto{\pgfqpoint{5.736276in}{0.413320in}}%
\pgfpathlineto{\pgfqpoint{5.733594in}{0.413320in}}%
\pgfpathlineto{\pgfqpoint{5.730919in}{0.413320in}}%
\pgfpathlineto{\pgfqpoint{5.728339in}{0.413320in}}%
\pgfpathlineto{\pgfqpoint{5.725548in}{0.413320in}}%
\pgfpathlineto{\pgfqpoint{5.722950in}{0.413320in}}%
\pgfpathlineto{\pgfqpoint{5.720201in}{0.413320in}}%
\pgfpathlineto{\pgfqpoint{5.717671in}{0.413320in}}%
\pgfpathlineto{\pgfqpoint{5.714834in}{0.413320in}}%
\pgfpathlineto{\pgfqpoint{5.712291in}{0.413320in}}%
\pgfpathlineto{\pgfqpoint{5.709490in}{0.413320in}}%
\pgfpathlineto{\pgfqpoint{5.706800in}{0.413320in}}%
\pgfpathlineto{\pgfqpoint{5.704130in}{0.413320in}}%
\pgfpathlineto{\pgfqpoint{5.701453in}{0.413320in}}%
\pgfpathlineto{\pgfqpoint{5.698775in}{0.413320in}}%
\pgfpathlineto{\pgfqpoint{5.696101in}{0.413320in}}%
\pgfpathlineto{\pgfqpoint{5.693473in}{0.413320in}}%
\pgfpathlineto{\pgfqpoint{5.690730in}{0.413320in}}%
\pgfpathlineto{\pgfqpoint{5.688159in}{0.413320in}}%
\pgfpathlineto{\pgfqpoint{5.685385in}{0.413320in}}%
\pgfpathlineto{\pgfqpoint{5.682836in}{0.413320in}}%
\pgfpathlineto{\pgfqpoint{5.680027in}{0.413320in}}%
\pgfpathlineto{\pgfqpoint{5.677486in}{0.413320in}}%
\pgfpathlineto{\pgfqpoint{5.674667in}{0.413320in}}%
\pgfpathlineto{\pgfqpoint{5.671991in}{0.413320in}}%
\pgfpathlineto{\pgfqpoint{5.669313in}{0.413320in}}%
\pgfpathlineto{\pgfqpoint{5.666632in}{0.413320in}}%
\pgfpathlineto{\pgfqpoint{5.664099in}{0.413320in}}%
\pgfpathlineto{\pgfqpoint{5.661273in}{0.413320in}}%
\pgfpathlineto{\pgfqpoint{5.658723in}{0.413320in}}%
\pgfpathlineto{\pgfqpoint{5.655919in}{0.413320in}}%
\pgfpathlineto{\pgfqpoint{5.653376in}{0.413320in}}%
\pgfpathlineto{\pgfqpoint{5.650563in}{0.413320in}}%
\pgfpathlineto{\pgfqpoint{5.648008in}{0.413320in}}%
\pgfpathlineto{\pgfqpoint{5.645243in}{0.413320in}}%
\pgfpathlineto{\pgfqpoint{5.642518in}{0.413320in}}%
\pgfpathlineto{\pgfqpoint{5.639852in}{0.413320in}}%
\pgfpathlineto{\pgfqpoint{5.637172in}{0.413320in}}%
\pgfpathlineto{\pgfqpoint{5.634496in}{0.413320in}}%
\pgfpathlineto{\pgfqpoint{5.631815in}{0.413320in}}%
\pgfpathlineto{\pgfqpoint{5.629232in}{0.413320in}}%
\pgfpathlineto{\pgfqpoint{5.626460in}{0.413320in}}%
\pgfpathlineto{\pgfqpoint{5.623868in}{0.413320in}}%
\pgfpathlineto{\pgfqpoint{5.621102in}{0.413320in}}%
\pgfpathlineto{\pgfqpoint{5.618526in}{0.413320in}}%
\pgfpathlineto{\pgfqpoint{5.615743in}{0.413320in}}%
\pgfpathlineto{\pgfqpoint{5.613235in}{0.413320in}}%
\pgfpathlineto{\pgfqpoint{5.610389in}{0.413320in}}%
\pgfpathlineto{\pgfqpoint{5.607698in}{0.413320in}}%
\pgfpathlineto{\pgfqpoint{5.605073in}{0.413320in}}%
\pgfpathlineto{\pgfqpoint{5.602352in}{0.413320in}}%
\pgfpathlineto{\pgfqpoint{5.599674in}{0.413320in}}%
\pgfpathlineto{\pgfqpoint{5.596999in}{0.413320in}}%
\pgfpathlineto{\pgfqpoint{5.594368in}{0.413320in}}%
\pgfpathlineto{\pgfqpoint{5.591641in}{0.413320in}}%
\pgfpathlineto{\pgfqpoint{5.589040in}{0.413320in}}%
\pgfpathlineto{\pgfqpoint{5.586269in}{0.413320in}}%
\pgfpathlineto{\pgfqpoint{5.583709in}{0.413320in}}%
\pgfpathlineto{\pgfqpoint{5.580914in}{0.413320in}}%
\pgfpathlineto{\pgfqpoint{5.578342in}{0.413320in}}%
\pgfpathlineto{\pgfqpoint{5.575596in}{0.413320in}}%
\pgfpathlineto{\pgfqpoint{5.572893in}{0.413320in}}%
\pgfpathlineto{\pgfqpoint{5.570215in}{0.413320in}}%
\pgfpathlineto{\pgfqpoint{5.567536in}{0.413320in}}%
\pgfpathlineto{\pgfqpoint{5.564940in}{0.413320in}}%
\pgfpathlineto{\pgfqpoint{5.562180in}{0.413320in}}%
\pgfpathlineto{\pgfqpoint{5.559612in}{0.413320in}}%
\pgfpathlineto{\pgfqpoint{5.556822in}{0.413320in}}%
\pgfpathlineto{\pgfqpoint{5.554198in}{0.413320in}}%
\pgfpathlineto{\pgfqpoint{5.551457in}{0.413320in}}%
\pgfpathlineto{\pgfqpoint{5.548921in}{0.413320in}}%
\pgfpathlineto{\pgfqpoint{5.546110in}{0.413320in}}%
\pgfpathlineto{\pgfqpoint{5.543421in}{0.413320in}}%
\pgfpathlineto{\pgfqpoint{5.540750in}{0.413320in}}%
\pgfpathlineto{\pgfqpoint{5.538074in}{0.413320in}}%
\pgfpathlineto{\pgfqpoint{5.535395in}{0.413320in}}%
\pgfpathlineto{\pgfqpoint{5.532717in}{0.413320in}}%
\pgfpathlineto{\pgfqpoint{5.530148in}{0.413320in}}%
\pgfpathlineto{\pgfqpoint{5.527360in}{0.413320in}}%
\pgfpathlineto{\pgfqpoint{5.524756in}{0.413320in}}%
\pgfpathlineto{\pgfqpoint{5.522003in}{0.413320in}}%
\pgfpathlineto{\pgfqpoint{5.519433in}{0.413320in}}%
\pgfpathlineto{\pgfqpoint{5.516646in}{0.413320in}}%
\pgfpathlineto{\pgfqpoint{5.514080in}{0.413320in}}%
\pgfpathlineto{\pgfqpoint{5.511290in}{0.413320in}}%
\pgfpathlineto{\pgfqpoint{5.508612in}{0.413320in}}%
\pgfpathlineto{\pgfqpoint{5.505933in}{0.413320in}}%
\pgfpathlineto{\pgfqpoint{5.503255in}{0.413320in}}%
\pgfpathlineto{\pgfqpoint{5.500687in}{0.413320in}}%
\pgfpathlineto{\pgfqpoint{5.497898in}{0.413320in}}%
\pgfpathlineto{\pgfqpoint{5.495346in}{0.413320in}}%
\pgfpathlineto{\pgfqpoint{5.492541in}{0.413320in}}%
\pgfpathlineto{\pgfqpoint{5.490000in}{0.413320in}}%
\pgfpathlineto{\pgfqpoint{5.487176in}{0.413320in}}%
\pgfpathlineto{\pgfqpoint{5.484641in}{0.413320in}}%
\pgfpathlineto{\pgfqpoint{5.481825in}{0.413320in}}%
\pgfpathlineto{\pgfqpoint{5.479152in}{0.413320in}}%
\pgfpathlineto{\pgfqpoint{5.476458in}{0.413320in}}%
\pgfpathlineto{\pgfqpoint{5.473792in}{0.413320in}}%
\pgfpathlineto{\pgfqpoint{5.471113in}{0.413320in}}%
\pgfpathlineto{\pgfqpoint{5.468425in}{0.413320in}}%
\pgfpathlineto{\pgfqpoint{5.465888in}{0.413320in}}%
\pgfpathlineto{\pgfqpoint{5.463079in}{0.413320in}}%
\pgfpathlineto{\pgfqpoint{5.460489in}{0.413320in}}%
\pgfpathlineto{\pgfqpoint{5.457721in}{0.413320in}}%
\pgfpathlineto{\pgfqpoint{5.455168in}{0.413320in}}%
\pgfpathlineto{\pgfqpoint{5.452365in}{0.413320in}}%
\pgfpathlineto{\pgfqpoint{5.449769in}{0.413320in}}%
\pgfpathlineto{\pgfqpoint{5.447021in}{0.413320in}}%
\pgfpathlineto{\pgfqpoint{5.444328in}{0.413320in}}%
\pgfpathlineto{\pgfqpoint{5.441698in}{0.413320in}}%
\pgfpathlineto{\pgfqpoint{5.438974in}{0.413320in}}%
\pgfpathlineto{\pgfqpoint{5.436295in}{0.413320in}}%
\pgfpathlineto{\pgfqpoint{5.433616in}{0.413320in}}%
\pgfpathlineto{\pgfqpoint{5.431015in}{0.413320in}}%
\pgfpathlineto{\pgfqpoint{5.428259in}{0.413320in}}%
\pgfpathlineto{\pgfqpoint{5.425661in}{0.413320in}}%
\pgfpathlineto{\pgfqpoint{5.422897in}{0.413320in}}%
\pgfpathlineto{\pgfqpoint{5.420304in}{0.413320in}}%
\pgfpathlineto{\pgfqpoint{5.417547in}{0.413320in}}%
\pgfpathlineto{\pgfqpoint{5.414954in}{0.413320in}}%
\pgfpathlineto{\pgfqpoint{5.412190in}{0.413320in}}%
\pgfpathlineto{\pgfqpoint{5.409507in}{0.413320in}}%
\pgfpathlineto{\pgfqpoint{5.406832in}{0.413320in}}%
\pgfpathlineto{\pgfqpoint{5.404154in}{0.413320in}}%
\pgfpathlineto{\pgfqpoint{5.401576in}{0.413320in}}%
\pgfpathlineto{\pgfqpoint{5.398784in}{0.413320in}}%
\pgfpathlineto{\pgfqpoint{5.396219in}{0.413320in}}%
\pgfpathlineto{\pgfqpoint{5.393441in}{0.413320in}}%
\pgfpathlineto{\pgfqpoint{5.390900in}{0.413320in}}%
\pgfpathlineto{\pgfqpoint{5.388083in}{0.413320in}}%
\pgfpathlineto{\pgfqpoint{5.385550in}{0.413320in}}%
\pgfpathlineto{\pgfqpoint{5.382725in}{0.413320in}}%
\pgfpathlineto{\pgfqpoint{5.380048in}{0.413320in}}%
\pgfpathlineto{\pgfqpoint{5.377370in}{0.413320in}}%
\pgfpathlineto{\pgfqpoint{5.374692in}{0.413320in}}%
\pgfpathlineto{\pgfqpoint{5.372013in}{0.413320in}}%
\pgfpathlineto{\pgfqpoint{5.369335in}{0.413320in}}%
\pgfpathlineto{\pgfqpoint{5.366727in}{0.413320in}}%
\pgfpathlineto{\pgfqpoint{5.363966in}{0.413320in}}%
\pgfpathlineto{\pgfqpoint{5.361370in}{0.413320in}}%
\pgfpathlineto{\pgfqpoint{5.358612in}{0.413320in}}%
\pgfpathlineto{\pgfqpoint{5.356056in}{0.413320in}}%
\pgfpathlineto{\pgfqpoint{5.353262in}{0.413320in}}%
\pgfpathlineto{\pgfqpoint{5.350723in}{0.413320in}}%
\pgfpathlineto{\pgfqpoint{5.347905in}{0.413320in}}%
\pgfpathlineto{\pgfqpoint{5.345224in}{0.413320in}}%
\pgfpathlineto{\pgfqpoint{5.342549in}{0.413320in}}%
\pgfpathlineto{\pgfqpoint{5.339872in}{0.413320in}}%
\pgfpathlineto{\pgfqpoint{5.337353in}{0.413320in}}%
\pgfpathlineto{\pgfqpoint{5.334510in}{0.413320in}}%
\pgfpathlineto{\pgfqpoint{5.331973in}{0.413320in}}%
\pgfpathlineto{\pgfqpoint{5.329159in}{0.413320in}}%
\pgfpathlineto{\pgfqpoint{5.326564in}{0.413320in}}%
\pgfpathlineto{\pgfqpoint{5.323802in}{0.413320in}}%
\pgfpathlineto{\pgfqpoint{5.321256in}{0.413320in}}%
\pgfpathlineto{\pgfqpoint{5.318430in}{0.413320in}}%
\pgfpathlineto{\pgfqpoint{5.315754in}{0.413320in}}%
\pgfpathlineto{\pgfqpoint{5.313089in}{0.413320in}}%
\pgfpathlineto{\pgfqpoint{5.310411in}{0.413320in}}%
\pgfpathlineto{\pgfqpoint{5.307731in}{0.413320in}}%
\pgfpathlineto{\pgfqpoint{5.305054in}{0.413320in}}%
\pgfpathlineto{\pgfqpoint{5.302443in}{0.413320in}}%
\pgfpathlineto{\pgfqpoint{5.299696in}{0.413320in}}%
\pgfpathlineto{\pgfqpoint{5.297140in}{0.413320in}}%
\pgfpathlineto{\pgfqpoint{5.294339in}{0.413320in}}%
\pgfpathlineto{\pgfqpoint{5.291794in}{0.413320in}}%
\pgfpathlineto{\pgfqpoint{5.288984in}{0.413320in}}%
\pgfpathlineto{\pgfqpoint{5.286436in}{0.413320in}}%
\pgfpathlineto{\pgfqpoint{5.283631in}{0.413320in}}%
\pgfpathlineto{\pgfqpoint{5.280947in}{0.413320in}}%
\pgfpathlineto{\pgfqpoint{5.278322in}{0.413320in}}%
\pgfpathlineto{\pgfqpoint{5.275589in}{0.413320in}}%
\pgfpathlineto{\pgfqpoint{5.272913in}{0.413320in}}%
\pgfpathlineto{\pgfqpoint{5.270238in}{0.413320in}}%
\pgfpathlineto{\pgfqpoint{5.267691in}{0.413320in}}%
\pgfpathlineto{\pgfqpoint{5.264876in}{0.413320in}}%
\pgfpathlineto{\pgfqpoint{5.262264in}{0.413320in}}%
\pgfpathlineto{\pgfqpoint{5.259511in}{0.413320in}}%
\pgfpathlineto{\pgfqpoint{5.256973in}{0.413320in}}%
\pgfpathlineto{\pgfqpoint{5.254236in}{0.413320in}}%
\pgfpathlineto{\pgfqpoint{5.251590in}{0.413320in}}%
\pgfpathlineto{\pgfqpoint{5.248816in}{0.413320in}}%
\pgfpathlineto{\pgfqpoint{5.246130in}{0.413320in}}%
\pgfpathlineto{\pgfqpoint{5.243445in}{0.413320in}}%
\pgfpathlineto{\pgfqpoint{5.240777in}{0.413320in}}%
\pgfpathlineto{\pgfqpoint{5.238173in}{0.413320in}}%
\pgfpathlineto{\pgfqpoint{5.235409in}{0.413320in}}%
\pgfpathlineto{\pgfqpoint{5.232855in}{0.413320in}}%
\pgfpathlineto{\pgfqpoint{5.230059in}{0.413320in}}%
\pgfpathlineto{\pgfqpoint{5.227470in}{0.413320in}}%
\pgfpathlineto{\pgfqpoint{5.224695in}{0.413320in}}%
\pgfpathlineto{\pgfqpoint{5.222151in}{0.413320in}}%
\pgfpathlineto{\pgfqpoint{5.219345in}{0.413320in}}%
\pgfpathlineto{\pgfqpoint{5.216667in}{0.413320in}}%
\pgfpathlineto{\pgfqpoint{5.214027in}{0.413320in}}%
\pgfpathlineto{\pgfqpoint{5.211299in}{0.413320in}}%
\pgfpathlineto{\pgfqpoint{5.208630in}{0.413320in}}%
\pgfpathlineto{\pgfqpoint{5.205952in}{0.413320in}}%
\pgfpathlineto{\pgfqpoint{5.203388in}{0.413320in}}%
\pgfpathlineto{\pgfqpoint{5.200594in}{0.413320in}}%
\pgfpathlineto{\pgfqpoint{5.198008in}{0.413320in}}%
\pgfpathlineto{\pgfqpoint{5.195239in}{0.413320in}}%
\pgfpathlineto{\pgfqpoint{5.192680in}{0.413320in}}%
\pgfpathlineto{\pgfqpoint{5.189880in}{0.413320in}}%
\pgfpathlineto{\pgfqpoint{5.187294in}{0.413320in}}%
\pgfpathlineto{\pgfqpoint{5.184522in}{0.413320in}}%
\pgfpathlineto{\pgfqpoint{5.181848in}{0.413320in}}%
\pgfpathlineto{\pgfqpoint{5.179188in}{0.413320in}}%
\pgfpathlineto{\pgfqpoint{5.176477in}{0.413320in}}%
\pgfpathlineto{\pgfqpoint{5.173925in}{0.413320in}}%
\pgfpathlineto{\pgfqpoint{5.171133in}{0.413320in}}%
\pgfpathlineto{\pgfqpoint{5.168591in}{0.413320in}}%
\pgfpathlineto{\pgfqpoint{5.165775in}{0.413320in}}%
\pgfpathlineto{\pgfqpoint{5.163243in}{0.413320in}}%
\pgfpathlineto{\pgfqpoint{5.160420in}{0.413320in}}%
\pgfpathlineto{\pgfqpoint{5.157815in}{0.413320in}}%
\pgfpathlineto{\pgfqpoint{5.155059in}{0.413320in}}%
\pgfpathlineto{\pgfqpoint{5.152382in}{0.413320in}}%
\pgfpathlineto{\pgfqpoint{5.149734in}{0.413320in}}%
\pgfpathlineto{\pgfqpoint{5.147029in}{0.413320in}}%
\pgfpathlineto{\pgfqpoint{5.144349in}{0.413320in}}%
\pgfpathlineto{\pgfqpoint{5.141660in}{0.413320in}}%
\pgfpathlineto{\pgfqpoint{5.139072in}{0.413320in}}%
\pgfpathlineto{\pgfqpoint{5.136311in}{0.413320in}}%
\pgfpathlineto{\pgfqpoint{5.133716in}{0.413320in}}%
\pgfpathlineto{\pgfqpoint{5.130953in}{0.413320in}}%
\pgfpathlineto{\pgfqpoint{5.128421in}{0.413320in}}%
\pgfpathlineto{\pgfqpoint{5.125599in}{0.413320in}}%
\pgfpathlineto{\pgfqpoint{5.123042in}{0.413320in}}%
\pgfpathlineto{\pgfqpoint{5.120243in}{0.413320in}}%
\pgfpathlineto{\pgfqpoint{5.117550in}{0.413320in}}%
\pgfpathlineto{\pgfqpoint{5.114887in}{0.413320in}}%
\pgfpathlineto{\pgfqpoint{5.112209in}{0.413320in}}%
\pgfpathlineto{\pgfqpoint{5.109530in}{0.413320in}}%
\pgfpathlineto{\pgfqpoint{5.106842in}{0.413320in}}%
\pgfpathlineto{\pgfqpoint{5.104312in}{0.413320in}}%
\pgfpathlineto{\pgfqpoint{5.101496in}{0.413320in}}%
\pgfpathlineto{\pgfqpoint{5.098948in}{0.413320in}}%
\pgfpathlineto{\pgfqpoint{5.096142in}{0.413320in}}%
\pgfpathlineto{\pgfqpoint{5.093579in}{0.413320in}}%
\pgfpathlineto{\pgfqpoint{5.090788in}{0.413320in}}%
\pgfpathlineto{\pgfqpoint{5.088103in}{0.413320in}}%
\pgfpathlineto{\pgfqpoint{5.085426in}{0.413320in}}%
\pgfpathlineto{\pgfqpoint{5.082746in}{0.413320in}}%
\pgfpathlineto{\pgfqpoint{5.080067in}{0.413320in}}%
\pgfpathlineto{\pgfqpoint{5.077390in}{0.413320in}}%
\pgfpathlineto{\pgfqpoint{5.074851in}{0.413320in}}%
\pgfpathlineto{\pgfqpoint{5.072030in}{0.413320in}}%
\pgfpathlineto{\pgfqpoint{5.069463in}{0.413320in}}%
\pgfpathlineto{\pgfqpoint{5.066677in}{0.413320in}}%
\pgfpathlineto{\pgfqpoint{5.064144in}{0.413320in}}%
\pgfpathlineto{\pgfqpoint{5.061315in}{0.413320in}}%
\pgfpathlineto{\pgfqpoint{5.058711in}{0.413320in}}%
\pgfpathlineto{\pgfqpoint{5.055952in}{0.413320in}}%
\pgfpathlineto{\pgfqpoint{5.053284in}{0.413320in}}%
\pgfpathlineto{\pgfqpoint{5.050606in}{0.413320in}}%
\pgfpathlineto{\pgfqpoint{5.047924in}{0.413320in}}%
\pgfpathlineto{\pgfqpoint{5.045249in}{0.413320in}}%
\pgfpathlineto{\pgfqpoint{5.042572in}{0.413320in}}%
\pgfpathlineto{\pgfqpoint{5.039962in}{0.413320in}}%
\pgfpathlineto{\pgfqpoint{5.037214in}{0.413320in}}%
\pgfpathlineto{\pgfqpoint{5.034649in}{0.413320in}}%
\pgfpathlineto{\pgfqpoint{5.031849in}{0.413320in}}%
\pgfpathlineto{\pgfqpoint{5.029275in}{0.413320in}}%
\pgfpathlineto{\pgfqpoint{5.026501in}{0.413320in}}%
\pgfpathlineto{\pgfqpoint{5.023927in}{0.413320in}}%
\pgfpathlineto{\pgfqpoint{5.021147in}{0.413320in}}%
\pgfpathlineto{\pgfqpoint{5.018466in}{0.413320in}}%
\pgfpathlineto{\pgfqpoint{5.015820in}{0.413320in}}%
\pgfpathlineto{\pgfqpoint{5.013104in}{0.413320in}}%
\pgfpathlineto{\pgfqpoint{5.010562in}{0.413320in}}%
\pgfpathlineto{\pgfqpoint{5.007751in}{0.413320in}}%
\pgfpathlineto{\pgfqpoint{5.005178in}{0.413320in}}%
\pgfpathlineto{\pgfqpoint{5.002384in}{0.413320in}}%
\pgfpathlineto{\pgfqpoint{4.999780in}{0.413320in}}%
\pgfpathlineto{\pgfqpoint{4.997028in}{0.413320in}}%
\pgfpathlineto{\pgfqpoint{4.994390in}{0.413320in}}%
\pgfpathlineto{\pgfqpoint{4.991683in}{0.413320in}}%
\pgfpathlineto{\pgfqpoint{4.989001in}{0.413320in}}%
\pgfpathlineto{\pgfqpoint{4.986325in}{0.413320in}}%
\pgfpathlineto{\pgfqpoint{4.983637in}{0.413320in}}%
\pgfpathlineto{\pgfqpoint{4.980967in}{0.413320in}}%
\pgfpathlineto{\pgfqpoint{4.978287in}{0.413320in}}%
\pgfpathlineto{\pgfqpoint{4.975703in}{0.413320in}}%
\pgfpathlineto{\pgfqpoint{4.972933in}{0.413320in}}%
\pgfpathlineto{\pgfqpoint{4.970314in}{0.413320in}}%
\pgfpathlineto{\pgfqpoint{4.967575in}{0.413320in}}%
\pgfpathlineto{\pgfqpoint{4.965002in}{0.413320in}}%
\pgfpathlineto{\pgfqpoint{4.962219in}{0.413320in}}%
\pgfpathlineto{\pgfqpoint{4.959689in}{0.413320in}}%
\pgfpathlineto{\pgfqpoint{4.956862in}{0.413320in}}%
\pgfpathlineto{\pgfqpoint{4.954182in}{0.413320in}}%
\pgfpathlineto{\pgfqpoint{4.951504in}{0.413320in}}%
\pgfpathlineto{\pgfqpoint{4.948827in}{0.413320in}}%
\pgfpathlineto{\pgfqpoint{4.946151in}{0.413320in}}%
\pgfpathlineto{\pgfqpoint{4.943466in}{0.413320in}}%
\pgfpathlineto{\pgfqpoint{4.940881in}{0.413320in}}%
\pgfpathlineto{\pgfqpoint{4.938112in}{0.413320in}}%
\pgfpathlineto{\pgfqpoint{4.935515in}{0.413320in}}%
\pgfpathlineto{\pgfqpoint{4.932742in}{0.413320in}}%
\pgfpathlineto{\pgfqpoint{4.930170in}{0.413320in}}%
\pgfpathlineto{\pgfqpoint{4.927400in}{0.413320in}}%
\pgfpathlineto{\pgfqpoint{4.924708in}{0.413320in}}%
\pgfpathlineto{\pgfqpoint{4.922041in}{0.413320in}}%
\pgfpathlineto{\pgfqpoint{4.919352in}{0.413320in}}%
\pgfpathlineto{\pgfqpoint{4.916681in}{0.413320in}}%
\pgfpathlineto{\pgfqpoint{4.914009in}{0.413320in}}%
\pgfpathlineto{\pgfqpoint{4.911435in}{0.413320in}}%
\pgfpathlineto{\pgfqpoint{4.908648in}{0.413320in}}%
\pgfpathlineto{\pgfqpoint{4.906096in}{0.413320in}}%
\pgfpathlineto{\pgfqpoint{4.903295in}{0.413320in}}%
\pgfpathlineto{\pgfqpoint{4.900712in}{0.413320in}}%
\pgfpathlineto{\pgfqpoint{4.897938in}{0.413320in}}%
\pgfpathlineto{\pgfqpoint{4.895399in}{0.413320in}}%
\pgfpathlineto{\pgfqpoint{4.892611in}{0.413320in}}%
\pgfpathlineto{\pgfqpoint{4.889902in}{0.413320in}}%
\pgfpathlineto{\pgfqpoint{4.887211in}{0.413320in}}%
\pgfpathlineto{\pgfqpoint{4.884540in}{0.413320in}}%
\pgfpathlineto{\pgfqpoint{4.881864in}{0.413320in}}%
\pgfpathlineto{\pgfqpoint{4.879180in}{0.413320in}}%
\pgfpathlineto{\pgfqpoint{4.876636in}{0.413320in}}%
\pgfpathlineto{\pgfqpoint{4.873832in}{0.413320in}}%
\pgfpathlineto{\pgfqpoint{4.871209in}{0.413320in}}%
\pgfpathlineto{\pgfqpoint{4.868474in}{0.413320in}}%
\pgfpathlineto{\pgfqpoint{4.865910in}{0.413320in}}%
\pgfpathlineto{\pgfqpoint{4.863116in}{0.413320in}}%
\pgfpathlineto{\pgfqpoint{4.860544in}{0.413320in}}%
\pgfpathlineto{\pgfqpoint{4.857807in}{0.413320in}}%
\pgfpathlineto{\pgfqpoint{4.855070in}{0.413320in}}%
\pgfpathlineto{\pgfqpoint{4.852404in}{0.413320in}}%
\pgfpathlineto{\pgfqpoint{4.849715in}{0.413320in}}%
\pgfpathlineto{\pgfqpoint{4.847127in}{0.413320in}}%
\pgfpathlineto{\pgfqpoint{4.844361in}{0.413320in}}%
\pgfpathlineto{\pgfqpoint{4.842380in}{0.413320in}}%
\pgfpathlineto{\pgfqpoint{4.839922in}{0.413320in}}%
\pgfpathlineto{\pgfqpoint{4.837992in}{0.413320in}}%
\pgfpathlineto{\pgfqpoint{4.833657in}{0.413320in}}%
\pgfpathlineto{\pgfqpoint{4.831045in}{0.413320in}}%
\pgfpathlineto{\pgfqpoint{4.828291in}{0.413320in}}%
\pgfpathlineto{\pgfqpoint{4.825619in}{0.413320in}}%
\pgfpathlineto{\pgfqpoint{4.822945in}{0.413320in}}%
\pgfpathlineto{\pgfqpoint{4.820265in}{0.413320in}}%
\pgfpathlineto{\pgfqpoint{4.817587in}{0.413320in}}%
\pgfpathlineto{\pgfqpoint{4.814907in}{0.413320in}}%
\pgfpathlineto{\pgfqpoint{4.812377in}{0.413320in}}%
\pgfpathlineto{\pgfqpoint{4.809538in}{0.413320in}}%
\pgfpathlineto{\pgfqpoint{4.807017in}{0.413320in}}%
\pgfpathlineto{\pgfqpoint{4.804193in}{0.413320in}}%
\pgfpathlineto{\pgfqpoint{4.801586in}{0.413320in}}%
\pgfpathlineto{\pgfqpoint{4.798830in}{0.413320in}}%
\pgfpathlineto{\pgfqpoint{4.796274in}{0.413320in}}%
\pgfpathlineto{\pgfqpoint{4.793512in}{0.413320in}}%
\pgfpathlineto{\pgfqpoint{4.790798in}{0.413320in}}%
\pgfpathlineto{\pgfqpoint{4.788116in}{0.413320in}}%
\pgfpathlineto{\pgfqpoint{4.785445in}{0.413320in}}%
\pgfpathlineto{\pgfqpoint{4.782872in}{0.413320in}}%
\pgfpathlineto{\pgfqpoint{4.780083in}{0.413320in}}%
\pgfpathlineto{\pgfqpoint{4.777535in}{0.413320in}}%
\pgfpathlineto{\pgfqpoint{4.774732in}{0.413320in}}%
\pgfpathlineto{\pgfqpoint{4.772198in}{0.413320in}}%
\pgfpathlineto{\pgfqpoint{4.769367in}{0.413320in}}%
\pgfpathlineto{\pgfqpoint{4.766783in}{0.413320in}}%
\pgfpathlineto{\pgfqpoint{4.764018in}{0.413320in}}%
\pgfpathlineto{\pgfqpoint{4.761337in}{0.413320in}}%
\pgfpathlineto{\pgfqpoint{4.758653in}{0.413320in}}%
\pgfpathlineto{\pgfqpoint{4.755983in}{0.413320in}}%
\pgfpathlineto{\pgfqpoint{4.753298in}{0.413320in}}%
\pgfpathlineto{\pgfqpoint{4.750627in}{0.413320in}}%
\pgfpathlineto{\pgfqpoint{4.748081in}{0.413320in}}%
\pgfpathlineto{\pgfqpoint{4.745256in}{0.413320in}}%
\pgfpathlineto{\pgfqpoint{4.742696in}{0.413320in}}%
\pgfpathlineto{\pgfqpoint{4.739912in}{0.413320in}}%
\pgfpathlineto{\pgfqpoint{4.737348in}{0.413320in}}%
\pgfpathlineto{\pgfqpoint{4.734552in}{0.413320in}}%
\pgfpathlineto{\pgfqpoint{4.731901in}{0.413320in}}%
\pgfpathlineto{\pgfqpoint{4.729233in}{0.413320in}}%
\pgfpathlineto{\pgfqpoint{4.726508in}{0.413320in}}%
\pgfpathlineto{\pgfqpoint{4.723873in}{0.413320in}}%
\pgfpathlineto{\pgfqpoint{4.721160in}{0.413320in}}%
\pgfpathlineto{\pgfqpoint{4.718486in}{0.413320in}}%
\pgfpathlineto{\pgfqpoint{4.715806in}{0.413320in}}%
\pgfpathlineto{\pgfqpoint{4.713275in}{0.413320in}}%
\pgfpathlineto{\pgfqpoint{4.710437in}{0.413320in}}%
\pgfpathlineto{\pgfqpoint{4.707824in}{0.413320in}}%
\pgfpathlineto{\pgfqpoint{4.705094in}{0.413320in}}%
\pgfpathlineto{\pgfqpoint{4.702517in}{0.413320in}}%
\pgfpathlineto{\pgfqpoint{4.699734in}{0.413320in}}%
\pgfpathlineto{\pgfqpoint{4.697170in}{0.413320in}}%
\pgfpathlineto{\pgfqpoint{4.694381in}{0.413320in}}%
\pgfpathlineto{\pgfqpoint{4.691694in}{0.413320in}}%
\pgfpathlineto{\pgfqpoint{4.689051in}{0.413320in}}%
\pgfpathlineto{\pgfqpoint{4.686337in}{0.413320in}}%
\pgfpathlineto{\pgfqpoint{4.683799in}{0.413320in}}%
\pgfpathlineto{\pgfqpoint{4.680988in}{0.413320in}}%
\pgfpathlineto{\pgfqpoint{4.678448in}{0.413320in}}%
\pgfpathlineto{\pgfqpoint{4.675619in}{0.413320in}}%
\pgfpathlineto{\pgfqpoint{4.673068in}{0.413320in}}%
\pgfpathlineto{\pgfqpoint{4.670261in}{0.413320in}}%
\pgfpathlineto{\pgfqpoint{4.667764in}{0.413320in}}%
\pgfpathlineto{\pgfqpoint{4.664923in}{0.413320in}}%
\pgfpathlineto{\pgfqpoint{4.662237in}{0.413320in}}%
\pgfpathlineto{\pgfqpoint{4.659590in}{0.413320in}}%
\pgfpathlineto{\pgfqpoint{4.656873in}{0.413320in}}%
\pgfpathlineto{\pgfqpoint{4.654203in}{0.413320in}}%
\pgfpathlineto{\pgfqpoint{4.651524in}{0.413320in}}%
\pgfpathlineto{\pgfqpoint{4.648922in}{0.413320in}}%
\pgfpathlineto{\pgfqpoint{4.646169in}{0.413320in}}%
\pgfpathlineto{\pgfqpoint{4.643628in}{0.413320in}}%
\pgfpathlineto{\pgfqpoint{4.640809in}{0.413320in}}%
\pgfpathlineto{\pgfqpoint{4.638204in}{0.413320in}}%
\pgfpathlineto{\pgfqpoint{4.635445in}{0.413320in}}%
\pgfpathlineto{\pgfqpoint{4.632902in}{0.413320in}}%
\pgfpathlineto{\pgfqpoint{4.630096in}{0.413320in}}%
\pgfpathlineto{\pgfqpoint{4.627411in}{0.413320in}}%
\pgfpathlineto{\pgfqpoint{4.624741in}{0.413320in}}%
\pgfpathlineto{\pgfqpoint{4.622056in}{0.413320in}}%
\pgfpathlineto{\pgfqpoint{4.619529in}{0.413320in}}%
\pgfpathlineto{\pgfqpoint{4.616702in}{0.413320in}}%
\pgfpathlineto{\pgfqpoint{4.614134in}{0.413320in}}%
\pgfpathlineto{\pgfqpoint{4.611350in}{0.413320in}}%
\pgfpathlineto{\pgfqpoint{4.608808in}{0.413320in}}%
\pgfpathlineto{\pgfqpoint{4.605990in}{0.413320in}}%
\pgfpathlineto{\pgfqpoint{4.603430in}{0.413320in}}%
\pgfpathlineto{\pgfqpoint{4.600633in}{0.413320in}}%
\pgfpathlineto{\pgfqpoint{4.597951in}{0.413320in}}%
\pgfpathlineto{\pgfqpoint{4.595281in}{0.413320in}}%
\pgfpathlineto{\pgfqpoint{4.592589in}{0.413320in}}%
\pgfpathlineto{\pgfqpoint{4.589920in}{0.413320in}}%
\pgfpathlineto{\pgfqpoint{4.587244in}{0.413320in}}%
\pgfpathlineto{\pgfqpoint{4.584672in}{0.413320in}}%
\pgfpathlineto{\pgfqpoint{4.581888in}{0.413320in}}%
\pgfpathlineto{\pgfqpoint{4.579305in}{0.413320in}}%
\pgfpathlineto{\pgfqpoint{4.576531in}{0.413320in}}%
\pgfpathlineto{\pgfqpoint{4.573947in}{0.413320in}}%
\pgfpathlineto{\pgfqpoint{4.571171in}{0.413320in}}%
\pgfpathlineto{\pgfqpoint{4.568612in}{0.413320in}}%
\pgfpathlineto{\pgfqpoint{4.565820in}{0.413320in}}%
\pgfpathlineto{\pgfqpoint{4.563125in}{0.413320in}}%
\pgfpathlineto{\pgfqpoint{4.560448in}{0.413320in}}%
\pgfpathlineto{\pgfqpoint{4.557777in}{0.413320in}}%
\pgfpathlineto{\pgfqpoint{4.555106in}{0.413320in}}%
\pgfpathlineto{\pgfqpoint{4.552425in}{0.413320in}}%
\pgfpathlineto{\pgfqpoint{4.549822in}{0.413320in}}%
\pgfpathlineto{\pgfqpoint{4.547064in}{0.413320in}}%
\pgfpathlineto{\pgfqpoint{4.544464in}{0.413320in}}%
\pgfpathlineto{\pgfqpoint{4.541711in}{0.413320in}}%
\pgfpathlineto{\pgfqpoint{4.539144in}{0.413320in}}%
\pgfpathlineto{\pgfqpoint{4.536400in}{0.413320in}}%
\pgfpathlineto{\pgfqpoint{4.533764in}{0.413320in}}%
\pgfpathlineto{\pgfqpoint{4.530990in}{0.413320in}}%
\pgfpathlineto{\pgfqpoint{4.528307in}{0.413320in}}%
\pgfpathlineto{\pgfqpoint{4.525640in}{0.413320in}}%
\pgfpathlineto{\pgfqpoint{4.522962in}{0.413320in}}%
\pgfpathlineto{\pgfqpoint{4.520345in}{0.413320in}}%
\pgfpathlineto{\pgfqpoint{4.517598in}{0.413320in}}%
\pgfpathlineto{\pgfqpoint{4.515080in}{0.413320in}}%
\pgfpathlineto{\pgfqpoint{4.512246in}{0.413320in}}%
\pgfpathlineto{\pgfqpoint{4.509643in}{0.413320in}}%
\pgfpathlineto{\pgfqpoint{4.506893in}{0.413320in}}%
\pgfpathlineto{\pgfqpoint{4.504305in}{0.413320in}}%
\pgfpathlineto{\pgfqpoint{4.501529in}{0.413320in}}%
\pgfpathlineto{\pgfqpoint{4.498850in}{0.413320in}}%
\pgfpathlineto{\pgfqpoint{4.496167in}{0.413320in}}%
\pgfpathlineto{\pgfqpoint{4.493492in}{0.413320in}}%
\pgfpathlineto{\pgfqpoint{4.490822in}{0.413320in}}%
\pgfpathlineto{\pgfqpoint{4.488130in}{0.413320in}}%
\pgfpathlineto{\pgfqpoint{4.485581in}{0.413320in}}%
\pgfpathlineto{\pgfqpoint{4.482778in}{0.413320in}}%
\pgfpathlineto{\pgfqpoint{4.480201in}{0.413320in}}%
\pgfpathlineto{\pgfqpoint{4.477430in}{0.413320in}}%
\pgfpathlineto{\pgfqpoint{4.474861in}{0.413320in}}%
\pgfpathlineto{\pgfqpoint{4.472059in}{0.413320in}}%
\pgfpathlineto{\pgfqpoint{4.469492in}{0.413320in}}%
\pgfpathlineto{\pgfqpoint{4.466717in}{0.413320in}}%
\pgfpathlineto{\pgfqpoint{4.464029in}{0.413320in}}%
\pgfpathlineto{\pgfqpoint{4.461367in}{0.413320in}}%
\pgfpathlineto{\pgfqpoint{4.458681in}{0.413320in}}%
\pgfpathlineto{\pgfqpoint{4.456138in}{0.413320in}}%
\pgfpathlineto{\pgfqpoint{4.453312in}{0.413320in}}%
\pgfpathlineto{\pgfqpoint{4.450767in}{0.413320in}}%
\pgfpathlineto{\pgfqpoint{4.447965in}{0.413320in}}%
\pgfpathlineto{\pgfqpoint{4.445423in}{0.413320in}}%
\pgfpathlineto{\pgfqpoint{4.442611in}{0.413320in}}%
\pgfpathlineto{\pgfqpoint{4.440041in}{0.413320in}}%
\pgfpathlineto{\pgfqpoint{4.437253in}{0.413320in}}%
\pgfpathlineto{\pgfqpoint{4.434569in}{0.413320in}}%
\pgfpathlineto{\pgfqpoint{4.431901in}{0.413320in}}%
\pgfpathlineto{\pgfqpoint{4.429220in}{0.413320in}}%
\pgfpathlineto{\pgfqpoint{4.426534in}{0.413320in}}%
\pgfpathlineto{\pgfqpoint{4.423863in}{0.413320in}}%
\pgfpathlineto{\pgfqpoint{4.421292in}{0.413320in}}%
\pgfpathlineto{\pgfqpoint{4.418506in}{0.413320in}}%
\pgfpathlineto{\pgfqpoint{4.415932in}{0.413320in}}%
\pgfpathlineto{\pgfqpoint{4.413149in}{0.413320in}}%
\pgfpathlineto{\pgfqpoint{4.410587in}{0.413320in}}%
\pgfpathlineto{\pgfqpoint{4.407788in}{0.413320in}}%
\pgfpathlineto{\pgfqpoint{4.405234in}{0.413320in}}%
\pgfpathlineto{\pgfqpoint{4.402468in}{0.413320in}}%
\pgfpathlineto{\pgfqpoint{4.399745in}{0.413320in}}%
\pgfpathlineto{\pgfqpoint{4.397076in}{0.413320in}}%
\pgfpathlineto{\pgfqpoint{4.394400in}{0.413320in}}%
\pgfpathlineto{\pgfqpoint{4.391721in}{0.413320in}}%
\pgfpathlineto{\pgfqpoint{4.389044in}{0.413320in}}%
\pgfpathlineto{\pgfqpoint{4.386431in}{0.413320in}}%
\pgfpathlineto{\pgfqpoint{4.383674in}{0.413320in}}%
\pgfpathlineto{\pgfqpoint{4.381097in}{0.413320in}}%
\pgfpathlineto{\pgfqpoint{4.378329in}{0.413320in}}%
\pgfpathlineto{\pgfqpoint{4.375761in}{0.413320in}}%
\pgfpathlineto{\pgfqpoint{4.372976in}{0.413320in}}%
\pgfpathlineto{\pgfqpoint{4.370437in}{0.413320in}}%
\pgfpathlineto{\pgfqpoint{4.367646in}{0.413320in}}%
\pgfpathlineto{\pgfqpoint{4.364936in}{0.413320in}}%
\pgfpathlineto{\pgfqpoint{4.362270in}{0.413320in}}%
\pgfpathlineto{\pgfqpoint{4.359582in}{0.413320in}}%
\pgfpathlineto{\pgfqpoint{4.357014in}{0.413320in}}%
\pgfpathlineto{\pgfqpoint{4.354224in}{0.413320in}}%
\pgfpathlineto{\pgfqpoint{4.351645in}{0.413320in}}%
\pgfpathlineto{\pgfqpoint{4.348868in}{0.413320in}}%
\pgfpathlineto{\pgfqpoint{4.346263in}{0.413320in}}%
\pgfpathlineto{\pgfqpoint{4.343510in}{0.413320in}}%
\pgfpathlineto{\pgfqpoint{4.340976in}{0.413320in}}%
\pgfpathlineto{\pgfqpoint{4.338154in}{0.413320in}}%
\pgfpathlineto{\pgfqpoint{4.335463in}{0.413320in}}%
\pgfpathlineto{\pgfqpoint{4.332796in}{0.413320in}}%
\pgfpathlineto{\pgfqpoint{4.330118in}{0.413320in}}%
\pgfpathlineto{\pgfqpoint{4.327440in}{0.413320in}}%
\pgfpathlineto{\pgfqpoint{4.324760in}{0.413320in}}%
\pgfpathlineto{\pgfqpoint{4.322181in}{0.413320in}}%
\pgfpathlineto{\pgfqpoint{4.319405in}{0.413320in}}%
\pgfpathlineto{\pgfqpoint{4.316856in}{0.413320in}}%
\pgfpathlineto{\pgfqpoint{4.314032in}{0.413320in}}%
\pgfpathlineto{\pgfqpoint{4.311494in}{0.413320in}}%
\pgfpathlineto{\pgfqpoint{4.308691in}{0.413320in}}%
\pgfpathlineto{\pgfqpoint{4.306118in}{0.413320in}}%
\pgfpathlineto{\pgfqpoint{4.303357in}{0.413320in}}%
\pgfpathlineto{\pgfqpoint{4.300656in}{0.413320in}}%
\pgfpathlineto{\pgfqpoint{4.297977in}{0.413320in}}%
\pgfpathlineto{\pgfqpoint{4.295299in}{0.413320in}}%
\pgfpathlineto{\pgfqpoint{4.292786in}{0.413320in}}%
\pgfpathlineto{\pgfqpoint{4.289936in}{0.413320in}}%
\pgfpathlineto{\pgfqpoint{4.287399in}{0.413320in}}%
\pgfpathlineto{\pgfqpoint{4.284586in}{0.413320in}}%
\pgfpathlineto{\pgfqpoint{4.282000in}{0.413320in}}%
\pgfpathlineto{\pgfqpoint{4.279212in}{0.413320in}}%
\pgfpathlineto{\pgfqpoint{4.276635in}{0.413320in}}%
\pgfpathlineto{\pgfqpoint{4.273874in}{0.413320in}}%
\pgfpathlineto{\pgfqpoint{4.271187in}{0.413320in}}%
\pgfpathlineto{\pgfqpoint{4.268590in}{0.413320in}}%
\pgfpathlineto{\pgfqpoint{4.265824in}{0.413320in}}%
\pgfpathlineto{\pgfqpoint{4.263157in}{0.413320in}}%
\pgfpathlineto{\pgfqpoint{4.260477in}{0.413320in}}%
\pgfpathlineto{\pgfqpoint{4.257958in}{0.413320in}}%
\pgfpathlineto{\pgfqpoint{4.255120in}{0.413320in}}%
\pgfpathlineto{\pgfqpoint{4.252581in}{0.413320in}}%
\pgfpathlineto{\pgfqpoint{4.249767in}{0.413320in}}%
\pgfpathlineto{\pgfqpoint{4.247225in}{0.413320in}}%
\pgfpathlineto{\pgfqpoint{4.244394in}{0.413320in}}%
\pgfpathlineto{\pgfqpoint{4.241900in}{0.413320in}}%
\pgfpathlineto{\pgfqpoint{4.239084in}{0.413320in}}%
\pgfpathlineto{\pgfqpoint{4.236375in}{0.413320in}}%
\pgfpathlineto{\pgfqpoint{4.233691in}{0.413320in}}%
\pgfpathlineto{\pgfqpoint{4.231013in}{0.413320in}}%
\pgfpathlineto{\pgfqpoint{4.228331in}{0.413320in}}%
\pgfpathlineto{\pgfqpoint{4.225654in}{0.413320in}}%
\pgfpathlineto{\pgfqpoint{4.223082in}{0.413320in}}%
\pgfpathlineto{\pgfqpoint{4.220304in}{0.413320in}}%
\pgfpathlineto{\pgfqpoint{4.217694in}{0.413320in}}%
\pgfpathlineto{\pgfqpoint{4.214948in}{0.413320in}}%
\pgfpathlineto{\pgfqpoint{4.212383in}{0.413320in}}%
\pgfpathlineto{\pgfqpoint{4.209597in}{0.413320in}}%
\pgfpathlineto{\pgfqpoint{4.207076in}{0.413320in}}%
\pgfpathlineto{\pgfqpoint{4.204240in}{0.413320in}}%
\pgfpathlineto{\pgfqpoint{4.201542in}{0.413320in}}%
\pgfpathlineto{\pgfqpoint{4.198878in}{0.413320in}}%
\pgfpathlineto{\pgfqpoint{4.196186in}{0.413320in}}%
\pgfpathlineto{\pgfqpoint{4.193638in}{0.413320in}}%
\pgfpathlineto{\pgfqpoint{4.190842in}{0.413320in}}%
\pgfpathlineto{\pgfqpoint{4.188318in}{0.413320in}}%
\pgfpathlineto{\pgfqpoint{4.185481in}{0.413320in}}%
\pgfpathlineto{\pgfqpoint{4.182899in}{0.413320in}}%
\pgfpathlineto{\pgfqpoint{4.180129in}{0.413320in}}%
\pgfpathlineto{\pgfqpoint{4.177593in}{0.413320in}}%
\pgfpathlineto{\pgfqpoint{4.174770in}{0.413320in}}%
\pgfpathlineto{\pgfqpoint{4.172093in}{0.413320in}}%
\pgfpathlineto{\pgfqpoint{4.169415in}{0.413320in}}%
\pgfpathlineto{\pgfqpoint{4.166737in}{0.413320in}}%
\pgfpathlineto{\pgfqpoint{4.164059in}{0.413320in}}%
\pgfpathlineto{\pgfqpoint{4.161380in}{0.413320in}}%
\pgfpathlineto{\pgfqpoint{4.158806in}{0.413320in}}%
\pgfpathlineto{\pgfqpoint{4.156016in}{0.413320in}}%
\pgfpathlineto{\pgfqpoint{4.153423in}{0.413320in}}%
\pgfpathlineto{\pgfqpoint{4.150665in}{0.413320in}}%
\pgfpathlineto{\pgfqpoint{4.148082in}{0.413320in}}%
\pgfpathlineto{\pgfqpoint{4.145310in}{0.413320in}}%
\pgfpathlineto{\pgfqpoint{4.142713in}{0.413320in}}%
\pgfpathlineto{\pgfqpoint{4.139963in}{0.413320in}}%
\pgfpathlineto{\pgfqpoint{4.137272in}{0.413320in}}%
\pgfpathlineto{\pgfqpoint{4.134615in}{0.413320in}}%
\pgfpathlineto{\pgfqpoint{4.131920in}{0.413320in}}%
\pgfpathlineto{\pgfqpoint{4.129349in}{0.413320in}}%
\pgfpathlineto{\pgfqpoint{4.126553in}{0.413320in}}%
\pgfpathlineto{\pgfqpoint{4.124019in}{0.413320in}}%
\pgfpathlineto{\pgfqpoint{4.121205in}{0.413320in}}%
\pgfpathlineto{\pgfqpoint{4.118554in}{0.413320in}}%
\pgfpathlineto{\pgfqpoint{4.115844in}{0.413320in}}%
\pgfpathlineto{\pgfqpoint{4.113252in}{0.413320in}}%
\pgfpathlineto{\pgfqpoint{4.110488in}{0.413320in}}%
\pgfpathlineto{\pgfqpoint{4.107814in}{0.413320in}}%
\pgfpathlineto{\pgfqpoint{4.105185in}{0.413320in}}%
\pgfpathlineto{\pgfqpoint{4.102456in}{0.413320in}}%
\pgfpathlineto{\pgfqpoint{4.099777in}{0.413320in}}%
\pgfpathlineto{\pgfqpoint{4.097092in}{0.413320in}}%
\pgfpathlineto{\pgfqpoint{4.094527in}{0.413320in}}%
\pgfpathlineto{\pgfqpoint{4.091729in}{0.413320in}}%
\pgfpathlineto{\pgfqpoint{4.089159in}{0.413320in}}%
\pgfpathlineto{\pgfqpoint{4.086385in}{0.413320in}}%
\pgfpathlineto{\pgfqpoint{4.083870in}{0.413320in}}%
\pgfpathlineto{\pgfqpoint{4.081018in}{0.413320in}}%
\pgfpathlineto{\pgfqpoint{4.078471in}{0.413320in}}%
\pgfpathlineto{\pgfqpoint{4.075705in}{0.413320in}}%
\pgfpathlineto{\pgfqpoint{4.072985in}{0.413320in}}%
\pgfpathlineto{\pgfqpoint{4.070313in}{0.413320in}}%
\pgfpathlineto{\pgfqpoint{4.067636in}{0.413320in}}%
\pgfpathlineto{\pgfqpoint{4.064957in}{0.413320in}}%
\pgfpathlineto{\pgfqpoint{4.062266in}{0.413320in}}%
\pgfpathlineto{\pgfqpoint{4.059702in}{0.413320in}}%
\pgfpathlineto{\pgfqpoint{4.056911in}{0.413320in}}%
\pgfpathlineto{\pgfqpoint{4.054326in}{0.413320in}}%
\pgfpathlineto{\pgfqpoint{4.051557in}{0.413320in}}%
\pgfpathlineto{\pgfqpoint{4.049006in}{0.413320in}}%
\pgfpathlineto{\pgfqpoint{4.046210in}{0.413320in}}%
\pgfpathlineto{\pgfqpoint{4.043667in}{0.413320in}}%
\pgfpathlineto{\pgfqpoint{4.040852in}{0.413320in}}%
\pgfpathlineto{\pgfqpoint{4.038174in}{0.413320in}}%
\pgfpathlineto{\pgfqpoint{4.035492in}{0.413320in}}%
\pgfpathlineto{\pgfqpoint{4.032817in}{0.413320in}}%
\pgfpathlineto{\pgfqpoint{4.030229in}{0.413320in}}%
\pgfpathlineto{\pgfqpoint{4.027447in}{0.413320in}}%
\pgfpathlineto{\pgfqpoint{4.024868in}{0.413320in}}%
\pgfpathlineto{\pgfqpoint{4.022097in}{0.413320in}}%
\pgfpathlineto{\pgfqpoint{4.019518in}{0.413320in}}%
\pgfpathlineto{\pgfqpoint{4.016744in}{0.413320in}}%
\pgfpathlineto{\pgfqpoint{4.014186in}{0.413320in}}%
\pgfpathlineto{\pgfqpoint{4.011394in}{0.413320in}}%
\pgfpathlineto{\pgfqpoint{4.008699in}{0.413320in}}%
\pgfpathlineto{\pgfqpoint{4.006034in}{0.413320in}}%
\pgfpathlineto{\pgfqpoint{4.003348in}{0.413320in}}%
\pgfpathlineto{\pgfqpoint{4.000674in}{0.413320in}}%
\pgfpathlineto{\pgfqpoint{3.997990in}{0.413320in}}%
\pgfpathlineto{\pgfqpoint{3.995417in}{0.413320in}}%
\pgfpathlineto{\pgfqpoint{3.992642in}{0.413320in}}%
\pgfpathlineto{\pgfqpoint{3.990055in}{0.413320in}}%
\pgfpathlineto{\pgfqpoint{3.987270in}{0.413320in}}%
\pgfpathlineto{\pgfqpoint{3.984714in}{0.413320in}}%
\pgfpathlineto{\pgfqpoint{3.981929in}{0.413320in}}%
\pgfpathlineto{\pgfqpoint{3.979389in}{0.413320in}}%
\pgfpathlineto{\pgfqpoint{3.976563in}{0.413320in}}%
\pgfpathlineto{\pgfqpoint{3.973885in}{0.413320in}}%
\pgfpathlineto{\pgfqpoint{3.971250in}{0.413320in}}%
\pgfpathlineto{\pgfqpoint{3.968523in}{0.413320in}}%
\pgfpathlineto{\pgfqpoint{3.966013in}{0.413320in}}%
\pgfpathlineto{\pgfqpoint{3.963176in}{0.413320in}}%
\pgfpathlineto{\pgfqpoint{3.960635in}{0.413320in}}%
\pgfpathlineto{\pgfqpoint{3.957823in}{0.413320in}}%
\pgfpathlineto{\pgfqpoint{3.955211in}{0.413320in}}%
\pgfpathlineto{\pgfqpoint{3.952464in}{0.413320in}}%
\pgfpathlineto{\pgfqpoint{3.949894in}{0.413320in}}%
\pgfpathlineto{\pgfqpoint{3.947101in}{0.413320in}}%
\pgfpathlineto{\pgfqpoint{3.944431in}{0.413320in}}%
\pgfpathlineto{\pgfqpoint{3.941778in}{0.413320in}}%
\pgfpathlineto{\pgfqpoint{3.939075in}{0.413320in}}%
\pgfpathlineto{\pgfqpoint{3.936395in}{0.413320in}}%
\pgfpathlineto{\pgfqpoint{3.933714in}{0.413320in}}%
\pgfpathlineto{\pgfqpoint{3.931202in}{0.413320in}}%
\pgfpathlineto{\pgfqpoint{3.928347in}{0.413320in}}%
\pgfpathlineto{\pgfqpoint{3.925778in}{0.413320in}}%
\pgfpathlineto{\pgfqpoint{3.923005in}{0.413320in}}%
\pgfpathlineto{\pgfqpoint{3.920412in}{0.413320in}}%
\pgfpathlineto{\pgfqpoint{3.917646in}{0.413320in}}%
\pgfpathlineto{\pgfqpoint{3.915107in}{0.413320in}}%
\pgfpathlineto{\pgfqpoint{3.912296in}{0.413320in}}%
\pgfpathlineto{\pgfqpoint{3.909602in}{0.413320in}}%
\pgfpathlineto{\pgfqpoint{3.906918in}{0.413320in}}%
\pgfpathlineto{\pgfqpoint{3.904252in}{0.413320in}}%
\pgfpathlineto{\pgfqpoint{3.901573in}{0.413320in}}%
\pgfpathlineto{\pgfqpoint{3.898891in}{0.413320in}}%
\pgfpathlineto{\pgfqpoint{3.896345in}{0.413320in}}%
\pgfpathlineto{\pgfqpoint{3.893541in}{0.413320in}}%
\pgfpathlineto{\pgfqpoint{3.890926in}{0.413320in}}%
\pgfpathlineto{\pgfqpoint{3.888188in}{0.413320in}}%
\pgfpathlineto{\pgfqpoint{3.885621in}{0.413320in}}%
\pgfpathlineto{\pgfqpoint{3.882850in}{0.413320in}}%
\pgfpathlineto{\pgfqpoint{3.880237in}{0.413320in}}%
\pgfpathlineto{\pgfqpoint{3.877466in}{0.413320in}}%
\pgfpathlineto{\pgfqpoint{3.874790in}{0.413320in}}%
\pgfpathlineto{\pgfqpoint{3.872114in}{0.413320in}}%
\pgfpathlineto{\pgfqpoint{3.869435in}{0.413320in}}%
\pgfpathlineto{\pgfqpoint{3.866815in}{0.413320in}}%
\pgfpathlineto{\pgfqpoint{3.864073in}{0.413320in}}%
\pgfpathlineto{\pgfqpoint{3.861561in}{0.413320in}}%
\pgfpathlineto{\pgfqpoint{3.858720in}{0.413320in}}%
\pgfpathlineto{\pgfqpoint{3.856100in}{0.413320in}}%
\pgfpathlineto{\pgfqpoint{3.853358in}{0.413320in}}%
\pgfpathlineto{\pgfqpoint{3.850814in}{0.413320in}}%
\pgfpathlineto{\pgfqpoint{3.848005in}{0.413320in}}%
\pgfpathlineto{\pgfqpoint{3.845329in}{0.413320in}}%
\pgfpathlineto{\pgfqpoint{3.842641in}{0.413320in}}%
\pgfpathlineto{\pgfqpoint{3.839960in}{0.413320in}}%
\pgfpathlineto{\pgfqpoint{3.837286in}{0.413320in}}%
\pgfpathlineto{\pgfqpoint{3.834616in}{0.413320in}}%
\pgfpathlineto{\pgfqpoint{3.832053in}{0.413320in}}%
\pgfpathlineto{\pgfqpoint{3.829252in}{0.413320in}}%
\pgfpathlineto{\pgfqpoint{3.826679in}{0.413320in}}%
\pgfpathlineto{\pgfqpoint{3.823903in}{0.413320in}}%
\pgfpathlineto{\pgfqpoint{3.821315in}{0.413320in}}%
\pgfpathlineto{\pgfqpoint{3.818546in}{0.413320in}}%
\pgfpathlineto{\pgfqpoint{3.815983in}{0.413320in}}%
\pgfpathlineto{\pgfqpoint{3.813172in}{0.413320in}}%
\pgfpathlineto{\pgfqpoint{3.810510in}{0.413320in}}%
\pgfpathlineto{\pgfqpoint{3.807832in}{0.413320in}}%
\pgfpathlineto{\pgfqpoint{3.805145in}{0.413320in}}%
\pgfpathlineto{\pgfqpoint{3.802569in}{0.413320in}}%
\pgfpathlineto{\pgfqpoint{3.799797in}{0.413320in}}%
\pgfpathlineto{\pgfqpoint{3.797265in}{0.413320in}}%
\pgfpathlineto{\pgfqpoint{3.794435in}{0.413320in}}%
\pgfpathlineto{\pgfqpoint{3.791897in}{0.413320in}}%
\pgfpathlineto{\pgfqpoint{3.789084in}{0.413320in}}%
\pgfpathlineto{\pgfqpoint{3.786504in}{0.413320in}}%
\pgfpathlineto{\pgfqpoint{3.783725in}{0.413320in}}%
\pgfpathlineto{\pgfqpoint{3.781046in}{0.413320in}}%
\pgfpathlineto{\pgfqpoint{3.778370in}{0.413320in}}%
\pgfpathlineto{\pgfqpoint{3.775691in}{0.413320in}}%
\pgfpathlineto{\pgfqpoint{3.773014in}{0.413320in}}%
\pgfpathlineto{\pgfqpoint{3.770323in}{0.413320in}}%
\pgfpathlineto{\pgfqpoint{3.767782in}{0.413320in}}%
\pgfpathlineto{\pgfqpoint{3.764966in}{0.413320in}}%
\pgfpathlineto{\pgfqpoint{3.762389in}{0.413320in}}%
\pgfpathlineto{\pgfqpoint{3.759622in}{0.413320in}}%
\pgfpathlineto{\pgfqpoint{3.757065in}{0.413320in}}%
\pgfpathlineto{\pgfqpoint{3.754265in}{0.413320in}}%
\pgfpathlineto{\pgfqpoint{3.751728in}{0.413320in}}%
\pgfpathlineto{\pgfqpoint{3.748903in}{0.413320in}}%
\pgfpathlineto{\pgfqpoint{3.746229in}{0.413320in}}%
\pgfpathlineto{\pgfqpoint{3.743548in}{0.413320in}}%
\pgfpathlineto{\pgfqpoint{3.740874in}{0.413320in}}%
\pgfpathlineto{\pgfqpoint{3.738194in}{0.413320in}}%
\pgfpathlineto{\pgfqpoint{3.735509in}{0.413320in}}%
\pgfpathlineto{\pgfqpoint{3.732950in}{0.413320in}}%
\pgfpathlineto{\pgfqpoint{3.730158in}{0.413320in}}%
\pgfpathlineto{\pgfqpoint{3.727581in}{0.413320in}}%
\pgfpathlineto{\pgfqpoint{3.724804in}{0.413320in}}%
\pgfpathlineto{\pgfqpoint{3.722228in}{0.413320in}}%
\pgfpathlineto{\pgfqpoint{3.719446in}{0.413320in}}%
\pgfpathlineto{\pgfqpoint{3.716875in}{0.413320in}}%
\pgfpathlineto{\pgfqpoint{3.714086in}{0.413320in}}%
\pgfpathlineto{\pgfqpoint{3.711410in}{0.413320in}}%
\pgfpathlineto{\pgfqpoint{3.708729in}{0.413320in}}%
\pgfpathlineto{\pgfqpoint{3.706053in}{0.413320in}}%
\pgfpathlineto{\pgfqpoint{3.703460in}{0.413320in}}%
\pgfpathlineto{\pgfqpoint{3.700684in}{0.413320in}}%
\pgfpathlineto{\pgfqpoint{3.698125in}{0.413320in}}%
\pgfpathlineto{\pgfqpoint{3.695331in}{0.413320in}}%
\pgfpathlineto{\pgfqpoint{3.692765in}{0.413320in}}%
\pgfpathlineto{\pgfqpoint{3.689983in}{0.413320in}}%
\pgfpathlineto{\pgfqpoint{3.687442in}{0.413320in}}%
\pgfpathlineto{\pgfqpoint{3.684620in}{0.413320in}}%
\pgfpathlineto{\pgfqpoint{3.681948in}{0.413320in}}%
\pgfpathlineto{\pgfqpoint{3.679273in}{0.413320in}}%
\pgfpathlineto{\pgfqpoint{3.676591in}{0.413320in}}%
\pgfpathlineto{\pgfqpoint{3.673911in}{0.413320in}}%
\pgfpathlineto{\pgfqpoint{3.671232in}{0.413320in}}%
\pgfpathlineto{\pgfqpoint{3.668665in}{0.413320in}}%
\pgfpathlineto{\pgfqpoint{3.665864in}{0.413320in}}%
\pgfpathlineto{\pgfqpoint{3.663276in}{0.413320in}}%
\pgfpathlineto{\pgfqpoint{3.660515in}{0.413320in}}%
\pgfpathlineto{\pgfqpoint{3.657917in}{0.413320in}}%
\pgfpathlineto{\pgfqpoint{3.655165in}{0.413320in}}%
\pgfpathlineto{\pgfqpoint{3.652628in}{0.413320in}}%
\pgfpathlineto{\pgfqpoint{3.649837in}{0.413320in}}%
\pgfpathlineto{\pgfqpoint{3.647130in}{0.413320in}}%
\pgfpathlineto{\pgfqpoint{3.644452in}{0.413320in}}%
\pgfpathlineto{\pgfqpoint{3.641773in}{0.413320in}}%
\pgfpathlineto{\pgfqpoint{3.639207in}{0.413320in}}%
\pgfpathlineto{\pgfqpoint{3.636413in}{0.413320in}}%
\pgfpathlineto{\pgfqpoint{3.633858in}{0.413320in}}%
\pgfpathlineto{\pgfqpoint{3.631058in}{0.413320in}}%
\pgfpathlineto{\pgfqpoint{3.628460in}{0.413320in}}%
\pgfpathlineto{\pgfqpoint{3.625689in}{0.413320in}}%
\pgfpathlineto{\pgfqpoint{3.623165in}{0.413320in}}%
\pgfpathlineto{\pgfqpoint{3.620345in}{0.413320in}}%
\pgfpathlineto{\pgfqpoint{3.617667in}{0.413320in}}%
\pgfpathlineto{\pgfqpoint{3.614982in}{0.413320in}}%
\pgfpathlineto{\pgfqpoint{3.612311in}{0.413320in}}%
\pgfpathlineto{\pgfqpoint{3.609632in}{0.413320in}}%
\pgfpathlineto{\pgfqpoint{3.606951in}{0.413320in}}%
\pgfpathlineto{\pgfqpoint{3.604387in}{0.413320in}}%
\pgfpathlineto{\pgfqpoint{3.601590in}{0.413320in}}%
\pgfpathlineto{\pgfqpoint{3.598998in}{0.413320in}}%
\pgfpathlineto{\pgfqpoint{3.596240in}{0.413320in}}%
\pgfpathlineto{\pgfqpoint{3.593620in}{0.413320in}}%
\pgfpathlineto{\pgfqpoint{3.590883in}{0.413320in}}%
\pgfpathlineto{\pgfqpoint{3.588258in}{0.413320in}}%
\pgfpathlineto{\pgfqpoint{3.585532in}{0.413320in}}%
\pgfpathlineto{\pgfqpoint{3.582851in}{0.413320in}}%
\pgfpathlineto{\pgfqpoint{3.580191in}{0.413320in}}%
\pgfpathlineto{\pgfqpoint{3.577487in}{0.413320in}}%
\pgfpathlineto{\pgfqpoint{3.574814in}{0.413320in}}%
\pgfpathlineto{\pgfqpoint{3.572126in}{0.413320in}}%
\pgfpathlineto{\pgfqpoint{3.569584in}{0.413320in}}%
\pgfpathlineto{\pgfqpoint{3.566774in}{0.413320in}}%
\pgfpathlineto{\pgfqpoint{3.564188in}{0.413320in}}%
\pgfpathlineto{\pgfqpoint{3.561420in}{0.413320in}}%
\pgfpathlineto{\pgfqpoint{3.558853in}{0.413320in}}%
\pgfpathlineto{\pgfqpoint{3.556061in}{0.413320in}}%
\pgfpathlineto{\pgfqpoint{3.553498in}{0.413320in}}%
\pgfpathlineto{\pgfqpoint{3.550713in}{0.413320in}}%
\pgfpathlineto{\pgfqpoint{3.548029in}{0.413320in}}%
\pgfpathlineto{\pgfqpoint{3.545349in}{0.413320in}}%
\pgfpathlineto{\pgfqpoint{3.542656in}{0.413320in}}%
\pgfpathlineto{\pgfqpoint{3.540093in}{0.413320in}}%
\pgfpathlineto{\pgfqpoint{3.537309in}{0.413320in}}%
\pgfpathlineto{\pgfqpoint{3.534783in}{0.413320in}}%
\pgfpathlineto{\pgfqpoint{3.531955in}{0.413320in}}%
\pgfpathlineto{\pgfqpoint{3.529327in}{0.413320in}}%
\pgfpathlineto{\pgfqpoint{3.526601in}{0.413320in}}%
\pgfpathlineto{\pgfqpoint{3.524041in}{0.413320in}}%
\pgfpathlineto{\pgfqpoint{3.521244in}{0.413320in}}%
\pgfpathlineto{\pgfqpoint{3.518565in}{0.413320in}}%
\pgfpathlineto{\pgfqpoint{3.515884in}{0.413320in}}%
\pgfpathlineto{\pgfqpoint{3.513209in}{0.413320in}}%
\pgfpathlineto{\pgfqpoint{3.510533in}{0.413320in}}%
\pgfpathlineto{\pgfqpoint{3.507840in}{0.413320in}}%
\pgfpathlineto{\pgfqpoint{3.505262in}{0.413320in}}%
\pgfpathlineto{\pgfqpoint{3.502488in}{0.413320in}}%
\pgfpathlineto{\pgfqpoint{3.499909in}{0.413320in}}%
\pgfpathlineto{\pgfqpoint{3.497139in}{0.413320in}}%
\pgfpathlineto{\pgfqpoint{3.494581in}{0.413320in}}%
\pgfpathlineto{\pgfqpoint{3.491783in}{0.413320in}}%
\pgfpathlineto{\pgfqpoint{3.489223in}{0.413320in}}%
\pgfpathlineto{\pgfqpoint{3.486442in}{0.413320in}}%
\pgfpathlineto{\pgfqpoint{3.483744in}{0.413320in}}%
\pgfpathlineto{\pgfqpoint{3.481072in}{0.413320in}}%
\pgfpathlineto{\pgfqpoint{3.478378in}{0.413320in}}%
\pgfpathlineto{\pgfqpoint{3.475821in}{0.413320in}}%
\pgfpathlineto{\pgfqpoint{3.473021in}{0.413320in}}%
\pgfpathlineto{\pgfqpoint{3.470466in}{0.413320in}}%
\pgfpathlineto{\pgfqpoint{3.467678in}{0.413320in}}%
\pgfpathlineto{\pgfqpoint{3.465072in}{0.413320in}}%
\pgfpathlineto{\pgfqpoint{3.462321in}{0.413320in}}%
\pgfpathlineto{\pgfqpoint{3.459695in}{0.413320in}}%
\pgfpathlineto{\pgfqpoint{3.456960in}{0.413320in}}%
\pgfpathlineto{\pgfqpoint{3.454285in}{0.413320in}}%
\pgfpathlineto{\pgfqpoint{3.451597in}{0.413320in}}%
\pgfpathlineto{\pgfqpoint{3.448926in}{0.413320in}}%
\pgfpathlineto{\pgfqpoint{3.446257in}{0.413320in}}%
\pgfpathlineto{\pgfqpoint{3.443574in}{0.413320in}}%
\pgfpathlineto{\pgfqpoint{3.440996in}{0.413320in}}%
\pgfpathlineto{\pgfqpoint{3.438210in}{0.413320in}}%
\pgfpathlineto{\pgfqpoint{3.435635in}{0.413320in}}%
\pgfpathlineto{\pgfqpoint{3.432851in}{0.413320in}}%
\pgfpathlineto{\pgfqpoint{3.430313in}{0.413320in}}%
\pgfpathlineto{\pgfqpoint{3.427501in}{0.413320in}}%
\pgfpathlineto{\pgfqpoint{3.424887in}{0.413320in}}%
\pgfpathlineto{\pgfqpoint{3.422142in}{0.413320in}}%
\pgfpathlineto{\pgfqpoint{3.419455in}{0.413320in}}%
\pgfpathlineto{\pgfqpoint{3.416780in}{0.413320in}}%
\pgfpathlineto{\pgfqpoint{3.414109in}{0.413320in}}%
\pgfpathlineto{\pgfqpoint{3.411431in}{0.413320in}}%
\pgfpathlineto{\pgfqpoint{3.408752in}{0.413320in}}%
\pgfpathlineto{\pgfqpoint{3.406202in}{0.413320in}}%
\pgfpathlineto{\pgfqpoint{3.403394in}{0.413320in}}%
\pgfpathlineto{\pgfqpoint{3.400783in}{0.413320in}}%
\pgfpathlineto{\pgfqpoint{3.398037in}{0.413320in}}%
\pgfpathlineto{\pgfqpoint{3.395461in}{0.413320in}}%
\pgfpathlineto{\pgfqpoint{3.392681in}{0.413320in}}%
\pgfpathlineto{\pgfqpoint{3.390102in}{0.413320in}}%
\pgfpathlineto{\pgfqpoint{3.387309in}{0.413320in}}%
\pgfpathlineto{\pgfqpoint{3.384647in}{0.413320in}}%
\pgfpathlineto{\pgfqpoint{3.381959in}{0.413320in}}%
\pgfpathlineto{\pgfqpoint{3.379290in}{0.413320in}}%
\pgfpathlineto{\pgfqpoint{3.376735in}{0.413320in}}%
\pgfpathlineto{\pgfqpoint{3.373921in}{0.413320in}}%
\pgfpathlineto{\pgfqpoint{3.371357in}{0.413320in}}%
\pgfpathlineto{\pgfqpoint{3.368577in}{0.413320in}}%
\pgfpathlineto{\pgfqpoint{3.365996in}{0.413320in}}%
\pgfpathlineto{\pgfqpoint{3.363221in}{0.413320in}}%
\pgfpathlineto{\pgfqpoint{3.360620in}{0.413320in}}%
\pgfpathlineto{\pgfqpoint{3.357862in}{0.413320in}}%
\pgfpathlineto{\pgfqpoint{3.355177in}{0.413320in}}%
\pgfpathlineto{\pgfqpoint{3.352505in}{0.413320in}}%
\pgfpathlineto{\pgfqpoint{3.349828in}{0.413320in}}%
\pgfpathlineto{\pgfqpoint{3.347139in}{0.413320in}}%
\pgfpathlineto{\pgfqpoint{3.344468in}{0.413320in}}%
\pgfpathlineto{\pgfqpoint{3.341893in}{0.413320in}}%
\pgfpathlineto{\pgfqpoint{3.339101in}{0.413320in}}%
\pgfpathlineto{\pgfqpoint{3.336541in}{0.413320in}}%
\pgfpathlineto{\pgfqpoint{3.333758in}{0.413320in}}%
\pgfpathlineto{\pgfqpoint{3.331183in}{0.413320in}}%
\pgfpathlineto{\pgfqpoint{3.328401in}{0.413320in}}%
\pgfpathlineto{\pgfqpoint{3.325860in}{0.413320in}}%
\pgfpathlineto{\pgfqpoint{3.323049in}{0.413320in}}%
\pgfpathlineto{\pgfqpoint{3.320366in}{0.413320in}}%
\pgfpathlineto{\pgfqpoint{3.317688in}{0.413320in}}%
\pgfpathlineto{\pgfqpoint{3.315008in}{0.413320in}}%
\pgfpathlineto{\pgfqpoint{3.312480in}{0.413320in}}%
\pgfpathlineto{\pgfqpoint{3.309652in}{0.413320in}}%
\pgfpathlineto{\pgfqpoint{3.307104in}{0.413320in}}%
\pgfpathlineto{\pgfqpoint{3.304295in}{0.413320in}}%
\pgfpathlineto{\pgfqpoint{3.301719in}{0.413320in}}%
\pgfpathlineto{\pgfqpoint{3.298937in}{0.413320in}}%
\pgfpathlineto{\pgfqpoint{3.296376in}{0.413320in}}%
\pgfpathlineto{\pgfqpoint{3.293574in}{0.413320in}}%
\pgfpathlineto{\pgfqpoint{3.290890in}{0.413320in}}%
\pgfpathlineto{\pgfqpoint{3.288225in}{0.413320in}}%
\pgfpathlineto{\pgfqpoint{3.285534in}{0.413320in}}%
\pgfpathlineto{\pgfqpoint{3.282870in}{0.413320in}}%
\pgfpathlineto{\pgfqpoint{3.280189in}{0.413320in}}%
\pgfpathlineto{\pgfqpoint{3.277603in}{0.413320in}}%
\pgfpathlineto{\pgfqpoint{3.274831in}{0.413320in}}%
\pgfpathlineto{\pgfqpoint{3.272254in}{0.413320in}}%
\pgfpathlineto{\pgfqpoint{3.269478in}{0.413320in}}%
\pgfpathlineto{\pgfqpoint{3.266849in}{0.413320in}}%
\pgfpathlineto{\pgfqpoint{3.264119in}{0.413320in}}%
\pgfpathlineto{\pgfqpoint{3.261594in}{0.413320in}}%
\pgfpathlineto{\pgfqpoint{3.258784in}{0.413320in}}%
\pgfpathlineto{\pgfqpoint{3.256083in}{0.413320in}}%
\pgfpathlineto{\pgfqpoint{3.253404in}{0.413320in}}%
\pgfpathlineto{\pgfqpoint{3.250716in}{0.413320in}}%
\pgfpathlineto{\pgfqpoint{3.248049in}{0.413320in}}%
\pgfpathlineto{\pgfqpoint{3.245363in}{0.413320in}}%
\pgfpathlineto{\pgfqpoint{3.242807in}{0.413320in}}%
\pgfpathlineto{\pgfqpoint{3.240010in}{0.413320in}}%
\pgfpathlineto{\pgfqpoint{3.237411in}{0.413320in}}%
\pgfpathlineto{\pgfqpoint{3.234658in}{0.413320in}}%
\pgfpathlineto{\pgfqpoint{3.232069in}{0.413320in}}%
\pgfpathlineto{\pgfqpoint{3.229310in}{0.413320in}}%
\pgfpathlineto{\pgfqpoint{3.226609in}{0.413320in}}%
\pgfpathlineto{\pgfqpoint{3.223942in}{0.413320in}}%
\pgfpathlineto{\pgfqpoint{3.221255in}{0.413320in}}%
\pgfpathlineto{\pgfqpoint{3.218586in}{0.413320in}}%
\pgfpathlineto{\pgfqpoint{3.215908in}{0.413320in}}%
\pgfpathlineto{\pgfqpoint{3.213342in}{0.413320in}}%
\pgfpathlineto{\pgfqpoint{3.210545in}{0.413320in}}%
\pgfpathlineto{\pgfqpoint{3.207984in}{0.413320in}}%
\pgfpathlineto{\pgfqpoint{3.205195in}{0.413320in}}%
\pgfpathlineto{\pgfqpoint{3.202562in}{0.413320in}}%
\pgfpathlineto{\pgfqpoint{3.199823in}{0.413320in}}%
\pgfpathlineto{\pgfqpoint{3.197226in}{0.413320in}}%
\pgfpathlineto{\pgfqpoint{3.194508in}{0.413320in}}%
\pgfpathlineto{\pgfqpoint{3.191796in}{0.413320in}}%
\pgfpathlineto{\pgfqpoint{3.189117in}{0.413320in}}%
\pgfpathlineto{\pgfqpoint{3.186440in}{0.413320in}}%
\pgfpathlineto{\pgfqpoint{3.183760in}{0.413320in}}%
\pgfpathlineto{\pgfqpoint{3.181089in}{0.413320in}}%
\pgfpathlineto{\pgfqpoint{3.178525in}{0.413320in}}%
\pgfpathlineto{\pgfqpoint{3.175724in}{0.413320in}}%
\pgfpathlineto{\pgfqpoint{3.173142in}{0.413320in}}%
\pgfpathlineto{\pgfqpoint{3.170375in}{0.413320in}}%
\pgfpathlineto{\pgfqpoint{3.167776in}{0.413320in}}%
\pgfpathlineto{\pgfqpoint{3.165019in}{0.413320in}}%
\pgfpathlineto{\pgfqpoint{3.162474in}{0.413320in}}%
\pgfpathlineto{\pgfqpoint{3.159675in}{0.413320in}}%
\pgfpathlineto{\pgfqpoint{3.156981in}{0.413320in}}%
\pgfpathlineto{\pgfqpoint{3.154327in}{0.413320in}}%
\pgfpathlineto{\pgfqpoint{3.151612in}{0.413320in}}%
\pgfpathlineto{\pgfqpoint{3.149057in}{0.413320in}}%
\pgfpathlineto{\pgfqpoint{3.146271in}{0.413320in}}%
\pgfpathlineto{\pgfqpoint{3.143740in}{0.413320in}}%
\pgfpathlineto{\pgfqpoint{3.140913in}{0.413320in}}%
\pgfpathlineto{\pgfqpoint{3.138375in}{0.413320in}}%
\pgfpathlineto{\pgfqpoint{3.135550in}{0.413320in}}%
\pgfpathlineto{\pgfqpoint{3.132946in}{0.413320in}}%
\pgfpathlineto{\pgfqpoint{3.130199in}{0.413320in}}%
\pgfpathlineto{\pgfqpoint{3.127512in}{0.413320in}}%
\pgfpathlineto{\pgfqpoint{3.124842in}{0.413320in}}%
\pgfpathlineto{\pgfqpoint{3.122163in}{0.413320in}}%
\pgfpathlineto{\pgfqpoint{3.119487in}{0.413320in}}%
\pgfpathlineto{\pgfqpoint{3.116807in}{0.413320in}}%
\pgfpathlineto{\pgfqpoint{3.114242in}{0.413320in}}%
\pgfpathlineto{\pgfqpoint{3.111451in}{0.413320in}}%
\pgfpathlineto{\pgfqpoint{3.108896in}{0.413320in}}%
\pgfpathlineto{\pgfqpoint{3.106094in}{0.413320in}}%
\pgfpathlineto{\pgfqpoint{3.103508in}{0.413320in}}%
\pgfpathlineto{\pgfqpoint{3.100737in}{0.413320in}}%
\pgfpathlineto{\pgfqpoint{3.098163in}{0.413320in}}%
\pgfpathlineto{\pgfqpoint{3.095388in}{0.413320in}}%
\pgfpathlineto{\pgfqpoint{3.092699in}{0.413320in}}%
\pgfpathlineto{\pgfqpoint{3.090023in}{0.413320in}}%
\pgfpathlineto{\pgfqpoint{3.087343in}{0.413320in}}%
\pgfpathlineto{\pgfqpoint{3.084671in}{0.413320in}}%
\pgfpathlineto{\pgfqpoint{3.081990in}{0.413320in}}%
\pgfpathlineto{\pgfqpoint{3.079381in}{0.413320in}}%
\pgfpathlineto{\pgfqpoint{3.076631in}{0.413320in}}%
\pgfpathlineto{\pgfqpoint{3.074056in}{0.413320in}}%
\pgfpathlineto{\pgfqpoint{3.071266in}{0.413320in}}%
\pgfpathlineto{\pgfqpoint{3.068709in}{0.413320in}}%
\pgfpathlineto{\pgfqpoint{3.065916in}{0.413320in}}%
\pgfpathlineto{\pgfqpoint{3.063230in}{0.413320in}}%
\pgfpathlineto{\pgfqpoint{3.060561in}{0.413320in}}%
\pgfpathlineto{\pgfqpoint{3.057884in}{0.413320in}}%
\pgfpathlineto{\pgfqpoint{3.055202in}{0.413320in}}%
\pgfpathlineto{\pgfqpoint{3.052526in}{0.413320in}}%
\pgfpathlineto{\pgfqpoint{3.049988in}{0.413320in}}%
\pgfpathlineto{\pgfqpoint{3.047157in}{0.413320in}}%
\pgfpathlineto{\pgfqpoint{3.044568in}{0.413320in}}%
\pgfpathlineto{\pgfqpoint{3.041813in}{0.413320in}}%
\pgfpathlineto{\pgfqpoint{3.039262in}{0.413320in}}%
\pgfpathlineto{\pgfqpoint{3.036456in}{0.413320in}}%
\pgfpathlineto{\pgfqpoint{3.033921in}{0.413320in}}%
\pgfpathlineto{\pgfqpoint{3.031091in}{0.413320in}}%
\pgfpathlineto{\pgfqpoint{3.028412in}{0.413320in}}%
\pgfpathlineto{\pgfqpoint{3.025803in}{0.413320in}}%
\pgfpathlineto{\pgfqpoint{3.023058in}{0.413320in}}%
\pgfpathlineto{\pgfqpoint{3.020382in}{0.413320in}}%
\pgfpathlineto{\pgfqpoint{3.017707in}{0.413320in}}%
\pgfpathlineto{\pgfqpoint{3.015097in}{0.413320in}}%
\pgfpathlineto{\pgfqpoint{3.012351in}{0.413320in}}%
\pgfpathlineto{\pgfqpoint{3.009784in}{0.413320in}}%
\pgfpathlineto{\pgfqpoint{3.006993in}{0.413320in}}%
\pgfpathlineto{\pgfqpoint{3.004419in}{0.413320in}}%
\pgfpathlineto{\pgfqpoint{3.001635in}{0.413320in}}%
\pgfpathlineto{\pgfqpoint{2.999103in}{0.413320in}}%
\pgfpathlineto{\pgfqpoint{2.996300in}{0.413320in}}%
\pgfpathlineto{\pgfqpoint{2.993595in}{0.413320in}}%
\pgfpathlineto{\pgfqpoint{2.990978in}{0.413320in}}%
\pgfpathlineto{\pgfqpoint{2.988238in}{0.413320in}}%
\pgfpathlineto{\pgfqpoint{2.985666in}{0.413320in}}%
\pgfpathlineto{\pgfqpoint{2.982885in}{0.413320in}}%
\pgfpathlineto{\pgfqpoint{2.980341in}{0.413320in}}%
\pgfpathlineto{\pgfqpoint{2.977517in}{0.413320in}}%
\pgfpathlineto{\pgfqpoint{2.974972in}{0.413320in}}%
\pgfpathlineto{\pgfqpoint{2.972177in}{0.413320in}}%
\pgfpathlineto{\pgfqpoint{2.969599in}{0.413320in}}%
\pgfpathlineto{\pgfqpoint{2.966812in}{0.413320in}}%
\pgfpathlineto{\pgfqpoint{2.964127in}{0.413320in}}%
\pgfpathlineto{\pgfqpoint{2.961460in}{0.413320in}}%
\pgfpathlineto{\pgfqpoint{2.958782in}{0.413320in}}%
\pgfpathlineto{\pgfqpoint{2.956103in}{0.413320in}}%
\pgfpathlineto{\pgfqpoint{2.953422in}{0.413320in}}%
\pgfpathlineto{\pgfqpoint{2.950884in}{0.413320in}}%
\pgfpathlineto{\pgfqpoint{2.948068in}{0.413320in}}%
\pgfpathlineto{\pgfqpoint{2.945461in}{0.413320in}}%
\pgfpathlineto{\pgfqpoint{2.942711in}{0.413320in}}%
\pgfpathlineto{\pgfqpoint{2.940120in}{0.413320in}}%
\pgfpathlineto{\pgfqpoint{2.937352in}{0.413320in}}%
\pgfpathlineto{\pgfqpoint{2.934759in}{0.413320in}}%
\pgfpathlineto{\pgfqpoint{2.932033in}{0.413320in}}%
\pgfpathlineto{\pgfqpoint{2.929321in}{0.413320in}}%
\pgfpathlineto{\pgfqpoint{2.926655in}{0.413320in}}%
\pgfpathlineto{\pgfqpoint{2.923963in}{0.413320in}}%
\pgfpathlineto{\pgfqpoint{2.921363in}{0.413320in}}%
\pgfpathlineto{\pgfqpoint{2.918606in}{0.413320in}}%
\pgfpathlineto{\pgfqpoint{2.916061in}{0.413320in}}%
\pgfpathlineto{\pgfqpoint{2.913243in}{0.413320in}}%
\pgfpathlineto{\pgfqpoint{2.910631in}{0.413320in}}%
\pgfpathlineto{\pgfqpoint{2.907882in}{0.413320in}}%
\pgfpathlineto{\pgfqpoint{2.905341in}{0.413320in}}%
\pgfpathlineto{\pgfqpoint{2.902535in}{0.413320in}}%
\pgfpathlineto{\pgfqpoint{2.899858in}{0.413320in}}%
\pgfpathlineto{\pgfqpoint{2.897179in}{0.413320in}}%
\pgfpathlineto{\pgfqpoint{2.894487in}{0.413320in}}%
\pgfpathlineto{\pgfqpoint{2.891809in}{0.413320in}}%
\pgfpathlineto{\pgfqpoint{2.889145in}{0.413320in}}%
\pgfpathlineto{\pgfqpoint{2.886578in}{0.413320in}}%
\pgfpathlineto{\pgfqpoint{2.883780in}{0.413320in}}%
\pgfpathlineto{\pgfqpoint{2.881254in}{0.413320in}}%
\pgfpathlineto{\pgfqpoint{2.878431in}{0.413320in}}%
\pgfpathlineto{\pgfqpoint{2.875882in}{0.413320in}}%
\pgfpathlineto{\pgfqpoint{2.873074in}{0.413320in}}%
\pgfpathlineto{\pgfqpoint{2.870475in}{0.413320in}}%
\pgfpathlineto{\pgfqpoint{2.867713in}{0.413320in}}%
\pgfpathlineto{\pgfqpoint{2.865031in}{0.413320in}}%
\pgfpathlineto{\pgfqpoint{2.862402in}{0.413320in}}%
\pgfpathlineto{\pgfqpoint{2.859668in}{0.413320in}}%
\pgfpathlineto{\pgfqpoint{2.857003in}{0.413320in}}%
\pgfpathlineto{\pgfqpoint{2.854325in}{0.413320in}}%
\pgfpathlineto{\pgfqpoint{2.851793in}{0.413320in}}%
\pgfpathlineto{\pgfqpoint{2.848960in}{0.413320in}}%
\pgfpathlineto{\pgfqpoint{2.846408in}{0.413320in}}%
\pgfpathlineto{\pgfqpoint{2.843611in}{0.413320in}}%
\pgfpathlineto{\pgfqpoint{2.841055in}{0.413320in}}%
\pgfpathlineto{\pgfqpoint{2.838254in}{0.413320in}}%
\pgfpathlineto{\pgfqpoint{2.835698in}{0.413320in}}%
\pgfpathlineto{\pgfqpoint{2.832894in}{0.413320in}}%
\pgfpathlineto{\pgfqpoint{2.830219in}{0.413320in}}%
\pgfpathlineto{\pgfqpoint{2.827567in}{0.413320in}}%
\pgfpathlineto{\pgfqpoint{2.824851in}{0.413320in}}%
\pgfpathlineto{\pgfqpoint{2.822303in}{0.413320in}}%
\pgfpathlineto{\pgfqpoint{2.819506in}{0.413320in}}%
\pgfpathlineto{\pgfqpoint{2.816867in}{0.413320in}}%
\pgfpathlineto{\pgfqpoint{2.814141in}{0.413320in}}%
\pgfpathlineto{\pgfqpoint{2.811597in}{0.413320in}}%
\pgfpathlineto{\pgfqpoint{2.808792in}{0.413320in}}%
\pgfpathlineto{\pgfqpoint{2.806175in}{0.413320in}}%
\pgfpathlineto{\pgfqpoint{2.803435in}{0.413320in}}%
\pgfpathlineto{\pgfqpoint{2.800756in}{0.413320in}}%
\pgfpathlineto{\pgfqpoint{2.798070in}{0.413320in}}%
\pgfpathlineto{\pgfqpoint{2.795398in}{0.413320in}}%
\pgfpathlineto{\pgfqpoint{2.792721in}{0.413320in}}%
\pgfpathlineto{\pgfqpoint{2.790044in}{0.413320in}}%
\pgfpathlineto{\pgfqpoint{2.787468in}{0.413320in}}%
\pgfpathlineto{\pgfqpoint{2.784687in}{0.413320in}}%
\pgfpathlineto{\pgfqpoint{2.782113in}{0.413320in}}%
\pgfpathlineto{\pgfqpoint{2.779330in}{0.413320in}}%
\pgfpathlineto{\pgfqpoint{2.776767in}{0.413320in}}%
\pgfpathlineto{\pgfqpoint{2.773972in}{0.413320in}}%
\pgfpathlineto{\pgfqpoint{2.771373in}{0.413320in}}%
\pgfpathlineto{\pgfqpoint{2.768617in}{0.413320in}}%
\pgfpathlineto{\pgfqpoint{2.765935in}{0.413320in}}%
\pgfpathlineto{\pgfqpoint{2.763253in}{0.413320in}}%
\pgfpathlineto{\pgfqpoint{2.760581in}{0.413320in}}%
\pgfpathlineto{\pgfqpoint{2.758028in}{0.413320in}}%
\pgfpathlineto{\pgfqpoint{2.755224in}{0.413320in}}%
\pgfpathlineto{\pgfqpoint{2.752614in}{0.413320in}}%
\pgfpathlineto{\pgfqpoint{2.749868in}{0.413320in}}%
\pgfpathlineto{\pgfqpoint{2.747260in}{0.413320in}}%
\pgfpathlineto{\pgfqpoint{2.744510in}{0.413320in}}%
\pgfpathlineto{\pgfqpoint{2.741928in}{0.413320in}}%
\pgfpathlineto{\pgfqpoint{2.739155in}{0.413320in}}%
\pgfpathlineto{\pgfqpoint{2.736476in}{0.413320in}}%
\pgfpathlineto{\pgfqpoint{2.733798in}{0.413320in}}%
\pgfpathlineto{\pgfqpoint{2.731119in}{0.413320in}}%
\pgfpathlineto{\pgfqpoint{2.728439in}{0.413320in}}%
\pgfpathlineto{\pgfqpoint{2.725760in}{0.413320in}}%
\pgfpathlineto{\pgfqpoint{2.723211in}{0.413320in}}%
\pgfpathlineto{\pgfqpoint{2.720404in}{0.413320in}}%
\pgfpathlineto{\pgfqpoint{2.717773in}{0.413320in}}%
\pgfpathlineto{\pgfqpoint{2.715036in}{0.413320in}}%
\pgfpathlineto{\pgfqpoint{2.712477in}{0.413320in}}%
\pgfpathlineto{\pgfqpoint{2.709683in}{0.413320in}}%
\pgfpathlineto{\pgfqpoint{2.707125in}{0.413320in}}%
\pgfpathlineto{\pgfqpoint{2.704326in}{0.413320in}}%
\pgfpathlineto{\pgfqpoint{2.701657in}{0.413320in}}%
\pgfpathlineto{\pgfqpoint{2.698968in}{0.413320in}}%
\pgfpathlineto{\pgfqpoint{2.696293in}{0.413320in}}%
\pgfpathlineto{\pgfqpoint{2.693611in}{0.413320in}}%
\pgfpathlineto{\pgfqpoint{2.690940in}{0.413320in}}%
\pgfpathlineto{\pgfqpoint{2.688328in}{0.413320in}}%
\pgfpathlineto{\pgfqpoint{2.685586in}{0.413320in}}%
\pgfpathlineto{\pgfqpoint{2.683009in}{0.413320in}}%
\pgfpathlineto{\pgfqpoint{2.680224in}{0.413320in}}%
\pgfpathlineto{\pgfqpoint{2.677650in}{0.413320in}}%
\pgfpathlineto{\pgfqpoint{2.674873in}{0.413320in}}%
\pgfpathlineto{\pgfqpoint{2.672301in}{0.413320in}}%
\pgfpathlineto{\pgfqpoint{2.669506in}{0.413320in}}%
\pgfpathlineto{\pgfqpoint{2.666836in}{0.413320in}}%
\pgfpathlineto{\pgfqpoint{2.664151in}{0.413320in}}%
\pgfpathlineto{\pgfqpoint{2.661481in}{0.413320in}}%
\pgfpathlineto{\pgfqpoint{2.658942in}{0.413320in}}%
\pgfpathlineto{\pgfqpoint{2.656124in}{0.413320in}}%
\pgfpathlineto{\pgfqpoint{2.653567in}{0.413320in}}%
\pgfpathlineto{\pgfqpoint{2.650767in}{0.413320in}}%
\pgfpathlineto{\pgfqpoint{2.648196in}{0.413320in}}%
\pgfpathlineto{\pgfqpoint{2.645408in}{0.413320in}}%
\pgfpathlineto{\pgfqpoint{2.642827in}{0.413320in}}%
\pgfpathlineto{\pgfqpoint{2.640053in}{0.413320in}}%
\pgfpathlineto{\pgfqpoint{2.637369in}{0.413320in}}%
\pgfpathlineto{\pgfqpoint{2.634700in}{0.413320in}}%
\pgfpathlineto{\pgfqpoint{2.632018in}{0.413320in}}%
\pgfpathlineto{\pgfqpoint{2.629340in}{0.413320in}}%
\pgfpathlineto{\pgfqpoint{2.626653in}{0.413320in}}%
\pgfpathlineto{\pgfqpoint{2.624077in}{0.413320in}}%
\pgfpathlineto{\pgfqpoint{2.621304in}{0.413320in}}%
\pgfpathlineto{\pgfqpoint{2.618773in}{0.413320in}}%
\pgfpathlineto{\pgfqpoint{2.615934in}{0.413320in}}%
\pgfpathlineto{\pgfqpoint{2.613393in}{0.413320in}}%
\pgfpathlineto{\pgfqpoint{2.610588in}{0.413320in}}%
\pgfpathlineto{\pgfqpoint{2.608004in}{0.413320in}}%
\pgfpathlineto{\pgfqpoint{2.605232in}{0.413320in}}%
\pgfpathlineto{\pgfqpoint{2.602557in}{0.413320in}}%
\pgfpathlineto{\pgfqpoint{2.599920in}{0.413320in}}%
\pgfpathlineto{\pgfqpoint{2.597196in}{0.413320in}}%
\pgfpathlineto{\pgfqpoint{2.594630in}{0.413320in}}%
\pgfpathlineto{\pgfqpoint{2.591842in}{0.413320in}}%
\pgfpathlineto{\pgfqpoint{2.589248in}{0.413320in}}%
\pgfpathlineto{\pgfqpoint{2.586484in}{0.413320in}}%
\pgfpathlineto{\pgfqpoint{2.583913in}{0.413320in}}%
\pgfpathlineto{\pgfqpoint{2.581129in}{0.413320in}}%
\pgfpathlineto{\pgfqpoint{2.578567in}{0.413320in}}%
\pgfpathlineto{\pgfqpoint{2.575779in}{0.413320in}}%
\pgfpathlineto{\pgfqpoint{2.573082in}{0.413320in}}%
\pgfpathlineto{\pgfqpoint{2.570411in}{0.413320in}}%
\pgfpathlineto{\pgfqpoint{2.567730in}{0.413320in}}%
\pgfpathlineto{\pgfqpoint{2.565045in}{0.413320in}}%
\pgfpathlineto{\pgfqpoint{2.562375in}{0.413320in}}%
\pgfpathlineto{\pgfqpoint{2.559790in}{0.413320in}}%
\pgfpathlineto{\pgfqpoint{2.557009in}{0.413320in}}%
\pgfpathlineto{\pgfqpoint{2.554493in}{0.413320in}}%
\pgfpathlineto{\pgfqpoint{2.551664in}{0.413320in}}%
\pgfpathlineto{\pgfqpoint{2.549114in}{0.413320in}}%
\pgfpathlineto{\pgfqpoint{2.546310in}{0.413320in}}%
\pgfpathlineto{\pgfqpoint{2.543765in}{0.413320in}}%
\pgfpathlineto{\pgfqpoint{2.540949in}{0.413320in}}%
\pgfpathlineto{\pgfqpoint{2.538274in}{0.413320in}}%
\pgfpathlineto{\pgfqpoint{2.535624in}{0.413320in}}%
\pgfpathlineto{\pgfqpoint{2.532917in}{0.413320in}}%
\pgfpathlineto{\pgfqpoint{2.530234in}{0.413320in}}%
\pgfpathlineto{\pgfqpoint{2.527560in}{0.413320in}}%
\pgfpathlineto{\pgfqpoint{2.524988in}{0.413320in}}%
\pgfpathlineto{\pgfqpoint{2.522197in}{0.413320in}}%
\pgfpathlineto{\pgfqpoint{2.519607in}{0.413320in}}%
\pgfpathlineto{\pgfqpoint{2.516845in}{0.413320in}}%
\pgfpathlineto{\pgfqpoint{2.514268in}{0.413320in}}%
\pgfpathlineto{\pgfqpoint{2.511478in}{0.413320in}}%
\pgfpathlineto{\pgfqpoint{2.508917in}{0.413320in}}%
\pgfpathlineto{\pgfqpoint{2.506163in}{0.413320in}}%
\pgfpathlineto{\pgfqpoint{2.503454in}{0.413320in}}%
\pgfpathlineto{\pgfqpoint{2.500801in}{0.413320in}}%
\pgfpathlineto{\pgfqpoint{2.498085in}{0.413320in}}%
\pgfpathlineto{\pgfqpoint{2.495542in}{0.413320in}}%
\pgfpathlineto{\pgfqpoint{2.492729in}{0.413320in}}%
\pgfpathlineto{\pgfqpoint{2.490183in}{0.413320in}}%
\pgfpathlineto{\pgfqpoint{2.487384in}{0.413320in}}%
\pgfpathlineto{\pgfqpoint{2.484870in}{0.413320in}}%
\pgfpathlineto{\pgfqpoint{2.482026in}{0.413320in}}%
\pgfpathlineto{\pgfqpoint{2.479420in}{0.413320in}}%
\pgfpathlineto{\pgfqpoint{2.476671in}{0.413320in}}%
\pgfpathlineto{\pgfqpoint{2.473989in}{0.413320in}}%
\pgfpathlineto{\pgfqpoint{2.471311in}{0.413320in}}%
\pgfpathlineto{\pgfqpoint{2.468635in}{0.413320in}}%
\pgfpathlineto{\pgfqpoint{2.465957in}{0.413320in}}%
\pgfpathlineto{\pgfqpoint{2.463280in}{0.413320in}}%
\pgfpathlineto{\pgfqpoint{2.460711in}{0.413320in}}%
\pgfpathlineto{\pgfqpoint{2.457917in}{0.413320in}}%
\pgfpathlineto{\pgfqpoint{2.455353in}{0.413320in}}%
\pgfpathlineto{\pgfqpoint{2.452562in}{0.413320in}}%
\pgfpathlineto{\pgfqpoint{2.450032in}{0.413320in}}%
\pgfpathlineto{\pgfqpoint{2.447209in}{0.413320in}}%
\pgfpathlineto{\pgfqpoint{2.444677in}{0.413320in}}%
\pgfpathlineto{\pgfqpoint{2.441876in}{0.413320in}}%
\pgfpathlineto{\pgfqpoint{2.439167in}{0.413320in}}%
\pgfpathlineto{\pgfqpoint{2.436518in}{0.413320in}}%
\pgfpathlineto{\pgfqpoint{2.433815in}{0.413320in}}%
\pgfpathlineto{\pgfqpoint{2.431251in}{0.413320in}}%
\pgfpathlineto{\pgfqpoint{2.428453in}{0.413320in}}%
\pgfpathlineto{\pgfqpoint{2.425878in}{0.413320in}}%
\pgfpathlineto{\pgfqpoint{2.423098in}{0.413320in}}%
\pgfpathlineto{\pgfqpoint{2.420528in}{0.413320in}}%
\pgfpathlineto{\pgfqpoint{2.417747in}{0.413320in}}%
\pgfpathlineto{\pgfqpoint{2.415184in}{0.413320in}}%
\pgfpathlineto{\pgfqpoint{2.412389in}{0.413320in}}%
\pgfpathlineto{\pgfqpoint{2.409699in}{0.413320in}}%
\pgfpathlineto{\pgfqpoint{2.407024in}{0.413320in}}%
\pgfpathlineto{\pgfqpoint{2.404352in}{0.413320in}}%
\pgfpathlineto{\pgfqpoint{2.401675in}{0.413320in}}%
\pgfpathlineto{\pgfqpoint{2.398995in}{0.413320in}}%
\pgfpathclose%
\pgfusepath{stroke,fill}%
\end{pgfscope}%
\begin{pgfscope}%
\pgfpathrectangle{\pgfqpoint{2.398995in}{0.319877in}}{\pgfqpoint{3.986877in}{1.993438in}} %
\pgfusepath{clip}%
\pgfsetbuttcap%
\pgfsetroundjoin%
\definecolor{currentfill}{rgb}{1.000000,1.000000,1.000000}%
\pgfsetfillcolor{currentfill}%
\pgfsetlinewidth{1.003750pt}%
\definecolor{currentstroke}{rgb}{0.845502,0.545143,0.195344}%
\pgfsetstrokecolor{currentstroke}%
\pgfsetdash{}{0pt}%
\pgfpathmoveto{\pgfqpoint{2.398995in}{0.413320in}}%
\pgfpathlineto{\pgfqpoint{2.398995in}{0.911391in}}%
\pgfpathlineto{\pgfqpoint{2.401675in}{0.909752in}}%
\pgfpathlineto{\pgfqpoint{2.404352in}{0.912806in}}%
\pgfpathlineto{\pgfqpoint{2.407024in}{0.911990in}}%
\pgfpathlineto{\pgfqpoint{2.409699in}{0.920092in}}%
\pgfpathlineto{\pgfqpoint{2.412389in}{0.920897in}}%
\pgfpathlineto{\pgfqpoint{2.415184in}{0.914047in}}%
\pgfpathlineto{\pgfqpoint{2.417747in}{0.913506in}}%
\pgfpathlineto{\pgfqpoint{2.420528in}{0.914240in}}%
\pgfpathlineto{\pgfqpoint{2.423098in}{0.915127in}}%
\pgfpathlineto{\pgfqpoint{2.425878in}{0.912675in}}%
\pgfpathlineto{\pgfqpoint{2.428453in}{0.924825in}}%
\pgfpathlineto{\pgfqpoint{2.431251in}{0.925863in}}%
\pgfpathlineto{\pgfqpoint{2.433815in}{0.920122in}}%
\pgfpathlineto{\pgfqpoint{2.436518in}{0.922491in}}%
\pgfpathlineto{\pgfqpoint{2.439167in}{0.911656in}}%
\pgfpathlineto{\pgfqpoint{2.441876in}{0.912683in}}%
\pgfpathlineto{\pgfqpoint{2.444677in}{0.909950in}}%
\pgfpathlineto{\pgfqpoint{2.447209in}{0.911690in}}%
\pgfpathlineto{\pgfqpoint{2.450032in}{0.913098in}}%
\pgfpathlineto{\pgfqpoint{2.452562in}{0.917464in}}%
\pgfpathlineto{\pgfqpoint{2.455353in}{0.919007in}}%
\pgfpathlineto{\pgfqpoint{2.457917in}{0.913417in}}%
\pgfpathlineto{\pgfqpoint{2.460711in}{0.920990in}}%
\pgfpathlineto{\pgfqpoint{2.463280in}{0.922225in}}%
\pgfpathlineto{\pgfqpoint{2.465957in}{0.919409in}}%
\pgfpathlineto{\pgfqpoint{2.468635in}{0.918441in}}%
\pgfpathlineto{\pgfqpoint{2.471311in}{0.918487in}}%
\pgfpathlineto{\pgfqpoint{2.473989in}{0.924788in}}%
\pgfpathlineto{\pgfqpoint{2.476671in}{0.916719in}}%
\pgfpathlineto{\pgfqpoint{2.479420in}{0.922934in}}%
\pgfpathlineto{\pgfqpoint{2.482026in}{0.982336in}}%
\pgfpathlineto{\pgfqpoint{2.484870in}{1.040056in}}%
\pgfpathlineto{\pgfqpoint{2.487384in}{1.019474in}}%
\pgfpathlineto{\pgfqpoint{2.490183in}{1.003162in}}%
\pgfpathlineto{\pgfqpoint{2.492729in}{0.987422in}}%
\pgfpathlineto{\pgfqpoint{2.495542in}{0.977085in}}%
\pgfpathlineto{\pgfqpoint{2.498085in}{0.961623in}}%
\pgfpathlineto{\pgfqpoint{2.500801in}{0.961116in}}%
\pgfpathlineto{\pgfqpoint{2.503454in}{0.948545in}}%
\pgfpathlineto{\pgfqpoint{2.506163in}{0.949074in}}%
\pgfpathlineto{\pgfqpoint{2.508917in}{0.940409in}}%
\pgfpathlineto{\pgfqpoint{2.511478in}{0.946597in}}%
\pgfpathlineto{\pgfqpoint{2.514268in}{0.943266in}}%
\pgfpathlineto{\pgfqpoint{2.516845in}{0.938258in}}%
\pgfpathlineto{\pgfqpoint{2.519607in}{0.935667in}}%
\pgfpathlineto{\pgfqpoint{2.522197in}{0.949990in}}%
\pgfpathlineto{\pgfqpoint{2.524988in}{0.991762in}}%
\pgfpathlineto{\pgfqpoint{2.527560in}{1.027011in}}%
\pgfpathlineto{\pgfqpoint{2.530234in}{1.013422in}}%
\pgfpathlineto{\pgfqpoint{2.532917in}{0.998902in}}%
\pgfpathlineto{\pgfqpoint{2.535624in}{0.993322in}}%
\pgfpathlineto{\pgfqpoint{2.538274in}{0.987947in}}%
\pgfpathlineto{\pgfqpoint{2.540949in}{0.975400in}}%
\pgfpathlineto{\pgfqpoint{2.543765in}{0.968391in}}%
\pgfpathlineto{\pgfqpoint{2.546310in}{0.957797in}}%
\pgfpathlineto{\pgfqpoint{2.549114in}{0.951339in}}%
\pgfpathlineto{\pgfqpoint{2.551664in}{0.955697in}}%
\pgfpathlineto{\pgfqpoint{2.554493in}{0.951251in}}%
\pgfpathlineto{\pgfqpoint{2.557009in}{0.936657in}}%
\pgfpathlineto{\pgfqpoint{2.559790in}{0.940133in}}%
\pgfpathlineto{\pgfqpoint{2.562375in}{0.942433in}}%
\pgfpathlineto{\pgfqpoint{2.565045in}{0.941304in}}%
\pgfpathlineto{\pgfqpoint{2.567730in}{0.937601in}}%
\pgfpathlineto{\pgfqpoint{2.570411in}{0.935603in}}%
\pgfpathlineto{\pgfqpoint{2.573082in}{0.921939in}}%
\pgfpathlineto{\pgfqpoint{2.575779in}{0.922208in}}%
\pgfpathlineto{\pgfqpoint{2.578567in}{0.917766in}}%
\pgfpathlineto{\pgfqpoint{2.581129in}{0.916623in}}%
\pgfpathlineto{\pgfqpoint{2.583913in}{0.920740in}}%
\pgfpathlineto{\pgfqpoint{2.586484in}{0.922222in}}%
\pgfpathlineto{\pgfqpoint{2.589248in}{0.923776in}}%
\pgfpathlineto{\pgfqpoint{2.591842in}{0.920211in}}%
\pgfpathlineto{\pgfqpoint{2.594630in}{0.917438in}}%
\pgfpathlineto{\pgfqpoint{2.597196in}{0.917325in}}%
\pgfpathlineto{\pgfqpoint{2.599920in}{0.915136in}}%
\pgfpathlineto{\pgfqpoint{2.602557in}{0.918268in}}%
\pgfpathlineto{\pgfqpoint{2.605232in}{0.913319in}}%
\pgfpathlineto{\pgfqpoint{2.608004in}{0.910546in}}%
\pgfpathlineto{\pgfqpoint{2.610588in}{0.915890in}}%
\pgfpathlineto{\pgfqpoint{2.613393in}{0.916598in}}%
\pgfpathlineto{\pgfqpoint{2.615934in}{0.922015in}}%
\pgfpathlineto{\pgfqpoint{2.618773in}{0.917701in}}%
\pgfpathlineto{\pgfqpoint{2.621304in}{0.917906in}}%
\pgfpathlineto{\pgfqpoint{2.624077in}{0.914294in}}%
\pgfpathlineto{\pgfqpoint{2.626653in}{0.922134in}}%
\pgfpathlineto{\pgfqpoint{2.629340in}{0.918746in}}%
\pgfpathlineto{\pgfqpoint{2.632018in}{0.916408in}}%
\pgfpathlineto{\pgfqpoint{2.634700in}{0.916740in}}%
\pgfpathlineto{\pgfqpoint{2.637369in}{0.918156in}}%
\pgfpathlineto{\pgfqpoint{2.640053in}{0.916395in}}%
\pgfpathlineto{\pgfqpoint{2.642827in}{0.912968in}}%
\pgfpathlineto{\pgfqpoint{2.645408in}{0.915292in}}%
\pgfpathlineto{\pgfqpoint{2.648196in}{0.910498in}}%
\pgfpathlineto{\pgfqpoint{2.650767in}{0.905079in}}%
\pgfpathlineto{\pgfqpoint{2.653567in}{0.907410in}}%
\pgfpathlineto{\pgfqpoint{2.656124in}{0.909821in}}%
\pgfpathlineto{\pgfqpoint{2.658942in}{0.908726in}}%
\pgfpathlineto{\pgfqpoint{2.661481in}{0.910090in}}%
\pgfpathlineto{\pgfqpoint{2.664151in}{0.908652in}}%
\pgfpathlineto{\pgfqpoint{2.666836in}{0.910190in}}%
\pgfpathlineto{\pgfqpoint{2.669506in}{0.911106in}}%
\pgfpathlineto{\pgfqpoint{2.672301in}{0.912478in}}%
\pgfpathlineto{\pgfqpoint{2.674873in}{0.917054in}}%
\pgfpathlineto{\pgfqpoint{2.677650in}{0.918313in}}%
\pgfpathlineto{\pgfqpoint{2.680224in}{0.918467in}}%
\pgfpathlineto{\pgfqpoint{2.683009in}{0.914360in}}%
\pgfpathlineto{\pgfqpoint{2.685586in}{0.918116in}}%
\pgfpathlineto{\pgfqpoint{2.688328in}{0.916833in}}%
\pgfpathlineto{\pgfqpoint{2.690940in}{0.920847in}}%
\pgfpathlineto{\pgfqpoint{2.693611in}{0.920204in}}%
\pgfpathlineto{\pgfqpoint{2.696293in}{0.916334in}}%
\pgfpathlineto{\pgfqpoint{2.698968in}{0.919813in}}%
\pgfpathlineto{\pgfqpoint{2.701657in}{0.921068in}}%
\pgfpathlineto{\pgfqpoint{2.704326in}{0.919199in}}%
\pgfpathlineto{\pgfqpoint{2.707125in}{0.915732in}}%
\pgfpathlineto{\pgfqpoint{2.709683in}{0.925027in}}%
\pgfpathlineto{\pgfqpoint{2.712477in}{0.925227in}}%
\pgfpathlineto{\pgfqpoint{2.715036in}{0.922024in}}%
\pgfpathlineto{\pgfqpoint{2.717773in}{0.919841in}}%
\pgfpathlineto{\pgfqpoint{2.720404in}{0.917880in}}%
\pgfpathlineto{\pgfqpoint{2.723211in}{0.919422in}}%
\pgfpathlineto{\pgfqpoint{2.725760in}{0.912094in}}%
\pgfpathlineto{\pgfqpoint{2.728439in}{0.919259in}}%
\pgfpathlineto{\pgfqpoint{2.731119in}{0.919714in}}%
\pgfpathlineto{\pgfqpoint{2.733798in}{0.924021in}}%
\pgfpathlineto{\pgfqpoint{2.736476in}{0.920575in}}%
\pgfpathlineto{\pgfqpoint{2.739155in}{0.915353in}}%
\pgfpathlineto{\pgfqpoint{2.741928in}{0.914346in}}%
\pgfpathlineto{\pgfqpoint{2.744510in}{0.910294in}}%
\pgfpathlineto{\pgfqpoint{2.747260in}{0.914147in}}%
\pgfpathlineto{\pgfqpoint{2.749868in}{0.915893in}}%
\pgfpathlineto{\pgfqpoint{2.752614in}{0.911390in}}%
\pgfpathlineto{\pgfqpoint{2.755224in}{0.914300in}}%
\pgfpathlineto{\pgfqpoint{2.758028in}{0.915579in}}%
\pgfpathlineto{\pgfqpoint{2.760581in}{0.925430in}}%
\pgfpathlineto{\pgfqpoint{2.763253in}{0.918077in}}%
\pgfpathlineto{\pgfqpoint{2.765935in}{0.920074in}}%
\pgfpathlineto{\pgfqpoint{2.768617in}{0.917595in}}%
\pgfpathlineto{\pgfqpoint{2.771373in}{0.914113in}}%
\pgfpathlineto{\pgfqpoint{2.773972in}{0.919781in}}%
\pgfpathlineto{\pgfqpoint{2.776767in}{0.917299in}}%
\pgfpathlineto{\pgfqpoint{2.779330in}{0.916206in}}%
\pgfpathlineto{\pgfqpoint{2.782113in}{0.915580in}}%
\pgfpathlineto{\pgfqpoint{2.784687in}{0.922554in}}%
\pgfpathlineto{\pgfqpoint{2.787468in}{0.915932in}}%
\pgfpathlineto{\pgfqpoint{2.790044in}{0.915765in}}%
\pgfpathlineto{\pgfqpoint{2.792721in}{0.918946in}}%
\pgfpathlineto{\pgfqpoint{2.795398in}{0.923134in}}%
\pgfpathlineto{\pgfqpoint{2.798070in}{0.923554in}}%
\pgfpathlineto{\pgfqpoint{2.800756in}{0.920345in}}%
\pgfpathlineto{\pgfqpoint{2.803435in}{0.916045in}}%
\pgfpathlineto{\pgfqpoint{2.806175in}{0.916199in}}%
\pgfpathlineto{\pgfqpoint{2.808792in}{0.918648in}}%
\pgfpathlineto{\pgfqpoint{2.811597in}{0.918362in}}%
\pgfpathlineto{\pgfqpoint{2.814141in}{0.918324in}}%
\pgfpathlineto{\pgfqpoint{2.816867in}{0.919339in}}%
\pgfpathlineto{\pgfqpoint{2.819506in}{0.909965in}}%
\pgfpathlineto{\pgfqpoint{2.822303in}{0.910745in}}%
\pgfpathlineto{\pgfqpoint{2.824851in}{0.911304in}}%
\pgfpathlineto{\pgfqpoint{2.827567in}{0.912045in}}%
\pgfpathlineto{\pgfqpoint{2.830219in}{0.916061in}}%
\pgfpathlineto{\pgfqpoint{2.832894in}{0.911905in}}%
\pgfpathlineto{\pgfqpoint{2.835698in}{0.912225in}}%
\pgfpathlineto{\pgfqpoint{2.838254in}{0.922512in}}%
\pgfpathlineto{\pgfqpoint{2.841055in}{0.935214in}}%
\pgfpathlineto{\pgfqpoint{2.843611in}{0.931759in}}%
\pgfpathlineto{\pgfqpoint{2.846408in}{0.922222in}}%
\pgfpathlineto{\pgfqpoint{2.848960in}{0.916964in}}%
\pgfpathlineto{\pgfqpoint{2.851793in}{0.917936in}}%
\pgfpathlineto{\pgfqpoint{2.854325in}{0.916151in}}%
\pgfpathlineto{\pgfqpoint{2.857003in}{0.914363in}}%
\pgfpathlineto{\pgfqpoint{2.859668in}{0.913845in}}%
\pgfpathlineto{\pgfqpoint{2.862402in}{0.918667in}}%
\pgfpathlineto{\pgfqpoint{2.865031in}{0.924410in}}%
\pgfpathlineto{\pgfqpoint{2.867713in}{0.920861in}}%
\pgfpathlineto{\pgfqpoint{2.870475in}{0.920242in}}%
\pgfpathlineto{\pgfqpoint{2.873074in}{0.924483in}}%
\pgfpathlineto{\pgfqpoint{2.875882in}{0.921635in}}%
\pgfpathlineto{\pgfqpoint{2.878431in}{0.915983in}}%
\pgfpathlineto{\pgfqpoint{2.881254in}{0.918545in}}%
\pgfpathlineto{\pgfqpoint{2.883780in}{0.913717in}}%
\pgfpathlineto{\pgfqpoint{2.886578in}{0.919014in}}%
\pgfpathlineto{\pgfqpoint{2.889145in}{0.916694in}}%
\pgfpathlineto{\pgfqpoint{2.891809in}{0.912450in}}%
\pgfpathlineto{\pgfqpoint{2.894487in}{0.919003in}}%
\pgfpathlineto{\pgfqpoint{2.897179in}{0.919692in}}%
\pgfpathlineto{\pgfqpoint{2.899858in}{0.918734in}}%
\pgfpathlineto{\pgfqpoint{2.902535in}{0.920908in}}%
\pgfpathlineto{\pgfqpoint{2.905341in}{0.919860in}}%
\pgfpathlineto{\pgfqpoint{2.907882in}{0.921730in}}%
\pgfpathlineto{\pgfqpoint{2.910631in}{0.919520in}}%
\pgfpathlineto{\pgfqpoint{2.913243in}{0.916730in}}%
\pgfpathlineto{\pgfqpoint{2.916061in}{0.907998in}}%
\pgfpathlineto{\pgfqpoint{2.918606in}{0.914515in}}%
\pgfpathlineto{\pgfqpoint{2.921363in}{0.911807in}}%
\pgfpathlineto{\pgfqpoint{2.923963in}{0.909604in}}%
\pgfpathlineto{\pgfqpoint{2.926655in}{0.912390in}}%
\pgfpathlineto{\pgfqpoint{2.929321in}{0.906887in}}%
\pgfpathlineto{\pgfqpoint{2.932033in}{0.910058in}}%
\pgfpathlineto{\pgfqpoint{2.934759in}{0.904540in}}%
\pgfpathlineto{\pgfqpoint{2.937352in}{0.907863in}}%
\pgfpathlineto{\pgfqpoint{2.940120in}{0.911059in}}%
\pgfpathlineto{\pgfqpoint{2.942711in}{0.905780in}}%
\pgfpathlineto{\pgfqpoint{2.945461in}{0.904309in}}%
\pgfpathlineto{\pgfqpoint{2.948068in}{0.915206in}}%
\pgfpathlineto{\pgfqpoint{2.950884in}{0.917569in}}%
\pgfpathlineto{\pgfqpoint{2.953422in}{0.918233in}}%
\pgfpathlineto{\pgfqpoint{2.956103in}{0.920897in}}%
\pgfpathlineto{\pgfqpoint{2.958782in}{0.914652in}}%
\pgfpathlineto{\pgfqpoint{2.961460in}{0.912237in}}%
\pgfpathlineto{\pgfqpoint{2.964127in}{0.914973in}}%
\pgfpathlineto{\pgfqpoint{2.966812in}{0.917044in}}%
\pgfpathlineto{\pgfqpoint{2.969599in}{0.911192in}}%
\pgfpathlineto{\pgfqpoint{2.972177in}{0.912025in}}%
\pgfpathlineto{\pgfqpoint{2.974972in}{0.915015in}}%
\pgfpathlineto{\pgfqpoint{2.977517in}{0.910766in}}%
\pgfpathlineto{\pgfqpoint{2.980341in}{0.912004in}}%
\pgfpathlineto{\pgfqpoint{2.982885in}{0.912600in}}%
\pgfpathlineto{\pgfqpoint{2.985666in}{0.914205in}}%
\pgfpathlineto{\pgfqpoint{2.988238in}{0.912346in}}%
\pgfpathlineto{\pgfqpoint{2.990978in}{0.918225in}}%
\pgfpathlineto{\pgfqpoint{2.993595in}{0.921775in}}%
\pgfpathlineto{\pgfqpoint{2.996300in}{0.929680in}}%
\pgfpathlineto{\pgfqpoint{2.999103in}{0.946807in}}%
\pgfpathlineto{\pgfqpoint{3.001635in}{0.940015in}}%
\pgfpathlineto{\pgfqpoint{3.004419in}{0.929837in}}%
\pgfpathlineto{\pgfqpoint{3.006993in}{0.925200in}}%
\pgfpathlineto{\pgfqpoint{3.009784in}{0.923725in}}%
\pgfpathlineto{\pgfqpoint{3.012351in}{0.936566in}}%
\pgfpathlineto{\pgfqpoint{3.015097in}{0.928530in}}%
\pgfpathlineto{\pgfqpoint{3.017707in}{0.926716in}}%
\pgfpathlineto{\pgfqpoint{3.020382in}{0.931522in}}%
\pgfpathlineto{\pgfqpoint{3.023058in}{0.942109in}}%
\pgfpathlineto{\pgfqpoint{3.025803in}{0.945820in}}%
\pgfpathlineto{\pgfqpoint{3.028412in}{0.944959in}}%
\pgfpathlineto{\pgfqpoint{3.031091in}{0.929267in}}%
\pgfpathlineto{\pgfqpoint{3.033921in}{0.923515in}}%
\pgfpathlineto{\pgfqpoint{3.036456in}{0.925694in}}%
\pgfpathlineto{\pgfqpoint{3.039262in}{0.922954in}}%
\pgfpathlineto{\pgfqpoint{3.041813in}{0.924682in}}%
\pgfpathlineto{\pgfqpoint{3.044568in}{0.921528in}}%
\pgfpathlineto{\pgfqpoint{3.047157in}{0.926340in}}%
\pgfpathlineto{\pgfqpoint{3.049988in}{0.926921in}}%
\pgfpathlineto{\pgfqpoint{3.052526in}{0.922982in}}%
\pgfpathlineto{\pgfqpoint{3.055202in}{0.918941in}}%
\pgfpathlineto{\pgfqpoint{3.057884in}{0.920456in}}%
\pgfpathlineto{\pgfqpoint{3.060561in}{0.916392in}}%
\pgfpathlineto{\pgfqpoint{3.063230in}{0.911968in}}%
\pgfpathlineto{\pgfqpoint{3.065916in}{0.919502in}}%
\pgfpathlineto{\pgfqpoint{3.068709in}{0.916160in}}%
\pgfpathlineto{\pgfqpoint{3.071266in}{0.916319in}}%
\pgfpathlineto{\pgfqpoint{3.074056in}{0.914673in}}%
\pgfpathlineto{\pgfqpoint{3.076631in}{0.915133in}}%
\pgfpathlineto{\pgfqpoint{3.079381in}{0.912817in}}%
\pgfpathlineto{\pgfqpoint{3.081990in}{0.911626in}}%
\pgfpathlineto{\pgfqpoint{3.084671in}{0.910722in}}%
\pgfpathlineto{\pgfqpoint{3.087343in}{0.911978in}}%
\pgfpathlineto{\pgfqpoint{3.090023in}{0.917915in}}%
\pgfpathlineto{\pgfqpoint{3.092699in}{0.907301in}}%
\pgfpathlineto{\pgfqpoint{3.095388in}{0.908884in}}%
\pgfpathlineto{\pgfqpoint{3.098163in}{0.913284in}}%
\pgfpathlineto{\pgfqpoint{3.100737in}{0.912349in}}%
\pgfpathlineto{\pgfqpoint{3.103508in}{0.913181in}}%
\pgfpathlineto{\pgfqpoint{3.106094in}{0.908850in}}%
\pgfpathlineto{\pgfqpoint{3.108896in}{0.911995in}}%
\pgfpathlineto{\pgfqpoint{3.111451in}{0.907793in}}%
\pgfpathlineto{\pgfqpoint{3.114242in}{0.909093in}}%
\pgfpathlineto{\pgfqpoint{3.116807in}{0.908159in}}%
\pgfpathlineto{\pgfqpoint{3.119487in}{0.907499in}}%
\pgfpathlineto{\pgfqpoint{3.122163in}{0.911846in}}%
\pgfpathlineto{\pgfqpoint{3.124842in}{0.916301in}}%
\pgfpathlineto{\pgfqpoint{3.127512in}{0.918194in}}%
\pgfpathlineto{\pgfqpoint{3.130199in}{0.912975in}}%
\pgfpathlineto{\pgfqpoint{3.132946in}{0.918893in}}%
\pgfpathlineto{\pgfqpoint{3.135550in}{0.919466in}}%
\pgfpathlineto{\pgfqpoint{3.138375in}{0.914322in}}%
\pgfpathlineto{\pgfqpoint{3.140913in}{0.906863in}}%
\pgfpathlineto{\pgfqpoint{3.143740in}{0.907686in}}%
\pgfpathlineto{\pgfqpoint{3.146271in}{0.904309in}}%
\pgfpathlineto{\pgfqpoint{3.149057in}{0.904309in}}%
\pgfpathlineto{\pgfqpoint{3.151612in}{0.906078in}}%
\pgfpathlineto{\pgfqpoint{3.154327in}{0.917533in}}%
\pgfpathlineto{\pgfqpoint{3.156981in}{0.921914in}}%
\pgfpathlineto{\pgfqpoint{3.159675in}{0.914919in}}%
\pgfpathlineto{\pgfqpoint{3.162474in}{0.913286in}}%
\pgfpathlineto{\pgfqpoint{3.165019in}{0.912189in}}%
\pgfpathlineto{\pgfqpoint{3.167776in}{0.904443in}}%
\pgfpathlineto{\pgfqpoint{3.170375in}{0.905279in}}%
\pgfpathlineto{\pgfqpoint{3.173142in}{0.909302in}}%
\pgfpathlineto{\pgfqpoint{3.175724in}{0.914865in}}%
\pgfpathlineto{\pgfqpoint{3.178525in}{0.920574in}}%
\pgfpathlineto{\pgfqpoint{3.181089in}{0.920333in}}%
\pgfpathlineto{\pgfqpoint{3.183760in}{0.921867in}}%
\pgfpathlineto{\pgfqpoint{3.186440in}{0.914980in}}%
\pgfpathlineto{\pgfqpoint{3.189117in}{0.919150in}}%
\pgfpathlineto{\pgfqpoint{3.191796in}{0.913539in}}%
\pgfpathlineto{\pgfqpoint{3.194508in}{0.917511in}}%
\pgfpathlineto{\pgfqpoint{3.197226in}{0.915080in}}%
\pgfpathlineto{\pgfqpoint{3.199823in}{0.910717in}}%
\pgfpathlineto{\pgfqpoint{3.202562in}{0.911713in}}%
\pgfpathlineto{\pgfqpoint{3.205195in}{0.915657in}}%
\pgfpathlineto{\pgfqpoint{3.207984in}{0.917791in}}%
\pgfpathlineto{\pgfqpoint{3.210545in}{0.907716in}}%
\pgfpathlineto{\pgfqpoint{3.213342in}{0.909336in}}%
\pgfpathlineto{\pgfqpoint{3.215908in}{0.911428in}}%
\pgfpathlineto{\pgfqpoint{3.218586in}{0.911115in}}%
\pgfpathlineto{\pgfqpoint{3.221255in}{0.915649in}}%
\pgfpathlineto{\pgfqpoint{3.223942in}{0.918500in}}%
\pgfpathlineto{\pgfqpoint{3.226609in}{0.921302in}}%
\pgfpathlineto{\pgfqpoint{3.229310in}{0.915454in}}%
\pgfpathlineto{\pgfqpoint{3.232069in}{0.918445in}}%
\pgfpathlineto{\pgfqpoint{3.234658in}{0.919181in}}%
\pgfpathlineto{\pgfqpoint{3.237411in}{0.917715in}}%
\pgfpathlineto{\pgfqpoint{3.240010in}{0.917477in}}%
\pgfpathlineto{\pgfqpoint{3.242807in}{0.918823in}}%
\pgfpathlineto{\pgfqpoint{3.245363in}{0.917493in}}%
\pgfpathlineto{\pgfqpoint{3.248049in}{0.920388in}}%
\pgfpathlineto{\pgfqpoint{3.250716in}{0.919588in}}%
\pgfpathlineto{\pgfqpoint{3.253404in}{0.919096in}}%
\pgfpathlineto{\pgfqpoint{3.256083in}{0.920502in}}%
\pgfpathlineto{\pgfqpoint{3.258784in}{0.916953in}}%
\pgfpathlineto{\pgfqpoint{3.261594in}{0.915409in}}%
\pgfpathlineto{\pgfqpoint{3.264119in}{0.906088in}}%
\pgfpathlineto{\pgfqpoint{3.266849in}{0.908359in}}%
\pgfpathlineto{\pgfqpoint{3.269478in}{0.908642in}}%
\pgfpathlineto{\pgfqpoint{3.272254in}{0.909622in}}%
\pgfpathlineto{\pgfqpoint{3.274831in}{0.907459in}}%
\pgfpathlineto{\pgfqpoint{3.277603in}{0.918331in}}%
\pgfpathlineto{\pgfqpoint{3.280189in}{0.918808in}}%
\pgfpathlineto{\pgfqpoint{3.282870in}{0.920652in}}%
\pgfpathlineto{\pgfqpoint{3.285534in}{0.915286in}}%
\pgfpathlineto{\pgfqpoint{3.288225in}{0.920205in}}%
\pgfpathlineto{\pgfqpoint{3.290890in}{0.920978in}}%
\pgfpathlineto{\pgfqpoint{3.293574in}{0.919442in}}%
\pgfpathlineto{\pgfqpoint{3.296376in}{0.917756in}}%
\pgfpathlineto{\pgfqpoint{3.298937in}{0.920392in}}%
\pgfpathlineto{\pgfqpoint{3.301719in}{0.918582in}}%
\pgfpathlineto{\pgfqpoint{3.304295in}{0.914967in}}%
\pgfpathlineto{\pgfqpoint{3.307104in}{0.916812in}}%
\pgfpathlineto{\pgfqpoint{3.309652in}{0.920598in}}%
\pgfpathlineto{\pgfqpoint{3.312480in}{0.912058in}}%
\pgfpathlineto{\pgfqpoint{3.315008in}{0.917476in}}%
\pgfpathlineto{\pgfqpoint{3.317688in}{0.914373in}}%
\pgfpathlineto{\pgfqpoint{3.320366in}{0.918487in}}%
\pgfpathlineto{\pgfqpoint{3.323049in}{0.919053in}}%
\pgfpathlineto{\pgfqpoint{3.325860in}{0.917986in}}%
\pgfpathlineto{\pgfqpoint{3.328401in}{0.917554in}}%
\pgfpathlineto{\pgfqpoint{3.331183in}{0.922441in}}%
\pgfpathlineto{\pgfqpoint{3.333758in}{0.917562in}}%
\pgfpathlineto{\pgfqpoint{3.336541in}{0.916761in}}%
\pgfpathlineto{\pgfqpoint{3.339101in}{0.917833in}}%
\pgfpathlineto{\pgfqpoint{3.341893in}{0.917752in}}%
\pgfpathlineto{\pgfqpoint{3.344468in}{0.917634in}}%
\pgfpathlineto{\pgfqpoint{3.347139in}{0.914114in}}%
\pgfpathlineto{\pgfqpoint{3.349828in}{0.919891in}}%
\pgfpathlineto{\pgfqpoint{3.352505in}{0.923049in}}%
\pgfpathlineto{\pgfqpoint{3.355177in}{0.925723in}}%
\pgfpathlineto{\pgfqpoint{3.357862in}{0.920463in}}%
\pgfpathlineto{\pgfqpoint{3.360620in}{0.923224in}}%
\pgfpathlineto{\pgfqpoint{3.363221in}{0.923812in}}%
\pgfpathlineto{\pgfqpoint{3.365996in}{0.924250in}}%
\pgfpathlineto{\pgfqpoint{3.368577in}{0.920192in}}%
\pgfpathlineto{\pgfqpoint{3.371357in}{0.916550in}}%
\pgfpathlineto{\pgfqpoint{3.373921in}{0.912785in}}%
\pgfpathlineto{\pgfqpoint{3.376735in}{0.914374in}}%
\pgfpathlineto{\pgfqpoint{3.379290in}{0.918016in}}%
\pgfpathlineto{\pgfqpoint{3.381959in}{0.917659in}}%
\pgfpathlineto{\pgfqpoint{3.384647in}{0.917565in}}%
\pgfpathlineto{\pgfqpoint{3.387309in}{0.916693in}}%
\pgfpathlineto{\pgfqpoint{3.390102in}{0.920776in}}%
\pgfpathlineto{\pgfqpoint{3.392681in}{0.918669in}}%
\pgfpathlineto{\pgfqpoint{3.395461in}{0.920780in}}%
\pgfpathlineto{\pgfqpoint{3.398037in}{0.916207in}}%
\pgfpathlineto{\pgfqpoint{3.400783in}{0.915680in}}%
\pgfpathlineto{\pgfqpoint{3.403394in}{0.921773in}}%
\pgfpathlineto{\pgfqpoint{3.406202in}{0.921323in}}%
\pgfpathlineto{\pgfqpoint{3.408752in}{0.922902in}}%
\pgfpathlineto{\pgfqpoint{3.411431in}{0.921403in}}%
\pgfpathlineto{\pgfqpoint{3.414109in}{0.921877in}}%
\pgfpathlineto{\pgfqpoint{3.416780in}{0.922372in}}%
\pgfpathlineto{\pgfqpoint{3.419455in}{0.921465in}}%
\pgfpathlineto{\pgfqpoint{3.422142in}{0.920319in}}%
\pgfpathlineto{\pgfqpoint{3.424887in}{0.926948in}}%
\pgfpathlineto{\pgfqpoint{3.427501in}{0.921746in}}%
\pgfpathlineto{\pgfqpoint{3.430313in}{0.920838in}}%
\pgfpathlineto{\pgfqpoint{3.432851in}{0.918804in}}%
\pgfpathlineto{\pgfqpoint{3.435635in}{0.921821in}}%
\pgfpathlineto{\pgfqpoint{3.438210in}{0.920202in}}%
\pgfpathlineto{\pgfqpoint{3.440996in}{0.920714in}}%
\pgfpathlineto{\pgfqpoint{3.443574in}{0.920297in}}%
\pgfpathlineto{\pgfqpoint{3.446257in}{0.920702in}}%
\pgfpathlineto{\pgfqpoint{3.448926in}{0.921552in}}%
\pgfpathlineto{\pgfqpoint{3.451597in}{0.920245in}}%
\pgfpathlineto{\pgfqpoint{3.454285in}{0.928414in}}%
\pgfpathlineto{\pgfqpoint{3.456960in}{0.932481in}}%
\pgfpathlineto{\pgfqpoint{3.459695in}{0.926202in}}%
\pgfpathlineto{\pgfqpoint{3.462321in}{0.927843in}}%
\pgfpathlineto{\pgfqpoint{3.465072in}{0.935724in}}%
\pgfpathlineto{\pgfqpoint{3.467678in}{0.951222in}}%
\pgfpathlineto{\pgfqpoint{3.470466in}{0.938945in}}%
\pgfpathlineto{\pgfqpoint{3.473021in}{0.933595in}}%
\pgfpathlineto{\pgfqpoint{3.475821in}{0.925337in}}%
\pgfpathlineto{\pgfqpoint{3.478378in}{0.920067in}}%
\pgfpathlineto{\pgfqpoint{3.481072in}{0.923589in}}%
\pgfpathlineto{\pgfqpoint{3.483744in}{0.933442in}}%
\pgfpathlineto{\pgfqpoint{3.486442in}{0.925932in}}%
\pgfpathlineto{\pgfqpoint{3.489223in}{0.921993in}}%
\pgfpathlineto{\pgfqpoint{3.491783in}{0.923075in}}%
\pgfpathlineto{\pgfqpoint{3.494581in}{0.920622in}}%
\pgfpathlineto{\pgfqpoint{3.497139in}{0.919953in}}%
\pgfpathlineto{\pgfqpoint{3.499909in}{0.922165in}}%
\pgfpathlineto{\pgfqpoint{3.502488in}{0.922062in}}%
\pgfpathlineto{\pgfqpoint{3.505262in}{0.921421in}}%
\pgfpathlineto{\pgfqpoint{3.507840in}{0.926880in}}%
\pgfpathlineto{\pgfqpoint{3.510533in}{0.923546in}}%
\pgfpathlineto{\pgfqpoint{3.513209in}{0.922350in}}%
\pgfpathlineto{\pgfqpoint{3.515884in}{0.917334in}}%
\pgfpathlineto{\pgfqpoint{3.518565in}{0.922546in}}%
\pgfpathlineto{\pgfqpoint{3.521244in}{0.917002in}}%
\pgfpathlineto{\pgfqpoint{3.524041in}{0.923213in}}%
\pgfpathlineto{\pgfqpoint{3.526601in}{0.917882in}}%
\pgfpathlineto{\pgfqpoint{3.529327in}{0.916610in}}%
\pgfpathlineto{\pgfqpoint{3.531955in}{0.917970in}}%
\pgfpathlineto{\pgfqpoint{3.534783in}{0.922997in}}%
\pgfpathlineto{\pgfqpoint{3.537309in}{0.920858in}}%
\pgfpathlineto{\pgfqpoint{3.540093in}{0.925632in}}%
\pgfpathlineto{\pgfqpoint{3.542656in}{0.919874in}}%
\pgfpathlineto{\pgfqpoint{3.545349in}{0.921389in}}%
\pgfpathlineto{\pgfqpoint{3.548029in}{0.920382in}}%
\pgfpathlineto{\pgfqpoint{3.550713in}{0.923326in}}%
\pgfpathlineto{\pgfqpoint{3.553498in}{0.928163in}}%
\pgfpathlineto{\pgfqpoint{3.556061in}{0.925784in}}%
\pgfpathlineto{\pgfqpoint{3.558853in}{0.923610in}}%
\pgfpathlineto{\pgfqpoint{3.561420in}{0.923027in}}%
\pgfpathlineto{\pgfqpoint{3.564188in}{0.922420in}}%
\pgfpathlineto{\pgfqpoint{3.566774in}{0.928196in}}%
\pgfpathlineto{\pgfqpoint{3.569584in}{0.922589in}}%
\pgfpathlineto{\pgfqpoint{3.572126in}{0.920244in}}%
\pgfpathlineto{\pgfqpoint{3.574814in}{0.921888in}}%
\pgfpathlineto{\pgfqpoint{3.577487in}{0.921677in}}%
\pgfpathlineto{\pgfqpoint{3.580191in}{0.923588in}}%
\pgfpathlineto{\pgfqpoint{3.582851in}{0.925194in}}%
\pgfpathlineto{\pgfqpoint{3.585532in}{0.921241in}}%
\pgfpathlineto{\pgfqpoint{3.588258in}{0.921469in}}%
\pgfpathlineto{\pgfqpoint{3.590883in}{0.917643in}}%
\pgfpathlineto{\pgfqpoint{3.593620in}{0.921984in}}%
\pgfpathlineto{\pgfqpoint{3.596240in}{0.921316in}}%
\pgfpathlineto{\pgfqpoint{3.598998in}{0.920385in}}%
\pgfpathlineto{\pgfqpoint{3.601590in}{0.920758in}}%
\pgfpathlineto{\pgfqpoint{3.604387in}{0.919696in}}%
\pgfpathlineto{\pgfqpoint{3.606951in}{0.921247in}}%
\pgfpathlineto{\pgfqpoint{3.609632in}{0.917127in}}%
\pgfpathlineto{\pgfqpoint{3.612311in}{0.913008in}}%
\pgfpathlineto{\pgfqpoint{3.614982in}{0.911682in}}%
\pgfpathlineto{\pgfqpoint{3.617667in}{0.906269in}}%
\pgfpathlineto{\pgfqpoint{3.620345in}{0.911864in}}%
\pgfpathlineto{\pgfqpoint{3.623165in}{0.911730in}}%
\pgfpathlineto{\pgfqpoint{3.625689in}{0.917241in}}%
\pgfpathlineto{\pgfqpoint{3.628460in}{0.915981in}}%
\pgfpathlineto{\pgfqpoint{3.631058in}{0.911592in}}%
\pgfpathlineto{\pgfqpoint{3.633858in}{0.917938in}}%
\pgfpathlineto{\pgfqpoint{3.636413in}{0.919304in}}%
\pgfpathlineto{\pgfqpoint{3.639207in}{0.923388in}}%
\pgfpathlineto{\pgfqpoint{3.641773in}{0.925341in}}%
\pgfpathlineto{\pgfqpoint{3.644452in}{0.925955in}}%
\pgfpathlineto{\pgfqpoint{3.647130in}{0.912886in}}%
\pgfpathlineto{\pgfqpoint{3.649837in}{0.913890in}}%
\pgfpathlineto{\pgfqpoint{3.652628in}{0.914622in}}%
\pgfpathlineto{\pgfqpoint{3.655165in}{0.916330in}}%
\pgfpathlineto{\pgfqpoint{3.657917in}{0.910502in}}%
\pgfpathlineto{\pgfqpoint{3.660515in}{0.907878in}}%
\pgfpathlineto{\pgfqpoint{3.663276in}{0.904309in}}%
\pgfpathlineto{\pgfqpoint{3.665864in}{0.909384in}}%
\pgfpathlineto{\pgfqpoint{3.668665in}{0.908476in}}%
\pgfpathlineto{\pgfqpoint{3.671232in}{0.905537in}}%
\pgfpathlineto{\pgfqpoint{3.673911in}{0.904309in}}%
\pgfpathlineto{\pgfqpoint{3.676591in}{0.904309in}}%
\pgfpathlineto{\pgfqpoint{3.679273in}{0.904309in}}%
\pgfpathlineto{\pgfqpoint{3.681948in}{0.904309in}}%
\pgfpathlineto{\pgfqpoint{3.684620in}{0.904309in}}%
\pgfpathlineto{\pgfqpoint{3.687442in}{0.906482in}}%
\pgfpathlineto{\pgfqpoint{3.689983in}{0.905856in}}%
\pgfpathlineto{\pgfqpoint{3.692765in}{0.907311in}}%
\pgfpathlineto{\pgfqpoint{3.695331in}{0.906285in}}%
\pgfpathlineto{\pgfqpoint{3.698125in}{0.906271in}}%
\pgfpathlineto{\pgfqpoint{3.700684in}{0.907394in}}%
\pgfpathlineto{\pgfqpoint{3.703460in}{0.912025in}}%
\pgfpathlineto{\pgfqpoint{3.706053in}{0.909624in}}%
\pgfpathlineto{\pgfqpoint{3.708729in}{0.912191in}}%
\pgfpathlineto{\pgfqpoint{3.711410in}{0.910911in}}%
\pgfpathlineto{\pgfqpoint{3.714086in}{0.908959in}}%
\pgfpathlineto{\pgfqpoint{3.716875in}{0.915196in}}%
\pgfpathlineto{\pgfqpoint{3.719446in}{0.937914in}}%
\pgfpathlineto{\pgfqpoint{3.722228in}{0.925459in}}%
\pgfpathlineto{\pgfqpoint{3.724804in}{0.917986in}}%
\pgfpathlineto{\pgfqpoint{3.727581in}{0.919869in}}%
\pgfpathlineto{\pgfqpoint{3.730158in}{0.915548in}}%
\pgfpathlineto{\pgfqpoint{3.732950in}{0.918685in}}%
\pgfpathlineto{\pgfqpoint{3.735509in}{0.921263in}}%
\pgfpathlineto{\pgfqpoint{3.738194in}{0.921150in}}%
\pgfpathlineto{\pgfqpoint{3.740874in}{0.918296in}}%
\pgfpathlineto{\pgfqpoint{3.743548in}{0.921193in}}%
\pgfpathlineto{\pgfqpoint{3.746229in}{0.920521in}}%
\pgfpathlineto{\pgfqpoint{3.748903in}{0.917150in}}%
\pgfpathlineto{\pgfqpoint{3.751728in}{0.916157in}}%
\pgfpathlineto{\pgfqpoint{3.754265in}{0.916621in}}%
\pgfpathlineto{\pgfqpoint{3.757065in}{0.916980in}}%
\pgfpathlineto{\pgfqpoint{3.759622in}{0.916913in}}%
\pgfpathlineto{\pgfqpoint{3.762389in}{0.914809in}}%
\pgfpathlineto{\pgfqpoint{3.764966in}{0.920572in}}%
\pgfpathlineto{\pgfqpoint{3.767782in}{0.917237in}}%
\pgfpathlineto{\pgfqpoint{3.770323in}{0.931614in}}%
\pgfpathlineto{\pgfqpoint{3.773014in}{0.939361in}}%
\pgfpathlineto{\pgfqpoint{3.775691in}{0.966189in}}%
\pgfpathlineto{\pgfqpoint{3.778370in}{0.962615in}}%
\pgfpathlineto{\pgfqpoint{3.781046in}{0.965253in}}%
\pgfpathlineto{\pgfqpoint{3.783725in}{0.937065in}}%
\pgfpathlineto{\pgfqpoint{3.786504in}{0.933292in}}%
\pgfpathlineto{\pgfqpoint{3.789084in}{0.924404in}}%
\pgfpathlineto{\pgfqpoint{3.791897in}{0.916301in}}%
\pgfpathlineto{\pgfqpoint{3.794435in}{0.917422in}}%
\pgfpathlineto{\pgfqpoint{3.797265in}{0.904309in}}%
\pgfpathlineto{\pgfqpoint{3.799797in}{0.904309in}}%
\pgfpathlineto{\pgfqpoint{3.802569in}{0.906372in}}%
\pgfpathlineto{\pgfqpoint{3.805145in}{0.904309in}}%
\pgfpathlineto{\pgfqpoint{3.807832in}{0.904309in}}%
\pgfpathlineto{\pgfqpoint{3.810510in}{0.906439in}}%
\pgfpathlineto{\pgfqpoint{3.813172in}{0.905358in}}%
\pgfpathlineto{\pgfqpoint{3.815983in}{0.906486in}}%
\pgfpathlineto{\pgfqpoint{3.818546in}{0.912316in}}%
\pgfpathlineto{\pgfqpoint{3.821315in}{0.920519in}}%
\pgfpathlineto{\pgfqpoint{3.823903in}{0.959080in}}%
\pgfpathlineto{\pgfqpoint{3.826679in}{0.940053in}}%
\pgfpathlineto{\pgfqpoint{3.829252in}{0.926693in}}%
\pgfpathlineto{\pgfqpoint{3.832053in}{0.921247in}}%
\pgfpathlineto{\pgfqpoint{3.834616in}{0.922140in}}%
\pgfpathlineto{\pgfqpoint{3.837286in}{0.919354in}}%
\pgfpathlineto{\pgfqpoint{3.839960in}{0.920978in}}%
\pgfpathlineto{\pgfqpoint{3.842641in}{0.917212in}}%
\pgfpathlineto{\pgfqpoint{3.845329in}{0.927906in}}%
\pgfpathlineto{\pgfqpoint{3.848005in}{0.915796in}}%
\pgfpathlineto{\pgfqpoint{3.850814in}{0.921405in}}%
\pgfpathlineto{\pgfqpoint{3.853358in}{0.918924in}}%
\pgfpathlineto{\pgfqpoint{3.856100in}{0.919086in}}%
\pgfpathlineto{\pgfqpoint{3.858720in}{0.917038in}}%
\pgfpathlineto{\pgfqpoint{3.861561in}{0.913506in}}%
\pgfpathlineto{\pgfqpoint{3.864073in}{0.919655in}}%
\pgfpathlineto{\pgfqpoint{3.866815in}{0.919458in}}%
\pgfpathlineto{\pgfqpoint{3.869435in}{0.916851in}}%
\pgfpathlineto{\pgfqpoint{3.872114in}{0.919957in}}%
\pgfpathlineto{\pgfqpoint{3.874790in}{0.919095in}}%
\pgfpathlineto{\pgfqpoint{3.877466in}{0.915745in}}%
\pgfpathlineto{\pgfqpoint{3.880237in}{0.913660in}}%
\pgfpathlineto{\pgfqpoint{3.882850in}{0.916073in}}%
\pgfpathlineto{\pgfqpoint{3.885621in}{0.914608in}}%
\pgfpathlineto{\pgfqpoint{3.888188in}{0.913462in}}%
\pgfpathlineto{\pgfqpoint{3.890926in}{0.911102in}}%
\pgfpathlineto{\pgfqpoint{3.893541in}{0.908556in}}%
\pgfpathlineto{\pgfqpoint{3.896345in}{0.907321in}}%
\pgfpathlineto{\pgfqpoint{3.898891in}{0.910580in}}%
\pgfpathlineto{\pgfqpoint{3.901573in}{0.912457in}}%
\pgfpathlineto{\pgfqpoint{3.904252in}{0.913134in}}%
\pgfpathlineto{\pgfqpoint{3.906918in}{0.914525in}}%
\pgfpathlineto{\pgfqpoint{3.909602in}{0.913551in}}%
\pgfpathlineto{\pgfqpoint{3.912296in}{0.914867in}}%
\pgfpathlineto{\pgfqpoint{3.915107in}{0.914904in}}%
\pgfpathlineto{\pgfqpoint{3.917646in}{0.916884in}}%
\pgfpathlineto{\pgfqpoint{3.920412in}{0.915842in}}%
\pgfpathlineto{\pgfqpoint{3.923005in}{0.917552in}}%
\pgfpathlineto{\pgfqpoint{3.925778in}{0.916386in}}%
\pgfpathlineto{\pgfqpoint{3.928347in}{0.916671in}}%
\pgfpathlineto{\pgfqpoint{3.931202in}{0.917786in}}%
\pgfpathlineto{\pgfqpoint{3.933714in}{0.912743in}}%
\pgfpathlineto{\pgfqpoint{3.936395in}{0.913986in}}%
\pgfpathlineto{\pgfqpoint{3.939075in}{0.913605in}}%
\pgfpathlineto{\pgfqpoint{3.941778in}{0.904865in}}%
\pgfpathlineto{\pgfqpoint{3.944431in}{0.908563in}}%
\pgfpathlineto{\pgfqpoint{3.947101in}{0.913260in}}%
\pgfpathlineto{\pgfqpoint{3.949894in}{0.921982in}}%
\pgfpathlineto{\pgfqpoint{3.952464in}{0.912698in}}%
\pgfpathlineto{\pgfqpoint{3.955211in}{0.911714in}}%
\pgfpathlineto{\pgfqpoint{3.957823in}{0.919903in}}%
\pgfpathlineto{\pgfqpoint{3.960635in}{0.914601in}}%
\pgfpathlineto{\pgfqpoint{3.963176in}{0.913254in}}%
\pgfpathlineto{\pgfqpoint{3.966013in}{0.915724in}}%
\pgfpathlineto{\pgfqpoint{3.968523in}{0.921451in}}%
\pgfpathlineto{\pgfqpoint{3.971250in}{0.918366in}}%
\pgfpathlineto{\pgfqpoint{3.973885in}{0.911023in}}%
\pgfpathlineto{\pgfqpoint{3.976563in}{0.907860in}}%
\pgfpathlineto{\pgfqpoint{3.979389in}{0.913326in}}%
\pgfpathlineto{\pgfqpoint{3.981929in}{0.915308in}}%
\pgfpathlineto{\pgfqpoint{3.984714in}{0.917450in}}%
\pgfpathlineto{\pgfqpoint{3.987270in}{0.915307in}}%
\pgfpathlineto{\pgfqpoint{3.990055in}{0.914094in}}%
\pgfpathlineto{\pgfqpoint{3.992642in}{0.917139in}}%
\pgfpathlineto{\pgfqpoint{3.995417in}{0.919711in}}%
\pgfpathlineto{\pgfqpoint{3.997990in}{0.915682in}}%
\pgfpathlineto{\pgfqpoint{4.000674in}{0.920510in}}%
\pgfpathlineto{\pgfqpoint{4.003348in}{0.916238in}}%
\pgfpathlineto{\pgfqpoint{4.006034in}{0.914906in}}%
\pgfpathlineto{\pgfqpoint{4.008699in}{0.915769in}}%
\pgfpathlineto{\pgfqpoint{4.011394in}{0.918425in}}%
\pgfpathlineto{\pgfqpoint{4.014186in}{0.914414in}}%
\pgfpathlineto{\pgfqpoint{4.016744in}{0.922790in}}%
\pgfpathlineto{\pgfqpoint{4.019518in}{0.952175in}}%
\pgfpathlineto{\pgfqpoint{4.022097in}{0.953039in}}%
\pgfpathlineto{\pgfqpoint{4.024868in}{0.964177in}}%
\pgfpathlineto{\pgfqpoint{4.027447in}{0.976702in}}%
\pgfpathlineto{\pgfqpoint{4.030229in}{0.961441in}}%
\pgfpathlineto{\pgfqpoint{4.032817in}{0.947686in}}%
\pgfpathlineto{\pgfqpoint{4.035492in}{0.948158in}}%
\pgfpathlineto{\pgfqpoint{4.038174in}{0.941189in}}%
\pgfpathlineto{\pgfqpoint{4.040852in}{0.939116in}}%
\pgfpathlineto{\pgfqpoint{4.043667in}{0.936455in}}%
\pgfpathlineto{\pgfqpoint{4.046210in}{0.930448in}}%
\pgfpathlineto{\pgfqpoint{4.049006in}{0.927939in}}%
\pgfpathlineto{\pgfqpoint{4.051557in}{0.926563in}}%
\pgfpathlineto{\pgfqpoint{4.054326in}{0.927125in}}%
\pgfpathlineto{\pgfqpoint{4.056911in}{0.920679in}}%
\pgfpathlineto{\pgfqpoint{4.059702in}{0.920227in}}%
\pgfpathlineto{\pgfqpoint{4.062266in}{0.921464in}}%
\pgfpathlineto{\pgfqpoint{4.064957in}{0.912995in}}%
\pgfpathlineto{\pgfqpoint{4.067636in}{0.917020in}}%
\pgfpathlineto{\pgfqpoint{4.070313in}{0.916920in}}%
\pgfpathlineto{\pgfqpoint{4.072985in}{0.916603in}}%
\pgfpathlineto{\pgfqpoint{4.075705in}{0.913980in}}%
\pgfpathlineto{\pgfqpoint{4.078471in}{0.922373in}}%
\pgfpathlineto{\pgfqpoint{4.081018in}{0.920356in}}%
\pgfpathlineto{\pgfqpoint{4.083870in}{0.917717in}}%
\pgfpathlineto{\pgfqpoint{4.086385in}{0.921649in}}%
\pgfpathlineto{\pgfqpoint{4.089159in}{0.918754in}}%
\pgfpathlineto{\pgfqpoint{4.091729in}{0.915676in}}%
\pgfpathlineto{\pgfqpoint{4.094527in}{0.920992in}}%
\pgfpathlineto{\pgfqpoint{4.097092in}{0.924720in}}%
\pgfpathlineto{\pgfqpoint{4.099777in}{0.919986in}}%
\pgfpathlineto{\pgfqpoint{4.102456in}{0.923973in}}%
\pgfpathlineto{\pgfqpoint{4.105185in}{0.922391in}}%
\pgfpathlineto{\pgfqpoint{4.107814in}{0.923722in}}%
\pgfpathlineto{\pgfqpoint{4.110488in}{0.922569in}}%
\pgfpathlineto{\pgfqpoint{4.113252in}{0.918902in}}%
\pgfpathlineto{\pgfqpoint{4.115844in}{0.918936in}}%
\pgfpathlineto{\pgfqpoint{4.118554in}{0.917521in}}%
\pgfpathlineto{\pgfqpoint{4.121205in}{0.918446in}}%
\pgfpathlineto{\pgfqpoint{4.124019in}{0.935595in}}%
\pgfpathlineto{\pgfqpoint{4.126553in}{0.924790in}}%
\pgfpathlineto{\pgfqpoint{4.129349in}{0.919877in}}%
\pgfpathlineto{\pgfqpoint{4.131920in}{0.917821in}}%
\pgfpathlineto{\pgfqpoint{4.134615in}{0.921253in}}%
\pgfpathlineto{\pgfqpoint{4.137272in}{0.919960in}}%
\pgfpathlineto{\pgfqpoint{4.139963in}{0.928535in}}%
\pgfpathlineto{\pgfqpoint{4.142713in}{0.926010in}}%
\pgfpathlineto{\pgfqpoint{4.145310in}{0.926608in}}%
\pgfpathlineto{\pgfqpoint{4.148082in}{0.921444in}}%
\pgfpathlineto{\pgfqpoint{4.150665in}{0.925554in}}%
\pgfpathlineto{\pgfqpoint{4.153423in}{0.922968in}}%
\pgfpathlineto{\pgfqpoint{4.156016in}{0.918064in}}%
\pgfpathlineto{\pgfqpoint{4.158806in}{0.926965in}}%
\pgfpathlineto{\pgfqpoint{4.161380in}{0.933878in}}%
\pgfpathlineto{\pgfqpoint{4.164059in}{0.925847in}}%
\pgfpathlineto{\pgfqpoint{4.166737in}{0.922338in}}%
\pgfpathlineto{\pgfqpoint{4.169415in}{0.918288in}}%
\pgfpathlineto{\pgfqpoint{4.172093in}{0.912907in}}%
\pgfpathlineto{\pgfqpoint{4.174770in}{0.917852in}}%
\pgfpathlineto{\pgfqpoint{4.177593in}{0.919291in}}%
\pgfpathlineto{\pgfqpoint{4.180129in}{0.933088in}}%
\pgfpathlineto{\pgfqpoint{4.182899in}{0.942868in}}%
\pgfpathlineto{\pgfqpoint{4.185481in}{0.925225in}}%
\pgfpathlineto{\pgfqpoint{4.188318in}{0.918996in}}%
\pgfpathlineto{\pgfqpoint{4.190842in}{0.916842in}}%
\pgfpathlineto{\pgfqpoint{4.193638in}{0.920032in}}%
\pgfpathlineto{\pgfqpoint{4.196186in}{0.919636in}}%
\pgfpathlineto{\pgfqpoint{4.198878in}{0.921130in}}%
\pgfpathlineto{\pgfqpoint{4.201542in}{0.925182in}}%
\pgfpathlineto{\pgfqpoint{4.204240in}{0.913296in}}%
\pgfpathlineto{\pgfqpoint{4.207076in}{0.906629in}}%
\pgfpathlineto{\pgfqpoint{4.209597in}{0.906042in}}%
\pgfpathlineto{\pgfqpoint{4.212383in}{0.917370in}}%
\pgfpathlineto{\pgfqpoint{4.214948in}{0.921243in}}%
\pgfpathlineto{\pgfqpoint{4.217694in}{0.920821in}}%
\pgfpathlineto{\pgfqpoint{4.220304in}{0.926641in}}%
\pgfpathlineto{\pgfqpoint{4.223082in}{0.925850in}}%
\pgfpathlineto{\pgfqpoint{4.225654in}{0.926837in}}%
\pgfpathlineto{\pgfqpoint{4.228331in}{0.929614in}}%
\pgfpathlineto{\pgfqpoint{4.231013in}{0.922841in}}%
\pgfpathlineto{\pgfqpoint{4.233691in}{0.922781in}}%
\pgfpathlineto{\pgfqpoint{4.236375in}{0.919669in}}%
\pgfpathlineto{\pgfqpoint{4.239084in}{0.916675in}}%
\pgfpathlineto{\pgfqpoint{4.241900in}{0.925344in}}%
\pgfpathlineto{\pgfqpoint{4.244394in}{0.925136in}}%
\pgfpathlineto{\pgfqpoint{4.247225in}{0.912108in}}%
\pgfpathlineto{\pgfqpoint{4.249767in}{0.910090in}}%
\pgfpathlineto{\pgfqpoint{4.252581in}{0.907262in}}%
\pgfpathlineto{\pgfqpoint{4.255120in}{0.907097in}}%
\pgfpathlineto{\pgfqpoint{4.257958in}{0.910275in}}%
\pgfpathlineto{\pgfqpoint{4.260477in}{0.911598in}}%
\pgfpathlineto{\pgfqpoint{4.263157in}{0.909919in}}%
\pgfpathlineto{\pgfqpoint{4.265824in}{0.910583in}}%
\pgfpathlineto{\pgfqpoint{4.268590in}{0.910214in}}%
\pgfpathlineto{\pgfqpoint{4.271187in}{0.910839in}}%
\pgfpathlineto{\pgfqpoint{4.273874in}{0.912651in}}%
\pgfpathlineto{\pgfqpoint{4.276635in}{0.911187in}}%
\pgfpathlineto{\pgfqpoint{4.279212in}{0.907197in}}%
\pgfpathlineto{\pgfqpoint{4.282000in}{0.921506in}}%
\pgfpathlineto{\pgfqpoint{4.284586in}{0.916979in}}%
\pgfpathlineto{\pgfqpoint{4.287399in}{0.917167in}}%
\pgfpathlineto{\pgfqpoint{4.289936in}{0.912283in}}%
\pgfpathlineto{\pgfqpoint{4.292786in}{0.914481in}}%
\pgfpathlineto{\pgfqpoint{4.295299in}{0.912480in}}%
\pgfpathlineto{\pgfqpoint{4.297977in}{0.918635in}}%
\pgfpathlineto{\pgfqpoint{4.300656in}{0.927481in}}%
\pgfpathlineto{\pgfqpoint{4.303357in}{0.923931in}}%
\pgfpathlineto{\pgfqpoint{4.306118in}{0.921965in}}%
\pgfpathlineto{\pgfqpoint{4.308691in}{0.922843in}}%
\pgfpathlineto{\pgfqpoint{4.311494in}{0.923568in}}%
\pgfpathlineto{\pgfqpoint{4.314032in}{0.918755in}}%
\pgfpathlineto{\pgfqpoint{4.316856in}{0.919672in}}%
\pgfpathlineto{\pgfqpoint{4.319405in}{0.925404in}}%
\pgfpathlineto{\pgfqpoint{4.322181in}{0.922134in}}%
\pgfpathlineto{\pgfqpoint{4.324760in}{0.922054in}}%
\pgfpathlineto{\pgfqpoint{4.327440in}{0.927126in}}%
\pgfpathlineto{\pgfqpoint{4.330118in}{0.925448in}}%
\pgfpathlineto{\pgfqpoint{4.332796in}{0.924296in}}%
\pgfpathlineto{\pgfqpoint{4.335463in}{0.921002in}}%
\pgfpathlineto{\pgfqpoint{4.338154in}{0.915747in}}%
\pgfpathlineto{\pgfqpoint{4.340976in}{0.919763in}}%
\pgfpathlineto{\pgfqpoint{4.343510in}{0.915231in}}%
\pgfpathlineto{\pgfqpoint{4.346263in}{0.917580in}}%
\pgfpathlineto{\pgfqpoint{4.348868in}{0.917637in}}%
\pgfpathlineto{\pgfqpoint{4.351645in}{0.920166in}}%
\pgfpathlineto{\pgfqpoint{4.354224in}{0.923069in}}%
\pgfpathlineto{\pgfqpoint{4.357014in}{0.920871in}}%
\pgfpathlineto{\pgfqpoint{4.359582in}{0.917185in}}%
\pgfpathlineto{\pgfqpoint{4.362270in}{0.913450in}}%
\pgfpathlineto{\pgfqpoint{4.364936in}{0.923039in}}%
\pgfpathlineto{\pgfqpoint{4.367646in}{0.920487in}}%
\pgfpathlineto{\pgfqpoint{4.370437in}{0.921527in}}%
\pgfpathlineto{\pgfqpoint{4.372976in}{0.919041in}}%
\pgfpathlineto{\pgfqpoint{4.375761in}{0.923068in}}%
\pgfpathlineto{\pgfqpoint{4.378329in}{0.922791in}}%
\pgfpathlineto{\pgfqpoint{4.381097in}{0.922437in}}%
\pgfpathlineto{\pgfqpoint{4.383674in}{0.920551in}}%
\pgfpathlineto{\pgfqpoint{4.386431in}{0.915076in}}%
\pgfpathlineto{\pgfqpoint{4.389044in}{0.924786in}}%
\pgfpathlineto{\pgfqpoint{4.391721in}{0.919688in}}%
\pgfpathlineto{\pgfqpoint{4.394400in}{0.919426in}}%
\pgfpathlineto{\pgfqpoint{4.397076in}{0.907188in}}%
\pgfpathlineto{\pgfqpoint{4.399745in}{0.922176in}}%
\pgfpathlineto{\pgfqpoint{4.402468in}{0.933517in}}%
\pgfpathlineto{\pgfqpoint{4.405234in}{0.934514in}}%
\pgfpathlineto{\pgfqpoint{4.407788in}{0.930897in}}%
\pgfpathlineto{\pgfqpoint{4.410587in}{0.924132in}}%
\pgfpathlineto{\pgfqpoint{4.413149in}{0.920600in}}%
\pgfpathlineto{\pgfqpoint{4.415932in}{0.918028in}}%
\pgfpathlineto{\pgfqpoint{4.418506in}{0.917470in}}%
\pgfpathlineto{\pgfqpoint{4.421292in}{0.924559in}}%
\pgfpathlineto{\pgfqpoint{4.423863in}{0.920109in}}%
\pgfpathlineto{\pgfqpoint{4.426534in}{0.915439in}}%
\pgfpathlineto{\pgfqpoint{4.429220in}{0.910523in}}%
\pgfpathlineto{\pgfqpoint{4.431901in}{0.916154in}}%
\pgfpathlineto{\pgfqpoint{4.434569in}{0.918402in}}%
\pgfpathlineto{\pgfqpoint{4.437253in}{0.922495in}}%
\pgfpathlineto{\pgfqpoint{4.440041in}{0.916611in}}%
\pgfpathlineto{\pgfqpoint{4.442611in}{0.913999in}}%
\pgfpathlineto{\pgfqpoint{4.445423in}{0.918365in}}%
\pgfpathlineto{\pgfqpoint{4.447965in}{0.918986in}}%
\pgfpathlineto{\pgfqpoint{4.450767in}{0.920075in}}%
\pgfpathlineto{\pgfqpoint{4.453312in}{0.919009in}}%
\pgfpathlineto{\pgfqpoint{4.456138in}{0.934177in}}%
\pgfpathlineto{\pgfqpoint{4.458681in}{0.953219in}}%
\pgfpathlineto{\pgfqpoint{4.461367in}{0.941362in}}%
\pgfpathlineto{\pgfqpoint{4.464029in}{0.938496in}}%
\pgfpathlineto{\pgfqpoint{4.466717in}{0.931753in}}%
\pgfpathlineto{\pgfqpoint{4.469492in}{0.920068in}}%
\pgfpathlineto{\pgfqpoint{4.472059in}{0.921080in}}%
\pgfpathlineto{\pgfqpoint{4.474861in}{0.918184in}}%
\pgfpathlineto{\pgfqpoint{4.477430in}{0.920679in}}%
\pgfpathlineto{\pgfqpoint{4.480201in}{0.921668in}}%
\pgfpathlineto{\pgfqpoint{4.482778in}{0.929142in}}%
\pgfpathlineto{\pgfqpoint{4.485581in}{0.927487in}}%
\pgfpathlineto{\pgfqpoint{4.488130in}{0.924869in}}%
\pgfpathlineto{\pgfqpoint{4.490822in}{0.924868in}}%
\pgfpathlineto{\pgfqpoint{4.493492in}{0.925765in}}%
\pgfpathlineto{\pgfqpoint{4.496167in}{0.924166in}}%
\pgfpathlineto{\pgfqpoint{4.498850in}{0.928043in}}%
\pgfpathlineto{\pgfqpoint{4.501529in}{0.926208in}}%
\pgfpathlineto{\pgfqpoint{4.504305in}{0.923089in}}%
\pgfpathlineto{\pgfqpoint{4.506893in}{0.935197in}}%
\pgfpathlineto{\pgfqpoint{4.509643in}{0.942076in}}%
\pgfpathlineto{\pgfqpoint{4.512246in}{0.952295in}}%
\pgfpathlineto{\pgfqpoint{4.515080in}{0.939996in}}%
\pgfpathlineto{\pgfqpoint{4.517598in}{0.931899in}}%
\pgfpathlineto{\pgfqpoint{4.520345in}{0.926624in}}%
\pgfpathlineto{\pgfqpoint{4.522962in}{0.932795in}}%
\pgfpathlineto{\pgfqpoint{4.525640in}{0.936574in}}%
\pgfpathlineto{\pgfqpoint{4.528307in}{0.930775in}}%
\pgfpathlineto{\pgfqpoint{4.530990in}{0.924781in}}%
\pgfpathlineto{\pgfqpoint{4.533764in}{0.926852in}}%
\pgfpathlineto{\pgfqpoint{4.536400in}{0.926610in}}%
\pgfpathlineto{\pgfqpoint{4.539144in}{0.925920in}}%
\pgfpathlineto{\pgfqpoint{4.541711in}{0.924010in}}%
\pgfpathlineto{\pgfqpoint{4.544464in}{0.921219in}}%
\pgfpathlineto{\pgfqpoint{4.547064in}{0.928022in}}%
\pgfpathlineto{\pgfqpoint{4.549822in}{0.924806in}}%
\pgfpathlineto{\pgfqpoint{4.552425in}{0.920947in}}%
\pgfpathlineto{\pgfqpoint{4.555106in}{0.918724in}}%
\pgfpathlineto{\pgfqpoint{4.557777in}{0.919823in}}%
\pgfpathlineto{\pgfqpoint{4.560448in}{0.920200in}}%
\pgfpathlineto{\pgfqpoint{4.563125in}{0.923644in}}%
\pgfpathlineto{\pgfqpoint{4.565820in}{0.923668in}}%
\pgfpathlineto{\pgfqpoint{4.568612in}{0.926464in}}%
\pgfpathlineto{\pgfqpoint{4.571171in}{0.923586in}}%
\pgfpathlineto{\pgfqpoint{4.573947in}{0.921724in}}%
\pgfpathlineto{\pgfqpoint{4.576531in}{0.920868in}}%
\pgfpathlineto{\pgfqpoint{4.579305in}{0.920636in}}%
\pgfpathlineto{\pgfqpoint{4.581888in}{0.921885in}}%
\pgfpathlineto{\pgfqpoint{4.584672in}{0.929699in}}%
\pgfpathlineto{\pgfqpoint{4.587244in}{0.922463in}}%
\pgfpathlineto{\pgfqpoint{4.589920in}{0.919641in}}%
\pgfpathlineto{\pgfqpoint{4.592589in}{0.934954in}}%
\pgfpathlineto{\pgfqpoint{4.595281in}{0.930742in}}%
\pgfpathlineto{\pgfqpoint{4.597951in}{0.935199in}}%
\pgfpathlineto{\pgfqpoint{4.600633in}{0.940146in}}%
\pgfpathlineto{\pgfqpoint{4.603430in}{0.936966in}}%
\pgfpathlineto{\pgfqpoint{4.605990in}{0.931235in}}%
\pgfpathlineto{\pgfqpoint{4.608808in}{0.929145in}}%
\pgfpathlineto{\pgfqpoint{4.611350in}{0.925839in}}%
\pgfpathlineto{\pgfqpoint{4.614134in}{0.925417in}}%
\pgfpathlineto{\pgfqpoint{4.616702in}{0.929884in}}%
\pgfpathlineto{\pgfqpoint{4.619529in}{0.924254in}}%
\pgfpathlineto{\pgfqpoint{4.622056in}{0.923023in}}%
\pgfpathlineto{\pgfqpoint{4.624741in}{0.930437in}}%
\pgfpathlineto{\pgfqpoint{4.627411in}{0.926956in}}%
\pgfpathlineto{\pgfqpoint{4.630096in}{0.927650in}}%
\pgfpathlineto{\pgfqpoint{4.632902in}{0.919961in}}%
\pgfpathlineto{\pgfqpoint{4.635445in}{0.922143in}}%
\pgfpathlineto{\pgfqpoint{4.638204in}{0.925347in}}%
\pgfpathlineto{\pgfqpoint{4.640809in}{0.922867in}}%
\pgfpathlineto{\pgfqpoint{4.643628in}{0.925031in}}%
\pgfpathlineto{\pgfqpoint{4.646169in}{0.922442in}}%
\pgfpathlineto{\pgfqpoint{4.648922in}{0.923845in}}%
\pgfpathlineto{\pgfqpoint{4.651524in}{0.924794in}}%
\pgfpathlineto{\pgfqpoint{4.654203in}{0.927656in}}%
\pgfpathlineto{\pgfqpoint{4.656873in}{0.929397in}}%
\pgfpathlineto{\pgfqpoint{4.659590in}{0.926305in}}%
\pgfpathlineto{\pgfqpoint{4.662237in}{0.944851in}}%
\pgfpathlineto{\pgfqpoint{4.664923in}{0.939210in}}%
\pgfpathlineto{\pgfqpoint{4.667764in}{0.935151in}}%
\pgfpathlineto{\pgfqpoint{4.670261in}{0.933723in}}%
\pgfpathlineto{\pgfqpoint{4.673068in}{0.928146in}}%
\pgfpathlineto{\pgfqpoint{4.675619in}{0.924671in}}%
\pgfpathlineto{\pgfqpoint{4.678448in}{0.924415in}}%
\pgfpathlineto{\pgfqpoint{4.680988in}{0.924282in}}%
\pgfpathlineto{\pgfqpoint{4.683799in}{0.921238in}}%
\pgfpathlineto{\pgfqpoint{4.686337in}{0.917548in}}%
\pgfpathlineto{\pgfqpoint{4.689051in}{0.917187in}}%
\pgfpathlineto{\pgfqpoint{4.691694in}{0.920744in}}%
\pgfpathlineto{\pgfqpoint{4.694381in}{0.919152in}}%
\pgfpathlineto{\pgfqpoint{4.697170in}{0.914102in}}%
\pgfpathlineto{\pgfqpoint{4.699734in}{0.918306in}}%
\pgfpathlineto{\pgfqpoint{4.702517in}{0.917530in}}%
\pgfpathlineto{\pgfqpoint{4.705094in}{0.913748in}}%
\pgfpathlineto{\pgfqpoint{4.707824in}{0.916271in}}%
\pgfpathlineto{\pgfqpoint{4.710437in}{0.915624in}}%
\pgfpathlineto{\pgfqpoint{4.713275in}{0.915698in}}%
\pgfpathlineto{\pgfqpoint{4.715806in}{0.918229in}}%
\pgfpathlineto{\pgfqpoint{4.718486in}{0.917016in}}%
\pgfpathlineto{\pgfqpoint{4.721160in}{0.918324in}}%
\pgfpathlineto{\pgfqpoint{4.723873in}{0.916597in}}%
\pgfpathlineto{\pgfqpoint{4.726508in}{0.923254in}}%
\pgfpathlineto{\pgfqpoint{4.729233in}{0.919653in}}%
\pgfpathlineto{\pgfqpoint{4.731901in}{0.917786in}}%
\pgfpathlineto{\pgfqpoint{4.734552in}{0.916793in}}%
\pgfpathlineto{\pgfqpoint{4.737348in}{0.913014in}}%
\pgfpathlineto{\pgfqpoint{4.739912in}{0.920579in}}%
\pgfpathlineto{\pgfqpoint{4.742696in}{0.912720in}}%
\pgfpathlineto{\pgfqpoint{4.745256in}{0.912222in}}%
\pgfpathlineto{\pgfqpoint{4.748081in}{0.913576in}}%
\pgfpathlineto{\pgfqpoint{4.750627in}{0.919903in}}%
\pgfpathlineto{\pgfqpoint{4.753298in}{0.917435in}}%
\pgfpathlineto{\pgfqpoint{4.755983in}{0.924863in}}%
\pgfpathlineto{\pgfqpoint{4.758653in}{0.922289in}}%
\pgfpathlineto{\pgfqpoint{4.761337in}{0.936304in}}%
\pgfpathlineto{\pgfqpoint{4.764018in}{0.925906in}}%
\pgfpathlineto{\pgfqpoint{4.766783in}{0.925889in}}%
\pgfpathlineto{\pgfqpoint{4.769367in}{0.920263in}}%
\pgfpathlineto{\pgfqpoint{4.772198in}{0.920956in}}%
\pgfpathlineto{\pgfqpoint{4.774732in}{0.924497in}}%
\pgfpathlineto{\pgfqpoint{4.777535in}{0.926993in}}%
\pgfpathlineto{\pgfqpoint{4.780083in}{0.921817in}}%
\pgfpathlineto{\pgfqpoint{4.782872in}{0.918491in}}%
\pgfpathlineto{\pgfqpoint{4.785445in}{0.924618in}}%
\pgfpathlineto{\pgfqpoint{4.788116in}{0.926669in}}%
\pgfpathlineto{\pgfqpoint{4.790798in}{0.934774in}}%
\pgfpathlineto{\pgfqpoint{4.793512in}{0.961857in}}%
\pgfpathlineto{\pgfqpoint{4.796274in}{0.955953in}}%
\pgfpathlineto{\pgfqpoint{4.798830in}{0.942762in}}%
\pgfpathlineto{\pgfqpoint{4.801586in}{0.930401in}}%
\pgfpathlineto{\pgfqpoint{4.804193in}{0.928044in}}%
\pgfpathlineto{\pgfqpoint{4.807017in}{0.924418in}}%
\pgfpathlineto{\pgfqpoint{4.809538in}{0.923719in}}%
\pgfpathlineto{\pgfqpoint{4.812377in}{0.931986in}}%
\pgfpathlineto{\pgfqpoint{4.814907in}{0.923755in}}%
\pgfpathlineto{\pgfqpoint{4.817587in}{0.907046in}}%
\pgfpathlineto{\pgfqpoint{4.820265in}{0.914957in}}%
\pgfpathlineto{\pgfqpoint{4.822945in}{0.904309in}}%
\pgfpathlineto{\pgfqpoint{4.825619in}{0.904309in}}%
\pgfpathlineto{\pgfqpoint{4.828291in}{0.904309in}}%
\pgfpathlineto{\pgfqpoint{4.831045in}{0.904665in}}%
\pgfpathlineto{\pgfqpoint{4.833657in}{0.910881in}}%
\pgfpathlineto{\pgfqpoint{4.837992in}{0.921247in}}%
\pgfpathlineto{\pgfqpoint{4.839922in}{0.923231in}}%
\pgfpathlineto{\pgfqpoint{4.842380in}{0.921711in}}%
\pgfpathlineto{\pgfqpoint{4.844361in}{0.918255in}}%
\pgfpathlineto{\pgfqpoint{4.847127in}{0.921853in}}%
\pgfpathlineto{\pgfqpoint{4.849715in}{0.919054in}}%
\pgfpathlineto{\pgfqpoint{4.852404in}{0.913612in}}%
\pgfpathlineto{\pgfqpoint{4.855070in}{0.919856in}}%
\pgfpathlineto{\pgfqpoint{4.857807in}{0.923035in}}%
\pgfpathlineto{\pgfqpoint{4.860544in}{0.920006in}}%
\pgfpathlineto{\pgfqpoint{4.863116in}{0.924276in}}%
\pgfpathlineto{\pgfqpoint{4.865910in}{0.921908in}}%
\pgfpathlineto{\pgfqpoint{4.868474in}{0.919294in}}%
\pgfpathlineto{\pgfqpoint{4.871209in}{0.917835in}}%
\pgfpathlineto{\pgfqpoint{4.873832in}{0.922782in}}%
\pgfpathlineto{\pgfqpoint{4.876636in}{0.919363in}}%
\pgfpathlineto{\pgfqpoint{4.879180in}{0.916985in}}%
\pgfpathlineto{\pgfqpoint{4.881864in}{0.922104in}}%
\pgfpathlineto{\pgfqpoint{4.884540in}{0.913836in}}%
\pgfpathlineto{\pgfqpoint{4.887211in}{0.919367in}}%
\pgfpathlineto{\pgfqpoint{4.889902in}{0.918764in}}%
\pgfpathlineto{\pgfqpoint{4.892611in}{0.915319in}}%
\pgfpathlineto{\pgfqpoint{4.895399in}{0.921129in}}%
\pgfpathlineto{\pgfqpoint{4.897938in}{0.917255in}}%
\pgfpathlineto{\pgfqpoint{4.900712in}{0.919628in}}%
\pgfpathlineto{\pgfqpoint{4.903295in}{0.927467in}}%
\pgfpathlineto{\pgfqpoint{4.906096in}{0.941975in}}%
\pgfpathlineto{\pgfqpoint{4.908648in}{0.932409in}}%
\pgfpathlineto{\pgfqpoint{4.911435in}{0.926603in}}%
\pgfpathlineto{\pgfqpoint{4.914009in}{0.921313in}}%
\pgfpathlineto{\pgfqpoint{4.916681in}{0.919870in}}%
\pgfpathlineto{\pgfqpoint{4.919352in}{0.913621in}}%
\pgfpathlineto{\pgfqpoint{4.922041in}{0.917548in}}%
\pgfpathlineto{\pgfqpoint{4.924708in}{0.921604in}}%
\pgfpathlineto{\pgfqpoint{4.927400in}{0.920215in}}%
\pgfpathlineto{\pgfqpoint{4.930170in}{0.922612in}}%
\pgfpathlineto{\pgfqpoint{4.932742in}{0.920511in}}%
\pgfpathlineto{\pgfqpoint{4.935515in}{0.920836in}}%
\pgfpathlineto{\pgfqpoint{4.938112in}{0.920155in}}%
\pgfpathlineto{\pgfqpoint{4.940881in}{0.916419in}}%
\pgfpathlineto{\pgfqpoint{4.943466in}{0.916004in}}%
\pgfpathlineto{\pgfqpoint{4.946151in}{0.909552in}}%
\pgfpathlineto{\pgfqpoint{4.948827in}{0.913916in}}%
\pgfpathlineto{\pgfqpoint{4.951504in}{0.913484in}}%
\pgfpathlineto{\pgfqpoint{4.954182in}{0.921141in}}%
\pgfpathlineto{\pgfqpoint{4.956862in}{0.942037in}}%
\pgfpathlineto{\pgfqpoint{4.959689in}{0.932730in}}%
\pgfpathlineto{\pgfqpoint{4.962219in}{0.924492in}}%
\pgfpathlineto{\pgfqpoint{4.965002in}{0.920083in}}%
\pgfpathlineto{\pgfqpoint{4.967575in}{0.916185in}}%
\pgfpathlineto{\pgfqpoint{4.970314in}{0.916187in}}%
\pgfpathlineto{\pgfqpoint{4.972933in}{0.919347in}}%
\pgfpathlineto{\pgfqpoint{4.975703in}{0.921263in}}%
\pgfpathlineto{\pgfqpoint{4.978287in}{0.922479in}}%
\pgfpathlineto{\pgfqpoint{4.980967in}{0.923557in}}%
\pgfpathlineto{\pgfqpoint{4.983637in}{0.917917in}}%
\pgfpathlineto{\pgfqpoint{4.986325in}{0.922350in}}%
\pgfpathlineto{\pgfqpoint{4.989001in}{0.919921in}}%
\pgfpathlineto{\pgfqpoint{4.991683in}{0.922829in}}%
\pgfpathlineto{\pgfqpoint{4.994390in}{0.924083in}}%
\pgfpathlineto{\pgfqpoint{4.997028in}{0.925122in}}%
\pgfpathlineto{\pgfqpoint{4.999780in}{0.924517in}}%
\pgfpathlineto{\pgfqpoint{5.002384in}{0.924784in}}%
\pgfpathlineto{\pgfqpoint{5.005178in}{0.920444in}}%
\pgfpathlineto{\pgfqpoint{5.007751in}{0.929119in}}%
\pgfpathlineto{\pgfqpoint{5.010562in}{0.924258in}}%
\pgfpathlineto{\pgfqpoint{5.013104in}{0.919746in}}%
\pgfpathlineto{\pgfqpoint{5.015820in}{0.917577in}}%
\pgfpathlineto{\pgfqpoint{5.018466in}{0.917197in}}%
\pgfpathlineto{\pgfqpoint{5.021147in}{0.921272in}}%
\pgfpathlineto{\pgfqpoint{5.023927in}{0.922796in}}%
\pgfpathlineto{\pgfqpoint{5.026501in}{0.921723in}}%
\pgfpathlineto{\pgfqpoint{5.029275in}{0.919554in}}%
\pgfpathlineto{\pgfqpoint{5.031849in}{0.919005in}}%
\pgfpathlineto{\pgfqpoint{5.034649in}{0.917577in}}%
\pgfpathlineto{\pgfqpoint{5.037214in}{0.906548in}}%
\pgfpathlineto{\pgfqpoint{5.039962in}{0.914984in}}%
\pgfpathlineto{\pgfqpoint{5.042572in}{0.923254in}}%
\pgfpathlineto{\pgfqpoint{5.045249in}{0.917954in}}%
\pgfpathlineto{\pgfqpoint{5.047924in}{0.917743in}}%
\pgfpathlineto{\pgfqpoint{5.050606in}{0.920788in}}%
\pgfpathlineto{\pgfqpoint{5.053284in}{0.922332in}}%
\pgfpathlineto{\pgfqpoint{5.055952in}{0.919808in}}%
\pgfpathlineto{\pgfqpoint{5.058711in}{0.919471in}}%
\pgfpathlineto{\pgfqpoint{5.061315in}{0.925580in}}%
\pgfpathlineto{\pgfqpoint{5.064144in}{0.927319in}}%
\pgfpathlineto{\pgfqpoint{5.066677in}{0.922967in}}%
\pgfpathlineto{\pgfqpoint{5.069463in}{0.919529in}}%
\pgfpathlineto{\pgfqpoint{5.072030in}{0.917850in}}%
\pgfpathlineto{\pgfqpoint{5.074851in}{0.910797in}}%
\pgfpathlineto{\pgfqpoint{5.077390in}{0.921016in}}%
\pgfpathlineto{\pgfqpoint{5.080067in}{0.925032in}}%
\pgfpathlineto{\pgfqpoint{5.082746in}{0.920969in}}%
\pgfpathlineto{\pgfqpoint{5.085426in}{0.919769in}}%
\pgfpathlineto{\pgfqpoint{5.088103in}{0.925693in}}%
\pgfpathlineto{\pgfqpoint{5.090788in}{0.924250in}}%
\pgfpathlineto{\pgfqpoint{5.093579in}{0.922588in}}%
\pgfpathlineto{\pgfqpoint{5.096142in}{0.923110in}}%
\pgfpathlineto{\pgfqpoint{5.098948in}{0.923046in}}%
\pgfpathlineto{\pgfqpoint{5.101496in}{0.923067in}}%
\pgfpathlineto{\pgfqpoint{5.104312in}{0.928520in}}%
\pgfpathlineto{\pgfqpoint{5.106842in}{0.924058in}}%
\pgfpathlineto{\pgfqpoint{5.109530in}{0.923221in}}%
\pgfpathlineto{\pgfqpoint{5.112209in}{0.924793in}}%
\pgfpathlineto{\pgfqpoint{5.114887in}{0.921650in}}%
\pgfpathlineto{\pgfqpoint{5.117550in}{0.922145in}}%
\pgfpathlineto{\pgfqpoint{5.120243in}{0.923254in}}%
\pgfpathlineto{\pgfqpoint{5.123042in}{0.925923in}}%
\pgfpathlineto{\pgfqpoint{5.125599in}{0.920243in}}%
\pgfpathlineto{\pgfqpoint{5.128421in}{0.925583in}}%
\pgfpathlineto{\pgfqpoint{5.130953in}{0.923885in}}%
\pgfpathlineto{\pgfqpoint{5.133716in}{0.923517in}}%
\pgfpathlineto{\pgfqpoint{5.136311in}{0.923803in}}%
\pgfpathlineto{\pgfqpoint{5.139072in}{0.927276in}}%
\pgfpathlineto{\pgfqpoint{5.141660in}{0.921604in}}%
\pgfpathlineto{\pgfqpoint{5.144349in}{0.926740in}}%
\pgfpathlineto{\pgfqpoint{5.147029in}{0.942843in}}%
\pgfpathlineto{\pgfqpoint{5.149734in}{0.987477in}}%
\pgfpathlineto{\pgfqpoint{5.152382in}{1.037544in}}%
\pgfpathlineto{\pgfqpoint{5.155059in}{1.053141in}}%
\pgfpathlineto{\pgfqpoint{5.157815in}{1.044271in}}%
\pgfpathlineto{\pgfqpoint{5.160420in}{1.032973in}}%
\pgfpathlineto{\pgfqpoint{5.163243in}{1.031463in}}%
\pgfpathlineto{\pgfqpoint{5.165775in}{1.020079in}}%
\pgfpathlineto{\pgfqpoint{5.168591in}{1.028194in}}%
\pgfpathlineto{\pgfqpoint{5.171133in}{1.010040in}}%
\pgfpathlineto{\pgfqpoint{5.173925in}{1.015949in}}%
\pgfpathlineto{\pgfqpoint{5.176477in}{1.014867in}}%
\pgfpathlineto{\pgfqpoint{5.179188in}{0.996229in}}%
\pgfpathlineto{\pgfqpoint{5.181848in}{0.979824in}}%
\pgfpathlineto{\pgfqpoint{5.184522in}{0.980577in}}%
\pgfpathlineto{\pgfqpoint{5.187294in}{0.988422in}}%
\pgfpathlineto{\pgfqpoint{5.189880in}{0.997206in}}%
\pgfpathlineto{\pgfqpoint{5.192680in}{0.992671in}}%
\pgfpathlineto{\pgfqpoint{5.195239in}{0.970426in}}%
\pgfpathlineto{\pgfqpoint{5.198008in}{0.961603in}}%
\pgfpathlineto{\pgfqpoint{5.200594in}{0.947572in}}%
\pgfpathlineto{\pgfqpoint{5.203388in}{0.944862in}}%
\pgfpathlineto{\pgfqpoint{5.205952in}{0.935981in}}%
\pgfpathlineto{\pgfqpoint{5.208630in}{0.937276in}}%
\pgfpathlineto{\pgfqpoint{5.211299in}{0.937939in}}%
\pgfpathlineto{\pgfqpoint{5.214027in}{0.928412in}}%
\pgfpathlineto{\pgfqpoint{5.216667in}{0.930132in}}%
\pgfpathlineto{\pgfqpoint{5.219345in}{0.935616in}}%
\pgfpathlineto{\pgfqpoint{5.222151in}{0.930078in}}%
\pgfpathlineto{\pgfqpoint{5.224695in}{0.932032in}}%
\pgfpathlineto{\pgfqpoint{5.227470in}{0.939624in}}%
\pgfpathlineto{\pgfqpoint{5.230059in}{0.947565in}}%
\pgfpathlineto{\pgfqpoint{5.232855in}{0.943514in}}%
\pgfpathlineto{\pgfqpoint{5.235409in}{0.946548in}}%
\pgfpathlineto{\pgfqpoint{5.238173in}{0.943407in}}%
\pgfpathlineto{\pgfqpoint{5.240777in}{0.944684in}}%
\pgfpathlineto{\pgfqpoint{5.243445in}{0.940096in}}%
\pgfpathlineto{\pgfqpoint{5.246130in}{0.928017in}}%
\pgfpathlineto{\pgfqpoint{5.248816in}{0.922249in}}%
\pgfpathlineto{\pgfqpoint{5.251590in}{0.914729in}}%
\pgfpathlineto{\pgfqpoint{5.254236in}{0.914731in}}%
\pgfpathlineto{\pgfqpoint{5.256973in}{0.919982in}}%
\pgfpathlineto{\pgfqpoint{5.259511in}{0.920100in}}%
\pgfpathlineto{\pgfqpoint{5.262264in}{0.916587in}}%
\pgfpathlineto{\pgfqpoint{5.264876in}{0.918457in}}%
\pgfpathlineto{\pgfqpoint{5.267691in}{0.916818in}}%
\pgfpathlineto{\pgfqpoint{5.270238in}{0.917877in}}%
\pgfpathlineto{\pgfqpoint{5.272913in}{0.923825in}}%
\pgfpathlineto{\pgfqpoint{5.275589in}{0.923097in}}%
\pgfpathlineto{\pgfqpoint{5.278322in}{0.911227in}}%
\pgfpathlineto{\pgfqpoint{5.280947in}{0.907028in}}%
\pgfpathlineto{\pgfqpoint{5.283631in}{0.915122in}}%
\pgfpathlineto{\pgfqpoint{5.286436in}{0.913140in}}%
\pgfpathlineto{\pgfqpoint{5.288984in}{0.918212in}}%
\pgfpathlineto{\pgfqpoint{5.291794in}{0.916160in}}%
\pgfpathlineto{\pgfqpoint{5.294339in}{0.914323in}}%
\pgfpathlineto{\pgfqpoint{5.297140in}{0.914609in}}%
\pgfpathlineto{\pgfqpoint{5.299696in}{0.915476in}}%
\pgfpathlineto{\pgfqpoint{5.302443in}{0.914917in}}%
\pgfpathlineto{\pgfqpoint{5.305054in}{0.917522in}}%
\pgfpathlineto{\pgfqpoint{5.307731in}{0.914555in}}%
\pgfpathlineto{\pgfqpoint{5.310411in}{0.920889in}}%
\pgfpathlineto{\pgfqpoint{5.313089in}{0.920289in}}%
\pgfpathlineto{\pgfqpoint{5.315754in}{0.917189in}}%
\pgfpathlineto{\pgfqpoint{5.318430in}{0.918636in}}%
\pgfpathlineto{\pgfqpoint{5.321256in}{0.917732in}}%
\pgfpathlineto{\pgfqpoint{5.323802in}{0.916163in}}%
\pgfpathlineto{\pgfqpoint{5.326564in}{0.917866in}}%
\pgfpathlineto{\pgfqpoint{5.329159in}{0.916492in}}%
\pgfpathlineto{\pgfqpoint{5.331973in}{0.926346in}}%
\pgfpathlineto{\pgfqpoint{5.334510in}{0.919443in}}%
\pgfpathlineto{\pgfqpoint{5.337353in}{0.924062in}}%
\pgfpathlineto{\pgfqpoint{5.339872in}{0.919823in}}%
\pgfpathlineto{\pgfqpoint{5.342549in}{0.915604in}}%
\pgfpathlineto{\pgfqpoint{5.345224in}{0.907670in}}%
\pgfpathlineto{\pgfqpoint{5.347905in}{0.910682in}}%
\pgfpathlineto{\pgfqpoint{5.350723in}{0.913311in}}%
\pgfpathlineto{\pgfqpoint{5.353262in}{0.920208in}}%
\pgfpathlineto{\pgfqpoint{5.356056in}{0.922275in}}%
\pgfpathlineto{\pgfqpoint{5.358612in}{0.917784in}}%
\pgfpathlineto{\pgfqpoint{5.361370in}{0.925575in}}%
\pgfpathlineto{\pgfqpoint{5.363966in}{0.915973in}}%
\pgfpathlineto{\pgfqpoint{5.366727in}{0.915982in}}%
\pgfpathlineto{\pgfqpoint{5.369335in}{0.919386in}}%
\pgfpathlineto{\pgfqpoint{5.372013in}{0.918243in}}%
\pgfpathlineto{\pgfqpoint{5.374692in}{0.920910in}}%
\pgfpathlineto{\pgfqpoint{5.377370in}{0.920983in}}%
\pgfpathlineto{\pgfqpoint{5.380048in}{0.918750in}}%
\pgfpathlineto{\pgfqpoint{5.382725in}{0.922746in}}%
\pgfpathlineto{\pgfqpoint{5.385550in}{0.920417in}}%
\pgfpathlineto{\pgfqpoint{5.388083in}{0.917424in}}%
\pgfpathlineto{\pgfqpoint{5.390900in}{0.918423in}}%
\pgfpathlineto{\pgfqpoint{5.393441in}{0.911090in}}%
\pgfpathlineto{\pgfqpoint{5.396219in}{0.920647in}}%
\pgfpathlineto{\pgfqpoint{5.398784in}{0.941975in}}%
\pgfpathlineto{\pgfqpoint{5.401576in}{0.937253in}}%
\pgfpathlineto{\pgfqpoint{5.404154in}{0.929021in}}%
\pgfpathlineto{\pgfqpoint{5.406832in}{0.925497in}}%
\pgfpathlineto{\pgfqpoint{5.409507in}{0.926441in}}%
\pgfpathlineto{\pgfqpoint{5.412190in}{0.922937in}}%
\pgfpathlineto{\pgfqpoint{5.414954in}{0.924799in}}%
\pgfpathlineto{\pgfqpoint{5.417547in}{0.925019in}}%
\pgfpathlineto{\pgfqpoint{5.420304in}{0.928046in}}%
\pgfpathlineto{\pgfqpoint{5.422897in}{0.924288in}}%
\pgfpathlineto{\pgfqpoint{5.425661in}{0.926733in}}%
\pgfpathlineto{\pgfqpoint{5.428259in}{0.923940in}}%
\pgfpathlineto{\pgfqpoint{5.431015in}{0.923221in}}%
\pgfpathlineto{\pgfqpoint{5.433616in}{0.923932in}}%
\pgfpathlineto{\pgfqpoint{5.436295in}{0.922954in}}%
\pgfpathlineto{\pgfqpoint{5.438974in}{0.921140in}}%
\pgfpathlineto{\pgfqpoint{5.441698in}{0.923899in}}%
\pgfpathlineto{\pgfqpoint{5.444328in}{0.917549in}}%
\pgfpathlineto{\pgfqpoint{5.447021in}{0.921313in}}%
\pgfpathlineto{\pgfqpoint{5.449769in}{0.932622in}}%
\pgfpathlineto{\pgfqpoint{5.452365in}{0.924988in}}%
\pgfpathlineto{\pgfqpoint{5.455168in}{0.922633in}}%
\pgfpathlineto{\pgfqpoint{5.457721in}{0.923181in}}%
\pgfpathlineto{\pgfqpoint{5.460489in}{0.919305in}}%
\pgfpathlineto{\pgfqpoint{5.463079in}{0.922950in}}%
\pgfpathlineto{\pgfqpoint{5.465888in}{0.918349in}}%
\pgfpathlineto{\pgfqpoint{5.468425in}{0.916315in}}%
\pgfpathlineto{\pgfqpoint{5.471113in}{0.916804in}}%
\pgfpathlineto{\pgfqpoint{5.473792in}{0.916690in}}%
\pgfpathlineto{\pgfqpoint{5.476458in}{0.920438in}}%
\pgfpathlineto{\pgfqpoint{5.479152in}{0.925434in}}%
\pgfpathlineto{\pgfqpoint{5.481825in}{0.924678in}}%
\pgfpathlineto{\pgfqpoint{5.484641in}{0.921590in}}%
\pgfpathlineto{\pgfqpoint{5.487176in}{0.924255in}}%
\pgfpathlineto{\pgfqpoint{5.490000in}{0.924799in}}%
\pgfpathlineto{\pgfqpoint{5.492541in}{0.923363in}}%
\pgfpathlineto{\pgfqpoint{5.495346in}{0.926267in}}%
\pgfpathlineto{\pgfqpoint{5.497898in}{0.926387in}}%
\pgfpathlineto{\pgfqpoint{5.500687in}{0.925750in}}%
\pgfpathlineto{\pgfqpoint{5.503255in}{0.928080in}}%
\pgfpathlineto{\pgfqpoint{5.505933in}{0.927516in}}%
\pgfpathlineto{\pgfqpoint{5.508612in}{0.928197in}}%
\pgfpathlineto{\pgfqpoint{5.511290in}{0.928371in}}%
\pgfpathlineto{\pgfqpoint{5.514080in}{0.922596in}}%
\pgfpathlineto{\pgfqpoint{5.516646in}{0.939116in}}%
\pgfpathlineto{\pgfqpoint{5.519433in}{0.938972in}}%
\pgfpathlineto{\pgfqpoint{5.522003in}{0.927266in}}%
\pgfpathlineto{\pgfqpoint{5.524756in}{0.925692in}}%
\pgfpathlineto{\pgfqpoint{5.527360in}{0.926142in}}%
\pgfpathlineto{\pgfqpoint{5.530148in}{0.921055in}}%
\pgfpathlineto{\pgfqpoint{5.532717in}{0.924434in}}%
\pgfpathlineto{\pgfqpoint{5.535395in}{0.924053in}}%
\pgfpathlineto{\pgfqpoint{5.538074in}{0.926536in}}%
\pgfpathlineto{\pgfqpoint{5.540750in}{0.927378in}}%
\pgfpathlineto{\pgfqpoint{5.543421in}{0.923574in}}%
\pgfpathlineto{\pgfqpoint{5.546110in}{0.924470in}}%
\pgfpathlineto{\pgfqpoint{5.548921in}{0.928187in}}%
\pgfpathlineto{\pgfqpoint{5.551457in}{0.925857in}}%
\pgfpathlineto{\pgfqpoint{5.554198in}{0.923779in}}%
\pgfpathlineto{\pgfqpoint{5.556822in}{0.920024in}}%
\pgfpathlineto{\pgfqpoint{5.559612in}{0.923040in}}%
\pgfpathlineto{\pgfqpoint{5.562180in}{0.918838in}}%
\pgfpathlineto{\pgfqpoint{5.564940in}{0.918645in}}%
\pgfpathlineto{\pgfqpoint{5.567536in}{0.917630in}}%
\pgfpathlineto{\pgfqpoint{5.570215in}{0.924151in}}%
\pgfpathlineto{\pgfqpoint{5.572893in}{0.922208in}}%
\pgfpathlineto{\pgfqpoint{5.575596in}{0.922719in}}%
\pgfpathlineto{\pgfqpoint{5.578342in}{0.919815in}}%
\pgfpathlineto{\pgfqpoint{5.580914in}{0.921679in}}%
\pgfpathlineto{\pgfqpoint{5.583709in}{0.918137in}}%
\pgfpathlineto{\pgfqpoint{5.586269in}{0.925799in}}%
\pgfpathlineto{\pgfqpoint{5.589040in}{0.920866in}}%
\pgfpathlineto{\pgfqpoint{5.591641in}{0.918259in}}%
\pgfpathlineto{\pgfqpoint{5.594368in}{0.919952in}}%
\pgfpathlineto{\pgfqpoint{5.596999in}{0.918276in}}%
\pgfpathlineto{\pgfqpoint{5.599674in}{0.919922in}}%
\pgfpathlineto{\pgfqpoint{5.602352in}{0.922558in}}%
\pgfpathlineto{\pgfqpoint{5.605073in}{0.920319in}}%
\pgfpathlineto{\pgfqpoint{5.607698in}{0.918326in}}%
\pgfpathlineto{\pgfqpoint{5.610389in}{0.925262in}}%
\pgfpathlineto{\pgfqpoint{5.613235in}{0.916928in}}%
\pgfpathlineto{\pgfqpoint{5.615743in}{0.916782in}}%
\pgfpathlineto{\pgfqpoint{5.618526in}{0.918782in}}%
\pgfpathlineto{\pgfqpoint{5.621102in}{0.917214in}}%
\pgfpathlineto{\pgfqpoint{5.623868in}{0.917221in}}%
\pgfpathlineto{\pgfqpoint{5.626460in}{0.919322in}}%
\pgfpathlineto{\pgfqpoint{5.629232in}{0.920874in}}%
\pgfpathlineto{\pgfqpoint{5.631815in}{0.920376in}}%
\pgfpathlineto{\pgfqpoint{5.634496in}{0.919930in}}%
\pgfpathlineto{\pgfqpoint{5.637172in}{0.929429in}}%
\pgfpathlineto{\pgfqpoint{5.639852in}{0.939642in}}%
\pgfpathlineto{\pgfqpoint{5.642518in}{0.975316in}}%
\pgfpathlineto{\pgfqpoint{5.645243in}{0.966948in}}%
\pgfpathlineto{\pgfqpoint{5.648008in}{1.015518in}}%
\pgfpathlineto{\pgfqpoint{5.650563in}{1.010737in}}%
\pgfpathlineto{\pgfqpoint{5.653376in}{0.973778in}}%
\pgfpathlineto{\pgfqpoint{5.655919in}{0.944743in}}%
\pgfpathlineto{\pgfqpoint{5.658723in}{0.935444in}}%
\pgfpathlineto{\pgfqpoint{5.661273in}{0.924148in}}%
\pgfpathlineto{\pgfqpoint{5.664099in}{0.917752in}}%
\pgfpathlineto{\pgfqpoint{5.666632in}{0.930009in}}%
\pgfpathlineto{\pgfqpoint{5.669313in}{0.935906in}}%
\pgfpathlineto{\pgfqpoint{5.671991in}{0.922395in}}%
\pgfpathlineto{\pgfqpoint{5.674667in}{0.921900in}}%
\pgfpathlineto{\pgfqpoint{5.677486in}{0.913282in}}%
\pgfpathlineto{\pgfqpoint{5.680027in}{0.915151in}}%
\pgfpathlineto{\pgfqpoint{5.682836in}{0.912566in}}%
\pgfpathlineto{\pgfqpoint{5.685385in}{0.915926in}}%
\pgfpathlineto{\pgfqpoint{5.688159in}{0.914167in}}%
\pgfpathlineto{\pgfqpoint{5.690730in}{0.926659in}}%
\pgfpathlineto{\pgfqpoint{5.693473in}{0.936644in}}%
\pgfpathlineto{\pgfqpoint{5.696101in}{0.930836in}}%
\pgfpathlineto{\pgfqpoint{5.698775in}{0.923437in}}%
\pgfpathlineto{\pgfqpoint{5.701453in}{0.918524in}}%
\pgfpathlineto{\pgfqpoint{5.704130in}{0.912877in}}%
\pgfpathlineto{\pgfqpoint{5.706800in}{0.911852in}}%
\pgfpathlineto{\pgfqpoint{5.709490in}{0.911046in}}%
\pgfpathlineto{\pgfqpoint{5.712291in}{0.912404in}}%
\pgfpathlineto{\pgfqpoint{5.714834in}{0.913105in}}%
\pgfpathlineto{\pgfqpoint{5.717671in}{0.915840in}}%
\pgfpathlineto{\pgfqpoint{5.720201in}{0.913315in}}%
\pgfpathlineto{\pgfqpoint{5.722950in}{0.912337in}}%
\pgfpathlineto{\pgfqpoint{5.725548in}{0.916621in}}%
\pgfpathlineto{\pgfqpoint{5.728339in}{0.914709in}}%
\pgfpathlineto{\pgfqpoint{5.730919in}{0.961471in}}%
\pgfpathlineto{\pgfqpoint{5.733594in}{0.953677in}}%
\pgfpathlineto{\pgfqpoint{5.736276in}{0.939792in}}%
\pgfpathlineto{\pgfqpoint{5.738974in}{0.933837in}}%
\pgfpathlineto{\pgfqpoint{5.741745in}{0.929948in}}%
\pgfpathlineto{\pgfqpoint{5.744310in}{0.927877in}}%
\pgfpathlineto{\pgfqpoint{5.744310in}{0.413320in}}%
\pgfpathlineto{\pgfqpoint{5.744310in}{0.413320in}}%
\pgfpathlineto{\pgfqpoint{5.741745in}{0.413320in}}%
\pgfpathlineto{\pgfqpoint{5.738974in}{0.413320in}}%
\pgfpathlineto{\pgfqpoint{5.736276in}{0.413320in}}%
\pgfpathlineto{\pgfqpoint{5.733594in}{0.413320in}}%
\pgfpathlineto{\pgfqpoint{5.730919in}{0.413320in}}%
\pgfpathlineto{\pgfqpoint{5.728339in}{0.413320in}}%
\pgfpathlineto{\pgfqpoint{5.725548in}{0.413320in}}%
\pgfpathlineto{\pgfqpoint{5.722950in}{0.413320in}}%
\pgfpathlineto{\pgfqpoint{5.720201in}{0.413320in}}%
\pgfpathlineto{\pgfqpoint{5.717671in}{0.413320in}}%
\pgfpathlineto{\pgfqpoint{5.714834in}{0.413320in}}%
\pgfpathlineto{\pgfqpoint{5.712291in}{0.413320in}}%
\pgfpathlineto{\pgfqpoint{5.709490in}{0.413320in}}%
\pgfpathlineto{\pgfqpoint{5.706800in}{0.413320in}}%
\pgfpathlineto{\pgfqpoint{5.704130in}{0.413320in}}%
\pgfpathlineto{\pgfqpoint{5.701453in}{0.413320in}}%
\pgfpathlineto{\pgfqpoint{5.698775in}{0.413320in}}%
\pgfpathlineto{\pgfqpoint{5.696101in}{0.413320in}}%
\pgfpathlineto{\pgfqpoint{5.693473in}{0.413320in}}%
\pgfpathlineto{\pgfqpoint{5.690730in}{0.413320in}}%
\pgfpathlineto{\pgfqpoint{5.688159in}{0.413320in}}%
\pgfpathlineto{\pgfqpoint{5.685385in}{0.413320in}}%
\pgfpathlineto{\pgfqpoint{5.682836in}{0.413320in}}%
\pgfpathlineto{\pgfqpoint{5.680027in}{0.413320in}}%
\pgfpathlineto{\pgfqpoint{5.677486in}{0.413320in}}%
\pgfpathlineto{\pgfqpoint{5.674667in}{0.413320in}}%
\pgfpathlineto{\pgfqpoint{5.671991in}{0.413320in}}%
\pgfpathlineto{\pgfqpoint{5.669313in}{0.413320in}}%
\pgfpathlineto{\pgfqpoint{5.666632in}{0.413320in}}%
\pgfpathlineto{\pgfqpoint{5.664099in}{0.413320in}}%
\pgfpathlineto{\pgfqpoint{5.661273in}{0.413320in}}%
\pgfpathlineto{\pgfqpoint{5.658723in}{0.413320in}}%
\pgfpathlineto{\pgfqpoint{5.655919in}{0.413320in}}%
\pgfpathlineto{\pgfqpoint{5.653376in}{0.413320in}}%
\pgfpathlineto{\pgfqpoint{5.650563in}{0.413320in}}%
\pgfpathlineto{\pgfqpoint{5.648008in}{0.413320in}}%
\pgfpathlineto{\pgfqpoint{5.645243in}{0.413320in}}%
\pgfpathlineto{\pgfqpoint{5.642518in}{0.413320in}}%
\pgfpathlineto{\pgfqpoint{5.639852in}{0.413320in}}%
\pgfpathlineto{\pgfqpoint{5.637172in}{0.413320in}}%
\pgfpathlineto{\pgfqpoint{5.634496in}{0.413320in}}%
\pgfpathlineto{\pgfqpoint{5.631815in}{0.413320in}}%
\pgfpathlineto{\pgfqpoint{5.629232in}{0.413320in}}%
\pgfpathlineto{\pgfqpoint{5.626460in}{0.413320in}}%
\pgfpathlineto{\pgfqpoint{5.623868in}{0.413320in}}%
\pgfpathlineto{\pgfqpoint{5.621102in}{0.413320in}}%
\pgfpathlineto{\pgfqpoint{5.618526in}{0.413320in}}%
\pgfpathlineto{\pgfqpoint{5.615743in}{0.413320in}}%
\pgfpathlineto{\pgfqpoint{5.613235in}{0.413320in}}%
\pgfpathlineto{\pgfqpoint{5.610389in}{0.413320in}}%
\pgfpathlineto{\pgfqpoint{5.607698in}{0.413320in}}%
\pgfpathlineto{\pgfqpoint{5.605073in}{0.413320in}}%
\pgfpathlineto{\pgfqpoint{5.602352in}{0.413320in}}%
\pgfpathlineto{\pgfqpoint{5.599674in}{0.413320in}}%
\pgfpathlineto{\pgfqpoint{5.596999in}{0.413320in}}%
\pgfpathlineto{\pgfqpoint{5.594368in}{0.413320in}}%
\pgfpathlineto{\pgfqpoint{5.591641in}{0.413320in}}%
\pgfpathlineto{\pgfqpoint{5.589040in}{0.413320in}}%
\pgfpathlineto{\pgfqpoint{5.586269in}{0.413320in}}%
\pgfpathlineto{\pgfqpoint{5.583709in}{0.413320in}}%
\pgfpathlineto{\pgfqpoint{5.580914in}{0.413320in}}%
\pgfpathlineto{\pgfqpoint{5.578342in}{0.413320in}}%
\pgfpathlineto{\pgfqpoint{5.575596in}{0.413320in}}%
\pgfpathlineto{\pgfqpoint{5.572893in}{0.413320in}}%
\pgfpathlineto{\pgfqpoint{5.570215in}{0.413320in}}%
\pgfpathlineto{\pgfqpoint{5.567536in}{0.413320in}}%
\pgfpathlineto{\pgfqpoint{5.564940in}{0.413320in}}%
\pgfpathlineto{\pgfqpoint{5.562180in}{0.413320in}}%
\pgfpathlineto{\pgfqpoint{5.559612in}{0.413320in}}%
\pgfpathlineto{\pgfqpoint{5.556822in}{0.413320in}}%
\pgfpathlineto{\pgfqpoint{5.554198in}{0.413320in}}%
\pgfpathlineto{\pgfqpoint{5.551457in}{0.413320in}}%
\pgfpathlineto{\pgfqpoint{5.548921in}{0.413320in}}%
\pgfpathlineto{\pgfqpoint{5.546110in}{0.413320in}}%
\pgfpathlineto{\pgfqpoint{5.543421in}{0.413320in}}%
\pgfpathlineto{\pgfqpoint{5.540750in}{0.413320in}}%
\pgfpathlineto{\pgfqpoint{5.538074in}{0.413320in}}%
\pgfpathlineto{\pgfqpoint{5.535395in}{0.413320in}}%
\pgfpathlineto{\pgfqpoint{5.532717in}{0.413320in}}%
\pgfpathlineto{\pgfqpoint{5.530148in}{0.413320in}}%
\pgfpathlineto{\pgfqpoint{5.527360in}{0.413320in}}%
\pgfpathlineto{\pgfqpoint{5.524756in}{0.413320in}}%
\pgfpathlineto{\pgfqpoint{5.522003in}{0.413320in}}%
\pgfpathlineto{\pgfqpoint{5.519433in}{0.413320in}}%
\pgfpathlineto{\pgfqpoint{5.516646in}{0.413320in}}%
\pgfpathlineto{\pgfqpoint{5.514080in}{0.413320in}}%
\pgfpathlineto{\pgfqpoint{5.511290in}{0.413320in}}%
\pgfpathlineto{\pgfqpoint{5.508612in}{0.413320in}}%
\pgfpathlineto{\pgfqpoint{5.505933in}{0.413320in}}%
\pgfpathlineto{\pgfqpoint{5.503255in}{0.413320in}}%
\pgfpathlineto{\pgfqpoint{5.500687in}{0.413320in}}%
\pgfpathlineto{\pgfqpoint{5.497898in}{0.413320in}}%
\pgfpathlineto{\pgfqpoint{5.495346in}{0.413320in}}%
\pgfpathlineto{\pgfqpoint{5.492541in}{0.413320in}}%
\pgfpathlineto{\pgfqpoint{5.490000in}{0.413320in}}%
\pgfpathlineto{\pgfqpoint{5.487176in}{0.413320in}}%
\pgfpathlineto{\pgfqpoint{5.484641in}{0.413320in}}%
\pgfpathlineto{\pgfqpoint{5.481825in}{0.413320in}}%
\pgfpathlineto{\pgfqpoint{5.479152in}{0.413320in}}%
\pgfpathlineto{\pgfqpoint{5.476458in}{0.413320in}}%
\pgfpathlineto{\pgfqpoint{5.473792in}{0.413320in}}%
\pgfpathlineto{\pgfqpoint{5.471113in}{0.413320in}}%
\pgfpathlineto{\pgfqpoint{5.468425in}{0.413320in}}%
\pgfpathlineto{\pgfqpoint{5.465888in}{0.413320in}}%
\pgfpathlineto{\pgfqpoint{5.463079in}{0.413320in}}%
\pgfpathlineto{\pgfqpoint{5.460489in}{0.413320in}}%
\pgfpathlineto{\pgfqpoint{5.457721in}{0.413320in}}%
\pgfpathlineto{\pgfqpoint{5.455168in}{0.413320in}}%
\pgfpathlineto{\pgfqpoint{5.452365in}{0.413320in}}%
\pgfpathlineto{\pgfqpoint{5.449769in}{0.413320in}}%
\pgfpathlineto{\pgfqpoint{5.447021in}{0.413320in}}%
\pgfpathlineto{\pgfqpoint{5.444328in}{0.413320in}}%
\pgfpathlineto{\pgfqpoint{5.441698in}{0.413320in}}%
\pgfpathlineto{\pgfqpoint{5.438974in}{0.413320in}}%
\pgfpathlineto{\pgfqpoint{5.436295in}{0.413320in}}%
\pgfpathlineto{\pgfqpoint{5.433616in}{0.413320in}}%
\pgfpathlineto{\pgfqpoint{5.431015in}{0.413320in}}%
\pgfpathlineto{\pgfqpoint{5.428259in}{0.413320in}}%
\pgfpathlineto{\pgfqpoint{5.425661in}{0.413320in}}%
\pgfpathlineto{\pgfqpoint{5.422897in}{0.413320in}}%
\pgfpathlineto{\pgfqpoint{5.420304in}{0.413320in}}%
\pgfpathlineto{\pgfqpoint{5.417547in}{0.413320in}}%
\pgfpathlineto{\pgfqpoint{5.414954in}{0.413320in}}%
\pgfpathlineto{\pgfqpoint{5.412190in}{0.413320in}}%
\pgfpathlineto{\pgfqpoint{5.409507in}{0.413320in}}%
\pgfpathlineto{\pgfqpoint{5.406832in}{0.413320in}}%
\pgfpathlineto{\pgfqpoint{5.404154in}{0.413320in}}%
\pgfpathlineto{\pgfqpoint{5.401576in}{0.413320in}}%
\pgfpathlineto{\pgfqpoint{5.398784in}{0.413320in}}%
\pgfpathlineto{\pgfqpoint{5.396219in}{0.413320in}}%
\pgfpathlineto{\pgfqpoint{5.393441in}{0.413320in}}%
\pgfpathlineto{\pgfqpoint{5.390900in}{0.413320in}}%
\pgfpathlineto{\pgfqpoint{5.388083in}{0.413320in}}%
\pgfpathlineto{\pgfqpoint{5.385550in}{0.413320in}}%
\pgfpathlineto{\pgfqpoint{5.382725in}{0.413320in}}%
\pgfpathlineto{\pgfqpoint{5.380048in}{0.413320in}}%
\pgfpathlineto{\pgfqpoint{5.377370in}{0.413320in}}%
\pgfpathlineto{\pgfqpoint{5.374692in}{0.413320in}}%
\pgfpathlineto{\pgfqpoint{5.372013in}{0.413320in}}%
\pgfpathlineto{\pgfqpoint{5.369335in}{0.413320in}}%
\pgfpathlineto{\pgfqpoint{5.366727in}{0.413320in}}%
\pgfpathlineto{\pgfqpoint{5.363966in}{0.413320in}}%
\pgfpathlineto{\pgfqpoint{5.361370in}{0.413320in}}%
\pgfpathlineto{\pgfqpoint{5.358612in}{0.413320in}}%
\pgfpathlineto{\pgfqpoint{5.356056in}{0.413320in}}%
\pgfpathlineto{\pgfqpoint{5.353262in}{0.413320in}}%
\pgfpathlineto{\pgfqpoint{5.350723in}{0.413320in}}%
\pgfpathlineto{\pgfqpoint{5.347905in}{0.413320in}}%
\pgfpathlineto{\pgfqpoint{5.345224in}{0.413320in}}%
\pgfpathlineto{\pgfqpoint{5.342549in}{0.413320in}}%
\pgfpathlineto{\pgfqpoint{5.339872in}{0.413320in}}%
\pgfpathlineto{\pgfqpoint{5.337353in}{0.413320in}}%
\pgfpathlineto{\pgfqpoint{5.334510in}{0.413320in}}%
\pgfpathlineto{\pgfqpoint{5.331973in}{0.413320in}}%
\pgfpathlineto{\pgfqpoint{5.329159in}{0.413320in}}%
\pgfpathlineto{\pgfqpoint{5.326564in}{0.413320in}}%
\pgfpathlineto{\pgfqpoint{5.323802in}{0.413320in}}%
\pgfpathlineto{\pgfqpoint{5.321256in}{0.413320in}}%
\pgfpathlineto{\pgfqpoint{5.318430in}{0.413320in}}%
\pgfpathlineto{\pgfqpoint{5.315754in}{0.413320in}}%
\pgfpathlineto{\pgfqpoint{5.313089in}{0.413320in}}%
\pgfpathlineto{\pgfqpoint{5.310411in}{0.413320in}}%
\pgfpathlineto{\pgfqpoint{5.307731in}{0.413320in}}%
\pgfpathlineto{\pgfqpoint{5.305054in}{0.413320in}}%
\pgfpathlineto{\pgfqpoint{5.302443in}{0.413320in}}%
\pgfpathlineto{\pgfqpoint{5.299696in}{0.413320in}}%
\pgfpathlineto{\pgfqpoint{5.297140in}{0.413320in}}%
\pgfpathlineto{\pgfqpoint{5.294339in}{0.413320in}}%
\pgfpathlineto{\pgfqpoint{5.291794in}{0.413320in}}%
\pgfpathlineto{\pgfqpoint{5.288984in}{0.413320in}}%
\pgfpathlineto{\pgfqpoint{5.286436in}{0.413320in}}%
\pgfpathlineto{\pgfqpoint{5.283631in}{0.413320in}}%
\pgfpathlineto{\pgfqpoint{5.280947in}{0.413320in}}%
\pgfpathlineto{\pgfqpoint{5.278322in}{0.413320in}}%
\pgfpathlineto{\pgfqpoint{5.275589in}{0.413320in}}%
\pgfpathlineto{\pgfqpoint{5.272913in}{0.413320in}}%
\pgfpathlineto{\pgfqpoint{5.270238in}{0.413320in}}%
\pgfpathlineto{\pgfqpoint{5.267691in}{0.413320in}}%
\pgfpathlineto{\pgfqpoint{5.264876in}{0.413320in}}%
\pgfpathlineto{\pgfqpoint{5.262264in}{0.413320in}}%
\pgfpathlineto{\pgfqpoint{5.259511in}{0.413320in}}%
\pgfpathlineto{\pgfqpoint{5.256973in}{0.413320in}}%
\pgfpathlineto{\pgfqpoint{5.254236in}{0.413320in}}%
\pgfpathlineto{\pgfqpoint{5.251590in}{0.413320in}}%
\pgfpathlineto{\pgfqpoint{5.248816in}{0.413320in}}%
\pgfpathlineto{\pgfqpoint{5.246130in}{0.413320in}}%
\pgfpathlineto{\pgfqpoint{5.243445in}{0.413320in}}%
\pgfpathlineto{\pgfqpoint{5.240777in}{0.413320in}}%
\pgfpathlineto{\pgfqpoint{5.238173in}{0.413320in}}%
\pgfpathlineto{\pgfqpoint{5.235409in}{0.413320in}}%
\pgfpathlineto{\pgfqpoint{5.232855in}{0.413320in}}%
\pgfpathlineto{\pgfqpoint{5.230059in}{0.413320in}}%
\pgfpathlineto{\pgfqpoint{5.227470in}{0.413320in}}%
\pgfpathlineto{\pgfqpoint{5.224695in}{0.413320in}}%
\pgfpathlineto{\pgfqpoint{5.222151in}{0.413320in}}%
\pgfpathlineto{\pgfqpoint{5.219345in}{0.413320in}}%
\pgfpathlineto{\pgfqpoint{5.216667in}{0.413320in}}%
\pgfpathlineto{\pgfqpoint{5.214027in}{0.413320in}}%
\pgfpathlineto{\pgfqpoint{5.211299in}{0.413320in}}%
\pgfpathlineto{\pgfqpoint{5.208630in}{0.413320in}}%
\pgfpathlineto{\pgfqpoint{5.205952in}{0.413320in}}%
\pgfpathlineto{\pgfqpoint{5.203388in}{0.413320in}}%
\pgfpathlineto{\pgfqpoint{5.200594in}{0.413320in}}%
\pgfpathlineto{\pgfqpoint{5.198008in}{0.413320in}}%
\pgfpathlineto{\pgfqpoint{5.195239in}{0.413320in}}%
\pgfpathlineto{\pgfqpoint{5.192680in}{0.413320in}}%
\pgfpathlineto{\pgfqpoint{5.189880in}{0.413320in}}%
\pgfpathlineto{\pgfqpoint{5.187294in}{0.413320in}}%
\pgfpathlineto{\pgfqpoint{5.184522in}{0.413320in}}%
\pgfpathlineto{\pgfqpoint{5.181848in}{0.413320in}}%
\pgfpathlineto{\pgfqpoint{5.179188in}{0.413320in}}%
\pgfpathlineto{\pgfqpoint{5.176477in}{0.413320in}}%
\pgfpathlineto{\pgfqpoint{5.173925in}{0.413320in}}%
\pgfpathlineto{\pgfqpoint{5.171133in}{0.413320in}}%
\pgfpathlineto{\pgfqpoint{5.168591in}{0.413320in}}%
\pgfpathlineto{\pgfqpoint{5.165775in}{0.413320in}}%
\pgfpathlineto{\pgfqpoint{5.163243in}{0.413320in}}%
\pgfpathlineto{\pgfqpoint{5.160420in}{0.413320in}}%
\pgfpathlineto{\pgfqpoint{5.157815in}{0.413320in}}%
\pgfpathlineto{\pgfqpoint{5.155059in}{0.413320in}}%
\pgfpathlineto{\pgfqpoint{5.152382in}{0.413320in}}%
\pgfpathlineto{\pgfqpoint{5.149734in}{0.413320in}}%
\pgfpathlineto{\pgfqpoint{5.147029in}{0.413320in}}%
\pgfpathlineto{\pgfqpoint{5.144349in}{0.413320in}}%
\pgfpathlineto{\pgfqpoint{5.141660in}{0.413320in}}%
\pgfpathlineto{\pgfqpoint{5.139072in}{0.413320in}}%
\pgfpathlineto{\pgfqpoint{5.136311in}{0.413320in}}%
\pgfpathlineto{\pgfqpoint{5.133716in}{0.413320in}}%
\pgfpathlineto{\pgfqpoint{5.130953in}{0.413320in}}%
\pgfpathlineto{\pgfqpoint{5.128421in}{0.413320in}}%
\pgfpathlineto{\pgfqpoint{5.125599in}{0.413320in}}%
\pgfpathlineto{\pgfqpoint{5.123042in}{0.413320in}}%
\pgfpathlineto{\pgfqpoint{5.120243in}{0.413320in}}%
\pgfpathlineto{\pgfqpoint{5.117550in}{0.413320in}}%
\pgfpathlineto{\pgfqpoint{5.114887in}{0.413320in}}%
\pgfpathlineto{\pgfqpoint{5.112209in}{0.413320in}}%
\pgfpathlineto{\pgfqpoint{5.109530in}{0.413320in}}%
\pgfpathlineto{\pgfqpoint{5.106842in}{0.413320in}}%
\pgfpathlineto{\pgfqpoint{5.104312in}{0.413320in}}%
\pgfpathlineto{\pgfqpoint{5.101496in}{0.413320in}}%
\pgfpathlineto{\pgfqpoint{5.098948in}{0.413320in}}%
\pgfpathlineto{\pgfqpoint{5.096142in}{0.413320in}}%
\pgfpathlineto{\pgfqpoint{5.093579in}{0.413320in}}%
\pgfpathlineto{\pgfqpoint{5.090788in}{0.413320in}}%
\pgfpathlineto{\pgfqpoint{5.088103in}{0.413320in}}%
\pgfpathlineto{\pgfqpoint{5.085426in}{0.413320in}}%
\pgfpathlineto{\pgfqpoint{5.082746in}{0.413320in}}%
\pgfpathlineto{\pgfqpoint{5.080067in}{0.413320in}}%
\pgfpathlineto{\pgfqpoint{5.077390in}{0.413320in}}%
\pgfpathlineto{\pgfqpoint{5.074851in}{0.413320in}}%
\pgfpathlineto{\pgfqpoint{5.072030in}{0.413320in}}%
\pgfpathlineto{\pgfqpoint{5.069463in}{0.413320in}}%
\pgfpathlineto{\pgfqpoint{5.066677in}{0.413320in}}%
\pgfpathlineto{\pgfqpoint{5.064144in}{0.413320in}}%
\pgfpathlineto{\pgfqpoint{5.061315in}{0.413320in}}%
\pgfpathlineto{\pgfqpoint{5.058711in}{0.413320in}}%
\pgfpathlineto{\pgfqpoint{5.055952in}{0.413320in}}%
\pgfpathlineto{\pgfqpoint{5.053284in}{0.413320in}}%
\pgfpathlineto{\pgfqpoint{5.050606in}{0.413320in}}%
\pgfpathlineto{\pgfqpoint{5.047924in}{0.413320in}}%
\pgfpathlineto{\pgfqpoint{5.045249in}{0.413320in}}%
\pgfpathlineto{\pgfqpoint{5.042572in}{0.413320in}}%
\pgfpathlineto{\pgfqpoint{5.039962in}{0.413320in}}%
\pgfpathlineto{\pgfqpoint{5.037214in}{0.413320in}}%
\pgfpathlineto{\pgfqpoint{5.034649in}{0.413320in}}%
\pgfpathlineto{\pgfqpoint{5.031849in}{0.413320in}}%
\pgfpathlineto{\pgfqpoint{5.029275in}{0.413320in}}%
\pgfpathlineto{\pgfqpoint{5.026501in}{0.413320in}}%
\pgfpathlineto{\pgfqpoint{5.023927in}{0.413320in}}%
\pgfpathlineto{\pgfqpoint{5.021147in}{0.413320in}}%
\pgfpathlineto{\pgfqpoint{5.018466in}{0.413320in}}%
\pgfpathlineto{\pgfqpoint{5.015820in}{0.413320in}}%
\pgfpathlineto{\pgfqpoint{5.013104in}{0.413320in}}%
\pgfpathlineto{\pgfqpoint{5.010562in}{0.413320in}}%
\pgfpathlineto{\pgfqpoint{5.007751in}{0.413320in}}%
\pgfpathlineto{\pgfqpoint{5.005178in}{0.413320in}}%
\pgfpathlineto{\pgfqpoint{5.002384in}{0.413320in}}%
\pgfpathlineto{\pgfqpoint{4.999780in}{0.413320in}}%
\pgfpathlineto{\pgfqpoint{4.997028in}{0.413320in}}%
\pgfpathlineto{\pgfqpoint{4.994390in}{0.413320in}}%
\pgfpathlineto{\pgfqpoint{4.991683in}{0.413320in}}%
\pgfpathlineto{\pgfqpoint{4.989001in}{0.413320in}}%
\pgfpathlineto{\pgfqpoint{4.986325in}{0.413320in}}%
\pgfpathlineto{\pgfqpoint{4.983637in}{0.413320in}}%
\pgfpathlineto{\pgfqpoint{4.980967in}{0.413320in}}%
\pgfpathlineto{\pgfqpoint{4.978287in}{0.413320in}}%
\pgfpathlineto{\pgfqpoint{4.975703in}{0.413320in}}%
\pgfpathlineto{\pgfqpoint{4.972933in}{0.413320in}}%
\pgfpathlineto{\pgfqpoint{4.970314in}{0.413320in}}%
\pgfpathlineto{\pgfqpoint{4.967575in}{0.413320in}}%
\pgfpathlineto{\pgfqpoint{4.965002in}{0.413320in}}%
\pgfpathlineto{\pgfqpoint{4.962219in}{0.413320in}}%
\pgfpathlineto{\pgfqpoint{4.959689in}{0.413320in}}%
\pgfpathlineto{\pgfqpoint{4.956862in}{0.413320in}}%
\pgfpathlineto{\pgfqpoint{4.954182in}{0.413320in}}%
\pgfpathlineto{\pgfqpoint{4.951504in}{0.413320in}}%
\pgfpathlineto{\pgfqpoint{4.948827in}{0.413320in}}%
\pgfpathlineto{\pgfqpoint{4.946151in}{0.413320in}}%
\pgfpathlineto{\pgfqpoint{4.943466in}{0.413320in}}%
\pgfpathlineto{\pgfqpoint{4.940881in}{0.413320in}}%
\pgfpathlineto{\pgfqpoint{4.938112in}{0.413320in}}%
\pgfpathlineto{\pgfqpoint{4.935515in}{0.413320in}}%
\pgfpathlineto{\pgfqpoint{4.932742in}{0.413320in}}%
\pgfpathlineto{\pgfqpoint{4.930170in}{0.413320in}}%
\pgfpathlineto{\pgfqpoint{4.927400in}{0.413320in}}%
\pgfpathlineto{\pgfqpoint{4.924708in}{0.413320in}}%
\pgfpathlineto{\pgfqpoint{4.922041in}{0.413320in}}%
\pgfpathlineto{\pgfqpoint{4.919352in}{0.413320in}}%
\pgfpathlineto{\pgfqpoint{4.916681in}{0.413320in}}%
\pgfpathlineto{\pgfqpoint{4.914009in}{0.413320in}}%
\pgfpathlineto{\pgfqpoint{4.911435in}{0.413320in}}%
\pgfpathlineto{\pgfqpoint{4.908648in}{0.413320in}}%
\pgfpathlineto{\pgfqpoint{4.906096in}{0.413320in}}%
\pgfpathlineto{\pgfqpoint{4.903295in}{0.413320in}}%
\pgfpathlineto{\pgfqpoint{4.900712in}{0.413320in}}%
\pgfpathlineto{\pgfqpoint{4.897938in}{0.413320in}}%
\pgfpathlineto{\pgfqpoint{4.895399in}{0.413320in}}%
\pgfpathlineto{\pgfqpoint{4.892611in}{0.413320in}}%
\pgfpathlineto{\pgfqpoint{4.889902in}{0.413320in}}%
\pgfpathlineto{\pgfqpoint{4.887211in}{0.413320in}}%
\pgfpathlineto{\pgfqpoint{4.884540in}{0.413320in}}%
\pgfpathlineto{\pgfqpoint{4.881864in}{0.413320in}}%
\pgfpathlineto{\pgfqpoint{4.879180in}{0.413320in}}%
\pgfpathlineto{\pgfqpoint{4.876636in}{0.413320in}}%
\pgfpathlineto{\pgfqpoint{4.873832in}{0.413320in}}%
\pgfpathlineto{\pgfqpoint{4.871209in}{0.413320in}}%
\pgfpathlineto{\pgfqpoint{4.868474in}{0.413320in}}%
\pgfpathlineto{\pgfqpoint{4.865910in}{0.413320in}}%
\pgfpathlineto{\pgfqpoint{4.863116in}{0.413320in}}%
\pgfpathlineto{\pgfqpoint{4.860544in}{0.413320in}}%
\pgfpathlineto{\pgfqpoint{4.857807in}{0.413320in}}%
\pgfpathlineto{\pgfqpoint{4.855070in}{0.413320in}}%
\pgfpathlineto{\pgfqpoint{4.852404in}{0.413320in}}%
\pgfpathlineto{\pgfqpoint{4.849715in}{0.413320in}}%
\pgfpathlineto{\pgfqpoint{4.847127in}{0.413320in}}%
\pgfpathlineto{\pgfqpoint{4.844361in}{0.413320in}}%
\pgfpathlineto{\pgfqpoint{4.842380in}{0.413320in}}%
\pgfpathlineto{\pgfqpoint{4.839922in}{0.413320in}}%
\pgfpathlineto{\pgfqpoint{4.837992in}{0.413320in}}%
\pgfpathlineto{\pgfqpoint{4.833657in}{0.413320in}}%
\pgfpathlineto{\pgfqpoint{4.831045in}{0.413320in}}%
\pgfpathlineto{\pgfqpoint{4.828291in}{0.413320in}}%
\pgfpathlineto{\pgfqpoint{4.825619in}{0.413320in}}%
\pgfpathlineto{\pgfqpoint{4.822945in}{0.413320in}}%
\pgfpathlineto{\pgfqpoint{4.820265in}{0.413320in}}%
\pgfpathlineto{\pgfqpoint{4.817587in}{0.413320in}}%
\pgfpathlineto{\pgfqpoint{4.814907in}{0.413320in}}%
\pgfpathlineto{\pgfqpoint{4.812377in}{0.413320in}}%
\pgfpathlineto{\pgfqpoint{4.809538in}{0.413320in}}%
\pgfpathlineto{\pgfqpoint{4.807017in}{0.413320in}}%
\pgfpathlineto{\pgfqpoint{4.804193in}{0.413320in}}%
\pgfpathlineto{\pgfqpoint{4.801586in}{0.413320in}}%
\pgfpathlineto{\pgfqpoint{4.798830in}{0.413320in}}%
\pgfpathlineto{\pgfqpoint{4.796274in}{0.413320in}}%
\pgfpathlineto{\pgfqpoint{4.793512in}{0.413320in}}%
\pgfpathlineto{\pgfqpoint{4.790798in}{0.413320in}}%
\pgfpathlineto{\pgfqpoint{4.788116in}{0.413320in}}%
\pgfpathlineto{\pgfqpoint{4.785445in}{0.413320in}}%
\pgfpathlineto{\pgfqpoint{4.782872in}{0.413320in}}%
\pgfpathlineto{\pgfqpoint{4.780083in}{0.413320in}}%
\pgfpathlineto{\pgfqpoint{4.777535in}{0.413320in}}%
\pgfpathlineto{\pgfqpoint{4.774732in}{0.413320in}}%
\pgfpathlineto{\pgfqpoint{4.772198in}{0.413320in}}%
\pgfpathlineto{\pgfqpoint{4.769367in}{0.413320in}}%
\pgfpathlineto{\pgfqpoint{4.766783in}{0.413320in}}%
\pgfpathlineto{\pgfqpoint{4.764018in}{0.413320in}}%
\pgfpathlineto{\pgfqpoint{4.761337in}{0.413320in}}%
\pgfpathlineto{\pgfqpoint{4.758653in}{0.413320in}}%
\pgfpathlineto{\pgfqpoint{4.755983in}{0.413320in}}%
\pgfpathlineto{\pgfqpoint{4.753298in}{0.413320in}}%
\pgfpathlineto{\pgfqpoint{4.750627in}{0.413320in}}%
\pgfpathlineto{\pgfqpoint{4.748081in}{0.413320in}}%
\pgfpathlineto{\pgfqpoint{4.745256in}{0.413320in}}%
\pgfpathlineto{\pgfqpoint{4.742696in}{0.413320in}}%
\pgfpathlineto{\pgfqpoint{4.739912in}{0.413320in}}%
\pgfpathlineto{\pgfqpoint{4.737348in}{0.413320in}}%
\pgfpathlineto{\pgfqpoint{4.734552in}{0.413320in}}%
\pgfpathlineto{\pgfqpoint{4.731901in}{0.413320in}}%
\pgfpathlineto{\pgfqpoint{4.729233in}{0.413320in}}%
\pgfpathlineto{\pgfqpoint{4.726508in}{0.413320in}}%
\pgfpathlineto{\pgfqpoint{4.723873in}{0.413320in}}%
\pgfpathlineto{\pgfqpoint{4.721160in}{0.413320in}}%
\pgfpathlineto{\pgfqpoint{4.718486in}{0.413320in}}%
\pgfpathlineto{\pgfqpoint{4.715806in}{0.413320in}}%
\pgfpathlineto{\pgfqpoint{4.713275in}{0.413320in}}%
\pgfpathlineto{\pgfqpoint{4.710437in}{0.413320in}}%
\pgfpathlineto{\pgfqpoint{4.707824in}{0.413320in}}%
\pgfpathlineto{\pgfqpoint{4.705094in}{0.413320in}}%
\pgfpathlineto{\pgfqpoint{4.702517in}{0.413320in}}%
\pgfpathlineto{\pgfqpoint{4.699734in}{0.413320in}}%
\pgfpathlineto{\pgfqpoint{4.697170in}{0.413320in}}%
\pgfpathlineto{\pgfqpoint{4.694381in}{0.413320in}}%
\pgfpathlineto{\pgfqpoint{4.691694in}{0.413320in}}%
\pgfpathlineto{\pgfqpoint{4.689051in}{0.413320in}}%
\pgfpathlineto{\pgfqpoint{4.686337in}{0.413320in}}%
\pgfpathlineto{\pgfqpoint{4.683799in}{0.413320in}}%
\pgfpathlineto{\pgfqpoint{4.680988in}{0.413320in}}%
\pgfpathlineto{\pgfqpoint{4.678448in}{0.413320in}}%
\pgfpathlineto{\pgfqpoint{4.675619in}{0.413320in}}%
\pgfpathlineto{\pgfqpoint{4.673068in}{0.413320in}}%
\pgfpathlineto{\pgfqpoint{4.670261in}{0.413320in}}%
\pgfpathlineto{\pgfqpoint{4.667764in}{0.413320in}}%
\pgfpathlineto{\pgfqpoint{4.664923in}{0.413320in}}%
\pgfpathlineto{\pgfqpoint{4.662237in}{0.413320in}}%
\pgfpathlineto{\pgfqpoint{4.659590in}{0.413320in}}%
\pgfpathlineto{\pgfqpoint{4.656873in}{0.413320in}}%
\pgfpathlineto{\pgfqpoint{4.654203in}{0.413320in}}%
\pgfpathlineto{\pgfqpoint{4.651524in}{0.413320in}}%
\pgfpathlineto{\pgfqpoint{4.648922in}{0.413320in}}%
\pgfpathlineto{\pgfqpoint{4.646169in}{0.413320in}}%
\pgfpathlineto{\pgfqpoint{4.643628in}{0.413320in}}%
\pgfpathlineto{\pgfqpoint{4.640809in}{0.413320in}}%
\pgfpathlineto{\pgfqpoint{4.638204in}{0.413320in}}%
\pgfpathlineto{\pgfqpoint{4.635445in}{0.413320in}}%
\pgfpathlineto{\pgfqpoint{4.632902in}{0.413320in}}%
\pgfpathlineto{\pgfqpoint{4.630096in}{0.413320in}}%
\pgfpathlineto{\pgfqpoint{4.627411in}{0.413320in}}%
\pgfpathlineto{\pgfqpoint{4.624741in}{0.413320in}}%
\pgfpathlineto{\pgfqpoint{4.622056in}{0.413320in}}%
\pgfpathlineto{\pgfqpoint{4.619529in}{0.413320in}}%
\pgfpathlineto{\pgfqpoint{4.616702in}{0.413320in}}%
\pgfpathlineto{\pgfqpoint{4.614134in}{0.413320in}}%
\pgfpathlineto{\pgfqpoint{4.611350in}{0.413320in}}%
\pgfpathlineto{\pgfqpoint{4.608808in}{0.413320in}}%
\pgfpathlineto{\pgfqpoint{4.605990in}{0.413320in}}%
\pgfpathlineto{\pgfqpoint{4.603430in}{0.413320in}}%
\pgfpathlineto{\pgfqpoint{4.600633in}{0.413320in}}%
\pgfpathlineto{\pgfqpoint{4.597951in}{0.413320in}}%
\pgfpathlineto{\pgfqpoint{4.595281in}{0.413320in}}%
\pgfpathlineto{\pgfqpoint{4.592589in}{0.413320in}}%
\pgfpathlineto{\pgfqpoint{4.589920in}{0.413320in}}%
\pgfpathlineto{\pgfqpoint{4.587244in}{0.413320in}}%
\pgfpathlineto{\pgfqpoint{4.584672in}{0.413320in}}%
\pgfpathlineto{\pgfqpoint{4.581888in}{0.413320in}}%
\pgfpathlineto{\pgfqpoint{4.579305in}{0.413320in}}%
\pgfpathlineto{\pgfqpoint{4.576531in}{0.413320in}}%
\pgfpathlineto{\pgfqpoint{4.573947in}{0.413320in}}%
\pgfpathlineto{\pgfqpoint{4.571171in}{0.413320in}}%
\pgfpathlineto{\pgfqpoint{4.568612in}{0.413320in}}%
\pgfpathlineto{\pgfqpoint{4.565820in}{0.413320in}}%
\pgfpathlineto{\pgfqpoint{4.563125in}{0.413320in}}%
\pgfpathlineto{\pgfqpoint{4.560448in}{0.413320in}}%
\pgfpathlineto{\pgfqpoint{4.557777in}{0.413320in}}%
\pgfpathlineto{\pgfqpoint{4.555106in}{0.413320in}}%
\pgfpathlineto{\pgfqpoint{4.552425in}{0.413320in}}%
\pgfpathlineto{\pgfqpoint{4.549822in}{0.413320in}}%
\pgfpathlineto{\pgfqpoint{4.547064in}{0.413320in}}%
\pgfpathlineto{\pgfqpoint{4.544464in}{0.413320in}}%
\pgfpathlineto{\pgfqpoint{4.541711in}{0.413320in}}%
\pgfpathlineto{\pgfqpoint{4.539144in}{0.413320in}}%
\pgfpathlineto{\pgfqpoint{4.536400in}{0.413320in}}%
\pgfpathlineto{\pgfqpoint{4.533764in}{0.413320in}}%
\pgfpathlineto{\pgfqpoint{4.530990in}{0.413320in}}%
\pgfpathlineto{\pgfqpoint{4.528307in}{0.413320in}}%
\pgfpathlineto{\pgfqpoint{4.525640in}{0.413320in}}%
\pgfpathlineto{\pgfqpoint{4.522962in}{0.413320in}}%
\pgfpathlineto{\pgfqpoint{4.520345in}{0.413320in}}%
\pgfpathlineto{\pgfqpoint{4.517598in}{0.413320in}}%
\pgfpathlineto{\pgfqpoint{4.515080in}{0.413320in}}%
\pgfpathlineto{\pgfqpoint{4.512246in}{0.413320in}}%
\pgfpathlineto{\pgfqpoint{4.509643in}{0.413320in}}%
\pgfpathlineto{\pgfqpoint{4.506893in}{0.413320in}}%
\pgfpathlineto{\pgfqpoint{4.504305in}{0.413320in}}%
\pgfpathlineto{\pgfqpoint{4.501529in}{0.413320in}}%
\pgfpathlineto{\pgfqpoint{4.498850in}{0.413320in}}%
\pgfpathlineto{\pgfqpoint{4.496167in}{0.413320in}}%
\pgfpathlineto{\pgfqpoint{4.493492in}{0.413320in}}%
\pgfpathlineto{\pgfqpoint{4.490822in}{0.413320in}}%
\pgfpathlineto{\pgfqpoint{4.488130in}{0.413320in}}%
\pgfpathlineto{\pgfqpoint{4.485581in}{0.413320in}}%
\pgfpathlineto{\pgfqpoint{4.482778in}{0.413320in}}%
\pgfpathlineto{\pgfqpoint{4.480201in}{0.413320in}}%
\pgfpathlineto{\pgfqpoint{4.477430in}{0.413320in}}%
\pgfpathlineto{\pgfqpoint{4.474861in}{0.413320in}}%
\pgfpathlineto{\pgfqpoint{4.472059in}{0.413320in}}%
\pgfpathlineto{\pgfqpoint{4.469492in}{0.413320in}}%
\pgfpathlineto{\pgfqpoint{4.466717in}{0.413320in}}%
\pgfpathlineto{\pgfqpoint{4.464029in}{0.413320in}}%
\pgfpathlineto{\pgfqpoint{4.461367in}{0.413320in}}%
\pgfpathlineto{\pgfqpoint{4.458681in}{0.413320in}}%
\pgfpathlineto{\pgfqpoint{4.456138in}{0.413320in}}%
\pgfpathlineto{\pgfqpoint{4.453312in}{0.413320in}}%
\pgfpathlineto{\pgfqpoint{4.450767in}{0.413320in}}%
\pgfpathlineto{\pgfqpoint{4.447965in}{0.413320in}}%
\pgfpathlineto{\pgfqpoint{4.445423in}{0.413320in}}%
\pgfpathlineto{\pgfqpoint{4.442611in}{0.413320in}}%
\pgfpathlineto{\pgfqpoint{4.440041in}{0.413320in}}%
\pgfpathlineto{\pgfqpoint{4.437253in}{0.413320in}}%
\pgfpathlineto{\pgfqpoint{4.434569in}{0.413320in}}%
\pgfpathlineto{\pgfqpoint{4.431901in}{0.413320in}}%
\pgfpathlineto{\pgfqpoint{4.429220in}{0.413320in}}%
\pgfpathlineto{\pgfqpoint{4.426534in}{0.413320in}}%
\pgfpathlineto{\pgfqpoint{4.423863in}{0.413320in}}%
\pgfpathlineto{\pgfqpoint{4.421292in}{0.413320in}}%
\pgfpathlineto{\pgfqpoint{4.418506in}{0.413320in}}%
\pgfpathlineto{\pgfqpoint{4.415932in}{0.413320in}}%
\pgfpathlineto{\pgfqpoint{4.413149in}{0.413320in}}%
\pgfpathlineto{\pgfqpoint{4.410587in}{0.413320in}}%
\pgfpathlineto{\pgfqpoint{4.407788in}{0.413320in}}%
\pgfpathlineto{\pgfqpoint{4.405234in}{0.413320in}}%
\pgfpathlineto{\pgfqpoint{4.402468in}{0.413320in}}%
\pgfpathlineto{\pgfqpoint{4.399745in}{0.413320in}}%
\pgfpathlineto{\pgfqpoint{4.397076in}{0.413320in}}%
\pgfpathlineto{\pgfqpoint{4.394400in}{0.413320in}}%
\pgfpathlineto{\pgfqpoint{4.391721in}{0.413320in}}%
\pgfpathlineto{\pgfqpoint{4.389044in}{0.413320in}}%
\pgfpathlineto{\pgfqpoint{4.386431in}{0.413320in}}%
\pgfpathlineto{\pgfqpoint{4.383674in}{0.413320in}}%
\pgfpathlineto{\pgfqpoint{4.381097in}{0.413320in}}%
\pgfpathlineto{\pgfqpoint{4.378329in}{0.413320in}}%
\pgfpathlineto{\pgfqpoint{4.375761in}{0.413320in}}%
\pgfpathlineto{\pgfqpoint{4.372976in}{0.413320in}}%
\pgfpathlineto{\pgfqpoint{4.370437in}{0.413320in}}%
\pgfpathlineto{\pgfqpoint{4.367646in}{0.413320in}}%
\pgfpathlineto{\pgfqpoint{4.364936in}{0.413320in}}%
\pgfpathlineto{\pgfqpoint{4.362270in}{0.413320in}}%
\pgfpathlineto{\pgfqpoint{4.359582in}{0.413320in}}%
\pgfpathlineto{\pgfqpoint{4.357014in}{0.413320in}}%
\pgfpathlineto{\pgfqpoint{4.354224in}{0.413320in}}%
\pgfpathlineto{\pgfqpoint{4.351645in}{0.413320in}}%
\pgfpathlineto{\pgfqpoint{4.348868in}{0.413320in}}%
\pgfpathlineto{\pgfqpoint{4.346263in}{0.413320in}}%
\pgfpathlineto{\pgfqpoint{4.343510in}{0.413320in}}%
\pgfpathlineto{\pgfqpoint{4.340976in}{0.413320in}}%
\pgfpathlineto{\pgfqpoint{4.338154in}{0.413320in}}%
\pgfpathlineto{\pgfqpoint{4.335463in}{0.413320in}}%
\pgfpathlineto{\pgfqpoint{4.332796in}{0.413320in}}%
\pgfpathlineto{\pgfqpoint{4.330118in}{0.413320in}}%
\pgfpathlineto{\pgfqpoint{4.327440in}{0.413320in}}%
\pgfpathlineto{\pgfqpoint{4.324760in}{0.413320in}}%
\pgfpathlineto{\pgfqpoint{4.322181in}{0.413320in}}%
\pgfpathlineto{\pgfqpoint{4.319405in}{0.413320in}}%
\pgfpathlineto{\pgfqpoint{4.316856in}{0.413320in}}%
\pgfpathlineto{\pgfqpoint{4.314032in}{0.413320in}}%
\pgfpathlineto{\pgfqpoint{4.311494in}{0.413320in}}%
\pgfpathlineto{\pgfqpoint{4.308691in}{0.413320in}}%
\pgfpathlineto{\pgfqpoint{4.306118in}{0.413320in}}%
\pgfpathlineto{\pgfqpoint{4.303357in}{0.413320in}}%
\pgfpathlineto{\pgfqpoint{4.300656in}{0.413320in}}%
\pgfpathlineto{\pgfqpoint{4.297977in}{0.413320in}}%
\pgfpathlineto{\pgfqpoint{4.295299in}{0.413320in}}%
\pgfpathlineto{\pgfqpoint{4.292786in}{0.413320in}}%
\pgfpathlineto{\pgfqpoint{4.289936in}{0.413320in}}%
\pgfpathlineto{\pgfqpoint{4.287399in}{0.413320in}}%
\pgfpathlineto{\pgfqpoint{4.284586in}{0.413320in}}%
\pgfpathlineto{\pgfqpoint{4.282000in}{0.413320in}}%
\pgfpathlineto{\pgfqpoint{4.279212in}{0.413320in}}%
\pgfpathlineto{\pgfqpoint{4.276635in}{0.413320in}}%
\pgfpathlineto{\pgfqpoint{4.273874in}{0.413320in}}%
\pgfpathlineto{\pgfqpoint{4.271187in}{0.413320in}}%
\pgfpathlineto{\pgfqpoint{4.268590in}{0.413320in}}%
\pgfpathlineto{\pgfqpoint{4.265824in}{0.413320in}}%
\pgfpathlineto{\pgfqpoint{4.263157in}{0.413320in}}%
\pgfpathlineto{\pgfqpoint{4.260477in}{0.413320in}}%
\pgfpathlineto{\pgfqpoint{4.257958in}{0.413320in}}%
\pgfpathlineto{\pgfqpoint{4.255120in}{0.413320in}}%
\pgfpathlineto{\pgfqpoint{4.252581in}{0.413320in}}%
\pgfpathlineto{\pgfqpoint{4.249767in}{0.413320in}}%
\pgfpathlineto{\pgfqpoint{4.247225in}{0.413320in}}%
\pgfpathlineto{\pgfqpoint{4.244394in}{0.413320in}}%
\pgfpathlineto{\pgfqpoint{4.241900in}{0.413320in}}%
\pgfpathlineto{\pgfqpoint{4.239084in}{0.413320in}}%
\pgfpathlineto{\pgfqpoint{4.236375in}{0.413320in}}%
\pgfpathlineto{\pgfqpoint{4.233691in}{0.413320in}}%
\pgfpathlineto{\pgfqpoint{4.231013in}{0.413320in}}%
\pgfpathlineto{\pgfqpoint{4.228331in}{0.413320in}}%
\pgfpathlineto{\pgfqpoint{4.225654in}{0.413320in}}%
\pgfpathlineto{\pgfqpoint{4.223082in}{0.413320in}}%
\pgfpathlineto{\pgfqpoint{4.220304in}{0.413320in}}%
\pgfpathlineto{\pgfqpoint{4.217694in}{0.413320in}}%
\pgfpathlineto{\pgfqpoint{4.214948in}{0.413320in}}%
\pgfpathlineto{\pgfqpoint{4.212383in}{0.413320in}}%
\pgfpathlineto{\pgfqpoint{4.209597in}{0.413320in}}%
\pgfpathlineto{\pgfqpoint{4.207076in}{0.413320in}}%
\pgfpathlineto{\pgfqpoint{4.204240in}{0.413320in}}%
\pgfpathlineto{\pgfqpoint{4.201542in}{0.413320in}}%
\pgfpathlineto{\pgfqpoint{4.198878in}{0.413320in}}%
\pgfpathlineto{\pgfqpoint{4.196186in}{0.413320in}}%
\pgfpathlineto{\pgfqpoint{4.193638in}{0.413320in}}%
\pgfpathlineto{\pgfqpoint{4.190842in}{0.413320in}}%
\pgfpathlineto{\pgfqpoint{4.188318in}{0.413320in}}%
\pgfpathlineto{\pgfqpoint{4.185481in}{0.413320in}}%
\pgfpathlineto{\pgfqpoint{4.182899in}{0.413320in}}%
\pgfpathlineto{\pgfqpoint{4.180129in}{0.413320in}}%
\pgfpathlineto{\pgfqpoint{4.177593in}{0.413320in}}%
\pgfpathlineto{\pgfqpoint{4.174770in}{0.413320in}}%
\pgfpathlineto{\pgfqpoint{4.172093in}{0.413320in}}%
\pgfpathlineto{\pgfqpoint{4.169415in}{0.413320in}}%
\pgfpathlineto{\pgfqpoint{4.166737in}{0.413320in}}%
\pgfpathlineto{\pgfqpoint{4.164059in}{0.413320in}}%
\pgfpathlineto{\pgfqpoint{4.161380in}{0.413320in}}%
\pgfpathlineto{\pgfqpoint{4.158806in}{0.413320in}}%
\pgfpathlineto{\pgfqpoint{4.156016in}{0.413320in}}%
\pgfpathlineto{\pgfqpoint{4.153423in}{0.413320in}}%
\pgfpathlineto{\pgfqpoint{4.150665in}{0.413320in}}%
\pgfpathlineto{\pgfqpoint{4.148082in}{0.413320in}}%
\pgfpathlineto{\pgfqpoint{4.145310in}{0.413320in}}%
\pgfpathlineto{\pgfqpoint{4.142713in}{0.413320in}}%
\pgfpathlineto{\pgfqpoint{4.139963in}{0.413320in}}%
\pgfpathlineto{\pgfqpoint{4.137272in}{0.413320in}}%
\pgfpathlineto{\pgfqpoint{4.134615in}{0.413320in}}%
\pgfpathlineto{\pgfqpoint{4.131920in}{0.413320in}}%
\pgfpathlineto{\pgfqpoint{4.129349in}{0.413320in}}%
\pgfpathlineto{\pgfqpoint{4.126553in}{0.413320in}}%
\pgfpathlineto{\pgfqpoint{4.124019in}{0.413320in}}%
\pgfpathlineto{\pgfqpoint{4.121205in}{0.413320in}}%
\pgfpathlineto{\pgfqpoint{4.118554in}{0.413320in}}%
\pgfpathlineto{\pgfqpoint{4.115844in}{0.413320in}}%
\pgfpathlineto{\pgfqpoint{4.113252in}{0.413320in}}%
\pgfpathlineto{\pgfqpoint{4.110488in}{0.413320in}}%
\pgfpathlineto{\pgfqpoint{4.107814in}{0.413320in}}%
\pgfpathlineto{\pgfqpoint{4.105185in}{0.413320in}}%
\pgfpathlineto{\pgfqpoint{4.102456in}{0.413320in}}%
\pgfpathlineto{\pgfqpoint{4.099777in}{0.413320in}}%
\pgfpathlineto{\pgfqpoint{4.097092in}{0.413320in}}%
\pgfpathlineto{\pgfqpoint{4.094527in}{0.413320in}}%
\pgfpathlineto{\pgfqpoint{4.091729in}{0.413320in}}%
\pgfpathlineto{\pgfqpoint{4.089159in}{0.413320in}}%
\pgfpathlineto{\pgfqpoint{4.086385in}{0.413320in}}%
\pgfpathlineto{\pgfqpoint{4.083870in}{0.413320in}}%
\pgfpathlineto{\pgfqpoint{4.081018in}{0.413320in}}%
\pgfpathlineto{\pgfqpoint{4.078471in}{0.413320in}}%
\pgfpathlineto{\pgfqpoint{4.075705in}{0.413320in}}%
\pgfpathlineto{\pgfqpoint{4.072985in}{0.413320in}}%
\pgfpathlineto{\pgfqpoint{4.070313in}{0.413320in}}%
\pgfpathlineto{\pgfqpoint{4.067636in}{0.413320in}}%
\pgfpathlineto{\pgfqpoint{4.064957in}{0.413320in}}%
\pgfpathlineto{\pgfqpoint{4.062266in}{0.413320in}}%
\pgfpathlineto{\pgfqpoint{4.059702in}{0.413320in}}%
\pgfpathlineto{\pgfqpoint{4.056911in}{0.413320in}}%
\pgfpathlineto{\pgfqpoint{4.054326in}{0.413320in}}%
\pgfpathlineto{\pgfqpoint{4.051557in}{0.413320in}}%
\pgfpathlineto{\pgfqpoint{4.049006in}{0.413320in}}%
\pgfpathlineto{\pgfqpoint{4.046210in}{0.413320in}}%
\pgfpathlineto{\pgfqpoint{4.043667in}{0.413320in}}%
\pgfpathlineto{\pgfqpoint{4.040852in}{0.413320in}}%
\pgfpathlineto{\pgfqpoint{4.038174in}{0.413320in}}%
\pgfpathlineto{\pgfqpoint{4.035492in}{0.413320in}}%
\pgfpathlineto{\pgfqpoint{4.032817in}{0.413320in}}%
\pgfpathlineto{\pgfqpoint{4.030229in}{0.413320in}}%
\pgfpathlineto{\pgfqpoint{4.027447in}{0.413320in}}%
\pgfpathlineto{\pgfqpoint{4.024868in}{0.413320in}}%
\pgfpathlineto{\pgfqpoint{4.022097in}{0.413320in}}%
\pgfpathlineto{\pgfqpoint{4.019518in}{0.413320in}}%
\pgfpathlineto{\pgfqpoint{4.016744in}{0.413320in}}%
\pgfpathlineto{\pgfqpoint{4.014186in}{0.413320in}}%
\pgfpathlineto{\pgfqpoint{4.011394in}{0.413320in}}%
\pgfpathlineto{\pgfqpoint{4.008699in}{0.413320in}}%
\pgfpathlineto{\pgfqpoint{4.006034in}{0.413320in}}%
\pgfpathlineto{\pgfqpoint{4.003348in}{0.413320in}}%
\pgfpathlineto{\pgfqpoint{4.000674in}{0.413320in}}%
\pgfpathlineto{\pgfqpoint{3.997990in}{0.413320in}}%
\pgfpathlineto{\pgfqpoint{3.995417in}{0.413320in}}%
\pgfpathlineto{\pgfqpoint{3.992642in}{0.413320in}}%
\pgfpathlineto{\pgfqpoint{3.990055in}{0.413320in}}%
\pgfpathlineto{\pgfqpoint{3.987270in}{0.413320in}}%
\pgfpathlineto{\pgfqpoint{3.984714in}{0.413320in}}%
\pgfpathlineto{\pgfqpoint{3.981929in}{0.413320in}}%
\pgfpathlineto{\pgfqpoint{3.979389in}{0.413320in}}%
\pgfpathlineto{\pgfqpoint{3.976563in}{0.413320in}}%
\pgfpathlineto{\pgfqpoint{3.973885in}{0.413320in}}%
\pgfpathlineto{\pgfqpoint{3.971250in}{0.413320in}}%
\pgfpathlineto{\pgfqpoint{3.968523in}{0.413320in}}%
\pgfpathlineto{\pgfqpoint{3.966013in}{0.413320in}}%
\pgfpathlineto{\pgfqpoint{3.963176in}{0.413320in}}%
\pgfpathlineto{\pgfqpoint{3.960635in}{0.413320in}}%
\pgfpathlineto{\pgfqpoint{3.957823in}{0.413320in}}%
\pgfpathlineto{\pgfqpoint{3.955211in}{0.413320in}}%
\pgfpathlineto{\pgfqpoint{3.952464in}{0.413320in}}%
\pgfpathlineto{\pgfqpoint{3.949894in}{0.413320in}}%
\pgfpathlineto{\pgfqpoint{3.947101in}{0.413320in}}%
\pgfpathlineto{\pgfqpoint{3.944431in}{0.413320in}}%
\pgfpathlineto{\pgfqpoint{3.941778in}{0.413320in}}%
\pgfpathlineto{\pgfqpoint{3.939075in}{0.413320in}}%
\pgfpathlineto{\pgfqpoint{3.936395in}{0.413320in}}%
\pgfpathlineto{\pgfqpoint{3.933714in}{0.413320in}}%
\pgfpathlineto{\pgfqpoint{3.931202in}{0.413320in}}%
\pgfpathlineto{\pgfqpoint{3.928347in}{0.413320in}}%
\pgfpathlineto{\pgfqpoint{3.925778in}{0.413320in}}%
\pgfpathlineto{\pgfqpoint{3.923005in}{0.413320in}}%
\pgfpathlineto{\pgfqpoint{3.920412in}{0.413320in}}%
\pgfpathlineto{\pgfqpoint{3.917646in}{0.413320in}}%
\pgfpathlineto{\pgfqpoint{3.915107in}{0.413320in}}%
\pgfpathlineto{\pgfqpoint{3.912296in}{0.413320in}}%
\pgfpathlineto{\pgfqpoint{3.909602in}{0.413320in}}%
\pgfpathlineto{\pgfqpoint{3.906918in}{0.413320in}}%
\pgfpathlineto{\pgfqpoint{3.904252in}{0.413320in}}%
\pgfpathlineto{\pgfqpoint{3.901573in}{0.413320in}}%
\pgfpathlineto{\pgfqpoint{3.898891in}{0.413320in}}%
\pgfpathlineto{\pgfqpoint{3.896345in}{0.413320in}}%
\pgfpathlineto{\pgfqpoint{3.893541in}{0.413320in}}%
\pgfpathlineto{\pgfqpoint{3.890926in}{0.413320in}}%
\pgfpathlineto{\pgfqpoint{3.888188in}{0.413320in}}%
\pgfpathlineto{\pgfqpoint{3.885621in}{0.413320in}}%
\pgfpathlineto{\pgfqpoint{3.882850in}{0.413320in}}%
\pgfpathlineto{\pgfqpoint{3.880237in}{0.413320in}}%
\pgfpathlineto{\pgfqpoint{3.877466in}{0.413320in}}%
\pgfpathlineto{\pgfqpoint{3.874790in}{0.413320in}}%
\pgfpathlineto{\pgfqpoint{3.872114in}{0.413320in}}%
\pgfpathlineto{\pgfqpoint{3.869435in}{0.413320in}}%
\pgfpathlineto{\pgfqpoint{3.866815in}{0.413320in}}%
\pgfpathlineto{\pgfqpoint{3.864073in}{0.413320in}}%
\pgfpathlineto{\pgfqpoint{3.861561in}{0.413320in}}%
\pgfpathlineto{\pgfqpoint{3.858720in}{0.413320in}}%
\pgfpathlineto{\pgfqpoint{3.856100in}{0.413320in}}%
\pgfpathlineto{\pgfqpoint{3.853358in}{0.413320in}}%
\pgfpathlineto{\pgfqpoint{3.850814in}{0.413320in}}%
\pgfpathlineto{\pgfqpoint{3.848005in}{0.413320in}}%
\pgfpathlineto{\pgfqpoint{3.845329in}{0.413320in}}%
\pgfpathlineto{\pgfqpoint{3.842641in}{0.413320in}}%
\pgfpathlineto{\pgfqpoint{3.839960in}{0.413320in}}%
\pgfpathlineto{\pgfqpoint{3.837286in}{0.413320in}}%
\pgfpathlineto{\pgfqpoint{3.834616in}{0.413320in}}%
\pgfpathlineto{\pgfqpoint{3.832053in}{0.413320in}}%
\pgfpathlineto{\pgfqpoint{3.829252in}{0.413320in}}%
\pgfpathlineto{\pgfqpoint{3.826679in}{0.413320in}}%
\pgfpathlineto{\pgfqpoint{3.823903in}{0.413320in}}%
\pgfpathlineto{\pgfqpoint{3.821315in}{0.413320in}}%
\pgfpathlineto{\pgfqpoint{3.818546in}{0.413320in}}%
\pgfpathlineto{\pgfqpoint{3.815983in}{0.413320in}}%
\pgfpathlineto{\pgfqpoint{3.813172in}{0.413320in}}%
\pgfpathlineto{\pgfqpoint{3.810510in}{0.413320in}}%
\pgfpathlineto{\pgfqpoint{3.807832in}{0.413320in}}%
\pgfpathlineto{\pgfqpoint{3.805145in}{0.413320in}}%
\pgfpathlineto{\pgfqpoint{3.802569in}{0.413320in}}%
\pgfpathlineto{\pgfqpoint{3.799797in}{0.413320in}}%
\pgfpathlineto{\pgfqpoint{3.797265in}{0.413320in}}%
\pgfpathlineto{\pgfqpoint{3.794435in}{0.413320in}}%
\pgfpathlineto{\pgfqpoint{3.791897in}{0.413320in}}%
\pgfpathlineto{\pgfqpoint{3.789084in}{0.413320in}}%
\pgfpathlineto{\pgfqpoint{3.786504in}{0.413320in}}%
\pgfpathlineto{\pgfqpoint{3.783725in}{0.413320in}}%
\pgfpathlineto{\pgfqpoint{3.781046in}{0.413320in}}%
\pgfpathlineto{\pgfqpoint{3.778370in}{0.413320in}}%
\pgfpathlineto{\pgfqpoint{3.775691in}{0.413320in}}%
\pgfpathlineto{\pgfqpoint{3.773014in}{0.413320in}}%
\pgfpathlineto{\pgfqpoint{3.770323in}{0.413320in}}%
\pgfpathlineto{\pgfqpoint{3.767782in}{0.413320in}}%
\pgfpathlineto{\pgfqpoint{3.764966in}{0.413320in}}%
\pgfpathlineto{\pgfqpoint{3.762389in}{0.413320in}}%
\pgfpathlineto{\pgfqpoint{3.759622in}{0.413320in}}%
\pgfpathlineto{\pgfqpoint{3.757065in}{0.413320in}}%
\pgfpathlineto{\pgfqpoint{3.754265in}{0.413320in}}%
\pgfpathlineto{\pgfqpoint{3.751728in}{0.413320in}}%
\pgfpathlineto{\pgfqpoint{3.748903in}{0.413320in}}%
\pgfpathlineto{\pgfqpoint{3.746229in}{0.413320in}}%
\pgfpathlineto{\pgfqpoint{3.743548in}{0.413320in}}%
\pgfpathlineto{\pgfqpoint{3.740874in}{0.413320in}}%
\pgfpathlineto{\pgfqpoint{3.738194in}{0.413320in}}%
\pgfpathlineto{\pgfqpoint{3.735509in}{0.413320in}}%
\pgfpathlineto{\pgfqpoint{3.732950in}{0.413320in}}%
\pgfpathlineto{\pgfqpoint{3.730158in}{0.413320in}}%
\pgfpathlineto{\pgfqpoint{3.727581in}{0.413320in}}%
\pgfpathlineto{\pgfqpoint{3.724804in}{0.413320in}}%
\pgfpathlineto{\pgfqpoint{3.722228in}{0.413320in}}%
\pgfpathlineto{\pgfqpoint{3.719446in}{0.413320in}}%
\pgfpathlineto{\pgfqpoint{3.716875in}{0.413320in}}%
\pgfpathlineto{\pgfqpoint{3.714086in}{0.413320in}}%
\pgfpathlineto{\pgfqpoint{3.711410in}{0.413320in}}%
\pgfpathlineto{\pgfqpoint{3.708729in}{0.413320in}}%
\pgfpathlineto{\pgfqpoint{3.706053in}{0.413320in}}%
\pgfpathlineto{\pgfqpoint{3.703460in}{0.413320in}}%
\pgfpathlineto{\pgfqpoint{3.700684in}{0.413320in}}%
\pgfpathlineto{\pgfqpoint{3.698125in}{0.413320in}}%
\pgfpathlineto{\pgfqpoint{3.695331in}{0.413320in}}%
\pgfpathlineto{\pgfqpoint{3.692765in}{0.413320in}}%
\pgfpathlineto{\pgfqpoint{3.689983in}{0.413320in}}%
\pgfpathlineto{\pgfqpoint{3.687442in}{0.413320in}}%
\pgfpathlineto{\pgfqpoint{3.684620in}{0.413320in}}%
\pgfpathlineto{\pgfqpoint{3.681948in}{0.413320in}}%
\pgfpathlineto{\pgfqpoint{3.679273in}{0.413320in}}%
\pgfpathlineto{\pgfqpoint{3.676591in}{0.413320in}}%
\pgfpathlineto{\pgfqpoint{3.673911in}{0.413320in}}%
\pgfpathlineto{\pgfqpoint{3.671232in}{0.413320in}}%
\pgfpathlineto{\pgfqpoint{3.668665in}{0.413320in}}%
\pgfpathlineto{\pgfqpoint{3.665864in}{0.413320in}}%
\pgfpathlineto{\pgfqpoint{3.663276in}{0.413320in}}%
\pgfpathlineto{\pgfqpoint{3.660515in}{0.413320in}}%
\pgfpathlineto{\pgfqpoint{3.657917in}{0.413320in}}%
\pgfpathlineto{\pgfqpoint{3.655165in}{0.413320in}}%
\pgfpathlineto{\pgfqpoint{3.652628in}{0.413320in}}%
\pgfpathlineto{\pgfqpoint{3.649837in}{0.413320in}}%
\pgfpathlineto{\pgfqpoint{3.647130in}{0.413320in}}%
\pgfpathlineto{\pgfqpoint{3.644452in}{0.413320in}}%
\pgfpathlineto{\pgfqpoint{3.641773in}{0.413320in}}%
\pgfpathlineto{\pgfqpoint{3.639207in}{0.413320in}}%
\pgfpathlineto{\pgfqpoint{3.636413in}{0.413320in}}%
\pgfpathlineto{\pgfqpoint{3.633858in}{0.413320in}}%
\pgfpathlineto{\pgfqpoint{3.631058in}{0.413320in}}%
\pgfpathlineto{\pgfqpoint{3.628460in}{0.413320in}}%
\pgfpathlineto{\pgfqpoint{3.625689in}{0.413320in}}%
\pgfpathlineto{\pgfqpoint{3.623165in}{0.413320in}}%
\pgfpathlineto{\pgfqpoint{3.620345in}{0.413320in}}%
\pgfpathlineto{\pgfqpoint{3.617667in}{0.413320in}}%
\pgfpathlineto{\pgfqpoint{3.614982in}{0.413320in}}%
\pgfpathlineto{\pgfqpoint{3.612311in}{0.413320in}}%
\pgfpathlineto{\pgfqpoint{3.609632in}{0.413320in}}%
\pgfpathlineto{\pgfqpoint{3.606951in}{0.413320in}}%
\pgfpathlineto{\pgfqpoint{3.604387in}{0.413320in}}%
\pgfpathlineto{\pgfqpoint{3.601590in}{0.413320in}}%
\pgfpathlineto{\pgfqpoint{3.598998in}{0.413320in}}%
\pgfpathlineto{\pgfqpoint{3.596240in}{0.413320in}}%
\pgfpathlineto{\pgfqpoint{3.593620in}{0.413320in}}%
\pgfpathlineto{\pgfqpoint{3.590883in}{0.413320in}}%
\pgfpathlineto{\pgfqpoint{3.588258in}{0.413320in}}%
\pgfpathlineto{\pgfqpoint{3.585532in}{0.413320in}}%
\pgfpathlineto{\pgfqpoint{3.582851in}{0.413320in}}%
\pgfpathlineto{\pgfqpoint{3.580191in}{0.413320in}}%
\pgfpathlineto{\pgfqpoint{3.577487in}{0.413320in}}%
\pgfpathlineto{\pgfqpoint{3.574814in}{0.413320in}}%
\pgfpathlineto{\pgfqpoint{3.572126in}{0.413320in}}%
\pgfpathlineto{\pgfqpoint{3.569584in}{0.413320in}}%
\pgfpathlineto{\pgfqpoint{3.566774in}{0.413320in}}%
\pgfpathlineto{\pgfqpoint{3.564188in}{0.413320in}}%
\pgfpathlineto{\pgfqpoint{3.561420in}{0.413320in}}%
\pgfpathlineto{\pgfqpoint{3.558853in}{0.413320in}}%
\pgfpathlineto{\pgfqpoint{3.556061in}{0.413320in}}%
\pgfpathlineto{\pgfqpoint{3.553498in}{0.413320in}}%
\pgfpathlineto{\pgfqpoint{3.550713in}{0.413320in}}%
\pgfpathlineto{\pgfqpoint{3.548029in}{0.413320in}}%
\pgfpathlineto{\pgfqpoint{3.545349in}{0.413320in}}%
\pgfpathlineto{\pgfqpoint{3.542656in}{0.413320in}}%
\pgfpathlineto{\pgfqpoint{3.540093in}{0.413320in}}%
\pgfpathlineto{\pgfqpoint{3.537309in}{0.413320in}}%
\pgfpathlineto{\pgfqpoint{3.534783in}{0.413320in}}%
\pgfpathlineto{\pgfqpoint{3.531955in}{0.413320in}}%
\pgfpathlineto{\pgfqpoint{3.529327in}{0.413320in}}%
\pgfpathlineto{\pgfqpoint{3.526601in}{0.413320in}}%
\pgfpathlineto{\pgfqpoint{3.524041in}{0.413320in}}%
\pgfpathlineto{\pgfqpoint{3.521244in}{0.413320in}}%
\pgfpathlineto{\pgfqpoint{3.518565in}{0.413320in}}%
\pgfpathlineto{\pgfqpoint{3.515884in}{0.413320in}}%
\pgfpathlineto{\pgfqpoint{3.513209in}{0.413320in}}%
\pgfpathlineto{\pgfqpoint{3.510533in}{0.413320in}}%
\pgfpathlineto{\pgfqpoint{3.507840in}{0.413320in}}%
\pgfpathlineto{\pgfqpoint{3.505262in}{0.413320in}}%
\pgfpathlineto{\pgfqpoint{3.502488in}{0.413320in}}%
\pgfpathlineto{\pgfqpoint{3.499909in}{0.413320in}}%
\pgfpathlineto{\pgfqpoint{3.497139in}{0.413320in}}%
\pgfpathlineto{\pgfqpoint{3.494581in}{0.413320in}}%
\pgfpathlineto{\pgfqpoint{3.491783in}{0.413320in}}%
\pgfpathlineto{\pgfqpoint{3.489223in}{0.413320in}}%
\pgfpathlineto{\pgfqpoint{3.486442in}{0.413320in}}%
\pgfpathlineto{\pgfqpoint{3.483744in}{0.413320in}}%
\pgfpathlineto{\pgfqpoint{3.481072in}{0.413320in}}%
\pgfpathlineto{\pgfqpoint{3.478378in}{0.413320in}}%
\pgfpathlineto{\pgfqpoint{3.475821in}{0.413320in}}%
\pgfpathlineto{\pgfqpoint{3.473021in}{0.413320in}}%
\pgfpathlineto{\pgfqpoint{3.470466in}{0.413320in}}%
\pgfpathlineto{\pgfqpoint{3.467678in}{0.413320in}}%
\pgfpathlineto{\pgfqpoint{3.465072in}{0.413320in}}%
\pgfpathlineto{\pgfqpoint{3.462321in}{0.413320in}}%
\pgfpathlineto{\pgfqpoint{3.459695in}{0.413320in}}%
\pgfpathlineto{\pgfqpoint{3.456960in}{0.413320in}}%
\pgfpathlineto{\pgfqpoint{3.454285in}{0.413320in}}%
\pgfpathlineto{\pgfqpoint{3.451597in}{0.413320in}}%
\pgfpathlineto{\pgfqpoint{3.448926in}{0.413320in}}%
\pgfpathlineto{\pgfqpoint{3.446257in}{0.413320in}}%
\pgfpathlineto{\pgfqpoint{3.443574in}{0.413320in}}%
\pgfpathlineto{\pgfqpoint{3.440996in}{0.413320in}}%
\pgfpathlineto{\pgfqpoint{3.438210in}{0.413320in}}%
\pgfpathlineto{\pgfqpoint{3.435635in}{0.413320in}}%
\pgfpathlineto{\pgfqpoint{3.432851in}{0.413320in}}%
\pgfpathlineto{\pgfqpoint{3.430313in}{0.413320in}}%
\pgfpathlineto{\pgfqpoint{3.427501in}{0.413320in}}%
\pgfpathlineto{\pgfqpoint{3.424887in}{0.413320in}}%
\pgfpathlineto{\pgfqpoint{3.422142in}{0.413320in}}%
\pgfpathlineto{\pgfqpoint{3.419455in}{0.413320in}}%
\pgfpathlineto{\pgfqpoint{3.416780in}{0.413320in}}%
\pgfpathlineto{\pgfqpoint{3.414109in}{0.413320in}}%
\pgfpathlineto{\pgfqpoint{3.411431in}{0.413320in}}%
\pgfpathlineto{\pgfqpoint{3.408752in}{0.413320in}}%
\pgfpathlineto{\pgfqpoint{3.406202in}{0.413320in}}%
\pgfpathlineto{\pgfqpoint{3.403394in}{0.413320in}}%
\pgfpathlineto{\pgfqpoint{3.400783in}{0.413320in}}%
\pgfpathlineto{\pgfqpoint{3.398037in}{0.413320in}}%
\pgfpathlineto{\pgfqpoint{3.395461in}{0.413320in}}%
\pgfpathlineto{\pgfqpoint{3.392681in}{0.413320in}}%
\pgfpathlineto{\pgfqpoint{3.390102in}{0.413320in}}%
\pgfpathlineto{\pgfqpoint{3.387309in}{0.413320in}}%
\pgfpathlineto{\pgfqpoint{3.384647in}{0.413320in}}%
\pgfpathlineto{\pgfqpoint{3.381959in}{0.413320in}}%
\pgfpathlineto{\pgfqpoint{3.379290in}{0.413320in}}%
\pgfpathlineto{\pgfqpoint{3.376735in}{0.413320in}}%
\pgfpathlineto{\pgfqpoint{3.373921in}{0.413320in}}%
\pgfpathlineto{\pgfqpoint{3.371357in}{0.413320in}}%
\pgfpathlineto{\pgfqpoint{3.368577in}{0.413320in}}%
\pgfpathlineto{\pgfqpoint{3.365996in}{0.413320in}}%
\pgfpathlineto{\pgfqpoint{3.363221in}{0.413320in}}%
\pgfpathlineto{\pgfqpoint{3.360620in}{0.413320in}}%
\pgfpathlineto{\pgfqpoint{3.357862in}{0.413320in}}%
\pgfpathlineto{\pgfqpoint{3.355177in}{0.413320in}}%
\pgfpathlineto{\pgfqpoint{3.352505in}{0.413320in}}%
\pgfpathlineto{\pgfqpoint{3.349828in}{0.413320in}}%
\pgfpathlineto{\pgfqpoint{3.347139in}{0.413320in}}%
\pgfpathlineto{\pgfqpoint{3.344468in}{0.413320in}}%
\pgfpathlineto{\pgfqpoint{3.341893in}{0.413320in}}%
\pgfpathlineto{\pgfqpoint{3.339101in}{0.413320in}}%
\pgfpathlineto{\pgfqpoint{3.336541in}{0.413320in}}%
\pgfpathlineto{\pgfqpoint{3.333758in}{0.413320in}}%
\pgfpathlineto{\pgfqpoint{3.331183in}{0.413320in}}%
\pgfpathlineto{\pgfqpoint{3.328401in}{0.413320in}}%
\pgfpathlineto{\pgfqpoint{3.325860in}{0.413320in}}%
\pgfpathlineto{\pgfqpoint{3.323049in}{0.413320in}}%
\pgfpathlineto{\pgfqpoint{3.320366in}{0.413320in}}%
\pgfpathlineto{\pgfqpoint{3.317688in}{0.413320in}}%
\pgfpathlineto{\pgfqpoint{3.315008in}{0.413320in}}%
\pgfpathlineto{\pgfqpoint{3.312480in}{0.413320in}}%
\pgfpathlineto{\pgfqpoint{3.309652in}{0.413320in}}%
\pgfpathlineto{\pgfqpoint{3.307104in}{0.413320in}}%
\pgfpathlineto{\pgfqpoint{3.304295in}{0.413320in}}%
\pgfpathlineto{\pgfqpoint{3.301719in}{0.413320in}}%
\pgfpathlineto{\pgfqpoint{3.298937in}{0.413320in}}%
\pgfpathlineto{\pgfqpoint{3.296376in}{0.413320in}}%
\pgfpathlineto{\pgfqpoint{3.293574in}{0.413320in}}%
\pgfpathlineto{\pgfqpoint{3.290890in}{0.413320in}}%
\pgfpathlineto{\pgfqpoint{3.288225in}{0.413320in}}%
\pgfpathlineto{\pgfqpoint{3.285534in}{0.413320in}}%
\pgfpathlineto{\pgfqpoint{3.282870in}{0.413320in}}%
\pgfpathlineto{\pgfqpoint{3.280189in}{0.413320in}}%
\pgfpathlineto{\pgfqpoint{3.277603in}{0.413320in}}%
\pgfpathlineto{\pgfqpoint{3.274831in}{0.413320in}}%
\pgfpathlineto{\pgfqpoint{3.272254in}{0.413320in}}%
\pgfpathlineto{\pgfqpoint{3.269478in}{0.413320in}}%
\pgfpathlineto{\pgfqpoint{3.266849in}{0.413320in}}%
\pgfpathlineto{\pgfqpoint{3.264119in}{0.413320in}}%
\pgfpathlineto{\pgfqpoint{3.261594in}{0.413320in}}%
\pgfpathlineto{\pgfqpoint{3.258784in}{0.413320in}}%
\pgfpathlineto{\pgfqpoint{3.256083in}{0.413320in}}%
\pgfpathlineto{\pgfqpoint{3.253404in}{0.413320in}}%
\pgfpathlineto{\pgfqpoint{3.250716in}{0.413320in}}%
\pgfpathlineto{\pgfqpoint{3.248049in}{0.413320in}}%
\pgfpathlineto{\pgfqpoint{3.245363in}{0.413320in}}%
\pgfpathlineto{\pgfqpoint{3.242807in}{0.413320in}}%
\pgfpathlineto{\pgfqpoint{3.240010in}{0.413320in}}%
\pgfpathlineto{\pgfqpoint{3.237411in}{0.413320in}}%
\pgfpathlineto{\pgfqpoint{3.234658in}{0.413320in}}%
\pgfpathlineto{\pgfqpoint{3.232069in}{0.413320in}}%
\pgfpathlineto{\pgfqpoint{3.229310in}{0.413320in}}%
\pgfpathlineto{\pgfqpoint{3.226609in}{0.413320in}}%
\pgfpathlineto{\pgfqpoint{3.223942in}{0.413320in}}%
\pgfpathlineto{\pgfqpoint{3.221255in}{0.413320in}}%
\pgfpathlineto{\pgfqpoint{3.218586in}{0.413320in}}%
\pgfpathlineto{\pgfqpoint{3.215908in}{0.413320in}}%
\pgfpathlineto{\pgfqpoint{3.213342in}{0.413320in}}%
\pgfpathlineto{\pgfqpoint{3.210545in}{0.413320in}}%
\pgfpathlineto{\pgfqpoint{3.207984in}{0.413320in}}%
\pgfpathlineto{\pgfqpoint{3.205195in}{0.413320in}}%
\pgfpathlineto{\pgfqpoint{3.202562in}{0.413320in}}%
\pgfpathlineto{\pgfqpoint{3.199823in}{0.413320in}}%
\pgfpathlineto{\pgfqpoint{3.197226in}{0.413320in}}%
\pgfpathlineto{\pgfqpoint{3.194508in}{0.413320in}}%
\pgfpathlineto{\pgfqpoint{3.191796in}{0.413320in}}%
\pgfpathlineto{\pgfqpoint{3.189117in}{0.413320in}}%
\pgfpathlineto{\pgfqpoint{3.186440in}{0.413320in}}%
\pgfpathlineto{\pgfqpoint{3.183760in}{0.413320in}}%
\pgfpathlineto{\pgfqpoint{3.181089in}{0.413320in}}%
\pgfpathlineto{\pgfqpoint{3.178525in}{0.413320in}}%
\pgfpathlineto{\pgfqpoint{3.175724in}{0.413320in}}%
\pgfpathlineto{\pgfqpoint{3.173142in}{0.413320in}}%
\pgfpathlineto{\pgfqpoint{3.170375in}{0.413320in}}%
\pgfpathlineto{\pgfqpoint{3.167776in}{0.413320in}}%
\pgfpathlineto{\pgfqpoint{3.165019in}{0.413320in}}%
\pgfpathlineto{\pgfqpoint{3.162474in}{0.413320in}}%
\pgfpathlineto{\pgfqpoint{3.159675in}{0.413320in}}%
\pgfpathlineto{\pgfqpoint{3.156981in}{0.413320in}}%
\pgfpathlineto{\pgfqpoint{3.154327in}{0.413320in}}%
\pgfpathlineto{\pgfqpoint{3.151612in}{0.413320in}}%
\pgfpathlineto{\pgfqpoint{3.149057in}{0.413320in}}%
\pgfpathlineto{\pgfqpoint{3.146271in}{0.413320in}}%
\pgfpathlineto{\pgfqpoint{3.143740in}{0.413320in}}%
\pgfpathlineto{\pgfqpoint{3.140913in}{0.413320in}}%
\pgfpathlineto{\pgfqpoint{3.138375in}{0.413320in}}%
\pgfpathlineto{\pgfqpoint{3.135550in}{0.413320in}}%
\pgfpathlineto{\pgfqpoint{3.132946in}{0.413320in}}%
\pgfpathlineto{\pgfqpoint{3.130199in}{0.413320in}}%
\pgfpathlineto{\pgfqpoint{3.127512in}{0.413320in}}%
\pgfpathlineto{\pgfqpoint{3.124842in}{0.413320in}}%
\pgfpathlineto{\pgfqpoint{3.122163in}{0.413320in}}%
\pgfpathlineto{\pgfqpoint{3.119487in}{0.413320in}}%
\pgfpathlineto{\pgfqpoint{3.116807in}{0.413320in}}%
\pgfpathlineto{\pgfqpoint{3.114242in}{0.413320in}}%
\pgfpathlineto{\pgfqpoint{3.111451in}{0.413320in}}%
\pgfpathlineto{\pgfqpoint{3.108896in}{0.413320in}}%
\pgfpathlineto{\pgfqpoint{3.106094in}{0.413320in}}%
\pgfpathlineto{\pgfqpoint{3.103508in}{0.413320in}}%
\pgfpathlineto{\pgfqpoint{3.100737in}{0.413320in}}%
\pgfpathlineto{\pgfqpoint{3.098163in}{0.413320in}}%
\pgfpathlineto{\pgfqpoint{3.095388in}{0.413320in}}%
\pgfpathlineto{\pgfqpoint{3.092699in}{0.413320in}}%
\pgfpathlineto{\pgfqpoint{3.090023in}{0.413320in}}%
\pgfpathlineto{\pgfqpoint{3.087343in}{0.413320in}}%
\pgfpathlineto{\pgfqpoint{3.084671in}{0.413320in}}%
\pgfpathlineto{\pgfqpoint{3.081990in}{0.413320in}}%
\pgfpathlineto{\pgfqpoint{3.079381in}{0.413320in}}%
\pgfpathlineto{\pgfqpoint{3.076631in}{0.413320in}}%
\pgfpathlineto{\pgfqpoint{3.074056in}{0.413320in}}%
\pgfpathlineto{\pgfqpoint{3.071266in}{0.413320in}}%
\pgfpathlineto{\pgfqpoint{3.068709in}{0.413320in}}%
\pgfpathlineto{\pgfqpoint{3.065916in}{0.413320in}}%
\pgfpathlineto{\pgfqpoint{3.063230in}{0.413320in}}%
\pgfpathlineto{\pgfqpoint{3.060561in}{0.413320in}}%
\pgfpathlineto{\pgfqpoint{3.057884in}{0.413320in}}%
\pgfpathlineto{\pgfqpoint{3.055202in}{0.413320in}}%
\pgfpathlineto{\pgfqpoint{3.052526in}{0.413320in}}%
\pgfpathlineto{\pgfqpoint{3.049988in}{0.413320in}}%
\pgfpathlineto{\pgfqpoint{3.047157in}{0.413320in}}%
\pgfpathlineto{\pgfqpoint{3.044568in}{0.413320in}}%
\pgfpathlineto{\pgfqpoint{3.041813in}{0.413320in}}%
\pgfpathlineto{\pgfqpoint{3.039262in}{0.413320in}}%
\pgfpathlineto{\pgfqpoint{3.036456in}{0.413320in}}%
\pgfpathlineto{\pgfqpoint{3.033921in}{0.413320in}}%
\pgfpathlineto{\pgfqpoint{3.031091in}{0.413320in}}%
\pgfpathlineto{\pgfqpoint{3.028412in}{0.413320in}}%
\pgfpathlineto{\pgfqpoint{3.025803in}{0.413320in}}%
\pgfpathlineto{\pgfqpoint{3.023058in}{0.413320in}}%
\pgfpathlineto{\pgfqpoint{3.020382in}{0.413320in}}%
\pgfpathlineto{\pgfqpoint{3.017707in}{0.413320in}}%
\pgfpathlineto{\pgfqpoint{3.015097in}{0.413320in}}%
\pgfpathlineto{\pgfqpoint{3.012351in}{0.413320in}}%
\pgfpathlineto{\pgfqpoint{3.009784in}{0.413320in}}%
\pgfpathlineto{\pgfqpoint{3.006993in}{0.413320in}}%
\pgfpathlineto{\pgfqpoint{3.004419in}{0.413320in}}%
\pgfpathlineto{\pgfqpoint{3.001635in}{0.413320in}}%
\pgfpathlineto{\pgfqpoint{2.999103in}{0.413320in}}%
\pgfpathlineto{\pgfqpoint{2.996300in}{0.413320in}}%
\pgfpathlineto{\pgfqpoint{2.993595in}{0.413320in}}%
\pgfpathlineto{\pgfqpoint{2.990978in}{0.413320in}}%
\pgfpathlineto{\pgfqpoint{2.988238in}{0.413320in}}%
\pgfpathlineto{\pgfqpoint{2.985666in}{0.413320in}}%
\pgfpathlineto{\pgfqpoint{2.982885in}{0.413320in}}%
\pgfpathlineto{\pgfqpoint{2.980341in}{0.413320in}}%
\pgfpathlineto{\pgfqpoint{2.977517in}{0.413320in}}%
\pgfpathlineto{\pgfqpoint{2.974972in}{0.413320in}}%
\pgfpathlineto{\pgfqpoint{2.972177in}{0.413320in}}%
\pgfpathlineto{\pgfqpoint{2.969599in}{0.413320in}}%
\pgfpathlineto{\pgfqpoint{2.966812in}{0.413320in}}%
\pgfpathlineto{\pgfqpoint{2.964127in}{0.413320in}}%
\pgfpathlineto{\pgfqpoint{2.961460in}{0.413320in}}%
\pgfpathlineto{\pgfqpoint{2.958782in}{0.413320in}}%
\pgfpathlineto{\pgfqpoint{2.956103in}{0.413320in}}%
\pgfpathlineto{\pgfqpoint{2.953422in}{0.413320in}}%
\pgfpathlineto{\pgfqpoint{2.950884in}{0.413320in}}%
\pgfpathlineto{\pgfqpoint{2.948068in}{0.413320in}}%
\pgfpathlineto{\pgfqpoint{2.945461in}{0.413320in}}%
\pgfpathlineto{\pgfqpoint{2.942711in}{0.413320in}}%
\pgfpathlineto{\pgfqpoint{2.940120in}{0.413320in}}%
\pgfpathlineto{\pgfqpoint{2.937352in}{0.413320in}}%
\pgfpathlineto{\pgfqpoint{2.934759in}{0.413320in}}%
\pgfpathlineto{\pgfqpoint{2.932033in}{0.413320in}}%
\pgfpathlineto{\pgfqpoint{2.929321in}{0.413320in}}%
\pgfpathlineto{\pgfqpoint{2.926655in}{0.413320in}}%
\pgfpathlineto{\pgfqpoint{2.923963in}{0.413320in}}%
\pgfpathlineto{\pgfqpoint{2.921363in}{0.413320in}}%
\pgfpathlineto{\pgfqpoint{2.918606in}{0.413320in}}%
\pgfpathlineto{\pgfqpoint{2.916061in}{0.413320in}}%
\pgfpathlineto{\pgfqpoint{2.913243in}{0.413320in}}%
\pgfpathlineto{\pgfqpoint{2.910631in}{0.413320in}}%
\pgfpathlineto{\pgfqpoint{2.907882in}{0.413320in}}%
\pgfpathlineto{\pgfqpoint{2.905341in}{0.413320in}}%
\pgfpathlineto{\pgfqpoint{2.902535in}{0.413320in}}%
\pgfpathlineto{\pgfqpoint{2.899858in}{0.413320in}}%
\pgfpathlineto{\pgfqpoint{2.897179in}{0.413320in}}%
\pgfpathlineto{\pgfqpoint{2.894487in}{0.413320in}}%
\pgfpathlineto{\pgfqpoint{2.891809in}{0.413320in}}%
\pgfpathlineto{\pgfqpoint{2.889145in}{0.413320in}}%
\pgfpathlineto{\pgfqpoint{2.886578in}{0.413320in}}%
\pgfpathlineto{\pgfqpoint{2.883780in}{0.413320in}}%
\pgfpathlineto{\pgfqpoint{2.881254in}{0.413320in}}%
\pgfpathlineto{\pgfqpoint{2.878431in}{0.413320in}}%
\pgfpathlineto{\pgfqpoint{2.875882in}{0.413320in}}%
\pgfpathlineto{\pgfqpoint{2.873074in}{0.413320in}}%
\pgfpathlineto{\pgfqpoint{2.870475in}{0.413320in}}%
\pgfpathlineto{\pgfqpoint{2.867713in}{0.413320in}}%
\pgfpathlineto{\pgfqpoint{2.865031in}{0.413320in}}%
\pgfpathlineto{\pgfqpoint{2.862402in}{0.413320in}}%
\pgfpathlineto{\pgfqpoint{2.859668in}{0.413320in}}%
\pgfpathlineto{\pgfqpoint{2.857003in}{0.413320in}}%
\pgfpathlineto{\pgfqpoint{2.854325in}{0.413320in}}%
\pgfpathlineto{\pgfqpoint{2.851793in}{0.413320in}}%
\pgfpathlineto{\pgfqpoint{2.848960in}{0.413320in}}%
\pgfpathlineto{\pgfqpoint{2.846408in}{0.413320in}}%
\pgfpathlineto{\pgfqpoint{2.843611in}{0.413320in}}%
\pgfpathlineto{\pgfqpoint{2.841055in}{0.413320in}}%
\pgfpathlineto{\pgfqpoint{2.838254in}{0.413320in}}%
\pgfpathlineto{\pgfqpoint{2.835698in}{0.413320in}}%
\pgfpathlineto{\pgfqpoint{2.832894in}{0.413320in}}%
\pgfpathlineto{\pgfqpoint{2.830219in}{0.413320in}}%
\pgfpathlineto{\pgfqpoint{2.827567in}{0.413320in}}%
\pgfpathlineto{\pgfqpoint{2.824851in}{0.413320in}}%
\pgfpathlineto{\pgfqpoint{2.822303in}{0.413320in}}%
\pgfpathlineto{\pgfqpoint{2.819506in}{0.413320in}}%
\pgfpathlineto{\pgfqpoint{2.816867in}{0.413320in}}%
\pgfpathlineto{\pgfqpoint{2.814141in}{0.413320in}}%
\pgfpathlineto{\pgfqpoint{2.811597in}{0.413320in}}%
\pgfpathlineto{\pgfqpoint{2.808792in}{0.413320in}}%
\pgfpathlineto{\pgfqpoint{2.806175in}{0.413320in}}%
\pgfpathlineto{\pgfqpoint{2.803435in}{0.413320in}}%
\pgfpathlineto{\pgfqpoint{2.800756in}{0.413320in}}%
\pgfpathlineto{\pgfqpoint{2.798070in}{0.413320in}}%
\pgfpathlineto{\pgfqpoint{2.795398in}{0.413320in}}%
\pgfpathlineto{\pgfqpoint{2.792721in}{0.413320in}}%
\pgfpathlineto{\pgfqpoint{2.790044in}{0.413320in}}%
\pgfpathlineto{\pgfqpoint{2.787468in}{0.413320in}}%
\pgfpathlineto{\pgfqpoint{2.784687in}{0.413320in}}%
\pgfpathlineto{\pgfqpoint{2.782113in}{0.413320in}}%
\pgfpathlineto{\pgfqpoint{2.779330in}{0.413320in}}%
\pgfpathlineto{\pgfqpoint{2.776767in}{0.413320in}}%
\pgfpathlineto{\pgfqpoint{2.773972in}{0.413320in}}%
\pgfpathlineto{\pgfqpoint{2.771373in}{0.413320in}}%
\pgfpathlineto{\pgfqpoint{2.768617in}{0.413320in}}%
\pgfpathlineto{\pgfqpoint{2.765935in}{0.413320in}}%
\pgfpathlineto{\pgfqpoint{2.763253in}{0.413320in}}%
\pgfpathlineto{\pgfqpoint{2.760581in}{0.413320in}}%
\pgfpathlineto{\pgfqpoint{2.758028in}{0.413320in}}%
\pgfpathlineto{\pgfqpoint{2.755224in}{0.413320in}}%
\pgfpathlineto{\pgfqpoint{2.752614in}{0.413320in}}%
\pgfpathlineto{\pgfqpoint{2.749868in}{0.413320in}}%
\pgfpathlineto{\pgfqpoint{2.747260in}{0.413320in}}%
\pgfpathlineto{\pgfqpoint{2.744510in}{0.413320in}}%
\pgfpathlineto{\pgfqpoint{2.741928in}{0.413320in}}%
\pgfpathlineto{\pgfqpoint{2.739155in}{0.413320in}}%
\pgfpathlineto{\pgfqpoint{2.736476in}{0.413320in}}%
\pgfpathlineto{\pgfqpoint{2.733798in}{0.413320in}}%
\pgfpathlineto{\pgfqpoint{2.731119in}{0.413320in}}%
\pgfpathlineto{\pgfqpoint{2.728439in}{0.413320in}}%
\pgfpathlineto{\pgfqpoint{2.725760in}{0.413320in}}%
\pgfpathlineto{\pgfqpoint{2.723211in}{0.413320in}}%
\pgfpathlineto{\pgfqpoint{2.720404in}{0.413320in}}%
\pgfpathlineto{\pgfqpoint{2.717773in}{0.413320in}}%
\pgfpathlineto{\pgfqpoint{2.715036in}{0.413320in}}%
\pgfpathlineto{\pgfqpoint{2.712477in}{0.413320in}}%
\pgfpathlineto{\pgfqpoint{2.709683in}{0.413320in}}%
\pgfpathlineto{\pgfqpoint{2.707125in}{0.413320in}}%
\pgfpathlineto{\pgfqpoint{2.704326in}{0.413320in}}%
\pgfpathlineto{\pgfqpoint{2.701657in}{0.413320in}}%
\pgfpathlineto{\pgfqpoint{2.698968in}{0.413320in}}%
\pgfpathlineto{\pgfqpoint{2.696293in}{0.413320in}}%
\pgfpathlineto{\pgfqpoint{2.693611in}{0.413320in}}%
\pgfpathlineto{\pgfqpoint{2.690940in}{0.413320in}}%
\pgfpathlineto{\pgfqpoint{2.688328in}{0.413320in}}%
\pgfpathlineto{\pgfqpoint{2.685586in}{0.413320in}}%
\pgfpathlineto{\pgfqpoint{2.683009in}{0.413320in}}%
\pgfpathlineto{\pgfqpoint{2.680224in}{0.413320in}}%
\pgfpathlineto{\pgfqpoint{2.677650in}{0.413320in}}%
\pgfpathlineto{\pgfqpoint{2.674873in}{0.413320in}}%
\pgfpathlineto{\pgfqpoint{2.672301in}{0.413320in}}%
\pgfpathlineto{\pgfqpoint{2.669506in}{0.413320in}}%
\pgfpathlineto{\pgfqpoint{2.666836in}{0.413320in}}%
\pgfpathlineto{\pgfqpoint{2.664151in}{0.413320in}}%
\pgfpathlineto{\pgfqpoint{2.661481in}{0.413320in}}%
\pgfpathlineto{\pgfqpoint{2.658942in}{0.413320in}}%
\pgfpathlineto{\pgfqpoint{2.656124in}{0.413320in}}%
\pgfpathlineto{\pgfqpoint{2.653567in}{0.413320in}}%
\pgfpathlineto{\pgfqpoint{2.650767in}{0.413320in}}%
\pgfpathlineto{\pgfqpoint{2.648196in}{0.413320in}}%
\pgfpathlineto{\pgfqpoint{2.645408in}{0.413320in}}%
\pgfpathlineto{\pgfqpoint{2.642827in}{0.413320in}}%
\pgfpathlineto{\pgfqpoint{2.640053in}{0.413320in}}%
\pgfpathlineto{\pgfqpoint{2.637369in}{0.413320in}}%
\pgfpathlineto{\pgfqpoint{2.634700in}{0.413320in}}%
\pgfpathlineto{\pgfqpoint{2.632018in}{0.413320in}}%
\pgfpathlineto{\pgfqpoint{2.629340in}{0.413320in}}%
\pgfpathlineto{\pgfqpoint{2.626653in}{0.413320in}}%
\pgfpathlineto{\pgfqpoint{2.624077in}{0.413320in}}%
\pgfpathlineto{\pgfqpoint{2.621304in}{0.413320in}}%
\pgfpathlineto{\pgfqpoint{2.618773in}{0.413320in}}%
\pgfpathlineto{\pgfqpoint{2.615934in}{0.413320in}}%
\pgfpathlineto{\pgfqpoint{2.613393in}{0.413320in}}%
\pgfpathlineto{\pgfqpoint{2.610588in}{0.413320in}}%
\pgfpathlineto{\pgfqpoint{2.608004in}{0.413320in}}%
\pgfpathlineto{\pgfqpoint{2.605232in}{0.413320in}}%
\pgfpathlineto{\pgfqpoint{2.602557in}{0.413320in}}%
\pgfpathlineto{\pgfqpoint{2.599920in}{0.413320in}}%
\pgfpathlineto{\pgfqpoint{2.597196in}{0.413320in}}%
\pgfpathlineto{\pgfqpoint{2.594630in}{0.413320in}}%
\pgfpathlineto{\pgfqpoint{2.591842in}{0.413320in}}%
\pgfpathlineto{\pgfqpoint{2.589248in}{0.413320in}}%
\pgfpathlineto{\pgfqpoint{2.586484in}{0.413320in}}%
\pgfpathlineto{\pgfqpoint{2.583913in}{0.413320in}}%
\pgfpathlineto{\pgfqpoint{2.581129in}{0.413320in}}%
\pgfpathlineto{\pgfqpoint{2.578567in}{0.413320in}}%
\pgfpathlineto{\pgfqpoint{2.575779in}{0.413320in}}%
\pgfpathlineto{\pgfqpoint{2.573082in}{0.413320in}}%
\pgfpathlineto{\pgfqpoint{2.570411in}{0.413320in}}%
\pgfpathlineto{\pgfqpoint{2.567730in}{0.413320in}}%
\pgfpathlineto{\pgfqpoint{2.565045in}{0.413320in}}%
\pgfpathlineto{\pgfqpoint{2.562375in}{0.413320in}}%
\pgfpathlineto{\pgfqpoint{2.559790in}{0.413320in}}%
\pgfpathlineto{\pgfqpoint{2.557009in}{0.413320in}}%
\pgfpathlineto{\pgfqpoint{2.554493in}{0.413320in}}%
\pgfpathlineto{\pgfqpoint{2.551664in}{0.413320in}}%
\pgfpathlineto{\pgfqpoint{2.549114in}{0.413320in}}%
\pgfpathlineto{\pgfqpoint{2.546310in}{0.413320in}}%
\pgfpathlineto{\pgfqpoint{2.543765in}{0.413320in}}%
\pgfpathlineto{\pgfqpoint{2.540949in}{0.413320in}}%
\pgfpathlineto{\pgfqpoint{2.538274in}{0.413320in}}%
\pgfpathlineto{\pgfqpoint{2.535624in}{0.413320in}}%
\pgfpathlineto{\pgfqpoint{2.532917in}{0.413320in}}%
\pgfpathlineto{\pgfqpoint{2.530234in}{0.413320in}}%
\pgfpathlineto{\pgfqpoint{2.527560in}{0.413320in}}%
\pgfpathlineto{\pgfqpoint{2.524988in}{0.413320in}}%
\pgfpathlineto{\pgfqpoint{2.522197in}{0.413320in}}%
\pgfpathlineto{\pgfqpoint{2.519607in}{0.413320in}}%
\pgfpathlineto{\pgfqpoint{2.516845in}{0.413320in}}%
\pgfpathlineto{\pgfqpoint{2.514268in}{0.413320in}}%
\pgfpathlineto{\pgfqpoint{2.511478in}{0.413320in}}%
\pgfpathlineto{\pgfqpoint{2.508917in}{0.413320in}}%
\pgfpathlineto{\pgfqpoint{2.506163in}{0.413320in}}%
\pgfpathlineto{\pgfqpoint{2.503454in}{0.413320in}}%
\pgfpathlineto{\pgfqpoint{2.500801in}{0.413320in}}%
\pgfpathlineto{\pgfqpoint{2.498085in}{0.413320in}}%
\pgfpathlineto{\pgfqpoint{2.495542in}{0.413320in}}%
\pgfpathlineto{\pgfqpoint{2.492729in}{0.413320in}}%
\pgfpathlineto{\pgfqpoint{2.490183in}{0.413320in}}%
\pgfpathlineto{\pgfqpoint{2.487384in}{0.413320in}}%
\pgfpathlineto{\pgfqpoint{2.484870in}{0.413320in}}%
\pgfpathlineto{\pgfqpoint{2.482026in}{0.413320in}}%
\pgfpathlineto{\pgfqpoint{2.479420in}{0.413320in}}%
\pgfpathlineto{\pgfqpoint{2.476671in}{0.413320in}}%
\pgfpathlineto{\pgfqpoint{2.473989in}{0.413320in}}%
\pgfpathlineto{\pgfqpoint{2.471311in}{0.413320in}}%
\pgfpathlineto{\pgfqpoint{2.468635in}{0.413320in}}%
\pgfpathlineto{\pgfqpoint{2.465957in}{0.413320in}}%
\pgfpathlineto{\pgfqpoint{2.463280in}{0.413320in}}%
\pgfpathlineto{\pgfqpoint{2.460711in}{0.413320in}}%
\pgfpathlineto{\pgfqpoint{2.457917in}{0.413320in}}%
\pgfpathlineto{\pgfqpoint{2.455353in}{0.413320in}}%
\pgfpathlineto{\pgfqpoint{2.452562in}{0.413320in}}%
\pgfpathlineto{\pgfqpoint{2.450032in}{0.413320in}}%
\pgfpathlineto{\pgfqpoint{2.447209in}{0.413320in}}%
\pgfpathlineto{\pgfqpoint{2.444677in}{0.413320in}}%
\pgfpathlineto{\pgfqpoint{2.441876in}{0.413320in}}%
\pgfpathlineto{\pgfqpoint{2.439167in}{0.413320in}}%
\pgfpathlineto{\pgfqpoint{2.436518in}{0.413320in}}%
\pgfpathlineto{\pgfqpoint{2.433815in}{0.413320in}}%
\pgfpathlineto{\pgfqpoint{2.431251in}{0.413320in}}%
\pgfpathlineto{\pgfqpoint{2.428453in}{0.413320in}}%
\pgfpathlineto{\pgfqpoint{2.425878in}{0.413320in}}%
\pgfpathlineto{\pgfqpoint{2.423098in}{0.413320in}}%
\pgfpathlineto{\pgfqpoint{2.420528in}{0.413320in}}%
\pgfpathlineto{\pgfqpoint{2.417747in}{0.413320in}}%
\pgfpathlineto{\pgfqpoint{2.415184in}{0.413320in}}%
\pgfpathlineto{\pgfqpoint{2.412389in}{0.413320in}}%
\pgfpathlineto{\pgfqpoint{2.409699in}{0.413320in}}%
\pgfpathlineto{\pgfqpoint{2.407024in}{0.413320in}}%
\pgfpathlineto{\pgfqpoint{2.404352in}{0.413320in}}%
\pgfpathlineto{\pgfqpoint{2.401675in}{0.413320in}}%
\pgfpathlineto{\pgfqpoint{2.398995in}{0.413320in}}%
\pgfpathclose%
\pgfusepath{stroke,fill}%
\end{pgfscope}%
\begin{pgfscope}%
\pgfpathrectangle{\pgfqpoint{2.398995in}{0.319877in}}{\pgfqpoint{3.986877in}{1.993438in}} %
\pgfusepath{clip}%
\pgfsetbuttcap%
\pgfsetroundjoin%
\definecolor{currentfill}{rgb}{1.000000,1.000000,1.000000}%
\pgfsetfillcolor{currentfill}%
\pgfsetlinewidth{1.003750pt}%
\definecolor{currentstroke}{rgb}{0.961777,0.393453,0.819319}%
\pgfsetstrokecolor{currentstroke}%
\pgfsetdash{}{0pt}%
\pgfpathmoveto{\pgfqpoint{2.398995in}{0.413320in}}%
\pgfpathlineto{\pgfqpoint{2.398995in}{0.851461in}}%
\pgfpathlineto{\pgfqpoint{2.401675in}{0.847950in}}%
\pgfpathlineto{\pgfqpoint{2.404352in}{0.851405in}}%
\pgfpathlineto{\pgfqpoint{2.407024in}{0.849678in}}%
\pgfpathlineto{\pgfqpoint{2.409699in}{0.848136in}}%
\pgfpathlineto{\pgfqpoint{2.412389in}{0.851796in}}%
\pgfpathlineto{\pgfqpoint{2.415184in}{0.854726in}}%
\pgfpathlineto{\pgfqpoint{2.417747in}{0.854252in}}%
\pgfpathlineto{\pgfqpoint{2.420528in}{0.852797in}}%
\pgfpathlineto{\pgfqpoint{2.423098in}{0.855055in}}%
\pgfpathlineto{\pgfqpoint{2.425878in}{0.852827in}}%
\pgfpathlineto{\pgfqpoint{2.428453in}{0.854211in}}%
\pgfpathlineto{\pgfqpoint{2.431251in}{0.846605in}}%
\pgfpathlineto{\pgfqpoint{2.433815in}{0.843379in}}%
\pgfpathlineto{\pgfqpoint{2.436518in}{0.844493in}}%
\pgfpathlineto{\pgfqpoint{2.439167in}{0.847646in}}%
\pgfpathlineto{\pgfqpoint{2.441876in}{0.846121in}}%
\pgfpathlineto{\pgfqpoint{2.444677in}{0.845016in}}%
\pgfpathlineto{\pgfqpoint{2.447209in}{0.847677in}}%
\pgfpathlineto{\pgfqpoint{2.450032in}{0.844837in}}%
\pgfpathlineto{\pgfqpoint{2.452562in}{0.844811in}}%
\pgfpathlineto{\pgfqpoint{2.455353in}{0.848600in}}%
\pgfpathlineto{\pgfqpoint{2.457917in}{0.847371in}}%
\pgfpathlineto{\pgfqpoint{2.460711in}{0.849057in}}%
\pgfpathlineto{\pgfqpoint{2.463280in}{0.849427in}}%
\pgfpathlineto{\pgfqpoint{2.465957in}{0.850891in}}%
\pgfpathlineto{\pgfqpoint{2.468635in}{0.850698in}}%
\pgfpathlineto{\pgfqpoint{2.471311in}{0.854542in}}%
\pgfpathlineto{\pgfqpoint{2.473989in}{0.850919in}}%
\pgfpathlineto{\pgfqpoint{2.476671in}{0.846790in}}%
\pgfpathlineto{\pgfqpoint{2.479420in}{0.848949in}}%
\pgfpathlineto{\pgfqpoint{2.482026in}{0.850926in}}%
\pgfpathlineto{\pgfqpoint{2.484870in}{0.843132in}}%
\pgfpathlineto{\pgfqpoint{2.487384in}{0.843040in}}%
\pgfpathlineto{\pgfqpoint{2.490183in}{0.847815in}}%
\pgfpathlineto{\pgfqpoint{2.492729in}{0.847124in}}%
\pgfpathlineto{\pgfqpoint{2.495542in}{0.848980in}}%
\pgfpathlineto{\pgfqpoint{2.498085in}{0.849177in}}%
\pgfpathlineto{\pgfqpoint{2.500801in}{0.848403in}}%
\pgfpathlineto{\pgfqpoint{2.503454in}{0.852534in}}%
\pgfpathlineto{\pgfqpoint{2.506163in}{0.849594in}}%
\pgfpathlineto{\pgfqpoint{2.508917in}{0.850097in}}%
\pgfpathlineto{\pgfqpoint{2.511478in}{0.847484in}}%
\pgfpathlineto{\pgfqpoint{2.514268in}{0.847133in}}%
\pgfpathlineto{\pgfqpoint{2.516845in}{0.849207in}}%
\pgfpathlineto{\pgfqpoint{2.519607in}{0.846541in}}%
\pgfpathlineto{\pgfqpoint{2.522197in}{0.848238in}}%
\pgfpathlineto{\pgfqpoint{2.524988in}{0.846271in}}%
\pgfpathlineto{\pgfqpoint{2.527560in}{0.846729in}}%
\pgfpathlineto{\pgfqpoint{2.530234in}{0.843040in}}%
\pgfpathlineto{\pgfqpoint{2.532917in}{0.849297in}}%
\pgfpathlineto{\pgfqpoint{2.535624in}{0.849687in}}%
\pgfpathlineto{\pgfqpoint{2.538274in}{0.849118in}}%
\pgfpathlineto{\pgfqpoint{2.540949in}{0.850053in}}%
\pgfpathlineto{\pgfqpoint{2.543765in}{0.849620in}}%
\pgfpathlineto{\pgfqpoint{2.546310in}{0.854947in}}%
\pgfpathlineto{\pgfqpoint{2.549114in}{0.863391in}}%
\pgfpathlineto{\pgfqpoint{2.551664in}{0.882591in}}%
\pgfpathlineto{\pgfqpoint{2.554493in}{0.922314in}}%
\pgfpathlineto{\pgfqpoint{2.557009in}{0.945139in}}%
\pgfpathlineto{\pgfqpoint{2.559790in}{0.943409in}}%
\pgfpathlineto{\pgfqpoint{2.562375in}{0.973012in}}%
\pgfpathlineto{\pgfqpoint{2.565045in}{0.984065in}}%
\pgfpathlineto{\pgfqpoint{2.567730in}{0.988522in}}%
\pgfpathlineto{\pgfqpoint{2.570411in}{0.992970in}}%
\pgfpathlineto{\pgfqpoint{2.573082in}{0.999372in}}%
\pgfpathlineto{\pgfqpoint{2.575779in}{0.991062in}}%
\pgfpathlineto{\pgfqpoint{2.578567in}{0.979445in}}%
\pgfpathlineto{\pgfqpoint{2.581129in}{0.977804in}}%
\pgfpathlineto{\pgfqpoint{2.583913in}{0.965463in}}%
\pgfpathlineto{\pgfqpoint{2.586484in}{0.958204in}}%
\pgfpathlineto{\pgfqpoint{2.589248in}{0.956151in}}%
\pgfpathlineto{\pgfqpoint{2.591842in}{0.946854in}}%
\pgfpathlineto{\pgfqpoint{2.594630in}{0.932592in}}%
\pgfpathlineto{\pgfqpoint{2.597196in}{0.926289in}}%
\pgfpathlineto{\pgfqpoint{2.599920in}{0.918880in}}%
\pgfpathlineto{\pgfqpoint{2.602557in}{0.907738in}}%
\pgfpathlineto{\pgfqpoint{2.605232in}{0.906117in}}%
\pgfpathlineto{\pgfqpoint{2.608004in}{0.898231in}}%
\pgfpathlineto{\pgfqpoint{2.610588in}{0.889350in}}%
\pgfpathlineto{\pgfqpoint{2.613393in}{0.888017in}}%
\pgfpathlineto{\pgfqpoint{2.615934in}{0.881856in}}%
\pgfpathlineto{\pgfqpoint{2.618773in}{0.876403in}}%
\pgfpathlineto{\pgfqpoint{2.621304in}{0.872646in}}%
\pgfpathlineto{\pgfqpoint{2.624077in}{0.866703in}}%
\pgfpathlineto{\pgfqpoint{2.626653in}{0.865881in}}%
\pgfpathlineto{\pgfqpoint{2.629340in}{0.867418in}}%
\pgfpathlineto{\pgfqpoint{2.632018in}{0.862728in}}%
\pgfpathlineto{\pgfqpoint{2.634700in}{0.861046in}}%
\pgfpathlineto{\pgfqpoint{2.637369in}{0.861307in}}%
\pgfpathlineto{\pgfqpoint{2.640053in}{0.859885in}}%
\pgfpathlineto{\pgfqpoint{2.642827in}{0.863033in}}%
\pgfpathlineto{\pgfqpoint{2.645408in}{0.864473in}}%
\pgfpathlineto{\pgfqpoint{2.648196in}{0.868139in}}%
\pgfpathlineto{\pgfqpoint{2.650767in}{0.872392in}}%
\pgfpathlineto{\pgfqpoint{2.653567in}{0.867180in}}%
\pgfpathlineto{\pgfqpoint{2.656124in}{0.864384in}}%
\pgfpathlineto{\pgfqpoint{2.658942in}{0.857655in}}%
\pgfpathlineto{\pgfqpoint{2.661481in}{0.858422in}}%
\pgfpathlineto{\pgfqpoint{2.664151in}{0.856976in}}%
\pgfpathlineto{\pgfqpoint{2.666836in}{0.857999in}}%
\pgfpathlineto{\pgfqpoint{2.669506in}{0.857260in}}%
\pgfpathlineto{\pgfqpoint{2.672301in}{0.854535in}}%
\pgfpathlineto{\pgfqpoint{2.674873in}{0.849041in}}%
\pgfpathlineto{\pgfqpoint{2.677650in}{0.851131in}}%
\pgfpathlineto{\pgfqpoint{2.680224in}{0.852679in}}%
\pgfpathlineto{\pgfqpoint{2.683009in}{0.853289in}}%
\pgfpathlineto{\pgfqpoint{2.685586in}{0.852803in}}%
\pgfpathlineto{\pgfqpoint{2.688328in}{0.852234in}}%
\pgfpathlineto{\pgfqpoint{2.690940in}{0.852245in}}%
\pgfpathlineto{\pgfqpoint{2.693611in}{0.849442in}}%
\pgfpathlineto{\pgfqpoint{2.696293in}{0.852962in}}%
\pgfpathlineto{\pgfqpoint{2.698968in}{0.851453in}}%
\pgfpathlineto{\pgfqpoint{2.701657in}{0.851941in}}%
\pgfpathlineto{\pgfqpoint{2.704326in}{0.853272in}}%
\pgfpathlineto{\pgfqpoint{2.707125in}{0.854451in}}%
\pgfpathlineto{\pgfqpoint{2.709683in}{0.853412in}}%
\pgfpathlineto{\pgfqpoint{2.712477in}{0.854222in}}%
\pgfpathlineto{\pgfqpoint{2.715036in}{0.857060in}}%
\pgfpathlineto{\pgfqpoint{2.717773in}{0.857861in}}%
\pgfpathlineto{\pgfqpoint{2.720404in}{0.858696in}}%
\pgfpathlineto{\pgfqpoint{2.723211in}{0.853090in}}%
\pgfpathlineto{\pgfqpoint{2.725760in}{0.851017in}}%
\pgfpathlineto{\pgfqpoint{2.728439in}{0.849134in}}%
\pgfpathlineto{\pgfqpoint{2.731119in}{0.847857in}}%
\pgfpathlineto{\pgfqpoint{2.733798in}{0.852548in}}%
\pgfpathlineto{\pgfqpoint{2.736476in}{0.852276in}}%
\pgfpathlineto{\pgfqpoint{2.739155in}{0.859722in}}%
\pgfpathlineto{\pgfqpoint{2.741928in}{0.855931in}}%
\pgfpathlineto{\pgfqpoint{2.744510in}{0.854013in}}%
\pgfpathlineto{\pgfqpoint{2.747260in}{0.856116in}}%
\pgfpathlineto{\pgfqpoint{2.749868in}{0.859546in}}%
\pgfpathlineto{\pgfqpoint{2.752614in}{0.854926in}}%
\pgfpathlineto{\pgfqpoint{2.755224in}{0.857068in}}%
\pgfpathlineto{\pgfqpoint{2.758028in}{0.855321in}}%
\pgfpathlineto{\pgfqpoint{2.760581in}{0.866068in}}%
\pgfpathlineto{\pgfqpoint{2.763253in}{0.856035in}}%
\pgfpathlineto{\pgfqpoint{2.765935in}{0.855831in}}%
\pgfpathlineto{\pgfqpoint{2.768617in}{0.851060in}}%
\pgfpathlineto{\pgfqpoint{2.771373in}{0.855343in}}%
\pgfpathlineto{\pgfqpoint{2.773972in}{0.856958in}}%
\pgfpathlineto{\pgfqpoint{2.776767in}{0.860864in}}%
\pgfpathlineto{\pgfqpoint{2.779330in}{0.862893in}}%
\pgfpathlineto{\pgfqpoint{2.782113in}{0.863283in}}%
\pgfpathlineto{\pgfqpoint{2.784687in}{0.856211in}}%
\pgfpathlineto{\pgfqpoint{2.787468in}{0.859045in}}%
\pgfpathlineto{\pgfqpoint{2.790044in}{0.862061in}}%
\pgfpathlineto{\pgfqpoint{2.792721in}{0.857942in}}%
\pgfpathlineto{\pgfqpoint{2.795398in}{0.872293in}}%
\pgfpathlineto{\pgfqpoint{2.798070in}{0.862785in}}%
\pgfpathlineto{\pgfqpoint{2.800756in}{0.857472in}}%
\pgfpathlineto{\pgfqpoint{2.803435in}{0.855736in}}%
\pgfpathlineto{\pgfqpoint{2.806175in}{0.862429in}}%
\pgfpathlineto{\pgfqpoint{2.808792in}{0.852886in}}%
\pgfpathlineto{\pgfqpoint{2.811597in}{0.851591in}}%
\pgfpathlineto{\pgfqpoint{2.814141in}{0.851327in}}%
\pgfpathlineto{\pgfqpoint{2.816867in}{0.848531in}}%
\pgfpathlineto{\pgfqpoint{2.819506in}{0.852708in}}%
\pgfpathlineto{\pgfqpoint{2.822303in}{0.852416in}}%
\pgfpathlineto{\pgfqpoint{2.824851in}{0.850414in}}%
\pgfpathlineto{\pgfqpoint{2.827567in}{0.849590in}}%
\pgfpathlineto{\pgfqpoint{2.830219in}{0.850943in}}%
\pgfpathlineto{\pgfqpoint{2.832894in}{0.853501in}}%
\pgfpathlineto{\pgfqpoint{2.835698in}{0.853257in}}%
\pgfpathlineto{\pgfqpoint{2.838254in}{0.857784in}}%
\pgfpathlineto{\pgfqpoint{2.841055in}{0.857384in}}%
\pgfpathlineto{\pgfqpoint{2.843611in}{0.851828in}}%
\pgfpathlineto{\pgfqpoint{2.846408in}{0.852403in}}%
\pgfpathlineto{\pgfqpoint{2.848960in}{0.851112in}}%
\pgfpathlineto{\pgfqpoint{2.851793in}{0.850899in}}%
\pgfpathlineto{\pgfqpoint{2.854325in}{0.848766in}}%
\pgfpathlineto{\pgfqpoint{2.857003in}{0.851197in}}%
\pgfpathlineto{\pgfqpoint{2.859668in}{0.850205in}}%
\pgfpathlineto{\pgfqpoint{2.862402in}{0.850492in}}%
\pgfpathlineto{\pgfqpoint{2.865031in}{0.850210in}}%
\pgfpathlineto{\pgfqpoint{2.867713in}{0.848576in}}%
\pgfpathlineto{\pgfqpoint{2.870475in}{0.850109in}}%
\pgfpathlineto{\pgfqpoint{2.873074in}{0.849288in}}%
\pgfpathlineto{\pgfqpoint{2.875882in}{0.852205in}}%
\pgfpathlineto{\pgfqpoint{2.878431in}{0.854332in}}%
\pgfpathlineto{\pgfqpoint{2.881254in}{0.851472in}}%
\pgfpathlineto{\pgfqpoint{2.883780in}{0.849090in}}%
\pgfpathlineto{\pgfqpoint{2.886578in}{0.854481in}}%
\pgfpathlineto{\pgfqpoint{2.889145in}{0.853282in}}%
\pgfpathlineto{\pgfqpoint{2.891809in}{0.851762in}}%
\pgfpathlineto{\pgfqpoint{2.894487in}{0.851588in}}%
\pgfpathlineto{\pgfqpoint{2.897179in}{0.848910in}}%
\pgfpathlineto{\pgfqpoint{2.899858in}{0.853272in}}%
\pgfpathlineto{\pgfqpoint{2.902535in}{0.852480in}}%
\pgfpathlineto{\pgfqpoint{2.905341in}{0.850113in}}%
\pgfpathlineto{\pgfqpoint{2.907882in}{0.848931in}}%
\pgfpathlineto{\pgfqpoint{2.910631in}{0.849716in}}%
\pgfpathlineto{\pgfqpoint{2.913243in}{0.854658in}}%
\pgfpathlineto{\pgfqpoint{2.916061in}{0.848086in}}%
\pgfpathlineto{\pgfqpoint{2.918606in}{0.849839in}}%
\pgfpathlineto{\pgfqpoint{2.921363in}{0.848855in}}%
\pgfpathlineto{\pgfqpoint{2.923963in}{0.848004in}}%
\pgfpathlineto{\pgfqpoint{2.926655in}{0.850754in}}%
\pgfpathlineto{\pgfqpoint{2.929321in}{0.849788in}}%
\pgfpathlineto{\pgfqpoint{2.932033in}{0.850559in}}%
\pgfpathlineto{\pgfqpoint{2.934759in}{0.852524in}}%
\pgfpathlineto{\pgfqpoint{2.937352in}{0.852206in}}%
\pgfpathlineto{\pgfqpoint{2.940120in}{0.854347in}}%
\pgfpathlineto{\pgfqpoint{2.942711in}{0.848841in}}%
\pgfpathlineto{\pgfqpoint{2.945461in}{0.845461in}}%
\pgfpathlineto{\pgfqpoint{2.948068in}{0.849081in}}%
\pgfpathlineto{\pgfqpoint{2.950884in}{0.849201in}}%
\pgfpathlineto{\pgfqpoint{2.953422in}{0.853409in}}%
\pgfpathlineto{\pgfqpoint{2.956103in}{0.851967in}}%
\pgfpathlineto{\pgfqpoint{2.958782in}{0.852182in}}%
\pgfpathlineto{\pgfqpoint{2.961460in}{0.855206in}}%
\pgfpathlineto{\pgfqpoint{2.964127in}{0.856333in}}%
\pgfpathlineto{\pgfqpoint{2.966812in}{0.852193in}}%
\pgfpathlineto{\pgfqpoint{2.969599in}{0.854835in}}%
\pgfpathlineto{\pgfqpoint{2.972177in}{0.851997in}}%
\pgfpathlineto{\pgfqpoint{2.974972in}{0.851764in}}%
\pgfpathlineto{\pgfqpoint{2.977517in}{0.853163in}}%
\pgfpathlineto{\pgfqpoint{2.980341in}{0.854581in}}%
\pgfpathlineto{\pgfqpoint{2.982885in}{0.854137in}}%
\pgfpathlineto{\pgfqpoint{2.985666in}{0.854566in}}%
\pgfpathlineto{\pgfqpoint{2.988238in}{0.855593in}}%
\pgfpathlineto{\pgfqpoint{2.990978in}{0.860593in}}%
\pgfpathlineto{\pgfqpoint{2.993595in}{0.856119in}}%
\pgfpathlineto{\pgfqpoint{2.996300in}{0.857135in}}%
\pgfpathlineto{\pgfqpoint{2.999103in}{0.855407in}}%
\pgfpathlineto{\pgfqpoint{3.001635in}{0.857642in}}%
\pgfpathlineto{\pgfqpoint{3.004419in}{0.859823in}}%
\pgfpathlineto{\pgfqpoint{3.006993in}{0.859168in}}%
\pgfpathlineto{\pgfqpoint{3.009784in}{0.855481in}}%
\pgfpathlineto{\pgfqpoint{3.012351in}{0.855306in}}%
\pgfpathlineto{\pgfqpoint{3.015097in}{0.851938in}}%
\pgfpathlineto{\pgfqpoint{3.017707in}{0.855403in}}%
\pgfpathlineto{\pgfqpoint{3.020382in}{0.854196in}}%
\pgfpathlineto{\pgfqpoint{3.023058in}{0.860774in}}%
\pgfpathlineto{\pgfqpoint{3.025803in}{0.866060in}}%
\pgfpathlineto{\pgfqpoint{3.028412in}{0.871600in}}%
\pgfpathlineto{\pgfqpoint{3.031091in}{0.866786in}}%
\pgfpathlineto{\pgfqpoint{3.033921in}{0.864860in}}%
\pgfpathlineto{\pgfqpoint{3.036456in}{0.859623in}}%
\pgfpathlineto{\pgfqpoint{3.039262in}{0.861420in}}%
\pgfpathlineto{\pgfqpoint{3.041813in}{0.862493in}}%
\pgfpathlineto{\pgfqpoint{3.044568in}{0.861750in}}%
\pgfpathlineto{\pgfqpoint{3.047157in}{0.864528in}}%
\pgfpathlineto{\pgfqpoint{3.049988in}{0.875654in}}%
\pgfpathlineto{\pgfqpoint{3.052526in}{0.874283in}}%
\pgfpathlineto{\pgfqpoint{3.055202in}{0.866321in}}%
\pgfpathlineto{\pgfqpoint{3.057884in}{0.865412in}}%
\pgfpathlineto{\pgfqpoint{3.060561in}{0.858528in}}%
\pgfpathlineto{\pgfqpoint{3.063230in}{0.865928in}}%
\pgfpathlineto{\pgfqpoint{3.065916in}{0.863863in}}%
\pgfpathlineto{\pgfqpoint{3.068709in}{0.857363in}}%
\pgfpathlineto{\pgfqpoint{3.071266in}{0.858708in}}%
\pgfpathlineto{\pgfqpoint{3.074056in}{0.863449in}}%
\pgfpathlineto{\pgfqpoint{3.076631in}{0.853937in}}%
\pgfpathlineto{\pgfqpoint{3.079381in}{0.855500in}}%
\pgfpathlineto{\pgfqpoint{3.081990in}{0.850984in}}%
\pgfpathlineto{\pgfqpoint{3.084671in}{0.847473in}}%
\pgfpathlineto{\pgfqpoint{3.087343in}{0.850659in}}%
\pgfpathlineto{\pgfqpoint{3.090023in}{0.848606in}}%
\pgfpathlineto{\pgfqpoint{3.092699in}{0.857845in}}%
\pgfpathlineto{\pgfqpoint{3.095388in}{0.857629in}}%
\pgfpathlineto{\pgfqpoint{3.098163in}{0.852575in}}%
\pgfpathlineto{\pgfqpoint{3.100737in}{0.865076in}}%
\pgfpathlineto{\pgfqpoint{3.103508in}{0.883279in}}%
\pgfpathlineto{\pgfqpoint{3.106094in}{0.871023in}}%
\pgfpathlineto{\pgfqpoint{3.108896in}{0.860716in}}%
\pgfpathlineto{\pgfqpoint{3.111451in}{0.856244in}}%
\pgfpathlineto{\pgfqpoint{3.114242in}{0.856091in}}%
\pgfpathlineto{\pgfqpoint{3.116807in}{0.850221in}}%
\pgfpathlineto{\pgfqpoint{3.119487in}{0.849810in}}%
\pgfpathlineto{\pgfqpoint{3.122163in}{0.849133in}}%
\pgfpathlineto{\pgfqpoint{3.124842in}{0.849455in}}%
\pgfpathlineto{\pgfqpoint{3.127512in}{0.850441in}}%
\pgfpathlineto{\pgfqpoint{3.130199in}{0.852025in}}%
\pgfpathlineto{\pgfqpoint{3.132946in}{0.855300in}}%
\pgfpathlineto{\pgfqpoint{3.135550in}{0.851663in}}%
\pgfpathlineto{\pgfqpoint{3.138375in}{0.848061in}}%
\pgfpathlineto{\pgfqpoint{3.140913in}{0.843040in}}%
\pgfpathlineto{\pgfqpoint{3.143740in}{0.843040in}}%
\pgfpathlineto{\pgfqpoint{3.146271in}{0.843040in}}%
\pgfpathlineto{\pgfqpoint{3.149057in}{0.843040in}}%
\pgfpathlineto{\pgfqpoint{3.151612in}{0.843040in}}%
\pgfpathlineto{\pgfqpoint{3.154327in}{0.843744in}}%
\pgfpathlineto{\pgfqpoint{3.156981in}{0.843040in}}%
\pgfpathlineto{\pgfqpoint{3.159675in}{0.843744in}}%
\pgfpathlineto{\pgfqpoint{3.162474in}{0.843040in}}%
\pgfpathlineto{\pgfqpoint{3.165019in}{0.843040in}}%
\pgfpathlineto{\pgfqpoint{3.167776in}{0.843040in}}%
\pgfpathlineto{\pgfqpoint{3.170375in}{0.843040in}}%
\pgfpathlineto{\pgfqpoint{3.173142in}{0.843040in}}%
\pgfpathlineto{\pgfqpoint{3.175724in}{0.843040in}}%
\pgfpathlineto{\pgfqpoint{3.178525in}{0.843040in}}%
\pgfpathlineto{\pgfqpoint{3.181089in}{0.843040in}}%
\pgfpathlineto{\pgfqpoint{3.183760in}{0.843040in}}%
\pgfpathlineto{\pgfqpoint{3.186440in}{0.843040in}}%
\pgfpathlineto{\pgfqpoint{3.189117in}{0.843040in}}%
\pgfpathlineto{\pgfqpoint{3.191796in}{0.843040in}}%
\pgfpathlineto{\pgfqpoint{3.194508in}{0.843040in}}%
\pgfpathlineto{\pgfqpoint{3.197226in}{0.843040in}}%
\pgfpathlineto{\pgfqpoint{3.199823in}{0.843040in}}%
\pgfpathlineto{\pgfqpoint{3.202562in}{0.843040in}}%
\pgfpathlineto{\pgfqpoint{3.205195in}{0.843040in}}%
\pgfpathlineto{\pgfqpoint{3.207984in}{0.843040in}}%
\pgfpathlineto{\pgfqpoint{3.210545in}{0.843040in}}%
\pgfpathlineto{\pgfqpoint{3.213342in}{0.843040in}}%
\pgfpathlineto{\pgfqpoint{3.215908in}{0.843040in}}%
\pgfpathlineto{\pgfqpoint{3.218586in}{0.843040in}}%
\pgfpathlineto{\pgfqpoint{3.221255in}{0.843040in}}%
\pgfpathlineto{\pgfqpoint{3.223942in}{0.843040in}}%
\pgfpathlineto{\pgfqpoint{3.226609in}{0.843040in}}%
\pgfpathlineto{\pgfqpoint{3.229310in}{0.843040in}}%
\pgfpathlineto{\pgfqpoint{3.232069in}{0.844923in}}%
\pgfpathlineto{\pgfqpoint{3.234658in}{0.844240in}}%
\pgfpathlineto{\pgfqpoint{3.237411in}{0.844435in}}%
\pgfpathlineto{\pgfqpoint{3.240010in}{0.846262in}}%
\pgfpathlineto{\pgfqpoint{3.242807in}{0.844354in}}%
\pgfpathlineto{\pgfqpoint{3.245363in}{0.843590in}}%
\pgfpathlineto{\pgfqpoint{3.248049in}{0.847300in}}%
\pgfpathlineto{\pgfqpoint{3.250716in}{0.845955in}}%
\pgfpathlineto{\pgfqpoint{3.253404in}{0.849532in}}%
\pgfpathlineto{\pgfqpoint{3.256083in}{0.848610in}}%
\pgfpathlineto{\pgfqpoint{3.258784in}{0.851639in}}%
\pgfpathlineto{\pgfqpoint{3.261594in}{0.853208in}}%
\pgfpathlineto{\pgfqpoint{3.264119in}{0.849110in}}%
\pgfpathlineto{\pgfqpoint{3.266849in}{0.849938in}}%
\pgfpathlineto{\pgfqpoint{3.269478in}{0.857710in}}%
\pgfpathlineto{\pgfqpoint{3.272254in}{0.855657in}}%
\pgfpathlineto{\pgfqpoint{3.274831in}{0.854291in}}%
\pgfpathlineto{\pgfqpoint{3.277603in}{0.856209in}}%
\pgfpathlineto{\pgfqpoint{3.280189in}{0.855243in}}%
\pgfpathlineto{\pgfqpoint{3.282870in}{0.856895in}}%
\pgfpathlineto{\pgfqpoint{3.285534in}{0.856225in}}%
\pgfpathlineto{\pgfqpoint{3.288225in}{0.855657in}}%
\pgfpathlineto{\pgfqpoint{3.290890in}{0.857276in}}%
\pgfpathlineto{\pgfqpoint{3.293574in}{0.852462in}}%
\pgfpathlineto{\pgfqpoint{3.296376in}{0.856735in}}%
\pgfpathlineto{\pgfqpoint{3.298937in}{0.860404in}}%
\pgfpathlineto{\pgfqpoint{3.301719in}{0.854362in}}%
\pgfpathlineto{\pgfqpoint{3.304295in}{0.854791in}}%
\pgfpathlineto{\pgfqpoint{3.307104in}{0.852154in}}%
\pgfpathlineto{\pgfqpoint{3.309652in}{0.849821in}}%
\pgfpathlineto{\pgfqpoint{3.312480in}{0.854731in}}%
\pgfpathlineto{\pgfqpoint{3.315008in}{0.850362in}}%
\pgfpathlineto{\pgfqpoint{3.317688in}{0.852324in}}%
\pgfpathlineto{\pgfqpoint{3.320366in}{0.851613in}}%
\pgfpathlineto{\pgfqpoint{3.323049in}{0.851147in}}%
\pgfpathlineto{\pgfqpoint{3.325860in}{0.849952in}}%
\pgfpathlineto{\pgfqpoint{3.328401in}{0.853185in}}%
\pgfpathlineto{\pgfqpoint{3.331183in}{0.851630in}}%
\pgfpathlineto{\pgfqpoint{3.333758in}{0.853583in}}%
\pgfpathlineto{\pgfqpoint{3.336541in}{0.851229in}}%
\pgfpathlineto{\pgfqpoint{3.339101in}{0.853347in}}%
\pgfpathlineto{\pgfqpoint{3.341893in}{0.852021in}}%
\pgfpathlineto{\pgfqpoint{3.344468in}{0.852436in}}%
\pgfpathlineto{\pgfqpoint{3.347139in}{0.851447in}}%
\pgfpathlineto{\pgfqpoint{3.349828in}{0.849646in}}%
\pgfpathlineto{\pgfqpoint{3.352505in}{0.852599in}}%
\pgfpathlineto{\pgfqpoint{3.355177in}{0.851029in}}%
\pgfpathlineto{\pgfqpoint{3.357862in}{0.851092in}}%
\pgfpathlineto{\pgfqpoint{3.360620in}{0.851855in}}%
\pgfpathlineto{\pgfqpoint{3.363221in}{0.850619in}}%
\pgfpathlineto{\pgfqpoint{3.365996in}{0.852587in}}%
\pgfpathlineto{\pgfqpoint{3.368577in}{0.851955in}}%
\pgfpathlineto{\pgfqpoint{3.371357in}{0.853843in}}%
\pgfpathlineto{\pgfqpoint{3.373921in}{0.853093in}}%
\pgfpathlineto{\pgfqpoint{3.376735in}{0.853171in}}%
\pgfpathlineto{\pgfqpoint{3.379290in}{0.852597in}}%
\pgfpathlineto{\pgfqpoint{3.381959in}{0.853335in}}%
\pgfpathlineto{\pgfqpoint{3.384647in}{0.850895in}}%
\pgfpathlineto{\pgfqpoint{3.387309in}{0.852021in}}%
\pgfpathlineto{\pgfqpoint{3.390102in}{0.851334in}}%
\pgfpathlineto{\pgfqpoint{3.392681in}{0.852446in}}%
\pgfpathlineto{\pgfqpoint{3.395461in}{0.854517in}}%
\pgfpathlineto{\pgfqpoint{3.398037in}{0.853142in}}%
\pgfpathlineto{\pgfqpoint{3.400783in}{0.853225in}}%
\pgfpathlineto{\pgfqpoint{3.403394in}{0.850548in}}%
\pgfpathlineto{\pgfqpoint{3.406202in}{0.854140in}}%
\pgfpathlineto{\pgfqpoint{3.408752in}{0.857213in}}%
\pgfpathlineto{\pgfqpoint{3.411431in}{0.855716in}}%
\pgfpathlineto{\pgfqpoint{3.414109in}{0.854172in}}%
\pgfpathlineto{\pgfqpoint{3.416780in}{0.855643in}}%
\pgfpathlineto{\pgfqpoint{3.419455in}{0.858382in}}%
\pgfpathlineto{\pgfqpoint{3.422142in}{0.856126in}}%
\pgfpathlineto{\pgfqpoint{3.424887in}{0.859419in}}%
\pgfpathlineto{\pgfqpoint{3.427501in}{0.857034in}}%
\pgfpathlineto{\pgfqpoint{3.430313in}{0.857045in}}%
\pgfpathlineto{\pgfqpoint{3.432851in}{0.859024in}}%
\pgfpathlineto{\pgfqpoint{3.435635in}{0.857128in}}%
\pgfpathlineto{\pgfqpoint{3.438210in}{0.855559in}}%
\pgfpathlineto{\pgfqpoint{3.440996in}{0.856805in}}%
\pgfpathlineto{\pgfqpoint{3.443574in}{0.853863in}}%
\pgfpathlineto{\pgfqpoint{3.446257in}{0.856192in}}%
\pgfpathlineto{\pgfqpoint{3.448926in}{0.856447in}}%
\pgfpathlineto{\pgfqpoint{3.451597in}{0.857180in}}%
\pgfpathlineto{\pgfqpoint{3.454285in}{0.855416in}}%
\pgfpathlineto{\pgfqpoint{3.456960in}{0.854361in}}%
\pgfpathlineto{\pgfqpoint{3.459695in}{0.855888in}}%
\pgfpathlineto{\pgfqpoint{3.462321in}{0.853052in}}%
\pgfpathlineto{\pgfqpoint{3.465072in}{0.849424in}}%
\pgfpathlineto{\pgfqpoint{3.467678in}{0.848935in}}%
\pgfpathlineto{\pgfqpoint{3.470466in}{0.849975in}}%
\pgfpathlineto{\pgfqpoint{3.473021in}{0.848818in}}%
\pgfpathlineto{\pgfqpoint{3.475821in}{0.848148in}}%
\pgfpathlineto{\pgfqpoint{3.478378in}{0.845035in}}%
\pgfpathlineto{\pgfqpoint{3.481072in}{0.849084in}}%
\pgfpathlineto{\pgfqpoint{3.483744in}{0.850566in}}%
\pgfpathlineto{\pgfqpoint{3.486442in}{0.849983in}}%
\pgfpathlineto{\pgfqpoint{3.489223in}{0.850285in}}%
\pgfpathlineto{\pgfqpoint{3.491783in}{0.857815in}}%
\pgfpathlineto{\pgfqpoint{3.494581in}{0.851444in}}%
\pgfpathlineto{\pgfqpoint{3.497139in}{0.859194in}}%
\pgfpathlineto{\pgfqpoint{3.499909in}{0.860911in}}%
\pgfpathlineto{\pgfqpoint{3.502488in}{0.859190in}}%
\pgfpathlineto{\pgfqpoint{3.505262in}{0.857013in}}%
\pgfpathlineto{\pgfqpoint{3.507840in}{0.863488in}}%
\pgfpathlineto{\pgfqpoint{3.510533in}{0.866628in}}%
\pgfpathlineto{\pgfqpoint{3.513209in}{0.862546in}}%
\pgfpathlineto{\pgfqpoint{3.515884in}{0.857262in}}%
\pgfpathlineto{\pgfqpoint{3.518565in}{0.856443in}}%
\pgfpathlineto{\pgfqpoint{3.521244in}{0.856260in}}%
\pgfpathlineto{\pgfqpoint{3.524041in}{0.855363in}}%
\pgfpathlineto{\pgfqpoint{3.526601in}{0.855552in}}%
\pgfpathlineto{\pgfqpoint{3.529327in}{0.855617in}}%
\pgfpathlineto{\pgfqpoint{3.531955in}{0.853918in}}%
\pgfpathlineto{\pgfqpoint{3.534783in}{0.853721in}}%
\pgfpathlineto{\pgfqpoint{3.537309in}{0.852698in}}%
\pgfpathlineto{\pgfqpoint{3.540093in}{0.852985in}}%
\pgfpathlineto{\pgfqpoint{3.542656in}{0.855168in}}%
\pgfpathlineto{\pgfqpoint{3.545349in}{0.857929in}}%
\pgfpathlineto{\pgfqpoint{3.548029in}{0.856138in}}%
\pgfpathlineto{\pgfqpoint{3.550713in}{0.852187in}}%
\pgfpathlineto{\pgfqpoint{3.553498in}{0.850716in}}%
\pgfpathlineto{\pgfqpoint{3.556061in}{0.852774in}}%
\pgfpathlineto{\pgfqpoint{3.558853in}{0.854041in}}%
\pgfpathlineto{\pgfqpoint{3.561420in}{0.853393in}}%
\pgfpathlineto{\pgfqpoint{3.564188in}{0.856773in}}%
\pgfpathlineto{\pgfqpoint{3.566774in}{0.856205in}}%
\pgfpathlineto{\pgfqpoint{3.569584in}{0.855987in}}%
\pgfpathlineto{\pgfqpoint{3.572126in}{0.857701in}}%
\pgfpathlineto{\pgfqpoint{3.574814in}{0.852904in}}%
\pgfpathlineto{\pgfqpoint{3.577487in}{0.852946in}}%
\pgfpathlineto{\pgfqpoint{3.580191in}{0.851524in}}%
\pgfpathlineto{\pgfqpoint{3.582851in}{0.854110in}}%
\pgfpathlineto{\pgfqpoint{3.585532in}{0.854357in}}%
\pgfpathlineto{\pgfqpoint{3.588258in}{0.857016in}}%
\pgfpathlineto{\pgfqpoint{3.590883in}{0.856734in}}%
\pgfpathlineto{\pgfqpoint{3.593620in}{0.857692in}}%
\pgfpathlineto{\pgfqpoint{3.596240in}{0.855024in}}%
\pgfpathlineto{\pgfqpoint{3.598998in}{0.858637in}}%
\pgfpathlineto{\pgfqpoint{3.601590in}{0.857121in}}%
\pgfpathlineto{\pgfqpoint{3.604387in}{0.853757in}}%
\pgfpathlineto{\pgfqpoint{3.606951in}{0.853502in}}%
\pgfpathlineto{\pgfqpoint{3.609632in}{0.855474in}}%
\pgfpathlineto{\pgfqpoint{3.612311in}{0.857225in}}%
\pgfpathlineto{\pgfqpoint{3.614982in}{0.855920in}}%
\pgfpathlineto{\pgfqpoint{3.617667in}{0.857139in}}%
\pgfpathlineto{\pgfqpoint{3.620345in}{0.856035in}}%
\pgfpathlineto{\pgfqpoint{3.623165in}{0.856004in}}%
\pgfpathlineto{\pgfqpoint{3.625689in}{0.856278in}}%
\pgfpathlineto{\pgfqpoint{3.628460in}{0.857956in}}%
\pgfpathlineto{\pgfqpoint{3.631058in}{0.855150in}}%
\pgfpathlineto{\pgfqpoint{3.633858in}{0.853282in}}%
\pgfpathlineto{\pgfqpoint{3.636413in}{0.857983in}}%
\pgfpathlineto{\pgfqpoint{3.639207in}{0.856289in}}%
\pgfpathlineto{\pgfqpoint{3.641773in}{0.859552in}}%
\pgfpathlineto{\pgfqpoint{3.644452in}{0.865744in}}%
\pgfpathlineto{\pgfqpoint{3.647130in}{0.861367in}}%
\pgfpathlineto{\pgfqpoint{3.649837in}{0.863037in}}%
\pgfpathlineto{\pgfqpoint{3.652628in}{0.864228in}}%
\pgfpathlineto{\pgfqpoint{3.655165in}{0.859667in}}%
\pgfpathlineto{\pgfqpoint{3.657917in}{0.862414in}}%
\pgfpathlineto{\pgfqpoint{3.660515in}{0.860343in}}%
\pgfpathlineto{\pgfqpoint{3.663276in}{0.863551in}}%
\pgfpathlineto{\pgfqpoint{3.665864in}{0.871068in}}%
\pgfpathlineto{\pgfqpoint{3.668665in}{0.869533in}}%
\pgfpathlineto{\pgfqpoint{3.671232in}{0.865612in}}%
\pgfpathlineto{\pgfqpoint{3.673911in}{0.858522in}}%
\pgfpathlineto{\pgfqpoint{3.676591in}{0.856850in}}%
\pgfpathlineto{\pgfqpoint{3.679273in}{0.863149in}}%
\pgfpathlineto{\pgfqpoint{3.681948in}{0.868696in}}%
\pgfpathlineto{\pgfqpoint{3.684620in}{0.870391in}}%
\pgfpathlineto{\pgfqpoint{3.687442in}{0.865843in}}%
\pgfpathlineto{\pgfqpoint{3.689983in}{0.865519in}}%
\pgfpathlineto{\pgfqpoint{3.692765in}{0.871713in}}%
\pgfpathlineto{\pgfqpoint{3.695331in}{0.867346in}}%
\pgfpathlineto{\pgfqpoint{3.698125in}{0.858729in}}%
\pgfpathlineto{\pgfqpoint{3.700684in}{0.855980in}}%
\pgfpathlineto{\pgfqpoint{3.703460in}{0.861855in}}%
\pgfpathlineto{\pgfqpoint{3.706053in}{0.861195in}}%
\pgfpathlineto{\pgfqpoint{3.708729in}{0.858931in}}%
\pgfpathlineto{\pgfqpoint{3.711410in}{0.858442in}}%
\pgfpathlineto{\pgfqpoint{3.714086in}{0.859910in}}%
\pgfpathlineto{\pgfqpoint{3.716875in}{0.858141in}}%
\pgfpathlineto{\pgfqpoint{3.719446in}{0.856395in}}%
\pgfpathlineto{\pgfqpoint{3.722228in}{0.863077in}}%
\pgfpathlineto{\pgfqpoint{3.724804in}{0.864638in}}%
\pgfpathlineto{\pgfqpoint{3.727581in}{0.863162in}}%
\pgfpathlineto{\pgfqpoint{3.730158in}{0.858992in}}%
\pgfpathlineto{\pgfqpoint{3.732950in}{0.856395in}}%
\pgfpathlineto{\pgfqpoint{3.735509in}{0.857851in}}%
\pgfpathlineto{\pgfqpoint{3.738194in}{0.855777in}}%
\pgfpathlineto{\pgfqpoint{3.740874in}{0.856448in}}%
\pgfpathlineto{\pgfqpoint{3.743548in}{0.857785in}}%
\pgfpathlineto{\pgfqpoint{3.746229in}{0.854041in}}%
\pgfpathlineto{\pgfqpoint{3.748903in}{0.857031in}}%
\pgfpathlineto{\pgfqpoint{3.751728in}{0.854355in}}%
\pgfpathlineto{\pgfqpoint{3.754265in}{0.855174in}}%
\pgfpathlineto{\pgfqpoint{3.757065in}{0.855251in}}%
\pgfpathlineto{\pgfqpoint{3.759622in}{0.855342in}}%
\pgfpathlineto{\pgfqpoint{3.762389in}{0.856444in}}%
\pgfpathlineto{\pgfqpoint{3.764966in}{0.890407in}}%
\pgfpathlineto{\pgfqpoint{3.767782in}{0.894595in}}%
\pgfpathlineto{\pgfqpoint{3.770323in}{0.963860in}}%
\pgfpathlineto{\pgfqpoint{3.773014in}{1.004076in}}%
\pgfpathlineto{\pgfqpoint{3.775691in}{1.059738in}}%
\pgfpathlineto{\pgfqpoint{3.778370in}{1.032408in}}%
\pgfpathlineto{\pgfqpoint{3.781046in}{0.997659in}}%
\pgfpathlineto{\pgfqpoint{3.783725in}{0.973406in}}%
\pgfpathlineto{\pgfqpoint{3.786504in}{0.958755in}}%
\pgfpathlineto{\pgfqpoint{3.789084in}{0.947548in}}%
\pgfpathlineto{\pgfqpoint{3.791897in}{0.940002in}}%
\pgfpathlineto{\pgfqpoint{3.794435in}{0.945666in}}%
\pgfpathlineto{\pgfqpoint{3.797265in}{0.937543in}}%
\pgfpathlineto{\pgfqpoint{3.799797in}{0.920711in}}%
\pgfpathlineto{\pgfqpoint{3.802569in}{0.917390in}}%
\pgfpathlineto{\pgfqpoint{3.805145in}{0.906511in}}%
\pgfpathlineto{\pgfqpoint{3.807832in}{0.897935in}}%
\pgfpathlineto{\pgfqpoint{3.810510in}{0.888743in}}%
\pgfpathlineto{\pgfqpoint{3.813172in}{0.890777in}}%
\pgfpathlineto{\pgfqpoint{3.815983in}{0.886917in}}%
\pgfpathlineto{\pgfqpoint{3.818546in}{0.881511in}}%
\pgfpathlineto{\pgfqpoint{3.821315in}{0.882033in}}%
\pgfpathlineto{\pgfqpoint{3.823903in}{0.874414in}}%
\pgfpathlineto{\pgfqpoint{3.826679in}{0.874050in}}%
\pgfpathlineto{\pgfqpoint{3.829252in}{0.870343in}}%
\pgfpathlineto{\pgfqpoint{3.832053in}{0.868051in}}%
\pgfpathlineto{\pgfqpoint{3.834616in}{0.867210in}}%
\pgfpathlineto{\pgfqpoint{3.837286in}{0.866970in}}%
\pgfpathlineto{\pgfqpoint{3.839960in}{0.863899in}}%
\pgfpathlineto{\pgfqpoint{3.842641in}{0.863537in}}%
\pgfpathlineto{\pgfqpoint{3.845329in}{0.867279in}}%
\pgfpathlineto{\pgfqpoint{3.848005in}{0.865243in}}%
\pgfpathlineto{\pgfqpoint{3.850814in}{0.860236in}}%
\pgfpathlineto{\pgfqpoint{3.853358in}{0.859826in}}%
\pgfpathlineto{\pgfqpoint{3.856100in}{0.856011in}}%
\pgfpathlineto{\pgfqpoint{3.858720in}{0.853036in}}%
\pgfpathlineto{\pgfqpoint{3.861561in}{0.855819in}}%
\pgfpathlineto{\pgfqpoint{3.864073in}{0.854257in}}%
\pgfpathlineto{\pgfqpoint{3.866815in}{0.854654in}}%
\pgfpathlineto{\pgfqpoint{3.869435in}{0.851981in}}%
\pgfpathlineto{\pgfqpoint{3.872114in}{0.854626in}}%
\pgfpathlineto{\pgfqpoint{3.874790in}{0.855534in}}%
\pgfpathlineto{\pgfqpoint{3.877466in}{0.855285in}}%
\pgfpathlineto{\pgfqpoint{3.880237in}{0.856614in}}%
\pgfpathlineto{\pgfqpoint{3.882850in}{0.856017in}}%
\pgfpathlineto{\pgfqpoint{3.885621in}{0.858718in}}%
\pgfpathlineto{\pgfqpoint{3.888188in}{0.855576in}}%
\pgfpathlineto{\pgfqpoint{3.890926in}{0.849795in}}%
\pgfpathlineto{\pgfqpoint{3.893541in}{0.843639in}}%
\pgfpathlineto{\pgfqpoint{3.896345in}{0.848088in}}%
\pgfpathlineto{\pgfqpoint{3.898891in}{0.848542in}}%
\pgfpathlineto{\pgfqpoint{3.901573in}{0.850902in}}%
\pgfpathlineto{\pgfqpoint{3.904252in}{0.851793in}}%
\pgfpathlineto{\pgfqpoint{3.906918in}{0.852132in}}%
\pgfpathlineto{\pgfqpoint{3.909602in}{0.853528in}}%
\pgfpathlineto{\pgfqpoint{3.912296in}{0.853121in}}%
\pgfpathlineto{\pgfqpoint{3.915107in}{0.851112in}}%
\pgfpathlineto{\pgfqpoint{3.917646in}{0.850960in}}%
\pgfpathlineto{\pgfqpoint{3.920412in}{0.849394in}}%
\pgfpathlineto{\pgfqpoint{3.923005in}{0.854730in}}%
\pgfpathlineto{\pgfqpoint{3.925778in}{0.851669in}}%
\pgfpathlineto{\pgfqpoint{3.928347in}{0.850557in}}%
\pgfpathlineto{\pgfqpoint{3.931202in}{0.848302in}}%
\pgfpathlineto{\pgfqpoint{3.933714in}{0.849485in}}%
\pgfpathlineto{\pgfqpoint{3.936395in}{0.854481in}}%
\pgfpathlineto{\pgfqpoint{3.939075in}{0.852486in}}%
\pgfpathlineto{\pgfqpoint{3.941778in}{0.851017in}}%
\pgfpathlineto{\pgfqpoint{3.944431in}{0.852222in}}%
\pgfpathlineto{\pgfqpoint{3.947101in}{0.850522in}}%
\pgfpathlineto{\pgfqpoint{3.949894in}{0.853582in}}%
\pgfpathlineto{\pgfqpoint{3.952464in}{0.843040in}}%
\pgfpathlineto{\pgfqpoint{3.955211in}{0.846720in}}%
\pgfpathlineto{\pgfqpoint{3.957823in}{0.849493in}}%
\pgfpathlineto{\pgfqpoint{3.960635in}{0.848891in}}%
\pgfpathlineto{\pgfqpoint{3.963176in}{0.853922in}}%
\pgfpathlineto{\pgfqpoint{3.966013in}{0.852075in}}%
\pgfpathlineto{\pgfqpoint{3.968523in}{0.852483in}}%
\pgfpathlineto{\pgfqpoint{3.971250in}{0.848722in}}%
\pgfpathlineto{\pgfqpoint{3.973885in}{0.850051in}}%
\pgfpathlineto{\pgfqpoint{3.976563in}{0.846974in}}%
\pgfpathlineto{\pgfqpoint{3.979389in}{0.850815in}}%
\pgfpathlineto{\pgfqpoint{3.981929in}{0.849187in}}%
\pgfpathlineto{\pgfqpoint{3.984714in}{0.849214in}}%
\pgfpathlineto{\pgfqpoint{3.987270in}{0.847674in}}%
\pgfpathlineto{\pgfqpoint{3.990055in}{0.851210in}}%
\pgfpathlineto{\pgfqpoint{3.992642in}{0.852600in}}%
\pgfpathlineto{\pgfqpoint{3.995417in}{0.846848in}}%
\pgfpathlineto{\pgfqpoint{3.997990in}{0.852002in}}%
\pgfpathlineto{\pgfqpoint{4.000674in}{0.849532in}}%
\pgfpathlineto{\pgfqpoint{4.003348in}{0.849974in}}%
\pgfpathlineto{\pgfqpoint{4.006034in}{0.852304in}}%
\pgfpathlineto{\pgfqpoint{4.008699in}{0.850288in}}%
\pgfpathlineto{\pgfqpoint{4.011394in}{0.855540in}}%
\pgfpathlineto{\pgfqpoint{4.014186in}{0.852313in}}%
\pgfpathlineto{\pgfqpoint{4.016744in}{0.852220in}}%
\pgfpathlineto{\pgfqpoint{4.019518in}{0.851962in}}%
\pgfpathlineto{\pgfqpoint{4.022097in}{0.851643in}}%
\pgfpathlineto{\pgfqpoint{4.024868in}{0.852499in}}%
\pgfpathlineto{\pgfqpoint{4.027447in}{0.852838in}}%
\pgfpathlineto{\pgfqpoint{4.030229in}{0.853314in}}%
\pgfpathlineto{\pgfqpoint{4.032817in}{0.852675in}}%
\pgfpathlineto{\pgfqpoint{4.035492in}{0.857714in}}%
\pgfpathlineto{\pgfqpoint{4.038174in}{0.855907in}}%
\pgfpathlineto{\pgfqpoint{4.040852in}{0.857382in}}%
\pgfpathlineto{\pgfqpoint{4.043667in}{0.851946in}}%
\pgfpathlineto{\pgfqpoint{4.046210in}{0.853943in}}%
\pgfpathlineto{\pgfqpoint{4.049006in}{0.854021in}}%
\pgfpathlineto{\pgfqpoint{4.051557in}{0.853851in}}%
\pgfpathlineto{\pgfqpoint{4.054326in}{0.852267in}}%
\pgfpathlineto{\pgfqpoint{4.056911in}{0.855307in}}%
\pgfpathlineto{\pgfqpoint{4.059702in}{0.857081in}}%
\pgfpathlineto{\pgfqpoint{4.062266in}{0.854920in}}%
\pgfpathlineto{\pgfqpoint{4.064957in}{0.856104in}}%
\pgfpathlineto{\pgfqpoint{4.067636in}{0.854794in}}%
\pgfpathlineto{\pgfqpoint{4.070313in}{0.857419in}}%
\pgfpathlineto{\pgfqpoint{4.072985in}{0.855675in}}%
\pgfpathlineto{\pgfqpoint{4.075705in}{0.857430in}}%
\pgfpathlineto{\pgfqpoint{4.078471in}{0.860937in}}%
\pgfpathlineto{\pgfqpoint{4.081018in}{0.859748in}}%
\pgfpathlineto{\pgfqpoint{4.083870in}{0.857504in}}%
\pgfpathlineto{\pgfqpoint{4.086385in}{0.857621in}}%
\pgfpathlineto{\pgfqpoint{4.089159in}{0.857663in}}%
\pgfpathlineto{\pgfqpoint{4.091729in}{0.855867in}}%
\pgfpathlineto{\pgfqpoint{4.094527in}{0.857217in}}%
\pgfpathlineto{\pgfqpoint{4.097092in}{0.851896in}}%
\pgfpathlineto{\pgfqpoint{4.099777in}{0.854799in}}%
\pgfpathlineto{\pgfqpoint{4.102456in}{0.855994in}}%
\pgfpathlineto{\pgfqpoint{4.105185in}{0.855765in}}%
\pgfpathlineto{\pgfqpoint{4.107814in}{0.853232in}}%
\pgfpathlineto{\pgfqpoint{4.110488in}{0.853731in}}%
\pgfpathlineto{\pgfqpoint{4.113252in}{0.853178in}}%
\pgfpathlineto{\pgfqpoint{4.115844in}{0.853940in}}%
\pgfpathlineto{\pgfqpoint{4.118554in}{0.856406in}}%
\pgfpathlineto{\pgfqpoint{4.121205in}{0.856494in}}%
\pgfpathlineto{\pgfqpoint{4.124019in}{0.853396in}}%
\pgfpathlineto{\pgfqpoint{4.126553in}{0.854165in}}%
\pgfpathlineto{\pgfqpoint{4.129349in}{0.854084in}}%
\pgfpathlineto{\pgfqpoint{4.131920in}{0.853412in}}%
\pgfpathlineto{\pgfqpoint{4.134615in}{0.854840in}}%
\pgfpathlineto{\pgfqpoint{4.137272in}{0.858026in}}%
\pgfpathlineto{\pgfqpoint{4.139963in}{0.856583in}}%
\pgfpathlineto{\pgfqpoint{4.142713in}{0.852638in}}%
\pgfpathlineto{\pgfqpoint{4.145310in}{0.851617in}}%
\pgfpathlineto{\pgfqpoint{4.148082in}{0.852785in}}%
\pgfpathlineto{\pgfqpoint{4.150665in}{0.853704in}}%
\pgfpathlineto{\pgfqpoint{4.153423in}{0.851404in}}%
\pgfpathlineto{\pgfqpoint{4.156016in}{0.855063in}}%
\pgfpathlineto{\pgfqpoint{4.158806in}{0.856091in}}%
\pgfpathlineto{\pgfqpoint{4.161380in}{0.856175in}}%
\pgfpathlineto{\pgfqpoint{4.164059in}{0.852293in}}%
\pgfpathlineto{\pgfqpoint{4.166737in}{0.856517in}}%
\pgfpathlineto{\pgfqpoint{4.169415in}{0.855531in}}%
\pgfpathlineto{\pgfqpoint{4.172093in}{0.855305in}}%
\pgfpathlineto{\pgfqpoint{4.174770in}{0.856486in}}%
\pgfpathlineto{\pgfqpoint{4.177593in}{0.855720in}}%
\pgfpathlineto{\pgfqpoint{4.180129in}{0.852419in}}%
\pgfpathlineto{\pgfqpoint{4.182899in}{0.853484in}}%
\pgfpathlineto{\pgfqpoint{4.185481in}{0.852740in}}%
\pgfpathlineto{\pgfqpoint{4.188318in}{0.849861in}}%
\pgfpathlineto{\pgfqpoint{4.190842in}{0.850587in}}%
\pgfpathlineto{\pgfqpoint{4.193638in}{0.850663in}}%
\pgfpathlineto{\pgfqpoint{4.196186in}{0.854569in}}%
\pgfpathlineto{\pgfqpoint{4.198878in}{0.853504in}}%
\pgfpathlineto{\pgfqpoint{4.201542in}{0.852598in}}%
\pgfpathlineto{\pgfqpoint{4.204240in}{0.854517in}}%
\pgfpathlineto{\pgfqpoint{4.207076in}{0.853694in}}%
\pgfpathlineto{\pgfqpoint{4.209597in}{0.854480in}}%
\pgfpathlineto{\pgfqpoint{4.212383in}{0.852622in}}%
\pgfpathlineto{\pgfqpoint{4.214948in}{0.850971in}}%
\pgfpathlineto{\pgfqpoint{4.217694in}{0.852794in}}%
\pgfpathlineto{\pgfqpoint{4.220304in}{0.871748in}}%
\pgfpathlineto{\pgfqpoint{4.223082in}{0.893441in}}%
\pgfpathlineto{\pgfqpoint{4.225654in}{0.914485in}}%
\pgfpathlineto{\pgfqpoint{4.228331in}{0.943121in}}%
\pgfpathlineto{\pgfqpoint{4.231013in}{0.932837in}}%
\pgfpathlineto{\pgfqpoint{4.233691in}{0.917806in}}%
\pgfpathlineto{\pgfqpoint{4.236375in}{0.902024in}}%
\pgfpathlineto{\pgfqpoint{4.239084in}{0.897351in}}%
\pgfpathlineto{\pgfqpoint{4.241900in}{0.891887in}}%
\pgfpathlineto{\pgfqpoint{4.244394in}{0.890012in}}%
\pgfpathlineto{\pgfqpoint{4.247225in}{0.884316in}}%
\pgfpathlineto{\pgfqpoint{4.249767in}{0.878241in}}%
\pgfpathlineto{\pgfqpoint{4.252581in}{0.874321in}}%
\pgfpathlineto{\pgfqpoint{4.255120in}{0.869228in}}%
\pgfpathlineto{\pgfqpoint{4.257958in}{0.868185in}}%
\pgfpathlineto{\pgfqpoint{4.260477in}{0.867903in}}%
\pgfpathlineto{\pgfqpoint{4.263157in}{0.861932in}}%
\pgfpathlineto{\pgfqpoint{4.265824in}{0.861556in}}%
\pgfpathlineto{\pgfqpoint{4.268590in}{0.860796in}}%
\pgfpathlineto{\pgfqpoint{4.271187in}{0.862958in}}%
\pgfpathlineto{\pgfqpoint{4.273874in}{0.862918in}}%
\pgfpathlineto{\pgfqpoint{4.276635in}{0.861134in}}%
\pgfpathlineto{\pgfqpoint{4.279212in}{0.864638in}}%
\pgfpathlineto{\pgfqpoint{4.282000in}{0.879138in}}%
\pgfpathlineto{\pgfqpoint{4.284586in}{0.879057in}}%
\pgfpathlineto{\pgfqpoint{4.287399in}{0.867893in}}%
\pgfpathlineto{\pgfqpoint{4.289936in}{0.868814in}}%
\pgfpathlineto{\pgfqpoint{4.292786in}{0.867106in}}%
\pgfpathlineto{\pgfqpoint{4.295299in}{0.862587in}}%
\pgfpathlineto{\pgfqpoint{4.297977in}{0.866357in}}%
\pgfpathlineto{\pgfqpoint{4.300656in}{0.865671in}}%
\pgfpathlineto{\pgfqpoint{4.303357in}{0.864440in}}%
\pgfpathlineto{\pgfqpoint{4.306118in}{0.862092in}}%
\pgfpathlineto{\pgfqpoint{4.308691in}{0.861691in}}%
\pgfpathlineto{\pgfqpoint{4.311494in}{0.861810in}}%
\pgfpathlineto{\pgfqpoint{4.314032in}{0.860984in}}%
\pgfpathlineto{\pgfqpoint{4.316856in}{0.859888in}}%
\pgfpathlineto{\pgfqpoint{4.319405in}{0.857146in}}%
\pgfpathlineto{\pgfqpoint{4.322181in}{0.856214in}}%
\pgfpathlineto{\pgfqpoint{4.324760in}{0.859174in}}%
\pgfpathlineto{\pgfqpoint{4.327440in}{0.859350in}}%
\pgfpathlineto{\pgfqpoint{4.330118in}{0.858665in}}%
\pgfpathlineto{\pgfqpoint{4.332796in}{0.856191in}}%
\pgfpathlineto{\pgfqpoint{4.335463in}{0.863608in}}%
\pgfpathlineto{\pgfqpoint{4.338154in}{0.862623in}}%
\pgfpathlineto{\pgfqpoint{4.340976in}{0.862714in}}%
\pgfpathlineto{\pgfqpoint{4.343510in}{0.861445in}}%
\pgfpathlineto{\pgfqpoint{4.346263in}{0.859456in}}%
\pgfpathlineto{\pgfqpoint{4.348868in}{0.860151in}}%
\pgfpathlineto{\pgfqpoint{4.351645in}{0.857557in}}%
\pgfpathlineto{\pgfqpoint{4.354224in}{0.856610in}}%
\pgfpathlineto{\pgfqpoint{4.357014in}{0.858342in}}%
\pgfpathlineto{\pgfqpoint{4.359582in}{0.855223in}}%
\pgfpathlineto{\pgfqpoint{4.362270in}{0.854638in}}%
\pgfpathlineto{\pgfqpoint{4.364936in}{0.860409in}}%
\pgfpathlineto{\pgfqpoint{4.367646in}{0.860140in}}%
\pgfpathlineto{\pgfqpoint{4.370437in}{0.856672in}}%
\pgfpathlineto{\pgfqpoint{4.372976in}{0.856226in}}%
\pgfpathlineto{\pgfqpoint{4.375761in}{0.853916in}}%
\pgfpathlineto{\pgfqpoint{4.378329in}{0.860118in}}%
\pgfpathlineto{\pgfqpoint{4.381097in}{0.857369in}}%
\pgfpathlineto{\pgfqpoint{4.383674in}{0.860156in}}%
\pgfpathlineto{\pgfqpoint{4.386431in}{0.857829in}}%
\pgfpathlineto{\pgfqpoint{4.389044in}{0.856745in}}%
\pgfpathlineto{\pgfqpoint{4.391721in}{0.856143in}}%
\pgfpathlineto{\pgfqpoint{4.394400in}{0.857905in}}%
\pgfpathlineto{\pgfqpoint{4.397076in}{0.860186in}}%
\pgfpathlineto{\pgfqpoint{4.399745in}{0.869598in}}%
\pgfpathlineto{\pgfqpoint{4.402468in}{0.872399in}}%
\pgfpathlineto{\pgfqpoint{4.405234in}{0.875692in}}%
\pgfpathlineto{\pgfqpoint{4.407788in}{0.874219in}}%
\pgfpathlineto{\pgfqpoint{4.410587in}{0.875264in}}%
\pgfpathlineto{\pgfqpoint{4.413149in}{0.868914in}}%
\pgfpathlineto{\pgfqpoint{4.415932in}{0.863145in}}%
\pgfpathlineto{\pgfqpoint{4.418506in}{0.864518in}}%
\pgfpathlineto{\pgfqpoint{4.421292in}{0.862641in}}%
\pgfpathlineto{\pgfqpoint{4.423863in}{0.865213in}}%
\pgfpathlineto{\pgfqpoint{4.426534in}{0.865650in}}%
\pgfpathlineto{\pgfqpoint{4.429220in}{0.863430in}}%
\pgfpathlineto{\pgfqpoint{4.431901in}{0.858117in}}%
\pgfpathlineto{\pgfqpoint{4.434569in}{0.858272in}}%
\pgfpathlineto{\pgfqpoint{4.437253in}{0.857021in}}%
\pgfpathlineto{\pgfqpoint{4.440041in}{0.852730in}}%
\pgfpathlineto{\pgfqpoint{4.442611in}{0.852587in}}%
\pgfpathlineto{\pgfqpoint{4.445423in}{0.853235in}}%
\pgfpathlineto{\pgfqpoint{4.447965in}{0.856864in}}%
\pgfpathlineto{\pgfqpoint{4.450767in}{0.861037in}}%
\pgfpathlineto{\pgfqpoint{4.453312in}{0.857376in}}%
\pgfpathlineto{\pgfqpoint{4.456138in}{0.855979in}}%
\pgfpathlineto{\pgfqpoint{4.458681in}{0.860135in}}%
\pgfpathlineto{\pgfqpoint{4.461367in}{0.859471in}}%
\pgfpathlineto{\pgfqpoint{4.464029in}{0.855945in}}%
\pgfpathlineto{\pgfqpoint{4.466717in}{0.855798in}}%
\pgfpathlineto{\pgfqpoint{4.469492in}{0.853873in}}%
\pgfpathlineto{\pgfqpoint{4.472059in}{0.853885in}}%
\pgfpathlineto{\pgfqpoint{4.474861in}{0.853723in}}%
\pgfpathlineto{\pgfqpoint{4.477430in}{0.854611in}}%
\pgfpathlineto{\pgfqpoint{4.480201in}{0.854042in}}%
\pgfpathlineto{\pgfqpoint{4.482778in}{0.857420in}}%
\pgfpathlineto{\pgfqpoint{4.485581in}{0.855050in}}%
\pgfpathlineto{\pgfqpoint{4.488130in}{0.857543in}}%
\pgfpathlineto{\pgfqpoint{4.490822in}{0.857476in}}%
\pgfpathlineto{\pgfqpoint{4.493492in}{0.855768in}}%
\pgfpathlineto{\pgfqpoint{4.496167in}{0.856937in}}%
\pgfpathlineto{\pgfqpoint{4.498850in}{0.852646in}}%
\pgfpathlineto{\pgfqpoint{4.501529in}{0.856767in}}%
\pgfpathlineto{\pgfqpoint{4.504305in}{0.857116in}}%
\pgfpathlineto{\pgfqpoint{4.506893in}{0.855869in}}%
\pgfpathlineto{\pgfqpoint{4.509643in}{0.858884in}}%
\pgfpathlineto{\pgfqpoint{4.512246in}{0.859226in}}%
\pgfpathlineto{\pgfqpoint{4.515080in}{0.857325in}}%
\pgfpathlineto{\pgfqpoint{4.517598in}{0.854376in}}%
\pgfpathlineto{\pgfqpoint{4.520345in}{0.852669in}}%
\pgfpathlineto{\pgfqpoint{4.522962in}{0.851897in}}%
\pgfpathlineto{\pgfqpoint{4.525640in}{0.859410in}}%
\pgfpathlineto{\pgfqpoint{4.528307in}{0.861782in}}%
\pgfpathlineto{\pgfqpoint{4.530990in}{0.865450in}}%
\pgfpathlineto{\pgfqpoint{4.533764in}{0.870514in}}%
\pgfpathlineto{\pgfqpoint{4.536400in}{0.875969in}}%
\pgfpathlineto{\pgfqpoint{4.539144in}{0.873712in}}%
\pgfpathlineto{\pgfqpoint{4.541711in}{0.864338in}}%
\pgfpathlineto{\pgfqpoint{4.544464in}{0.864654in}}%
\pgfpathlineto{\pgfqpoint{4.547064in}{0.865664in}}%
\pgfpathlineto{\pgfqpoint{4.549822in}{0.860485in}}%
\pgfpathlineto{\pgfqpoint{4.552425in}{0.861358in}}%
\pgfpathlineto{\pgfqpoint{4.555106in}{0.858805in}}%
\pgfpathlineto{\pgfqpoint{4.557777in}{0.862803in}}%
\pgfpathlineto{\pgfqpoint{4.560448in}{0.861282in}}%
\pgfpathlineto{\pgfqpoint{4.563125in}{0.859185in}}%
\pgfpathlineto{\pgfqpoint{4.565820in}{0.858197in}}%
\pgfpathlineto{\pgfqpoint{4.568612in}{0.861470in}}%
\pgfpathlineto{\pgfqpoint{4.571171in}{0.869587in}}%
\pgfpathlineto{\pgfqpoint{4.573947in}{0.871025in}}%
\pgfpathlineto{\pgfqpoint{4.576531in}{0.867755in}}%
\pgfpathlineto{\pgfqpoint{4.579305in}{0.866013in}}%
\pgfpathlineto{\pgfqpoint{4.581888in}{0.863343in}}%
\pgfpathlineto{\pgfqpoint{4.584672in}{0.867206in}}%
\pgfpathlineto{\pgfqpoint{4.587244in}{0.863549in}}%
\pgfpathlineto{\pgfqpoint{4.589920in}{0.869783in}}%
\pgfpathlineto{\pgfqpoint{4.592589in}{0.872943in}}%
\pgfpathlineto{\pgfqpoint{4.595281in}{0.873219in}}%
\pgfpathlineto{\pgfqpoint{4.597951in}{0.868981in}}%
\pgfpathlineto{\pgfqpoint{4.600633in}{0.871466in}}%
\pgfpathlineto{\pgfqpoint{4.603430in}{0.869203in}}%
\pgfpathlineto{\pgfqpoint{4.605990in}{0.868386in}}%
\pgfpathlineto{\pgfqpoint{4.608808in}{0.864277in}}%
\pgfpathlineto{\pgfqpoint{4.611350in}{0.867354in}}%
\pgfpathlineto{\pgfqpoint{4.614134in}{0.864188in}}%
\pgfpathlineto{\pgfqpoint{4.616702in}{0.861858in}}%
\pgfpathlineto{\pgfqpoint{4.619529in}{0.863542in}}%
\pgfpathlineto{\pgfqpoint{4.622056in}{0.864068in}}%
\pgfpathlineto{\pgfqpoint{4.624741in}{0.860793in}}%
\pgfpathlineto{\pgfqpoint{4.627411in}{0.859243in}}%
\pgfpathlineto{\pgfqpoint{4.630096in}{0.856373in}}%
\pgfpathlineto{\pgfqpoint{4.632902in}{0.856041in}}%
\pgfpathlineto{\pgfqpoint{4.635445in}{0.854139in}}%
\pgfpathlineto{\pgfqpoint{4.638204in}{0.857014in}}%
\pgfpathlineto{\pgfqpoint{4.640809in}{0.855909in}}%
\pgfpathlineto{\pgfqpoint{4.643628in}{0.855191in}}%
\pgfpathlineto{\pgfqpoint{4.646169in}{0.857253in}}%
\pgfpathlineto{\pgfqpoint{4.648922in}{0.857133in}}%
\pgfpathlineto{\pgfqpoint{4.651524in}{0.856227in}}%
\pgfpathlineto{\pgfqpoint{4.654203in}{0.857141in}}%
\pgfpathlineto{\pgfqpoint{4.656873in}{0.854790in}}%
\pgfpathlineto{\pgfqpoint{4.659590in}{0.854183in}}%
\pgfpathlineto{\pgfqpoint{4.662237in}{0.853770in}}%
\pgfpathlineto{\pgfqpoint{4.664923in}{0.855659in}}%
\pgfpathlineto{\pgfqpoint{4.667764in}{0.855993in}}%
\pgfpathlineto{\pgfqpoint{4.670261in}{0.854252in}}%
\pgfpathlineto{\pgfqpoint{4.673068in}{0.855412in}}%
\pgfpathlineto{\pgfqpoint{4.675619in}{0.860735in}}%
\pgfpathlineto{\pgfqpoint{4.678448in}{0.863904in}}%
\pgfpathlineto{\pgfqpoint{4.680988in}{0.860992in}}%
\pgfpathlineto{\pgfqpoint{4.683799in}{0.856869in}}%
\pgfpathlineto{\pgfqpoint{4.686337in}{0.852951in}}%
\pgfpathlineto{\pgfqpoint{4.689051in}{0.855316in}}%
\pgfpathlineto{\pgfqpoint{4.691694in}{0.857778in}}%
\pgfpathlineto{\pgfqpoint{4.694381in}{0.855866in}}%
\pgfpathlineto{\pgfqpoint{4.697170in}{0.856396in}}%
\pgfpathlineto{\pgfqpoint{4.699734in}{0.854779in}}%
\pgfpathlineto{\pgfqpoint{4.702517in}{0.857890in}}%
\pgfpathlineto{\pgfqpoint{4.705094in}{0.856422in}}%
\pgfpathlineto{\pgfqpoint{4.707824in}{0.854476in}}%
\pgfpathlineto{\pgfqpoint{4.710437in}{0.853925in}}%
\pgfpathlineto{\pgfqpoint{4.713275in}{0.851709in}}%
\pgfpathlineto{\pgfqpoint{4.715806in}{0.853527in}}%
\pgfpathlineto{\pgfqpoint{4.718486in}{0.854843in}}%
\pgfpathlineto{\pgfqpoint{4.721160in}{0.854232in}}%
\pgfpathlineto{\pgfqpoint{4.723873in}{0.852169in}}%
\pgfpathlineto{\pgfqpoint{4.726508in}{0.855949in}}%
\pgfpathlineto{\pgfqpoint{4.729233in}{0.854067in}}%
\pgfpathlineto{\pgfqpoint{4.731901in}{0.852993in}}%
\pgfpathlineto{\pgfqpoint{4.734552in}{0.858636in}}%
\pgfpathlineto{\pgfqpoint{4.737348in}{0.860280in}}%
\pgfpathlineto{\pgfqpoint{4.739912in}{0.856338in}}%
\pgfpathlineto{\pgfqpoint{4.742696in}{0.860127in}}%
\pgfpathlineto{\pgfqpoint{4.745256in}{0.861823in}}%
\pgfpathlineto{\pgfqpoint{4.748081in}{0.858457in}}%
\pgfpathlineto{\pgfqpoint{4.750627in}{0.855828in}}%
\pgfpathlineto{\pgfqpoint{4.753298in}{0.856548in}}%
\pgfpathlineto{\pgfqpoint{4.755983in}{0.855905in}}%
\pgfpathlineto{\pgfqpoint{4.758653in}{0.857270in}}%
\pgfpathlineto{\pgfqpoint{4.761337in}{0.867677in}}%
\pgfpathlineto{\pgfqpoint{4.764018in}{0.861976in}}%
\pgfpathlineto{\pgfqpoint{4.766783in}{0.855183in}}%
\pgfpathlineto{\pgfqpoint{4.769367in}{0.851255in}}%
\pgfpathlineto{\pgfqpoint{4.772198in}{0.855259in}}%
\pgfpathlineto{\pgfqpoint{4.774732in}{0.853894in}}%
\pgfpathlineto{\pgfqpoint{4.777535in}{0.854533in}}%
\pgfpathlineto{\pgfqpoint{4.780083in}{0.856290in}}%
\pgfpathlineto{\pgfqpoint{4.782872in}{0.857824in}}%
\pgfpathlineto{\pgfqpoint{4.785445in}{0.857782in}}%
\pgfpathlineto{\pgfqpoint{4.788116in}{0.852855in}}%
\pgfpathlineto{\pgfqpoint{4.790798in}{0.853588in}}%
\pgfpathlineto{\pgfqpoint{4.793512in}{0.850740in}}%
\pgfpathlineto{\pgfqpoint{4.796274in}{0.854129in}}%
\pgfpathlineto{\pgfqpoint{4.798830in}{0.853517in}}%
\pgfpathlineto{\pgfqpoint{4.801586in}{0.853015in}}%
\pgfpathlineto{\pgfqpoint{4.804193in}{0.852470in}}%
\pgfpathlineto{\pgfqpoint{4.807017in}{0.852555in}}%
\pgfpathlineto{\pgfqpoint{4.809538in}{0.850300in}}%
\pgfpathlineto{\pgfqpoint{4.812377in}{0.851342in}}%
\pgfpathlineto{\pgfqpoint{4.814907in}{0.848459in}}%
\pgfpathlineto{\pgfqpoint{4.817587in}{0.849571in}}%
\pgfpathlineto{\pgfqpoint{4.820265in}{0.852280in}}%
\pgfpathlineto{\pgfqpoint{4.822945in}{0.860387in}}%
\pgfpathlineto{\pgfqpoint{4.825619in}{0.853074in}}%
\pgfpathlineto{\pgfqpoint{4.828291in}{0.850183in}}%
\pgfpathlineto{\pgfqpoint{4.831045in}{0.847822in}}%
\pgfpathlineto{\pgfqpoint{4.833657in}{0.848666in}}%
\pgfpathlineto{\pgfqpoint{4.837992in}{0.852181in}}%
\pgfpathlineto{\pgfqpoint{4.839922in}{0.855982in}}%
\pgfpathlineto{\pgfqpoint{4.842380in}{0.852807in}}%
\pgfpathlineto{\pgfqpoint{4.844361in}{0.853705in}}%
\pgfpathlineto{\pgfqpoint{4.847127in}{0.854412in}}%
\pgfpathlineto{\pgfqpoint{4.849715in}{0.853307in}}%
\pgfpathlineto{\pgfqpoint{4.852404in}{0.852293in}}%
\pgfpathlineto{\pgfqpoint{4.855070in}{0.857234in}}%
\pgfpathlineto{\pgfqpoint{4.857807in}{0.857615in}}%
\pgfpathlineto{\pgfqpoint{4.860544in}{0.857342in}}%
\pgfpathlineto{\pgfqpoint{4.863116in}{0.856759in}}%
\pgfpathlineto{\pgfqpoint{4.865910in}{0.859404in}}%
\pgfpathlineto{\pgfqpoint{4.868474in}{0.859454in}}%
\pgfpathlineto{\pgfqpoint{4.871209in}{0.860928in}}%
\pgfpathlineto{\pgfqpoint{4.873832in}{0.859862in}}%
\pgfpathlineto{\pgfqpoint{4.876636in}{0.859753in}}%
\pgfpathlineto{\pgfqpoint{4.879180in}{0.860980in}}%
\pgfpathlineto{\pgfqpoint{4.881864in}{0.860459in}}%
\pgfpathlineto{\pgfqpoint{4.884540in}{0.859373in}}%
\pgfpathlineto{\pgfqpoint{4.887211in}{0.860411in}}%
\pgfpathlineto{\pgfqpoint{4.889902in}{0.859081in}}%
\pgfpathlineto{\pgfqpoint{4.892611in}{0.860547in}}%
\pgfpathlineto{\pgfqpoint{4.895399in}{0.860132in}}%
\pgfpathlineto{\pgfqpoint{4.897938in}{0.860996in}}%
\pgfpathlineto{\pgfqpoint{4.900712in}{0.859578in}}%
\pgfpathlineto{\pgfqpoint{4.903295in}{0.856666in}}%
\pgfpathlineto{\pgfqpoint{4.906096in}{0.854139in}}%
\pgfpathlineto{\pgfqpoint{4.908648in}{0.854988in}}%
\pgfpathlineto{\pgfqpoint{4.911435in}{0.859449in}}%
\pgfpathlineto{\pgfqpoint{4.914009in}{0.857697in}}%
\pgfpathlineto{\pgfqpoint{4.916681in}{0.857608in}}%
\pgfpathlineto{\pgfqpoint{4.919352in}{0.857413in}}%
\pgfpathlineto{\pgfqpoint{4.922041in}{0.854458in}}%
\pgfpathlineto{\pgfqpoint{4.924708in}{0.852315in}}%
\pgfpathlineto{\pgfqpoint{4.927400in}{0.855341in}}%
\pgfpathlineto{\pgfqpoint{4.930170in}{0.854285in}}%
\pgfpathlineto{\pgfqpoint{4.932742in}{0.860408in}}%
\pgfpathlineto{\pgfqpoint{4.935515in}{0.860817in}}%
\pgfpathlineto{\pgfqpoint{4.938112in}{0.857473in}}%
\pgfpathlineto{\pgfqpoint{4.940881in}{0.860808in}}%
\pgfpathlineto{\pgfqpoint{4.943466in}{0.857608in}}%
\pgfpathlineto{\pgfqpoint{4.946151in}{0.858224in}}%
\pgfpathlineto{\pgfqpoint{4.948827in}{0.859915in}}%
\pgfpathlineto{\pgfqpoint{4.951504in}{0.866138in}}%
\pgfpathlineto{\pgfqpoint{4.954182in}{0.874115in}}%
\pgfpathlineto{\pgfqpoint{4.956862in}{0.889863in}}%
\pgfpathlineto{\pgfqpoint{4.959689in}{0.877501in}}%
\pgfpathlineto{\pgfqpoint{4.962219in}{0.873181in}}%
\pgfpathlineto{\pgfqpoint{4.965002in}{0.870029in}}%
\pgfpathlineto{\pgfqpoint{4.967575in}{0.869881in}}%
\pgfpathlineto{\pgfqpoint{4.970314in}{0.866778in}}%
\pgfpathlineto{\pgfqpoint{4.972933in}{0.864927in}}%
\pgfpathlineto{\pgfqpoint{4.975703in}{0.869444in}}%
\pgfpathlineto{\pgfqpoint{4.978287in}{0.862601in}}%
\pgfpathlineto{\pgfqpoint{4.980967in}{0.861776in}}%
\pgfpathlineto{\pgfqpoint{4.983637in}{0.859166in}}%
\pgfpathlineto{\pgfqpoint{4.986325in}{0.858760in}}%
\pgfpathlineto{\pgfqpoint{4.989001in}{0.863731in}}%
\pgfpathlineto{\pgfqpoint{4.991683in}{0.855690in}}%
\pgfpathlineto{\pgfqpoint{4.994390in}{0.856166in}}%
\pgfpathlineto{\pgfqpoint{4.997028in}{0.856890in}}%
\pgfpathlineto{\pgfqpoint{4.999780in}{0.857925in}}%
\pgfpathlineto{\pgfqpoint{5.002384in}{0.853798in}}%
\pgfpathlineto{\pgfqpoint{5.005178in}{0.857540in}}%
\pgfpathlineto{\pgfqpoint{5.007751in}{0.856783in}}%
\pgfpathlineto{\pgfqpoint{5.010562in}{0.858196in}}%
\pgfpathlineto{\pgfqpoint{5.013104in}{0.858033in}}%
\pgfpathlineto{\pgfqpoint{5.015820in}{0.857965in}}%
\pgfpathlineto{\pgfqpoint{5.018466in}{0.853148in}}%
\pgfpathlineto{\pgfqpoint{5.021147in}{0.854402in}}%
\pgfpathlineto{\pgfqpoint{5.023927in}{0.856241in}}%
\pgfpathlineto{\pgfqpoint{5.026501in}{0.852449in}}%
\pgfpathlineto{\pgfqpoint{5.029275in}{0.856946in}}%
\pgfpathlineto{\pgfqpoint{5.031849in}{0.853128in}}%
\pgfpathlineto{\pgfqpoint{5.034649in}{0.875901in}}%
\pgfpathlineto{\pgfqpoint{5.037214in}{0.871756in}}%
\pgfpathlineto{\pgfqpoint{5.039962in}{0.861077in}}%
\pgfpathlineto{\pgfqpoint{5.042572in}{0.859865in}}%
\pgfpathlineto{\pgfqpoint{5.045249in}{0.858865in}}%
\pgfpathlineto{\pgfqpoint{5.047924in}{0.862661in}}%
\pgfpathlineto{\pgfqpoint{5.050606in}{0.861411in}}%
\pgfpathlineto{\pgfqpoint{5.053284in}{0.861010in}}%
\pgfpathlineto{\pgfqpoint{5.055952in}{0.858713in}}%
\pgfpathlineto{\pgfqpoint{5.058711in}{0.858305in}}%
\pgfpathlineto{\pgfqpoint{5.061315in}{0.858436in}}%
\pgfpathlineto{\pgfqpoint{5.064144in}{0.860865in}}%
\pgfpathlineto{\pgfqpoint{5.066677in}{0.858413in}}%
\pgfpathlineto{\pgfqpoint{5.069463in}{0.856473in}}%
\pgfpathlineto{\pgfqpoint{5.072030in}{0.854596in}}%
\pgfpathlineto{\pgfqpoint{5.074851in}{0.858365in}}%
\pgfpathlineto{\pgfqpoint{5.077390in}{0.853120in}}%
\pgfpathlineto{\pgfqpoint{5.080067in}{0.852793in}}%
\pgfpathlineto{\pgfqpoint{5.082746in}{0.853928in}}%
\pgfpathlineto{\pgfqpoint{5.085426in}{0.858503in}}%
\pgfpathlineto{\pgfqpoint{5.088103in}{0.857029in}}%
\pgfpathlineto{\pgfqpoint{5.090788in}{0.858150in}}%
\pgfpathlineto{\pgfqpoint{5.093579in}{0.857690in}}%
\pgfpathlineto{\pgfqpoint{5.096142in}{0.855688in}}%
\pgfpathlineto{\pgfqpoint{5.098948in}{0.856520in}}%
\pgfpathlineto{\pgfqpoint{5.101496in}{0.855355in}}%
\pgfpathlineto{\pgfqpoint{5.104312in}{0.852495in}}%
\pgfpathlineto{\pgfqpoint{5.106842in}{0.854417in}}%
\pgfpathlineto{\pgfqpoint{5.109530in}{0.854043in}}%
\pgfpathlineto{\pgfqpoint{5.112209in}{0.855979in}}%
\pgfpathlineto{\pgfqpoint{5.114887in}{0.856578in}}%
\pgfpathlineto{\pgfqpoint{5.117550in}{0.857735in}}%
\pgfpathlineto{\pgfqpoint{5.120243in}{0.857202in}}%
\pgfpathlineto{\pgfqpoint{5.123042in}{0.855408in}}%
\pgfpathlineto{\pgfqpoint{5.125599in}{0.857811in}}%
\pgfpathlineto{\pgfqpoint{5.128421in}{0.855301in}}%
\pgfpathlineto{\pgfqpoint{5.130953in}{0.858618in}}%
\pgfpathlineto{\pgfqpoint{5.133716in}{0.856058in}}%
\pgfpathlineto{\pgfqpoint{5.136311in}{0.856133in}}%
\pgfpathlineto{\pgfqpoint{5.139072in}{0.852887in}}%
\pgfpathlineto{\pgfqpoint{5.141660in}{0.854468in}}%
\pgfpathlineto{\pgfqpoint{5.144349in}{0.856841in}}%
\pgfpathlineto{\pgfqpoint{5.147029in}{0.853417in}}%
\pgfpathlineto{\pgfqpoint{5.149734in}{0.851866in}}%
\pgfpathlineto{\pgfqpoint{5.152382in}{0.846315in}}%
\pgfpathlineto{\pgfqpoint{5.155059in}{0.850160in}}%
\pgfpathlineto{\pgfqpoint{5.157815in}{0.850192in}}%
\pgfpathlineto{\pgfqpoint{5.160420in}{0.851516in}}%
\pgfpathlineto{\pgfqpoint{5.163243in}{0.854854in}}%
\pgfpathlineto{\pgfqpoint{5.165775in}{0.854327in}}%
\pgfpathlineto{\pgfqpoint{5.168591in}{0.855118in}}%
\pgfpathlineto{\pgfqpoint{5.171133in}{0.853820in}}%
\pgfpathlineto{\pgfqpoint{5.173925in}{0.855400in}}%
\pgfpathlineto{\pgfqpoint{5.176477in}{0.854873in}}%
\pgfpathlineto{\pgfqpoint{5.179188in}{0.857047in}}%
\pgfpathlineto{\pgfqpoint{5.181848in}{0.855254in}}%
\pgfpathlineto{\pgfqpoint{5.184522in}{0.855621in}}%
\pgfpathlineto{\pgfqpoint{5.187294in}{0.855152in}}%
\pgfpathlineto{\pgfqpoint{5.189880in}{0.855696in}}%
\pgfpathlineto{\pgfqpoint{5.192680in}{0.854668in}}%
\pgfpathlineto{\pgfqpoint{5.195239in}{0.853423in}}%
\pgfpathlineto{\pgfqpoint{5.198008in}{0.855834in}}%
\pgfpathlineto{\pgfqpoint{5.200594in}{0.857560in}}%
\pgfpathlineto{\pgfqpoint{5.203388in}{0.857384in}}%
\pgfpathlineto{\pgfqpoint{5.205952in}{0.860435in}}%
\pgfpathlineto{\pgfqpoint{5.208630in}{0.860632in}}%
\pgfpathlineto{\pgfqpoint{5.211299in}{0.857961in}}%
\pgfpathlineto{\pgfqpoint{5.214027in}{0.857764in}}%
\pgfpathlineto{\pgfqpoint{5.216667in}{0.857756in}}%
\pgfpathlineto{\pgfqpoint{5.219345in}{0.855122in}}%
\pgfpathlineto{\pgfqpoint{5.222151in}{0.853262in}}%
\pgfpathlineto{\pgfqpoint{5.224695in}{0.855431in}}%
\pgfpathlineto{\pgfqpoint{5.227470in}{0.853788in}}%
\pgfpathlineto{\pgfqpoint{5.230059in}{0.852984in}}%
\pgfpathlineto{\pgfqpoint{5.232855in}{0.853486in}}%
\pgfpathlineto{\pgfqpoint{5.235409in}{0.854986in}}%
\pgfpathlineto{\pgfqpoint{5.238173in}{0.855100in}}%
\pgfpathlineto{\pgfqpoint{5.240777in}{0.855059in}}%
\pgfpathlineto{\pgfqpoint{5.243445in}{0.853810in}}%
\pgfpathlineto{\pgfqpoint{5.246130in}{0.854159in}}%
\pgfpathlineto{\pgfqpoint{5.248816in}{0.854640in}}%
\pgfpathlineto{\pgfqpoint{5.251590in}{0.856627in}}%
\pgfpathlineto{\pgfqpoint{5.254236in}{0.856775in}}%
\pgfpathlineto{\pgfqpoint{5.256973in}{0.856181in}}%
\pgfpathlineto{\pgfqpoint{5.259511in}{0.853741in}}%
\pgfpathlineto{\pgfqpoint{5.262264in}{0.857420in}}%
\pgfpathlineto{\pgfqpoint{5.264876in}{0.855794in}}%
\pgfpathlineto{\pgfqpoint{5.267691in}{0.857780in}}%
\pgfpathlineto{\pgfqpoint{5.270238in}{0.856855in}}%
\pgfpathlineto{\pgfqpoint{5.272913in}{0.878456in}}%
\pgfpathlineto{\pgfqpoint{5.275589in}{0.892969in}}%
\pgfpathlineto{\pgfqpoint{5.278322in}{0.881264in}}%
\pgfpathlineto{\pgfqpoint{5.280947in}{0.881995in}}%
\pgfpathlineto{\pgfqpoint{5.283631in}{0.875511in}}%
\pgfpathlineto{\pgfqpoint{5.286436in}{0.881291in}}%
\pgfpathlineto{\pgfqpoint{5.288984in}{0.880622in}}%
\pgfpathlineto{\pgfqpoint{5.291794in}{0.876705in}}%
\pgfpathlineto{\pgfqpoint{5.294339in}{0.873402in}}%
\pgfpathlineto{\pgfqpoint{5.297140in}{0.871110in}}%
\pgfpathlineto{\pgfqpoint{5.299696in}{0.869810in}}%
\pgfpathlineto{\pgfqpoint{5.302443in}{0.865800in}}%
\pgfpathlineto{\pgfqpoint{5.305054in}{0.864720in}}%
\pgfpathlineto{\pgfqpoint{5.307731in}{0.865456in}}%
\pgfpathlineto{\pgfqpoint{5.310411in}{0.864528in}}%
\pgfpathlineto{\pgfqpoint{5.313089in}{0.865390in}}%
\pgfpathlineto{\pgfqpoint{5.315754in}{0.861832in}}%
\pgfpathlineto{\pgfqpoint{5.318430in}{0.860720in}}%
\pgfpathlineto{\pgfqpoint{5.321256in}{0.859570in}}%
\pgfpathlineto{\pgfqpoint{5.323802in}{0.857745in}}%
\pgfpathlineto{\pgfqpoint{5.326564in}{0.857319in}}%
\pgfpathlineto{\pgfqpoint{5.329159in}{0.855577in}}%
\pgfpathlineto{\pgfqpoint{5.331973in}{0.854196in}}%
\pgfpathlineto{\pgfqpoint{5.334510in}{0.853007in}}%
\pgfpathlineto{\pgfqpoint{5.337353in}{0.854793in}}%
\pgfpathlineto{\pgfqpoint{5.339872in}{0.852549in}}%
\pgfpathlineto{\pgfqpoint{5.342549in}{0.852161in}}%
\pgfpathlineto{\pgfqpoint{5.345224in}{0.848597in}}%
\pgfpathlineto{\pgfqpoint{5.347905in}{0.844984in}}%
\pgfpathlineto{\pgfqpoint{5.350723in}{0.848946in}}%
\pgfpathlineto{\pgfqpoint{5.353262in}{0.849761in}}%
\pgfpathlineto{\pgfqpoint{5.356056in}{0.851671in}}%
\pgfpathlineto{\pgfqpoint{5.358612in}{0.852420in}}%
\pgfpathlineto{\pgfqpoint{5.361370in}{0.850872in}}%
\pgfpathlineto{\pgfqpoint{5.363966in}{0.855137in}}%
\pgfpathlineto{\pgfqpoint{5.366727in}{0.854944in}}%
\pgfpathlineto{\pgfqpoint{5.369335in}{0.858327in}}%
\pgfpathlineto{\pgfqpoint{5.372013in}{0.858747in}}%
\pgfpathlineto{\pgfqpoint{5.374692in}{0.857971in}}%
\pgfpathlineto{\pgfqpoint{5.377370in}{0.858561in}}%
\pgfpathlineto{\pgfqpoint{5.380048in}{0.855093in}}%
\pgfpathlineto{\pgfqpoint{5.382725in}{0.857234in}}%
\pgfpathlineto{\pgfqpoint{5.385550in}{0.854781in}}%
\pgfpathlineto{\pgfqpoint{5.388083in}{0.856383in}}%
\pgfpathlineto{\pgfqpoint{5.390900in}{0.853044in}}%
\pgfpathlineto{\pgfqpoint{5.393441in}{0.853359in}}%
\pgfpathlineto{\pgfqpoint{5.396219in}{0.855609in}}%
\pgfpathlineto{\pgfqpoint{5.398784in}{0.878756in}}%
\pgfpathlineto{\pgfqpoint{5.401576in}{0.880468in}}%
\pgfpathlineto{\pgfqpoint{5.404154in}{0.870164in}}%
\pgfpathlineto{\pgfqpoint{5.406832in}{0.868408in}}%
\pgfpathlineto{\pgfqpoint{5.409507in}{0.864523in}}%
\pgfpathlineto{\pgfqpoint{5.412190in}{0.860034in}}%
\pgfpathlineto{\pgfqpoint{5.414954in}{0.863052in}}%
\pgfpathlineto{\pgfqpoint{5.417547in}{0.861887in}}%
\pgfpathlineto{\pgfqpoint{5.420304in}{0.857782in}}%
\pgfpathlineto{\pgfqpoint{5.422897in}{0.858034in}}%
\pgfpathlineto{\pgfqpoint{5.425661in}{0.858004in}}%
\pgfpathlineto{\pgfqpoint{5.428259in}{0.858449in}}%
\pgfpathlineto{\pgfqpoint{5.431015in}{0.857554in}}%
\pgfpathlineto{\pgfqpoint{5.433616in}{0.857988in}}%
\pgfpathlineto{\pgfqpoint{5.436295in}{0.858544in}}%
\pgfpathlineto{\pgfqpoint{5.438974in}{0.857488in}}%
\pgfpathlineto{\pgfqpoint{5.441698in}{0.858848in}}%
\pgfpathlineto{\pgfqpoint{5.444328in}{0.864900in}}%
\pgfpathlineto{\pgfqpoint{5.447021in}{0.858658in}}%
\pgfpathlineto{\pgfqpoint{5.449769in}{0.873460in}}%
\pgfpathlineto{\pgfqpoint{5.452365in}{0.869173in}}%
\pgfpathlineto{\pgfqpoint{5.455168in}{0.865974in}}%
\pgfpathlineto{\pgfqpoint{5.457721in}{0.865039in}}%
\pgfpathlineto{\pgfqpoint{5.460489in}{0.862755in}}%
\pgfpathlineto{\pgfqpoint{5.463079in}{0.861547in}}%
\pgfpathlineto{\pgfqpoint{5.465888in}{0.858538in}}%
\pgfpathlineto{\pgfqpoint{5.468425in}{0.860531in}}%
\pgfpathlineto{\pgfqpoint{5.471113in}{0.858913in}}%
\pgfpathlineto{\pgfqpoint{5.473792in}{0.860575in}}%
\pgfpathlineto{\pgfqpoint{5.476458in}{0.855577in}}%
\pgfpathlineto{\pgfqpoint{5.479152in}{0.857903in}}%
\pgfpathlineto{\pgfqpoint{5.481825in}{0.858201in}}%
\pgfpathlineto{\pgfqpoint{5.484641in}{0.857057in}}%
\pgfpathlineto{\pgfqpoint{5.487176in}{0.857792in}}%
\pgfpathlineto{\pgfqpoint{5.490000in}{0.855994in}}%
\pgfpathlineto{\pgfqpoint{5.492541in}{0.859629in}}%
\pgfpathlineto{\pgfqpoint{5.495346in}{0.860321in}}%
\pgfpathlineto{\pgfqpoint{5.497898in}{0.856531in}}%
\pgfpathlineto{\pgfqpoint{5.500687in}{0.860116in}}%
\pgfpathlineto{\pgfqpoint{5.503255in}{0.857680in}}%
\pgfpathlineto{\pgfqpoint{5.505933in}{0.856883in}}%
\pgfpathlineto{\pgfqpoint{5.508612in}{0.852523in}}%
\pgfpathlineto{\pgfqpoint{5.511290in}{0.851089in}}%
\pgfpathlineto{\pgfqpoint{5.514080in}{0.863344in}}%
\pgfpathlineto{\pgfqpoint{5.516646in}{0.856488in}}%
\pgfpathlineto{\pgfqpoint{5.519433in}{0.858882in}}%
\pgfpathlineto{\pgfqpoint{5.522003in}{0.858999in}}%
\pgfpathlineto{\pgfqpoint{5.524756in}{0.863867in}}%
\pgfpathlineto{\pgfqpoint{5.527360in}{0.857757in}}%
\pgfpathlineto{\pgfqpoint{5.530148in}{0.854661in}}%
\pgfpathlineto{\pgfqpoint{5.532717in}{0.854705in}}%
\pgfpathlineto{\pgfqpoint{5.535395in}{0.865419in}}%
\pgfpathlineto{\pgfqpoint{5.538074in}{0.855945in}}%
\pgfpathlineto{\pgfqpoint{5.540750in}{0.855258in}}%
\pgfpathlineto{\pgfqpoint{5.543421in}{0.857310in}}%
\pgfpathlineto{\pgfqpoint{5.546110in}{0.858480in}}%
\pgfpathlineto{\pgfqpoint{5.548921in}{0.856130in}}%
\pgfpathlineto{\pgfqpoint{5.551457in}{0.855755in}}%
\pgfpathlineto{\pgfqpoint{5.554198in}{0.854249in}}%
\pgfpathlineto{\pgfqpoint{5.556822in}{0.853522in}}%
\pgfpathlineto{\pgfqpoint{5.559612in}{0.853220in}}%
\pgfpathlineto{\pgfqpoint{5.562180in}{0.849673in}}%
\pgfpathlineto{\pgfqpoint{5.564940in}{0.850737in}}%
\pgfpathlineto{\pgfqpoint{5.567536in}{0.852792in}}%
\pgfpathlineto{\pgfqpoint{5.570215in}{0.849244in}}%
\pgfpathlineto{\pgfqpoint{5.572893in}{0.854201in}}%
\pgfpathlineto{\pgfqpoint{5.575596in}{0.853588in}}%
\pgfpathlineto{\pgfqpoint{5.578342in}{0.857471in}}%
\pgfpathlineto{\pgfqpoint{5.580914in}{0.856783in}}%
\pgfpathlineto{\pgfqpoint{5.583709in}{0.847705in}}%
\pgfpathlineto{\pgfqpoint{5.586269in}{0.853019in}}%
\pgfpathlineto{\pgfqpoint{5.589040in}{0.859433in}}%
\pgfpathlineto{\pgfqpoint{5.591641in}{0.859184in}}%
\pgfpathlineto{\pgfqpoint{5.594368in}{0.856081in}}%
\pgfpathlineto{\pgfqpoint{5.596999in}{0.858000in}}%
\pgfpathlineto{\pgfqpoint{5.599674in}{0.858094in}}%
\pgfpathlineto{\pgfqpoint{5.602352in}{0.856362in}}%
\pgfpathlineto{\pgfqpoint{5.605073in}{0.857730in}}%
\pgfpathlineto{\pgfqpoint{5.607698in}{0.858961in}}%
\pgfpathlineto{\pgfqpoint{5.610389in}{0.855829in}}%
\pgfpathlineto{\pgfqpoint{5.613235in}{0.857952in}}%
\pgfpathlineto{\pgfqpoint{5.615743in}{0.855368in}}%
\pgfpathlineto{\pgfqpoint{5.618526in}{0.858724in}}%
\pgfpathlineto{\pgfqpoint{5.621102in}{0.858726in}}%
\pgfpathlineto{\pgfqpoint{5.623868in}{0.860344in}}%
\pgfpathlineto{\pgfqpoint{5.626460in}{0.860804in}}%
\pgfpathlineto{\pgfqpoint{5.629232in}{0.859277in}}%
\pgfpathlineto{\pgfqpoint{5.631815in}{0.859510in}}%
\pgfpathlineto{\pgfqpoint{5.634496in}{0.857800in}}%
\pgfpathlineto{\pgfqpoint{5.637172in}{0.870809in}}%
\pgfpathlineto{\pgfqpoint{5.639852in}{0.894488in}}%
\pgfpathlineto{\pgfqpoint{5.642518in}{0.948446in}}%
\pgfpathlineto{\pgfqpoint{5.645243in}{0.957755in}}%
\pgfpathlineto{\pgfqpoint{5.648008in}{0.968919in}}%
\pgfpathlineto{\pgfqpoint{5.650563in}{0.999811in}}%
\pgfpathlineto{\pgfqpoint{5.653376in}{0.999914in}}%
\pgfpathlineto{\pgfqpoint{5.655919in}{0.973853in}}%
\pgfpathlineto{\pgfqpoint{5.658723in}{0.961761in}}%
\pgfpathlineto{\pgfqpoint{5.661273in}{0.939900in}}%
\pgfpathlineto{\pgfqpoint{5.664099in}{0.917518in}}%
\pgfpathlineto{\pgfqpoint{5.666632in}{0.932906in}}%
\pgfpathlineto{\pgfqpoint{5.669313in}{0.945033in}}%
\pgfpathlineto{\pgfqpoint{5.671991in}{0.930012in}}%
\pgfpathlineto{\pgfqpoint{5.674667in}{0.918994in}}%
\pgfpathlineto{\pgfqpoint{5.677486in}{0.905163in}}%
\pgfpathlineto{\pgfqpoint{5.680027in}{0.887766in}}%
\pgfpathlineto{\pgfqpoint{5.682836in}{0.878842in}}%
\pgfpathlineto{\pgfqpoint{5.685385in}{0.881682in}}%
\pgfpathlineto{\pgfqpoint{5.688159in}{0.870899in}}%
\pgfpathlineto{\pgfqpoint{5.690730in}{0.870939in}}%
\pgfpathlineto{\pgfqpoint{5.693473in}{0.908559in}}%
\pgfpathlineto{\pgfqpoint{5.696101in}{0.924994in}}%
\pgfpathlineto{\pgfqpoint{5.698775in}{0.915754in}}%
\pgfpathlineto{\pgfqpoint{5.701453in}{0.895974in}}%
\pgfpathlineto{\pgfqpoint{5.704130in}{0.884091in}}%
\pgfpathlineto{\pgfqpoint{5.706800in}{0.870291in}}%
\pgfpathlineto{\pgfqpoint{5.709490in}{0.870281in}}%
\pgfpathlineto{\pgfqpoint{5.712291in}{0.868273in}}%
\pgfpathlineto{\pgfqpoint{5.714834in}{0.865637in}}%
\pgfpathlineto{\pgfqpoint{5.717671in}{0.864398in}}%
\pgfpathlineto{\pgfqpoint{5.720201in}{0.863401in}}%
\pgfpathlineto{\pgfqpoint{5.722950in}{0.863405in}}%
\pgfpathlineto{\pgfqpoint{5.725548in}{0.861002in}}%
\pgfpathlineto{\pgfqpoint{5.728339in}{0.859259in}}%
\pgfpathlineto{\pgfqpoint{5.730919in}{0.858569in}}%
\pgfpathlineto{\pgfqpoint{5.733594in}{0.859485in}}%
\pgfpathlineto{\pgfqpoint{5.736276in}{0.855536in}}%
\pgfpathlineto{\pgfqpoint{5.738974in}{0.854933in}}%
\pgfpathlineto{\pgfqpoint{5.741745in}{0.852141in}}%
\pgfpathlineto{\pgfqpoint{5.744310in}{0.856237in}}%
\pgfpathlineto{\pgfqpoint{5.744310in}{0.413320in}}%
\pgfpathlineto{\pgfqpoint{5.744310in}{0.413320in}}%
\pgfpathlineto{\pgfqpoint{5.741745in}{0.413320in}}%
\pgfpathlineto{\pgfqpoint{5.738974in}{0.413320in}}%
\pgfpathlineto{\pgfqpoint{5.736276in}{0.413320in}}%
\pgfpathlineto{\pgfqpoint{5.733594in}{0.413320in}}%
\pgfpathlineto{\pgfqpoint{5.730919in}{0.413320in}}%
\pgfpathlineto{\pgfqpoint{5.728339in}{0.413320in}}%
\pgfpathlineto{\pgfqpoint{5.725548in}{0.413320in}}%
\pgfpathlineto{\pgfqpoint{5.722950in}{0.413320in}}%
\pgfpathlineto{\pgfqpoint{5.720201in}{0.413320in}}%
\pgfpathlineto{\pgfqpoint{5.717671in}{0.413320in}}%
\pgfpathlineto{\pgfqpoint{5.714834in}{0.413320in}}%
\pgfpathlineto{\pgfqpoint{5.712291in}{0.413320in}}%
\pgfpathlineto{\pgfqpoint{5.709490in}{0.413320in}}%
\pgfpathlineto{\pgfqpoint{5.706800in}{0.413320in}}%
\pgfpathlineto{\pgfqpoint{5.704130in}{0.413320in}}%
\pgfpathlineto{\pgfqpoint{5.701453in}{0.413320in}}%
\pgfpathlineto{\pgfqpoint{5.698775in}{0.413320in}}%
\pgfpathlineto{\pgfqpoint{5.696101in}{0.413320in}}%
\pgfpathlineto{\pgfqpoint{5.693473in}{0.413320in}}%
\pgfpathlineto{\pgfqpoint{5.690730in}{0.413320in}}%
\pgfpathlineto{\pgfqpoint{5.688159in}{0.413320in}}%
\pgfpathlineto{\pgfqpoint{5.685385in}{0.413320in}}%
\pgfpathlineto{\pgfqpoint{5.682836in}{0.413320in}}%
\pgfpathlineto{\pgfqpoint{5.680027in}{0.413320in}}%
\pgfpathlineto{\pgfqpoint{5.677486in}{0.413320in}}%
\pgfpathlineto{\pgfqpoint{5.674667in}{0.413320in}}%
\pgfpathlineto{\pgfqpoint{5.671991in}{0.413320in}}%
\pgfpathlineto{\pgfqpoint{5.669313in}{0.413320in}}%
\pgfpathlineto{\pgfqpoint{5.666632in}{0.413320in}}%
\pgfpathlineto{\pgfqpoint{5.664099in}{0.413320in}}%
\pgfpathlineto{\pgfqpoint{5.661273in}{0.413320in}}%
\pgfpathlineto{\pgfqpoint{5.658723in}{0.413320in}}%
\pgfpathlineto{\pgfqpoint{5.655919in}{0.413320in}}%
\pgfpathlineto{\pgfqpoint{5.653376in}{0.413320in}}%
\pgfpathlineto{\pgfqpoint{5.650563in}{0.413320in}}%
\pgfpathlineto{\pgfqpoint{5.648008in}{0.413320in}}%
\pgfpathlineto{\pgfqpoint{5.645243in}{0.413320in}}%
\pgfpathlineto{\pgfqpoint{5.642518in}{0.413320in}}%
\pgfpathlineto{\pgfqpoint{5.639852in}{0.413320in}}%
\pgfpathlineto{\pgfqpoint{5.637172in}{0.413320in}}%
\pgfpathlineto{\pgfqpoint{5.634496in}{0.413320in}}%
\pgfpathlineto{\pgfqpoint{5.631815in}{0.413320in}}%
\pgfpathlineto{\pgfqpoint{5.629232in}{0.413320in}}%
\pgfpathlineto{\pgfqpoint{5.626460in}{0.413320in}}%
\pgfpathlineto{\pgfqpoint{5.623868in}{0.413320in}}%
\pgfpathlineto{\pgfqpoint{5.621102in}{0.413320in}}%
\pgfpathlineto{\pgfqpoint{5.618526in}{0.413320in}}%
\pgfpathlineto{\pgfqpoint{5.615743in}{0.413320in}}%
\pgfpathlineto{\pgfqpoint{5.613235in}{0.413320in}}%
\pgfpathlineto{\pgfqpoint{5.610389in}{0.413320in}}%
\pgfpathlineto{\pgfqpoint{5.607698in}{0.413320in}}%
\pgfpathlineto{\pgfqpoint{5.605073in}{0.413320in}}%
\pgfpathlineto{\pgfqpoint{5.602352in}{0.413320in}}%
\pgfpathlineto{\pgfqpoint{5.599674in}{0.413320in}}%
\pgfpathlineto{\pgfqpoint{5.596999in}{0.413320in}}%
\pgfpathlineto{\pgfqpoint{5.594368in}{0.413320in}}%
\pgfpathlineto{\pgfqpoint{5.591641in}{0.413320in}}%
\pgfpathlineto{\pgfqpoint{5.589040in}{0.413320in}}%
\pgfpathlineto{\pgfqpoint{5.586269in}{0.413320in}}%
\pgfpathlineto{\pgfqpoint{5.583709in}{0.413320in}}%
\pgfpathlineto{\pgfqpoint{5.580914in}{0.413320in}}%
\pgfpathlineto{\pgfqpoint{5.578342in}{0.413320in}}%
\pgfpathlineto{\pgfqpoint{5.575596in}{0.413320in}}%
\pgfpathlineto{\pgfqpoint{5.572893in}{0.413320in}}%
\pgfpathlineto{\pgfqpoint{5.570215in}{0.413320in}}%
\pgfpathlineto{\pgfqpoint{5.567536in}{0.413320in}}%
\pgfpathlineto{\pgfqpoint{5.564940in}{0.413320in}}%
\pgfpathlineto{\pgfqpoint{5.562180in}{0.413320in}}%
\pgfpathlineto{\pgfqpoint{5.559612in}{0.413320in}}%
\pgfpathlineto{\pgfqpoint{5.556822in}{0.413320in}}%
\pgfpathlineto{\pgfqpoint{5.554198in}{0.413320in}}%
\pgfpathlineto{\pgfqpoint{5.551457in}{0.413320in}}%
\pgfpathlineto{\pgfqpoint{5.548921in}{0.413320in}}%
\pgfpathlineto{\pgfqpoint{5.546110in}{0.413320in}}%
\pgfpathlineto{\pgfqpoint{5.543421in}{0.413320in}}%
\pgfpathlineto{\pgfqpoint{5.540750in}{0.413320in}}%
\pgfpathlineto{\pgfqpoint{5.538074in}{0.413320in}}%
\pgfpathlineto{\pgfqpoint{5.535395in}{0.413320in}}%
\pgfpathlineto{\pgfqpoint{5.532717in}{0.413320in}}%
\pgfpathlineto{\pgfqpoint{5.530148in}{0.413320in}}%
\pgfpathlineto{\pgfqpoint{5.527360in}{0.413320in}}%
\pgfpathlineto{\pgfqpoint{5.524756in}{0.413320in}}%
\pgfpathlineto{\pgfqpoint{5.522003in}{0.413320in}}%
\pgfpathlineto{\pgfqpoint{5.519433in}{0.413320in}}%
\pgfpathlineto{\pgfqpoint{5.516646in}{0.413320in}}%
\pgfpathlineto{\pgfqpoint{5.514080in}{0.413320in}}%
\pgfpathlineto{\pgfqpoint{5.511290in}{0.413320in}}%
\pgfpathlineto{\pgfqpoint{5.508612in}{0.413320in}}%
\pgfpathlineto{\pgfqpoint{5.505933in}{0.413320in}}%
\pgfpathlineto{\pgfqpoint{5.503255in}{0.413320in}}%
\pgfpathlineto{\pgfqpoint{5.500687in}{0.413320in}}%
\pgfpathlineto{\pgfqpoint{5.497898in}{0.413320in}}%
\pgfpathlineto{\pgfqpoint{5.495346in}{0.413320in}}%
\pgfpathlineto{\pgfqpoint{5.492541in}{0.413320in}}%
\pgfpathlineto{\pgfqpoint{5.490000in}{0.413320in}}%
\pgfpathlineto{\pgfqpoint{5.487176in}{0.413320in}}%
\pgfpathlineto{\pgfqpoint{5.484641in}{0.413320in}}%
\pgfpathlineto{\pgfqpoint{5.481825in}{0.413320in}}%
\pgfpathlineto{\pgfqpoint{5.479152in}{0.413320in}}%
\pgfpathlineto{\pgfqpoint{5.476458in}{0.413320in}}%
\pgfpathlineto{\pgfqpoint{5.473792in}{0.413320in}}%
\pgfpathlineto{\pgfqpoint{5.471113in}{0.413320in}}%
\pgfpathlineto{\pgfqpoint{5.468425in}{0.413320in}}%
\pgfpathlineto{\pgfqpoint{5.465888in}{0.413320in}}%
\pgfpathlineto{\pgfqpoint{5.463079in}{0.413320in}}%
\pgfpathlineto{\pgfqpoint{5.460489in}{0.413320in}}%
\pgfpathlineto{\pgfqpoint{5.457721in}{0.413320in}}%
\pgfpathlineto{\pgfqpoint{5.455168in}{0.413320in}}%
\pgfpathlineto{\pgfqpoint{5.452365in}{0.413320in}}%
\pgfpathlineto{\pgfqpoint{5.449769in}{0.413320in}}%
\pgfpathlineto{\pgfqpoint{5.447021in}{0.413320in}}%
\pgfpathlineto{\pgfqpoint{5.444328in}{0.413320in}}%
\pgfpathlineto{\pgfqpoint{5.441698in}{0.413320in}}%
\pgfpathlineto{\pgfqpoint{5.438974in}{0.413320in}}%
\pgfpathlineto{\pgfqpoint{5.436295in}{0.413320in}}%
\pgfpathlineto{\pgfqpoint{5.433616in}{0.413320in}}%
\pgfpathlineto{\pgfqpoint{5.431015in}{0.413320in}}%
\pgfpathlineto{\pgfqpoint{5.428259in}{0.413320in}}%
\pgfpathlineto{\pgfqpoint{5.425661in}{0.413320in}}%
\pgfpathlineto{\pgfqpoint{5.422897in}{0.413320in}}%
\pgfpathlineto{\pgfqpoint{5.420304in}{0.413320in}}%
\pgfpathlineto{\pgfqpoint{5.417547in}{0.413320in}}%
\pgfpathlineto{\pgfqpoint{5.414954in}{0.413320in}}%
\pgfpathlineto{\pgfqpoint{5.412190in}{0.413320in}}%
\pgfpathlineto{\pgfqpoint{5.409507in}{0.413320in}}%
\pgfpathlineto{\pgfqpoint{5.406832in}{0.413320in}}%
\pgfpathlineto{\pgfqpoint{5.404154in}{0.413320in}}%
\pgfpathlineto{\pgfqpoint{5.401576in}{0.413320in}}%
\pgfpathlineto{\pgfqpoint{5.398784in}{0.413320in}}%
\pgfpathlineto{\pgfqpoint{5.396219in}{0.413320in}}%
\pgfpathlineto{\pgfqpoint{5.393441in}{0.413320in}}%
\pgfpathlineto{\pgfqpoint{5.390900in}{0.413320in}}%
\pgfpathlineto{\pgfqpoint{5.388083in}{0.413320in}}%
\pgfpathlineto{\pgfqpoint{5.385550in}{0.413320in}}%
\pgfpathlineto{\pgfqpoint{5.382725in}{0.413320in}}%
\pgfpathlineto{\pgfqpoint{5.380048in}{0.413320in}}%
\pgfpathlineto{\pgfqpoint{5.377370in}{0.413320in}}%
\pgfpathlineto{\pgfqpoint{5.374692in}{0.413320in}}%
\pgfpathlineto{\pgfqpoint{5.372013in}{0.413320in}}%
\pgfpathlineto{\pgfqpoint{5.369335in}{0.413320in}}%
\pgfpathlineto{\pgfqpoint{5.366727in}{0.413320in}}%
\pgfpathlineto{\pgfqpoint{5.363966in}{0.413320in}}%
\pgfpathlineto{\pgfqpoint{5.361370in}{0.413320in}}%
\pgfpathlineto{\pgfqpoint{5.358612in}{0.413320in}}%
\pgfpathlineto{\pgfqpoint{5.356056in}{0.413320in}}%
\pgfpathlineto{\pgfqpoint{5.353262in}{0.413320in}}%
\pgfpathlineto{\pgfqpoint{5.350723in}{0.413320in}}%
\pgfpathlineto{\pgfqpoint{5.347905in}{0.413320in}}%
\pgfpathlineto{\pgfqpoint{5.345224in}{0.413320in}}%
\pgfpathlineto{\pgfqpoint{5.342549in}{0.413320in}}%
\pgfpathlineto{\pgfqpoint{5.339872in}{0.413320in}}%
\pgfpathlineto{\pgfqpoint{5.337353in}{0.413320in}}%
\pgfpathlineto{\pgfqpoint{5.334510in}{0.413320in}}%
\pgfpathlineto{\pgfqpoint{5.331973in}{0.413320in}}%
\pgfpathlineto{\pgfqpoint{5.329159in}{0.413320in}}%
\pgfpathlineto{\pgfqpoint{5.326564in}{0.413320in}}%
\pgfpathlineto{\pgfqpoint{5.323802in}{0.413320in}}%
\pgfpathlineto{\pgfqpoint{5.321256in}{0.413320in}}%
\pgfpathlineto{\pgfqpoint{5.318430in}{0.413320in}}%
\pgfpathlineto{\pgfqpoint{5.315754in}{0.413320in}}%
\pgfpathlineto{\pgfqpoint{5.313089in}{0.413320in}}%
\pgfpathlineto{\pgfqpoint{5.310411in}{0.413320in}}%
\pgfpathlineto{\pgfqpoint{5.307731in}{0.413320in}}%
\pgfpathlineto{\pgfqpoint{5.305054in}{0.413320in}}%
\pgfpathlineto{\pgfqpoint{5.302443in}{0.413320in}}%
\pgfpathlineto{\pgfqpoint{5.299696in}{0.413320in}}%
\pgfpathlineto{\pgfqpoint{5.297140in}{0.413320in}}%
\pgfpathlineto{\pgfqpoint{5.294339in}{0.413320in}}%
\pgfpathlineto{\pgfqpoint{5.291794in}{0.413320in}}%
\pgfpathlineto{\pgfqpoint{5.288984in}{0.413320in}}%
\pgfpathlineto{\pgfqpoint{5.286436in}{0.413320in}}%
\pgfpathlineto{\pgfqpoint{5.283631in}{0.413320in}}%
\pgfpathlineto{\pgfqpoint{5.280947in}{0.413320in}}%
\pgfpathlineto{\pgfqpoint{5.278322in}{0.413320in}}%
\pgfpathlineto{\pgfqpoint{5.275589in}{0.413320in}}%
\pgfpathlineto{\pgfqpoint{5.272913in}{0.413320in}}%
\pgfpathlineto{\pgfqpoint{5.270238in}{0.413320in}}%
\pgfpathlineto{\pgfqpoint{5.267691in}{0.413320in}}%
\pgfpathlineto{\pgfqpoint{5.264876in}{0.413320in}}%
\pgfpathlineto{\pgfqpoint{5.262264in}{0.413320in}}%
\pgfpathlineto{\pgfqpoint{5.259511in}{0.413320in}}%
\pgfpathlineto{\pgfqpoint{5.256973in}{0.413320in}}%
\pgfpathlineto{\pgfqpoint{5.254236in}{0.413320in}}%
\pgfpathlineto{\pgfqpoint{5.251590in}{0.413320in}}%
\pgfpathlineto{\pgfqpoint{5.248816in}{0.413320in}}%
\pgfpathlineto{\pgfqpoint{5.246130in}{0.413320in}}%
\pgfpathlineto{\pgfqpoint{5.243445in}{0.413320in}}%
\pgfpathlineto{\pgfqpoint{5.240777in}{0.413320in}}%
\pgfpathlineto{\pgfqpoint{5.238173in}{0.413320in}}%
\pgfpathlineto{\pgfqpoint{5.235409in}{0.413320in}}%
\pgfpathlineto{\pgfqpoint{5.232855in}{0.413320in}}%
\pgfpathlineto{\pgfqpoint{5.230059in}{0.413320in}}%
\pgfpathlineto{\pgfqpoint{5.227470in}{0.413320in}}%
\pgfpathlineto{\pgfqpoint{5.224695in}{0.413320in}}%
\pgfpathlineto{\pgfqpoint{5.222151in}{0.413320in}}%
\pgfpathlineto{\pgfqpoint{5.219345in}{0.413320in}}%
\pgfpathlineto{\pgfqpoint{5.216667in}{0.413320in}}%
\pgfpathlineto{\pgfqpoint{5.214027in}{0.413320in}}%
\pgfpathlineto{\pgfqpoint{5.211299in}{0.413320in}}%
\pgfpathlineto{\pgfqpoint{5.208630in}{0.413320in}}%
\pgfpathlineto{\pgfqpoint{5.205952in}{0.413320in}}%
\pgfpathlineto{\pgfqpoint{5.203388in}{0.413320in}}%
\pgfpathlineto{\pgfqpoint{5.200594in}{0.413320in}}%
\pgfpathlineto{\pgfqpoint{5.198008in}{0.413320in}}%
\pgfpathlineto{\pgfqpoint{5.195239in}{0.413320in}}%
\pgfpathlineto{\pgfqpoint{5.192680in}{0.413320in}}%
\pgfpathlineto{\pgfqpoint{5.189880in}{0.413320in}}%
\pgfpathlineto{\pgfqpoint{5.187294in}{0.413320in}}%
\pgfpathlineto{\pgfqpoint{5.184522in}{0.413320in}}%
\pgfpathlineto{\pgfqpoint{5.181848in}{0.413320in}}%
\pgfpathlineto{\pgfqpoint{5.179188in}{0.413320in}}%
\pgfpathlineto{\pgfqpoint{5.176477in}{0.413320in}}%
\pgfpathlineto{\pgfqpoint{5.173925in}{0.413320in}}%
\pgfpathlineto{\pgfqpoint{5.171133in}{0.413320in}}%
\pgfpathlineto{\pgfqpoint{5.168591in}{0.413320in}}%
\pgfpathlineto{\pgfqpoint{5.165775in}{0.413320in}}%
\pgfpathlineto{\pgfqpoint{5.163243in}{0.413320in}}%
\pgfpathlineto{\pgfqpoint{5.160420in}{0.413320in}}%
\pgfpathlineto{\pgfqpoint{5.157815in}{0.413320in}}%
\pgfpathlineto{\pgfqpoint{5.155059in}{0.413320in}}%
\pgfpathlineto{\pgfqpoint{5.152382in}{0.413320in}}%
\pgfpathlineto{\pgfqpoint{5.149734in}{0.413320in}}%
\pgfpathlineto{\pgfqpoint{5.147029in}{0.413320in}}%
\pgfpathlineto{\pgfqpoint{5.144349in}{0.413320in}}%
\pgfpathlineto{\pgfqpoint{5.141660in}{0.413320in}}%
\pgfpathlineto{\pgfqpoint{5.139072in}{0.413320in}}%
\pgfpathlineto{\pgfqpoint{5.136311in}{0.413320in}}%
\pgfpathlineto{\pgfqpoint{5.133716in}{0.413320in}}%
\pgfpathlineto{\pgfqpoint{5.130953in}{0.413320in}}%
\pgfpathlineto{\pgfqpoint{5.128421in}{0.413320in}}%
\pgfpathlineto{\pgfqpoint{5.125599in}{0.413320in}}%
\pgfpathlineto{\pgfqpoint{5.123042in}{0.413320in}}%
\pgfpathlineto{\pgfqpoint{5.120243in}{0.413320in}}%
\pgfpathlineto{\pgfqpoint{5.117550in}{0.413320in}}%
\pgfpathlineto{\pgfqpoint{5.114887in}{0.413320in}}%
\pgfpathlineto{\pgfqpoint{5.112209in}{0.413320in}}%
\pgfpathlineto{\pgfqpoint{5.109530in}{0.413320in}}%
\pgfpathlineto{\pgfqpoint{5.106842in}{0.413320in}}%
\pgfpathlineto{\pgfqpoint{5.104312in}{0.413320in}}%
\pgfpathlineto{\pgfqpoint{5.101496in}{0.413320in}}%
\pgfpathlineto{\pgfqpoint{5.098948in}{0.413320in}}%
\pgfpathlineto{\pgfqpoint{5.096142in}{0.413320in}}%
\pgfpathlineto{\pgfqpoint{5.093579in}{0.413320in}}%
\pgfpathlineto{\pgfqpoint{5.090788in}{0.413320in}}%
\pgfpathlineto{\pgfqpoint{5.088103in}{0.413320in}}%
\pgfpathlineto{\pgfqpoint{5.085426in}{0.413320in}}%
\pgfpathlineto{\pgfqpoint{5.082746in}{0.413320in}}%
\pgfpathlineto{\pgfqpoint{5.080067in}{0.413320in}}%
\pgfpathlineto{\pgfqpoint{5.077390in}{0.413320in}}%
\pgfpathlineto{\pgfqpoint{5.074851in}{0.413320in}}%
\pgfpathlineto{\pgfqpoint{5.072030in}{0.413320in}}%
\pgfpathlineto{\pgfqpoint{5.069463in}{0.413320in}}%
\pgfpathlineto{\pgfqpoint{5.066677in}{0.413320in}}%
\pgfpathlineto{\pgfqpoint{5.064144in}{0.413320in}}%
\pgfpathlineto{\pgfqpoint{5.061315in}{0.413320in}}%
\pgfpathlineto{\pgfqpoint{5.058711in}{0.413320in}}%
\pgfpathlineto{\pgfqpoint{5.055952in}{0.413320in}}%
\pgfpathlineto{\pgfqpoint{5.053284in}{0.413320in}}%
\pgfpathlineto{\pgfqpoint{5.050606in}{0.413320in}}%
\pgfpathlineto{\pgfqpoint{5.047924in}{0.413320in}}%
\pgfpathlineto{\pgfqpoint{5.045249in}{0.413320in}}%
\pgfpathlineto{\pgfqpoint{5.042572in}{0.413320in}}%
\pgfpathlineto{\pgfqpoint{5.039962in}{0.413320in}}%
\pgfpathlineto{\pgfqpoint{5.037214in}{0.413320in}}%
\pgfpathlineto{\pgfqpoint{5.034649in}{0.413320in}}%
\pgfpathlineto{\pgfqpoint{5.031849in}{0.413320in}}%
\pgfpathlineto{\pgfqpoint{5.029275in}{0.413320in}}%
\pgfpathlineto{\pgfqpoint{5.026501in}{0.413320in}}%
\pgfpathlineto{\pgfqpoint{5.023927in}{0.413320in}}%
\pgfpathlineto{\pgfqpoint{5.021147in}{0.413320in}}%
\pgfpathlineto{\pgfqpoint{5.018466in}{0.413320in}}%
\pgfpathlineto{\pgfqpoint{5.015820in}{0.413320in}}%
\pgfpathlineto{\pgfqpoint{5.013104in}{0.413320in}}%
\pgfpathlineto{\pgfqpoint{5.010562in}{0.413320in}}%
\pgfpathlineto{\pgfqpoint{5.007751in}{0.413320in}}%
\pgfpathlineto{\pgfqpoint{5.005178in}{0.413320in}}%
\pgfpathlineto{\pgfqpoint{5.002384in}{0.413320in}}%
\pgfpathlineto{\pgfqpoint{4.999780in}{0.413320in}}%
\pgfpathlineto{\pgfqpoint{4.997028in}{0.413320in}}%
\pgfpathlineto{\pgfqpoint{4.994390in}{0.413320in}}%
\pgfpathlineto{\pgfqpoint{4.991683in}{0.413320in}}%
\pgfpathlineto{\pgfqpoint{4.989001in}{0.413320in}}%
\pgfpathlineto{\pgfqpoint{4.986325in}{0.413320in}}%
\pgfpathlineto{\pgfqpoint{4.983637in}{0.413320in}}%
\pgfpathlineto{\pgfqpoint{4.980967in}{0.413320in}}%
\pgfpathlineto{\pgfqpoint{4.978287in}{0.413320in}}%
\pgfpathlineto{\pgfqpoint{4.975703in}{0.413320in}}%
\pgfpathlineto{\pgfqpoint{4.972933in}{0.413320in}}%
\pgfpathlineto{\pgfqpoint{4.970314in}{0.413320in}}%
\pgfpathlineto{\pgfqpoint{4.967575in}{0.413320in}}%
\pgfpathlineto{\pgfqpoint{4.965002in}{0.413320in}}%
\pgfpathlineto{\pgfqpoint{4.962219in}{0.413320in}}%
\pgfpathlineto{\pgfqpoint{4.959689in}{0.413320in}}%
\pgfpathlineto{\pgfqpoint{4.956862in}{0.413320in}}%
\pgfpathlineto{\pgfqpoint{4.954182in}{0.413320in}}%
\pgfpathlineto{\pgfqpoint{4.951504in}{0.413320in}}%
\pgfpathlineto{\pgfqpoint{4.948827in}{0.413320in}}%
\pgfpathlineto{\pgfqpoint{4.946151in}{0.413320in}}%
\pgfpathlineto{\pgfqpoint{4.943466in}{0.413320in}}%
\pgfpathlineto{\pgfqpoint{4.940881in}{0.413320in}}%
\pgfpathlineto{\pgfqpoint{4.938112in}{0.413320in}}%
\pgfpathlineto{\pgfqpoint{4.935515in}{0.413320in}}%
\pgfpathlineto{\pgfqpoint{4.932742in}{0.413320in}}%
\pgfpathlineto{\pgfqpoint{4.930170in}{0.413320in}}%
\pgfpathlineto{\pgfqpoint{4.927400in}{0.413320in}}%
\pgfpathlineto{\pgfqpoint{4.924708in}{0.413320in}}%
\pgfpathlineto{\pgfqpoint{4.922041in}{0.413320in}}%
\pgfpathlineto{\pgfqpoint{4.919352in}{0.413320in}}%
\pgfpathlineto{\pgfqpoint{4.916681in}{0.413320in}}%
\pgfpathlineto{\pgfqpoint{4.914009in}{0.413320in}}%
\pgfpathlineto{\pgfqpoint{4.911435in}{0.413320in}}%
\pgfpathlineto{\pgfqpoint{4.908648in}{0.413320in}}%
\pgfpathlineto{\pgfqpoint{4.906096in}{0.413320in}}%
\pgfpathlineto{\pgfqpoint{4.903295in}{0.413320in}}%
\pgfpathlineto{\pgfqpoint{4.900712in}{0.413320in}}%
\pgfpathlineto{\pgfqpoint{4.897938in}{0.413320in}}%
\pgfpathlineto{\pgfqpoint{4.895399in}{0.413320in}}%
\pgfpathlineto{\pgfqpoint{4.892611in}{0.413320in}}%
\pgfpathlineto{\pgfqpoint{4.889902in}{0.413320in}}%
\pgfpathlineto{\pgfqpoint{4.887211in}{0.413320in}}%
\pgfpathlineto{\pgfqpoint{4.884540in}{0.413320in}}%
\pgfpathlineto{\pgfqpoint{4.881864in}{0.413320in}}%
\pgfpathlineto{\pgfqpoint{4.879180in}{0.413320in}}%
\pgfpathlineto{\pgfqpoint{4.876636in}{0.413320in}}%
\pgfpathlineto{\pgfqpoint{4.873832in}{0.413320in}}%
\pgfpathlineto{\pgfqpoint{4.871209in}{0.413320in}}%
\pgfpathlineto{\pgfqpoint{4.868474in}{0.413320in}}%
\pgfpathlineto{\pgfqpoint{4.865910in}{0.413320in}}%
\pgfpathlineto{\pgfqpoint{4.863116in}{0.413320in}}%
\pgfpathlineto{\pgfqpoint{4.860544in}{0.413320in}}%
\pgfpathlineto{\pgfqpoint{4.857807in}{0.413320in}}%
\pgfpathlineto{\pgfqpoint{4.855070in}{0.413320in}}%
\pgfpathlineto{\pgfqpoint{4.852404in}{0.413320in}}%
\pgfpathlineto{\pgfqpoint{4.849715in}{0.413320in}}%
\pgfpathlineto{\pgfqpoint{4.847127in}{0.413320in}}%
\pgfpathlineto{\pgfqpoint{4.844361in}{0.413320in}}%
\pgfpathlineto{\pgfqpoint{4.842380in}{0.413320in}}%
\pgfpathlineto{\pgfqpoint{4.839922in}{0.413320in}}%
\pgfpathlineto{\pgfqpoint{4.837992in}{0.413320in}}%
\pgfpathlineto{\pgfqpoint{4.833657in}{0.413320in}}%
\pgfpathlineto{\pgfqpoint{4.831045in}{0.413320in}}%
\pgfpathlineto{\pgfqpoint{4.828291in}{0.413320in}}%
\pgfpathlineto{\pgfqpoint{4.825619in}{0.413320in}}%
\pgfpathlineto{\pgfqpoint{4.822945in}{0.413320in}}%
\pgfpathlineto{\pgfqpoint{4.820265in}{0.413320in}}%
\pgfpathlineto{\pgfqpoint{4.817587in}{0.413320in}}%
\pgfpathlineto{\pgfqpoint{4.814907in}{0.413320in}}%
\pgfpathlineto{\pgfqpoint{4.812377in}{0.413320in}}%
\pgfpathlineto{\pgfqpoint{4.809538in}{0.413320in}}%
\pgfpathlineto{\pgfqpoint{4.807017in}{0.413320in}}%
\pgfpathlineto{\pgfqpoint{4.804193in}{0.413320in}}%
\pgfpathlineto{\pgfqpoint{4.801586in}{0.413320in}}%
\pgfpathlineto{\pgfqpoint{4.798830in}{0.413320in}}%
\pgfpathlineto{\pgfqpoint{4.796274in}{0.413320in}}%
\pgfpathlineto{\pgfqpoint{4.793512in}{0.413320in}}%
\pgfpathlineto{\pgfqpoint{4.790798in}{0.413320in}}%
\pgfpathlineto{\pgfqpoint{4.788116in}{0.413320in}}%
\pgfpathlineto{\pgfqpoint{4.785445in}{0.413320in}}%
\pgfpathlineto{\pgfqpoint{4.782872in}{0.413320in}}%
\pgfpathlineto{\pgfqpoint{4.780083in}{0.413320in}}%
\pgfpathlineto{\pgfqpoint{4.777535in}{0.413320in}}%
\pgfpathlineto{\pgfqpoint{4.774732in}{0.413320in}}%
\pgfpathlineto{\pgfqpoint{4.772198in}{0.413320in}}%
\pgfpathlineto{\pgfqpoint{4.769367in}{0.413320in}}%
\pgfpathlineto{\pgfqpoint{4.766783in}{0.413320in}}%
\pgfpathlineto{\pgfqpoint{4.764018in}{0.413320in}}%
\pgfpathlineto{\pgfqpoint{4.761337in}{0.413320in}}%
\pgfpathlineto{\pgfqpoint{4.758653in}{0.413320in}}%
\pgfpathlineto{\pgfqpoint{4.755983in}{0.413320in}}%
\pgfpathlineto{\pgfqpoint{4.753298in}{0.413320in}}%
\pgfpathlineto{\pgfqpoint{4.750627in}{0.413320in}}%
\pgfpathlineto{\pgfqpoint{4.748081in}{0.413320in}}%
\pgfpathlineto{\pgfqpoint{4.745256in}{0.413320in}}%
\pgfpathlineto{\pgfqpoint{4.742696in}{0.413320in}}%
\pgfpathlineto{\pgfqpoint{4.739912in}{0.413320in}}%
\pgfpathlineto{\pgfqpoint{4.737348in}{0.413320in}}%
\pgfpathlineto{\pgfqpoint{4.734552in}{0.413320in}}%
\pgfpathlineto{\pgfqpoint{4.731901in}{0.413320in}}%
\pgfpathlineto{\pgfqpoint{4.729233in}{0.413320in}}%
\pgfpathlineto{\pgfqpoint{4.726508in}{0.413320in}}%
\pgfpathlineto{\pgfqpoint{4.723873in}{0.413320in}}%
\pgfpathlineto{\pgfqpoint{4.721160in}{0.413320in}}%
\pgfpathlineto{\pgfqpoint{4.718486in}{0.413320in}}%
\pgfpathlineto{\pgfqpoint{4.715806in}{0.413320in}}%
\pgfpathlineto{\pgfqpoint{4.713275in}{0.413320in}}%
\pgfpathlineto{\pgfqpoint{4.710437in}{0.413320in}}%
\pgfpathlineto{\pgfqpoint{4.707824in}{0.413320in}}%
\pgfpathlineto{\pgfqpoint{4.705094in}{0.413320in}}%
\pgfpathlineto{\pgfqpoint{4.702517in}{0.413320in}}%
\pgfpathlineto{\pgfqpoint{4.699734in}{0.413320in}}%
\pgfpathlineto{\pgfqpoint{4.697170in}{0.413320in}}%
\pgfpathlineto{\pgfqpoint{4.694381in}{0.413320in}}%
\pgfpathlineto{\pgfqpoint{4.691694in}{0.413320in}}%
\pgfpathlineto{\pgfqpoint{4.689051in}{0.413320in}}%
\pgfpathlineto{\pgfqpoint{4.686337in}{0.413320in}}%
\pgfpathlineto{\pgfqpoint{4.683799in}{0.413320in}}%
\pgfpathlineto{\pgfqpoint{4.680988in}{0.413320in}}%
\pgfpathlineto{\pgfqpoint{4.678448in}{0.413320in}}%
\pgfpathlineto{\pgfqpoint{4.675619in}{0.413320in}}%
\pgfpathlineto{\pgfqpoint{4.673068in}{0.413320in}}%
\pgfpathlineto{\pgfqpoint{4.670261in}{0.413320in}}%
\pgfpathlineto{\pgfqpoint{4.667764in}{0.413320in}}%
\pgfpathlineto{\pgfqpoint{4.664923in}{0.413320in}}%
\pgfpathlineto{\pgfqpoint{4.662237in}{0.413320in}}%
\pgfpathlineto{\pgfqpoint{4.659590in}{0.413320in}}%
\pgfpathlineto{\pgfqpoint{4.656873in}{0.413320in}}%
\pgfpathlineto{\pgfqpoint{4.654203in}{0.413320in}}%
\pgfpathlineto{\pgfqpoint{4.651524in}{0.413320in}}%
\pgfpathlineto{\pgfqpoint{4.648922in}{0.413320in}}%
\pgfpathlineto{\pgfqpoint{4.646169in}{0.413320in}}%
\pgfpathlineto{\pgfqpoint{4.643628in}{0.413320in}}%
\pgfpathlineto{\pgfqpoint{4.640809in}{0.413320in}}%
\pgfpathlineto{\pgfqpoint{4.638204in}{0.413320in}}%
\pgfpathlineto{\pgfqpoint{4.635445in}{0.413320in}}%
\pgfpathlineto{\pgfqpoint{4.632902in}{0.413320in}}%
\pgfpathlineto{\pgfqpoint{4.630096in}{0.413320in}}%
\pgfpathlineto{\pgfqpoint{4.627411in}{0.413320in}}%
\pgfpathlineto{\pgfqpoint{4.624741in}{0.413320in}}%
\pgfpathlineto{\pgfqpoint{4.622056in}{0.413320in}}%
\pgfpathlineto{\pgfqpoint{4.619529in}{0.413320in}}%
\pgfpathlineto{\pgfqpoint{4.616702in}{0.413320in}}%
\pgfpathlineto{\pgfqpoint{4.614134in}{0.413320in}}%
\pgfpathlineto{\pgfqpoint{4.611350in}{0.413320in}}%
\pgfpathlineto{\pgfqpoint{4.608808in}{0.413320in}}%
\pgfpathlineto{\pgfqpoint{4.605990in}{0.413320in}}%
\pgfpathlineto{\pgfqpoint{4.603430in}{0.413320in}}%
\pgfpathlineto{\pgfqpoint{4.600633in}{0.413320in}}%
\pgfpathlineto{\pgfqpoint{4.597951in}{0.413320in}}%
\pgfpathlineto{\pgfqpoint{4.595281in}{0.413320in}}%
\pgfpathlineto{\pgfqpoint{4.592589in}{0.413320in}}%
\pgfpathlineto{\pgfqpoint{4.589920in}{0.413320in}}%
\pgfpathlineto{\pgfqpoint{4.587244in}{0.413320in}}%
\pgfpathlineto{\pgfqpoint{4.584672in}{0.413320in}}%
\pgfpathlineto{\pgfqpoint{4.581888in}{0.413320in}}%
\pgfpathlineto{\pgfqpoint{4.579305in}{0.413320in}}%
\pgfpathlineto{\pgfqpoint{4.576531in}{0.413320in}}%
\pgfpathlineto{\pgfqpoint{4.573947in}{0.413320in}}%
\pgfpathlineto{\pgfqpoint{4.571171in}{0.413320in}}%
\pgfpathlineto{\pgfqpoint{4.568612in}{0.413320in}}%
\pgfpathlineto{\pgfqpoint{4.565820in}{0.413320in}}%
\pgfpathlineto{\pgfqpoint{4.563125in}{0.413320in}}%
\pgfpathlineto{\pgfqpoint{4.560448in}{0.413320in}}%
\pgfpathlineto{\pgfqpoint{4.557777in}{0.413320in}}%
\pgfpathlineto{\pgfqpoint{4.555106in}{0.413320in}}%
\pgfpathlineto{\pgfqpoint{4.552425in}{0.413320in}}%
\pgfpathlineto{\pgfqpoint{4.549822in}{0.413320in}}%
\pgfpathlineto{\pgfqpoint{4.547064in}{0.413320in}}%
\pgfpathlineto{\pgfqpoint{4.544464in}{0.413320in}}%
\pgfpathlineto{\pgfqpoint{4.541711in}{0.413320in}}%
\pgfpathlineto{\pgfqpoint{4.539144in}{0.413320in}}%
\pgfpathlineto{\pgfqpoint{4.536400in}{0.413320in}}%
\pgfpathlineto{\pgfqpoint{4.533764in}{0.413320in}}%
\pgfpathlineto{\pgfqpoint{4.530990in}{0.413320in}}%
\pgfpathlineto{\pgfqpoint{4.528307in}{0.413320in}}%
\pgfpathlineto{\pgfqpoint{4.525640in}{0.413320in}}%
\pgfpathlineto{\pgfqpoint{4.522962in}{0.413320in}}%
\pgfpathlineto{\pgfqpoint{4.520345in}{0.413320in}}%
\pgfpathlineto{\pgfqpoint{4.517598in}{0.413320in}}%
\pgfpathlineto{\pgfqpoint{4.515080in}{0.413320in}}%
\pgfpathlineto{\pgfqpoint{4.512246in}{0.413320in}}%
\pgfpathlineto{\pgfqpoint{4.509643in}{0.413320in}}%
\pgfpathlineto{\pgfqpoint{4.506893in}{0.413320in}}%
\pgfpathlineto{\pgfqpoint{4.504305in}{0.413320in}}%
\pgfpathlineto{\pgfqpoint{4.501529in}{0.413320in}}%
\pgfpathlineto{\pgfqpoint{4.498850in}{0.413320in}}%
\pgfpathlineto{\pgfqpoint{4.496167in}{0.413320in}}%
\pgfpathlineto{\pgfqpoint{4.493492in}{0.413320in}}%
\pgfpathlineto{\pgfqpoint{4.490822in}{0.413320in}}%
\pgfpathlineto{\pgfqpoint{4.488130in}{0.413320in}}%
\pgfpathlineto{\pgfqpoint{4.485581in}{0.413320in}}%
\pgfpathlineto{\pgfqpoint{4.482778in}{0.413320in}}%
\pgfpathlineto{\pgfqpoint{4.480201in}{0.413320in}}%
\pgfpathlineto{\pgfqpoint{4.477430in}{0.413320in}}%
\pgfpathlineto{\pgfqpoint{4.474861in}{0.413320in}}%
\pgfpathlineto{\pgfqpoint{4.472059in}{0.413320in}}%
\pgfpathlineto{\pgfqpoint{4.469492in}{0.413320in}}%
\pgfpathlineto{\pgfqpoint{4.466717in}{0.413320in}}%
\pgfpathlineto{\pgfqpoint{4.464029in}{0.413320in}}%
\pgfpathlineto{\pgfqpoint{4.461367in}{0.413320in}}%
\pgfpathlineto{\pgfqpoint{4.458681in}{0.413320in}}%
\pgfpathlineto{\pgfqpoint{4.456138in}{0.413320in}}%
\pgfpathlineto{\pgfqpoint{4.453312in}{0.413320in}}%
\pgfpathlineto{\pgfqpoint{4.450767in}{0.413320in}}%
\pgfpathlineto{\pgfqpoint{4.447965in}{0.413320in}}%
\pgfpathlineto{\pgfqpoint{4.445423in}{0.413320in}}%
\pgfpathlineto{\pgfqpoint{4.442611in}{0.413320in}}%
\pgfpathlineto{\pgfqpoint{4.440041in}{0.413320in}}%
\pgfpathlineto{\pgfqpoint{4.437253in}{0.413320in}}%
\pgfpathlineto{\pgfqpoint{4.434569in}{0.413320in}}%
\pgfpathlineto{\pgfqpoint{4.431901in}{0.413320in}}%
\pgfpathlineto{\pgfqpoint{4.429220in}{0.413320in}}%
\pgfpathlineto{\pgfqpoint{4.426534in}{0.413320in}}%
\pgfpathlineto{\pgfqpoint{4.423863in}{0.413320in}}%
\pgfpathlineto{\pgfqpoint{4.421292in}{0.413320in}}%
\pgfpathlineto{\pgfqpoint{4.418506in}{0.413320in}}%
\pgfpathlineto{\pgfqpoint{4.415932in}{0.413320in}}%
\pgfpathlineto{\pgfqpoint{4.413149in}{0.413320in}}%
\pgfpathlineto{\pgfqpoint{4.410587in}{0.413320in}}%
\pgfpathlineto{\pgfqpoint{4.407788in}{0.413320in}}%
\pgfpathlineto{\pgfqpoint{4.405234in}{0.413320in}}%
\pgfpathlineto{\pgfqpoint{4.402468in}{0.413320in}}%
\pgfpathlineto{\pgfqpoint{4.399745in}{0.413320in}}%
\pgfpathlineto{\pgfqpoint{4.397076in}{0.413320in}}%
\pgfpathlineto{\pgfqpoint{4.394400in}{0.413320in}}%
\pgfpathlineto{\pgfqpoint{4.391721in}{0.413320in}}%
\pgfpathlineto{\pgfqpoint{4.389044in}{0.413320in}}%
\pgfpathlineto{\pgfqpoint{4.386431in}{0.413320in}}%
\pgfpathlineto{\pgfqpoint{4.383674in}{0.413320in}}%
\pgfpathlineto{\pgfqpoint{4.381097in}{0.413320in}}%
\pgfpathlineto{\pgfqpoint{4.378329in}{0.413320in}}%
\pgfpathlineto{\pgfqpoint{4.375761in}{0.413320in}}%
\pgfpathlineto{\pgfqpoint{4.372976in}{0.413320in}}%
\pgfpathlineto{\pgfqpoint{4.370437in}{0.413320in}}%
\pgfpathlineto{\pgfqpoint{4.367646in}{0.413320in}}%
\pgfpathlineto{\pgfqpoint{4.364936in}{0.413320in}}%
\pgfpathlineto{\pgfqpoint{4.362270in}{0.413320in}}%
\pgfpathlineto{\pgfqpoint{4.359582in}{0.413320in}}%
\pgfpathlineto{\pgfqpoint{4.357014in}{0.413320in}}%
\pgfpathlineto{\pgfqpoint{4.354224in}{0.413320in}}%
\pgfpathlineto{\pgfqpoint{4.351645in}{0.413320in}}%
\pgfpathlineto{\pgfqpoint{4.348868in}{0.413320in}}%
\pgfpathlineto{\pgfqpoint{4.346263in}{0.413320in}}%
\pgfpathlineto{\pgfqpoint{4.343510in}{0.413320in}}%
\pgfpathlineto{\pgfqpoint{4.340976in}{0.413320in}}%
\pgfpathlineto{\pgfqpoint{4.338154in}{0.413320in}}%
\pgfpathlineto{\pgfqpoint{4.335463in}{0.413320in}}%
\pgfpathlineto{\pgfqpoint{4.332796in}{0.413320in}}%
\pgfpathlineto{\pgfqpoint{4.330118in}{0.413320in}}%
\pgfpathlineto{\pgfqpoint{4.327440in}{0.413320in}}%
\pgfpathlineto{\pgfqpoint{4.324760in}{0.413320in}}%
\pgfpathlineto{\pgfqpoint{4.322181in}{0.413320in}}%
\pgfpathlineto{\pgfqpoint{4.319405in}{0.413320in}}%
\pgfpathlineto{\pgfqpoint{4.316856in}{0.413320in}}%
\pgfpathlineto{\pgfqpoint{4.314032in}{0.413320in}}%
\pgfpathlineto{\pgfqpoint{4.311494in}{0.413320in}}%
\pgfpathlineto{\pgfqpoint{4.308691in}{0.413320in}}%
\pgfpathlineto{\pgfqpoint{4.306118in}{0.413320in}}%
\pgfpathlineto{\pgfqpoint{4.303357in}{0.413320in}}%
\pgfpathlineto{\pgfqpoint{4.300656in}{0.413320in}}%
\pgfpathlineto{\pgfqpoint{4.297977in}{0.413320in}}%
\pgfpathlineto{\pgfqpoint{4.295299in}{0.413320in}}%
\pgfpathlineto{\pgfqpoint{4.292786in}{0.413320in}}%
\pgfpathlineto{\pgfqpoint{4.289936in}{0.413320in}}%
\pgfpathlineto{\pgfqpoint{4.287399in}{0.413320in}}%
\pgfpathlineto{\pgfqpoint{4.284586in}{0.413320in}}%
\pgfpathlineto{\pgfqpoint{4.282000in}{0.413320in}}%
\pgfpathlineto{\pgfqpoint{4.279212in}{0.413320in}}%
\pgfpathlineto{\pgfqpoint{4.276635in}{0.413320in}}%
\pgfpathlineto{\pgfqpoint{4.273874in}{0.413320in}}%
\pgfpathlineto{\pgfqpoint{4.271187in}{0.413320in}}%
\pgfpathlineto{\pgfqpoint{4.268590in}{0.413320in}}%
\pgfpathlineto{\pgfqpoint{4.265824in}{0.413320in}}%
\pgfpathlineto{\pgfqpoint{4.263157in}{0.413320in}}%
\pgfpathlineto{\pgfqpoint{4.260477in}{0.413320in}}%
\pgfpathlineto{\pgfqpoint{4.257958in}{0.413320in}}%
\pgfpathlineto{\pgfqpoint{4.255120in}{0.413320in}}%
\pgfpathlineto{\pgfqpoint{4.252581in}{0.413320in}}%
\pgfpathlineto{\pgfqpoint{4.249767in}{0.413320in}}%
\pgfpathlineto{\pgfqpoint{4.247225in}{0.413320in}}%
\pgfpathlineto{\pgfqpoint{4.244394in}{0.413320in}}%
\pgfpathlineto{\pgfqpoint{4.241900in}{0.413320in}}%
\pgfpathlineto{\pgfqpoint{4.239084in}{0.413320in}}%
\pgfpathlineto{\pgfqpoint{4.236375in}{0.413320in}}%
\pgfpathlineto{\pgfqpoint{4.233691in}{0.413320in}}%
\pgfpathlineto{\pgfqpoint{4.231013in}{0.413320in}}%
\pgfpathlineto{\pgfqpoint{4.228331in}{0.413320in}}%
\pgfpathlineto{\pgfqpoint{4.225654in}{0.413320in}}%
\pgfpathlineto{\pgfqpoint{4.223082in}{0.413320in}}%
\pgfpathlineto{\pgfqpoint{4.220304in}{0.413320in}}%
\pgfpathlineto{\pgfqpoint{4.217694in}{0.413320in}}%
\pgfpathlineto{\pgfqpoint{4.214948in}{0.413320in}}%
\pgfpathlineto{\pgfqpoint{4.212383in}{0.413320in}}%
\pgfpathlineto{\pgfqpoint{4.209597in}{0.413320in}}%
\pgfpathlineto{\pgfqpoint{4.207076in}{0.413320in}}%
\pgfpathlineto{\pgfqpoint{4.204240in}{0.413320in}}%
\pgfpathlineto{\pgfqpoint{4.201542in}{0.413320in}}%
\pgfpathlineto{\pgfqpoint{4.198878in}{0.413320in}}%
\pgfpathlineto{\pgfqpoint{4.196186in}{0.413320in}}%
\pgfpathlineto{\pgfqpoint{4.193638in}{0.413320in}}%
\pgfpathlineto{\pgfqpoint{4.190842in}{0.413320in}}%
\pgfpathlineto{\pgfqpoint{4.188318in}{0.413320in}}%
\pgfpathlineto{\pgfqpoint{4.185481in}{0.413320in}}%
\pgfpathlineto{\pgfqpoint{4.182899in}{0.413320in}}%
\pgfpathlineto{\pgfqpoint{4.180129in}{0.413320in}}%
\pgfpathlineto{\pgfqpoint{4.177593in}{0.413320in}}%
\pgfpathlineto{\pgfqpoint{4.174770in}{0.413320in}}%
\pgfpathlineto{\pgfqpoint{4.172093in}{0.413320in}}%
\pgfpathlineto{\pgfqpoint{4.169415in}{0.413320in}}%
\pgfpathlineto{\pgfqpoint{4.166737in}{0.413320in}}%
\pgfpathlineto{\pgfqpoint{4.164059in}{0.413320in}}%
\pgfpathlineto{\pgfqpoint{4.161380in}{0.413320in}}%
\pgfpathlineto{\pgfqpoint{4.158806in}{0.413320in}}%
\pgfpathlineto{\pgfqpoint{4.156016in}{0.413320in}}%
\pgfpathlineto{\pgfqpoint{4.153423in}{0.413320in}}%
\pgfpathlineto{\pgfqpoint{4.150665in}{0.413320in}}%
\pgfpathlineto{\pgfqpoint{4.148082in}{0.413320in}}%
\pgfpathlineto{\pgfqpoint{4.145310in}{0.413320in}}%
\pgfpathlineto{\pgfqpoint{4.142713in}{0.413320in}}%
\pgfpathlineto{\pgfqpoint{4.139963in}{0.413320in}}%
\pgfpathlineto{\pgfqpoint{4.137272in}{0.413320in}}%
\pgfpathlineto{\pgfqpoint{4.134615in}{0.413320in}}%
\pgfpathlineto{\pgfqpoint{4.131920in}{0.413320in}}%
\pgfpathlineto{\pgfqpoint{4.129349in}{0.413320in}}%
\pgfpathlineto{\pgfqpoint{4.126553in}{0.413320in}}%
\pgfpathlineto{\pgfqpoint{4.124019in}{0.413320in}}%
\pgfpathlineto{\pgfqpoint{4.121205in}{0.413320in}}%
\pgfpathlineto{\pgfqpoint{4.118554in}{0.413320in}}%
\pgfpathlineto{\pgfqpoint{4.115844in}{0.413320in}}%
\pgfpathlineto{\pgfqpoint{4.113252in}{0.413320in}}%
\pgfpathlineto{\pgfqpoint{4.110488in}{0.413320in}}%
\pgfpathlineto{\pgfqpoint{4.107814in}{0.413320in}}%
\pgfpathlineto{\pgfqpoint{4.105185in}{0.413320in}}%
\pgfpathlineto{\pgfqpoint{4.102456in}{0.413320in}}%
\pgfpathlineto{\pgfqpoint{4.099777in}{0.413320in}}%
\pgfpathlineto{\pgfqpoint{4.097092in}{0.413320in}}%
\pgfpathlineto{\pgfqpoint{4.094527in}{0.413320in}}%
\pgfpathlineto{\pgfqpoint{4.091729in}{0.413320in}}%
\pgfpathlineto{\pgfqpoint{4.089159in}{0.413320in}}%
\pgfpathlineto{\pgfqpoint{4.086385in}{0.413320in}}%
\pgfpathlineto{\pgfqpoint{4.083870in}{0.413320in}}%
\pgfpathlineto{\pgfqpoint{4.081018in}{0.413320in}}%
\pgfpathlineto{\pgfqpoint{4.078471in}{0.413320in}}%
\pgfpathlineto{\pgfqpoint{4.075705in}{0.413320in}}%
\pgfpathlineto{\pgfqpoint{4.072985in}{0.413320in}}%
\pgfpathlineto{\pgfqpoint{4.070313in}{0.413320in}}%
\pgfpathlineto{\pgfqpoint{4.067636in}{0.413320in}}%
\pgfpathlineto{\pgfqpoint{4.064957in}{0.413320in}}%
\pgfpathlineto{\pgfqpoint{4.062266in}{0.413320in}}%
\pgfpathlineto{\pgfqpoint{4.059702in}{0.413320in}}%
\pgfpathlineto{\pgfqpoint{4.056911in}{0.413320in}}%
\pgfpathlineto{\pgfqpoint{4.054326in}{0.413320in}}%
\pgfpathlineto{\pgfqpoint{4.051557in}{0.413320in}}%
\pgfpathlineto{\pgfqpoint{4.049006in}{0.413320in}}%
\pgfpathlineto{\pgfqpoint{4.046210in}{0.413320in}}%
\pgfpathlineto{\pgfqpoint{4.043667in}{0.413320in}}%
\pgfpathlineto{\pgfqpoint{4.040852in}{0.413320in}}%
\pgfpathlineto{\pgfqpoint{4.038174in}{0.413320in}}%
\pgfpathlineto{\pgfqpoint{4.035492in}{0.413320in}}%
\pgfpathlineto{\pgfqpoint{4.032817in}{0.413320in}}%
\pgfpathlineto{\pgfqpoint{4.030229in}{0.413320in}}%
\pgfpathlineto{\pgfqpoint{4.027447in}{0.413320in}}%
\pgfpathlineto{\pgfqpoint{4.024868in}{0.413320in}}%
\pgfpathlineto{\pgfqpoint{4.022097in}{0.413320in}}%
\pgfpathlineto{\pgfqpoint{4.019518in}{0.413320in}}%
\pgfpathlineto{\pgfqpoint{4.016744in}{0.413320in}}%
\pgfpathlineto{\pgfqpoint{4.014186in}{0.413320in}}%
\pgfpathlineto{\pgfqpoint{4.011394in}{0.413320in}}%
\pgfpathlineto{\pgfqpoint{4.008699in}{0.413320in}}%
\pgfpathlineto{\pgfqpoint{4.006034in}{0.413320in}}%
\pgfpathlineto{\pgfqpoint{4.003348in}{0.413320in}}%
\pgfpathlineto{\pgfqpoint{4.000674in}{0.413320in}}%
\pgfpathlineto{\pgfqpoint{3.997990in}{0.413320in}}%
\pgfpathlineto{\pgfqpoint{3.995417in}{0.413320in}}%
\pgfpathlineto{\pgfqpoint{3.992642in}{0.413320in}}%
\pgfpathlineto{\pgfqpoint{3.990055in}{0.413320in}}%
\pgfpathlineto{\pgfqpoint{3.987270in}{0.413320in}}%
\pgfpathlineto{\pgfqpoint{3.984714in}{0.413320in}}%
\pgfpathlineto{\pgfqpoint{3.981929in}{0.413320in}}%
\pgfpathlineto{\pgfqpoint{3.979389in}{0.413320in}}%
\pgfpathlineto{\pgfqpoint{3.976563in}{0.413320in}}%
\pgfpathlineto{\pgfqpoint{3.973885in}{0.413320in}}%
\pgfpathlineto{\pgfqpoint{3.971250in}{0.413320in}}%
\pgfpathlineto{\pgfqpoint{3.968523in}{0.413320in}}%
\pgfpathlineto{\pgfqpoint{3.966013in}{0.413320in}}%
\pgfpathlineto{\pgfqpoint{3.963176in}{0.413320in}}%
\pgfpathlineto{\pgfqpoint{3.960635in}{0.413320in}}%
\pgfpathlineto{\pgfqpoint{3.957823in}{0.413320in}}%
\pgfpathlineto{\pgfqpoint{3.955211in}{0.413320in}}%
\pgfpathlineto{\pgfqpoint{3.952464in}{0.413320in}}%
\pgfpathlineto{\pgfqpoint{3.949894in}{0.413320in}}%
\pgfpathlineto{\pgfqpoint{3.947101in}{0.413320in}}%
\pgfpathlineto{\pgfqpoint{3.944431in}{0.413320in}}%
\pgfpathlineto{\pgfqpoint{3.941778in}{0.413320in}}%
\pgfpathlineto{\pgfqpoint{3.939075in}{0.413320in}}%
\pgfpathlineto{\pgfqpoint{3.936395in}{0.413320in}}%
\pgfpathlineto{\pgfqpoint{3.933714in}{0.413320in}}%
\pgfpathlineto{\pgfqpoint{3.931202in}{0.413320in}}%
\pgfpathlineto{\pgfqpoint{3.928347in}{0.413320in}}%
\pgfpathlineto{\pgfqpoint{3.925778in}{0.413320in}}%
\pgfpathlineto{\pgfqpoint{3.923005in}{0.413320in}}%
\pgfpathlineto{\pgfqpoint{3.920412in}{0.413320in}}%
\pgfpathlineto{\pgfqpoint{3.917646in}{0.413320in}}%
\pgfpathlineto{\pgfqpoint{3.915107in}{0.413320in}}%
\pgfpathlineto{\pgfqpoint{3.912296in}{0.413320in}}%
\pgfpathlineto{\pgfqpoint{3.909602in}{0.413320in}}%
\pgfpathlineto{\pgfqpoint{3.906918in}{0.413320in}}%
\pgfpathlineto{\pgfqpoint{3.904252in}{0.413320in}}%
\pgfpathlineto{\pgfqpoint{3.901573in}{0.413320in}}%
\pgfpathlineto{\pgfqpoint{3.898891in}{0.413320in}}%
\pgfpathlineto{\pgfqpoint{3.896345in}{0.413320in}}%
\pgfpathlineto{\pgfqpoint{3.893541in}{0.413320in}}%
\pgfpathlineto{\pgfqpoint{3.890926in}{0.413320in}}%
\pgfpathlineto{\pgfqpoint{3.888188in}{0.413320in}}%
\pgfpathlineto{\pgfqpoint{3.885621in}{0.413320in}}%
\pgfpathlineto{\pgfqpoint{3.882850in}{0.413320in}}%
\pgfpathlineto{\pgfqpoint{3.880237in}{0.413320in}}%
\pgfpathlineto{\pgfqpoint{3.877466in}{0.413320in}}%
\pgfpathlineto{\pgfqpoint{3.874790in}{0.413320in}}%
\pgfpathlineto{\pgfqpoint{3.872114in}{0.413320in}}%
\pgfpathlineto{\pgfqpoint{3.869435in}{0.413320in}}%
\pgfpathlineto{\pgfqpoint{3.866815in}{0.413320in}}%
\pgfpathlineto{\pgfqpoint{3.864073in}{0.413320in}}%
\pgfpathlineto{\pgfqpoint{3.861561in}{0.413320in}}%
\pgfpathlineto{\pgfqpoint{3.858720in}{0.413320in}}%
\pgfpathlineto{\pgfqpoint{3.856100in}{0.413320in}}%
\pgfpathlineto{\pgfqpoint{3.853358in}{0.413320in}}%
\pgfpathlineto{\pgfqpoint{3.850814in}{0.413320in}}%
\pgfpathlineto{\pgfqpoint{3.848005in}{0.413320in}}%
\pgfpathlineto{\pgfqpoint{3.845329in}{0.413320in}}%
\pgfpathlineto{\pgfqpoint{3.842641in}{0.413320in}}%
\pgfpathlineto{\pgfqpoint{3.839960in}{0.413320in}}%
\pgfpathlineto{\pgfqpoint{3.837286in}{0.413320in}}%
\pgfpathlineto{\pgfqpoint{3.834616in}{0.413320in}}%
\pgfpathlineto{\pgfqpoint{3.832053in}{0.413320in}}%
\pgfpathlineto{\pgfqpoint{3.829252in}{0.413320in}}%
\pgfpathlineto{\pgfqpoint{3.826679in}{0.413320in}}%
\pgfpathlineto{\pgfqpoint{3.823903in}{0.413320in}}%
\pgfpathlineto{\pgfqpoint{3.821315in}{0.413320in}}%
\pgfpathlineto{\pgfqpoint{3.818546in}{0.413320in}}%
\pgfpathlineto{\pgfqpoint{3.815983in}{0.413320in}}%
\pgfpathlineto{\pgfqpoint{3.813172in}{0.413320in}}%
\pgfpathlineto{\pgfqpoint{3.810510in}{0.413320in}}%
\pgfpathlineto{\pgfqpoint{3.807832in}{0.413320in}}%
\pgfpathlineto{\pgfqpoint{3.805145in}{0.413320in}}%
\pgfpathlineto{\pgfqpoint{3.802569in}{0.413320in}}%
\pgfpathlineto{\pgfqpoint{3.799797in}{0.413320in}}%
\pgfpathlineto{\pgfqpoint{3.797265in}{0.413320in}}%
\pgfpathlineto{\pgfqpoint{3.794435in}{0.413320in}}%
\pgfpathlineto{\pgfqpoint{3.791897in}{0.413320in}}%
\pgfpathlineto{\pgfqpoint{3.789084in}{0.413320in}}%
\pgfpathlineto{\pgfqpoint{3.786504in}{0.413320in}}%
\pgfpathlineto{\pgfqpoint{3.783725in}{0.413320in}}%
\pgfpathlineto{\pgfqpoint{3.781046in}{0.413320in}}%
\pgfpathlineto{\pgfqpoint{3.778370in}{0.413320in}}%
\pgfpathlineto{\pgfqpoint{3.775691in}{0.413320in}}%
\pgfpathlineto{\pgfqpoint{3.773014in}{0.413320in}}%
\pgfpathlineto{\pgfqpoint{3.770323in}{0.413320in}}%
\pgfpathlineto{\pgfqpoint{3.767782in}{0.413320in}}%
\pgfpathlineto{\pgfqpoint{3.764966in}{0.413320in}}%
\pgfpathlineto{\pgfqpoint{3.762389in}{0.413320in}}%
\pgfpathlineto{\pgfqpoint{3.759622in}{0.413320in}}%
\pgfpathlineto{\pgfqpoint{3.757065in}{0.413320in}}%
\pgfpathlineto{\pgfqpoint{3.754265in}{0.413320in}}%
\pgfpathlineto{\pgfqpoint{3.751728in}{0.413320in}}%
\pgfpathlineto{\pgfqpoint{3.748903in}{0.413320in}}%
\pgfpathlineto{\pgfqpoint{3.746229in}{0.413320in}}%
\pgfpathlineto{\pgfqpoint{3.743548in}{0.413320in}}%
\pgfpathlineto{\pgfqpoint{3.740874in}{0.413320in}}%
\pgfpathlineto{\pgfqpoint{3.738194in}{0.413320in}}%
\pgfpathlineto{\pgfqpoint{3.735509in}{0.413320in}}%
\pgfpathlineto{\pgfqpoint{3.732950in}{0.413320in}}%
\pgfpathlineto{\pgfqpoint{3.730158in}{0.413320in}}%
\pgfpathlineto{\pgfqpoint{3.727581in}{0.413320in}}%
\pgfpathlineto{\pgfqpoint{3.724804in}{0.413320in}}%
\pgfpathlineto{\pgfqpoint{3.722228in}{0.413320in}}%
\pgfpathlineto{\pgfqpoint{3.719446in}{0.413320in}}%
\pgfpathlineto{\pgfqpoint{3.716875in}{0.413320in}}%
\pgfpathlineto{\pgfqpoint{3.714086in}{0.413320in}}%
\pgfpathlineto{\pgfqpoint{3.711410in}{0.413320in}}%
\pgfpathlineto{\pgfqpoint{3.708729in}{0.413320in}}%
\pgfpathlineto{\pgfqpoint{3.706053in}{0.413320in}}%
\pgfpathlineto{\pgfqpoint{3.703460in}{0.413320in}}%
\pgfpathlineto{\pgfqpoint{3.700684in}{0.413320in}}%
\pgfpathlineto{\pgfqpoint{3.698125in}{0.413320in}}%
\pgfpathlineto{\pgfqpoint{3.695331in}{0.413320in}}%
\pgfpathlineto{\pgfqpoint{3.692765in}{0.413320in}}%
\pgfpathlineto{\pgfqpoint{3.689983in}{0.413320in}}%
\pgfpathlineto{\pgfqpoint{3.687442in}{0.413320in}}%
\pgfpathlineto{\pgfqpoint{3.684620in}{0.413320in}}%
\pgfpathlineto{\pgfqpoint{3.681948in}{0.413320in}}%
\pgfpathlineto{\pgfqpoint{3.679273in}{0.413320in}}%
\pgfpathlineto{\pgfqpoint{3.676591in}{0.413320in}}%
\pgfpathlineto{\pgfqpoint{3.673911in}{0.413320in}}%
\pgfpathlineto{\pgfqpoint{3.671232in}{0.413320in}}%
\pgfpathlineto{\pgfqpoint{3.668665in}{0.413320in}}%
\pgfpathlineto{\pgfqpoint{3.665864in}{0.413320in}}%
\pgfpathlineto{\pgfqpoint{3.663276in}{0.413320in}}%
\pgfpathlineto{\pgfqpoint{3.660515in}{0.413320in}}%
\pgfpathlineto{\pgfqpoint{3.657917in}{0.413320in}}%
\pgfpathlineto{\pgfqpoint{3.655165in}{0.413320in}}%
\pgfpathlineto{\pgfqpoint{3.652628in}{0.413320in}}%
\pgfpathlineto{\pgfqpoint{3.649837in}{0.413320in}}%
\pgfpathlineto{\pgfqpoint{3.647130in}{0.413320in}}%
\pgfpathlineto{\pgfqpoint{3.644452in}{0.413320in}}%
\pgfpathlineto{\pgfqpoint{3.641773in}{0.413320in}}%
\pgfpathlineto{\pgfqpoint{3.639207in}{0.413320in}}%
\pgfpathlineto{\pgfqpoint{3.636413in}{0.413320in}}%
\pgfpathlineto{\pgfqpoint{3.633858in}{0.413320in}}%
\pgfpathlineto{\pgfqpoint{3.631058in}{0.413320in}}%
\pgfpathlineto{\pgfqpoint{3.628460in}{0.413320in}}%
\pgfpathlineto{\pgfqpoint{3.625689in}{0.413320in}}%
\pgfpathlineto{\pgfqpoint{3.623165in}{0.413320in}}%
\pgfpathlineto{\pgfqpoint{3.620345in}{0.413320in}}%
\pgfpathlineto{\pgfqpoint{3.617667in}{0.413320in}}%
\pgfpathlineto{\pgfqpoint{3.614982in}{0.413320in}}%
\pgfpathlineto{\pgfqpoint{3.612311in}{0.413320in}}%
\pgfpathlineto{\pgfqpoint{3.609632in}{0.413320in}}%
\pgfpathlineto{\pgfqpoint{3.606951in}{0.413320in}}%
\pgfpathlineto{\pgfqpoint{3.604387in}{0.413320in}}%
\pgfpathlineto{\pgfqpoint{3.601590in}{0.413320in}}%
\pgfpathlineto{\pgfqpoint{3.598998in}{0.413320in}}%
\pgfpathlineto{\pgfqpoint{3.596240in}{0.413320in}}%
\pgfpathlineto{\pgfqpoint{3.593620in}{0.413320in}}%
\pgfpathlineto{\pgfqpoint{3.590883in}{0.413320in}}%
\pgfpathlineto{\pgfqpoint{3.588258in}{0.413320in}}%
\pgfpathlineto{\pgfqpoint{3.585532in}{0.413320in}}%
\pgfpathlineto{\pgfqpoint{3.582851in}{0.413320in}}%
\pgfpathlineto{\pgfqpoint{3.580191in}{0.413320in}}%
\pgfpathlineto{\pgfqpoint{3.577487in}{0.413320in}}%
\pgfpathlineto{\pgfqpoint{3.574814in}{0.413320in}}%
\pgfpathlineto{\pgfqpoint{3.572126in}{0.413320in}}%
\pgfpathlineto{\pgfqpoint{3.569584in}{0.413320in}}%
\pgfpathlineto{\pgfqpoint{3.566774in}{0.413320in}}%
\pgfpathlineto{\pgfqpoint{3.564188in}{0.413320in}}%
\pgfpathlineto{\pgfqpoint{3.561420in}{0.413320in}}%
\pgfpathlineto{\pgfqpoint{3.558853in}{0.413320in}}%
\pgfpathlineto{\pgfqpoint{3.556061in}{0.413320in}}%
\pgfpathlineto{\pgfqpoint{3.553498in}{0.413320in}}%
\pgfpathlineto{\pgfqpoint{3.550713in}{0.413320in}}%
\pgfpathlineto{\pgfqpoint{3.548029in}{0.413320in}}%
\pgfpathlineto{\pgfqpoint{3.545349in}{0.413320in}}%
\pgfpathlineto{\pgfqpoint{3.542656in}{0.413320in}}%
\pgfpathlineto{\pgfqpoint{3.540093in}{0.413320in}}%
\pgfpathlineto{\pgfqpoint{3.537309in}{0.413320in}}%
\pgfpathlineto{\pgfqpoint{3.534783in}{0.413320in}}%
\pgfpathlineto{\pgfqpoint{3.531955in}{0.413320in}}%
\pgfpathlineto{\pgfqpoint{3.529327in}{0.413320in}}%
\pgfpathlineto{\pgfqpoint{3.526601in}{0.413320in}}%
\pgfpathlineto{\pgfqpoint{3.524041in}{0.413320in}}%
\pgfpathlineto{\pgfqpoint{3.521244in}{0.413320in}}%
\pgfpathlineto{\pgfqpoint{3.518565in}{0.413320in}}%
\pgfpathlineto{\pgfqpoint{3.515884in}{0.413320in}}%
\pgfpathlineto{\pgfqpoint{3.513209in}{0.413320in}}%
\pgfpathlineto{\pgfqpoint{3.510533in}{0.413320in}}%
\pgfpathlineto{\pgfqpoint{3.507840in}{0.413320in}}%
\pgfpathlineto{\pgfqpoint{3.505262in}{0.413320in}}%
\pgfpathlineto{\pgfqpoint{3.502488in}{0.413320in}}%
\pgfpathlineto{\pgfqpoint{3.499909in}{0.413320in}}%
\pgfpathlineto{\pgfqpoint{3.497139in}{0.413320in}}%
\pgfpathlineto{\pgfqpoint{3.494581in}{0.413320in}}%
\pgfpathlineto{\pgfqpoint{3.491783in}{0.413320in}}%
\pgfpathlineto{\pgfqpoint{3.489223in}{0.413320in}}%
\pgfpathlineto{\pgfqpoint{3.486442in}{0.413320in}}%
\pgfpathlineto{\pgfqpoint{3.483744in}{0.413320in}}%
\pgfpathlineto{\pgfqpoint{3.481072in}{0.413320in}}%
\pgfpathlineto{\pgfqpoint{3.478378in}{0.413320in}}%
\pgfpathlineto{\pgfqpoint{3.475821in}{0.413320in}}%
\pgfpathlineto{\pgfqpoint{3.473021in}{0.413320in}}%
\pgfpathlineto{\pgfqpoint{3.470466in}{0.413320in}}%
\pgfpathlineto{\pgfqpoint{3.467678in}{0.413320in}}%
\pgfpathlineto{\pgfqpoint{3.465072in}{0.413320in}}%
\pgfpathlineto{\pgfqpoint{3.462321in}{0.413320in}}%
\pgfpathlineto{\pgfqpoint{3.459695in}{0.413320in}}%
\pgfpathlineto{\pgfqpoint{3.456960in}{0.413320in}}%
\pgfpathlineto{\pgfqpoint{3.454285in}{0.413320in}}%
\pgfpathlineto{\pgfqpoint{3.451597in}{0.413320in}}%
\pgfpathlineto{\pgfqpoint{3.448926in}{0.413320in}}%
\pgfpathlineto{\pgfqpoint{3.446257in}{0.413320in}}%
\pgfpathlineto{\pgfqpoint{3.443574in}{0.413320in}}%
\pgfpathlineto{\pgfqpoint{3.440996in}{0.413320in}}%
\pgfpathlineto{\pgfqpoint{3.438210in}{0.413320in}}%
\pgfpathlineto{\pgfqpoint{3.435635in}{0.413320in}}%
\pgfpathlineto{\pgfqpoint{3.432851in}{0.413320in}}%
\pgfpathlineto{\pgfqpoint{3.430313in}{0.413320in}}%
\pgfpathlineto{\pgfqpoint{3.427501in}{0.413320in}}%
\pgfpathlineto{\pgfqpoint{3.424887in}{0.413320in}}%
\pgfpathlineto{\pgfqpoint{3.422142in}{0.413320in}}%
\pgfpathlineto{\pgfqpoint{3.419455in}{0.413320in}}%
\pgfpathlineto{\pgfqpoint{3.416780in}{0.413320in}}%
\pgfpathlineto{\pgfqpoint{3.414109in}{0.413320in}}%
\pgfpathlineto{\pgfqpoint{3.411431in}{0.413320in}}%
\pgfpathlineto{\pgfqpoint{3.408752in}{0.413320in}}%
\pgfpathlineto{\pgfqpoint{3.406202in}{0.413320in}}%
\pgfpathlineto{\pgfqpoint{3.403394in}{0.413320in}}%
\pgfpathlineto{\pgfqpoint{3.400783in}{0.413320in}}%
\pgfpathlineto{\pgfqpoint{3.398037in}{0.413320in}}%
\pgfpathlineto{\pgfqpoint{3.395461in}{0.413320in}}%
\pgfpathlineto{\pgfqpoint{3.392681in}{0.413320in}}%
\pgfpathlineto{\pgfqpoint{3.390102in}{0.413320in}}%
\pgfpathlineto{\pgfqpoint{3.387309in}{0.413320in}}%
\pgfpathlineto{\pgfqpoint{3.384647in}{0.413320in}}%
\pgfpathlineto{\pgfqpoint{3.381959in}{0.413320in}}%
\pgfpathlineto{\pgfqpoint{3.379290in}{0.413320in}}%
\pgfpathlineto{\pgfqpoint{3.376735in}{0.413320in}}%
\pgfpathlineto{\pgfqpoint{3.373921in}{0.413320in}}%
\pgfpathlineto{\pgfqpoint{3.371357in}{0.413320in}}%
\pgfpathlineto{\pgfqpoint{3.368577in}{0.413320in}}%
\pgfpathlineto{\pgfqpoint{3.365996in}{0.413320in}}%
\pgfpathlineto{\pgfqpoint{3.363221in}{0.413320in}}%
\pgfpathlineto{\pgfqpoint{3.360620in}{0.413320in}}%
\pgfpathlineto{\pgfqpoint{3.357862in}{0.413320in}}%
\pgfpathlineto{\pgfqpoint{3.355177in}{0.413320in}}%
\pgfpathlineto{\pgfqpoint{3.352505in}{0.413320in}}%
\pgfpathlineto{\pgfqpoint{3.349828in}{0.413320in}}%
\pgfpathlineto{\pgfqpoint{3.347139in}{0.413320in}}%
\pgfpathlineto{\pgfqpoint{3.344468in}{0.413320in}}%
\pgfpathlineto{\pgfqpoint{3.341893in}{0.413320in}}%
\pgfpathlineto{\pgfqpoint{3.339101in}{0.413320in}}%
\pgfpathlineto{\pgfqpoint{3.336541in}{0.413320in}}%
\pgfpathlineto{\pgfqpoint{3.333758in}{0.413320in}}%
\pgfpathlineto{\pgfqpoint{3.331183in}{0.413320in}}%
\pgfpathlineto{\pgfqpoint{3.328401in}{0.413320in}}%
\pgfpathlineto{\pgfqpoint{3.325860in}{0.413320in}}%
\pgfpathlineto{\pgfqpoint{3.323049in}{0.413320in}}%
\pgfpathlineto{\pgfqpoint{3.320366in}{0.413320in}}%
\pgfpathlineto{\pgfqpoint{3.317688in}{0.413320in}}%
\pgfpathlineto{\pgfqpoint{3.315008in}{0.413320in}}%
\pgfpathlineto{\pgfqpoint{3.312480in}{0.413320in}}%
\pgfpathlineto{\pgfqpoint{3.309652in}{0.413320in}}%
\pgfpathlineto{\pgfqpoint{3.307104in}{0.413320in}}%
\pgfpathlineto{\pgfqpoint{3.304295in}{0.413320in}}%
\pgfpathlineto{\pgfqpoint{3.301719in}{0.413320in}}%
\pgfpathlineto{\pgfqpoint{3.298937in}{0.413320in}}%
\pgfpathlineto{\pgfqpoint{3.296376in}{0.413320in}}%
\pgfpathlineto{\pgfqpoint{3.293574in}{0.413320in}}%
\pgfpathlineto{\pgfqpoint{3.290890in}{0.413320in}}%
\pgfpathlineto{\pgfqpoint{3.288225in}{0.413320in}}%
\pgfpathlineto{\pgfqpoint{3.285534in}{0.413320in}}%
\pgfpathlineto{\pgfqpoint{3.282870in}{0.413320in}}%
\pgfpathlineto{\pgfqpoint{3.280189in}{0.413320in}}%
\pgfpathlineto{\pgfqpoint{3.277603in}{0.413320in}}%
\pgfpathlineto{\pgfqpoint{3.274831in}{0.413320in}}%
\pgfpathlineto{\pgfqpoint{3.272254in}{0.413320in}}%
\pgfpathlineto{\pgfqpoint{3.269478in}{0.413320in}}%
\pgfpathlineto{\pgfqpoint{3.266849in}{0.413320in}}%
\pgfpathlineto{\pgfqpoint{3.264119in}{0.413320in}}%
\pgfpathlineto{\pgfqpoint{3.261594in}{0.413320in}}%
\pgfpathlineto{\pgfqpoint{3.258784in}{0.413320in}}%
\pgfpathlineto{\pgfqpoint{3.256083in}{0.413320in}}%
\pgfpathlineto{\pgfqpoint{3.253404in}{0.413320in}}%
\pgfpathlineto{\pgfqpoint{3.250716in}{0.413320in}}%
\pgfpathlineto{\pgfqpoint{3.248049in}{0.413320in}}%
\pgfpathlineto{\pgfqpoint{3.245363in}{0.413320in}}%
\pgfpathlineto{\pgfqpoint{3.242807in}{0.413320in}}%
\pgfpathlineto{\pgfqpoint{3.240010in}{0.413320in}}%
\pgfpathlineto{\pgfqpoint{3.237411in}{0.413320in}}%
\pgfpathlineto{\pgfqpoint{3.234658in}{0.413320in}}%
\pgfpathlineto{\pgfqpoint{3.232069in}{0.413320in}}%
\pgfpathlineto{\pgfqpoint{3.229310in}{0.413320in}}%
\pgfpathlineto{\pgfqpoint{3.226609in}{0.413320in}}%
\pgfpathlineto{\pgfqpoint{3.223942in}{0.413320in}}%
\pgfpathlineto{\pgfqpoint{3.221255in}{0.413320in}}%
\pgfpathlineto{\pgfqpoint{3.218586in}{0.413320in}}%
\pgfpathlineto{\pgfqpoint{3.215908in}{0.413320in}}%
\pgfpathlineto{\pgfqpoint{3.213342in}{0.413320in}}%
\pgfpathlineto{\pgfqpoint{3.210545in}{0.413320in}}%
\pgfpathlineto{\pgfqpoint{3.207984in}{0.413320in}}%
\pgfpathlineto{\pgfqpoint{3.205195in}{0.413320in}}%
\pgfpathlineto{\pgfqpoint{3.202562in}{0.413320in}}%
\pgfpathlineto{\pgfqpoint{3.199823in}{0.413320in}}%
\pgfpathlineto{\pgfqpoint{3.197226in}{0.413320in}}%
\pgfpathlineto{\pgfqpoint{3.194508in}{0.413320in}}%
\pgfpathlineto{\pgfqpoint{3.191796in}{0.413320in}}%
\pgfpathlineto{\pgfqpoint{3.189117in}{0.413320in}}%
\pgfpathlineto{\pgfqpoint{3.186440in}{0.413320in}}%
\pgfpathlineto{\pgfqpoint{3.183760in}{0.413320in}}%
\pgfpathlineto{\pgfqpoint{3.181089in}{0.413320in}}%
\pgfpathlineto{\pgfqpoint{3.178525in}{0.413320in}}%
\pgfpathlineto{\pgfqpoint{3.175724in}{0.413320in}}%
\pgfpathlineto{\pgfqpoint{3.173142in}{0.413320in}}%
\pgfpathlineto{\pgfqpoint{3.170375in}{0.413320in}}%
\pgfpathlineto{\pgfqpoint{3.167776in}{0.413320in}}%
\pgfpathlineto{\pgfqpoint{3.165019in}{0.413320in}}%
\pgfpathlineto{\pgfqpoint{3.162474in}{0.413320in}}%
\pgfpathlineto{\pgfqpoint{3.159675in}{0.413320in}}%
\pgfpathlineto{\pgfqpoint{3.156981in}{0.413320in}}%
\pgfpathlineto{\pgfqpoint{3.154327in}{0.413320in}}%
\pgfpathlineto{\pgfqpoint{3.151612in}{0.413320in}}%
\pgfpathlineto{\pgfqpoint{3.149057in}{0.413320in}}%
\pgfpathlineto{\pgfqpoint{3.146271in}{0.413320in}}%
\pgfpathlineto{\pgfqpoint{3.143740in}{0.413320in}}%
\pgfpathlineto{\pgfqpoint{3.140913in}{0.413320in}}%
\pgfpathlineto{\pgfqpoint{3.138375in}{0.413320in}}%
\pgfpathlineto{\pgfqpoint{3.135550in}{0.413320in}}%
\pgfpathlineto{\pgfqpoint{3.132946in}{0.413320in}}%
\pgfpathlineto{\pgfqpoint{3.130199in}{0.413320in}}%
\pgfpathlineto{\pgfqpoint{3.127512in}{0.413320in}}%
\pgfpathlineto{\pgfqpoint{3.124842in}{0.413320in}}%
\pgfpathlineto{\pgfqpoint{3.122163in}{0.413320in}}%
\pgfpathlineto{\pgfqpoint{3.119487in}{0.413320in}}%
\pgfpathlineto{\pgfqpoint{3.116807in}{0.413320in}}%
\pgfpathlineto{\pgfqpoint{3.114242in}{0.413320in}}%
\pgfpathlineto{\pgfqpoint{3.111451in}{0.413320in}}%
\pgfpathlineto{\pgfqpoint{3.108896in}{0.413320in}}%
\pgfpathlineto{\pgfqpoint{3.106094in}{0.413320in}}%
\pgfpathlineto{\pgfqpoint{3.103508in}{0.413320in}}%
\pgfpathlineto{\pgfqpoint{3.100737in}{0.413320in}}%
\pgfpathlineto{\pgfqpoint{3.098163in}{0.413320in}}%
\pgfpathlineto{\pgfqpoint{3.095388in}{0.413320in}}%
\pgfpathlineto{\pgfqpoint{3.092699in}{0.413320in}}%
\pgfpathlineto{\pgfqpoint{3.090023in}{0.413320in}}%
\pgfpathlineto{\pgfqpoint{3.087343in}{0.413320in}}%
\pgfpathlineto{\pgfqpoint{3.084671in}{0.413320in}}%
\pgfpathlineto{\pgfqpoint{3.081990in}{0.413320in}}%
\pgfpathlineto{\pgfqpoint{3.079381in}{0.413320in}}%
\pgfpathlineto{\pgfqpoint{3.076631in}{0.413320in}}%
\pgfpathlineto{\pgfqpoint{3.074056in}{0.413320in}}%
\pgfpathlineto{\pgfqpoint{3.071266in}{0.413320in}}%
\pgfpathlineto{\pgfqpoint{3.068709in}{0.413320in}}%
\pgfpathlineto{\pgfqpoint{3.065916in}{0.413320in}}%
\pgfpathlineto{\pgfqpoint{3.063230in}{0.413320in}}%
\pgfpathlineto{\pgfqpoint{3.060561in}{0.413320in}}%
\pgfpathlineto{\pgfqpoint{3.057884in}{0.413320in}}%
\pgfpathlineto{\pgfqpoint{3.055202in}{0.413320in}}%
\pgfpathlineto{\pgfqpoint{3.052526in}{0.413320in}}%
\pgfpathlineto{\pgfqpoint{3.049988in}{0.413320in}}%
\pgfpathlineto{\pgfqpoint{3.047157in}{0.413320in}}%
\pgfpathlineto{\pgfqpoint{3.044568in}{0.413320in}}%
\pgfpathlineto{\pgfqpoint{3.041813in}{0.413320in}}%
\pgfpathlineto{\pgfqpoint{3.039262in}{0.413320in}}%
\pgfpathlineto{\pgfqpoint{3.036456in}{0.413320in}}%
\pgfpathlineto{\pgfqpoint{3.033921in}{0.413320in}}%
\pgfpathlineto{\pgfqpoint{3.031091in}{0.413320in}}%
\pgfpathlineto{\pgfqpoint{3.028412in}{0.413320in}}%
\pgfpathlineto{\pgfqpoint{3.025803in}{0.413320in}}%
\pgfpathlineto{\pgfqpoint{3.023058in}{0.413320in}}%
\pgfpathlineto{\pgfqpoint{3.020382in}{0.413320in}}%
\pgfpathlineto{\pgfqpoint{3.017707in}{0.413320in}}%
\pgfpathlineto{\pgfqpoint{3.015097in}{0.413320in}}%
\pgfpathlineto{\pgfqpoint{3.012351in}{0.413320in}}%
\pgfpathlineto{\pgfqpoint{3.009784in}{0.413320in}}%
\pgfpathlineto{\pgfqpoint{3.006993in}{0.413320in}}%
\pgfpathlineto{\pgfqpoint{3.004419in}{0.413320in}}%
\pgfpathlineto{\pgfqpoint{3.001635in}{0.413320in}}%
\pgfpathlineto{\pgfqpoint{2.999103in}{0.413320in}}%
\pgfpathlineto{\pgfqpoint{2.996300in}{0.413320in}}%
\pgfpathlineto{\pgfqpoint{2.993595in}{0.413320in}}%
\pgfpathlineto{\pgfqpoint{2.990978in}{0.413320in}}%
\pgfpathlineto{\pgfqpoint{2.988238in}{0.413320in}}%
\pgfpathlineto{\pgfqpoint{2.985666in}{0.413320in}}%
\pgfpathlineto{\pgfqpoint{2.982885in}{0.413320in}}%
\pgfpathlineto{\pgfqpoint{2.980341in}{0.413320in}}%
\pgfpathlineto{\pgfqpoint{2.977517in}{0.413320in}}%
\pgfpathlineto{\pgfqpoint{2.974972in}{0.413320in}}%
\pgfpathlineto{\pgfqpoint{2.972177in}{0.413320in}}%
\pgfpathlineto{\pgfqpoint{2.969599in}{0.413320in}}%
\pgfpathlineto{\pgfqpoint{2.966812in}{0.413320in}}%
\pgfpathlineto{\pgfqpoint{2.964127in}{0.413320in}}%
\pgfpathlineto{\pgfqpoint{2.961460in}{0.413320in}}%
\pgfpathlineto{\pgfqpoint{2.958782in}{0.413320in}}%
\pgfpathlineto{\pgfqpoint{2.956103in}{0.413320in}}%
\pgfpathlineto{\pgfqpoint{2.953422in}{0.413320in}}%
\pgfpathlineto{\pgfqpoint{2.950884in}{0.413320in}}%
\pgfpathlineto{\pgfqpoint{2.948068in}{0.413320in}}%
\pgfpathlineto{\pgfqpoint{2.945461in}{0.413320in}}%
\pgfpathlineto{\pgfqpoint{2.942711in}{0.413320in}}%
\pgfpathlineto{\pgfqpoint{2.940120in}{0.413320in}}%
\pgfpathlineto{\pgfqpoint{2.937352in}{0.413320in}}%
\pgfpathlineto{\pgfqpoint{2.934759in}{0.413320in}}%
\pgfpathlineto{\pgfqpoint{2.932033in}{0.413320in}}%
\pgfpathlineto{\pgfqpoint{2.929321in}{0.413320in}}%
\pgfpathlineto{\pgfqpoint{2.926655in}{0.413320in}}%
\pgfpathlineto{\pgfqpoint{2.923963in}{0.413320in}}%
\pgfpathlineto{\pgfqpoint{2.921363in}{0.413320in}}%
\pgfpathlineto{\pgfqpoint{2.918606in}{0.413320in}}%
\pgfpathlineto{\pgfqpoint{2.916061in}{0.413320in}}%
\pgfpathlineto{\pgfqpoint{2.913243in}{0.413320in}}%
\pgfpathlineto{\pgfqpoint{2.910631in}{0.413320in}}%
\pgfpathlineto{\pgfqpoint{2.907882in}{0.413320in}}%
\pgfpathlineto{\pgfqpoint{2.905341in}{0.413320in}}%
\pgfpathlineto{\pgfqpoint{2.902535in}{0.413320in}}%
\pgfpathlineto{\pgfqpoint{2.899858in}{0.413320in}}%
\pgfpathlineto{\pgfqpoint{2.897179in}{0.413320in}}%
\pgfpathlineto{\pgfqpoint{2.894487in}{0.413320in}}%
\pgfpathlineto{\pgfqpoint{2.891809in}{0.413320in}}%
\pgfpathlineto{\pgfqpoint{2.889145in}{0.413320in}}%
\pgfpathlineto{\pgfqpoint{2.886578in}{0.413320in}}%
\pgfpathlineto{\pgfqpoint{2.883780in}{0.413320in}}%
\pgfpathlineto{\pgfqpoint{2.881254in}{0.413320in}}%
\pgfpathlineto{\pgfqpoint{2.878431in}{0.413320in}}%
\pgfpathlineto{\pgfqpoint{2.875882in}{0.413320in}}%
\pgfpathlineto{\pgfqpoint{2.873074in}{0.413320in}}%
\pgfpathlineto{\pgfqpoint{2.870475in}{0.413320in}}%
\pgfpathlineto{\pgfqpoint{2.867713in}{0.413320in}}%
\pgfpathlineto{\pgfqpoint{2.865031in}{0.413320in}}%
\pgfpathlineto{\pgfqpoint{2.862402in}{0.413320in}}%
\pgfpathlineto{\pgfqpoint{2.859668in}{0.413320in}}%
\pgfpathlineto{\pgfqpoint{2.857003in}{0.413320in}}%
\pgfpathlineto{\pgfqpoint{2.854325in}{0.413320in}}%
\pgfpathlineto{\pgfqpoint{2.851793in}{0.413320in}}%
\pgfpathlineto{\pgfqpoint{2.848960in}{0.413320in}}%
\pgfpathlineto{\pgfqpoint{2.846408in}{0.413320in}}%
\pgfpathlineto{\pgfqpoint{2.843611in}{0.413320in}}%
\pgfpathlineto{\pgfqpoint{2.841055in}{0.413320in}}%
\pgfpathlineto{\pgfqpoint{2.838254in}{0.413320in}}%
\pgfpathlineto{\pgfqpoint{2.835698in}{0.413320in}}%
\pgfpathlineto{\pgfqpoint{2.832894in}{0.413320in}}%
\pgfpathlineto{\pgfqpoint{2.830219in}{0.413320in}}%
\pgfpathlineto{\pgfqpoint{2.827567in}{0.413320in}}%
\pgfpathlineto{\pgfqpoint{2.824851in}{0.413320in}}%
\pgfpathlineto{\pgfqpoint{2.822303in}{0.413320in}}%
\pgfpathlineto{\pgfqpoint{2.819506in}{0.413320in}}%
\pgfpathlineto{\pgfqpoint{2.816867in}{0.413320in}}%
\pgfpathlineto{\pgfqpoint{2.814141in}{0.413320in}}%
\pgfpathlineto{\pgfqpoint{2.811597in}{0.413320in}}%
\pgfpathlineto{\pgfqpoint{2.808792in}{0.413320in}}%
\pgfpathlineto{\pgfqpoint{2.806175in}{0.413320in}}%
\pgfpathlineto{\pgfqpoint{2.803435in}{0.413320in}}%
\pgfpathlineto{\pgfqpoint{2.800756in}{0.413320in}}%
\pgfpathlineto{\pgfqpoint{2.798070in}{0.413320in}}%
\pgfpathlineto{\pgfqpoint{2.795398in}{0.413320in}}%
\pgfpathlineto{\pgfqpoint{2.792721in}{0.413320in}}%
\pgfpathlineto{\pgfqpoint{2.790044in}{0.413320in}}%
\pgfpathlineto{\pgfqpoint{2.787468in}{0.413320in}}%
\pgfpathlineto{\pgfqpoint{2.784687in}{0.413320in}}%
\pgfpathlineto{\pgfqpoint{2.782113in}{0.413320in}}%
\pgfpathlineto{\pgfqpoint{2.779330in}{0.413320in}}%
\pgfpathlineto{\pgfqpoint{2.776767in}{0.413320in}}%
\pgfpathlineto{\pgfqpoint{2.773972in}{0.413320in}}%
\pgfpathlineto{\pgfqpoint{2.771373in}{0.413320in}}%
\pgfpathlineto{\pgfqpoint{2.768617in}{0.413320in}}%
\pgfpathlineto{\pgfqpoint{2.765935in}{0.413320in}}%
\pgfpathlineto{\pgfqpoint{2.763253in}{0.413320in}}%
\pgfpathlineto{\pgfqpoint{2.760581in}{0.413320in}}%
\pgfpathlineto{\pgfqpoint{2.758028in}{0.413320in}}%
\pgfpathlineto{\pgfqpoint{2.755224in}{0.413320in}}%
\pgfpathlineto{\pgfqpoint{2.752614in}{0.413320in}}%
\pgfpathlineto{\pgfqpoint{2.749868in}{0.413320in}}%
\pgfpathlineto{\pgfqpoint{2.747260in}{0.413320in}}%
\pgfpathlineto{\pgfqpoint{2.744510in}{0.413320in}}%
\pgfpathlineto{\pgfqpoint{2.741928in}{0.413320in}}%
\pgfpathlineto{\pgfqpoint{2.739155in}{0.413320in}}%
\pgfpathlineto{\pgfqpoint{2.736476in}{0.413320in}}%
\pgfpathlineto{\pgfqpoint{2.733798in}{0.413320in}}%
\pgfpathlineto{\pgfqpoint{2.731119in}{0.413320in}}%
\pgfpathlineto{\pgfqpoint{2.728439in}{0.413320in}}%
\pgfpathlineto{\pgfqpoint{2.725760in}{0.413320in}}%
\pgfpathlineto{\pgfqpoint{2.723211in}{0.413320in}}%
\pgfpathlineto{\pgfqpoint{2.720404in}{0.413320in}}%
\pgfpathlineto{\pgfqpoint{2.717773in}{0.413320in}}%
\pgfpathlineto{\pgfqpoint{2.715036in}{0.413320in}}%
\pgfpathlineto{\pgfqpoint{2.712477in}{0.413320in}}%
\pgfpathlineto{\pgfqpoint{2.709683in}{0.413320in}}%
\pgfpathlineto{\pgfqpoint{2.707125in}{0.413320in}}%
\pgfpathlineto{\pgfqpoint{2.704326in}{0.413320in}}%
\pgfpathlineto{\pgfqpoint{2.701657in}{0.413320in}}%
\pgfpathlineto{\pgfqpoint{2.698968in}{0.413320in}}%
\pgfpathlineto{\pgfqpoint{2.696293in}{0.413320in}}%
\pgfpathlineto{\pgfqpoint{2.693611in}{0.413320in}}%
\pgfpathlineto{\pgfqpoint{2.690940in}{0.413320in}}%
\pgfpathlineto{\pgfqpoint{2.688328in}{0.413320in}}%
\pgfpathlineto{\pgfqpoint{2.685586in}{0.413320in}}%
\pgfpathlineto{\pgfqpoint{2.683009in}{0.413320in}}%
\pgfpathlineto{\pgfqpoint{2.680224in}{0.413320in}}%
\pgfpathlineto{\pgfqpoint{2.677650in}{0.413320in}}%
\pgfpathlineto{\pgfqpoint{2.674873in}{0.413320in}}%
\pgfpathlineto{\pgfqpoint{2.672301in}{0.413320in}}%
\pgfpathlineto{\pgfqpoint{2.669506in}{0.413320in}}%
\pgfpathlineto{\pgfqpoint{2.666836in}{0.413320in}}%
\pgfpathlineto{\pgfqpoint{2.664151in}{0.413320in}}%
\pgfpathlineto{\pgfqpoint{2.661481in}{0.413320in}}%
\pgfpathlineto{\pgfqpoint{2.658942in}{0.413320in}}%
\pgfpathlineto{\pgfqpoint{2.656124in}{0.413320in}}%
\pgfpathlineto{\pgfqpoint{2.653567in}{0.413320in}}%
\pgfpathlineto{\pgfqpoint{2.650767in}{0.413320in}}%
\pgfpathlineto{\pgfqpoint{2.648196in}{0.413320in}}%
\pgfpathlineto{\pgfqpoint{2.645408in}{0.413320in}}%
\pgfpathlineto{\pgfqpoint{2.642827in}{0.413320in}}%
\pgfpathlineto{\pgfqpoint{2.640053in}{0.413320in}}%
\pgfpathlineto{\pgfqpoint{2.637369in}{0.413320in}}%
\pgfpathlineto{\pgfqpoint{2.634700in}{0.413320in}}%
\pgfpathlineto{\pgfqpoint{2.632018in}{0.413320in}}%
\pgfpathlineto{\pgfqpoint{2.629340in}{0.413320in}}%
\pgfpathlineto{\pgfqpoint{2.626653in}{0.413320in}}%
\pgfpathlineto{\pgfqpoint{2.624077in}{0.413320in}}%
\pgfpathlineto{\pgfqpoint{2.621304in}{0.413320in}}%
\pgfpathlineto{\pgfqpoint{2.618773in}{0.413320in}}%
\pgfpathlineto{\pgfqpoint{2.615934in}{0.413320in}}%
\pgfpathlineto{\pgfqpoint{2.613393in}{0.413320in}}%
\pgfpathlineto{\pgfqpoint{2.610588in}{0.413320in}}%
\pgfpathlineto{\pgfqpoint{2.608004in}{0.413320in}}%
\pgfpathlineto{\pgfqpoint{2.605232in}{0.413320in}}%
\pgfpathlineto{\pgfqpoint{2.602557in}{0.413320in}}%
\pgfpathlineto{\pgfqpoint{2.599920in}{0.413320in}}%
\pgfpathlineto{\pgfqpoint{2.597196in}{0.413320in}}%
\pgfpathlineto{\pgfqpoint{2.594630in}{0.413320in}}%
\pgfpathlineto{\pgfqpoint{2.591842in}{0.413320in}}%
\pgfpathlineto{\pgfqpoint{2.589248in}{0.413320in}}%
\pgfpathlineto{\pgfqpoint{2.586484in}{0.413320in}}%
\pgfpathlineto{\pgfqpoint{2.583913in}{0.413320in}}%
\pgfpathlineto{\pgfqpoint{2.581129in}{0.413320in}}%
\pgfpathlineto{\pgfqpoint{2.578567in}{0.413320in}}%
\pgfpathlineto{\pgfqpoint{2.575779in}{0.413320in}}%
\pgfpathlineto{\pgfqpoint{2.573082in}{0.413320in}}%
\pgfpathlineto{\pgfqpoint{2.570411in}{0.413320in}}%
\pgfpathlineto{\pgfqpoint{2.567730in}{0.413320in}}%
\pgfpathlineto{\pgfqpoint{2.565045in}{0.413320in}}%
\pgfpathlineto{\pgfqpoint{2.562375in}{0.413320in}}%
\pgfpathlineto{\pgfqpoint{2.559790in}{0.413320in}}%
\pgfpathlineto{\pgfqpoint{2.557009in}{0.413320in}}%
\pgfpathlineto{\pgfqpoint{2.554493in}{0.413320in}}%
\pgfpathlineto{\pgfqpoint{2.551664in}{0.413320in}}%
\pgfpathlineto{\pgfqpoint{2.549114in}{0.413320in}}%
\pgfpathlineto{\pgfqpoint{2.546310in}{0.413320in}}%
\pgfpathlineto{\pgfqpoint{2.543765in}{0.413320in}}%
\pgfpathlineto{\pgfqpoint{2.540949in}{0.413320in}}%
\pgfpathlineto{\pgfqpoint{2.538274in}{0.413320in}}%
\pgfpathlineto{\pgfqpoint{2.535624in}{0.413320in}}%
\pgfpathlineto{\pgfqpoint{2.532917in}{0.413320in}}%
\pgfpathlineto{\pgfqpoint{2.530234in}{0.413320in}}%
\pgfpathlineto{\pgfqpoint{2.527560in}{0.413320in}}%
\pgfpathlineto{\pgfqpoint{2.524988in}{0.413320in}}%
\pgfpathlineto{\pgfqpoint{2.522197in}{0.413320in}}%
\pgfpathlineto{\pgfqpoint{2.519607in}{0.413320in}}%
\pgfpathlineto{\pgfqpoint{2.516845in}{0.413320in}}%
\pgfpathlineto{\pgfqpoint{2.514268in}{0.413320in}}%
\pgfpathlineto{\pgfqpoint{2.511478in}{0.413320in}}%
\pgfpathlineto{\pgfqpoint{2.508917in}{0.413320in}}%
\pgfpathlineto{\pgfqpoint{2.506163in}{0.413320in}}%
\pgfpathlineto{\pgfqpoint{2.503454in}{0.413320in}}%
\pgfpathlineto{\pgfqpoint{2.500801in}{0.413320in}}%
\pgfpathlineto{\pgfqpoint{2.498085in}{0.413320in}}%
\pgfpathlineto{\pgfqpoint{2.495542in}{0.413320in}}%
\pgfpathlineto{\pgfqpoint{2.492729in}{0.413320in}}%
\pgfpathlineto{\pgfqpoint{2.490183in}{0.413320in}}%
\pgfpathlineto{\pgfqpoint{2.487384in}{0.413320in}}%
\pgfpathlineto{\pgfqpoint{2.484870in}{0.413320in}}%
\pgfpathlineto{\pgfqpoint{2.482026in}{0.413320in}}%
\pgfpathlineto{\pgfqpoint{2.479420in}{0.413320in}}%
\pgfpathlineto{\pgfqpoint{2.476671in}{0.413320in}}%
\pgfpathlineto{\pgfqpoint{2.473989in}{0.413320in}}%
\pgfpathlineto{\pgfqpoint{2.471311in}{0.413320in}}%
\pgfpathlineto{\pgfqpoint{2.468635in}{0.413320in}}%
\pgfpathlineto{\pgfqpoint{2.465957in}{0.413320in}}%
\pgfpathlineto{\pgfqpoint{2.463280in}{0.413320in}}%
\pgfpathlineto{\pgfqpoint{2.460711in}{0.413320in}}%
\pgfpathlineto{\pgfqpoint{2.457917in}{0.413320in}}%
\pgfpathlineto{\pgfqpoint{2.455353in}{0.413320in}}%
\pgfpathlineto{\pgfqpoint{2.452562in}{0.413320in}}%
\pgfpathlineto{\pgfqpoint{2.450032in}{0.413320in}}%
\pgfpathlineto{\pgfqpoint{2.447209in}{0.413320in}}%
\pgfpathlineto{\pgfqpoint{2.444677in}{0.413320in}}%
\pgfpathlineto{\pgfqpoint{2.441876in}{0.413320in}}%
\pgfpathlineto{\pgfqpoint{2.439167in}{0.413320in}}%
\pgfpathlineto{\pgfqpoint{2.436518in}{0.413320in}}%
\pgfpathlineto{\pgfqpoint{2.433815in}{0.413320in}}%
\pgfpathlineto{\pgfqpoint{2.431251in}{0.413320in}}%
\pgfpathlineto{\pgfqpoint{2.428453in}{0.413320in}}%
\pgfpathlineto{\pgfqpoint{2.425878in}{0.413320in}}%
\pgfpathlineto{\pgfqpoint{2.423098in}{0.413320in}}%
\pgfpathlineto{\pgfqpoint{2.420528in}{0.413320in}}%
\pgfpathlineto{\pgfqpoint{2.417747in}{0.413320in}}%
\pgfpathlineto{\pgfqpoint{2.415184in}{0.413320in}}%
\pgfpathlineto{\pgfqpoint{2.412389in}{0.413320in}}%
\pgfpathlineto{\pgfqpoint{2.409699in}{0.413320in}}%
\pgfpathlineto{\pgfqpoint{2.407024in}{0.413320in}}%
\pgfpathlineto{\pgfqpoint{2.404352in}{0.413320in}}%
\pgfpathlineto{\pgfqpoint{2.401675in}{0.413320in}}%
\pgfpathlineto{\pgfqpoint{2.398995in}{0.413320in}}%
\pgfpathclose%
\pgfusepath{stroke,fill}%
\end{pgfscope}%
\begin{pgfscope}%
\pgfpathrectangle{\pgfqpoint{2.398995in}{0.319877in}}{\pgfqpoint{3.986877in}{1.993438in}} %
\pgfusepath{clip}%
\pgfsetbuttcap%
\pgfsetroundjoin%
\definecolor{currentfill}{rgb}{1.000000,1.000000,1.000000}%
\pgfsetfillcolor{currentfill}%
\pgfsetlinewidth{1.003750pt}%
\definecolor{currentstroke}{rgb}{0.868441,0.433082,0.957794}%
\pgfsetstrokecolor{currentstroke}%
\pgfsetdash{}{0pt}%
\pgfpathmoveto{\pgfqpoint{2.398995in}{0.413320in}}%
\pgfpathlineto{\pgfqpoint{2.398995in}{0.798819in}}%
\pgfpathlineto{\pgfqpoint{2.401675in}{0.799103in}}%
\pgfpathlineto{\pgfqpoint{2.404352in}{0.794808in}}%
\pgfpathlineto{\pgfqpoint{2.407024in}{0.792350in}}%
\pgfpathlineto{\pgfqpoint{2.409699in}{0.779516in}}%
\pgfpathlineto{\pgfqpoint{2.412389in}{0.784383in}}%
\pgfpathlineto{\pgfqpoint{2.415184in}{0.799486in}}%
\pgfpathlineto{\pgfqpoint{2.417747in}{0.805400in}}%
\pgfpathlineto{\pgfqpoint{2.420528in}{0.795884in}}%
\pgfpathlineto{\pgfqpoint{2.423098in}{0.802065in}}%
\pgfpathlineto{\pgfqpoint{2.425878in}{0.806194in}}%
\pgfpathlineto{\pgfqpoint{2.428453in}{0.805582in}}%
\pgfpathlineto{\pgfqpoint{2.431251in}{0.807753in}}%
\pgfpathlineto{\pgfqpoint{2.433815in}{0.805325in}}%
\pgfpathlineto{\pgfqpoint{2.436518in}{0.813974in}}%
\pgfpathlineto{\pgfqpoint{2.439167in}{0.810988in}}%
\pgfpathlineto{\pgfqpoint{2.441876in}{0.804307in}}%
\pgfpathlineto{\pgfqpoint{2.444677in}{0.806050in}}%
\pgfpathlineto{\pgfqpoint{2.447209in}{0.802257in}}%
\pgfpathlineto{\pgfqpoint{2.450032in}{0.795591in}}%
\pgfpathlineto{\pgfqpoint{2.452562in}{0.796057in}}%
\pgfpathlineto{\pgfqpoint{2.455353in}{0.797284in}}%
\pgfpathlineto{\pgfqpoint{2.457917in}{0.793842in}}%
\pgfpathlineto{\pgfqpoint{2.460711in}{0.794550in}}%
\pgfpathlineto{\pgfqpoint{2.463280in}{0.799026in}}%
\pgfpathlineto{\pgfqpoint{2.465957in}{0.801001in}}%
\pgfpathlineto{\pgfqpoint{2.468635in}{0.796089in}}%
\pgfpathlineto{\pgfqpoint{2.471311in}{0.796301in}}%
\pgfpathlineto{\pgfqpoint{2.473989in}{0.794759in}}%
\pgfpathlineto{\pgfqpoint{2.476671in}{0.797439in}}%
\pgfpathlineto{\pgfqpoint{2.479420in}{0.802633in}}%
\pgfpathlineto{\pgfqpoint{2.482026in}{0.810830in}}%
\pgfpathlineto{\pgfqpoint{2.484870in}{0.812061in}}%
\pgfpathlineto{\pgfqpoint{2.487384in}{0.806786in}}%
\pgfpathlineto{\pgfqpoint{2.490183in}{0.791838in}}%
\pgfpathlineto{\pgfqpoint{2.492729in}{0.793780in}}%
\pgfpathlineto{\pgfqpoint{2.495542in}{0.801849in}}%
\pgfpathlineto{\pgfqpoint{2.498085in}{0.793724in}}%
\pgfpathlineto{\pgfqpoint{2.500801in}{0.796499in}}%
\pgfpathlineto{\pgfqpoint{2.503454in}{0.800964in}}%
\pgfpathlineto{\pgfqpoint{2.506163in}{0.794914in}}%
\pgfpathlineto{\pgfqpoint{2.508917in}{0.797075in}}%
\pgfpathlineto{\pgfqpoint{2.511478in}{0.799360in}}%
\pgfpathlineto{\pgfqpoint{2.514268in}{0.801423in}}%
\pgfpathlineto{\pgfqpoint{2.516845in}{0.809073in}}%
\pgfpathlineto{\pgfqpoint{2.519607in}{0.813593in}}%
\pgfpathlineto{\pgfqpoint{2.522197in}{0.813612in}}%
\pgfpathlineto{\pgfqpoint{2.524988in}{0.813912in}}%
\pgfpathlineto{\pgfqpoint{2.527560in}{0.811548in}}%
\pgfpathlineto{\pgfqpoint{2.530234in}{0.805268in}}%
\pgfpathlineto{\pgfqpoint{2.532917in}{0.800813in}}%
\pgfpathlineto{\pgfqpoint{2.535624in}{0.796965in}}%
\pgfpathlineto{\pgfqpoint{2.538274in}{0.798666in}}%
\pgfpathlineto{\pgfqpoint{2.540949in}{0.797806in}}%
\pgfpathlineto{\pgfqpoint{2.543765in}{0.794837in}}%
\pgfpathlineto{\pgfqpoint{2.546310in}{0.797856in}}%
\pgfpathlineto{\pgfqpoint{2.549114in}{0.793018in}}%
\pgfpathlineto{\pgfqpoint{2.551664in}{0.793760in}}%
\pgfpathlineto{\pgfqpoint{2.554493in}{0.782241in}}%
\pgfpathlineto{\pgfqpoint{2.557009in}{0.793877in}}%
\pgfpathlineto{\pgfqpoint{2.559790in}{0.805772in}}%
\pgfpathlineto{\pgfqpoint{2.562375in}{0.803908in}}%
\pgfpathlineto{\pgfqpoint{2.565045in}{0.795046in}}%
\pgfpathlineto{\pgfqpoint{2.567730in}{0.793123in}}%
\pgfpathlineto{\pgfqpoint{2.570411in}{0.811092in}}%
\pgfpathlineto{\pgfqpoint{2.573082in}{0.882926in}}%
\pgfpathlineto{\pgfqpoint{2.575779in}{0.934798in}}%
\pgfpathlineto{\pgfqpoint{2.578567in}{0.884471in}}%
\pgfpathlineto{\pgfqpoint{2.581129in}{0.856196in}}%
\pgfpathlineto{\pgfqpoint{2.583913in}{0.841091in}}%
\pgfpathlineto{\pgfqpoint{2.586484in}{0.837101in}}%
\pgfpathlineto{\pgfqpoint{2.589248in}{0.824087in}}%
\pgfpathlineto{\pgfqpoint{2.591842in}{0.812330in}}%
\pgfpathlineto{\pgfqpoint{2.594630in}{0.811351in}}%
\pgfpathlineto{\pgfqpoint{2.597196in}{0.803357in}}%
\pgfpathlineto{\pgfqpoint{2.599920in}{0.794741in}}%
\pgfpathlineto{\pgfqpoint{2.602557in}{0.802331in}}%
\pgfpathlineto{\pgfqpoint{2.605232in}{0.797906in}}%
\pgfpathlineto{\pgfqpoint{2.608004in}{0.796937in}}%
\pgfpathlineto{\pgfqpoint{2.610588in}{0.796278in}}%
\pgfpathlineto{\pgfqpoint{2.613393in}{0.791153in}}%
\pgfpathlineto{\pgfqpoint{2.615934in}{0.789406in}}%
\pgfpathlineto{\pgfqpoint{2.618773in}{0.793966in}}%
\pgfpathlineto{\pgfqpoint{2.621304in}{0.797429in}}%
\pgfpathlineto{\pgfqpoint{2.624077in}{0.797912in}}%
\pgfpathlineto{\pgfqpoint{2.626653in}{0.801725in}}%
\pgfpathlineto{\pgfqpoint{2.629340in}{0.799854in}}%
\pgfpathlineto{\pgfqpoint{2.632018in}{0.801536in}}%
\pgfpathlineto{\pgfqpoint{2.634700in}{0.798790in}}%
\pgfpathlineto{\pgfqpoint{2.637369in}{0.797317in}}%
\pgfpathlineto{\pgfqpoint{2.640053in}{0.800016in}}%
\pgfpathlineto{\pgfqpoint{2.642827in}{0.797689in}}%
\pgfpathlineto{\pgfqpoint{2.645408in}{0.799516in}}%
\pgfpathlineto{\pgfqpoint{2.648196in}{0.803662in}}%
\pgfpathlineto{\pgfqpoint{2.650767in}{0.804334in}}%
\pgfpathlineto{\pgfqpoint{2.653567in}{0.798598in}}%
\pgfpathlineto{\pgfqpoint{2.656124in}{0.793991in}}%
\pgfpathlineto{\pgfqpoint{2.658942in}{0.791558in}}%
\pgfpathlineto{\pgfqpoint{2.661481in}{0.793465in}}%
\pgfpathlineto{\pgfqpoint{2.664151in}{0.795773in}}%
\pgfpathlineto{\pgfqpoint{2.666836in}{0.796646in}}%
\pgfpathlineto{\pgfqpoint{2.669506in}{0.793347in}}%
\pgfpathlineto{\pgfqpoint{2.672301in}{0.795401in}}%
\pgfpathlineto{\pgfqpoint{2.674873in}{0.795179in}}%
\pgfpathlineto{\pgfqpoint{2.677650in}{0.801837in}}%
\pgfpathlineto{\pgfqpoint{2.680224in}{0.801228in}}%
\pgfpathlineto{\pgfqpoint{2.683009in}{0.805600in}}%
\pgfpathlineto{\pgfqpoint{2.685586in}{0.804361in}}%
\pgfpathlineto{\pgfqpoint{2.688328in}{0.798135in}}%
\pgfpathlineto{\pgfqpoint{2.690940in}{0.800144in}}%
\pgfpathlineto{\pgfqpoint{2.693611in}{0.801112in}}%
\pgfpathlineto{\pgfqpoint{2.696293in}{0.798800in}}%
\pgfpathlineto{\pgfqpoint{2.698968in}{0.796625in}}%
\pgfpathlineto{\pgfqpoint{2.701657in}{0.798238in}}%
\pgfpathlineto{\pgfqpoint{2.704326in}{0.799852in}}%
\pgfpathlineto{\pgfqpoint{2.707125in}{0.796018in}}%
\pgfpathlineto{\pgfqpoint{2.709683in}{0.793582in}}%
\pgfpathlineto{\pgfqpoint{2.712477in}{0.796092in}}%
\pgfpathlineto{\pgfqpoint{2.715036in}{0.797273in}}%
\pgfpathlineto{\pgfqpoint{2.717773in}{0.796177in}}%
\pgfpathlineto{\pgfqpoint{2.720404in}{0.794201in}}%
\pgfpathlineto{\pgfqpoint{2.723211in}{0.790828in}}%
\pgfpathlineto{\pgfqpoint{2.725760in}{0.790784in}}%
\pgfpathlineto{\pgfqpoint{2.728439in}{0.787092in}}%
\pgfpathlineto{\pgfqpoint{2.731119in}{0.785490in}}%
\pgfpathlineto{\pgfqpoint{2.733798in}{0.783942in}}%
\pgfpathlineto{\pgfqpoint{2.736476in}{0.774425in}}%
\pgfpathlineto{\pgfqpoint{2.739155in}{0.774425in}}%
\pgfpathlineto{\pgfqpoint{2.741928in}{0.782282in}}%
\pgfpathlineto{\pgfqpoint{2.744510in}{0.788432in}}%
\pgfpathlineto{\pgfqpoint{2.747260in}{0.789765in}}%
\pgfpathlineto{\pgfqpoint{2.749868in}{0.784741in}}%
\pgfpathlineto{\pgfqpoint{2.752614in}{0.787979in}}%
\pgfpathlineto{\pgfqpoint{2.755224in}{0.785528in}}%
\pgfpathlineto{\pgfqpoint{2.758028in}{0.785413in}}%
\pgfpathlineto{\pgfqpoint{2.760581in}{0.790734in}}%
\pgfpathlineto{\pgfqpoint{2.763253in}{0.784670in}}%
\pgfpathlineto{\pgfqpoint{2.765935in}{0.788055in}}%
\pgfpathlineto{\pgfqpoint{2.768617in}{0.783960in}}%
\pgfpathlineto{\pgfqpoint{2.771373in}{0.782734in}}%
\pgfpathlineto{\pgfqpoint{2.773972in}{0.787200in}}%
\pgfpathlineto{\pgfqpoint{2.776767in}{0.785084in}}%
\pgfpathlineto{\pgfqpoint{2.779330in}{0.788802in}}%
\pgfpathlineto{\pgfqpoint{2.782113in}{0.788593in}}%
\pgfpathlineto{\pgfqpoint{2.784687in}{0.784609in}}%
\pgfpathlineto{\pgfqpoint{2.787468in}{0.791002in}}%
\pgfpathlineto{\pgfqpoint{2.790044in}{0.791936in}}%
\pgfpathlineto{\pgfqpoint{2.792721in}{0.794530in}}%
\pgfpathlineto{\pgfqpoint{2.795398in}{0.797981in}}%
\pgfpathlineto{\pgfqpoint{2.798070in}{0.792402in}}%
\pgfpathlineto{\pgfqpoint{2.800756in}{0.797513in}}%
\pgfpathlineto{\pgfqpoint{2.803435in}{0.800163in}}%
\pgfpathlineto{\pgfqpoint{2.806175in}{0.802235in}}%
\pgfpathlineto{\pgfqpoint{2.808792in}{0.802210in}}%
\pgfpathlineto{\pgfqpoint{2.811597in}{0.804742in}}%
\pgfpathlineto{\pgfqpoint{2.814141in}{0.797245in}}%
\pgfpathlineto{\pgfqpoint{2.816867in}{0.795907in}}%
\pgfpathlineto{\pgfqpoint{2.819506in}{0.798121in}}%
\pgfpathlineto{\pgfqpoint{2.822303in}{0.793637in}}%
\pgfpathlineto{\pgfqpoint{2.824851in}{0.797088in}}%
\pgfpathlineto{\pgfqpoint{2.827567in}{0.801890in}}%
\pgfpathlineto{\pgfqpoint{2.830219in}{0.798205in}}%
\pgfpathlineto{\pgfqpoint{2.832894in}{0.799964in}}%
\pgfpathlineto{\pgfqpoint{2.835698in}{0.793273in}}%
\pgfpathlineto{\pgfqpoint{2.838254in}{0.794826in}}%
\pgfpathlineto{\pgfqpoint{2.841055in}{0.789034in}}%
\pgfpathlineto{\pgfqpoint{2.843611in}{0.788079in}}%
\pgfpathlineto{\pgfqpoint{2.846408in}{0.793518in}}%
\pgfpathlineto{\pgfqpoint{2.848960in}{0.792795in}}%
\pgfpathlineto{\pgfqpoint{2.851793in}{0.791520in}}%
\pgfpathlineto{\pgfqpoint{2.854325in}{0.795637in}}%
\pgfpathlineto{\pgfqpoint{2.857003in}{0.794421in}}%
\pgfpathlineto{\pgfqpoint{2.859668in}{0.794136in}}%
\pgfpathlineto{\pgfqpoint{2.862402in}{0.800430in}}%
\pgfpathlineto{\pgfqpoint{2.865031in}{0.799597in}}%
\pgfpathlineto{\pgfqpoint{2.867713in}{0.794574in}}%
\pgfpathlineto{\pgfqpoint{2.870475in}{0.796334in}}%
\pgfpathlineto{\pgfqpoint{2.873074in}{0.797019in}}%
\pgfpathlineto{\pgfqpoint{2.875882in}{0.791731in}}%
\pgfpathlineto{\pgfqpoint{2.878431in}{0.792970in}}%
\pgfpathlineto{\pgfqpoint{2.881254in}{0.793943in}}%
\pgfpathlineto{\pgfqpoint{2.883780in}{0.788621in}}%
\pgfpathlineto{\pgfqpoint{2.886578in}{0.792847in}}%
\pgfpathlineto{\pgfqpoint{2.889145in}{0.792774in}}%
\pgfpathlineto{\pgfqpoint{2.891809in}{0.788808in}}%
\pgfpathlineto{\pgfqpoint{2.894487in}{0.794274in}}%
\pgfpathlineto{\pgfqpoint{2.897179in}{0.799034in}}%
\pgfpathlineto{\pgfqpoint{2.899858in}{0.787267in}}%
\pgfpathlineto{\pgfqpoint{2.902535in}{0.794026in}}%
\pgfpathlineto{\pgfqpoint{2.905341in}{0.794314in}}%
\pgfpathlineto{\pgfqpoint{2.907882in}{0.793847in}}%
\pgfpathlineto{\pgfqpoint{2.910631in}{0.789720in}}%
\pgfpathlineto{\pgfqpoint{2.913243in}{0.789790in}}%
\pgfpathlineto{\pgfqpoint{2.916061in}{0.797944in}}%
\pgfpathlineto{\pgfqpoint{2.918606in}{0.794143in}}%
\pgfpathlineto{\pgfqpoint{2.921363in}{0.798366in}}%
\pgfpathlineto{\pgfqpoint{2.923963in}{0.796222in}}%
\pgfpathlineto{\pgfqpoint{2.926655in}{0.794081in}}%
\pgfpathlineto{\pgfqpoint{2.929321in}{0.794783in}}%
\pgfpathlineto{\pgfqpoint{2.932033in}{0.793848in}}%
\pgfpathlineto{\pgfqpoint{2.934759in}{0.788840in}}%
\pgfpathlineto{\pgfqpoint{2.937352in}{0.788918in}}%
\pgfpathlineto{\pgfqpoint{2.940120in}{0.781528in}}%
\pgfpathlineto{\pgfqpoint{2.942711in}{0.787275in}}%
\pgfpathlineto{\pgfqpoint{2.945461in}{0.792894in}}%
\pgfpathlineto{\pgfqpoint{2.948068in}{0.793748in}}%
\pgfpathlineto{\pgfqpoint{2.950884in}{0.793081in}}%
\pgfpathlineto{\pgfqpoint{2.953422in}{0.794090in}}%
\pgfpathlineto{\pgfqpoint{2.956103in}{0.792121in}}%
\pgfpathlineto{\pgfqpoint{2.958782in}{0.791760in}}%
\pgfpathlineto{\pgfqpoint{2.961460in}{0.792801in}}%
\pgfpathlineto{\pgfqpoint{2.964127in}{0.793737in}}%
\pgfpathlineto{\pgfqpoint{2.966812in}{0.799502in}}%
\pgfpathlineto{\pgfqpoint{2.969599in}{0.795497in}}%
\pgfpathlineto{\pgfqpoint{2.972177in}{0.798095in}}%
\pgfpathlineto{\pgfqpoint{2.974972in}{0.796446in}}%
\pgfpathlineto{\pgfqpoint{2.977517in}{0.796042in}}%
\pgfpathlineto{\pgfqpoint{2.980341in}{0.793540in}}%
\pgfpathlineto{\pgfqpoint{2.982885in}{0.792815in}}%
\pgfpathlineto{\pgfqpoint{2.985666in}{0.792525in}}%
\pgfpathlineto{\pgfqpoint{2.988238in}{0.796873in}}%
\pgfpathlineto{\pgfqpoint{2.990978in}{0.798443in}}%
\pgfpathlineto{\pgfqpoint{2.993595in}{0.799060in}}%
\pgfpathlineto{\pgfqpoint{2.996300in}{0.795638in}}%
\pgfpathlineto{\pgfqpoint{2.999103in}{0.801345in}}%
\pgfpathlineto{\pgfqpoint{3.001635in}{0.793841in}}%
\pgfpathlineto{\pgfqpoint{3.004419in}{0.795736in}}%
\pgfpathlineto{\pgfqpoint{3.006993in}{0.787376in}}%
\pgfpathlineto{\pgfqpoint{3.009784in}{0.787170in}}%
\pgfpathlineto{\pgfqpoint{3.012351in}{0.787403in}}%
\pgfpathlineto{\pgfqpoint{3.015097in}{0.795398in}}%
\pgfpathlineto{\pgfqpoint{3.017707in}{0.787068in}}%
\pgfpathlineto{\pgfqpoint{3.020382in}{0.788494in}}%
\pgfpathlineto{\pgfqpoint{3.023058in}{0.787312in}}%
\pgfpathlineto{\pgfqpoint{3.025803in}{0.787595in}}%
\pgfpathlineto{\pgfqpoint{3.028412in}{0.791847in}}%
\pgfpathlineto{\pgfqpoint{3.031091in}{0.796184in}}%
\pgfpathlineto{\pgfqpoint{3.033921in}{0.793326in}}%
\pgfpathlineto{\pgfqpoint{3.036456in}{0.800889in}}%
\pgfpathlineto{\pgfqpoint{3.039262in}{0.800986in}}%
\pgfpathlineto{\pgfqpoint{3.041813in}{0.796544in}}%
\pgfpathlineto{\pgfqpoint{3.044568in}{0.786946in}}%
\pgfpathlineto{\pgfqpoint{3.047157in}{0.792283in}}%
\pgfpathlineto{\pgfqpoint{3.049988in}{0.789398in}}%
\pgfpathlineto{\pgfqpoint{3.052526in}{0.789726in}}%
\pgfpathlineto{\pgfqpoint{3.055202in}{0.787740in}}%
\pgfpathlineto{\pgfqpoint{3.057884in}{0.789971in}}%
\pgfpathlineto{\pgfqpoint{3.060561in}{0.793431in}}%
\pgfpathlineto{\pgfqpoint{3.063230in}{0.789929in}}%
\pgfpathlineto{\pgfqpoint{3.065916in}{0.795580in}}%
\pgfpathlineto{\pgfqpoint{3.068709in}{0.792654in}}%
\pgfpathlineto{\pgfqpoint{3.071266in}{0.790773in}}%
\pgfpathlineto{\pgfqpoint{3.074056in}{0.796724in}}%
\pgfpathlineto{\pgfqpoint{3.076631in}{0.794965in}}%
\pgfpathlineto{\pgfqpoint{3.079381in}{0.801074in}}%
\pgfpathlineto{\pgfqpoint{3.081990in}{0.789888in}}%
\pgfpathlineto{\pgfqpoint{3.084671in}{0.795295in}}%
\pgfpathlineto{\pgfqpoint{3.087343in}{0.794573in}}%
\pgfpathlineto{\pgfqpoint{3.090023in}{0.787551in}}%
\pgfpathlineto{\pgfqpoint{3.092699in}{0.784135in}}%
\pgfpathlineto{\pgfqpoint{3.095388in}{0.781966in}}%
\pgfpathlineto{\pgfqpoint{3.098163in}{0.782132in}}%
\pgfpathlineto{\pgfqpoint{3.100737in}{0.774425in}}%
\pgfpathlineto{\pgfqpoint{3.103508in}{0.774425in}}%
\pgfpathlineto{\pgfqpoint{3.106094in}{0.784551in}}%
\pgfpathlineto{\pgfqpoint{3.108896in}{0.787112in}}%
\pgfpathlineto{\pgfqpoint{3.111451in}{0.793264in}}%
\pgfpathlineto{\pgfqpoint{3.114242in}{0.783453in}}%
\pgfpathlineto{\pgfqpoint{3.116807in}{0.792116in}}%
\pgfpathlineto{\pgfqpoint{3.119487in}{0.787968in}}%
\pgfpathlineto{\pgfqpoint{3.122163in}{0.794414in}}%
\pgfpathlineto{\pgfqpoint{3.124842in}{0.793921in}}%
\pgfpathlineto{\pgfqpoint{3.127512in}{0.794335in}}%
\pgfpathlineto{\pgfqpoint{3.130199in}{0.799121in}}%
\pgfpathlineto{\pgfqpoint{3.132946in}{0.798435in}}%
\pgfpathlineto{\pgfqpoint{3.135550in}{0.797963in}}%
\pgfpathlineto{\pgfqpoint{3.138375in}{0.796530in}}%
\pgfpathlineto{\pgfqpoint{3.140913in}{0.780845in}}%
\pgfpathlineto{\pgfqpoint{3.143740in}{0.774425in}}%
\pgfpathlineto{\pgfqpoint{3.146271in}{0.774425in}}%
\pgfpathlineto{\pgfqpoint{3.149057in}{0.774425in}}%
\pgfpathlineto{\pgfqpoint{3.151612in}{0.774425in}}%
\pgfpathlineto{\pgfqpoint{3.154327in}{0.775306in}}%
\pgfpathlineto{\pgfqpoint{3.156981in}{0.786537in}}%
\pgfpathlineto{\pgfqpoint{3.159675in}{0.781775in}}%
\pgfpathlineto{\pgfqpoint{3.162474in}{0.782660in}}%
\pgfpathlineto{\pgfqpoint{3.165019in}{0.776942in}}%
\pgfpathlineto{\pgfqpoint{3.167776in}{0.774425in}}%
\pgfpathlineto{\pgfqpoint{3.170375in}{0.774425in}}%
\pgfpathlineto{\pgfqpoint{3.173142in}{0.774425in}}%
\pgfpathlineto{\pgfqpoint{3.175724in}{0.775400in}}%
\pgfpathlineto{\pgfqpoint{3.178525in}{0.779126in}}%
\pgfpathlineto{\pgfqpoint{3.181089in}{0.787704in}}%
\pgfpathlineto{\pgfqpoint{3.183760in}{0.780955in}}%
\pgfpathlineto{\pgfqpoint{3.186440in}{0.789569in}}%
\pgfpathlineto{\pgfqpoint{3.189117in}{0.790517in}}%
\pgfpathlineto{\pgfqpoint{3.191796in}{0.786267in}}%
\pgfpathlineto{\pgfqpoint{3.194508in}{0.786805in}}%
\pgfpathlineto{\pgfqpoint{3.197226in}{0.779296in}}%
\pgfpathlineto{\pgfqpoint{3.199823in}{0.774425in}}%
\pgfpathlineto{\pgfqpoint{3.202562in}{0.774493in}}%
\pgfpathlineto{\pgfqpoint{3.205195in}{0.785272in}}%
\pgfpathlineto{\pgfqpoint{3.207984in}{0.792385in}}%
\pgfpathlineto{\pgfqpoint{3.210545in}{0.789202in}}%
\pgfpathlineto{\pgfqpoint{3.213342in}{0.787353in}}%
\pgfpathlineto{\pgfqpoint{3.215908in}{0.790264in}}%
\pgfpathlineto{\pgfqpoint{3.218586in}{0.789541in}}%
\pgfpathlineto{\pgfqpoint{3.221255in}{0.788688in}}%
\pgfpathlineto{\pgfqpoint{3.223942in}{0.801703in}}%
\pgfpathlineto{\pgfqpoint{3.226609in}{0.793088in}}%
\pgfpathlineto{\pgfqpoint{3.229310in}{0.788348in}}%
\pgfpathlineto{\pgfqpoint{3.232069in}{0.789686in}}%
\pgfpathlineto{\pgfqpoint{3.234658in}{0.789887in}}%
\pgfpathlineto{\pgfqpoint{3.237411in}{0.794158in}}%
\pgfpathlineto{\pgfqpoint{3.240010in}{0.796026in}}%
\pgfpathlineto{\pgfqpoint{3.242807in}{0.800322in}}%
\pgfpathlineto{\pgfqpoint{3.245363in}{0.797521in}}%
\pgfpathlineto{\pgfqpoint{3.248049in}{0.802156in}}%
\pgfpathlineto{\pgfqpoint{3.250716in}{0.799361in}}%
\pgfpathlineto{\pgfqpoint{3.253404in}{0.795255in}}%
\pgfpathlineto{\pgfqpoint{3.256083in}{0.797586in}}%
\pgfpathlineto{\pgfqpoint{3.258784in}{0.795167in}}%
\pgfpathlineto{\pgfqpoint{3.261594in}{0.797963in}}%
\pgfpathlineto{\pgfqpoint{3.264119in}{0.801305in}}%
\pgfpathlineto{\pgfqpoint{3.266849in}{0.791189in}}%
\pgfpathlineto{\pgfqpoint{3.269478in}{0.790901in}}%
\pgfpathlineto{\pgfqpoint{3.272254in}{0.787798in}}%
\pgfpathlineto{\pgfqpoint{3.274831in}{0.790143in}}%
\pgfpathlineto{\pgfqpoint{3.277603in}{0.790007in}}%
\pgfpathlineto{\pgfqpoint{3.280189in}{0.792455in}}%
\pgfpathlineto{\pgfqpoint{3.282870in}{0.793357in}}%
\pgfpathlineto{\pgfqpoint{3.285534in}{0.803644in}}%
\pgfpathlineto{\pgfqpoint{3.288225in}{0.800616in}}%
\pgfpathlineto{\pgfqpoint{3.290890in}{0.793654in}}%
\pgfpathlineto{\pgfqpoint{3.293574in}{0.793309in}}%
\pgfpathlineto{\pgfqpoint{3.296376in}{0.793722in}}%
\pgfpathlineto{\pgfqpoint{3.298937in}{0.795915in}}%
\pgfpathlineto{\pgfqpoint{3.301719in}{0.795887in}}%
\pgfpathlineto{\pgfqpoint{3.304295in}{0.794850in}}%
\pgfpathlineto{\pgfqpoint{3.307104in}{0.795828in}}%
\pgfpathlineto{\pgfqpoint{3.309652in}{0.795503in}}%
\pgfpathlineto{\pgfqpoint{3.312480in}{0.793000in}}%
\pgfpathlineto{\pgfqpoint{3.315008in}{0.793462in}}%
\pgfpathlineto{\pgfqpoint{3.317688in}{0.796566in}}%
\pgfpathlineto{\pgfqpoint{3.320366in}{0.798205in}}%
\pgfpathlineto{\pgfqpoint{3.323049in}{0.798620in}}%
\pgfpathlineto{\pgfqpoint{3.325860in}{0.793037in}}%
\pgfpathlineto{\pgfqpoint{3.328401in}{0.796524in}}%
\pgfpathlineto{\pgfqpoint{3.331183in}{0.800959in}}%
\pgfpathlineto{\pgfqpoint{3.333758in}{0.794094in}}%
\pgfpathlineto{\pgfqpoint{3.336541in}{0.790508in}}%
\pgfpathlineto{\pgfqpoint{3.339101in}{0.794025in}}%
\pgfpathlineto{\pgfqpoint{3.341893in}{0.800013in}}%
\pgfpathlineto{\pgfqpoint{3.344468in}{0.798595in}}%
\pgfpathlineto{\pgfqpoint{3.347139in}{0.797509in}}%
\pgfpathlineto{\pgfqpoint{3.349828in}{0.796882in}}%
\pgfpathlineto{\pgfqpoint{3.352505in}{0.801213in}}%
\pgfpathlineto{\pgfqpoint{3.355177in}{0.801555in}}%
\pgfpathlineto{\pgfqpoint{3.357862in}{0.802092in}}%
\pgfpathlineto{\pgfqpoint{3.360620in}{0.809503in}}%
\pgfpathlineto{\pgfqpoint{3.363221in}{0.803205in}}%
\pgfpathlineto{\pgfqpoint{3.365996in}{0.798175in}}%
\pgfpathlineto{\pgfqpoint{3.368577in}{0.795869in}}%
\pgfpathlineto{\pgfqpoint{3.371357in}{0.798580in}}%
\pgfpathlineto{\pgfqpoint{3.373921in}{0.804807in}}%
\pgfpathlineto{\pgfqpoint{3.376735in}{0.798888in}}%
\pgfpathlineto{\pgfqpoint{3.379290in}{0.800407in}}%
\pgfpathlineto{\pgfqpoint{3.381959in}{0.800211in}}%
\pgfpathlineto{\pgfqpoint{3.384647in}{0.799837in}}%
\pgfpathlineto{\pgfqpoint{3.387309in}{0.802655in}}%
\pgfpathlineto{\pgfqpoint{3.390102in}{0.802407in}}%
\pgfpathlineto{\pgfqpoint{3.392681in}{0.806341in}}%
\pgfpathlineto{\pgfqpoint{3.395461in}{0.802434in}}%
\pgfpathlineto{\pgfqpoint{3.398037in}{0.803206in}}%
\pgfpathlineto{\pgfqpoint{3.400783in}{0.794265in}}%
\pgfpathlineto{\pgfqpoint{3.403394in}{0.804912in}}%
\pgfpathlineto{\pgfqpoint{3.406202in}{0.801266in}}%
\pgfpathlineto{\pgfqpoint{3.408752in}{0.796652in}}%
\pgfpathlineto{\pgfqpoint{3.411431in}{0.793428in}}%
\pgfpathlineto{\pgfqpoint{3.414109in}{0.800541in}}%
\pgfpathlineto{\pgfqpoint{3.416780in}{0.799095in}}%
\pgfpathlineto{\pgfqpoint{3.419455in}{0.791213in}}%
\pgfpathlineto{\pgfqpoint{3.422142in}{0.795749in}}%
\pgfpathlineto{\pgfqpoint{3.424887in}{0.795264in}}%
\pgfpathlineto{\pgfqpoint{3.427501in}{0.797122in}}%
\pgfpathlineto{\pgfqpoint{3.430313in}{0.794935in}}%
\pgfpathlineto{\pgfqpoint{3.432851in}{0.794506in}}%
\pgfpathlineto{\pgfqpoint{3.435635in}{0.795950in}}%
\pgfpathlineto{\pgfqpoint{3.438210in}{0.796220in}}%
\pgfpathlineto{\pgfqpoint{3.440996in}{0.792836in}}%
\pgfpathlineto{\pgfqpoint{3.443574in}{0.802261in}}%
\pgfpathlineto{\pgfqpoint{3.446257in}{0.797342in}}%
\pgfpathlineto{\pgfqpoint{3.448926in}{0.791222in}}%
\pgfpathlineto{\pgfqpoint{3.451597in}{0.793866in}}%
\pgfpathlineto{\pgfqpoint{3.454285in}{0.788996in}}%
\pgfpathlineto{\pgfqpoint{3.456960in}{0.788805in}}%
\pgfpathlineto{\pgfqpoint{3.459695in}{0.793061in}}%
\pgfpathlineto{\pgfqpoint{3.462321in}{0.792337in}}%
\pgfpathlineto{\pgfqpoint{3.465072in}{0.790135in}}%
\pgfpathlineto{\pgfqpoint{3.467678in}{0.796321in}}%
\pgfpathlineto{\pgfqpoint{3.470466in}{0.793564in}}%
\pgfpathlineto{\pgfqpoint{3.473021in}{0.798897in}}%
\pgfpathlineto{\pgfqpoint{3.475821in}{0.798326in}}%
\pgfpathlineto{\pgfqpoint{3.478378in}{0.795492in}}%
\pgfpathlineto{\pgfqpoint{3.481072in}{0.797607in}}%
\pgfpathlineto{\pgfqpoint{3.483744in}{0.804263in}}%
\pgfpathlineto{\pgfqpoint{3.486442in}{0.799927in}}%
\pgfpathlineto{\pgfqpoint{3.489223in}{0.794631in}}%
\pgfpathlineto{\pgfqpoint{3.491783in}{0.796665in}}%
\pgfpathlineto{\pgfqpoint{3.494581in}{0.800762in}}%
\pgfpathlineto{\pgfqpoint{3.497139in}{0.800891in}}%
\pgfpathlineto{\pgfqpoint{3.499909in}{0.796251in}}%
\pgfpathlineto{\pgfqpoint{3.502488in}{0.794933in}}%
\pgfpathlineto{\pgfqpoint{3.505262in}{0.802655in}}%
\pgfpathlineto{\pgfqpoint{3.507840in}{0.804258in}}%
\pgfpathlineto{\pgfqpoint{3.510533in}{0.805782in}}%
\pgfpathlineto{\pgfqpoint{3.513209in}{0.799755in}}%
\pgfpathlineto{\pgfqpoint{3.515884in}{0.797256in}}%
\pgfpathlineto{\pgfqpoint{3.518565in}{0.798292in}}%
\pgfpathlineto{\pgfqpoint{3.521244in}{0.796918in}}%
\pgfpathlineto{\pgfqpoint{3.524041in}{0.800933in}}%
\pgfpathlineto{\pgfqpoint{3.526601in}{0.796239in}}%
\pgfpathlineto{\pgfqpoint{3.529327in}{0.796129in}}%
\pgfpathlineto{\pgfqpoint{3.531955in}{0.795082in}}%
\pgfpathlineto{\pgfqpoint{3.534783in}{0.792752in}}%
\pgfpathlineto{\pgfqpoint{3.537309in}{0.796095in}}%
\pgfpathlineto{\pgfqpoint{3.540093in}{0.793316in}}%
\pgfpathlineto{\pgfqpoint{3.542656in}{0.795352in}}%
\pgfpathlineto{\pgfqpoint{3.545349in}{0.796080in}}%
\pgfpathlineto{\pgfqpoint{3.548029in}{0.796542in}}%
\pgfpathlineto{\pgfqpoint{3.550713in}{0.797688in}}%
\pgfpathlineto{\pgfqpoint{3.553498in}{0.793940in}}%
\pgfpathlineto{\pgfqpoint{3.556061in}{0.799243in}}%
\pgfpathlineto{\pgfqpoint{3.558853in}{0.800984in}}%
\pgfpathlineto{\pgfqpoint{3.561420in}{0.792187in}}%
\pgfpathlineto{\pgfqpoint{3.564188in}{0.791607in}}%
\pgfpathlineto{\pgfqpoint{3.566774in}{0.793563in}}%
\pgfpathlineto{\pgfqpoint{3.569584in}{0.796302in}}%
\pgfpathlineto{\pgfqpoint{3.572126in}{0.792350in}}%
\pgfpathlineto{\pgfqpoint{3.574814in}{0.796591in}}%
\pgfpathlineto{\pgfqpoint{3.577487in}{0.794178in}}%
\pgfpathlineto{\pgfqpoint{3.580191in}{0.795189in}}%
\pgfpathlineto{\pgfqpoint{3.582851in}{0.794241in}}%
\pgfpathlineto{\pgfqpoint{3.585532in}{0.799275in}}%
\pgfpathlineto{\pgfqpoint{3.588258in}{0.794785in}}%
\pgfpathlineto{\pgfqpoint{3.590883in}{0.794128in}}%
\pgfpathlineto{\pgfqpoint{3.593620in}{0.795234in}}%
\pgfpathlineto{\pgfqpoint{3.596240in}{0.793289in}}%
\pgfpathlineto{\pgfqpoint{3.598998in}{0.796453in}}%
\pgfpathlineto{\pgfqpoint{3.601590in}{0.796777in}}%
\pgfpathlineto{\pgfqpoint{3.604387in}{0.799397in}}%
\pgfpathlineto{\pgfqpoint{3.606951in}{0.798192in}}%
\pgfpathlineto{\pgfqpoint{3.609632in}{0.792757in}}%
\pgfpathlineto{\pgfqpoint{3.612311in}{0.782926in}}%
\pgfpathlineto{\pgfqpoint{3.614982in}{0.787492in}}%
\pgfpathlineto{\pgfqpoint{3.617667in}{0.788716in}}%
\pgfpathlineto{\pgfqpoint{3.620345in}{0.787137in}}%
\pgfpathlineto{\pgfqpoint{3.623165in}{0.789865in}}%
\pgfpathlineto{\pgfqpoint{3.625689in}{0.792339in}}%
\pgfpathlineto{\pgfqpoint{3.628460in}{0.787744in}}%
\pgfpathlineto{\pgfqpoint{3.631058in}{0.787200in}}%
\pgfpathlineto{\pgfqpoint{3.633858in}{0.788188in}}%
\pgfpathlineto{\pgfqpoint{3.636413in}{0.795263in}}%
\pgfpathlineto{\pgfqpoint{3.639207in}{0.789393in}}%
\pgfpathlineto{\pgfqpoint{3.641773in}{0.789301in}}%
\pgfpathlineto{\pgfqpoint{3.644452in}{0.784852in}}%
\pgfpathlineto{\pgfqpoint{3.647130in}{0.780082in}}%
\pgfpathlineto{\pgfqpoint{3.649837in}{0.784956in}}%
\pgfpathlineto{\pgfqpoint{3.652628in}{0.785326in}}%
\pgfpathlineto{\pgfqpoint{3.655165in}{0.790715in}}%
\pgfpathlineto{\pgfqpoint{3.657917in}{0.786104in}}%
\pgfpathlineto{\pgfqpoint{3.660515in}{0.775843in}}%
\pgfpathlineto{\pgfqpoint{3.663276in}{0.774816in}}%
\pgfpathlineto{\pgfqpoint{3.665864in}{0.780967in}}%
\pgfpathlineto{\pgfqpoint{3.668665in}{0.786598in}}%
\pgfpathlineto{\pgfqpoint{3.671232in}{0.790176in}}%
\pgfpathlineto{\pgfqpoint{3.673911in}{0.801224in}}%
\pgfpathlineto{\pgfqpoint{3.676591in}{0.788419in}}%
\pgfpathlineto{\pgfqpoint{3.679273in}{0.784261in}}%
\pgfpathlineto{\pgfqpoint{3.681948in}{0.785029in}}%
\pgfpathlineto{\pgfqpoint{3.684620in}{0.783856in}}%
\pgfpathlineto{\pgfqpoint{3.687442in}{0.785764in}}%
\pgfpathlineto{\pgfqpoint{3.689983in}{0.794587in}}%
\pgfpathlineto{\pgfqpoint{3.692765in}{0.794437in}}%
\pgfpathlineto{\pgfqpoint{3.695331in}{0.801020in}}%
\pgfpathlineto{\pgfqpoint{3.698125in}{0.793794in}}%
\pgfpathlineto{\pgfqpoint{3.700684in}{0.795148in}}%
\pgfpathlineto{\pgfqpoint{3.703460in}{0.797512in}}%
\pgfpathlineto{\pgfqpoint{3.706053in}{0.799207in}}%
\pgfpathlineto{\pgfqpoint{3.708729in}{0.799958in}}%
\pgfpathlineto{\pgfqpoint{3.711410in}{0.795518in}}%
\pgfpathlineto{\pgfqpoint{3.714086in}{0.796616in}}%
\pgfpathlineto{\pgfqpoint{3.716875in}{0.800380in}}%
\pgfpathlineto{\pgfqpoint{3.719446in}{0.801873in}}%
\pgfpathlineto{\pgfqpoint{3.722228in}{0.796292in}}%
\pgfpathlineto{\pgfqpoint{3.724804in}{0.790656in}}%
\pgfpathlineto{\pgfqpoint{3.727581in}{0.794240in}}%
\pgfpathlineto{\pgfqpoint{3.730158in}{0.797371in}}%
\pgfpathlineto{\pgfqpoint{3.732950in}{0.796260in}}%
\pgfpathlineto{\pgfqpoint{3.735509in}{0.794996in}}%
\pgfpathlineto{\pgfqpoint{3.738194in}{0.792532in}}%
\pgfpathlineto{\pgfqpoint{3.740874in}{0.795341in}}%
\pgfpathlineto{\pgfqpoint{3.743548in}{0.794877in}}%
\pgfpathlineto{\pgfqpoint{3.746229in}{0.798477in}}%
\pgfpathlineto{\pgfqpoint{3.748903in}{0.798293in}}%
\pgfpathlineto{\pgfqpoint{3.751728in}{0.798387in}}%
\pgfpathlineto{\pgfqpoint{3.754265in}{0.803164in}}%
\pgfpathlineto{\pgfqpoint{3.757065in}{0.803063in}}%
\pgfpathlineto{\pgfqpoint{3.759622in}{0.799827in}}%
\pgfpathlineto{\pgfqpoint{3.762389in}{0.797732in}}%
\pgfpathlineto{\pgfqpoint{3.764966in}{0.799727in}}%
\pgfpathlineto{\pgfqpoint{3.767782in}{0.797489in}}%
\pgfpathlineto{\pgfqpoint{3.770323in}{0.787987in}}%
\pgfpathlineto{\pgfqpoint{3.773014in}{0.790198in}}%
\pgfpathlineto{\pgfqpoint{3.775691in}{0.786181in}}%
\pgfpathlineto{\pgfqpoint{3.778370in}{0.780083in}}%
\pgfpathlineto{\pgfqpoint{3.781046in}{0.782289in}}%
\pgfpathlineto{\pgfqpoint{3.783725in}{0.781933in}}%
\pgfpathlineto{\pgfqpoint{3.786504in}{0.783813in}}%
\pgfpathlineto{\pgfqpoint{3.789084in}{0.787544in}}%
\pgfpathlineto{\pgfqpoint{3.791897in}{0.786665in}}%
\pgfpathlineto{\pgfqpoint{3.794435in}{0.782959in}}%
\pgfpathlineto{\pgfqpoint{3.797265in}{0.787911in}}%
\pgfpathlineto{\pgfqpoint{3.799797in}{0.782962in}}%
\pgfpathlineto{\pgfqpoint{3.802569in}{0.790760in}}%
\pgfpathlineto{\pgfqpoint{3.805145in}{0.794800in}}%
\pgfpathlineto{\pgfqpoint{3.807832in}{0.795567in}}%
\pgfpathlineto{\pgfqpoint{3.810510in}{0.795712in}}%
\pgfpathlineto{\pgfqpoint{3.813172in}{0.801439in}}%
\pgfpathlineto{\pgfqpoint{3.815983in}{0.798928in}}%
\pgfpathlineto{\pgfqpoint{3.818546in}{0.797303in}}%
\pgfpathlineto{\pgfqpoint{3.821315in}{0.794424in}}%
\pgfpathlineto{\pgfqpoint{3.823903in}{0.792563in}}%
\pgfpathlineto{\pgfqpoint{3.826679in}{0.788939in}}%
\pgfpathlineto{\pgfqpoint{3.829252in}{0.789434in}}%
\pgfpathlineto{\pgfqpoint{3.832053in}{0.793393in}}%
\pgfpathlineto{\pgfqpoint{3.834616in}{0.791010in}}%
\pgfpathlineto{\pgfqpoint{3.837286in}{0.792274in}}%
\pgfpathlineto{\pgfqpoint{3.839960in}{0.793973in}}%
\pgfpathlineto{\pgfqpoint{3.842641in}{0.798106in}}%
\pgfpathlineto{\pgfqpoint{3.845329in}{0.794220in}}%
\pgfpathlineto{\pgfqpoint{3.848005in}{0.795789in}}%
\pgfpathlineto{\pgfqpoint{3.850814in}{0.797951in}}%
\pgfpathlineto{\pgfqpoint{3.853358in}{0.798498in}}%
\pgfpathlineto{\pgfqpoint{3.856100in}{0.800679in}}%
\pgfpathlineto{\pgfqpoint{3.858720in}{0.800007in}}%
\pgfpathlineto{\pgfqpoint{3.861561in}{0.800910in}}%
\pgfpathlineto{\pgfqpoint{3.864073in}{0.798760in}}%
\pgfpathlineto{\pgfqpoint{3.866815in}{0.799792in}}%
\pgfpathlineto{\pgfqpoint{3.869435in}{0.796901in}}%
\pgfpathlineto{\pgfqpoint{3.872114in}{0.802183in}}%
\pgfpathlineto{\pgfqpoint{3.874790in}{0.805334in}}%
\pgfpathlineto{\pgfqpoint{3.877466in}{0.794875in}}%
\pgfpathlineto{\pgfqpoint{3.880237in}{0.796076in}}%
\pgfpathlineto{\pgfqpoint{3.882850in}{0.797240in}}%
\pgfpathlineto{\pgfqpoint{3.885621in}{0.794488in}}%
\pgfpathlineto{\pgfqpoint{3.888188in}{0.798474in}}%
\pgfpathlineto{\pgfqpoint{3.890926in}{0.794835in}}%
\pgfpathlineto{\pgfqpoint{3.893541in}{0.797386in}}%
\pgfpathlineto{\pgfqpoint{3.896345in}{0.799667in}}%
\pgfpathlineto{\pgfqpoint{3.898891in}{0.809141in}}%
\pgfpathlineto{\pgfqpoint{3.901573in}{0.804485in}}%
\pgfpathlineto{\pgfqpoint{3.904252in}{0.803152in}}%
\pgfpathlineto{\pgfqpoint{3.906918in}{0.805361in}}%
\pgfpathlineto{\pgfqpoint{3.909602in}{0.798386in}}%
\pgfpathlineto{\pgfqpoint{3.912296in}{0.801027in}}%
\pgfpathlineto{\pgfqpoint{3.915107in}{0.806606in}}%
\pgfpathlineto{\pgfqpoint{3.917646in}{0.810590in}}%
\pgfpathlineto{\pgfqpoint{3.920412in}{0.811925in}}%
\pgfpathlineto{\pgfqpoint{3.923005in}{0.798515in}}%
\pgfpathlineto{\pgfqpoint{3.925778in}{0.799697in}}%
\pgfpathlineto{\pgfqpoint{3.928347in}{0.800514in}}%
\pgfpathlineto{\pgfqpoint{3.931202in}{0.805902in}}%
\pgfpathlineto{\pgfqpoint{3.933714in}{0.800743in}}%
\pgfpathlineto{\pgfqpoint{3.936395in}{0.798182in}}%
\pgfpathlineto{\pgfqpoint{3.939075in}{0.797981in}}%
\pgfpathlineto{\pgfqpoint{3.941778in}{0.794003in}}%
\pgfpathlineto{\pgfqpoint{3.944431in}{0.790375in}}%
\pgfpathlineto{\pgfqpoint{3.947101in}{0.793348in}}%
\pgfpathlineto{\pgfqpoint{3.949894in}{0.786366in}}%
\pgfpathlineto{\pgfqpoint{3.952464in}{0.783816in}}%
\pgfpathlineto{\pgfqpoint{3.955211in}{0.794246in}}%
\pgfpathlineto{\pgfqpoint{3.957823in}{0.797413in}}%
\pgfpathlineto{\pgfqpoint{3.960635in}{0.799149in}}%
\pgfpathlineto{\pgfqpoint{3.963176in}{0.798807in}}%
\pgfpathlineto{\pgfqpoint{3.966013in}{0.798034in}}%
\pgfpathlineto{\pgfqpoint{3.968523in}{0.800166in}}%
\pgfpathlineto{\pgfqpoint{3.971250in}{0.793562in}}%
\pgfpathlineto{\pgfqpoint{3.973885in}{0.790343in}}%
\pgfpathlineto{\pgfqpoint{3.976563in}{0.794175in}}%
\pgfpathlineto{\pgfqpoint{3.979389in}{0.798517in}}%
\pgfpathlineto{\pgfqpoint{3.981929in}{0.796049in}}%
\pgfpathlineto{\pgfqpoint{3.984714in}{0.795748in}}%
\pgfpathlineto{\pgfqpoint{3.987270in}{0.795151in}}%
\pgfpathlineto{\pgfqpoint{3.990055in}{0.797674in}}%
\pgfpathlineto{\pgfqpoint{3.992642in}{0.797132in}}%
\pgfpathlineto{\pgfqpoint{3.995417in}{0.801393in}}%
\pgfpathlineto{\pgfqpoint{3.997990in}{0.798908in}}%
\pgfpathlineto{\pgfqpoint{4.000674in}{0.796111in}}%
\pgfpathlineto{\pgfqpoint{4.003348in}{0.796610in}}%
\pgfpathlineto{\pgfqpoint{4.006034in}{0.798819in}}%
\pgfpathlineto{\pgfqpoint{4.008699in}{0.796858in}}%
\pgfpathlineto{\pgfqpoint{4.011394in}{0.797235in}}%
\pgfpathlineto{\pgfqpoint{4.014186in}{0.799724in}}%
\pgfpathlineto{\pgfqpoint{4.016744in}{0.800281in}}%
\pgfpathlineto{\pgfqpoint{4.019518in}{0.800305in}}%
\pgfpathlineto{\pgfqpoint{4.022097in}{0.801194in}}%
\pgfpathlineto{\pgfqpoint{4.024868in}{0.797558in}}%
\pgfpathlineto{\pgfqpoint{4.027447in}{0.798736in}}%
\pgfpathlineto{\pgfqpoint{4.030229in}{0.797587in}}%
\pgfpathlineto{\pgfqpoint{4.032817in}{0.798551in}}%
\pgfpathlineto{\pgfqpoint{4.035492in}{0.798310in}}%
\pgfpathlineto{\pgfqpoint{4.038174in}{0.798079in}}%
\pgfpathlineto{\pgfqpoint{4.040852in}{0.799797in}}%
\pgfpathlineto{\pgfqpoint{4.043667in}{0.803339in}}%
\pgfpathlineto{\pgfqpoint{4.046210in}{0.800756in}}%
\pgfpathlineto{\pgfqpoint{4.049006in}{0.801080in}}%
\pgfpathlineto{\pgfqpoint{4.051557in}{0.802989in}}%
\pgfpathlineto{\pgfqpoint{4.054326in}{0.805701in}}%
\pgfpathlineto{\pgfqpoint{4.056911in}{0.797151in}}%
\pgfpathlineto{\pgfqpoint{4.059702in}{0.798400in}}%
\pgfpathlineto{\pgfqpoint{4.062266in}{0.796868in}}%
\pgfpathlineto{\pgfqpoint{4.064957in}{0.800873in}}%
\pgfpathlineto{\pgfqpoint{4.067636in}{0.798498in}}%
\pgfpathlineto{\pgfqpoint{4.070313in}{0.793903in}}%
\pgfpathlineto{\pgfqpoint{4.072985in}{0.800332in}}%
\pgfpathlineto{\pgfqpoint{4.075705in}{0.795984in}}%
\pgfpathlineto{\pgfqpoint{4.078471in}{0.791963in}}%
\pgfpathlineto{\pgfqpoint{4.081018in}{0.790069in}}%
\pgfpathlineto{\pgfqpoint{4.083870in}{0.795217in}}%
\pgfpathlineto{\pgfqpoint{4.086385in}{0.792130in}}%
\pgfpathlineto{\pgfqpoint{4.089159in}{0.795987in}}%
\pgfpathlineto{\pgfqpoint{4.091729in}{0.795739in}}%
\pgfpathlineto{\pgfqpoint{4.094527in}{0.796741in}}%
\pgfpathlineto{\pgfqpoint{4.097092in}{0.802344in}}%
\pgfpathlineto{\pgfqpoint{4.099777in}{0.795465in}}%
\pgfpathlineto{\pgfqpoint{4.102456in}{0.801293in}}%
\pgfpathlineto{\pgfqpoint{4.105185in}{0.798150in}}%
\pgfpathlineto{\pgfqpoint{4.107814in}{0.798625in}}%
\pgfpathlineto{\pgfqpoint{4.110488in}{0.797902in}}%
\pgfpathlineto{\pgfqpoint{4.113252in}{0.797847in}}%
\pgfpathlineto{\pgfqpoint{4.115844in}{0.801333in}}%
\pgfpathlineto{\pgfqpoint{4.118554in}{0.795559in}}%
\pgfpathlineto{\pgfqpoint{4.121205in}{0.797939in}}%
\pgfpathlineto{\pgfqpoint{4.124019in}{0.792258in}}%
\pgfpathlineto{\pgfqpoint{4.126553in}{0.794788in}}%
\pgfpathlineto{\pgfqpoint{4.129349in}{0.797086in}}%
\pgfpathlineto{\pgfqpoint{4.131920in}{0.799942in}}%
\pgfpathlineto{\pgfqpoint{4.134615in}{0.795948in}}%
\pgfpathlineto{\pgfqpoint{4.137272in}{0.797545in}}%
\pgfpathlineto{\pgfqpoint{4.139963in}{0.793096in}}%
\pgfpathlineto{\pgfqpoint{4.142713in}{0.800659in}}%
\pgfpathlineto{\pgfqpoint{4.145310in}{0.796861in}}%
\pgfpathlineto{\pgfqpoint{4.148082in}{0.795229in}}%
\pgfpathlineto{\pgfqpoint{4.150665in}{0.796729in}}%
\pgfpathlineto{\pgfqpoint{4.153423in}{0.796018in}}%
\pgfpathlineto{\pgfqpoint{4.156016in}{0.796327in}}%
\pgfpathlineto{\pgfqpoint{4.158806in}{0.797321in}}%
\pgfpathlineto{\pgfqpoint{4.161380in}{0.800130in}}%
\pgfpathlineto{\pgfqpoint{4.164059in}{0.802482in}}%
\pgfpathlineto{\pgfqpoint{4.166737in}{0.799275in}}%
\pgfpathlineto{\pgfqpoint{4.169415in}{0.798695in}}%
\pgfpathlineto{\pgfqpoint{4.172093in}{0.797999in}}%
\pgfpathlineto{\pgfqpoint{4.174770in}{0.798621in}}%
\pgfpathlineto{\pgfqpoint{4.177593in}{0.799537in}}%
\pgfpathlineto{\pgfqpoint{4.180129in}{0.797103in}}%
\pgfpathlineto{\pgfqpoint{4.182899in}{0.792956in}}%
\pgfpathlineto{\pgfqpoint{4.185481in}{0.795920in}}%
\pgfpathlineto{\pgfqpoint{4.188318in}{0.797289in}}%
\pgfpathlineto{\pgfqpoint{4.190842in}{0.793146in}}%
\pgfpathlineto{\pgfqpoint{4.193638in}{0.793957in}}%
\pgfpathlineto{\pgfqpoint{4.196186in}{0.795617in}}%
\pgfpathlineto{\pgfqpoint{4.198878in}{0.798493in}}%
\pgfpathlineto{\pgfqpoint{4.201542in}{0.793193in}}%
\pgfpathlineto{\pgfqpoint{4.204240in}{0.791851in}}%
\pgfpathlineto{\pgfqpoint{4.207076in}{0.783596in}}%
\pgfpathlineto{\pgfqpoint{4.209597in}{0.785029in}}%
\pgfpathlineto{\pgfqpoint{4.212383in}{0.791516in}}%
\pgfpathlineto{\pgfqpoint{4.214948in}{0.793640in}}%
\pgfpathlineto{\pgfqpoint{4.217694in}{0.790358in}}%
\pgfpathlineto{\pgfqpoint{4.220304in}{0.795166in}}%
\pgfpathlineto{\pgfqpoint{4.223082in}{0.802878in}}%
\pgfpathlineto{\pgfqpoint{4.225654in}{0.794597in}}%
\pgfpathlineto{\pgfqpoint{4.228331in}{0.794316in}}%
\pgfpathlineto{\pgfqpoint{4.231013in}{0.790938in}}%
\pgfpathlineto{\pgfqpoint{4.233691in}{0.791375in}}%
\pgfpathlineto{\pgfqpoint{4.236375in}{0.792019in}}%
\pgfpathlineto{\pgfqpoint{4.239084in}{0.793700in}}%
\pgfpathlineto{\pgfqpoint{4.241900in}{0.796831in}}%
\pgfpathlineto{\pgfqpoint{4.244394in}{0.796637in}}%
\pgfpathlineto{\pgfqpoint{4.247225in}{0.791864in}}%
\pgfpathlineto{\pgfqpoint{4.249767in}{0.797344in}}%
\pgfpathlineto{\pgfqpoint{4.252581in}{0.796382in}}%
\pgfpathlineto{\pgfqpoint{4.255120in}{0.798777in}}%
\pgfpathlineto{\pgfqpoint{4.257958in}{0.796539in}}%
\pgfpathlineto{\pgfqpoint{4.260477in}{0.794349in}}%
\pgfpathlineto{\pgfqpoint{4.263157in}{0.794337in}}%
\pgfpathlineto{\pgfqpoint{4.265824in}{0.793064in}}%
\pgfpathlineto{\pgfqpoint{4.268590in}{0.794080in}}%
\pgfpathlineto{\pgfqpoint{4.271187in}{0.795475in}}%
\pgfpathlineto{\pgfqpoint{4.273874in}{0.789494in}}%
\pgfpathlineto{\pgfqpoint{4.276635in}{0.791047in}}%
\pgfpathlineto{\pgfqpoint{4.279212in}{0.797402in}}%
\pgfpathlineto{\pgfqpoint{4.282000in}{0.794049in}}%
\pgfpathlineto{\pgfqpoint{4.284586in}{0.792720in}}%
\pgfpathlineto{\pgfqpoint{4.287399in}{0.796814in}}%
\pgfpathlineto{\pgfqpoint{4.289936in}{0.794849in}}%
\pgfpathlineto{\pgfqpoint{4.292786in}{0.793047in}}%
\pgfpathlineto{\pgfqpoint{4.295299in}{0.793445in}}%
\pgfpathlineto{\pgfqpoint{4.297977in}{0.791239in}}%
\pgfpathlineto{\pgfqpoint{4.300656in}{0.791304in}}%
\pgfpathlineto{\pgfqpoint{4.303357in}{0.794815in}}%
\pgfpathlineto{\pgfqpoint{4.306118in}{0.794933in}}%
\pgfpathlineto{\pgfqpoint{4.308691in}{0.794112in}}%
\pgfpathlineto{\pgfqpoint{4.311494in}{0.793162in}}%
\pgfpathlineto{\pgfqpoint{4.314032in}{0.797882in}}%
\pgfpathlineto{\pgfqpoint{4.316856in}{0.790989in}}%
\pgfpathlineto{\pgfqpoint{4.319405in}{0.794818in}}%
\pgfpathlineto{\pgfqpoint{4.322181in}{0.800490in}}%
\pgfpathlineto{\pgfqpoint{4.324760in}{0.794642in}}%
\pgfpathlineto{\pgfqpoint{4.327440in}{0.798002in}}%
\pgfpathlineto{\pgfqpoint{4.330118in}{0.791453in}}%
\pgfpathlineto{\pgfqpoint{4.332796in}{0.797279in}}%
\pgfpathlineto{\pgfqpoint{4.335463in}{0.799804in}}%
\pgfpathlineto{\pgfqpoint{4.338154in}{0.802837in}}%
\pgfpathlineto{\pgfqpoint{4.340976in}{0.799401in}}%
\pgfpathlineto{\pgfqpoint{4.343510in}{0.797939in}}%
\pgfpathlineto{\pgfqpoint{4.346263in}{0.795568in}}%
\pgfpathlineto{\pgfqpoint{4.348868in}{0.793588in}}%
\pgfpathlineto{\pgfqpoint{4.351645in}{0.797170in}}%
\pgfpathlineto{\pgfqpoint{4.354224in}{0.798395in}}%
\pgfpathlineto{\pgfqpoint{4.357014in}{0.796969in}}%
\pgfpathlineto{\pgfqpoint{4.359582in}{0.799456in}}%
\pgfpathlineto{\pgfqpoint{4.362270in}{0.799090in}}%
\pgfpathlineto{\pgfqpoint{4.364936in}{0.795027in}}%
\pgfpathlineto{\pgfqpoint{4.367646in}{0.792435in}}%
\pgfpathlineto{\pgfqpoint{4.370437in}{0.800173in}}%
\pgfpathlineto{\pgfqpoint{4.372976in}{0.806430in}}%
\pgfpathlineto{\pgfqpoint{4.375761in}{0.799696in}}%
\pgfpathlineto{\pgfqpoint{4.378329in}{0.797525in}}%
\pgfpathlineto{\pgfqpoint{4.381097in}{0.795661in}}%
\pgfpathlineto{\pgfqpoint{4.383674in}{0.802493in}}%
\pgfpathlineto{\pgfqpoint{4.386431in}{0.796574in}}%
\pgfpathlineto{\pgfqpoint{4.389044in}{0.802763in}}%
\pgfpathlineto{\pgfqpoint{4.391721in}{0.801108in}}%
\pgfpathlineto{\pgfqpoint{4.394400in}{0.796600in}}%
\pgfpathlineto{\pgfqpoint{4.397076in}{0.789041in}}%
\pgfpathlineto{\pgfqpoint{4.399745in}{0.789405in}}%
\pgfpathlineto{\pgfqpoint{4.402468in}{0.790461in}}%
\pgfpathlineto{\pgfqpoint{4.405234in}{0.796336in}}%
\pgfpathlineto{\pgfqpoint{4.407788in}{0.791186in}}%
\pgfpathlineto{\pgfqpoint{4.410587in}{0.795669in}}%
\pgfpathlineto{\pgfqpoint{4.413149in}{0.793003in}}%
\pgfpathlineto{\pgfqpoint{4.415932in}{0.791775in}}%
\pgfpathlineto{\pgfqpoint{4.418506in}{0.791040in}}%
\pgfpathlineto{\pgfqpoint{4.421292in}{0.788283in}}%
\pgfpathlineto{\pgfqpoint{4.423863in}{0.792347in}}%
\pgfpathlineto{\pgfqpoint{4.426534in}{0.786441in}}%
\pgfpathlineto{\pgfqpoint{4.429220in}{0.790344in}}%
\pgfpathlineto{\pgfqpoint{4.431901in}{0.792739in}}%
\pgfpathlineto{\pgfqpoint{4.434569in}{0.795743in}}%
\pgfpathlineto{\pgfqpoint{4.437253in}{0.800401in}}%
\pgfpathlineto{\pgfqpoint{4.440041in}{0.805376in}}%
\pgfpathlineto{\pgfqpoint{4.442611in}{0.807218in}}%
\pgfpathlineto{\pgfqpoint{4.445423in}{0.807676in}}%
\pgfpathlineto{\pgfqpoint{4.447965in}{0.796509in}}%
\pgfpathlineto{\pgfqpoint{4.450767in}{0.794454in}}%
\pgfpathlineto{\pgfqpoint{4.453312in}{0.789130in}}%
\pgfpathlineto{\pgfqpoint{4.456138in}{0.787468in}}%
\pgfpathlineto{\pgfqpoint{4.458681in}{0.786873in}}%
\pgfpathlineto{\pgfqpoint{4.461367in}{0.795166in}}%
\pgfpathlineto{\pgfqpoint{4.464029in}{0.792311in}}%
\pgfpathlineto{\pgfqpoint{4.466717in}{0.800496in}}%
\pgfpathlineto{\pgfqpoint{4.469492in}{0.804247in}}%
\pgfpathlineto{\pgfqpoint{4.472059in}{0.799125in}}%
\pgfpathlineto{\pgfqpoint{4.474861in}{0.806565in}}%
\pgfpathlineto{\pgfqpoint{4.477430in}{0.798911in}}%
\pgfpathlineto{\pgfqpoint{4.480201in}{0.800484in}}%
\pgfpathlineto{\pgfqpoint{4.482778in}{0.797136in}}%
\pgfpathlineto{\pgfqpoint{4.485581in}{0.807488in}}%
\pgfpathlineto{\pgfqpoint{4.488130in}{0.807899in}}%
\pgfpathlineto{\pgfqpoint{4.490822in}{0.803977in}}%
\pgfpathlineto{\pgfqpoint{4.493492in}{0.808573in}}%
\pgfpathlineto{\pgfqpoint{4.496167in}{0.800587in}}%
\pgfpathlineto{\pgfqpoint{4.498850in}{0.807332in}}%
\pgfpathlineto{\pgfqpoint{4.501529in}{0.798937in}}%
\pgfpathlineto{\pgfqpoint{4.504305in}{0.801060in}}%
\pgfpathlineto{\pgfqpoint{4.506893in}{0.797955in}}%
\pgfpathlineto{\pgfqpoint{4.509643in}{0.801486in}}%
\pgfpathlineto{\pgfqpoint{4.512246in}{0.799484in}}%
\pgfpathlineto{\pgfqpoint{4.515080in}{0.800961in}}%
\pgfpathlineto{\pgfqpoint{4.517598in}{0.794229in}}%
\pgfpathlineto{\pgfqpoint{4.520345in}{0.802023in}}%
\pgfpathlineto{\pgfqpoint{4.522962in}{0.811200in}}%
\pgfpathlineto{\pgfqpoint{4.525640in}{0.799604in}}%
\pgfpathlineto{\pgfqpoint{4.528307in}{0.797253in}}%
\pgfpathlineto{\pgfqpoint{4.530990in}{0.792257in}}%
\pgfpathlineto{\pgfqpoint{4.533764in}{0.799737in}}%
\pgfpathlineto{\pgfqpoint{4.536400in}{0.816390in}}%
\pgfpathlineto{\pgfqpoint{4.539144in}{0.803801in}}%
\pgfpathlineto{\pgfqpoint{4.541711in}{0.796788in}}%
\pgfpathlineto{\pgfqpoint{4.544464in}{0.792611in}}%
\pgfpathlineto{\pgfqpoint{4.547064in}{0.800296in}}%
\pgfpathlineto{\pgfqpoint{4.549822in}{0.793412in}}%
\pgfpathlineto{\pgfqpoint{4.552425in}{0.791867in}}%
\pgfpathlineto{\pgfqpoint{4.555106in}{0.787017in}}%
\pgfpathlineto{\pgfqpoint{4.557777in}{0.791212in}}%
\pgfpathlineto{\pgfqpoint{4.560448in}{0.791339in}}%
\pgfpathlineto{\pgfqpoint{4.563125in}{0.789825in}}%
\pgfpathlineto{\pgfqpoint{4.565820in}{0.794359in}}%
\pgfpathlineto{\pgfqpoint{4.568612in}{0.792146in}}%
\pgfpathlineto{\pgfqpoint{4.571171in}{0.787336in}}%
\pgfpathlineto{\pgfqpoint{4.573947in}{0.783114in}}%
\pgfpathlineto{\pgfqpoint{4.576531in}{0.787554in}}%
\pgfpathlineto{\pgfqpoint{4.579305in}{0.788752in}}%
\pgfpathlineto{\pgfqpoint{4.581888in}{0.791278in}}%
\pgfpathlineto{\pgfqpoint{4.584672in}{0.796430in}}%
\pgfpathlineto{\pgfqpoint{4.587244in}{0.792567in}}%
\pgfpathlineto{\pgfqpoint{4.589920in}{0.789736in}}%
\pgfpathlineto{\pgfqpoint{4.592589in}{0.783280in}}%
\pgfpathlineto{\pgfqpoint{4.595281in}{0.783418in}}%
\pgfpathlineto{\pgfqpoint{4.597951in}{0.790446in}}%
\pgfpathlineto{\pgfqpoint{4.600633in}{0.792770in}}%
\pgfpathlineto{\pgfqpoint{4.603430in}{0.796822in}}%
\pgfpathlineto{\pgfqpoint{4.605990in}{0.795062in}}%
\pgfpathlineto{\pgfqpoint{4.608808in}{0.795893in}}%
\pgfpathlineto{\pgfqpoint{4.611350in}{0.795691in}}%
\pgfpathlineto{\pgfqpoint{4.614134in}{0.795726in}}%
\pgfpathlineto{\pgfqpoint{4.616702in}{0.792094in}}%
\pgfpathlineto{\pgfqpoint{4.619529in}{0.797645in}}%
\pgfpathlineto{\pgfqpoint{4.622056in}{0.796696in}}%
\pgfpathlineto{\pgfqpoint{4.624741in}{0.798500in}}%
\pgfpathlineto{\pgfqpoint{4.627411in}{0.796028in}}%
\pgfpathlineto{\pgfqpoint{4.630096in}{0.811784in}}%
\pgfpathlineto{\pgfqpoint{4.632902in}{0.809556in}}%
\pgfpathlineto{\pgfqpoint{4.635445in}{0.803960in}}%
\pgfpathlineto{\pgfqpoint{4.638204in}{0.802086in}}%
\pgfpathlineto{\pgfqpoint{4.640809in}{0.798255in}}%
\pgfpathlineto{\pgfqpoint{4.643628in}{0.798250in}}%
\pgfpathlineto{\pgfqpoint{4.646169in}{0.799766in}}%
\pgfpathlineto{\pgfqpoint{4.648922in}{0.797307in}}%
\pgfpathlineto{\pgfqpoint{4.651524in}{0.796435in}}%
\pgfpathlineto{\pgfqpoint{4.654203in}{0.804689in}}%
\pgfpathlineto{\pgfqpoint{4.656873in}{0.811416in}}%
\pgfpathlineto{\pgfqpoint{4.659590in}{0.806630in}}%
\pgfpathlineto{\pgfqpoint{4.662237in}{0.811429in}}%
\pgfpathlineto{\pgfqpoint{4.664923in}{0.805020in}}%
\pgfpathlineto{\pgfqpoint{4.667764in}{0.795656in}}%
\pgfpathlineto{\pgfqpoint{4.670261in}{0.789734in}}%
\pgfpathlineto{\pgfqpoint{4.673068in}{0.794029in}}%
\pgfpathlineto{\pgfqpoint{4.675619in}{0.798202in}}%
\pgfpathlineto{\pgfqpoint{4.678448in}{0.789980in}}%
\pgfpathlineto{\pgfqpoint{4.680988in}{0.797437in}}%
\pgfpathlineto{\pgfqpoint{4.683799in}{0.799867in}}%
\pgfpathlineto{\pgfqpoint{4.686337in}{0.789406in}}%
\pgfpathlineto{\pgfqpoint{4.689051in}{0.793723in}}%
\pgfpathlineto{\pgfqpoint{4.691694in}{0.792189in}}%
\pgfpathlineto{\pgfqpoint{4.694381in}{0.790921in}}%
\pgfpathlineto{\pgfqpoint{4.697170in}{0.792676in}}%
\pgfpathlineto{\pgfqpoint{4.699734in}{0.792277in}}%
\pgfpathlineto{\pgfqpoint{4.702517in}{0.789310in}}%
\pgfpathlineto{\pgfqpoint{4.705094in}{0.792463in}}%
\pgfpathlineto{\pgfqpoint{4.707824in}{0.798275in}}%
\pgfpathlineto{\pgfqpoint{4.710437in}{0.791422in}}%
\pgfpathlineto{\pgfqpoint{4.713275in}{0.788220in}}%
\pgfpathlineto{\pgfqpoint{4.715806in}{0.794027in}}%
\pgfpathlineto{\pgfqpoint{4.718486in}{0.799966in}}%
\pgfpathlineto{\pgfqpoint{4.721160in}{0.796155in}}%
\pgfpathlineto{\pgfqpoint{4.723873in}{0.797622in}}%
\pgfpathlineto{\pgfqpoint{4.726508in}{0.801149in}}%
\pgfpathlineto{\pgfqpoint{4.729233in}{0.791401in}}%
\pgfpathlineto{\pgfqpoint{4.731901in}{0.798587in}}%
\pgfpathlineto{\pgfqpoint{4.734552in}{0.797512in}}%
\pgfpathlineto{\pgfqpoint{4.737348in}{0.795632in}}%
\pgfpathlineto{\pgfqpoint{4.739912in}{0.796236in}}%
\pgfpathlineto{\pgfqpoint{4.742696in}{0.793636in}}%
\pgfpathlineto{\pgfqpoint{4.745256in}{0.788069in}}%
\pgfpathlineto{\pgfqpoint{4.748081in}{0.793085in}}%
\pgfpathlineto{\pgfqpoint{4.750627in}{0.791861in}}%
\pgfpathlineto{\pgfqpoint{4.753298in}{0.791223in}}%
\pgfpathlineto{\pgfqpoint{4.755983in}{0.789093in}}%
\pgfpathlineto{\pgfqpoint{4.758653in}{0.787228in}}%
\pgfpathlineto{\pgfqpoint{4.761337in}{0.788434in}}%
\pgfpathlineto{\pgfqpoint{4.764018in}{0.784549in}}%
\pgfpathlineto{\pgfqpoint{4.766783in}{0.795836in}}%
\pgfpathlineto{\pgfqpoint{4.769367in}{0.791450in}}%
\pgfpathlineto{\pgfqpoint{4.772198in}{0.788964in}}%
\pgfpathlineto{\pgfqpoint{4.774732in}{0.785690in}}%
\pgfpathlineto{\pgfqpoint{4.777535in}{0.792443in}}%
\pgfpathlineto{\pgfqpoint{4.780083in}{0.790694in}}%
\pgfpathlineto{\pgfqpoint{4.782872in}{0.794989in}}%
\pgfpathlineto{\pgfqpoint{4.785445in}{0.792118in}}%
\pgfpathlineto{\pgfqpoint{4.788116in}{0.792272in}}%
\pgfpathlineto{\pgfqpoint{4.790798in}{0.791904in}}%
\pgfpathlineto{\pgfqpoint{4.793512in}{0.796752in}}%
\pgfpathlineto{\pgfqpoint{4.796274in}{0.796290in}}%
\pgfpathlineto{\pgfqpoint{4.798830in}{0.795846in}}%
\pgfpathlineto{\pgfqpoint{4.801586in}{0.794649in}}%
\pgfpathlineto{\pgfqpoint{4.804193in}{0.798111in}}%
\pgfpathlineto{\pgfqpoint{4.807017in}{0.792724in}}%
\pgfpathlineto{\pgfqpoint{4.809538in}{0.785961in}}%
\pgfpathlineto{\pgfqpoint{4.812377in}{0.780782in}}%
\pgfpathlineto{\pgfqpoint{4.814907in}{0.781887in}}%
\pgfpathlineto{\pgfqpoint{4.817587in}{0.775080in}}%
\pgfpathlineto{\pgfqpoint{4.820265in}{0.777190in}}%
\pgfpathlineto{\pgfqpoint{4.822945in}{0.774425in}}%
\pgfpathlineto{\pgfqpoint{4.825619in}{0.774425in}}%
\pgfpathlineto{\pgfqpoint{4.828291in}{0.777665in}}%
\pgfpathlineto{\pgfqpoint{4.831045in}{0.778337in}}%
\pgfpathlineto{\pgfqpoint{4.833657in}{0.788473in}}%
\pgfpathlineto{\pgfqpoint{4.837992in}{0.786639in}}%
\pgfpathlineto{\pgfqpoint{4.839922in}{0.787645in}}%
\pgfpathlineto{\pgfqpoint{4.842380in}{0.793117in}}%
\pgfpathlineto{\pgfqpoint{4.844361in}{0.792974in}}%
\pgfpathlineto{\pgfqpoint{4.847127in}{0.792082in}}%
\pgfpathlineto{\pgfqpoint{4.849715in}{0.783974in}}%
\pgfpathlineto{\pgfqpoint{4.852404in}{0.787311in}}%
\pgfpathlineto{\pgfqpoint{4.855070in}{0.786330in}}%
\pgfpathlineto{\pgfqpoint{4.857807in}{0.785980in}}%
\pgfpathlineto{\pgfqpoint{4.860544in}{0.790205in}}%
\pgfpathlineto{\pgfqpoint{4.863116in}{0.789420in}}%
\pgfpathlineto{\pgfqpoint{4.865910in}{0.792093in}}%
\pgfpathlineto{\pgfqpoint{4.868474in}{0.791798in}}%
\pgfpathlineto{\pgfqpoint{4.871209in}{0.791506in}}%
\pgfpathlineto{\pgfqpoint{4.873832in}{0.792654in}}%
\pgfpathlineto{\pgfqpoint{4.876636in}{0.794018in}}%
\pgfpathlineto{\pgfqpoint{4.879180in}{0.790260in}}%
\pgfpathlineto{\pgfqpoint{4.881864in}{0.788289in}}%
\pgfpathlineto{\pgfqpoint{4.884540in}{0.784532in}}%
\pgfpathlineto{\pgfqpoint{4.887211in}{0.789564in}}%
\pgfpathlineto{\pgfqpoint{4.889902in}{0.784332in}}%
\pgfpathlineto{\pgfqpoint{4.892611in}{0.791484in}}%
\pgfpathlineto{\pgfqpoint{4.895399in}{0.789735in}}%
\pgfpathlineto{\pgfqpoint{4.897938in}{0.789193in}}%
\pgfpathlineto{\pgfqpoint{4.900712in}{0.789240in}}%
\pgfpathlineto{\pgfqpoint{4.903295in}{0.787899in}}%
\pgfpathlineto{\pgfqpoint{4.906096in}{0.803402in}}%
\pgfpathlineto{\pgfqpoint{4.908648in}{0.807374in}}%
\pgfpathlineto{\pgfqpoint{4.911435in}{0.807482in}}%
\pgfpathlineto{\pgfqpoint{4.914009in}{0.802433in}}%
\pgfpathlineto{\pgfqpoint{4.916681in}{0.803754in}}%
\pgfpathlineto{\pgfqpoint{4.919352in}{0.801663in}}%
\pgfpathlineto{\pgfqpoint{4.922041in}{0.804194in}}%
\pgfpathlineto{\pgfqpoint{4.924708in}{0.805596in}}%
\pgfpathlineto{\pgfqpoint{4.927400in}{0.799015in}}%
\pgfpathlineto{\pgfqpoint{4.930170in}{0.801228in}}%
\pgfpathlineto{\pgfqpoint{4.932742in}{0.793737in}}%
\pgfpathlineto{\pgfqpoint{4.935515in}{0.804473in}}%
\pgfpathlineto{\pgfqpoint{4.938112in}{0.800489in}}%
\pgfpathlineto{\pgfqpoint{4.940881in}{0.797832in}}%
\pgfpathlineto{\pgfqpoint{4.943466in}{0.795595in}}%
\pgfpathlineto{\pgfqpoint{4.946151in}{0.794549in}}%
\pgfpathlineto{\pgfqpoint{4.948827in}{0.798073in}}%
\pgfpathlineto{\pgfqpoint{4.951504in}{0.793089in}}%
\pgfpathlineto{\pgfqpoint{4.954182in}{0.788819in}}%
\pgfpathlineto{\pgfqpoint{4.956862in}{0.792140in}}%
\pgfpathlineto{\pgfqpoint{4.959689in}{0.789871in}}%
\pgfpathlineto{\pgfqpoint{4.962219in}{0.785905in}}%
\pgfpathlineto{\pgfqpoint{4.965002in}{0.790441in}}%
\pgfpathlineto{\pgfqpoint{4.967575in}{0.793872in}}%
\pgfpathlineto{\pgfqpoint{4.970314in}{0.799134in}}%
\pgfpathlineto{\pgfqpoint{4.972933in}{0.795698in}}%
\pgfpathlineto{\pgfqpoint{4.975703in}{0.797823in}}%
\pgfpathlineto{\pgfqpoint{4.978287in}{0.799752in}}%
\pgfpathlineto{\pgfqpoint{4.980967in}{0.799497in}}%
\pgfpathlineto{\pgfqpoint{4.983637in}{0.803251in}}%
\pgfpathlineto{\pgfqpoint{4.986325in}{0.808491in}}%
\pgfpathlineto{\pgfqpoint{4.989001in}{0.805167in}}%
\pgfpathlineto{\pgfqpoint{4.991683in}{0.803208in}}%
\pgfpathlineto{\pgfqpoint{4.994390in}{0.797673in}}%
\pgfpathlineto{\pgfqpoint{4.997028in}{0.799829in}}%
\pgfpathlineto{\pgfqpoint{4.999780in}{0.805288in}}%
\pgfpathlineto{\pgfqpoint{5.002384in}{0.807366in}}%
\pgfpathlineto{\pgfqpoint{5.005178in}{0.797675in}}%
\pgfpathlineto{\pgfqpoint{5.007751in}{0.803042in}}%
\pgfpathlineto{\pgfqpoint{5.010562in}{0.804977in}}%
\pgfpathlineto{\pgfqpoint{5.013104in}{0.802809in}}%
\pgfpathlineto{\pgfqpoint{5.015820in}{0.827301in}}%
\pgfpathlineto{\pgfqpoint{5.018466in}{0.838946in}}%
\pgfpathlineto{\pgfqpoint{5.021147in}{0.825206in}}%
\pgfpathlineto{\pgfqpoint{5.023927in}{0.829102in}}%
\pgfpathlineto{\pgfqpoint{5.026501in}{0.829075in}}%
\pgfpathlineto{\pgfqpoint{5.029275in}{0.817987in}}%
\pgfpathlineto{\pgfqpoint{5.031849in}{0.819019in}}%
\pgfpathlineto{\pgfqpoint{5.034649in}{0.835065in}}%
\pgfpathlineto{\pgfqpoint{5.037214in}{0.826255in}}%
\pgfpathlineto{\pgfqpoint{5.039962in}{0.820772in}}%
\pgfpathlineto{\pgfqpoint{5.042572in}{0.815696in}}%
\pgfpathlineto{\pgfqpoint{5.045249in}{0.812378in}}%
\pgfpathlineto{\pgfqpoint{5.047924in}{0.809554in}}%
\pgfpathlineto{\pgfqpoint{5.050606in}{0.823835in}}%
\pgfpathlineto{\pgfqpoint{5.053284in}{0.824085in}}%
\pgfpathlineto{\pgfqpoint{5.055952in}{0.824606in}}%
\pgfpathlineto{\pgfqpoint{5.058711in}{0.819718in}}%
\pgfpathlineto{\pgfqpoint{5.061315in}{0.833730in}}%
\pgfpathlineto{\pgfqpoint{5.064144in}{0.827395in}}%
\pgfpathlineto{\pgfqpoint{5.066677in}{0.813288in}}%
\pgfpathlineto{\pgfqpoint{5.069463in}{0.814398in}}%
\pgfpathlineto{\pgfqpoint{5.072030in}{0.811759in}}%
\pgfpathlineto{\pgfqpoint{5.074851in}{0.800631in}}%
\pgfpathlineto{\pgfqpoint{5.077390in}{0.808294in}}%
\pgfpathlineto{\pgfqpoint{5.080067in}{0.805298in}}%
\pgfpathlineto{\pgfqpoint{5.082746in}{0.803211in}}%
\pgfpathlineto{\pgfqpoint{5.085426in}{0.802545in}}%
\pgfpathlineto{\pgfqpoint{5.088103in}{0.804523in}}%
\pgfpathlineto{\pgfqpoint{5.090788in}{0.807770in}}%
\pgfpathlineto{\pgfqpoint{5.093579in}{0.799937in}}%
\pgfpathlineto{\pgfqpoint{5.096142in}{0.797194in}}%
\pgfpathlineto{\pgfqpoint{5.098948in}{0.798518in}}%
\pgfpathlineto{\pgfqpoint{5.101496in}{0.800211in}}%
\pgfpathlineto{\pgfqpoint{5.104312in}{0.803217in}}%
\pgfpathlineto{\pgfqpoint{5.106842in}{0.805188in}}%
\pgfpathlineto{\pgfqpoint{5.109530in}{0.801585in}}%
\pgfpathlineto{\pgfqpoint{5.112209in}{0.802021in}}%
\pgfpathlineto{\pgfqpoint{5.114887in}{0.797926in}}%
\pgfpathlineto{\pgfqpoint{5.117550in}{0.803158in}}%
\pgfpathlineto{\pgfqpoint{5.120243in}{0.804243in}}%
\pgfpathlineto{\pgfqpoint{5.123042in}{0.798718in}}%
\pgfpathlineto{\pgfqpoint{5.125599in}{0.802854in}}%
\pgfpathlineto{\pgfqpoint{5.128421in}{0.797183in}}%
\pgfpathlineto{\pgfqpoint{5.130953in}{0.800419in}}%
\pgfpathlineto{\pgfqpoint{5.133716in}{0.794931in}}%
\pgfpathlineto{\pgfqpoint{5.136311in}{0.802647in}}%
\pgfpathlineto{\pgfqpoint{5.139072in}{0.804863in}}%
\pgfpathlineto{\pgfqpoint{5.141660in}{0.800597in}}%
\pgfpathlineto{\pgfqpoint{5.144349in}{0.805991in}}%
\pgfpathlineto{\pgfqpoint{5.147029in}{0.803515in}}%
\pgfpathlineto{\pgfqpoint{5.149734in}{0.801136in}}%
\pgfpathlineto{\pgfqpoint{5.152382in}{0.798370in}}%
\pgfpathlineto{\pgfqpoint{5.155059in}{0.794195in}}%
\pgfpathlineto{\pgfqpoint{5.157815in}{0.800061in}}%
\pgfpathlineto{\pgfqpoint{5.160420in}{0.799688in}}%
\pgfpathlineto{\pgfqpoint{5.163243in}{0.795836in}}%
\pgfpathlineto{\pgfqpoint{5.165775in}{0.795359in}}%
\pgfpathlineto{\pgfqpoint{5.168591in}{0.791964in}}%
\pgfpathlineto{\pgfqpoint{5.171133in}{0.794512in}}%
\pgfpathlineto{\pgfqpoint{5.173925in}{0.794593in}}%
\pgfpathlineto{\pgfqpoint{5.176477in}{0.797733in}}%
\pgfpathlineto{\pgfqpoint{5.179188in}{0.793314in}}%
\pgfpathlineto{\pgfqpoint{5.181848in}{0.791583in}}%
\pgfpathlineto{\pgfqpoint{5.184522in}{0.795141in}}%
\pgfpathlineto{\pgfqpoint{5.187294in}{0.795099in}}%
\pgfpathlineto{\pgfqpoint{5.189880in}{0.799349in}}%
\pgfpathlineto{\pgfqpoint{5.192680in}{0.795138in}}%
\pgfpathlineto{\pgfqpoint{5.195239in}{0.791266in}}%
\pgfpathlineto{\pgfqpoint{5.198008in}{0.790509in}}%
\pgfpathlineto{\pgfqpoint{5.200594in}{0.798972in}}%
\pgfpathlineto{\pgfqpoint{5.203388in}{0.792443in}}%
\pgfpathlineto{\pgfqpoint{5.205952in}{0.787915in}}%
\pgfpathlineto{\pgfqpoint{5.208630in}{0.785588in}}%
\pgfpathlineto{\pgfqpoint{5.211299in}{0.789353in}}%
\pgfpathlineto{\pgfqpoint{5.214027in}{0.787979in}}%
\pgfpathlineto{\pgfqpoint{5.216667in}{0.792442in}}%
\pgfpathlineto{\pgfqpoint{5.219345in}{0.796740in}}%
\pgfpathlineto{\pgfqpoint{5.222151in}{0.799171in}}%
\pgfpathlineto{\pgfqpoint{5.224695in}{0.796480in}}%
\pgfpathlineto{\pgfqpoint{5.227470in}{0.801743in}}%
\pgfpathlineto{\pgfqpoint{5.230059in}{0.799349in}}%
\pgfpathlineto{\pgfqpoint{5.232855in}{0.800486in}}%
\pgfpathlineto{\pgfqpoint{5.235409in}{0.799415in}}%
\pgfpathlineto{\pgfqpoint{5.238173in}{0.795423in}}%
\pgfpathlineto{\pgfqpoint{5.240777in}{0.795097in}}%
\pgfpathlineto{\pgfqpoint{5.243445in}{0.799958in}}%
\pgfpathlineto{\pgfqpoint{5.246130in}{0.797058in}}%
\pgfpathlineto{\pgfqpoint{5.248816in}{0.795439in}}%
\pgfpathlineto{\pgfqpoint{5.251590in}{0.794385in}}%
\pgfpathlineto{\pgfqpoint{5.254236in}{0.793108in}}%
\pgfpathlineto{\pgfqpoint{5.256973in}{0.798465in}}%
\pgfpathlineto{\pgfqpoint{5.259511in}{0.797163in}}%
\pgfpathlineto{\pgfqpoint{5.262264in}{0.794910in}}%
\pgfpathlineto{\pgfqpoint{5.264876in}{0.793475in}}%
\pgfpathlineto{\pgfqpoint{5.267691in}{0.794525in}}%
\pgfpathlineto{\pgfqpoint{5.270238in}{0.800804in}}%
\pgfpathlineto{\pgfqpoint{5.272913in}{0.805186in}}%
\pgfpathlineto{\pgfqpoint{5.275589in}{0.812053in}}%
\pgfpathlineto{\pgfqpoint{5.278322in}{0.801483in}}%
\pgfpathlineto{\pgfqpoint{5.280947in}{0.795226in}}%
\pgfpathlineto{\pgfqpoint{5.283631in}{0.793320in}}%
\pgfpathlineto{\pgfqpoint{5.286436in}{0.796650in}}%
\pgfpathlineto{\pgfqpoint{5.288984in}{0.795668in}}%
\pgfpathlineto{\pgfqpoint{5.291794in}{0.794036in}}%
\pgfpathlineto{\pgfqpoint{5.294339in}{0.793170in}}%
\pgfpathlineto{\pgfqpoint{5.297140in}{0.793927in}}%
\pgfpathlineto{\pgfqpoint{5.299696in}{0.790096in}}%
\pgfpathlineto{\pgfqpoint{5.302443in}{0.789018in}}%
\pgfpathlineto{\pgfqpoint{5.305054in}{0.790930in}}%
\pgfpathlineto{\pgfqpoint{5.307731in}{0.790821in}}%
\pgfpathlineto{\pgfqpoint{5.310411in}{0.794365in}}%
\pgfpathlineto{\pgfqpoint{5.313089in}{0.790237in}}%
\pgfpathlineto{\pgfqpoint{5.315754in}{0.787686in}}%
\pgfpathlineto{\pgfqpoint{5.318430in}{0.796863in}}%
\pgfpathlineto{\pgfqpoint{5.321256in}{0.803342in}}%
\pgfpathlineto{\pgfqpoint{5.323802in}{0.793477in}}%
\pgfpathlineto{\pgfqpoint{5.326564in}{0.786699in}}%
\pgfpathlineto{\pgfqpoint{5.329159in}{0.785382in}}%
\pgfpathlineto{\pgfqpoint{5.331973in}{0.784386in}}%
\pgfpathlineto{\pgfqpoint{5.334510in}{0.781621in}}%
\pgfpathlineto{\pgfqpoint{5.337353in}{0.790579in}}%
\pgfpathlineto{\pgfqpoint{5.339872in}{0.787555in}}%
\pgfpathlineto{\pgfqpoint{5.342549in}{0.783973in}}%
\pgfpathlineto{\pgfqpoint{5.345224in}{0.775132in}}%
\pgfpathlineto{\pgfqpoint{5.347905in}{0.774425in}}%
\pgfpathlineto{\pgfqpoint{5.350723in}{0.778632in}}%
\pgfpathlineto{\pgfqpoint{5.353262in}{0.789757in}}%
\pgfpathlineto{\pgfqpoint{5.356056in}{0.786045in}}%
\pgfpathlineto{\pgfqpoint{5.358612in}{0.789850in}}%
\pgfpathlineto{\pgfqpoint{5.361370in}{0.788926in}}%
\pgfpathlineto{\pgfqpoint{5.363966in}{0.787610in}}%
\pgfpathlineto{\pgfqpoint{5.366727in}{0.786014in}}%
\pgfpathlineto{\pgfqpoint{5.369335in}{0.786738in}}%
\pgfpathlineto{\pgfqpoint{5.372013in}{0.791286in}}%
\pgfpathlineto{\pgfqpoint{5.374692in}{0.791726in}}%
\pgfpathlineto{\pgfqpoint{5.377370in}{0.790157in}}%
\pgfpathlineto{\pgfqpoint{5.380048in}{0.791575in}}%
\pgfpathlineto{\pgfqpoint{5.382725in}{0.795604in}}%
\pgfpathlineto{\pgfqpoint{5.385550in}{0.796673in}}%
\pgfpathlineto{\pgfqpoint{5.388083in}{0.794194in}}%
\pgfpathlineto{\pgfqpoint{5.390900in}{0.786878in}}%
\pgfpathlineto{\pgfqpoint{5.393441in}{0.790729in}}%
\pgfpathlineto{\pgfqpoint{5.396219in}{0.792244in}}%
\pgfpathlineto{\pgfqpoint{5.398784in}{0.781563in}}%
\pgfpathlineto{\pgfqpoint{5.401576in}{0.782136in}}%
\pgfpathlineto{\pgfqpoint{5.404154in}{0.789116in}}%
\pgfpathlineto{\pgfqpoint{5.406832in}{0.791504in}}%
\pgfpathlineto{\pgfqpoint{5.409507in}{0.791016in}}%
\pgfpathlineto{\pgfqpoint{5.412190in}{0.794800in}}%
\pgfpathlineto{\pgfqpoint{5.414954in}{0.789499in}}%
\pgfpathlineto{\pgfqpoint{5.417547in}{0.793966in}}%
\pgfpathlineto{\pgfqpoint{5.420304in}{0.794786in}}%
\pgfpathlineto{\pgfqpoint{5.422897in}{0.793530in}}%
\pgfpathlineto{\pgfqpoint{5.425661in}{0.794886in}}%
\pgfpathlineto{\pgfqpoint{5.428259in}{0.797425in}}%
\pgfpathlineto{\pgfqpoint{5.431015in}{0.794136in}}%
\pgfpathlineto{\pgfqpoint{5.433616in}{0.794788in}}%
\pgfpathlineto{\pgfqpoint{5.436295in}{0.798892in}}%
\pgfpathlineto{\pgfqpoint{5.438974in}{0.791624in}}%
\pgfpathlineto{\pgfqpoint{5.441698in}{0.796591in}}%
\pgfpathlineto{\pgfqpoint{5.444328in}{0.808045in}}%
\pgfpathlineto{\pgfqpoint{5.447021in}{0.802430in}}%
\pgfpathlineto{\pgfqpoint{5.449769in}{0.791644in}}%
\pgfpathlineto{\pgfqpoint{5.452365in}{0.792445in}}%
\pgfpathlineto{\pgfqpoint{5.455168in}{0.794745in}}%
\pgfpathlineto{\pgfqpoint{5.457721in}{0.790887in}}%
\pgfpathlineto{\pgfqpoint{5.460489in}{0.786283in}}%
\pgfpathlineto{\pgfqpoint{5.463079in}{0.794170in}}%
\pgfpathlineto{\pgfqpoint{5.465888in}{0.795514in}}%
\pgfpathlineto{\pgfqpoint{5.468425in}{0.798274in}}%
\pgfpathlineto{\pgfqpoint{5.471113in}{0.799901in}}%
\pgfpathlineto{\pgfqpoint{5.473792in}{0.800855in}}%
\pgfpathlineto{\pgfqpoint{5.476458in}{0.797606in}}%
\pgfpathlineto{\pgfqpoint{5.479152in}{0.799169in}}%
\pgfpathlineto{\pgfqpoint{5.481825in}{0.801175in}}%
\pgfpathlineto{\pgfqpoint{5.484641in}{0.799420in}}%
\pgfpathlineto{\pgfqpoint{5.487176in}{0.800987in}}%
\pgfpathlineto{\pgfqpoint{5.490000in}{0.800819in}}%
\pgfpathlineto{\pgfqpoint{5.492541in}{0.795137in}}%
\pgfpathlineto{\pgfqpoint{5.495346in}{0.795490in}}%
\pgfpathlineto{\pgfqpoint{5.497898in}{0.801603in}}%
\pgfpathlineto{\pgfqpoint{5.500687in}{0.796250in}}%
\pgfpathlineto{\pgfqpoint{5.503255in}{0.799836in}}%
\pgfpathlineto{\pgfqpoint{5.505933in}{0.804276in}}%
\pgfpathlineto{\pgfqpoint{5.508612in}{0.812957in}}%
\pgfpathlineto{\pgfqpoint{5.511290in}{0.809364in}}%
\pgfpathlineto{\pgfqpoint{5.514080in}{0.823236in}}%
\pgfpathlineto{\pgfqpoint{5.516646in}{0.819682in}}%
\pgfpathlineto{\pgfqpoint{5.519433in}{0.809626in}}%
\pgfpathlineto{\pgfqpoint{5.522003in}{0.797203in}}%
\pgfpathlineto{\pgfqpoint{5.524756in}{0.809714in}}%
\pgfpathlineto{\pgfqpoint{5.527360in}{0.800945in}}%
\pgfpathlineto{\pgfqpoint{5.530148in}{0.796804in}}%
\pgfpathlineto{\pgfqpoint{5.532717in}{0.795563in}}%
\pgfpathlineto{\pgfqpoint{5.535395in}{0.804310in}}%
\pgfpathlineto{\pgfqpoint{5.538074in}{0.797901in}}%
\pgfpathlineto{\pgfqpoint{5.540750in}{0.798839in}}%
\pgfpathlineto{\pgfqpoint{5.543421in}{0.798555in}}%
\pgfpathlineto{\pgfqpoint{5.546110in}{0.800398in}}%
\pgfpathlineto{\pgfqpoint{5.548921in}{0.798272in}}%
\pgfpathlineto{\pgfqpoint{5.551457in}{0.798313in}}%
\pgfpathlineto{\pgfqpoint{5.554198in}{0.805029in}}%
\pgfpathlineto{\pgfqpoint{5.556822in}{0.802698in}}%
\pgfpathlineto{\pgfqpoint{5.559612in}{0.809281in}}%
\pgfpathlineto{\pgfqpoint{5.562180in}{0.815045in}}%
\pgfpathlineto{\pgfqpoint{5.564940in}{0.811690in}}%
\pgfpathlineto{\pgfqpoint{5.567536in}{0.820434in}}%
\pgfpathlineto{\pgfqpoint{5.570215in}{0.818534in}}%
\pgfpathlineto{\pgfqpoint{5.572893in}{0.801662in}}%
\pgfpathlineto{\pgfqpoint{5.575596in}{0.801837in}}%
\pgfpathlineto{\pgfqpoint{5.578342in}{0.797936in}}%
\pgfpathlineto{\pgfqpoint{5.580914in}{0.808520in}}%
\pgfpathlineto{\pgfqpoint{5.583709in}{0.814775in}}%
\pgfpathlineto{\pgfqpoint{5.586269in}{0.799938in}}%
\pgfpathlineto{\pgfqpoint{5.589040in}{0.796146in}}%
\pgfpathlineto{\pgfqpoint{5.591641in}{0.800691in}}%
\pgfpathlineto{\pgfqpoint{5.594368in}{0.803210in}}%
\pgfpathlineto{\pgfqpoint{5.596999in}{0.796118in}}%
\pgfpathlineto{\pgfqpoint{5.599674in}{0.795942in}}%
\pgfpathlineto{\pgfqpoint{5.602352in}{0.796287in}}%
\pgfpathlineto{\pgfqpoint{5.605073in}{0.792575in}}%
\pgfpathlineto{\pgfqpoint{5.607698in}{0.791825in}}%
\pgfpathlineto{\pgfqpoint{5.610389in}{0.786248in}}%
\pgfpathlineto{\pgfqpoint{5.613235in}{0.785809in}}%
\pgfpathlineto{\pgfqpoint{5.615743in}{0.786463in}}%
\pgfpathlineto{\pgfqpoint{5.618526in}{0.790151in}}%
\pgfpathlineto{\pgfqpoint{5.621102in}{0.792271in}}%
\pgfpathlineto{\pgfqpoint{5.623868in}{0.791559in}}%
\pgfpathlineto{\pgfqpoint{5.626460in}{0.793703in}}%
\pgfpathlineto{\pgfqpoint{5.629232in}{0.793352in}}%
\pgfpathlineto{\pgfqpoint{5.631815in}{0.788599in}}%
\pgfpathlineto{\pgfqpoint{5.634496in}{0.789874in}}%
\pgfpathlineto{\pgfqpoint{5.637172in}{0.793717in}}%
\pgfpathlineto{\pgfqpoint{5.639852in}{0.800535in}}%
\pgfpathlineto{\pgfqpoint{5.642518in}{0.804616in}}%
\pgfpathlineto{\pgfqpoint{5.645243in}{0.807479in}}%
\pgfpathlineto{\pgfqpoint{5.648008in}{0.805496in}}%
\pgfpathlineto{\pgfqpoint{5.650563in}{0.808703in}}%
\pgfpathlineto{\pgfqpoint{5.653376in}{0.799728in}}%
\pgfpathlineto{\pgfqpoint{5.655919in}{0.796696in}}%
\pgfpathlineto{\pgfqpoint{5.658723in}{0.792089in}}%
\pgfpathlineto{\pgfqpoint{5.661273in}{0.778688in}}%
\pgfpathlineto{\pgfqpoint{5.664099in}{0.779599in}}%
\pgfpathlineto{\pgfqpoint{5.666632in}{0.781809in}}%
\pgfpathlineto{\pgfqpoint{5.669313in}{0.787558in}}%
\pgfpathlineto{\pgfqpoint{5.671991in}{0.785049in}}%
\pgfpathlineto{\pgfqpoint{5.674667in}{0.783997in}}%
\pgfpathlineto{\pgfqpoint{5.677486in}{0.787199in}}%
\pgfpathlineto{\pgfqpoint{5.680027in}{0.791561in}}%
\pgfpathlineto{\pgfqpoint{5.682836in}{0.790277in}}%
\pgfpathlineto{\pgfqpoint{5.685385in}{0.789082in}}%
\pgfpathlineto{\pgfqpoint{5.688159in}{0.793984in}}%
\pgfpathlineto{\pgfqpoint{5.690730in}{0.796101in}}%
\pgfpathlineto{\pgfqpoint{5.693473in}{0.790874in}}%
\pgfpathlineto{\pgfqpoint{5.696101in}{0.791672in}}%
\pgfpathlineto{\pgfqpoint{5.698775in}{0.788553in}}%
\pgfpathlineto{\pgfqpoint{5.701453in}{0.796612in}}%
\pgfpathlineto{\pgfqpoint{5.704130in}{0.783490in}}%
\pgfpathlineto{\pgfqpoint{5.706800in}{0.785280in}}%
\pgfpathlineto{\pgfqpoint{5.709490in}{0.787312in}}%
\pgfpathlineto{\pgfqpoint{5.712291in}{0.783084in}}%
\pgfpathlineto{\pgfqpoint{5.714834in}{0.790011in}}%
\pgfpathlineto{\pgfqpoint{5.717671in}{0.788039in}}%
\pgfpathlineto{\pgfqpoint{5.720201in}{0.795331in}}%
\pgfpathlineto{\pgfqpoint{5.722950in}{0.791678in}}%
\pgfpathlineto{\pgfqpoint{5.725548in}{0.794971in}}%
\pgfpathlineto{\pgfqpoint{5.728339in}{0.800366in}}%
\pgfpathlineto{\pgfqpoint{5.730919in}{0.801955in}}%
\pgfpathlineto{\pgfqpoint{5.733594in}{0.803196in}}%
\pgfpathlineto{\pgfqpoint{5.736276in}{0.797362in}}%
\pgfpathlineto{\pgfqpoint{5.738974in}{0.797611in}}%
\pgfpathlineto{\pgfqpoint{5.741745in}{0.799435in}}%
\pgfpathlineto{\pgfqpoint{5.744310in}{0.799310in}}%
\pgfpathlineto{\pgfqpoint{5.744310in}{0.413320in}}%
\pgfpathlineto{\pgfqpoint{5.744310in}{0.413320in}}%
\pgfpathlineto{\pgfqpoint{5.741745in}{0.413320in}}%
\pgfpathlineto{\pgfqpoint{5.738974in}{0.413320in}}%
\pgfpathlineto{\pgfqpoint{5.736276in}{0.413320in}}%
\pgfpathlineto{\pgfqpoint{5.733594in}{0.413320in}}%
\pgfpathlineto{\pgfqpoint{5.730919in}{0.413320in}}%
\pgfpathlineto{\pgfqpoint{5.728339in}{0.413320in}}%
\pgfpathlineto{\pgfqpoint{5.725548in}{0.413320in}}%
\pgfpathlineto{\pgfqpoint{5.722950in}{0.413320in}}%
\pgfpathlineto{\pgfqpoint{5.720201in}{0.413320in}}%
\pgfpathlineto{\pgfqpoint{5.717671in}{0.413320in}}%
\pgfpathlineto{\pgfqpoint{5.714834in}{0.413320in}}%
\pgfpathlineto{\pgfqpoint{5.712291in}{0.413320in}}%
\pgfpathlineto{\pgfqpoint{5.709490in}{0.413320in}}%
\pgfpathlineto{\pgfqpoint{5.706800in}{0.413320in}}%
\pgfpathlineto{\pgfqpoint{5.704130in}{0.413320in}}%
\pgfpathlineto{\pgfqpoint{5.701453in}{0.413320in}}%
\pgfpathlineto{\pgfqpoint{5.698775in}{0.413320in}}%
\pgfpathlineto{\pgfqpoint{5.696101in}{0.413320in}}%
\pgfpathlineto{\pgfqpoint{5.693473in}{0.413320in}}%
\pgfpathlineto{\pgfqpoint{5.690730in}{0.413320in}}%
\pgfpathlineto{\pgfqpoint{5.688159in}{0.413320in}}%
\pgfpathlineto{\pgfqpoint{5.685385in}{0.413320in}}%
\pgfpathlineto{\pgfqpoint{5.682836in}{0.413320in}}%
\pgfpathlineto{\pgfqpoint{5.680027in}{0.413320in}}%
\pgfpathlineto{\pgfqpoint{5.677486in}{0.413320in}}%
\pgfpathlineto{\pgfqpoint{5.674667in}{0.413320in}}%
\pgfpathlineto{\pgfqpoint{5.671991in}{0.413320in}}%
\pgfpathlineto{\pgfqpoint{5.669313in}{0.413320in}}%
\pgfpathlineto{\pgfqpoint{5.666632in}{0.413320in}}%
\pgfpathlineto{\pgfqpoint{5.664099in}{0.413320in}}%
\pgfpathlineto{\pgfqpoint{5.661273in}{0.413320in}}%
\pgfpathlineto{\pgfqpoint{5.658723in}{0.413320in}}%
\pgfpathlineto{\pgfqpoint{5.655919in}{0.413320in}}%
\pgfpathlineto{\pgfqpoint{5.653376in}{0.413320in}}%
\pgfpathlineto{\pgfqpoint{5.650563in}{0.413320in}}%
\pgfpathlineto{\pgfqpoint{5.648008in}{0.413320in}}%
\pgfpathlineto{\pgfqpoint{5.645243in}{0.413320in}}%
\pgfpathlineto{\pgfqpoint{5.642518in}{0.413320in}}%
\pgfpathlineto{\pgfqpoint{5.639852in}{0.413320in}}%
\pgfpathlineto{\pgfqpoint{5.637172in}{0.413320in}}%
\pgfpathlineto{\pgfqpoint{5.634496in}{0.413320in}}%
\pgfpathlineto{\pgfqpoint{5.631815in}{0.413320in}}%
\pgfpathlineto{\pgfqpoint{5.629232in}{0.413320in}}%
\pgfpathlineto{\pgfqpoint{5.626460in}{0.413320in}}%
\pgfpathlineto{\pgfqpoint{5.623868in}{0.413320in}}%
\pgfpathlineto{\pgfqpoint{5.621102in}{0.413320in}}%
\pgfpathlineto{\pgfqpoint{5.618526in}{0.413320in}}%
\pgfpathlineto{\pgfqpoint{5.615743in}{0.413320in}}%
\pgfpathlineto{\pgfqpoint{5.613235in}{0.413320in}}%
\pgfpathlineto{\pgfqpoint{5.610389in}{0.413320in}}%
\pgfpathlineto{\pgfqpoint{5.607698in}{0.413320in}}%
\pgfpathlineto{\pgfqpoint{5.605073in}{0.413320in}}%
\pgfpathlineto{\pgfqpoint{5.602352in}{0.413320in}}%
\pgfpathlineto{\pgfqpoint{5.599674in}{0.413320in}}%
\pgfpathlineto{\pgfqpoint{5.596999in}{0.413320in}}%
\pgfpathlineto{\pgfqpoint{5.594368in}{0.413320in}}%
\pgfpathlineto{\pgfqpoint{5.591641in}{0.413320in}}%
\pgfpathlineto{\pgfqpoint{5.589040in}{0.413320in}}%
\pgfpathlineto{\pgfqpoint{5.586269in}{0.413320in}}%
\pgfpathlineto{\pgfqpoint{5.583709in}{0.413320in}}%
\pgfpathlineto{\pgfqpoint{5.580914in}{0.413320in}}%
\pgfpathlineto{\pgfqpoint{5.578342in}{0.413320in}}%
\pgfpathlineto{\pgfqpoint{5.575596in}{0.413320in}}%
\pgfpathlineto{\pgfqpoint{5.572893in}{0.413320in}}%
\pgfpathlineto{\pgfqpoint{5.570215in}{0.413320in}}%
\pgfpathlineto{\pgfqpoint{5.567536in}{0.413320in}}%
\pgfpathlineto{\pgfqpoint{5.564940in}{0.413320in}}%
\pgfpathlineto{\pgfqpoint{5.562180in}{0.413320in}}%
\pgfpathlineto{\pgfqpoint{5.559612in}{0.413320in}}%
\pgfpathlineto{\pgfqpoint{5.556822in}{0.413320in}}%
\pgfpathlineto{\pgfqpoint{5.554198in}{0.413320in}}%
\pgfpathlineto{\pgfqpoint{5.551457in}{0.413320in}}%
\pgfpathlineto{\pgfqpoint{5.548921in}{0.413320in}}%
\pgfpathlineto{\pgfqpoint{5.546110in}{0.413320in}}%
\pgfpathlineto{\pgfqpoint{5.543421in}{0.413320in}}%
\pgfpathlineto{\pgfqpoint{5.540750in}{0.413320in}}%
\pgfpathlineto{\pgfqpoint{5.538074in}{0.413320in}}%
\pgfpathlineto{\pgfqpoint{5.535395in}{0.413320in}}%
\pgfpathlineto{\pgfqpoint{5.532717in}{0.413320in}}%
\pgfpathlineto{\pgfqpoint{5.530148in}{0.413320in}}%
\pgfpathlineto{\pgfqpoint{5.527360in}{0.413320in}}%
\pgfpathlineto{\pgfqpoint{5.524756in}{0.413320in}}%
\pgfpathlineto{\pgfqpoint{5.522003in}{0.413320in}}%
\pgfpathlineto{\pgfqpoint{5.519433in}{0.413320in}}%
\pgfpathlineto{\pgfqpoint{5.516646in}{0.413320in}}%
\pgfpathlineto{\pgfqpoint{5.514080in}{0.413320in}}%
\pgfpathlineto{\pgfqpoint{5.511290in}{0.413320in}}%
\pgfpathlineto{\pgfqpoint{5.508612in}{0.413320in}}%
\pgfpathlineto{\pgfqpoint{5.505933in}{0.413320in}}%
\pgfpathlineto{\pgfqpoint{5.503255in}{0.413320in}}%
\pgfpathlineto{\pgfqpoint{5.500687in}{0.413320in}}%
\pgfpathlineto{\pgfqpoint{5.497898in}{0.413320in}}%
\pgfpathlineto{\pgfqpoint{5.495346in}{0.413320in}}%
\pgfpathlineto{\pgfqpoint{5.492541in}{0.413320in}}%
\pgfpathlineto{\pgfqpoint{5.490000in}{0.413320in}}%
\pgfpathlineto{\pgfqpoint{5.487176in}{0.413320in}}%
\pgfpathlineto{\pgfqpoint{5.484641in}{0.413320in}}%
\pgfpathlineto{\pgfqpoint{5.481825in}{0.413320in}}%
\pgfpathlineto{\pgfqpoint{5.479152in}{0.413320in}}%
\pgfpathlineto{\pgfqpoint{5.476458in}{0.413320in}}%
\pgfpathlineto{\pgfqpoint{5.473792in}{0.413320in}}%
\pgfpathlineto{\pgfqpoint{5.471113in}{0.413320in}}%
\pgfpathlineto{\pgfqpoint{5.468425in}{0.413320in}}%
\pgfpathlineto{\pgfqpoint{5.465888in}{0.413320in}}%
\pgfpathlineto{\pgfqpoint{5.463079in}{0.413320in}}%
\pgfpathlineto{\pgfqpoint{5.460489in}{0.413320in}}%
\pgfpathlineto{\pgfqpoint{5.457721in}{0.413320in}}%
\pgfpathlineto{\pgfqpoint{5.455168in}{0.413320in}}%
\pgfpathlineto{\pgfqpoint{5.452365in}{0.413320in}}%
\pgfpathlineto{\pgfqpoint{5.449769in}{0.413320in}}%
\pgfpathlineto{\pgfqpoint{5.447021in}{0.413320in}}%
\pgfpathlineto{\pgfqpoint{5.444328in}{0.413320in}}%
\pgfpathlineto{\pgfqpoint{5.441698in}{0.413320in}}%
\pgfpathlineto{\pgfqpoint{5.438974in}{0.413320in}}%
\pgfpathlineto{\pgfqpoint{5.436295in}{0.413320in}}%
\pgfpathlineto{\pgfqpoint{5.433616in}{0.413320in}}%
\pgfpathlineto{\pgfqpoint{5.431015in}{0.413320in}}%
\pgfpathlineto{\pgfqpoint{5.428259in}{0.413320in}}%
\pgfpathlineto{\pgfqpoint{5.425661in}{0.413320in}}%
\pgfpathlineto{\pgfqpoint{5.422897in}{0.413320in}}%
\pgfpathlineto{\pgfqpoint{5.420304in}{0.413320in}}%
\pgfpathlineto{\pgfqpoint{5.417547in}{0.413320in}}%
\pgfpathlineto{\pgfqpoint{5.414954in}{0.413320in}}%
\pgfpathlineto{\pgfqpoint{5.412190in}{0.413320in}}%
\pgfpathlineto{\pgfqpoint{5.409507in}{0.413320in}}%
\pgfpathlineto{\pgfqpoint{5.406832in}{0.413320in}}%
\pgfpathlineto{\pgfqpoint{5.404154in}{0.413320in}}%
\pgfpathlineto{\pgfqpoint{5.401576in}{0.413320in}}%
\pgfpathlineto{\pgfqpoint{5.398784in}{0.413320in}}%
\pgfpathlineto{\pgfqpoint{5.396219in}{0.413320in}}%
\pgfpathlineto{\pgfqpoint{5.393441in}{0.413320in}}%
\pgfpathlineto{\pgfqpoint{5.390900in}{0.413320in}}%
\pgfpathlineto{\pgfqpoint{5.388083in}{0.413320in}}%
\pgfpathlineto{\pgfqpoint{5.385550in}{0.413320in}}%
\pgfpathlineto{\pgfqpoint{5.382725in}{0.413320in}}%
\pgfpathlineto{\pgfqpoint{5.380048in}{0.413320in}}%
\pgfpathlineto{\pgfqpoint{5.377370in}{0.413320in}}%
\pgfpathlineto{\pgfqpoint{5.374692in}{0.413320in}}%
\pgfpathlineto{\pgfqpoint{5.372013in}{0.413320in}}%
\pgfpathlineto{\pgfqpoint{5.369335in}{0.413320in}}%
\pgfpathlineto{\pgfqpoint{5.366727in}{0.413320in}}%
\pgfpathlineto{\pgfqpoint{5.363966in}{0.413320in}}%
\pgfpathlineto{\pgfqpoint{5.361370in}{0.413320in}}%
\pgfpathlineto{\pgfqpoint{5.358612in}{0.413320in}}%
\pgfpathlineto{\pgfqpoint{5.356056in}{0.413320in}}%
\pgfpathlineto{\pgfqpoint{5.353262in}{0.413320in}}%
\pgfpathlineto{\pgfqpoint{5.350723in}{0.413320in}}%
\pgfpathlineto{\pgfqpoint{5.347905in}{0.413320in}}%
\pgfpathlineto{\pgfqpoint{5.345224in}{0.413320in}}%
\pgfpathlineto{\pgfqpoint{5.342549in}{0.413320in}}%
\pgfpathlineto{\pgfqpoint{5.339872in}{0.413320in}}%
\pgfpathlineto{\pgfqpoint{5.337353in}{0.413320in}}%
\pgfpathlineto{\pgfqpoint{5.334510in}{0.413320in}}%
\pgfpathlineto{\pgfqpoint{5.331973in}{0.413320in}}%
\pgfpathlineto{\pgfqpoint{5.329159in}{0.413320in}}%
\pgfpathlineto{\pgfqpoint{5.326564in}{0.413320in}}%
\pgfpathlineto{\pgfqpoint{5.323802in}{0.413320in}}%
\pgfpathlineto{\pgfqpoint{5.321256in}{0.413320in}}%
\pgfpathlineto{\pgfqpoint{5.318430in}{0.413320in}}%
\pgfpathlineto{\pgfqpoint{5.315754in}{0.413320in}}%
\pgfpathlineto{\pgfqpoint{5.313089in}{0.413320in}}%
\pgfpathlineto{\pgfqpoint{5.310411in}{0.413320in}}%
\pgfpathlineto{\pgfqpoint{5.307731in}{0.413320in}}%
\pgfpathlineto{\pgfqpoint{5.305054in}{0.413320in}}%
\pgfpathlineto{\pgfqpoint{5.302443in}{0.413320in}}%
\pgfpathlineto{\pgfqpoint{5.299696in}{0.413320in}}%
\pgfpathlineto{\pgfqpoint{5.297140in}{0.413320in}}%
\pgfpathlineto{\pgfqpoint{5.294339in}{0.413320in}}%
\pgfpathlineto{\pgfqpoint{5.291794in}{0.413320in}}%
\pgfpathlineto{\pgfqpoint{5.288984in}{0.413320in}}%
\pgfpathlineto{\pgfqpoint{5.286436in}{0.413320in}}%
\pgfpathlineto{\pgfqpoint{5.283631in}{0.413320in}}%
\pgfpathlineto{\pgfqpoint{5.280947in}{0.413320in}}%
\pgfpathlineto{\pgfqpoint{5.278322in}{0.413320in}}%
\pgfpathlineto{\pgfqpoint{5.275589in}{0.413320in}}%
\pgfpathlineto{\pgfqpoint{5.272913in}{0.413320in}}%
\pgfpathlineto{\pgfqpoint{5.270238in}{0.413320in}}%
\pgfpathlineto{\pgfqpoint{5.267691in}{0.413320in}}%
\pgfpathlineto{\pgfqpoint{5.264876in}{0.413320in}}%
\pgfpathlineto{\pgfqpoint{5.262264in}{0.413320in}}%
\pgfpathlineto{\pgfqpoint{5.259511in}{0.413320in}}%
\pgfpathlineto{\pgfqpoint{5.256973in}{0.413320in}}%
\pgfpathlineto{\pgfqpoint{5.254236in}{0.413320in}}%
\pgfpathlineto{\pgfqpoint{5.251590in}{0.413320in}}%
\pgfpathlineto{\pgfqpoint{5.248816in}{0.413320in}}%
\pgfpathlineto{\pgfqpoint{5.246130in}{0.413320in}}%
\pgfpathlineto{\pgfqpoint{5.243445in}{0.413320in}}%
\pgfpathlineto{\pgfqpoint{5.240777in}{0.413320in}}%
\pgfpathlineto{\pgfqpoint{5.238173in}{0.413320in}}%
\pgfpathlineto{\pgfqpoint{5.235409in}{0.413320in}}%
\pgfpathlineto{\pgfqpoint{5.232855in}{0.413320in}}%
\pgfpathlineto{\pgfqpoint{5.230059in}{0.413320in}}%
\pgfpathlineto{\pgfqpoint{5.227470in}{0.413320in}}%
\pgfpathlineto{\pgfqpoint{5.224695in}{0.413320in}}%
\pgfpathlineto{\pgfqpoint{5.222151in}{0.413320in}}%
\pgfpathlineto{\pgfqpoint{5.219345in}{0.413320in}}%
\pgfpathlineto{\pgfqpoint{5.216667in}{0.413320in}}%
\pgfpathlineto{\pgfqpoint{5.214027in}{0.413320in}}%
\pgfpathlineto{\pgfqpoint{5.211299in}{0.413320in}}%
\pgfpathlineto{\pgfqpoint{5.208630in}{0.413320in}}%
\pgfpathlineto{\pgfqpoint{5.205952in}{0.413320in}}%
\pgfpathlineto{\pgfqpoint{5.203388in}{0.413320in}}%
\pgfpathlineto{\pgfqpoint{5.200594in}{0.413320in}}%
\pgfpathlineto{\pgfqpoint{5.198008in}{0.413320in}}%
\pgfpathlineto{\pgfqpoint{5.195239in}{0.413320in}}%
\pgfpathlineto{\pgfqpoint{5.192680in}{0.413320in}}%
\pgfpathlineto{\pgfqpoint{5.189880in}{0.413320in}}%
\pgfpathlineto{\pgfqpoint{5.187294in}{0.413320in}}%
\pgfpathlineto{\pgfqpoint{5.184522in}{0.413320in}}%
\pgfpathlineto{\pgfqpoint{5.181848in}{0.413320in}}%
\pgfpathlineto{\pgfqpoint{5.179188in}{0.413320in}}%
\pgfpathlineto{\pgfqpoint{5.176477in}{0.413320in}}%
\pgfpathlineto{\pgfqpoint{5.173925in}{0.413320in}}%
\pgfpathlineto{\pgfqpoint{5.171133in}{0.413320in}}%
\pgfpathlineto{\pgfqpoint{5.168591in}{0.413320in}}%
\pgfpathlineto{\pgfqpoint{5.165775in}{0.413320in}}%
\pgfpathlineto{\pgfqpoint{5.163243in}{0.413320in}}%
\pgfpathlineto{\pgfqpoint{5.160420in}{0.413320in}}%
\pgfpathlineto{\pgfqpoint{5.157815in}{0.413320in}}%
\pgfpathlineto{\pgfqpoint{5.155059in}{0.413320in}}%
\pgfpathlineto{\pgfqpoint{5.152382in}{0.413320in}}%
\pgfpathlineto{\pgfqpoint{5.149734in}{0.413320in}}%
\pgfpathlineto{\pgfqpoint{5.147029in}{0.413320in}}%
\pgfpathlineto{\pgfqpoint{5.144349in}{0.413320in}}%
\pgfpathlineto{\pgfqpoint{5.141660in}{0.413320in}}%
\pgfpathlineto{\pgfqpoint{5.139072in}{0.413320in}}%
\pgfpathlineto{\pgfqpoint{5.136311in}{0.413320in}}%
\pgfpathlineto{\pgfqpoint{5.133716in}{0.413320in}}%
\pgfpathlineto{\pgfqpoint{5.130953in}{0.413320in}}%
\pgfpathlineto{\pgfqpoint{5.128421in}{0.413320in}}%
\pgfpathlineto{\pgfqpoint{5.125599in}{0.413320in}}%
\pgfpathlineto{\pgfqpoint{5.123042in}{0.413320in}}%
\pgfpathlineto{\pgfqpoint{5.120243in}{0.413320in}}%
\pgfpathlineto{\pgfqpoint{5.117550in}{0.413320in}}%
\pgfpathlineto{\pgfqpoint{5.114887in}{0.413320in}}%
\pgfpathlineto{\pgfqpoint{5.112209in}{0.413320in}}%
\pgfpathlineto{\pgfqpoint{5.109530in}{0.413320in}}%
\pgfpathlineto{\pgfqpoint{5.106842in}{0.413320in}}%
\pgfpathlineto{\pgfqpoint{5.104312in}{0.413320in}}%
\pgfpathlineto{\pgfqpoint{5.101496in}{0.413320in}}%
\pgfpathlineto{\pgfqpoint{5.098948in}{0.413320in}}%
\pgfpathlineto{\pgfqpoint{5.096142in}{0.413320in}}%
\pgfpathlineto{\pgfqpoint{5.093579in}{0.413320in}}%
\pgfpathlineto{\pgfqpoint{5.090788in}{0.413320in}}%
\pgfpathlineto{\pgfqpoint{5.088103in}{0.413320in}}%
\pgfpathlineto{\pgfqpoint{5.085426in}{0.413320in}}%
\pgfpathlineto{\pgfqpoint{5.082746in}{0.413320in}}%
\pgfpathlineto{\pgfqpoint{5.080067in}{0.413320in}}%
\pgfpathlineto{\pgfqpoint{5.077390in}{0.413320in}}%
\pgfpathlineto{\pgfqpoint{5.074851in}{0.413320in}}%
\pgfpathlineto{\pgfqpoint{5.072030in}{0.413320in}}%
\pgfpathlineto{\pgfqpoint{5.069463in}{0.413320in}}%
\pgfpathlineto{\pgfqpoint{5.066677in}{0.413320in}}%
\pgfpathlineto{\pgfqpoint{5.064144in}{0.413320in}}%
\pgfpathlineto{\pgfqpoint{5.061315in}{0.413320in}}%
\pgfpathlineto{\pgfqpoint{5.058711in}{0.413320in}}%
\pgfpathlineto{\pgfqpoint{5.055952in}{0.413320in}}%
\pgfpathlineto{\pgfqpoint{5.053284in}{0.413320in}}%
\pgfpathlineto{\pgfqpoint{5.050606in}{0.413320in}}%
\pgfpathlineto{\pgfqpoint{5.047924in}{0.413320in}}%
\pgfpathlineto{\pgfqpoint{5.045249in}{0.413320in}}%
\pgfpathlineto{\pgfqpoint{5.042572in}{0.413320in}}%
\pgfpathlineto{\pgfqpoint{5.039962in}{0.413320in}}%
\pgfpathlineto{\pgfqpoint{5.037214in}{0.413320in}}%
\pgfpathlineto{\pgfqpoint{5.034649in}{0.413320in}}%
\pgfpathlineto{\pgfqpoint{5.031849in}{0.413320in}}%
\pgfpathlineto{\pgfqpoint{5.029275in}{0.413320in}}%
\pgfpathlineto{\pgfqpoint{5.026501in}{0.413320in}}%
\pgfpathlineto{\pgfqpoint{5.023927in}{0.413320in}}%
\pgfpathlineto{\pgfqpoint{5.021147in}{0.413320in}}%
\pgfpathlineto{\pgfqpoint{5.018466in}{0.413320in}}%
\pgfpathlineto{\pgfqpoint{5.015820in}{0.413320in}}%
\pgfpathlineto{\pgfqpoint{5.013104in}{0.413320in}}%
\pgfpathlineto{\pgfqpoint{5.010562in}{0.413320in}}%
\pgfpathlineto{\pgfqpoint{5.007751in}{0.413320in}}%
\pgfpathlineto{\pgfqpoint{5.005178in}{0.413320in}}%
\pgfpathlineto{\pgfqpoint{5.002384in}{0.413320in}}%
\pgfpathlineto{\pgfqpoint{4.999780in}{0.413320in}}%
\pgfpathlineto{\pgfqpoint{4.997028in}{0.413320in}}%
\pgfpathlineto{\pgfqpoint{4.994390in}{0.413320in}}%
\pgfpathlineto{\pgfqpoint{4.991683in}{0.413320in}}%
\pgfpathlineto{\pgfqpoint{4.989001in}{0.413320in}}%
\pgfpathlineto{\pgfqpoint{4.986325in}{0.413320in}}%
\pgfpathlineto{\pgfqpoint{4.983637in}{0.413320in}}%
\pgfpathlineto{\pgfqpoint{4.980967in}{0.413320in}}%
\pgfpathlineto{\pgfqpoint{4.978287in}{0.413320in}}%
\pgfpathlineto{\pgfqpoint{4.975703in}{0.413320in}}%
\pgfpathlineto{\pgfqpoint{4.972933in}{0.413320in}}%
\pgfpathlineto{\pgfqpoint{4.970314in}{0.413320in}}%
\pgfpathlineto{\pgfqpoint{4.967575in}{0.413320in}}%
\pgfpathlineto{\pgfqpoint{4.965002in}{0.413320in}}%
\pgfpathlineto{\pgfqpoint{4.962219in}{0.413320in}}%
\pgfpathlineto{\pgfqpoint{4.959689in}{0.413320in}}%
\pgfpathlineto{\pgfqpoint{4.956862in}{0.413320in}}%
\pgfpathlineto{\pgfqpoint{4.954182in}{0.413320in}}%
\pgfpathlineto{\pgfqpoint{4.951504in}{0.413320in}}%
\pgfpathlineto{\pgfqpoint{4.948827in}{0.413320in}}%
\pgfpathlineto{\pgfqpoint{4.946151in}{0.413320in}}%
\pgfpathlineto{\pgfqpoint{4.943466in}{0.413320in}}%
\pgfpathlineto{\pgfqpoint{4.940881in}{0.413320in}}%
\pgfpathlineto{\pgfqpoint{4.938112in}{0.413320in}}%
\pgfpathlineto{\pgfqpoint{4.935515in}{0.413320in}}%
\pgfpathlineto{\pgfqpoint{4.932742in}{0.413320in}}%
\pgfpathlineto{\pgfqpoint{4.930170in}{0.413320in}}%
\pgfpathlineto{\pgfqpoint{4.927400in}{0.413320in}}%
\pgfpathlineto{\pgfqpoint{4.924708in}{0.413320in}}%
\pgfpathlineto{\pgfqpoint{4.922041in}{0.413320in}}%
\pgfpathlineto{\pgfqpoint{4.919352in}{0.413320in}}%
\pgfpathlineto{\pgfqpoint{4.916681in}{0.413320in}}%
\pgfpathlineto{\pgfqpoint{4.914009in}{0.413320in}}%
\pgfpathlineto{\pgfqpoint{4.911435in}{0.413320in}}%
\pgfpathlineto{\pgfqpoint{4.908648in}{0.413320in}}%
\pgfpathlineto{\pgfqpoint{4.906096in}{0.413320in}}%
\pgfpathlineto{\pgfqpoint{4.903295in}{0.413320in}}%
\pgfpathlineto{\pgfqpoint{4.900712in}{0.413320in}}%
\pgfpathlineto{\pgfqpoint{4.897938in}{0.413320in}}%
\pgfpathlineto{\pgfqpoint{4.895399in}{0.413320in}}%
\pgfpathlineto{\pgfqpoint{4.892611in}{0.413320in}}%
\pgfpathlineto{\pgfqpoint{4.889902in}{0.413320in}}%
\pgfpathlineto{\pgfqpoint{4.887211in}{0.413320in}}%
\pgfpathlineto{\pgfqpoint{4.884540in}{0.413320in}}%
\pgfpathlineto{\pgfqpoint{4.881864in}{0.413320in}}%
\pgfpathlineto{\pgfqpoint{4.879180in}{0.413320in}}%
\pgfpathlineto{\pgfqpoint{4.876636in}{0.413320in}}%
\pgfpathlineto{\pgfqpoint{4.873832in}{0.413320in}}%
\pgfpathlineto{\pgfqpoint{4.871209in}{0.413320in}}%
\pgfpathlineto{\pgfqpoint{4.868474in}{0.413320in}}%
\pgfpathlineto{\pgfqpoint{4.865910in}{0.413320in}}%
\pgfpathlineto{\pgfqpoint{4.863116in}{0.413320in}}%
\pgfpathlineto{\pgfqpoint{4.860544in}{0.413320in}}%
\pgfpathlineto{\pgfqpoint{4.857807in}{0.413320in}}%
\pgfpathlineto{\pgfqpoint{4.855070in}{0.413320in}}%
\pgfpathlineto{\pgfqpoint{4.852404in}{0.413320in}}%
\pgfpathlineto{\pgfqpoint{4.849715in}{0.413320in}}%
\pgfpathlineto{\pgfqpoint{4.847127in}{0.413320in}}%
\pgfpathlineto{\pgfqpoint{4.844361in}{0.413320in}}%
\pgfpathlineto{\pgfqpoint{4.842380in}{0.413320in}}%
\pgfpathlineto{\pgfqpoint{4.839922in}{0.413320in}}%
\pgfpathlineto{\pgfqpoint{4.837992in}{0.413320in}}%
\pgfpathlineto{\pgfqpoint{4.833657in}{0.413320in}}%
\pgfpathlineto{\pgfqpoint{4.831045in}{0.413320in}}%
\pgfpathlineto{\pgfqpoint{4.828291in}{0.413320in}}%
\pgfpathlineto{\pgfqpoint{4.825619in}{0.413320in}}%
\pgfpathlineto{\pgfqpoint{4.822945in}{0.413320in}}%
\pgfpathlineto{\pgfqpoint{4.820265in}{0.413320in}}%
\pgfpathlineto{\pgfqpoint{4.817587in}{0.413320in}}%
\pgfpathlineto{\pgfqpoint{4.814907in}{0.413320in}}%
\pgfpathlineto{\pgfqpoint{4.812377in}{0.413320in}}%
\pgfpathlineto{\pgfqpoint{4.809538in}{0.413320in}}%
\pgfpathlineto{\pgfqpoint{4.807017in}{0.413320in}}%
\pgfpathlineto{\pgfqpoint{4.804193in}{0.413320in}}%
\pgfpathlineto{\pgfqpoint{4.801586in}{0.413320in}}%
\pgfpathlineto{\pgfqpoint{4.798830in}{0.413320in}}%
\pgfpathlineto{\pgfqpoint{4.796274in}{0.413320in}}%
\pgfpathlineto{\pgfqpoint{4.793512in}{0.413320in}}%
\pgfpathlineto{\pgfqpoint{4.790798in}{0.413320in}}%
\pgfpathlineto{\pgfqpoint{4.788116in}{0.413320in}}%
\pgfpathlineto{\pgfqpoint{4.785445in}{0.413320in}}%
\pgfpathlineto{\pgfqpoint{4.782872in}{0.413320in}}%
\pgfpathlineto{\pgfqpoint{4.780083in}{0.413320in}}%
\pgfpathlineto{\pgfqpoint{4.777535in}{0.413320in}}%
\pgfpathlineto{\pgfqpoint{4.774732in}{0.413320in}}%
\pgfpathlineto{\pgfqpoint{4.772198in}{0.413320in}}%
\pgfpathlineto{\pgfqpoint{4.769367in}{0.413320in}}%
\pgfpathlineto{\pgfqpoint{4.766783in}{0.413320in}}%
\pgfpathlineto{\pgfqpoint{4.764018in}{0.413320in}}%
\pgfpathlineto{\pgfqpoint{4.761337in}{0.413320in}}%
\pgfpathlineto{\pgfqpoint{4.758653in}{0.413320in}}%
\pgfpathlineto{\pgfqpoint{4.755983in}{0.413320in}}%
\pgfpathlineto{\pgfqpoint{4.753298in}{0.413320in}}%
\pgfpathlineto{\pgfqpoint{4.750627in}{0.413320in}}%
\pgfpathlineto{\pgfqpoint{4.748081in}{0.413320in}}%
\pgfpathlineto{\pgfqpoint{4.745256in}{0.413320in}}%
\pgfpathlineto{\pgfqpoint{4.742696in}{0.413320in}}%
\pgfpathlineto{\pgfqpoint{4.739912in}{0.413320in}}%
\pgfpathlineto{\pgfqpoint{4.737348in}{0.413320in}}%
\pgfpathlineto{\pgfqpoint{4.734552in}{0.413320in}}%
\pgfpathlineto{\pgfqpoint{4.731901in}{0.413320in}}%
\pgfpathlineto{\pgfqpoint{4.729233in}{0.413320in}}%
\pgfpathlineto{\pgfqpoint{4.726508in}{0.413320in}}%
\pgfpathlineto{\pgfqpoint{4.723873in}{0.413320in}}%
\pgfpathlineto{\pgfqpoint{4.721160in}{0.413320in}}%
\pgfpathlineto{\pgfqpoint{4.718486in}{0.413320in}}%
\pgfpathlineto{\pgfqpoint{4.715806in}{0.413320in}}%
\pgfpathlineto{\pgfqpoint{4.713275in}{0.413320in}}%
\pgfpathlineto{\pgfqpoint{4.710437in}{0.413320in}}%
\pgfpathlineto{\pgfqpoint{4.707824in}{0.413320in}}%
\pgfpathlineto{\pgfqpoint{4.705094in}{0.413320in}}%
\pgfpathlineto{\pgfqpoint{4.702517in}{0.413320in}}%
\pgfpathlineto{\pgfqpoint{4.699734in}{0.413320in}}%
\pgfpathlineto{\pgfqpoint{4.697170in}{0.413320in}}%
\pgfpathlineto{\pgfqpoint{4.694381in}{0.413320in}}%
\pgfpathlineto{\pgfqpoint{4.691694in}{0.413320in}}%
\pgfpathlineto{\pgfqpoint{4.689051in}{0.413320in}}%
\pgfpathlineto{\pgfqpoint{4.686337in}{0.413320in}}%
\pgfpathlineto{\pgfqpoint{4.683799in}{0.413320in}}%
\pgfpathlineto{\pgfqpoint{4.680988in}{0.413320in}}%
\pgfpathlineto{\pgfqpoint{4.678448in}{0.413320in}}%
\pgfpathlineto{\pgfqpoint{4.675619in}{0.413320in}}%
\pgfpathlineto{\pgfqpoint{4.673068in}{0.413320in}}%
\pgfpathlineto{\pgfqpoint{4.670261in}{0.413320in}}%
\pgfpathlineto{\pgfqpoint{4.667764in}{0.413320in}}%
\pgfpathlineto{\pgfqpoint{4.664923in}{0.413320in}}%
\pgfpathlineto{\pgfqpoint{4.662237in}{0.413320in}}%
\pgfpathlineto{\pgfqpoint{4.659590in}{0.413320in}}%
\pgfpathlineto{\pgfqpoint{4.656873in}{0.413320in}}%
\pgfpathlineto{\pgfqpoint{4.654203in}{0.413320in}}%
\pgfpathlineto{\pgfqpoint{4.651524in}{0.413320in}}%
\pgfpathlineto{\pgfqpoint{4.648922in}{0.413320in}}%
\pgfpathlineto{\pgfqpoint{4.646169in}{0.413320in}}%
\pgfpathlineto{\pgfqpoint{4.643628in}{0.413320in}}%
\pgfpathlineto{\pgfqpoint{4.640809in}{0.413320in}}%
\pgfpathlineto{\pgfqpoint{4.638204in}{0.413320in}}%
\pgfpathlineto{\pgfqpoint{4.635445in}{0.413320in}}%
\pgfpathlineto{\pgfqpoint{4.632902in}{0.413320in}}%
\pgfpathlineto{\pgfqpoint{4.630096in}{0.413320in}}%
\pgfpathlineto{\pgfqpoint{4.627411in}{0.413320in}}%
\pgfpathlineto{\pgfqpoint{4.624741in}{0.413320in}}%
\pgfpathlineto{\pgfqpoint{4.622056in}{0.413320in}}%
\pgfpathlineto{\pgfqpoint{4.619529in}{0.413320in}}%
\pgfpathlineto{\pgfqpoint{4.616702in}{0.413320in}}%
\pgfpathlineto{\pgfqpoint{4.614134in}{0.413320in}}%
\pgfpathlineto{\pgfqpoint{4.611350in}{0.413320in}}%
\pgfpathlineto{\pgfqpoint{4.608808in}{0.413320in}}%
\pgfpathlineto{\pgfqpoint{4.605990in}{0.413320in}}%
\pgfpathlineto{\pgfqpoint{4.603430in}{0.413320in}}%
\pgfpathlineto{\pgfqpoint{4.600633in}{0.413320in}}%
\pgfpathlineto{\pgfqpoint{4.597951in}{0.413320in}}%
\pgfpathlineto{\pgfqpoint{4.595281in}{0.413320in}}%
\pgfpathlineto{\pgfqpoint{4.592589in}{0.413320in}}%
\pgfpathlineto{\pgfqpoint{4.589920in}{0.413320in}}%
\pgfpathlineto{\pgfqpoint{4.587244in}{0.413320in}}%
\pgfpathlineto{\pgfqpoint{4.584672in}{0.413320in}}%
\pgfpathlineto{\pgfqpoint{4.581888in}{0.413320in}}%
\pgfpathlineto{\pgfqpoint{4.579305in}{0.413320in}}%
\pgfpathlineto{\pgfqpoint{4.576531in}{0.413320in}}%
\pgfpathlineto{\pgfqpoint{4.573947in}{0.413320in}}%
\pgfpathlineto{\pgfqpoint{4.571171in}{0.413320in}}%
\pgfpathlineto{\pgfqpoint{4.568612in}{0.413320in}}%
\pgfpathlineto{\pgfqpoint{4.565820in}{0.413320in}}%
\pgfpathlineto{\pgfqpoint{4.563125in}{0.413320in}}%
\pgfpathlineto{\pgfqpoint{4.560448in}{0.413320in}}%
\pgfpathlineto{\pgfqpoint{4.557777in}{0.413320in}}%
\pgfpathlineto{\pgfqpoint{4.555106in}{0.413320in}}%
\pgfpathlineto{\pgfqpoint{4.552425in}{0.413320in}}%
\pgfpathlineto{\pgfqpoint{4.549822in}{0.413320in}}%
\pgfpathlineto{\pgfqpoint{4.547064in}{0.413320in}}%
\pgfpathlineto{\pgfqpoint{4.544464in}{0.413320in}}%
\pgfpathlineto{\pgfqpoint{4.541711in}{0.413320in}}%
\pgfpathlineto{\pgfqpoint{4.539144in}{0.413320in}}%
\pgfpathlineto{\pgfqpoint{4.536400in}{0.413320in}}%
\pgfpathlineto{\pgfqpoint{4.533764in}{0.413320in}}%
\pgfpathlineto{\pgfqpoint{4.530990in}{0.413320in}}%
\pgfpathlineto{\pgfqpoint{4.528307in}{0.413320in}}%
\pgfpathlineto{\pgfqpoint{4.525640in}{0.413320in}}%
\pgfpathlineto{\pgfqpoint{4.522962in}{0.413320in}}%
\pgfpathlineto{\pgfqpoint{4.520345in}{0.413320in}}%
\pgfpathlineto{\pgfqpoint{4.517598in}{0.413320in}}%
\pgfpathlineto{\pgfqpoint{4.515080in}{0.413320in}}%
\pgfpathlineto{\pgfqpoint{4.512246in}{0.413320in}}%
\pgfpathlineto{\pgfqpoint{4.509643in}{0.413320in}}%
\pgfpathlineto{\pgfqpoint{4.506893in}{0.413320in}}%
\pgfpathlineto{\pgfqpoint{4.504305in}{0.413320in}}%
\pgfpathlineto{\pgfqpoint{4.501529in}{0.413320in}}%
\pgfpathlineto{\pgfqpoint{4.498850in}{0.413320in}}%
\pgfpathlineto{\pgfqpoint{4.496167in}{0.413320in}}%
\pgfpathlineto{\pgfqpoint{4.493492in}{0.413320in}}%
\pgfpathlineto{\pgfqpoint{4.490822in}{0.413320in}}%
\pgfpathlineto{\pgfqpoint{4.488130in}{0.413320in}}%
\pgfpathlineto{\pgfqpoint{4.485581in}{0.413320in}}%
\pgfpathlineto{\pgfqpoint{4.482778in}{0.413320in}}%
\pgfpathlineto{\pgfqpoint{4.480201in}{0.413320in}}%
\pgfpathlineto{\pgfqpoint{4.477430in}{0.413320in}}%
\pgfpathlineto{\pgfqpoint{4.474861in}{0.413320in}}%
\pgfpathlineto{\pgfqpoint{4.472059in}{0.413320in}}%
\pgfpathlineto{\pgfqpoint{4.469492in}{0.413320in}}%
\pgfpathlineto{\pgfqpoint{4.466717in}{0.413320in}}%
\pgfpathlineto{\pgfqpoint{4.464029in}{0.413320in}}%
\pgfpathlineto{\pgfqpoint{4.461367in}{0.413320in}}%
\pgfpathlineto{\pgfqpoint{4.458681in}{0.413320in}}%
\pgfpathlineto{\pgfqpoint{4.456138in}{0.413320in}}%
\pgfpathlineto{\pgfqpoint{4.453312in}{0.413320in}}%
\pgfpathlineto{\pgfqpoint{4.450767in}{0.413320in}}%
\pgfpathlineto{\pgfqpoint{4.447965in}{0.413320in}}%
\pgfpathlineto{\pgfqpoint{4.445423in}{0.413320in}}%
\pgfpathlineto{\pgfqpoint{4.442611in}{0.413320in}}%
\pgfpathlineto{\pgfqpoint{4.440041in}{0.413320in}}%
\pgfpathlineto{\pgfqpoint{4.437253in}{0.413320in}}%
\pgfpathlineto{\pgfqpoint{4.434569in}{0.413320in}}%
\pgfpathlineto{\pgfqpoint{4.431901in}{0.413320in}}%
\pgfpathlineto{\pgfqpoint{4.429220in}{0.413320in}}%
\pgfpathlineto{\pgfqpoint{4.426534in}{0.413320in}}%
\pgfpathlineto{\pgfqpoint{4.423863in}{0.413320in}}%
\pgfpathlineto{\pgfqpoint{4.421292in}{0.413320in}}%
\pgfpathlineto{\pgfqpoint{4.418506in}{0.413320in}}%
\pgfpathlineto{\pgfqpoint{4.415932in}{0.413320in}}%
\pgfpathlineto{\pgfqpoint{4.413149in}{0.413320in}}%
\pgfpathlineto{\pgfqpoint{4.410587in}{0.413320in}}%
\pgfpathlineto{\pgfqpoint{4.407788in}{0.413320in}}%
\pgfpathlineto{\pgfqpoint{4.405234in}{0.413320in}}%
\pgfpathlineto{\pgfqpoint{4.402468in}{0.413320in}}%
\pgfpathlineto{\pgfqpoint{4.399745in}{0.413320in}}%
\pgfpathlineto{\pgfqpoint{4.397076in}{0.413320in}}%
\pgfpathlineto{\pgfqpoint{4.394400in}{0.413320in}}%
\pgfpathlineto{\pgfqpoint{4.391721in}{0.413320in}}%
\pgfpathlineto{\pgfqpoint{4.389044in}{0.413320in}}%
\pgfpathlineto{\pgfqpoint{4.386431in}{0.413320in}}%
\pgfpathlineto{\pgfqpoint{4.383674in}{0.413320in}}%
\pgfpathlineto{\pgfqpoint{4.381097in}{0.413320in}}%
\pgfpathlineto{\pgfqpoint{4.378329in}{0.413320in}}%
\pgfpathlineto{\pgfqpoint{4.375761in}{0.413320in}}%
\pgfpathlineto{\pgfqpoint{4.372976in}{0.413320in}}%
\pgfpathlineto{\pgfqpoint{4.370437in}{0.413320in}}%
\pgfpathlineto{\pgfqpoint{4.367646in}{0.413320in}}%
\pgfpathlineto{\pgfqpoint{4.364936in}{0.413320in}}%
\pgfpathlineto{\pgfqpoint{4.362270in}{0.413320in}}%
\pgfpathlineto{\pgfqpoint{4.359582in}{0.413320in}}%
\pgfpathlineto{\pgfqpoint{4.357014in}{0.413320in}}%
\pgfpathlineto{\pgfqpoint{4.354224in}{0.413320in}}%
\pgfpathlineto{\pgfqpoint{4.351645in}{0.413320in}}%
\pgfpathlineto{\pgfqpoint{4.348868in}{0.413320in}}%
\pgfpathlineto{\pgfqpoint{4.346263in}{0.413320in}}%
\pgfpathlineto{\pgfqpoint{4.343510in}{0.413320in}}%
\pgfpathlineto{\pgfqpoint{4.340976in}{0.413320in}}%
\pgfpathlineto{\pgfqpoint{4.338154in}{0.413320in}}%
\pgfpathlineto{\pgfqpoint{4.335463in}{0.413320in}}%
\pgfpathlineto{\pgfqpoint{4.332796in}{0.413320in}}%
\pgfpathlineto{\pgfqpoint{4.330118in}{0.413320in}}%
\pgfpathlineto{\pgfqpoint{4.327440in}{0.413320in}}%
\pgfpathlineto{\pgfqpoint{4.324760in}{0.413320in}}%
\pgfpathlineto{\pgfqpoint{4.322181in}{0.413320in}}%
\pgfpathlineto{\pgfqpoint{4.319405in}{0.413320in}}%
\pgfpathlineto{\pgfqpoint{4.316856in}{0.413320in}}%
\pgfpathlineto{\pgfqpoint{4.314032in}{0.413320in}}%
\pgfpathlineto{\pgfqpoint{4.311494in}{0.413320in}}%
\pgfpathlineto{\pgfqpoint{4.308691in}{0.413320in}}%
\pgfpathlineto{\pgfqpoint{4.306118in}{0.413320in}}%
\pgfpathlineto{\pgfqpoint{4.303357in}{0.413320in}}%
\pgfpathlineto{\pgfqpoint{4.300656in}{0.413320in}}%
\pgfpathlineto{\pgfqpoint{4.297977in}{0.413320in}}%
\pgfpathlineto{\pgfqpoint{4.295299in}{0.413320in}}%
\pgfpathlineto{\pgfqpoint{4.292786in}{0.413320in}}%
\pgfpathlineto{\pgfqpoint{4.289936in}{0.413320in}}%
\pgfpathlineto{\pgfqpoint{4.287399in}{0.413320in}}%
\pgfpathlineto{\pgfqpoint{4.284586in}{0.413320in}}%
\pgfpathlineto{\pgfqpoint{4.282000in}{0.413320in}}%
\pgfpathlineto{\pgfqpoint{4.279212in}{0.413320in}}%
\pgfpathlineto{\pgfqpoint{4.276635in}{0.413320in}}%
\pgfpathlineto{\pgfqpoint{4.273874in}{0.413320in}}%
\pgfpathlineto{\pgfqpoint{4.271187in}{0.413320in}}%
\pgfpathlineto{\pgfqpoint{4.268590in}{0.413320in}}%
\pgfpathlineto{\pgfqpoint{4.265824in}{0.413320in}}%
\pgfpathlineto{\pgfqpoint{4.263157in}{0.413320in}}%
\pgfpathlineto{\pgfqpoint{4.260477in}{0.413320in}}%
\pgfpathlineto{\pgfqpoint{4.257958in}{0.413320in}}%
\pgfpathlineto{\pgfqpoint{4.255120in}{0.413320in}}%
\pgfpathlineto{\pgfqpoint{4.252581in}{0.413320in}}%
\pgfpathlineto{\pgfqpoint{4.249767in}{0.413320in}}%
\pgfpathlineto{\pgfqpoint{4.247225in}{0.413320in}}%
\pgfpathlineto{\pgfqpoint{4.244394in}{0.413320in}}%
\pgfpathlineto{\pgfqpoint{4.241900in}{0.413320in}}%
\pgfpathlineto{\pgfqpoint{4.239084in}{0.413320in}}%
\pgfpathlineto{\pgfqpoint{4.236375in}{0.413320in}}%
\pgfpathlineto{\pgfqpoint{4.233691in}{0.413320in}}%
\pgfpathlineto{\pgfqpoint{4.231013in}{0.413320in}}%
\pgfpathlineto{\pgfqpoint{4.228331in}{0.413320in}}%
\pgfpathlineto{\pgfqpoint{4.225654in}{0.413320in}}%
\pgfpathlineto{\pgfqpoint{4.223082in}{0.413320in}}%
\pgfpathlineto{\pgfqpoint{4.220304in}{0.413320in}}%
\pgfpathlineto{\pgfqpoint{4.217694in}{0.413320in}}%
\pgfpathlineto{\pgfqpoint{4.214948in}{0.413320in}}%
\pgfpathlineto{\pgfqpoint{4.212383in}{0.413320in}}%
\pgfpathlineto{\pgfqpoint{4.209597in}{0.413320in}}%
\pgfpathlineto{\pgfqpoint{4.207076in}{0.413320in}}%
\pgfpathlineto{\pgfqpoint{4.204240in}{0.413320in}}%
\pgfpathlineto{\pgfqpoint{4.201542in}{0.413320in}}%
\pgfpathlineto{\pgfqpoint{4.198878in}{0.413320in}}%
\pgfpathlineto{\pgfqpoint{4.196186in}{0.413320in}}%
\pgfpathlineto{\pgfqpoint{4.193638in}{0.413320in}}%
\pgfpathlineto{\pgfqpoint{4.190842in}{0.413320in}}%
\pgfpathlineto{\pgfqpoint{4.188318in}{0.413320in}}%
\pgfpathlineto{\pgfqpoint{4.185481in}{0.413320in}}%
\pgfpathlineto{\pgfqpoint{4.182899in}{0.413320in}}%
\pgfpathlineto{\pgfqpoint{4.180129in}{0.413320in}}%
\pgfpathlineto{\pgfqpoint{4.177593in}{0.413320in}}%
\pgfpathlineto{\pgfqpoint{4.174770in}{0.413320in}}%
\pgfpathlineto{\pgfqpoint{4.172093in}{0.413320in}}%
\pgfpathlineto{\pgfqpoint{4.169415in}{0.413320in}}%
\pgfpathlineto{\pgfqpoint{4.166737in}{0.413320in}}%
\pgfpathlineto{\pgfqpoint{4.164059in}{0.413320in}}%
\pgfpathlineto{\pgfqpoint{4.161380in}{0.413320in}}%
\pgfpathlineto{\pgfqpoint{4.158806in}{0.413320in}}%
\pgfpathlineto{\pgfqpoint{4.156016in}{0.413320in}}%
\pgfpathlineto{\pgfqpoint{4.153423in}{0.413320in}}%
\pgfpathlineto{\pgfqpoint{4.150665in}{0.413320in}}%
\pgfpathlineto{\pgfqpoint{4.148082in}{0.413320in}}%
\pgfpathlineto{\pgfqpoint{4.145310in}{0.413320in}}%
\pgfpathlineto{\pgfqpoint{4.142713in}{0.413320in}}%
\pgfpathlineto{\pgfqpoint{4.139963in}{0.413320in}}%
\pgfpathlineto{\pgfqpoint{4.137272in}{0.413320in}}%
\pgfpathlineto{\pgfqpoint{4.134615in}{0.413320in}}%
\pgfpathlineto{\pgfqpoint{4.131920in}{0.413320in}}%
\pgfpathlineto{\pgfqpoint{4.129349in}{0.413320in}}%
\pgfpathlineto{\pgfqpoint{4.126553in}{0.413320in}}%
\pgfpathlineto{\pgfqpoint{4.124019in}{0.413320in}}%
\pgfpathlineto{\pgfqpoint{4.121205in}{0.413320in}}%
\pgfpathlineto{\pgfqpoint{4.118554in}{0.413320in}}%
\pgfpathlineto{\pgfqpoint{4.115844in}{0.413320in}}%
\pgfpathlineto{\pgfqpoint{4.113252in}{0.413320in}}%
\pgfpathlineto{\pgfqpoint{4.110488in}{0.413320in}}%
\pgfpathlineto{\pgfqpoint{4.107814in}{0.413320in}}%
\pgfpathlineto{\pgfqpoint{4.105185in}{0.413320in}}%
\pgfpathlineto{\pgfqpoint{4.102456in}{0.413320in}}%
\pgfpathlineto{\pgfqpoint{4.099777in}{0.413320in}}%
\pgfpathlineto{\pgfqpoint{4.097092in}{0.413320in}}%
\pgfpathlineto{\pgfqpoint{4.094527in}{0.413320in}}%
\pgfpathlineto{\pgfqpoint{4.091729in}{0.413320in}}%
\pgfpathlineto{\pgfqpoint{4.089159in}{0.413320in}}%
\pgfpathlineto{\pgfqpoint{4.086385in}{0.413320in}}%
\pgfpathlineto{\pgfqpoint{4.083870in}{0.413320in}}%
\pgfpathlineto{\pgfqpoint{4.081018in}{0.413320in}}%
\pgfpathlineto{\pgfqpoint{4.078471in}{0.413320in}}%
\pgfpathlineto{\pgfqpoint{4.075705in}{0.413320in}}%
\pgfpathlineto{\pgfqpoint{4.072985in}{0.413320in}}%
\pgfpathlineto{\pgfqpoint{4.070313in}{0.413320in}}%
\pgfpathlineto{\pgfqpoint{4.067636in}{0.413320in}}%
\pgfpathlineto{\pgfqpoint{4.064957in}{0.413320in}}%
\pgfpathlineto{\pgfqpoint{4.062266in}{0.413320in}}%
\pgfpathlineto{\pgfqpoint{4.059702in}{0.413320in}}%
\pgfpathlineto{\pgfqpoint{4.056911in}{0.413320in}}%
\pgfpathlineto{\pgfqpoint{4.054326in}{0.413320in}}%
\pgfpathlineto{\pgfqpoint{4.051557in}{0.413320in}}%
\pgfpathlineto{\pgfqpoint{4.049006in}{0.413320in}}%
\pgfpathlineto{\pgfqpoint{4.046210in}{0.413320in}}%
\pgfpathlineto{\pgfqpoint{4.043667in}{0.413320in}}%
\pgfpathlineto{\pgfqpoint{4.040852in}{0.413320in}}%
\pgfpathlineto{\pgfqpoint{4.038174in}{0.413320in}}%
\pgfpathlineto{\pgfqpoint{4.035492in}{0.413320in}}%
\pgfpathlineto{\pgfqpoint{4.032817in}{0.413320in}}%
\pgfpathlineto{\pgfqpoint{4.030229in}{0.413320in}}%
\pgfpathlineto{\pgfqpoint{4.027447in}{0.413320in}}%
\pgfpathlineto{\pgfqpoint{4.024868in}{0.413320in}}%
\pgfpathlineto{\pgfqpoint{4.022097in}{0.413320in}}%
\pgfpathlineto{\pgfqpoint{4.019518in}{0.413320in}}%
\pgfpathlineto{\pgfqpoint{4.016744in}{0.413320in}}%
\pgfpathlineto{\pgfqpoint{4.014186in}{0.413320in}}%
\pgfpathlineto{\pgfqpoint{4.011394in}{0.413320in}}%
\pgfpathlineto{\pgfqpoint{4.008699in}{0.413320in}}%
\pgfpathlineto{\pgfqpoint{4.006034in}{0.413320in}}%
\pgfpathlineto{\pgfqpoint{4.003348in}{0.413320in}}%
\pgfpathlineto{\pgfqpoint{4.000674in}{0.413320in}}%
\pgfpathlineto{\pgfqpoint{3.997990in}{0.413320in}}%
\pgfpathlineto{\pgfqpoint{3.995417in}{0.413320in}}%
\pgfpathlineto{\pgfqpoint{3.992642in}{0.413320in}}%
\pgfpathlineto{\pgfqpoint{3.990055in}{0.413320in}}%
\pgfpathlineto{\pgfqpoint{3.987270in}{0.413320in}}%
\pgfpathlineto{\pgfqpoint{3.984714in}{0.413320in}}%
\pgfpathlineto{\pgfqpoint{3.981929in}{0.413320in}}%
\pgfpathlineto{\pgfqpoint{3.979389in}{0.413320in}}%
\pgfpathlineto{\pgfqpoint{3.976563in}{0.413320in}}%
\pgfpathlineto{\pgfqpoint{3.973885in}{0.413320in}}%
\pgfpathlineto{\pgfqpoint{3.971250in}{0.413320in}}%
\pgfpathlineto{\pgfqpoint{3.968523in}{0.413320in}}%
\pgfpathlineto{\pgfqpoint{3.966013in}{0.413320in}}%
\pgfpathlineto{\pgfqpoint{3.963176in}{0.413320in}}%
\pgfpathlineto{\pgfqpoint{3.960635in}{0.413320in}}%
\pgfpathlineto{\pgfqpoint{3.957823in}{0.413320in}}%
\pgfpathlineto{\pgfqpoint{3.955211in}{0.413320in}}%
\pgfpathlineto{\pgfqpoint{3.952464in}{0.413320in}}%
\pgfpathlineto{\pgfqpoint{3.949894in}{0.413320in}}%
\pgfpathlineto{\pgfqpoint{3.947101in}{0.413320in}}%
\pgfpathlineto{\pgfqpoint{3.944431in}{0.413320in}}%
\pgfpathlineto{\pgfqpoint{3.941778in}{0.413320in}}%
\pgfpathlineto{\pgfqpoint{3.939075in}{0.413320in}}%
\pgfpathlineto{\pgfqpoint{3.936395in}{0.413320in}}%
\pgfpathlineto{\pgfqpoint{3.933714in}{0.413320in}}%
\pgfpathlineto{\pgfqpoint{3.931202in}{0.413320in}}%
\pgfpathlineto{\pgfqpoint{3.928347in}{0.413320in}}%
\pgfpathlineto{\pgfqpoint{3.925778in}{0.413320in}}%
\pgfpathlineto{\pgfqpoint{3.923005in}{0.413320in}}%
\pgfpathlineto{\pgfqpoint{3.920412in}{0.413320in}}%
\pgfpathlineto{\pgfqpoint{3.917646in}{0.413320in}}%
\pgfpathlineto{\pgfqpoint{3.915107in}{0.413320in}}%
\pgfpathlineto{\pgfqpoint{3.912296in}{0.413320in}}%
\pgfpathlineto{\pgfqpoint{3.909602in}{0.413320in}}%
\pgfpathlineto{\pgfqpoint{3.906918in}{0.413320in}}%
\pgfpathlineto{\pgfqpoint{3.904252in}{0.413320in}}%
\pgfpathlineto{\pgfqpoint{3.901573in}{0.413320in}}%
\pgfpathlineto{\pgfqpoint{3.898891in}{0.413320in}}%
\pgfpathlineto{\pgfqpoint{3.896345in}{0.413320in}}%
\pgfpathlineto{\pgfqpoint{3.893541in}{0.413320in}}%
\pgfpathlineto{\pgfqpoint{3.890926in}{0.413320in}}%
\pgfpathlineto{\pgfqpoint{3.888188in}{0.413320in}}%
\pgfpathlineto{\pgfqpoint{3.885621in}{0.413320in}}%
\pgfpathlineto{\pgfqpoint{3.882850in}{0.413320in}}%
\pgfpathlineto{\pgfqpoint{3.880237in}{0.413320in}}%
\pgfpathlineto{\pgfqpoint{3.877466in}{0.413320in}}%
\pgfpathlineto{\pgfqpoint{3.874790in}{0.413320in}}%
\pgfpathlineto{\pgfqpoint{3.872114in}{0.413320in}}%
\pgfpathlineto{\pgfqpoint{3.869435in}{0.413320in}}%
\pgfpathlineto{\pgfqpoint{3.866815in}{0.413320in}}%
\pgfpathlineto{\pgfqpoint{3.864073in}{0.413320in}}%
\pgfpathlineto{\pgfqpoint{3.861561in}{0.413320in}}%
\pgfpathlineto{\pgfqpoint{3.858720in}{0.413320in}}%
\pgfpathlineto{\pgfqpoint{3.856100in}{0.413320in}}%
\pgfpathlineto{\pgfqpoint{3.853358in}{0.413320in}}%
\pgfpathlineto{\pgfqpoint{3.850814in}{0.413320in}}%
\pgfpathlineto{\pgfqpoint{3.848005in}{0.413320in}}%
\pgfpathlineto{\pgfqpoint{3.845329in}{0.413320in}}%
\pgfpathlineto{\pgfqpoint{3.842641in}{0.413320in}}%
\pgfpathlineto{\pgfqpoint{3.839960in}{0.413320in}}%
\pgfpathlineto{\pgfqpoint{3.837286in}{0.413320in}}%
\pgfpathlineto{\pgfqpoint{3.834616in}{0.413320in}}%
\pgfpathlineto{\pgfqpoint{3.832053in}{0.413320in}}%
\pgfpathlineto{\pgfqpoint{3.829252in}{0.413320in}}%
\pgfpathlineto{\pgfqpoint{3.826679in}{0.413320in}}%
\pgfpathlineto{\pgfqpoint{3.823903in}{0.413320in}}%
\pgfpathlineto{\pgfqpoint{3.821315in}{0.413320in}}%
\pgfpathlineto{\pgfqpoint{3.818546in}{0.413320in}}%
\pgfpathlineto{\pgfqpoint{3.815983in}{0.413320in}}%
\pgfpathlineto{\pgfqpoint{3.813172in}{0.413320in}}%
\pgfpathlineto{\pgfqpoint{3.810510in}{0.413320in}}%
\pgfpathlineto{\pgfqpoint{3.807832in}{0.413320in}}%
\pgfpathlineto{\pgfqpoint{3.805145in}{0.413320in}}%
\pgfpathlineto{\pgfqpoint{3.802569in}{0.413320in}}%
\pgfpathlineto{\pgfqpoint{3.799797in}{0.413320in}}%
\pgfpathlineto{\pgfqpoint{3.797265in}{0.413320in}}%
\pgfpathlineto{\pgfqpoint{3.794435in}{0.413320in}}%
\pgfpathlineto{\pgfqpoint{3.791897in}{0.413320in}}%
\pgfpathlineto{\pgfqpoint{3.789084in}{0.413320in}}%
\pgfpathlineto{\pgfqpoint{3.786504in}{0.413320in}}%
\pgfpathlineto{\pgfqpoint{3.783725in}{0.413320in}}%
\pgfpathlineto{\pgfqpoint{3.781046in}{0.413320in}}%
\pgfpathlineto{\pgfqpoint{3.778370in}{0.413320in}}%
\pgfpathlineto{\pgfqpoint{3.775691in}{0.413320in}}%
\pgfpathlineto{\pgfqpoint{3.773014in}{0.413320in}}%
\pgfpathlineto{\pgfqpoint{3.770323in}{0.413320in}}%
\pgfpathlineto{\pgfqpoint{3.767782in}{0.413320in}}%
\pgfpathlineto{\pgfqpoint{3.764966in}{0.413320in}}%
\pgfpathlineto{\pgfqpoint{3.762389in}{0.413320in}}%
\pgfpathlineto{\pgfqpoint{3.759622in}{0.413320in}}%
\pgfpathlineto{\pgfqpoint{3.757065in}{0.413320in}}%
\pgfpathlineto{\pgfqpoint{3.754265in}{0.413320in}}%
\pgfpathlineto{\pgfqpoint{3.751728in}{0.413320in}}%
\pgfpathlineto{\pgfqpoint{3.748903in}{0.413320in}}%
\pgfpathlineto{\pgfqpoint{3.746229in}{0.413320in}}%
\pgfpathlineto{\pgfqpoint{3.743548in}{0.413320in}}%
\pgfpathlineto{\pgfqpoint{3.740874in}{0.413320in}}%
\pgfpathlineto{\pgfqpoint{3.738194in}{0.413320in}}%
\pgfpathlineto{\pgfqpoint{3.735509in}{0.413320in}}%
\pgfpathlineto{\pgfqpoint{3.732950in}{0.413320in}}%
\pgfpathlineto{\pgfqpoint{3.730158in}{0.413320in}}%
\pgfpathlineto{\pgfqpoint{3.727581in}{0.413320in}}%
\pgfpathlineto{\pgfqpoint{3.724804in}{0.413320in}}%
\pgfpathlineto{\pgfqpoint{3.722228in}{0.413320in}}%
\pgfpathlineto{\pgfqpoint{3.719446in}{0.413320in}}%
\pgfpathlineto{\pgfqpoint{3.716875in}{0.413320in}}%
\pgfpathlineto{\pgfqpoint{3.714086in}{0.413320in}}%
\pgfpathlineto{\pgfqpoint{3.711410in}{0.413320in}}%
\pgfpathlineto{\pgfqpoint{3.708729in}{0.413320in}}%
\pgfpathlineto{\pgfqpoint{3.706053in}{0.413320in}}%
\pgfpathlineto{\pgfqpoint{3.703460in}{0.413320in}}%
\pgfpathlineto{\pgfqpoint{3.700684in}{0.413320in}}%
\pgfpathlineto{\pgfqpoint{3.698125in}{0.413320in}}%
\pgfpathlineto{\pgfqpoint{3.695331in}{0.413320in}}%
\pgfpathlineto{\pgfqpoint{3.692765in}{0.413320in}}%
\pgfpathlineto{\pgfqpoint{3.689983in}{0.413320in}}%
\pgfpathlineto{\pgfqpoint{3.687442in}{0.413320in}}%
\pgfpathlineto{\pgfqpoint{3.684620in}{0.413320in}}%
\pgfpathlineto{\pgfqpoint{3.681948in}{0.413320in}}%
\pgfpathlineto{\pgfqpoint{3.679273in}{0.413320in}}%
\pgfpathlineto{\pgfqpoint{3.676591in}{0.413320in}}%
\pgfpathlineto{\pgfqpoint{3.673911in}{0.413320in}}%
\pgfpathlineto{\pgfqpoint{3.671232in}{0.413320in}}%
\pgfpathlineto{\pgfqpoint{3.668665in}{0.413320in}}%
\pgfpathlineto{\pgfqpoint{3.665864in}{0.413320in}}%
\pgfpathlineto{\pgfqpoint{3.663276in}{0.413320in}}%
\pgfpathlineto{\pgfqpoint{3.660515in}{0.413320in}}%
\pgfpathlineto{\pgfqpoint{3.657917in}{0.413320in}}%
\pgfpathlineto{\pgfqpoint{3.655165in}{0.413320in}}%
\pgfpathlineto{\pgfqpoint{3.652628in}{0.413320in}}%
\pgfpathlineto{\pgfqpoint{3.649837in}{0.413320in}}%
\pgfpathlineto{\pgfqpoint{3.647130in}{0.413320in}}%
\pgfpathlineto{\pgfqpoint{3.644452in}{0.413320in}}%
\pgfpathlineto{\pgfqpoint{3.641773in}{0.413320in}}%
\pgfpathlineto{\pgfqpoint{3.639207in}{0.413320in}}%
\pgfpathlineto{\pgfqpoint{3.636413in}{0.413320in}}%
\pgfpathlineto{\pgfqpoint{3.633858in}{0.413320in}}%
\pgfpathlineto{\pgfqpoint{3.631058in}{0.413320in}}%
\pgfpathlineto{\pgfqpoint{3.628460in}{0.413320in}}%
\pgfpathlineto{\pgfqpoint{3.625689in}{0.413320in}}%
\pgfpathlineto{\pgfqpoint{3.623165in}{0.413320in}}%
\pgfpathlineto{\pgfqpoint{3.620345in}{0.413320in}}%
\pgfpathlineto{\pgfqpoint{3.617667in}{0.413320in}}%
\pgfpathlineto{\pgfqpoint{3.614982in}{0.413320in}}%
\pgfpathlineto{\pgfqpoint{3.612311in}{0.413320in}}%
\pgfpathlineto{\pgfqpoint{3.609632in}{0.413320in}}%
\pgfpathlineto{\pgfqpoint{3.606951in}{0.413320in}}%
\pgfpathlineto{\pgfqpoint{3.604387in}{0.413320in}}%
\pgfpathlineto{\pgfqpoint{3.601590in}{0.413320in}}%
\pgfpathlineto{\pgfqpoint{3.598998in}{0.413320in}}%
\pgfpathlineto{\pgfqpoint{3.596240in}{0.413320in}}%
\pgfpathlineto{\pgfqpoint{3.593620in}{0.413320in}}%
\pgfpathlineto{\pgfqpoint{3.590883in}{0.413320in}}%
\pgfpathlineto{\pgfqpoint{3.588258in}{0.413320in}}%
\pgfpathlineto{\pgfqpoint{3.585532in}{0.413320in}}%
\pgfpathlineto{\pgfqpoint{3.582851in}{0.413320in}}%
\pgfpathlineto{\pgfqpoint{3.580191in}{0.413320in}}%
\pgfpathlineto{\pgfqpoint{3.577487in}{0.413320in}}%
\pgfpathlineto{\pgfqpoint{3.574814in}{0.413320in}}%
\pgfpathlineto{\pgfqpoint{3.572126in}{0.413320in}}%
\pgfpathlineto{\pgfqpoint{3.569584in}{0.413320in}}%
\pgfpathlineto{\pgfqpoint{3.566774in}{0.413320in}}%
\pgfpathlineto{\pgfqpoint{3.564188in}{0.413320in}}%
\pgfpathlineto{\pgfqpoint{3.561420in}{0.413320in}}%
\pgfpathlineto{\pgfqpoint{3.558853in}{0.413320in}}%
\pgfpathlineto{\pgfqpoint{3.556061in}{0.413320in}}%
\pgfpathlineto{\pgfqpoint{3.553498in}{0.413320in}}%
\pgfpathlineto{\pgfqpoint{3.550713in}{0.413320in}}%
\pgfpathlineto{\pgfqpoint{3.548029in}{0.413320in}}%
\pgfpathlineto{\pgfqpoint{3.545349in}{0.413320in}}%
\pgfpathlineto{\pgfqpoint{3.542656in}{0.413320in}}%
\pgfpathlineto{\pgfqpoint{3.540093in}{0.413320in}}%
\pgfpathlineto{\pgfqpoint{3.537309in}{0.413320in}}%
\pgfpathlineto{\pgfqpoint{3.534783in}{0.413320in}}%
\pgfpathlineto{\pgfqpoint{3.531955in}{0.413320in}}%
\pgfpathlineto{\pgfqpoint{3.529327in}{0.413320in}}%
\pgfpathlineto{\pgfqpoint{3.526601in}{0.413320in}}%
\pgfpathlineto{\pgfqpoint{3.524041in}{0.413320in}}%
\pgfpathlineto{\pgfqpoint{3.521244in}{0.413320in}}%
\pgfpathlineto{\pgfqpoint{3.518565in}{0.413320in}}%
\pgfpathlineto{\pgfqpoint{3.515884in}{0.413320in}}%
\pgfpathlineto{\pgfqpoint{3.513209in}{0.413320in}}%
\pgfpathlineto{\pgfqpoint{3.510533in}{0.413320in}}%
\pgfpathlineto{\pgfqpoint{3.507840in}{0.413320in}}%
\pgfpathlineto{\pgfqpoint{3.505262in}{0.413320in}}%
\pgfpathlineto{\pgfqpoint{3.502488in}{0.413320in}}%
\pgfpathlineto{\pgfqpoint{3.499909in}{0.413320in}}%
\pgfpathlineto{\pgfqpoint{3.497139in}{0.413320in}}%
\pgfpathlineto{\pgfqpoint{3.494581in}{0.413320in}}%
\pgfpathlineto{\pgfqpoint{3.491783in}{0.413320in}}%
\pgfpathlineto{\pgfqpoint{3.489223in}{0.413320in}}%
\pgfpathlineto{\pgfqpoint{3.486442in}{0.413320in}}%
\pgfpathlineto{\pgfqpoint{3.483744in}{0.413320in}}%
\pgfpathlineto{\pgfqpoint{3.481072in}{0.413320in}}%
\pgfpathlineto{\pgfqpoint{3.478378in}{0.413320in}}%
\pgfpathlineto{\pgfqpoint{3.475821in}{0.413320in}}%
\pgfpathlineto{\pgfqpoint{3.473021in}{0.413320in}}%
\pgfpathlineto{\pgfqpoint{3.470466in}{0.413320in}}%
\pgfpathlineto{\pgfqpoint{3.467678in}{0.413320in}}%
\pgfpathlineto{\pgfqpoint{3.465072in}{0.413320in}}%
\pgfpathlineto{\pgfqpoint{3.462321in}{0.413320in}}%
\pgfpathlineto{\pgfqpoint{3.459695in}{0.413320in}}%
\pgfpathlineto{\pgfqpoint{3.456960in}{0.413320in}}%
\pgfpathlineto{\pgfqpoint{3.454285in}{0.413320in}}%
\pgfpathlineto{\pgfqpoint{3.451597in}{0.413320in}}%
\pgfpathlineto{\pgfqpoint{3.448926in}{0.413320in}}%
\pgfpathlineto{\pgfqpoint{3.446257in}{0.413320in}}%
\pgfpathlineto{\pgfqpoint{3.443574in}{0.413320in}}%
\pgfpathlineto{\pgfqpoint{3.440996in}{0.413320in}}%
\pgfpathlineto{\pgfqpoint{3.438210in}{0.413320in}}%
\pgfpathlineto{\pgfqpoint{3.435635in}{0.413320in}}%
\pgfpathlineto{\pgfqpoint{3.432851in}{0.413320in}}%
\pgfpathlineto{\pgfqpoint{3.430313in}{0.413320in}}%
\pgfpathlineto{\pgfqpoint{3.427501in}{0.413320in}}%
\pgfpathlineto{\pgfqpoint{3.424887in}{0.413320in}}%
\pgfpathlineto{\pgfqpoint{3.422142in}{0.413320in}}%
\pgfpathlineto{\pgfqpoint{3.419455in}{0.413320in}}%
\pgfpathlineto{\pgfqpoint{3.416780in}{0.413320in}}%
\pgfpathlineto{\pgfqpoint{3.414109in}{0.413320in}}%
\pgfpathlineto{\pgfqpoint{3.411431in}{0.413320in}}%
\pgfpathlineto{\pgfqpoint{3.408752in}{0.413320in}}%
\pgfpathlineto{\pgfqpoint{3.406202in}{0.413320in}}%
\pgfpathlineto{\pgfqpoint{3.403394in}{0.413320in}}%
\pgfpathlineto{\pgfqpoint{3.400783in}{0.413320in}}%
\pgfpathlineto{\pgfqpoint{3.398037in}{0.413320in}}%
\pgfpathlineto{\pgfqpoint{3.395461in}{0.413320in}}%
\pgfpathlineto{\pgfqpoint{3.392681in}{0.413320in}}%
\pgfpathlineto{\pgfqpoint{3.390102in}{0.413320in}}%
\pgfpathlineto{\pgfqpoint{3.387309in}{0.413320in}}%
\pgfpathlineto{\pgfqpoint{3.384647in}{0.413320in}}%
\pgfpathlineto{\pgfqpoint{3.381959in}{0.413320in}}%
\pgfpathlineto{\pgfqpoint{3.379290in}{0.413320in}}%
\pgfpathlineto{\pgfqpoint{3.376735in}{0.413320in}}%
\pgfpathlineto{\pgfqpoint{3.373921in}{0.413320in}}%
\pgfpathlineto{\pgfqpoint{3.371357in}{0.413320in}}%
\pgfpathlineto{\pgfqpoint{3.368577in}{0.413320in}}%
\pgfpathlineto{\pgfqpoint{3.365996in}{0.413320in}}%
\pgfpathlineto{\pgfqpoint{3.363221in}{0.413320in}}%
\pgfpathlineto{\pgfqpoint{3.360620in}{0.413320in}}%
\pgfpathlineto{\pgfqpoint{3.357862in}{0.413320in}}%
\pgfpathlineto{\pgfqpoint{3.355177in}{0.413320in}}%
\pgfpathlineto{\pgfqpoint{3.352505in}{0.413320in}}%
\pgfpathlineto{\pgfqpoint{3.349828in}{0.413320in}}%
\pgfpathlineto{\pgfqpoint{3.347139in}{0.413320in}}%
\pgfpathlineto{\pgfqpoint{3.344468in}{0.413320in}}%
\pgfpathlineto{\pgfqpoint{3.341893in}{0.413320in}}%
\pgfpathlineto{\pgfqpoint{3.339101in}{0.413320in}}%
\pgfpathlineto{\pgfqpoint{3.336541in}{0.413320in}}%
\pgfpathlineto{\pgfqpoint{3.333758in}{0.413320in}}%
\pgfpathlineto{\pgfqpoint{3.331183in}{0.413320in}}%
\pgfpathlineto{\pgfqpoint{3.328401in}{0.413320in}}%
\pgfpathlineto{\pgfqpoint{3.325860in}{0.413320in}}%
\pgfpathlineto{\pgfqpoint{3.323049in}{0.413320in}}%
\pgfpathlineto{\pgfqpoint{3.320366in}{0.413320in}}%
\pgfpathlineto{\pgfqpoint{3.317688in}{0.413320in}}%
\pgfpathlineto{\pgfqpoint{3.315008in}{0.413320in}}%
\pgfpathlineto{\pgfqpoint{3.312480in}{0.413320in}}%
\pgfpathlineto{\pgfqpoint{3.309652in}{0.413320in}}%
\pgfpathlineto{\pgfqpoint{3.307104in}{0.413320in}}%
\pgfpathlineto{\pgfqpoint{3.304295in}{0.413320in}}%
\pgfpathlineto{\pgfqpoint{3.301719in}{0.413320in}}%
\pgfpathlineto{\pgfqpoint{3.298937in}{0.413320in}}%
\pgfpathlineto{\pgfqpoint{3.296376in}{0.413320in}}%
\pgfpathlineto{\pgfqpoint{3.293574in}{0.413320in}}%
\pgfpathlineto{\pgfqpoint{3.290890in}{0.413320in}}%
\pgfpathlineto{\pgfqpoint{3.288225in}{0.413320in}}%
\pgfpathlineto{\pgfqpoint{3.285534in}{0.413320in}}%
\pgfpathlineto{\pgfqpoint{3.282870in}{0.413320in}}%
\pgfpathlineto{\pgfqpoint{3.280189in}{0.413320in}}%
\pgfpathlineto{\pgfqpoint{3.277603in}{0.413320in}}%
\pgfpathlineto{\pgfqpoint{3.274831in}{0.413320in}}%
\pgfpathlineto{\pgfqpoint{3.272254in}{0.413320in}}%
\pgfpathlineto{\pgfqpoint{3.269478in}{0.413320in}}%
\pgfpathlineto{\pgfqpoint{3.266849in}{0.413320in}}%
\pgfpathlineto{\pgfqpoint{3.264119in}{0.413320in}}%
\pgfpathlineto{\pgfqpoint{3.261594in}{0.413320in}}%
\pgfpathlineto{\pgfqpoint{3.258784in}{0.413320in}}%
\pgfpathlineto{\pgfqpoint{3.256083in}{0.413320in}}%
\pgfpathlineto{\pgfqpoint{3.253404in}{0.413320in}}%
\pgfpathlineto{\pgfqpoint{3.250716in}{0.413320in}}%
\pgfpathlineto{\pgfqpoint{3.248049in}{0.413320in}}%
\pgfpathlineto{\pgfqpoint{3.245363in}{0.413320in}}%
\pgfpathlineto{\pgfqpoint{3.242807in}{0.413320in}}%
\pgfpathlineto{\pgfqpoint{3.240010in}{0.413320in}}%
\pgfpathlineto{\pgfqpoint{3.237411in}{0.413320in}}%
\pgfpathlineto{\pgfqpoint{3.234658in}{0.413320in}}%
\pgfpathlineto{\pgfqpoint{3.232069in}{0.413320in}}%
\pgfpathlineto{\pgfqpoint{3.229310in}{0.413320in}}%
\pgfpathlineto{\pgfqpoint{3.226609in}{0.413320in}}%
\pgfpathlineto{\pgfqpoint{3.223942in}{0.413320in}}%
\pgfpathlineto{\pgfqpoint{3.221255in}{0.413320in}}%
\pgfpathlineto{\pgfqpoint{3.218586in}{0.413320in}}%
\pgfpathlineto{\pgfqpoint{3.215908in}{0.413320in}}%
\pgfpathlineto{\pgfqpoint{3.213342in}{0.413320in}}%
\pgfpathlineto{\pgfqpoint{3.210545in}{0.413320in}}%
\pgfpathlineto{\pgfqpoint{3.207984in}{0.413320in}}%
\pgfpathlineto{\pgfqpoint{3.205195in}{0.413320in}}%
\pgfpathlineto{\pgfqpoint{3.202562in}{0.413320in}}%
\pgfpathlineto{\pgfqpoint{3.199823in}{0.413320in}}%
\pgfpathlineto{\pgfqpoint{3.197226in}{0.413320in}}%
\pgfpathlineto{\pgfqpoint{3.194508in}{0.413320in}}%
\pgfpathlineto{\pgfqpoint{3.191796in}{0.413320in}}%
\pgfpathlineto{\pgfqpoint{3.189117in}{0.413320in}}%
\pgfpathlineto{\pgfqpoint{3.186440in}{0.413320in}}%
\pgfpathlineto{\pgfqpoint{3.183760in}{0.413320in}}%
\pgfpathlineto{\pgfqpoint{3.181089in}{0.413320in}}%
\pgfpathlineto{\pgfqpoint{3.178525in}{0.413320in}}%
\pgfpathlineto{\pgfqpoint{3.175724in}{0.413320in}}%
\pgfpathlineto{\pgfqpoint{3.173142in}{0.413320in}}%
\pgfpathlineto{\pgfqpoint{3.170375in}{0.413320in}}%
\pgfpathlineto{\pgfqpoint{3.167776in}{0.413320in}}%
\pgfpathlineto{\pgfqpoint{3.165019in}{0.413320in}}%
\pgfpathlineto{\pgfqpoint{3.162474in}{0.413320in}}%
\pgfpathlineto{\pgfqpoint{3.159675in}{0.413320in}}%
\pgfpathlineto{\pgfqpoint{3.156981in}{0.413320in}}%
\pgfpathlineto{\pgfqpoint{3.154327in}{0.413320in}}%
\pgfpathlineto{\pgfqpoint{3.151612in}{0.413320in}}%
\pgfpathlineto{\pgfqpoint{3.149057in}{0.413320in}}%
\pgfpathlineto{\pgfqpoint{3.146271in}{0.413320in}}%
\pgfpathlineto{\pgfqpoint{3.143740in}{0.413320in}}%
\pgfpathlineto{\pgfqpoint{3.140913in}{0.413320in}}%
\pgfpathlineto{\pgfqpoint{3.138375in}{0.413320in}}%
\pgfpathlineto{\pgfqpoint{3.135550in}{0.413320in}}%
\pgfpathlineto{\pgfqpoint{3.132946in}{0.413320in}}%
\pgfpathlineto{\pgfqpoint{3.130199in}{0.413320in}}%
\pgfpathlineto{\pgfqpoint{3.127512in}{0.413320in}}%
\pgfpathlineto{\pgfqpoint{3.124842in}{0.413320in}}%
\pgfpathlineto{\pgfqpoint{3.122163in}{0.413320in}}%
\pgfpathlineto{\pgfqpoint{3.119487in}{0.413320in}}%
\pgfpathlineto{\pgfqpoint{3.116807in}{0.413320in}}%
\pgfpathlineto{\pgfqpoint{3.114242in}{0.413320in}}%
\pgfpathlineto{\pgfqpoint{3.111451in}{0.413320in}}%
\pgfpathlineto{\pgfqpoint{3.108896in}{0.413320in}}%
\pgfpathlineto{\pgfqpoint{3.106094in}{0.413320in}}%
\pgfpathlineto{\pgfqpoint{3.103508in}{0.413320in}}%
\pgfpathlineto{\pgfqpoint{3.100737in}{0.413320in}}%
\pgfpathlineto{\pgfqpoint{3.098163in}{0.413320in}}%
\pgfpathlineto{\pgfqpoint{3.095388in}{0.413320in}}%
\pgfpathlineto{\pgfqpoint{3.092699in}{0.413320in}}%
\pgfpathlineto{\pgfqpoint{3.090023in}{0.413320in}}%
\pgfpathlineto{\pgfqpoint{3.087343in}{0.413320in}}%
\pgfpathlineto{\pgfqpoint{3.084671in}{0.413320in}}%
\pgfpathlineto{\pgfqpoint{3.081990in}{0.413320in}}%
\pgfpathlineto{\pgfqpoint{3.079381in}{0.413320in}}%
\pgfpathlineto{\pgfqpoint{3.076631in}{0.413320in}}%
\pgfpathlineto{\pgfqpoint{3.074056in}{0.413320in}}%
\pgfpathlineto{\pgfqpoint{3.071266in}{0.413320in}}%
\pgfpathlineto{\pgfqpoint{3.068709in}{0.413320in}}%
\pgfpathlineto{\pgfqpoint{3.065916in}{0.413320in}}%
\pgfpathlineto{\pgfqpoint{3.063230in}{0.413320in}}%
\pgfpathlineto{\pgfqpoint{3.060561in}{0.413320in}}%
\pgfpathlineto{\pgfqpoint{3.057884in}{0.413320in}}%
\pgfpathlineto{\pgfqpoint{3.055202in}{0.413320in}}%
\pgfpathlineto{\pgfqpoint{3.052526in}{0.413320in}}%
\pgfpathlineto{\pgfqpoint{3.049988in}{0.413320in}}%
\pgfpathlineto{\pgfqpoint{3.047157in}{0.413320in}}%
\pgfpathlineto{\pgfqpoint{3.044568in}{0.413320in}}%
\pgfpathlineto{\pgfqpoint{3.041813in}{0.413320in}}%
\pgfpathlineto{\pgfqpoint{3.039262in}{0.413320in}}%
\pgfpathlineto{\pgfqpoint{3.036456in}{0.413320in}}%
\pgfpathlineto{\pgfqpoint{3.033921in}{0.413320in}}%
\pgfpathlineto{\pgfqpoint{3.031091in}{0.413320in}}%
\pgfpathlineto{\pgfqpoint{3.028412in}{0.413320in}}%
\pgfpathlineto{\pgfqpoint{3.025803in}{0.413320in}}%
\pgfpathlineto{\pgfqpoint{3.023058in}{0.413320in}}%
\pgfpathlineto{\pgfqpoint{3.020382in}{0.413320in}}%
\pgfpathlineto{\pgfqpoint{3.017707in}{0.413320in}}%
\pgfpathlineto{\pgfqpoint{3.015097in}{0.413320in}}%
\pgfpathlineto{\pgfqpoint{3.012351in}{0.413320in}}%
\pgfpathlineto{\pgfqpoint{3.009784in}{0.413320in}}%
\pgfpathlineto{\pgfqpoint{3.006993in}{0.413320in}}%
\pgfpathlineto{\pgfqpoint{3.004419in}{0.413320in}}%
\pgfpathlineto{\pgfqpoint{3.001635in}{0.413320in}}%
\pgfpathlineto{\pgfqpoint{2.999103in}{0.413320in}}%
\pgfpathlineto{\pgfqpoint{2.996300in}{0.413320in}}%
\pgfpathlineto{\pgfqpoint{2.993595in}{0.413320in}}%
\pgfpathlineto{\pgfqpoint{2.990978in}{0.413320in}}%
\pgfpathlineto{\pgfqpoint{2.988238in}{0.413320in}}%
\pgfpathlineto{\pgfqpoint{2.985666in}{0.413320in}}%
\pgfpathlineto{\pgfqpoint{2.982885in}{0.413320in}}%
\pgfpathlineto{\pgfqpoint{2.980341in}{0.413320in}}%
\pgfpathlineto{\pgfqpoint{2.977517in}{0.413320in}}%
\pgfpathlineto{\pgfqpoint{2.974972in}{0.413320in}}%
\pgfpathlineto{\pgfqpoint{2.972177in}{0.413320in}}%
\pgfpathlineto{\pgfqpoint{2.969599in}{0.413320in}}%
\pgfpathlineto{\pgfqpoint{2.966812in}{0.413320in}}%
\pgfpathlineto{\pgfqpoint{2.964127in}{0.413320in}}%
\pgfpathlineto{\pgfqpoint{2.961460in}{0.413320in}}%
\pgfpathlineto{\pgfqpoint{2.958782in}{0.413320in}}%
\pgfpathlineto{\pgfqpoint{2.956103in}{0.413320in}}%
\pgfpathlineto{\pgfqpoint{2.953422in}{0.413320in}}%
\pgfpathlineto{\pgfqpoint{2.950884in}{0.413320in}}%
\pgfpathlineto{\pgfqpoint{2.948068in}{0.413320in}}%
\pgfpathlineto{\pgfqpoint{2.945461in}{0.413320in}}%
\pgfpathlineto{\pgfqpoint{2.942711in}{0.413320in}}%
\pgfpathlineto{\pgfqpoint{2.940120in}{0.413320in}}%
\pgfpathlineto{\pgfqpoint{2.937352in}{0.413320in}}%
\pgfpathlineto{\pgfqpoint{2.934759in}{0.413320in}}%
\pgfpathlineto{\pgfqpoint{2.932033in}{0.413320in}}%
\pgfpathlineto{\pgfqpoint{2.929321in}{0.413320in}}%
\pgfpathlineto{\pgfqpoint{2.926655in}{0.413320in}}%
\pgfpathlineto{\pgfqpoint{2.923963in}{0.413320in}}%
\pgfpathlineto{\pgfqpoint{2.921363in}{0.413320in}}%
\pgfpathlineto{\pgfqpoint{2.918606in}{0.413320in}}%
\pgfpathlineto{\pgfqpoint{2.916061in}{0.413320in}}%
\pgfpathlineto{\pgfqpoint{2.913243in}{0.413320in}}%
\pgfpathlineto{\pgfqpoint{2.910631in}{0.413320in}}%
\pgfpathlineto{\pgfqpoint{2.907882in}{0.413320in}}%
\pgfpathlineto{\pgfqpoint{2.905341in}{0.413320in}}%
\pgfpathlineto{\pgfqpoint{2.902535in}{0.413320in}}%
\pgfpathlineto{\pgfqpoint{2.899858in}{0.413320in}}%
\pgfpathlineto{\pgfqpoint{2.897179in}{0.413320in}}%
\pgfpathlineto{\pgfqpoint{2.894487in}{0.413320in}}%
\pgfpathlineto{\pgfqpoint{2.891809in}{0.413320in}}%
\pgfpathlineto{\pgfqpoint{2.889145in}{0.413320in}}%
\pgfpathlineto{\pgfqpoint{2.886578in}{0.413320in}}%
\pgfpathlineto{\pgfqpoint{2.883780in}{0.413320in}}%
\pgfpathlineto{\pgfqpoint{2.881254in}{0.413320in}}%
\pgfpathlineto{\pgfqpoint{2.878431in}{0.413320in}}%
\pgfpathlineto{\pgfqpoint{2.875882in}{0.413320in}}%
\pgfpathlineto{\pgfqpoint{2.873074in}{0.413320in}}%
\pgfpathlineto{\pgfqpoint{2.870475in}{0.413320in}}%
\pgfpathlineto{\pgfqpoint{2.867713in}{0.413320in}}%
\pgfpathlineto{\pgfqpoint{2.865031in}{0.413320in}}%
\pgfpathlineto{\pgfqpoint{2.862402in}{0.413320in}}%
\pgfpathlineto{\pgfqpoint{2.859668in}{0.413320in}}%
\pgfpathlineto{\pgfqpoint{2.857003in}{0.413320in}}%
\pgfpathlineto{\pgfqpoint{2.854325in}{0.413320in}}%
\pgfpathlineto{\pgfqpoint{2.851793in}{0.413320in}}%
\pgfpathlineto{\pgfqpoint{2.848960in}{0.413320in}}%
\pgfpathlineto{\pgfqpoint{2.846408in}{0.413320in}}%
\pgfpathlineto{\pgfqpoint{2.843611in}{0.413320in}}%
\pgfpathlineto{\pgfqpoint{2.841055in}{0.413320in}}%
\pgfpathlineto{\pgfqpoint{2.838254in}{0.413320in}}%
\pgfpathlineto{\pgfqpoint{2.835698in}{0.413320in}}%
\pgfpathlineto{\pgfqpoint{2.832894in}{0.413320in}}%
\pgfpathlineto{\pgfqpoint{2.830219in}{0.413320in}}%
\pgfpathlineto{\pgfqpoint{2.827567in}{0.413320in}}%
\pgfpathlineto{\pgfqpoint{2.824851in}{0.413320in}}%
\pgfpathlineto{\pgfqpoint{2.822303in}{0.413320in}}%
\pgfpathlineto{\pgfqpoint{2.819506in}{0.413320in}}%
\pgfpathlineto{\pgfqpoint{2.816867in}{0.413320in}}%
\pgfpathlineto{\pgfqpoint{2.814141in}{0.413320in}}%
\pgfpathlineto{\pgfqpoint{2.811597in}{0.413320in}}%
\pgfpathlineto{\pgfqpoint{2.808792in}{0.413320in}}%
\pgfpathlineto{\pgfqpoint{2.806175in}{0.413320in}}%
\pgfpathlineto{\pgfqpoint{2.803435in}{0.413320in}}%
\pgfpathlineto{\pgfqpoint{2.800756in}{0.413320in}}%
\pgfpathlineto{\pgfqpoint{2.798070in}{0.413320in}}%
\pgfpathlineto{\pgfqpoint{2.795398in}{0.413320in}}%
\pgfpathlineto{\pgfqpoint{2.792721in}{0.413320in}}%
\pgfpathlineto{\pgfqpoint{2.790044in}{0.413320in}}%
\pgfpathlineto{\pgfqpoint{2.787468in}{0.413320in}}%
\pgfpathlineto{\pgfqpoint{2.784687in}{0.413320in}}%
\pgfpathlineto{\pgfqpoint{2.782113in}{0.413320in}}%
\pgfpathlineto{\pgfqpoint{2.779330in}{0.413320in}}%
\pgfpathlineto{\pgfqpoint{2.776767in}{0.413320in}}%
\pgfpathlineto{\pgfqpoint{2.773972in}{0.413320in}}%
\pgfpathlineto{\pgfqpoint{2.771373in}{0.413320in}}%
\pgfpathlineto{\pgfqpoint{2.768617in}{0.413320in}}%
\pgfpathlineto{\pgfqpoint{2.765935in}{0.413320in}}%
\pgfpathlineto{\pgfqpoint{2.763253in}{0.413320in}}%
\pgfpathlineto{\pgfqpoint{2.760581in}{0.413320in}}%
\pgfpathlineto{\pgfqpoint{2.758028in}{0.413320in}}%
\pgfpathlineto{\pgfqpoint{2.755224in}{0.413320in}}%
\pgfpathlineto{\pgfqpoint{2.752614in}{0.413320in}}%
\pgfpathlineto{\pgfqpoint{2.749868in}{0.413320in}}%
\pgfpathlineto{\pgfqpoint{2.747260in}{0.413320in}}%
\pgfpathlineto{\pgfqpoint{2.744510in}{0.413320in}}%
\pgfpathlineto{\pgfqpoint{2.741928in}{0.413320in}}%
\pgfpathlineto{\pgfqpoint{2.739155in}{0.413320in}}%
\pgfpathlineto{\pgfqpoint{2.736476in}{0.413320in}}%
\pgfpathlineto{\pgfqpoint{2.733798in}{0.413320in}}%
\pgfpathlineto{\pgfqpoint{2.731119in}{0.413320in}}%
\pgfpathlineto{\pgfqpoint{2.728439in}{0.413320in}}%
\pgfpathlineto{\pgfqpoint{2.725760in}{0.413320in}}%
\pgfpathlineto{\pgfqpoint{2.723211in}{0.413320in}}%
\pgfpathlineto{\pgfqpoint{2.720404in}{0.413320in}}%
\pgfpathlineto{\pgfqpoint{2.717773in}{0.413320in}}%
\pgfpathlineto{\pgfqpoint{2.715036in}{0.413320in}}%
\pgfpathlineto{\pgfqpoint{2.712477in}{0.413320in}}%
\pgfpathlineto{\pgfqpoint{2.709683in}{0.413320in}}%
\pgfpathlineto{\pgfqpoint{2.707125in}{0.413320in}}%
\pgfpathlineto{\pgfqpoint{2.704326in}{0.413320in}}%
\pgfpathlineto{\pgfqpoint{2.701657in}{0.413320in}}%
\pgfpathlineto{\pgfqpoint{2.698968in}{0.413320in}}%
\pgfpathlineto{\pgfqpoint{2.696293in}{0.413320in}}%
\pgfpathlineto{\pgfqpoint{2.693611in}{0.413320in}}%
\pgfpathlineto{\pgfqpoint{2.690940in}{0.413320in}}%
\pgfpathlineto{\pgfqpoint{2.688328in}{0.413320in}}%
\pgfpathlineto{\pgfqpoint{2.685586in}{0.413320in}}%
\pgfpathlineto{\pgfqpoint{2.683009in}{0.413320in}}%
\pgfpathlineto{\pgfqpoint{2.680224in}{0.413320in}}%
\pgfpathlineto{\pgfqpoint{2.677650in}{0.413320in}}%
\pgfpathlineto{\pgfqpoint{2.674873in}{0.413320in}}%
\pgfpathlineto{\pgfqpoint{2.672301in}{0.413320in}}%
\pgfpathlineto{\pgfqpoint{2.669506in}{0.413320in}}%
\pgfpathlineto{\pgfqpoint{2.666836in}{0.413320in}}%
\pgfpathlineto{\pgfqpoint{2.664151in}{0.413320in}}%
\pgfpathlineto{\pgfqpoint{2.661481in}{0.413320in}}%
\pgfpathlineto{\pgfqpoint{2.658942in}{0.413320in}}%
\pgfpathlineto{\pgfqpoint{2.656124in}{0.413320in}}%
\pgfpathlineto{\pgfqpoint{2.653567in}{0.413320in}}%
\pgfpathlineto{\pgfqpoint{2.650767in}{0.413320in}}%
\pgfpathlineto{\pgfqpoint{2.648196in}{0.413320in}}%
\pgfpathlineto{\pgfqpoint{2.645408in}{0.413320in}}%
\pgfpathlineto{\pgfqpoint{2.642827in}{0.413320in}}%
\pgfpathlineto{\pgfqpoint{2.640053in}{0.413320in}}%
\pgfpathlineto{\pgfqpoint{2.637369in}{0.413320in}}%
\pgfpathlineto{\pgfqpoint{2.634700in}{0.413320in}}%
\pgfpathlineto{\pgfqpoint{2.632018in}{0.413320in}}%
\pgfpathlineto{\pgfqpoint{2.629340in}{0.413320in}}%
\pgfpathlineto{\pgfqpoint{2.626653in}{0.413320in}}%
\pgfpathlineto{\pgfqpoint{2.624077in}{0.413320in}}%
\pgfpathlineto{\pgfqpoint{2.621304in}{0.413320in}}%
\pgfpathlineto{\pgfqpoint{2.618773in}{0.413320in}}%
\pgfpathlineto{\pgfqpoint{2.615934in}{0.413320in}}%
\pgfpathlineto{\pgfqpoint{2.613393in}{0.413320in}}%
\pgfpathlineto{\pgfqpoint{2.610588in}{0.413320in}}%
\pgfpathlineto{\pgfqpoint{2.608004in}{0.413320in}}%
\pgfpathlineto{\pgfqpoint{2.605232in}{0.413320in}}%
\pgfpathlineto{\pgfqpoint{2.602557in}{0.413320in}}%
\pgfpathlineto{\pgfqpoint{2.599920in}{0.413320in}}%
\pgfpathlineto{\pgfqpoint{2.597196in}{0.413320in}}%
\pgfpathlineto{\pgfqpoint{2.594630in}{0.413320in}}%
\pgfpathlineto{\pgfqpoint{2.591842in}{0.413320in}}%
\pgfpathlineto{\pgfqpoint{2.589248in}{0.413320in}}%
\pgfpathlineto{\pgfqpoint{2.586484in}{0.413320in}}%
\pgfpathlineto{\pgfqpoint{2.583913in}{0.413320in}}%
\pgfpathlineto{\pgfqpoint{2.581129in}{0.413320in}}%
\pgfpathlineto{\pgfqpoint{2.578567in}{0.413320in}}%
\pgfpathlineto{\pgfqpoint{2.575779in}{0.413320in}}%
\pgfpathlineto{\pgfqpoint{2.573082in}{0.413320in}}%
\pgfpathlineto{\pgfqpoint{2.570411in}{0.413320in}}%
\pgfpathlineto{\pgfqpoint{2.567730in}{0.413320in}}%
\pgfpathlineto{\pgfqpoint{2.565045in}{0.413320in}}%
\pgfpathlineto{\pgfqpoint{2.562375in}{0.413320in}}%
\pgfpathlineto{\pgfqpoint{2.559790in}{0.413320in}}%
\pgfpathlineto{\pgfqpoint{2.557009in}{0.413320in}}%
\pgfpathlineto{\pgfqpoint{2.554493in}{0.413320in}}%
\pgfpathlineto{\pgfqpoint{2.551664in}{0.413320in}}%
\pgfpathlineto{\pgfqpoint{2.549114in}{0.413320in}}%
\pgfpathlineto{\pgfqpoint{2.546310in}{0.413320in}}%
\pgfpathlineto{\pgfqpoint{2.543765in}{0.413320in}}%
\pgfpathlineto{\pgfqpoint{2.540949in}{0.413320in}}%
\pgfpathlineto{\pgfqpoint{2.538274in}{0.413320in}}%
\pgfpathlineto{\pgfqpoint{2.535624in}{0.413320in}}%
\pgfpathlineto{\pgfqpoint{2.532917in}{0.413320in}}%
\pgfpathlineto{\pgfqpoint{2.530234in}{0.413320in}}%
\pgfpathlineto{\pgfqpoint{2.527560in}{0.413320in}}%
\pgfpathlineto{\pgfqpoint{2.524988in}{0.413320in}}%
\pgfpathlineto{\pgfqpoint{2.522197in}{0.413320in}}%
\pgfpathlineto{\pgfqpoint{2.519607in}{0.413320in}}%
\pgfpathlineto{\pgfqpoint{2.516845in}{0.413320in}}%
\pgfpathlineto{\pgfqpoint{2.514268in}{0.413320in}}%
\pgfpathlineto{\pgfqpoint{2.511478in}{0.413320in}}%
\pgfpathlineto{\pgfqpoint{2.508917in}{0.413320in}}%
\pgfpathlineto{\pgfqpoint{2.506163in}{0.413320in}}%
\pgfpathlineto{\pgfqpoint{2.503454in}{0.413320in}}%
\pgfpathlineto{\pgfqpoint{2.500801in}{0.413320in}}%
\pgfpathlineto{\pgfqpoint{2.498085in}{0.413320in}}%
\pgfpathlineto{\pgfqpoint{2.495542in}{0.413320in}}%
\pgfpathlineto{\pgfqpoint{2.492729in}{0.413320in}}%
\pgfpathlineto{\pgfqpoint{2.490183in}{0.413320in}}%
\pgfpathlineto{\pgfqpoint{2.487384in}{0.413320in}}%
\pgfpathlineto{\pgfqpoint{2.484870in}{0.413320in}}%
\pgfpathlineto{\pgfqpoint{2.482026in}{0.413320in}}%
\pgfpathlineto{\pgfqpoint{2.479420in}{0.413320in}}%
\pgfpathlineto{\pgfqpoint{2.476671in}{0.413320in}}%
\pgfpathlineto{\pgfqpoint{2.473989in}{0.413320in}}%
\pgfpathlineto{\pgfqpoint{2.471311in}{0.413320in}}%
\pgfpathlineto{\pgfqpoint{2.468635in}{0.413320in}}%
\pgfpathlineto{\pgfqpoint{2.465957in}{0.413320in}}%
\pgfpathlineto{\pgfqpoint{2.463280in}{0.413320in}}%
\pgfpathlineto{\pgfqpoint{2.460711in}{0.413320in}}%
\pgfpathlineto{\pgfqpoint{2.457917in}{0.413320in}}%
\pgfpathlineto{\pgfqpoint{2.455353in}{0.413320in}}%
\pgfpathlineto{\pgfqpoint{2.452562in}{0.413320in}}%
\pgfpathlineto{\pgfqpoint{2.450032in}{0.413320in}}%
\pgfpathlineto{\pgfqpoint{2.447209in}{0.413320in}}%
\pgfpathlineto{\pgfqpoint{2.444677in}{0.413320in}}%
\pgfpathlineto{\pgfqpoint{2.441876in}{0.413320in}}%
\pgfpathlineto{\pgfqpoint{2.439167in}{0.413320in}}%
\pgfpathlineto{\pgfqpoint{2.436518in}{0.413320in}}%
\pgfpathlineto{\pgfqpoint{2.433815in}{0.413320in}}%
\pgfpathlineto{\pgfqpoint{2.431251in}{0.413320in}}%
\pgfpathlineto{\pgfqpoint{2.428453in}{0.413320in}}%
\pgfpathlineto{\pgfqpoint{2.425878in}{0.413320in}}%
\pgfpathlineto{\pgfqpoint{2.423098in}{0.413320in}}%
\pgfpathlineto{\pgfqpoint{2.420528in}{0.413320in}}%
\pgfpathlineto{\pgfqpoint{2.417747in}{0.413320in}}%
\pgfpathlineto{\pgfqpoint{2.415184in}{0.413320in}}%
\pgfpathlineto{\pgfqpoint{2.412389in}{0.413320in}}%
\pgfpathlineto{\pgfqpoint{2.409699in}{0.413320in}}%
\pgfpathlineto{\pgfqpoint{2.407024in}{0.413320in}}%
\pgfpathlineto{\pgfqpoint{2.404352in}{0.413320in}}%
\pgfpathlineto{\pgfqpoint{2.401675in}{0.413320in}}%
\pgfpathlineto{\pgfqpoint{2.398995in}{0.413320in}}%
\pgfpathclose%
\pgfusepath{stroke,fill}%
\end{pgfscope}%
\begin{pgfscope}%
\pgfpathrectangle{\pgfqpoint{2.398995in}{0.319877in}}{\pgfqpoint{3.986877in}{1.993438in}} %
\pgfusepath{clip}%
\pgfsetbuttcap%
\pgfsetroundjoin%
\definecolor{currentfill}{rgb}{1.000000,1.000000,1.000000}%
\pgfsetfillcolor{currentfill}%
\pgfsetlinewidth{1.003750pt}%
\definecolor{currentstroke}{rgb}{0.969626,0.454651,0.396033}%
\pgfsetstrokecolor{currentstroke}%
\pgfsetdash{}{0pt}%
\pgfpathmoveto{\pgfqpoint{2.398995in}{0.413320in}}%
\pgfpathlineto{\pgfqpoint{2.398995in}{0.725515in}}%
\pgfpathlineto{\pgfqpoint{2.401675in}{0.730539in}}%
\pgfpathlineto{\pgfqpoint{2.404352in}{0.727586in}}%
\pgfpathlineto{\pgfqpoint{2.407024in}{0.738797in}}%
\pgfpathlineto{\pgfqpoint{2.409699in}{0.764153in}}%
\pgfpathlineto{\pgfqpoint{2.412389in}{0.756569in}}%
\pgfpathlineto{\pgfqpoint{2.415184in}{0.749529in}}%
\pgfpathlineto{\pgfqpoint{2.417747in}{0.745452in}}%
\pgfpathlineto{\pgfqpoint{2.420528in}{0.742946in}}%
\pgfpathlineto{\pgfqpoint{2.423098in}{0.741534in}}%
\pgfpathlineto{\pgfqpoint{2.425878in}{0.735457in}}%
\pgfpathlineto{\pgfqpoint{2.428453in}{0.737101in}}%
\pgfpathlineto{\pgfqpoint{2.431251in}{0.737393in}}%
\pgfpathlineto{\pgfqpoint{2.433815in}{0.744606in}}%
\pgfpathlineto{\pgfqpoint{2.436518in}{0.746092in}}%
\pgfpathlineto{\pgfqpoint{2.439167in}{0.742840in}}%
\pgfpathlineto{\pgfqpoint{2.441876in}{0.744033in}}%
\pgfpathlineto{\pgfqpoint{2.444677in}{0.742117in}}%
\pgfpathlineto{\pgfqpoint{2.447209in}{0.743494in}}%
\pgfpathlineto{\pgfqpoint{2.450032in}{0.733663in}}%
\pgfpathlineto{\pgfqpoint{2.452562in}{0.730834in}}%
\pgfpathlineto{\pgfqpoint{2.455353in}{0.734878in}}%
\pgfpathlineto{\pgfqpoint{2.457917in}{0.732303in}}%
\pgfpathlineto{\pgfqpoint{2.460711in}{0.736456in}}%
\pgfpathlineto{\pgfqpoint{2.463280in}{0.735763in}}%
\pgfpathlineto{\pgfqpoint{2.465957in}{0.736174in}}%
\pgfpathlineto{\pgfqpoint{2.468635in}{0.732789in}}%
\pgfpathlineto{\pgfqpoint{2.471311in}{0.733856in}}%
\pgfpathlineto{\pgfqpoint{2.473989in}{0.734872in}}%
\pgfpathlineto{\pgfqpoint{2.476671in}{0.732696in}}%
\pgfpathlineto{\pgfqpoint{2.479420in}{0.731805in}}%
\pgfpathlineto{\pgfqpoint{2.482026in}{0.721225in}}%
\pgfpathlineto{\pgfqpoint{2.484870in}{0.716345in}}%
\pgfpathlineto{\pgfqpoint{2.487384in}{0.722530in}}%
\pgfpathlineto{\pgfqpoint{2.490183in}{0.722059in}}%
\pgfpathlineto{\pgfqpoint{2.492729in}{0.727008in}}%
\pgfpathlineto{\pgfqpoint{2.495542in}{0.728752in}}%
\pgfpathlineto{\pgfqpoint{2.498085in}{0.726448in}}%
\pgfpathlineto{\pgfqpoint{2.500801in}{0.727394in}}%
\pgfpathlineto{\pgfqpoint{2.503454in}{0.730921in}}%
\pgfpathlineto{\pgfqpoint{2.506163in}{0.727702in}}%
\pgfpathlineto{\pgfqpoint{2.508917in}{0.727898in}}%
\pgfpathlineto{\pgfqpoint{2.511478in}{0.723794in}}%
\pgfpathlineto{\pgfqpoint{2.514268in}{0.722676in}}%
\pgfpathlineto{\pgfqpoint{2.516845in}{0.727501in}}%
\pgfpathlineto{\pgfqpoint{2.519607in}{0.727776in}}%
\pgfpathlineto{\pgfqpoint{2.522197in}{0.723496in}}%
\pgfpathlineto{\pgfqpoint{2.524988in}{0.723420in}}%
\pgfpathlineto{\pgfqpoint{2.527560in}{0.723763in}}%
\pgfpathlineto{\pgfqpoint{2.530234in}{0.718864in}}%
\pgfpathlineto{\pgfqpoint{2.532917in}{0.726761in}}%
\pgfpathlineto{\pgfqpoint{2.535624in}{0.726021in}}%
\pgfpathlineto{\pgfqpoint{2.538274in}{0.730911in}}%
\pgfpathlineto{\pgfqpoint{2.540949in}{0.731761in}}%
\pgfpathlineto{\pgfqpoint{2.543765in}{0.727789in}}%
\pgfpathlineto{\pgfqpoint{2.546310in}{0.728835in}}%
\pgfpathlineto{\pgfqpoint{2.549114in}{0.728033in}}%
\pgfpathlineto{\pgfqpoint{2.551664in}{0.732395in}}%
\pgfpathlineto{\pgfqpoint{2.554493in}{0.736450in}}%
\pgfpathlineto{\pgfqpoint{2.557009in}{0.731948in}}%
\pgfpathlineto{\pgfqpoint{2.559790in}{0.722467in}}%
\pgfpathlineto{\pgfqpoint{2.562375in}{0.723500in}}%
\pgfpathlineto{\pgfqpoint{2.565045in}{0.729416in}}%
\pgfpathlineto{\pgfqpoint{2.567730in}{0.729113in}}%
\pgfpathlineto{\pgfqpoint{2.570411in}{0.736621in}}%
\pgfpathlineto{\pgfqpoint{2.573082in}{0.736965in}}%
\pgfpathlineto{\pgfqpoint{2.575779in}{0.727202in}}%
\pgfpathlineto{\pgfqpoint{2.578567in}{0.732025in}}%
\pgfpathlineto{\pgfqpoint{2.581129in}{0.728252in}}%
\pgfpathlineto{\pgfqpoint{2.583913in}{0.728402in}}%
\pgfpathlineto{\pgfqpoint{2.586484in}{0.731921in}}%
\pgfpathlineto{\pgfqpoint{2.589248in}{0.729766in}}%
\pgfpathlineto{\pgfqpoint{2.591842in}{0.733577in}}%
\pgfpathlineto{\pgfqpoint{2.594630in}{0.733862in}}%
\pgfpathlineto{\pgfqpoint{2.597196in}{0.729127in}}%
\pgfpathlineto{\pgfqpoint{2.599920in}{0.731796in}}%
\pgfpathlineto{\pgfqpoint{2.602557in}{0.726413in}}%
\pgfpathlineto{\pgfqpoint{2.605232in}{0.732470in}}%
\pgfpathlineto{\pgfqpoint{2.608004in}{0.731540in}}%
\pgfpathlineto{\pgfqpoint{2.610588in}{0.731175in}}%
\pgfpathlineto{\pgfqpoint{2.613393in}{0.731757in}}%
\pgfpathlineto{\pgfqpoint{2.615934in}{0.730475in}}%
\pgfpathlineto{\pgfqpoint{2.618773in}{0.732670in}}%
\pgfpathlineto{\pgfqpoint{2.621304in}{0.730033in}}%
\pgfpathlineto{\pgfqpoint{2.624077in}{0.724965in}}%
\pgfpathlineto{\pgfqpoint{2.626653in}{0.726168in}}%
\pgfpathlineto{\pgfqpoint{2.629340in}{0.725620in}}%
\pgfpathlineto{\pgfqpoint{2.632018in}{0.725229in}}%
\pgfpathlineto{\pgfqpoint{2.634700in}{0.728277in}}%
\pgfpathlineto{\pgfqpoint{2.637369in}{0.724499in}}%
\pgfpathlineto{\pgfqpoint{2.640053in}{0.726241in}}%
\pgfpathlineto{\pgfqpoint{2.642827in}{0.729885in}}%
\pgfpathlineto{\pgfqpoint{2.645408in}{0.728363in}}%
\pgfpathlineto{\pgfqpoint{2.648196in}{0.731852in}}%
\pgfpathlineto{\pgfqpoint{2.650767in}{0.736243in}}%
\pgfpathlineto{\pgfqpoint{2.653567in}{0.734089in}}%
\pgfpathlineto{\pgfqpoint{2.656124in}{0.727041in}}%
\pgfpathlineto{\pgfqpoint{2.658942in}{0.728159in}}%
\pgfpathlineto{\pgfqpoint{2.661481in}{0.730827in}}%
\pgfpathlineto{\pgfqpoint{2.664151in}{0.731795in}}%
\pgfpathlineto{\pgfqpoint{2.666836in}{0.730775in}}%
\pgfpathlineto{\pgfqpoint{2.669506in}{0.730166in}}%
\pgfpathlineto{\pgfqpoint{2.672301in}{0.728686in}}%
\pgfpathlineto{\pgfqpoint{2.674873in}{0.724155in}}%
\pgfpathlineto{\pgfqpoint{2.677650in}{0.727653in}}%
\pgfpathlineto{\pgfqpoint{2.680224in}{0.727204in}}%
\pgfpathlineto{\pgfqpoint{2.683009in}{0.724232in}}%
\pgfpathlineto{\pgfqpoint{2.685586in}{0.726522in}}%
\pgfpathlineto{\pgfqpoint{2.688328in}{0.729561in}}%
\pgfpathlineto{\pgfqpoint{2.690940in}{0.728149in}}%
\pgfpathlineto{\pgfqpoint{2.693611in}{0.727310in}}%
\pgfpathlineto{\pgfqpoint{2.696293in}{0.725390in}}%
\pgfpathlineto{\pgfqpoint{2.698968in}{0.727849in}}%
\pgfpathlineto{\pgfqpoint{2.701657in}{0.728568in}}%
\pgfpathlineto{\pgfqpoint{2.704326in}{0.729274in}}%
\pgfpathlineto{\pgfqpoint{2.707125in}{0.729894in}}%
\pgfpathlineto{\pgfqpoint{2.709683in}{0.730018in}}%
\pgfpathlineto{\pgfqpoint{2.712477in}{0.735590in}}%
\pgfpathlineto{\pgfqpoint{2.715036in}{0.731179in}}%
\pgfpathlineto{\pgfqpoint{2.717773in}{0.738168in}}%
\pgfpathlineto{\pgfqpoint{2.720404in}{0.739053in}}%
\pgfpathlineto{\pgfqpoint{2.723211in}{0.731514in}}%
\pgfpathlineto{\pgfqpoint{2.725760in}{0.730212in}}%
\pgfpathlineto{\pgfqpoint{2.728439in}{0.728556in}}%
\pgfpathlineto{\pgfqpoint{2.731119in}{0.728447in}}%
\pgfpathlineto{\pgfqpoint{2.733798in}{0.724248in}}%
\pgfpathlineto{\pgfqpoint{2.736476in}{0.716345in}}%
\pgfpathlineto{\pgfqpoint{2.739155in}{0.722777in}}%
\pgfpathlineto{\pgfqpoint{2.741928in}{0.723471in}}%
\pgfpathlineto{\pgfqpoint{2.744510in}{0.730987in}}%
\pgfpathlineto{\pgfqpoint{2.747260in}{0.730863in}}%
\pgfpathlineto{\pgfqpoint{2.749868in}{0.731944in}}%
\pgfpathlineto{\pgfqpoint{2.752614in}{0.733541in}}%
\pgfpathlineto{\pgfqpoint{2.755224in}{0.733531in}}%
\pgfpathlineto{\pgfqpoint{2.758028in}{0.733393in}}%
\pgfpathlineto{\pgfqpoint{2.760581in}{0.730073in}}%
\pgfpathlineto{\pgfqpoint{2.763253in}{0.728006in}}%
\pgfpathlineto{\pgfqpoint{2.765935in}{0.725761in}}%
\pgfpathlineto{\pgfqpoint{2.768617in}{0.726372in}}%
\pgfpathlineto{\pgfqpoint{2.771373in}{0.724258in}}%
\pgfpathlineto{\pgfqpoint{2.773972in}{0.725427in}}%
\pgfpathlineto{\pgfqpoint{2.776767in}{0.732605in}}%
\pgfpathlineto{\pgfqpoint{2.779330in}{0.729533in}}%
\pgfpathlineto{\pgfqpoint{2.782113in}{0.726172in}}%
\pgfpathlineto{\pgfqpoint{2.784687in}{0.721491in}}%
\pgfpathlineto{\pgfqpoint{2.787468in}{0.727463in}}%
\pgfpathlineto{\pgfqpoint{2.790044in}{0.727736in}}%
\pgfpathlineto{\pgfqpoint{2.792721in}{0.723661in}}%
\pgfpathlineto{\pgfqpoint{2.795398in}{0.727936in}}%
\pgfpathlineto{\pgfqpoint{2.798070in}{0.723153in}}%
\pgfpathlineto{\pgfqpoint{2.800756in}{0.723535in}}%
\pgfpathlineto{\pgfqpoint{2.803435in}{0.725154in}}%
\pgfpathlineto{\pgfqpoint{2.806175in}{0.726487in}}%
\pgfpathlineto{\pgfqpoint{2.808792in}{0.723181in}}%
\pgfpathlineto{\pgfqpoint{2.811597in}{0.724293in}}%
\pgfpathlineto{\pgfqpoint{2.814141in}{0.727790in}}%
\pgfpathlineto{\pgfqpoint{2.816867in}{0.728567in}}%
\pgfpathlineto{\pgfqpoint{2.819506in}{0.727714in}}%
\pgfpathlineto{\pgfqpoint{2.822303in}{0.727284in}}%
\pgfpathlineto{\pgfqpoint{2.824851in}{0.727546in}}%
\pgfpathlineto{\pgfqpoint{2.827567in}{0.727393in}}%
\pgfpathlineto{\pgfqpoint{2.830219in}{0.727604in}}%
\pgfpathlineto{\pgfqpoint{2.832894in}{0.728615in}}%
\pgfpathlineto{\pgfqpoint{2.835698in}{0.729735in}}%
\pgfpathlineto{\pgfqpoint{2.838254in}{0.727069in}}%
\pgfpathlineto{\pgfqpoint{2.841055in}{0.720652in}}%
\pgfpathlineto{\pgfqpoint{2.843611in}{0.718220in}}%
\pgfpathlineto{\pgfqpoint{2.846408in}{0.725971in}}%
\pgfpathlineto{\pgfqpoint{2.848960in}{0.725285in}}%
\pgfpathlineto{\pgfqpoint{2.851793in}{0.725436in}}%
\pgfpathlineto{\pgfqpoint{2.854325in}{0.726656in}}%
\pgfpathlineto{\pgfqpoint{2.857003in}{0.722184in}}%
\pgfpathlineto{\pgfqpoint{2.859668in}{0.722820in}}%
\pgfpathlineto{\pgfqpoint{2.862402in}{0.725714in}}%
\pgfpathlineto{\pgfqpoint{2.865031in}{0.724140in}}%
\pgfpathlineto{\pgfqpoint{2.867713in}{0.726852in}}%
\pgfpathlineto{\pgfqpoint{2.870475in}{0.726445in}}%
\pgfpathlineto{\pgfqpoint{2.873074in}{0.727878in}}%
\pgfpathlineto{\pgfqpoint{2.875882in}{0.727777in}}%
\pgfpathlineto{\pgfqpoint{2.878431in}{0.731509in}}%
\pgfpathlineto{\pgfqpoint{2.881254in}{0.732018in}}%
\pgfpathlineto{\pgfqpoint{2.883780in}{0.727002in}}%
\pgfpathlineto{\pgfqpoint{2.886578in}{0.734870in}}%
\pgfpathlineto{\pgfqpoint{2.889145in}{0.733545in}}%
\pgfpathlineto{\pgfqpoint{2.891809in}{0.730096in}}%
\pgfpathlineto{\pgfqpoint{2.894487in}{0.726783in}}%
\pgfpathlineto{\pgfqpoint{2.897179in}{0.730963in}}%
\pgfpathlineto{\pgfqpoint{2.899858in}{0.730662in}}%
\pgfpathlineto{\pgfqpoint{2.902535in}{0.729523in}}%
\pgfpathlineto{\pgfqpoint{2.905341in}{0.724513in}}%
\pgfpathlineto{\pgfqpoint{2.907882in}{0.724796in}}%
\pgfpathlineto{\pgfqpoint{2.910631in}{0.727536in}}%
\pgfpathlineto{\pgfqpoint{2.913243in}{0.730778in}}%
\pgfpathlineto{\pgfqpoint{2.916061in}{0.726135in}}%
\pgfpathlineto{\pgfqpoint{2.918606in}{0.727264in}}%
\pgfpathlineto{\pgfqpoint{2.921363in}{0.727285in}}%
\pgfpathlineto{\pgfqpoint{2.923963in}{0.728759in}}%
\pgfpathlineto{\pgfqpoint{2.926655in}{0.727763in}}%
\pgfpathlineto{\pgfqpoint{2.929321in}{0.729455in}}%
\pgfpathlineto{\pgfqpoint{2.932033in}{0.728389in}}%
\pgfpathlineto{\pgfqpoint{2.934759in}{0.728535in}}%
\pgfpathlineto{\pgfqpoint{2.937352in}{0.728879in}}%
\pgfpathlineto{\pgfqpoint{2.940120in}{0.723091in}}%
\pgfpathlineto{\pgfqpoint{2.942711in}{0.723264in}}%
\pgfpathlineto{\pgfqpoint{2.945461in}{0.725121in}}%
\pgfpathlineto{\pgfqpoint{2.948068in}{0.727776in}}%
\pgfpathlineto{\pgfqpoint{2.950884in}{0.726776in}}%
\pgfpathlineto{\pgfqpoint{2.953422in}{0.732548in}}%
\pgfpathlineto{\pgfqpoint{2.956103in}{0.731873in}}%
\pgfpathlineto{\pgfqpoint{2.958782in}{0.733608in}}%
\pgfpathlineto{\pgfqpoint{2.961460in}{0.729375in}}%
\pgfpathlineto{\pgfqpoint{2.964127in}{0.730585in}}%
\pgfpathlineto{\pgfqpoint{2.966812in}{0.728787in}}%
\pgfpathlineto{\pgfqpoint{2.969599in}{0.733277in}}%
\pgfpathlineto{\pgfqpoint{2.972177in}{0.729391in}}%
\pgfpathlineto{\pgfqpoint{2.974972in}{0.734429in}}%
\pgfpathlineto{\pgfqpoint{2.977517in}{0.734863in}}%
\pgfpathlineto{\pgfqpoint{2.980341in}{0.730881in}}%
\pgfpathlineto{\pgfqpoint{2.982885in}{0.732606in}}%
\pgfpathlineto{\pgfqpoint{2.985666in}{0.736733in}}%
\pgfpathlineto{\pgfqpoint{2.988238in}{0.733307in}}%
\pgfpathlineto{\pgfqpoint{2.990978in}{0.743277in}}%
\pgfpathlineto{\pgfqpoint{2.993595in}{0.746751in}}%
\pgfpathlineto{\pgfqpoint{2.996300in}{0.736278in}}%
\pgfpathlineto{\pgfqpoint{2.999103in}{0.732261in}}%
\pgfpathlineto{\pgfqpoint{3.001635in}{0.734782in}}%
\pgfpathlineto{\pgfqpoint{3.004419in}{0.729662in}}%
\pgfpathlineto{\pgfqpoint{3.006993in}{0.731809in}}%
\pgfpathlineto{\pgfqpoint{3.009784in}{0.731969in}}%
\pgfpathlineto{\pgfqpoint{3.012351in}{0.733753in}}%
\pgfpathlineto{\pgfqpoint{3.015097in}{0.731467in}}%
\pgfpathlineto{\pgfqpoint{3.017707in}{0.731774in}}%
\pgfpathlineto{\pgfqpoint{3.020382in}{0.731815in}}%
\pgfpathlineto{\pgfqpoint{3.023058in}{0.731190in}}%
\pgfpathlineto{\pgfqpoint{3.025803in}{0.732344in}}%
\pgfpathlineto{\pgfqpoint{3.028412in}{0.730990in}}%
\pgfpathlineto{\pgfqpoint{3.031091in}{0.729957in}}%
\pgfpathlineto{\pgfqpoint{3.033921in}{0.737670in}}%
\pgfpathlineto{\pgfqpoint{3.036456in}{0.743865in}}%
\pgfpathlineto{\pgfqpoint{3.039262in}{0.739971in}}%
\pgfpathlineto{\pgfqpoint{3.041813in}{0.734584in}}%
\pgfpathlineto{\pgfqpoint{3.044568in}{0.737633in}}%
\pgfpathlineto{\pgfqpoint{3.047157in}{0.732879in}}%
\pgfpathlineto{\pgfqpoint{3.049988in}{0.743464in}}%
\pgfpathlineto{\pgfqpoint{3.052526in}{0.746347in}}%
\pgfpathlineto{\pgfqpoint{3.055202in}{0.740625in}}%
\pgfpathlineto{\pgfqpoint{3.057884in}{0.737407in}}%
\pgfpathlineto{\pgfqpoint{3.060561in}{0.726556in}}%
\pgfpathlineto{\pgfqpoint{3.063230in}{0.736584in}}%
\pgfpathlineto{\pgfqpoint{3.065916in}{0.733280in}}%
\pgfpathlineto{\pgfqpoint{3.068709in}{0.728279in}}%
\pgfpathlineto{\pgfqpoint{3.071266in}{0.735999in}}%
\pgfpathlineto{\pgfqpoint{3.074056in}{0.743094in}}%
\pgfpathlineto{\pgfqpoint{3.076631in}{0.742876in}}%
\pgfpathlineto{\pgfqpoint{3.079381in}{0.732366in}}%
\pgfpathlineto{\pgfqpoint{3.081990in}{0.732581in}}%
\pgfpathlineto{\pgfqpoint{3.084671in}{0.730685in}}%
\pgfpathlineto{\pgfqpoint{3.087343in}{0.729929in}}%
\pgfpathlineto{\pgfqpoint{3.090023in}{0.732604in}}%
\pgfpathlineto{\pgfqpoint{3.092699in}{0.734886in}}%
\pgfpathlineto{\pgfqpoint{3.095388in}{0.732996in}}%
\pgfpathlineto{\pgfqpoint{3.098163in}{0.726809in}}%
\pgfpathlineto{\pgfqpoint{3.100737in}{0.722947in}}%
\pgfpathlineto{\pgfqpoint{3.103508in}{0.726350in}}%
\pgfpathlineto{\pgfqpoint{3.106094in}{0.722582in}}%
\pgfpathlineto{\pgfqpoint{3.108896in}{0.723039in}}%
\pgfpathlineto{\pgfqpoint{3.111451in}{0.725336in}}%
\pgfpathlineto{\pgfqpoint{3.114242in}{0.724216in}}%
\pgfpathlineto{\pgfqpoint{3.116807in}{0.722308in}}%
\pgfpathlineto{\pgfqpoint{3.119487in}{0.720213in}}%
\pgfpathlineto{\pgfqpoint{3.122163in}{0.716937in}}%
\pgfpathlineto{\pgfqpoint{3.124842in}{0.720970in}}%
\pgfpathlineto{\pgfqpoint{3.127512in}{0.724169in}}%
\pgfpathlineto{\pgfqpoint{3.130199in}{0.726681in}}%
\pgfpathlineto{\pgfqpoint{3.132946in}{0.727540in}}%
\pgfpathlineto{\pgfqpoint{3.135550in}{0.726789in}}%
\pgfpathlineto{\pgfqpoint{3.138375in}{0.722532in}}%
\pgfpathlineto{\pgfqpoint{3.140913in}{0.716345in}}%
\pgfpathlineto{\pgfqpoint{3.143740in}{0.716345in}}%
\pgfpathlineto{\pgfqpoint{3.146271in}{0.716345in}}%
\pgfpathlineto{\pgfqpoint{3.149057in}{0.716345in}}%
\pgfpathlineto{\pgfqpoint{3.151612in}{0.716345in}}%
\pgfpathlineto{\pgfqpoint{3.154327in}{0.716345in}}%
\pgfpathlineto{\pgfqpoint{3.156981in}{0.716345in}}%
\pgfpathlineto{\pgfqpoint{3.159675in}{0.716345in}}%
\pgfpathlineto{\pgfqpoint{3.162474in}{0.716345in}}%
\pgfpathlineto{\pgfqpoint{3.165019in}{0.716345in}}%
\pgfpathlineto{\pgfqpoint{3.167776in}{0.716345in}}%
\pgfpathlineto{\pgfqpoint{3.170375in}{0.716345in}}%
\pgfpathlineto{\pgfqpoint{3.173142in}{0.716345in}}%
\pgfpathlineto{\pgfqpoint{3.175724in}{0.716345in}}%
\pgfpathlineto{\pgfqpoint{3.178525in}{0.716345in}}%
\pgfpathlineto{\pgfqpoint{3.181089in}{0.716345in}}%
\pgfpathlineto{\pgfqpoint{3.183760in}{0.716345in}}%
\pgfpathlineto{\pgfqpoint{3.186440in}{0.716345in}}%
\pgfpathlineto{\pgfqpoint{3.189117in}{0.716345in}}%
\pgfpathlineto{\pgfqpoint{3.191796in}{0.717946in}}%
\pgfpathlineto{\pgfqpoint{3.194508in}{0.718396in}}%
\pgfpathlineto{\pgfqpoint{3.197226in}{0.716345in}}%
\pgfpathlineto{\pgfqpoint{3.199823in}{0.716345in}}%
\pgfpathlineto{\pgfqpoint{3.202562in}{0.716345in}}%
\pgfpathlineto{\pgfqpoint{3.205195in}{0.716345in}}%
\pgfpathlineto{\pgfqpoint{3.207984in}{0.716345in}}%
\pgfpathlineto{\pgfqpoint{3.210545in}{0.716345in}}%
\pgfpathlineto{\pgfqpoint{3.213342in}{0.718029in}}%
\pgfpathlineto{\pgfqpoint{3.215908in}{0.723149in}}%
\pgfpathlineto{\pgfqpoint{3.218586in}{0.723598in}}%
\pgfpathlineto{\pgfqpoint{3.221255in}{0.724640in}}%
\pgfpathlineto{\pgfqpoint{3.223942in}{0.721681in}}%
\pgfpathlineto{\pgfqpoint{3.226609in}{0.718274in}}%
\pgfpathlineto{\pgfqpoint{3.229310in}{0.718936in}}%
\pgfpathlineto{\pgfqpoint{3.232069in}{0.725853in}}%
\pgfpathlineto{\pgfqpoint{3.234658in}{0.721054in}}%
\pgfpathlineto{\pgfqpoint{3.237411in}{0.721686in}}%
\pgfpathlineto{\pgfqpoint{3.240010in}{0.725306in}}%
\pgfpathlineto{\pgfqpoint{3.242807in}{0.726954in}}%
\pgfpathlineto{\pgfqpoint{3.245363in}{0.724329in}}%
\pgfpathlineto{\pgfqpoint{3.248049in}{0.724518in}}%
\pgfpathlineto{\pgfqpoint{3.250716in}{0.726739in}}%
\pgfpathlineto{\pgfqpoint{3.253404in}{0.728486in}}%
\pgfpathlineto{\pgfqpoint{3.256083in}{0.726557in}}%
\pgfpathlineto{\pgfqpoint{3.258784in}{0.729443in}}%
\pgfpathlineto{\pgfqpoint{3.261594in}{0.727756in}}%
\pgfpathlineto{\pgfqpoint{3.264119in}{0.728183in}}%
\pgfpathlineto{\pgfqpoint{3.266849in}{0.727790in}}%
\pgfpathlineto{\pgfqpoint{3.269478in}{0.731081in}}%
\pgfpathlineto{\pgfqpoint{3.272254in}{0.728995in}}%
\pgfpathlineto{\pgfqpoint{3.274831in}{0.731805in}}%
\pgfpathlineto{\pgfqpoint{3.277603in}{0.732987in}}%
\pgfpathlineto{\pgfqpoint{3.280189in}{0.734238in}}%
\pgfpathlineto{\pgfqpoint{3.282870in}{0.733996in}}%
\pgfpathlineto{\pgfqpoint{3.285534in}{0.735760in}}%
\pgfpathlineto{\pgfqpoint{3.288225in}{0.732620in}}%
\pgfpathlineto{\pgfqpoint{3.290890in}{0.735071in}}%
\pgfpathlineto{\pgfqpoint{3.293574in}{0.731642in}}%
\pgfpathlineto{\pgfqpoint{3.296376in}{0.731610in}}%
\pgfpathlineto{\pgfqpoint{3.298937in}{0.733553in}}%
\pgfpathlineto{\pgfqpoint{3.301719in}{0.729607in}}%
\pgfpathlineto{\pgfqpoint{3.304295in}{0.729512in}}%
\pgfpathlineto{\pgfqpoint{3.307104in}{0.731150in}}%
\pgfpathlineto{\pgfqpoint{3.309652in}{0.731318in}}%
\pgfpathlineto{\pgfqpoint{3.312480in}{0.729799in}}%
\pgfpathlineto{\pgfqpoint{3.315008in}{0.728371in}}%
\pgfpathlineto{\pgfqpoint{3.317688in}{0.733373in}}%
\pgfpathlineto{\pgfqpoint{3.320366in}{0.730165in}}%
\pgfpathlineto{\pgfqpoint{3.323049in}{0.735471in}}%
\pgfpathlineto{\pgfqpoint{3.325860in}{0.729118in}}%
\pgfpathlineto{\pgfqpoint{3.328401in}{0.733006in}}%
\pgfpathlineto{\pgfqpoint{3.331183in}{0.732643in}}%
\pgfpathlineto{\pgfqpoint{3.333758in}{0.731607in}}%
\pgfpathlineto{\pgfqpoint{3.336541in}{0.736008in}}%
\pgfpathlineto{\pgfqpoint{3.339101in}{0.732037in}}%
\pgfpathlineto{\pgfqpoint{3.341893in}{0.733801in}}%
\pgfpathlineto{\pgfqpoint{3.344468in}{0.733105in}}%
\pgfpathlineto{\pgfqpoint{3.347139in}{0.731946in}}%
\pgfpathlineto{\pgfqpoint{3.349828in}{0.731010in}}%
\pgfpathlineto{\pgfqpoint{3.352505in}{0.732174in}}%
\pgfpathlineto{\pgfqpoint{3.355177in}{0.733467in}}%
\pgfpathlineto{\pgfqpoint{3.357862in}{0.733956in}}%
\pgfpathlineto{\pgfqpoint{3.360620in}{0.736300in}}%
\pgfpathlineto{\pgfqpoint{3.363221in}{0.730451in}}%
\pgfpathlineto{\pgfqpoint{3.365996in}{0.732604in}}%
\pgfpathlineto{\pgfqpoint{3.368577in}{0.730998in}}%
\pgfpathlineto{\pgfqpoint{3.371357in}{0.732868in}}%
\pgfpathlineto{\pgfqpoint{3.373921in}{0.732514in}}%
\pgfpathlineto{\pgfqpoint{3.376735in}{0.729579in}}%
\pgfpathlineto{\pgfqpoint{3.379290in}{0.731916in}}%
\pgfpathlineto{\pgfqpoint{3.381959in}{0.730504in}}%
\pgfpathlineto{\pgfqpoint{3.384647in}{0.726790in}}%
\pgfpathlineto{\pgfqpoint{3.387309in}{0.724214in}}%
\pgfpathlineto{\pgfqpoint{3.390102in}{0.729705in}}%
\pgfpathlineto{\pgfqpoint{3.392681in}{0.728515in}}%
\pgfpathlineto{\pgfqpoint{3.395461in}{0.729976in}}%
\pgfpathlineto{\pgfqpoint{3.398037in}{0.726534in}}%
\pgfpathlineto{\pgfqpoint{3.400783in}{0.732071in}}%
\pgfpathlineto{\pgfqpoint{3.403394in}{0.730610in}}%
\pgfpathlineto{\pgfqpoint{3.406202in}{0.731185in}}%
\pgfpathlineto{\pgfqpoint{3.408752in}{0.729669in}}%
\pgfpathlineto{\pgfqpoint{3.411431in}{0.731391in}}%
\pgfpathlineto{\pgfqpoint{3.414109in}{0.733751in}}%
\pgfpathlineto{\pgfqpoint{3.416780in}{0.728612in}}%
\pgfpathlineto{\pgfqpoint{3.419455in}{0.732424in}}%
\pgfpathlineto{\pgfqpoint{3.422142in}{0.734406in}}%
\pgfpathlineto{\pgfqpoint{3.424887in}{0.734938in}}%
\pgfpathlineto{\pgfqpoint{3.427501in}{0.738168in}}%
\pgfpathlineto{\pgfqpoint{3.430313in}{0.733226in}}%
\pgfpathlineto{\pgfqpoint{3.432851in}{0.736664in}}%
\pgfpathlineto{\pgfqpoint{3.435635in}{0.736569in}}%
\pgfpathlineto{\pgfqpoint{3.438210in}{0.734345in}}%
\pgfpathlineto{\pgfqpoint{3.440996in}{0.736625in}}%
\pgfpathlineto{\pgfqpoint{3.443574in}{0.736991in}}%
\pgfpathlineto{\pgfqpoint{3.446257in}{0.738121in}}%
\pgfpathlineto{\pgfqpoint{3.448926in}{0.737492in}}%
\pgfpathlineto{\pgfqpoint{3.451597in}{0.739442in}}%
\pgfpathlineto{\pgfqpoint{3.454285in}{0.737023in}}%
\pgfpathlineto{\pgfqpoint{3.456960in}{0.735267in}}%
\pgfpathlineto{\pgfqpoint{3.459695in}{0.737845in}}%
\pgfpathlineto{\pgfqpoint{3.462321in}{0.735312in}}%
\pgfpathlineto{\pgfqpoint{3.465072in}{0.734090in}}%
\pgfpathlineto{\pgfqpoint{3.467678in}{0.734804in}}%
\pgfpathlineto{\pgfqpoint{3.470466in}{0.734875in}}%
\pgfpathlineto{\pgfqpoint{3.473021in}{0.733819in}}%
\pgfpathlineto{\pgfqpoint{3.475821in}{0.729497in}}%
\pgfpathlineto{\pgfqpoint{3.478378in}{0.729838in}}%
\pgfpathlineto{\pgfqpoint{3.481072in}{0.729518in}}%
\pgfpathlineto{\pgfqpoint{3.483744in}{0.732953in}}%
\pgfpathlineto{\pgfqpoint{3.486442in}{0.730782in}}%
\pgfpathlineto{\pgfqpoint{3.489223in}{0.730797in}}%
\pgfpathlineto{\pgfqpoint{3.491783in}{0.731973in}}%
\pgfpathlineto{\pgfqpoint{3.494581in}{0.728024in}}%
\pgfpathlineto{\pgfqpoint{3.497139in}{0.731374in}}%
\pgfpathlineto{\pgfqpoint{3.499909in}{0.733396in}}%
\pgfpathlineto{\pgfqpoint{3.502488in}{0.734056in}}%
\pgfpathlineto{\pgfqpoint{3.505262in}{0.738332in}}%
\pgfpathlineto{\pgfqpoint{3.507840in}{0.741998in}}%
\pgfpathlineto{\pgfqpoint{3.510533in}{0.768434in}}%
\pgfpathlineto{\pgfqpoint{3.513209in}{0.757710in}}%
\pgfpathlineto{\pgfqpoint{3.515884in}{0.749804in}}%
\pgfpathlineto{\pgfqpoint{3.518565in}{0.738252in}}%
\pgfpathlineto{\pgfqpoint{3.521244in}{0.736819in}}%
\pgfpathlineto{\pgfqpoint{3.524041in}{0.736718in}}%
\pgfpathlineto{\pgfqpoint{3.526601in}{0.739018in}}%
\pgfpathlineto{\pgfqpoint{3.529327in}{0.738549in}}%
\pgfpathlineto{\pgfqpoint{3.531955in}{0.735377in}}%
\pgfpathlineto{\pgfqpoint{3.534783in}{0.735854in}}%
\pgfpathlineto{\pgfqpoint{3.537309in}{0.732559in}}%
\pgfpathlineto{\pgfqpoint{3.540093in}{0.733493in}}%
\pgfpathlineto{\pgfqpoint{3.542656in}{0.732996in}}%
\pgfpathlineto{\pgfqpoint{3.545349in}{0.734405in}}%
\pgfpathlineto{\pgfqpoint{3.548029in}{0.733404in}}%
\pgfpathlineto{\pgfqpoint{3.550713in}{0.729239in}}%
\pgfpathlineto{\pgfqpoint{3.553498in}{0.727970in}}%
\pgfpathlineto{\pgfqpoint{3.556061in}{0.729525in}}%
\pgfpathlineto{\pgfqpoint{3.558853in}{0.732321in}}%
\pgfpathlineto{\pgfqpoint{3.561420in}{0.732049in}}%
\pgfpathlineto{\pgfqpoint{3.564188in}{0.729803in}}%
\pgfpathlineto{\pgfqpoint{3.566774in}{0.729717in}}%
\pgfpathlineto{\pgfqpoint{3.569584in}{0.733622in}}%
\pgfpathlineto{\pgfqpoint{3.572126in}{0.734795in}}%
\pgfpathlineto{\pgfqpoint{3.574814in}{0.733088in}}%
\pgfpathlineto{\pgfqpoint{3.577487in}{0.728661in}}%
\pgfpathlineto{\pgfqpoint{3.580191in}{0.734131in}}%
\pgfpathlineto{\pgfqpoint{3.582851in}{0.731502in}}%
\pgfpathlineto{\pgfqpoint{3.585532in}{0.731241in}}%
\pgfpathlineto{\pgfqpoint{3.588258in}{0.732977in}}%
\pgfpathlineto{\pgfqpoint{3.590883in}{0.737306in}}%
\pgfpathlineto{\pgfqpoint{3.593620in}{0.738158in}}%
\pgfpathlineto{\pgfqpoint{3.596240in}{0.741053in}}%
\pgfpathlineto{\pgfqpoint{3.598998in}{0.736031in}}%
\pgfpathlineto{\pgfqpoint{3.601590in}{0.735982in}}%
\pgfpathlineto{\pgfqpoint{3.604387in}{0.734840in}}%
\pgfpathlineto{\pgfqpoint{3.606951in}{0.731339in}}%
\pgfpathlineto{\pgfqpoint{3.609632in}{0.736258in}}%
\pgfpathlineto{\pgfqpoint{3.612311in}{0.737672in}}%
\pgfpathlineto{\pgfqpoint{3.614982in}{0.734644in}}%
\pgfpathlineto{\pgfqpoint{3.617667in}{0.733616in}}%
\pgfpathlineto{\pgfqpoint{3.620345in}{0.734473in}}%
\pgfpathlineto{\pgfqpoint{3.623165in}{0.735109in}}%
\pgfpathlineto{\pgfqpoint{3.625689in}{0.734224in}}%
\pgfpathlineto{\pgfqpoint{3.628460in}{0.736255in}}%
\pgfpathlineto{\pgfqpoint{3.631058in}{0.738086in}}%
\pgfpathlineto{\pgfqpoint{3.633858in}{0.735613in}}%
\pgfpathlineto{\pgfqpoint{3.636413in}{0.736646in}}%
\pgfpathlineto{\pgfqpoint{3.639207in}{0.738485in}}%
\pgfpathlineto{\pgfqpoint{3.641773in}{0.742565in}}%
\pgfpathlineto{\pgfqpoint{3.644452in}{0.738663in}}%
\pgfpathlineto{\pgfqpoint{3.647130in}{0.737593in}}%
\pgfpathlineto{\pgfqpoint{3.649837in}{0.738782in}}%
\pgfpathlineto{\pgfqpoint{3.652628in}{0.741190in}}%
\pgfpathlineto{\pgfqpoint{3.655165in}{0.739221in}}%
\pgfpathlineto{\pgfqpoint{3.657917in}{0.758567in}}%
\pgfpathlineto{\pgfqpoint{3.660515in}{0.779527in}}%
\pgfpathlineto{\pgfqpoint{3.663276in}{0.798718in}}%
\pgfpathlineto{\pgfqpoint{3.665864in}{0.805627in}}%
\pgfpathlineto{\pgfqpoint{3.668665in}{0.791435in}}%
\pgfpathlineto{\pgfqpoint{3.671232in}{0.773772in}}%
\pgfpathlineto{\pgfqpoint{3.673911in}{0.765680in}}%
\pgfpathlineto{\pgfqpoint{3.676591in}{0.776582in}}%
\pgfpathlineto{\pgfqpoint{3.679273in}{0.798634in}}%
\pgfpathlineto{\pgfqpoint{3.681948in}{0.803738in}}%
\pgfpathlineto{\pgfqpoint{3.684620in}{0.800657in}}%
\pgfpathlineto{\pgfqpoint{3.687442in}{0.784439in}}%
\pgfpathlineto{\pgfqpoint{3.689983in}{0.775932in}}%
\pgfpathlineto{\pgfqpoint{3.692765in}{0.766322in}}%
\pgfpathlineto{\pgfqpoint{3.695331in}{0.758933in}}%
\pgfpathlineto{\pgfqpoint{3.698125in}{0.755288in}}%
\pgfpathlineto{\pgfqpoint{3.700684in}{0.749767in}}%
\pgfpathlineto{\pgfqpoint{3.703460in}{0.746559in}}%
\pgfpathlineto{\pgfqpoint{3.706053in}{0.745470in}}%
\pgfpathlineto{\pgfqpoint{3.708729in}{0.743306in}}%
\pgfpathlineto{\pgfqpoint{3.711410in}{0.744923in}}%
\pgfpathlineto{\pgfqpoint{3.714086in}{0.744383in}}%
\pgfpathlineto{\pgfqpoint{3.716875in}{0.739045in}}%
\pgfpathlineto{\pgfqpoint{3.719446in}{0.732757in}}%
\pgfpathlineto{\pgfqpoint{3.722228in}{0.728839in}}%
\pgfpathlineto{\pgfqpoint{3.724804in}{0.733440in}}%
\pgfpathlineto{\pgfqpoint{3.727581in}{0.734433in}}%
\pgfpathlineto{\pgfqpoint{3.730158in}{0.731946in}}%
\pgfpathlineto{\pgfqpoint{3.732950in}{0.733980in}}%
\pgfpathlineto{\pgfqpoint{3.735509in}{0.731319in}}%
\pgfpathlineto{\pgfqpoint{3.738194in}{0.732789in}}%
\pgfpathlineto{\pgfqpoint{3.740874in}{0.736152in}}%
\pgfpathlineto{\pgfqpoint{3.743548in}{0.733081in}}%
\pgfpathlineto{\pgfqpoint{3.746229in}{0.730251in}}%
\pgfpathlineto{\pgfqpoint{3.748903in}{0.731154in}}%
\pgfpathlineto{\pgfqpoint{3.751728in}{0.730688in}}%
\pgfpathlineto{\pgfqpoint{3.754265in}{0.732594in}}%
\pgfpathlineto{\pgfqpoint{3.757065in}{0.736361in}}%
\pgfpathlineto{\pgfqpoint{3.759622in}{0.733501in}}%
\pgfpathlineto{\pgfqpoint{3.762389in}{0.733550in}}%
\pgfpathlineto{\pgfqpoint{3.764966in}{0.745406in}}%
\pgfpathlineto{\pgfqpoint{3.767782in}{0.737140in}}%
\pgfpathlineto{\pgfqpoint{3.770323in}{0.748457in}}%
\pgfpathlineto{\pgfqpoint{3.773014in}{0.762613in}}%
\pgfpathlineto{\pgfqpoint{3.775691in}{0.754204in}}%
\pgfpathlineto{\pgfqpoint{3.778370in}{0.735983in}}%
\pgfpathlineto{\pgfqpoint{3.781046in}{0.730769in}}%
\pgfpathlineto{\pgfqpoint{3.783725in}{0.725186in}}%
\pgfpathlineto{\pgfqpoint{3.786504in}{0.729896in}}%
\pgfpathlineto{\pgfqpoint{3.789084in}{0.727664in}}%
\pgfpathlineto{\pgfqpoint{3.791897in}{0.740552in}}%
\pgfpathlineto{\pgfqpoint{3.794435in}{0.733824in}}%
\pgfpathlineto{\pgfqpoint{3.797265in}{0.739398in}}%
\pgfpathlineto{\pgfqpoint{3.799797in}{0.730539in}}%
\pgfpathlineto{\pgfqpoint{3.802569in}{0.731626in}}%
\pgfpathlineto{\pgfqpoint{3.805145in}{0.731963in}}%
\pgfpathlineto{\pgfqpoint{3.807832in}{0.731504in}}%
\pgfpathlineto{\pgfqpoint{3.810510in}{0.732517in}}%
\pgfpathlineto{\pgfqpoint{3.813172in}{0.730199in}}%
\pgfpathlineto{\pgfqpoint{3.815983in}{0.727704in}}%
\pgfpathlineto{\pgfqpoint{3.818546in}{0.727475in}}%
\pgfpathlineto{\pgfqpoint{3.821315in}{0.732080in}}%
\pgfpathlineto{\pgfqpoint{3.823903in}{0.729626in}}%
\pgfpathlineto{\pgfqpoint{3.826679in}{0.730100in}}%
\pgfpathlineto{\pgfqpoint{3.829252in}{0.733052in}}%
\pgfpathlineto{\pgfqpoint{3.832053in}{0.727934in}}%
\pgfpathlineto{\pgfqpoint{3.834616in}{0.735016in}}%
\pgfpathlineto{\pgfqpoint{3.837286in}{0.734664in}}%
\pgfpathlineto{\pgfqpoint{3.839960in}{0.738089in}}%
\pgfpathlineto{\pgfqpoint{3.842641in}{0.734609in}}%
\pgfpathlineto{\pgfqpoint{3.845329in}{0.735402in}}%
\pgfpathlineto{\pgfqpoint{3.848005in}{0.732756in}}%
\pgfpathlineto{\pgfqpoint{3.850814in}{0.727308in}}%
\pgfpathlineto{\pgfqpoint{3.853358in}{0.735598in}}%
\pgfpathlineto{\pgfqpoint{3.856100in}{0.731513in}}%
\pgfpathlineto{\pgfqpoint{3.858720in}{0.727013in}}%
\pgfpathlineto{\pgfqpoint{3.861561in}{0.726726in}}%
\pgfpathlineto{\pgfqpoint{3.864073in}{0.730433in}}%
\pgfpathlineto{\pgfqpoint{3.866815in}{0.730658in}}%
\pgfpathlineto{\pgfqpoint{3.869435in}{0.725964in}}%
\pgfpathlineto{\pgfqpoint{3.872114in}{0.729942in}}%
\pgfpathlineto{\pgfqpoint{3.874790in}{0.730824in}}%
\pgfpathlineto{\pgfqpoint{3.877466in}{0.731755in}}%
\pgfpathlineto{\pgfqpoint{3.880237in}{0.734776in}}%
\pgfpathlineto{\pgfqpoint{3.882850in}{0.733858in}}%
\pgfpathlineto{\pgfqpoint{3.885621in}{0.728747in}}%
\pgfpathlineto{\pgfqpoint{3.888188in}{0.730974in}}%
\pgfpathlineto{\pgfqpoint{3.890926in}{0.730300in}}%
\pgfpathlineto{\pgfqpoint{3.893541in}{0.722823in}}%
\pgfpathlineto{\pgfqpoint{3.896345in}{0.723909in}}%
\pgfpathlineto{\pgfqpoint{3.898891in}{0.725824in}}%
\pgfpathlineto{\pgfqpoint{3.901573in}{0.725700in}}%
\pgfpathlineto{\pgfqpoint{3.904252in}{0.729103in}}%
\pgfpathlineto{\pgfqpoint{3.906918in}{0.727881in}}%
\pgfpathlineto{\pgfqpoint{3.909602in}{0.729925in}}%
\pgfpathlineto{\pgfqpoint{3.912296in}{0.726699in}}%
\pgfpathlineto{\pgfqpoint{3.915107in}{0.730057in}}%
\pgfpathlineto{\pgfqpoint{3.917646in}{0.728275in}}%
\pgfpathlineto{\pgfqpoint{3.920412in}{0.726240in}}%
\pgfpathlineto{\pgfqpoint{3.923005in}{0.731736in}}%
\pgfpathlineto{\pgfqpoint{3.925778in}{0.728614in}}%
\pgfpathlineto{\pgfqpoint{3.928347in}{0.730933in}}%
\pgfpathlineto{\pgfqpoint{3.931202in}{0.731660in}}%
\pgfpathlineto{\pgfqpoint{3.933714in}{0.728430in}}%
\pgfpathlineto{\pgfqpoint{3.936395in}{0.730330in}}%
\pgfpathlineto{\pgfqpoint{3.939075in}{0.732252in}}%
\pgfpathlineto{\pgfqpoint{3.941778in}{0.730538in}}%
\pgfpathlineto{\pgfqpoint{3.944431in}{0.732358in}}%
\pgfpathlineto{\pgfqpoint{3.947101in}{0.734565in}}%
\pgfpathlineto{\pgfqpoint{3.949894in}{0.730722in}}%
\pgfpathlineto{\pgfqpoint{3.952464in}{0.722842in}}%
\pgfpathlineto{\pgfqpoint{3.955211in}{0.723652in}}%
\pgfpathlineto{\pgfqpoint{3.957823in}{0.728036in}}%
\pgfpathlineto{\pgfqpoint{3.960635in}{0.724741in}}%
\pgfpathlineto{\pgfqpoint{3.963176in}{0.726061in}}%
\pgfpathlineto{\pgfqpoint{3.966013in}{0.732427in}}%
\pgfpathlineto{\pgfqpoint{3.968523in}{0.732295in}}%
\pgfpathlineto{\pgfqpoint{3.971250in}{0.733244in}}%
\pgfpathlineto{\pgfqpoint{3.973885in}{0.727750in}}%
\pgfpathlineto{\pgfqpoint{3.976563in}{0.728928in}}%
\pgfpathlineto{\pgfqpoint{3.979389in}{0.727556in}}%
\pgfpathlineto{\pgfqpoint{3.981929in}{0.727833in}}%
\pgfpathlineto{\pgfqpoint{3.984714in}{0.725936in}}%
\pgfpathlineto{\pgfqpoint{3.987270in}{0.728354in}}%
\pgfpathlineto{\pgfqpoint{3.990055in}{0.729262in}}%
\pgfpathlineto{\pgfqpoint{3.992642in}{0.731183in}}%
\pgfpathlineto{\pgfqpoint{3.995417in}{0.729191in}}%
\pgfpathlineto{\pgfqpoint{3.997990in}{0.728842in}}%
\pgfpathlineto{\pgfqpoint{4.000674in}{0.729934in}}%
\pgfpathlineto{\pgfqpoint{4.003348in}{0.730739in}}%
\pgfpathlineto{\pgfqpoint{4.006034in}{0.727343in}}%
\pgfpathlineto{\pgfqpoint{4.008699in}{0.726855in}}%
\pgfpathlineto{\pgfqpoint{4.011394in}{0.731978in}}%
\pgfpathlineto{\pgfqpoint{4.014186in}{0.728381in}}%
\pgfpathlineto{\pgfqpoint{4.016744in}{0.730454in}}%
\pgfpathlineto{\pgfqpoint{4.019518in}{0.726892in}}%
\pgfpathlineto{\pgfqpoint{4.022097in}{0.729835in}}%
\pgfpathlineto{\pgfqpoint{4.024868in}{0.727614in}}%
\pgfpathlineto{\pgfqpoint{4.027447in}{0.724392in}}%
\pgfpathlineto{\pgfqpoint{4.030229in}{0.730410in}}%
\pgfpathlineto{\pgfqpoint{4.032817in}{0.734314in}}%
\pgfpathlineto{\pgfqpoint{4.035492in}{0.750827in}}%
\pgfpathlineto{\pgfqpoint{4.038174in}{0.742319in}}%
\pgfpathlineto{\pgfqpoint{4.040852in}{0.738403in}}%
\pgfpathlineto{\pgfqpoint{4.043667in}{0.734741in}}%
\pgfpathlineto{\pgfqpoint{4.046210in}{0.733157in}}%
\pgfpathlineto{\pgfqpoint{4.049006in}{0.735409in}}%
\pgfpathlineto{\pgfqpoint{4.051557in}{0.731192in}}%
\pgfpathlineto{\pgfqpoint{4.054326in}{0.727124in}}%
\pgfpathlineto{\pgfqpoint{4.056911in}{0.734527in}}%
\pgfpathlineto{\pgfqpoint{4.059702in}{0.733735in}}%
\pgfpathlineto{\pgfqpoint{4.062266in}{0.737181in}}%
\pgfpathlineto{\pgfqpoint{4.064957in}{0.730281in}}%
\pgfpathlineto{\pgfqpoint{4.067636in}{0.730377in}}%
\pgfpathlineto{\pgfqpoint{4.070313in}{0.737395in}}%
\pgfpathlineto{\pgfqpoint{4.072985in}{0.731840in}}%
\pgfpathlineto{\pgfqpoint{4.075705in}{0.734387in}}%
\pgfpathlineto{\pgfqpoint{4.078471in}{0.735866in}}%
\pgfpathlineto{\pgfqpoint{4.081018in}{0.736322in}}%
\pgfpathlineto{\pgfqpoint{4.083870in}{0.736067in}}%
\pgfpathlineto{\pgfqpoint{4.086385in}{0.734595in}}%
\pgfpathlineto{\pgfqpoint{4.089159in}{0.731076in}}%
\pgfpathlineto{\pgfqpoint{4.091729in}{0.736967in}}%
\pgfpathlineto{\pgfqpoint{4.094527in}{0.731335in}}%
\pgfpathlineto{\pgfqpoint{4.097092in}{0.729979in}}%
\pgfpathlineto{\pgfqpoint{4.099777in}{0.733756in}}%
\pgfpathlineto{\pgfqpoint{4.102456in}{0.731301in}}%
\pgfpathlineto{\pgfqpoint{4.105185in}{0.726409in}}%
\pgfpathlineto{\pgfqpoint{4.107814in}{0.727641in}}%
\pgfpathlineto{\pgfqpoint{4.110488in}{0.731642in}}%
\pgfpathlineto{\pgfqpoint{4.113252in}{0.729990in}}%
\pgfpathlineto{\pgfqpoint{4.115844in}{0.734571in}}%
\pgfpathlineto{\pgfqpoint{4.118554in}{0.737141in}}%
\pgfpathlineto{\pgfqpoint{4.121205in}{0.732749in}}%
\pgfpathlineto{\pgfqpoint{4.124019in}{0.731818in}}%
\pgfpathlineto{\pgfqpoint{4.126553in}{0.730523in}}%
\pgfpathlineto{\pgfqpoint{4.129349in}{0.732080in}}%
\pgfpathlineto{\pgfqpoint{4.131920in}{0.730017in}}%
\pgfpathlineto{\pgfqpoint{4.134615in}{0.735646in}}%
\pgfpathlineto{\pgfqpoint{4.137272in}{0.733044in}}%
\pgfpathlineto{\pgfqpoint{4.139963in}{0.733072in}}%
\pgfpathlineto{\pgfqpoint{4.142713in}{0.729897in}}%
\pgfpathlineto{\pgfqpoint{4.145310in}{0.731039in}}%
\pgfpathlineto{\pgfqpoint{4.148082in}{0.731356in}}%
\pgfpathlineto{\pgfqpoint{4.150665in}{0.732876in}}%
\pgfpathlineto{\pgfqpoint{4.153423in}{0.731952in}}%
\pgfpathlineto{\pgfqpoint{4.156016in}{0.730659in}}%
\pgfpathlineto{\pgfqpoint{4.158806in}{0.730355in}}%
\pgfpathlineto{\pgfqpoint{4.161380in}{0.734244in}}%
\pgfpathlineto{\pgfqpoint{4.164059in}{0.729856in}}%
\pgfpathlineto{\pgfqpoint{4.166737in}{0.727278in}}%
\pgfpathlineto{\pgfqpoint{4.169415in}{0.727918in}}%
\pgfpathlineto{\pgfqpoint{4.172093in}{0.730528in}}%
\pgfpathlineto{\pgfqpoint{4.174770in}{0.730540in}}%
\pgfpathlineto{\pgfqpoint{4.177593in}{0.728239in}}%
\pgfpathlineto{\pgfqpoint{4.180129in}{0.725801in}}%
\pgfpathlineto{\pgfqpoint{4.182899in}{0.722767in}}%
\pgfpathlineto{\pgfqpoint{4.185481in}{0.724226in}}%
\pgfpathlineto{\pgfqpoint{4.188318in}{0.726493in}}%
\pgfpathlineto{\pgfqpoint{4.190842in}{0.727657in}}%
\pgfpathlineto{\pgfqpoint{4.193638in}{0.730192in}}%
\pgfpathlineto{\pgfqpoint{4.196186in}{0.724824in}}%
\pgfpathlineto{\pgfqpoint{4.198878in}{0.726385in}}%
\pgfpathlineto{\pgfqpoint{4.201542in}{0.728359in}}%
\pgfpathlineto{\pgfqpoint{4.204240in}{0.722075in}}%
\pgfpathlineto{\pgfqpoint{4.207076in}{0.720341in}}%
\pgfpathlineto{\pgfqpoint{4.209597in}{0.726094in}}%
\pgfpathlineto{\pgfqpoint{4.212383in}{0.721866in}}%
\pgfpathlineto{\pgfqpoint{4.214948in}{0.725451in}}%
\pgfpathlineto{\pgfqpoint{4.217694in}{0.725878in}}%
\pgfpathlineto{\pgfqpoint{4.220304in}{0.727038in}}%
\pgfpathlineto{\pgfqpoint{4.223082in}{0.723966in}}%
\pgfpathlineto{\pgfqpoint{4.225654in}{0.724331in}}%
\pgfpathlineto{\pgfqpoint{4.228331in}{0.734744in}}%
\pgfpathlineto{\pgfqpoint{4.231013in}{0.731522in}}%
\pgfpathlineto{\pgfqpoint{4.233691in}{0.730553in}}%
\pgfpathlineto{\pgfqpoint{4.236375in}{0.729007in}}%
\pgfpathlineto{\pgfqpoint{4.239084in}{0.726816in}}%
\pgfpathlineto{\pgfqpoint{4.241900in}{0.728427in}}%
\pgfpathlineto{\pgfqpoint{4.244394in}{0.727806in}}%
\pgfpathlineto{\pgfqpoint{4.247225in}{0.728795in}}%
\pgfpathlineto{\pgfqpoint{4.249767in}{0.727239in}}%
\pgfpathlineto{\pgfqpoint{4.252581in}{0.728009in}}%
\pgfpathlineto{\pgfqpoint{4.255120in}{0.726167in}}%
\pgfpathlineto{\pgfqpoint{4.257958in}{0.730230in}}%
\pgfpathlineto{\pgfqpoint{4.260477in}{0.729593in}}%
\pgfpathlineto{\pgfqpoint{4.263157in}{0.729509in}}%
\pgfpathlineto{\pgfqpoint{4.265824in}{0.729936in}}%
\pgfpathlineto{\pgfqpoint{4.268590in}{0.730766in}}%
\pgfpathlineto{\pgfqpoint{4.271187in}{0.731380in}}%
\pgfpathlineto{\pgfqpoint{4.273874in}{0.727333in}}%
\pgfpathlineto{\pgfqpoint{4.276635in}{0.729357in}}%
\pgfpathlineto{\pgfqpoint{4.279212in}{0.733895in}}%
\pgfpathlineto{\pgfqpoint{4.282000in}{0.737076in}}%
\pgfpathlineto{\pgfqpoint{4.284586in}{0.745448in}}%
\pgfpathlineto{\pgfqpoint{4.287399in}{0.741184in}}%
\pgfpathlineto{\pgfqpoint{4.289936in}{0.741794in}}%
\pgfpathlineto{\pgfqpoint{4.292786in}{0.736602in}}%
\pgfpathlineto{\pgfqpoint{4.295299in}{0.730905in}}%
\pgfpathlineto{\pgfqpoint{4.297977in}{0.739800in}}%
\pgfpathlineto{\pgfqpoint{4.300656in}{0.737885in}}%
\pgfpathlineto{\pgfqpoint{4.303357in}{0.738826in}}%
\pgfpathlineto{\pgfqpoint{4.306118in}{0.737298in}}%
\pgfpathlineto{\pgfqpoint{4.308691in}{0.735115in}}%
\pgfpathlineto{\pgfqpoint{4.311494in}{0.732638in}}%
\pgfpathlineto{\pgfqpoint{4.314032in}{0.733316in}}%
\pgfpathlineto{\pgfqpoint{4.316856in}{0.731941in}}%
\pgfpathlineto{\pgfqpoint{4.319405in}{0.729027in}}%
\pgfpathlineto{\pgfqpoint{4.322181in}{0.731871in}}%
\pgfpathlineto{\pgfqpoint{4.324760in}{0.736740in}}%
\pgfpathlineto{\pgfqpoint{4.327440in}{0.739554in}}%
\pgfpathlineto{\pgfqpoint{4.330118in}{0.737526in}}%
\pgfpathlineto{\pgfqpoint{4.332796in}{0.737929in}}%
\pgfpathlineto{\pgfqpoint{4.335463in}{0.747921in}}%
\pgfpathlineto{\pgfqpoint{4.338154in}{0.750500in}}%
\pgfpathlineto{\pgfqpoint{4.340976in}{0.751962in}}%
\pgfpathlineto{\pgfqpoint{4.343510in}{0.743796in}}%
\pgfpathlineto{\pgfqpoint{4.346263in}{0.736829in}}%
\pgfpathlineto{\pgfqpoint{4.348868in}{0.734914in}}%
\pgfpathlineto{\pgfqpoint{4.351645in}{0.733793in}}%
\pgfpathlineto{\pgfqpoint{4.354224in}{0.736318in}}%
\pgfpathlineto{\pgfqpoint{4.357014in}{0.736655in}}%
\pgfpathlineto{\pgfqpoint{4.359582in}{0.734399in}}%
\pgfpathlineto{\pgfqpoint{4.362270in}{0.732691in}}%
\pgfpathlineto{\pgfqpoint{4.364936in}{0.739147in}}%
\pgfpathlineto{\pgfqpoint{4.367646in}{0.738297in}}%
\pgfpathlineto{\pgfqpoint{4.370437in}{0.734461in}}%
\pgfpathlineto{\pgfqpoint{4.372976in}{0.733166in}}%
\pgfpathlineto{\pgfqpoint{4.375761in}{0.737643in}}%
\pgfpathlineto{\pgfqpoint{4.378329in}{0.741439in}}%
\pgfpathlineto{\pgfqpoint{4.381097in}{0.741772in}}%
\pgfpathlineto{\pgfqpoint{4.383674in}{0.741027in}}%
\pgfpathlineto{\pgfqpoint{4.386431in}{0.737105in}}%
\pgfpathlineto{\pgfqpoint{4.389044in}{0.732254in}}%
\pgfpathlineto{\pgfqpoint{4.391721in}{0.731678in}}%
\pgfpathlineto{\pgfqpoint{4.394400in}{0.728352in}}%
\pgfpathlineto{\pgfqpoint{4.397076in}{0.733014in}}%
\pgfpathlineto{\pgfqpoint{4.399745in}{0.741671in}}%
\pgfpathlineto{\pgfqpoint{4.402468in}{0.747970in}}%
\pgfpathlineto{\pgfqpoint{4.405234in}{0.749872in}}%
\pgfpathlineto{\pgfqpoint{4.407788in}{0.746352in}}%
\pgfpathlineto{\pgfqpoint{4.410587in}{0.741692in}}%
\pgfpathlineto{\pgfqpoint{4.413149in}{0.738399in}}%
\pgfpathlineto{\pgfqpoint{4.415932in}{0.738022in}}%
\pgfpathlineto{\pgfqpoint{4.418506in}{0.739743in}}%
\pgfpathlineto{\pgfqpoint{4.421292in}{0.738874in}}%
\pgfpathlineto{\pgfqpoint{4.423863in}{0.739789in}}%
\pgfpathlineto{\pgfqpoint{4.426534in}{0.737286in}}%
\pgfpathlineto{\pgfqpoint{4.429220in}{0.736372in}}%
\pgfpathlineto{\pgfqpoint{4.431901in}{0.736693in}}%
\pgfpathlineto{\pgfqpoint{4.434569in}{0.737756in}}%
\pgfpathlineto{\pgfqpoint{4.437253in}{0.735349in}}%
\pgfpathlineto{\pgfqpoint{4.440041in}{0.737243in}}%
\pgfpathlineto{\pgfqpoint{4.442611in}{0.737406in}}%
\pgfpathlineto{\pgfqpoint{4.445423in}{0.735736in}}%
\pgfpathlineto{\pgfqpoint{4.447965in}{0.748721in}}%
\pgfpathlineto{\pgfqpoint{4.450767in}{0.759451in}}%
\pgfpathlineto{\pgfqpoint{4.453312in}{0.751065in}}%
\pgfpathlineto{\pgfqpoint{4.456138in}{0.739856in}}%
\pgfpathlineto{\pgfqpoint{4.458681in}{0.737679in}}%
\pgfpathlineto{\pgfqpoint{4.461367in}{0.738977in}}%
\pgfpathlineto{\pgfqpoint{4.464029in}{0.733980in}}%
\pgfpathlineto{\pgfqpoint{4.466717in}{0.731957in}}%
\pgfpathlineto{\pgfqpoint{4.469492in}{0.729172in}}%
\pgfpathlineto{\pgfqpoint{4.472059in}{0.732450in}}%
\pgfpathlineto{\pgfqpoint{4.474861in}{0.731478in}}%
\pgfpathlineto{\pgfqpoint{4.477430in}{0.732173in}}%
\pgfpathlineto{\pgfqpoint{4.480201in}{0.729970in}}%
\pgfpathlineto{\pgfqpoint{4.482778in}{0.729904in}}%
\pgfpathlineto{\pgfqpoint{4.485581in}{0.727126in}}%
\pgfpathlineto{\pgfqpoint{4.488130in}{0.724643in}}%
\pgfpathlineto{\pgfqpoint{4.490822in}{0.728934in}}%
\pgfpathlineto{\pgfqpoint{4.493492in}{0.728611in}}%
\pgfpathlineto{\pgfqpoint{4.496167in}{0.729554in}}%
\pgfpathlineto{\pgfqpoint{4.498850in}{0.727893in}}%
\pgfpathlineto{\pgfqpoint{4.501529in}{0.730433in}}%
\pgfpathlineto{\pgfqpoint{4.504305in}{0.731615in}}%
\pgfpathlineto{\pgfqpoint{4.506893in}{0.735772in}}%
\pgfpathlineto{\pgfqpoint{4.509643in}{0.734859in}}%
\pgfpathlineto{\pgfqpoint{4.512246in}{0.731739in}}%
\pgfpathlineto{\pgfqpoint{4.515080in}{0.726870in}}%
\pgfpathlineto{\pgfqpoint{4.517598in}{0.726995in}}%
\pgfpathlineto{\pgfqpoint{4.520345in}{0.727800in}}%
\pgfpathlineto{\pgfqpoint{4.522962in}{0.724393in}}%
\pgfpathlineto{\pgfqpoint{4.525640in}{0.730030in}}%
\pgfpathlineto{\pgfqpoint{4.528307in}{0.734085in}}%
\pgfpathlineto{\pgfqpoint{4.530990in}{0.732297in}}%
\pgfpathlineto{\pgfqpoint{4.533764in}{0.736450in}}%
\pgfpathlineto{\pgfqpoint{4.536400in}{0.764208in}}%
\pgfpathlineto{\pgfqpoint{4.539144in}{0.752940in}}%
\pgfpathlineto{\pgfqpoint{4.541711in}{0.742664in}}%
\pgfpathlineto{\pgfqpoint{4.544464in}{0.737764in}}%
\pgfpathlineto{\pgfqpoint{4.547064in}{0.734007in}}%
\pgfpathlineto{\pgfqpoint{4.549822in}{0.734818in}}%
\pgfpathlineto{\pgfqpoint{4.552425in}{0.733292in}}%
\pgfpathlineto{\pgfqpoint{4.555106in}{0.732987in}}%
\pgfpathlineto{\pgfqpoint{4.557777in}{0.734088in}}%
\pgfpathlineto{\pgfqpoint{4.560448in}{0.736581in}}%
\pgfpathlineto{\pgfqpoint{4.563125in}{0.736689in}}%
\pgfpathlineto{\pgfqpoint{4.565820in}{0.732727in}}%
\pgfpathlineto{\pgfqpoint{4.568612in}{0.734288in}}%
\pgfpathlineto{\pgfqpoint{4.571171in}{0.731455in}}%
\pgfpathlineto{\pgfqpoint{4.573947in}{0.730306in}}%
\pgfpathlineto{\pgfqpoint{4.576531in}{0.733636in}}%
\pgfpathlineto{\pgfqpoint{4.579305in}{0.729086in}}%
\pgfpathlineto{\pgfqpoint{4.581888in}{0.730918in}}%
\pgfpathlineto{\pgfqpoint{4.584672in}{0.731484in}}%
\pgfpathlineto{\pgfqpoint{4.587244in}{0.728190in}}%
\pgfpathlineto{\pgfqpoint{4.589920in}{0.728975in}}%
\pgfpathlineto{\pgfqpoint{4.592589in}{0.722782in}}%
\pgfpathlineto{\pgfqpoint{4.595281in}{0.719839in}}%
\pgfpathlineto{\pgfqpoint{4.597951in}{0.722778in}}%
\pgfpathlineto{\pgfqpoint{4.600633in}{0.724178in}}%
\pgfpathlineto{\pgfqpoint{4.603430in}{0.721065in}}%
\pgfpathlineto{\pgfqpoint{4.605990in}{0.728636in}}%
\pgfpathlineto{\pgfqpoint{4.608808in}{0.727789in}}%
\pgfpathlineto{\pgfqpoint{4.611350in}{0.732165in}}%
\pgfpathlineto{\pgfqpoint{4.614134in}{0.729206in}}%
\pgfpathlineto{\pgfqpoint{4.616702in}{0.724515in}}%
\pgfpathlineto{\pgfqpoint{4.619529in}{0.727722in}}%
\pgfpathlineto{\pgfqpoint{4.622056in}{0.731195in}}%
\pgfpathlineto{\pgfqpoint{4.624741in}{0.730173in}}%
\pgfpathlineto{\pgfqpoint{4.627411in}{0.729436in}}%
\pgfpathlineto{\pgfqpoint{4.630096in}{0.726372in}}%
\pgfpathlineto{\pgfqpoint{4.632902in}{0.725990in}}%
\pgfpathlineto{\pgfqpoint{4.635445in}{0.728307in}}%
\pgfpathlineto{\pgfqpoint{4.638204in}{0.726284in}}%
\pgfpathlineto{\pgfqpoint{4.640809in}{0.725864in}}%
\pgfpathlineto{\pgfqpoint{4.643628in}{0.726199in}}%
\pgfpathlineto{\pgfqpoint{4.646169in}{0.728171in}}%
\pgfpathlineto{\pgfqpoint{4.648922in}{0.728409in}}%
\pgfpathlineto{\pgfqpoint{4.651524in}{0.730413in}}%
\pgfpathlineto{\pgfqpoint{4.654203in}{0.728759in}}%
\pgfpathlineto{\pgfqpoint{4.656873in}{0.732107in}}%
\pgfpathlineto{\pgfqpoint{4.659590in}{0.728328in}}%
\pgfpathlineto{\pgfqpoint{4.662237in}{0.725495in}}%
\pgfpathlineto{\pgfqpoint{4.664923in}{0.727968in}}%
\pgfpathlineto{\pgfqpoint{4.667764in}{0.730044in}}%
\pgfpathlineto{\pgfqpoint{4.670261in}{0.745588in}}%
\pgfpathlineto{\pgfqpoint{4.673068in}{0.737278in}}%
\pgfpathlineto{\pgfqpoint{4.675619in}{0.736889in}}%
\pgfpathlineto{\pgfqpoint{4.678448in}{0.735110in}}%
\pgfpathlineto{\pgfqpoint{4.680988in}{0.735669in}}%
\pgfpathlineto{\pgfqpoint{4.683799in}{0.733959in}}%
\pgfpathlineto{\pgfqpoint{4.686337in}{0.728779in}}%
\pgfpathlineto{\pgfqpoint{4.689051in}{0.734668in}}%
\pgfpathlineto{\pgfqpoint{4.691694in}{0.733579in}}%
\pgfpathlineto{\pgfqpoint{4.694381in}{0.733589in}}%
\pgfpathlineto{\pgfqpoint{4.697170in}{0.733351in}}%
\pgfpathlineto{\pgfqpoint{4.699734in}{0.735298in}}%
\pgfpathlineto{\pgfqpoint{4.702517in}{0.735143in}}%
\pgfpathlineto{\pgfqpoint{4.705094in}{0.729754in}}%
\pgfpathlineto{\pgfqpoint{4.707824in}{0.732025in}}%
\pgfpathlineto{\pgfqpoint{4.710437in}{0.725644in}}%
\pgfpathlineto{\pgfqpoint{4.713275in}{0.726292in}}%
\pgfpathlineto{\pgfqpoint{4.715806in}{0.725146in}}%
\pgfpathlineto{\pgfqpoint{4.718486in}{0.729154in}}%
\pgfpathlineto{\pgfqpoint{4.721160in}{0.730736in}}%
\pgfpathlineto{\pgfqpoint{4.723873in}{0.733298in}}%
\pgfpathlineto{\pgfqpoint{4.726508in}{0.740308in}}%
\pgfpathlineto{\pgfqpoint{4.729233in}{0.736712in}}%
\pgfpathlineto{\pgfqpoint{4.731901in}{0.733376in}}%
\pgfpathlineto{\pgfqpoint{4.734552in}{0.729706in}}%
\pgfpathlineto{\pgfqpoint{4.737348in}{0.731090in}}%
\pgfpathlineto{\pgfqpoint{4.739912in}{0.730385in}}%
\pgfpathlineto{\pgfqpoint{4.742696in}{0.734725in}}%
\pgfpathlineto{\pgfqpoint{4.745256in}{0.734420in}}%
\pgfpathlineto{\pgfqpoint{4.748081in}{0.736056in}}%
\pgfpathlineto{\pgfqpoint{4.750627in}{0.729915in}}%
\pgfpathlineto{\pgfqpoint{4.753298in}{0.732730in}}%
\pgfpathlineto{\pgfqpoint{4.755983in}{0.732166in}}%
\pgfpathlineto{\pgfqpoint{4.758653in}{0.729859in}}%
\pgfpathlineto{\pgfqpoint{4.761337in}{0.729083in}}%
\pgfpathlineto{\pgfqpoint{4.764018in}{0.725559in}}%
\pgfpathlineto{\pgfqpoint{4.766783in}{0.728839in}}%
\pgfpathlineto{\pgfqpoint{4.769367in}{0.729511in}}%
\pgfpathlineto{\pgfqpoint{4.772198in}{0.732364in}}%
\pgfpathlineto{\pgfqpoint{4.774732in}{0.737125in}}%
\pgfpathlineto{\pgfqpoint{4.777535in}{0.735401in}}%
\pgfpathlineto{\pgfqpoint{4.780083in}{0.733967in}}%
\pgfpathlineto{\pgfqpoint{4.782872in}{0.741890in}}%
\pgfpathlineto{\pgfqpoint{4.785445in}{0.734501in}}%
\pgfpathlineto{\pgfqpoint{4.788116in}{0.732865in}}%
\pgfpathlineto{\pgfqpoint{4.790798in}{0.731293in}}%
\pgfpathlineto{\pgfqpoint{4.793512in}{0.727860in}}%
\pgfpathlineto{\pgfqpoint{4.796274in}{0.730101in}}%
\pgfpathlineto{\pgfqpoint{4.798830in}{0.730174in}}%
\pgfpathlineto{\pgfqpoint{4.801586in}{0.730637in}}%
\pgfpathlineto{\pgfqpoint{4.804193in}{0.730676in}}%
\pgfpathlineto{\pgfqpoint{4.807017in}{0.730666in}}%
\pgfpathlineto{\pgfqpoint{4.809538in}{0.729304in}}%
\pgfpathlineto{\pgfqpoint{4.812377in}{0.725891in}}%
\pgfpathlineto{\pgfqpoint{4.814907in}{0.726087in}}%
\pgfpathlineto{\pgfqpoint{4.817587in}{0.726250in}}%
\pgfpathlineto{\pgfqpoint{4.820265in}{0.723524in}}%
\pgfpathlineto{\pgfqpoint{4.822945in}{0.726443in}}%
\pgfpathlineto{\pgfqpoint{4.825619in}{0.723122in}}%
\pgfpathlineto{\pgfqpoint{4.828291in}{0.725974in}}%
\pgfpathlineto{\pgfqpoint{4.831045in}{0.725654in}}%
\pgfpathlineto{\pgfqpoint{4.833657in}{0.722471in}}%
\pgfpathlineto{\pgfqpoint{4.837992in}{0.723829in}}%
\pgfpathlineto{\pgfqpoint{4.839922in}{0.728577in}}%
\pgfpathlineto{\pgfqpoint{4.842380in}{0.730511in}}%
\pgfpathlineto{\pgfqpoint{4.844361in}{0.731618in}}%
\pgfpathlineto{\pgfqpoint{4.847127in}{0.727928in}}%
\pgfpathlineto{\pgfqpoint{4.849715in}{0.723921in}}%
\pgfpathlineto{\pgfqpoint{4.852404in}{0.723847in}}%
\pgfpathlineto{\pgfqpoint{4.855070in}{0.724453in}}%
\pgfpathlineto{\pgfqpoint{4.857807in}{0.728608in}}%
\pgfpathlineto{\pgfqpoint{4.860544in}{0.727156in}}%
\pgfpathlineto{\pgfqpoint{4.863116in}{0.731269in}}%
\pgfpathlineto{\pgfqpoint{4.865910in}{0.728813in}}%
\pgfpathlineto{\pgfqpoint{4.868474in}{0.733309in}}%
\pgfpathlineto{\pgfqpoint{4.871209in}{0.734462in}}%
\pgfpathlineto{\pgfqpoint{4.873832in}{0.734757in}}%
\pgfpathlineto{\pgfqpoint{4.876636in}{0.735967in}}%
\pgfpathlineto{\pgfqpoint{4.879180in}{0.734415in}}%
\pgfpathlineto{\pgfqpoint{4.881864in}{0.730730in}}%
\pgfpathlineto{\pgfqpoint{4.884540in}{0.733544in}}%
\pgfpathlineto{\pgfqpoint{4.887211in}{0.731910in}}%
\pgfpathlineto{\pgfqpoint{4.889902in}{0.736339in}}%
\pgfpathlineto{\pgfqpoint{4.892611in}{0.734902in}}%
\pgfpathlineto{\pgfqpoint{4.895399in}{0.736021in}}%
\pgfpathlineto{\pgfqpoint{4.897938in}{0.736424in}}%
\pgfpathlineto{\pgfqpoint{4.900712in}{0.735505in}}%
\pgfpathlineto{\pgfqpoint{4.903295in}{0.732873in}}%
\pgfpathlineto{\pgfqpoint{4.906096in}{0.725461in}}%
\pgfpathlineto{\pgfqpoint{4.908648in}{0.731618in}}%
\pgfpathlineto{\pgfqpoint{4.911435in}{0.746334in}}%
\pgfpathlineto{\pgfqpoint{4.914009in}{0.736324in}}%
\pgfpathlineto{\pgfqpoint{4.916681in}{0.735985in}}%
\pgfpathlineto{\pgfqpoint{4.919352in}{0.735377in}}%
\pgfpathlineto{\pgfqpoint{4.922041in}{0.729904in}}%
\pgfpathlineto{\pgfqpoint{4.924708in}{0.731969in}}%
\pgfpathlineto{\pgfqpoint{4.927400in}{0.735116in}}%
\pgfpathlineto{\pgfqpoint{4.930170in}{0.730063in}}%
\pgfpathlineto{\pgfqpoint{4.932742in}{0.731768in}}%
\pgfpathlineto{\pgfqpoint{4.935515in}{0.730667in}}%
\pgfpathlineto{\pgfqpoint{4.938112in}{0.725999in}}%
\pgfpathlineto{\pgfqpoint{4.940881in}{0.727535in}}%
\pgfpathlineto{\pgfqpoint{4.943466in}{0.731559in}}%
\pgfpathlineto{\pgfqpoint{4.946151in}{0.732594in}}%
\pgfpathlineto{\pgfqpoint{4.948827in}{0.729533in}}%
\pgfpathlineto{\pgfqpoint{4.951504in}{0.734259in}}%
\pgfpathlineto{\pgfqpoint{4.954182in}{0.731441in}}%
\pgfpathlineto{\pgfqpoint{4.956862in}{0.724573in}}%
\pgfpathlineto{\pgfqpoint{4.959689in}{0.723344in}}%
\pgfpathlineto{\pgfqpoint{4.962219in}{0.729343in}}%
\pgfpathlineto{\pgfqpoint{4.965002in}{0.733733in}}%
\pgfpathlineto{\pgfqpoint{4.967575in}{0.728881in}}%
\pgfpathlineto{\pgfqpoint{4.970314in}{0.731604in}}%
\pgfpathlineto{\pgfqpoint{4.972933in}{0.727147in}}%
\pgfpathlineto{\pgfqpoint{4.975703in}{0.735744in}}%
\pgfpathlineto{\pgfqpoint{4.978287in}{0.731364in}}%
\pgfpathlineto{\pgfqpoint{4.980967in}{0.733491in}}%
\pgfpathlineto{\pgfqpoint{4.983637in}{0.729708in}}%
\pgfpathlineto{\pgfqpoint{4.986325in}{0.731594in}}%
\pgfpathlineto{\pgfqpoint{4.989001in}{0.737474in}}%
\pgfpathlineto{\pgfqpoint{4.991683in}{0.737985in}}%
\pgfpathlineto{\pgfqpoint{4.994390in}{0.735208in}}%
\pgfpathlineto{\pgfqpoint{4.997028in}{0.732404in}}%
\pgfpathlineto{\pgfqpoint{4.999780in}{0.729777in}}%
\pgfpathlineto{\pgfqpoint{5.002384in}{0.723762in}}%
\pgfpathlineto{\pgfqpoint{5.005178in}{0.728909in}}%
\pgfpathlineto{\pgfqpoint{5.007751in}{0.726759in}}%
\pgfpathlineto{\pgfqpoint{5.010562in}{0.726088in}}%
\pgfpathlineto{\pgfqpoint{5.013104in}{0.732040in}}%
\pgfpathlineto{\pgfqpoint{5.015820in}{0.729988in}}%
\pgfpathlineto{\pgfqpoint{5.018466in}{0.728965in}}%
\pgfpathlineto{\pgfqpoint{5.021147in}{0.730087in}}%
\pgfpathlineto{\pgfqpoint{5.023927in}{0.729566in}}%
\pgfpathlineto{\pgfqpoint{5.026501in}{0.728260in}}%
\pgfpathlineto{\pgfqpoint{5.029275in}{0.728047in}}%
\pgfpathlineto{\pgfqpoint{5.031849in}{0.728136in}}%
\pgfpathlineto{\pgfqpoint{5.034649in}{0.728529in}}%
\pgfpathlineto{\pgfqpoint{5.037214in}{0.723780in}}%
\pgfpathlineto{\pgfqpoint{5.039962in}{0.726747in}}%
\pgfpathlineto{\pgfqpoint{5.042572in}{0.726508in}}%
\pgfpathlineto{\pgfqpoint{5.045249in}{0.727047in}}%
\pgfpathlineto{\pgfqpoint{5.047924in}{0.729935in}}%
\pgfpathlineto{\pgfqpoint{5.050606in}{0.726742in}}%
\pgfpathlineto{\pgfqpoint{5.053284in}{0.723776in}}%
\pgfpathlineto{\pgfqpoint{5.055952in}{0.726443in}}%
\pgfpathlineto{\pgfqpoint{5.058711in}{0.728139in}}%
\pgfpathlineto{\pgfqpoint{5.061315in}{0.725352in}}%
\pgfpathlineto{\pgfqpoint{5.064144in}{0.729787in}}%
\pgfpathlineto{\pgfqpoint{5.066677in}{0.729136in}}%
\pgfpathlineto{\pgfqpoint{5.069463in}{0.727497in}}%
\pgfpathlineto{\pgfqpoint{5.072030in}{0.727989in}}%
\pgfpathlineto{\pgfqpoint{5.074851in}{0.726222in}}%
\pgfpathlineto{\pgfqpoint{5.077390in}{0.727950in}}%
\pgfpathlineto{\pgfqpoint{5.080067in}{0.725886in}}%
\pgfpathlineto{\pgfqpoint{5.082746in}{0.732496in}}%
\pgfpathlineto{\pgfqpoint{5.085426in}{0.731576in}}%
\pgfpathlineto{\pgfqpoint{5.088103in}{0.733660in}}%
\pgfpathlineto{\pgfqpoint{5.090788in}{0.729768in}}%
\pgfpathlineto{\pgfqpoint{5.093579in}{0.729643in}}%
\pgfpathlineto{\pgfqpoint{5.096142in}{0.726961in}}%
\pgfpathlineto{\pgfqpoint{5.098948in}{0.728178in}}%
\pgfpathlineto{\pgfqpoint{5.101496in}{0.729682in}}%
\pgfpathlineto{\pgfqpoint{5.104312in}{0.728570in}}%
\pgfpathlineto{\pgfqpoint{5.106842in}{0.733661in}}%
\pgfpathlineto{\pgfqpoint{5.109530in}{0.727186in}}%
\pgfpathlineto{\pgfqpoint{5.112209in}{0.733662in}}%
\pgfpathlineto{\pgfqpoint{5.114887in}{0.729560in}}%
\pgfpathlineto{\pgfqpoint{5.117550in}{0.731477in}}%
\pgfpathlineto{\pgfqpoint{5.120243in}{0.730300in}}%
\pgfpathlineto{\pgfqpoint{5.123042in}{0.732479in}}%
\pgfpathlineto{\pgfqpoint{5.125599in}{0.731547in}}%
\pgfpathlineto{\pgfqpoint{5.128421in}{0.731078in}}%
\pgfpathlineto{\pgfqpoint{5.130953in}{0.732913in}}%
\pgfpathlineto{\pgfqpoint{5.133716in}{0.732273in}}%
\pgfpathlineto{\pgfqpoint{5.136311in}{0.730092in}}%
\pgfpathlineto{\pgfqpoint{5.139072in}{0.727296in}}%
\pgfpathlineto{\pgfqpoint{5.141660in}{0.732161in}}%
\pgfpathlineto{\pgfqpoint{5.144349in}{0.727263in}}%
\pgfpathlineto{\pgfqpoint{5.147029in}{0.728481in}}%
\pgfpathlineto{\pgfqpoint{5.149734in}{0.730416in}}%
\pgfpathlineto{\pgfqpoint{5.152382in}{0.723758in}}%
\pgfpathlineto{\pgfqpoint{5.155059in}{0.722435in}}%
\pgfpathlineto{\pgfqpoint{5.157815in}{0.728990in}}%
\pgfpathlineto{\pgfqpoint{5.160420in}{0.728344in}}%
\pgfpathlineto{\pgfqpoint{5.163243in}{0.728246in}}%
\pgfpathlineto{\pgfqpoint{5.165775in}{0.729322in}}%
\pgfpathlineto{\pgfqpoint{5.168591in}{0.731456in}}%
\pgfpathlineto{\pgfqpoint{5.171133in}{0.728984in}}%
\pgfpathlineto{\pgfqpoint{5.173925in}{0.743986in}}%
\pgfpathlineto{\pgfqpoint{5.176477in}{0.736152in}}%
\pgfpathlineto{\pgfqpoint{5.179188in}{0.733037in}}%
\pgfpathlineto{\pgfqpoint{5.181848in}{0.734813in}}%
\pgfpathlineto{\pgfqpoint{5.184522in}{0.736348in}}%
\pgfpathlineto{\pgfqpoint{5.187294in}{0.733515in}}%
\pgfpathlineto{\pgfqpoint{5.189880in}{0.731255in}}%
\pgfpathlineto{\pgfqpoint{5.192680in}{0.732166in}}%
\pgfpathlineto{\pgfqpoint{5.195239in}{0.730886in}}%
\pgfpathlineto{\pgfqpoint{5.198008in}{0.736575in}}%
\pgfpathlineto{\pgfqpoint{5.200594in}{0.735956in}}%
\pgfpathlineto{\pgfqpoint{5.203388in}{0.743408in}}%
\pgfpathlineto{\pgfqpoint{5.205952in}{0.738859in}}%
\pgfpathlineto{\pgfqpoint{5.208630in}{0.737525in}}%
\pgfpathlineto{\pgfqpoint{5.211299in}{0.736498in}}%
\pgfpathlineto{\pgfqpoint{5.214027in}{0.736906in}}%
\pgfpathlineto{\pgfqpoint{5.216667in}{0.733694in}}%
\pgfpathlineto{\pgfqpoint{5.219345in}{0.732362in}}%
\pgfpathlineto{\pgfqpoint{5.222151in}{0.732696in}}%
\pgfpathlineto{\pgfqpoint{5.224695in}{0.730205in}}%
\pgfpathlineto{\pgfqpoint{5.227470in}{0.728125in}}%
\pgfpathlineto{\pgfqpoint{5.230059in}{0.729754in}}%
\pgfpathlineto{\pgfqpoint{5.232855in}{0.728913in}}%
\pgfpathlineto{\pgfqpoint{5.235409in}{0.729026in}}%
\pgfpathlineto{\pgfqpoint{5.238173in}{0.729779in}}%
\pgfpathlineto{\pgfqpoint{5.240777in}{0.727231in}}%
\pgfpathlineto{\pgfqpoint{5.243445in}{0.727331in}}%
\pgfpathlineto{\pgfqpoint{5.246130in}{0.731999in}}%
\pgfpathlineto{\pgfqpoint{5.248816in}{0.727746in}}%
\pgfpathlineto{\pgfqpoint{5.251590in}{0.728700in}}%
\pgfpathlineto{\pgfqpoint{5.254236in}{0.729548in}}%
\pgfpathlineto{\pgfqpoint{5.256973in}{0.727341in}}%
\pgfpathlineto{\pgfqpoint{5.259511in}{0.728634in}}%
\pgfpathlineto{\pgfqpoint{5.262264in}{0.726269in}}%
\pgfpathlineto{\pgfqpoint{5.264876in}{0.735282in}}%
\pgfpathlineto{\pgfqpoint{5.267691in}{0.731368in}}%
\pgfpathlineto{\pgfqpoint{5.270238in}{0.731334in}}%
\pgfpathlineto{\pgfqpoint{5.272913in}{0.732513in}}%
\pgfpathlineto{\pgfqpoint{5.275589in}{0.730127in}}%
\pgfpathlineto{\pgfqpoint{5.278322in}{0.724585in}}%
\pgfpathlineto{\pgfqpoint{5.280947in}{0.723272in}}%
\pgfpathlineto{\pgfqpoint{5.283631in}{0.728747in}}%
\pgfpathlineto{\pgfqpoint{5.286436in}{0.733166in}}%
\pgfpathlineto{\pgfqpoint{5.288984in}{0.733321in}}%
\pgfpathlineto{\pgfqpoint{5.291794in}{0.735530in}}%
\pgfpathlineto{\pgfqpoint{5.294339in}{0.739687in}}%
\pgfpathlineto{\pgfqpoint{5.297140in}{0.736223in}}%
\pgfpathlineto{\pgfqpoint{5.299696in}{0.734352in}}%
\pgfpathlineto{\pgfqpoint{5.302443in}{0.736420in}}%
\pgfpathlineto{\pgfqpoint{5.305054in}{0.736973in}}%
\pgfpathlineto{\pgfqpoint{5.307731in}{0.738760in}}%
\pgfpathlineto{\pgfqpoint{5.310411in}{0.736356in}}%
\pgfpathlineto{\pgfqpoint{5.313089in}{0.738292in}}%
\pgfpathlineto{\pgfqpoint{5.315754in}{0.734970in}}%
\pgfpathlineto{\pgfqpoint{5.318430in}{0.738826in}}%
\pgfpathlineto{\pgfqpoint{5.321256in}{0.738950in}}%
\pgfpathlineto{\pgfqpoint{5.323802in}{0.736689in}}%
\pgfpathlineto{\pgfqpoint{5.326564in}{0.733444in}}%
\pgfpathlineto{\pgfqpoint{5.329159in}{0.734532in}}%
\pgfpathlineto{\pgfqpoint{5.331973in}{0.733457in}}%
\pgfpathlineto{\pgfqpoint{5.334510in}{0.731646in}}%
\pgfpathlineto{\pgfqpoint{5.337353in}{0.735841in}}%
\pgfpathlineto{\pgfqpoint{5.339872in}{0.734334in}}%
\pgfpathlineto{\pgfqpoint{5.342549in}{0.732515in}}%
\pgfpathlineto{\pgfqpoint{5.345224in}{0.725862in}}%
\pgfpathlineto{\pgfqpoint{5.347905in}{0.726409in}}%
\pgfpathlineto{\pgfqpoint{5.350723in}{0.731326in}}%
\pgfpathlineto{\pgfqpoint{5.353262in}{0.730582in}}%
\pgfpathlineto{\pgfqpoint{5.356056in}{0.731329in}}%
\pgfpathlineto{\pgfqpoint{5.358612in}{0.732418in}}%
\pgfpathlineto{\pgfqpoint{5.361370in}{0.729990in}}%
\pgfpathlineto{\pgfqpoint{5.363966in}{0.728929in}}%
\pgfpathlineto{\pgfqpoint{5.366727in}{0.729413in}}%
\pgfpathlineto{\pgfqpoint{5.369335in}{0.728492in}}%
\pgfpathlineto{\pgfqpoint{5.372013in}{0.730180in}}%
\pgfpathlineto{\pgfqpoint{5.374692in}{0.733730in}}%
\pgfpathlineto{\pgfqpoint{5.377370in}{0.733169in}}%
\pgfpathlineto{\pgfqpoint{5.380048in}{0.734478in}}%
\pgfpathlineto{\pgfqpoint{5.382725in}{0.731573in}}%
\pgfpathlineto{\pgfqpoint{5.385550in}{0.731985in}}%
\pgfpathlineto{\pgfqpoint{5.388083in}{0.730016in}}%
\pgfpathlineto{\pgfqpoint{5.390900in}{0.727629in}}%
\pgfpathlineto{\pgfqpoint{5.393441in}{0.722397in}}%
\pgfpathlineto{\pgfqpoint{5.396219in}{0.727157in}}%
\pgfpathlineto{\pgfqpoint{5.398784in}{0.738247in}}%
\pgfpathlineto{\pgfqpoint{5.401576in}{0.738537in}}%
\pgfpathlineto{\pgfqpoint{5.404154in}{0.734652in}}%
\pgfpathlineto{\pgfqpoint{5.406832in}{0.734814in}}%
\pgfpathlineto{\pgfqpoint{5.409507in}{0.733194in}}%
\pgfpathlineto{\pgfqpoint{5.412190in}{0.737549in}}%
\pgfpathlineto{\pgfqpoint{5.414954in}{0.737498in}}%
\pgfpathlineto{\pgfqpoint{5.417547in}{0.731396in}}%
\pgfpathlineto{\pgfqpoint{5.420304in}{0.731388in}}%
\pgfpathlineto{\pgfqpoint{5.422897in}{0.734646in}}%
\pgfpathlineto{\pgfqpoint{5.425661in}{0.734849in}}%
\pgfpathlineto{\pgfqpoint{5.428259in}{0.738378in}}%
\pgfpathlineto{\pgfqpoint{5.431015in}{0.736307in}}%
\pgfpathlineto{\pgfqpoint{5.433616in}{0.738164in}}%
\pgfpathlineto{\pgfqpoint{5.436295in}{0.737572in}}%
\pgfpathlineto{\pgfqpoint{5.438974in}{0.735912in}}%
\pgfpathlineto{\pgfqpoint{5.441698in}{0.737075in}}%
\pgfpathlineto{\pgfqpoint{5.444328in}{0.739791in}}%
\pgfpathlineto{\pgfqpoint{5.447021in}{0.736280in}}%
\pgfpathlineto{\pgfqpoint{5.449769in}{0.738860in}}%
\pgfpathlineto{\pgfqpoint{5.452365in}{0.738308in}}%
\pgfpathlineto{\pgfqpoint{5.455168in}{0.734678in}}%
\pgfpathlineto{\pgfqpoint{5.457721in}{0.739012in}}%
\pgfpathlineto{\pgfqpoint{5.460489in}{0.739141in}}%
\pgfpathlineto{\pgfqpoint{5.463079in}{0.734165in}}%
\pgfpathlineto{\pgfqpoint{5.465888in}{0.737396in}}%
\pgfpathlineto{\pgfqpoint{5.468425in}{0.733961in}}%
\pgfpathlineto{\pgfqpoint{5.471113in}{0.732983in}}%
\pgfpathlineto{\pgfqpoint{5.473792in}{0.731173in}}%
\pgfpathlineto{\pgfqpoint{5.476458in}{0.731556in}}%
\pgfpathlineto{\pgfqpoint{5.479152in}{0.733547in}}%
\pgfpathlineto{\pgfqpoint{5.481825in}{0.732684in}}%
\pgfpathlineto{\pgfqpoint{5.484641in}{0.732612in}}%
\pgfpathlineto{\pgfqpoint{5.487176in}{0.729038in}}%
\pgfpathlineto{\pgfqpoint{5.490000in}{0.733747in}}%
\pgfpathlineto{\pgfqpoint{5.492541in}{0.731753in}}%
\pgfpathlineto{\pgfqpoint{5.495346in}{0.734491in}}%
\pgfpathlineto{\pgfqpoint{5.497898in}{0.731799in}}%
\pgfpathlineto{\pgfqpoint{5.500687in}{0.731496in}}%
\pgfpathlineto{\pgfqpoint{5.503255in}{0.733918in}}%
\pgfpathlineto{\pgfqpoint{5.505933in}{0.732074in}}%
\pgfpathlineto{\pgfqpoint{5.508612in}{0.728386in}}%
\pgfpathlineto{\pgfqpoint{5.511290in}{0.728960in}}%
\pgfpathlineto{\pgfqpoint{5.514080in}{0.735550in}}%
\pgfpathlineto{\pgfqpoint{5.516646in}{0.734832in}}%
\pgfpathlineto{\pgfqpoint{5.519433in}{0.736760in}}%
\pgfpathlineto{\pgfqpoint{5.522003in}{0.736838in}}%
\pgfpathlineto{\pgfqpoint{5.524756in}{0.734780in}}%
\pgfpathlineto{\pgfqpoint{5.527360in}{0.731542in}}%
\pgfpathlineto{\pgfqpoint{5.530148in}{0.733818in}}%
\pgfpathlineto{\pgfqpoint{5.532717in}{0.731202in}}%
\pgfpathlineto{\pgfqpoint{5.535395in}{0.731008in}}%
\pgfpathlineto{\pgfqpoint{5.538074in}{0.732759in}}%
\pgfpathlineto{\pgfqpoint{5.540750in}{0.731997in}}%
\pgfpathlineto{\pgfqpoint{5.543421in}{0.733219in}}%
\pgfpathlineto{\pgfqpoint{5.546110in}{0.733745in}}%
\pgfpathlineto{\pgfqpoint{5.548921in}{0.732537in}}%
\pgfpathlineto{\pgfqpoint{5.551457in}{0.734899in}}%
\pgfpathlineto{\pgfqpoint{5.554198in}{0.731404in}}%
\pgfpathlineto{\pgfqpoint{5.556822in}{0.732771in}}%
\pgfpathlineto{\pgfqpoint{5.559612in}{0.728189in}}%
\pgfpathlineto{\pgfqpoint{5.562180in}{0.722407in}}%
\pgfpathlineto{\pgfqpoint{5.564940in}{0.723502in}}%
\pgfpathlineto{\pgfqpoint{5.567536in}{0.728858in}}%
\pgfpathlineto{\pgfqpoint{5.570215in}{0.724780in}}%
\pgfpathlineto{\pgfqpoint{5.572893in}{0.726655in}}%
\pgfpathlineto{\pgfqpoint{5.575596in}{0.731128in}}%
\pgfpathlineto{\pgfqpoint{5.578342in}{0.733879in}}%
\pgfpathlineto{\pgfqpoint{5.580914in}{0.726441in}}%
\pgfpathlineto{\pgfqpoint{5.583709in}{0.729266in}}%
\pgfpathlineto{\pgfqpoint{5.586269in}{0.735811in}}%
\pgfpathlineto{\pgfqpoint{5.589040in}{0.735922in}}%
\pgfpathlineto{\pgfqpoint{5.591641in}{0.732103in}}%
\pgfpathlineto{\pgfqpoint{5.594368in}{0.733156in}}%
\pgfpathlineto{\pgfqpoint{5.596999in}{0.737794in}}%
\pgfpathlineto{\pgfqpoint{5.599674in}{0.734858in}}%
\pgfpathlineto{\pgfqpoint{5.602352in}{0.734840in}}%
\pgfpathlineto{\pgfqpoint{5.605073in}{0.738623in}}%
\pgfpathlineto{\pgfqpoint{5.607698in}{0.736900in}}%
\pgfpathlineto{\pgfqpoint{5.610389in}{0.736596in}}%
\pgfpathlineto{\pgfqpoint{5.613235in}{0.737381in}}%
\pgfpathlineto{\pgfqpoint{5.615743in}{0.737625in}}%
\pgfpathlineto{\pgfqpoint{5.618526in}{0.738121in}}%
\pgfpathlineto{\pgfqpoint{5.621102in}{0.733420in}}%
\pgfpathlineto{\pgfqpoint{5.623868in}{0.737179in}}%
\pgfpathlineto{\pgfqpoint{5.626460in}{0.737650in}}%
\pgfpathlineto{\pgfqpoint{5.629232in}{0.736804in}}%
\pgfpathlineto{\pgfqpoint{5.631815in}{0.737225in}}%
\pgfpathlineto{\pgfqpoint{5.634496in}{0.739897in}}%
\pgfpathlineto{\pgfqpoint{5.637172in}{0.743704in}}%
\pgfpathlineto{\pgfqpoint{5.639852in}{0.756024in}}%
\pgfpathlineto{\pgfqpoint{5.642518in}{0.767257in}}%
\pgfpathlineto{\pgfqpoint{5.645243in}{0.756800in}}%
\pgfpathlineto{\pgfqpoint{5.648008in}{0.780212in}}%
\pgfpathlineto{\pgfqpoint{5.650563in}{0.791736in}}%
\pgfpathlineto{\pgfqpoint{5.653376in}{0.778820in}}%
\pgfpathlineto{\pgfqpoint{5.655919in}{0.773663in}}%
\pgfpathlineto{\pgfqpoint{5.658723in}{0.764549in}}%
\pgfpathlineto{\pgfqpoint{5.661273in}{0.754742in}}%
\pgfpathlineto{\pgfqpoint{5.664099in}{0.746642in}}%
\pgfpathlineto{\pgfqpoint{5.666632in}{0.750897in}}%
\pgfpathlineto{\pgfqpoint{5.669313in}{0.764120in}}%
\pgfpathlineto{\pgfqpoint{5.671991in}{0.758385in}}%
\pgfpathlineto{\pgfqpoint{5.674667in}{0.752764in}}%
\pgfpathlineto{\pgfqpoint{5.677486in}{0.746130in}}%
\pgfpathlineto{\pgfqpoint{5.680027in}{0.741043in}}%
\pgfpathlineto{\pgfqpoint{5.682836in}{0.742603in}}%
\pgfpathlineto{\pgfqpoint{5.685385in}{0.740947in}}%
\pgfpathlineto{\pgfqpoint{5.688159in}{0.735071in}}%
\pgfpathlineto{\pgfqpoint{5.690730in}{0.735932in}}%
\pgfpathlineto{\pgfqpoint{5.693473in}{0.753065in}}%
\pgfpathlineto{\pgfqpoint{5.696101in}{0.759217in}}%
\pgfpathlineto{\pgfqpoint{5.698775in}{0.756529in}}%
\pgfpathlineto{\pgfqpoint{5.701453in}{0.745627in}}%
\pgfpathlineto{\pgfqpoint{5.704130in}{0.738512in}}%
\pgfpathlineto{\pgfqpoint{5.706800in}{0.727201in}}%
\pgfpathlineto{\pgfqpoint{5.709490in}{0.731671in}}%
\pgfpathlineto{\pgfqpoint{5.712291in}{0.731735in}}%
\pgfpathlineto{\pgfqpoint{5.714834in}{0.727795in}}%
\pgfpathlineto{\pgfqpoint{5.717671in}{0.726676in}}%
\pgfpathlineto{\pgfqpoint{5.720201in}{0.723964in}}%
\pgfpathlineto{\pgfqpoint{5.722950in}{0.731480in}}%
\pgfpathlineto{\pgfqpoint{5.725548in}{0.731669in}}%
\pgfpathlineto{\pgfqpoint{5.728339in}{0.730114in}}%
\pgfpathlineto{\pgfqpoint{5.730919in}{0.729880in}}%
\pgfpathlineto{\pgfqpoint{5.733594in}{0.733562in}}%
\pgfpathlineto{\pgfqpoint{5.736276in}{0.733978in}}%
\pgfpathlineto{\pgfqpoint{5.738974in}{0.730408in}}%
\pgfpathlineto{\pgfqpoint{5.741745in}{0.730075in}}%
\pgfpathlineto{\pgfqpoint{5.744310in}{0.731743in}}%
\pgfpathlineto{\pgfqpoint{5.744310in}{0.413320in}}%
\pgfpathlineto{\pgfqpoint{5.744310in}{0.413320in}}%
\pgfpathlineto{\pgfqpoint{5.741745in}{0.413320in}}%
\pgfpathlineto{\pgfqpoint{5.738974in}{0.413320in}}%
\pgfpathlineto{\pgfqpoint{5.736276in}{0.413320in}}%
\pgfpathlineto{\pgfqpoint{5.733594in}{0.413320in}}%
\pgfpathlineto{\pgfqpoint{5.730919in}{0.413320in}}%
\pgfpathlineto{\pgfqpoint{5.728339in}{0.413320in}}%
\pgfpathlineto{\pgfqpoint{5.725548in}{0.413320in}}%
\pgfpathlineto{\pgfqpoint{5.722950in}{0.413320in}}%
\pgfpathlineto{\pgfqpoint{5.720201in}{0.413320in}}%
\pgfpathlineto{\pgfqpoint{5.717671in}{0.413320in}}%
\pgfpathlineto{\pgfqpoint{5.714834in}{0.413320in}}%
\pgfpathlineto{\pgfqpoint{5.712291in}{0.413320in}}%
\pgfpathlineto{\pgfqpoint{5.709490in}{0.413320in}}%
\pgfpathlineto{\pgfqpoint{5.706800in}{0.413320in}}%
\pgfpathlineto{\pgfqpoint{5.704130in}{0.413320in}}%
\pgfpathlineto{\pgfqpoint{5.701453in}{0.413320in}}%
\pgfpathlineto{\pgfqpoint{5.698775in}{0.413320in}}%
\pgfpathlineto{\pgfqpoint{5.696101in}{0.413320in}}%
\pgfpathlineto{\pgfqpoint{5.693473in}{0.413320in}}%
\pgfpathlineto{\pgfqpoint{5.690730in}{0.413320in}}%
\pgfpathlineto{\pgfqpoint{5.688159in}{0.413320in}}%
\pgfpathlineto{\pgfqpoint{5.685385in}{0.413320in}}%
\pgfpathlineto{\pgfqpoint{5.682836in}{0.413320in}}%
\pgfpathlineto{\pgfqpoint{5.680027in}{0.413320in}}%
\pgfpathlineto{\pgfqpoint{5.677486in}{0.413320in}}%
\pgfpathlineto{\pgfqpoint{5.674667in}{0.413320in}}%
\pgfpathlineto{\pgfqpoint{5.671991in}{0.413320in}}%
\pgfpathlineto{\pgfqpoint{5.669313in}{0.413320in}}%
\pgfpathlineto{\pgfqpoint{5.666632in}{0.413320in}}%
\pgfpathlineto{\pgfqpoint{5.664099in}{0.413320in}}%
\pgfpathlineto{\pgfqpoint{5.661273in}{0.413320in}}%
\pgfpathlineto{\pgfqpoint{5.658723in}{0.413320in}}%
\pgfpathlineto{\pgfqpoint{5.655919in}{0.413320in}}%
\pgfpathlineto{\pgfqpoint{5.653376in}{0.413320in}}%
\pgfpathlineto{\pgfqpoint{5.650563in}{0.413320in}}%
\pgfpathlineto{\pgfqpoint{5.648008in}{0.413320in}}%
\pgfpathlineto{\pgfqpoint{5.645243in}{0.413320in}}%
\pgfpathlineto{\pgfqpoint{5.642518in}{0.413320in}}%
\pgfpathlineto{\pgfqpoint{5.639852in}{0.413320in}}%
\pgfpathlineto{\pgfqpoint{5.637172in}{0.413320in}}%
\pgfpathlineto{\pgfqpoint{5.634496in}{0.413320in}}%
\pgfpathlineto{\pgfqpoint{5.631815in}{0.413320in}}%
\pgfpathlineto{\pgfqpoint{5.629232in}{0.413320in}}%
\pgfpathlineto{\pgfqpoint{5.626460in}{0.413320in}}%
\pgfpathlineto{\pgfqpoint{5.623868in}{0.413320in}}%
\pgfpathlineto{\pgfqpoint{5.621102in}{0.413320in}}%
\pgfpathlineto{\pgfqpoint{5.618526in}{0.413320in}}%
\pgfpathlineto{\pgfqpoint{5.615743in}{0.413320in}}%
\pgfpathlineto{\pgfqpoint{5.613235in}{0.413320in}}%
\pgfpathlineto{\pgfqpoint{5.610389in}{0.413320in}}%
\pgfpathlineto{\pgfqpoint{5.607698in}{0.413320in}}%
\pgfpathlineto{\pgfqpoint{5.605073in}{0.413320in}}%
\pgfpathlineto{\pgfqpoint{5.602352in}{0.413320in}}%
\pgfpathlineto{\pgfqpoint{5.599674in}{0.413320in}}%
\pgfpathlineto{\pgfqpoint{5.596999in}{0.413320in}}%
\pgfpathlineto{\pgfqpoint{5.594368in}{0.413320in}}%
\pgfpathlineto{\pgfqpoint{5.591641in}{0.413320in}}%
\pgfpathlineto{\pgfqpoint{5.589040in}{0.413320in}}%
\pgfpathlineto{\pgfqpoint{5.586269in}{0.413320in}}%
\pgfpathlineto{\pgfqpoint{5.583709in}{0.413320in}}%
\pgfpathlineto{\pgfqpoint{5.580914in}{0.413320in}}%
\pgfpathlineto{\pgfqpoint{5.578342in}{0.413320in}}%
\pgfpathlineto{\pgfqpoint{5.575596in}{0.413320in}}%
\pgfpathlineto{\pgfqpoint{5.572893in}{0.413320in}}%
\pgfpathlineto{\pgfqpoint{5.570215in}{0.413320in}}%
\pgfpathlineto{\pgfqpoint{5.567536in}{0.413320in}}%
\pgfpathlineto{\pgfqpoint{5.564940in}{0.413320in}}%
\pgfpathlineto{\pgfqpoint{5.562180in}{0.413320in}}%
\pgfpathlineto{\pgfqpoint{5.559612in}{0.413320in}}%
\pgfpathlineto{\pgfqpoint{5.556822in}{0.413320in}}%
\pgfpathlineto{\pgfqpoint{5.554198in}{0.413320in}}%
\pgfpathlineto{\pgfqpoint{5.551457in}{0.413320in}}%
\pgfpathlineto{\pgfqpoint{5.548921in}{0.413320in}}%
\pgfpathlineto{\pgfqpoint{5.546110in}{0.413320in}}%
\pgfpathlineto{\pgfqpoint{5.543421in}{0.413320in}}%
\pgfpathlineto{\pgfqpoint{5.540750in}{0.413320in}}%
\pgfpathlineto{\pgfqpoint{5.538074in}{0.413320in}}%
\pgfpathlineto{\pgfqpoint{5.535395in}{0.413320in}}%
\pgfpathlineto{\pgfqpoint{5.532717in}{0.413320in}}%
\pgfpathlineto{\pgfqpoint{5.530148in}{0.413320in}}%
\pgfpathlineto{\pgfqpoint{5.527360in}{0.413320in}}%
\pgfpathlineto{\pgfqpoint{5.524756in}{0.413320in}}%
\pgfpathlineto{\pgfqpoint{5.522003in}{0.413320in}}%
\pgfpathlineto{\pgfqpoint{5.519433in}{0.413320in}}%
\pgfpathlineto{\pgfqpoint{5.516646in}{0.413320in}}%
\pgfpathlineto{\pgfqpoint{5.514080in}{0.413320in}}%
\pgfpathlineto{\pgfqpoint{5.511290in}{0.413320in}}%
\pgfpathlineto{\pgfqpoint{5.508612in}{0.413320in}}%
\pgfpathlineto{\pgfqpoint{5.505933in}{0.413320in}}%
\pgfpathlineto{\pgfqpoint{5.503255in}{0.413320in}}%
\pgfpathlineto{\pgfqpoint{5.500687in}{0.413320in}}%
\pgfpathlineto{\pgfqpoint{5.497898in}{0.413320in}}%
\pgfpathlineto{\pgfqpoint{5.495346in}{0.413320in}}%
\pgfpathlineto{\pgfqpoint{5.492541in}{0.413320in}}%
\pgfpathlineto{\pgfqpoint{5.490000in}{0.413320in}}%
\pgfpathlineto{\pgfqpoint{5.487176in}{0.413320in}}%
\pgfpathlineto{\pgfqpoint{5.484641in}{0.413320in}}%
\pgfpathlineto{\pgfqpoint{5.481825in}{0.413320in}}%
\pgfpathlineto{\pgfqpoint{5.479152in}{0.413320in}}%
\pgfpathlineto{\pgfqpoint{5.476458in}{0.413320in}}%
\pgfpathlineto{\pgfqpoint{5.473792in}{0.413320in}}%
\pgfpathlineto{\pgfqpoint{5.471113in}{0.413320in}}%
\pgfpathlineto{\pgfqpoint{5.468425in}{0.413320in}}%
\pgfpathlineto{\pgfqpoint{5.465888in}{0.413320in}}%
\pgfpathlineto{\pgfqpoint{5.463079in}{0.413320in}}%
\pgfpathlineto{\pgfqpoint{5.460489in}{0.413320in}}%
\pgfpathlineto{\pgfqpoint{5.457721in}{0.413320in}}%
\pgfpathlineto{\pgfqpoint{5.455168in}{0.413320in}}%
\pgfpathlineto{\pgfqpoint{5.452365in}{0.413320in}}%
\pgfpathlineto{\pgfqpoint{5.449769in}{0.413320in}}%
\pgfpathlineto{\pgfqpoint{5.447021in}{0.413320in}}%
\pgfpathlineto{\pgfqpoint{5.444328in}{0.413320in}}%
\pgfpathlineto{\pgfqpoint{5.441698in}{0.413320in}}%
\pgfpathlineto{\pgfqpoint{5.438974in}{0.413320in}}%
\pgfpathlineto{\pgfqpoint{5.436295in}{0.413320in}}%
\pgfpathlineto{\pgfqpoint{5.433616in}{0.413320in}}%
\pgfpathlineto{\pgfqpoint{5.431015in}{0.413320in}}%
\pgfpathlineto{\pgfqpoint{5.428259in}{0.413320in}}%
\pgfpathlineto{\pgfqpoint{5.425661in}{0.413320in}}%
\pgfpathlineto{\pgfqpoint{5.422897in}{0.413320in}}%
\pgfpathlineto{\pgfqpoint{5.420304in}{0.413320in}}%
\pgfpathlineto{\pgfqpoint{5.417547in}{0.413320in}}%
\pgfpathlineto{\pgfqpoint{5.414954in}{0.413320in}}%
\pgfpathlineto{\pgfqpoint{5.412190in}{0.413320in}}%
\pgfpathlineto{\pgfqpoint{5.409507in}{0.413320in}}%
\pgfpathlineto{\pgfqpoint{5.406832in}{0.413320in}}%
\pgfpathlineto{\pgfqpoint{5.404154in}{0.413320in}}%
\pgfpathlineto{\pgfqpoint{5.401576in}{0.413320in}}%
\pgfpathlineto{\pgfqpoint{5.398784in}{0.413320in}}%
\pgfpathlineto{\pgfqpoint{5.396219in}{0.413320in}}%
\pgfpathlineto{\pgfqpoint{5.393441in}{0.413320in}}%
\pgfpathlineto{\pgfqpoint{5.390900in}{0.413320in}}%
\pgfpathlineto{\pgfqpoint{5.388083in}{0.413320in}}%
\pgfpathlineto{\pgfqpoint{5.385550in}{0.413320in}}%
\pgfpathlineto{\pgfqpoint{5.382725in}{0.413320in}}%
\pgfpathlineto{\pgfqpoint{5.380048in}{0.413320in}}%
\pgfpathlineto{\pgfqpoint{5.377370in}{0.413320in}}%
\pgfpathlineto{\pgfqpoint{5.374692in}{0.413320in}}%
\pgfpathlineto{\pgfqpoint{5.372013in}{0.413320in}}%
\pgfpathlineto{\pgfqpoint{5.369335in}{0.413320in}}%
\pgfpathlineto{\pgfqpoint{5.366727in}{0.413320in}}%
\pgfpathlineto{\pgfqpoint{5.363966in}{0.413320in}}%
\pgfpathlineto{\pgfqpoint{5.361370in}{0.413320in}}%
\pgfpathlineto{\pgfqpoint{5.358612in}{0.413320in}}%
\pgfpathlineto{\pgfqpoint{5.356056in}{0.413320in}}%
\pgfpathlineto{\pgfqpoint{5.353262in}{0.413320in}}%
\pgfpathlineto{\pgfqpoint{5.350723in}{0.413320in}}%
\pgfpathlineto{\pgfqpoint{5.347905in}{0.413320in}}%
\pgfpathlineto{\pgfqpoint{5.345224in}{0.413320in}}%
\pgfpathlineto{\pgfqpoint{5.342549in}{0.413320in}}%
\pgfpathlineto{\pgfqpoint{5.339872in}{0.413320in}}%
\pgfpathlineto{\pgfqpoint{5.337353in}{0.413320in}}%
\pgfpathlineto{\pgfqpoint{5.334510in}{0.413320in}}%
\pgfpathlineto{\pgfqpoint{5.331973in}{0.413320in}}%
\pgfpathlineto{\pgfqpoint{5.329159in}{0.413320in}}%
\pgfpathlineto{\pgfqpoint{5.326564in}{0.413320in}}%
\pgfpathlineto{\pgfqpoint{5.323802in}{0.413320in}}%
\pgfpathlineto{\pgfqpoint{5.321256in}{0.413320in}}%
\pgfpathlineto{\pgfqpoint{5.318430in}{0.413320in}}%
\pgfpathlineto{\pgfqpoint{5.315754in}{0.413320in}}%
\pgfpathlineto{\pgfqpoint{5.313089in}{0.413320in}}%
\pgfpathlineto{\pgfqpoint{5.310411in}{0.413320in}}%
\pgfpathlineto{\pgfqpoint{5.307731in}{0.413320in}}%
\pgfpathlineto{\pgfqpoint{5.305054in}{0.413320in}}%
\pgfpathlineto{\pgfqpoint{5.302443in}{0.413320in}}%
\pgfpathlineto{\pgfqpoint{5.299696in}{0.413320in}}%
\pgfpathlineto{\pgfqpoint{5.297140in}{0.413320in}}%
\pgfpathlineto{\pgfqpoint{5.294339in}{0.413320in}}%
\pgfpathlineto{\pgfqpoint{5.291794in}{0.413320in}}%
\pgfpathlineto{\pgfqpoint{5.288984in}{0.413320in}}%
\pgfpathlineto{\pgfqpoint{5.286436in}{0.413320in}}%
\pgfpathlineto{\pgfqpoint{5.283631in}{0.413320in}}%
\pgfpathlineto{\pgfqpoint{5.280947in}{0.413320in}}%
\pgfpathlineto{\pgfqpoint{5.278322in}{0.413320in}}%
\pgfpathlineto{\pgfqpoint{5.275589in}{0.413320in}}%
\pgfpathlineto{\pgfqpoint{5.272913in}{0.413320in}}%
\pgfpathlineto{\pgfqpoint{5.270238in}{0.413320in}}%
\pgfpathlineto{\pgfqpoint{5.267691in}{0.413320in}}%
\pgfpathlineto{\pgfqpoint{5.264876in}{0.413320in}}%
\pgfpathlineto{\pgfqpoint{5.262264in}{0.413320in}}%
\pgfpathlineto{\pgfqpoint{5.259511in}{0.413320in}}%
\pgfpathlineto{\pgfqpoint{5.256973in}{0.413320in}}%
\pgfpathlineto{\pgfqpoint{5.254236in}{0.413320in}}%
\pgfpathlineto{\pgfqpoint{5.251590in}{0.413320in}}%
\pgfpathlineto{\pgfqpoint{5.248816in}{0.413320in}}%
\pgfpathlineto{\pgfqpoint{5.246130in}{0.413320in}}%
\pgfpathlineto{\pgfqpoint{5.243445in}{0.413320in}}%
\pgfpathlineto{\pgfqpoint{5.240777in}{0.413320in}}%
\pgfpathlineto{\pgfqpoint{5.238173in}{0.413320in}}%
\pgfpathlineto{\pgfqpoint{5.235409in}{0.413320in}}%
\pgfpathlineto{\pgfqpoint{5.232855in}{0.413320in}}%
\pgfpathlineto{\pgfqpoint{5.230059in}{0.413320in}}%
\pgfpathlineto{\pgfqpoint{5.227470in}{0.413320in}}%
\pgfpathlineto{\pgfqpoint{5.224695in}{0.413320in}}%
\pgfpathlineto{\pgfqpoint{5.222151in}{0.413320in}}%
\pgfpathlineto{\pgfqpoint{5.219345in}{0.413320in}}%
\pgfpathlineto{\pgfqpoint{5.216667in}{0.413320in}}%
\pgfpathlineto{\pgfqpoint{5.214027in}{0.413320in}}%
\pgfpathlineto{\pgfqpoint{5.211299in}{0.413320in}}%
\pgfpathlineto{\pgfqpoint{5.208630in}{0.413320in}}%
\pgfpathlineto{\pgfqpoint{5.205952in}{0.413320in}}%
\pgfpathlineto{\pgfqpoint{5.203388in}{0.413320in}}%
\pgfpathlineto{\pgfqpoint{5.200594in}{0.413320in}}%
\pgfpathlineto{\pgfqpoint{5.198008in}{0.413320in}}%
\pgfpathlineto{\pgfqpoint{5.195239in}{0.413320in}}%
\pgfpathlineto{\pgfqpoint{5.192680in}{0.413320in}}%
\pgfpathlineto{\pgfqpoint{5.189880in}{0.413320in}}%
\pgfpathlineto{\pgfqpoint{5.187294in}{0.413320in}}%
\pgfpathlineto{\pgfqpoint{5.184522in}{0.413320in}}%
\pgfpathlineto{\pgfqpoint{5.181848in}{0.413320in}}%
\pgfpathlineto{\pgfqpoint{5.179188in}{0.413320in}}%
\pgfpathlineto{\pgfqpoint{5.176477in}{0.413320in}}%
\pgfpathlineto{\pgfqpoint{5.173925in}{0.413320in}}%
\pgfpathlineto{\pgfqpoint{5.171133in}{0.413320in}}%
\pgfpathlineto{\pgfqpoint{5.168591in}{0.413320in}}%
\pgfpathlineto{\pgfqpoint{5.165775in}{0.413320in}}%
\pgfpathlineto{\pgfqpoint{5.163243in}{0.413320in}}%
\pgfpathlineto{\pgfqpoint{5.160420in}{0.413320in}}%
\pgfpathlineto{\pgfqpoint{5.157815in}{0.413320in}}%
\pgfpathlineto{\pgfqpoint{5.155059in}{0.413320in}}%
\pgfpathlineto{\pgfqpoint{5.152382in}{0.413320in}}%
\pgfpathlineto{\pgfqpoint{5.149734in}{0.413320in}}%
\pgfpathlineto{\pgfqpoint{5.147029in}{0.413320in}}%
\pgfpathlineto{\pgfqpoint{5.144349in}{0.413320in}}%
\pgfpathlineto{\pgfqpoint{5.141660in}{0.413320in}}%
\pgfpathlineto{\pgfqpoint{5.139072in}{0.413320in}}%
\pgfpathlineto{\pgfqpoint{5.136311in}{0.413320in}}%
\pgfpathlineto{\pgfqpoint{5.133716in}{0.413320in}}%
\pgfpathlineto{\pgfqpoint{5.130953in}{0.413320in}}%
\pgfpathlineto{\pgfqpoint{5.128421in}{0.413320in}}%
\pgfpathlineto{\pgfqpoint{5.125599in}{0.413320in}}%
\pgfpathlineto{\pgfqpoint{5.123042in}{0.413320in}}%
\pgfpathlineto{\pgfqpoint{5.120243in}{0.413320in}}%
\pgfpathlineto{\pgfqpoint{5.117550in}{0.413320in}}%
\pgfpathlineto{\pgfqpoint{5.114887in}{0.413320in}}%
\pgfpathlineto{\pgfqpoint{5.112209in}{0.413320in}}%
\pgfpathlineto{\pgfqpoint{5.109530in}{0.413320in}}%
\pgfpathlineto{\pgfqpoint{5.106842in}{0.413320in}}%
\pgfpathlineto{\pgfqpoint{5.104312in}{0.413320in}}%
\pgfpathlineto{\pgfqpoint{5.101496in}{0.413320in}}%
\pgfpathlineto{\pgfqpoint{5.098948in}{0.413320in}}%
\pgfpathlineto{\pgfqpoint{5.096142in}{0.413320in}}%
\pgfpathlineto{\pgfqpoint{5.093579in}{0.413320in}}%
\pgfpathlineto{\pgfqpoint{5.090788in}{0.413320in}}%
\pgfpathlineto{\pgfqpoint{5.088103in}{0.413320in}}%
\pgfpathlineto{\pgfqpoint{5.085426in}{0.413320in}}%
\pgfpathlineto{\pgfqpoint{5.082746in}{0.413320in}}%
\pgfpathlineto{\pgfqpoint{5.080067in}{0.413320in}}%
\pgfpathlineto{\pgfqpoint{5.077390in}{0.413320in}}%
\pgfpathlineto{\pgfqpoint{5.074851in}{0.413320in}}%
\pgfpathlineto{\pgfqpoint{5.072030in}{0.413320in}}%
\pgfpathlineto{\pgfqpoint{5.069463in}{0.413320in}}%
\pgfpathlineto{\pgfqpoint{5.066677in}{0.413320in}}%
\pgfpathlineto{\pgfqpoint{5.064144in}{0.413320in}}%
\pgfpathlineto{\pgfqpoint{5.061315in}{0.413320in}}%
\pgfpathlineto{\pgfqpoint{5.058711in}{0.413320in}}%
\pgfpathlineto{\pgfqpoint{5.055952in}{0.413320in}}%
\pgfpathlineto{\pgfqpoint{5.053284in}{0.413320in}}%
\pgfpathlineto{\pgfqpoint{5.050606in}{0.413320in}}%
\pgfpathlineto{\pgfqpoint{5.047924in}{0.413320in}}%
\pgfpathlineto{\pgfqpoint{5.045249in}{0.413320in}}%
\pgfpathlineto{\pgfqpoint{5.042572in}{0.413320in}}%
\pgfpathlineto{\pgfqpoint{5.039962in}{0.413320in}}%
\pgfpathlineto{\pgfqpoint{5.037214in}{0.413320in}}%
\pgfpathlineto{\pgfqpoint{5.034649in}{0.413320in}}%
\pgfpathlineto{\pgfqpoint{5.031849in}{0.413320in}}%
\pgfpathlineto{\pgfqpoint{5.029275in}{0.413320in}}%
\pgfpathlineto{\pgfqpoint{5.026501in}{0.413320in}}%
\pgfpathlineto{\pgfqpoint{5.023927in}{0.413320in}}%
\pgfpathlineto{\pgfqpoint{5.021147in}{0.413320in}}%
\pgfpathlineto{\pgfqpoint{5.018466in}{0.413320in}}%
\pgfpathlineto{\pgfqpoint{5.015820in}{0.413320in}}%
\pgfpathlineto{\pgfqpoint{5.013104in}{0.413320in}}%
\pgfpathlineto{\pgfqpoint{5.010562in}{0.413320in}}%
\pgfpathlineto{\pgfqpoint{5.007751in}{0.413320in}}%
\pgfpathlineto{\pgfqpoint{5.005178in}{0.413320in}}%
\pgfpathlineto{\pgfqpoint{5.002384in}{0.413320in}}%
\pgfpathlineto{\pgfqpoint{4.999780in}{0.413320in}}%
\pgfpathlineto{\pgfqpoint{4.997028in}{0.413320in}}%
\pgfpathlineto{\pgfqpoint{4.994390in}{0.413320in}}%
\pgfpathlineto{\pgfqpoint{4.991683in}{0.413320in}}%
\pgfpathlineto{\pgfqpoint{4.989001in}{0.413320in}}%
\pgfpathlineto{\pgfqpoint{4.986325in}{0.413320in}}%
\pgfpathlineto{\pgfqpoint{4.983637in}{0.413320in}}%
\pgfpathlineto{\pgfqpoint{4.980967in}{0.413320in}}%
\pgfpathlineto{\pgfqpoint{4.978287in}{0.413320in}}%
\pgfpathlineto{\pgfqpoint{4.975703in}{0.413320in}}%
\pgfpathlineto{\pgfqpoint{4.972933in}{0.413320in}}%
\pgfpathlineto{\pgfqpoint{4.970314in}{0.413320in}}%
\pgfpathlineto{\pgfqpoint{4.967575in}{0.413320in}}%
\pgfpathlineto{\pgfqpoint{4.965002in}{0.413320in}}%
\pgfpathlineto{\pgfqpoint{4.962219in}{0.413320in}}%
\pgfpathlineto{\pgfqpoint{4.959689in}{0.413320in}}%
\pgfpathlineto{\pgfqpoint{4.956862in}{0.413320in}}%
\pgfpathlineto{\pgfqpoint{4.954182in}{0.413320in}}%
\pgfpathlineto{\pgfqpoint{4.951504in}{0.413320in}}%
\pgfpathlineto{\pgfqpoint{4.948827in}{0.413320in}}%
\pgfpathlineto{\pgfqpoint{4.946151in}{0.413320in}}%
\pgfpathlineto{\pgfqpoint{4.943466in}{0.413320in}}%
\pgfpathlineto{\pgfqpoint{4.940881in}{0.413320in}}%
\pgfpathlineto{\pgfqpoint{4.938112in}{0.413320in}}%
\pgfpathlineto{\pgfqpoint{4.935515in}{0.413320in}}%
\pgfpathlineto{\pgfqpoint{4.932742in}{0.413320in}}%
\pgfpathlineto{\pgfqpoint{4.930170in}{0.413320in}}%
\pgfpathlineto{\pgfqpoint{4.927400in}{0.413320in}}%
\pgfpathlineto{\pgfqpoint{4.924708in}{0.413320in}}%
\pgfpathlineto{\pgfqpoint{4.922041in}{0.413320in}}%
\pgfpathlineto{\pgfqpoint{4.919352in}{0.413320in}}%
\pgfpathlineto{\pgfqpoint{4.916681in}{0.413320in}}%
\pgfpathlineto{\pgfqpoint{4.914009in}{0.413320in}}%
\pgfpathlineto{\pgfqpoint{4.911435in}{0.413320in}}%
\pgfpathlineto{\pgfqpoint{4.908648in}{0.413320in}}%
\pgfpathlineto{\pgfqpoint{4.906096in}{0.413320in}}%
\pgfpathlineto{\pgfqpoint{4.903295in}{0.413320in}}%
\pgfpathlineto{\pgfqpoint{4.900712in}{0.413320in}}%
\pgfpathlineto{\pgfqpoint{4.897938in}{0.413320in}}%
\pgfpathlineto{\pgfqpoint{4.895399in}{0.413320in}}%
\pgfpathlineto{\pgfqpoint{4.892611in}{0.413320in}}%
\pgfpathlineto{\pgfqpoint{4.889902in}{0.413320in}}%
\pgfpathlineto{\pgfqpoint{4.887211in}{0.413320in}}%
\pgfpathlineto{\pgfqpoint{4.884540in}{0.413320in}}%
\pgfpathlineto{\pgfqpoint{4.881864in}{0.413320in}}%
\pgfpathlineto{\pgfqpoint{4.879180in}{0.413320in}}%
\pgfpathlineto{\pgfqpoint{4.876636in}{0.413320in}}%
\pgfpathlineto{\pgfqpoint{4.873832in}{0.413320in}}%
\pgfpathlineto{\pgfqpoint{4.871209in}{0.413320in}}%
\pgfpathlineto{\pgfqpoint{4.868474in}{0.413320in}}%
\pgfpathlineto{\pgfqpoint{4.865910in}{0.413320in}}%
\pgfpathlineto{\pgfqpoint{4.863116in}{0.413320in}}%
\pgfpathlineto{\pgfqpoint{4.860544in}{0.413320in}}%
\pgfpathlineto{\pgfqpoint{4.857807in}{0.413320in}}%
\pgfpathlineto{\pgfqpoint{4.855070in}{0.413320in}}%
\pgfpathlineto{\pgfqpoint{4.852404in}{0.413320in}}%
\pgfpathlineto{\pgfqpoint{4.849715in}{0.413320in}}%
\pgfpathlineto{\pgfqpoint{4.847127in}{0.413320in}}%
\pgfpathlineto{\pgfqpoint{4.844361in}{0.413320in}}%
\pgfpathlineto{\pgfqpoint{4.842380in}{0.413320in}}%
\pgfpathlineto{\pgfqpoint{4.839922in}{0.413320in}}%
\pgfpathlineto{\pgfqpoint{4.837992in}{0.413320in}}%
\pgfpathlineto{\pgfqpoint{4.833657in}{0.413320in}}%
\pgfpathlineto{\pgfqpoint{4.831045in}{0.413320in}}%
\pgfpathlineto{\pgfqpoint{4.828291in}{0.413320in}}%
\pgfpathlineto{\pgfqpoint{4.825619in}{0.413320in}}%
\pgfpathlineto{\pgfqpoint{4.822945in}{0.413320in}}%
\pgfpathlineto{\pgfqpoint{4.820265in}{0.413320in}}%
\pgfpathlineto{\pgfqpoint{4.817587in}{0.413320in}}%
\pgfpathlineto{\pgfqpoint{4.814907in}{0.413320in}}%
\pgfpathlineto{\pgfqpoint{4.812377in}{0.413320in}}%
\pgfpathlineto{\pgfqpoint{4.809538in}{0.413320in}}%
\pgfpathlineto{\pgfqpoint{4.807017in}{0.413320in}}%
\pgfpathlineto{\pgfqpoint{4.804193in}{0.413320in}}%
\pgfpathlineto{\pgfqpoint{4.801586in}{0.413320in}}%
\pgfpathlineto{\pgfqpoint{4.798830in}{0.413320in}}%
\pgfpathlineto{\pgfqpoint{4.796274in}{0.413320in}}%
\pgfpathlineto{\pgfqpoint{4.793512in}{0.413320in}}%
\pgfpathlineto{\pgfqpoint{4.790798in}{0.413320in}}%
\pgfpathlineto{\pgfqpoint{4.788116in}{0.413320in}}%
\pgfpathlineto{\pgfqpoint{4.785445in}{0.413320in}}%
\pgfpathlineto{\pgfqpoint{4.782872in}{0.413320in}}%
\pgfpathlineto{\pgfqpoint{4.780083in}{0.413320in}}%
\pgfpathlineto{\pgfqpoint{4.777535in}{0.413320in}}%
\pgfpathlineto{\pgfqpoint{4.774732in}{0.413320in}}%
\pgfpathlineto{\pgfqpoint{4.772198in}{0.413320in}}%
\pgfpathlineto{\pgfqpoint{4.769367in}{0.413320in}}%
\pgfpathlineto{\pgfqpoint{4.766783in}{0.413320in}}%
\pgfpathlineto{\pgfqpoint{4.764018in}{0.413320in}}%
\pgfpathlineto{\pgfqpoint{4.761337in}{0.413320in}}%
\pgfpathlineto{\pgfqpoint{4.758653in}{0.413320in}}%
\pgfpathlineto{\pgfqpoint{4.755983in}{0.413320in}}%
\pgfpathlineto{\pgfqpoint{4.753298in}{0.413320in}}%
\pgfpathlineto{\pgfqpoint{4.750627in}{0.413320in}}%
\pgfpathlineto{\pgfqpoint{4.748081in}{0.413320in}}%
\pgfpathlineto{\pgfqpoint{4.745256in}{0.413320in}}%
\pgfpathlineto{\pgfqpoint{4.742696in}{0.413320in}}%
\pgfpathlineto{\pgfqpoint{4.739912in}{0.413320in}}%
\pgfpathlineto{\pgfqpoint{4.737348in}{0.413320in}}%
\pgfpathlineto{\pgfqpoint{4.734552in}{0.413320in}}%
\pgfpathlineto{\pgfqpoint{4.731901in}{0.413320in}}%
\pgfpathlineto{\pgfqpoint{4.729233in}{0.413320in}}%
\pgfpathlineto{\pgfqpoint{4.726508in}{0.413320in}}%
\pgfpathlineto{\pgfqpoint{4.723873in}{0.413320in}}%
\pgfpathlineto{\pgfqpoint{4.721160in}{0.413320in}}%
\pgfpathlineto{\pgfqpoint{4.718486in}{0.413320in}}%
\pgfpathlineto{\pgfqpoint{4.715806in}{0.413320in}}%
\pgfpathlineto{\pgfqpoint{4.713275in}{0.413320in}}%
\pgfpathlineto{\pgfqpoint{4.710437in}{0.413320in}}%
\pgfpathlineto{\pgfqpoint{4.707824in}{0.413320in}}%
\pgfpathlineto{\pgfqpoint{4.705094in}{0.413320in}}%
\pgfpathlineto{\pgfqpoint{4.702517in}{0.413320in}}%
\pgfpathlineto{\pgfqpoint{4.699734in}{0.413320in}}%
\pgfpathlineto{\pgfqpoint{4.697170in}{0.413320in}}%
\pgfpathlineto{\pgfqpoint{4.694381in}{0.413320in}}%
\pgfpathlineto{\pgfqpoint{4.691694in}{0.413320in}}%
\pgfpathlineto{\pgfqpoint{4.689051in}{0.413320in}}%
\pgfpathlineto{\pgfqpoint{4.686337in}{0.413320in}}%
\pgfpathlineto{\pgfqpoint{4.683799in}{0.413320in}}%
\pgfpathlineto{\pgfqpoint{4.680988in}{0.413320in}}%
\pgfpathlineto{\pgfqpoint{4.678448in}{0.413320in}}%
\pgfpathlineto{\pgfqpoint{4.675619in}{0.413320in}}%
\pgfpathlineto{\pgfqpoint{4.673068in}{0.413320in}}%
\pgfpathlineto{\pgfqpoint{4.670261in}{0.413320in}}%
\pgfpathlineto{\pgfqpoint{4.667764in}{0.413320in}}%
\pgfpathlineto{\pgfqpoint{4.664923in}{0.413320in}}%
\pgfpathlineto{\pgfqpoint{4.662237in}{0.413320in}}%
\pgfpathlineto{\pgfqpoint{4.659590in}{0.413320in}}%
\pgfpathlineto{\pgfqpoint{4.656873in}{0.413320in}}%
\pgfpathlineto{\pgfqpoint{4.654203in}{0.413320in}}%
\pgfpathlineto{\pgfqpoint{4.651524in}{0.413320in}}%
\pgfpathlineto{\pgfqpoint{4.648922in}{0.413320in}}%
\pgfpathlineto{\pgfqpoint{4.646169in}{0.413320in}}%
\pgfpathlineto{\pgfqpoint{4.643628in}{0.413320in}}%
\pgfpathlineto{\pgfqpoint{4.640809in}{0.413320in}}%
\pgfpathlineto{\pgfqpoint{4.638204in}{0.413320in}}%
\pgfpathlineto{\pgfqpoint{4.635445in}{0.413320in}}%
\pgfpathlineto{\pgfqpoint{4.632902in}{0.413320in}}%
\pgfpathlineto{\pgfqpoint{4.630096in}{0.413320in}}%
\pgfpathlineto{\pgfqpoint{4.627411in}{0.413320in}}%
\pgfpathlineto{\pgfqpoint{4.624741in}{0.413320in}}%
\pgfpathlineto{\pgfqpoint{4.622056in}{0.413320in}}%
\pgfpathlineto{\pgfqpoint{4.619529in}{0.413320in}}%
\pgfpathlineto{\pgfqpoint{4.616702in}{0.413320in}}%
\pgfpathlineto{\pgfqpoint{4.614134in}{0.413320in}}%
\pgfpathlineto{\pgfqpoint{4.611350in}{0.413320in}}%
\pgfpathlineto{\pgfqpoint{4.608808in}{0.413320in}}%
\pgfpathlineto{\pgfqpoint{4.605990in}{0.413320in}}%
\pgfpathlineto{\pgfqpoint{4.603430in}{0.413320in}}%
\pgfpathlineto{\pgfqpoint{4.600633in}{0.413320in}}%
\pgfpathlineto{\pgfqpoint{4.597951in}{0.413320in}}%
\pgfpathlineto{\pgfqpoint{4.595281in}{0.413320in}}%
\pgfpathlineto{\pgfqpoint{4.592589in}{0.413320in}}%
\pgfpathlineto{\pgfqpoint{4.589920in}{0.413320in}}%
\pgfpathlineto{\pgfqpoint{4.587244in}{0.413320in}}%
\pgfpathlineto{\pgfqpoint{4.584672in}{0.413320in}}%
\pgfpathlineto{\pgfqpoint{4.581888in}{0.413320in}}%
\pgfpathlineto{\pgfqpoint{4.579305in}{0.413320in}}%
\pgfpathlineto{\pgfqpoint{4.576531in}{0.413320in}}%
\pgfpathlineto{\pgfqpoint{4.573947in}{0.413320in}}%
\pgfpathlineto{\pgfqpoint{4.571171in}{0.413320in}}%
\pgfpathlineto{\pgfqpoint{4.568612in}{0.413320in}}%
\pgfpathlineto{\pgfqpoint{4.565820in}{0.413320in}}%
\pgfpathlineto{\pgfqpoint{4.563125in}{0.413320in}}%
\pgfpathlineto{\pgfqpoint{4.560448in}{0.413320in}}%
\pgfpathlineto{\pgfqpoint{4.557777in}{0.413320in}}%
\pgfpathlineto{\pgfqpoint{4.555106in}{0.413320in}}%
\pgfpathlineto{\pgfqpoint{4.552425in}{0.413320in}}%
\pgfpathlineto{\pgfqpoint{4.549822in}{0.413320in}}%
\pgfpathlineto{\pgfqpoint{4.547064in}{0.413320in}}%
\pgfpathlineto{\pgfqpoint{4.544464in}{0.413320in}}%
\pgfpathlineto{\pgfqpoint{4.541711in}{0.413320in}}%
\pgfpathlineto{\pgfqpoint{4.539144in}{0.413320in}}%
\pgfpathlineto{\pgfqpoint{4.536400in}{0.413320in}}%
\pgfpathlineto{\pgfqpoint{4.533764in}{0.413320in}}%
\pgfpathlineto{\pgfqpoint{4.530990in}{0.413320in}}%
\pgfpathlineto{\pgfqpoint{4.528307in}{0.413320in}}%
\pgfpathlineto{\pgfqpoint{4.525640in}{0.413320in}}%
\pgfpathlineto{\pgfqpoint{4.522962in}{0.413320in}}%
\pgfpathlineto{\pgfqpoint{4.520345in}{0.413320in}}%
\pgfpathlineto{\pgfqpoint{4.517598in}{0.413320in}}%
\pgfpathlineto{\pgfqpoint{4.515080in}{0.413320in}}%
\pgfpathlineto{\pgfqpoint{4.512246in}{0.413320in}}%
\pgfpathlineto{\pgfqpoint{4.509643in}{0.413320in}}%
\pgfpathlineto{\pgfqpoint{4.506893in}{0.413320in}}%
\pgfpathlineto{\pgfqpoint{4.504305in}{0.413320in}}%
\pgfpathlineto{\pgfqpoint{4.501529in}{0.413320in}}%
\pgfpathlineto{\pgfqpoint{4.498850in}{0.413320in}}%
\pgfpathlineto{\pgfqpoint{4.496167in}{0.413320in}}%
\pgfpathlineto{\pgfqpoint{4.493492in}{0.413320in}}%
\pgfpathlineto{\pgfqpoint{4.490822in}{0.413320in}}%
\pgfpathlineto{\pgfqpoint{4.488130in}{0.413320in}}%
\pgfpathlineto{\pgfqpoint{4.485581in}{0.413320in}}%
\pgfpathlineto{\pgfqpoint{4.482778in}{0.413320in}}%
\pgfpathlineto{\pgfqpoint{4.480201in}{0.413320in}}%
\pgfpathlineto{\pgfqpoint{4.477430in}{0.413320in}}%
\pgfpathlineto{\pgfqpoint{4.474861in}{0.413320in}}%
\pgfpathlineto{\pgfqpoint{4.472059in}{0.413320in}}%
\pgfpathlineto{\pgfqpoint{4.469492in}{0.413320in}}%
\pgfpathlineto{\pgfqpoint{4.466717in}{0.413320in}}%
\pgfpathlineto{\pgfqpoint{4.464029in}{0.413320in}}%
\pgfpathlineto{\pgfqpoint{4.461367in}{0.413320in}}%
\pgfpathlineto{\pgfqpoint{4.458681in}{0.413320in}}%
\pgfpathlineto{\pgfqpoint{4.456138in}{0.413320in}}%
\pgfpathlineto{\pgfqpoint{4.453312in}{0.413320in}}%
\pgfpathlineto{\pgfqpoint{4.450767in}{0.413320in}}%
\pgfpathlineto{\pgfqpoint{4.447965in}{0.413320in}}%
\pgfpathlineto{\pgfqpoint{4.445423in}{0.413320in}}%
\pgfpathlineto{\pgfqpoint{4.442611in}{0.413320in}}%
\pgfpathlineto{\pgfqpoint{4.440041in}{0.413320in}}%
\pgfpathlineto{\pgfqpoint{4.437253in}{0.413320in}}%
\pgfpathlineto{\pgfqpoint{4.434569in}{0.413320in}}%
\pgfpathlineto{\pgfqpoint{4.431901in}{0.413320in}}%
\pgfpathlineto{\pgfqpoint{4.429220in}{0.413320in}}%
\pgfpathlineto{\pgfqpoint{4.426534in}{0.413320in}}%
\pgfpathlineto{\pgfqpoint{4.423863in}{0.413320in}}%
\pgfpathlineto{\pgfqpoint{4.421292in}{0.413320in}}%
\pgfpathlineto{\pgfqpoint{4.418506in}{0.413320in}}%
\pgfpathlineto{\pgfqpoint{4.415932in}{0.413320in}}%
\pgfpathlineto{\pgfqpoint{4.413149in}{0.413320in}}%
\pgfpathlineto{\pgfqpoint{4.410587in}{0.413320in}}%
\pgfpathlineto{\pgfqpoint{4.407788in}{0.413320in}}%
\pgfpathlineto{\pgfqpoint{4.405234in}{0.413320in}}%
\pgfpathlineto{\pgfqpoint{4.402468in}{0.413320in}}%
\pgfpathlineto{\pgfqpoint{4.399745in}{0.413320in}}%
\pgfpathlineto{\pgfqpoint{4.397076in}{0.413320in}}%
\pgfpathlineto{\pgfqpoint{4.394400in}{0.413320in}}%
\pgfpathlineto{\pgfqpoint{4.391721in}{0.413320in}}%
\pgfpathlineto{\pgfqpoint{4.389044in}{0.413320in}}%
\pgfpathlineto{\pgfqpoint{4.386431in}{0.413320in}}%
\pgfpathlineto{\pgfqpoint{4.383674in}{0.413320in}}%
\pgfpathlineto{\pgfqpoint{4.381097in}{0.413320in}}%
\pgfpathlineto{\pgfqpoint{4.378329in}{0.413320in}}%
\pgfpathlineto{\pgfqpoint{4.375761in}{0.413320in}}%
\pgfpathlineto{\pgfqpoint{4.372976in}{0.413320in}}%
\pgfpathlineto{\pgfqpoint{4.370437in}{0.413320in}}%
\pgfpathlineto{\pgfqpoint{4.367646in}{0.413320in}}%
\pgfpathlineto{\pgfqpoint{4.364936in}{0.413320in}}%
\pgfpathlineto{\pgfqpoint{4.362270in}{0.413320in}}%
\pgfpathlineto{\pgfqpoint{4.359582in}{0.413320in}}%
\pgfpathlineto{\pgfqpoint{4.357014in}{0.413320in}}%
\pgfpathlineto{\pgfqpoint{4.354224in}{0.413320in}}%
\pgfpathlineto{\pgfqpoint{4.351645in}{0.413320in}}%
\pgfpathlineto{\pgfqpoint{4.348868in}{0.413320in}}%
\pgfpathlineto{\pgfqpoint{4.346263in}{0.413320in}}%
\pgfpathlineto{\pgfqpoint{4.343510in}{0.413320in}}%
\pgfpathlineto{\pgfqpoint{4.340976in}{0.413320in}}%
\pgfpathlineto{\pgfqpoint{4.338154in}{0.413320in}}%
\pgfpathlineto{\pgfqpoint{4.335463in}{0.413320in}}%
\pgfpathlineto{\pgfqpoint{4.332796in}{0.413320in}}%
\pgfpathlineto{\pgfqpoint{4.330118in}{0.413320in}}%
\pgfpathlineto{\pgfqpoint{4.327440in}{0.413320in}}%
\pgfpathlineto{\pgfqpoint{4.324760in}{0.413320in}}%
\pgfpathlineto{\pgfqpoint{4.322181in}{0.413320in}}%
\pgfpathlineto{\pgfqpoint{4.319405in}{0.413320in}}%
\pgfpathlineto{\pgfqpoint{4.316856in}{0.413320in}}%
\pgfpathlineto{\pgfqpoint{4.314032in}{0.413320in}}%
\pgfpathlineto{\pgfqpoint{4.311494in}{0.413320in}}%
\pgfpathlineto{\pgfqpoint{4.308691in}{0.413320in}}%
\pgfpathlineto{\pgfqpoint{4.306118in}{0.413320in}}%
\pgfpathlineto{\pgfqpoint{4.303357in}{0.413320in}}%
\pgfpathlineto{\pgfqpoint{4.300656in}{0.413320in}}%
\pgfpathlineto{\pgfqpoint{4.297977in}{0.413320in}}%
\pgfpathlineto{\pgfqpoint{4.295299in}{0.413320in}}%
\pgfpathlineto{\pgfqpoint{4.292786in}{0.413320in}}%
\pgfpathlineto{\pgfqpoint{4.289936in}{0.413320in}}%
\pgfpathlineto{\pgfqpoint{4.287399in}{0.413320in}}%
\pgfpathlineto{\pgfqpoint{4.284586in}{0.413320in}}%
\pgfpathlineto{\pgfqpoint{4.282000in}{0.413320in}}%
\pgfpathlineto{\pgfqpoint{4.279212in}{0.413320in}}%
\pgfpathlineto{\pgfqpoint{4.276635in}{0.413320in}}%
\pgfpathlineto{\pgfqpoint{4.273874in}{0.413320in}}%
\pgfpathlineto{\pgfqpoint{4.271187in}{0.413320in}}%
\pgfpathlineto{\pgfqpoint{4.268590in}{0.413320in}}%
\pgfpathlineto{\pgfqpoint{4.265824in}{0.413320in}}%
\pgfpathlineto{\pgfqpoint{4.263157in}{0.413320in}}%
\pgfpathlineto{\pgfqpoint{4.260477in}{0.413320in}}%
\pgfpathlineto{\pgfqpoint{4.257958in}{0.413320in}}%
\pgfpathlineto{\pgfqpoint{4.255120in}{0.413320in}}%
\pgfpathlineto{\pgfqpoint{4.252581in}{0.413320in}}%
\pgfpathlineto{\pgfqpoint{4.249767in}{0.413320in}}%
\pgfpathlineto{\pgfqpoint{4.247225in}{0.413320in}}%
\pgfpathlineto{\pgfqpoint{4.244394in}{0.413320in}}%
\pgfpathlineto{\pgfqpoint{4.241900in}{0.413320in}}%
\pgfpathlineto{\pgfqpoint{4.239084in}{0.413320in}}%
\pgfpathlineto{\pgfqpoint{4.236375in}{0.413320in}}%
\pgfpathlineto{\pgfqpoint{4.233691in}{0.413320in}}%
\pgfpathlineto{\pgfqpoint{4.231013in}{0.413320in}}%
\pgfpathlineto{\pgfqpoint{4.228331in}{0.413320in}}%
\pgfpathlineto{\pgfqpoint{4.225654in}{0.413320in}}%
\pgfpathlineto{\pgfqpoint{4.223082in}{0.413320in}}%
\pgfpathlineto{\pgfqpoint{4.220304in}{0.413320in}}%
\pgfpathlineto{\pgfqpoint{4.217694in}{0.413320in}}%
\pgfpathlineto{\pgfqpoint{4.214948in}{0.413320in}}%
\pgfpathlineto{\pgfqpoint{4.212383in}{0.413320in}}%
\pgfpathlineto{\pgfqpoint{4.209597in}{0.413320in}}%
\pgfpathlineto{\pgfqpoint{4.207076in}{0.413320in}}%
\pgfpathlineto{\pgfqpoint{4.204240in}{0.413320in}}%
\pgfpathlineto{\pgfqpoint{4.201542in}{0.413320in}}%
\pgfpathlineto{\pgfqpoint{4.198878in}{0.413320in}}%
\pgfpathlineto{\pgfqpoint{4.196186in}{0.413320in}}%
\pgfpathlineto{\pgfqpoint{4.193638in}{0.413320in}}%
\pgfpathlineto{\pgfqpoint{4.190842in}{0.413320in}}%
\pgfpathlineto{\pgfqpoint{4.188318in}{0.413320in}}%
\pgfpathlineto{\pgfqpoint{4.185481in}{0.413320in}}%
\pgfpathlineto{\pgfqpoint{4.182899in}{0.413320in}}%
\pgfpathlineto{\pgfqpoint{4.180129in}{0.413320in}}%
\pgfpathlineto{\pgfqpoint{4.177593in}{0.413320in}}%
\pgfpathlineto{\pgfqpoint{4.174770in}{0.413320in}}%
\pgfpathlineto{\pgfqpoint{4.172093in}{0.413320in}}%
\pgfpathlineto{\pgfqpoint{4.169415in}{0.413320in}}%
\pgfpathlineto{\pgfqpoint{4.166737in}{0.413320in}}%
\pgfpathlineto{\pgfqpoint{4.164059in}{0.413320in}}%
\pgfpathlineto{\pgfqpoint{4.161380in}{0.413320in}}%
\pgfpathlineto{\pgfqpoint{4.158806in}{0.413320in}}%
\pgfpathlineto{\pgfqpoint{4.156016in}{0.413320in}}%
\pgfpathlineto{\pgfqpoint{4.153423in}{0.413320in}}%
\pgfpathlineto{\pgfqpoint{4.150665in}{0.413320in}}%
\pgfpathlineto{\pgfqpoint{4.148082in}{0.413320in}}%
\pgfpathlineto{\pgfqpoint{4.145310in}{0.413320in}}%
\pgfpathlineto{\pgfqpoint{4.142713in}{0.413320in}}%
\pgfpathlineto{\pgfqpoint{4.139963in}{0.413320in}}%
\pgfpathlineto{\pgfqpoint{4.137272in}{0.413320in}}%
\pgfpathlineto{\pgfqpoint{4.134615in}{0.413320in}}%
\pgfpathlineto{\pgfqpoint{4.131920in}{0.413320in}}%
\pgfpathlineto{\pgfqpoint{4.129349in}{0.413320in}}%
\pgfpathlineto{\pgfqpoint{4.126553in}{0.413320in}}%
\pgfpathlineto{\pgfqpoint{4.124019in}{0.413320in}}%
\pgfpathlineto{\pgfqpoint{4.121205in}{0.413320in}}%
\pgfpathlineto{\pgfqpoint{4.118554in}{0.413320in}}%
\pgfpathlineto{\pgfqpoint{4.115844in}{0.413320in}}%
\pgfpathlineto{\pgfqpoint{4.113252in}{0.413320in}}%
\pgfpathlineto{\pgfqpoint{4.110488in}{0.413320in}}%
\pgfpathlineto{\pgfqpoint{4.107814in}{0.413320in}}%
\pgfpathlineto{\pgfqpoint{4.105185in}{0.413320in}}%
\pgfpathlineto{\pgfqpoint{4.102456in}{0.413320in}}%
\pgfpathlineto{\pgfqpoint{4.099777in}{0.413320in}}%
\pgfpathlineto{\pgfqpoint{4.097092in}{0.413320in}}%
\pgfpathlineto{\pgfqpoint{4.094527in}{0.413320in}}%
\pgfpathlineto{\pgfqpoint{4.091729in}{0.413320in}}%
\pgfpathlineto{\pgfqpoint{4.089159in}{0.413320in}}%
\pgfpathlineto{\pgfqpoint{4.086385in}{0.413320in}}%
\pgfpathlineto{\pgfqpoint{4.083870in}{0.413320in}}%
\pgfpathlineto{\pgfqpoint{4.081018in}{0.413320in}}%
\pgfpathlineto{\pgfqpoint{4.078471in}{0.413320in}}%
\pgfpathlineto{\pgfqpoint{4.075705in}{0.413320in}}%
\pgfpathlineto{\pgfqpoint{4.072985in}{0.413320in}}%
\pgfpathlineto{\pgfqpoint{4.070313in}{0.413320in}}%
\pgfpathlineto{\pgfqpoint{4.067636in}{0.413320in}}%
\pgfpathlineto{\pgfqpoint{4.064957in}{0.413320in}}%
\pgfpathlineto{\pgfqpoint{4.062266in}{0.413320in}}%
\pgfpathlineto{\pgfqpoint{4.059702in}{0.413320in}}%
\pgfpathlineto{\pgfqpoint{4.056911in}{0.413320in}}%
\pgfpathlineto{\pgfqpoint{4.054326in}{0.413320in}}%
\pgfpathlineto{\pgfqpoint{4.051557in}{0.413320in}}%
\pgfpathlineto{\pgfqpoint{4.049006in}{0.413320in}}%
\pgfpathlineto{\pgfqpoint{4.046210in}{0.413320in}}%
\pgfpathlineto{\pgfqpoint{4.043667in}{0.413320in}}%
\pgfpathlineto{\pgfqpoint{4.040852in}{0.413320in}}%
\pgfpathlineto{\pgfqpoint{4.038174in}{0.413320in}}%
\pgfpathlineto{\pgfqpoint{4.035492in}{0.413320in}}%
\pgfpathlineto{\pgfqpoint{4.032817in}{0.413320in}}%
\pgfpathlineto{\pgfqpoint{4.030229in}{0.413320in}}%
\pgfpathlineto{\pgfqpoint{4.027447in}{0.413320in}}%
\pgfpathlineto{\pgfqpoint{4.024868in}{0.413320in}}%
\pgfpathlineto{\pgfqpoint{4.022097in}{0.413320in}}%
\pgfpathlineto{\pgfqpoint{4.019518in}{0.413320in}}%
\pgfpathlineto{\pgfqpoint{4.016744in}{0.413320in}}%
\pgfpathlineto{\pgfqpoint{4.014186in}{0.413320in}}%
\pgfpathlineto{\pgfqpoint{4.011394in}{0.413320in}}%
\pgfpathlineto{\pgfqpoint{4.008699in}{0.413320in}}%
\pgfpathlineto{\pgfqpoint{4.006034in}{0.413320in}}%
\pgfpathlineto{\pgfqpoint{4.003348in}{0.413320in}}%
\pgfpathlineto{\pgfqpoint{4.000674in}{0.413320in}}%
\pgfpathlineto{\pgfqpoint{3.997990in}{0.413320in}}%
\pgfpathlineto{\pgfqpoint{3.995417in}{0.413320in}}%
\pgfpathlineto{\pgfqpoint{3.992642in}{0.413320in}}%
\pgfpathlineto{\pgfqpoint{3.990055in}{0.413320in}}%
\pgfpathlineto{\pgfqpoint{3.987270in}{0.413320in}}%
\pgfpathlineto{\pgfqpoint{3.984714in}{0.413320in}}%
\pgfpathlineto{\pgfqpoint{3.981929in}{0.413320in}}%
\pgfpathlineto{\pgfqpoint{3.979389in}{0.413320in}}%
\pgfpathlineto{\pgfqpoint{3.976563in}{0.413320in}}%
\pgfpathlineto{\pgfqpoint{3.973885in}{0.413320in}}%
\pgfpathlineto{\pgfqpoint{3.971250in}{0.413320in}}%
\pgfpathlineto{\pgfqpoint{3.968523in}{0.413320in}}%
\pgfpathlineto{\pgfqpoint{3.966013in}{0.413320in}}%
\pgfpathlineto{\pgfqpoint{3.963176in}{0.413320in}}%
\pgfpathlineto{\pgfqpoint{3.960635in}{0.413320in}}%
\pgfpathlineto{\pgfqpoint{3.957823in}{0.413320in}}%
\pgfpathlineto{\pgfqpoint{3.955211in}{0.413320in}}%
\pgfpathlineto{\pgfqpoint{3.952464in}{0.413320in}}%
\pgfpathlineto{\pgfqpoint{3.949894in}{0.413320in}}%
\pgfpathlineto{\pgfqpoint{3.947101in}{0.413320in}}%
\pgfpathlineto{\pgfqpoint{3.944431in}{0.413320in}}%
\pgfpathlineto{\pgfqpoint{3.941778in}{0.413320in}}%
\pgfpathlineto{\pgfqpoint{3.939075in}{0.413320in}}%
\pgfpathlineto{\pgfqpoint{3.936395in}{0.413320in}}%
\pgfpathlineto{\pgfqpoint{3.933714in}{0.413320in}}%
\pgfpathlineto{\pgfqpoint{3.931202in}{0.413320in}}%
\pgfpathlineto{\pgfqpoint{3.928347in}{0.413320in}}%
\pgfpathlineto{\pgfqpoint{3.925778in}{0.413320in}}%
\pgfpathlineto{\pgfqpoint{3.923005in}{0.413320in}}%
\pgfpathlineto{\pgfqpoint{3.920412in}{0.413320in}}%
\pgfpathlineto{\pgfqpoint{3.917646in}{0.413320in}}%
\pgfpathlineto{\pgfqpoint{3.915107in}{0.413320in}}%
\pgfpathlineto{\pgfqpoint{3.912296in}{0.413320in}}%
\pgfpathlineto{\pgfqpoint{3.909602in}{0.413320in}}%
\pgfpathlineto{\pgfqpoint{3.906918in}{0.413320in}}%
\pgfpathlineto{\pgfqpoint{3.904252in}{0.413320in}}%
\pgfpathlineto{\pgfqpoint{3.901573in}{0.413320in}}%
\pgfpathlineto{\pgfqpoint{3.898891in}{0.413320in}}%
\pgfpathlineto{\pgfqpoint{3.896345in}{0.413320in}}%
\pgfpathlineto{\pgfqpoint{3.893541in}{0.413320in}}%
\pgfpathlineto{\pgfqpoint{3.890926in}{0.413320in}}%
\pgfpathlineto{\pgfqpoint{3.888188in}{0.413320in}}%
\pgfpathlineto{\pgfqpoint{3.885621in}{0.413320in}}%
\pgfpathlineto{\pgfqpoint{3.882850in}{0.413320in}}%
\pgfpathlineto{\pgfqpoint{3.880237in}{0.413320in}}%
\pgfpathlineto{\pgfqpoint{3.877466in}{0.413320in}}%
\pgfpathlineto{\pgfqpoint{3.874790in}{0.413320in}}%
\pgfpathlineto{\pgfqpoint{3.872114in}{0.413320in}}%
\pgfpathlineto{\pgfqpoint{3.869435in}{0.413320in}}%
\pgfpathlineto{\pgfqpoint{3.866815in}{0.413320in}}%
\pgfpathlineto{\pgfqpoint{3.864073in}{0.413320in}}%
\pgfpathlineto{\pgfqpoint{3.861561in}{0.413320in}}%
\pgfpathlineto{\pgfqpoint{3.858720in}{0.413320in}}%
\pgfpathlineto{\pgfqpoint{3.856100in}{0.413320in}}%
\pgfpathlineto{\pgfqpoint{3.853358in}{0.413320in}}%
\pgfpathlineto{\pgfqpoint{3.850814in}{0.413320in}}%
\pgfpathlineto{\pgfqpoint{3.848005in}{0.413320in}}%
\pgfpathlineto{\pgfqpoint{3.845329in}{0.413320in}}%
\pgfpathlineto{\pgfqpoint{3.842641in}{0.413320in}}%
\pgfpathlineto{\pgfqpoint{3.839960in}{0.413320in}}%
\pgfpathlineto{\pgfqpoint{3.837286in}{0.413320in}}%
\pgfpathlineto{\pgfqpoint{3.834616in}{0.413320in}}%
\pgfpathlineto{\pgfqpoint{3.832053in}{0.413320in}}%
\pgfpathlineto{\pgfqpoint{3.829252in}{0.413320in}}%
\pgfpathlineto{\pgfqpoint{3.826679in}{0.413320in}}%
\pgfpathlineto{\pgfqpoint{3.823903in}{0.413320in}}%
\pgfpathlineto{\pgfqpoint{3.821315in}{0.413320in}}%
\pgfpathlineto{\pgfqpoint{3.818546in}{0.413320in}}%
\pgfpathlineto{\pgfqpoint{3.815983in}{0.413320in}}%
\pgfpathlineto{\pgfqpoint{3.813172in}{0.413320in}}%
\pgfpathlineto{\pgfqpoint{3.810510in}{0.413320in}}%
\pgfpathlineto{\pgfqpoint{3.807832in}{0.413320in}}%
\pgfpathlineto{\pgfqpoint{3.805145in}{0.413320in}}%
\pgfpathlineto{\pgfqpoint{3.802569in}{0.413320in}}%
\pgfpathlineto{\pgfqpoint{3.799797in}{0.413320in}}%
\pgfpathlineto{\pgfqpoint{3.797265in}{0.413320in}}%
\pgfpathlineto{\pgfqpoint{3.794435in}{0.413320in}}%
\pgfpathlineto{\pgfqpoint{3.791897in}{0.413320in}}%
\pgfpathlineto{\pgfqpoint{3.789084in}{0.413320in}}%
\pgfpathlineto{\pgfqpoint{3.786504in}{0.413320in}}%
\pgfpathlineto{\pgfqpoint{3.783725in}{0.413320in}}%
\pgfpathlineto{\pgfqpoint{3.781046in}{0.413320in}}%
\pgfpathlineto{\pgfqpoint{3.778370in}{0.413320in}}%
\pgfpathlineto{\pgfqpoint{3.775691in}{0.413320in}}%
\pgfpathlineto{\pgfqpoint{3.773014in}{0.413320in}}%
\pgfpathlineto{\pgfqpoint{3.770323in}{0.413320in}}%
\pgfpathlineto{\pgfqpoint{3.767782in}{0.413320in}}%
\pgfpathlineto{\pgfqpoint{3.764966in}{0.413320in}}%
\pgfpathlineto{\pgfqpoint{3.762389in}{0.413320in}}%
\pgfpathlineto{\pgfqpoint{3.759622in}{0.413320in}}%
\pgfpathlineto{\pgfqpoint{3.757065in}{0.413320in}}%
\pgfpathlineto{\pgfqpoint{3.754265in}{0.413320in}}%
\pgfpathlineto{\pgfqpoint{3.751728in}{0.413320in}}%
\pgfpathlineto{\pgfqpoint{3.748903in}{0.413320in}}%
\pgfpathlineto{\pgfqpoint{3.746229in}{0.413320in}}%
\pgfpathlineto{\pgfqpoint{3.743548in}{0.413320in}}%
\pgfpathlineto{\pgfqpoint{3.740874in}{0.413320in}}%
\pgfpathlineto{\pgfqpoint{3.738194in}{0.413320in}}%
\pgfpathlineto{\pgfqpoint{3.735509in}{0.413320in}}%
\pgfpathlineto{\pgfqpoint{3.732950in}{0.413320in}}%
\pgfpathlineto{\pgfqpoint{3.730158in}{0.413320in}}%
\pgfpathlineto{\pgfqpoint{3.727581in}{0.413320in}}%
\pgfpathlineto{\pgfqpoint{3.724804in}{0.413320in}}%
\pgfpathlineto{\pgfqpoint{3.722228in}{0.413320in}}%
\pgfpathlineto{\pgfqpoint{3.719446in}{0.413320in}}%
\pgfpathlineto{\pgfqpoint{3.716875in}{0.413320in}}%
\pgfpathlineto{\pgfqpoint{3.714086in}{0.413320in}}%
\pgfpathlineto{\pgfqpoint{3.711410in}{0.413320in}}%
\pgfpathlineto{\pgfqpoint{3.708729in}{0.413320in}}%
\pgfpathlineto{\pgfqpoint{3.706053in}{0.413320in}}%
\pgfpathlineto{\pgfqpoint{3.703460in}{0.413320in}}%
\pgfpathlineto{\pgfqpoint{3.700684in}{0.413320in}}%
\pgfpathlineto{\pgfqpoint{3.698125in}{0.413320in}}%
\pgfpathlineto{\pgfqpoint{3.695331in}{0.413320in}}%
\pgfpathlineto{\pgfqpoint{3.692765in}{0.413320in}}%
\pgfpathlineto{\pgfqpoint{3.689983in}{0.413320in}}%
\pgfpathlineto{\pgfqpoint{3.687442in}{0.413320in}}%
\pgfpathlineto{\pgfqpoint{3.684620in}{0.413320in}}%
\pgfpathlineto{\pgfqpoint{3.681948in}{0.413320in}}%
\pgfpathlineto{\pgfqpoint{3.679273in}{0.413320in}}%
\pgfpathlineto{\pgfqpoint{3.676591in}{0.413320in}}%
\pgfpathlineto{\pgfqpoint{3.673911in}{0.413320in}}%
\pgfpathlineto{\pgfqpoint{3.671232in}{0.413320in}}%
\pgfpathlineto{\pgfqpoint{3.668665in}{0.413320in}}%
\pgfpathlineto{\pgfqpoint{3.665864in}{0.413320in}}%
\pgfpathlineto{\pgfqpoint{3.663276in}{0.413320in}}%
\pgfpathlineto{\pgfqpoint{3.660515in}{0.413320in}}%
\pgfpathlineto{\pgfqpoint{3.657917in}{0.413320in}}%
\pgfpathlineto{\pgfqpoint{3.655165in}{0.413320in}}%
\pgfpathlineto{\pgfqpoint{3.652628in}{0.413320in}}%
\pgfpathlineto{\pgfqpoint{3.649837in}{0.413320in}}%
\pgfpathlineto{\pgfqpoint{3.647130in}{0.413320in}}%
\pgfpathlineto{\pgfqpoint{3.644452in}{0.413320in}}%
\pgfpathlineto{\pgfqpoint{3.641773in}{0.413320in}}%
\pgfpathlineto{\pgfqpoint{3.639207in}{0.413320in}}%
\pgfpathlineto{\pgfqpoint{3.636413in}{0.413320in}}%
\pgfpathlineto{\pgfqpoint{3.633858in}{0.413320in}}%
\pgfpathlineto{\pgfqpoint{3.631058in}{0.413320in}}%
\pgfpathlineto{\pgfqpoint{3.628460in}{0.413320in}}%
\pgfpathlineto{\pgfqpoint{3.625689in}{0.413320in}}%
\pgfpathlineto{\pgfqpoint{3.623165in}{0.413320in}}%
\pgfpathlineto{\pgfqpoint{3.620345in}{0.413320in}}%
\pgfpathlineto{\pgfqpoint{3.617667in}{0.413320in}}%
\pgfpathlineto{\pgfqpoint{3.614982in}{0.413320in}}%
\pgfpathlineto{\pgfqpoint{3.612311in}{0.413320in}}%
\pgfpathlineto{\pgfqpoint{3.609632in}{0.413320in}}%
\pgfpathlineto{\pgfqpoint{3.606951in}{0.413320in}}%
\pgfpathlineto{\pgfqpoint{3.604387in}{0.413320in}}%
\pgfpathlineto{\pgfqpoint{3.601590in}{0.413320in}}%
\pgfpathlineto{\pgfqpoint{3.598998in}{0.413320in}}%
\pgfpathlineto{\pgfqpoint{3.596240in}{0.413320in}}%
\pgfpathlineto{\pgfqpoint{3.593620in}{0.413320in}}%
\pgfpathlineto{\pgfqpoint{3.590883in}{0.413320in}}%
\pgfpathlineto{\pgfqpoint{3.588258in}{0.413320in}}%
\pgfpathlineto{\pgfqpoint{3.585532in}{0.413320in}}%
\pgfpathlineto{\pgfqpoint{3.582851in}{0.413320in}}%
\pgfpathlineto{\pgfqpoint{3.580191in}{0.413320in}}%
\pgfpathlineto{\pgfqpoint{3.577487in}{0.413320in}}%
\pgfpathlineto{\pgfqpoint{3.574814in}{0.413320in}}%
\pgfpathlineto{\pgfqpoint{3.572126in}{0.413320in}}%
\pgfpathlineto{\pgfqpoint{3.569584in}{0.413320in}}%
\pgfpathlineto{\pgfqpoint{3.566774in}{0.413320in}}%
\pgfpathlineto{\pgfqpoint{3.564188in}{0.413320in}}%
\pgfpathlineto{\pgfqpoint{3.561420in}{0.413320in}}%
\pgfpathlineto{\pgfqpoint{3.558853in}{0.413320in}}%
\pgfpathlineto{\pgfqpoint{3.556061in}{0.413320in}}%
\pgfpathlineto{\pgfqpoint{3.553498in}{0.413320in}}%
\pgfpathlineto{\pgfqpoint{3.550713in}{0.413320in}}%
\pgfpathlineto{\pgfqpoint{3.548029in}{0.413320in}}%
\pgfpathlineto{\pgfqpoint{3.545349in}{0.413320in}}%
\pgfpathlineto{\pgfqpoint{3.542656in}{0.413320in}}%
\pgfpathlineto{\pgfqpoint{3.540093in}{0.413320in}}%
\pgfpathlineto{\pgfqpoint{3.537309in}{0.413320in}}%
\pgfpathlineto{\pgfqpoint{3.534783in}{0.413320in}}%
\pgfpathlineto{\pgfqpoint{3.531955in}{0.413320in}}%
\pgfpathlineto{\pgfqpoint{3.529327in}{0.413320in}}%
\pgfpathlineto{\pgfqpoint{3.526601in}{0.413320in}}%
\pgfpathlineto{\pgfqpoint{3.524041in}{0.413320in}}%
\pgfpathlineto{\pgfqpoint{3.521244in}{0.413320in}}%
\pgfpathlineto{\pgfqpoint{3.518565in}{0.413320in}}%
\pgfpathlineto{\pgfqpoint{3.515884in}{0.413320in}}%
\pgfpathlineto{\pgfqpoint{3.513209in}{0.413320in}}%
\pgfpathlineto{\pgfqpoint{3.510533in}{0.413320in}}%
\pgfpathlineto{\pgfqpoint{3.507840in}{0.413320in}}%
\pgfpathlineto{\pgfqpoint{3.505262in}{0.413320in}}%
\pgfpathlineto{\pgfqpoint{3.502488in}{0.413320in}}%
\pgfpathlineto{\pgfqpoint{3.499909in}{0.413320in}}%
\pgfpathlineto{\pgfqpoint{3.497139in}{0.413320in}}%
\pgfpathlineto{\pgfqpoint{3.494581in}{0.413320in}}%
\pgfpathlineto{\pgfqpoint{3.491783in}{0.413320in}}%
\pgfpathlineto{\pgfqpoint{3.489223in}{0.413320in}}%
\pgfpathlineto{\pgfqpoint{3.486442in}{0.413320in}}%
\pgfpathlineto{\pgfqpoint{3.483744in}{0.413320in}}%
\pgfpathlineto{\pgfqpoint{3.481072in}{0.413320in}}%
\pgfpathlineto{\pgfqpoint{3.478378in}{0.413320in}}%
\pgfpathlineto{\pgfqpoint{3.475821in}{0.413320in}}%
\pgfpathlineto{\pgfqpoint{3.473021in}{0.413320in}}%
\pgfpathlineto{\pgfqpoint{3.470466in}{0.413320in}}%
\pgfpathlineto{\pgfqpoint{3.467678in}{0.413320in}}%
\pgfpathlineto{\pgfqpoint{3.465072in}{0.413320in}}%
\pgfpathlineto{\pgfqpoint{3.462321in}{0.413320in}}%
\pgfpathlineto{\pgfqpoint{3.459695in}{0.413320in}}%
\pgfpathlineto{\pgfqpoint{3.456960in}{0.413320in}}%
\pgfpathlineto{\pgfqpoint{3.454285in}{0.413320in}}%
\pgfpathlineto{\pgfqpoint{3.451597in}{0.413320in}}%
\pgfpathlineto{\pgfqpoint{3.448926in}{0.413320in}}%
\pgfpathlineto{\pgfqpoint{3.446257in}{0.413320in}}%
\pgfpathlineto{\pgfqpoint{3.443574in}{0.413320in}}%
\pgfpathlineto{\pgfqpoint{3.440996in}{0.413320in}}%
\pgfpathlineto{\pgfqpoint{3.438210in}{0.413320in}}%
\pgfpathlineto{\pgfqpoint{3.435635in}{0.413320in}}%
\pgfpathlineto{\pgfqpoint{3.432851in}{0.413320in}}%
\pgfpathlineto{\pgfqpoint{3.430313in}{0.413320in}}%
\pgfpathlineto{\pgfqpoint{3.427501in}{0.413320in}}%
\pgfpathlineto{\pgfqpoint{3.424887in}{0.413320in}}%
\pgfpathlineto{\pgfqpoint{3.422142in}{0.413320in}}%
\pgfpathlineto{\pgfqpoint{3.419455in}{0.413320in}}%
\pgfpathlineto{\pgfqpoint{3.416780in}{0.413320in}}%
\pgfpathlineto{\pgfqpoint{3.414109in}{0.413320in}}%
\pgfpathlineto{\pgfqpoint{3.411431in}{0.413320in}}%
\pgfpathlineto{\pgfqpoint{3.408752in}{0.413320in}}%
\pgfpathlineto{\pgfqpoint{3.406202in}{0.413320in}}%
\pgfpathlineto{\pgfqpoint{3.403394in}{0.413320in}}%
\pgfpathlineto{\pgfqpoint{3.400783in}{0.413320in}}%
\pgfpathlineto{\pgfqpoint{3.398037in}{0.413320in}}%
\pgfpathlineto{\pgfqpoint{3.395461in}{0.413320in}}%
\pgfpathlineto{\pgfqpoint{3.392681in}{0.413320in}}%
\pgfpathlineto{\pgfqpoint{3.390102in}{0.413320in}}%
\pgfpathlineto{\pgfqpoint{3.387309in}{0.413320in}}%
\pgfpathlineto{\pgfqpoint{3.384647in}{0.413320in}}%
\pgfpathlineto{\pgfqpoint{3.381959in}{0.413320in}}%
\pgfpathlineto{\pgfqpoint{3.379290in}{0.413320in}}%
\pgfpathlineto{\pgfqpoint{3.376735in}{0.413320in}}%
\pgfpathlineto{\pgfqpoint{3.373921in}{0.413320in}}%
\pgfpathlineto{\pgfqpoint{3.371357in}{0.413320in}}%
\pgfpathlineto{\pgfqpoint{3.368577in}{0.413320in}}%
\pgfpathlineto{\pgfqpoint{3.365996in}{0.413320in}}%
\pgfpathlineto{\pgfqpoint{3.363221in}{0.413320in}}%
\pgfpathlineto{\pgfqpoint{3.360620in}{0.413320in}}%
\pgfpathlineto{\pgfqpoint{3.357862in}{0.413320in}}%
\pgfpathlineto{\pgfqpoint{3.355177in}{0.413320in}}%
\pgfpathlineto{\pgfqpoint{3.352505in}{0.413320in}}%
\pgfpathlineto{\pgfqpoint{3.349828in}{0.413320in}}%
\pgfpathlineto{\pgfqpoint{3.347139in}{0.413320in}}%
\pgfpathlineto{\pgfqpoint{3.344468in}{0.413320in}}%
\pgfpathlineto{\pgfqpoint{3.341893in}{0.413320in}}%
\pgfpathlineto{\pgfqpoint{3.339101in}{0.413320in}}%
\pgfpathlineto{\pgfqpoint{3.336541in}{0.413320in}}%
\pgfpathlineto{\pgfqpoint{3.333758in}{0.413320in}}%
\pgfpathlineto{\pgfqpoint{3.331183in}{0.413320in}}%
\pgfpathlineto{\pgfqpoint{3.328401in}{0.413320in}}%
\pgfpathlineto{\pgfqpoint{3.325860in}{0.413320in}}%
\pgfpathlineto{\pgfqpoint{3.323049in}{0.413320in}}%
\pgfpathlineto{\pgfqpoint{3.320366in}{0.413320in}}%
\pgfpathlineto{\pgfqpoint{3.317688in}{0.413320in}}%
\pgfpathlineto{\pgfqpoint{3.315008in}{0.413320in}}%
\pgfpathlineto{\pgfqpoint{3.312480in}{0.413320in}}%
\pgfpathlineto{\pgfqpoint{3.309652in}{0.413320in}}%
\pgfpathlineto{\pgfqpoint{3.307104in}{0.413320in}}%
\pgfpathlineto{\pgfqpoint{3.304295in}{0.413320in}}%
\pgfpathlineto{\pgfqpoint{3.301719in}{0.413320in}}%
\pgfpathlineto{\pgfqpoint{3.298937in}{0.413320in}}%
\pgfpathlineto{\pgfqpoint{3.296376in}{0.413320in}}%
\pgfpathlineto{\pgfqpoint{3.293574in}{0.413320in}}%
\pgfpathlineto{\pgfqpoint{3.290890in}{0.413320in}}%
\pgfpathlineto{\pgfqpoint{3.288225in}{0.413320in}}%
\pgfpathlineto{\pgfqpoint{3.285534in}{0.413320in}}%
\pgfpathlineto{\pgfqpoint{3.282870in}{0.413320in}}%
\pgfpathlineto{\pgfqpoint{3.280189in}{0.413320in}}%
\pgfpathlineto{\pgfqpoint{3.277603in}{0.413320in}}%
\pgfpathlineto{\pgfqpoint{3.274831in}{0.413320in}}%
\pgfpathlineto{\pgfqpoint{3.272254in}{0.413320in}}%
\pgfpathlineto{\pgfqpoint{3.269478in}{0.413320in}}%
\pgfpathlineto{\pgfqpoint{3.266849in}{0.413320in}}%
\pgfpathlineto{\pgfqpoint{3.264119in}{0.413320in}}%
\pgfpathlineto{\pgfqpoint{3.261594in}{0.413320in}}%
\pgfpathlineto{\pgfqpoint{3.258784in}{0.413320in}}%
\pgfpathlineto{\pgfqpoint{3.256083in}{0.413320in}}%
\pgfpathlineto{\pgfqpoint{3.253404in}{0.413320in}}%
\pgfpathlineto{\pgfqpoint{3.250716in}{0.413320in}}%
\pgfpathlineto{\pgfqpoint{3.248049in}{0.413320in}}%
\pgfpathlineto{\pgfqpoint{3.245363in}{0.413320in}}%
\pgfpathlineto{\pgfqpoint{3.242807in}{0.413320in}}%
\pgfpathlineto{\pgfqpoint{3.240010in}{0.413320in}}%
\pgfpathlineto{\pgfqpoint{3.237411in}{0.413320in}}%
\pgfpathlineto{\pgfqpoint{3.234658in}{0.413320in}}%
\pgfpathlineto{\pgfqpoint{3.232069in}{0.413320in}}%
\pgfpathlineto{\pgfqpoint{3.229310in}{0.413320in}}%
\pgfpathlineto{\pgfqpoint{3.226609in}{0.413320in}}%
\pgfpathlineto{\pgfqpoint{3.223942in}{0.413320in}}%
\pgfpathlineto{\pgfqpoint{3.221255in}{0.413320in}}%
\pgfpathlineto{\pgfqpoint{3.218586in}{0.413320in}}%
\pgfpathlineto{\pgfqpoint{3.215908in}{0.413320in}}%
\pgfpathlineto{\pgfqpoint{3.213342in}{0.413320in}}%
\pgfpathlineto{\pgfqpoint{3.210545in}{0.413320in}}%
\pgfpathlineto{\pgfqpoint{3.207984in}{0.413320in}}%
\pgfpathlineto{\pgfqpoint{3.205195in}{0.413320in}}%
\pgfpathlineto{\pgfqpoint{3.202562in}{0.413320in}}%
\pgfpathlineto{\pgfqpoint{3.199823in}{0.413320in}}%
\pgfpathlineto{\pgfqpoint{3.197226in}{0.413320in}}%
\pgfpathlineto{\pgfqpoint{3.194508in}{0.413320in}}%
\pgfpathlineto{\pgfqpoint{3.191796in}{0.413320in}}%
\pgfpathlineto{\pgfqpoint{3.189117in}{0.413320in}}%
\pgfpathlineto{\pgfqpoint{3.186440in}{0.413320in}}%
\pgfpathlineto{\pgfqpoint{3.183760in}{0.413320in}}%
\pgfpathlineto{\pgfqpoint{3.181089in}{0.413320in}}%
\pgfpathlineto{\pgfqpoint{3.178525in}{0.413320in}}%
\pgfpathlineto{\pgfqpoint{3.175724in}{0.413320in}}%
\pgfpathlineto{\pgfqpoint{3.173142in}{0.413320in}}%
\pgfpathlineto{\pgfqpoint{3.170375in}{0.413320in}}%
\pgfpathlineto{\pgfqpoint{3.167776in}{0.413320in}}%
\pgfpathlineto{\pgfqpoint{3.165019in}{0.413320in}}%
\pgfpathlineto{\pgfqpoint{3.162474in}{0.413320in}}%
\pgfpathlineto{\pgfqpoint{3.159675in}{0.413320in}}%
\pgfpathlineto{\pgfqpoint{3.156981in}{0.413320in}}%
\pgfpathlineto{\pgfqpoint{3.154327in}{0.413320in}}%
\pgfpathlineto{\pgfqpoint{3.151612in}{0.413320in}}%
\pgfpathlineto{\pgfqpoint{3.149057in}{0.413320in}}%
\pgfpathlineto{\pgfqpoint{3.146271in}{0.413320in}}%
\pgfpathlineto{\pgfqpoint{3.143740in}{0.413320in}}%
\pgfpathlineto{\pgfqpoint{3.140913in}{0.413320in}}%
\pgfpathlineto{\pgfqpoint{3.138375in}{0.413320in}}%
\pgfpathlineto{\pgfqpoint{3.135550in}{0.413320in}}%
\pgfpathlineto{\pgfqpoint{3.132946in}{0.413320in}}%
\pgfpathlineto{\pgfqpoint{3.130199in}{0.413320in}}%
\pgfpathlineto{\pgfqpoint{3.127512in}{0.413320in}}%
\pgfpathlineto{\pgfqpoint{3.124842in}{0.413320in}}%
\pgfpathlineto{\pgfqpoint{3.122163in}{0.413320in}}%
\pgfpathlineto{\pgfqpoint{3.119487in}{0.413320in}}%
\pgfpathlineto{\pgfqpoint{3.116807in}{0.413320in}}%
\pgfpathlineto{\pgfqpoint{3.114242in}{0.413320in}}%
\pgfpathlineto{\pgfqpoint{3.111451in}{0.413320in}}%
\pgfpathlineto{\pgfqpoint{3.108896in}{0.413320in}}%
\pgfpathlineto{\pgfqpoint{3.106094in}{0.413320in}}%
\pgfpathlineto{\pgfqpoint{3.103508in}{0.413320in}}%
\pgfpathlineto{\pgfqpoint{3.100737in}{0.413320in}}%
\pgfpathlineto{\pgfqpoint{3.098163in}{0.413320in}}%
\pgfpathlineto{\pgfqpoint{3.095388in}{0.413320in}}%
\pgfpathlineto{\pgfqpoint{3.092699in}{0.413320in}}%
\pgfpathlineto{\pgfqpoint{3.090023in}{0.413320in}}%
\pgfpathlineto{\pgfqpoint{3.087343in}{0.413320in}}%
\pgfpathlineto{\pgfqpoint{3.084671in}{0.413320in}}%
\pgfpathlineto{\pgfqpoint{3.081990in}{0.413320in}}%
\pgfpathlineto{\pgfqpoint{3.079381in}{0.413320in}}%
\pgfpathlineto{\pgfqpoint{3.076631in}{0.413320in}}%
\pgfpathlineto{\pgfqpoint{3.074056in}{0.413320in}}%
\pgfpathlineto{\pgfqpoint{3.071266in}{0.413320in}}%
\pgfpathlineto{\pgfqpoint{3.068709in}{0.413320in}}%
\pgfpathlineto{\pgfqpoint{3.065916in}{0.413320in}}%
\pgfpathlineto{\pgfqpoint{3.063230in}{0.413320in}}%
\pgfpathlineto{\pgfqpoint{3.060561in}{0.413320in}}%
\pgfpathlineto{\pgfqpoint{3.057884in}{0.413320in}}%
\pgfpathlineto{\pgfqpoint{3.055202in}{0.413320in}}%
\pgfpathlineto{\pgfqpoint{3.052526in}{0.413320in}}%
\pgfpathlineto{\pgfqpoint{3.049988in}{0.413320in}}%
\pgfpathlineto{\pgfqpoint{3.047157in}{0.413320in}}%
\pgfpathlineto{\pgfqpoint{3.044568in}{0.413320in}}%
\pgfpathlineto{\pgfqpoint{3.041813in}{0.413320in}}%
\pgfpathlineto{\pgfqpoint{3.039262in}{0.413320in}}%
\pgfpathlineto{\pgfqpoint{3.036456in}{0.413320in}}%
\pgfpathlineto{\pgfqpoint{3.033921in}{0.413320in}}%
\pgfpathlineto{\pgfqpoint{3.031091in}{0.413320in}}%
\pgfpathlineto{\pgfqpoint{3.028412in}{0.413320in}}%
\pgfpathlineto{\pgfqpoint{3.025803in}{0.413320in}}%
\pgfpathlineto{\pgfqpoint{3.023058in}{0.413320in}}%
\pgfpathlineto{\pgfqpoint{3.020382in}{0.413320in}}%
\pgfpathlineto{\pgfqpoint{3.017707in}{0.413320in}}%
\pgfpathlineto{\pgfqpoint{3.015097in}{0.413320in}}%
\pgfpathlineto{\pgfqpoint{3.012351in}{0.413320in}}%
\pgfpathlineto{\pgfqpoint{3.009784in}{0.413320in}}%
\pgfpathlineto{\pgfqpoint{3.006993in}{0.413320in}}%
\pgfpathlineto{\pgfqpoint{3.004419in}{0.413320in}}%
\pgfpathlineto{\pgfqpoint{3.001635in}{0.413320in}}%
\pgfpathlineto{\pgfqpoint{2.999103in}{0.413320in}}%
\pgfpathlineto{\pgfqpoint{2.996300in}{0.413320in}}%
\pgfpathlineto{\pgfqpoint{2.993595in}{0.413320in}}%
\pgfpathlineto{\pgfqpoint{2.990978in}{0.413320in}}%
\pgfpathlineto{\pgfqpoint{2.988238in}{0.413320in}}%
\pgfpathlineto{\pgfqpoint{2.985666in}{0.413320in}}%
\pgfpathlineto{\pgfqpoint{2.982885in}{0.413320in}}%
\pgfpathlineto{\pgfqpoint{2.980341in}{0.413320in}}%
\pgfpathlineto{\pgfqpoint{2.977517in}{0.413320in}}%
\pgfpathlineto{\pgfqpoint{2.974972in}{0.413320in}}%
\pgfpathlineto{\pgfqpoint{2.972177in}{0.413320in}}%
\pgfpathlineto{\pgfqpoint{2.969599in}{0.413320in}}%
\pgfpathlineto{\pgfqpoint{2.966812in}{0.413320in}}%
\pgfpathlineto{\pgfqpoint{2.964127in}{0.413320in}}%
\pgfpathlineto{\pgfqpoint{2.961460in}{0.413320in}}%
\pgfpathlineto{\pgfqpoint{2.958782in}{0.413320in}}%
\pgfpathlineto{\pgfqpoint{2.956103in}{0.413320in}}%
\pgfpathlineto{\pgfqpoint{2.953422in}{0.413320in}}%
\pgfpathlineto{\pgfqpoint{2.950884in}{0.413320in}}%
\pgfpathlineto{\pgfqpoint{2.948068in}{0.413320in}}%
\pgfpathlineto{\pgfqpoint{2.945461in}{0.413320in}}%
\pgfpathlineto{\pgfqpoint{2.942711in}{0.413320in}}%
\pgfpathlineto{\pgfqpoint{2.940120in}{0.413320in}}%
\pgfpathlineto{\pgfqpoint{2.937352in}{0.413320in}}%
\pgfpathlineto{\pgfqpoint{2.934759in}{0.413320in}}%
\pgfpathlineto{\pgfqpoint{2.932033in}{0.413320in}}%
\pgfpathlineto{\pgfqpoint{2.929321in}{0.413320in}}%
\pgfpathlineto{\pgfqpoint{2.926655in}{0.413320in}}%
\pgfpathlineto{\pgfqpoint{2.923963in}{0.413320in}}%
\pgfpathlineto{\pgfqpoint{2.921363in}{0.413320in}}%
\pgfpathlineto{\pgfqpoint{2.918606in}{0.413320in}}%
\pgfpathlineto{\pgfqpoint{2.916061in}{0.413320in}}%
\pgfpathlineto{\pgfqpoint{2.913243in}{0.413320in}}%
\pgfpathlineto{\pgfqpoint{2.910631in}{0.413320in}}%
\pgfpathlineto{\pgfqpoint{2.907882in}{0.413320in}}%
\pgfpathlineto{\pgfqpoint{2.905341in}{0.413320in}}%
\pgfpathlineto{\pgfqpoint{2.902535in}{0.413320in}}%
\pgfpathlineto{\pgfqpoint{2.899858in}{0.413320in}}%
\pgfpathlineto{\pgfqpoint{2.897179in}{0.413320in}}%
\pgfpathlineto{\pgfqpoint{2.894487in}{0.413320in}}%
\pgfpathlineto{\pgfqpoint{2.891809in}{0.413320in}}%
\pgfpathlineto{\pgfqpoint{2.889145in}{0.413320in}}%
\pgfpathlineto{\pgfqpoint{2.886578in}{0.413320in}}%
\pgfpathlineto{\pgfqpoint{2.883780in}{0.413320in}}%
\pgfpathlineto{\pgfqpoint{2.881254in}{0.413320in}}%
\pgfpathlineto{\pgfqpoint{2.878431in}{0.413320in}}%
\pgfpathlineto{\pgfqpoint{2.875882in}{0.413320in}}%
\pgfpathlineto{\pgfqpoint{2.873074in}{0.413320in}}%
\pgfpathlineto{\pgfqpoint{2.870475in}{0.413320in}}%
\pgfpathlineto{\pgfqpoint{2.867713in}{0.413320in}}%
\pgfpathlineto{\pgfqpoint{2.865031in}{0.413320in}}%
\pgfpathlineto{\pgfqpoint{2.862402in}{0.413320in}}%
\pgfpathlineto{\pgfqpoint{2.859668in}{0.413320in}}%
\pgfpathlineto{\pgfqpoint{2.857003in}{0.413320in}}%
\pgfpathlineto{\pgfqpoint{2.854325in}{0.413320in}}%
\pgfpathlineto{\pgfqpoint{2.851793in}{0.413320in}}%
\pgfpathlineto{\pgfqpoint{2.848960in}{0.413320in}}%
\pgfpathlineto{\pgfqpoint{2.846408in}{0.413320in}}%
\pgfpathlineto{\pgfqpoint{2.843611in}{0.413320in}}%
\pgfpathlineto{\pgfqpoint{2.841055in}{0.413320in}}%
\pgfpathlineto{\pgfqpoint{2.838254in}{0.413320in}}%
\pgfpathlineto{\pgfqpoint{2.835698in}{0.413320in}}%
\pgfpathlineto{\pgfqpoint{2.832894in}{0.413320in}}%
\pgfpathlineto{\pgfqpoint{2.830219in}{0.413320in}}%
\pgfpathlineto{\pgfqpoint{2.827567in}{0.413320in}}%
\pgfpathlineto{\pgfqpoint{2.824851in}{0.413320in}}%
\pgfpathlineto{\pgfqpoint{2.822303in}{0.413320in}}%
\pgfpathlineto{\pgfqpoint{2.819506in}{0.413320in}}%
\pgfpathlineto{\pgfqpoint{2.816867in}{0.413320in}}%
\pgfpathlineto{\pgfqpoint{2.814141in}{0.413320in}}%
\pgfpathlineto{\pgfqpoint{2.811597in}{0.413320in}}%
\pgfpathlineto{\pgfqpoint{2.808792in}{0.413320in}}%
\pgfpathlineto{\pgfqpoint{2.806175in}{0.413320in}}%
\pgfpathlineto{\pgfqpoint{2.803435in}{0.413320in}}%
\pgfpathlineto{\pgfqpoint{2.800756in}{0.413320in}}%
\pgfpathlineto{\pgfqpoint{2.798070in}{0.413320in}}%
\pgfpathlineto{\pgfqpoint{2.795398in}{0.413320in}}%
\pgfpathlineto{\pgfqpoint{2.792721in}{0.413320in}}%
\pgfpathlineto{\pgfqpoint{2.790044in}{0.413320in}}%
\pgfpathlineto{\pgfqpoint{2.787468in}{0.413320in}}%
\pgfpathlineto{\pgfqpoint{2.784687in}{0.413320in}}%
\pgfpathlineto{\pgfqpoint{2.782113in}{0.413320in}}%
\pgfpathlineto{\pgfqpoint{2.779330in}{0.413320in}}%
\pgfpathlineto{\pgfqpoint{2.776767in}{0.413320in}}%
\pgfpathlineto{\pgfqpoint{2.773972in}{0.413320in}}%
\pgfpathlineto{\pgfqpoint{2.771373in}{0.413320in}}%
\pgfpathlineto{\pgfqpoint{2.768617in}{0.413320in}}%
\pgfpathlineto{\pgfqpoint{2.765935in}{0.413320in}}%
\pgfpathlineto{\pgfqpoint{2.763253in}{0.413320in}}%
\pgfpathlineto{\pgfqpoint{2.760581in}{0.413320in}}%
\pgfpathlineto{\pgfqpoint{2.758028in}{0.413320in}}%
\pgfpathlineto{\pgfqpoint{2.755224in}{0.413320in}}%
\pgfpathlineto{\pgfqpoint{2.752614in}{0.413320in}}%
\pgfpathlineto{\pgfqpoint{2.749868in}{0.413320in}}%
\pgfpathlineto{\pgfqpoint{2.747260in}{0.413320in}}%
\pgfpathlineto{\pgfqpoint{2.744510in}{0.413320in}}%
\pgfpathlineto{\pgfqpoint{2.741928in}{0.413320in}}%
\pgfpathlineto{\pgfqpoint{2.739155in}{0.413320in}}%
\pgfpathlineto{\pgfqpoint{2.736476in}{0.413320in}}%
\pgfpathlineto{\pgfqpoint{2.733798in}{0.413320in}}%
\pgfpathlineto{\pgfqpoint{2.731119in}{0.413320in}}%
\pgfpathlineto{\pgfqpoint{2.728439in}{0.413320in}}%
\pgfpathlineto{\pgfqpoint{2.725760in}{0.413320in}}%
\pgfpathlineto{\pgfqpoint{2.723211in}{0.413320in}}%
\pgfpathlineto{\pgfqpoint{2.720404in}{0.413320in}}%
\pgfpathlineto{\pgfqpoint{2.717773in}{0.413320in}}%
\pgfpathlineto{\pgfqpoint{2.715036in}{0.413320in}}%
\pgfpathlineto{\pgfqpoint{2.712477in}{0.413320in}}%
\pgfpathlineto{\pgfqpoint{2.709683in}{0.413320in}}%
\pgfpathlineto{\pgfqpoint{2.707125in}{0.413320in}}%
\pgfpathlineto{\pgfqpoint{2.704326in}{0.413320in}}%
\pgfpathlineto{\pgfqpoint{2.701657in}{0.413320in}}%
\pgfpathlineto{\pgfqpoint{2.698968in}{0.413320in}}%
\pgfpathlineto{\pgfqpoint{2.696293in}{0.413320in}}%
\pgfpathlineto{\pgfqpoint{2.693611in}{0.413320in}}%
\pgfpathlineto{\pgfqpoint{2.690940in}{0.413320in}}%
\pgfpathlineto{\pgfqpoint{2.688328in}{0.413320in}}%
\pgfpathlineto{\pgfqpoint{2.685586in}{0.413320in}}%
\pgfpathlineto{\pgfqpoint{2.683009in}{0.413320in}}%
\pgfpathlineto{\pgfqpoint{2.680224in}{0.413320in}}%
\pgfpathlineto{\pgfqpoint{2.677650in}{0.413320in}}%
\pgfpathlineto{\pgfqpoint{2.674873in}{0.413320in}}%
\pgfpathlineto{\pgfqpoint{2.672301in}{0.413320in}}%
\pgfpathlineto{\pgfqpoint{2.669506in}{0.413320in}}%
\pgfpathlineto{\pgfqpoint{2.666836in}{0.413320in}}%
\pgfpathlineto{\pgfqpoint{2.664151in}{0.413320in}}%
\pgfpathlineto{\pgfqpoint{2.661481in}{0.413320in}}%
\pgfpathlineto{\pgfqpoint{2.658942in}{0.413320in}}%
\pgfpathlineto{\pgfqpoint{2.656124in}{0.413320in}}%
\pgfpathlineto{\pgfqpoint{2.653567in}{0.413320in}}%
\pgfpathlineto{\pgfqpoint{2.650767in}{0.413320in}}%
\pgfpathlineto{\pgfqpoint{2.648196in}{0.413320in}}%
\pgfpathlineto{\pgfqpoint{2.645408in}{0.413320in}}%
\pgfpathlineto{\pgfqpoint{2.642827in}{0.413320in}}%
\pgfpathlineto{\pgfqpoint{2.640053in}{0.413320in}}%
\pgfpathlineto{\pgfqpoint{2.637369in}{0.413320in}}%
\pgfpathlineto{\pgfqpoint{2.634700in}{0.413320in}}%
\pgfpathlineto{\pgfqpoint{2.632018in}{0.413320in}}%
\pgfpathlineto{\pgfqpoint{2.629340in}{0.413320in}}%
\pgfpathlineto{\pgfqpoint{2.626653in}{0.413320in}}%
\pgfpathlineto{\pgfqpoint{2.624077in}{0.413320in}}%
\pgfpathlineto{\pgfqpoint{2.621304in}{0.413320in}}%
\pgfpathlineto{\pgfqpoint{2.618773in}{0.413320in}}%
\pgfpathlineto{\pgfqpoint{2.615934in}{0.413320in}}%
\pgfpathlineto{\pgfqpoint{2.613393in}{0.413320in}}%
\pgfpathlineto{\pgfqpoint{2.610588in}{0.413320in}}%
\pgfpathlineto{\pgfqpoint{2.608004in}{0.413320in}}%
\pgfpathlineto{\pgfqpoint{2.605232in}{0.413320in}}%
\pgfpathlineto{\pgfqpoint{2.602557in}{0.413320in}}%
\pgfpathlineto{\pgfqpoint{2.599920in}{0.413320in}}%
\pgfpathlineto{\pgfqpoint{2.597196in}{0.413320in}}%
\pgfpathlineto{\pgfqpoint{2.594630in}{0.413320in}}%
\pgfpathlineto{\pgfqpoint{2.591842in}{0.413320in}}%
\pgfpathlineto{\pgfqpoint{2.589248in}{0.413320in}}%
\pgfpathlineto{\pgfqpoint{2.586484in}{0.413320in}}%
\pgfpathlineto{\pgfqpoint{2.583913in}{0.413320in}}%
\pgfpathlineto{\pgfqpoint{2.581129in}{0.413320in}}%
\pgfpathlineto{\pgfqpoint{2.578567in}{0.413320in}}%
\pgfpathlineto{\pgfqpoint{2.575779in}{0.413320in}}%
\pgfpathlineto{\pgfqpoint{2.573082in}{0.413320in}}%
\pgfpathlineto{\pgfqpoint{2.570411in}{0.413320in}}%
\pgfpathlineto{\pgfqpoint{2.567730in}{0.413320in}}%
\pgfpathlineto{\pgfqpoint{2.565045in}{0.413320in}}%
\pgfpathlineto{\pgfqpoint{2.562375in}{0.413320in}}%
\pgfpathlineto{\pgfqpoint{2.559790in}{0.413320in}}%
\pgfpathlineto{\pgfqpoint{2.557009in}{0.413320in}}%
\pgfpathlineto{\pgfqpoint{2.554493in}{0.413320in}}%
\pgfpathlineto{\pgfqpoint{2.551664in}{0.413320in}}%
\pgfpathlineto{\pgfqpoint{2.549114in}{0.413320in}}%
\pgfpathlineto{\pgfqpoint{2.546310in}{0.413320in}}%
\pgfpathlineto{\pgfqpoint{2.543765in}{0.413320in}}%
\pgfpathlineto{\pgfqpoint{2.540949in}{0.413320in}}%
\pgfpathlineto{\pgfqpoint{2.538274in}{0.413320in}}%
\pgfpathlineto{\pgfqpoint{2.535624in}{0.413320in}}%
\pgfpathlineto{\pgfqpoint{2.532917in}{0.413320in}}%
\pgfpathlineto{\pgfqpoint{2.530234in}{0.413320in}}%
\pgfpathlineto{\pgfqpoint{2.527560in}{0.413320in}}%
\pgfpathlineto{\pgfqpoint{2.524988in}{0.413320in}}%
\pgfpathlineto{\pgfqpoint{2.522197in}{0.413320in}}%
\pgfpathlineto{\pgfqpoint{2.519607in}{0.413320in}}%
\pgfpathlineto{\pgfqpoint{2.516845in}{0.413320in}}%
\pgfpathlineto{\pgfqpoint{2.514268in}{0.413320in}}%
\pgfpathlineto{\pgfqpoint{2.511478in}{0.413320in}}%
\pgfpathlineto{\pgfqpoint{2.508917in}{0.413320in}}%
\pgfpathlineto{\pgfqpoint{2.506163in}{0.413320in}}%
\pgfpathlineto{\pgfqpoint{2.503454in}{0.413320in}}%
\pgfpathlineto{\pgfqpoint{2.500801in}{0.413320in}}%
\pgfpathlineto{\pgfqpoint{2.498085in}{0.413320in}}%
\pgfpathlineto{\pgfqpoint{2.495542in}{0.413320in}}%
\pgfpathlineto{\pgfqpoint{2.492729in}{0.413320in}}%
\pgfpathlineto{\pgfqpoint{2.490183in}{0.413320in}}%
\pgfpathlineto{\pgfqpoint{2.487384in}{0.413320in}}%
\pgfpathlineto{\pgfqpoint{2.484870in}{0.413320in}}%
\pgfpathlineto{\pgfqpoint{2.482026in}{0.413320in}}%
\pgfpathlineto{\pgfqpoint{2.479420in}{0.413320in}}%
\pgfpathlineto{\pgfqpoint{2.476671in}{0.413320in}}%
\pgfpathlineto{\pgfqpoint{2.473989in}{0.413320in}}%
\pgfpathlineto{\pgfqpoint{2.471311in}{0.413320in}}%
\pgfpathlineto{\pgfqpoint{2.468635in}{0.413320in}}%
\pgfpathlineto{\pgfqpoint{2.465957in}{0.413320in}}%
\pgfpathlineto{\pgfqpoint{2.463280in}{0.413320in}}%
\pgfpathlineto{\pgfqpoint{2.460711in}{0.413320in}}%
\pgfpathlineto{\pgfqpoint{2.457917in}{0.413320in}}%
\pgfpathlineto{\pgfqpoint{2.455353in}{0.413320in}}%
\pgfpathlineto{\pgfqpoint{2.452562in}{0.413320in}}%
\pgfpathlineto{\pgfqpoint{2.450032in}{0.413320in}}%
\pgfpathlineto{\pgfqpoint{2.447209in}{0.413320in}}%
\pgfpathlineto{\pgfqpoint{2.444677in}{0.413320in}}%
\pgfpathlineto{\pgfqpoint{2.441876in}{0.413320in}}%
\pgfpathlineto{\pgfqpoint{2.439167in}{0.413320in}}%
\pgfpathlineto{\pgfqpoint{2.436518in}{0.413320in}}%
\pgfpathlineto{\pgfqpoint{2.433815in}{0.413320in}}%
\pgfpathlineto{\pgfqpoint{2.431251in}{0.413320in}}%
\pgfpathlineto{\pgfqpoint{2.428453in}{0.413320in}}%
\pgfpathlineto{\pgfqpoint{2.425878in}{0.413320in}}%
\pgfpathlineto{\pgfqpoint{2.423098in}{0.413320in}}%
\pgfpathlineto{\pgfqpoint{2.420528in}{0.413320in}}%
\pgfpathlineto{\pgfqpoint{2.417747in}{0.413320in}}%
\pgfpathlineto{\pgfqpoint{2.415184in}{0.413320in}}%
\pgfpathlineto{\pgfqpoint{2.412389in}{0.413320in}}%
\pgfpathlineto{\pgfqpoint{2.409699in}{0.413320in}}%
\pgfpathlineto{\pgfqpoint{2.407024in}{0.413320in}}%
\pgfpathlineto{\pgfqpoint{2.404352in}{0.413320in}}%
\pgfpathlineto{\pgfqpoint{2.401675in}{0.413320in}}%
\pgfpathlineto{\pgfqpoint{2.398995in}{0.413320in}}%
\pgfpathclose%
\pgfusepath{stroke,fill}%
\end{pgfscope}%
\begin{pgfscope}%
\pgfpathrectangle{\pgfqpoint{2.398995in}{0.319877in}}{\pgfqpoint{3.986877in}{1.993438in}} %
\pgfusepath{clip}%
\pgfsetbuttcap%
\pgfsetroundjoin%
\definecolor{currentfill}{rgb}{1.000000,1.000000,1.000000}%
\pgfsetfillcolor{currentfill}%
\pgfsetlinewidth{1.003750pt}%
\definecolor{currentstroke}{rgb}{0.504902,0.590912,0.958466}%
\pgfsetstrokecolor{currentstroke}%
\pgfsetdash{}{0pt}%
\pgfpathmoveto{\pgfqpoint{2.398995in}{0.413320in}}%
\pgfpathlineto{\pgfqpoint{2.398995in}{0.667499in}}%
\pgfpathlineto{\pgfqpoint{2.401675in}{0.666406in}}%
\pgfpathlineto{\pgfqpoint{2.404352in}{0.670491in}}%
\pgfpathlineto{\pgfqpoint{2.407024in}{0.667399in}}%
\pgfpathlineto{\pgfqpoint{2.409699in}{0.666832in}}%
\pgfpathlineto{\pgfqpoint{2.412389in}{0.666510in}}%
\pgfpathlineto{\pgfqpoint{2.415184in}{0.668154in}}%
\pgfpathlineto{\pgfqpoint{2.417747in}{0.665635in}}%
\pgfpathlineto{\pgfqpoint{2.420528in}{0.664839in}}%
\pgfpathlineto{\pgfqpoint{2.423098in}{0.665958in}}%
\pgfpathlineto{\pgfqpoint{2.425878in}{0.673116in}}%
\pgfpathlineto{\pgfqpoint{2.428453in}{0.670178in}}%
\pgfpathlineto{\pgfqpoint{2.431251in}{0.668210in}}%
\pgfpathlineto{\pgfqpoint{2.433815in}{0.669849in}}%
\pgfpathlineto{\pgfqpoint{2.436518in}{0.677173in}}%
\pgfpathlineto{\pgfqpoint{2.439167in}{0.676448in}}%
\pgfpathlineto{\pgfqpoint{2.441876in}{0.676965in}}%
\pgfpathlineto{\pgfqpoint{2.444677in}{0.673690in}}%
\pgfpathlineto{\pgfqpoint{2.447209in}{0.670645in}}%
\pgfpathlineto{\pgfqpoint{2.450032in}{0.667947in}}%
\pgfpathlineto{\pgfqpoint{2.452562in}{0.662616in}}%
\pgfpathlineto{\pgfqpoint{2.455353in}{0.665092in}}%
\pgfpathlineto{\pgfqpoint{2.457917in}{0.666768in}}%
\pgfpathlineto{\pgfqpoint{2.460711in}{0.665773in}}%
\pgfpathlineto{\pgfqpoint{2.463280in}{0.667565in}}%
\pgfpathlineto{\pgfqpoint{2.465957in}{0.666915in}}%
\pgfpathlineto{\pgfqpoint{2.468635in}{0.665584in}}%
\pgfpathlineto{\pgfqpoint{2.471311in}{0.666463in}}%
\pgfpathlineto{\pgfqpoint{2.473989in}{0.663627in}}%
\pgfpathlineto{\pgfqpoint{2.476671in}{0.664882in}}%
\pgfpathlineto{\pgfqpoint{2.479420in}{0.663556in}}%
\pgfpathlineto{\pgfqpoint{2.482026in}{0.664392in}}%
\pgfpathlineto{\pgfqpoint{2.484870in}{0.665076in}}%
\pgfpathlineto{\pgfqpoint{2.487384in}{0.659894in}}%
\pgfpathlineto{\pgfqpoint{2.490183in}{0.662025in}}%
\pgfpathlineto{\pgfqpoint{2.492729in}{0.661136in}}%
\pgfpathlineto{\pgfqpoint{2.495542in}{0.661763in}}%
\pgfpathlineto{\pgfqpoint{2.498085in}{0.660617in}}%
\pgfpathlineto{\pgfqpoint{2.500801in}{0.662593in}}%
\pgfpathlineto{\pgfqpoint{2.503454in}{0.662387in}}%
\pgfpathlineto{\pgfqpoint{2.506163in}{0.667100in}}%
\pgfpathlineto{\pgfqpoint{2.508917in}{0.666642in}}%
\pgfpathlineto{\pgfqpoint{2.511478in}{0.663376in}}%
\pgfpathlineto{\pgfqpoint{2.514268in}{0.664267in}}%
\pgfpathlineto{\pgfqpoint{2.516845in}{0.667430in}}%
\pgfpathlineto{\pgfqpoint{2.519607in}{0.667146in}}%
\pgfpathlineto{\pgfqpoint{2.522197in}{0.667992in}}%
\pgfpathlineto{\pgfqpoint{2.524988in}{0.666781in}}%
\pgfpathlineto{\pgfqpoint{2.527560in}{0.666021in}}%
\pgfpathlineto{\pgfqpoint{2.530234in}{0.664368in}}%
\pgfpathlineto{\pgfqpoint{2.532917in}{0.664343in}}%
\pgfpathlineto{\pgfqpoint{2.535624in}{0.662439in}}%
\pgfpathlineto{\pgfqpoint{2.538274in}{0.661538in}}%
\pgfpathlineto{\pgfqpoint{2.540949in}{0.663167in}}%
\pgfpathlineto{\pgfqpoint{2.543765in}{0.663169in}}%
\pgfpathlineto{\pgfqpoint{2.546310in}{0.665457in}}%
\pgfpathlineto{\pgfqpoint{2.549114in}{0.664725in}}%
\pgfpathlineto{\pgfqpoint{2.551664in}{0.663824in}}%
\pgfpathlineto{\pgfqpoint{2.554493in}{0.669402in}}%
\pgfpathlineto{\pgfqpoint{2.557009in}{0.671477in}}%
\pgfpathlineto{\pgfqpoint{2.559790in}{0.665798in}}%
\pgfpathlineto{\pgfqpoint{2.562375in}{0.667947in}}%
\pgfpathlineto{\pgfqpoint{2.565045in}{0.668229in}}%
\pgfpathlineto{\pgfqpoint{2.567730in}{0.667617in}}%
\pgfpathlineto{\pgfqpoint{2.570411in}{0.667555in}}%
\pgfpathlineto{\pgfqpoint{2.573082in}{0.672442in}}%
\pgfpathlineto{\pgfqpoint{2.575779in}{0.677476in}}%
\pgfpathlineto{\pgfqpoint{2.578567in}{0.672900in}}%
\pgfpathlineto{\pgfqpoint{2.581129in}{0.675466in}}%
\pgfpathlineto{\pgfqpoint{2.583913in}{0.668742in}}%
\pgfpathlineto{\pgfqpoint{2.586484in}{0.666122in}}%
\pgfpathlineto{\pgfqpoint{2.589248in}{0.663989in}}%
\pgfpathlineto{\pgfqpoint{2.591842in}{0.668871in}}%
\pgfpathlineto{\pgfqpoint{2.594630in}{0.667027in}}%
\pgfpathlineto{\pgfqpoint{2.597196in}{0.664270in}}%
\pgfpathlineto{\pgfqpoint{2.599920in}{0.665295in}}%
\pgfpathlineto{\pgfqpoint{2.602557in}{0.663990in}}%
\pgfpathlineto{\pgfqpoint{2.605232in}{0.660713in}}%
\pgfpathlineto{\pgfqpoint{2.608004in}{0.665639in}}%
\pgfpathlineto{\pgfqpoint{2.610588in}{0.664922in}}%
\pgfpathlineto{\pgfqpoint{2.613393in}{0.669468in}}%
\pgfpathlineto{\pgfqpoint{2.615934in}{0.668461in}}%
\pgfpathlineto{\pgfqpoint{2.618773in}{0.669939in}}%
\pgfpathlineto{\pgfqpoint{2.621304in}{0.669960in}}%
\pgfpathlineto{\pgfqpoint{2.624077in}{0.668580in}}%
\pgfpathlineto{\pgfqpoint{2.626653in}{0.663788in}}%
\pgfpathlineto{\pgfqpoint{2.629340in}{0.665711in}}%
\pgfpathlineto{\pgfqpoint{2.632018in}{0.667650in}}%
\pgfpathlineto{\pgfqpoint{2.634700in}{0.665618in}}%
\pgfpathlineto{\pgfqpoint{2.637369in}{0.664879in}}%
\pgfpathlineto{\pgfqpoint{2.640053in}{0.665125in}}%
\pgfpathlineto{\pgfqpoint{2.642827in}{0.665669in}}%
\pgfpathlineto{\pgfqpoint{2.645408in}{0.664963in}}%
\pgfpathlineto{\pgfqpoint{2.648196in}{0.669520in}}%
\pgfpathlineto{\pgfqpoint{2.650767in}{0.672641in}}%
\pgfpathlineto{\pgfqpoint{2.653567in}{0.671028in}}%
\pgfpathlineto{\pgfqpoint{2.656124in}{0.668778in}}%
\pgfpathlineto{\pgfqpoint{2.658942in}{0.671802in}}%
\pgfpathlineto{\pgfqpoint{2.661481in}{0.666211in}}%
\pgfpathlineto{\pgfqpoint{2.664151in}{0.663390in}}%
\pgfpathlineto{\pgfqpoint{2.666836in}{0.668274in}}%
\pgfpathlineto{\pgfqpoint{2.669506in}{0.669723in}}%
\pgfpathlineto{\pgfqpoint{2.672301in}{0.671322in}}%
\pgfpathlineto{\pgfqpoint{2.674873in}{0.667496in}}%
\pgfpathlineto{\pgfqpoint{2.677650in}{0.666230in}}%
\pgfpathlineto{\pgfqpoint{2.680224in}{0.666393in}}%
\pgfpathlineto{\pgfqpoint{2.683009in}{0.664437in}}%
\pgfpathlineto{\pgfqpoint{2.685586in}{0.666330in}}%
\pgfpathlineto{\pgfqpoint{2.688328in}{0.661731in}}%
\pgfpathlineto{\pgfqpoint{2.690940in}{0.665028in}}%
\pgfpathlineto{\pgfqpoint{2.693611in}{0.665718in}}%
\pgfpathlineto{\pgfqpoint{2.696293in}{0.664883in}}%
\pgfpathlineto{\pgfqpoint{2.698968in}{0.664919in}}%
\pgfpathlineto{\pgfqpoint{2.701657in}{0.661953in}}%
\pgfpathlineto{\pgfqpoint{2.704326in}{0.667748in}}%
\pgfpathlineto{\pgfqpoint{2.707125in}{0.666962in}}%
\pgfpathlineto{\pgfqpoint{2.709683in}{0.664750in}}%
\pgfpathlineto{\pgfqpoint{2.712477in}{0.667685in}}%
\pgfpathlineto{\pgfqpoint{2.715036in}{0.663611in}}%
\pgfpathlineto{\pgfqpoint{2.717773in}{0.662951in}}%
\pgfpathlineto{\pgfqpoint{2.720404in}{0.666404in}}%
\pgfpathlineto{\pgfqpoint{2.723211in}{0.663904in}}%
\pgfpathlineto{\pgfqpoint{2.725760in}{0.663023in}}%
\pgfpathlineto{\pgfqpoint{2.728439in}{0.659894in}}%
\pgfpathlineto{\pgfqpoint{2.731119in}{0.660475in}}%
\pgfpathlineto{\pgfqpoint{2.733798in}{0.659894in}}%
\pgfpathlineto{\pgfqpoint{2.736476in}{0.659894in}}%
\pgfpathlineto{\pgfqpoint{2.739155in}{0.659894in}}%
\pgfpathlineto{\pgfqpoint{2.741928in}{0.659894in}}%
\pgfpathlineto{\pgfqpoint{2.744510in}{0.663315in}}%
\pgfpathlineto{\pgfqpoint{2.747260in}{0.664048in}}%
\pgfpathlineto{\pgfqpoint{2.749868in}{0.666255in}}%
\pgfpathlineto{\pgfqpoint{2.752614in}{0.663070in}}%
\pgfpathlineto{\pgfqpoint{2.755224in}{0.665772in}}%
\pgfpathlineto{\pgfqpoint{2.758028in}{0.665611in}}%
\pgfpathlineto{\pgfqpoint{2.760581in}{0.665985in}}%
\pgfpathlineto{\pgfqpoint{2.763253in}{0.664191in}}%
\pgfpathlineto{\pgfqpoint{2.765935in}{0.664376in}}%
\pgfpathlineto{\pgfqpoint{2.768617in}{0.662862in}}%
\pgfpathlineto{\pgfqpoint{2.771373in}{0.661455in}}%
\pgfpathlineto{\pgfqpoint{2.773972in}{0.659894in}}%
\pgfpathlineto{\pgfqpoint{2.776767in}{0.662128in}}%
\pgfpathlineto{\pgfqpoint{2.779330in}{0.659894in}}%
\pgfpathlineto{\pgfqpoint{2.782113in}{0.659894in}}%
\pgfpathlineto{\pgfqpoint{2.784687in}{0.659894in}}%
\pgfpathlineto{\pgfqpoint{2.787468in}{0.659894in}}%
\pgfpathlineto{\pgfqpoint{2.790044in}{0.659894in}}%
\pgfpathlineto{\pgfqpoint{2.792721in}{0.663020in}}%
\pgfpathlineto{\pgfqpoint{2.795398in}{0.666164in}}%
\pgfpathlineto{\pgfqpoint{2.798070in}{0.664859in}}%
\pgfpathlineto{\pgfqpoint{2.800756in}{0.671671in}}%
\pgfpathlineto{\pgfqpoint{2.803435in}{0.670519in}}%
\pgfpathlineto{\pgfqpoint{2.806175in}{0.664383in}}%
\pgfpathlineto{\pgfqpoint{2.808792in}{0.664905in}}%
\pgfpathlineto{\pgfqpoint{2.811597in}{0.669889in}}%
\pgfpathlineto{\pgfqpoint{2.814141in}{0.666766in}}%
\pgfpathlineto{\pgfqpoint{2.816867in}{0.668816in}}%
\pgfpathlineto{\pgfqpoint{2.819506in}{0.668989in}}%
\pgfpathlineto{\pgfqpoint{2.822303in}{0.667937in}}%
\pgfpathlineto{\pgfqpoint{2.824851in}{0.662789in}}%
\pgfpathlineto{\pgfqpoint{2.827567in}{0.665857in}}%
\pgfpathlineto{\pgfqpoint{2.830219in}{0.662202in}}%
\pgfpathlineto{\pgfqpoint{2.832894in}{0.665219in}}%
\pgfpathlineto{\pgfqpoint{2.835698in}{0.664645in}}%
\pgfpathlineto{\pgfqpoint{2.838254in}{0.664322in}}%
\pgfpathlineto{\pgfqpoint{2.841055in}{0.664371in}}%
\pgfpathlineto{\pgfqpoint{2.843611in}{0.664727in}}%
\pgfpathlineto{\pgfqpoint{2.846408in}{0.667739in}}%
\pgfpathlineto{\pgfqpoint{2.848960in}{0.666914in}}%
\pgfpathlineto{\pgfqpoint{2.851793in}{0.666281in}}%
\pgfpathlineto{\pgfqpoint{2.854325in}{0.669625in}}%
\pgfpathlineto{\pgfqpoint{2.857003in}{0.666246in}}%
\pgfpathlineto{\pgfqpoint{2.859668in}{0.665192in}}%
\pgfpathlineto{\pgfqpoint{2.862402in}{0.662827in}}%
\pgfpathlineto{\pgfqpoint{2.865031in}{0.669602in}}%
\pgfpathlineto{\pgfqpoint{2.867713in}{0.668735in}}%
\pgfpathlineto{\pgfqpoint{2.870475in}{0.667081in}}%
\pgfpathlineto{\pgfqpoint{2.873074in}{0.666072in}}%
\pgfpathlineto{\pgfqpoint{2.875882in}{0.665687in}}%
\pgfpathlineto{\pgfqpoint{2.878431in}{0.666562in}}%
\pgfpathlineto{\pgfqpoint{2.881254in}{0.665947in}}%
\pgfpathlineto{\pgfqpoint{2.883780in}{0.667860in}}%
\pgfpathlineto{\pgfqpoint{2.886578in}{0.664920in}}%
\pgfpathlineto{\pgfqpoint{2.889145in}{0.666356in}}%
\pgfpathlineto{\pgfqpoint{2.891809in}{0.662598in}}%
\pgfpathlineto{\pgfqpoint{2.894487in}{0.660913in}}%
\pgfpathlineto{\pgfqpoint{2.897179in}{0.663401in}}%
\pgfpathlineto{\pgfqpoint{2.899858in}{0.663402in}}%
\pgfpathlineto{\pgfqpoint{2.902535in}{0.662302in}}%
\pgfpathlineto{\pgfqpoint{2.905341in}{0.661694in}}%
\pgfpathlineto{\pgfqpoint{2.907882in}{0.664711in}}%
\pgfpathlineto{\pgfqpoint{2.910631in}{0.660322in}}%
\pgfpathlineto{\pgfqpoint{2.913243in}{0.664583in}}%
\pgfpathlineto{\pgfqpoint{2.916061in}{0.665360in}}%
\pgfpathlineto{\pgfqpoint{2.918606in}{0.665636in}}%
\pgfpathlineto{\pgfqpoint{2.921363in}{0.666868in}}%
\pgfpathlineto{\pgfqpoint{2.923963in}{0.666472in}}%
\pgfpathlineto{\pgfqpoint{2.926655in}{0.664521in}}%
\pgfpathlineto{\pgfqpoint{2.929321in}{0.665906in}}%
\pgfpathlineto{\pgfqpoint{2.932033in}{0.667019in}}%
\pgfpathlineto{\pgfqpoint{2.934759in}{0.667924in}}%
\pgfpathlineto{\pgfqpoint{2.937352in}{0.665956in}}%
\pgfpathlineto{\pgfqpoint{2.940120in}{0.664041in}}%
\pgfpathlineto{\pgfqpoint{2.942711in}{0.659894in}}%
\pgfpathlineto{\pgfqpoint{2.945461in}{0.662611in}}%
\pgfpathlineto{\pgfqpoint{2.948068in}{0.665735in}}%
\pgfpathlineto{\pgfqpoint{2.950884in}{0.665090in}}%
\pgfpathlineto{\pgfqpoint{2.953422in}{0.661325in}}%
\pgfpathlineto{\pgfqpoint{2.956103in}{0.662881in}}%
\pgfpathlineto{\pgfqpoint{2.958782in}{0.664207in}}%
\pgfpathlineto{\pgfqpoint{2.961460in}{0.666197in}}%
\pgfpathlineto{\pgfqpoint{2.964127in}{0.667776in}}%
\pgfpathlineto{\pgfqpoint{2.966812in}{0.665348in}}%
\pgfpathlineto{\pgfqpoint{2.969599in}{0.668448in}}%
\pgfpathlineto{\pgfqpoint{2.972177in}{0.665957in}}%
\pgfpathlineto{\pgfqpoint{2.974972in}{0.666301in}}%
\pgfpathlineto{\pgfqpoint{2.977517in}{0.668301in}}%
\pgfpathlineto{\pgfqpoint{2.980341in}{0.669526in}}%
\pgfpathlineto{\pgfqpoint{2.982885in}{0.670490in}}%
\pgfpathlineto{\pgfqpoint{2.985666in}{0.669072in}}%
\pgfpathlineto{\pgfqpoint{2.988238in}{0.670447in}}%
\pgfpathlineto{\pgfqpoint{2.990978in}{0.668493in}}%
\pgfpathlineto{\pgfqpoint{2.993595in}{0.668256in}}%
\pgfpathlineto{\pgfqpoint{2.996300in}{0.668449in}}%
\pgfpathlineto{\pgfqpoint{2.999103in}{0.673100in}}%
\pgfpathlineto{\pgfqpoint{3.001635in}{0.666037in}}%
\pgfpathlineto{\pgfqpoint{3.004419in}{0.665432in}}%
\pgfpathlineto{\pgfqpoint{3.006993in}{0.663756in}}%
\pgfpathlineto{\pgfqpoint{3.009784in}{0.665673in}}%
\pgfpathlineto{\pgfqpoint{3.012351in}{0.668252in}}%
\pgfpathlineto{\pgfqpoint{3.015097in}{0.667762in}}%
\pgfpathlineto{\pgfqpoint{3.017707in}{0.670689in}}%
\pgfpathlineto{\pgfqpoint{3.020382in}{0.669995in}}%
\pgfpathlineto{\pgfqpoint{3.023058in}{0.665053in}}%
\pgfpathlineto{\pgfqpoint{3.025803in}{0.666122in}}%
\pgfpathlineto{\pgfqpoint{3.028412in}{0.668584in}}%
\pgfpathlineto{\pgfqpoint{3.031091in}{0.665461in}}%
\pgfpathlineto{\pgfqpoint{3.033921in}{0.665567in}}%
\pgfpathlineto{\pgfqpoint{3.036456in}{0.666943in}}%
\pgfpathlineto{\pgfqpoint{3.039262in}{0.664604in}}%
\pgfpathlineto{\pgfqpoint{3.041813in}{0.664293in}}%
\pgfpathlineto{\pgfqpoint{3.044568in}{0.664308in}}%
\pgfpathlineto{\pgfqpoint{3.047157in}{0.671970in}}%
\pgfpathlineto{\pgfqpoint{3.049988in}{0.686268in}}%
\pgfpathlineto{\pgfqpoint{3.052526in}{0.676975in}}%
\pgfpathlineto{\pgfqpoint{3.055202in}{0.668506in}}%
\pgfpathlineto{\pgfqpoint{3.057884in}{0.663488in}}%
\pgfpathlineto{\pgfqpoint{3.060561in}{0.661585in}}%
\pgfpathlineto{\pgfqpoint{3.063230in}{0.659894in}}%
\pgfpathlineto{\pgfqpoint{3.065916in}{0.662485in}}%
\pgfpathlineto{\pgfqpoint{3.068709in}{0.661478in}}%
\pgfpathlineto{\pgfqpoint{3.071266in}{0.663065in}}%
\pgfpathlineto{\pgfqpoint{3.074056in}{0.666411in}}%
\pgfpathlineto{\pgfqpoint{3.076631in}{0.663096in}}%
\pgfpathlineto{\pgfqpoint{3.079381in}{0.670843in}}%
\pgfpathlineto{\pgfqpoint{3.081990in}{0.671728in}}%
\pgfpathlineto{\pgfqpoint{3.084671in}{0.670331in}}%
\pgfpathlineto{\pgfqpoint{3.087343in}{0.670554in}}%
\pgfpathlineto{\pgfqpoint{3.090023in}{0.662919in}}%
\pgfpathlineto{\pgfqpoint{3.092699in}{0.660625in}}%
\pgfpathlineto{\pgfqpoint{3.095388in}{0.660827in}}%
\pgfpathlineto{\pgfqpoint{3.098163in}{0.659894in}}%
\pgfpathlineto{\pgfqpoint{3.100737in}{0.659894in}}%
\pgfpathlineto{\pgfqpoint{3.103508in}{0.659894in}}%
\pgfpathlineto{\pgfqpoint{3.106094in}{0.663342in}}%
\pgfpathlineto{\pgfqpoint{3.108896in}{0.667562in}}%
\pgfpathlineto{\pgfqpoint{3.111451in}{0.669857in}}%
\pgfpathlineto{\pgfqpoint{3.114242in}{0.669044in}}%
\pgfpathlineto{\pgfqpoint{3.116807in}{0.674640in}}%
\pgfpathlineto{\pgfqpoint{3.119487in}{0.666849in}}%
\pgfpathlineto{\pgfqpoint{3.122163in}{0.661487in}}%
\pgfpathlineto{\pgfqpoint{3.124842in}{0.659894in}}%
\pgfpathlineto{\pgfqpoint{3.127512in}{0.660642in}}%
\pgfpathlineto{\pgfqpoint{3.130199in}{0.663841in}}%
\pgfpathlineto{\pgfqpoint{3.132946in}{0.660887in}}%
\pgfpathlineto{\pgfqpoint{3.135550in}{0.661255in}}%
\pgfpathlineto{\pgfqpoint{3.138375in}{0.659894in}}%
\pgfpathlineto{\pgfqpoint{3.140913in}{0.659894in}}%
\pgfpathlineto{\pgfqpoint{3.143740in}{0.661646in}}%
\pgfpathlineto{\pgfqpoint{3.146271in}{0.659894in}}%
\pgfpathlineto{\pgfqpoint{3.149057in}{0.659894in}}%
\pgfpathlineto{\pgfqpoint{3.151612in}{0.659894in}}%
\pgfpathlineto{\pgfqpoint{3.154327in}{0.659894in}}%
\pgfpathlineto{\pgfqpoint{3.156981in}{0.659894in}}%
\pgfpathlineto{\pgfqpoint{3.159675in}{0.659894in}}%
\pgfpathlineto{\pgfqpoint{3.162474in}{0.659894in}}%
\pgfpathlineto{\pgfqpoint{3.165019in}{0.659894in}}%
\pgfpathlineto{\pgfqpoint{3.167776in}{0.659894in}}%
\pgfpathlineto{\pgfqpoint{3.170375in}{0.659894in}}%
\pgfpathlineto{\pgfqpoint{3.173142in}{0.659894in}}%
\pgfpathlineto{\pgfqpoint{3.175724in}{0.659894in}}%
\pgfpathlineto{\pgfqpoint{3.178525in}{0.659894in}}%
\pgfpathlineto{\pgfqpoint{3.181089in}{0.659894in}}%
\pgfpathlineto{\pgfqpoint{3.183760in}{0.659894in}}%
\pgfpathlineto{\pgfqpoint{3.186440in}{0.659894in}}%
\pgfpathlineto{\pgfqpoint{3.189117in}{0.659894in}}%
\pgfpathlineto{\pgfqpoint{3.191796in}{0.659894in}}%
\pgfpathlineto{\pgfqpoint{3.194508in}{0.659894in}}%
\pgfpathlineto{\pgfqpoint{3.197226in}{0.659894in}}%
\pgfpathlineto{\pgfqpoint{3.199823in}{0.659894in}}%
\pgfpathlineto{\pgfqpoint{3.202562in}{0.659894in}}%
\pgfpathlineto{\pgfqpoint{3.205195in}{0.659894in}}%
\pgfpathlineto{\pgfqpoint{3.207984in}{0.659894in}}%
\pgfpathlineto{\pgfqpoint{3.210545in}{0.659894in}}%
\pgfpathlineto{\pgfqpoint{3.213342in}{0.659894in}}%
\pgfpathlineto{\pgfqpoint{3.215908in}{0.659894in}}%
\pgfpathlineto{\pgfqpoint{3.218586in}{0.659894in}}%
\pgfpathlineto{\pgfqpoint{3.221255in}{0.659894in}}%
\pgfpathlineto{\pgfqpoint{3.223942in}{0.661404in}}%
\pgfpathlineto{\pgfqpoint{3.226609in}{0.659894in}}%
\pgfpathlineto{\pgfqpoint{3.229310in}{0.659894in}}%
\pgfpathlineto{\pgfqpoint{3.232069in}{0.659894in}}%
\pgfpathlineto{\pgfqpoint{3.234658in}{0.662612in}}%
\pgfpathlineto{\pgfqpoint{3.237411in}{0.659894in}}%
\pgfpathlineto{\pgfqpoint{3.240010in}{0.661116in}}%
\pgfpathlineto{\pgfqpoint{3.242807in}{0.660038in}}%
\pgfpathlineto{\pgfqpoint{3.245363in}{0.662610in}}%
\pgfpathlineto{\pgfqpoint{3.248049in}{0.665920in}}%
\pgfpathlineto{\pgfqpoint{3.250716in}{0.662387in}}%
\pgfpathlineto{\pgfqpoint{3.253404in}{0.664124in}}%
\pgfpathlineto{\pgfqpoint{3.256083in}{0.667494in}}%
\pgfpathlineto{\pgfqpoint{3.258784in}{0.669591in}}%
\pgfpathlineto{\pgfqpoint{3.261594in}{0.666926in}}%
\pgfpathlineto{\pgfqpoint{3.264119in}{0.671378in}}%
\pgfpathlineto{\pgfqpoint{3.266849in}{0.672160in}}%
\pgfpathlineto{\pgfqpoint{3.269478in}{0.672983in}}%
\pgfpathlineto{\pgfqpoint{3.272254in}{0.668667in}}%
\pgfpathlineto{\pgfqpoint{3.274831in}{0.671824in}}%
\pgfpathlineto{\pgfqpoint{3.277603in}{0.669984in}}%
\pgfpathlineto{\pgfqpoint{3.280189in}{0.667854in}}%
\pgfpathlineto{\pgfqpoint{3.282870in}{0.671881in}}%
\pgfpathlineto{\pgfqpoint{3.285534in}{0.668247in}}%
\pgfpathlineto{\pgfqpoint{3.288225in}{0.666588in}}%
\pgfpathlineto{\pgfqpoint{3.290890in}{0.665318in}}%
\pgfpathlineto{\pgfqpoint{3.293574in}{0.672653in}}%
\pgfpathlineto{\pgfqpoint{3.296376in}{0.670217in}}%
\pgfpathlineto{\pgfqpoint{3.298937in}{0.667582in}}%
\pgfpathlineto{\pgfqpoint{3.301719in}{0.665946in}}%
\pgfpathlineto{\pgfqpoint{3.304295in}{0.667578in}}%
\pgfpathlineto{\pgfqpoint{3.307104in}{0.670330in}}%
\pgfpathlineto{\pgfqpoint{3.309652in}{0.667107in}}%
\pgfpathlineto{\pgfqpoint{3.312480in}{0.666602in}}%
\pgfpathlineto{\pgfqpoint{3.315008in}{0.664409in}}%
\pgfpathlineto{\pgfqpoint{3.317688in}{0.666030in}}%
\pgfpathlineto{\pgfqpoint{3.320366in}{0.664718in}}%
\pgfpathlineto{\pgfqpoint{3.323049in}{0.665589in}}%
\pgfpathlineto{\pgfqpoint{3.325860in}{0.666660in}}%
\pgfpathlineto{\pgfqpoint{3.328401in}{0.662508in}}%
\pgfpathlineto{\pgfqpoint{3.331183in}{0.667850in}}%
\pgfpathlineto{\pgfqpoint{3.333758in}{0.664448in}}%
\pgfpathlineto{\pgfqpoint{3.336541in}{0.667871in}}%
\pgfpathlineto{\pgfqpoint{3.339101in}{0.666445in}}%
\pgfpathlineto{\pgfqpoint{3.341893in}{0.664938in}}%
\pgfpathlineto{\pgfqpoint{3.344468in}{0.670382in}}%
\pgfpathlineto{\pgfqpoint{3.347139in}{0.669600in}}%
\pgfpathlineto{\pgfqpoint{3.349828in}{0.668812in}}%
\pgfpathlineto{\pgfqpoint{3.352505in}{0.664381in}}%
\pgfpathlineto{\pgfqpoint{3.355177in}{0.663318in}}%
\pgfpathlineto{\pgfqpoint{3.357862in}{0.664318in}}%
\pgfpathlineto{\pgfqpoint{3.360620in}{0.664864in}}%
\pgfpathlineto{\pgfqpoint{3.363221in}{0.665790in}}%
\pgfpathlineto{\pgfqpoint{3.365996in}{0.665634in}}%
\pgfpathlineto{\pgfqpoint{3.368577in}{0.668781in}}%
\pgfpathlineto{\pgfqpoint{3.371357in}{0.666377in}}%
\pgfpathlineto{\pgfqpoint{3.373921in}{0.665004in}}%
\pgfpathlineto{\pgfqpoint{3.376735in}{0.668968in}}%
\pgfpathlineto{\pgfqpoint{3.379290in}{0.668261in}}%
\pgfpathlineto{\pgfqpoint{3.381959in}{0.671233in}}%
\pgfpathlineto{\pgfqpoint{3.384647in}{0.668149in}}%
\pgfpathlineto{\pgfqpoint{3.387309in}{0.670010in}}%
\pgfpathlineto{\pgfqpoint{3.390102in}{0.664827in}}%
\pgfpathlineto{\pgfqpoint{3.392681in}{0.670466in}}%
\pgfpathlineto{\pgfqpoint{3.395461in}{0.668376in}}%
\pgfpathlineto{\pgfqpoint{3.398037in}{0.668824in}}%
\pgfpathlineto{\pgfqpoint{3.400783in}{0.666677in}}%
\pgfpathlineto{\pgfqpoint{3.403394in}{0.670049in}}%
\pgfpathlineto{\pgfqpoint{3.406202in}{0.668593in}}%
\pgfpathlineto{\pgfqpoint{3.408752in}{0.666707in}}%
\pgfpathlineto{\pgfqpoint{3.411431in}{0.668432in}}%
\pgfpathlineto{\pgfqpoint{3.414109in}{0.666452in}}%
\pgfpathlineto{\pgfqpoint{3.416780in}{0.669070in}}%
\pgfpathlineto{\pgfqpoint{3.419455in}{0.671785in}}%
\pgfpathlineto{\pgfqpoint{3.422142in}{0.668497in}}%
\pgfpathlineto{\pgfqpoint{3.424887in}{0.665487in}}%
\pgfpathlineto{\pgfqpoint{3.427501in}{0.668094in}}%
\pgfpathlineto{\pgfqpoint{3.430313in}{0.668529in}}%
\pgfpathlineto{\pgfqpoint{3.432851in}{0.667750in}}%
\pgfpathlineto{\pgfqpoint{3.435635in}{0.666205in}}%
\pgfpathlineto{\pgfqpoint{3.438210in}{0.671411in}}%
\pgfpathlineto{\pgfqpoint{3.440996in}{0.670995in}}%
\pgfpathlineto{\pgfqpoint{3.443574in}{0.672128in}}%
\pgfpathlineto{\pgfqpoint{3.446257in}{0.671621in}}%
\pgfpathlineto{\pgfqpoint{3.448926in}{0.670348in}}%
\pgfpathlineto{\pgfqpoint{3.451597in}{0.671449in}}%
\pgfpathlineto{\pgfqpoint{3.454285in}{0.670185in}}%
\pgfpathlineto{\pgfqpoint{3.456960in}{0.669068in}}%
\pgfpathlineto{\pgfqpoint{3.459695in}{0.669915in}}%
\pgfpathlineto{\pgfqpoint{3.462321in}{0.670902in}}%
\pgfpathlineto{\pgfqpoint{3.465072in}{0.671404in}}%
\pgfpathlineto{\pgfqpoint{3.467678in}{0.668440in}}%
\pgfpathlineto{\pgfqpoint{3.470466in}{0.666898in}}%
\pgfpathlineto{\pgfqpoint{3.473021in}{0.666914in}}%
\pgfpathlineto{\pgfqpoint{3.475821in}{0.668532in}}%
\pgfpathlineto{\pgfqpoint{3.478378in}{0.669074in}}%
\pgfpathlineto{\pgfqpoint{3.481072in}{0.668488in}}%
\pgfpathlineto{\pgfqpoint{3.483744in}{0.675453in}}%
\pgfpathlineto{\pgfqpoint{3.486442in}{0.670763in}}%
\pgfpathlineto{\pgfqpoint{3.489223in}{0.668196in}}%
\pgfpathlineto{\pgfqpoint{3.491783in}{0.670663in}}%
\pgfpathlineto{\pgfqpoint{3.494581in}{0.670130in}}%
\pgfpathlineto{\pgfqpoint{3.497139in}{0.668466in}}%
\pgfpathlineto{\pgfqpoint{3.499909in}{0.672223in}}%
\pgfpathlineto{\pgfqpoint{3.502488in}{0.667788in}}%
\pgfpathlineto{\pgfqpoint{3.505262in}{0.668922in}}%
\pgfpathlineto{\pgfqpoint{3.507840in}{0.665929in}}%
\pgfpathlineto{\pgfqpoint{3.510533in}{0.668277in}}%
\pgfpathlineto{\pgfqpoint{3.513209in}{0.666200in}}%
\pgfpathlineto{\pgfqpoint{3.515884in}{0.667513in}}%
\pgfpathlineto{\pgfqpoint{3.518565in}{0.666349in}}%
\pgfpathlineto{\pgfqpoint{3.521244in}{0.665413in}}%
\pgfpathlineto{\pgfqpoint{3.524041in}{0.665258in}}%
\pgfpathlineto{\pgfqpoint{3.526601in}{0.666371in}}%
\pgfpathlineto{\pgfqpoint{3.529327in}{0.668271in}}%
\pgfpathlineto{\pgfqpoint{3.531955in}{0.665493in}}%
\pgfpathlineto{\pgfqpoint{3.534783in}{0.666262in}}%
\pgfpathlineto{\pgfqpoint{3.537309in}{0.663865in}}%
\pgfpathlineto{\pgfqpoint{3.540093in}{0.664074in}}%
\pgfpathlineto{\pgfqpoint{3.542656in}{0.665938in}}%
\pgfpathlineto{\pgfqpoint{3.545349in}{0.666054in}}%
\pgfpathlineto{\pgfqpoint{3.548029in}{0.669086in}}%
\pgfpathlineto{\pgfqpoint{3.550713in}{0.663030in}}%
\pgfpathlineto{\pgfqpoint{3.553498in}{0.667457in}}%
\pgfpathlineto{\pgfqpoint{3.556061in}{0.668762in}}%
\pgfpathlineto{\pgfqpoint{3.558853in}{0.669733in}}%
\pgfpathlineto{\pgfqpoint{3.561420in}{0.666800in}}%
\pgfpathlineto{\pgfqpoint{3.564188in}{0.668972in}}%
\pgfpathlineto{\pgfqpoint{3.566774in}{0.668265in}}%
\pgfpathlineto{\pgfqpoint{3.569584in}{0.664969in}}%
\pgfpathlineto{\pgfqpoint{3.572126in}{0.664150in}}%
\pgfpathlineto{\pgfqpoint{3.574814in}{0.666757in}}%
\pgfpathlineto{\pgfqpoint{3.577487in}{0.665501in}}%
\pgfpathlineto{\pgfqpoint{3.580191in}{0.664261in}}%
\pgfpathlineto{\pgfqpoint{3.582851in}{0.663394in}}%
\pgfpathlineto{\pgfqpoint{3.585532in}{0.664838in}}%
\pgfpathlineto{\pgfqpoint{3.588258in}{0.664986in}}%
\pgfpathlineto{\pgfqpoint{3.590883in}{0.668806in}}%
\pgfpathlineto{\pgfqpoint{3.593620in}{0.668878in}}%
\pgfpathlineto{\pgfqpoint{3.596240in}{0.668841in}}%
\pgfpathlineto{\pgfqpoint{3.598998in}{0.667941in}}%
\pgfpathlineto{\pgfqpoint{3.601590in}{0.667342in}}%
\pgfpathlineto{\pgfqpoint{3.604387in}{0.666004in}}%
\pgfpathlineto{\pgfqpoint{3.606951in}{0.665836in}}%
\pgfpathlineto{\pgfqpoint{3.609632in}{0.662426in}}%
\pgfpathlineto{\pgfqpoint{3.612311in}{0.662930in}}%
\pgfpathlineto{\pgfqpoint{3.614982in}{0.665033in}}%
\pgfpathlineto{\pgfqpoint{3.617667in}{0.664970in}}%
\pgfpathlineto{\pgfqpoint{3.620345in}{0.664302in}}%
\pgfpathlineto{\pgfqpoint{3.623165in}{0.662562in}}%
\pgfpathlineto{\pgfqpoint{3.625689in}{0.666243in}}%
\pgfpathlineto{\pgfqpoint{3.628460in}{0.665180in}}%
\pgfpathlineto{\pgfqpoint{3.631058in}{0.664539in}}%
\pgfpathlineto{\pgfqpoint{3.633858in}{0.666131in}}%
\pgfpathlineto{\pgfqpoint{3.636413in}{0.666039in}}%
\pgfpathlineto{\pgfqpoint{3.639207in}{0.667348in}}%
\pgfpathlineto{\pgfqpoint{3.641773in}{0.664137in}}%
\pgfpathlineto{\pgfqpoint{3.644452in}{0.661068in}}%
\pgfpathlineto{\pgfqpoint{3.647130in}{0.659894in}}%
\pgfpathlineto{\pgfqpoint{3.649837in}{0.663220in}}%
\pgfpathlineto{\pgfqpoint{3.652628in}{0.664613in}}%
\pgfpathlineto{\pgfqpoint{3.655165in}{0.667207in}}%
\pgfpathlineto{\pgfqpoint{3.657917in}{0.667436in}}%
\pgfpathlineto{\pgfqpoint{3.660515in}{0.665577in}}%
\pgfpathlineto{\pgfqpoint{3.663276in}{0.672685in}}%
\pgfpathlineto{\pgfqpoint{3.665864in}{0.673322in}}%
\pgfpathlineto{\pgfqpoint{3.668665in}{0.669664in}}%
\pgfpathlineto{\pgfqpoint{3.671232in}{0.669692in}}%
\pgfpathlineto{\pgfqpoint{3.673911in}{0.667824in}}%
\pgfpathlineto{\pgfqpoint{3.676591in}{0.668925in}}%
\pgfpathlineto{\pgfqpoint{3.679273in}{0.669482in}}%
\pgfpathlineto{\pgfqpoint{3.681948in}{0.663621in}}%
\pgfpathlineto{\pgfqpoint{3.684620in}{0.662774in}}%
\pgfpathlineto{\pgfqpoint{3.687442in}{0.659976in}}%
\pgfpathlineto{\pgfqpoint{3.689983in}{0.659988in}}%
\pgfpathlineto{\pgfqpoint{3.692765in}{0.665450in}}%
\pgfpathlineto{\pgfqpoint{3.695331in}{0.663631in}}%
\pgfpathlineto{\pgfqpoint{3.698125in}{0.667370in}}%
\pgfpathlineto{\pgfqpoint{3.700684in}{0.663469in}}%
\pgfpathlineto{\pgfqpoint{3.703460in}{0.664889in}}%
\pgfpathlineto{\pgfqpoint{3.706053in}{0.661919in}}%
\pgfpathlineto{\pgfqpoint{3.708729in}{0.662116in}}%
\pgfpathlineto{\pgfqpoint{3.711410in}{0.664417in}}%
\pgfpathlineto{\pgfqpoint{3.714086in}{0.665622in}}%
\pgfpathlineto{\pgfqpoint{3.716875in}{0.668209in}}%
\pgfpathlineto{\pgfqpoint{3.719446in}{0.663361in}}%
\pgfpathlineto{\pgfqpoint{3.722228in}{0.659894in}}%
\pgfpathlineto{\pgfqpoint{3.724804in}{0.661516in}}%
\pgfpathlineto{\pgfqpoint{3.727581in}{0.662823in}}%
\pgfpathlineto{\pgfqpoint{3.730158in}{0.665266in}}%
\pgfpathlineto{\pgfqpoint{3.732950in}{0.663722in}}%
\pgfpathlineto{\pgfqpoint{3.735509in}{0.664020in}}%
\pgfpathlineto{\pgfqpoint{3.738194in}{0.666297in}}%
\pgfpathlineto{\pgfqpoint{3.740874in}{0.663743in}}%
\pgfpathlineto{\pgfqpoint{3.743548in}{0.667208in}}%
\pgfpathlineto{\pgfqpoint{3.746229in}{0.665445in}}%
\pgfpathlineto{\pgfqpoint{3.748903in}{0.665110in}}%
\pgfpathlineto{\pgfqpoint{3.751728in}{0.665809in}}%
\pgfpathlineto{\pgfqpoint{3.754265in}{0.667849in}}%
\pgfpathlineto{\pgfqpoint{3.757065in}{0.668638in}}%
\pgfpathlineto{\pgfqpoint{3.759622in}{0.664373in}}%
\pgfpathlineto{\pgfqpoint{3.762389in}{0.666240in}}%
\pgfpathlineto{\pgfqpoint{3.764966in}{0.671751in}}%
\pgfpathlineto{\pgfqpoint{3.767782in}{0.668008in}}%
\pgfpathlineto{\pgfqpoint{3.770323in}{0.668822in}}%
\pgfpathlineto{\pgfqpoint{3.773014in}{0.670745in}}%
\pgfpathlineto{\pgfqpoint{3.775691in}{0.667801in}}%
\pgfpathlineto{\pgfqpoint{3.778370in}{0.659951in}}%
\pgfpathlineto{\pgfqpoint{3.781046in}{0.659894in}}%
\pgfpathlineto{\pgfqpoint{3.783725in}{0.659894in}}%
\pgfpathlineto{\pgfqpoint{3.786504in}{0.661519in}}%
\pgfpathlineto{\pgfqpoint{3.789084in}{0.660545in}}%
\pgfpathlineto{\pgfqpoint{3.791897in}{0.665850in}}%
\pgfpathlineto{\pgfqpoint{3.794435in}{0.666058in}}%
\pgfpathlineto{\pgfqpoint{3.797265in}{0.661600in}}%
\pgfpathlineto{\pgfqpoint{3.799797in}{0.661618in}}%
\pgfpathlineto{\pgfqpoint{3.802569in}{0.659894in}}%
\pgfpathlineto{\pgfqpoint{3.805145in}{0.662731in}}%
\pgfpathlineto{\pgfqpoint{3.807832in}{0.664560in}}%
\pgfpathlineto{\pgfqpoint{3.810510in}{0.665017in}}%
\pgfpathlineto{\pgfqpoint{3.813172in}{0.663958in}}%
\pgfpathlineto{\pgfqpoint{3.815983in}{0.666017in}}%
\pgfpathlineto{\pgfqpoint{3.818546in}{0.662474in}}%
\pgfpathlineto{\pgfqpoint{3.821315in}{0.664307in}}%
\pgfpathlineto{\pgfqpoint{3.823903in}{0.662707in}}%
\pgfpathlineto{\pgfqpoint{3.826679in}{0.664013in}}%
\pgfpathlineto{\pgfqpoint{3.829252in}{0.666870in}}%
\pgfpathlineto{\pgfqpoint{3.832053in}{0.665747in}}%
\pgfpathlineto{\pgfqpoint{3.834616in}{0.666718in}}%
\pgfpathlineto{\pgfqpoint{3.837286in}{0.667121in}}%
\pgfpathlineto{\pgfqpoint{3.839960in}{0.669630in}}%
\pgfpathlineto{\pgfqpoint{3.842641in}{0.669424in}}%
\pgfpathlineto{\pgfqpoint{3.845329in}{0.666665in}}%
\pgfpathlineto{\pgfqpoint{3.848005in}{0.669500in}}%
\pgfpathlineto{\pgfqpoint{3.850814in}{0.666405in}}%
\pgfpathlineto{\pgfqpoint{3.853358in}{0.662460in}}%
\pgfpathlineto{\pgfqpoint{3.856100in}{0.664756in}}%
\pgfpathlineto{\pgfqpoint{3.858720in}{0.662876in}}%
\pgfpathlineto{\pgfqpoint{3.861561in}{0.667752in}}%
\pgfpathlineto{\pgfqpoint{3.864073in}{0.666718in}}%
\pgfpathlineto{\pgfqpoint{3.866815in}{0.668022in}}%
\pgfpathlineto{\pgfqpoint{3.869435in}{0.667964in}}%
\pgfpathlineto{\pgfqpoint{3.872114in}{0.666166in}}%
\pgfpathlineto{\pgfqpoint{3.874790in}{0.668473in}}%
\pgfpathlineto{\pgfqpoint{3.877466in}{0.670743in}}%
\pgfpathlineto{\pgfqpoint{3.880237in}{0.666759in}}%
\pgfpathlineto{\pgfqpoint{3.882850in}{0.670140in}}%
\pgfpathlineto{\pgfqpoint{3.885621in}{0.667908in}}%
\pgfpathlineto{\pgfqpoint{3.888188in}{0.667201in}}%
\pgfpathlineto{\pgfqpoint{3.890926in}{0.668379in}}%
\pgfpathlineto{\pgfqpoint{3.893541in}{0.670318in}}%
\pgfpathlineto{\pgfqpoint{3.896345in}{0.675365in}}%
\pgfpathlineto{\pgfqpoint{3.898891in}{0.672469in}}%
\pgfpathlineto{\pgfqpoint{3.901573in}{0.671093in}}%
\pgfpathlineto{\pgfqpoint{3.904252in}{0.673384in}}%
\pgfpathlineto{\pgfqpoint{3.906918in}{0.669444in}}%
\pgfpathlineto{\pgfqpoint{3.909602in}{0.668509in}}%
\pgfpathlineto{\pgfqpoint{3.912296in}{0.670492in}}%
\pgfpathlineto{\pgfqpoint{3.915107in}{0.670733in}}%
\pgfpathlineto{\pgfqpoint{3.917646in}{0.671833in}}%
\pgfpathlineto{\pgfqpoint{3.920412in}{0.669638in}}%
\pgfpathlineto{\pgfqpoint{3.923005in}{0.670188in}}%
\pgfpathlineto{\pgfqpoint{3.925778in}{0.671325in}}%
\pgfpathlineto{\pgfqpoint{3.928347in}{0.670720in}}%
\pgfpathlineto{\pgfqpoint{3.931202in}{0.668594in}}%
\pgfpathlineto{\pgfqpoint{3.933714in}{0.674241in}}%
\pgfpathlineto{\pgfqpoint{3.936395in}{0.674488in}}%
\pgfpathlineto{\pgfqpoint{3.939075in}{0.672279in}}%
\pgfpathlineto{\pgfqpoint{3.941778in}{0.678145in}}%
\pgfpathlineto{\pgfqpoint{3.944431in}{0.674144in}}%
\pgfpathlineto{\pgfqpoint{3.947101in}{0.669271in}}%
\pgfpathlineto{\pgfqpoint{3.949894in}{0.670375in}}%
\pgfpathlineto{\pgfqpoint{3.952464in}{0.662888in}}%
\pgfpathlineto{\pgfqpoint{3.955211in}{0.667453in}}%
\pgfpathlineto{\pgfqpoint{3.957823in}{0.669386in}}%
\pgfpathlineto{\pgfqpoint{3.960635in}{0.669493in}}%
\pgfpathlineto{\pgfqpoint{3.963176in}{0.670824in}}%
\pgfpathlineto{\pgfqpoint{3.966013in}{0.668801in}}%
\pgfpathlineto{\pgfqpoint{3.968523in}{0.668674in}}%
\pgfpathlineto{\pgfqpoint{3.971250in}{0.666619in}}%
\pgfpathlineto{\pgfqpoint{3.973885in}{0.667824in}}%
\pgfpathlineto{\pgfqpoint{3.976563in}{0.667268in}}%
\pgfpathlineto{\pgfqpoint{3.979389in}{0.666403in}}%
\pgfpathlineto{\pgfqpoint{3.981929in}{0.661003in}}%
\pgfpathlineto{\pgfqpoint{3.984714in}{0.667889in}}%
\pgfpathlineto{\pgfqpoint{3.987270in}{0.662691in}}%
\pgfpathlineto{\pgfqpoint{3.990055in}{0.669221in}}%
\pgfpathlineto{\pgfqpoint{3.992642in}{0.669263in}}%
\pgfpathlineto{\pgfqpoint{3.995417in}{0.668266in}}%
\pgfpathlineto{\pgfqpoint{3.997990in}{0.671049in}}%
\pgfpathlineto{\pgfqpoint{4.000674in}{0.666926in}}%
\pgfpathlineto{\pgfqpoint{4.003348in}{0.670412in}}%
\pgfpathlineto{\pgfqpoint{4.006034in}{0.670450in}}%
\pgfpathlineto{\pgfqpoint{4.008699in}{0.666344in}}%
\pgfpathlineto{\pgfqpoint{4.011394in}{0.666248in}}%
\pgfpathlineto{\pgfqpoint{4.014186in}{0.664639in}}%
\pgfpathlineto{\pgfqpoint{4.016744in}{0.669273in}}%
\pgfpathlineto{\pgfqpoint{4.019518in}{0.667180in}}%
\pgfpathlineto{\pgfqpoint{4.022097in}{0.668393in}}%
\pgfpathlineto{\pgfqpoint{4.024868in}{0.666597in}}%
\pgfpathlineto{\pgfqpoint{4.027447in}{0.666710in}}%
\pgfpathlineto{\pgfqpoint{4.030229in}{0.667791in}}%
\pgfpathlineto{\pgfqpoint{4.032817in}{0.665830in}}%
\pgfpathlineto{\pgfqpoint{4.035492in}{0.671613in}}%
\pgfpathlineto{\pgfqpoint{4.038174in}{0.669397in}}%
\pgfpathlineto{\pgfqpoint{4.040852in}{0.668803in}}%
\pgfpathlineto{\pgfqpoint{4.043667in}{0.671574in}}%
\pgfpathlineto{\pgfqpoint{4.046210in}{0.680558in}}%
\pgfpathlineto{\pgfqpoint{4.049006in}{0.675550in}}%
\pgfpathlineto{\pgfqpoint{4.051557in}{0.671573in}}%
\pgfpathlineto{\pgfqpoint{4.054326in}{0.667891in}}%
\pgfpathlineto{\pgfqpoint{4.056911in}{0.670318in}}%
\pgfpathlineto{\pgfqpoint{4.059702in}{0.670046in}}%
\pgfpathlineto{\pgfqpoint{4.062266in}{0.667722in}}%
\pgfpathlineto{\pgfqpoint{4.064957in}{0.675376in}}%
\pgfpathlineto{\pgfqpoint{4.067636in}{0.669464in}}%
\pgfpathlineto{\pgfqpoint{4.070313in}{0.673677in}}%
\pgfpathlineto{\pgfqpoint{4.072985in}{0.673256in}}%
\pgfpathlineto{\pgfqpoint{4.075705in}{0.673601in}}%
\pgfpathlineto{\pgfqpoint{4.078471in}{0.677072in}}%
\pgfpathlineto{\pgfqpoint{4.081018in}{0.675822in}}%
\pgfpathlineto{\pgfqpoint{4.083870in}{0.670555in}}%
\pgfpathlineto{\pgfqpoint{4.086385in}{0.672962in}}%
\pgfpathlineto{\pgfqpoint{4.089159in}{0.676303in}}%
\pgfpathlineto{\pgfqpoint{4.091729in}{0.672410in}}%
\pgfpathlineto{\pgfqpoint{4.094527in}{0.672190in}}%
\pgfpathlineto{\pgfqpoint{4.097092in}{0.673655in}}%
\pgfpathlineto{\pgfqpoint{4.099777in}{0.675687in}}%
\pgfpathlineto{\pgfqpoint{4.102456in}{0.670092in}}%
\pgfpathlineto{\pgfqpoint{4.105185in}{0.675380in}}%
\pgfpathlineto{\pgfqpoint{4.107814in}{0.675676in}}%
\pgfpathlineto{\pgfqpoint{4.110488in}{0.674547in}}%
\pgfpathlineto{\pgfqpoint{4.113252in}{0.670923in}}%
\pgfpathlineto{\pgfqpoint{4.115844in}{0.670798in}}%
\pgfpathlineto{\pgfqpoint{4.118554in}{0.669953in}}%
\pgfpathlineto{\pgfqpoint{4.121205in}{0.674170in}}%
\pgfpathlineto{\pgfqpoint{4.124019in}{0.674435in}}%
\pgfpathlineto{\pgfqpoint{4.126553in}{0.671559in}}%
\pgfpathlineto{\pgfqpoint{4.129349in}{0.676920in}}%
\pgfpathlineto{\pgfqpoint{4.131920in}{0.676135in}}%
\pgfpathlineto{\pgfqpoint{4.134615in}{0.671224in}}%
\pgfpathlineto{\pgfqpoint{4.137272in}{0.670181in}}%
\pgfpathlineto{\pgfqpoint{4.139963in}{0.670782in}}%
\pgfpathlineto{\pgfqpoint{4.142713in}{0.671937in}}%
\pgfpathlineto{\pgfqpoint{4.145310in}{0.668449in}}%
\pgfpathlineto{\pgfqpoint{4.148082in}{0.667905in}}%
\pgfpathlineto{\pgfqpoint{4.150665in}{0.667072in}}%
\pgfpathlineto{\pgfqpoint{4.153423in}{0.666260in}}%
\pgfpathlineto{\pgfqpoint{4.156016in}{0.664151in}}%
\pgfpathlineto{\pgfqpoint{4.158806in}{0.665590in}}%
\pgfpathlineto{\pgfqpoint{4.161380in}{0.670419in}}%
\pgfpathlineto{\pgfqpoint{4.164059in}{0.668052in}}%
\pgfpathlineto{\pgfqpoint{4.166737in}{0.666388in}}%
\pgfpathlineto{\pgfqpoint{4.169415in}{0.663928in}}%
\pgfpathlineto{\pgfqpoint{4.172093in}{0.665855in}}%
\pgfpathlineto{\pgfqpoint{4.174770in}{0.664942in}}%
\pgfpathlineto{\pgfqpoint{4.177593in}{0.667201in}}%
\pgfpathlineto{\pgfqpoint{4.180129in}{0.672029in}}%
\pgfpathlineto{\pgfqpoint{4.182899in}{0.670232in}}%
\pgfpathlineto{\pgfqpoint{4.185481in}{0.672262in}}%
\pgfpathlineto{\pgfqpoint{4.188318in}{0.674364in}}%
\pgfpathlineto{\pgfqpoint{4.190842in}{0.666941in}}%
\pgfpathlineto{\pgfqpoint{4.193638in}{0.669926in}}%
\pgfpathlineto{\pgfqpoint{4.196186in}{0.672470in}}%
\pgfpathlineto{\pgfqpoint{4.198878in}{0.673185in}}%
\pgfpathlineto{\pgfqpoint{4.201542in}{0.673965in}}%
\pgfpathlineto{\pgfqpoint{4.204240in}{0.668769in}}%
\pgfpathlineto{\pgfqpoint{4.207076in}{0.662306in}}%
\pgfpathlineto{\pgfqpoint{4.209597in}{0.661212in}}%
\pgfpathlineto{\pgfqpoint{4.212383in}{0.663631in}}%
\pgfpathlineto{\pgfqpoint{4.214948in}{0.666029in}}%
\pgfpathlineto{\pgfqpoint{4.217694in}{0.666709in}}%
\pgfpathlineto{\pgfqpoint{4.220304in}{0.670838in}}%
\pgfpathlineto{\pgfqpoint{4.223082in}{0.680671in}}%
\pgfpathlineto{\pgfqpoint{4.225654in}{0.672422in}}%
\pgfpathlineto{\pgfqpoint{4.228331in}{0.676805in}}%
\pgfpathlineto{\pgfqpoint{4.231013in}{0.669091in}}%
\pgfpathlineto{\pgfqpoint{4.233691in}{0.669636in}}%
\pgfpathlineto{\pgfqpoint{4.236375in}{0.672525in}}%
\pgfpathlineto{\pgfqpoint{4.239084in}{0.670834in}}%
\pgfpathlineto{\pgfqpoint{4.241900in}{0.664090in}}%
\pgfpathlineto{\pgfqpoint{4.244394in}{0.667169in}}%
\pgfpathlineto{\pgfqpoint{4.247225in}{0.669202in}}%
\pgfpathlineto{\pgfqpoint{4.249767in}{0.667624in}}%
\pgfpathlineto{\pgfqpoint{4.252581in}{0.668132in}}%
\pgfpathlineto{\pgfqpoint{4.255120in}{0.666569in}}%
\pgfpathlineto{\pgfqpoint{4.257958in}{0.670432in}}%
\pgfpathlineto{\pgfqpoint{4.260477in}{0.668925in}}%
\pgfpathlineto{\pgfqpoint{4.263157in}{0.670160in}}%
\pgfpathlineto{\pgfqpoint{4.265824in}{0.672364in}}%
\pgfpathlineto{\pgfqpoint{4.268590in}{0.673460in}}%
\pgfpathlineto{\pgfqpoint{4.271187in}{0.674397in}}%
\pgfpathlineto{\pgfqpoint{4.273874in}{0.673184in}}%
\pgfpathlineto{\pgfqpoint{4.276635in}{0.671706in}}%
\pgfpathlineto{\pgfqpoint{4.279212in}{0.668327in}}%
\pgfpathlineto{\pgfqpoint{4.282000in}{0.666640in}}%
\pgfpathlineto{\pgfqpoint{4.284586in}{0.662509in}}%
\pgfpathlineto{\pgfqpoint{4.287399in}{0.660490in}}%
\pgfpathlineto{\pgfqpoint{4.289936in}{0.663572in}}%
\pgfpathlineto{\pgfqpoint{4.292786in}{0.668142in}}%
\pgfpathlineto{\pgfqpoint{4.295299in}{0.666252in}}%
\pgfpathlineto{\pgfqpoint{4.297977in}{0.665742in}}%
\pgfpathlineto{\pgfqpoint{4.300656in}{0.664171in}}%
\pgfpathlineto{\pgfqpoint{4.303357in}{0.667618in}}%
\pgfpathlineto{\pgfqpoint{4.306118in}{0.669449in}}%
\pgfpathlineto{\pgfqpoint{4.308691in}{0.668357in}}%
\pgfpathlineto{\pgfqpoint{4.311494in}{0.666090in}}%
\pgfpathlineto{\pgfqpoint{4.314032in}{0.665864in}}%
\pgfpathlineto{\pgfqpoint{4.316856in}{0.668802in}}%
\pgfpathlineto{\pgfqpoint{4.319405in}{0.663556in}}%
\pgfpathlineto{\pgfqpoint{4.322181in}{0.665044in}}%
\pgfpathlineto{\pgfqpoint{4.324760in}{0.664767in}}%
\pgfpathlineto{\pgfqpoint{4.327440in}{0.667125in}}%
\pgfpathlineto{\pgfqpoint{4.330118in}{0.663917in}}%
\pgfpathlineto{\pgfqpoint{4.332796in}{0.667950in}}%
\pgfpathlineto{\pgfqpoint{4.335463in}{0.668885in}}%
\pgfpathlineto{\pgfqpoint{4.338154in}{0.669626in}}%
\pgfpathlineto{\pgfqpoint{4.340976in}{0.669148in}}%
\pgfpathlineto{\pgfqpoint{4.343510in}{0.663805in}}%
\pgfpathlineto{\pgfqpoint{4.346263in}{0.660377in}}%
\pgfpathlineto{\pgfqpoint{4.348868in}{0.666564in}}%
\pgfpathlineto{\pgfqpoint{4.351645in}{0.661839in}}%
\pgfpathlineto{\pgfqpoint{4.354224in}{0.666706in}}%
\pgfpathlineto{\pgfqpoint{4.357014in}{0.662326in}}%
\pgfpathlineto{\pgfqpoint{4.359582in}{0.667143in}}%
\pgfpathlineto{\pgfqpoint{4.362270in}{0.666742in}}%
\pgfpathlineto{\pgfqpoint{4.364936in}{0.667297in}}%
\pgfpathlineto{\pgfqpoint{4.367646in}{0.666490in}}%
\pgfpathlineto{\pgfqpoint{4.370437in}{0.666806in}}%
\pgfpathlineto{\pgfqpoint{4.372976in}{0.669335in}}%
\pgfpathlineto{\pgfqpoint{4.375761in}{0.670242in}}%
\pgfpathlineto{\pgfqpoint{4.378329in}{0.672476in}}%
\pgfpathlineto{\pgfqpoint{4.381097in}{0.673625in}}%
\pgfpathlineto{\pgfqpoint{4.383674in}{0.665324in}}%
\pgfpathlineto{\pgfqpoint{4.386431in}{0.666269in}}%
\pgfpathlineto{\pgfqpoint{4.389044in}{0.666426in}}%
\pgfpathlineto{\pgfqpoint{4.391721in}{0.665359in}}%
\pgfpathlineto{\pgfqpoint{4.394400in}{0.670139in}}%
\pgfpathlineto{\pgfqpoint{4.397076in}{0.668047in}}%
\pgfpathlineto{\pgfqpoint{4.399745in}{0.668126in}}%
\pgfpathlineto{\pgfqpoint{4.402468in}{0.669441in}}%
\pgfpathlineto{\pgfqpoint{4.405234in}{0.663279in}}%
\pgfpathlineto{\pgfqpoint{4.407788in}{0.663626in}}%
\pgfpathlineto{\pgfqpoint{4.410587in}{0.665532in}}%
\pgfpathlineto{\pgfqpoint{4.413149in}{0.666614in}}%
\pgfpathlineto{\pgfqpoint{4.415932in}{0.664896in}}%
\pgfpathlineto{\pgfqpoint{4.418506in}{0.667677in}}%
\pgfpathlineto{\pgfqpoint{4.421292in}{0.666636in}}%
\pgfpathlineto{\pgfqpoint{4.423863in}{0.667834in}}%
\pgfpathlineto{\pgfqpoint{4.426534in}{0.666704in}}%
\pgfpathlineto{\pgfqpoint{4.429220in}{0.670691in}}%
\pgfpathlineto{\pgfqpoint{4.431901in}{0.665866in}}%
\pgfpathlineto{\pgfqpoint{4.434569in}{0.668760in}}%
\pgfpathlineto{\pgfqpoint{4.437253in}{0.673776in}}%
\pgfpathlineto{\pgfqpoint{4.440041in}{0.683814in}}%
\pgfpathlineto{\pgfqpoint{4.442611in}{0.683303in}}%
\pgfpathlineto{\pgfqpoint{4.445423in}{0.682526in}}%
\pgfpathlineto{\pgfqpoint{4.447965in}{0.679574in}}%
\pgfpathlineto{\pgfqpoint{4.450767in}{0.672345in}}%
\pgfpathlineto{\pgfqpoint{4.453312in}{0.670071in}}%
\pgfpathlineto{\pgfqpoint{4.456138in}{0.669617in}}%
\pgfpathlineto{\pgfqpoint{4.458681in}{0.667085in}}%
\pgfpathlineto{\pgfqpoint{4.461367in}{0.669148in}}%
\pgfpathlineto{\pgfqpoint{4.464029in}{0.672009in}}%
\pgfpathlineto{\pgfqpoint{4.466717in}{0.671538in}}%
\pgfpathlineto{\pgfqpoint{4.469492in}{0.673208in}}%
\pgfpathlineto{\pgfqpoint{4.472059in}{0.671458in}}%
\pgfpathlineto{\pgfqpoint{4.474861in}{0.669439in}}%
\pgfpathlineto{\pgfqpoint{4.477430in}{0.672006in}}%
\pgfpathlineto{\pgfqpoint{4.480201in}{0.673074in}}%
\pgfpathlineto{\pgfqpoint{4.482778in}{0.669334in}}%
\pgfpathlineto{\pgfqpoint{4.485581in}{0.668632in}}%
\pgfpathlineto{\pgfqpoint{4.488130in}{0.670069in}}%
\pgfpathlineto{\pgfqpoint{4.490822in}{0.670767in}}%
\pgfpathlineto{\pgfqpoint{4.493492in}{0.671945in}}%
\pgfpathlineto{\pgfqpoint{4.496167in}{0.669335in}}%
\pgfpathlineto{\pgfqpoint{4.498850in}{0.667394in}}%
\pgfpathlineto{\pgfqpoint{4.501529in}{0.673166in}}%
\pgfpathlineto{\pgfqpoint{4.504305in}{0.672055in}}%
\pgfpathlineto{\pgfqpoint{4.506893in}{0.671558in}}%
\pgfpathlineto{\pgfqpoint{4.509643in}{0.670778in}}%
\pgfpathlineto{\pgfqpoint{4.512246in}{0.670080in}}%
\pgfpathlineto{\pgfqpoint{4.515080in}{0.669465in}}%
\pgfpathlineto{\pgfqpoint{4.517598in}{0.670281in}}%
\pgfpathlineto{\pgfqpoint{4.520345in}{0.671411in}}%
\pgfpathlineto{\pgfqpoint{4.522962in}{0.676924in}}%
\pgfpathlineto{\pgfqpoint{4.525640in}{0.672235in}}%
\pgfpathlineto{\pgfqpoint{4.528307in}{0.669701in}}%
\pgfpathlineto{\pgfqpoint{4.530990in}{0.670983in}}%
\pgfpathlineto{\pgfqpoint{4.533764in}{0.675550in}}%
\pgfpathlineto{\pgfqpoint{4.536400in}{0.680318in}}%
\pgfpathlineto{\pgfqpoint{4.539144in}{0.671987in}}%
\pgfpathlineto{\pgfqpoint{4.541711in}{0.664052in}}%
\pgfpathlineto{\pgfqpoint{4.544464in}{0.667120in}}%
\pgfpathlineto{\pgfqpoint{4.547064in}{0.664502in}}%
\pgfpathlineto{\pgfqpoint{4.549822in}{0.666108in}}%
\pgfpathlineto{\pgfqpoint{4.552425in}{0.665375in}}%
\pgfpathlineto{\pgfqpoint{4.555106in}{0.664300in}}%
\pgfpathlineto{\pgfqpoint{4.557777in}{0.669073in}}%
\pgfpathlineto{\pgfqpoint{4.560448in}{0.664316in}}%
\pgfpathlineto{\pgfqpoint{4.563125in}{0.663418in}}%
\pgfpathlineto{\pgfqpoint{4.565820in}{0.667033in}}%
\pgfpathlineto{\pgfqpoint{4.568612in}{0.666432in}}%
\pgfpathlineto{\pgfqpoint{4.571171in}{0.667648in}}%
\pgfpathlineto{\pgfqpoint{4.573947in}{0.664650in}}%
\pgfpathlineto{\pgfqpoint{4.576531in}{0.664837in}}%
\pgfpathlineto{\pgfqpoint{4.579305in}{0.664030in}}%
\pgfpathlineto{\pgfqpoint{4.581888in}{0.663675in}}%
\pgfpathlineto{\pgfqpoint{4.584672in}{0.666948in}}%
\pgfpathlineto{\pgfqpoint{4.587244in}{0.665240in}}%
\pgfpathlineto{\pgfqpoint{4.589920in}{0.665964in}}%
\pgfpathlineto{\pgfqpoint{4.592589in}{0.667690in}}%
\pgfpathlineto{\pgfqpoint{4.595281in}{0.666967in}}%
\pgfpathlineto{\pgfqpoint{4.597951in}{0.665666in}}%
\pgfpathlineto{\pgfqpoint{4.600633in}{0.663577in}}%
\pgfpathlineto{\pgfqpoint{4.603430in}{0.663407in}}%
\pgfpathlineto{\pgfqpoint{4.605990in}{0.667283in}}%
\pgfpathlineto{\pgfqpoint{4.608808in}{0.666949in}}%
\pgfpathlineto{\pgfqpoint{4.611350in}{0.666155in}}%
\pgfpathlineto{\pgfqpoint{4.614134in}{0.666075in}}%
\pgfpathlineto{\pgfqpoint{4.616702in}{0.670419in}}%
\pgfpathlineto{\pgfqpoint{4.619529in}{0.669014in}}%
\pgfpathlineto{\pgfqpoint{4.622056in}{0.667651in}}%
\pgfpathlineto{\pgfqpoint{4.624741in}{0.669431in}}%
\pgfpathlineto{\pgfqpoint{4.627411in}{0.668307in}}%
\pgfpathlineto{\pgfqpoint{4.630096in}{0.670160in}}%
\pgfpathlineto{\pgfqpoint{4.632902in}{0.670645in}}%
\pgfpathlineto{\pgfqpoint{4.635445in}{0.668420in}}%
\pgfpathlineto{\pgfqpoint{4.638204in}{0.667136in}}%
\pgfpathlineto{\pgfqpoint{4.640809in}{0.667864in}}%
\pgfpathlineto{\pgfqpoint{4.643628in}{0.668319in}}%
\pgfpathlineto{\pgfqpoint{4.646169in}{0.668594in}}%
\pgfpathlineto{\pgfqpoint{4.648922in}{0.669451in}}%
\pgfpathlineto{\pgfqpoint{4.651524in}{0.669540in}}%
\pgfpathlineto{\pgfqpoint{4.654203in}{0.674263in}}%
\pgfpathlineto{\pgfqpoint{4.656873in}{0.667635in}}%
\pgfpathlineto{\pgfqpoint{4.659590in}{0.668119in}}%
\pgfpathlineto{\pgfqpoint{4.662237in}{0.666525in}}%
\pgfpathlineto{\pgfqpoint{4.664923in}{0.669024in}}%
\pgfpathlineto{\pgfqpoint{4.667764in}{0.669485in}}%
\pgfpathlineto{\pgfqpoint{4.670261in}{0.663964in}}%
\pgfpathlineto{\pgfqpoint{4.673068in}{0.664925in}}%
\pgfpathlineto{\pgfqpoint{4.675619in}{0.668164in}}%
\pgfpathlineto{\pgfqpoint{4.678448in}{0.667117in}}%
\pgfpathlineto{\pgfqpoint{4.680988in}{0.665651in}}%
\pgfpathlineto{\pgfqpoint{4.683799in}{0.668103in}}%
\pgfpathlineto{\pgfqpoint{4.686337in}{0.668053in}}%
\pgfpathlineto{\pgfqpoint{4.689051in}{0.663974in}}%
\pgfpathlineto{\pgfqpoint{4.691694in}{0.662930in}}%
\pgfpathlineto{\pgfqpoint{4.694381in}{0.662728in}}%
\pgfpathlineto{\pgfqpoint{4.697170in}{0.661333in}}%
\pgfpathlineto{\pgfqpoint{4.699734in}{0.662701in}}%
\pgfpathlineto{\pgfqpoint{4.702517in}{0.665150in}}%
\pgfpathlineto{\pgfqpoint{4.705094in}{0.667931in}}%
\pgfpathlineto{\pgfqpoint{4.707824in}{0.665176in}}%
\pgfpathlineto{\pgfqpoint{4.710437in}{0.669162in}}%
\pgfpathlineto{\pgfqpoint{4.713275in}{0.670549in}}%
\pgfpathlineto{\pgfqpoint{4.715806in}{0.666236in}}%
\pgfpathlineto{\pgfqpoint{4.718486in}{0.664073in}}%
\pgfpathlineto{\pgfqpoint{4.721160in}{0.665793in}}%
\pgfpathlineto{\pgfqpoint{4.723873in}{0.667016in}}%
\pgfpathlineto{\pgfqpoint{4.726508in}{0.666182in}}%
\pgfpathlineto{\pgfqpoint{4.729233in}{0.662501in}}%
\pgfpathlineto{\pgfqpoint{4.731901in}{0.664963in}}%
\pgfpathlineto{\pgfqpoint{4.734552in}{0.668941in}}%
\pgfpathlineto{\pgfqpoint{4.737348in}{0.672112in}}%
\pgfpathlineto{\pgfqpoint{4.739912in}{0.667583in}}%
\pgfpathlineto{\pgfqpoint{4.742696in}{0.671139in}}%
\pgfpathlineto{\pgfqpoint{4.745256in}{0.675849in}}%
\pgfpathlineto{\pgfqpoint{4.748081in}{0.675827in}}%
\pgfpathlineto{\pgfqpoint{4.750627in}{0.675935in}}%
\pgfpathlineto{\pgfqpoint{4.753298in}{0.676078in}}%
\pgfpathlineto{\pgfqpoint{4.755983in}{0.678151in}}%
\pgfpathlineto{\pgfqpoint{4.758653in}{0.670348in}}%
\pgfpathlineto{\pgfqpoint{4.761337in}{0.665725in}}%
\pgfpathlineto{\pgfqpoint{4.764018in}{0.668105in}}%
\pgfpathlineto{\pgfqpoint{4.766783in}{0.668584in}}%
\pgfpathlineto{\pgfqpoint{4.769367in}{0.673645in}}%
\pgfpathlineto{\pgfqpoint{4.772198in}{0.673570in}}%
\pgfpathlineto{\pgfqpoint{4.774732in}{0.671495in}}%
\pgfpathlineto{\pgfqpoint{4.777535in}{0.672514in}}%
\pgfpathlineto{\pgfqpoint{4.780083in}{0.671666in}}%
\pgfpathlineto{\pgfqpoint{4.782872in}{0.672061in}}%
\pgfpathlineto{\pgfqpoint{4.785445in}{0.671366in}}%
\pgfpathlineto{\pgfqpoint{4.788116in}{0.671923in}}%
\pgfpathlineto{\pgfqpoint{4.790798in}{0.670199in}}%
\pgfpathlineto{\pgfqpoint{4.793512in}{0.669184in}}%
\pgfpathlineto{\pgfqpoint{4.796274in}{0.672642in}}%
\pgfpathlineto{\pgfqpoint{4.798830in}{0.666519in}}%
\pgfpathlineto{\pgfqpoint{4.801586in}{0.667744in}}%
\pgfpathlineto{\pgfqpoint{4.804193in}{0.664711in}}%
\pgfpathlineto{\pgfqpoint{4.807017in}{0.666890in}}%
\pgfpathlineto{\pgfqpoint{4.809538in}{0.663629in}}%
\pgfpathlineto{\pgfqpoint{4.812377in}{0.663635in}}%
\pgfpathlineto{\pgfqpoint{4.814907in}{0.661209in}}%
\pgfpathlineto{\pgfqpoint{4.817587in}{0.659894in}}%
\pgfpathlineto{\pgfqpoint{4.820265in}{0.659894in}}%
\pgfpathlineto{\pgfqpoint{4.822945in}{0.659894in}}%
\pgfpathlineto{\pgfqpoint{4.825619in}{0.661440in}}%
\pgfpathlineto{\pgfqpoint{4.828291in}{0.662754in}}%
\pgfpathlineto{\pgfqpoint{4.831045in}{0.661918in}}%
\pgfpathlineto{\pgfqpoint{4.833657in}{0.665182in}}%
\pgfpathlineto{\pgfqpoint{4.837992in}{0.667492in}}%
\pgfpathlineto{\pgfqpoint{4.839922in}{0.663217in}}%
\pgfpathlineto{\pgfqpoint{4.842380in}{0.665571in}}%
\pgfpathlineto{\pgfqpoint{4.844361in}{0.664872in}}%
\pgfpathlineto{\pgfqpoint{4.847127in}{0.662099in}}%
\pgfpathlineto{\pgfqpoint{4.849715in}{0.669827in}}%
\pgfpathlineto{\pgfqpoint{4.852404in}{0.671436in}}%
\pgfpathlineto{\pgfqpoint{4.855070in}{0.664861in}}%
\pgfpathlineto{\pgfqpoint{4.857807in}{0.667599in}}%
\pgfpathlineto{\pgfqpoint{4.860544in}{0.669791in}}%
\pgfpathlineto{\pgfqpoint{4.863116in}{0.664004in}}%
\pgfpathlineto{\pgfqpoint{4.865910in}{0.664946in}}%
\pgfpathlineto{\pgfqpoint{4.868474in}{0.667134in}}%
\pgfpathlineto{\pgfqpoint{4.871209in}{0.667277in}}%
\pgfpathlineto{\pgfqpoint{4.873832in}{0.670745in}}%
\pgfpathlineto{\pgfqpoint{4.876636in}{0.672614in}}%
\pgfpathlineto{\pgfqpoint{4.879180in}{0.668733in}}%
\pgfpathlineto{\pgfqpoint{4.881864in}{0.667319in}}%
\pgfpathlineto{\pgfqpoint{4.884540in}{0.666510in}}%
\pgfpathlineto{\pgfqpoint{4.887211in}{0.669194in}}%
\pgfpathlineto{\pgfqpoint{4.889902in}{0.666728in}}%
\pgfpathlineto{\pgfqpoint{4.892611in}{0.667712in}}%
\pgfpathlineto{\pgfqpoint{4.895399in}{0.668217in}}%
\pgfpathlineto{\pgfqpoint{4.897938in}{0.670988in}}%
\pgfpathlineto{\pgfqpoint{4.900712in}{0.668128in}}%
\pgfpathlineto{\pgfqpoint{4.903295in}{0.671812in}}%
\pgfpathlineto{\pgfqpoint{4.906096in}{0.678035in}}%
\pgfpathlineto{\pgfqpoint{4.908648in}{0.708594in}}%
\pgfpathlineto{\pgfqpoint{4.911435in}{0.713393in}}%
\pgfpathlineto{\pgfqpoint{4.914009in}{0.703430in}}%
\pgfpathlineto{\pgfqpoint{4.916681in}{0.697602in}}%
\pgfpathlineto{\pgfqpoint{4.919352in}{0.696815in}}%
\pgfpathlineto{\pgfqpoint{4.922041in}{0.693040in}}%
\pgfpathlineto{\pgfqpoint{4.924708in}{0.687417in}}%
\pgfpathlineto{\pgfqpoint{4.927400in}{0.685807in}}%
\pgfpathlineto{\pgfqpoint{4.930170in}{0.689983in}}%
\pgfpathlineto{\pgfqpoint{4.932742in}{0.685956in}}%
\pgfpathlineto{\pgfqpoint{4.935515in}{0.680686in}}%
\pgfpathlineto{\pgfqpoint{4.938112in}{0.676541in}}%
\pgfpathlineto{\pgfqpoint{4.940881in}{0.668933in}}%
\pgfpathlineto{\pgfqpoint{4.943466in}{0.673233in}}%
\pgfpathlineto{\pgfqpoint{4.946151in}{0.670687in}}%
\pgfpathlineto{\pgfqpoint{4.948827in}{0.669345in}}%
\pgfpathlineto{\pgfqpoint{4.951504in}{0.669520in}}%
\pgfpathlineto{\pgfqpoint{4.954182in}{0.672807in}}%
\pgfpathlineto{\pgfqpoint{4.956862in}{0.666448in}}%
\pgfpathlineto{\pgfqpoint{4.959689in}{0.662254in}}%
\pgfpathlineto{\pgfqpoint{4.962219in}{0.663468in}}%
\pgfpathlineto{\pgfqpoint{4.965002in}{0.667963in}}%
\pgfpathlineto{\pgfqpoint{4.967575in}{0.666057in}}%
\pgfpathlineto{\pgfqpoint{4.970314in}{0.666176in}}%
\pgfpathlineto{\pgfqpoint{4.972933in}{0.665019in}}%
\pgfpathlineto{\pgfqpoint{4.975703in}{0.663892in}}%
\pgfpathlineto{\pgfqpoint{4.978287in}{0.665663in}}%
\pgfpathlineto{\pgfqpoint{4.980967in}{0.664414in}}%
\pgfpathlineto{\pgfqpoint{4.983637in}{0.666027in}}%
\pgfpathlineto{\pgfqpoint{4.986325in}{0.661657in}}%
\pgfpathlineto{\pgfqpoint{4.989001in}{0.660853in}}%
\pgfpathlineto{\pgfqpoint{4.991683in}{0.659894in}}%
\pgfpathlineto{\pgfqpoint{4.994390in}{0.664390in}}%
\pgfpathlineto{\pgfqpoint{4.997028in}{0.667566in}}%
\pgfpathlineto{\pgfqpoint{4.999780in}{0.664193in}}%
\pgfpathlineto{\pgfqpoint{5.002384in}{0.665257in}}%
\pgfpathlineto{\pgfqpoint{5.005178in}{0.666246in}}%
\pgfpathlineto{\pgfqpoint{5.007751in}{0.663684in}}%
\pgfpathlineto{\pgfqpoint{5.010562in}{0.671071in}}%
\pgfpathlineto{\pgfqpoint{5.013104in}{0.670440in}}%
\pgfpathlineto{\pgfqpoint{5.015820in}{0.672951in}}%
\pgfpathlineto{\pgfqpoint{5.018466in}{0.677736in}}%
\pgfpathlineto{\pgfqpoint{5.021147in}{0.677511in}}%
\pgfpathlineto{\pgfqpoint{5.023927in}{0.674589in}}%
\pgfpathlineto{\pgfqpoint{5.026501in}{0.670889in}}%
\pgfpathlineto{\pgfqpoint{5.029275in}{0.672066in}}%
\pgfpathlineto{\pgfqpoint{5.031849in}{0.672662in}}%
\pgfpathlineto{\pgfqpoint{5.034649in}{0.677452in}}%
\pgfpathlineto{\pgfqpoint{5.037214in}{0.678565in}}%
\pgfpathlineto{\pgfqpoint{5.039962in}{0.676769in}}%
\pgfpathlineto{\pgfqpoint{5.042572in}{0.680138in}}%
\pgfpathlineto{\pgfqpoint{5.045249in}{0.675596in}}%
\pgfpathlineto{\pgfqpoint{5.047924in}{0.672118in}}%
\pgfpathlineto{\pgfqpoint{5.050606in}{0.671866in}}%
\pgfpathlineto{\pgfqpoint{5.053284in}{0.673499in}}%
\pgfpathlineto{\pgfqpoint{5.055952in}{0.675377in}}%
\pgfpathlineto{\pgfqpoint{5.058711in}{0.671902in}}%
\pgfpathlineto{\pgfqpoint{5.061315in}{0.669178in}}%
\pgfpathlineto{\pgfqpoint{5.064144in}{0.672123in}}%
\pgfpathlineto{\pgfqpoint{5.066677in}{0.672887in}}%
\pgfpathlineto{\pgfqpoint{5.069463in}{0.668895in}}%
\pgfpathlineto{\pgfqpoint{5.072030in}{0.669418in}}%
\pgfpathlineto{\pgfqpoint{5.074851in}{0.667903in}}%
\pgfpathlineto{\pgfqpoint{5.077390in}{0.668579in}}%
\pgfpathlineto{\pgfqpoint{5.080067in}{0.666607in}}%
\pgfpathlineto{\pgfqpoint{5.082746in}{0.665963in}}%
\pgfpathlineto{\pgfqpoint{5.085426in}{0.668239in}}%
\pgfpathlineto{\pgfqpoint{5.088103in}{0.669579in}}%
\pgfpathlineto{\pgfqpoint{5.090788in}{0.665596in}}%
\pgfpathlineto{\pgfqpoint{5.093579in}{0.667398in}}%
\pgfpathlineto{\pgfqpoint{5.096142in}{0.669356in}}%
\pgfpathlineto{\pgfqpoint{5.098948in}{0.669729in}}%
\pgfpathlineto{\pgfqpoint{5.101496in}{0.671358in}}%
\pgfpathlineto{\pgfqpoint{5.104312in}{0.669814in}}%
\pgfpathlineto{\pgfqpoint{5.106842in}{0.668734in}}%
\pgfpathlineto{\pgfqpoint{5.109530in}{0.668824in}}%
\pgfpathlineto{\pgfqpoint{5.112209in}{0.668024in}}%
\pgfpathlineto{\pgfqpoint{5.114887in}{0.668916in}}%
\pgfpathlineto{\pgfqpoint{5.117550in}{0.669221in}}%
\pgfpathlineto{\pgfqpoint{5.120243in}{0.668122in}}%
\pgfpathlineto{\pgfqpoint{5.123042in}{0.668865in}}%
\pgfpathlineto{\pgfqpoint{5.125599in}{0.672984in}}%
\pgfpathlineto{\pgfqpoint{5.128421in}{0.667640in}}%
\pgfpathlineto{\pgfqpoint{5.130953in}{0.667568in}}%
\pgfpathlineto{\pgfqpoint{5.133716in}{0.672349in}}%
\pgfpathlineto{\pgfqpoint{5.136311in}{0.670192in}}%
\pgfpathlineto{\pgfqpoint{5.139072in}{0.672398in}}%
\pgfpathlineto{\pgfqpoint{5.141660in}{0.672693in}}%
\pgfpathlineto{\pgfqpoint{5.144349in}{0.673774in}}%
\pgfpathlineto{\pgfqpoint{5.147029in}{0.669792in}}%
\pgfpathlineto{\pgfqpoint{5.149734in}{0.665832in}}%
\pgfpathlineto{\pgfqpoint{5.152382in}{0.659894in}}%
\pgfpathlineto{\pgfqpoint{5.155059in}{0.664395in}}%
\pgfpathlineto{\pgfqpoint{5.157815in}{0.665831in}}%
\pgfpathlineto{\pgfqpoint{5.160420in}{0.669109in}}%
\pgfpathlineto{\pgfqpoint{5.163243in}{0.671517in}}%
\pgfpathlineto{\pgfqpoint{5.165775in}{0.667836in}}%
\pgfpathlineto{\pgfqpoint{5.168591in}{0.669332in}}%
\pgfpathlineto{\pgfqpoint{5.171133in}{0.666940in}}%
\pgfpathlineto{\pgfqpoint{5.173925in}{0.670093in}}%
\pgfpathlineto{\pgfqpoint{5.176477in}{0.670374in}}%
\pgfpathlineto{\pgfqpoint{5.179188in}{0.666367in}}%
\pgfpathlineto{\pgfqpoint{5.181848in}{0.664506in}}%
\pgfpathlineto{\pgfqpoint{5.184522in}{0.668560in}}%
\pgfpathlineto{\pgfqpoint{5.187294in}{0.668581in}}%
\pgfpathlineto{\pgfqpoint{5.189880in}{0.668913in}}%
\pgfpathlineto{\pgfqpoint{5.192680in}{0.670841in}}%
\pgfpathlineto{\pgfqpoint{5.195239in}{0.672889in}}%
\pgfpathlineto{\pgfqpoint{5.198008in}{0.672527in}}%
\pgfpathlineto{\pgfqpoint{5.200594in}{0.673269in}}%
\pgfpathlineto{\pgfqpoint{5.203388in}{0.670762in}}%
\pgfpathlineto{\pgfqpoint{5.205952in}{0.671527in}}%
\pgfpathlineto{\pgfqpoint{5.208630in}{0.669140in}}%
\pgfpathlineto{\pgfqpoint{5.211299in}{0.667556in}}%
\pgfpathlineto{\pgfqpoint{5.214027in}{0.673502in}}%
\pgfpathlineto{\pgfqpoint{5.216667in}{0.672067in}}%
\pgfpathlineto{\pgfqpoint{5.219345in}{0.669280in}}%
\pgfpathlineto{\pgfqpoint{5.222151in}{0.672883in}}%
\pgfpathlineto{\pgfqpoint{5.224695in}{0.671913in}}%
\pgfpathlineto{\pgfqpoint{5.227470in}{0.671711in}}%
\pgfpathlineto{\pgfqpoint{5.230059in}{0.672042in}}%
\pgfpathlineto{\pgfqpoint{5.232855in}{0.670779in}}%
\pgfpathlineto{\pgfqpoint{5.235409in}{0.670080in}}%
\pgfpathlineto{\pgfqpoint{5.238173in}{0.673512in}}%
\pgfpathlineto{\pgfqpoint{5.240777in}{0.672011in}}%
\pgfpathlineto{\pgfqpoint{5.243445in}{0.672291in}}%
\pgfpathlineto{\pgfqpoint{5.246130in}{0.673274in}}%
\pgfpathlineto{\pgfqpoint{5.248816in}{0.669713in}}%
\pgfpathlineto{\pgfqpoint{5.251590in}{0.669920in}}%
\pgfpathlineto{\pgfqpoint{5.254236in}{0.669376in}}%
\pgfpathlineto{\pgfqpoint{5.256973in}{0.671735in}}%
\pgfpathlineto{\pgfqpoint{5.259511in}{0.670855in}}%
\pgfpathlineto{\pgfqpoint{5.262264in}{0.670485in}}%
\pgfpathlineto{\pgfqpoint{5.264876in}{0.665992in}}%
\pgfpathlineto{\pgfqpoint{5.267691in}{0.664952in}}%
\pgfpathlineto{\pgfqpoint{5.270238in}{0.672616in}}%
\pgfpathlineto{\pgfqpoint{5.272913in}{0.673164in}}%
\pgfpathlineto{\pgfqpoint{5.275589in}{0.668528in}}%
\pgfpathlineto{\pgfqpoint{5.278322in}{0.664957in}}%
\pgfpathlineto{\pgfqpoint{5.280947in}{0.663637in}}%
\pgfpathlineto{\pgfqpoint{5.283631in}{0.663931in}}%
\pgfpathlineto{\pgfqpoint{5.286436in}{0.665262in}}%
\pgfpathlineto{\pgfqpoint{5.288984in}{0.666046in}}%
\pgfpathlineto{\pgfqpoint{5.291794in}{0.666338in}}%
\pgfpathlineto{\pgfqpoint{5.294339in}{0.670590in}}%
\pgfpathlineto{\pgfqpoint{5.297140in}{0.668916in}}%
\pgfpathlineto{\pgfqpoint{5.299696in}{0.671922in}}%
\pgfpathlineto{\pgfqpoint{5.302443in}{0.667260in}}%
\pgfpathlineto{\pgfqpoint{5.305054in}{0.669582in}}%
\pgfpathlineto{\pgfqpoint{5.307731in}{0.674281in}}%
\pgfpathlineto{\pgfqpoint{5.310411in}{0.674217in}}%
\pgfpathlineto{\pgfqpoint{5.313089in}{0.671067in}}%
\pgfpathlineto{\pgfqpoint{5.315754in}{0.673171in}}%
\pgfpathlineto{\pgfqpoint{5.318430in}{0.673854in}}%
\pgfpathlineto{\pgfqpoint{5.321256in}{0.673032in}}%
\pgfpathlineto{\pgfqpoint{5.323802in}{0.670618in}}%
\pgfpathlineto{\pgfqpoint{5.326564in}{0.669443in}}%
\pgfpathlineto{\pgfqpoint{5.329159in}{0.669830in}}%
\pgfpathlineto{\pgfqpoint{5.331973in}{0.668695in}}%
\pgfpathlineto{\pgfqpoint{5.334510in}{0.674554in}}%
\pgfpathlineto{\pgfqpoint{5.337353in}{0.672237in}}%
\pgfpathlineto{\pgfqpoint{5.339872in}{0.666083in}}%
\pgfpathlineto{\pgfqpoint{5.342549in}{0.688517in}}%
\pgfpathlineto{\pgfqpoint{5.345224in}{0.730323in}}%
\pgfpathlineto{\pgfqpoint{5.347905in}{0.751576in}}%
\pgfpathlineto{\pgfqpoint{5.350723in}{0.729544in}}%
\pgfpathlineto{\pgfqpoint{5.353262in}{0.716638in}}%
\pgfpathlineto{\pgfqpoint{5.356056in}{0.709206in}}%
\pgfpathlineto{\pgfqpoint{5.358612in}{0.702906in}}%
\pgfpathlineto{\pgfqpoint{5.361370in}{0.694381in}}%
\pgfpathlineto{\pgfqpoint{5.363966in}{0.681957in}}%
\pgfpathlineto{\pgfqpoint{5.366727in}{0.679775in}}%
\pgfpathlineto{\pgfqpoint{5.369335in}{0.675176in}}%
\pgfpathlineto{\pgfqpoint{5.372013in}{0.671158in}}%
\pgfpathlineto{\pgfqpoint{5.374692in}{0.672798in}}%
\pgfpathlineto{\pgfqpoint{5.377370in}{0.671489in}}%
\pgfpathlineto{\pgfqpoint{5.380048in}{0.671254in}}%
\pgfpathlineto{\pgfqpoint{5.382725in}{0.669800in}}%
\pgfpathlineto{\pgfqpoint{5.385550in}{0.671411in}}%
\pgfpathlineto{\pgfqpoint{5.388083in}{0.698746in}}%
\pgfpathlineto{\pgfqpoint{5.390900in}{0.685126in}}%
\pgfpathlineto{\pgfqpoint{5.393441in}{0.682891in}}%
\pgfpathlineto{\pgfqpoint{5.396219in}{0.681060in}}%
\pgfpathlineto{\pgfqpoint{5.398784in}{0.689406in}}%
\pgfpathlineto{\pgfqpoint{5.401576in}{0.680714in}}%
\pgfpathlineto{\pgfqpoint{5.404154in}{0.676814in}}%
\pgfpathlineto{\pgfqpoint{5.406832in}{0.672730in}}%
\pgfpathlineto{\pgfqpoint{5.409507in}{0.674228in}}%
\pgfpathlineto{\pgfqpoint{5.412190in}{0.669374in}}%
\pgfpathlineto{\pgfqpoint{5.414954in}{0.668573in}}%
\pgfpathlineto{\pgfqpoint{5.417547in}{0.671044in}}%
\pgfpathlineto{\pgfqpoint{5.420304in}{0.672358in}}%
\pgfpathlineto{\pgfqpoint{5.422897in}{0.669660in}}%
\pgfpathlineto{\pgfqpoint{5.425661in}{0.668067in}}%
\pgfpathlineto{\pgfqpoint{5.428259in}{0.663911in}}%
\pgfpathlineto{\pgfqpoint{5.431015in}{0.666741in}}%
\pgfpathlineto{\pgfqpoint{5.433616in}{0.665661in}}%
\pgfpathlineto{\pgfqpoint{5.436295in}{0.669034in}}%
\pgfpathlineto{\pgfqpoint{5.438974in}{0.663392in}}%
\pgfpathlineto{\pgfqpoint{5.441698in}{0.666674in}}%
\pgfpathlineto{\pgfqpoint{5.444328in}{0.672047in}}%
\pgfpathlineto{\pgfqpoint{5.447021in}{0.670859in}}%
\pgfpathlineto{\pgfqpoint{5.449769in}{0.670185in}}%
\pgfpathlineto{\pgfqpoint{5.452365in}{0.668744in}}%
\pgfpathlineto{\pgfqpoint{5.455168in}{0.670976in}}%
\pgfpathlineto{\pgfqpoint{5.457721in}{0.668902in}}%
\pgfpathlineto{\pgfqpoint{5.460489in}{0.668013in}}%
\pgfpathlineto{\pgfqpoint{5.463079in}{0.666119in}}%
\pgfpathlineto{\pgfqpoint{5.465888in}{0.673304in}}%
\pgfpathlineto{\pgfqpoint{5.468425in}{0.668901in}}%
\pgfpathlineto{\pgfqpoint{5.471113in}{0.668117in}}%
\pgfpathlineto{\pgfqpoint{5.473792in}{0.668053in}}%
\pgfpathlineto{\pgfqpoint{5.476458in}{0.667238in}}%
\pgfpathlineto{\pgfqpoint{5.479152in}{0.666763in}}%
\pgfpathlineto{\pgfqpoint{5.481825in}{0.668308in}}%
\pgfpathlineto{\pgfqpoint{5.484641in}{0.670986in}}%
\pgfpathlineto{\pgfqpoint{5.487176in}{0.671898in}}%
\pgfpathlineto{\pgfqpoint{5.490000in}{0.673724in}}%
\pgfpathlineto{\pgfqpoint{5.492541in}{0.671435in}}%
\pgfpathlineto{\pgfqpoint{5.495346in}{0.672497in}}%
\pgfpathlineto{\pgfqpoint{5.497898in}{0.673061in}}%
\pgfpathlineto{\pgfqpoint{5.500687in}{0.671519in}}%
\pgfpathlineto{\pgfqpoint{5.503255in}{0.669717in}}%
\pgfpathlineto{\pgfqpoint{5.505933in}{0.667138in}}%
\pgfpathlineto{\pgfqpoint{5.508612in}{0.668197in}}%
\pgfpathlineto{\pgfqpoint{5.511290in}{0.669422in}}%
\pgfpathlineto{\pgfqpoint{5.514080in}{0.671864in}}%
\pgfpathlineto{\pgfqpoint{5.516646in}{0.674049in}}%
\pgfpathlineto{\pgfqpoint{5.519433in}{0.675491in}}%
\pgfpathlineto{\pgfqpoint{5.522003in}{0.670581in}}%
\pgfpathlineto{\pgfqpoint{5.524756in}{0.668901in}}%
\pgfpathlineto{\pgfqpoint{5.527360in}{0.667633in}}%
\pgfpathlineto{\pgfqpoint{5.530148in}{0.664800in}}%
\pgfpathlineto{\pgfqpoint{5.532717in}{0.663581in}}%
\pgfpathlineto{\pgfqpoint{5.535395in}{0.669190in}}%
\pgfpathlineto{\pgfqpoint{5.538074in}{0.666041in}}%
\pgfpathlineto{\pgfqpoint{5.540750in}{0.664398in}}%
\pgfpathlineto{\pgfqpoint{5.543421in}{0.668533in}}%
\pgfpathlineto{\pgfqpoint{5.546110in}{0.666648in}}%
\pgfpathlineto{\pgfqpoint{5.548921in}{0.669740in}}%
\pgfpathlineto{\pgfqpoint{5.551457in}{0.669340in}}%
\pgfpathlineto{\pgfqpoint{5.554198in}{0.665352in}}%
\pgfpathlineto{\pgfqpoint{5.556822in}{0.669894in}}%
\pgfpathlineto{\pgfqpoint{5.559612in}{0.668903in}}%
\pgfpathlineto{\pgfqpoint{5.562180in}{0.678312in}}%
\pgfpathlineto{\pgfqpoint{5.564940in}{0.683052in}}%
\pgfpathlineto{\pgfqpoint{5.567536in}{0.673431in}}%
\pgfpathlineto{\pgfqpoint{5.570215in}{0.669971in}}%
\pgfpathlineto{\pgfqpoint{5.572893in}{0.673201in}}%
\pgfpathlineto{\pgfqpoint{5.575596in}{0.670401in}}%
\pgfpathlineto{\pgfqpoint{5.578342in}{0.671869in}}%
\pgfpathlineto{\pgfqpoint{5.580914in}{0.671914in}}%
\pgfpathlineto{\pgfqpoint{5.583709in}{0.672037in}}%
\pgfpathlineto{\pgfqpoint{5.586269in}{0.671910in}}%
\pgfpathlineto{\pgfqpoint{5.589040in}{0.674229in}}%
\pgfpathlineto{\pgfqpoint{5.591641in}{0.672094in}}%
\pgfpathlineto{\pgfqpoint{5.594368in}{0.668926in}}%
\pgfpathlineto{\pgfqpoint{5.596999in}{0.668434in}}%
\pgfpathlineto{\pgfqpoint{5.599674in}{0.673765in}}%
\pgfpathlineto{\pgfqpoint{5.602352in}{0.669428in}}%
\pgfpathlineto{\pgfqpoint{5.605073in}{0.676052in}}%
\pgfpathlineto{\pgfqpoint{5.607698in}{0.674307in}}%
\pgfpathlineto{\pgfqpoint{5.610389in}{0.673197in}}%
\pgfpathlineto{\pgfqpoint{5.613235in}{0.673765in}}%
\pgfpathlineto{\pgfqpoint{5.615743in}{0.674083in}}%
\pgfpathlineto{\pgfqpoint{5.618526in}{0.675984in}}%
\pgfpathlineto{\pgfqpoint{5.621102in}{0.671447in}}%
\pgfpathlineto{\pgfqpoint{5.623868in}{0.670922in}}%
\pgfpathlineto{\pgfqpoint{5.626460in}{0.670579in}}%
\pgfpathlineto{\pgfqpoint{5.629232in}{0.669990in}}%
\pgfpathlineto{\pgfqpoint{5.631815in}{0.669736in}}%
\pgfpathlineto{\pgfqpoint{5.634496in}{0.667902in}}%
\pgfpathlineto{\pgfqpoint{5.637172in}{0.666776in}}%
\pgfpathlineto{\pgfqpoint{5.639852in}{0.675238in}}%
\pgfpathlineto{\pgfqpoint{5.642518in}{0.686880in}}%
\pgfpathlineto{\pgfqpoint{5.645243in}{0.682830in}}%
\pgfpathlineto{\pgfqpoint{5.648008in}{0.697129in}}%
\pgfpathlineto{\pgfqpoint{5.650563in}{0.702468in}}%
\pgfpathlineto{\pgfqpoint{5.653376in}{0.682448in}}%
\pgfpathlineto{\pgfqpoint{5.655919in}{0.673090in}}%
\pgfpathlineto{\pgfqpoint{5.658723in}{0.670996in}}%
\pgfpathlineto{\pgfqpoint{5.661273in}{0.672515in}}%
\pgfpathlineto{\pgfqpoint{5.664099in}{0.673170in}}%
\pgfpathlineto{\pgfqpoint{5.666632in}{0.675926in}}%
\pgfpathlineto{\pgfqpoint{5.669313in}{0.675724in}}%
\pgfpathlineto{\pgfqpoint{5.671991in}{0.665989in}}%
\pgfpathlineto{\pgfqpoint{5.674667in}{0.670364in}}%
\pgfpathlineto{\pgfqpoint{5.677486in}{0.670338in}}%
\pgfpathlineto{\pgfqpoint{5.680027in}{0.671217in}}%
\pgfpathlineto{\pgfqpoint{5.682836in}{0.671961in}}%
\pgfpathlineto{\pgfqpoint{5.685385in}{0.671480in}}%
\pgfpathlineto{\pgfqpoint{5.688159in}{0.671237in}}%
\pgfpathlineto{\pgfqpoint{5.690730in}{0.666506in}}%
\pgfpathlineto{\pgfqpoint{5.693473in}{0.673467in}}%
\pgfpathlineto{\pgfqpoint{5.696101in}{0.667036in}}%
\pgfpathlineto{\pgfqpoint{5.698775in}{0.669893in}}%
\pgfpathlineto{\pgfqpoint{5.701453in}{0.673568in}}%
\pgfpathlineto{\pgfqpoint{5.704130in}{0.738133in}}%
\pgfpathlineto{\pgfqpoint{5.706800in}{0.770985in}}%
\pgfpathlineto{\pgfqpoint{5.709490in}{0.821601in}}%
\pgfpathlineto{\pgfqpoint{5.712291in}{0.826971in}}%
\pgfpathlineto{\pgfqpoint{5.714834in}{0.791399in}}%
\pgfpathlineto{\pgfqpoint{5.717671in}{0.764486in}}%
\pgfpathlineto{\pgfqpoint{5.720201in}{0.741966in}}%
\pgfpathlineto{\pgfqpoint{5.722950in}{0.728129in}}%
\pgfpathlineto{\pgfqpoint{5.725548in}{0.725023in}}%
\pgfpathlineto{\pgfqpoint{5.728339in}{0.722356in}}%
\pgfpathlineto{\pgfqpoint{5.730919in}{0.705333in}}%
\pgfpathlineto{\pgfqpoint{5.733594in}{0.702750in}}%
\pgfpathlineto{\pgfqpoint{5.736276in}{0.698930in}}%
\pgfpathlineto{\pgfqpoint{5.738974in}{0.693591in}}%
\pgfpathlineto{\pgfqpoint{5.741745in}{0.691302in}}%
\pgfpathlineto{\pgfqpoint{5.744310in}{0.689431in}}%
\pgfpathlineto{\pgfqpoint{5.744310in}{0.413320in}}%
\pgfpathlineto{\pgfqpoint{5.744310in}{0.413320in}}%
\pgfpathlineto{\pgfqpoint{5.741745in}{0.413320in}}%
\pgfpathlineto{\pgfqpoint{5.738974in}{0.413320in}}%
\pgfpathlineto{\pgfqpoint{5.736276in}{0.413320in}}%
\pgfpathlineto{\pgfqpoint{5.733594in}{0.413320in}}%
\pgfpathlineto{\pgfqpoint{5.730919in}{0.413320in}}%
\pgfpathlineto{\pgfqpoint{5.728339in}{0.413320in}}%
\pgfpathlineto{\pgfqpoint{5.725548in}{0.413320in}}%
\pgfpathlineto{\pgfqpoint{5.722950in}{0.413320in}}%
\pgfpathlineto{\pgfqpoint{5.720201in}{0.413320in}}%
\pgfpathlineto{\pgfqpoint{5.717671in}{0.413320in}}%
\pgfpathlineto{\pgfqpoint{5.714834in}{0.413320in}}%
\pgfpathlineto{\pgfqpoint{5.712291in}{0.413320in}}%
\pgfpathlineto{\pgfqpoint{5.709490in}{0.413320in}}%
\pgfpathlineto{\pgfqpoint{5.706800in}{0.413320in}}%
\pgfpathlineto{\pgfqpoint{5.704130in}{0.413320in}}%
\pgfpathlineto{\pgfqpoint{5.701453in}{0.413320in}}%
\pgfpathlineto{\pgfqpoint{5.698775in}{0.413320in}}%
\pgfpathlineto{\pgfqpoint{5.696101in}{0.413320in}}%
\pgfpathlineto{\pgfqpoint{5.693473in}{0.413320in}}%
\pgfpathlineto{\pgfqpoint{5.690730in}{0.413320in}}%
\pgfpathlineto{\pgfqpoint{5.688159in}{0.413320in}}%
\pgfpathlineto{\pgfqpoint{5.685385in}{0.413320in}}%
\pgfpathlineto{\pgfqpoint{5.682836in}{0.413320in}}%
\pgfpathlineto{\pgfqpoint{5.680027in}{0.413320in}}%
\pgfpathlineto{\pgfqpoint{5.677486in}{0.413320in}}%
\pgfpathlineto{\pgfqpoint{5.674667in}{0.413320in}}%
\pgfpathlineto{\pgfqpoint{5.671991in}{0.413320in}}%
\pgfpathlineto{\pgfqpoint{5.669313in}{0.413320in}}%
\pgfpathlineto{\pgfqpoint{5.666632in}{0.413320in}}%
\pgfpathlineto{\pgfqpoint{5.664099in}{0.413320in}}%
\pgfpathlineto{\pgfqpoint{5.661273in}{0.413320in}}%
\pgfpathlineto{\pgfqpoint{5.658723in}{0.413320in}}%
\pgfpathlineto{\pgfqpoint{5.655919in}{0.413320in}}%
\pgfpathlineto{\pgfqpoint{5.653376in}{0.413320in}}%
\pgfpathlineto{\pgfqpoint{5.650563in}{0.413320in}}%
\pgfpathlineto{\pgfqpoint{5.648008in}{0.413320in}}%
\pgfpathlineto{\pgfqpoint{5.645243in}{0.413320in}}%
\pgfpathlineto{\pgfqpoint{5.642518in}{0.413320in}}%
\pgfpathlineto{\pgfqpoint{5.639852in}{0.413320in}}%
\pgfpathlineto{\pgfqpoint{5.637172in}{0.413320in}}%
\pgfpathlineto{\pgfqpoint{5.634496in}{0.413320in}}%
\pgfpathlineto{\pgfqpoint{5.631815in}{0.413320in}}%
\pgfpathlineto{\pgfqpoint{5.629232in}{0.413320in}}%
\pgfpathlineto{\pgfqpoint{5.626460in}{0.413320in}}%
\pgfpathlineto{\pgfqpoint{5.623868in}{0.413320in}}%
\pgfpathlineto{\pgfqpoint{5.621102in}{0.413320in}}%
\pgfpathlineto{\pgfqpoint{5.618526in}{0.413320in}}%
\pgfpathlineto{\pgfqpoint{5.615743in}{0.413320in}}%
\pgfpathlineto{\pgfqpoint{5.613235in}{0.413320in}}%
\pgfpathlineto{\pgfqpoint{5.610389in}{0.413320in}}%
\pgfpathlineto{\pgfqpoint{5.607698in}{0.413320in}}%
\pgfpathlineto{\pgfqpoint{5.605073in}{0.413320in}}%
\pgfpathlineto{\pgfqpoint{5.602352in}{0.413320in}}%
\pgfpathlineto{\pgfqpoint{5.599674in}{0.413320in}}%
\pgfpathlineto{\pgfqpoint{5.596999in}{0.413320in}}%
\pgfpathlineto{\pgfqpoint{5.594368in}{0.413320in}}%
\pgfpathlineto{\pgfqpoint{5.591641in}{0.413320in}}%
\pgfpathlineto{\pgfqpoint{5.589040in}{0.413320in}}%
\pgfpathlineto{\pgfqpoint{5.586269in}{0.413320in}}%
\pgfpathlineto{\pgfqpoint{5.583709in}{0.413320in}}%
\pgfpathlineto{\pgfqpoint{5.580914in}{0.413320in}}%
\pgfpathlineto{\pgfqpoint{5.578342in}{0.413320in}}%
\pgfpathlineto{\pgfqpoint{5.575596in}{0.413320in}}%
\pgfpathlineto{\pgfqpoint{5.572893in}{0.413320in}}%
\pgfpathlineto{\pgfqpoint{5.570215in}{0.413320in}}%
\pgfpathlineto{\pgfqpoint{5.567536in}{0.413320in}}%
\pgfpathlineto{\pgfqpoint{5.564940in}{0.413320in}}%
\pgfpathlineto{\pgfqpoint{5.562180in}{0.413320in}}%
\pgfpathlineto{\pgfqpoint{5.559612in}{0.413320in}}%
\pgfpathlineto{\pgfqpoint{5.556822in}{0.413320in}}%
\pgfpathlineto{\pgfqpoint{5.554198in}{0.413320in}}%
\pgfpathlineto{\pgfqpoint{5.551457in}{0.413320in}}%
\pgfpathlineto{\pgfqpoint{5.548921in}{0.413320in}}%
\pgfpathlineto{\pgfqpoint{5.546110in}{0.413320in}}%
\pgfpathlineto{\pgfqpoint{5.543421in}{0.413320in}}%
\pgfpathlineto{\pgfqpoint{5.540750in}{0.413320in}}%
\pgfpathlineto{\pgfqpoint{5.538074in}{0.413320in}}%
\pgfpathlineto{\pgfqpoint{5.535395in}{0.413320in}}%
\pgfpathlineto{\pgfqpoint{5.532717in}{0.413320in}}%
\pgfpathlineto{\pgfqpoint{5.530148in}{0.413320in}}%
\pgfpathlineto{\pgfqpoint{5.527360in}{0.413320in}}%
\pgfpathlineto{\pgfqpoint{5.524756in}{0.413320in}}%
\pgfpathlineto{\pgfqpoint{5.522003in}{0.413320in}}%
\pgfpathlineto{\pgfqpoint{5.519433in}{0.413320in}}%
\pgfpathlineto{\pgfqpoint{5.516646in}{0.413320in}}%
\pgfpathlineto{\pgfqpoint{5.514080in}{0.413320in}}%
\pgfpathlineto{\pgfqpoint{5.511290in}{0.413320in}}%
\pgfpathlineto{\pgfqpoint{5.508612in}{0.413320in}}%
\pgfpathlineto{\pgfqpoint{5.505933in}{0.413320in}}%
\pgfpathlineto{\pgfqpoint{5.503255in}{0.413320in}}%
\pgfpathlineto{\pgfqpoint{5.500687in}{0.413320in}}%
\pgfpathlineto{\pgfqpoint{5.497898in}{0.413320in}}%
\pgfpathlineto{\pgfqpoint{5.495346in}{0.413320in}}%
\pgfpathlineto{\pgfqpoint{5.492541in}{0.413320in}}%
\pgfpathlineto{\pgfqpoint{5.490000in}{0.413320in}}%
\pgfpathlineto{\pgfqpoint{5.487176in}{0.413320in}}%
\pgfpathlineto{\pgfqpoint{5.484641in}{0.413320in}}%
\pgfpathlineto{\pgfqpoint{5.481825in}{0.413320in}}%
\pgfpathlineto{\pgfqpoint{5.479152in}{0.413320in}}%
\pgfpathlineto{\pgfqpoint{5.476458in}{0.413320in}}%
\pgfpathlineto{\pgfqpoint{5.473792in}{0.413320in}}%
\pgfpathlineto{\pgfqpoint{5.471113in}{0.413320in}}%
\pgfpathlineto{\pgfqpoint{5.468425in}{0.413320in}}%
\pgfpathlineto{\pgfqpoint{5.465888in}{0.413320in}}%
\pgfpathlineto{\pgfqpoint{5.463079in}{0.413320in}}%
\pgfpathlineto{\pgfqpoint{5.460489in}{0.413320in}}%
\pgfpathlineto{\pgfqpoint{5.457721in}{0.413320in}}%
\pgfpathlineto{\pgfqpoint{5.455168in}{0.413320in}}%
\pgfpathlineto{\pgfqpoint{5.452365in}{0.413320in}}%
\pgfpathlineto{\pgfqpoint{5.449769in}{0.413320in}}%
\pgfpathlineto{\pgfqpoint{5.447021in}{0.413320in}}%
\pgfpathlineto{\pgfqpoint{5.444328in}{0.413320in}}%
\pgfpathlineto{\pgfqpoint{5.441698in}{0.413320in}}%
\pgfpathlineto{\pgfqpoint{5.438974in}{0.413320in}}%
\pgfpathlineto{\pgfqpoint{5.436295in}{0.413320in}}%
\pgfpathlineto{\pgfqpoint{5.433616in}{0.413320in}}%
\pgfpathlineto{\pgfqpoint{5.431015in}{0.413320in}}%
\pgfpathlineto{\pgfqpoint{5.428259in}{0.413320in}}%
\pgfpathlineto{\pgfqpoint{5.425661in}{0.413320in}}%
\pgfpathlineto{\pgfqpoint{5.422897in}{0.413320in}}%
\pgfpathlineto{\pgfqpoint{5.420304in}{0.413320in}}%
\pgfpathlineto{\pgfqpoint{5.417547in}{0.413320in}}%
\pgfpathlineto{\pgfqpoint{5.414954in}{0.413320in}}%
\pgfpathlineto{\pgfqpoint{5.412190in}{0.413320in}}%
\pgfpathlineto{\pgfqpoint{5.409507in}{0.413320in}}%
\pgfpathlineto{\pgfqpoint{5.406832in}{0.413320in}}%
\pgfpathlineto{\pgfqpoint{5.404154in}{0.413320in}}%
\pgfpathlineto{\pgfqpoint{5.401576in}{0.413320in}}%
\pgfpathlineto{\pgfqpoint{5.398784in}{0.413320in}}%
\pgfpathlineto{\pgfqpoint{5.396219in}{0.413320in}}%
\pgfpathlineto{\pgfqpoint{5.393441in}{0.413320in}}%
\pgfpathlineto{\pgfqpoint{5.390900in}{0.413320in}}%
\pgfpathlineto{\pgfqpoint{5.388083in}{0.413320in}}%
\pgfpathlineto{\pgfqpoint{5.385550in}{0.413320in}}%
\pgfpathlineto{\pgfqpoint{5.382725in}{0.413320in}}%
\pgfpathlineto{\pgfqpoint{5.380048in}{0.413320in}}%
\pgfpathlineto{\pgfqpoint{5.377370in}{0.413320in}}%
\pgfpathlineto{\pgfqpoint{5.374692in}{0.413320in}}%
\pgfpathlineto{\pgfqpoint{5.372013in}{0.413320in}}%
\pgfpathlineto{\pgfqpoint{5.369335in}{0.413320in}}%
\pgfpathlineto{\pgfqpoint{5.366727in}{0.413320in}}%
\pgfpathlineto{\pgfqpoint{5.363966in}{0.413320in}}%
\pgfpathlineto{\pgfqpoint{5.361370in}{0.413320in}}%
\pgfpathlineto{\pgfqpoint{5.358612in}{0.413320in}}%
\pgfpathlineto{\pgfqpoint{5.356056in}{0.413320in}}%
\pgfpathlineto{\pgfqpoint{5.353262in}{0.413320in}}%
\pgfpathlineto{\pgfqpoint{5.350723in}{0.413320in}}%
\pgfpathlineto{\pgfqpoint{5.347905in}{0.413320in}}%
\pgfpathlineto{\pgfqpoint{5.345224in}{0.413320in}}%
\pgfpathlineto{\pgfqpoint{5.342549in}{0.413320in}}%
\pgfpathlineto{\pgfqpoint{5.339872in}{0.413320in}}%
\pgfpathlineto{\pgfqpoint{5.337353in}{0.413320in}}%
\pgfpathlineto{\pgfqpoint{5.334510in}{0.413320in}}%
\pgfpathlineto{\pgfqpoint{5.331973in}{0.413320in}}%
\pgfpathlineto{\pgfqpoint{5.329159in}{0.413320in}}%
\pgfpathlineto{\pgfqpoint{5.326564in}{0.413320in}}%
\pgfpathlineto{\pgfqpoint{5.323802in}{0.413320in}}%
\pgfpathlineto{\pgfqpoint{5.321256in}{0.413320in}}%
\pgfpathlineto{\pgfqpoint{5.318430in}{0.413320in}}%
\pgfpathlineto{\pgfqpoint{5.315754in}{0.413320in}}%
\pgfpathlineto{\pgfqpoint{5.313089in}{0.413320in}}%
\pgfpathlineto{\pgfqpoint{5.310411in}{0.413320in}}%
\pgfpathlineto{\pgfqpoint{5.307731in}{0.413320in}}%
\pgfpathlineto{\pgfqpoint{5.305054in}{0.413320in}}%
\pgfpathlineto{\pgfqpoint{5.302443in}{0.413320in}}%
\pgfpathlineto{\pgfqpoint{5.299696in}{0.413320in}}%
\pgfpathlineto{\pgfqpoint{5.297140in}{0.413320in}}%
\pgfpathlineto{\pgfqpoint{5.294339in}{0.413320in}}%
\pgfpathlineto{\pgfqpoint{5.291794in}{0.413320in}}%
\pgfpathlineto{\pgfqpoint{5.288984in}{0.413320in}}%
\pgfpathlineto{\pgfqpoint{5.286436in}{0.413320in}}%
\pgfpathlineto{\pgfqpoint{5.283631in}{0.413320in}}%
\pgfpathlineto{\pgfqpoint{5.280947in}{0.413320in}}%
\pgfpathlineto{\pgfqpoint{5.278322in}{0.413320in}}%
\pgfpathlineto{\pgfqpoint{5.275589in}{0.413320in}}%
\pgfpathlineto{\pgfqpoint{5.272913in}{0.413320in}}%
\pgfpathlineto{\pgfqpoint{5.270238in}{0.413320in}}%
\pgfpathlineto{\pgfqpoint{5.267691in}{0.413320in}}%
\pgfpathlineto{\pgfqpoint{5.264876in}{0.413320in}}%
\pgfpathlineto{\pgfqpoint{5.262264in}{0.413320in}}%
\pgfpathlineto{\pgfqpoint{5.259511in}{0.413320in}}%
\pgfpathlineto{\pgfqpoint{5.256973in}{0.413320in}}%
\pgfpathlineto{\pgfqpoint{5.254236in}{0.413320in}}%
\pgfpathlineto{\pgfqpoint{5.251590in}{0.413320in}}%
\pgfpathlineto{\pgfqpoint{5.248816in}{0.413320in}}%
\pgfpathlineto{\pgfqpoint{5.246130in}{0.413320in}}%
\pgfpathlineto{\pgfqpoint{5.243445in}{0.413320in}}%
\pgfpathlineto{\pgfqpoint{5.240777in}{0.413320in}}%
\pgfpathlineto{\pgfqpoint{5.238173in}{0.413320in}}%
\pgfpathlineto{\pgfqpoint{5.235409in}{0.413320in}}%
\pgfpathlineto{\pgfqpoint{5.232855in}{0.413320in}}%
\pgfpathlineto{\pgfqpoint{5.230059in}{0.413320in}}%
\pgfpathlineto{\pgfqpoint{5.227470in}{0.413320in}}%
\pgfpathlineto{\pgfqpoint{5.224695in}{0.413320in}}%
\pgfpathlineto{\pgfqpoint{5.222151in}{0.413320in}}%
\pgfpathlineto{\pgfqpoint{5.219345in}{0.413320in}}%
\pgfpathlineto{\pgfqpoint{5.216667in}{0.413320in}}%
\pgfpathlineto{\pgfqpoint{5.214027in}{0.413320in}}%
\pgfpathlineto{\pgfqpoint{5.211299in}{0.413320in}}%
\pgfpathlineto{\pgfqpoint{5.208630in}{0.413320in}}%
\pgfpathlineto{\pgfqpoint{5.205952in}{0.413320in}}%
\pgfpathlineto{\pgfqpoint{5.203388in}{0.413320in}}%
\pgfpathlineto{\pgfqpoint{5.200594in}{0.413320in}}%
\pgfpathlineto{\pgfqpoint{5.198008in}{0.413320in}}%
\pgfpathlineto{\pgfqpoint{5.195239in}{0.413320in}}%
\pgfpathlineto{\pgfqpoint{5.192680in}{0.413320in}}%
\pgfpathlineto{\pgfqpoint{5.189880in}{0.413320in}}%
\pgfpathlineto{\pgfqpoint{5.187294in}{0.413320in}}%
\pgfpathlineto{\pgfqpoint{5.184522in}{0.413320in}}%
\pgfpathlineto{\pgfqpoint{5.181848in}{0.413320in}}%
\pgfpathlineto{\pgfqpoint{5.179188in}{0.413320in}}%
\pgfpathlineto{\pgfqpoint{5.176477in}{0.413320in}}%
\pgfpathlineto{\pgfqpoint{5.173925in}{0.413320in}}%
\pgfpathlineto{\pgfqpoint{5.171133in}{0.413320in}}%
\pgfpathlineto{\pgfqpoint{5.168591in}{0.413320in}}%
\pgfpathlineto{\pgfqpoint{5.165775in}{0.413320in}}%
\pgfpathlineto{\pgfqpoint{5.163243in}{0.413320in}}%
\pgfpathlineto{\pgfqpoint{5.160420in}{0.413320in}}%
\pgfpathlineto{\pgfqpoint{5.157815in}{0.413320in}}%
\pgfpathlineto{\pgfqpoint{5.155059in}{0.413320in}}%
\pgfpathlineto{\pgfqpoint{5.152382in}{0.413320in}}%
\pgfpathlineto{\pgfqpoint{5.149734in}{0.413320in}}%
\pgfpathlineto{\pgfqpoint{5.147029in}{0.413320in}}%
\pgfpathlineto{\pgfqpoint{5.144349in}{0.413320in}}%
\pgfpathlineto{\pgfqpoint{5.141660in}{0.413320in}}%
\pgfpathlineto{\pgfqpoint{5.139072in}{0.413320in}}%
\pgfpathlineto{\pgfqpoint{5.136311in}{0.413320in}}%
\pgfpathlineto{\pgfqpoint{5.133716in}{0.413320in}}%
\pgfpathlineto{\pgfqpoint{5.130953in}{0.413320in}}%
\pgfpathlineto{\pgfqpoint{5.128421in}{0.413320in}}%
\pgfpathlineto{\pgfqpoint{5.125599in}{0.413320in}}%
\pgfpathlineto{\pgfqpoint{5.123042in}{0.413320in}}%
\pgfpathlineto{\pgfqpoint{5.120243in}{0.413320in}}%
\pgfpathlineto{\pgfqpoint{5.117550in}{0.413320in}}%
\pgfpathlineto{\pgfqpoint{5.114887in}{0.413320in}}%
\pgfpathlineto{\pgfqpoint{5.112209in}{0.413320in}}%
\pgfpathlineto{\pgfqpoint{5.109530in}{0.413320in}}%
\pgfpathlineto{\pgfqpoint{5.106842in}{0.413320in}}%
\pgfpathlineto{\pgfqpoint{5.104312in}{0.413320in}}%
\pgfpathlineto{\pgfqpoint{5.101496in}{0.413320in}}%
\pgfpathlineto{\pgfqpoint{5.098948in}{0.413320in}}%
\pgfpathlineto{\pgfqpoint{5.096142in}{0.413320in}}%
\pgfpathlineto{\pgfqpoint{5.093579in}{0.413320in}}%
\pgfpathlineto{\pgfqpoint{5.090788in}{0.413320in}}%
\pgfpathlineto{\pgfqpoint{5.088103in}{0.413320in}}%
\pgfpathlineto{\pgfqpoint{5.085426in}{0.413320in}}%
\pgfpathlineto{\pgfqpoint{5.082746in}{0.413320in}}%
\pgfpathlineto{\pgfqpoint{5.080067in}{0.413320in}}%
\pgfpathlineto{\pgfqpoint{5.077390in}{0.413320in}}%
\pgfpathlineto{\pgfqpoint{5.074851in}{0.413320in}}%
\pgfpathlineto{\pgfqpoint{5.072030in}{0.413320in}}%
\pgfpathlineto{\pgfqpoint{5.069463in}{0.413320in}}%
\pgfpathlineto{\pgfqpoint{5.066677in}{0.413320in}}%
\pgfpathlineto{\pgfqpoint{5.064144in}{0.413320in}}%
\pgfpathlineto{\pgfqpoint{5.061315in}{0.413320in}}%
\pgfpathlineto{\pgfqpoint{5.058711in}{0.413320in}}%
\pgfpathlineto{\pgfqpoint{5.055952in}{0.413320in}}%
\pgfpathlineto{\pgfqpoint{5.053284in}{0.413320in}}%
\pgfpathlineto{\pgfqpoint{5.050606in}{0.413320in}}%
\pgfpathlineto{\pgfqpoint{5.047924in}{0.413320in}}%
\pgfpathlineto{\pgfqpoint{5.045249in}{0.413320in}}%
\pgfpathlineto{\pgfqpoint{5.042572in}{0.413320in}}%
\pgfpathlineto{\pgfqpoint{5.039962in}{0.413320in}}%
\pgfpathlineto{\pgfqpoint{5.037214in}{0.413320in}}%
\pgfpathlineto{\pgfqpoint{5.034649in}{0.413320in}}%
\pgfpathlineto{\pgfqpoint{5.031849in}{0.413320in}}%
\pgfpathlineto{\pgfqpoint{5.029275in}{0.413320in}}%
\pgfpathlineto{\pgfqpoint{5.026501in}{0.413320in}}%
\pgfpathlineto{\pgfqpoint{5.023927in}{0.413320in}}%
\pgfpathlineto{\pgfqpoint{5.021147in}{0.413320in}}%
\pgfpathlineto{\pgfqpoint{5.018466in}{0.413320in}}%
\pgfpathlineto{\pgfqpoint{5.015820in}{0.413320in}}%
\pgfpathlineto{\pgfqpoint{5.013104in}{0.413320in}}%
\pgfpathlineto{\pgfqpoint{5.010562in}{0.413320in}}%
\pgfpathlineto{\pgfqpoint{5.007751in}{0.413320in}}%
\pgfpathlineto{\pgfqpoint{5.005178in}{0.413320in}}%
\pgfpathlineto{\pgfqpoint{5.002384in}{0.413320in}}%
\pgfpathlineto{\pgfqpoint{4.999780in}{0.413320in}}%
\pgfpathlineto{\pgfqpoint{4.997028in}{0.413320in}}%
\pgfpathlineto{\pgfqpoint{4.994390in}{0.413320in}}%
\pgfpathlineto{\pgfqpoint{4.991683in}{0.413320in}}%
\pgfpathlineto{\pgfqpoint{4.989001in}{0.413320in}}%
\pgfpathlineto{\pgfqpoint{4.986325in}{0.413320in}}%
\pgfpathlineto{\pgfqpoint{4.983637in}{0.413320in}}%
\pgfpathlineto{\pgfqpoint{4.980967in}{0.413320in}}%
\pgfpathlineto{\pgfqpoint{4.978287in}{0.413320in}}%
\pgfpathlineto{\pgfqpoint{4.975703in}{0.413320in}}%
\pgfpathlineto{\pgfqpoint{4.972933in}{0.413320in}}%
\pgfpathlineto{\pgfqpoint{4.970314in}{0.413320in}}%
\pgfpathlineto{\pgfqpoint{4.967575in}{0.413320in}}%
\pgfpathlineto{\pgfqpoint{4.965002in}{0.413320in}}%
\pgfpathlineto{\pgfqpoint{4.962219in}{0.413320in}}%
\pgfpathlineto{\pgfqpoint{4.959689in}{0.413320in}}%
\pgfpathlineto{\pgfqpoint{4.956862in}{0.413320in}}%
\pgfpathlineto{\pgfqpoint{4.954182in}{0.413320in}}%
\pgfpathlineto{\pgfqpoint{4.951504in}{0.413320in}}%
\pgfpathlineto{\pgfqpoint{4.948827in}{0.413320in}}%
\pgfpathlineto{\pgfqpoint{4.946151in}{0.413320in}}%
\pgfpathlineto{\pgfqpoint{4.943466in}{0.413320in}}%
\pgfpathlineto{\pgfqpoint{4.940881in}{0.413320in}}%
\pgfpathlineto{\pgfqpoint{4.938112in}{0.413320in}}%
\pgfpathlineto{\pgfqpoint{4.935515in}{0.413320in}}%
\pgfpathlineto{\pgfqpoint{4.932742in}{0.413320in}}%
\pgfpathlineto{\pgfqpoint{4.930170in}{0.413320in}}%
\pgfpathlineto{\pgfqpoint{4.927400in}{0.413320in}}%
\pgfpathlineto{\pgfqpoint{4.924708in}{0.413320in}}%
\pgfpathlineto{\pgfqpoint{4.922041in}{0.413320in}}%
\pgfpathlineto{\pgfqpoint{4.919352in}{0.413320in}}%
\pgfpathlineto{\pgfqpoint{4.916681in}{0.413320in}}%
\pgfpathlineto{\pgfqpoint{4.914009in}{0.413320in}}%
\pgfpathlineto{\pgfqpoint{4.911435in}{0.413320in}}%
\pgfpathlineto{\pgfqpoint{4.908648in}{0.413320in}}%
\pgfpathlineto{\pgfqpoint{4.906096in}{0.413320in}}%
\pgfpathlineto{\pgfqpoint{4.903295in}{0.413320in}}%
\pgfpathlineto{\pgfqpoint{4.900712in}{0.413320in}}%
\pgfpathlineto{\pgfqpoint{4.897938in}{0.413320in}}%
\pgfpathlineto{\pgfqpoint{4.895399in}{0.413320in}}%
\pgfpathlineto{\pgfqpoint{4.892611in}{0.413320in}}%
\pgfpathlineto{\pgfqpoint{4.889902in}{0.413320in}}%
\pgfpathlineto{\pgfqpoint{4.887211in}{0.413320in}}%
\pgfpathlineto{\pgfqpoint{4.884540in}{0.413320in}}%
\pgfpathlineto{\pgfqpoint{4.881864in}{0.413320in}}%
\pgfpathlineto{\pgfqpoint{4.879180in}{0.413320in}}%
\pgfpathlineto{\pgfqpoint{4.876636in}{0.413320in}}%
\pgfpathlineto{\pgfqpoint{4.873832in}{0.413320in}}%
\pgfpathlineto{\pgfqpoint{4.871209in}{0.413320in}}%
\pgfpathlineto{\pgfqpoint{4.868474in}{0.413320in}}%
\pgfpathlineto{\pgfqpoint{4.865910in}{0.413320in}}%
\pgfpathlineto{\pgfqpoint{4.863116in}{0.413320in}}%
\pgfpathlineto{\pgfqpoint{4.860544in}{0.413320in}}%
\pgfpathlineto{\pgfqpoint{4.857807in}{0.413320in}}%
\pgfpathlineto{\pgfqpoint{4.855070in}{0.413320in}}%
\pgfpathlineto{\pgfqpoint{4.852404in}{0.413320in}}%
\pgfpathlineto{\pgfqpoint{4.849715in}{0.413320in}}%
\pgfpathlineto{\pgfqpoint{4.847127in}{0.413320in}}%
\pgfpathlineto{\pgfqpoint{4.844361in}{0.413320in}}%
\pgfpathlineto{\pgfqpoint{4.842380in}{0.413320in}}%
\pgfpathlineto{\pgfqpoint{4.839922in}{0.413320in}}%
\pgfpathlineto{\pgfqpoint{4.837992in}{0.413320in}}%
\pgfpathlineto{\pgfqpoint{4.833657in}{0.413320in}}%
\pgfpathlineto{\pgfqpoint{4.831045in}{0.413320in}}%
\pgfpathlineto{\pgfqpoint{4.828291in}{0.413320in}}%
\pgfpathlineto{\pgfqpoint{4.825619in}{0.413320in}}%
\pgfpathlineto{\pgfqpoint{4.822945in}{0.413320in}}%
\pgfpathlineto{\pgfqpoint{4.820265in}{0.413320in}}%
\pgfpathlineto{\pgfqpoint{4.817587in}{0.413320in}}%
\pgfpathlineto{\pgfqpoint{4.814907in}{0.413320in}}%
\pgfpathlineto{\pgfqpoint{4.812377in}{0.413320in}}%
\pgfpathlineto{\pgfqpoint{4.809538in}{0.413320in}}%
\pgfpathlineto{\pgfqpoint{4.807017in}{0.413320in}}%
\pgfpathlineto{\pgfqpoint{4.804193in}{0.413320in}}%
\pgfpathlineto{\pgfqpoint{4.801586in}{0.413320in}}%
\pgfpathlineto{\pgfqpoint{4.798830in}{0.413320in}}%
\pgfpathlineto{\pgfqpoint{4.796274in}{0.413320in}}%
\pgfpathlineto{\pgfqpoint{4.793512in}{0.413320in}}%
\pgfpathlineto{\pgfqpoint{4.790798in}{0.413320in}}%
\pgfpathlineto{\pgfqpoint{4.788116in}{0.413320in}}%
\pgfpathlineto{\pgfqpoint{4.785445in}{0.413320in}}%
\pgfpathlineto{\pgfqpoint{4.782872in}{0.413320in}}%
\pgfpathlineto{\pgfqpoint{4.780083in}{0.413320in}}%
\pgfpathlineto{\pgfqpoint{4.777535in}{0.413320in}}%
\pgfpathlineto{\pgfqpoint{4.774732in}{0.413320in}}%
\pgfpathlineto{\pgfqpoint{4.772198in}{0.413320in}}%
\pgfpathlineto{\pgfqpoint{4.769367in}{0.413320in}}%
\pgfpathlineto{\pgfqpoint{4.766783in}{0.413320in}}%
\pgfpathlineto{\pgfqpoint{4.764018in}{0.413320in}}%
\pgfpathlineto{\pgfqpoint{4.761337in}{0.413320in}}%
\pgfpathlineto{\pgfqpoint{4.758653in}{0.413320in}}%
\pgfpathlineto{\pgfqpoint{4.755983in}{0.413320in}}%
\pgfpathlineto{\pgfqpoint{4.753298in}{0.413320in}}%
\pgfpathlineto{\pgfqpoint{4.750627in}{0.413320in}}%
\pgfpathlineto{\pgfqpoint{4.748081in}{0.413320in}}%
\pgfpathlineto{\pgfqpoint{4.745256in}{0.413320in}}%
\pgfpathlineto{\pgfqpoint{4.742696in}{0.413320in}}%
\pgfpathlineto{\pgfqpoint{4.739912in}{0.413320in}}%
\pgfpathlineto{\pgfqpoint{4.737348in}{0.413320in}}%
\pgfpathlineto{\pgfqpoint{4.734552in}{0.413320in}}%
\pgfpathlineto{\pgfqpoint{4.731901in}{0.413320in}}%
\pgfpathlineto{\pgfqpoint{4.729233in}{0.413320in}}%
\pgfpathlineto{\pgfqpoint{4.726508in}{0.413320in}}%
\pgfpathlineto{\pgfqpoint{4.723873in}{0.413320in}}%
\pgfpathlineto{\pgfqpoint{4.721160in}{0.413320in}}%
\pgfpathlineto{\pgfqpoint{4.718486in}{0.413320in}}%
\pgfpathlineto{\pgfqpoint{4.715806in}{0.413320in}}%
\pgfpathlineto{\pgfqpoint{4.713275in}{0.413320in}}%
\pgfpathlineto{\pgfqpoint{4.710437in}{0.413320in}}%
\pgfpathlineto{\pgfqpoint{4.707824in}{0.413320in}}%
\pgfpathlineto{\pgfqpoint{4.705094in}{0.413320in}}%
\pgfpathlineto{\pgfqpoint{4.702517in}{0.413320in}}%
\pgfpathlineto{\pgfqpoint{4.699734in}{0.413320in}}%
\pgfpathlineto{\pgfqpoint{4.697170in}{0.413320in}}%
\pgfpathlineto{\pgfqpoint{4.694381in}{0.413320in}}%
\pgfpathlineto{\pgfqpoint{4.691694in}{0.413320in}}%
\pgfpathlineto{\pgfqpoint{4.689051in}{0.413320in}}%
\pgfpathlineto{\pgfqpoint{4.686337in}{0.413320in}}%
\pgfpathlineto{\pgfqpoint{4.683799in}{0.413320in}}%
\pgfpathlineto{\pgfqpoint{4.680988in}{0.413320in}}%
\pgfpathlineto{\pgfqpoint{4.678448in}{0.413320in}}%
\pgfpathlineto{\pgfqpoint{4.675619in}{0.413320in}}%
\pgfpathlineto{\pgfqpoint{4.673068in}{0.413320in}}%
\pgfpathlineto{\pgfqpoint{4.670261in}{0.413320in}}%
\pgfpathlineto{\pgfqpoint{4.667764in}{0.413320in}}%
\pgfpathlineto{\pgfqpoint{4.664923in}{0.413320in}}%
\pgfpathlineto{\pgfqpoint{4.662237in}{0.413320in}}%
\pgfpathlineto{\pgfqpoint{4.659590in}{0.413320in}}%
\pgfpathlineto{\pgfqpoint{4.656873in}{0.413320in}}%
\pgfpathlineto{\pgfqpoint{4.654203in}{0.413320in}}%
\pgfpathlineto{\pgfqpoint{4.651524in}{0.413320in}}%
\pgfpathlineto{\pgfqpoint{4.648922in}{0.413320in}}%
\pgfpathlineto{\pgfqpoint{4.646169in}{0.413320in}}%
\pgfpathlineto{\pgfqpoint{4.643628in}{0.413320in}}%
\pgfpathlineto{\pgfqpoint{4.640809in}{0.413320in}}%
\pgfpathlineto{\pgfqpoint{4.638204in}{0.413320in}}%
\pgfpathlineto{\pgfqpoint{4.635445in}{0.413320in}}%
\pgfpathlineto{\pgfqpoint{4.632902in}{0.413320in}}%
\pgfpathlineto{\pgfqpoint{4.630096in}{0.413320in}}%
\pgfpathlineto{\pgfqpoint{4.627411in}{0.413320in}}%
\pgfpathlineto{\pgfqpoint{4.624741in}{0.413320in}}%
\pgfpathlineto{\pgfqpoint{4.622056in}{0.413320in}}%
\pgfpathlineto{\pgfqpoint{4.619529in}{0.413320in}}%
\pgfpathlineto{\pgfqpoint{4.616702in}{0.413320in}}%
\pgfpathlineto{\pgfqpoint{4.614134in}{0.413320in}}%
\pgfpathlineto{\pgfqpoint{4.611350in}{0.413320in}}%
\pgfpathlineto{\pgfqpoint{4.608808in}{0.413320in}}%
\pgfpathlineto{\pgfqpoint{4.605990in}{0.413320in}}%
\pgfpathlineto{\pgfqpoint{4.603430in}{0.413320in}}%
\pgfpathlineto{\pgfqpoint{4.600633in}{0.413320in}}%
\pgfpathlineto{\pgfqpoint{4.597951in}{0.413320in}}%
\pgfpathlineto{\pgfqpoint{4.595281in}{0.413320in}}%
\pgfpathlineto{\pgfqpoint{4.592589in}{0.413320in}}%
\pgfpathlineto{\pgfqpoint{4.589920in}{0.413320in}}%
\pgfpathlineto{\pgfqpoint{4.587244in}{0.413320in}}%
\pgfpathlineto{\pgfqpoint{4.584672in}{0.413320in}}%
\pgfpathlineto{\pgfqpoint{4.581888in}{0.413320in}}%
\pgfpathlineto{\pgfqpoint{4.579305in}{0.413320in}}%
\pgfpathlineto{\pgfqpoint{4.576531in}{0.413320in}}%
\pgfpathlineto{\pgfqpoint{4.573947in}{0.413320in}}%
\pgfpathlineto{\pgfqpoint{4.571171in}{0.413320in}}%
\pgfpathlineto{\pgfqpoint{4.568612in}{0.413320in}}%
\pgfpathlineto{\pgfqpoint{4.565820in}{0.413320in}}%
\pgfpathlineto{\pgfqpoint{4.563125in}{0.413320in}}%
\pgfpathlineto{\pgfqpoint{4.560448in}{0.413320in}}%
\pgfpathlineto{\pgfqpoint{4.557777in}{0.413320in}}%
\pgfpathlineto{\pgfqpoint{4.555106in}{0.413320in}}%
\pgfpathlineto{\pgfqpoint{4.552425in}{0.413320in}}%
\pgfpathlineto{\pgfqpoint{4.549822in}{0.413320in}}%
\pgfpathlineto{\pgfqpoint{4.547064in}{0.413320in}}%
\pgfpathlineto{\pgfqpoint{4.544464in}{0.413320in}}%
\pgfpathlineto{\pgfqpoint{4.541711in}{0.413320in}}%
\pgfpathlineto{\pgfqpoint{4.539144in}{0.413320in}}%
\pgfpathlineto{\pgfqpoint{4.536400in}{0.413320in}}%
\pgfpathlineto{\pgfqpoint{4.533764in}{0.413320in}}%
\pgfpathlineto{\pgfqpoint{4.530990in}{0.413320in}}%
\pgfpathlineto{\pgfqpoint{4.528307in}{0.413320in}}%
\pgfpathlineto{\pgfqpoint{4.525640in}{0.413320in}}%
\pgfpathlineto{\pgfqpoint{4.522962in}{0.413320in}}%
\pgfpathlineto{\pgfqpoint{4.520345in}{0.413320in}}%
\pgfpathlineto{\pgfqpoint{4.517598in}{0.413320in}}%
\pgfpathlineto{\pgfqpoint{4.515080in}{0.413320in}}%
\pgfpathlineto{\pgfqpoint{4.512246in}{0.413320in}}%
\pgfpathlineto{\pgfqpoint{4.509643in}{0.413320in}}%
\pgfpathlineto{\pgfqpoint{4.506893in}{0.413320in}}%
\pgfpathlineto{\pgfqpoint{4.504305in}{0.413320in}}%
\pgfpathlineto{\pgfqpoint{4.501529in}{0.413320in}}%
\pgfpathlineto{\pgfqpoint{4.498850in}{0.413320in}}%
\pgfpathlineto{\pgfqpoint{4.496167in}{0.413320in}}%
\pgfpathlineto{\pgfqpoint{4.493492in}{0.413320in}}%
\pgfpathlineto{\pgfqpoint{4.490822in}{0.413320in}}%
\pgfpathlineto{\pgfqpoint{4.488130in}{0.413320in}}%
\pgfpathlineto{\pgfqpoint{4.485581in}{0.413320in}}%
\pgfpathlineto{\pgfqpoint{4.482778in}{0.413320in}}%
\pgfpathlineto{\pgfqpoint{4.480201in}{0.413320in}}%
\pgfpathlineto{\pgfqpoint{4.477430in}{0.413320in}}%
\pgfpathlineto{\pgfqpoint{4.474861in}{0.413320in}}%
\pgfpathlineto{\pgfqpoint{4.472059in}{0.413320in}}%
\pgfpathlineto{\pgfqpoint{4.469492in}{0.413320in}}%
\pgfpathlineto{\pgfqpoint{4.466717in}{0.413320in}}%
\pgfpathlineto{\pgfqpoint{4.464029in}{0.413320in}}%
\pgfpathlineto{\pgfqpoint{4.461367in}{0.413320in}}%
\pgfpathlineto{\pgfqpoint{4.458681in}{0.413320in}}%
\pgfpathlineto{\pgfqpoint{4.456138in}{0.413320in}}%
\pgfpathlineto{\pgfqpoint{4.453312in}{0.413320in}}%
\pgfpathlineto{\pgfqpoint{4.450767in}{0.413320in}}%
\pgfpathlineto{\pgfqpoint{4.447965in}{0.413320in}}%
\pgfpathlineto{\pgfqpoint{4.445423in}{0.413320in}}%
\pgfpathlineto{\pgfqpoint{4.442611in}{0.413320in}}%
\pgfpathlineto{\pgfqpoint{4.440041in}{0.413320in}}%
\pgfpathlineto{\pgfqpoint{4.437253in}{0.413320in}}%
\pgfpathlineto{\pgfqpoint{4.434569in}{0.413320in}}%
\pgfpathlineto{\pgfqpoint{4.431901in}{0.413320in}}%
\pgfpathlineto{\pgfqpoint{4.429220in}{0.413320in}}%
\pgfpathlineto{\pgfqpoint{4.426534in}{0.413320in}}%
\pgfpathlineto{\pgfqpoint{4.423863in}{0.413320in}}%
\pgfpathlineto{\pgfqpoint{4.421292in}{0.413320in}}%
\pgfpathlineto{\pgfqpoint{4.418506in}{0.413320in}}%
\pgfpathlineto{\pgfqpoint{4.415932in}{0.413320in}}%
\pgfpathlineto{\pgfqpoint{4.413149in}{0.413320in}}%
\pgfpathlineto{\pgfqpoint{4.410587in}{0.413320in}}%
\pgfpathlineto{\pgfqpoint{4.407788in}{0.413320in}}%
\pgfpathlineto{\pgfqpoint{4.405234in}{0.413320in}}%
\pgfpathlineto{\pgfqpoint{4.402468in}{0.413320in}}%
\pgfpathlineto{\pgfqpoint{4.399745in}{0.413320in}}%
\pgfpathlineto{\pgfqpoint{4.397076in}{0.413320in}}%
\pgfpathlineto{\pgfqpoint{4.394400in}{0.413320in}}%
\pgfpathlineto{\pgfqpoint{4.391721in}{0.413320in}}%
\pgfpathlineto{\pgfqpoint{4.389044in}{0.413320in}}%
\pgfpathlineto{\pgfqpoint{4.386431in}{0.413320in}}%
\pgfpathlineto{\pgfqpoint{4.383674in}{0.413320in}}%
\pgfpathlineto{\pgfqpoint{4.381097in}{0.413320in}}%
\pgfpathlineto{\pgfqpoint{4.378329in}{0.413320in}}%
\pgfpathlineto{\pgfqpoint{4.375761in}{0.413320in}}%
\pgfpathlineto{\pgfqpoint{4.372976in}{0.413320in}}%
\pgfpathlineto{\pgfqpoint{4.370437in}{0.413320in}}%
\pgfpathlineto{\pgfqpoint{4.367646in}{0.413320in}}%
\pgfpathlineto{\pgfqpoint{4.364936in}{0.413320in}}%
\pgfpathlineto{\pgfqpoint{4.362270in}{0.413320in}}%
\pgfpathlineto{\pgfqpoint{4.359582in}{0.413320in}}%
\pgfpathlineto{\pgfqpoint{4.357014in}{0.413320in}}%
\pgfpathlineto{\pgfqpoint{4.354224in}{0.413320in}}%
\pgfpathlineto{\pgfqpoint{4.351645in}{0.413320in}}%
\pgfpathlineto{\pgfqpoint{4.348868in}{0.413320in}}%
\pgfpathlineto{\pgfqpoint{4.346263in}{0.413320in}}%
\pgfpathlineto{\pgfqpoint{4.343510in}{0.413320in}}%
\pgfpathlineto{\pgfqpoint{4.340976in}{0.413320in}}%
\pgfpathlineto{\pgfqpoint{4.338154in}{0.413320in}}%
\pgfpathlineto{\pgfqpoint{4.335463in}{0.413320in}}%
\pgfpathlineto{\pgfqpoint{4.332796in}{0.413320in}}%
\pgfpathlineto{\pgfqpoint{4.330118in}{0.413320in}}%
\pgfpathlineto{\pgfqpoint{4.327440in}{0.413320in}}%
\pgfpathlineto{\pgfqpoint{4.324760in}{0.413320in}}%
\pgfpathlineto{\pgfqpoint{4.322181in}{0.413320in}}%
\pgfpathlineto{\pgfqpoint{4.319405in}{0.413320in}}%
\pgfpathlineto{\pgfqpoint{4.316856in}{0.413320in}}%
\pgfpathlineto{\pgfqpoint{4.314032in}{0.413320in}}%
\pgfpathlineto{\pgfqpoint{4.311494in}{0.413320in}}%
\pgfpathlineto{\pgfqpoint{4.308691in}{0.413320in}}%
\pgfpathlineto{\pgfqpoint{4.306118in}{0.413320in}}%
\pgfpathlineto{\pgfqpoint{4.303357in}{0.413320in}}%
\pgfpathlineto{\pgfqpoint{4.300656in}{0.413320in}}%
\pgfpathlineto{\pgfqpoint{4.297977in}{0.413320in}}%
\pgfpathlineto{\pgfqpoint{4.295299in}{0.413320in}}%
\pgfpathlineto{\pgfqpoint{4.292786in}{0.413320in}}%
\pgfpathlineto{\pgfqpoint{4.289936in}{0.413320in}}%
\pgfpathlineto{\pgfqpoint{4.287399in}{0.413320in}}%
\pgfpathlineto{\pgfqpoint{4.284586in}{0.413320in}}%
\pgfpathlineto{\pgfqpoint{4.282000in}{0.413320in}}%
\pgfpathlineto{\pgfqpoint{4.279212in}{0.413320in}}%
\pgfpathlineto{\pgfqpoint{4.276635in}{0.413320in}}%
\pgfpathlineto{\pgfqpoint{4.273874in}{0.413320in}}%
\pgfpathlineto{\pgfqpoint{4.271187in}{0.413320in}}%
\pgfpathlineto{\pgfqpoint{4.268590in}{0.413320in}}%
\pgfpathlineto{\pgfqpoint{4.265824in}{0.413320in}}%
\pgfpathlineto{\pgfqpoint{4.263157in}{0.413320in}}%
\pgfpathlineto{\pgfqpoint{4.260477in}{0.413320in}}%
\pgfpathlineto{\pgfqpoint{4.257958in}{0.413320in}}%
\pgfpathlineto{\pgfqpoint{4.255120in}{0.413320in}}%
\pgfpathlineto{\pgfqpoint{4.252581in}{0.413320in}}%
\pgfpathlineto{\pgfqpoint{4.249767in}{0.413320in}}%
\pgfpathlineto{\pgfqpoint{4.247225in}{0.413320in}}%
\pgfpathlineto{\pgfqpoint{4.244394in}{0.413320in}}%
\pgfpathlineto{\pgfqpoint{4.241900in}{0.413320in}}%
\pgfpathlineto{\pgfqpoint{4.239084in}{0.413320in}}%
\pgfpathlineto{\pgfqpoint{4.236375in}{0.413320in}}%
\pgfpathlineto{\pgfqpoint{4.233691in}{0.413320in}}%
\pgfpathlineto{\pgfqpoint{4.231013in}{0.413320in}}%
\pgfpathlineto{\pgfqpoint{4.228331in}{0.413320in}}%
\pgfpathlineto{\pgfqpoint{4.225654in}{0.413320in}}%
\pgfpathlineto{\pgfqpoint{4.223082in}{0.413320in}}%
\pgfpathlineto{\pgfqpoint{4.220304in}{0.413320in}}%
\pgfpathlineto{\pgfqpoint{4.217694in}{0.413320in}}%
\pgfpathlineto{\pgfqpoint{4.214948in}{0.413320in}}%
\pgfpathlineto{\pgfqpoint{4.212383in}{0.413320in}}%
\pgfpathlineto{\pgfqpoint{4.209597in}{0.413320in}}%
\pgfpathlineto{\pgfqpoint{4.207076in}{0.413320in}}%
\pgfpathlineto{\pgfqpoint{4.204240in}{0.413320in}}%
\pgfpathlineto{\pgfqpoint{4.201542in}{0.413320in}}%
\pgfpathlineto{\pgfqpoint{4.198878in}{0.413320in}}%
\pgfpathlineto{\pgfqpoint{4.196186in}{0.413320in}}%
\pgfpathlineto{\pgfqpoint{4.193638in}{0.413320in}}%
\pgfpathlineto{\pgfqpoint{4.190842in}{0.413320in}}%
\pgfpathlineto{\pgfqpoint{4.188318in}{0.413320in}}%
\pgfpathlineto{\pgfqpoint{4.185481in}{0.413320in}}%
\pgfpathlineto{\pgfqpoint{4.182899in}{0.413320in}}%
\pgfpathlineto{\pgfqpoint{4.180129in}{0.413320in}}%
\pgfpathlineto{\pgfqpoint{4.177593in}{0.413320in}}%
\pgfpathlineto{\pgfqpoint{4.174770in}{0.413320in}}%
\pgfpathlineto{\pgfqpoint{4.172093in}{0.413320in}}%
\pgfpathlineto{\pgfqpoint{4.169415in}{0.413320in}}%
\pgfpathlineto{\pgfqpoint{4.166737in}{0.413320in}}%
\pgfpathlineto{\pgfqpoint{4.164059in}{0.413320in}}%
\pgfpathlineto{\pgfqpoint{4.161380in}{0.413320in}}%
\pgfpathlineto{\pgfqpoint{4.158806in}{0.413320in}}%
\pgfpathlineto{\pgfqpoint{4.156016in}{0.413320in}}%
\pgfpathlineto{\pgfqpoint{4.153423in}{0.413320in}}%
\pgfpathlineto{\pgfqpoint{4.150665in}{0.413320in}}%
\pgfpathlineto{\pgfqpoint{4.148082in}{0.413320in}}%
\pgfpathlineto{\pgfqpoint{4.145310in}{0.413320in}}%
\pgfpathlineto{\pgfqpoint{4.142713in}{0.413320in}}%
\pgfpathlineto{\pgfqpoint{4.139963in}{0.413320in}}%
\pgfpathlineto{\pgfqpoint{4.137272in}{0.413320in}}%
\pgfpathlineto{\pgfqpoint{4.134615in}{0.413320in}}%
\pgfpathlineto{\pgfqpoint{4.131920in}{0.413320in}}%
\pgfpathlineto{\pgfqpoint{4.129349in}{0.413320in}}%
\pgfpathlineto{\pgfqpoint{4.126553in}{0.413320in}}%
\pgfpathlineto{\pgfqpoint{4.124019in}{0.413320in}}%
\pgfpathlineto{\pgfqpoint{4.121205in}{0.413320in}}%
\pgfpathlineto{\pgfqpoint{4.118554in}{0.413320in}}%
\pgfpathlineto{\pgfqpoint{4.115844in}{0.413320in}}%
\pgfpathlineto{\pgfqpoint{4.113252in}{0.413320in}}%
\pgfpathlineto{\pgfqpoint{4.110488in}{0.413320in}}%
\pgfpathlineto{\pgfqpoint{4.107814in}{0.413320in}}%
\pgfpathlineto{\pgfqpoint{4.105185in}{0.413320in}}%
\pgfpathlineto{\pgfqpoint{4.102456in}{0.413320in}}%
\pgfpathlineto{\pgfqpoint{4.099777in}{0.413320in}}%
\pgfpathlineto{\pgfqpoint{4.097092in}{0.413320in}}%
\pgfpathlineto{\pgfqpoint{4.094527in}{0.413320in}}%
\pgfpathlineto{\pgfqpoint{4.091729in}{0.413320in}}%
\pgfpathlineto{\pgfqpoint{4.089159in}{0.413320in}}%
\pgfpathlineto{\pgfqpoint{4.086385in}{0.413320in}}%
\pgfpathlineto{\pgfqpoint{4.083870in}{0.413320in}}%
\pgfpathlineto{\pgfqpoint{4.081018in}{0.413320in}}%
\pgfpathlineto{\pgfqpoint{4.078471in}{0.413320in}}%
\pgfpathlineto{\pgfqpoint{4.075705in}{0.413320in}}%
\pgfpathlineto{\pgfqpoint{4.072985in}{0.413320in}}%
\pgfpathlineto{\pgfqpoint{4.070313in}{0.413320in}}%
\pgfpathlineto{\pgfqpoint{4.067636in}{0.413320in}}%
\pgfpathlineto{\pgfqpoint{4.064957in}{0.413320in}}%
\pgfpathlineto{\pgfqpoint{4.062266in}{0.413320in}}%
\pgfpathlineto{\pgfqpoint{4.059702in}{0.413320in}}%
\pgfpathlineto{\pgfqpoint{4.056911in}{0.413320in}}%
\pgfpathlineto{\pgfqpoint{4.054326in}{0.413320in}}%
\pgfpathlineto{\pgfqpoint{4.051557in}{0.413320in}}%
\pgfpathlineto{\pgfqpoint{4.049006in}{0.413320in}}%
\pgfpathlineto{\pgfqpoint{4.046210in}{0.413320in}}%
\pgfpathlineto{\pgfqpoint{4.043667in}{0.413320in}}%
\pgfpathlineto{\pgfqpoint{4.040852in}{0.413320in}}%
\pgfpathlineto{\pgfqpoint{4.038174in}{0.413320in}}%
\pgfpathlineto{\pgfqpoint{4.035492in}{0.413320in}}%
\pgfpathlineto{\pgfqpoint{4.032817in}{0.413320in}}%
\pgfpathlineto{\pgfqpoint{4.030229in}{0.413320in}}%
\pgfpathlineto{\pgfqpoint{4.027447in}{0.413320in}}%
\pgfpathlineto{\pgfqpoint{4.024868in}{0.413320in}}%
\pgfpathlineto{\pgfqpoint{4.022097in}{0.413320in}}%
\pgfpathlineto{\pgfqpoint{4.019518in}{0.413320in}}%
\pgfpathlineto{\pgfqpoint{4.016744in}{0.413320in}}%
\pgfpathlineto{\pgfqpoint{4.014186in}{0.413320in}}%
\pgfpathlineto{\pgfqpoint{4.011394in}{0.413320in}}%
\pgfpathlineto{\pgfqpoint{4.008699in}{0.413320in}}%
\pgfpathlineto{\pgfqpoint{4.006034in}{0.413320in}}%
\pgfpathlineto{\pgfqpoint{4.003348in}{0.413320in}}%
\pgfpathlineto{\pgfqpoint{4.000674in}{0.413320in}}%
\pgfpathlineto{\pgfqpoint{3.997990in}{0.413320in}}%
\pgfpathlineto{\pgfqpoint{3.995417in}{0.413320in}}%
\pgfpathlineto{\pgfqpoint{3.992642in}{0.413320in}}%
\pgfpathlineto{\pgfqpoint{3.990055in}{0.413320in}}%
\pgfpathlineto{\pgfqpoint{3.987270in}{0.413320in}}%
\pgfpathlineto{\pgfqpoint{3.984714in}{0.413320in}}%
\pgfpathlineto{\pgfqpoint{3.981929in}{0.413320in}}%
\pgfpathlineto{\pgfqpoint{3.979389in}{0.413320in}}%
\pgfpathlineto{\pgfqpoint{3.976563in}{0.413320in}}%
\pgfpathlineto{\pgfqpoint{3.973885in}{0.413320in}}%
\pgfpathlineto{\pgfqpoint{3.971250in}{0.413320in}}%
\pgfpathlineto{\pgfqpoint{3.968523in}{0.413320in}}%
\pgfpathlineto{\pgfqpoint{3.966013in}{0.413320in}}%
\pgfpathlineto{\pgfqpoint{3.963176in}{0.413320in}}%
\pgfpathlineto{\pgfqpoint{3.960635in}{0.413320in}}%
\pgfpathlineto{\pgfqpoint{3.957823in}{0.413320in}}%
\pgfpathlineto{\pgfqpoint{3.955211in}{0.413320in}}%
\pgfpathlineto{\pgfqpoint{3.952464in}{0.413320in}}%
\pgfpathlineto{\pgfqpoint{3.949894in}{0.413320in}}%
\pgfpathlineto{\pgfqpoint{3.947101in}{0.413320in}}%
\pgfpathlineto{\pgfqpoint{3.944431in}{0.413320in}}%
\pgfpathlineto{\pgfqpoint{3.941778in}{0.413320in}}%
\pgfpathlineto{\pgfqpoint{3.939075in}{0.413320in}}%
\pgfpathlineto{\pgfqpoint{3.936395in}{0.413320in}}%
\pgfpathlineto{\pgfqpoint{3.933714in}{0.413320in}}%
\pgfpathlineto{\pgfqpoint{3.931202in}{0.413320in}}%
\pgfpathlineto{\pgfqpoint{3.928347in}{0.413320in}}%
\pgfpathlineto{\pgfqpoint{3.925778in}{0.413320in}}%
\pgfpathlineto{\pgfqpoint{3.923005in}{0.413320in}}%
\pgfpathlineto{\pgfqpoint{3.920412in}{0.413320in}}%
\pgfpathlineto{\pgfqpoint{3.917646in}{0.413320in}}%
\pgfpathlineto{\pgfqpoint{3.915107in}{0.413320in}}%
\pgfpathlineto{\pgfqpoint{3.912296in}{0.413320in}}%
\pgfpathlineto{\pgfqpoint{3.909602in}{0.413320in}}%
\pgfpathlineto{\pgfqpoint{3.906918in}{0.413320in}}%
\pgfpathlineto{\pgfqpoint{3.904252in}{0.413320in}}%
\pgfpathlineto{\pgfqpoint{3.901573in}{0.413320in}}%
\pgfpathlineto{\pgfqpoint{3.898891in}{0.413320in}}%
\pgfpathlineto{\pgfqpoint{3.896345in}{0.413320in}}%
\pgfpathlineto{\pgfqpoint{3.893541in}{0.413320in}}%
\pgfpathlineto{\pgfqpoint{3.890926in}{0.413320in}}%
\pgfpathlineto{\pgfqpoint{3.888188in}{0.413320in}}%
\pgfpathlineto{\pgfqpoint{3.885621in}{0.413320in}}%
\pgfpathlineto{\pgfqpoint{3.882850in}{0.413320in}}%
\pgfpathlineto{\pgfqpoint{3.880237in}{0.413320in}}%
\pgfpathlineto{\pgfqpoint{3.877466in}{0.413320in}}%
\pgfpathlineto{\pgfqpoint{3.874790in}{0.413320in}}%
\pgfpathlineto{\pgfqpoint{3.872114in}{0.413320in}}%
\pgfpathlineto{\pgfqpoint{3.869435in}{0.413320in}}%
\pgfpathlineto{\pgfqpoint{3.866815in}{0.413320in}}%
\pgfpathlineto{\pgfqpoint{3.864073in}{0.413320in}}%
\pgfpathlineto{\pgfqpoint{3.861561in}{0.413320in}}%
\pgfpathlineto{\pgfqpoint{3.858720in}{0.413320in}}%
\pgfpathlineto{\pgfqpoint{3.856100in}{0.413320in}}%
\pgfpathlineto{\pgfqpoint{3.853358in}{0.413320in}}%
\pgfpathlineto{\pgfqpoint{3.850814in}{0.413320in}}%
\pgfpathlineto{\pgfqpoint{3.848005in}{0.413320in}}%
\pgfpathlineto{\pgfqpoint{3.845329in}{0.413320in}}%
\pgfpathlineto{\pgfqpoint{3.842641in}{0.413320in}}%
\pgfpathlineto{\pgfqpoint{3.839960in}{0.413320in}}%
\pgfpathlineto{\pgfqpoint{3.837286in}{0.413320in}}%
\pgfpathlineto{\pgfqpoint{3.834616in}{0.413320in}}%
\pgfpathlineto{\pgfqpoint{3.832053in}{0.413320in}}%
\pgfpathlineto{\pgfqpoint{3.829252in}{0.413320in}}%
\pgfpathlineto{\pgfqpoint{3.826679in}{0.413320in}}%
\pgfpathlineto{\pgfqpoint{3.823903in}{0.413320in}}%
\pgfpathlineto{\pgfqpoint{3.821315in}{0.413320in}}%
\pgfpathlineto{\pgfqpoint{3.818546in}{0.413320in}}%
\pgfpathlineto{\pgfqpoint{3.815983in}{0.413320in}}%
\pgfpathlineto{\pgfqpoint{3.813172in}{0.413320in}}%
\pgfpathlineto{\pgfqpoint{3.810510in}{0.413320in}}%
\pgfpathlineto{\pgfqpoint{3.807832in}{0.413320in}}%
\pgfpathlineto{\pgfqpoint{3.805145in}{0.413320in}}%
\pgfpathlineto{\pgfqpoint{3.802569in}{0.413320in}}%
\pgfpathlineto{\pgfqpoint{3.799797in}{0.413320in}}%
\pgfpathlineto{\pgfqpoint{3.797265in}{0.413320in}}%
\pgfpathlineto{\pgfqpoint{3.794435in}{0.413320in}}%
\pgfpathlineto{\pgfqpoint{3.791897in}{0.413320in}}%
\pgfpathlineto{\pgfqpoint{3.789084in}{0.413320in}}%
\pgfpathlineto{\pgfqpoint{3.786504in}{0.413320in}}%
\pgfpathlineto{\pgfqpoint{3.783725in}{0.413320in}}%
\pgfpathlineto{\pgfqpoint{3.781046in}{0.413320in}}%
\pgfpathlineto{\pgfqpoint{3.778370in}{0.413320in}}%
\pgfpathlineto{\pgfqpoint{3.775691in}{0.413320in}}%
\pgfpathlineto{\pgfqpoint{3.773014in}{0.413320in}}%
\pgfpathlineto{\pgfqpoint{3.770323in}{0.413320in}}%
\pgfpathlineto{\pgfqpoint{3.767782in}{0.413320in}}%
\pgfpathlineto{\pgfqpoint{3.764966in}{0.413320in}}%
\pgfpathlineto{\pgfqpoint{3.762389in}{0.413320in}}%
\pgfpathlineto{\pgfqpoint{3.759622in}{0.413320in}}%
\pgfpathlineto{\pgfqpoint{3.757065in}{0.413320in}}%
\pgfpathlineto{\pgfqpoint{3.754265in}{0.413320in}}%
\pgfpathlineto{\pgfqpoint{3.751728in}{0.413320in}}%
\pgfpathlineto{\pgfqpoint{3.748903in}{0.413320in}}%
\pgfpathlineto{\pgfqpoint{3.746229in}{0.413320in}}%
\pgfpathlineto{\pgfqpoint{3.743548in}{0.413320in}}%
\pgfpathlineto{\pgfqpoint{3.740874in}{0.413320in}}%
\pgfpathlineto{\pgfqpoint{3.738194in}{0.413320in}}%
\pgfpathlineto{\pgfqpoint{3.735509in}{0.413320in}}%
\pgfpathlineto{\pgfqpoint{3.732950in}{0.413320in}}%
\pgfpathlineto{\pgfqpoint{3.730158in}{0.413320in}}%
\pgfpathlineto{\pgfqpoint{3.727581in}{0.413320in}}%
\pgfpathlineto{\pgfqpoint{3.724804in}{0.413320in}}%
\pgfpathlineto{\pgfqpoint{3.722228in}{0.413320in}}%
\pgfpathlineto{\pgfqpoint{3.719446in}{0.413320in}}%
\pgfpathlineto{\pgfqpoint{3.716875in}{0.413320in}}%
\pgfpathlineto{\pgfqpoint{3.714086in}{0.413320in}}%
\pgfpathlineto{\pgfqpoint{3.711410in}{0.413320in}}%
\pgfpathlineto{\pgfqpoint{3.708729in}{0.413320in}}%
\pgfpathlineto{\pgfqpoint{3.706053in}{0.413320in}}%
\pgfpathlineto{\pgfqpoint{3.703460in}{0.413320in}}%
\pgfpathlineto{\pgfqpoint{3.700684in}{0.413320in}}%
\pgfpathlineto{\pgfqpoint{3.698125in}{0.413320in}}%
\pgfpathlineto{\pgfqpoint{3.695331in}{0.413320in}}%
\pgfpathlineto{\pgfqpoint{3.692765in}{0.413320in}}%
\pgfpathlineto{\pgfqpoint{3.689983in}{0.413320in}}%
\pgfpathlineto{\pgfqpoint{3.687442in}{0.413320in}}%
\pgfpathlineto{\pgfqpoint{3.684620in}{0.413320in}}%
\pgfpathlineto{\pgfqpoint{3.681948in}{0.413320in}}%
\pgfpathlineto{\pgfqpoint{3.679273in}{0.413320in}}%
\pgfpathlineto{\pgfqpoint{3.676591in}{0.413320in}}%
\pgfpathlineto{\pgfqpoint{3.673911in}{0.413320in}}%
\pgfpathlineto{\pgfqpoint{3.671232in}{0.413320in}}%
\pgfpathlineto{\pgfqpoint{3.668665in}{0.413320in}}%
\pgfpathlineto{\pgfqpoint{3.665864in}{0.413320in}}%
\pgfpathlineto{\pgfqpoint{3.663276in}{0.413320in}}%
\pgfpathlineto{\pgfqpoint{3.660515in}{0.413320in}}%
\pgfpathlineto{\pgfqpoint{3.657917in}{0.413320in}}%
\pgfpathlineto{\pgfqpoint{3.655165in}{0.413320in}}%
\pgfpathlineto{\pgfqpoint{3.652628in}{0.413320in}}%
\pgfpathlineto{\pgfqpoint{3.649837in}{0.413320in}}%
\pgfpathlineto{\pgfqpoint{3.647130in}{0.413320in}}%
\pgfpathlineto{\pgfqpoint{3.644452in}{0.413320in}}%
\pgfpathlineto{\pgfqpoint{3.641773in}{0.413320in}}%
\pgfpathlineto{\pgfqpoint{3.639207in}{0.413320in}}%
\pgfpathlineto{\pgfqpoint{3.636413in}{0.413320in}}%
\pgfpathlineto{\pgfqpoint{3.633858in}{0.413320in}}%
\pgfpathlineto{\pgfqpoint{3.631058in}{0.413320in}}%
\pgfpathlineto{\pgfqpoint{3.628460in}{0.413320in}}%
\pgfpathlineto{\pgfqpoint{3.625689in}{0.413320in}}%
\pgfpathlineto{\pgfqpoint{3.623165in}{0.413320in}}%
\pgfpathlineto{\pgfqpoint{3.620345in}{0.413320in}}%
\pgfpathlineto{\pgfqpoint{3.617667in}{0.413320in}}%
\pgfpathlineto{\pgfqpoint{3.614982in}{0.413320in}}%
\pgfpathlineto{\pgfqpoint{3.612311in}{0.413320in}}%
\pgfpathlineto{\pgfqpoint{3.609632in}{0.413320in}}%
\pgfpathlineto{\pgfqpoint{3.606951in}{0.413320in}}%
\pgfpathlineto{\pgfqpoint{3.604387in}{0.413320in}}%
\pgfpathlineto{\pgfqpoint{3.601590in}{0.413320in}}%
\pgfpathlineto{\pgfqpoint{3.598998in}{0.413320in}}%
\pgfpathlineto{\pgfqpoint{3.596240in}{0.413320in}}%
\pgfpathlineto{\pgfqpoint{3.593620in}{0.413320in}}%
\pgfpathlineto{\pgfqpoint{3.590883in}{0.413320in}}%
\pgfpathlineto{\pgfqpoint{3.588258in}{0.413320in}}%
\pgfpathlineto{\pgfqpoint{3.585532in}{0.413320in}}%
\pgfpathlineto{\pgfqpoint{3.582851in}{0.413320in}}%
\pgfpathlineto{\pgfqpoint{3.580191in}{0.413320in}}%
\pgfpathlineto{\pgfqpoint{3.577487in}{0.413320in}}%
\pgfpathlineto{\pgfqpoint{3.574814in}{0.413320in}}%
\pgfpathlineto{\pgfqpoint{3.572126in}{0.413320in}}%
\pgfpathlineto{\pgfqpoint{3.569584in}{0.413320in}}%
\pgfpathlineto{\pgfqpoint{3.566774in}{0.413320in}}%
\pgfpathlineto{\pgfqpoint{3.564188in}{0.413320in}}%
\pgfpathlineto{\pgfqpoint{3.561420in}{0.413320in}}%
\pgfpathlineto{\pgfqpoint{3.558853in}{0.413320in}}%
\pgfpathlineto{\pgfqpoint{3.556061in}{0.413320in}}%
\pgfpathlineto{\pgfqpoint{3.553498in}{0.413320in}}%
\pgfpathlineto{\pgfqpoint{3.550713in}{0.413320in}}%
\pgfpathlineto{\pgfqpoint{3.548029in}{0.413320in}}%
\pgfpathlineto{\pgfqpoint{3.545349in}{0.413320in}}%
\pgfpathlineto{\pgfqpoint{3.542656in}{0.413320in}}%
\pgfpathlineto{\pgfqpoint{3.540093in}{0.413320in}}%
\pgfpathlineto{\pgfqpoint{3.537309in}{0.413320in}}%
\pgfpathlineto{\pgfqpoint{3.534783in}{0.413320in}}%
\pgfpathlineto{\pgfqpoint{3.531955in}{0.413320in}}%
\pgfpathlineto{\pgfqpoint{3.529327in}{0.413320in}}%
\pgfpathlineto{\pgfqpoint{3.526601in}{0.413320in}}%
\pgfpathlineto{\pgfqpoint{3.524041in}{0.413320in}}%
\pgfpathlineto{\pgfqpoint{3.521244in}{0.413320in}}%
\pgfpathlineto{\pgfqpoint{3.518565in}{0.413320in}}%
\pgfpathlineto{\pgfqpoint{3.515884in}{0.413320in}}%
\pgfpathlineto{\pgfqpoint{3.513209in}{0.413320in}}%
\pgfpathlineto{\pgfqpoint{3.510533in}{0.413320in}}%
\pgfpathlineto{\pgfqpoint{3.507840in}{0.413320in}}%
\pgfpathlineto{\pgfqpoint{3.505262in}{0.413320in}}%
\pgfpathlineto{\pgfqpoint{3.502488in}{0.413320in}}%
\pgfpathlineto{\pgfqpoint{3.499909in}{0.413320in}}%
\pgfpathlineto{\pgfqpoint{3.497139in}{0.413320in}}%
\pgfpathlineto{\pgfqpoint{3.494581in}{0.413320in}}%
\pgfpathlineto{\pgfqpoint{3.491783in}{0.413320in}}%
\pgfpathlineto{\pgfqpoint{3.489223in}{0.413320in}}%
\pgfpathlineto{\pgfqpoint{3.486442in}{0.413320in}}%
\pgfpathlineto{\pgfqpoint{3.483744in}{0.413320in}}%
\pgfpathlineto{\pgfqpoint{3.481072in}{0.413320in}}%
\pgfpathlineto{\pgfqpoint{3.478378in}{0.413320in}}%
\pgfpathlineto{\pgfqpoint{3.475821in}{0.413320in}}%
\pgfpathlineto{\pgfqpoint{3.473021in}{0.413320in}}%
\pgfpathlineto{\pgfqpoint{3.470466in}{0.413320in}}%
\pgfpathlineto{\pgfqpoint{3.467678in}{0.413320in}}%
\pgfpathlineto{\pgfqpoint{3.465072in}{0.413320in}}%
\pgfpathlineto{\pgfqpoint{3.462321in}{0.413320in}}%
\pgfpathlineto{\pgfqpoint{3.459695in}{0.413320in}}%
\pgfpathlineto{\pgfqpoint{3.456960in}{0.413320in}}%
\pgfpathlineto{\pgfqpoint{3.454285in}{0.413320in}}%
\pgfpathlineto{\pgfqpoint{3.451597in}{0.413320in}}%
\pgfpathlineto{\pgfqpoint{3.448926in}{0.413320in}}%
\pgfpathlineto{\pgfqpoint{3.446257in}{0.413320in}}%
\pgfpathlineto{\pgfqpoint{3.443574in}{0.413320in}}%
\pgfpathlineto{\pgfqpoint{3.440996in}{0.413320in}}%
\pgfpathlineto{\pgfqpoint{3.438210in}{0.413320in}}%
\pgfpathlineto{\pgfqpoint{3.435635in}{0.413320in}}%
\pgfpathlineto{\pgfqpoint{3.432851in}{0.413320in}}%
\pgfpathlineto{\pgfqpoint{3.430313in}{0.413320in}}%
\pgfpathlineto{\pgfqpoint{3.427501in}{0.413320in}}%
\pgfpathlineto{\pgfqpoint{3.424887in}{0.413320in}}%
\pgfpathlineto{\pgfqpoint{3.422142in}{0.413320in}}%
\pgfpathlineto{\pgfqpoint{3.419455in}{0.413320in}}%
\pgfpathlineto{\pgfqpoint{3.416780in}{0.413320in}}%
\pgfpathlineto{\pgfqpoint{3.414109in}{0.413320in}}%
\pgfpathlineto{\pgfqpoint{3.411431in}{0.413320in}}%
\pgfpathlineto{\pgfqpoint{3.408752in}{0.413320in}}%
\pgfpathlineto{\pgfqpoint{3.406202in}{0.413320in}}%
\pgfpathlineto{\pgfqpoint{3.403394in}{0.413320in}}%
\pgfpathlineto{\pgfqpoint{3.400783in}{0.413320in}}%
\pgfpathlineto{\pgfqpoint{3.398037in}{0.413320in}}%
\pgfpathlineto{\pgfqpoint{3.395461in}{0.413320in}}%
\pgfpathlineto{\pgfqpoint{3.392681in}{0.413320in}}%
\pgfpathlineto{\pgfqpoint{3.390102in}{0.413320in}}%
\pgfpathlineto{\pgfqpoint{3.387309in}{0.413320in}}%
\pgfpathlineto{\pgfqpoint{3.384647in}{0.413320in}}%
\pgfpathlineto{\pgfqpoint{3.381959in}{0.413320in}}%
\pgfpathlineto{\pgfqpoint{3.379290in}{0.413320in}}%
\pgfpathlineto{\pgfqpoint{3.376735in}{0.413320in}}%
\pgfpathlineto{\pgfqpoint{3.373921in}{0.413320in}}%
\pgfpathlineto{\pgfqpoint{3.371357in}{0.413320in}}%
\pgfpathlineto{\pgfqpoint{3.368577in}{0.413320in}}%
\pgfpathlineto{\pgfqpoint{3.365996in}{0.413320in}}%
\pgfpathlineto{\pgfqpoint{3.363221in}{0.413320in}}%
\pgfpathlineto{\pgfqpoint{3.360620in}{0.413320in}}%
\pgfpathlineto{\pgfqpoint{3.357862in}{0.413320in}}%
\pgfpathlineto{\pgfqpoint{3.355177in}{0.413320in}}%
\pgfpathlineto{\pgfqpoint{3.352505in}{0.413320in}}%
\pgfpathlineto{\pgfqpoint{3.349828in}{0.413320in}}%
\pgfpathlineto{\pgfqpoint{3.347139in}{0.413320in}}%
\pgfpathlineto{\pgfqpoint{3.344468in}{0.413320in}}%
\pgfpathlineto{\pgfqpoint{3.341893in}{0.413320in}}%
\pgfpathlineto{\pgfqpoint{3.339101in}{0.413320in}}%
\pgfpathlineto{\pgfqpoint{3.336541in}{0.413320in}}%
\pgfpathlineto{\pgfqpoint{3.333758in}{0.413320in}}%
\pgfpathlineto{\pgfqpoint{3.331183in}{0.413320in}}%
\pgfpathlineto{\pgfqpoint{3.328401in}{0.413320in}}%
\pgfpathlineto{\pgfqpoint{3.325860in}{0.413320in}}%
\pgfpathlineto{\pgfqpoint{3.323049in}{0.413320in}}%
\pgfpathlineto{\pgfqpoint{3.320366in}{0.413320in}}%
\pgfpathlineto{\pgfqpoint{3.317688in}{0.413320in}}%
\pgfpathlineto{\pgfqpoint{3.315008in}{0.413320in}}%
\pgfpathlineto{\pgfqpoint{3.312480in}{0.413320in}}%
\pgfpathlineto{\pgfqpoint{3.309652in}{0.413320in}}%
\pgfpathlineto{\pgfqpoint{3.307104in}{0.413320in}}%
\pgfpathlineto{\pgfqpoint{3.304295in}{0.413320in}}%
\pgfpathlineto{\pgfqpoint{3.301719in}{0.413320in}}%
\pgfpathlineto{\pgfqpoint{3.298937in}{0.413320in}}%
\pgfpathlineto{\pgfqpoint{3.296376in}{0.413320in}}%
\pgfpathlineto{\pgfqpoint{3.293574in}{0.413320in}}%
\pgfpathlineto{\pgfqpoint{3.290890in}{0.413320in}}%
\pgfpathlineto{\pgfqpoint{3.288225in}{0.413320in}}%
\pgfpathlineto{\pgfqpoint{3.285534in}{0.413320in}}%
\pgfpathlineto{\pgfqpoint{3.282870in}{0.413320in}}%
\pgfpathlineto{\pgfqpoint{3.280189in}{0.413320in}}%
\pgfpathlineto{\pgfqpoint{3.277603in}{0.413320in}}%
\pgfpathlineto{\pgfqpoint{3.274831in}{0.413320in}}%
\pgfpathlineto{\pgfqpoint{3.272254in}{0.413320in}}%
\pgfpathlineto{\pgfqpoint{3.269478in}{0.413320in}}%
\pgfpathlineto{\pgfqpoint{3.266849in}{0.413320in}}%
\pgfpathlineto{\pgfqpoint{3.264119in}{0.413320in}}%
\pgfpathlineto{\pgfqpoint{3.261594in}{0.413320in}}%
\pgfpathlineto{\pgfqpoint{3.258784in}{0.413320in}}%
\pgfpathlineto{\pgfqpoint{3.256083in}{0.413320in}}%
\pgfpathlineto{\pgfqpoint{3.253404in}{0.413320in}}%
\pgfpathlineto{\pgfqpoint{3.250716in}{0.413320in}}%
\pgfpathlineto{\pgfqpoint{3.248049in}{0.413320in}}%
\pgfpathlineto{\pgfqpoint{3.245363in}{0.413320in}}%
\pgfpathlineto{\pgfqpoint{3.242807in}{0.413320in}}%
\pgfpathlineto{\pgfqpoint{3.240010in}{0.413320in}}%
\pgfpathlineto{\pgfqpoint{3.237411in}{0.413320in}}%
\pgfpathlineto{\pgfqpoint{3.234658in}{0.413320in}}%
\pgfpathlineto{\pgfqpoint{3.232069in}{0.413320in}}%
\pgfpathlineto{\pgfqpoint{3.229310in}{0.413320in}}%
\pgfpathlineto{\pgfqpoint{3.226609in}{0.413320in}}%
\pgfpathlineto{\pgfqpoint{3.223942in}{0.413320in}}%
\pgfpathlineto{\pgfqpoint{3.221255in}{0.413320in}}%
\pgfpathlineto{\pgfqpoint{3.218586in}{0.413320in}}%
\pgfpathlineto{\pgfqpoint{3.215908in}{0.413320in}}%
\pgfpathlineto{\pgfqpoint{3.213342in}{0.413320in}}%
\pgfpathlineto{\pgfqpoint{3.210545in}{0.413320in}}%
\pgfpathlineto{\pgfqpoint{3.207984in}{0.413320in}}%
\pgfpathlineto{\pgfqpoint{3.205195in}{0.413320in}}%
\pgfpathlineto{\pgfqpoint{3.202562in}{0.413320in}}%
\pgfpathlineto{\pgfqpoint{3.199823in}{0.413320in}}%
\pgfpathlineto{\pgfqpoint{3.197226in}{0.413320in}}%
\pgfpathlineto{\pgfqpoint{3.194508in}{0.413320in}}%
\pgfpathlineto{\pgfqpoint{3.191796in}{0.413320in}}%
\pgfpathlineto{\pgfqpoint{3.189117in}{0.413320in}}%
\pgfpathlineto{\pgfqpoint{3.186440in}{0.413320in}}%
\pgfpathlineto{\pgfqpoint{3.183760in}{0.413320in}}%
\pgfpathlineto{\pgfqpoint{3.181089in}{0.413320in}}%
\pgfpathlineto{\pgfqpoint{3.178525in}{0.413320in}}%
\pgfpathlineto{\pgfqpoint{3.175724in}{0.413320in}}%
\pgfpathlineto{\pgfqpoint{3.173142in}{0.413320in}}%
\pgfpathlineto{\pgfqpoint{3.170375in}{0.413320in}}%
\pgfpathlineto{\pgfqpoint{3.167776in}{0.413320in}}%
\pgfpathlineto{\pgfqpoint{3.165019in}{0.413320in}}%
\pgfpathlineto{\pgfqpoint{3.162474in}{0.413320in}}%
\pgfpathlineto{\pgfqpoint{3.159675in}{0.413320in}}%
\pgfpathlineto{\pgfqpoint{3.156981in}{0.413320in}}%
\pgfpathlineto{\pgfqpoint{3.154327in}{0.413320in}}%
\pgfpathlineto{\pgfqpoint{3.151612in}{0.413320in}}%
\pgfpathlineto{\pgfqpoint{3.149057in}{0.413320in}}%
\pgfpathlineto{\pgfqpoint{3.146271in}{0.413320in}}%
\pgfpathlineto{\pgfqpoint{3.143740in}{0.413320in}}%
\pgfpathlineto{\pgfqpoint{3.140913in}{0.413320in}}%
\pgfpathlineto{\pgfqpoint{3.138375in}{0.413320in}}%
\pgfpathlineto{\pgfqpoint{3.135550in}{0.413320in}}%
\pgfpathlineto{\pgfqpoint{3.132946in}{0.413320in}}%
\pgfpathlineto{\pgfqpoint{3.130199in}{0.413320in}}%
\pgfpathlineto{\pgfqpoint{3.127512in}{0.413320in}}%
\pgfpathlineto{\pgfqpoint{3.124842in}{0.413320in}}%
\pgfpathlineto{\pgfqpoint{3.122163in}{0.413320in}}%
\pgfpathlineto{\pgfqpoint{3.119487in}{0.413320in}}%
\pgfpathlineto{\pgfqpoint{3.116807in}{0.413320in}}%
\pgfpathlineto{\pgfqpoint{3.114242in}{0.413320in}}%
\pgfpathlineto{\pgfqpoint{3.111451in}{0.413320in}}%
\pgfpathlineto{\pgfqpoint{3.108896in}{0.413320in}}%
\pgfpathlineto{\pgfqpoint{3.106094in}{0.413320in}}%
\pgfpathlineto{\pgfqpoint{3.103508in}{0.413320in}}%
\pgfpathlineto{\pgfqpoint{3.100737in}{0.413320in}}%
\pgfpathlineto{\pgfqpoint{3.098163in}{0.413320in}}%
\pgfpathlineto{\pgfqpoint{3.095388in}{0.413320in}}%
\pgfpathlineto{\pgfqpoint{3.092699in}{0.413320in}}%
\pgfpathlineto{\pgfqpoint{3.090023in}{0.413320in}}%
\pgfpathlineto{\pgfqpoint{3.087343in}{0.413320in}}%
\pgfpathlineto{\pgfqpoint{3.084671in}{0.413320in}}%
\pgfpathlineto{\pgfqpoint{3.081990in}{0.413320in}}%
\pgfpathlineto{\pgfqpoint{3.079381in}{0.413320in}}%
\pgfpathlineto{\pgfqpoint{3.076631in}{0.413320in}}%
\pgfpathlineto{\pgfqpoint{3.074056in}{0.413320in}}%
\pgfpathlineto{\pgfqpoint{3.071266in}{0.413320in}}%
\pgfpathlineto{\pgfqpoint{3.068709in}{0.413320in}}%
\pgfpathlineto{\pgfqpoint{3.065916in}{0.413320in}}%
\pgfpathlineto{\pgfqpoint{3.063230in}{0.413320in}}%
\pgfpathlineto{\pgfqpoint{3.060561in}{0.413320in}}%
\pgfpathlineto{\pgfqpoint{3.057884in}{0.413320in}}%
\pgfpathlineto{\pgfqpoint{3.055202in}{0.413320in}}%
\pgfpathlineto{\pgfqpoint{3.052526in}{0.413320in}}%
\pgfpathlineto{\pgfqpoint{3.049988in}{0.413320in}}%
\pgfpathlineto{\pgfqpoint{3.047157in}{0.413320in}}%
\pgfpathlineto{\pgfqpoint{3.044568in}{0.413320in}}%
\pgfpathlineto{\pgfqpoint{3.041813in}{0.413320in}}%
\pgfpathlineto{\pgfqpoint{3.039262in}{0.413320in}}%
\pgfpathlineto{\pgfqpoint{3.036456in}{0.413320in}}%
\pgfpathlineto{\pgfqpoint{3.033921in}{0.413320in}}%
\pgfpathlineto{\pgfqpoint{3.031091in}{0.413320in}}%
\pgfpathlineto{\pgfqpoint{3.028412in}{0.413320in}}%
\pgfpathlineto{\pgfqpoint{3.025803in}{0.413320in}}%
\pgfpathlineto{\pgfqpoint{3.023058in}{0.413320in}}%
\pgfpathlineto{\pgfqpoint{3.020382in}{0.413320in}}%
\pgfpathlineto{\pgfqpoint{3.017707in}{0.413320in}}%
\pgfpathlineto{\pgfqpoint{3.015097in}{0.413320in}}%
\pgfpathlineto{\pgfqpoint{3.012351in}{0.413320in}}%
\pgfpathlineto{\pgfqpoint{3.009784in}{0.413320in}}%
\pgfpathlineto{\pgfqpoint{3.006993in}{0.413320in}}%
\pgfpathlineto{\pgfqpoint{3.004419in}{0.413320in}}%
\pgfpathlineto{\pgfqpoint{3.001635in}{0.413320in}}%
\pgfpathlineto{\pgfqpoint{2.999103in}{0.413320in}}%
\pgfpathlineto{\pgfqpoint{2.996300in}{0.413320in}}%
\pgfpathlineto{\pgfqpoint{2.993595in}{0.413320in}}%
\pgfpathlineto{\pgfqpoint{2.990978in}{0.413320in}}%
\pgfpathlineto{\pgfqpoint{2.988238in}{0.413320in}}%
\pgfpathlineto{\pgfqpoint{2.985666in}{0.413320in}}%
\pgfpathlineto{\pgfqpoint{2.982885in}{0.413320in}}%
\pgfpathlineto{\pgfqpoint{2.980341in}{0.413320in}}%
\pgfpathlineto{\pgfqpoint{2.977517in}{0.413320in}}%
\pgfpathlineto{\pgfqpoint{2.974972in}{0.413320in}}%
\pgfpathlineto{\pgfqpoint{2.972177in}{0.413320in}}%
\pgfpathlineto{\pgfqpoint{2.969599in}{0.413320in}}%
\pgfpathlineto{\pgfqpoint{2.966812in}{0.413320in}}%
\pgfpathlineto{\pgfqpoint{2.964127in}{0.413320in}}%
\pgfpathlineto{\pgfqpoint{2.961460in}{0.413320in}}%
\pgfpathlineto{\pgfqpoint{2.958782in}{0.413320in}}%
\pgfpathlineto{\pgfqpoint{2.956103in}{0.413320in}}%
\pgfpathlineto{\pgfqpoint{2.953422in}{0.413320in}}%
\pgfpathlineto{\pgfqpoint{2.950884in}{0.413320in}}%
\pgfpathlineto{\pgfqpoint{2.948068in}{0.413320in}}%
\pgfpathlineto{\pgfqpoint{2.945461in}{0.413320in}}%
\pgfpathlineto{\pgfqpoint{2.942711in}{0.413320in}}%
\pgfpathlineto{\pgfqpoint{2.940120in}{0.413320in}}%
\pgfpathlineto{\pgfqpoint{2.937352in}{0.413320in}}%
\pgfpathlineto{\pgfqpoint{2.934759in}{0.413320in}}%
\pgfpathlineto{\pgfqpoint{2.932033in}{0.413320in}}%
\pgfpathlineto{\pgfqpoint{2.929321in}{0.413320in}}%
\pgfpathlineto{\pgfqpoint{2.926655in}{0.413320in}}%
\pgfpathlineto{\pgfqpoint{2.923963in}{0.413320in}}%
\pgfpathlineto{\pgfqpoint{2.921363in}{0.413320in}}%
\pgfpathlineto{\pgfqpoint{2.918606in}{0.413320in}}%
\pgfpathlineto{\pgfqpoint{2.916061in}{0.413320in}}%
\pgfpathlineto{\pgfqpoint{2.913243in}{0.413320in}}%
\pgfpathlineto{\pgfqpoint{2.910631in}{0.413320in}}%
\pgfpathlineto{\pgfqpoint{2.907882in}{0.413320in}}%
\pgfpathlineto{\pgfqpoint{2.905341in}{0.413320in}}%
\pgfpathlineto{\pgfqpoint{2.902535in}{0.413320in}}%
\pgfpathlineto{\pgfqpoint{2.899858in}{0.413320in}}%
\pgfpathlineto{\pgfqpoint{2.897179in}{0.413320in}}%
\pgfpathlineto{\pgfqpoint{2.894487in}{0.413320in}}%
\pgfpathlineto{\pgfqpoint{2.891809in}{0.413320in}}%
\pgfpathlineto{\pgfqpoint{2.889145in}{0.413320in}}%
\pgfpathlineto{\pgfqpoint{2.886578in}{0.413320in}}%
\pgfpathlineto{\pgfqpoint{2.883780in}{0.413320in}}%
\pgfpathlineto{\pgfqpoint{2.881254in}{0.413320in}}%
\pgfpathlineto{\pgfqpoint{2.878431in}{0.413320in}}%
\pgfpathlineto{\pgfqpoint{2.875882in}{0.413320in}}%
\pgfpathlineto{\pgfqpoint{2.873074in}{0.413320in}}%
\pgfpathlineto{\pgfqpoint{2.870475in}{0.413320in}}%
\pgfpathlineto{\pgfqpoint{2.867713in}{0.413320in}}%
\pgfpathlineto{\pgfqpoint{2.865031in}{0.413320in}}%
\pgfpathlineto{\pgfqpoint{2.862402in}{0.413320in}}%
\pgfpathlineto{\pgfqpoint{2.859668in}{0.413320in}}%
\pgfpathlineto{\pgfqpoint{2.857003in}{0.413320in}}%
\pgfpathlineto{\pgfqpoint{2.854325in}{0.413320in}}%
\pgfpathlineto{\pgfqpoint{2.851793in}{0.413320in}}%
\pgfpathlineto{\pgfqpoint{2.848960in}{0.413320in}}%
\pgfpathlineto{\pgfqpoint{2.846408in}{0.413320in}}%
\pgfpathlineto{\pgfqpoint{2.843611in}{0.413320in}}%
\pgfpathlineto{\pgfqpoint{2.841055in}{0.413320in}}%
\pgfpathlineto{\pgfqpoint{2.838254in}{0.413320in}}%
\pgfpathlineto{\pgfqpoint{2.835698in}{0.413320in}}%
\pgfpathlineto{\pgfqpoint{2.832894in}{0.413320in}}%
\pgfpathlineto{\pgfqpoint{2.830219in}{0.413320in}}%
\pgfpathlineto{\pgfqpoint{2.827567in}{0.413320in}}%
\pgfpathlineto{\pgfqpoint{2.824851in}{0.413320in}}%
\pgfpathlineto{\pgfqpoint{2.822303in}{0.413320in}}%
\pgfpathlineto{\pgfqpoint{2.819506in}{0.413320in}}%
\pgfpathlineto{\pgfqpoint{2.816867in}{0.413320in}}%
\pgfpathlineto{\pgfqpoint{2.814141in}{0.413320in}}%
\pgfpathlineto{\pgfqpoint{2.811597in}{0.413320in}}%
\pgfpathlineto{\pgfqpoint{2.808792in}{0.413320in}}%
\pgfpathlineto{\pgfqpoint{2.806175in}{0.413320in}}%
\pgfpathlineto{\pgfqpoint{2.803435in}{0.413320in}}%
\pgfpathlineto{\pgfqpoint{2.800756in}{0.413320in}}%
\pgfpathlineto{\pgfqpoint{2.798070in}{0.413320in}}%
\pgfpathlineto{\pgfqpoint{2.795398in}{0.413320in}}%
\pgfpathlineto{\pgfqpoint{2.792721in}{0.413320in}}%
\pgfpathlineto{\pgfqpoint{2.790044in}{0.413320in}}%
\pgfpathlineto{\pgfqpoint{2.787468in}{0.413320in}}%
\pgfpathlineto{\pgfqpoint{2.784687in}{0.413320in}}%
\pgfpathlineto{\pgfqpoint{2.782113in}{0.413320in}}%
\pgfpathlineto{\pgfqpoint{2.779330in}{0.413320in}}%
\pgfpathlineto{\pgfqpoint{2.776767in}{0.413320in}}%
\pgfpathlineto{\pgfqpoint{2.773972in}{0.413320in}}%
\pgfpathlineto{\pgfqpoint{2.771373in}{0.413320in}}%
\pgfpathlineto{\pgfqpoint{2.768617in}{0.413320in}}%
\pgfpathlineto{\pgfqpoint{2.765935in}{0.413320in}}%
\pgfpathlineto{\pgfqpoint{2.763253in}{0.413320in}}%
\pgfpathlineto{\pgfqpoint{2.760581in}{0.413320in}}%
\pgfpathlineto{\pgfqpoint{2.758028in}{0.413320in}}%
\pgfpathlineto{\pgfqpoint{2.755224in}{0.413320in}}%
\pgfpathlineto{\pgfqpoint{2.752614in}{0.413320in}}%
\pgfpathlineto{\pgfqpoint{2.749868in}{0.413320in}}%
\pgfpathlineto{\pgfqpoint{2.747260in}{0.413320in}}%
\pgfpathlineto{\pgfqpoint{2.744510in}{0.413320in}}%
\pgfpathlineto{\pgfqpoint{2.741928in}{0.413320in}}%
\pgfpathlineto{\pgfqpoint{2.739155in}{0.413320in}}%
\pgfpathlineto{\pgfqpoint{2.736476in}{0.413320in}}%
\pgfpathlineto{\pgfqpoint{2.733798in}{0.413320in}}%
\pgfpathlineto{\pgfqpoint{2.731119in}{0.413320in}}%
\pgfpathlineto{\pgfqpoint{2.728439in}{0.413320in}}%
\pgfpathlineto{\pgfqpoint{2.725760in}{0.413320in}}%
\pgfpathlineto{\pgfqpoint{2.723211in}{0.413320in}}%
\pgfpathlineto{\pgfqpoint{2.720404in}{0.413320in}}%
\pgfpathlineto{\pgfqpoint{2.717773in}{0.413320in}}%
\pgfpathlineto{\pgfqpoint{2.715036in}{0.413320in}}%
\pgfpathlineto{\pgfqpoint{2.712477in}{0.413320in}}%
\pgfpathlineto{\pgfqpoint{2.709683in}{0.413320in}}%
\pgfpathlineto{\pgfqpoint{2.707125in}{0.413320in}}%
\pgfpathlineto{\pgfqpoint{2.704326in}{0.413320in}}%
\pgfpathlineto{\pgfqpoint{2.701657in}{0.413320in}}%
\pgfpathlineto{\pgfqpoint{2.698968in}{0.413320in}}%
\pgfpathlineto{\pgfqpoint{2.696293in}{0.413320in}}%
\pgfpathlineto{\pgfqpoint{2.693611in}{0.413320in}}%
\pgfpathlineto{\pgfqpoint{2.690940in}{0.413320in}}%
\pgfpathlineto{\pgfqpoint{2.688328in}{0.413320in}}%
\pgfpathlineto{\pgfqpoint{2.685586in}{0.413320in}}%
\pgfpathlineto{\pgfqpoint{2.683009in}{0.413320in}}%
\pgfpathlineto{\pgfqpoint{2.680224in}{0.413320in}}%
\pgfpathlineto{\pgfqpoint{2.677650in}{0.413320in}}%
\pgfpathlineto{\pgfqpoint{2.674873in}{0.413320in}}%
\pgfpathlineto{\pgfqpoint{2.672301in}{0.413320in}}%
\pgfpathlineto{\pgfqpoint{2.669506in}{0.413320in}}%
\pgfpathlineto{\pgfqpoint{2.666836in}{0.413320in}}%
\pgfpathlineto{\pgfqpoint{2.664151in}{0.413320in}}%
\pgfpathlineto{\pgfqpoint{2.661481in}{0.413320in}}%
\pgfpathlineto{\pgfqpoint{2.658942in}{0.413320in}}%
\pgfpathlineto{\pgfqpoint{2.656124in}{0.413320in}}%
\pgfpathlineto{\pgfqpoint{2.653567in}{0.413320in}}%
\pgfpathlineto{\pgfqpoint{2.650767in}{0.413320in}}%
\pgfpathlineto{\pgfqpoint{2.648196in}{0.413320in}}%
\pgfpathlineto{\pgfqpoint{2.645408in}{0.413320in}}%
\pgfpathlineto{\pgfqpoint{2.642827in}{0.413320in}}%
\pgfpathlineto{\pgfqpoint{2.640053in}{0.413320in}}%
\pgfpathlineto{\pgfqpoint{2.637369in}{0.413320in}}%
\pgfpathlineto{\pgfqpoint{2.634700in}{0.413320in}}%
\pgfpathlineto{\pgfqpoint{2.632018in}{0.413320in}}%
\pgfpathlineto{\pgfqpoint{2.629340in}{0.413320in}}%
\pgfpathlineto{\pgfqpoint{2.626653in}{0.413320in}}%
\pgfpathlineto{\pgfqpoint{2.624077in}{0.413320in}}%
\pgfpathlineto{\pgfqpoint{2.621304in}{0.413320in}}%
\pgfpathlineto{\pgfqpoint{2.618773in}{0.413320in}}%
\pgfpathlineto{\pgfqpoint{2.615934in}{0.413320in}}%
\pgfpathlineto{\pgfqpoint{2.613393in}{0.413320in}}%
\pgfpathlineto{\pgfqpoint{2.610588in}{0.413320in}}%
\pgfpathlineto{\pgfqpoint{2.608004in}{0.413320in}}%
\pgfpathlineto{\pgfqpoint{2.605232in}{0.413320in}}%
\pgfpathlineto{\pgfqpoint{2.602557in}{0.413320in}}%
\pgfpathlineto{\pgfqpoint{2.599920in}{0.413320in}}%
\pgfpathlineto{\pgfqpoint{2.597196in}{0.413320in}}%
\pgfpathlineto{\pgfqpoint{2.594630in}{0.413320in}}%
\pgfpathlineto{\pgfqpoint{2.591842in}{0.413320in}}%
\pgfpathlineto{\pgfqpoint{2.589248in}{0.413320in}}%
\pgfpathlineto{\pgfqpoint{2.586484in}{0.413320in}}%
\pgfpathlineto{\pgfqpoint{2.583913in}{0.413320in}}%
\pgfpathlineto{\pgfqpoint{2.581129in}{0.413320in}}%
\pgfpathlineto{\pgfqpoint{2.578567in}{0.413320in}}%
\pgfpathlineto{\pgfqpoint{2.575779in}{0.413320in}}%
\pgfpathlineto{\pgfqpoint{2.573082in}{0.413320in}}%
\pgfpathlineto{\pgfqpoint{2.570411in}{0.413320in}}%
\pgfpathlineto{\pgfqpoint{2.567730in}{0.413320in}}%
\pgfpathlineto{\pgfqpoint{2.565045in}{0.413320in}}%
\pgfpathlineto{\pgfqpoint{2.562375in}{0.413320in}}%
\pgfpathlineto{\pgfqpoint{2.559790in}{0.413320in}}%
\pgfpathlineto{\pgfqpoint{2.557009in}{0.413320in}}%
\pgfpathlineto{\pgfqpoint{2.554493in}{0.413320in}}%
\pgfpathlineto{\pgfqpoint{2.551664in}{0.413320in}}%
\pgfpathlineto{\pgfqpoint{2.549114in}{0.413320in}}%
\pgfpathlineto{\pgfqpoint{2.546310in}{0.413320in}}%
\pgfpathlineto{\pgfqpoint{2.543765in}{0.413320in}}%
\pgfpathlineto{\pgfqpoint{2.540949in}{0.413320in}}%
\pgfpathlineto{\pgfqpoint{2.538274in}{0.413320in}}%
\pgfpathlineto{\pgfqpoint{2.535624in}{0.413320in}}%
\pgfpathlineto{\pgfqpoint{2.532917in}{0.413320in}}%
\pgfpathlineto{\pgfqpoint{2.530234in}{0.413320in}}%
\pgfpathlineto{\pgfqpoint{2.527560in}{0.413320in}}%
\pgfpathlineto{\pgfqpoint{2.524988in}{0.413320in}}%
\pgfpathlineto{\pgfqpoint{2.522197in}{0.413320in}}%
\pgfpathlineto{\pgfqpoint{2.519607in}{0.413320in}}%
\pgfpathlineto{\pgfqpoint{2.516845in}{0.413320in}}%
\pgfpathlineto{\pgfqpoint{2.514268in}{0.413320in}}%
\pgfpathlineto{\pgfqpoint{2.511478in}{0.413320in}}%
\pgfpathlineto{\pgfqpoint{2.508917in}{0.413320in}}%
\pgfpathlineto{\pgfqpoint{2.506163in}{0.413320in}}%
\pgfpathlineto{\pgfqpoint{2.503454in}{0.413320in}}%
\pgfpathlineto{\pgfqpoint{2.500801in}{0.413320in}}%
\pgfpathlineto{\pgfqpoint{2.498085in}{0.413320in}}%
\pgfpathlineto{\pgfqpoint{2.495542in}{0.413320in}}%
\pgfpathlineto{\pgfqpoint{2.492729in}{0.413320in}}%
\pgfpathlineto{\pgfqpoint{2.490183in}{0.413320in}}%
\pgfpathlineto{\pgfqpoint{2.487384in}{0.413320in}}%
\pgfpathlineto{\pgfqpoint{2.484870in}{0.413320in}}%
\pgfpathlineto{\pgfqpoint{2.482026in}{0.413320in}}%
\pgfpathlineto{\pgfqpoint{2.479420in}{0.413320in}}%
\pgfpathlineto{\pgfqpoint{2.476671in}{0.413320in}}%
\pgfpathlineto{\pgfqpoint{2.473989in}{0.413320in}}%
\pgfpathlineto{\pgfqpoint{2.471311in}{0.413320in}}%
\pgfpathlineto{\pgfqpoint{2.468635in}{0.413320in}}%
\pgfpathlineto{\pgfqpoint{2.465957in}{0.413320in}}%
\pgfpathlineto{\pgfqpoint{2.463280in}{0.413320in}}%
\pgfpathlineto{\pgfqpoint{2.460711in}{0.413320in}}%
\pgfpathlineto{\pgfqpoint{2.457917in}{0.413320in}}%
\pgfpathlineto{\pgfqpoint{2.455353in}{0.413320in}}%
\pgfpathlineto{\pgfqpoint{2.452562in}{0.413320in}}%
\pgfpathlineto{\pgfqpoint{2.450032in}{0.413320in}}%
\pgfpathlineto{\pgfqpoint{2.447209in}{0.413320in}}%
\pgfpathlineto{\pgfqpoint{2.444677in}{0.413320in}}%
\pgfpathlineto{\pgfqpoint{2.441876in}{0.413320in}}%
\pgfpathlineto{\pgfqpoint{2.439167in}{0.413320in}}%
\pgfpathlineto{\pgfqpoint{2.436518in}{0.413320in}}%
\pgfpathlineto{\pgfqpoint{2.433815in}{0.413320in}}%
\pgfpathlineto{\pgfqpoint{2.431251in}{0.413320in}}%
\pgfpathlineto{\pgfqpoint{2.428453in}{0.413320in}}%
\pgfpathlineto{\pgfqpoint{2.425878in}{0.413320in}}%
\pgfpathlineto{\pgfqpoint{2.423098in}{0.413320in}}%
\pgfpathlineto{\pgfqpoint{2.420528in}{0.413320in}}%
\pgfpathlineto{\pgfqpoint{2.417747in}{0.413320in}}%
\pgfpathlineto{\pgfqpoint{2.415184in}{0.413320in}}%
\pgfpathlineto{\pgfqpoint{2.412389in}{0.413320in}}%
\pgfpathlineto{\pgfqpoint{2.409699in}{0.413320in}}%
\pgfpathlineto{\pgfqpoint{2.407024in}{0.413320in}}%
\pgfpathlineto{\pgfqpoint{2.404352in}{0.413320in}}%
\pgfpathlineto{\pgfqpoint{2.401675in}{0.413320in}}%
\pgfpathlineto{\pgfqpoint{2.398995in}{0.413320in}}%
\pgfpathclose%
\pgfusepath{stroke,fill}%
\end{pgfscope}%
\begin{pgfscope}%
\pgfpathrectangle{\pgfqpoint{2.398995in}{0.319877in}}{\pgfqpoint{3.986877in}{1.993438in}} %
\pgfusepath{clip}%
\pgfsetbuttcap%
\pgfsetroundjoin%
\definecolor{currentfill}{rgb}{1.000000,1.000000,1.000000}%
\pgfsetfillcolor{currentfill}%
\pgfsetlinewidth{1.003750pt}%
\definecolor{currentstroke}{rgb}{0.775732,0.578493,0.194756}%
\pgfsetstrokecolor{currentstroke}%
\pgfsetdash{}{0pt}%
\pgfpathmoveto{\pgfqpoint{2.398995in}{0.413320in}}%
\pgfpathlineto{\pgfqpoint{2.398995in}{0.607013in}}%
\pgfpathlineto{\pgfqpoint{2.401675in}{0.613251in}}%
\pgfpathlineto{\pgfqpoint{2.404352in}{0.609522in}}%
\pgfpathlineto{\pgfqpoint{2.407024in}{0.614397in}}%
\pgfpathlineto{\pgfqpoint{2.409699in}{0.620118in}}%
\pgfpathlineto{\pgfqpoint{2.412389in}{0.613004in}}%
\pgfpathlineto{\pgfqpoint{2.415184in}{0.610922in}}%
\pgfpathlineto{\pgfqpoint{2.417747in}{0.607678in}}%
\pgfpathlineto{\pgfqpoint{2.420528in}{0.603293in}}%
\pgfpathlineto{\pgfqpoint{2.423098in}{0.601929in}}%
\pgfpathlineto{\pgfqpoint{2.425878in}{0.604708in}}%
\pgfpathlineto{\pgfqpoint{2.428453in}{0.608386in}}%
\pgfpathlineto{\pgfqpoint{2.431251in}{0.623096in}}%
\pgfpathlineto{\pgfqpoint{2.433815in}{0.627319in}}%
\pgfpathlineto{\pgfqpoint{2.436518in}{0.651350in}}%
\pgfpathlineto{\pgfqpoint{2.439167in}{0.632455in}}%
\pgfpathlineto{\pgfqpoint{2.441876in}{0.630962in}}%
\pgfpathlineto{\pgfqpoint{2.444677in}{0.623113in}}%
\pgfpathlineto{\pgfqpoint{2.447209in}{0.616440in}}%
\pgfpathlineto{\pgfqpoint{2.450032in}{0.617156in}}%
\pgfpathlineto{\pgfqpoint{2.452562in}{0.616750in}}%
\pgfpathlineto{\pgfqpoint{2.455353in}{0.613640in}}%
\pgfpathlineto{\pgfqpoint{2.457917in}{0.614549in}}%
\pgfpathlineto{\pgfqpoint{2.460711in}{0.612388in}}%
\pgfpathlineto{\pgfqpoint{2.463280in}{0.611629in}}%
\pgfpathlineto{\pgfqpoint{2.465957in}{0.611498in}}%
\pgfpathlineto{\pgfqpoint{2.468635in}{0.607022in}}%
\pgfpathlineto{\pgfqpoint{2.471311in}{0.607652in}}%
\pgfpathlineto{\pgfqpoint{2.473989in}{0.608093in}}%
\pgfpathlineto{\pgfqpoint{2.476671in}{0.603797in}}%
\pgfpathlineto{\pgfqpoint{2.479420in}{0.607767in}}%
\pgfpathlineto{\pgfqpoint{2.482026in}{0.611926in}}%
\pgfpathlineto{\pgfqpoint{2.484870in}{0.618176in}}%
\pgfpathlineto{\pgfqpoint{2.487384in}{0.616064in}}%
\pgfpathlineto{\pgfqpoint{2.490183in}{0.610673in}}%
\pgfpathlineto{\pgfqpoint{2.492729in}{0.608717in}}%
\pgfpathlineto{\pgfqpoint{2.495542in}{0.607968in}}%
\pgfpathlineto{\pgfqpoint{2.498085in}{0.606162in}}%
\pgfpathlineto{\pgfqpoint{2.500801in}{0.608464in}}%
\pgfpathlineto{\pgfqpoint{2.503454in}{0.606012in}}%
\pgfpathlineto{\pgfqpoint{2.506163in}{0.606409in}}%
\pgfpathlineto{\pgfqpoint{2.508917in}{0.608127in}}%
\pgfpathlineto{\pgfqpoint{2.511478in}{0.611816in}}%
\pgfpathlineto{\pgfqpoint{2.514268in}{0.607413in}}%
\pgfpathlineto{\pgfqpoint{2.516845in}{0.605264in}}%
\pgfpathlineto{\pgfqpoint{2.519607in}{0.606192in}}%
\pgfpathlineto{\pgfqpoint{2.522197in}{0.602416in}}%
\pgfpathlineto{\pgfqpoint{2.524988in}{0.605300in}}%
\pgfpathlineto{\pgfqpoint{2.527560in}{0.603610in}}%
\pgfpathlineto{\pgfqpoint{2.530234in}{0.607553in}}%
\pgfpathlineto{\pgfqpoint{2.532917in}{0.603972in}}%
\pgfpathlineto{\pgfqpoint{2.535624in}{0.608914in}}%
\pgfpathlineto{\pgfqpoint{2.538274in}{0.606996in}}%
\pgfpathlineto{\pgfqpoint{2.540949in}{0.603117in}}%
\pgfpathlineto{\pgfqpoint{2.543765in}{0.602836in}}%
\pgfpathlineto{\pgfqpoint{2.546310in}{0.606455in}}%
\pgfpathlineto{\pgfqpoint{2.549114in}{0.607919in}}%
\pgfpathlineto{\pgfqpoint{2.551664in}{0.613036in}}%
\pgfpathlineto{\pgfqpoint{2.554493in}{0.618845in}}%
\pgfpathlineto{\pgfqpoint{2.557009in}{0.620484in}}%
\pgfpathlineto{\pgfqpoint{2.559790in}{0.614047in}}%
\pgfpathlineto{\pgfqpoint{2.562375in}{0.622465in}}%
\pgfpathlineto{\pgfqpoint{2.565045in}{0.615380in}}%
\pgfpathlineto{\pgfqpoint{2.567730in}{0.617156in}}%
\pgfpathlineto{\pgfqpoint{2.570411in}{0.615781in}}%
\pgfpathlineto{\pgfqpoint{2.573082in}{0.668387in}}%
\pgfpathlineto{\pgfqpoint{2.575779in}{0.720810in}}%
\pgfpathlineto{\pgfqpoint{2.578567in}{0.701642in}}%
\pgfpathlineto{\pgfqpoint{2.581129in}{0.688566in}}%
\pgfpathlineto{\pgfqpoint{2.583913in}{0.674191in}}%
\pgfpathlineto{\pgfqpoint{2.586484in}{0.661857in}}%
\pgfpathlineto{\pgfqpoint{2.589248in}{0.654877in}}%
\pgfpathlineto{\pgfqpoint{2.591842in}{0.650297in}}%
\pgfpathlineto{\pgfqpoint{2.594630in}{0.646885in}}%
\pgfpathlineto{\pgfqpoint{2.597196in}{0.635723in}}%
\pgfpathlineto{\pgfqpoint{2.599920in}{0.631241in}}%
\pgfpathlineto{\pgfqpoint{2.602557in}{0.623067in}}%
\pgfpathlineto{\pgfqpoint{2.605232in}{0.622245in}}%
\pgfpathlineto{\pgfqpoint{2.608004in}{0.617486in}}%
\pgfpathlineto{\pgfqpoint{2.610588in}{0.616712in}}%
\pgfpathlineto{\pgfqpoint{2.613393in}{0.614563in}}%
\pgfpathlineto{\pgfqpoint{2.615934in}{0.615197in}}%
\pgfpathlineto{\pgfqpoint{2.618773in}{0.609002in}}%
\pgfpathlineto{\pgfqpoint{2.621304in}{0.607388in}}%
\pgfpathlineto{\pgfqpoint{2.624077in}{0.605503in}}%
\pgfpathlineto{\pgfqpoint{2.626653in}{0.611355in}}%
\pgfpathlineto{\pgfqpoint{2.629340in}{0.609569in}}%
\pgfpathlineto{\pgfqpoint{2.632018in}{0.609891in}}%
\pgfpathlineto{\pgfqpoint{2.634700in}{0.606185in}}%
\pgfpathlineto{\pgfqpoint{2.637369in}{0.603472in}}%
\pgfpathlineto{\pgfqpoint{2.640053in}{0.604113in}}%
\pgfpathlineto{\pgfqpoint{2.642827in}{0.604921in}}%
\pgfpathlineto{\pgfqpoint{2.645408in}{0.604238in}}%
\pgfpathlineto{\pgfqpoint{2.648196in}{0.601949in}}%
\pgfpathlineto{\pgfqpoint{2.650767in}{0.597063in}}%
\pgfpathlineto{\pgfqpoint{2.653567in}{0.595041in}}%
\pgfpathlineto{\pgfqpoint{2.656124in}{0.596788in}}%
\pgfpathlineto{\pgfqpoint{2.658942in}{0.598308in}}%
\pgfpathlineto{\pgfqpoint{2.661481in}{0.597206in}}%
\pgfpathlineto{\pgfqpoint{2.664151in}{0.596700in}}%
\pgfpathlineto{\pgfqpoint{2.666836in}{0.595041in}}%
\pgfpathlineto{\pgfqpoint{2.669506in}{0.599836in}}%
\pgfpathlineto{\pgfqpoint{2.672301in}{0.600358in}}%
\pgfpathlineto{\pgfqpoint{2.674873in}{0.601812in}}%
\pgfpathlineto{\pgfqpoint{2.677650in}{0.599209in}}%
\pgfpathlineto{\pgfqpoint{2.680224in}{0.607068in}}%
\pgfpathlineto{\pgfqpoint{2.683009in}{0.602462in}}%
\pgfpathlineto{\pgfqpoint{2.685586in}{0.603719in}}%
\pgfpathlineto{\pgfqpoint{2.688328in}{0.604754in}}%
\pgfpathlineto{\pgfqpoint{2.690940in}{0.605240in}}%
\pgfpathlineto{\pgfqpoint{2.693611in}{0.605538in}}%
\pgfpathlineto{\pgfqpoint{2.696293in}{0.607418in}}%
\pgfpathlineto{\pgfqpoint{2.698968in}{0.606289in}}%
\pgfpathlineto{\pgfqpoint{2.701657in}{0.606339in}}%
\pgfpathlineto{\pgfqpoint{2.704326in}{0.606630in}}%
\pgfpathlineto{\pgfqpoint{2.707125in}{0.605188in}}%
\pgfpathlineto{\pgfqpoint{2.709683in}{0.607945in}}%
\pgfpathlineto{\pgfqpoint{2.712477in}{0.602644in}}%
\pgfpathlineto{\pgfqpoint{2.715036in}{0.604105in}}%
\pgfpathlineto{\pgfqpoint{2.717773in}{0.608827in}}%
\pgfpathlineto{\pgfqpoint{2.720404in}{0.606446in}}%
\pgfpathlineto{\pgfqpoint{2.723211in}{0.602008in}}%
\pgfpathlineto{\pgfqpoint{2.725760in}{0.603918in}}%
\pgfpathlineto{\pgfqpoint{2.728439in}{0.609749in}}%
\pgfpathlineto{\pgfqpoint{2.731119in}{0.622574in}}%
\pgfpathlineto{\pgfqpoint{2.733798in}{0.620816in}}%
\pgfpathlineto{\pgfqpoint{2.736476in}{0.614636in}}%
\pgfpathlineto{\pgfqpoint{2.739155in}{0.608324in}}%
\pgfpathlineto{\pgfqpoint{2.741928in}{0.603383in}}%
\pgfpathlineto{\pgfqpoint{2.744510in}{0.606582in}}%
\pgfpathlineto{\pgfqpoint{2.747260in}{0.601975in}}%
\pgfpathlineto{\pgfqpoint{2.749868in}{0.603170in}}%
\pgfpathlineto{\pgfqpoint{2.752614in}{0.602734in}}%
\pgfpathlineto{\pgfqpoint{2.755224in}{0.604703in}}%
\pgfpathlineto{\pgfqpoint{2.758028in}{0.602628in}}%
\pgfpathlineto{\pgfqpoint{2.760581in}{0.605243in}}%
\pgfpathlineto{\pgfqpoint{2.763253in}{0.601257in}}%
\pgfpathlineto{\pgfqpoint{2.765935in}{0.604068in}}%
\pgfpathlineto{\pgfqpoint{2.768617in}{0.605929in}}%
\pgfpathlineto{\pgfqpoint{2.771373in}{0.606443in}}%
\pgfpathlineto{\pgfqpoint{2.773972in}{0.610057in}}%
\pgfpathlineto{\pgfqpoint{2.776767in}{0.606495in}}%
\pgfpathlineto{\pgfqpoint{2.779330in}{0.608632in}}%
\pgfpathlineto{\pgfqpoint{2.782113in}{0.610134in}}%
\pgfpathlineto{\pgfqpoint{2.784687in}{0.610776in}}%
\pgfpathlineto{\pgfqpoint{2.787468in}{0.605432in}}%
\pgfpathlineto{\pgfqpoint{2.790044in}{0.609171in}}%
\pgfpathlineto{\pgfqpoint{2.792721in}{0.607067in}}%
\pgfpathlineto{\pgfqpoint{2.795398in}{0.609099in}}%
\pgfpathlineto{\pgfqpoint{2.798070in}{0.609375in}}%
\pgfpathlineto{\pgfqpoint{2.800756in}{0.608735in}}%
\pgfpathlineto{\pgfqpoint{2.803435in}{0.608871in}}%
\pgfpathlineto{\pgfqpoint{2.806175in}{0.607735in}}%
\pgfpathlineto{\pgfqpoint{2.808792in}{0.608740in}}%
\pgfpathlineto{\pgfqpoint{2.811597in}{0.607626in}}%
\pgfpathlineto{\pgfqpoint{2.814141in}{0.608196in}}%
\pgfpathlineto{\pgfqpoint{2.816867in}{0.607374in}}%
\pgfpathlineto{\pgfqpoint{2.819506in}{0.599784in}}%
\pgfpathlineto{\pgfqpoint{2.822303in}{0.602390in}}%
\pgfpathlineto{\pgfqpoint{2.824851in}{0.602348in}}%
\pgfpathlineto{\pgfqpoint{2.827567in}{0.601621in}}%
\pgfpathlineto{\pgfqpoint{2.830219in}{0.603598in}}%
\pgfpathlineto{\pgfqpoint{2.832894in}{0.603617in}}%
\pgfpathlineto{\pgfqpoint{2.835698in}{0.600095in}}%
\pgfpathlineto{\pgfqpoint{2.838254in}{0.601173in}}%
\pgfpathlineto{\pgfqpoint{2.841055in}{0.608921in}}%
\pgfpathlineto{\pgfqpoint{2.843611in}{0.613549in}}%
\pgfpathlineto{\pgfqpoint{2.846408in}{0.609153in}}%
\pgfpathlineto{\pgfqpoint{2.848960in}{0.604514in}}%
\pgfpathlineto{\pgfqpoint{2.851793in}{0.603155in}}%
\pgfpathlineto{\pgfqpoint{2.854325in}{0.602360in}}%
\pgfpathlineto{\pgfqpoint{2.857003in}{0.610294in}}%
\pgfpathlineto{\pgfqpoint{2.859668in}{0.608548in}}%
\pgfpathlineto{\pgfqpoint{2.862402in}{0.607816in}}%
\pgfpathlineto{\pgfqpoint{2.865031in}{0.609765in}}%
\pgfpathlineto{\pgfqpoint{2.867713in}{0.606814in}}%
\pgfpathlineto{\pgfqpoint{2.870475in}{0.604866in}}%
\pgfpathlineto{\pgfqpoint{2.873074in}{0.609386in}}%
\pgfpathlineto{\pgfqpoint{2.875882in}{0.611784in}}%
\pgfpathlineto{\pgfqpoint{2.878431in}{0.610772in}}%
\pgfpathlineto{\pgfqpoint{2.881254in}{0.608750in}}%
\pgfpathlineto{\pgfqpoint{2.883780in}{0.610967in}}%
\pgfpathlineto{\pgfqpoint{2.886578in}{0.605541in}}%
\pgfpathlineto{\pgfqpoint{2.889145in}{0.603903in}}%
\pgfpathlineto{\pgfqpoint{2.891809in}{0.608833in}}%
\pgfpathlineto{\pgfqpoint{2.894487in}{0.602926in}}%
\pgfpathlineto{\pgfqpoint{2.897179in}{0.603663in}}%
\pgfpathlineto{\pgfqpoint{2.899858in}{0.600696in}}%
\pgfpathlineto{\pgfqpoint{2.902535in}{0.607372in}}%
\pgfpathlineto{\pgfqpoint{2.905341in}{0.604887in}}%
\pgfpathlineto{\pgfqpoint{2.907882in}{0.606886in}}%
\pgfpathlineto{\pgfqpoint{2.910631in}{0.605549in}}%
\pgfpathlineto{\pgfqpoint{2.913243in}{0.599491in}}%
\pgfpathlineto{\pgfqpoint{2.916061in}{0.601926in}}%
\pgfpathlineto{\pgfqpoint{2.918606in}{0.609132in}}%
\pgfpathlineto{\pgfqpoint{2.921363in}{0.601455in}}%
\pgfpathlineto{\pgfqpoint{2.923963in}{0.606995in}}%
\pgfpathlineto{\pgfqpoint{2.926655in}{0.599301in}}%
\pgfpathlineto{\pgfqpoint{2.929321in}{0.597677in}}%
\pgfpathlineto{\pgfqpoint{2.932033in}{0.596045in}}%
\pgfpathlineto{\pgfqpoint{2.934759in}{0.598990in}}%
\pgfpathlineto{\pgfqpoint{2.937352in}{0.602181in}}%
\pgfpathlineto{\pgfqpoint{2.940120in}{0.606570in}}%
\pgfpathlineto{\pgfqpoint{2.942711in}{0.603917in}}%
\pgfpathlineto{\pgfqpoint{2.945461in}{0.602922in}}%
\pgfpathlineto{\pgfqpoint{2.948068in}{0.607115in}}%
\pgfpathlineto{\pgfqpoint{2.950884in}{0.610767in}}%
\pgfpathlineto{\pgfqpoint{2.953422in}{0.603300in}}%
\pgfpathlineto{\pgfqpoint{2.956103in}{0.606641in}}%
\pgfpathlineto{\pgfqpoint{2.958782in}{0.610894in}}%
\pgfpathlineto{\pgfqpoint{2.961460in}{0.608222in}}%
\pgfpathlineto{\pgfqpoint{2.964127in}{0.606001in}}%
\pgfpathlineto{\pgfqpoint{2.966812in}{0.608542in}}%
\pgfpathlineto{\pgfqpoint{2.969599in}{0.607205in}}%
\pgfpathlineto{\pgfqpoint{2.972177in}{0.604773in}}%
\pgfpathlineto{\pgfqpoint{2.974972in}{0.605514in}}%
\pgfpathlineto{\pgfqpoint{2.977517in}{0.607774in}}%
\pgfpathlineto{\pgfqpoint{2.980341in}{0.607176in}}%
\pgfpathlineto{\pgfqpoint{2.982885in}{0.604689in}}%
\pgfpathlineto{\pgfqpoint{2.985666in}{0.606788in}}%
\pgfpathlineto{\pgfqpoint{2.988238in}{0.609050in}}%
\pgfpathlineto{\pgfqpoint{2.990978in}{0.610796in}}%
\pgfpathlineto{\pgfqpoint{2.993595in}{0.610800in}}%
\pgfpathlineto{\pgfqpoint{2.996300in}{0.609486in}}%
\pgfpathlineto{\pgfqpoint{2.999103in}{0.599321in}}%
\pgfpathlineto{\pgfqpoint{3.001635in}{0.598419in}}%
\pgfpathlineto{\pgfqpoint{3.004419in}{0.601720in}}%
\pgfpathlineto{\pgfqpoint{3.006993in}{0.602433in}}%
\pgfpathlineto{\pgfqpoint{3.009784in}{0.606734in}}%
\pgfpathlineto{\pgfqpoint{3.012351in}{0.603414in}}%
\pgfpathlineto{\pgfqpoint{3.015097in}{0.606507in}}%
\pgfpathlineto{\pgfqpoint{3.017707in}{0.607060in}}%
\pgfpathlineto{\pgfqpoint{3.020382in}{0.606869in}}%
\pgfpathlineto{\pgfqpoint{3.023058in}{0.619623in}}%
\pgfpathlineto{\pgfqpoint{3.025803in}{0.623390in}}%
\pgfpathlineto{\pgfqpoint{3.028412in}{0.611404in}}%
\pgfpathlineto{\pgfqpoint{3.031091in}{0.603302in}}%
\pgfpathlineto{\pgfqpoint{3.033921in}{0.601428in}}%
\pgfpathlineto{\pgfqpoint{3.036456in}{0.599764in}}%
\pgfpathlineto{\pgfqpoint{3.039262in}{0.602798in}}%
\pgfpathlineto{\pgfqpoint{3.041813in}{0.602073in}}%
\pgfpathlineto{\pgfqpoint{3.044568in}{0.602431in}}%
\pgfpathlineto{\pgfqpoint{3.047157in}{0.598720in}}%
\pgfpathlineto{\pgfqpoint{3.049988in}{0.600232in}}%
\pgfpathlineto{\pgfqpoint{3.052526in}{0.595041in}}%
\pgfpathlineto{\pgfqpoint{3.055202in}{0.597353in}}%
\pgfpathlineto{\pgfqpoint{3.057884in}{0.597789in}}%
\pgfpathlineto{\pgfqpoint{3.060561in}{0.601702in}}%
\pgfpathlineto{\pgfqpoint{3.063230in}{0.600658in}}%
\pgfpathlineto{\pgfqpoint{3.065916in}{0.601694in}}%
\pgfpathlineto{\pgfqpoint{3.068709in}{0.603273in}}%
\pgfpathlineto{\pgfqpoint{3.071266in}{0.599914in}}%
\pgfpathlineto{\pgfqpoint{3.074056in}{0.602917in}}%
\pgfpathlineto{\pgfqpoint{3.076631in}{0.601377in}}%
\pgfpathlineto{\pgfqpoint{3.079381in}{0.603003in}}%
\pgfpathlineto{\pgfqpoint{3.081990in}{0.600872in}}%
\pgfpathlineto{\pgfqpoint{3.084671in}{0.602370in}}%
\pgfpathlineto{\pgfqpoint{3.087343in}{0.602552in}}%
\pgfpathlineto{\pgfqpoint{3.090023in}{0.603352in}}%
\pgfpathlineto{\pgfqpoint{3.092699in}{0.598719in}}%
\pgfpathlineto{\pgfqpoint{3.095388in}{0.601156in}}%
\pgfpathlineto{\pgfqpoint{3.098163in}{0.599874in}}%
\pgfpathlineto{\pgfqpoint{3.100737in}{0.600031in}}%
\pgfpathlineto{\pgfqpoint{3.103508in}{0.612734in}}%
\pgfpathlineto{\pgfqpoint{3.106094in}{0.604831in}}%
\pgfpathlineto{\pgfqpoint{3.108896in}{0.601899in}}%
\pgfpathlineto{\pgfqpoint{3.111451in}{0.605497in}}%
\pgfpathlineto{\pgfqpoint{3.114242in}{0.609662in}}%
\pgfpathlineto{\pgfqpoint{3.116807in}{0.609237in}}%
\pgfpathlineto{\pgfqpoint{3.119487in}{0.606304in}}%
\pgfpathlineto{\pgfqpoint{3.122163in}{0.610200in}}%
\pgfpathlineto{\pgfqpoint{3.124842in}{0.610918in}}%
\pgfpathlineto{\pgfqpoint{3.127512in}{0.609887in}}%
\pgfpathlineto{\pgfqpoint{3.130199in}{0.608688in}}%
\pgfpathlineto{\pgfqpoint{3.132946in}{0.609594in}}%
\pgfpathlineto{\pgfqpoint{3.135550in}{0.613442in}}%
\pgfpathlineto{\pgfqpoint{3.138375in}{0.605559in}}%
\pgfpathlineto{\pgfqpoint{3.140913in}{0.607473in}}%
\pgfpathlineto{\pgfqpoint{3.143740in}{0.602227in}}%
\pgfpathlineto{\pgfqpoint{3.146271in}{0.597815in}}%
\pgfpathlineto{\pgfqpoint{3.149057in}{0.595041in}}%
\pgfpathlineto{\pgfqpoint{3.151612in}{0.595041in}}%
\pgfpathlineto{\pgfqpoint{3.154327in}{0.605707in}}%
\pgfpathlineto{\pgfqpoint{3.156981in}{0.606326in}}%
\pgfpathlineto{\pgfqpoint{3.159675in}{0.606423in}}%
\pgfpathlineto{\pgfqpoint{3.162474in}{0.603947in}}%
\pgfpathlineto{\pgfqpoint{3.165019in}{0.601571in}}%
\pgfpathlineto{\pgfqpoint{3.167776in}{0.596875in}}%
\pgfpathlineto{\pgfqpoint{3.170375in}{0.601724in}}%
\pgfpathlineto{\pgfqpoint{3.173142in}{0.595656in}}%
\pgfpathlineto{\pgfqpoint{3.175724in}{0.599483in}}%
\pgfpathlineto{\pgfqpoint{3.178525in}{0.606707in}}%
\pgfpathlineto{\pgfqpoint{3.181089in}{0.603340in}}%
\pgfpathlineto{\pgfqpoint{3.183760in}{0.603190in}}%
\pgfpathlineto{\pgfqpoint{3.186440in}{0.604949in}}%
\pgfpathlineto{\pgfqpoint{3.189117in}{0.605696in}}%
\pgfpathlineto{\pgfqpoint{3.191796in}{0.605698in}}%
\pgfpathlineto{\pgfqpoint{3.194508in}{0.601488in}}%
\pgfpathlineto{\pgfqpoint{3.197226in}{0.604224in}}%
\pgfpathlineto{\pgfqpoint{3.199823in}{0.596863in}}%
\pgfpathlineto{\pgfqpoint{3.202562in}{0.598143in}}%
\pgfpathlineto{\pgfqpoint{3.205195in}{0.600844in}}%
\pgfpathlineto{\pgfqpoint{3.207984in}{0.606443in}}%
\pgfpathlineto{\pgfqpoint{3.210545in}{0.600541in}}%
\pgfpathlineto{\pgfqpoint{3.213342in}{0.601029in}}%
\pgfpathlineto{\pgfqpoint{3.215908in}{0.598279in}}%
\pgfpathlineto{\pgfqpoint{3.218586in}{0.602698in}}%
\pgfpathlineto{\pgfqpoint{3.221255in}{0.608786in}}%
\pgfpathlineto{\pgfqpoint{3.223942in}{0.624713in}}%
\pgfpathlineto{\pgfqpoint{3.226609in}{0.676031in}}%
\pgfpathlineto{\pgfqpoint{3.229310in}{0.664510in}}%
\pgfpathlineto{\pgfqpoint{3.232069in}{0.646124in}}%
\pgfpathlineto{\pgfqpoint{3.234658in}{0.647438in}}%
\pgfpathlineto{\pgfqpoint{3.237411in}{0.642266in}}%
\pgfpathlineto{\pgfqpoint{3.240010in}{0.641224in}}%
\pgfpathlineto{\pgfqpoint{3.242807in}{0.642700in}}%
\pgfpathlineto{\pgfqpoint{3.245363in}{0.659533in}}%
\pgfpathlineto{\pgfqpoint{3.248049in}{0.647721in}}%
\pgfpathlineto{\pgfqpoint{3.250716in}{0.642689in}}%
\pgfpathlineto{\pgfqpoint{3.253404in}{0.639709in}}%
\pgfpathlineto{\pgfqpoint{3.256083in}{0.628098in}}%
\pgfpathlineto{\pgfqpoint{3.258784in}{0.624722in}}%
\pgfpathlineto{\pgfqpoint{3.261594in}{0.618466in}}%
\pgfpathlineto{\pgfqpoint{3.264119in}{0.613433in}}%
\pgfpathlineto{\pgfqpoint{3.266849in}{0.612464in}}%
\pgfpathlineto{\pgfqpoint{3.269478in}{0.611931in}}%
\pgfpathlineto{\pgfqpoint{3.272254in}{0.612060in}}%
\pgfpathlineto{\pgfqpoint{3.274831in}{0.613539in}}%
\pgfpathlineto{\pgfqpoint{3.277603in}{0.615155in}}%
\pgfpathlineto{\pgfqpoint{3.280189in}{0.614928in}}%
\pgfpathlineto{\pgfqpoint{3.282870in}{0.615950in}}%
\pgfpathlineto{\pgfqpoint{3.285534in}{0.612994in}}%
\pgfpathlineto{\pgfqpoint{3.288225in}{0.613445in}}%
\pgfpathlineto{\pgfqpoint{3.290890in}{0.614682in}}%
\pgfpathlineto{\pgfqpoint{3.293574in}{0.610484in}}%
\pgfpathlineto{\pgfqpoint{3.296376in}{0.609939in}}%
\pgfpathlineto{\pgfqpoint{3.298937in}{0.606123in}}%
\pgfpathlineto{\pgfqpoint{3.301719in}{0.601770in}}%
\pgfpathlineto{\pgfqpoint{3.304295in}{0.606960in}}%
\pgfpathlineto{\pgfqpoint{3.307104in}{0.603523in}}%
\pgfpathlineto{\pgfqpoint{3.309652in}{0.602437in}}%
\pgfpathlineto{\pgfqpoint{3.312480in}{0.602028in}}%
\pgfpathlineto{\pgfqpoint{3.315008in}{0.606558in}}%
\pgfpathlineto{\pgfqpoint{3.317688in}{0.602347in}}%
\pgfpathlineto{\pgfqpoint{3.320366in}{0.603590in}}%
\pgfpathlineto{\pgfqpoint{3.323049in}{0.599922in}}%
\pgfpathlineto{\pgfqpoint{3.325860in}{0.603751in}}%
\pgfpathlineto{\pgfqpoint{3.328401in}{0.603957in}}%
\pgfpathlineto{\pgfqpoint{3.331183in}{0.604823in}}%
\pgfpathlineto{\pgfqpoint{3.333758in}{0.601824in}}%
\pgfpathlineto{\pgfqpoint{3.336541in}{0.603311in}}%
\pgfpathlineto{\pgfqpoint{3.339101in}{0.603341in}}%
\pgfpathlineto{\pgfqpoint{3.341893in}{0.603320in}}%
\pgfpathlineto{\pgfqpoint{3.344468in}{0.603702in}}%
\pgfpathlineto{\pgfqpoint{3.347139in}{0.603769in}}%
\pgfpathlineto{\pgfqpoint{3.349828in}{0.605824in}}%
\pgfpathlineto{\pgfqpoint{3.352505in}{0.607965in}}%
\pgfpathlineto{\pgfqpoint{3.355177in}{0.605143in}}%
\pgfpathlineto{\pgfqpoint{3.357862in}{0.606938in}}%
\pgfpathlineto{\pgfqpoint{3.360620in}{0.605624in}}%
\pgfpathlineto{\pgfqpoint{3.363221in}{0.608299in}}%
\pgfpathlineto{\pgfqpoint{3.365996in}{0.607377in}}%
\pgfpathlineto{\pgfqpoint{3.368577in}{0.610973in}}%
\pgfpathlineto{\pgfqpoint{3.371357in}{0.603800in}}%
\pgfpathlineto{\pgfqpoint{3.373921in}{0.596038in}}%
\pgfpathlineto{\pgfqpoint{3.376735in}{0.602244in}}%
\pgfpathlineto{\pgfqpoint{3.379290in}{0.606407in}}%
\pgfpathlineto{\pgfqpoint{3.381959in}{0.610378in}}%
\pgfpathlineto{\pgfqpoint{3.384647in}{0.608328in}}%
\pgfpathlineto{\pgfqpoint{3.387309in}{0.611042in}}%
\pgfpathlineto{\pgfqpoint{3.390102in}{0.613436in}}%
\pgfpathlineto{\pgfqpoint{3.392681in}{0.607702in}}%
\pgfpathlineto{\pgfqpoint{3.395461in}{0.608817in}}%
\pgfpathlineto{\pgfqpoint{3.398037in}{0.602484in}}%
\pgfpathlineto{\pgfqpoint{3.400783in}{0.607678in}}%
\pgfpathlineto{\pgfqpoint{3.403394in}{0.610369in}}%
\pgfpathlineto{\pgfqpoint{3.406202in}{0.608027in}}%
\pgfpathlineto{\pgfqpoint{3.408752in}{0.610422in}}%
\pgfpathlineto{\pgfqpoint{3.411431in}{0.607130in}}%
\pgfpathlineto{\pgfqpoint{3.414109in}{0.608415in}}%
\pgfpathlineto{\pgfqpoint{3.416780in}{0.621662in}}%
\pgfpathlineto{\pgfqpoint{3.419455in}{0.614074in}}%
\pgfpathlineto{\pgfqpoint{3.422142in}{0.619967in}}%
\pgfpathlineto{\pgfqpoint{3.424887in}{0.615154in}}%
\pgfpathlineto{\pgfqpoint{3.427501in}{0.612113in}}%
\pgfpathlineto{\pgfqpoint{3.430313in}{0.606852in}}%
\pgfpathlineto{\pgfqpoint{3.432851in}{0.608806in}}%
\pgfpathlineto{\pgfqpoint{3.435635in}{0.611491in}}%
\pgfpathlineto{\pgfqpoint{3.438210in}{0.610180in}}%
\pgfpathlineto{\pgfqpoint{3.440996in}{0.610174in}}%
\pgfpathlineto{\pgfqpoint{3.443574in}{0.611553in}}%
\pgfpathlineto{\pgfqpoint{3.446257in}{0.607781in}}%
\pgfpathlineto{\pgfqpoint{3.448926in}{0.609873in}}%
\pgfpathlineto{\pgfqpoint{3.451597in}{0.608434in}}%
\pgfpathlineto{\pgfqpoint{3.454285in}{0.610412in}}%
\pgfpathlineto{\pgfqpoint{3.456960in}{0.607511in}}%
\pgfpathlineto{\pgfqpoint{3.459695in}{0.609492in}}%
\pgfpathlineto{\pgfqpoint{3.462321in}{0.608595in}}%
\pgfpathlineto{\pgfqpoint{3.465072in}{0.611194in}}%
\pgfpathlineto{\pgfqpoint{3.467678in}{0.604928in}}%
\pgfpathlineto{\pgfqpoint{3.470466in}{0.606341in}}%
\pgfpathlineto{\pgfqpoint{3.473021in}{0.604638in}}%
\pgfpathlineto{\pgfqpoint{3.475821in}{0.606095in}}%
\pgfpathlineto{\pgfqpoint{3.478378in}{0.604832in}}%
\pgfpathlineto{\pgfqpoint{3.481072in}{0.601011in}}%
\pgfpathlineto{\pgfqpoint{3.483744in}{0.603837in}}%
\pgfpathlineto{\pgfqpoint{3.486442in}{0.604897in}}%
\pgfpathlineto{\pgfqpoint{3.489223in}{0.602732in}}%
\pgfpathlineto{\pgfqpoint{3.491783in}{0.605348in}}%
\pgfpathlineto{\pgfqpoint{3.494581in}{0.606865in}}%
\pgfpathlineto{\pgfqpoint{3.497139in}{0.607444in}}%
\pgfpathlineto{\pgfqpoint{3.499909in}{0.606564in}}%
\pgfpathlineto{\pgfqpoint{3.502488in}{0.600380in}}%
\pgfpathlineto{\pgfqpoint{3.505262in}{0.604964in}}%
\pgfpathlineto{\pgfqpoint{3.507840in}{0.607273in}}%
\pgfpathlineto{\pgfqpoint{3.510533in}{0.603870in}}%
\pgfpathlineto{\pgfqpoint{3.513209in}{0.607308in}}%
\pgfpathlineto{\pgfqpoint{3.515884in}{0.607038in}}%
\pgfpathlineto{\pgfqpoint{3.518565in}{0.604312in}}%
\pgfpathlineto{\pgfqpoint{3.521244in}{0.605939in}}%
\pgfpathlineto{\pgfqpoint{3.524041in}{0.607998in}}%
\pgfpathlineto{\pgfqpoint{3.526601in}{0.609636in}}%
\pgfpathlineto{\pgfqpoint{3.529327in}{0.606157in}}%
\pgfpathlineto{\pgfqpoint{3.531955in}{0.604495in}}%
\pgfpathlineto{\pgfqpoint{3.534783in}{0.604204in}}%
\pgfpathlineto{\pgfqpoint{3.537309in}{0.605073in}}%
\pgfpathlineto{\pgfqpoint{3.540093in}{0.606441in}}%
\pgfpathlineto{\pgfqpoint{3.542656in}{0.600982in}}%
\pgfpathlineto{\pgfqpoint{3.545349in}{0.602527in}}%
\pgfpathlineto{\pgfqpoint{3.548029in}{0.608144in}}%
\pgfpathlineto{\pgfqpoint{3.550713in}{0.611355in}}%
\pgfpathlineto{\pgfqpoint{3.553498in}{0.610679in}}%
\pgfpathlineto{\pgfqpoint{3.556061in}{0.609586in}}%
\pgfpathlineto{\pgfqpoint{3.558853in}{0.608661in}}%
\pgfpathlineto{\pgfqpoint{3.561420in}{0.609686in}}%
\pgfpathlineto{\pgfqpoint{3.564188in}{0.606329in}}%
\pgfpathlineto{\pgfqpoint{3.566774in}{0.605386in}}%
\pgfpathlineto{\pgfqpoint{3.569584in}{0.604424in}}%
\pgfpathlineto{\pgfqpoint{3.572126in}{0.609144in}}%
\pgfpathlineto{\pgfqpoint{3.574814in}{0.602094in}}%
\pgfpathlineto{\pgfqpoint{3.577487in}{0.602711in}}%
\pgfpathlineto{\pgfqpoint{3.580191in}{0.604204in}}%
\pgfpathlineto{\pgfqpoint{3.582851in}{0.604240in}}%
\pgfpathlineto{\pgfqpoint{3.585532in}{0.607847in}}%
\pgfpathlineto{\pgfqpoint{3.588258in}{0.604047in}}%
\pgfpathlineto{\pgfqpoint{3.590883in}{0.604440in}}%
\pgfpathlineto{\pgfqpoint{3.593620in}{0.609038in}}%
\pgfpathlineto{\pgfqpoint{3.596240in}{0.608351in}}%
\pgfpathlineto{\pgfqpoint{3.598998in}{0.604594in}}%
\pgfpathlineto{\pgfqpoint{3.601590in}{0.605426in}}%
\pgfpathlineto{\pgfqpoint{3.604387in}{0.607204in}}%
\pgfpathlineto{\pgfqpoint{3.606951in}{0.607030in}}%
\pgfpathlineto{\pgfqpoint{3.609632in}{0.608853in}}%
\pgfpathlineto{\pgfqpoint{3.612311in}{0.621336in}}%
\pgfpathlineto{\pgfqpoint{3.614982in}{0.617904in}}%
\pgfpathlineto{\pgfqpoint{3.617667in}{0.613420in}}%
\pgfpathlineto{\pgfqpoint{3.620345in}{0.610480in}}%
\pgfpathlineto{\pgfqpoint{3.623165in}{0.610269in}}%
\pgfpathlineto{\pgfqpoint{3.625689in}{0.607838in}}%
\pgfpathlineto{\pgfqpoint{3.628460in}{0.613687in}}%
\pgfpathlineto{\pgfqpoint{3.631058in}{0.614336in}}%
\pgfpathlineto{\pgfqpoint{3.633858in}{0.610465in}}%
\pgfpathlineto{\pgfqpoint{3.636413in}{0.613709in}}%
\pgfpathlineto{\pgfqpoint{3.639207in}{0.612643in}}%
\pgfpathlineto{\pgfqpoint{3.641773in}{0.613094in}}%
\pgfpathlineto{\pgfqpoint{3.644452in}{0.608142in}}%
\pgfpathlineto{\pgfqpoint{3.647130in}{0.601886in}}%
\pgfpathlineto{\pgfqpoint{3.649837in}{0.603463in}}%
\pgfpathlineto{\pgfqpoint{3.652628in}{0.606011in}}%
\pgfpathlineto{\pgfqpoint{3.655165in}{0.608966in}}%
\pgfpathlineto{\pgfqpoint{3.657917in}{0.598206in}}%
\pgfpathlineto{\pgfqpoint{3.660515in}{0.595041in}}%
\pgfpathlineto{\pgfqpoint{3.663276in}{0.595041in}}%
\pgfpathlineto{\pgfqpoint{3.665864in}{0.595041in}}%
\pgfpathlineto{\pgfqpoint{3.668665in}{0.597965in}}%
\pgfpathlineto{\pgfqpoint{3.671232in}{0.602449in}}%
\pgfpathlineto{\pgfqpoint{3.673911in}{0.599655in}}%
\pgfpathlineto{\pgfqpoint{3.676591in}{0.596884in}}%
\pgfpathlineto{\pgfqpoint{3.679273in}{0.595041in}}%
\pgfpathlineto{\pgfqpoint{3.681948in}{0.598212in}}%
\pgfpathlineto{\pgfqpoint{3.684620in}{0.595491in}}%
\pgfpathlineto{\pgfqpoint{3.687442in}{0.599622in}}%
\pgfpathlineto{\pgfqpoint{3.689983in}{0.595929in}}%
\pgfpathlineto{\pgfqpoint{3.692765in}{0.598160in}}%
\pgfpathlineto{\pgfqpoint{3.695331in}{0.601654in}}%
\pgfpathlineto{\pgfqpoint{3.698125in}{0.603790in}}%
\pgfpathlineto{\pgfqpoint{3.700684in}{0.603437in}}%
\pgfpathlineto{\pgfqpoint{3.703460in}{0.596421in}}%
\pgfpathlineto{\pgfqpoint{3.706053in}{0.602875in}}%
\pgfpathlineto{\pgfqpoint{3.708729in}{0.605082in}}%
\pgfpathlineto{\pgfqpoint{3.711410in}{0.602736in}}%
\pgfpathlineto{\pgfqpoint{3.714086in}{0.605248in}}%
\pgfpathlineto{\pgfqpoint{3.716875in}{0.601356in}}%
\pgfpathlineto{\pgfqpoint{3.719446in}{0.599603in}}%
\pgfpathlineto{\pgfqpoint{3.722228in}{0.600835in}}%
\pgfpathlineto{\pgfqpoint{3.724804in}{0.598754in}}%
\pgfpathlineto{\pgfqpoint{3.727581in}{0.603903in}}%
\pgfpathlineto{\pgfqpoint{3.730158in}{0.602252in}}%
\pgfpathlineto{\pgfqpoint{3.732950in}{0.604720in}}%
\pgfpathlineto{\pgfqpoint{3.735509in}{0.601820in}}%
\pgfpathlineto{\pgfqpoint{3.738194in}{0.600612in}}%
\pgfpathlineto{\pgfqpoint{3.740874in}{0.600374in}}%
\pgfpathlineto{\pgfqpoint{3.743548in}{0.603054in}}%
\pgfpathlineto{\pgfqpoint{3.746229in}{0.602568in}}%
\pgfpathlineto{\pgfqpoint{3.748903in}{0.604366in}}%
\pgfpathlineto{\pgfqpoint{3.751728in}{0.603069in}}%
\pgfpathlineto{\pgfqpoint{3.754265in}{0.601333in}}%
\pgfpathlineto{\pgfqpoint{3.757065in}{0.601434in}}%
\pgfpathlineto{\pgfqpoint{3.759622in}{0.601511in}}%
\pgfpathlineto{\pgfqpoint{3.762389in}{0.599365in}}%
\pgfpathlineto{\pgfqpoint{3.764966in}{0.638535in}}%
\pgfpathlineto{\pgfqpoint{3.767782in}{0.654108in}}%
\pgfpathlineto{\pgfqpoint{3.770323in}{0.654761in}}%
\pgfpathlineto{\pgfqpoint{3.773014in}{0.657512in}}%
\pgfpathlineto{\pgfqpoint{3.775691in}{0.681731in}}%
\pgfpathlineto{\pgfqpoint{3.778370in}{0.699789in}}%
\pgfpathlineto{\pgfqpoint{3.781046in}{0.714905in}}%
\pgfpathlineto{\pgfqpoint{3.783725in}{0.673021in}}%
\pgfpathlineto{\pgfqpoint{3.786504in}{0.659140in}}%
\pgfpathlineto{\pgfqpoint{3.789084in}{0.650961in}}%
\pgfpathlineto{\pgfqpoint{3.791897in}{0.640538in}}%
\pgfpathlineto{\pgfqpoint{3.794435in}{0.659531in}}%
\pgfpathlineto{\pgfqpoint{3.797265in}{0.675602in}}%
\pgfpathlineto{\pgfqpoint{3.799797in}{0.684668in}}%
\pgfpathlineto{\pgfqpoint{3.802569in}{0.679869in}}%
\pgfpathlineto{\pgfqpoint{3.805145in}{0.668270in}}%
\pgfpathlineto{\pgfqpoint{3.807832in}{0.660747in}}%
\pgfpathlineto{\pgfqpoint{3.810510in}{0.646173in}}%
\pgfpathlineto{\pgfqpoint{3.813172in}{0.645013in}}%
\pgfpathlineto{\pgfqpoint{3.815983in}{0.642268in}}%
\pgfpathlineto{\pgfqpoint{3.818546in}{0.634859in}}%
\pgfpathlineto{\pgfqpoint{3.821315in}{0.630781in}}%
\pgfpathlineto{\pgfqpoint{3.823903in}{0.627039in}}%
\pgfpathlineto{\pgfqpoint{3.826679in}{0.618684in}}%
\pgfpathlineto{\pgfqpoint{3.829252in}{0.617704in}}%
\pgfpathlineto{\pgfqpoint{3.832053in}{0.613307in}}%
\pgfpathlineto{\pgfqpoint{3.834616in}{0.613657in}}%
\pgfpathlineto{\pgfqpoint{3.837286in}{0.610698in}}%
\pgfpathlineto{\pgfqpoint{3.839960in}{0.610820in}}%
\pgfpathlineto{\pgfqpoint{3.842641in}{0.608842in}}%
\pgfpathlineto{\pgfqpoint{3.845329in}{0.610352in}}%
\pgfpathlineto{\pgfqpoint{3.848005in}{0.611034in}}%
\pgfpathlineto{\pgfqpoint{3.850814in}{0.606313in}}%
\pgfpathlineto{\pgfqpoint{3.853358in}{0.601629in}}%
\pgfpathlineto{\pgfqpoint{3.856100in}{0.602417in}}%
\pgfpathlineto{\pgfqpoint{3.858720in}{0.605244in}}%
\pgfpathlineto{\pgfqpoint{3.861561in}{0.599165in}}%
\pgfpathlineto{\pgfqpoint{3.864073in}{0.602063in}}%
\pgfpathlineto{\pgfqpoint{3.866815in}{0.606208in}}%
\pgfpathlineto{\pgfqpoint{3.869435in}{0.604686in}}%
\pgfpathlineto{\pgfqpoint{3.872114in}{0.604223in}}%
\pgfpathlineto{\pgfqpoint{3.874790in}{0.608009in}}%
\pgfpathlineto{\pgfqpoint{3.877466in}{0.601016in}}%
\pgfpathlineto{\pgfqpoint{3.880237in}{0.605614in}}%
\pgfpathlineto{\pgfqpoint{3.882850in}{0.604348in}}%
\pgfpathlineto{\pgfqpoint{3.885621in}{0.600589in}}%
\pgfpathlineto{\pgfqpoint{3.888188in}{0.603415in}}%
\pgfpathlineto{\pgfqpoint{3.890926in}{0.599056in}}%
\pgfpathlineto{\pgfqpoint{3.893541in}{0.600408in}}%
\pgfpathlineto{\pgfqpoint{3.896345in}{0.599309in}}%
\pgfpathlineto{\pgfqpoint{3.898891in}{0.599706in}}%
\pgfpathlineto{\pgfqpoint{3.901573in}{0.598560in}}%
\pgfpathlineto{\pgfqpoint{3.904252in}{0.600795in}}%
\pgfpathlineto{\pgfqpoint{3.906918in}{0.600022in}}%
\pgfpathlineto{\pgfqpoint{3.909602in}{0.601210in}}%
\pgfpathlineto{\pgfqpoint{3.912296in}{0.601632in}}%
\pgfpathlineto{\pgfqpoint{3.915107in}{0.600717in}}%
\pgfpathlineto{\pgfqpoint{3.917646in}{0.597299in}}%
\pgfpathlineto{\pgfqpoint{3.920412in}{0.601220in}}%
\pgfpathlineto{\pgfqpoint{3.923005in}{0.599057in}}%
\pgfpathlineto{\pgfqpoint{3.925778in}{0.602073in}}%
\pgfpathlineto{\pgfqpoint{3.928347in}{0.600405in}}%
\pgfpathlineto{\pgfqpoint{3.931202in}{0.600309in}}%
\pgfpathlineto{\pgfqpoint{3.933714in}{0.604781in}}%
\pgfpathlineto{\pgfqpoint{3.936395in}{0.598347in}}%
\pgfpathlineto{\pgfqpoint{3.939075in}{0.597003in}}%
\pgfpathlineto{\pgfqpoint{3.941778in}{0.595332in}}%
\pgfpathlineto{\pgfqpoint{3.944431in}{0.595041in}}%
\pgfpathlineto{\pgfqpoint{3.947101in}{0.595041in}}%
\pgfpathlineto{\pgfqpoint{3.949894in}{0.595041in}}%
\pgfpathlineto{\pgfqpoint{3.952464in}{0.595041in}}%
\pgfpathlineto{\pgfqpoint{3.955211in}{0.595041in}}%
\pgfpathlineto{\pgfqpoint{3.957823in}{0.595461in}}%
\pgfpathlineto{\pgfqpoint{3.960635in}{0.600708in}}%
\pgfpathlineto{\pgfqpoint{3.963176in}{0.595652in}}%
\pgfpathlineto{\pgfqpoint{3.966013in}{0.597068in}}%
\pgfpathlineto{\pgfqpoint{3.968523in}{0.598055in}}%
\pgfpathlineto{\pgfqpoint{3.971250in}{0.599450in}}%
\pgfpathlineto{\pgfqpoint{3.973885in}{0.596496in}}%
\pgfpathlineto{\pgfqpoint{3.976563in}{0.595041in}}%
\pgfpathlineto{\pgfqpoint{3.979389in}{0.600851in}}%
\pgfpathlineto{\pgfqpoint{3.981929in}{0.600027in}}%
\pgfpathlineto{\pgfqpoint{3.984714in}{0.595041in}}%
\pgfpathlineto{\pgfqpoint{3.987270in}{0.595041in}}%
\pgfpathlineto{\pgfqpoint{3.990055in}{0.597154in}}%
\pgfpathlineto{\pgfqpoint{3.992642in}{0.599278in}}%
\pgfpathlineto{\pgfqpoint{3.995417in}{0.601992in}}%
\pgfpathlineto{\pgfqpoint{3.997990in}{0.601421in}}%
\pgfpathlineto{\pgfqpoint{4.000674in}{0.595182in}}%
\pgfpathlineto{\pgfqpoint{4.003348in}{0.599901in}}%
\pgfpathlineto{\pgfqpoint{4.006034in}{0.599092in}}%
\pgfpathlineto{\pgfqpoint{4.008699in}{0.602204in}}%
\pgfpathlineto{\pgfqpoint{4.011394in}{0.603198in}}%
\pgfpathlineto{\pgfqpoint{4.014186in}{0.601396in}}%
\pgfpathlineto{\pgfqpoint{4.016744in}{0.600343in}}%
\pgfpathlineto{\pgfqpoint{4.019518in}{0.604428in}}%
\pgfpathlineto{\pgfqpoint{4.022097in}{0.603418in}}%
\pgfpathlineto{\pgfqpoint{4.024868in}{0.604642in}}%
\pgfpathlineto{\pgfqpoint{4.027447in}{0.606423in}}%
\pgfpathlineto{\pgfqpoint{4.030229in}{0.605634in}}%
\pgfpathlineto{\pgfqpoint{4.032817in}{0.599814in}}%
\pgfpathlineto{\pgfqpoint{4.035492in}{0.602761in}}%
\pgfpathlineto{\pgfqpoint{4.038174in}{0.601313in}}%
\pgfpathlineto{\pgfqpoint{4.040852in}{0.605653in}}%
\pgfpathlineto{\pgfqpoint{4.043667in}{0.596922in}}%
\pgfpathlineto{\pgfqpoint{4.046210in}{0.601689in}}%
\pgfpathlineto{\pgfqpoint{4.049006in}{0.603643in}}%
\pgfpathlineto{\pgfqpoint{4.051557in}{0.603411in}}%
\pgfpathlineto{\pgfqpoint{4.054326in}{0.603534in}}%
\pgfpathlineto{\pgfqpoint{4.056911in}{0.606948in}}%
\pgfpathlineto{\pgfqpoint{4.059702in}{0.604494in}}%
\pgfpathlineto{\pgfqpoint{4.062266in}{0.602499in}}%
\pgfpathlineto{\pgfqpoint{4.064957in}{0.600519in}}%
\pgfpathlineto{\pgfqpoint{4.067636in}{0.602418in}}%
\pgfpathlineto{\pgfqpoint{4.070313in}{0.600740in}}%
\pgfpathlineto{\pgfqpoint{4.072985in}{0.601596in}}%
\pgfpathlineto{\pgfqpoint{4.075705in}{0.604657in}}%
\pgfpathlineto{\pgfqpoint{4.078471in}{0.604969in}}%
\pgfpathlineto{\pgfqpoint{4.081018in}{0.604019in}}%
\pgfpathlineto{\pgfqpoint{4.083870in}{0.606198in}}%
\pgfpathlineto{\pgfqpoint{4.086385in}{0.605570in}}%
\pgfpathlineto{\pgfqpoint{4.089159in}{0.603764in}}%
\pgfpathlineto{\pgfqpoint{4.091729in}{0.602053in}}%
\pgfpathlineto{\pgfqpoint{4.094527in}{0.606252in}}%
\pgfpathlineto{\pgfqpoint{4.097092in}{0.601598in}}%
\pgfpathlineto{\pgfqpoint{4.099777in}{0.604476in}}%
\pgfpathlineto{\pgfqpoint{4.102456in}{0.605135in}}%
\pgfpathlineto{\pgfqpoint{4.105185in}{0.602349in}}%
\pgfpathlineto{\pgfqpoint{4.107814in}{0.606291in}}%
\pgfpathlineto{\pgfqpoint{4.110488in}{0.604426in}}%
\pgfpathlineto{\pgfqpoint{4.113252in}{0.603831in}}%
\pgfpathlineto{\pgfqpoint{4.115844in}{0.600810in}}%
\pgfpathlineto{\pgfqpoint{4.118554in}{0.598409in}}%
\pgfpathlineto{\pgfqpoint{4.121205in}{0.598725in}}%
\pgfpathlineto{\pgfqpoint{4.124019in}{0.602998in}}%
\pgfpathlineto{\pgfqpoint{4.126553in}{0.599915in}}%
\pgfpathlineto{\pgfqpoint{4.129349in}{0.597342in}}%
\pgfpathlineto{\pgfqpoint{4.131920in}{0.598452in}}%
\pgfpathlineto{\pgfqpoint{4.134615in}{0.601822in}}%
\pgfpathlineto{\pgfqpoint{4.137272in}{0.601227in}}%
\pgfpathlineto{\pgfqpoint{4.139963in}{0.602976in}}%
\pgfpathlineto{\pgfqpoint{4.142713in}{0.609002in}}%
\pgfpathlineto{\pgfqpoint{4.145310in}{0.604883in}}%
\pgfpathlineto{\pgfqpoint{4.148082in}{0.602486in}}%
\pgfpathlineto{\pgfqpoint{4.150665in}{0.602237in}}%
\pgfpathlineto{\pgfqpoint{4.153423in}{0.603125in}}%
\pgfpathlineto{\pgfqpoint{4.156016in}{0.603690in}}%
\pgfpathlineto{\pgfqpoint{4.158806in}{0.600125in}}%
\pgfpathlineto{\pgfqpoint{4.161380in}{0.601823in}}%
\pgfpathlineto{\pgfqpoint{4.164059in}{0.600575in}}%
\pgfpathlineto{\pgfqpoint{4.166737in}{0.606006in}}%
\pgfpathlineto{\pgfqpoint{4.169415in}{0.603265in}}%
\pgfpathlineto{\pgfqpoint{4.172093in}{0.603965in}}%
\pgfpathlineto{\pgfqpoint{4.174770in}{0.604736in}}%
\pgfpathlineto{\pgfqpoint{4.177593in}{0.605279in}}%
\pgfpathlineto{\pgfqpoint{4.180129in}{0.604532in}}%
\pgfpathlineto{\pgfqpoint{4.182899in}{0.606422in}}%
\pgfpathlineto{\pgfqpoint{4.185481in}{0.602579in}}%
\pgfpathlineto{\pgfqpoint{4.188318in}{0.600368in}}%
\pgfpathlineto{\pgfqpoint{4.190842in}{0.599536in}}%
\pgfpathlineto{\pgfqpoint{4.193638in}{0.605813in}}%
\pgfpathlineto{\pgfqpoint{4.196186in}{0.605946in}}%
\pgfpathlineto{\pgfqpoint{4.198878in}{0.607117in}}%
\pgfpathlineto{\pgfqpoint{4.201542in}{0.600802in}}%
\pgfpathlineto{\pgfqpoint{4.204240in}{0.599068in}}%
\pgfpathlineto{\pgfqpoint{4.207076in}{0.601750in}}%
\pgfpathlineto{\pgfqpoint{4.209597in}{0.600278in}}%
\pgfpathlineto{\pgfqpoint{4.212383in}{0.600213in}}%
\pgfpathlineto{\pgfqpoint{4.214948in}{0.603080in}}%
\pgfpathlineto{\pgfqpoint{4.217694in}{0.605674in}}%
\pgfpathlineto{\pgfqpoint{4.220304in}{0.610323in}}%
\pgfpathlineto{\pgfqpoint{4.223082in}{0.620352in}}%
\pgfpathlineto{\pgfqpoint{4.225654in}{0.616002in}}%
\pgfpathlineto{\pgfqpoint{4.228331in}{0.628578in}}%
\pgfpathlineto{\pgfqpoint{4.231013in}{0.616162in}}%
\pgfpathlineto{\pgfqpoint{4.233691in}{0.607668in}}%
\pgfpathlineto{\pgfqpoint{4.236375in}{0.606170in}}%
\pgfpathlineto{\pgfqpoint{4.239084in}{0.602027in}}%
\pgfpathlineto{\pgfqpoint{4.241900in}{0.603978in}}%
\pgfpathlineto{\pgfqpoint{4.244394in}{0.611393in}}%
\pgfpathlineto{\pgfqpoint{4.247225in}{0.603766in}}%
\pgfpathlineto{\pgfqpoint{4.249767in}{0.603426in}}%
\pgfpathlineto{\pgfqpoint{4.252581in}{0.598930in}}%
\pgfpathlineto{\pgfqpoint{4.255120in}{0.599595in}}%
\pgfpathlineto{\pgfqpoint{4.257958in}{0.598726in}}%
\pgfpathlineto{\pgfqpoint{4.260477in}{0.599019in}}%
\pgfpathlineto{\pgfqpoint{4.263157in}{0.598105in}}%
\pgfpathlineto{\pgfqpoint{4.265824in}{0.597022in}}%
\pgfpathlineto{\pgfqpoint{4.268590in}{0.600529in}}%
\pgfpathlineto{\pgfqpoint{4.271187in}{0.602300in}}%
\pgfpathlineto{\pgfqpoint{4.273874in}{0.599685in}}%
\pgfpathlineto{\pgfqpoint{4.276635in}{0.599876in}}%
\pgfpathlineto{\pgfqpoint{4.279212in}{0.611213in}}%
\pgfpathlineto{\pgfqpoint{4.282000in}{0.607363in}}%
\pgfpathlineto{\pgfqpoint{4.284586in}{0.610378in}}%
\pgfpathlineto{\pgfqpoint{4.287399in}{0.598912in}}%
\pgfpathlineto{\pgfqpoint{4.289936in}{0.601853in}}%
\pgfpathlineto{\pgfqpoint{4.292786in}{0.600813in}}%
\pgfpathlineto{\pgfqpoint{4.295299in}{0.598291in}}%
\pgfpathlineto{\pgfqpoint{4.297977in}{0.604837in}}%
\pgfpathlineto{\pgfqpoint{4.300656in}{0.607329in}}%
\pgfpathlineto{\pgfqpoint{4.303357in}{0.607695in}}%
\pgfpathlineto{\pgfqpoint{4.306118in}{0.609982in}}%
\pgfpathlineto{\pgfqpoint{4.308691in}{0.610505in}}%
\pgfpathlineto{\pgfqpoint{4.311494in}{0.609185in}}%
\pgfpathlineto{\pgfqpoint{4.314032in}{0.606597in}}%
\pgfpathlineto{\pgfqpoint{4.316856in}{0.609422in}}%
\pgfpathlineto{\pgfqpoint{4.319405in}{0.612831in}}%
\pgfpathlineto{\pgfqpoint{4.322181in}{0.609386in}}%
\pgfpathlineto{\pgfqpoint{4.324760in}{0.605578in}}%
\pgfpathlineto{\pgfqpoint{4.327440in}{0.602134in}}%
\pgfpathlineto{\pgfqpoint{4.330118in}{0.605076in}}%
\pgfpathlineto{\pgfqpoint{4.332796in}{0.604445in}}%
\pgfpathlineto{\pgfqpoint{4.335463in}{0.610444in}}%
\pgfpathlineto{\pgfqpoint{4.338154in}{0.608427in}}%
\pgfpathlineto{\pgfqpoint{4.340976in}{0.609299in}}%
\pgfpathlineto{\pgfqpoint{4.343510in}{0.602263in}}%
\pgfpathlineto{\pgfqpoint{4.346263in}{0.604970in}}%
\pgfpathlineto{\pgfqpoint{4.348868in}{0.606013in}}%
\pgfpathlineto{\pgfqpoint{4.351645in}{0.605013in}}%
\pgfpathlineto{\pgfqpoint{4.354224in}{0.610055in}}%
\pgfpathlineto{\pgfqpoint{4.357014in}{0.606966in}}%
\pgfpathlineto{\pgfqpoint{4.359582in}{0.608677in}}%
\pgfpathlineto{\pgfqpoint{4.362270in}{0.610829in}}%
\pgfpathlineto{\pgfqpoint{4.364936in}{0.609808in}}%
\pgfpathlineto{\pgfqpoint{4.367646in}{0.609579in}}%
\pgfpathlineto{\pgfqpoint{4.370437in}{0.609259in}}%
\pgfpathlineto{\pgfqpoint{4.372976in}{0.607826in}}%
\pgfpathlineto{\pgfqpoint{4.375761in}{0.610745in}}%
\pgfpathlineto{\pgfqpoint{4.378329in}{0.620939in}}%
\pgfpathlineto{\pgfqpoint{4.381097in}{0.622769in}}%
\pgfpathlineto{\pgfqpoint{4.383674in}{0.613432in}}%
\pgfpathlineto{\pgfqpoint{4.386431in}{0.613187in}}%
\pgfpathlineto{\pgfqpoint{4.389044in}{0.607369in}}%
\pgfpathlineto{\pgfqpoint{4.391721in}{0.609098in}}%
\pgfpathlineto{\pgfqpoint{4.394400in}{0.611658in}}%
\pgfpathlineto{\pgfqpoint{4.397076in}{0.602592in}}%
\pgfpathlineto{\pgfqpoint{4.399745in}{0.604445in}}%
\pgfpathlineto{\pgfqpoint{4.402468in}{0.610500in}}%
\pgfpathlineto{\pgfqpoint{4.405234in}{0.609466in}}%
\pgfpathlineto{\pgfqpoint{4.407788in}{0.609141in}}%
\pgfpathlineto{\pgfqpoint{4.410587in}{0.609407in}}%
\pgfpathlineto{\pgfqpoint{4.413149in}{0.608888in}}%
\pgfpathlineto{\pgfqpoint{4.415932in}{0.605441in}}%
\pgfpathlineto{\pgfqpoint{4.418506in}{0.607128in}}%
\pgfpathlineto{\pgfqpoint{4.421292in}{0.604531in}}%
\pgfpathlineto{\pgfqpoint{4.423863in}{0.602999in}}%
\pgfpathlineto{\pgfqpoint{4.426534in}{0.607296in}}%
\pgfpathlineto{\pgfqpoint{4.429220in}{0.605663in}}%
\pgfpathlineto{\pgfqpoint{4.431901in}{0.601021in}}%
\pgfpathlineto{\pgfqpoint{4.434569in}{0.604216in}}%
\pgfpathlineto{\pgfqpoint{4.437253in}{0.602353in}}%
\pgfpathlineto{\pgfqpoint{4.440041in}{0.604932in}}%
\pgfpathlineto{\pgfqpoint{4.442611in}{0.595557in}}%
\pgfpathlineto{\pgfqpoint{4.445423in}{0.597466in}}%
\pgfpathlineto{\pgfqpoint{4.447965in}{0.601286in}}%
\pgfpathlineto{\pgfqpoint{4.450767in}{0.599687in}}%
\pgfpathlineto{\pgfqpoint{4.453312in}{0.606587in}}%
\pgfpathlineto{\pgfqpoint{4.456138in}{0.613979in}}%
\pgfpathlineto{\pgfqpoint{4.458681in}{0.605433in}}%
\pgfpathlineto{\pgfqpoint{4.461367in}{0.607737in}}%
\pgfpathlineto{\pgfqpoint{4.464029in}{0.606112in}}%
\pgfpathlineto{\pgfqpoint{4.466717in}{0.602021in}}%
\pgfpathlineto{\pgfqpoint{4.469492in}{0.603459in}}%
\pgfpathlineto{\pgfqpoint{4.472059in}{0.601373in}}%
\pgfpathlineto{\pgfqpoint{4.474861in}{0.606212in}}%
\pgfpathlineto{\pgfqpoint{4.477430in}{0.602760in}}%
\pgfpathlineto{\pgfqpoint{4.480201in}{0.600312in}}%
\pgfpathlineto{\pgfqpoint{4.482778in}{0.609148in}}%
\pgfpathlineto{\pgfqpoint{4.485581in}{0.611145in}}%
\pgfpathlineto{\pgfqpoint{4.488130in}{0.610164in}}%
\pgfpathlineto{\pgfqpoint{4.490822in}{0.611852in}}%
\pgfpathlineto{\pgfqpoint{4.493492in}{0.609359in}}%
\pgfpathlineto{\pgfqpoint{4.496167in}{0.609248in}}%
\pgfpathlineto{\pgfqpoint{4.498850in}{0.612742in}}%
\pgfpathlineto{\pgfqpoint{4.501529in}{0.608856in}}%
\pgfpathlineto{\pgfqpoint{4.504305in}{0.609480in}}%
\pgfpathlineto{\pgfqpoint{4.506893in}{0.614736in}}%
\pgfpathlineto{\pgfqpoint{4.509643in}{0.608936in}}%
\pgfpathlineto{\pgfqpoint{4.512246in}{0.616933in}}%
\pgfpathlineto{\pgfqpoint{4.515080in}{0.611950in}}%
\pgfpathlineto{\pgfqpoint{4.517598in}{0.613641in}}%
\pgfpathlineto{\pgfqpoint{4.520345in}{0.610949in}}%
\pgfpathlineto{\pgfqpoint{4.522962in}{0.612758in}}%
\pgfpathlineto{\pgfqpoint{4.525640in}{0.608827in}}%
\pgfpathlineto{\pgfqpoint{4.528307in}{0.612914in}}%
\pgfpathlineto{\pgfqpoint{4.530990in}{0.611606in}}%
\pgfpathlineto{\pgfqpoint{4.533764in}{0.608474in}}%
\pgfpathlineto{\pgfqpoint{4.536400in}{0.617431in}}%
\pgfpathlineto{\pgfqpoint{4.539144in}{0.608091in}}%
\pgfpathlineto{\pgfqpoint{4.541711in}{0.608208in}}%
\pgfpathlineto{\pgfqpoint{4.544464in}{0.607569in}}%
\pgfpathlineto{\pgfqpoint{4.547064in}{0.609609in}}%
\pgfpathlineto{\pgfqpoint{4.549822in}{0.606479in}}%
\pgfpathlineto{\pgfqpoint{4.552425in}{0.612525in}}%
\pgfpathlineto{\pgfqpoint{4.555106in}{0.610385in}}%
\pgfpathlineto{\pgfqpoint{4.557777in}{0.611035in}}%
\pgfpathlineto{\pgfqpoint{4.560448in}{0.609065in}}%
\pgfpathlineto{\pgfqpoint{4.563125in}{0.607617in}}%
\pgfpathlineto{\pgfqpoint{4.565820in}{0.609831in}}%
\pgfpathlineto{\pgfqpoint{4.568612in}{0.603673in}}%
\pgfpathlineto{\pgfqpoint{4.571171in}{0.598174in}}%
\pgfpathlineto{\pgfqpoint{4.573947in}{0.600721in}}%
\pgfpathlineto{\pgfqpoint{4.576531in}{0.600722in}}%
\pgfpathlineto{\pgfqpoint{4.579305in}{0.607209in}}%
\pgfpathlineto{\pgfqpoint{4.581888in}{0.606785in}}%
\pgfpathlineto{\pgfqpoint{4.584672in}{0.610152in}}%
\pgfpathlineto{\pgfqpoint{4.587244in}{0.603540in}}%
\pgfpathlineto{\pgfqpoint{4.589920in}{0.602492in}}%
\pgfpathlineto{\pgfqpoint{4.592589in}{0.605475in}}%
\pgfpathlineto{\pgfqpoint{4.595281in}{0.605577in}}%
\pgfpathlineto{\pgfqpoint{4.597951in}{0.606176in}}%
\pgfpathlineto{\pgfqpoint{4.600633in}{0.606407in}}%
\pgfpathlineto{\pgfqpoint{4.603430in}{0.608819in}}%
\pgfpathlineto{\pgfqpoint{4.605990in}{0.609463in}}%
\pgfpathlineto{\pgfqpoint{4.608808in}{0.608084in}}%
\pgfpathlineto{\pgfqpoint{4.611350in}{0.609651in}}%
\pgfpathlineto{\pgfqpoint{4.614134in}{0.609070in}}%
\pgfpathlineto{\pgfqpoint{4.616702in}{0.610462in}}%
\pgfpathlineto{\pgfqpoint{4.619529in}{0.608525in}}%
\pgfpathlineto{\pgfqpoint{4.622056in}{0.605088in}}%
\pgfpathlineto{\pgfqpoint{4.624741in}{0.612513in}}%
\pgfpathlineto{\pgfqpoint{4.627411in}{0.610917in}}%
\pgfpathlineto{\pgfqpoint{4.630096in}{0.612259in}}%
\pgfpathlineto{\pgfqpoint{4.632902in}{0.613080in}}%
\pgfpathlineto{\pgfqpoint{4.635445in}{0.611676in}}%
\pgfpathlineto{\pgfqpoint{4.638204in}{0.608895in}}%
\pgfpathlineto{\pgfqpoint{4.640809in}{0.610125in}}%
\pgfpathlineto{\pgfqpoint{4.643628in}{0.609218in}}%
\pgfpathlineto{\pgfqpoint{4.646169in}{0.610439in}}%
\pgfpathlineto{\pgfqpoint{4.648922in}{0.610353in}}%
\pgfpathlineto{\pgfqpoint{4.651524in}{0.611573in}}%
\pgfpathlineto{\pgfqpoint{4.654203in}{0.609254in}}%
\pgfpathlineto{\pgfqpoint{4.656873in}{0.610732in}}%
\pgfpathlineto{\pgfqpoint{4.659590in}{0.613808in}}%
\pgfpathlineto{\pgfqpoint{4.662237in}{0.608119in}}%
\pgfpathlineto{\pgfqpoint{4.664923in}{0.604345in}}%
\pgfpathlineto{\pgfqpoint{4.667764in}{0.609135in}}%
\pgfpathlineto{\pgfqpoint{4.670261in}{0.604731in}}%
\pgfpathlineto{\pgfqpoint{4.673068in}{0.607344in}}%
\pgfpathlineto{\pgfqpoint{4.675619in}{0.618872in}}%
\pgfpathlineto{\pgfqpoint{4.678448in}{0.629388in}}%
\pgfpathlineto{\pgfqpoint{4.680988in}{0.619462in}}%
\pgfpathlineto{\pgfqpoint{4.683799in}{0.614729in}}%
\pgfpathlineto{\pgfqpoint{4.686337in}{0.647468in}}%
\pgfpathlineto{\pgfqpoint{4.689051in}{0.643884in}}%
\pgfpathlineto{\pgfqpoint{4.691694in}{0.638698in}}%
\pgfpathlineto{\pgfqpoint{4.694381in}{0.626124in}}%
\pgfpathlineto{\pgfqpoint{4.697170in}{0.636137in}}%
\pgfpathlineto{\pgfqpoint{4.699734in}{0.627732in}}%
\pgfpathlineto{\pgfqpoint{4.702517in}{0.621832in}}%
\pgfpathlineto{\pgfqpoint{4.705094in}{0.620421in}}%
\pgfpathlineto{\pgfqpoint{4.707824in}{0.618442in}}%
\pgfpathlineto{\pgfqpoint{4.710437in}{0.617028in}}%
\pgfpathlineto{\pgfqpoint{4.713275in}{0.609477in}}%
\pgfpathlineto{\pgfqpoint{4.715806in}{0.609376in}}%
\pgfpathlineto{\pgfqpoint{4.718486in}{0.611784in}}%
\pgfpathlineto{\pgfqpoint{4.721160in}{0.612039in}}%
\pgfpathlineto{\pgfqpoint{4.723873in}{0.613292in}}%
\pgfpathlineto{\pgfqpoint{4.726508in}{0.611103in}}%
\pgfpathlineto{\pgfqpoint{4.729233in}{0.608598in}}%
\pgfpathlineto{\pgfqpoint{4.731901in}{0.609138in}}%
\pgfpathlineto{\pgfqpoint{4.734552in}{0.609960in}}%
\pgfpathlineto{\pgfqpoint{4.737348in}{0.608375in}}%
\pgfpathlineto{\pgfqpoint{4.739912in}{0.611616in}}%
\pgfpathlineto{\pgfqpoint{4.742696in}{0.607936in}}%
\pgfpathlineto{\pgfqpoint{4.745256in}{0.605833in}}%
\pgfpathlineto{\pgfqpoint{4.748081in}{0.609669in}}%
\pgfpathlineto{\pgfqpoint{4.750627in}{0.607581in}}%
\pgfpathlineto{\pgfqpoint{4.753298in}{0.612159in}}%
\pgfpathlineto{\pgfqpoint{4.755983in}{0.608011in}}%
\pgfpathlineto{\pgfqpoint{4.758653in}{0.610663in}}%
\pgfpathlineto{\pgfqpoint{4.761337in}{0.613881in}}%
\pgfpathlineto{\pgfqpoint{4.764018in}{0.605464in}}%
\pgfpathlineto{\pgfqpoint{4.766783in}{0.603107in}}%
\pgfpathlineto{\pgfqpoint{4.769367in}{0.602334in}}%
\pgfpathlineto{\pgfqpoint{4.772198in}{0.604857in}}%
\pgfpathlineto{\pgfqpoint{4.774732in}{0.609633in}}%
\pgfpathlineto{\pgfqpoint{4.777535in}{0.606997in}}%
\pgfpathlineto{\pgfqpoint{4.780083in}{0.604880in}}%
\pgfpathlineto{\pgfqpoint{4.782872in}{0.607109in}}%
\pgfpathlineto{\pgfqpoint{4.785445in}{0.609256in}}%
\pgfpathlineto{\pgfqpoint{4.788116in}{0.605930in}}%
\pgfpathlineto{\pgfqpoint{4.790798in}{0.606488in}}%
\pgfpathlineto{\pgfqpoint{4.793512in}{0.609802in}}%
\pgfpathlineto{\pgfqpoint{4.796274in}{0.608725in}}%
\pgfpathlineto{\pgfqpoint{4.798830in}{0.604603in}}%
\pgfpathlineto{\pgfqpoint{4.801586in}{0.602377in}}%
\pgfpathlineto{\pgfqpoint{4.804193in}{0.600840in}}%
\pgfpathlineto{\pgfqpoint{4.807017in}{0.600330in}}%
\pgfpathlineto{\pgfqpoint{4.809538in}{0.602080in}}%
\pgfpathlineto{\pgfqpoint{4.812377in}{0.603799in}}%
\pgfpathlineto{\pgfqpoint{4.814907in}{0.600868in}}%
\pgfpathlineto{\pgfqpoint{4.817587in}{0.597615in}}%
\pgfpathlineto{\pgfqpoint{4.820265in}{0.603930in}}%
\pgfpathlineto{\pgfqpoint{4.822945in}{0.596023in}}%
\pgfpathlineto{\pgfqpoint{4.825619in}{0.595041in}}%
\pgfpathlineto{\pgfqpoint{4.828291in}{0.595041in}}%
\pgfpathlineto{\pgfqpoint{4.831045in}{0.595041in}}%
\pgfpathlineto{\pgfqpoint{4.833657in}{0.601089in}}%
\pgfpathlineto{\pgfqpoint{4.837992in}{0.603882in}}%
\pgfpathlineto{\pgfqpoint{4.839922in}{0.601737in}}%
\pgfpathlineto{\pgfqpoint{4.842380in}{0.601974in}}%
\pgfpathlineto{\pgfqpoint{4.844361in}{0.606674in}}%
\pgfpathlineto{\pgfqpoint{4.847127in}{0.602715in}}%
\pgfpathlineto{\pgfqpoint{4.849715in}{0.607094in}}%
\pgfpathlineto{\pgfqpoint{4.852404in}{0.608857in}}%
\pgfpathlineto{\pgfqpoint{4.855070in}{0.606097in}}%
\pgfpathlineto{\pgfqpoint{4.857807in}{0.604954in}}%
\pgfpathlineto{\pgfqpoint{4.860544in}{0.605643in}}%
\pgfpathlineto{\pgfqpoint{4.863116in}{0.606984in}}%
\pgfpathlineto{\pgfqpoint{4.865910in}{0.607868in}}%
\pgfpathlineto{\pgfqpoint{4.868474in}{0.605574in}}%
\pgfpathlineto{\pgfqpoint{4.871209in}{0.605874in}}%
\pgfpathlineto{\pgfqpoint{4.873832in}{0.604905in}}%
\pgfpathlineto{\pgfqpoint{4.876636in}{0.606441in}}%
\pgfpathlineto{\pgfqpoint{4.879180in}{0.603865in}}%
\pgfpathlineto{\pgfqpoint{4.881864in}{0.606872in}}%
\pgfpathlineto{\pgfqpoint{4.884540in}{0.605584in}}%
\pgfpathlineto{\pgfqpoint{4.887211in}{0.604809in}}%
\pgfpathlineto{\pgfqpoint{4.889902in}{0.605621in}}%
\pgfpathlineto{\pgfqpoint{4.892611in}{0.603502in}}%
\pgfpathlineto{\pgfqpoint{4.895399in}{0.605055in}}%
\pgfpathlineto{\pgfqpoint{4.897938in}{0.604768in}}%
\pgfpathlineto{\pgfqpoint{4.900712in}{0.606708in}}%
\pgfpathlineto{\pgfqpoint{4.903295in}{0.608413in}}%
\pgfpathlineto{\pgfqpoint{4.906096in}{0.608999in}}%
\pgfpathlineto{\pgfqpoint{4.908648in}{0.598701in}}%
\pgfpathlineto{\pgfqpoint{4.911435in}{0.599101in}}%
\pgfpathlineto{\pgfqpoint{4.914009in}{0.602405in}}%
\pgfpathlineto{\pgfqpoint{4.916681in}{0.601723in}}%
\pgfpathlineto{\pgfqpoint{4.919352in}{0.600821in}}%
\pgfpathlineto{\pgfqpoint{4.922041in}{0.601827in}}%
\pgfpathlineto{\pgfqpoint{4.924708in}{0.595925in}}%
\pgfpathlineto{\pgfqpoint{4.927400in}{0.604043in}}%
\pgfpathlineto{\pgfqpoint{4.930170in}{0.599838in}}%
\pgfpathlineto{\pgfqpoint{4.932742in}{0.602928in}}%
\pgfpathlineto{\pgfqpoint{4.935515in}{0.607303in}}%
\pgfpathlineto{\pgfqpoint{4.938112in}{0.606913in}}%
\pgfpathlineto{\pgfqpoint{4.940881in}{0.609854in}}%
\pgfpathlineto{\pgfqpoint{4.943466in}{0.607822in}}%
\pgfpathlineto{\pgfqpoint{4.946151in}{0.608559in}}%
\pgfpathlineto{\pgfqpoint{4.948827in}{0.609757in}}%
\pgfpathlineto{\pgfqpoint{4.951504in}{0.610021in}}%
\pgfpathlineto{\pgfqpoint{4.954182in}{0.606248in}}%
\pgfpathlineto{\pgfqpoint{4.956862in}{0.607878in}}%
\pgfpathlineto{\pgfqpoint{4.959689in}{0.603204in}}%
\pgfpathlineto{\pgfqpoint{4.962219in}{0.603183in}}%
\pgfpathlineto{\pgfqpoint{4.965002in}{0.605299in}}%
\pgfpathlineto{\pgfqpoint{4.967575in}{0.608881in}}%
\pgfpathlineto{\pgfqpoint{4.970314in}{0.609092in}}%
\pgfpathlineto{\pgfqpoint{4.972933in}{0.611354in}}%
\pgfpathlineto{\pgfqpoint{4.975703in}{0.609250in}}%
\pgfpathlineto{\pgfqpoint{4.978287in}{0.612574in}}%
\pgfpathlineto{\pgfqpoint{4.980967in}{0.619238in}}%
\pgfpathlineto{\pgfqpoint{4.983637in}{0.609602in}}%
\pgfpathlineto{\pgfqpoint{4.986325in}{0.612123in}}%
\pgfpathlineto{\pgfqpoint{4.989001in}{0.610521in}}%
\pgfpathlineto{\pgfqpoint{4.991683in}{0.605310in}}%
\pgfpathlineto{\pgfqpoint{4.994390in}{0.609827in}}%
\pgfpathlineto{\pgfqpoint{4.997028in}{0.605488in}}%
\pgfpathlineto{\pgfqpoint{4.999780in}{0.607580in}}%
\pgfpathlineto{\pgfqpoint{5.002384in}{0.615064in}}%
\pgfpathlineto{\pgfqpoint{5.005178in}{0.611486in}}%
\pgfpathlineto{\pgfqpoint{5.007751in}{0.619991in}}%
\pgfpathlineto{\pgfqpoint{5.010562in}{0.615233in}}%
\pgfpathlineto{\pgfqpoint{5.013104in}{0.612999in}}%
\pgfpathlineto{\pgfqpoint{5.015820in}{0.612177in}}%
\pgfpathlineto{\pgfqpoint{5.018466in}{0.605770in}}%
\pgfpathlineto{\pgfqpoint{5.021147in}{0.605094in}}%
\pgfpathlineto{\pgfqpoint{5.023927in}{0.612439in}}%
\pgfpathlineto{\pgfqpoint{5.026501in}{0.613155in}}%
\pgfpathlineto{\pgfqpoint{5.029275in}{0.614608in}}%
\pgfpathlineto{\pgfqpoint{5.031849in}{0.614360in}}%
\pgfpathlineto{\pgfqpoint{5.034649in}{0.620328in}}%
\pgfpathlineto{\pgfqpoint{5.037214in}{0.614881in}}%
\pgfpathlineto{\pgfqpoint{5.039962in}{0.607696in}}%
\pgfpathlineto{\pgfqpoint{5.042572in}{0.607451in}}%
\pgfpathlineto{\pgfqpoint{5.045249in}{0.615308in}}%
\pgfpathlineto{\pgfqpoint{5.047924in}{0.608068in}}%
\pgfpathlineto{\pgfqpoint{5.050606in}{0.610359in}}%
\pgfpathlineto{\pgfqpoint{5.053284in}{0.605785in}}%
\pgfpathlineto{\pgfqpoint{5.055952in}{0.603000in}}%
\pgfpathlineto{\pgfqpoint{5.058711in}{0.610555in}}%
\pgfpathlineto{\pgfqpoint{5.061315in}{0.611468in}}%
\pgfpathlineto{\pgfqpoint{5.064144in}{0.609768in}}%
\pgfpathlineto{\pgfqpoint{5.066677in}{0.611302in}}%
\pgfpathlineto{\pgfqpoint{5.069463in}{0.606677in}}%
\pgfpathlineto{\pgfqpoint{5.072030in}{0.607357in}}%
\pgfpathlineto{\pgfqpoint{5.074851in}{0.601673in}}%
\pgfpathlineto{\pgfqpoint{5.077390in}{0.612860in}}%
\pgfpathlineto{\pgfqpoint{5.080067in}{0.611369in}}%
\pgfpathlineto{\pgfqpoint{5.082746in}{0.609947in}}%
\pgfpathlineto{\pgfqpoint{5.085426in}{0.606040in}}%
\pgfpathlineto{\pgfqpoint{5.088103in}{0.607716in}}%
\pgfpathlineto{\pgfqpoint{5.090788in}{0.607740in}}%
\pgfpathlineto{\pgfqpoint{5.093579in}{0.610807in}}%
\pgfpathlineto{\pgfqpoint{5.096142in}{0.609962in}}%
\pgfpathlineto{\pgfqpoint{5.098948in}{0.608183in}}%
\pgfpathlineto{\pgfqpoint{5.101496in}{0.606459in}}%
\pgfpathlineto{\pgfqpoint{5.104312in}{0.608900in}}%
\pgfpathlineto{\pgfqpoint{5.106842in}{0.606834in}}%
\pgfpathlineto{\pgfqpoint{5.109530in}{0.608872in}}%
\pgfpathlineto{\pgfqpoint{5.112209in}{0.605270in}}%
\pgfpathlineto{\pgfqpoint{5.114887in}{0.606798in}}%
\pgfpathlineto{\pgfqpoint{5.117550in}{0.610476in}}%
\pgfpathlineto{\pgfqpoint{5.120243in}{0.606326in}}%
\pgfpathlineto{\pgfqpoint{5.123042in}{0.607915in}}%
\pgfpathlineto{\pgfqpoint{5.125599in}{0.616667in}}%
\pgfpathlineto{\pgfqpoint{5.128421in}{0.610329in}}%
\pgfpathlineto{\pgfqpoint{5.130953in}{0.614842in}}%
\pgfpathlineto{\pgfqpoint{5.133716in}{0.609585in}}%
\pgfpathlineto{\pgfqpoint{5.136311in}{0.608858in}}%
\pgfpathlineto{\pgfqpoint{5.139072in}{0.612741in}}%
\pgfpathlineto{\pgfqpoint{5.141660in}{0.610801in}}%
\pgfpathlineto{\pgfqpoint{5.144349in}{0.610432in}}%
\pgfpathlineto{\pgfqpoint{5.147029in}{0.610421in}}%
\pgfpathlineto{\pgfqpoint{5.149734in}{0.612075in}}%
\pgfpathlineto{\pgfqpoint{5.152382in}{0.604218in}}%
\pgfpathlineto{\pgfqpoint{5.155059in}{0.603129in}}%
\pgfpathlineto{\pgfqpoint{5.157815in}{0.603755in}}%
\pgfpathlineto{\pgfqpoint{5.160420in}{0.612033in}}%
\pgfpathlineto{\pgfqpoint{5.163243in}{0.611974in}}%
\pgfpathlineto{\pgfqpoint{5.165775in}{0.612842in}}%
\pgfpathlineto{\pgfqpoint{5.168591in}{0.610146in}}%
\pgfpathlineto{\pgfqpoint{5.171133in}{0.612295in}}%
\pgfpathlineto{\pgfqpoint{5.173925in}{0.615781in}}%
\pgfpathlineto{\pgfqpoint{5.176477in}{0.611474in}}%
\pgfpathlineto{\pgfqpoint{5.179188in}{0.612798in}}%
\pgfpathlineto{\pgfqpoint{5.181848in}{0.635239in}}%
\pgfpathlineto{\pgfqpoint{5.184522in}{0.651032in}}%
\pgfpathlineto{\pgfqpoint{5.187294in}{0.637274in}}%
\pgfpathlineto{\pgfqpoint{5.189880in}{0.634999in}}%
\pgfpathlineto{\pgfqpoint{5.192680in}{0.627819in}}%
\pgfpathlineto{\pgfqpoint{5.195239in}{0.621471in}}%
\pgfpathlineto{\pgfqpoint{5.198008in}{0.621567in}}%
\pgfpathlineto{\pgfqpoint{5.200594in}{0.616380in}}%
\pgfpathlineto{\pgfqpoint{5.203388in}{0.620015in}}%
\pgfpathlineto{\pgfqpoint{5.205952in}{0.614565in}}%
\pgfpathlineto{\pgfqpoint{5.208630in}{0.613498in}}%
\pgfpathlineto{\pgfqpoint{5.211299in}{0.610325in}}%
\pgfpathlineto{\pgfqpoint{5.214027in}{0.606457in}}%
\pgfpathlineto{\pgfqpoint{5.216667in}{0.606620in}}%
\pgfpathlineto{\pgfqpoint{5.219345in}{0.607922in}}%
\pgfpathlineto{\pgfqpoint{5.222151in}{0.610823in}}%
\pgfpathlineto{\pgfqpoint{5.224695in}{0.604438in}}%
\pgfpathlineto{\pgfqpoint{5.227470in}{0.613890in}}%
\pgfpathlineto{\pgfqpoint{5.230059in}{0.607707in}}%
\pgfpathlineto{\pgfqpoint{5.232855in}{0.612444in}}%
\pgfpathlineto{\pgfqpoint{5.235409in}{0.607924in}}%
\pgfpathlineto{\pgfqpoint{5.238173in}{0.610213in}}%
\pgfpathlineto{\pgfqpoint{5.240777in}{0.615429in}}%
\pgfpathlineto{\pgfqpoint{5.243445in}{0.611610in}}%
\pgfpathlineto{\pgfqpoint{5.246130in}{0.604793in}}%
\pgfpathlineto{\pgfqpoint{5.248816in}{0.607816in}}%
\pgfpathlineto{\pgfqpoint{5.251590in}{0.603687in}}%
\pgfpathlineto{\pgfqpoint{5.254236in}{0.603210in}}%
\pgfpathlineto{\pgfqpoint{5.256973in}{0.608097in}}%
\pgfpathlineto{\pgfqpoint{5.259511in}{0.607106in}}%
\pgfpathlineto{\pgfqpoint{5.262264in}{0.604505in}}%
\pgfpathlineto{\pgfqpoint{5.264876in}{0.602442in}}%
\pgfpathlineto{\pgfqpoint{5.267691in}{0.608514in}}%
\pgfpathlineto{\pgfqpoint{5.270238in}{0.603397in}}%
\pgfpathlineto{\pgfqpoint{5.272913in}{0.610054in}}%
\pgfpathlineto{\pgfqpoint{5.275589in}{0.609984in}}%
\pgfpathlineto{\pgfqpoint{5.278322in}{0.603028in}}%
\pgfpathlineto{\pgfqpoint{5.280947in}{0.604396in}}%
\pgfpathlineto{\pgfqpoint{5.283631in}{0.597925in}}%
\pgfpathlineto{\pgfqpoint{5.286436in}{0.603341in}}%
\pgfpathlineto{\pgfqpoint{5.288984in}{0.598560in}}%
\pgfpathlineto{\pgfqpoint{5.291794in}{0.603452in}}%
\pgfpathlineto{\pgfqpoint{5.294339in}{0.603820in}}%
\pgfpathlineto{\pgfqpoint{5.297140in}{0.603763in}}%
\pgfpathlineto{\pgfqpoint{5.299696in}{0.602475in}}%
\pgfpathlineto{\pgfqpoint{5.302443in}{0.601936in}}%
\pgfpathlineto{\pgfqpoint{5.305054in}{0.597789in}}%
\pgfpathlineto{\pgfqpoint{5.307731in}{0.601620in}}%
\pgfpathlineto{\pgfqpoint{5.310411in}{0.604499in}}%
\pgfpathlineto{\pgfqpoint{5.313089in}{0.600651in}}%
\pgfpathlineto{\pgfqpoint{5.315754in}{0.604924in}}%
\pgfpathlineto{\pgfqpoint{5.318430in}{0.607175in}}%
\pgfpathlineto{\pgfqpoint{5.321256in}{0.607472in}}%
\pgfpathlineto{\pgfqpoint{5.323802in}{0.605976in}}%
\pgfpathlineto{\pgfqpoint{5.326564in}{0.601377in}}%
\pgfpathlineto{\pgfqpoint{5.329159in}{0.603849in}}%
\pgfpathlineto{\pgfqpoint{5.331973in}{0.614269in}}%
\pgfpathlineto{\pgfqpoint{5.334510in}{0.612074in}}%
\pgfpathlineto{\pgfqpoint{5.337353in}{0.609010in}}%
\pgfpathlineto{\pgfqpoint{5.339872in}{0.607394in}}%
\pgfpathlineto{\pgfqpoint{5.342549in}{0.600071in}}%
\pgfpathlineto{\pgfqpoint{5.345224in}{0.598970in}}%
\pgfpathlineto{\pgfqpoint{5.347905in}{0.599979in}}%
\pgfpathlineto{\pgfqpoint{5.350723in}{0.602536in}}%
\pgfpathlineto{\pgfqpoint{5.353262in}{0.605203in}}%
\pgfpathlineto{\pgfqpoint{5.356056in}{0.606043in}}%
\pgfpathlineto{\pgfqpoint{5.358612in}{0.609178in}}%
\pgfpathlineto{\pgfqpoint{5.361370in}{0.611442in}}%
\pgfpathlineto{\pgfqpoint{5.363966in}{0.614292in}}%
\pgfpathlineto{\pgfqpoint{5.366727in}{0.614260in}}%
\pgfpathlineto{\pgfqpoint{5.369335in}{0.613937in}}%
\pgfpathlineto{\pgfqpoint{5.372013in}{0.614829in}}%
\pgfpathlineto{\pgfqpoint{5.374692in}{0.613775in}}%
\pgfpathlineto{\pgfqpoint{5.377370in}{0.611910in}}%
\pgfpathlineto{\pgfqpoint{5.380048in}{0.616313in}}%
\pgfpathlineto{\pgfqpoint{5.382725in}{0.614922in}}%
\pgfpathlineto{\pgfqpoint{5.385550in}{0.611467in}}%
\pgfpathlineto{\pgfqpoint{5.388083in}{0.627069in}}%
\pgfpathlineto{\pgfqpoint{5.390900in}{0.613769in}}%
\pgfpathlineto{\pgfqpoint{5.393441in}{0.615522in}}%
\pgfpathlineto{\pgfqpoint{5.396219in}{0.615381in}}%
\pgfpathlineto{\pgfqpoint{5.398784in}{0.614775in}}%
\pgfpathlineto{\pgfqpoint{5.401576in}{0.611596in}}%
\pgfpathlineto{\pgfqpoint{5.404154in}{0.611912in}}%
\pgfpathlineto{\pgfqpoint{5.406832in}{0.607794in}}%
\pgfpathlineto{\pgfqpoint{5.409507in}{0.613328in}}%
\pgfpathlineto{\pgfqpoint{5.412190in}{0.611070in}}%
\pgfpathlineto{\pgfqpoint{5.414954in}{0.612665in}}%
\pgfpathlineto{\pgfqpoint{5.417547in}{0.611184in}}%
\pgfpathlineto{\pgfqpoint{5.420304in}{0.606399in}}%
\pgfpathlineto{\pgfqpoint{5.422897in}{0.608505in}}%
\pgfpathlineto{\pgfqpoint{5.425661in}{0.609056in}}%
\pgfpathlineto{\pgfqpoint{5.428259in}{0.611831in}}%
\pgfpathlineto{\pgfqpoint{5.431015in}{0.609778in}}%
\pgfpathlineto{\pgfqpoint{5.433616in}{0.609529in}}%
\pgfpathlineto{\pgfqpoint{5.436295in}{0.610797in}}%
\pgfpathlineto{\pgfqpoint{5.438974in}{0.612305in}}%
\pgfpathlineto{\pgfqpoint{5.441698in}{0.609363in}}%
\pgfpathlineto{\pgfqpoint{5.444328in}{0.605289in}}%
\pgfpathlineto{\pgfqpoint{5.447021in}{0.604961in}}%
\pgfpathlineto{\pgfqpoint{5.449769in}{0.602358in}}%
\pgfpathlineto{\pgfqpoint{5.452365in}{0.605567in}}%
\pgfpathlineto{\pgfqpoint{5.455168in}{0.604548in}}%
\pgfpathlineto{\pgfqpoint{5.457721in}{0.609701in}}%
\pgfpathlineto{\pgfqpoint{5.460489in}{0.606796in}}%
\pgfpathlineto{\pgfqpoint{5.463079in}{0.607068in}}%
\pgfpathlineto{\pgfqpoint{5.465888in}{0.603468in}}%
\pgfpathlineto{\pgfqpoint{5.468425in}{0.608139in}}%
\pgfpathlineto{\pgfqpoint{5.471113in}{0.611946in}}%
\pgfpathlineto{\pgfqpoint{5.473792in}{0.615461in}}%
\pgfpathlineto{\pgfqpoint{5.476458in}{0.613527in}}%
\pgfpathlineto{\pgfqpoint{5.479152in}{0.612816in}}%
\pgfpathlineto{\pgfqpoint{5.481825in}{0.614490in}}%
\pgfpathlineto{\pgfqpoint{5.484641in}{0.612981in}}%
\pgfpathlineto{\pgfqpoint{5.487176in}{0.615927in}}%
\pgfpathlineto{\pgfqpoint{5.490000in}{0.613008in}}%
\pgfpathlineto{\pgfqpoint{5.492541in}{0.616033in}}%
\pgfpathlineto{\pgfqpoint{5.495346in}{0.614876in}}%
\pgfpathlineto{\pgfqpoint{5.497898in}{0.607282in}}%
\pgfpathlineto{\pgfqpoint{5.500687in}{0.613751in}}%
\pgfpathlineto{\pgfqpoint{5.503255in}{0.612732in}}%
\pgfpathlineto{\pgfqpoint{5.505933in}{0.616675in}}%
\pgfpathlineto{\pgfqpoint{5.508612in}{0.611814in}}%
\pgfpathlineto{\pgfqpoint{5.511290in}{0.609970in}}%
\pgfpathlineto{\pgfqpoint{5.514080in}{0.610160in}}%
\pgfpathlineto{\pgfqpoint{5.516646in}{0.614452in}}%
\pgfpathlineto{\pgfqpoint{5.519433in}{0.617303in}}%
\pgfpathlineto{\pgfqpoint{5.522003in}{0.611916in}}%
\pgfpathlineto{\pgfqpoint{5.524756in}{0.612028in}}%
\pgfpathlineto{\pgfqpoint{5.527360in}{0.609679in}}%
\pgfpathlineto{\pgfqpoint{5.530148in}{0.611057in}}%
\pgfpathlineto{\pgfqpoint{5.532717in}{0.610665in}}%
\pgfpathlineto{\pgfqpoint{5.535395in}{0.609350in}}%
\pgfpathlineto{\pgfqpoint{5.538074in}{0.610345in}}%
\pgfpathlineto{\pgfqpoint{5.540750in}{0.608856in}}%
\pgfpathlineto{\pgfqpoint{5.543421in}{0.613206in}}%
\pgfpathlineto{\pgfqpoint{5.546110in}{0.609661in}}%
\pgfpathlineto{\pgfqpoint{5.548921in}{0.607274in}}%
\pgfpathlineto{\pgfqpoint{5.551457in}{0.610321in}}%
\pgfpathlineto{\pgfqpoint{5.554198in}{0.605814in}}%
\pgfpathlineto{\pgfqpoint{5.556822in}{0.607991in}}%
\pgfpathlineto{\pgfqpoint{5.559612in}{0.608212in}}%
\pgfpathlineto{\pgfqpoint{5.562180in}{0.609401in}}%
\pgfpathlineto{\pgfqpoint{5.564940in}{0.607154in}}%
\pgfpathlineto{\pgfqpoint{5.567536in}{0.614766in}}%
\pgfpathlineto{\pgfqpoint{5.570215in}{0.612902in}}%
\pgfpathlineto{\pgfqpoint{5.572893in}{0.612536in}}%
\pgfpathlineto{\pgfqpoint{5.575596in}{0.608510in}}%
\pgfpathlineto{\pgfqpoint{5.578342in}{0.608383in}}%
\pgfpathlineto{\pgfqpoint{5.580914in}{0.612557in}}%
\pgfpathlineto{\pgfqpoint{5.583709in}{0.617728in}}%
\pgfpathlineto{\pgfqpoint{5.586269in}{0.608619in}}%
\pgfpathlineto{\pgfqpoint{5.589040in}{0.608595in}}%
\pgfpathlineto{\pgfqpoint{5.591641in}{0.610964in}}%
\pgfpathlineto{\pgfqpoint{5.594368in}{0.610914in}}%
\pgfpathlineto{\pgfqpoint{5.596999in}{0.609228in}}%
\pgfpathlineto{\pgfqpoint{5.599674in}{0.608319in}}%
\pgfpathlineto{\pgfqpoint{5.602352in}{0.610382in}}%
\pgfpathlineto{\pgfqpoint{5.605073in}{0.610706in}}%
\pgfpathlineto{\pgfqpoint{5.607698in}{0.609241in}}%
\pgfpathlineto{\pgfqpoint{5.610389in}{0.607688in}}%
\pgfpathlineto{\pgfqpoint{5.613235in}{0.610991in}}%
\pgfpathlineto{\pgfqpoint{5.615743in}{0.610263in}}%
\pgfpathlineto{\pgfqpoint{5.618526in}{0.605432in}}%
\pgfpathlineto{\pgfqpoint{5.621102in}{0.609501in}}%
\pgfpathlineto{\pgfqpoint{5.623868in}{0.614189in}}%
\pgfpathlineto{\pgfqpoint{5.626460in}{0.610070in}}%
\pgfpathlineto{\pgfqpoint{5.629232in}{0.611605in}}%
\pgfpathlineto{\pgfqpoint{5.631815in}{0.608414in}}%
\pgfpathlineto{\pgfqpoint{5.634496in}{0.609107in}}%
\pgfpathlineto{\pgfqpoint{5.637172in}{0.614008in}}%
\pgfpathlineto{\pgfqpoint{5.639852in}{0.609313in}}%
\pgfpathlineto{\pgfqpoint{5.642518in}{0.617925in}}%
\pgfpathlineto{\pgfqpoint{5.645243in}{0.609353in}}%
\pgfpathlineto{\pgfqpoint{5.648008in}{0.615646in}}%
\pgfpathlineto{\pgfqpoint{5.650563in}{0.616831in}}%
\pgfpathlineto{\pgfqpoint{5.653376in}{0.617947in}}%
\pgfpathlineto{\pgfqpoint{5.655919in}{0.611151in}}%
\pgfpathlineto{\pgfqpoint{5.658723in}{0.611290in}}%
\pgfpathlineto{\pgfqpoint{5.661273in}{0.606724in}}%
\pgfpathlineto{\pgfqpoint{5.664099in}{0.605073in}}%
\pgfpathlineto{\pgfqpoint{5.666632in}{0.609677in}}%
\pgfpathlineto{\pgfqpoint{5.669313in}{0.612362in}}%
\pgfpathlineto{\pgfqpoint{5.671991in}{0.605256in}}%
\pgfpathlineto{\pgfqpoint{5.674667in}{0.604751in}}%
\pgfpathlineto{\pgfqpoint{5.677486in}{0.604468in}}%
\pgfpathlineto{\pgfqpoint{5.680027in}{0.605455in}}%
\pgfpathlineto{\pgfqpoint{5.682836in}{0.604724in}}%
\pgfpathlineto{\pgfqpoint{5.685385in}{0.608174in}}%
\pgfpathlineto{\pgfqpoint{5.688159in}{0.613727in}}%
\pgfpathlineto{\pgfqpoint{5.690730in}{0.608964in}}%
\pgfpathlineto{\pgfqpoint{5.693473in}{0.609996in}}%
\pgfpathlineto{\pgfqpoint{5.696101in}{0.614511in}}%
\pgfpathlineto{\pgfqpoint{5.698775in}{0.612232in}}%
\pgfpathlineto{\pgfqpoint{5.701453in}{0.610687in}}%
\pgfpathlineto{\pgfqpoint{5.704130in}{0.600765in}}%
\pgfpathlineto{\pgfqpoint{5.706800in}{0.611791in}}%
\pgfpathlineto{\pgfqpoint{5.709490in}{0.635939in}}%
\pgfpathlineto{\pgfqpoint{5.712291in}{0.656693in}}%
\pgfpathlineto{\pgfqpoint{5.714834in}{0.643827in}}%
\pgfpathlineto{\pgfqpoint{5.717671in}{0.647468in}}%
\pgfpathlineto{\pgfqpoint{5.720201in}{0.659916in}}%
\pgfpathlineto{\pgfqpoint{5.722950in}{0.646091in}}%
\pgfpathlineto{\pgfqpoint{5.725548in}{0.640231in}}%
\pgfpathlineto{\pgfqpoint{5.728339in}{0.628866in}}%
\pgfpathlineto{\pgfqpoint{5.730919in}{0.629884in}}%
\pgfpathlineto{\pgfqpoint{5.733594in}{0.624074in}}%
\pgfpathlineto{\pgfqpoint{5.736276in}{0.624474in}}%
\pgfpathlineto{\pgfqpoint{5.738974in}{0.630126in}}%
\pgfpathlineto{\pgfqpoint{5.741745in}{0.629575in}}%
\pgfpathlineto{\pgfqpoint{5.744310in}{0.623994in}}%
\pgfpathlineto{\pgfqpoint{5.744310in}{0.413320in}}%
\pgfpathlineto{\pgfqpoint{5.744310in}{0.413320in}}%
\pgfpathlineto{\pgfqpoint{5.741745in}{0.413320in}}%
\pgfpathlineto{\pgfqpoint{5.738974in}{0.413320in}}%
\pgfpathlineto{\pgfqpoint{5.736276in}{0.413320in}}%
\pgfpathlineto{\pgfqpoint{5.733594in}{0.413320in}}%
\pgfpathlineto{\pgfqpoint{5.730919in}{0.413320in}}%
\pgfpathlineto{\pgfqpoint{5.728339in}{0.413320in}}%
\pgfpathlineto{\pgfqpoint{5.725548in}{0.413320in}}%
\pgfpathlineto{\pgfqpoint{5.722950in}{0.413320in}}%
\pgfpathlineto{\pgfqpoint{5.720201in}{0.413320in}}%
\pgfpathlineto{\pgfqpoint{5.717671in}{0.413320in}}%
\pgfpathlineto{\pgfqpoint{5.714834in}{0.413320in}}%
\pgfpathlineto{\pgfqpoint{5.712291in}{0.413320in}}%
\pgfpathlineto{\pgfqpoint{5.709490in}{0.413320in}}%
\pgfpathlineto{\pgfqpoint{5.706800in}{0.413320in}}%
\pgfpathlineto{\pgfqpoint{5.704130in}{0.413320in}}%
\pgfpathlineto{\pgfqpoint{5.701453in}{0.413320in}}%
\pgfpathlineto{\pgfqpoint{5.698775in}{0.413320in}}%
\pgfpathlineto{\pgfqpoint{5.696101in}{0.413320in}}%
\pgfpathlineto{\pgfqpoint{5.693473in}{0.413320in}}%
\pgfpathlineto{\pgfqpoint{5.690730in}{0.413320in}}%
\pgfpathlineto{\pgfqpoint{5.688159in}{0.413320in}}%
\pgfpathlineto{\pgfqpoint{5.685385in}{0.413320in}}%
\pgfpathlineto{\pgfqpoint{5.682836in}{0.413320in}}%
\pgfpathlineto{\pgfqpoint{5.680027in}{0.413320in}}%
\pgfpathlineto{\pgfqpoint{5.677486in}{0.413320in}}%
\pgfpathlineto{\pgfqpoint{5.674667in}{0.413320in}}%
\pgfpathlineto{\pgfqpoint{5.671991in}{0.413320in}}%
\pgfpathlineto{\pgfqpoint{5.669313in}{0.413320in}}%
\pgfpathlineto{\pgfqpoint{5.666632in}{0.413320in}}%
\pgfpathlineto{\pgfqpoint{5.664099in}{0.413320in}}%
\pgfpathlineto{\pgfqpoint{5.661273in}{0.413320in}}%
\pgfpathlineto{\pgfqpoint{5.658723in}{0.413320in}}%
\pgfpathlineto{\pgfqpoint{5.655919in}{0.413320in}}%
\pgfpathlineto{\pgfqpoint{5.653376in}{0.413320in}}%
\pgfpathlineto{\pgfqpoint{5.650563in}{0.413320in}}%
\pgfpathlineto{\pgfqpoint{5.648008in}{0.413320in}}%
\pgfpathlineto{\pgfqpoint{5.645243in}{0.413320in}}%
\pgfpathlineto{\pgfqpoint{5.642518in}{0.413320in}}%
\pgfpathlineto{\pgfqpoint{5.639852in}{0.413320in}}%
\pgfpathlineto{\pgfqpoint{5.637172in}{0.413320in}}%
\pgfpathlineto{\pgfqpoint{5.634496in}{0.413320in}}%
\pgfpathlineto{\pgfqpoint{5.631815in}{0.413320in}}%
\pgfpathlineto{\pgfqpoint{5.629232in}{0.413320in}}%
\pgfpathlineto{\pgfqpoint{5.626460in}{0.413320in}}%
\pgfpathlineto{\pgfqpoint{5.623868in}{0.413320in}}%
\pgfpathlineto{\pgfqpoint{5.621102in}{0.413320in}}%
\pgfpathlineto{\pgfqpoint{5.618526in}{0.413320in}}%
\pgfpathlineto{\pgfqpoint{5.615743in}{0.413320in}}%
\pgfpathlineto{\pgfqpoint{5.613235in}{0.413320in}}%
\pgfpathlineto{\pgfqpoint{5.610389in}{0.413320in}}%
\pgfpathlineto{\pgfqpoint{5.607698in}{0.413320in}}%
\pgfpathlineto{\pgfqpoint{5.605073in}{0.413320in}}%
\pgfpathlineto{\pgfqpoint{5.602352in}{0.413320in}}%
\pgfpathlineto{\pgfqpoint{5.599674in}{0.413320in}}%
\pgfpathlineto{\pgfqpoint{5.596999in}{0.413320in}}%
\pgfpathlineto{\pgfqpoint{5.594368in}{0.413320in}}%
\pgfpathlineto{\pgfqpoint{5.591641in}{0.413320in}}%
\pgfpathlineto{\pgfqpoint{5.589040in}{0.413320in}}%
\pgfpathlineto{\pgfqpoint{5.586269in}{0.413320in}}%
\pgfpathlineto{\pgfqpoint{5.583709in}{0.413320in}}%
\pgfpathlineto{\pgfqpoint{5.580914in}{0.413320in}}%
\pgfpathlineto{\pgfqpoint{5.578342in}{0.413320in}}%
\pgfpathlineto{\pgfqpoint{5.575596in}{0.413320in}}%
\pgfpathlineto{\pgfqpoint{5.572893in}{0.413320in}}%
\pgfpathlineto{\pgfqpoint{5.570215in}{0.413320in}}%
\pgfpathlineto{\pgfqpoint{5.567536in}{0.413320in}}%
\pgfpathlineto{\pgfqpoint{5.564940in}{0.413320in}}%
\pgfpathlineto{\pgfqpoint{5.562180in}{0.413320in}}%
\pgfpathlineto{\pgfqpoint{5.559612in}{0.413320in}}%
\pgfpathlineto{\pgfqpoint{5.556822in}{0.413320in}}%
\pgfpathlineto{\pgfqpoint{5.554198in}{0.413320in}}%
\pgfpathlineto{\pgfqpoint{5.551457in}{0.413320in}}%
\pgfpathlineto{\pgfqpoint{5.548921in}{0.413320in}}%
\pgfpathlineto{\pgfqpoint{5.546110in}{0.413320in}}%
\pgfpathlineto{\pgfqpoint{5.543421in}{0.413320in}}%
\pgfpathlineto{\pgfqpoint{5.540750in}{0.413320in}}%
\pgfpathlineto{\pgfqpoint{5.538074in}{0.413320in}}%
\pgfpathlineto{\pgfqpoint{5.535395in}{0.413320in}}%
\pgfpathlineto{\pgfqpoint{5.532717in}{0.413320in}}%
\pgfpathlineto{\pgfqpoint{5.530148in}{0.413320in}}%
\pgfpathlineto{\pgfqpoint{5.527360in}{0.413320in}}%
\pgfpathlineto{\pgfqpoint{5.524756in}{0.413320in}}%
\pgfpathlineto{\pgfqpoint{5.522003in}{0.413320in}}%
\pgfpathlineto{\pgfqpoint{5.519433in}{0.413320in}}%
\pgfpathlineto{\pgfqpoint{5.516646in}{0.413320in}}%
\pgfpathlineto{\pgfqpoint{5.514080in}{0.413320in}}%
\pgfpathlineto{\pgfqpoint{5.511290in}{0.413320in}}%
\pgfpathlineto{\pgfqpoint{5.508612in}{0.413320in}}%
\pgfpathlineto{\pgfqpoint{5.505933in}{0.413320in}}%
\pgfpathlineto{\pgfqpoint{5.503255in}{0.413320in}}%
\pgfpathlineto{\pgfqpoint{5.500687in}{0.413320in}}%
\pgfpathlineto{\pgfqpoint{5.497898in}{0.413320in}}%
\pgfpathlineto{\pgfqpoint{5.495346in}{0.413320in}}%
\pgfpathlineto{\pgfqpoint{5.492541in}{0.413320in}}%
\pgfpathlineto{\pgfqpoint{5.490000in}{0.413320in}}%
\pgfpathlineto{\pgfqpoint{5.487176in}{0.413320in}}%
\pgfpathlineto{\pgfqpoint{5.484641in}{0.413320in}}%
\pgfpathlineto{\pgfqpoint{5.481825in}{0.413320in}}%
\pgfpathlineto{\pgfqpoint{5.479152in}{0.413320in}}%
\pgfpathlineto{\pgfqpoint{5.476458in}{0.413320in}}%
\pgfpathlineto{\pgfqpoint{5.473792in}{0.413320in}}%
\pgfpathlineto{\pgfqpoint{5.471113in}{0.413320in}}%
\pgfpathlineto{\pgfqpoint{5.468425in}{0.413320in}}%
\pgfpathlineto{\pgfqpoint{5.465888in}{0.413320in}}%
\pgfpathlineto{\pgfqpoint{5.463079in}{0.413320in}}%
\pgfpathlineto{\pgfqpoint{5.460489in}{0.413320in}}%
\pgfpathlineto{\pgfqpoint{5.457721in}{0.413320in}}%
\pgfpathlineto{\pgfqpoint{5.455168in}{0.413320in}}%
\pgfpathlineto{\pgfqpoint{5.452365in}{0.413320in}}%
\pgfpathlineto{\pgfqpoint{5.449769in}{0.413320in}}%
\pgfpathlineto{\pgfqpoint{5.447021in}{0.413320in}}%
\pgfpathlineto{\pgfqpoint{5.444328in}{0.413320in}}%
\pgfpathlineto{\pgfqpoint{5.441698in}{0.413320in}}%
\pgfpathlineto{\pgfqpoint{5.438974in}{0.413320in}}%
\pgfpathlineto{\pgfqpoint{5.436295in}{0.413320in}}%
\pgfpathlineto{\pgfqpoint{5.433616in}{0.413320in}}%
\pgfpathlineto{\pgfqpoint{5.431015in}{0.413320in}}%
\pgfpathlineto{\pgfqpoint{5.428259in}{0.413320in}}%
\pgfpathlineto{\pgfqpoint{5.425661in}{0.413320in}}%
\pgfpathlineto{\pgfqpoint{5.422897in}{0.413320in}}%
\pgfpathlineto{\pgfqpoint{5.420304in}{0.413320in}}%
\pgfpathlineto{\pgfqpoint{5.417547in}{0.413320in}}%
\pgfpathlineto{\pgfqpoint{5.414954in}{0.413320in}}%
\pgfpathlineto{\pgfqpoint{5.412190in}{0.413320in}}%
\pgfpathlineto{\pgfqpoint{5.409507in}{0.413320in}}%
\pgfpathlineto{\pgfqpoint{5.406832in}{0.413320in}}%
\pgfpathlineto{\pgfqpoint{5.404154in}{0.413320in}}%
\pgfpathlineto{\pgfqpoint{5.401576in}{0.413320in}}%
\pgfpathlineto{\pgfqpoint{5.398784in}{0.413320in}}%
\pgfpathlineto{\pgfqpoint{5.396219in}{0.413320in}}%
\pgfpathlineto{\pgfqpoint{5.393441in}{0.413320in}}%
\pgfpathlineto{\pgfqpoint{5.390900in}{0.413320in}}%
\pgfpathlineto{\pgfqpoint{5.388083in}{0.413320in}}%
\pgfpathlineto{\pgfqpoint{5.385550in}{0.413320in}}%
\pgfpathlineto{\pgfqpoint{5.382725in}{0.413320in}}%
\pgfpathlineto{\pgfqpoint{5.380048in}{0.413320in}}%
\pgfpathlineto{\pgfqpoint{5.377370in}{0.413320in}}%
\pgfpathlineto{\pgfqpoint{5.374692in}{0.413320in}}%
\pgfpathlineto{\pgfqpoint{5.372013in}{0.413320in}}%
\pgfpathlineto{\pgfqpoint{5.369335in}{0.413320in}}%
\pgfpathlineto{\pgfqpoint{5.366727in}{0.413320in}}%
\pgfpathlineto{\pgfqpoint{5.363966in}{0.413320in}}%
\pgfpathlineto{\pgfqpoint{5.361370in}{0.413320in}}%
\pgfpathlineto{\pgfqpoint{5.358612in}{0.413320in}}%
\pgfpathlineto{\pgfqpoint{5.356056in}{0.413320in}}%
\pgfpathlineto{\pgfqpoint{5.353262in}{0.413320in}}%
\pgfpathlineto{\pgfqpoint{5.350723in}{0.413320in}}%
\pgfpathlineto{\pgfqpoint{5.347905in}{0.413320in}}%
\pgfpathlineto{\pgfqpoint{5.345224in}{0.413320in}}%
\pgfpathlineto{\pgfqpoint{5.342549in}{0.413320in}}%
\pgfpathlineto{\pgfqpoint{5.339872in}{0.413320in}}%
\pgfpathlineto{\pgfqpoint{5.337353in}{0.413320in}}%
\pgfpathlineto{\pgfqpoint{5.334510in}{0.413320in}}%
\pgfpathlineto{\pgfqpoint{5.331973in}{0.413320in}}%
\pgfpathlineto{\pgfqpoint{5.329159in}{0.413320in}}%
\pgfpathlineto{\pgfqpoint{5.326564in}{0.413320in}}%
\pgfpathlineto{\pgfqpoint{5.323802in}{0.413320in}}%
\pgfpathlineto{\pgfqpoint{5.321256in}{0.413320in}}%
\pgfpathlineto{\pgfqpoint{5.318430in}{0.413320in}}%
\pgfpathlineto{\pgfqpoint{5.315754in}{0.413320in}}%
\pgfpathlineto{\pgfqpoint{5.313089in}{0.413320in}}%
\pgfpathlineto{\pgfqpoint{5.310411in}{0.413320in}}%
\pgfpathlineto{\pgfqpoint{5.307731in}{0.413320in}}%
\pgfpathlineto{\pgfqpoint{5.305054in}{0.413320in}}%
\pgfpathlineto{\pgfqpoint{5.302443in}{0.413320in}}%
\pgfpathlineto{\pgfqpoint{5.299696in}{0.413320in}}%
\pgfpathlineto{\pgfqpoint{5.297140in}{0.413320in}}%
\pgfpathlineto{\pgfqpoint{5.294339in}{0.413320in}}%
\pgfpathlineto{\pgfqpoint{5.291794in}{0.413320in}}%
\pgfpathlineto{\pgfqpoint{5.288984in}{0.413320in}}%
\pgfpathlineto{\pgfqpoint{5.286436in}{0.413320in}}%
\pgfpathlineto{\pgfqpoint{5.283631in}{0.413320in}}%
\pgfpathlineto{\pgfqpoint{5.280947in}{0.413320in}}%
\pgfpathlineto{\pgfqpoint{5.278322in}{0.413320in}}%
\pgfpathlineto{\pgfqpoint{5.275589in}{0.413320in}}%
\pgfpathlineto{\pgfqpoint{5.272913in}{0.413320in}}%
\pgfpathlineto{\pgfqpoint{5.270238in}{0.413320in}}%
\pgfpathlineto{\pgfqpoint{5.267691in}{0.413320in}}%
\pgfpathlineto{\pgfqpoint{5.264876in}{0.413320in}}%
\pgfpathlineto{\pgfqpoint{5.262264in}{0.413320in}}%
\pgfpathlineto{\pgfqpoint{5.259511in}{0.413320in}}%
\pgfpathlineto{\pgfqpoint{5.256973in}{0.413320in}}%
\pgfpathlineto{\pgfqpoint{5.254236in}{0.413320in}}%
\pgfpathlineto{\pgfqpoint{5.251590in}{0.413320in}}%
\pgfpathlineto{\pgfqpoint{5.248816in}{0.413320in}}%
\pgfpathlineto{\pgfqpoint{5.246130in}{0.413320in}}%
\pgfpathlineto{\pgfqpoint{5.243445in}{0.413320in}}%
\pgfpathlineto{\pgfqpoint{5.240777in}{0.413320in}}%
\pgfpathlineto{\pgfqpoint{5.238173in}{0.413320in}}%
\pgfpathlineto{\pgfqpoint{5.235409in}{0.413320in}}%
\pgfpathlineto{\pgfqpoint{5.232855in}{0.413320in}}%
\pgfpathlineto{\pgfqpoint{5.230059in}{0.413320in}}%
\pgfpathlineto{\pgfqpoint{5.227470in}{0.413320in}}%
\pgfpathlineto{\pgfqpoint{5.224695in}{0.413320in}}%
\pgfpathlineto{\pgfqpoint{5.222151in}{0.413320in}}%
\pgfpathlineto{\pgfqpoint{5.219345in}{0.413320in}}%
\pgfpathlineto{\pgfqpoint{5.216667in}{0.413320in}}%
\pgfpathlineto{\pgfqpoint{5.214027in}{0.413320in}}%
\pgfpathlineto{\pgfqpoint{5.211299in}{0.413320in}}%
\pgfpathlineto{\pgfqpoint{5.208630in}{0.413320in}}%
\pgfpathlineto{\pgfqpoint{5.205952in}{0.413320in}}%
\pgfpathlineto{\pgfqpoint{5.203388in}{0.413320in}}%
\pgfpathlineto{\pgfqpoint{5.200594in}{0.413320in}}%
\pgfpathlineto{\pgfqpoint{5.198008in}{0.413320in}}%
\pgfpathlineto{\pgfqpoint{5.195239in}{0.413320in}}%
\pgfpathlineto{\pgfqpoint{5.192680in}{0.413320in}}%
\pgfpathlineto{\pgfqpoint{5.189880in}{0.413320in}}%
\pgfpathlineto{\pgfqpoint{5.187294in}{0.413320in}}%
\pgfpathlineto{\pgfqpoint{5.184522in}{0.413320in}}%
\pgfpathlineto{\pgfqpoint{5.181848in}{0.413320in}}%
\pgfpathlineto{\pgfqpoint{5.179188in}{0.413320in}}%
\pgfpathlineto{\pgfqpoint{5.176477in}{0.413320in}}%
\pgfpathlineto{\pgfqpoint{5.173925in}{0.413320in}}%
\pgfpathlineto{\pgfqpoint{5.171133in}{0.413320in}}%
\pgfpathlineto{\pgfqpoint{5.168591in}{0.413320in}}%
\pgfpathlineto{\pgfqpoint{5.165775in}{0.413320in}}%
\pgfpathlineto{\pgfqpoint{5.163243in}{0.413320in}}%
\pgfpathlineto{\pgfqpoint{5.160420in}{0.413320in}}%
\pgfpathlineto{\pgfqpoint{5.157815in}{0.413320in}}%
\pgfpathlineto{\pgfqpoint{5.155059in}{0.413320in}}%
\pgfpathlineto{\pgfqpoint{5.152382in}{0.413320in}}%
\pgfpathlineto{\pgfqpoint{5.149734in}{0.413320in}}%
\pgfpathlineto{\pgfqpoint{5.147029in}{0.413320in}}%
\pgfpathlineto{\pgfqpoint{5.144349in}{0.413320in}}%
\pgfpathlineto{\pgfqpoint{5.141660in}{0.413320in}}%
\pgfpathlineto{\pgfqpoint{5.139072in}{0.413320in}}%
\pgfpathlineto{\pgfqpoint{5.136311in}{0.413320in}}%
\pgfpathlineto{\pgfqpoint{5.133716in}{0.413320in}}%
\pgfpathlineto{\pgfqpoint{5.130953in}{0.413320in}}%
\pgfpathlineto{\pgfqpoint{5.128421in}{0.413320in}}%
\pgfpathlineto{\pgfqpoint{5.125599in}{0.413320in}}%
\pgfpathlineto{\pgfqpoint{5.123042in}{0.413320in}}%
\pgfpathlineto{\pgfqpoint{5.120243in}{0.413320in}}%
\pgfpathlineto{\pgfqpoint{5.117550in}{0.413320in}}%
\pgfpathlineto{\pgfqpoint{5.114887in}{0.413320in}}%
\pgfpathlineto{\pgfqpoint{5.112209in}{0.413320in}}%
\pgfpathlineto{\pgfqpoint{5.109530in}{0.413320in}}%
\pgfpathlineto{\pgfqpoint{5.106842in}{0.413320in}}%
\pgfpathlineto{\pgfqpoint{5.104312in}{0.413320in}}%
\pgfpathlineto{\pgfqpoint{5.101496in}{0.413320in}}%
\pgfpathlineto{\pgfqpoint{5.098948in}{0.413320in}}%
\pgfpathlineto{\pgfqpoint{5.096142in}{0.413320in}}%
\pgfpathlineto{\pgfqpoint{5.093579in}{0.413320in}}%
\pgfpathlineto{\pgfqpoint{5.090788in}{0.413320in}}%
\pgfpathlineto{\pgfqpoint{5.088103in}{0.413320in}}%
\pgfpathlineto{\pgfqpoint{5.085426in}{0.413320in}}%
\pgfpathlineto{\pgfqpoint{5.082746in}{0.413320in}}%
\pgfpathlineto{\pgfqpoint{5.080067in}{0.413320in}}%
\pgfpathlineto{\pgfqpoint{5.077390in}{0.413320in}}%
\pgfpathlineto{\pgfqpoint{5.074851in}{0.413320in}}%
\pgfpathlineto{\pgfqpoint{5.072030in}{0.413320in}}%
\pgfpathlineto{\pgfqpoint{5.069463in}{0.413320in}}%
\pgfpathlineto{\pgfqpoint{5.066677in}{0.413320in}}%
\pgfpathlineto{\pgfqpoint{5.064144in}{0.413320in}}%
\pgfpathlineto{\pgfqpoint{5.061315in}{0.413320in}}%
\pgfpathlineto{\pgfqpoint{5.058711in}{0.413320in}}%
\pgfpathlineto{\pgfqpoint{5.055952in}{0.413320in}}%
\pgfpathlineto{\pgfqpoint{5.053284in}{0.413320in}}%
\pgfpathlineto{\pgfqpoint{5.050606in}{0.413320in}}%
\pgfpathlineto{\pgfqpoint{5.047924in}{0.413320in}}%
\pgfpathlineto{\pgfqpoint{5.045249in}{0.413320in}}%
\pgfpathlineto{\pgfqpoint{5.042572in}{0.413320in}}%
\pgfpathlineto{\pgfqpoint{5.039962in}{0.413320in}}%
\pgfpathlineto{\pgfqpoint{5.037214in}{0.413320in}}%
\pgfpathlineto{\pgfqpoint{5.034649in}{0.413320in}}%
\pgfpathlineto{\pgfqpoint{5.031849in}{0.413320in}}%
\pgfpathlineto{\pgfqpoint{5.029275in}{0.413320in}}%
\pgfpathlineto{\pgfqpoint{5.026501in}{0.413320in}}%
\pgfpathlineto{\pgfqpoint{5.023927in}{0.413320in}}%
\pgfpathlineto{\pgfqpoint{5.021147in}{0.413320in}}%
\pgfpathlineto{\pgfqpoint{5.018466in}{0.413320in}}%
\pgfpathlineto{\pgfqpoint{5.015820in}{0.413320in}}%
\pgfpathlineto{\pgfqpoint{5.013104in}{0.413320in}}%
\pgfpathlineto{\pgfqpoint{5.010562in}{0.413320in}}%
\pgfpathlineto{\pgfqpoint{5.007751in}{0.413320in}}%
\pgfpathlineto{\pgfqpoint{5.005178in}{0.413320in}}%
\pgfpathlineto{\pgfqpoint{5.002384in}{0.413320in}}%
\pgfpathlineto{\pgfqpoint{4.999780in}{0.413320in}}%
\pgfpathlineto{\pgfqpoint{4.997028in}{0.413320in}}%
\pgfpathlineto{\pgfqpoint{4.994390in}{0.413320in}}%
\pgfpathlineto{\pgfqpoint{4.991683in}{0.413320in}}%
\pgfpathlineto{\pgfqpoint{4.989001in}{0.413320in}}%
\pgfpathlineto{\pgfqpoint{4.986325in}{0.413320in}}%
\pgfpathlineto{\pgfqpoint{4.983637in}{0.413320in}}%
\pgfpathlineto{\pgfqpoint{4.980967in}{0.413320in}}%
\pgfpathlineto{\pgfqpoint{4.978287in}{0.413320in}}%
\pgfpathlineto{\pgfqpoint{4.975703in}{0.413320in}}%
\pgfpathlineto{\pgfqpoint{4.972933in}{0.413320in}}%
\pgfpathlineto{\pgfqpoint{4.970314in}{0.413320in}}%
\pgfpathlineto{\pgfqpoint{4.967575in}{0.413320in}}%
\pgfpathlineto{\pgfqpoint{4.965002in}{0.413320in}}%
\pgfpathlineto{\pgfqpoint{4.962219in}{0.413320in}}%
\pgfpathlineto{\pgfqpoint{4.959689in}{0.413320in}}%
\pgfpathlineto{\pgfqpoint{4.956862in}{0.413320in}}%
\pgfpathlineto{\pgfqpoint{4.954182in}{0.413320in}}%
\pgfpathlineto{\pgfqpoint{4.951504in}{0.413320in}}%
\pgfpathlineto{\pgfqpoint{4.948827in}{0.413320in}}%
\pgfpathlineto{\pgfqpoint{4.946151in}{0.413320in}}%
\pgfpathlineto{\pgfqpoint{4.943466in}{0.413320in}}%
\pgfpathlineto{\pgfqpoint{4.940881in}{0.413320in}}%
\pgfpathlineto{\pgfqpoint{4.938112in}{0.413320in}}%
\pgfpathlineto{\pgfqpoint{4.935515in}{0.413320in}}%
\pgfpathlineto{\pgfqpoint{4.932742in}{0.413320in}}%
\pgfpathlineto{\pgfqpoint{4.930170in}{0.413320in}}%
\pgfpathlineto{\pgfqpoint{4.927400in}{0.413320in}}%
\pgfpathlineto{\pgfqpoint{4.924708in}{0.413320in}}%
\pgfpathlineto{\pgfqpoint{4.922041in}{0.413320in}}%
\pgfpathlineto{\pgfqpoint{4.919352in}{0.413320in}}%
\pgfpathlineto{\pgfqpoint{4.916681in}{0.413320in}}%
\pgfpathlineto{\pgfqpoint{4.914009in}{0.413320in}}%
\pgfpathlineto{\pgfqpoint{4.911435in}{0.413320in}}%
\pgfpathlineto{\pgfqpoint{4.908648in}{0.413320in}}%
\pgfpathlineto{\pgfqpoint{4.906096in}{0.413320in}}%
\pgfpathlineto{\pgfqpoint{4.903295in}{0.413320in}}%
\pgfpathlineto{\pgfqpoint{4.900712in}{0.413320in}}%
\pgfpathlineto{\pgfqpoint{4.897938in}{0.413320in}}%
\pgfpathlineto{\pgfqpoint{4.895399in}{0.413320in}}%
\pgfpathlineto{\pgfqpoint{4.892611in}{0.413320in}}%
\pgfpathlineto{\pgfqpoint{4.889902in}{0.413320in}}%
\pgfpathlineto{\pgfqpoint{4.887211in}{0.413320in}}%
\pgfpathlineto{\pgfqpoint{4.884540in}{0.413320in}}%
\pgfpathlineto{\pgfqpoint{4.881864in}{0.413320in}}%
\pgfpathlineto{\pgfqpoint{4.879180in}{0.413320in}}%
\pgfpathlineto{\pgfqpoint{4.876636in}{0.413320in}}%
\pgfpathlineto{\pgfqpoint{4.873832in}{0.413320in}}%
\pgfpathlineto{\pgfqpoint{4.871209in}{0.413320in}}%
\pgfpathlineto{\pgfqpoint{4.868474in}{0.413320in}}%
\pgfpathlineto{\pgfqpoint{4.865910in}{0.413320in}}%
\pgfpathlineto{\pgfqpoint{4.863116in}{0.413320in}}%
\pgfpathlineto{\pgfqpoint{4.860544in}{0.413320in}}%
\pgfpathlineto{\pgfqpoint{4.857807in}{0.413320in}}%
\pgfpathlineto{\pgfqpoint{4.855070in}{0.413320in}}%
\pgfpathlineto{\pgfqpoint{4.852404in}{0.413320in}}%
\pgfpathlineto{\pgfqpoint{4.849715in}{0.413320in}}%
\pgfpathlineto{\pgfqpoint{4.847127in}{0.413320in}}%
\pgfpathlineto{\pgfqpoint{4.844361in}{0.413320in}}%
\pgfpathlineto{\pgfqpoint{4.842380in}{0.413320in}}%
\pgfpathlineto{\pgfqpoint{4.839922in}{0.413320in}}%
\pgfpathlineto{\pgfqpoint{4.837992in}{0.413320in}}%
\pgfpathlineto{\pgfqpoint{4.833657in}{0.413320in}}%
\pgfpathlineto{\pgfqpoint{4.831045in}{0.413320in}}%
\pgfpathlineto{\pgfqpoint{4.828291in}{0.413320in}}%
\pgfpathlineto{\pgfqpoint{4.825619in}{0.413320in}}%
\pgfpathlineto{\pgfqpoint{4.822945in}{0.413320in}}%
\pgfpathlineto{\pgfqpoint{4.820265in}{0.413320in}}%
\pgfpathlineto{\pgfqpoint{4.817587in}{0.413320in}}%
\pgfpathlineto{\pgfqpoint{4.814907in}{0.413320in}}%
\pgfpathlineto{\pgfqpoint{4.812377in}{0.413320in}}%
\pgfpathlineto{\pgfqpoint{4.809538in}{0.413320in}}%
\pgfpathlineto{\pgfqpoint{4.807017in}{0.413320in}}%
\pgfpathlineto{\pgfqpoint{4.804193in}{0.413320in}}%
\pgfpathlineto{\pgfqpoint{4.801586in}{0.413320in}}%
\pgfpathlineto{\pgfqpoint{4.798830in}{0.413320in}}%
\pgfpathlineto{\pgfqpoint{4.796274in}{0.413320in}}%
\pgfpathlineto{\pgfqpoint{4.793512in}{0.413320in}}%
\pgfpathlineto{\pgfqpoint{4.790798in}{0.413320in}}%
\pgfpathlineto{\pgfqpoint{4.788116in}{0.413320in}}%
\pgfpathlineto{\pgfqpoint{4.785445in}{0.413320in}}%
\pgfpathlineto{\pgfqpoint{4.782872in}{0.413320in}}%
\pgfpathlineto{\pgfqpoint{4.780083in}{0.413320in}}%
\pgfpathlineto{\pgfqpoint{4.777535in}{0.413320in}}%
\pgfpathlineto{\pgfqpoint{4.774732in}{0.413320in}}%
\pgfpathlineto{\pgfqpoint{4.772198in}{0.413320in}}%
\pgfpathlineto{\pgfqpoint{4.769367in}{0.413320in}}%
\pgfpathlineto{\pgfqpoint{4.766783in}{0.413320in}}%
\pgfpathlineto{\pgfqpoint{4.764018in}{0.413320in}}%
\pgfpathlineto{\pgfqpoint{4.761337in}{0.413320in}}%
\pgfpathlineto{\pgfqpoint{4.758653in}{0.413320in}}%
\pgfpathlineto{\pgfqpoint{4.755983in}{0.413320in}}%
\pgfpathlineto{\pgfqpoint{4.753298in}{0.413320in}}%
\pgfpathlineto{\pgfqpoint{4.750627in}{0.413320in}}%
\pgfpathlineto{\pgfqpoint{4.748081in}{0.413320in}}%
\pgfpathlineto{\pgfqpoint{4.745256in}{0.413320in}}%
\pgfpathlineto{\pgfqpoint{4.742696in}{0.413320in}}%
\pgfpathlineto{\pgfqpoint{4.739912in}{0.413320in}}%
\pgfpathlineto{\pgfqpoint{4.737348in}{0.413320in}}%
\pgfpathlineto{\pgfqpoint{4.734552in}{0.413320in}}%
\pgfpathlineto{\pgfqpoint{4.731901in}{0.413320in}}%
\pgfpathlineto{\pgfqpoint{4.729233in}{0.413320in}}%
\pgfpathlineto{\pgfqpoint{4.726508in}{0.413320in}}%
\pgfpathlineto{\pgfqpoint{4.723873in}{0.413320in}}%
\pgfpathlineto{\pgfqpoint{4.721160in}{0.413320in}}%
\pgfpathlineto{\pgfqpoint{4.718486in}{0.413320in}}%
\pgfpathlineto{\pgfqpoint{4.715806in}{0.413320in}}%
\pgfpathlineto{\pgfqpoint{4.713275in}{0.413320in}}%
\pgfpathlineto{\pgfqpoint{4.710437in}{0.413320in}}%
\pgfpathlineto{\pgfqpoint{4.707824in}{0.413320in}}%
\pgfpathlineto{\pgfqpoint{4.705094in}{0.413320in}}%
\pgfpathlineto{\pgfqpoint{4.702517in}{0.413320in}}%
\pgfpathlineto{\pgfqpoint{4.699734in}{0.413320in}}%
\pgfpathlineto{\pgfqpoint{4.697170in}{0.413320in}}%
\pgfpathlineto{\pgfqpoint{4.694381in}{0.413320in}}%
\pgfpathlineto{\pgfqpoint{4.691694in}{0.413320in}}%
\pgfpathlineto{\pgfqpoint{4.689051in}{0.413320in}}%
\pgfpathlineto{\pgfqpoint{4.686337in}{0.413320in}}%
\pgfpathlineto{\pgfqpoint{4.683799in}{0.413320in}}%
\pgfpathlineto{\pgfqpoint{4.680988in}{0.413320in}}%
\pgfpathlineto{\pgfqpoint{4.678448in}{0.413320in}}%
\pgfpathlineto{\pgfqpoint{4.675619in}{0.413320in}}%
\pgfpathlineto{\pgfqpoint{4.673068in}{0.413320in}}%
\pgfpathlineto{\pgfqpoint{4.670261in}{0.413320in}}%
\pgfpathlineto{\pgfqpoint{4.667764in}{0.413320in}}%
\pgfpathlineto{\pgfqpoint{4.664923in}{0.413320in}}%
\pgfpathlineto{\pgfqpoint{4.662237in}{0.413320in}}%
\pgfpathlineto{\pgfqpoint{4.659590in}{0.413320in}}%
\pgfpathlineto{\pgfqpoint{4.656873in}{0.413320in}}%
\pgfpathlineto{\pgfqpoint{4.654203in}{0.413320in}}%
\pgfpathlineto{\pgfqpoint{4.651524in}{0.413320in}}%
\pgfpathlineto{\pgfqpoint{4.648922in}{0.413320in}}%
\pgfpathlineto{\pgfqpoint{4.646169in}{0.413320in}}%
\pgfpathlineto{\pgfqpoint{4.643628in}{0.413320in}}%
\pgfpathlineto{\pgfqpoint{4.640809in}{0.413320in}}%
\pgfpathlineto{\pgfqpoint{4.638204in}{0.413320in}}%
\pgfpathlineto{\pgfqpoint{4.635445in}{0.413320in}}%
\pgfpathlineto{\pgfqpoint{4.632902in}{0.413320in}}%
\pgfpathlineto{\pgfqpoint{4.630096in}{0.413320in}}%
\pgfpathlineto{\pgfqpoint{4.627411in}{0.413320in}}%
\pgfpathlineto{\pgfqpoint{4.624741in}{0.413320in}}%
\pgfpathlineto{\pgfqpoint{4.622056in}{0.413320in}}%
\pgfpathlineto{\pgfqpoint{4.619529in}{0.413320in}}%
\pgfpathlineto{\pgfqpoint{4.616702in}{0.413320in}}%
\pgfpathlineto{\pgfqpoint{4.614134in}{0.413320in}}%
\pgfpathlineto{\pgfqpoint{4.611350in}{0.413320in}}%
\pgfpathlineto{\pgfqpoint{4.608808in}{0.413320in}}%
\pgfpathlineto{\pgfqpoint{4.605990in}{0.413320in}}%
\pgfpathlineto{\pgfqpoint{4.603430in}{0.413320in}}%
\pgfpathlineto{\pgfqpoint{4.600633in}{0.413320in}}%
\pgfpathlineto{\pgfqpoint{4.597951in}{0.413320in}}%
\pgfpathlineto{\pgfqpoint{4.595281in}{0.413320in}}%
\pgfpathlineto{\pgfqpoint{4.592589in}{0.413320in}}%
\pgfpathlineto{\pgfqpoint{4.589920in}{0.413320in}}%
\pgfpathlineto{\pgfqpoint{4.587244in}{0.413320in}}%
\pgfpathlineto{\pgfqpoint{4.584672in}{0.413320in}}%
\pgfpathlineto{\pgfqpoint{4.581888in}{0.413320in}}%
\pgfpathlineto{\pgfqpoint{4.579305in}{0.413320in}}%
\pgfpathlineto{\pgfqpoint{4.576531in}{0.413320in}}%
\pgfpathlineto{\pgfqpoint{4.573947in}{0.413320in}}%
\pgfpathlineto{\pgfqpoint{4.571171in}{0.413320in}}%
\pgfpathlineto{\pgfqpoint{4.568612in}{0.413320in}}%
\pgfpathlineto{\pgfqpoint{4.565820in}{0.413320in}}%
\pgfpathlineto{\pgfqpoint{4.563125in}{0.413320in}}%
\pgfpathlineto{\pgfqpoint{4.560448in}{0.413320in}}%
\pgfpathlineto{\pgfqpoint{4.557777in}{0.413320in}}%
\pgfpathlineto{\pgfqpoint{4.555106in}{0.413320in}}%
\pgfpathlineto{\pgfqpoint{4.552425in}{0.413320in}}%
\pgfpathlineto{\pgfqpoint{4.549822in}{0.413320in}}%
\pgfpathlineto{\pgfqpoint{4.547064in}{0.413320in}}%
\pgfpathlineto{\pgfqpoint{4.544464in}{0.413320in}}%
\pgfpathlineto{\pgfqpoint{4.541711in}{0.413320in}}%
\pgfpathlineto{\pgfqpoint{4.539144in}{0.413320in}}%
\pgfpathlineto{\pgfqpoint{4.536400in}{0.413320in}}%
\pgfpathlineto{\pgfqpoint{4.533764in}{0.413320in}}%
\pgfpathlineto{\pgfqpoint{4.530990in}{0.413320in}}%
\pgfpathlineto{\pgfqpoint{4.528307in}{0.413320in}}%
\pgfpathlineto{\pgfqpoint{4.525640in}{0.413320in}}%
\pgfpathlineto{\pgfqpoint{4.522962in}{0.413320in}}%
\pgfpathlineto{\pgfqpoint{4.520345in}{0.413320in}}%
\pgfpathlineto{\pgfqpoint{4.517598in}{0.413320in}}%
\pgfpathlineto{\pgfqpoint{4.515080in}{0.413320in}}%
\pgfpathlineto{\pgfqpoint{4.512246in}{0.413320in}}%
\pgfpathlineto{\pgfqpoint{4.509643in}{0.413320in}}%
\pgfpathlineto{\pgfqpoint{4.506893in}{0.413320in}}%
\pgfpathlineto{\pgfqpoint{4.504305in}{0.413320in}}%
\pgfpathlineto{\pgfqpoint{4.501529in}{0.413320in}}%
\pgfpathlineto{\pgfqpoint{4.498850in}{0.413320in}}%
\pgfpathlineto{\pgfqpoint{4.496167in}{0.413320in}}%
\pgfpathlineto{\pgfqpoint{4.493492in}{0.413320in}}%
\pgfpathlineto{\pgfqpoint{4.490822in}{0.413320in}}%
\pgfpathlineto{\pgfqpoint{4.488130in}{0.413320in}}%
\pgfpathlineto{\pgfqpoint{4.485581in}{0.413320in}}%
\pgfpathlineto{\pgfqpoint{4.482778in}{0.413320in}}%
\pgfpathlineto{\pgfqpoint{4.480201in}{0.413320in}}%
\pgfpathlineto{\pgfqpoint{4.477430in}{0.413320in}}%
\pgfpathlineto{\pgfqpoint{4.474861in}{0.413320in}}%
\pgfpathlineto{\pgfqpoint{4.472059in}{0.413320in}}%
\pgfpathlineto{\pgfqpoint{4.469492in}{0.413320in}}%
\pgfpathlineto{\pgfqpoint{4.466717in}{0.413320in}}%
\pgfpathlineto{\pgfqpoint{4.464029in}{0.413320in}}%
\pgfpathlineto{\pgfqpoint{4.461367in}{0.413320in}}%
\pgfpathlineto{\pgfqpoint{4.458681in}{0.413320in}}%
\pgfpathlineto{\pgfqpoint{4.456138in}{0.413320in}}%
\pgfpathlineto{\pgfqpoint{4.453312in}{0.413320in}}%
\pgfpathlineto{\pgfqpoint{4.450767in}{0.413320in}}%
\pgfpathlineto{\pgfqpoint{4.447965in}{0.413320in}}%
\pgfpathlineto{\pgfqpoint{4.445423in}{0.413320in}}%
\pgfpathlineto{\pgfqpoint{4.442611in}{0.413320in}}%
\pgfpathlineto{\pgfqpoint{4.440041in}{0.413320in}}%
\pgfpathlineto{\pgfqpoint{4.437253in}{0.413320in}}%
\pgfpathlineto{\pgfqpoint{4.434569in}{0.413320in}}%
\pgfpathlineto{\pgfqpoint{4.431901in}{0.413320in}}%
\pgfpathlineto{\pgfqpoint{4.429220in}{0.413320in}}%
\pgfpathlineto{\pgfqpoint{4.426534in}{0.413320in}}%
\pgfpathlineto{\pgfqpoint{4.423863in}{0.413320in}}%
\pgfpathlineto{\pgfqpoint{4.421292in}{0.413320in}}%
\pgfpathlineto{\pgfqpoint{4.418506in}{0.413320in}}%
\pgfpathlineto{\pgfqpoint{4.415932in}{0.413320in}}%
\pgfpathlineto{\pgfqpoint{4.413149in}{0.413320in}}%
\pgfpathlineto{\pgfqpoint{4.410587in}{0.413320in}}%
\pgfpathlineto{\pgfqpoint{4.407788in}{0.413320in}}%
\pgfpathlineto{\pgfqpoint{4.405234in}{0.413320in}}%
\pgfpathlineto{\pgfqpoint{4.402468in}{0.413320in}}%
\pgfpathlineto{\pgfqpoint{4.399745in}{0.413320in}}%
\pgfpathlineto{\pgfqpoint{4.397076in}{0.413320in}}%
\pgfpathlineto{\pgfqpoint{4.394400in}{0.413320in}}%
\pgfpathlineto{\pgfqpoint{4.391721in}{0.413320in}}%
\pgfpathlineto{\pgfqpoint{4.389044in}{0.413320in}}%
\pgfpathlineto{\pgfqpoint{4.386431in}{0.413320in}}%
\pgfpathlineto{\pgfqpoint{4.383674in}{0.413320in}}%
\pgfpathlineto{\pgfqpoint{4.381097in}{0.413320in}}%
\pgfpathlineto{\pgfqpoint{4.378329in}{0.413320in}}%
\pgfpathlineto{\pgfqpoint{4.375761in}{0.413320in}}%
\pgfpathlineto{\pgfqpoint{4.372976in}{0.413320in}}%
\pgfpathlineto{\pgfqpoint{4.370437in}{0.413320in}}%
\pgfpathlineto{\pgfqpoint{4.367646in}{0.413320in}}%
\pgfpathlineto{\pgfqpoint{4.364936in}{0.413320in}}%
\pgfpathlineto{\pgfqpoint{4.362270in}{0.413320in}}%
\pgfpathlineto{\pgfqpoint{4.359582in}{0.413320in}}%
\pgfpathlineto{\pgfqpoint{4.357014in}{0.413320in}}%
\pgfpathlineto{\pgfqpoint{4.354224in}{0.413320in}}%
\pgfpathlineto{\pgfqpoint{4.351645in}{0.413320in}}%
\pgfpathlineto{\pgfqpoint{4.348868in}{0.413320in}}%
\pgfpathlineto{\pgfqpoint{4.346263in}{0.413320in}}%
\pgfpathlineto{\pgfqpoint{4.343510in}{0.413320in}}%
\pgfpathlineto{\pgfqpoint{4.340976in}{0.413320in}}%
\pgfpathlineto{\pgfqpoint{4.338154in}{0.413320in}}%
\pgfpathlineto{\pgfqpoint{4.335463in}{0.413320in}}%
\pgfpathlineto{\pgfqpoint{4.332796in}{0.413320in}}%
\pgfpathlineto{\pgfqpoint{4.330118in}{0.413320in}}%
\pgfpathlineto{\pgfqpoint{4.327440in}{0.413320in}}%
\pgfpathlineto{\pgfqpoint{4.324760in}{0.413320in}}%
\pgfpathlineto{\pgfqpoint{4.322181in}{0.413320in}}%
\pgfpathlineto{\pgfqpoint{4.319405in}{0.413320in}}%
\pgfpathlineto{\pgfqpoint{4.316856in}{0.413320in}}%
\pgfpathlineto{\pgfqpoint{4.314032in}{0.413320in}}%
\pgfpathlineto{\pgfqpoint{4.311494in}{0.413320in}}%
\pgfpathlineto{\pgfqpoint{4.308691in}{0.413320in}}%
\pgfpathlineto{\pgfqpoint{4.306118in}{0.413320in}}%
\pgfpathlineto{\pgfqpoint{4.303357in}{0.413320in}}%
\pgfpathlineto{\pgfqpoint{4.300656in}{0.413320in}}%
\pgfpathlineto{\pgfqpoint{4.297977in}{0.413320in}}%
\pgfpathlineto{\pgfqpoint{4.295299in}{0.413320in}}%
\pgfpathlineto{\pgfqpoint{4.292786in}{0.413320in}}%
\pgfpathlineto{\pgfqpoint{4.289936in}{0.413320in}}%
\pgfpathlineto{\pgfqpoint{4.287399in}{0.413320in}}%
\pgfpathlineto{\pgfqpoint{4.284586in}{0.413320in}}%
\pgfpathlineto{\pgfqpoint{4.282000in}{0.413320in}}%
\pgfpathlineto{\pgfqpoint{4.279212in}{0.413320in}}%
\pgfpathlineto{\pgfqpoint{4.276635in}{0.413320in}}%
\pgfpathlineto{\pgfqpoint{4.273874in}{0.413320in}}%
\pgfpathlineto{\pgfqpoint{4.271187in}{0.413320in}}%
\pgfpathlineto{\pgfqpoint{4.268590in}{0.413320in}}%
\pgfpathlineto{\pgfqpoint{4.265824in}{0.413320in}}%
\pgfpathlineto{\pgfqpoint{4.263157in}{0.413320in}}%
\pgfpathlineto{\pgfqpoint{4.260477in}{0.413320in}}%
\pgfpathlineto{\pgfqpoint{4.257958in}{0.413320in}}%
\pgfpathlineto{\pgfqpoint{4.255120in}{0.413320in}}%
\pgfpathlineto{\pgfqpoint{4.252581in}{0.413320in}}%
\pgfpathlineto{\pgfqpoint{4.249767in}{0.413320in}}%
\pgfpathlineto{\pgfqpoint{4.247225in}{0.413320in}}%
\pgfpathlineto{\pgfqpoint{4.244394in}{0.413320in}}%
\pgfpathlineto{\pgfqpoint{4.241900in}{0.413320in}}%
\pgfpathlineto{\pgfqpoint{4.239084in}{0.413320in}}%
\pgfpathlineto{\pgfqpoint{4.236375in}{0.413320in}}%
\pgfpathlineto{\pgfqpoint{4.233691in}{0.413320in}}%
\pgfpathlineto{\pgfqpoint{4.231013in}{0.413320in}}%
\pgfpathlineto{\pgfqpoint{4.228331in}{0.413320in}}%
\pgfpathlineto{\pgfqpoint{4.225654in}{0.413320in}}%
\pgfpathlineto{\pgfqpoint{4.223082in}{0.413320in}}%
\pgfpathlineto{\pgfqpoint{4.220304in}{0.413320in}}%
\pgfpathlineto{\pgfqpoint{4.217694in}{0.413320in}}%
\pgfpathlineto{\pgfqpoint{4.214948in}{0.413320in}}%
\pgfpathlineto{\pgfqpoint{4.212383in}{0.413320in}}%
\pgfpathlineto{\pgfqpoint{4.209597in}{0.413320in}}%
\pgfpathlineto{\pgfqpoint{4.207076in}{0.413320in}}%
\pgfpathlineto{\pgfqpoint{4.204240in}{0.413320in}}%
\pgfpathlineto{\pgfqpoint{4.201542in}{0.413320in}}%
\pgfpathlineto{\pgfqpoint{4.198878in}{0.413320in}}%
\pgfpathlineto{\pgfqpoint{4.196186in}{0.413320in}}%
\pgfpathlineto{\pgfqpoint{4.193638in}{0.413320in}}%
\pgfpathlineto{\pgfqpoint{4.190842in}{0.413320in}}%
\pgfpathlineto{\pgfqpoint{4.188318in}{0.413320in}}%
\pgfpathlineto{\pgfqpoint{4.185481in}{0.413320in}}%
\pgfpathlineto{\pgfqpoint{4.182899in}{0.413320in}}%
\pgfpathlineto{\pgfqpoint{4.180129in}{0.413320in}}%
\pgfpathlineto{\pgfqpoint{4.177593in}{0.413320in}}%
\pgfpathlineto{\pgfqpoint{4.174770in}{0.413320in}}%
\pgfpathlineto{\pgfqpoint{4.172093in}{0.413320in}}%
\pgfpathlineto{\pgfqpoint{4.169415in}{0.413320in}}%
\pgfpathlineto{\pgfqpoint{4.166737in}{0.413320in}}%
\pgfpathlineto{\pgfqpoint{4.164059in}{0.413320in}}%
\pgfpathlineto{\pgfqpoint{4.161380in}{0.413320in}}%
\pgfpathlineto{\pgfqpoint{4.158806in}{0.413320in}}%
\pgfpathlineto{\pgfqpoint{4.156016in}{0.413320in}}%
\pgfpathlineto{\pgfqpoint{4.153423in}{0.413320in}}%
\pgfpathlineto{\pgfqpoint{4.150665in}{0.413320in}}%
\pgfpathlineto{\pgfqpoint{4.148082in}{0.413320in}}%
\pgfpathlineto{\pgfqpoint{4.145310in}{0.413320in}}%
\pgfpathlineto{\pgfqpoint{4.142713in}{0.413320in}}%
\pgfpathlineto{\pgfqpoint{4.139963in}{0.413320in}}%
\pgfpathlineto{\pgfqpoint{4.137272in}{0.413320in}}%
\pgfpathlineto{\pgfqpoint{4.134615in}{0.413320in}}%
\pgfpathlineto{\pgfqpoint{4.131920in}{0.413320in}}%
\pgfpathlineto{\pgfqpoint{4.129349in}{0.413320in}}%
\pgfpathlineto{\pgfqpoint{4.126553in}{0.413320in}}%
\pgfpathlineto{\pgfqpoint{4.124019in}{0.413320in}}%
\pgfpathlineto{\pgfqpoint{4.121205in}{0.413320in}}%
\pgfpathlineto{\pgfqpoint{4.118554in}{0.413320in}}%
\pgfpathlineto{\pgfqpoint{4.115844in}{0.413320in}}%
\pgfpathlineto{\pgfqpoint{4.113252in}{0.413320in}}%
\pgfpathlineto{\pgfqpoint{4.110488in}{0.413320in}}%
\pgfpathlineto{\pgfqpoint{4.107814in}{0.413320in}}%
\pgfpathlineto{\pgfqpoint{4.105185in}{0.413320in}}%
\pgfpathlineto{\pgfqpoint{4.102456in}{0.413320in}}%
\pgfpathlineto{\pgfqpoint{4.099777in}{0.413320in}}%
\pgfpathlineto{\pgfqpoint{4.097092in}{0.413320in}}%
\pgfpathlineto{\pgfqpoint{4.094527in}{0.413320in}}%
\pgfpathlineto{\pgfqpoint{4.091729in}{0.413320in}}%
\pgfpathlineto{\pgfqpoint{4.089159in}{0.413320in}}%
\pgfpathlineto{\pgfqpoint{4.086385in}{0.413320in}}%
\pgfpathlineto{\pgfqpoint{4.083870in}{0.413320in}}%
\pgfpathlineto{\pgfqpoint{4.081018in}{0.413320in}}%
\pgfpathlineto{\pgfqpoint{4.078471in}{0.413320in}}%
\pgfpathlineto{\pgfqpoint{4.075705in}{0.413320in}}%
\pgfpathlineto{\pgfqpoint{4.072985in}{0.413320in}}%
\pgfpathlineto{\pgfqpoint{4.070313in}{0.413320in}}%
\pgfpathlineto{\pgfqpoint{4.067636in}{0.413320in}}%
\pgfpathlineto{\pgfqpoint{4.064957in}{0.413320in}}%
\pgfpathlineto{\pgfqpoint{4.062266in}{0.413320in}}%
\pgfpathlineto{\pgfqpoint{4.059702in}{0.413320in}}%
\pgfpathlineto{\pgfqpoint{4.056911in}{0.413320in}}%
\pgfpathlineto{\pgfqpoint{4.054326in}{0.413320in}}%
\pgfpathlineto{\pgfqpoint{4.051557in}{0.413320in}}%
\pgfpathlineto{\pgfqpoint{4.049006in}{0.413320in}}%
\pgfpathlineto{\pgfqpoint{4.046210in}{0.413320in}}%
\pgfpathlineto{\pgfqpoint{4.043667in}{0.413320in}}%
\pgfpathlineto{\pgfqpoint{4.040852in}{0.413320in}}%
\pgfpathlineto{\pgfqpoint{4.038174in}{0.413320in}}%
\pgfpathlineto{\pgfqpoint{4.035492in}{0.413320in}}%
\pgfpathlineto{\pgfqpoint{4.032817in}{0.413320in}}%
\pgfpathlineto{\pgfqpoint{4.030229in}{0.413320in}}%
\pgfpathlineto{\pgfqpoint{4.027447in}{0.413320in}}%
\pgfpathlineto{\pgfqpoint{4.024868in}{0.413320in}}%
\pgfpathlineto{\pgfqpoint{4.022097in}{0.413320in}}%
\pgfpathlineto{\pgfqpoint{4.019518in}{0.413320in}}%
\pgfpathlineto{\pgfqpoint{4.016744in}{0.413320in}}%
\pgfpathlineto{\pgfqpoint{4.014186in}{0.413320in}}%
\pgfpathlineto{\pgfqpoint{4.011394in}{0.413320in}}%
\pgfpathlineto{\pgfqpoint{4.008699in}{0.413320in}}%
\pgfpathlineto{\pgfqpoint{4.006034in}{0.413320in}}%
\pgfpathlineto{\pgfqpoint{4.003348in}{0.413320in}}%
\pgfpathlineto{\pgfqpoint{4.000674in}{0.413320in}}%
\pgfpathlineto{\pgfqpoint{3.997990in}{0.413320in}}%
\pgfpathlineto{\pgfqpoint{3.995417in}{0.413320in}}%
\pgfpathlineto{\pgfqpoint{3.992642in}{0.413320in}}%
\pgfpathlineto{\pgfqpoint{3.990055in}{0.413320in}}%
\pgfpathlineto{\pgfqpoint{3.987270in}{0.413320in}}%
\pgfpathlineto{\pgfqpoint{3.984714in}{0.413320in}}%
\pgfpathlineto{\pgfqpoint{3.981929in}{0.413320in}}%
\pgfpathlineto{\pgfqpoint{3.979389in}{0.413320in}}%
\pgfpathlineto{\pgfqpoint{3.976563in}{0.413320in}}%
\pgfpathlineto{\pgfqpoint{3.973885in}{0.413320in}}%
\pgfpathlineto{\pgfqpoint{3.971250in}{0.413320in}}%
\pgfpathlineto{\pgfqpoint{3.968523in}{0.413320in}}%
\pgfpathlineto{\pgfqpoint{3.966013in}{0.413320in}}%
\pgfpathlineto{\pgfqpoint{3.963176in}{0.413320in}}%
\pgfpathlineto{\pgfqpoint{3.960635in}{0.413320in}}%
\pgfpathlineto{\pgfqpoint{3.957823in}{0.413320in}}%
\pgfpathlineto{\pgfqpoint{3.955211in}{0.413320in}}%
\pgfpathlineto{\pgfqpoint{3.952464in}{0.413320in}}%
\pgfpathlineto{\pgfqpoint{3.949894in}{0.413320in}}%
\pgfpathlineto{\pgfqpoint{3.947101in}{0.413320in}}%
\pgfpathlineto{\pgfqpoint{3.944431in}{0.413320in}}%
\pgfpathlineto{\pgfqpoint{3.941778in}{0.413320in}}%
\pgfpathlineto{\pgfqpoint{3.939075in}{0.413320in}}%
\pgfpathlineto{\pgfqpoint{3.936395in}{0.413320in}}%
\pgfpathlineto{\pgfqpoint{3.933714in}{0.413320in}}%
\pgfpathlineto{\pgfqpoint{3.931202in}{0.413320in}}%
\pgfpathlineto{\pgfqpoint{3.928347in}{0.413320in}}%
\pgfpathlineto{\pgfqpoint{3.925778in}{0.413320in}}%
\pgfpathlineto{\pgfqpoint{3.923005in}{0.413320in}}%
\pgfpathlineto{\pgfqpoint{3.920412in}{0.413320in}}%
\pgfpathlineto{\pgfqpoint{3.917646in}{0.413320in}}%
\pgfpathlineto{\pgfqpoint{3.915107in}{0.413320in}}%
\pgfpathlineto{\pgfqpoint{3.912296in}{0.413320in}}%
\pgfpathlineto{\pgfqpoint{3.909602in}{0.413320in}}%
\pgfpathlineto{\pgfqpoint{3.906918in}{0.413320in}}%
\pgfpathlineto{\pgfqpoint{3.904252in}{0.413320in}}%
\pgfpathlineto{\pgfqpoint{3.901573in}{0.413320in}}%
\pgfpathlineto{\pgfqpoint{3.898891in}{0.413320in}}%
\pgfpathlineto{\pgfqpoint{3.896345in}{0.413320in}}%
\pgfpathlineto{\pgfqpoint{3.893541in}{0.413320in}}%
\pgfpathlineto{\pgfqpoint{3.890926in}{0.413320in}}%
\pgfpathlineto{\pgfqpoint{3.888188in}{0.413320in}}%
\pgfpathlineto{\pgfqpoint{3.885621in}{0.413320in}}%
\pgfpathlineto{\pgfqpoint{3.882850in}{0.413320in}}%
\pgfpathlineto{\pgfqpoint{3.880237in}{0.413320in}}%
\pgfpathlineto{\pgfqpoint{3.877466in}{0.413320in}}%
\pgfpathlineto{\pgfqpoint{3.874790in}{0.413320in}}%
\pgfpathlineto{\pgfqpoint{3.872114in}{0.413320in}}%
\pgfpathlineto{\pgfqpoint{3.869435in}{0.413320in}}%
\pgfpathlineto{\pgfqpoint{3.866815in}{0.413320in}}%
\pgfpathlineto{\pgfqpoint{3.864073in}{0.413320in}}%
\pgfpathlineto{\pgfqpoint{3.861561in}{0.413320in}}%
\pgfpathlineto{\pgfqpoint{3.858720in}{0.413320in}}%
\pgfpathlineto{\pgfqpoint{3.856100in}{0.413320in}}%
\pgfpathlineto{\pgfqpoint{3.853358in}{0.413320in}}%
\pgfpathlineto{\pgfqpoint{3.850814in}{0.413320in}}%
\pgfpathlineto{\pgfqpoint{3.848005in}{0.413320in}}%
\pgfpathlineto{\pgfqpoint{3.845329in}{0.413320in}}%
\pgfpathlineto{\pgfqpoint{3.842641in}{0.413320in}}%
\pgfpathlineto{\pgfqpoint{3.839960in}{0.413320in}}%
\pgfpathlineto{\pgfqpoint{3.837286in}{0.413320in}}%
\pgfpathlineto{\pgfqpoint{3.834616in}{0.413320in}}%
\pgfpathlineto{\pgfqpoint{3.832053in}{0.413320in}}%
\pgfpathlineto{\pgfqpoint{3.829252in}{0.413320in}}%
\pgfpathlineto{\pgfqpoint{3.826679in}{0.413320in}}%
\pgfpathlineto{\pgfqpoint{3.823903in}{0.413320in}}%
\pgfpathlineto{\pgfqpoint{3.821315in}{0.413320in}}%
\pgfpathlineto{\pgfqpoint{3.818546in}{0.413320in}}%
\pgfpathlineto{\pgfqpoint{3.815983in}{0.413320in}}%
\pgfpathlineto{\pgfqpoint{3.813172in}{0.413320in}}%
\pgfpathlineto{\pgfqpoint{3.810510in}{0.413320in}}%
\pgfpathlineto{\pgfqpoint{3.807832in}{0.413320in}}%
\pgfpathlineto{\pgfqpoint{3.805145in}{0.413320in}}%
\pgfpathlineto{\pgfqpoint{3.802569in}{0.413320in}}%
\pgfpathlineto{\pgfqpoint{3.799797in}{0.413320in}}%
\pgfpathlineto{\pgfqpoint{3.797265in}{0.413320in}}%
\pgfpathlineto{\pgfqpoint{3.794435in}{0.413320in}}%
\pgfpathlineto{\pgfqpoint{3.791897in}{0.413320in}}%
\pgfpathlineto{\pgfqpoint{3.789084in}{0.413320in}}%
\pgfpathlineto{\pgfqpoint{3.786504in}{0.413320in}}%
\pgfpathlineto{\pgfqpoint{3.783725in}{0.413320in}}%
\pgfpathlineto{\pgfqpoint{3.781046in}{0.413320in}}%
\pgfpathlineto{\pgfqpoint{3.778370in}{0.413320in}}%
\pgfpathlineto{\pgfqpoint{3.775691in}{0.413320in}}%
\pgfpathlineto{\pgfqpoint{3.773014in}{0.413320in}}%
\pgfpathlineto{\pgfqpoint{3.770323in}{0.413320in}}%
\pgfpathlineto{\pgfqpoint{3.767782in}{0.413320in}}%
\pgfpathlineto{\pgfqpoint{3.764966in}{0.413320in}}%
\pgfpathlineto{\pgfqpoint{3.762389in}{0.413320in}}%
\pgfpathlineto{\pgfqpoint{3.759622in}{0.413320in}}%
\pgfpathlineto{\pgfqpoint{3.757065in}{0.413320in}}%
\pgfpathlineto{\pgfqpoint{3.754265in}{0.413320in}}%
\pgfpathlineto{\pgfqpoint{3.751728in}{0.413320in}}%
\pgfpathlineto{\pgfqpoint{3.748903in}{0.413320in}}%
\pgfpathlineto{\pgfqpoint{3.746229in}{0.413320in}}%
\pgfpathlineto{\pgfqpoint{3.743548in}{0.413320in}}%
\pgfpathlineto{\pgfqpoint{3.740874in}{0.413320in}}%
\pgfpathlineto{\pgfqpoint{3.738194in}{0.413320in}}%
\pgfpathlineto{\pgfqpoint{3.735509in}{0.413320in}}%
\pgfpathlineto{\pgfqpoint{3.732950in}{0.413320in}}%
\pgfpathlineto{\pgfqpoint{3.730158in}{0.413320in}}%
\pgfpathlineto{\pgfqpoint{3.727581in}{0.413320in}}%
\pgfpathlineto{\pgfqpoint{3.724804in}{0.413320in}}%
\pgfpathlineto{\pgfqpoint{3.722228in}{0.413320in}}%
\pgfpathlineto{\pgfqpoint{3.719446in}{0.413320in}}%
\pgfpathlineto{\pgfqpoint{3.716875in}{0.413320in}}%
\pgfpathlineto{\pgfqpoint{3.714086in}{0.413320in}}%
\pgfpathlineto{\pgfqpoint{3.711410in}{0.413320in}}%
\pgfpathlineto{\pgfqpoint{3.708729in}{0.413320in}}%
\pgfpathlineto{\pgfqpoint{3.706053in}{0.413320in}}%
\pgfpathlineto{\pgfqpoint{3.703460in}{0.413320in}}%
\pgfpathlineto{\pgfqpoint{3.700684in}{0.413320in}}%
\pgfpathlineto{\pgfqpoint{3.698125in}{0.413320in}}%
\pgfpathlineto{\pgfqpoint{3.695331in}{0.413320in}}%
\pgfpathlineto{\pgfqpoint{3.692765in}{0.413320in}}%
\pgfpathlineto{\pgfqpoint{3.689983in}{0.413320in}}%
\pgfpathlineto{\pgfqpoint{3.687442in}{0.413320in}}%
\pgfpathlineto{\pgfqpoint{3.684620in}{0.413320in}}%
\pgfpathlineto{\pgfqpoint{3.681948in}{0.413320in}}%
\pgfpathlineto{\pgfqpoint{3.679273in}{0.413320in}}%
\pgfpathlineto{\pgfqpoint{3.676591in}{0.413320in}}%
\pgfpathlineto{\pgfqpoint{3.673911in}{0.413320in}}%
\pgfpathlineto{\pgfqpoint{3.671232in}{0.413320in}}%
\pgfpathlineto{\pgfqpoint{3.668665in}{0.413320in}}%
\pgfpathlineto{\pgfqpoint{3.665864in}{0.413320in}}%
\pgfpathlineto{\pgfqpoint{3.663276in}{0.413320in}}%
\pgfpathlineto{\pgfqpoint{3.660515in}{0.413320in}}%
\pgfpathlineto{\pgfqpoint{3.657917in}{0.413320in}}%
\pgfpathlineto{\pgfqpoint{3.655165in}{0.413320in}}%
\pgfpathlineto{\pgfqpoint{3.652628in}{0.413320in}}%
\pgfpathlineto{\pgfqpoint{3.649837in}{0.413320in}}%
\pgfpathlineto{\pgfqpoint{3.647130in}{0.413320in}}%
\pgfpathlineto{\pgfqpoint{3.644452in}{0.413320in}}%
\pgfpathlineto{\pgfqpoint{3.641773in}{0.413320in}}%
\pgfpathlineto{\pgfqpoint{3.639207in}{0.413320in}}%
\pgfpathlineto{\pgfqpoint{3.636413in}{0.413320in}}%
\pgfpathlineto{\pgfqpoint{3.633858in}{0.413320in}}%
\pgfpathlineto{\pgfqpoint{3.631058in}{0.413320in}}%
\pgfpathlineto{\pgfqpoint{3.628460in}{0.413320in}}%
\pgfpathlineto{\pgfqpoint{3.625689in}{0.413320in}}%
\pgfpathlineto{\pgfqpoint{3.623165in}{0.413320in}}%
\pgfpathlineto{\pgfqpoint{3.620345in}{0.413320in}}%
\pgfpathlineto{\pgfqpoint{3.617667in}{0.413320in}}%
\pgfpathlineto{\pgfqpoint{3.614982in}{0.413320in}}%
\pgfpathlineto{\pgfqpoint{3.612311in}{0.413320in}}%
\pgfpathlineto{\pgfqpoint{3.609632in}{0.413320in}}%
\pgfpathlineto{\pgfqpoint{3.606951in}{0.413320in}}%
\pgfpathlineto{\pgfqpoint{3.604387in}{0.413320in}}%
\pgfpathlineto{\pgfqpoint{3.601590in}{0.413320in}}%
\pgfpathlineto{\pgfqpoint{3.598998in}{0.413320in}}%
\pgfpathlineto{\pgfqpoint{3.596240in}{0.413320in}}%
\pgfpathlineto{\pgfqpoint{3.593620in}{0.413320in}}%
\pgfpathlineto{\pgfqpoint{3.590883in}{0.413320in}}%
\pgfpathlineto{\pgfqpoint{3.588258in}{0.413320in}}%
\pgfpathlineto{\pgfqpoint{3.585532in}{0.413320in}}%
\pgfpathlineto{\pgfqpoint{3.582851in}{0.413320in}}%
\pgfpathlineto{\pgfqpoint{3.580191in}{0.413320in}}%
\pgfpathlineto{\pgfqpoint{3.577487in}{0.413320in}}%
\pgfpathlineto{\pgfqpoint{3.574814in}{0.413320in}}%
\pgfpathlineto{\pgfqpoint{3.572126in}{0.413320in}}%
\pgfpathlineto{\pgfqpoint{3.569584in}{0.413320in}}%
\pgfpathlineto{\pgfqpoint{3.566774in}{0.413320in}}%
\pgfpathlineto{\pgfqpoint{3.564188in}{0.413320in}}%
\pgfpathlineto{\pgfqpoint{3.561420in}{0.413320in}}%
\pgfpathlineto{\pgfqpoint{3.558853in}{0.413320in}}%
\pgfpathlineto{\pgfqpoint{3.556061in}{0.413320in}}%
\pgfpathlineto{\pgfqpoint{3.553498in}{0.413320in}}%
\pgfpathlineto{\pgfqpoint{3.550713in}{0.413320in}}%
\pgfpathlineto{\pgfqpoint{3.548029in}{0.413320in}}%
\pgfpathlineto{\pgfqpoint{3.545349in}{0.413320in}}%
\pgfpathlineto{\pgfqpoint{3.542656in}{0.413320in}}%
\pgfpathlineto{\pgfqpoint{3.540093in}{0.413320in}}%
\pgfpathlineto{\pgfqpoint{3.537309in}{0.413320in}}%
\pgfpathlineto{\pgfqpoint{3.534783in}{0.413320in}}%
\pgfpathlineto{\pgfqpoint{3.531955in}{0.413320in}}%
\pgfpathlineto{\pgfqpoint{3.529327in}{0.413320in}}%
\pgfpathlineto{\pgfqpoint{3.526601in}{0.413320in}}%
\pgfpathlineto{\pgfqpoint{3.524041in}{0.413320in}}%
\pgfpathlineto{\pgfqpoint{3.521244in}{0.413320in}}%
\pgfpathlineto{\pgfqpoint{3.518565in}{0.413320in}}%
\pgfpathlineto{\pgfqpoint{3.515884in}{0.413320in}}%
\pgfpathlineto{\pgfqpoint{3.513209in}{0.413320in}}%
\pgfpathlineto{\pgfqpoint{3.510533in}{0.413320in}}%
\pgfpathlineto{\pgfqpoint{3.507840in}{0.413320in}}%
\pgfpathlineto{\pgfqpoint{3.505262in}{0.413320in}}%
\pgfpathlineto{\pgfqpoint{3.502488in}{0.413320in}}%
\pgfpathlineto{\pgfqpoint{3.499909in}{0.413320in}}%
\pgfpathlineto{\pgfqpoint{3.497139in}{0.413320in}}%
\pgfpathlineto{\pgfqpoint{3.494581in}{0.413320in}}%
\pgfpathlineto{\pgfqpoint{3.491783in}{0.413320in}}%
\pgfpathlineto{\pgfqpoint{3.489223in}{0.413320in}}%
\pgfpathlineto{\pgfqpoint{3.486442in}{0.413320in}}%
\pgfpathlineto{\pgfqpoint{3.483744in}{0.413320in}}%
\pgfpathlineto{\pgfqpoint{3.481072in}{0.413320in}}%
\pgfpathlineto{\pgfqpoint{3.478378in}{0.413320in}}%
\pgfpathlineto{\pgfqpoint{3.475821in}{0.413320in}}%
\pgfpathlineto{\pgfqpoint{3.473021in}{0.413320in}}%
\pgfpathlineto{\pgfqpoint{3.470466in}{0.413320in}}%
\pgfpathlineto{\pgfqpoint{3.467678in}{0.413320in}}%
\pgfpathlineto{\pgfqpoint{3.465072in}{0.413320in}}%
\pgfpathlineto{\pgfqpoint{3.462321in}{0.413320in}}%
\pgfpathlineto{\pgfqpoint{3.459695in}{0.413320in}}%
\pgfpathlineto{\pgfqpoint{3.456960in}{0.413320in}}%
\pgfpathlineto{\pgfqpoint{3.454285in}{0.413320in}}%
\pgfpathlineto{\pgfqpoint{3.451597in}{0.413320in}}%
\pgfpathlineto{\pgfqpoint{3.448926in}{0.413320in}}%
\pgfpathlineto{\pgfqpoint{3.446257in}{0.413320in}}%
\pgfpathlineto{\pgfqpoint{3.443574in}{0.413320in}}%
\pgfpathlineto{\pgfqpoint{3.440996in}{0.413320in}}%
\pgfpathlineto{\pgfqpoint{3.438210in}{0.413320in}}%
\pgfpathlineto{\pgfqpoint{3.435635in}{0.413320in}}%
\pgfpathlineto{\pgfqpoint{3.432851in}{0.413320in}}%
\pgfpathlineto{\pgfqpoint{3.430313in}{0.413320in}}%
\pgfpathlineto{\pgfqpoint{3.427501in}{0.413320in}}%
\pgfpathlineto{\pgfqpoint{3.424887in}{0.413320in}}%
\pgfpathlineto{\pgfqpoint{3.422142in}{0.413320in}}%
\pgfpathlineto{\pgfqpoint{3.419455in}{0.413320in}}%
\pgfpathlineto{\pgfqpoint{3.416780in}{0.413320in}}%
\pgfpathlineto{\pgfqpoint{3.414109in}{0.413320in}}%
\pgfpathlineto{\pgfqpoint{3.411431in}{0.413320in}}%
\pgfpathlineto{\pgfqpoint{3.408752in}{0.413320in}}%
\pgfpathlineto{\pgfqpoint{3.406202in}{0.413320in}}%
\pgfpathlineto{\pgfqpoint{3.403394in}{0.413320in}}%
\pgfpathlineto{\pgfqpoint{3.400783in}{0.413320in}}%
\pgfpathlineto{\pgfqpoint{3.398037in}{0.413320in}}%
\pgfpathlineto{\pgfqpoint{3.395461in}{0.413320in}}%
\pgfpathlineto{\pgfqpoint{3.392681in}{0.413320in}}%
\pgfpathlineto{\pgfqpoint{3.390102in}{0.413320in}}%
\pgfpathlineto{\pgfqpoint{3.387309in}{0.413320in}}%
\pgfpathlineto{\pgfqpoint{3.384647in}{0.413320in}}%
\pgfpathlineto{\pgfqpoint{3.381959in}{0.413320in}}%
\pgfpathlineto{\pgfqpoint{3.379290in}{0.413320in}}%
\pgfpathlineto{\pgfqpoint{3.376735in}{0.413320in}}%
\pgfpathlineto{\pgfqpoint{3.373921in}{0.413320in}}%
\pgfpathlineto{\pgfqpoint{3.371357in}{0.413320in}}%
\pgfpathlineto{\pgfqpoint{3.368577in}{0.413320in}}%
\pgfpathlineto{\pgfqpoint{3.365996in}{0.413320in}}%
\pgfpathlineto{\pgfqpoint{3.363221in}{0.413320in}}%
\pgfpathlineto{\pgfqpoint{3.360620in}{0.413320in}}%
\pgfpathlineto{\pgfqpoint{3.357862in}{0.413320in}}%
\pgfpathlineto{\pgfqpoint{3.355177in}{0.413320in}}%
\pgfpathlineto{\pgfqpoint{3.352505in}{0.413320in}}%
\pgfpathlineto{\pgfqpoint{3.349828in}{0.413320in}}%
\pgfpathlineto{\pgfqpoint{3.347139in}{0.413320in}}%
\pgfpathlineto{\pgfqpoint{3.344468in}{0.413320in}}%
\pgfpathlineto{\pgfqpoint{3.341893in}{0.413320in}}%
\pgfpathlineto{\pgfqpoint{3.339101in}{0.413320in}}%
\pgfpathlineto{\pgfqpoint{3.336541in}{0.413320in}}%
\pgfpathlineto{\pgfqpoint{3.333758in}{0.413320in}}%
\pgfpathlineto{\pgfqpoint{3.331183in}{0.413320in}}%
\pgfpathlineto{\pgfqpoint{3.328401in}{0.413320in}}%
\pgfpathlineto{\pgfqpoint{3.325860in}{0.413320in}}%
\pgfpathlineto{\pgfqpoint{3.323049in}{0.413320in}}%
\pgfpathlineto{\pgfqpoint{3.320366in}{0.413320in}}%
\pgfpathlineto{\pgfqpoint{3.317688in}{0.413320in}}%
\pgfpathlineto{\pgfqpoint{3.315008in}{0.413320in}}%
\pgfpathlineto{\pgfqpoint{3.312480in}{0.413320in}}%
\pgfpathlineto{\pgfqpoint{3.309652in}{0.413320in}}%
\pgfpathlineto{\pgfqpoint{3.307104in}{0.413320in}}%
\pgfpathlineto{\pgfqpoint{3.304295in}{0.413320in}}%
\pgfpathlineto{\pgfqpoint{3.301719in}{0.413320in}}%
\pgfpathlineto{\pgfqpoint{3.298937in}{0.413320in}}%
\pgfpathlineto{\pgfqpoint{3.296376in}{0.413320in}}%
\pgfpathlineto{\pgfqpoint{3.293574in}{0.413320in}}%
\pgfpathlineto{\pgfqpoint{3.290890in}{0.413320in}}%
\pgfpathlineto{\pgfqpoint{3.288225in}{0.413320in}}%
\pgfpathlineto{\pgfqpoint{3.285534in}{0.413320in}}%
\pgfpathlineto{\pgfqpoint{3.282870in}{0.413320in}}%
\pgfpathlineto{\pgfqpoint{3.280189in}{0.413320in}}%
\pgfpathlineto{\pgfqpoint{3.277603in}{0.413320in}}%
\pgfpathlineto{\pgfqpoint{3.274831in}{0.413320in}}%
\pgfpathlineto{\pgfqpoint{3.272254in}{0.413320in}}%
\pgfpathlineto{\pgfqpoint{3.269478in}{0.413320in}}%
\pgfpathlineto{\pgfqpoint{3.266849in}{0.413320in}}%
\pgfpathlineto{\pgfqpoint{3.264119in}{0.413320in}}%
\pgfpathlineto{\pgfqpoint{3.261594in}{0.413320in}}%
\pgfpathlineto{\pgfqpoint{3.258784in}{0.413320in}}%
\pgfpathlineto{\pgfqpoint{3.256083in}{0.413320in}}%
\pgfpathlineto{\pgfqpoint{3.253404in}{0.413320in}}%
\pgfpathlineto{\pgfqpoint{3.250716in}{0.413320in}}%
\pgfpathlineto{\pgfqpoint{3.248049in}{0.413320in}}%
\pgfpathlineto{\pgfqpoint{3.245363in}{0.413320in}}%
\pgfpathlineto{\pgfqpoint{3.242807in}{0.413320in}}%
\pgfpathlineto{\pgfqpoint{3.240010in}{0.413320in}}%
\pgfpathlineto{\pgfqpoint{3.237411in}{0.413320in}}%
\pgfpathlineto{\pgfqpoint{3.234658in}{0.413320in}}%
\pgfpathlineto{\pgfqpoint{3.232069in}{0.413320in}}%
\pgfpathlineto{\pgfqpoint{3.229310in}{0.413320in}}%
\pgfpathlineto{\pgfqpoint{3.226609in}{0.413320in}}%
\pgfpathlineto{\pgfqpoint{3.223942in}{0.413320in}}%
\pgfpathlineto{\pgfqpoint{3.221255in}{0.413320in}}%
\pgfpathlineto{\pgfqpoint{3.218586in}{0.413320in}}%
\pgfpathlineto{\pgfqpoint{3.215908in}{0.413320in}}%
\pgfpathlineto{\pgfqpoint{3.213342in}{0.413320in}}%
\pgfpathlineto{\pgfqpoint{3.210545in}{0.413320in}}%
\pgfpathlineto{\pgfqpoint{3.207984in}{0.413320in}}%
\pgfpathlineto{\pgfqpoint{3.205195in}{0.413320in}}%
\pgfpathlineto{\pgfqpoint{3.202562in}{0.413320in}}%
\pgfpathlineto{\pgfqpoint{3.199823in}{0.413320in}}%
\pgfpathlineto{\pgfqpoint{3.197226in}{0.413320in}}%
\pgfpathlineto{\pgfqpoint{3.194508in}{0.413320in}}%
\pgfpathlineto{\pgfqpoint{3.191796in}{0.413320in}}%
\pgfpathlineto{\pgfqpoint{3.189117in}{0.413320in}}%
\pgfpathlineto{\pgfqpoint{3.186440in}{0.413320in}}%
\pgfpathlineto{\pgfqpoint{3.183760in}{0.413320in}}%
\pgfpathlineto{\pgfqpoint{3.181089in}{0.413320in}}%
\pgfpathlineto{\pgfqpoint{3.178525in}{0.413320in}}%
\pgfpathlineto{\pgfqpoint{3.175724in}{0.413320in}}%
\pgfpathlineto{\pgfqpoint{3.173142in}{0.413320in}}%
\pgfpathlineto{\pgfqpoint{3.170375in}{0.413320in}}%
\pgfpathlineto{\pgfqpoint{3.167776in}{0.413320in}}%
\pgfpathlineto{\pgfqpoint{3.165019in}{0.413320in}}%
\pgfpathlineto{\pgfqpoint{3.162474in}{0.413320in}}%
\pgfpathlineto{\pgfqpoint{3.159675in}{0.413320in}}%
\pgfpathlineto{\pgfqpoint{3.156981in}{0.413320in}}%
\pgfpathlineto{\pgfqpoint{3.154327in}{0.413320in}}%
\pgfpathlineto{\pgfqpoint{3.151612in}{0.413320in}}%
\pgfpathlineto{\pgfqpoint{3.149057in}{0.413320in}}%
\pgfpathlineto{\pgfqpoint{3.146271in}{0.413320in}}%
\pgfpathlineto{\pgfqpoint{3.143740in}{0.413320in}}%
\pgfpathlineto{\pgfqpoint{3.140913in}{0.413320in}}%
\pgfpathlineto{\pgfqpoint{3.138375in}{0.413320in}}%
\pgfpathlineto{\pgfqpoint{3.135550in}{0.413320in}}%
\pgfpathlineto{\pgfqpoint{3.132946in}{0.413320in}}%
\pgfpathlineto{\pgfqpoint{3.130199in}{0.413320in}}%
\pgfpathlineto{\pgfqpoint{3.127512in}{0.413320in}}%
\pgfpathlineto{\pgfqpoint{3.124842in}{0.413320in}}%
\pgfpathlineto{\pgfqpoint{3.122163in}{0.413320in}}%
\pgfpathlineto{\pgfqpoint{3.119487in}{0.413320in}}%
\pgfpathlineto{\pgfqpoint{3.116807in}{0.413320in}}%
\pgfpathlineto{\pgfqpoint{3.114242in}{0.413320in}}%
\pgfpathlineto{\pgfqpoint{3.111451in}{0.413320in}}%
\pgfpathlineto{\pgfqpoint{3.108896in}{0.413320in}}%
\pgfpathlineto{\pgfqpoint{3.106094in}{0.413320in}}%
\pgfpathlineto{\pgfqpoint{3.103508in}{0.413320in}}%
\pgfpathlineto{\pgfqpoint{3.100737in}{0.413320in}}%
\pgfpathlineto{\pgfqpoint{3.098163in}{0.413320in}}%
\pgfpathlineto{\pgfqpoint{3.095388in}{0.413320in}}%
\pgfpathlineto{\pgfqpoint{3.092699in}{0.413320in}}%
\pgfpathlineto{\pgfqpoint{3.090023in}{0.413320in}}%
\pgfpathlineto{\pgfqpoint{3.087343in}{0.413320in}}%
\pgfpathlineto{\pgfqpoint{3.084671in}{0.413320in}}%
\pgfpathlineto{\pgfqpoint{3.081990in}{0.413320in}}%
\pgfpathlineto{\pgfqpoint{3.079381in}{0.413320in}}%
\pgfpathlineto{\pgfqpoint{3.076631in}{0.413320in}}%
\pgfpathlineto{\pgfqpoint{3.074056in}{0.413320in}}%
\pgfpathlineto{\pgfqpoint{3.071266in}{0.413320in}}%
\pgfpathlineto{\pgfqpoint{3.068709in}{0.413320in}}%
\pgfpathlineto{\pgfqpoint{3.065916in}{0.413320in}}%
\pgfpathlineto{\pgfqpoint{3.063230in}{0.413320in}}%
\pgfpathlineto{\pgfqpoint{3.060561in}{0.413320in}}%
\pgfpathlineto{\pgfqpoint{3.057884in}{0.413320in}}%
\pgfpathlineto{\pgfqpoint{3.055202in}{0.413320in}}%
\pgfpathlineto{\pgfqpoint{3.052526in}{0.413320in}}%
\pgfpathlineto{\pgfqpoint{3.049988in}{0.413320in}}%
\pgfpathlineto{\pgfqpoint{3.047157in}{0.413320in}}%
\pgfpathlineto{\pgfqpoint{3.044568in}{0.413320in}}%
\pgfpathlineto{\pgfqpoint{3.041813in}{0.413320in}}%
\pgfpathlineto{\pgfqpoint{3.039262in}{0.413320in}}%
\pgfpathlineto{\pgfqpoint{3.036456in}{0.413320in}}%
\pgfpathlineto{\pgfqpoint{3.033921in}{0.413320in}}%
\pgfpathlineto{\pgfqpoint{3.031091in}{0.413320in}}%
\pgfpathlineto{\pgfqpoint{3.028412in}{0.413320in}}%
\pgfpathlineto{\pgfqpoint{3.025803in}{0.413320in}}%
\pgfpathlineto{\pgfqpoint{3.023058in}{0.413320in}}%
\pgfpathlineto{\pgfqpoint{3.020382in}{0.413320in}}%
\pgfpathlineto{\pgfqpoint{3.017707in}{0.413320in}}%
\pgfpathlineto{\pgfqpoint{3.015097in}{0.413320in}}%
\pgfpathlineto{\pgfqpoint{3.012351in}{0.413320in}}%
\pgfpathlineto{\pgfqpoint{3.009784in}{0.413320in}}%
\pgfpathlineto{\pgfqpoint{3.006993in}{0.413320in}}%
\pgfpathlineto{\pgfqpoint{3.004419in}{0.413320in}}%
\pgfpathlineto{\pgfqpoint{3.001635in}{0.413320in}}%
\pgfpathlineto{\pgfqpoint{2.999103in}{0.413320in}}%
\pgfpathlineto{\pgfqpoint{2.996300in}{0.413320in}}%
\pgfpathlineto{\pgfqpoint{2.993595in}{0.413320in}}%
\pgfpathlineto{\pgfqpoint{2.990978in}{0.413320in}}%
\pgfpathlineto{\pgfqpoint{2.988238in}{0.413320in}}%
\pgfpathlineto{\pgfqpoint{2.985666in}{0.413320in}}%
\pgfpathlineto{\pgfqpoint{2.982885in}{0.413320in}}%
\pgfpathlineto{\pgfqpoint{2.980341in}{0.413320in}}%
\pgfpathlineto{\pgfqpoint{2.977517in}{0.413320in}}%
\pgfpathlineto{\pgfqpoint{2.974972in}{0.413320in}}%
\pgfpathlineto{\pgfqpoint{2.972177in}{0.413320in}}%
\pgfpathlineto{\pgfqpoint{2.969599in}{0.413320in}}%
\pgfpathlineto{\pgfqpoint{2.966812in}{0.413320in}}%
\pgfpathlineto{\pgfqpoint{2.964127in}{0.413320in}}%
\pgfpathlineto{\pgfqpoint{2.961460in}{0.413320in}}%
\pgfpathlineto{\pgfqpoint{2.958782in}{0.413320in}}%
\pgfpathlineto{\pgfqpoint{2.956103in}{0.413320in}}%
\pgfpathlineto{\pgfqpoint{2.953422in}{0.413320in}}%
\pgfpathlineto{\pgfqpoint{2.950884in}{0.413320in}}%
\pgfpathlineto{\pgfqpoint{2.948068in}{0.413320in}}%
\pgfpathlineto{\pgfqpoint{2.945461in}{0.413320in}}%
\pgfpathlineto{\pgfqpoint{2.942711in}{0.413320in}}%
\pgfpathlineto{\pgfqpoint{2.940120in}{0.413320in}}%
\pgfpathlineto{\pgfqpoint{2.937352in}{0.413320in}}%
\pgfpathlineto{\pgfqpoint{2.934759in}{0.413320in}}%
\pgfpathlineto{\pgfqpoint{2.932033in}{0.413320in}}%
\pgfpathlineto{\pgfqpoint{2.929321in}{0.413320in}}%
\pgfpathlineto{\pgfqpoint{2.926655in}{0.413320in}}%
\pgfpathlineto{\pgfqpoint{2.923963in}{0.413320in}}%
\pgfpathlineto{\pgfqpoint{2.921363in}{0.413320in}}%
\pgfpathlineto{\pgfqpoint{2.918606in}{0.413320in}}%
\pgfpathlineto{\pgfqpoint{2.916061in}{0.413320in}}%
\pgfpathlineto{\pgfqpoint{2.913243in}{0.413320in}}%
\pgfpathlineto{\pgfqpoint{2.910631in}{0.413320in}}%
\pgfpathlineto{\pgfqpoint{2.907882in}{0.413320in}}%
\pgfpathlineto{\pgfqpoint{2.905341in}{0.413320in}}%
\pgfpathlineto{\pgfqpoint{2.902535in}{0.413320in}}%
\pgfpathlineto{\pgfqpoint{2.899858in}{0.413320in}}%
\pgfpathlineto{\pgfqpoint{2.897179in}{0.413320in}}%
\pgfpathlineto{\pgfqpoint{2.894487in}{0.413320in}}%
\pgfpathlineto{\pgfqpoint{2.891809in}{0.413320in}}%
\pgfpathlineto{\pgfqpoint{2.889145in}{0.413320in}}%
\pgfpathlineto{\pgfqpoint{2.886578in}{0.413320in}}%
\pgfpathlineto{\pgfqpoint{2.883780in}{0.413320in}}%
\pgfpathlineto{\pgfqpoint{2.881254in}{0.413320in}}%
\pgfpathlineto{\pgfqpoint{2.878431in}{0.413320in}}%
\pgfpathlineto{\pgfqpoint{2.875882in}{0.413320in}}%
\pgfpathlineto{\pgfqpoint{2.873074in}{0.413320in}}%
\pgfpathlineto{\pgfqpoint{2.870475in}{0.413320in}}%
\pgfpathlineto{\pgfqpoint{2.867713in}{0.413320in}}%
\pgfpathlineto{\pgfqpoint{2.865031in}{0.413320in}}%
\pgfpathlineto{\pgfqpoint{2.862402in}{0.413320in}}%
\pgfpathlineto{\pgfqpoint{2.859668in}{0.413320in}}%
\pgfpathlineto{\pgfqpoint{2.857003in}{0.413320in}}%
\pgfpathlineto{\pgfqpoint{2.854325in}{0.413320in}}%
\pgfpathlineto{\pgfqpoint{2.851793in}{0.413320in}}%
\pgfpathlineto{\pgfqpoint{2.848960in}{0.413320in}}%
\pgfpathlineto{\pgfqpoint{2.846408in}{0.413320in}}%
\pgfpathlineto{\pgfqpoint{2.843611in}{0.413320in}}%
\pgfpathlineto{\pgfqpoint{2.841055in}{0.413320in}}%
\pgfpathlineto{\pgfqpoint{2.838254in}{0.413320in}}%
\pgfpathlineto{\pgfqpoint{2.835698in}{0.413320in}}%
\pgfpathlineto{\pgfqpoint{2.832894in}{0.413320in}}%
\pgfpathlineto{\pgfqpoint{2.830219in}{0.413320in}}%
\pgfpathlineto{\pgfqpoint{2.827567in}{0.413320in}}%
\pgfpathlineto{\pgfqpoint{2.824851in}{0.413320in}}%
\pgfpathlineto{\pgfqpoint{2.822303in}{0.413320in}}%
\pgfpathlineto{\pgfqpoint{2.819506in}{0.413320in}}%
\pgfpathlineto{\pgfqpoint{2.816867in}{0.413320in}}%
\pgfpathlineto{\pgfqpoint{2.814141in}{0.413320in}}%
\pgfpathlineto{\pgfqpoint{2.811597in}{0.413320in}}%
\pgfpathlineto{\pgfqpoint{2.808792in}{0.413320in}}%
\pgfpathlineto{\pgfqpoint{2.806175in}{0.413320in}}%
\pgfpathlineto{\pgfqpoint{2.803435in}{0.413320in}}%
\pgfpathlineto{\pgfqpoint{2.800756in}{0.413320in}}%
\pgfpathlineto{\pgfqpoint{2.798070in}{0.413320in}}%
\pgfpathlineto{\pgfqpoint{2.795398in}{0.413320in}}%
\pgfpathlineto{\pgfqpoint{2.792721in}{0.413320in}}%
\pgfpathlineto{\pgfqpoint{2.790044in}{0.413320in}}%
\pgfpathlineto{\pgfqpoint{2.787468in}{0.413320in}}%
\pgfpathlineto{\pgfqpoint{2.784687in}{0.413320in}}%
\pgfpathlineto{\pgfqpoint{2.782113in}{0.413320in}}%
\pgfpathlineto{\pgfqpoint{2.779330in}{0.413320in}}%
\pgfpathlineto{\pgfqpoint{2.776767in}{0.413320in}}%
\pgfpathlineto{\pgfqpoint{2.773972in}{0.413320in}}%
\pgfpathlineto{\pgfqpoint{2.771373in}{0.413320in}}%
\pgfpathlineto{\pgfqpoint{2.768617in}{0.413320in}}%
\pgfpathlineto{\pgfqpoint{2.765935in}{0.413320in}}%
\pgfpathlineto{\pgfqpoint{2.763253in}{0.413320in}}%
\pgfpathlineto{\pgfqpoint{2.760581in}{0.413320in}}%
\pgfpathlineto{\pgfqpoint{2.758028in}{0.413320in}}%
\pgfpathlineto{\pgfqpoint{2.755224in}{0.413320in}}%
\pgfpathlineto{\pgfqpoint{2.752614in}{0.413320in}}%
\pgfpathlineto{\pgfqpoint{2.749868in}{0.413320in}}%
\pgfpathlineto{\pgfqpoint{2.747260in}{0.413320in}}%
\pgfpathlineto{\pgfqpoint{2.744510in}{0.413320in}}%
\pgfpathlineto{\pgfqpoint{2.741928in}{0.413320in}}%
\pgfpathlineto{\pgfqpoint{2.739155in}{0.413320in}}%
\pgfpathlineto{\pgfqpoint{2.736476in}{0.413320in}}%
\pgfpathlineto{\pgfqpoint{2.733798in}{0.413320in}}%
\pgfpathlineto{\pgfqpoint{2.731119in}{0.413320in}}%
\pgfpathlineto{\pgfqpoint{2.728439in}{0.413320in}}%
\pgfpathlineto{\pgfqpoint{2.725760in}{0.413320in}}%
\pgfpathlineto{\pgfqpoint{2.723211in}{0.413320in}}%
\pgfpathlineto{\pgfqpoint{2.720404in}{0.413320in}}%
\pgfpathlineto{\pgfqpoint{2.717773in}{0.413320in}}%
\pgfpathlineto{\pgfqpoint{2.715036in}{0.413320in}}%
\pgfpathlineto{\pgfqpoint{2.712477in}{0.413320in}}%
\pgfpathlineto{\pgfqpoint{2.709683in}{0.413320in}}%
\pgfpathlineto{\pgfqpoint{2.707125in}{0.413320in}}%
\pgfpathlineto{\pgfqpoint{2.704326in}{0.413320in}}%
\pgfpathlineto{\pgfqpoint{2.701657in}{0.413320in}}%
\pgfpathlineto{\pgfqpoint{2.698968in}{0.413320in}}%
\pgfpathlineto{\pgfqpoint{2.696293in}{0.413320in}}%
\pgfpathlineto{\pgfqpoint{2.693611in}{0.413320in}}%
\pgfpathlineto{\pgfqpoint{2.690940in}{0.413320in}}%
\pgfpathlineto{\pgfqpoint{2.688328in}{0.413320in}}%
\pgfpathlineto{\pgfqpoint{2.685586in}{0.413320in}}%
\pgfpathlineto{\pgfqpoint{2.683009in}{0.413320in}}%
\pgfpathlineto{\pgfqpoint{2.680224in}{0.413320in}}%
\pgfpathlineto{\pgfqpoint{2.677650in}{0.413320in}}%
\pgfpathlineto{\pgfqpoint{2.674873in}{0.413320in}}%
\pgfpathlineto{\pgfqpoint{2.672301in}{0.413320in}}%
\pgfpathlineto{\pgfqpoint{2.669506in}{0.413320in}}%
\pgfpathlineto{\pgfqpoint{2.666836in}{0.413320in}}%
\pgfpathlineto{\pgfqpoint{2.664151in}{0.413320in}}%
\pgfpathlineto{\pgfqpoint{2.661481in}{0.413320in}}%
\pgfpathlineto{\pgfqpoint{2.658942in}{0.413320in}}%
\pgfpathlineto{\pgfqpoint{2.656124in}{0.413320in}}%
\pgfpathlineto{\pgfqpoint{2.653567in}{0.413320in}}%
\pgfpathlineto{\pgfqpoint{2.650767in}{0.413320in}}%
\pgfpathlineto{\pgfqpoint{2.648196in}{0.413320in}}%
\pgfpathlineto{\pgfqpoint{2.645408in}{0.413320in}}%
\pgfpathlineto{\pgfqpoint{2.642827in}{0.413320in}}%
\pgfpathlineto{\pgfqpoint{2.640053in}{0.413320in}}%
\pgfpathlineto{\pgfqpoint{2.637369in}{0.413320in}}%
\pgfpathlineto{\pgfqpoint{2.634700in}{0.413320in}}%
\pgfpathlineto{\pgfqpoint{2.632018in}{0.413320in}}%
\pgfpathlineto{\pgfqpoint{2.629340in}{0.413320in}}%
\pgfpathlineto{\pgfqpoint{2.626653in}{0.413320in}}%
\pgfpathlineto{\pgfqpoint{2.624077in}{0.413320in}}%
\pgfpathlineto{\pgfqpoint{2.621304in}{0.413320in}}%
\pgfpathlineto{\pgfqpoint{2.618773in}{0.413320in}}%
\pgfpathlineto{\pgfqpoint{2.615934in}{0.413320in}}%
\pgfpathlineto{\pgfqpoint{2.613393in}{0.413320in}}%
\pgfpathlineto{\pgfqpoint{2.610588in}{0.413320in}}%
\pgfpathlineto{\pgfqpoint{2.608004in}{0.413320in}}%
\pgfpathlineto{\pgfqpoint{2.605232in}{0.413320in}}%
\pgfpathlineto{\pgfqpoint{2.602557in}{0.413320in}}%
\pgfpathlineto{\pgfqpoint{2.599920in}{0.413320in}}%
\pgfpathlineto{\pgfqpoint{2.597196in}{0.413320in}}%
\pgfpathlineto{\pgfqpoint{2.594630in}{0.413320in}}%
\pgfpathlineto{\pgfqpoint{2.591842in}{0.413320in}}%
\pgfpathlineto{\pgfqpoint{2.589248in}{0.413320in}}%
\pgfpathlineto{\pgfqpoint{2.586484in}{0.413320in}}%
\pgfpathlineto{\pgfqpoint{2.583913in}{0.413320in}}%
\pgfpathlineto{\pgfqpoint{2.581129in}{0.413320in}}%
\pgfpathlineto{\pgfqpoint{2.578567in}{0.413320in}}%
\pgfpathlineto{\pgfqpoint{2.575779in}{0.413320in}}%
\pgfpathlineto{\pgfqpoint{2.573082in}{0.413320in}}%
\pgfpathlineto{\pgfqpoint{2.570411in}{0.413320in}}%
\pgfpathlineto{\pgfqpoint{2.567730in}{0.413320in}}%
\pgfpathlineto{\pgfqpoint{2.565045in}{0.413320in}}%
\pgfpathlineto{\pgfqpoint{2.562375in}{0.413320in}}%
\pgfpathlineto{\pgfqpoint{2.559790in}{0.413320in}}%
\pgfpathlineto{\pgfqpoint{2.557009in}{0.413320in}}%
\pgfpathlineto{\pgfqpoint{2.554493in}{0.413320in}}%
\pgfpathlineto{\pgfqpoint{2.551664in}{0.413320in}}%
\pgfpathlineto{\pgfqpoint{2.549114in}{0.413320in}}%
\pgfpathlineto{\pgfqpoint{2.546310in}{0.413320in}}%
\pgfpathlineto{\pgfqpoint{2.543765in}{0.413320in}}%
\pgfpathlineto{\pgfqpoint{2.540949in}{0.413320in}}%
\pgfpathlineto{\pgfqpoint{2.538274in}{0.413320in}}%
\pgfpathlineto{\pgfqpoint{2.535624in}{0.413320in}}%
\pgfpathlineto{\pgfqpoint{2.532917in}{0.413320in}}%
\pgfpathlineto{\pgfqpoint{2.530234in}{0.413320in}}%
\pgfpathlineto{\pgfqpoint{2.527560in}{0.413320in}}%
\pgfpathlineto{\pgfqpoint{2.524988in}{0.413320in}}%
\pgfpathlineto{\pgfqpoint{2.522197in}{0.413320in}}%
\pgfpathlineto{\pgfqpoint{2.519607in}{0.413320in}}%
\pgfpathlineto{\pgfqpoint{2.516845in}{0.413320in}}%
\pgfpathlineto{\pgfqpoint{2.514268in}{0.413320in}}%
\pgfpathlineto{\pgfqpoint{2.511478in}{0.413320in}}%
\pgfpathlineto{\pgfqpoint{2.508917in}{0.413320in}}%
\pgfpathlineto{\pgfqpoint{2.506163in}{0.413320in}}%
\pgfpathlineto{\pgfqpoint{2.503454in}{0.413320in}}%
\pgfpathlineto{\pgfqpoint{2.500801in}{0.413320in}}%
\pgfpathlineto{\pgfqpoint{2.498085in}{0.413320in}}%
\pgfpathlineto{\pgfqpoint{2.495542in}{0.413320in}}%
\pgfpathlineto{\pgfqpoint{2.492729in}{0.413320in}}%
\pgfpathlineto{\pgfqpoint{2.490183in}{0.413320in}}%
\pgfpathlineto{\pgfqpoint{2.487384in}{0.413320in}}%
\pgfpathlineto{\pgfqpoint{2.484870in}{0.413320in}}%
\pgfpathlineto{\pgfqpoint{2.482026in}{0.413320in}}%
\pgfpathlineto{\pgfqpoint{2.479420in}{0.413320in}}%
\pgfpathlineto{\pgfqpoint{2.476671in}{0.413320in}}%
\pgfpathlineto{\pgfqpoint{2.473989in}{0.413320in}}%
\pgfpathlineto{\pgfqpoint{2.471311in}{0.413320in}}%
\pgfpathlineto{\pgfqpoint{2.468635in}{0.413320in}}%
\pgfpathlineto{\pgfqpoint{2.465957in}{0.413320in}}%
\pgfpathlineto{\pgfqpoint{2.463280in}{0.413320in}}%
\pgfpathlineto{\pgfqpoint{2.460711in}{0.413320in}}%
\pgfpathlineto{\pgfqpoint{2.457917in}{0.413320in}}%
\pgfpathlineto{\pgfqpoint{2.455353in}{0.413320in}}%
\pgfpathlineto{\pgfqpoint{2.452562in}{0.413320in}}%
\pgfpathlineto{\pgfqpoint{2.450032in}{0.413320in}}%
\pgfpathlineto{\pgfqpoint{2.447209in}{0.413320in}}%
\pgfpathlineto{\pgfqpoint{2.444677in}{0.413320in}}%
\pgfpathlineto{\pgfqpoint{2.441876in}{0.413320in}}%
\pgfpathlineto{\pgfqpoint{2.439167in}{0.413320in}}%
\pgfpathlineto{\pgfqpoint{2.436518in}{0.413320in}}%
\pgfpathlineto{\pgfqpoint{2.433815in}{0.413320in}}%
\pgfpathlineto{\pgfqpoint{2.431251in}{0.413320in}}%
\pgfpathlineto{\pgfqpoint{2.428453in}{0.413320in}}%
\pgfpathlineto{\pgfqpoint{2.425878in}{0.413320in}}%
\pgfpathlineto{\pgfqpoint{2.423098in}{0.413320in}}%
\pgfpathlineto{\pgfqpoint{2.420528in}{0.413320in}}%
\pgfpathlineto{\pgfqpoint{2.417747in}{0.413320in}}%
\pgfpathlineto{\pgfqpoint{2.415184in}{0.413320in}}%
\pgfpathlineto{\pgfqpoint{2.412389in}{0.413320in}}%
\pgfpathlineto{\pgfqpoint{2.409699in}{0.413320in}}%
\pgfpathlineto{\pgfqpoint{2.407024in}{0.413320in}}%
\pgfpathlineto{\pgfqpoint{2.404352in}{0.413320in}}%
\pgfpathlineto{\pgfqpoint{2.401675in}{0.413320in}}%
\pgfpathlineto{\pgfqpoint{2.398995in}{0.413320in}}%
\pgfpathclose%
\pgfusepath{stroke,fill}%
\end{pgfscope}%
\begin{pgfscope}%
\pgfpathrectangle{\pgfqpoint{2.398995in}{0.319877in}}{\pgfqpoint{3.986877in}{1.993438in}} %
\pgfusepath{clip}%
\pgfsetbuttcap%
\pgfsetroundjoin%
\definecolor{currentfill}{rgb}{1.000000,1.000000,1.000000}%
\pgfsetfillcolor{currentfill}%
\pgfsetlinewidth{1.003750pt}%
\definecolor{currentstroke}{rgb}{0.226681,0.650872,0.856189}%
\pgfsetstrokecolor{currentstroke}%
\pgfsetdash{}{0pt}%
\pgfpathmoveto{\pgfqpoint{2.398995in}{0.413320in}}%
\pgfpathlineto{\pgfqpoint{2.398995in}{0.539642in}}%
\pgfpathlineto{\pgfqpoint{2.401675in}{0.542534in}}%
\pgfpathlineto{\pgfqpoint{2.404352in}{0.546483in}}%
\pgfpathlineto{\pgfqpoint{2.407024in}{0.549924in}}%
\pgfpathlineto{\pgfqpoint{2.409699in}{0.545103in}}%
\pgfpathlineto{\pgfqpoint{2.412389in}{0.544308in}}%
\pgfpathlineto{\pgfqpoint{2.415184in}{0.543198in}}%
\pgfpathlineto{\pgfqpoint{2.417747in}{0.541200in}}%
\pgfpathlineto{\pgfqpoint{2.420528in}{0.542465in}}%
\pgfpathlineto{\pgfqpoint{2.423098in}{0.542537in}}%
\pgfpathlineto{\pgfqpoint{2.425878in}{0.541234in}}%
\pgfpathlineto{\pgfqpoint{2.428453in}{0.550494in}}%
\pgfpathlineto{\pgfqpoint{2.431251in}{0.554047in}}%
\pgfpathlineto{\pgfqpoint{2.433815in}{0.553697in}}%
\pgfpathlineto{\pgfqpoint{2.436518in}{0.541246in}}%
\pgfpathlineto{\pgfqpoint{2.439167in}{0.544559in}}%
\pgfpathlineto{\pgfqpoint{2.441876in}{0.535535in}}%
\pgfpathlineto{\pgfqpoint{2.444677in}{0.537889in}}%
\pgfpathlineto{\pgfqpoint{2.447209in}{0.541273in}}%
\pgfpathlineto{\pgfqpoint{2.450032in}{0.547355in}}%
\pgfpathlineto{\pgfqpoint{2.452562in}{0.543362in}}%
\pgfpathlineto{\pgfqpoint{2.455353in}{0.543002in}}%
\pgfpathlineto{\pgfqpoint{2.457917in}{0.541890in}}%
\pgfpathlineto{\pgfqpoint{2.460711in}{0.539771in}}%
\pgfpathlineto{\pgfqpoint{2.463280in}{0.542867in}}%
\pgfpathlineto{\pgfqpoint{2.465957in}{0.545608in}}%
\pgfpathlineto{\pgfqpoint{2.468635in}{0.543468in}}%
\pgfpathlineto{\pgfqpoint{2.471311in}{0.544029in}}%
\pgfpathlineto{\pgfqpoint{2.473989in}{0.539601in}}%
\pgfpathlineto{\pgfqpoint{2.476671in}{0.542742in}}%
\pgfpathlineto{\pgfqpoint{2.479420in}{0.539904in}}%
\pgfpathlineto{\pgfqpoint{2.482026in}{0.536194in}}%
\pgfpathlineto{\pgfqpoint{2.484870in}{0.553029in}}%
\pgfpathlineto{\pgfqpoint{2.487384in}{0.542509in}}%
\pgfpathlineto{\pgfqpoint{2.490183in}{0.545815in}}%
\pgfpathlineto{\pgfqpoint{2.492729in}{0.547460in}}%
\pgfpathlineto{\pgfqpoint{2.495542in}{0.545093in}}%
\pgfpathlineto{\pgfqpoint{2.498085in}{0.550823in}}%
\pgfpathlineto{\pgfqpoint{2.500801in}{0.548101in}}%
\pgfpathlineto{\pgfqpoint{2.503454in}{0.542127in}}%
\pgfpathlineto{\pgfqpoint{2.506163in}{0.548133in}}%
\pgfpathlineto{\pgfqpoint{2.508917in}{0.544525in}}%
\pgfpathlineto{\pgfqpoint{2.511478in}{0.548275in}}%
\pgfpathlineto{\pgfqpoint{2.514268in}{0.545978in}}%
\pgfpathlineto{\pgfqpoint{2.516845in}{0.547506in}}%
\pgfpathlineto{\pgfqpoint{2.519607in}{0.541024in}}%
\pgfpathlineto{\pgfqpoint{2.522197in}{0.540061in}}%
\pgfpathlineto{\pgfqpoint{2.524988in}{0.542121in}}%
\pgfpathlineto{\pgfqpoint{2.527560in}{0.539249in}}%
\pgfpathlineto{\pgfqpoint{2.530234in}{0.543022in}}%
\pgfpathlineto{\pgfqpoint{2.532917in}{0.540638in}}%
\pgfpathlineto{\pgfqpoint{2.535624in}{0.540893in}}%
\pgfpathlineto{\pgfqpoint{2.538274in}{0.541012in}}%
\pgfpathlineto{\pgfqpoint{2.540949in}{0.544632in}}%
\pgfpathlineto{\pgfqpoint{2.543765in}{0.546267in}}%
\pgfpathlineto{\pgfqpoint{2.546310in}{0.544597in}}%
\pgfpathlineto{\pgfqpoint{2.549114in}{0.551736in}}%
\pgfpathlineto{\pgfqpoint{2.551664in}{0.567869in}}%
\pgfpathlineto{\pgfqpoint{2.554493in}{0.578970in}}%
\pgfpathlineto{\pgfqpoint{2.557009in}{0.573853in}}%
\pgfpathlineto{\pgfqpoint{2.559790in}{0.561354in}}%
\pgfpathlineto{\pgfqpoint{2.562375in}{0.578209in}}%
\pgfpathlineto{\pgfqpoint{2.565045in}{0.582631in}}%
\pgfpathlineto{\pgfqpoint{2.567730in}{0.581025in}}%
\pgfpathlineto{\pgfqpoint{2.570411in}{0.582196in}}%
\pgfpathlineto{\pgfqpoint{2.573082in}{0.579930in}}%
\pgfpathlineto{\pgfqpoint{2.575779in}{0.577642in}}%
\pgfpathlineto{\pgfqpoint{2.578567in}{0.569441in}}%
\pgfpathlineto{\pgfqpoint{2.581129in}{0.567466in}}%
\pgfpathlineto{\pgfqpoint{2.583913in}{0.554558in}}%
\pgfpathlineto{\pgfqpoint{2.586484in}{0.553006in}}%
\pgfpathlineto{\pgfqpoint{2.589248in}{0.556717in}}%
\pgfpathlineto{\pgfqpoint{2.591842in}{0.545860in}}%
\pgfpathlineto{\pgfqpoint{2.594630in}{0.542598in}}%
\pgfpathlineto{\pgfqpoint{2.597196in}{0.542371in}}%
\pgfpathlineto{\pgfqpoint{2.599920in}{0.544248in}}%
\pgfpathlineto{\pgfqpoint{2.602557in}{0.536617in}}%
\pgfpathlineto{\pgfqpoint{2.605232in}{0.539733in}}%
\pgfpathlineto{\pgfqpoint{2.608004in}{0.547280in}}%
\pgfpathlineto{\pgfqpoint{2.610588in}{0.540571in}}%
\pgfpathlineto{\pgfqpoint{2.613393in}{0.540610in}}%
\pgfpathlineto{\pgfqpoint{2.615934in}{0.546931in}}%
\pgfpathlineto{\pgfqpoint{2.618773in}{0.545644in}}%
\pgfpathlineto{\pgfqpoint{2.621304in}{0.542242in}}%
\pgfpathlineto{\pgfqpoint{2.624077in}{0.540405in}}%
\pgfpathlineto{\pgfqpoint{2.626653in}{0.541190in}}%
\pgfpathlineto{\pgfqpoint{2.629340in}{0.540695in}}%
\pgfpathlineto{\pgfqpoint{2.632018in}{0.541601in}}%
\pgfpathlineto{\pgfqpoint{2.634700in}{0.537641in}}%
\pgfpathlineto{\pgfqpoint{2.637369in}{0.543048in}}%
\pgfpathlineto{\pgfqpoint{2.640053in}{0.540493in}}%
\pgfpathlineto{\pgfqpoint{2.642827in}{0.540096in}}%
\pgfpathlineto{\pgfqpoint{2.645408in}{0.540881in}}%
\pgfpathlineto{\pgfqpoint{2.648196in}{0.543074in}}%
\pgfpathlineto{\pgfqpoint{2.650767in}{0.550521in}}%
\pgfpathlineto{\pgfqpoint{2.653567in}{0.541862in}}%
\pgfpathlineto{\pgfqpoint{2.656124in}{0.534669in}}%
\pgfpathlineto{\pgfqpoint{2.658942in}{0.542052in}}%
\pgfpathlineto{\pgfqpoint{2.661481in}{0.536196in}}%
\pgfpathlineto{\pgfqpoint{2.664151in}{0.529878in}}%
\pgfpathlineto{\pgfqpoint{2.666836in}{0.537327in}}%
\pgfpathlineto{\pgfqpoint{2.669506in}{0.542204in}}%
\pgfpathlineto{\pgfqpoint{2.672301in}{0.539800in}}%
\pgfpathlineto{\pgfqpoint{2.674873in}{0.536692in}}%
\pgfpathlineto{\pgfqpoint{2.677650in}{0.540401in}}%
\pgfpathlineto{\pgfqpoint{2.680224in}{0.540690in}}%
\pgfpathlineto{\pgfqpoint{2.683009in}{0.540697in}}%
\pgfpathlineto{\pgfqpoint{2.685586in}{0.542271in}}%
\pgfpathlineto{\pgfqpoint{2.688328in}{0.545405in}}%
\pgfpathlineto{\pgfqpoint{2.690940in}{0.541733in}}%
\pgfpathlineto{\pgfqpoint{2.693611in}{0.542925in}}%
\pgfpathlineto{\pgfqpoint{2.696293in}{0.540798in}}%
\pgfpathlineto{\pgfqpoint{2.698968in}{0.547786in}}%
\pgfpathlineto{\pgfqpoint{2.701657in}{0.547501in}}%
\pgfpathlineto{\pgfqpoint{2.704326in}{0.544289in}}%
\pgfpathlineto{\pgfqpoint{2.707125in}{0.550282in}}%
\pgfpathlineto{\pgfqpoint{2.709683in}{0.550505in}}%
\pgfpathlineto{\pgfqpoint{2.712477in}{0.551403in}}%
\pgfpathlineto{\pgfqpoint{2.715036in}{0.551677in}}%
\pgfpathlineto{\pgfqpoint{2.717773in}{0.545665in}}%
\pgfpathlineto{\pgfqpoint{2.720404in}{0.545964in}}%
\pgfpathlineto{\pgfqpoint{2.723211in}{0.546048in}}%
\pgfpathlineto{\pgfqpoint{2.725760in}{0.545139in}}%
\pgfpathlineto{\pgfqpoint{2.728439in}{0.553751in}}%
\pgfpathlineto{\pgfqpoint{2.731119in}{0.554135in}}%
\pgfpathlineto{\pgfqpoint{2.733798in}{0.550732in}}%
\pgfpathlineto{\pgfqpoint{2.736476in}{0.548720in}}%
\pgfpathlineto{\pgfqpoint{2.739155in}{0.544273in}}%
\pgfpathlineto{\pgfqpoint{2.741928in}{0.540569in}}%
\pgfpathlineto{\pgfqpoint{2.744510in}{0.544078in}}%
\pgfpathlineto{\pgfqpoint{2.747260in}{0.545624in}}%
\pgfpathlineto{\pgfqpoint{2.749868in}{0.545863in}}%
\pgfpathlineto{\pgfqpoint{2.752614in}{0.541936in}}%
\pgfpathlineto{\pgfqpoint{2.755224in}{0.548420in}}%
\pgfpathlineto{\pgfqpoint{2.758028in}{0.539694in}}%
\pgfpathlineto{\pgfqpoint{2.760581in}{0.550564in}}%
\pgfpathlineto{\pgfqpoint{2.763253in}{0.537829in}}%
\pgfpathlineto{\pgfqpoint{2.765935in}{0.541003in}}%
\pgfpathlineto{\pgfqpoint{2.768617in}{0.539089in}}%
\pgfpathlineto{\pgfqpoint{2.771373in}{0.546695in}}%
\pgfpathlineto{\pgfqpoint{2.773972in}{0.547311in}}%
\pgfpathlineto{\pgfqpoint{2.776767in}{0.544660in}}%
\pgfpathlineto{\pgfqpoint{2.779330in}{0.547745in}}%
\pgfpathlineto{\pgfqpoint{2.782113in}{0.546111in}}%
\pgfpathlineto{\pgfqpoint{2.784687in}{0.547512in}}%
\pgfpathlineto{\pgfqpoint{2.787468in}{0.545603in}}%
\pgfpathlineto{\pgfqpoint{2.790044in}{0.543205in}}%
\pgfpathlineto{\pgfqpoint{2.792721in}{0.542507in}}%
\pgfpathlineto{\pgfqpoint{2.795398in}{0.546637in}}%
\pgfpathlineto{\pgfqpoint{2.798070in}{0.544081in}}%
\pgfpathlineto{\pgfqpoint{2.800756in}{0.544083in}}%
\pgfpathlineto{\pgfqpoint{2.803435in}{0.541550in}}%
\pgfpathlineto{\pgfqpoint{2.806175in}{0.540005in}}%
\pgfpathlineto{\pgfqpoint{2.808792in}{0.549125in}}%
\pgfpathlineto{\pgfqpoint{2.811597in}{0.543390in}}%
\pgfpathlineto{\pgfqpoint{2.814141in}{0.544809in}}%
\pgfpathlineto{\pgfqpoint{2.816867in}{0.545747in}}%
\pgfpathlineto{\pgfqpoint{2.819506in}{0.538015in}}%
\pgfpathlineto{\pgfqpoint{2.822303in}{0.543294in}}%
\pgfpathlineto{\pgfqpoint{2.824851in}{0.543313in}}%
\pgfpathlineto{\pgfqpoint{2.827567in}{0.539262in}}%
\pgfpathlineto{\pgfqpoint{2.830219in}{0.540525in}}%
\pgfpathlineto{\pgfqpoint{2.832894in}{0.543769in}}%
\pgfpathlineto{\pgfqpoint{2.835698in}{0.546564in}}%
\pgfpathlineto{\pgfqpoint{2.838254in}{0.544956in}}%
\pgfpathlineto{\pgfqpoint{2.841055in}{0.550380in}}%
\pgfpathlineto{\pgfqpoint{2.843611in}{0.545050in}}%
\pgfpathlineto{\pgfqpoint{2.846408in}{0.545987in}}%
\pgfpathlineto{\pgfqpoint{2.848960in}{0.539840in}}%
\pgfpathlineto{\pgfqpoint{2.851793in}{0.543251in}}%
\pgfpathlineto{\pgfqpoint{2.854325in}{0.538673in}}%
\pgfpathlineto{\pgfqpoint{2.857003in}{0.545179in}}%
\pgfpathlineto{\pgfqpoint{2.859668in}{0.541822in}}%
\pgfpathlineto{\pgfqpoint{2.862402in}{0.542752in}}%
\pgfpathlineto{\pgfqpoint{2.865031in}{0.538520in}}%
\pgfpathlineto{\pgfqpoint{2.867713in}{0.545011in}}%
\pgfpathlineto{\pgfqpoint{2.870475in}{0.544508in}}%
\pgfpathlineto{\pgfqpoint{2.873074in}{0.549767in}}%
\pgfpathlineto{\pgfqpoint{2.875882in}{0.543055in}}%
\pgfpathlineto{\pgfqpoint{2.878431in}{0.544242in}}%
\pgfpathlineto{\pgfqpoint{2.881254in}{0.542944in}}%
\pgfpathlineto{\pgfqpoint{2.883780in}{0.542192in}}%
\pgfpathlineto{\pgfqpoint{2.886578in}{0.539707in}}%
\pgfpathlineto{\pgfqpoint{2.889145in}{0.544809in}}%
\pgfpathlineto{\pgfqpoint{2.891809in}{0.538285in}}%
\pgfpathlineto{\pgfqpoint{2.894487in}{0.544606in}}%
\pgfpathlineto{\pgfqpoint{2.897179in}{0.549676in}}%
\pgfpathlineto{\pgfqpoint{2.899858in}{0.544696in}}%
\pgfpathlineto{\pgfqpoint{2.902535in}{0.547012in}}%
\pgfpathlineto{\pgfqpoint{2.905341in}{0.544146in}}%
\pgfpathlineto{\pgfqpoint{2.907882in}{0.545683in}}%
\pgfpathlineto{\pgfqpoint{2.910631in}{0.545679in}}%
\pgfpathlineto{\pgfqpoint{2.913243in}{0.537768in}}%
\pgfpathlineto{\pgfqpoint{2.916061in}{0.539292in}}%
\pgfpathlineto{\pgfqpoint{2.918606in}{0.545784in}}%
\pgfpathlineto{\pgfqpoint{2.921363in}{0.545369in}}%
\pgfpathlineto{\pgfqpoint{2.923963in}{0.547094in}}%
\pgfpathlineto{\pgfqpoint{2.926655in}{0.546897in}}%
\pgfpathlineto{\pgfqpoint{2.929321in}{0.544339in}}%
\pgfpathlineto{\pgfqpoint{2.932033in}{0.545710in}}%
\pgfpathlineto{\pgfqpoint{2.934759in}{0.538125in}}%
\pgfpathlineto{\pgfqpoint{2.937352in}{0.543339in}}%
\pgfpathlineto{\pgfqpoint{2.940120in}{0.540751in}}%
\pgfpathlineto{\pgfqpoint{2.942711in}{0.532961in}}%
\pgfpathlineto{\pgfqpoint{2.945461in}{0.538638in}}%
\pgfpathlineto{\pgfqpoint{2.948068in}{0.536115in}}%
\pgfpathlineto{\pgfqpoint{2.950884in}{0.540425in}}%
\pgfpathlineto{\pgfqpoint{2.953422in}{0.541520in}}%
\pgfpathlineto{\pgfqpoint{2.956103in}{0.536967in}}%
\pgfpathlineto{\pgfqpoint{2.958782in}{0.539909in}}%
\pgfpathlineto{\pgfqpoint{2.961460in}{0.540558in}}%
\pgfpathlineto{\pgfqpoint{2.964127in}{0.540418in}}%
\pgfpathlineto{\pgfqpoint{2.966812in}{0.540695in}}%
\pgfpathlineto{\pgfqpoint{2.969599in}{0.541813in}}%
\pgfpathlineto{\pgfqpoint{2.972177in}{0.542346in}}%
\pgfpathlineto{\pgfqpoint{2.974972in}{0.545588in}}%
\pgfpathlineto{\pgfqpoint{2.977517in}{0.548160in}}%
\pgfpathlineto{\pgfqpoint{2.980341in}{0.543714in}}%
\pgfpathlineto{\pgfqpoint{2.982885in}{0.546458in}}%
\pgfpathlineto{\pgfqpoint{2.985666in}{0.542508in}}%
\pgfpathlineto{\pgfqpoint{2.988238in}{0.544353in}}%
\pgfpathlineto{\pgfqpoint{2.990978in}{0.544337in}}%
\pgfpathlineto{\pgfqpoint{2.993595in}{0.556752in}}%
\pgfpathlineto{\pgfqpoint{2.996300in}{0.558747in}}%
\pgfpathlineto{\pgfqpoint{2.999103in}{0.561091in}}%
\pgfpathlineto{\pgfqpoint{3.001635in}{0.560892in}}%
\pgfpathlineto{\pgfqpoint{3.004419in}{0.549161in}}%
\pgfpathlineto{\pgfqpoint{3.006993in}{0.554322in}}%
\pgfpathlineto{\pgfqpoint{3.009784in}{0.546739in}}%
\pgfpathlineto{\pgfqpoint{3.012351in}{0.548203in}}%
\pgfpathlineto{\pgfqpoint{3.015097in}{0.546290in}}%
\pgfpathlineto{\pgfqpoint{3.017707in}{0.546418in}}%
\pgfpathlineto{\pgfqpoint{3.020382in}{0.541500in}}%
\pgfpathlineto{\pgfqpoint{3.023058in}{0.562500in}}%
\pgfpathlineto{\pgfqpoint{3.025803in}{0.567547in}}%
\pgfpathlineto{\pgfqpoint{3.028412in}{0.569078in}}%
\pgfpathlineto{\pgfqpoint{3.031091in}{0.557734in}}%
\pgfpathlineto{\pgfqpoint{3.033921in}{0.550627in}}%
\pgfpathlineto{\pgfqpoint{3.036456in}{0.550152in}}%
\pgfpathlineto{\pgfqpoint{3.039262in}{0.548940in}}%
\pgfpathlineto{\pgfqpoint{3.041813in}{0.547790in}}%
\pgfpathlineto{\pgfqpoint{3.044568in}{0.546319in}}%
\pgfpathlineto{\pgfqpoint{3.047157in}{0.541690in}}%
\pgfpathlineto{\pgfqpoint{3.049988in}{0.535662in}}%
\pgfpathlineto{\pgfqpoint{3.052526in}{0.531984in}}%
\pgfpathlineto{\pgfqpoint{3.055202in}{0.535655in}}%
\pgfpathlineto{\pgfqpoint{3.057884in}{0.537884in}}%
\pgfpathlineto{\pgfqpoint{3.060561in}{0.544124in}}%
\pgfpathlineto{\pgfqpoint{3.063230in}{0.538396in}}%
\pgfpathlineto{\pgfqpoint{3.065916in}{0.541557in}}%
\pgfpathlineto{\pgfqpoint{3.068709in}{0.542332in}}%
\pgfpathlineto{\pgfqpoint{3.071266in}{0.543668in}}%
\pgfpathlineto{\pgfqpoint{3.074056in}{0.542900in}}%
\pgfpathlineto{\pgfqpoint{3.076631in}{0.545385in}}%
\pgfpathlineto{\pgfqpoint{3.079381in}{0.537160in}}%
\pgfpathlineto{\pgfqpoint{3.081990in}{0.538839in}}%
\pgfpathlineto{\pgfqpoint{3.084671in}{0.538228in}}%
\pgfpathlineto{\pgfqpoint{3.087343in}{0.543460in}}%
\pgfpathlineto{\pgfqpoint{3.090023in}{0.539457in}}%
\pgfpathlineto{\pgfqpoint{3.092699in}{0.537403in}}%
\pgfpathlineto{\pgfqpoint{3.095388in}{0.542502in}}%
\pgfpathlineto{\pgfqpoint{3.098163in}{0.541394in}}%
\pgfpathlineto{\pgfqpoint{3.100737in}{0.552140in}}%
\pgfpathlineto{\pgfqpoint{3.103508in}{0.557938in}}%
\pgfpathlineto{\pgfqpoint{3.106094in}{0.549015in}}%
\pgfpathlineto{\pgfqpoint{3.108896in}{0.533238in}}%
\pgfpathlineto{\pgfqpoint{3.111451in}{0.543941in}}%
\pgfpathlineto{\pgfqpoint{3.114242in}{0.543341in}}%
\pgfpathlineto{\pgfqpoint{3.116807in}{0.541028in}}%
\pgfpathlineto{\pgfqpoint{3.119487in}{0.536606in}}%
\pgfpathlineto{\pgfqpoint{3.122163in}{0.542111in}}%
\pgfpathlineto{\pgfqpoint{3.124842in}{0.544881in}}%
\pgfpathlineto{\pgfqpoint{3.127512in}{0.547466in}}%
\pgfpathlineto{\pgfqpoint{3.130199in}{0.548567in}}%
\pgfpathlineto{\pgfqpoint{3.132946in}{0.543963in}}%
\pgfpathlineto{\pgfqpoint{3.135550in}{0.550483in}}%
\pgfpathlineto{\pgfqpoint{3.138375in}{0.547603in}}%
\pgfpathlineto{\pgfqpoint{3.140913in}{0.548023in}}%
\pgfpathlineto{\pgfqpoint{3.143740in}{0.534871in}}%
\pgfpathlineto{\pgfqpoint{3.146271in}{0.533597in}}%
\pgfpathlineto{\pgfqpoint{3.149057in}{0.531385in}}%
\pgfpathlineto{\pgfqpoint{3.151612in}{0.529918in}}%
\pgfpathlineto{\pgfqpoint{3.154327in}{0.543840in}}%
\pgfpathlineto{\pgfqpoint{3.156981in}{0.548171in}}%
\pgfpathlineto{\pgfqpoint{3.159675in}{0.541726in}}%
\pgfpathlineto{\pgfqpoint{3.162474in}{0.541023in}}%
\pgfpathlineto{\pgfqpoint{3.165019in}{0.538681in}}%
\pgfpathlineto{\pgfqpoint{3.167776in}{0.529878in}}%
\pgfpathlineto{\pgfqpoint{3.170375in}{0.529878in}}%
\pgfpathlineto{\pgfqpoint{3.173142in}{0.529878in}}%
\pgfpathlineto{\pgfqpoint{3.175724in}{0.533904in}}%
\pgfpathlineto{\pgfqpoint{3.178525in}{0.536361in}}%
\pgfpathlineto{\pgfqpoint{3.181089in}{0.541654in}}%
\pgfpathlineto{\pgfqpoint{3.183760in}{0.535411in}}%
\pgfpathlineto{\pgfqpoint{3.186440in}{0.536659in}}%
\pgfpathlineto{\pgfqpoint{3.189117in}{0.537603in}}%
\pgfpathlineto{\pgfqpoint{3.191796in}{0.535091in}}%
\pgfpathlineto{\pgfqpoint{3.194508in}{0.540008in}}%
\pgfpathlineto{\pgfqpoint{3.197226in}{0.533766in}}%
\pgfpathlineto{\pgfqpoint{3.199823in}{0.532569in}}%
\pgfpathlineto{\pgfqpoint{3.202562in}{0.529878in}}%
\pgfpathlineto{\pgfqpoint{3.205195in}{0.536403in}}%
\pgfpathlineto{\pgfqpoint{3.207984in}{0.537477in}}%
\pgfpathlineto{\pgfqpoint{3.210545in}{0.537568in}}%
\pgfpathlineto{\pgfqpoint{3.213342in}{0.533930in}}%
\pgfpathlineto{\pgfqpoint{3.215908in}{0.539574in}}%
\pgfpathlineto{\pgfqpoint{3.218586in}{0.543396in}}%
\pgfpathlineto{\pgfqpoint{3.221255in}{0.538382in}}%
\pgfpathlineto{\pgfqpoint{3.223942in}{0.545539in}}%
\pgfpathlineto{\pgfqpoint{3.226609in}{0.559895in}}%
\pgfpathlineto{\pgfqpoint{3.229310in}{0.554374in}}%
\pgfpathlineto{\pgfqpoint{3.232069in}{0.540537in}}%
\pgfpathlineto{\pgfqpoint{3.234658in}{0.545522in}}%
\pgfpathlineto{\pgfqpoint{3.237411in}{0.545207in}}%
\pgfpathlineto{\pgfqpoint{3.240010in}{0.551272in}}%
\pgfpathlineto{\pgfqpoint{3.242807in}{0.558412in}}%
\pgfpathlineto{\pgfqpoint{3.245363in}{0.555464in}}%
\pgfpathlineto{\pgfqpoint{3.248049in}{0.549953in}}%
\pgfpathlineto{\pgfqpoint{3.250716in}{0.551763in}}%
\pgfpathlineto{\pgfqpoint{3.253404in}{0.549375in}}%
\pgfpathlineto{\pgfqpoint{3.256083in}{0.542613in}}%
\pgfpathlineto{\pgfqpoint{3.258784in}{0.546302in}}%
\pgfpathlineto{\pgfqpoint{3.261594in}{0.546318in}}%
\pgfpathlineto{\pgfqpoint{3.264119in}{0.544429in}}%
\pgfpathlineto{\pgfqpoint{3.266849in}{0.549092in}}%
\pgfpathlineto{\pgfqpoint{3.269478in}{0.548523in}}%
\pgfpathlineto{\pgfqpoint{3.272254in}{0.549113in}}%
\pgfpathlineto{\pgfqpoint{3.274831in}{0.545732in}}%
\pgfpathlineto{\pgfqpoint{3.277603in}{0.549525in}}%
\pgfpathlineto{\pgfqpoint{3.280189in}{0.539752in}}%
\pgfpathlineto{\pgfqpoint{3.282870in}{0.549628in}}%
\pgfpathlineto{\pgfqpoint{3.285534in}{0.552506in}}%
\pgfpathlineto{\pgfqpoint{3.288225in}{0.546502in}}%
\pgfpathlineto{\pgfqpoint{3.290890in}{0.550931in}}%
\pgfpathlineto{\pgfqpoint{3.293574in}{0.547969in}}%
\pgfpathlineto{\pgfqpoint{3.296376in}{0.546899in}}%
\pgfpathlineto{\pgfqpoint{3.298937in}{0.548251in}}%
\pgfpathlineto{\pgfqpoint{3.301719in}{0.551588in}}%
\pgfpathlineto{\pgfqpoint{3.304295in}{0.544601in}}%
\pgfpathlineto{\pgfqpoint{3.307104in}{0.546292in}}%
\pgfpathlineto{\pgfqpoint{3.309652in}{0.545182in}}%
\pgfpathlineto{\pgfqpoint{3.312480in}{0.544561in}}%
\pgfpathlineto{\pgfqpoint{3.315008in}{0.546557in}}%
\pgfpathlineto{\pgfqpoint{3.317688in}{0.543193in}}%
\pgfpathlineto{\pgfqpoint{3.320366in}{0.544623in}}%
\pgfpathlineto{\pgfqpoint{3.323049in}{0.543898in}}%
\pgfpathlineto{\pgfqpoint{3.325860in}{0.541275in}}%
\pgfpathlineto{\pgfqpoint{3.328401in}{0.541383in}}%
\pgfpathlineto{\pgfqpoint{3.331183in}{0.544163in}}%
\pgfpathlineto{\pgfqpoint{3.333758in}{0.545530in}}%
\pgfpathlineto{\pgfqpoint{3.336541in}{0.541776in}}%
\pgfpathlineto{\pgfqpoint{3.339101in}{0.546331in}}%
\pgfpathlineto{\pgfqpoint{3.341893in}{0.547763in}}%
\pgfpathlineto{\pgfqpoint{3.344468in}{0.546799in}}%
\pgfpathlineto{\pgfqpoint{3.347139in}{0.542833in}}%
\pgfpathlineto{\pgfqpoint{3.349828in}{0.544591in}}%
\pgfpathlineto{\pgfqpoint{3.352505in}{0.542449in}}%
\pgfpathlineto{\pgfqpoint{3.355177in}{0.547972in}}%
\pgfpathlineto{\pgfqpoint{3.357862in}{0.549844in}}%
\pgfpathlineto{\pgfqpoint{3.360620in}{0.545003in}}%
\pgfpathlineto{\pgfqpoint{3.363221in}{0.547398in}}%
\pgfpathlineto{\pgfqpoint{3.365996in}{0.544075in}}%
\pgfpathlineto{\pgfqpoint{3.368577in}{0.545802in}}%
\pgfpathlineto{\pgfqpoint{3.371357in}{0.540840in}}%
\pgfpathlineto{\pgfqpoint{3.373921in}{0.538039in}}%
\pgfpathlineto{\pgfqpoint{3.376735in}{0.537635in}}%
\pgfpathlineto{\pgfqpoint{3.379290in}{0.539572in}}%
\pgfpathlineto{\pgfqpoint{3.381959in}{0.546814in}}%
\pgfpathlineto{\pgfqpoint{3.384647in}{0.545341in}}%
\pgfpathlineto{\pgfqpoint{3.387309in}{0.544045in}}%
\pgfpathlineto{\pgfqpoint{3.390102in}{0.544723in}}%
\pgfpathlineto{\pgfqpoint{3.392681in}{0.548764in}}%
\pgfpathlineto{\pgfqpoint{3.395461in}{0.542992in}}%
\pgfpathlineto{\pgfqpoint{3.398037in}{0.542425in}}%
\pgfpathlineto{\pgfqpoint{3.400783in}{0.542383in}}%
\pgfpathlineto{\pgfqpoint{3.403394in}{0.550906in}}%
\pgfpathlineto{\pgfqpoint{3.406202in}{0.553418in}}%
\pgfpathlineto{\pgfqpoint{3.408752in}{0.544564in}}%
\pgfpathlineto{\pgfqpoint{3.411431in}{0.551907in}}%
\pgfpathlineto{\pgfqpoint{3.414109in}{0.545677in}}%
\pgfpathlineto{\pgfqpoint{3.416780in}{0.550940in}}%
\pgfpathlineto{\pgfqpoint{3.419455in}{0.553259in}}%
\pgfpathlineto{\pgfqpoint{3.422142in}{0.551930in}}%
\pgfpathlineto{\pgfqpoint{3.424887in}{0.544928in}}%
\pgfpathlineto{\pgfqpoint{3.427501in}{0.541953in}}%
\pgfpathlineto{\pgfqpoint{3.430313in}{0.545889in}}%
\pgfpathlineto{\pgfqpoint{3.432851in}{0.549109in}}%
\pgfpathlineto{\pgfqpoint{3.435635in}{0.549903in}}%
\pgfpathlineto{\pgfqpoint{3.438210in}{0.547126in}}%
\pgfpathlineto{\pgfqpoint{3.440996in}{0.548305in}}%
\pgfpathlineto{\pgfqpoint{3.443574in}{0.545864in}}%
\pgfpathlineto{\pgfqpoint{3.446257in}{0.552206in}}%
\pgfpathlineto{\pgfqpoint{3.448926in}{0.549547in}}%
\pgfpathlineto{\pgfqpoint{3.451597in}{0.546524in}}%
\pgfpathlineto{\pgfqpoint{3.454285in}{0.550185in}}%
\pgfpathlineto{\pgfqpoint{3.456960in}{0.545033in}}%
\pgfpathlineto{\pgfqpoint{3.459695in}{0.540650in}}%
\pgfpathlineto{\pgfqpoint{3.462321in}{0.543838in}}%
\pgfpathlineto{\pgfqpoint{3.465072in}{0.538361in}}%
\pgfpathlineto{\pgfqpoint{3.467678in}{0.543438in}}%
\pgfpathlineto{\pgfqpoint{3.470466in}{0.542702in}}%
\pgfpathlineto{\pgfqpoint{3.473021in}{0.541284in}}%
\pgfpathlineto{\pgfqpoint{3.475821in}{0.545698in}}%
\pgfpathlineto{\pgfqpoint{3.478378in}{0.540894in}}%
\pgfpathlineto{\pgfqpoint{3.481072in}{0.543399in}}%
\pgfpathlineto{\pgfqpoint{3.483744in}{0.547395in}}%
\pgfpathlineto{\pgfqpoint{3.486442in}{0.551167in}}%
\pgfpathlineto{\pgfqpoint{3.489223in}{0.541307in}}%
\pgfpathlineto{\pgfqpoint{3.491783in}{0.544035in}}%
\pgfpathlineto{\pgfqpoint{3.494581in}{0.539720in}}%
\pgfpathlineto{\pgfqpoint{3.497139in}{0.546274in}}%
\pgfpathlineto{\pgfqpoint{3.499909in}{0.537514in}}%
\pgfpathlineto{\pgfqpoint{3.502488in}{0.544407in}}%
\pgfpathlineto{\pgfqpoint{3.505262in}{0.543088in}}%
\pgfpathlineto{\pgfqpoint{3.507840in}{0.551859in}}%
\pgfpathlineto{\pgfqpoint{3.510533in}{0.555165in}}%
\pgfpathlineto{\pgfqpoint{3.513209in}{0.549451in}}%
\pgfpathlineto{\pgfqpoint{3.515884in}{0.551347in}}%
\pgfpathlineto{\pgfqpoint{3.518565in}{0.547100in}}%
\pgfpathlineto{\pgfqpoint{3.521244in}{0.544058in}}%
\pgfpathlineto{\pgfqpoint{3.524041in}{0.547004in}}%
\pgfpathlineto{\pgfqpoint{3.526601in}{0.546568in}}%
\pgfpathlineto{\pgfqpoint{3.529327in}{0.545542in}}%
\pgfpathlineto{\pgfqpoint{3.531955in}{0.545612in}}%
\pgfpathlineto{\pgfqpoint{3.534783in}{0.546311in}}%
\pgfpathlineto{\pgfqpoint{3.537309in}{0.546935in}}%
\pgfpathlineto{\pgfqpoint{3.540093in}{0.541189in}}%
\pgfpathlineto{\pgfqpoint{3.542656in}{0.548081in}}%
\pgfpathlineto{\pgfqpoint{3.545349in}{0.547403in}}%
\pgfpathlineto{\pgfqpoint{3.548029in}{0.543015in}}%
\pgfpathlineto{\pgfqpoint{3.550713in}{0.546171in}}%
\pgfpathlineto{\pgfqpoint{3.553498in}{0.549086in}}%
\pgfpathlineto{\pgfqpoint{3.556061in}{0.549280in}}%
\pgfpathlineto{\pgfqpoint{3.558853in}{0.546614in}}%
\pgfpathlineto{\pgfqpoint{3.561420in}{0.543953in}}%
\pgfpathlineto{\pgfqpoint{3.564188in}{0.550425in}}%
\pgfpathlineto{\pgfqpoint{3.566774in}{0.546235in}}%
\pgfpathlineto{\pgfqpoint{3.569584in}{0.553273in}}%
\pgfpathlineto{\pgfqpoint{3.572126in}{0.550390in}}%
\pgfpathlineto{\pgfqpoint{3.574814in}{0.543693in}}%
\pgfpathlineto{\pgfqpoint{3.577487in}{0.549327in}}%
\pgfpathlineto{\pgfqpoint{3.580191in}{0.549097in}}%
\pgfpathlineto{\pgfqpoint{3.582851in}{0.549854in}}%
\pgfpathlineto{\pgfqpoint{3.585532in}{0.556109in}}%
\pgfpathlineto{\pgfqpoint{3.588258in}{0.554710in}}%
\pgfpathlineto{\pgfqpoint{3.590883in}{0.555496in}}%
\pgfpathlineto{\pgfqpoint{3.593620in}{0.547112in}}%
\pgfpathlineto{\pgfqpoint{3.596240in}{0.549464in}}%
\pgfpathlineto{\pgfqpoint{3.598998in}{0.549520in}}%
\pgfpathlineto{\pgfqpoint{3.601590in}{0.544540in}}%
\pgfpathlineto{\pgfqpoint{3.604387in}{0.544517in}}%
\pgfpathlineto{\pgfqpoint{3.606951in}{0.545046in}}%
\pgfpathlineto{\pgfqpoint{3.609632in}{0.545592in}}%
\pgfpathlineto{\pgfqpoint{3.612311in}{0.543353in}}%
\pgfpathlineto{\pgfqpoint{3.614982in}{0.535798in}}%
\pgfpathlineto{\pgfqpoint{3.617667in}{0.540173in}}%
\pgfpathlineto{\pgfqpoint{3.620345in}{0.541403in}}%
\pgfpathlineto{\pgfqpoint{3.623165in}{0.541676in}}%
\pgfpathlineto{\pgfqpoint{3.625689in}{0.540680in}}%
\pgfpathlineto{\pgfqpoint{3.628460in}{0.536865in}}%
\pgfpathlineto{\pgfqpoint{3.631058in}{0.540001in}}%
\pgfpathlineto{\pgfqpoint{3.633858in}{0.549472in}}%
\pgfpathlineto{\pgfqpoint{3.636413in}{0.546083in}}%
\pgfpathlineto{\pgfqpoint{3.639207in}{0.539361in}}%
\pgfpathlineto{\pgfqpoint{3.641773in}{0.541470in}}%
\pgfpathlineto{\pgfqpoint{3.644452in}{0.537829in}}%
\pgfpathlineto{\pgfqpoint{3.647130in}{0.529878in}}%
\pgfpathlineto{\pgfqpoint{3.649837in}{0.537459in}}%
\pgfpathlineto{\pgfqpoint{3.652628in}{0.541087in}}%
\pgfpathlineto{\pgfqpoint{3.655165in}{0.534586in}}%
\pgfpathlineto{\pgfqpoint{3.657917in}{0.529878in}}%
\pgfpathlineto{\pgfqpoint{3.660515in}{0.529878in}}%
\pgfpathlineto{\pgfqpoint{3.663276in}{0.529878in}}%
\pgfpathlineto{\pgfqpoint{3.665864in}{0.529878in}}%
\pgfpathlineto{\pgfqpoint{3.668665in}{0.532013in}}%
\pgfpathlineto{\pgfqpoint{3.671232in}{0.534873in}}%
\pgfpathlineto{\pgfqpoint{3.673911in}{0.539992in}}%
\pgfpathlineto{\pgfqpoint{3.676591in}{0.535436in}}%
\pgfpathlineto{\pgfqpoint{3.679273in}{0.529878in}}%
\pgfpathlineto{\pgfqpoint{3.681948in}{0.534021in}}%
\pgfpathlineto{\pgfqpoint{3.684620in}{0.533094in}}%
\pgfpathlineto{\pgfqpoint{3.687442in}{0.541708in}}%
\pgfpathlineto{\pgfqpoint{3.689983in}{0.541686in}}%
\pgfpathlineto{\pgfqpoint{3.692765in}{0.541500in}}%
\pgfpathlineto{\pgfqpoint{3.695331in}{0.543310in}}%
\pgfpathlineto{\pgfqpoint{3.698125in}{0.546032in}}%
\pgfpathlineto{\pgfqpoint{3.700684in}{0.542353in}}%
\pgfpathlineto{\pgfqpoint{3.703460in}{0.540067in}}%
\pgfpathlineto{\pgfqpoint{3.706053in}{0.540865in}}%
\pgfpathlineto{\pgfqpoint{3.708729in}{0.534629in}}%
\pgfpathlineto{\pgfqpoint{3.711410in}{0.542327in}}%
\pgfpathlineto{\pgfqpoint{3.714086in}{0.543337in}}%
\pgfpathlineto{\pgfqpoint{3.716875in}{0.545700in}}%
\pgfpathlineto{\pgfqpoint{3.719446in}{0.542978in}}%
\pgfpathlineto{\pgfqpoint{3.722228in}{0.535262in}}%
\pgfpathlineto{\pgfqpoint{3.724804in}{0.538191in}}%
\pgfpathlineto{\pgfqpoint{3.727581in}{0.540915in}}%
\pgfpathlineto{\pgfqpoint{3.730158in}{0.540496in}}%
\pgfpathlineto{\pgfqpoint{3.732950in}{0.541572in}}%
\pgfpathlineto{\pgfqpoint{3.735509in}{0.544685in}}%
\pgfpathlineto{\pgfqpoint{3.738194in}{0.544667in}}%
\pgfpathlineto{\pgfqpoint{3.740874in}{0.544458in}}%
\pgfpathlineto{\pgfqpoint{3.743548in}{0.548012in}}%
\pgfpathlineto{\pgfqpoint{3.746229in}{0.542242in}}%
\pgfpathlineto{\pgfqpoint{3.748903in}{0.542155in}}%
\pgfpathlineto{\pgfqpoint{3.751728in}{0.541407in}}%
\pgfpathlineto{\pgfqpoint{3.754265in}{0.546721in}}%
\pgfpathlineto{\pgfqpoint{3.757065in}{0.540148in}}%
\pgfpathlineto{\pgfqpoint{3.759622in}{0.548380in}}%
\pgfpathlineto{\pgfqpoint{3.762389in}{0.546704in}}%
\pgfpathlineto{\pgfqpoint{3.764966in}{0.566301in}}%
\pgfpathlineto{\pgfqpoint{3.767782in}{0.569301in}}%
\pgfpathlineto{\pgfqpoint{3.770323in}{0.603508in}}%
\pgfpathlineto{\pgfqpoint{3.773014in}{0.608055in}}%
\pgfpathlineto{\pgfqpoint{3.775691in}{0.674271in}}%
\pgfpathlineto{\pgfqpoint{3.778370in}{0.693566in}}%
\pgfpathlineto{\pgfqpoint{3.781046in}{0.718686in}}%
\pgfpathlineto{\pgfqpoint{3.783725in}{0.665159in}}%
\pgfpathlineto{\pgfqpoint{3.786504in}{0.624674in}}%
\pgfpathlineto{\pgfqpoint{3.789084in}{0.612012in}}%
\pgfpathlineto{\pgfqpoint{3.791897in}{0.594794in}}%
\pgfpathlineto{\pgfqpoint{3.794435in}{0.660769in}}%
\pgfpathlineto{\pgfqpoint{3.797265in}{0.703501in}}%
\pgfpathlineto{\pgfqpoint{3.799797in}{0.734345in}}%
\pgfpathlineto{\pgfqpoint{3.802569in}{0.715069in}}%
\pgfpathlineto{\pgfqpoint{3.805145in}{0.680210in}}%
\pgfpathlineto{\pgfqpoint{3.807832in}{0.657822in}}%
\pgfpathlineto{\pgfqpoint{3.810510in}{0.637672in}}%
\pgfpathlineto{\pgfqpoint{3.813172in}{0.615440in}}%
\pgfpathlineto{\pgfqpoint{3.815983in}{0.605579in}}%
\pgfpathlineto{\pgfqpoint{3.818546in}{0.592469in}}%
\pgfpathlineto{\pgfqpoint{3.821315in}{0.602184in}}%
\pgfpathlineto{\pgfqpoint{3.823903in}{0.610514in}}%
\pgfpathlineto{\pgfqpoint{3.826679in}{0.593523in}}%
\pgfpathlineto{\pgfqpoint{3.829252in}{0.581365in}}%
\pgfpathlineto{\pgfqpoint{3.832053in}{0.574264in}}%
\pgfpathlineto{\pgfqpoint{3.834616in}{0.570065in}}%
\pgfpathlineto{\pgfqpoint{3.837286in}{0.564670in}}%
\pgfpathlineto{\pgfqpoint{3.839960in}{0.559618in}}%
\pgfpathlineto{\pgfqpoint{3.842641in}{0.558494in}}%
\pgfpathlineto{\pgfqpoint{3.845329in}{0.570178in}}%
\pgfpathlineto{\pgfqpoint{3.848005in}{0.558041in}}%
\pgfpathlineto{\pgfqpoint{3.850814in}{0.558536in}}%
\pgfpathlineto{\pgfqpoint{3.853358in}{0.554406in}}%
\pgfpathlineto{\pgfqpoint{3.856100in}{0.550310in}}%
\pgfpathlineto{\pgfqpoint{3.858720in}{0.547503in}}%
\pgfpathlineto{\pgfqpoint{3.861561in}{0.550605in}}%
\pgfpathlineto{\pgfqpoint{3.864073in}{0.546746in}}%
\pgfpathlineto{\pgfqpoint{3.866815in}{0.551215in}}%
\pgfpathlineto{\pgfqpoint{3.869435in}{0.546612in}}%
\pgfpathlineto{\pgfqpoint{3.872114in}{0.550532in}}%
\pgfpathlineto{\pgfqpoint{3.874790in}{0.548428in}}%
\pgfpathlineto{\pgfqpoint{3.877466in}{0.546500in}}%
\pgfpathlineto{\pgfqpoint{3.880237in}{0.545588in}}%
\pgfpathlineto{\pgfqpoint{3.882850in}{0.544032in}}%
\pgfpathlineto{\pgfqpoint{3.885621in}{0.541096in}}%
\pgfpathlineto{\pgfqpoint{3.888188in}{0.544443in}}%
\pgfpathlineto{\pgfqpoint{3.890926in}{0.539624in}}%
\pgfpathlineto{\pgfqpoint{3.893541in}{0.540241in}}%
\pgfpathlineto{\pgfqpoint{3.896345in}{0.538838in}}%
\pgfpathlineto{\pgfqpoint{3.898891in}{0.542025in}}%
\pgfpathlineto{\pgfqpoint{3.901573in}{0.546297in}}%
\pgfpathlineto{\pgfqpoint{3.904252in}{0.540214in}}%
\pgfpathlineto{\pgfqpoint{3.906918in}{0.543666in}}%
\pgfpathlineto{\pgfqpoint{3.909602in}{0.550558in}}%
\pgfpathlineto{\pgfqpoint{3.912296in}{0.549649in}}%
\pgfpathlineto{\pgfqpoint{3.915107in}{0.549377in}}%
\pgfpathlineto{\pgfqpoint{3.917646in}{0.547219in}}%
\pgfpathlineto{\pgfqpoint{3.920412in}{0.544602in}}%
\pgfpathlineto{\pgfqpoint{3.923005in}{0.544319in}}%
\pgfpathlineto{\pgfqpoint{3.925778in}{0.545744in}}%
\pgfpathlineto{\pgfqpoint{3.928347in}{0.537030in}}%
\pgfpathlineto{\pgfqpoint{3.931202in}{0.543836in}}%
\pgfpathlineto{\pgfqpoint{3.933714in}{0.537613in}}%
\pgfpathlineto{\pgfqpoint{3.936395in}{0.537217in}}%
\pgfpathlineto{\pgfqpoint{3.939075in}{0.538025in}}%
\pgfpathlineto{\pgfqpoint{3.941778in}{0.535821in}}%
\pgfpathlineto{\pgfqpoint{3.944431in}{0.532772in}}%
\pgfpathlineto{\pgfqpoint{3.947101in}{0.537401in}}%
\pgfpathlineto{\pgfqpoint{3.949894in}{0.541838in}}%
\pgfpathlineto{\pgfqpoint{3.952464in}{0.531403in}}%
\pgfpathlineto{\pgfqpoint{3.955211in}{0.538728in}}%
\pgfpathlineto{\pgfqpoint{3.957823in}{0.541980in}}%
\pgfpathlineto{\pgfqpoint{3.960635in}{0.542584in}}%
\pgfpathlineto{\pgfqpoint{3.963176in}{0.541340in}}%
\pgfpathlineto{\pgfqpoint{3.966013in}{0.540230in}}%
\pgfpathlineto{\pgfqpoint{3.968523in}{0.540033in}}%
\pgfpathlineto{\pgfqpoint{3.971250in}{0.537725in}}%
\pgfpathlineto{\pgfqpoint{3.973885in}{0.537432in}}%
\pgfpathlineto{\pgfqpoint{3.976563in}{0.529878in}}%
\pgfpathlineto{\pgfqpoint{3.979389in}{0.539006in}}%
\pgfpathlineto{\pgfqpoint{3.981929in}{0.536085in}}%
\pgfpathlineto{\pgfqpoint{3.984714in}{0.532270in}}%
\pgfpathlineto{\pgfqpoint{3.987270in}{0.540181in}}%
\pgfpathlineto{\pgfqpoint{3.990055in}{0.535267in}}%
\pgfpathlineto{\pgfqpoint{3.992642in}{0.543191in}}%
\pgfpathlineto{\pgfqpoint{3.995417in}{0.546088in}}%
\pgfpathlineto{\pgfqpoint{3.997990in}{0.543106in}}%
\pgfpathlineto{\pgfqpoint{4.000674in}{0.537129in}}%
\pgfpathlineto{\pgfqpoint{4.003348in}{0.541562in}}%
\pgfpathlineto{\pgfqpoint{4.006034in}{0.542987in}}%
\pgfpathlineto{\pgfqpoint{4.008699in}{0.542833in}}%
\pgfpathlineto{\pgfqpoint{4.011394in}{0.546819in}}%
\pgfpathlineto{\pgfqpoint{4.014186in}{0.543893in}}%
\pgfpathlineto{\pgfqpoint{4.016744in}{0.544082in}}%
\pgfpathlineto{\pgfqpoint{4.019518in}{0.544590in}}%
\pgfpathlineto{\pgfqpoint{4.022097in}{0.540215in}}%
\pgfpathlineto{\pgfqpoint{4.024868in}{0.540740in}}%
\pgfpathlineto{\pgfqpoint{4.027447in}{0.545086in}}%
\pgfpathlineto{\pgfqpoint{4.030229in}{0.542919in}}%
\pgfpathlineto{\pgfqpoint{4.032817in}{0.545467in}}%
\pgfpathlineto{\pgfqpoint{4.035492in}{0.545056in}}%
\pgfpathlineto{\pgfqpoint{4.038174in}{0.546395in}}%
\pgfpathlineto{\pgfqpoint{4.040852in}{0.543153in}}%
\pgfpathlineto{\pgfqpoint{4.043667in}{0.542975in}}%
\pgfpathlineto{\pgfqpoint{4.046210in}{0.549580in}}%
\pgfpathlineto{\pgfqpoint{4.049006in}{0.543862in}}%
\pgfpathlineto{\pgfqpoint{4.051557in}{0.543043in}}%
\pgfpathlineto{\pgfqpoint{4.054326in}{0.548228in}}%
\pgfpathlineto{\pgfqpoint{4.056911in}{0.549345in}}%
\pgfpathlineto{\pgfqpoint{4.059702in}{0.550482in}}%
\pgfpathlineto{\pgfqpoint{4.062266in}{0.544833in}}%
\pgfpathlineto{\pgfqpoint{4.064957in}{0.544933in}}%
\pgfpathlineto{\pgfqpoint{4.067636in}{0.549247in}}%
\pgfpathlineto{\pgfqpoint{4.070313in}{0.546435in}}%
\pgfpathlineto{\pgfqpoint{4.072985in}{0.541520in}}%
\pgfpathlineto{\pgfqpoint{4.075705in}{0.550135in}}%
\pgfpathlineto{\pgfqpoint{4.078471in}{0.544708in}}%
\pgfpathlineto{\pgfqpoint{4.081018in}{0.546761in}}%
\pgfpathlineto{\pgfqpoint{4.083870in}{0.546039in}}%
\pgfpathlineto{\pgfqpoint{4.086385in}{0.548421in}}%
\pgfpathlineto{\pgfqpoint{4.089159in}{0.551477in}}%
\pgfpathlineto{\pgfqpoint{4.091729in}{0.549507in}}%
\pgfpathlineto{\pgfqpoint{4.094527in}{0.541190in}}%
\pgfpathlineto{\pgfqpoint{4.097092in}{0.544972in}}%
\pgfpathlineto{\pgfqpoint{4.099777in}{0.549578in}}%
\pgfpathlineto{\pgfqpoint{4.102456in}{0.546314in}}%
\pgfpathlineto{\pgfqpoint{4.105185in}{0.548954in}}%
\pgfpathlineto{\pgfqpoint{4.107814in}{0.546924in}}%
\pgfpathlineto{\pgfqpoint{4.110488in}{0.542825in}}%
\pgfpathlineto{\pgfqpoint{4.113252in}{0.544450in}}%
\pgfpathlineto{\pgfqpoint{4.115844in}{0.535577in}}%
\pgfpathlineto{\pgfqpoint{4.118554in}{0.542679in}}%
\pgfpathlineto{\pgfqpoint{4.121205in}{0.541857in}}%
\pgfpathlineto{\pgfqpoint{4.124019in}{0.542605in}}%
\pgfpathlineto{\pgfqpoint{4.126553in}{0.543015in}}%
\pgfpathlineto{\pgfqpoint{4.129349in}{0.540927in}}%
\pgfpathlineto{\pgfqpoint{4.131920in}{0.544981in}}%
\pgfpathlineto{\pgfqpoint{4.134615in}{0.547678in}}%
\pgfpathlineto{\pgfqpoint{4.137272in}{0.543076in}}%
\pgfpathlineto{\pgfqpoint{4.139963in}{0.550405in}}%
\pgfpathlineto{\pgfqpoint{4.142713in}{0.549846in}}%
\pgfpathlineto{\pgfqpoint{4.145310in}{0.541643in}}%
\pgfpathlineto{\pgfqpoint{4.148082in}{0.548337in}}%
\pgfpathlineto{\pgfqpoint{4.150665in}{0.546321in}}%
\pgfpathlineto{\pgfqpoint{4.153423in}{0.548264in}}%
\pgfpathlineto{\pgfqpoint{4.156016in}{0.545386in}}%
\pgfpathlineto{\pgfqpoint{4.158806in}{0.543731in}}%
\pgfpathlineto{\pgfqpoint{4.161380in}{0.554031in}}%
\pgfpathlineto{\pgfqpoint{4.164059in}{0.546299in}}%
\pgfpathlineto{\pgfqpoint{4.166737in}{0.545454in}}%
\pgfpathlineto{\pgfqpoint{4.169415in}{0.549931in}}%
\pgfpathlineto{\pgfqpoint{4.172093in}{0.548788in}}%
\pgfpathlineto{\pgfqpoint{4.174770in}{0.547947in}}%
\pgfpathlineto{\pgfqpoint{4.177593in}{0.544541in}}%
\pgfpathlineto{\pgfqpoint{4.180129in}{0.549184in}}%
\pgfpathlineto{\pgfqpoint{4.182899in}{0.551254in}}%
\pgfpathlineto{\pgfqpoint{4.185481in}{0.546234in}}%
\pgfpathlineto{\pgfqpoint{4.188318in}{0.551188in}}%
\pgfpathlineto{\pgfqpoint{4.190842in}{0.551957in}}%
\pgfpathlineto{\pgfqpoint{4.193638in}{0.543228in}}%
\pgfpathlineto{\pgfqpoint{4.196186in}{0.545881in}}%
\pgfpathlineto{\pgfqpoint{4.198878in}{0.546325in}}%
\pgfpathlineto{\pgfqpoint{4.201542in}{0.547813in}}%
\pgfpathlineto{\pgfqpoint{4.204240in}{0.545802in}}%
\pgfpathlineto{\pgfqpoint{4.207076in}{0.541404in}}%
\pgfpathlineto{\pgfqpoint{4.209597in}{0.538069in}}%
\pgfpathlineto{\pgfqpoint{4.212383in}{0.538285in}}%
\pgfpathlineto{\pgfqpoint{4.214948in}{0.542485in}}%
\pgfpathlineto{\pgfqpoint{4.217694in}{0.541075in}}%
\pgfpathlineto{\pgfqpoint{4.220304in}{0.553678in}}%
\pgfpathlineto{\pgfqpoint{4.223082in}{0.563545in}}%
\pgfpathlineto{\pgfqpoint{4.225654in}{0.558630in}}%
\pgfpathlineto{\pgfqpoint{4.228331in}{0.571447in}}%
\pgfpathlineto{\pgfqpoint{4.231013in}{0.553700in}}%
\pgfpathlineto{\pgfqpoint{4.233691in}{0.543391in}}%
\pgfpathlineto{\pgfqpoint{4.236375in}{0.538014in}}%
\pgfpathlineto{\pgfqpoint{4.239084in}{0.541632in}}%
\pgfpathlineto{\pgfqpoint{4.241900in}{0.539554in}}%
\pgfpathlineto{\pgfqpoint{4.244394in}{0.542059in}}%
\pgfpathlineto{\pgfqpoint{4.247225in}{0.539565in}}%
\pgfpathlineto{\pgfqpoint{4.249767in}{0.547905in}}%
\pgfpathlineto{\pgfqpoint{4.252581in}{0.545843in}}%
\pgfpathlineto{\pgfqpoint{4.255120in}{0.544569in}}%
\pgfpathlineto{\pgfqpoint{4.257958in}{0.545859in}}%
\pgfpathlineto{\pgfqpoint{4.260477in}{0.545103in}}%
\pgfpathlineto{\pgfqpoint{4.263157in}{0.541766in}}%
\pgfpathlineto{\pgfqpoint{4.265824in}{0.544021in}}%
\pgfpathlineto{\pgfqpoint{4.268590in}{0.539260in}}%
\pgfpathlineto{\pgfqpoint{4.271187in}{0.543122in}}%
\pgfpathlineto{\pgfqpoint{4.273874in}{0.541397in}}%
\pgfpathlineto{\pgfqpoint{4.276635in}{0.540111in}}%
\pgfpathlineto{\pgfqpoint{4.279212in}{0.540499in}}%
\pgfpathlineto{\pgfqpoint{4.282000in}{0.539511in}}%
\pgfpathlineto{\pgfqpoint{4.284586in}{0.539402in}}%
\pgfpathlineto{\pgfqpoint{4.287399in}{0.539823in}}%
\pgfpathlineto{\pgfqpoint{4.289936in}{0.534430in}}%
\pgfpathlineto{\pgfqpoint{4.292786in}{0.536741in}}%
\pgfpathlineto{\pgfqpoint{4.295299in}{0.540185in}}%
\pgfpathlineto{\pgfqpoint{4.297977in}{0.542263in}}%
\pgfpathlineto{\pgfqpoint{4.300656in}{0.542436in}}%
\pgfpathlineto{\pgfqpoint{4.303357in}{0.543071in}}%
\pgfpathlineto{\pgfqpoint{4.306118in}{0.546189in}}%
\pgfpathlineto{\pgfqpoint{4.308691in}{0.545864in}}%
\pgfpathlineto{\pgfqpoint{4.311494in}{0.549397in}}%
\pgfpathlineto{\pgfqpoint{4.314032in}{0.544816in}}%
\pgfpathlineto{\pgfqpoint{4.316856in}{0.546561in}}%
\pgfpathlineto{\pgfqpoint{4.319405in}{0.546792in}}%
\pgfpathlineto{\pgfqpoint{4.322181in}{0.543174in}}%
\pgfpathlineto{\pgfqpoint{4.324760in}{0.549004in}}%
\pgfpathlineto{\pgfqpoint{4.327440in}{0.543959in}}%
\pgfpathlineto{\pgfqpoint{4.330118in}{0.541805in}}%
\pgfpathlineto{\pgfqpoint{4.332796in}{0.543481in}}%
\pgfpathlineto{\pgfqpoint{4.335463in}{0.542165in}}%
\pgfpathlineto{\pgfqpoint{4.338154in}{0.545152in}}%
\pgfpathlineto{\pgfqpoint{4.340976in}{0.544680in}}%
\pgfpathlineto{\pgfqpoint{4.343510in}{0.540964in}}%
\pgfpathlineto{\pgfqpoint{4.346263in}{0.544581in}}%
\pgfpathlineto{\pgfqpoint{4.348868in}{0.543779in}}%
\pgfpathlineto{\pgfqpoint{4.351645in}{0.542112in}}%
\pgfpathlineto{\pgfqpoint{4.354224in}{0.546195in}}%
\pgfpathlineto{\pgfqpoint{4.357014in}{0.547645in}}%
\pgfpathlineto{\pgfqpoint{4.359582in}{0.543643in}}%
\pgfpathlineto{\pgfqpoint{4.362270in}{0.542857in}}%
\pgfpathlineto{\pgfqpoint{4.364936in}{0.544600in}}%
\pgfpathlineto{\pgfqpoint{4.367646in}{0.549590in}}%
\pgfpathlineto{\pgfqpoint{4.370437in}{0.548689in}}%
\pgfpathlineto{\pgfqpoint{4.372976in}{0.540276in}}%
\pgfpathlineto{\pgfqpoint{4.375761in}{0.546548in}}%
\pgfpathlineto{\pgfqpoint{4.378329in}{0.546411in}}%
\pgfpathlineto{\pgfqpoint{4.381097in}{0.548504in}}%
\pgfpathlineto{\pgfqpoint{4.383674in}{0.545920in}}%
\pgfpathlineto{\pgfqpoint{4.386431in}{0.551312in}}%
\pgfpathlineto{\pgfqpoint{4.389044in}{0.544916in}}%
\pgfpathlineto{\pgfqpoint{4.391721in}{0.547329in}}%
\pgfpathlineto{\pgfqpoint{4.394400in}{0.537340in}}%
\pgfpathlineto{\pgfqpoint{4.397076in}{0.539181in}}%
\pgfpathlineto{\pgfqpoint{4.399745in}{0.546900in}}%
\pgfpathlineto{\pgfqpoint{4.402468in}{0.557608in}}%
\pgfpathlineto{\pgfqpoint{4.405234in}{0.556509in}}%
\pgfpathlineto{\pgfqpoint{4.407788in}{0.553471in}}%
\pgfpathlineto{\pgfqpoint{4.410587in}{0.552641in}}%
\pgfpathlineto{\pgfqpoint{4.413149in}{0.543618in}}%
\pgfpathlineto{\pgfqpoint{4.415932in}{0.544378in}}%
\pgfpathlineto{\pgfqpoint{4.418506in}{0.543966in}}%
\pgfpathlineto{\pgfqpoint{4.421292in}{0.547703in}}%
\pgfpathlineto{\pgfqpoint{4.423863in}{0.553842in}}%
\pgfpathlineto{\pgfqpoint{4.426534in}{0.547083in}}%
\pgfpathlineto{\pgfqpoint{4.429220in}{0.547608in}}%
\pgfpathlineto{\pgfqpoint{4.431901in}{0.546584in}}%
\pgfpathlineto{\pgfqpoint{4.434569in}{0.544667in}}%
\pgfpathlineto{\pgfqpoint{4.437253in}{0.543668in}}%
\pgfpathlineto{\pgfqpoint{4.440041in}{0.536586in}}%
\pgfpathlineto{\pgfqpoint{4.442611in}{0.533669in}}%
\pgfpathlineto{\pgfqpoint{4.445423in}{0.536321in}}%
\pgfpathlineto{\pgfqpoint{4.447965in}{0.535694in}}%
\pgfpathlineto{\pgfqpoint{4.450767in}{0.540062in}}%
\pgfpathlineto{\pgfqpoint{4.453312in}{0.539330in}}%
\pgfpathlineto{\pgfqpoint{4.456138in}{0.546598in}}%
\pgfpathlineto{\pgfqpoint{4.458681in}{0.557839in}}%
\pgfpathlineto{\pgfqpoint{4.461367in}{0.554690in}}%
\pgfpathlineto{\pgfqpoint{4.464029in}{0.561221in}}%
\pgfpathlineto{\pgfqpoint{4.466717in}{0.547946in}}%
\pgfpathlineto{\pgfqpoint{4.469492in}{0.547984in}}%
\pgfpathlineto{\pgfqpoint{4.472059in}{0.543679in}}%
\pgfpathlineto{\pgfqpoint{4.474861in}{0.542248in}}%
\pgfpathlineto{\pgfqpoint{4.477430in}{0.537527in}}%
\pgfpathlineto{\pgfqpoint{4.480201in}{0.539980in}}%
\pgfpathlineto{\pgfqpoint{4.482778in}{0.556585in}}%
\pgfpathlineto{\pgfqpoint{4.485581in}{0.544809in}}%
\pgfpathlineto{\pgfqpoint{4.488130in}{0.543092in}}%
\pgfpathlineto{\pgfqpoint{4.490822in}{0.544217in}}%
\pgfpathlineto{\pgfqpoint{4.493492in}{0.546795in}}%
\pgfpathlineto{\pgfqpoint{4.496167in}{0.545800in}}%
\pgfpathlineto{\pgfqpoint{4.498850in}{0.550032in}}%
\pgfpathlineto{\pgfqpoint{4.501529in}{0.545916in}}%
\pgfpathlineto{\pgfqpoint{4.504305in}{0.549003in}}%
\pgfpathlineto{\pgfqpoint{4.506893in}{0.559446in}}%
\pgfpathlineto{\pgfqpoint{4.509643in}{0.564031in}}%
\pgfpathlineto{\pgfqpoint{4.512246in}{0.575551in}}%
\pgfpathlineto{\pgfqpoint{4.515080in}{0.562074in}}%
\pgfpathlineto{\pgfqpoint{4.517598in}{0.554267in}}%
\pgfpathlineto{\pgfqpoint{4.520345in}{0.548991in}}%
\pgfpathlineto{\pgfqpoint{4.522962in}{0.549786in}}%
\pgfpathlineto{\pgfqpoint{4.525640in}{0.556955in}}%
\pgfpathlineto{\pgfqpoint{4.528307in}{0.553470in}}%
\pgfpathlineto{\pgfqpoint{4.530990in}{0.551854in}}%
\pgfpathlineto{\pgfqpoint{4.533764in}{0.551204in}}%
\pgfpathlineto{\pgfqpoint{4.536400in}{0.556936in}}%
\pgfpathlineto{\pgfqpoint{4.539144in}{0.552414in}}%
\pgfpathlineto{\pgfqpoint{4.541711in}{0.544869in}}%
\pgfpathlineto{\pgfqpoint{4.544464in}{0.546965in}}%
\pgfpathlineto{\pgfqpoint{4.547064in}{0.550953in}}%
\pgfpathlineto{\pgfqpoint{4.549822in}{0.555559in}}%
\pgfpathlineto{\pgfqpoint{4.552425in}{0.550571in}}%
\pgfpathlineto{\pgfqpoint{4.555106in}{0.546385in}}%
\pgfpathlineto{\pgfqpoint{4.557777in}{0.544947in}}%
\pgfpathlineto{\pgfqpoint{4.560448in}{0.545210in}}%
\pgfpathlineto{\pgfqpoint{4.563125in}{0.548705in}}%
\pgfpathlineto{\pgfqpoint{4.565820in}{0.548738in}}%
\pgfpathlineto{\pgfqpoint{4.568612in}{0.546471in}}%
\pgfpathlineto{\pgfqpoint{4.571171in}{0.535205in}}%
\pgfpathlineto{\pgfqpoint{4.573947in}{0.538379in}}%
\pgfpathlineto{\pgfqpoint{4.576531in}{0.537257in}}%
\pgfpathlineto{\pgfqpoint{4.579305in}{0.543428in}}%
\pgfpathlineto{\pgfqpoint{4.581888in}{0.546173in}}%
\pgfpathlineto{\pgfqpoint{4.584672in}{0.552740in}}%
\pgfpathlineto{\pgfqpoint{4.587244in}{0.546259in}}%
\pgfpathlineto{\pgfqpoint{4.589920in}{0.537392in}}%
\pgfpathlineto{\pgfqpoint{4.592589in}{0.538261in}}%
\pgfpathlineto{\pgfqpoint{4.595281in}{0.546112in}}%
\pgfpathlineto{\pgfqpoint{4.597951in}{0.557234in}}%
\pgfpathlineto{\pgfqpoint{4.600633in}{0.564630in}}%
\pgfpathlineto{\pgfqpoint{4.603430in}{0.551280in}}%
\pgfpathlineto{\pgfqpoint{4.605990in}{0.547540in}}%
\pgfpathlineto{\pgfqpoint{4.608808in}{0.549998in}}%
\pgfpathlineto{\pgfqpoint{4.611350in}{0.547831in}}%
\pgfpathlineto{\pgfqpoint{4.614134in}{0.543711in}}%
\pgfpathlineto{\pgfqpoint{4.616702in}{0.542151in}}%
\pgfpathlineto{\pgfqpoint{4.619529in}{0.543919in}}%
\pgfpathlineto{\pgfqpoint{4.622056in}{0.543305in}}%
\pgfpathlineto{\pgfqpoint{4.624741in}{0.547897in}}%
\pgfpathlineto{\pgfqpoint{4.627411in}{0.551375in}}%
\pgfpathlineto{\pgfqpoint{4.630096in}{0.549104in}}%
\pgfpathlineto{\pgfqpoint{4.632902in}{0.549359in}}%
\pgfpathlineto{\pgfqpoint{4.635445in}{0.545504in}}%
\pgfpathlineto{\pgfqpoint{4.638204in}{0.543062in}}%
\pgfpathlineto{\pgfqpoint{4.640809in}{0.547072in}}%
\pgfpathlineto{\pgfqpoint{4.643628in}{0.552763in}}%
\pgfpathlineto{\pgfqpoint{4.646169in}{0.553009in}}%
\pgfpathlineto{\pgfqpoint{4.648922in}{0.549040in}}%
\pgfpathlineto{\pgfqpoint{4.651524in}{0.547297in}}%
\pgfpathlineto{\pgfqpoint{4.654203in}{0.551584in}}%
\pgfpathlineto{\pgfqpoint{4.656873in}{0.551204in}}%
\pgfpathlineto{\pgfqpoint{4.659590in}{0.545752in}}%
\pgfpathlineto{\pgfqpoint{4.662237in}{0.547244in}}%
\pgfpathlineto{\pgfqpoint{4.664923in}{0.541472in}}%
\pgfpathlineto{\pgfqpoint{4.667764in}{0.548227in}}%
\pgfpathlineto{\pgfqpoint{4.670261in}{0.542275in}}%
\pgfpathlineto{\pgfqpoint{4.673068in}{0.544688in}}%
\pgfpathlineto{\pgfqpoint{4.675619in}{0.550449in}}%
\pgfpathlineto{\pgfqpoint{4.678448in}{0.549115in}}%
\pgfpathlineto{\pgfqpoint{4.680988in}{0.549472in}}%
\pgfpathlineto{\pgfqpoint{4.683799in}{0.548593in}}%
\pgfpathlineto{\pgfqpoint{4.686337in}{0.543151in}}%
\pgfpathlineto{\pgfqpoint{4.689051in}{0.554817in}}%
\pgfpathlineto{\pgfqpoint{4.691694in}{0.551301in}}%
\pgfpathlineto{\pgfqpoint{4.694381in}{0.545527in}}%
\pgfpathlineto{\pgfqpoint{4.697170in}{0.549267in}}%
\pgfpathlineto{\pgfqpoint{4.699734in}{0.542635in}}%
\pgfpathlineto{\pgfqpoint{4.702517in}{0.544546in}}%
\pgfpathlineto{\pgfqpoint{4.705094in}{0.544479in}}%
\pgfpathlineto{\pgfqpoint{4.707824in}{0.545077in}}%
\pgfpathlineto{\pgfqpoint{4.710437in}{0.544917in}}%
\pgfpathlineto{\pgfqpoint{4.713275in}{0.537788in}}%
\pgfpathlineto{\pgfqpoint{4.715806in}{0.541270in}}%
\pgfpathlineto{\pgfqpoint{4.718486in}{0.541971in}}%
\pgfpathlineto{\pgfqpoint{4.721160in}{0.543621in}}%
\pgfpathlineto{\pgfqpoint{4.723873in}{0.540241in}}%
\pgfpathlineto{\pgfqpoint{4.726508in}{0.543146in}}%
\pgfpathlineto{\pgfqpoint{4.729233in}{0.543120in}}%
\pgfpathlineto{\pgfqpoint{4.731901in}{0.542010in}}%
\pgfpathlineto{\pgfqpoint{4.734552in}{0.544982in}}%
\pgfpathlineto{\pgfqpoint{4.737348in}{0.545828in}}%
\pgfpathlineto{\pgfqpoint{4.739912in}{0.547144in}}%
\pgfpathlineto{\pgfqpoint{4.742696in}{0.541755in}}%
\pgfpathlineto{\pgfqpoint{4.745256in}{0.537954in}}%
\pgfpathlineto{\pgfqpoint{4.748081in}{0.539573in}}%
\pgfpathlineto{\pgfqpoint{4.750627in}{0.539207in}}%
\pgfpathlineto{\pgfqpoint{4.753298in}{0.543813in}}%
\pgfpathlineto{\pgfqpoint{4.755983in}{0.541797in}}%
\pgfpathlineto{\pgfqpoint{4.758653in}{0.537912in}}%
\pgfpathlineto{\pgfqpoint{4.761337in}{0.539503in}}%
\pgfpathlineto{\pgfqpoint{4.764018in}{0.538565in}}%
\pgfpathlineto{\pgfqpoint{4.766783in}{0.535551in}}%
\pgfpathlineto{\pgfqpoint{4.769367in}{0.542327in}}%
\pgfpathlineto{\pgfqpoint{4.772198in}{0.540566in}}%
\pgfpathlineto{\pgfqpoint{4.774732in}{0.539145in}}%
\pgfpathlineto{\pgfqpoint{4.777535in}{0.540640in}}%
\pgfpathlineto{\pgfqpoint{4.780083in}{0.541713in}}%
\pgfpathlineto{\pgfqpoint{4.782872in}{0.542175in}}%
\pgfpathlineto{\pgfqpoint{4.785445in}{0.544970in}}%
\pgfpathlineto{\pgfqpoint{4.788116in}{0.542371in}}%
\pgfpathlineto{\pgfqpoint{4.790798in}{0.541273in}}%
\pgfpathlineto{\pgfqpoint{4.793512in}{0.537963in}}%
\pgfpathlineto{\pgfqpoint{4.796274in}{0.539408in}}%
\pgfpathlineto{\pgfqpoint{4.798830in}{0.539783in}}%
\pgfpathlineto{\pgfqpoint{4.801586in}{0.537369in}}%
\pgfpathlineto{\pgfqpoint{4.804193in}{0.535742in}}%
\pgfpathlineto{\pgfqpoint{4.807017in}{0.537025in}}%
\pgfpathlineto{\pgfqpoint{4.809538in}{0.540045in}}%
\pgfpathlineto{\pgfqpoint{4.812377in}{0.552569in}}%
\pgfpathlineto{\pgfqpoint{4.814907in}{0.539101in}}%
\pgfpathlineto{\pgfqpoint{4.817587in}{0.529878in}}%
\pgfpathlineto{\pgfqpoint{4.820265in}{0.538721in}}%
\pgfpathlineto{\pgfqpoint{4.822945in}{0.529878in}}%
\pgfpathlineto{\pgfqpoint{4.825619in}{0.529878in}}%
\pgfpathlineto{\pgfqpoint{4.828291in}{0.529878in}}%
\pgfpathlineto{\pgfqpoint{4.831045in}{0.529878in}}%
\pgfpathlineto{\pgfqpoint{4.833657in}{0.534015in}}%
\pgfpathlineto{\pgfqpoint{4.837992in}{0.542712in}}%
\pgfpathlineto{\pgfqpoint{4.839922in}{0.543376in}}%
\pgfpathlineto{\pgfqpoint{4.842380in}{0.537629in}}%
\pgfpathlineto{\pgfqpoint{4.844361in}{0.540370in}}%
\pgfpathlineto{\pgfqpoint{4.847127in}{0.541271in}}%
\pgfpathlineto{\pgfqpoint{4.849715in}{0.537295in}}%
\pgfpathlineto{\pgfqpoint{4.852404in}{0.536903in}}%
\pgfpathlineto{\pgfqpoint{4.855070in}{0.540789in}}%
\pgfpathlineto{\pgfqpoint{4.857807in}{0.540775in}}%
\pgfpathlineto{\pgfqpoint{4.860544in}{0.544240in}}%
\pgfpathlineto{\pgfqpoint{4.863116in}{0.539713in}}%
\pgfpathlineto{\pgfqpoint{4.865910in}{0.550085in}}%
\pgfpathlineto{\pgfqpoint{4.868474in}{0.550245in}}%
\pgfpathlineto{\pgfqpoint{4.871209in}{0.541106in}}%
\pgfpathlineto{\pgfqpoint{4.873832in}{0.551101in}}%
\pgfpathlineto{\pgfqpoint{4.876636in}{0.546293in}}%
\pgfpathlineto{\pgfqpoint{4.879180in}{0.547960in}}%
\pgfpathlineto{\pgfqpoint{4.881864in}{0.546236in}}%
\pgfpathlineto{\pgfqpoint{4.884540in}{0.540221in}}%
\pgfpathlineto{\pgfqpoint{4.887211in}{0.538941in}}%
\pgfpathlineto{\pgfqpoint{4.889902in}{0.541792in}}%
\pgfpathlineto{\pgfqpoint{4.892611in}{0.543961in}}%
\pgfpathlineto{\pgfqpoint{4.895399in}{0.541456in}}%
\pgfpathlineto{\pgfqpoint{4.897938in}{0.545588in}}%
\pgfpathlineto{\pgfqpoint{4.900712in}{0.541308in}}%
\pgfpathlineto{\pgfqpoint{4.903295in}{0.546198in}}%
\pgfpathlineto{\pgfqpoint{4.906096in}{0.548268in}}%
\pgfpathlineto{\pgfqpoint{4.908648in}{0.536537in}}%
\pgfpathlineto{\pgfqpoint{4.911435in}{0.532319in}}%
\pgfpathlineto{\pgfqpoint{4.914009in}{0.532427in}}%
\pgfpathlineto{\pgfqpoint{4.916681in}{0.529878in}}%
\pgfpathlineto{\pgfqpoint{4.919352in}{0.534101in}}%
\pgfpathlineto{\pgfqpoint{4.922041in}{0.529966in}}%
\pgfpathlineto{\pgfqpoint{4.924708in}{0.529878in}}%
\pgfpathlineto{\pgfqpoint{4.927400in}{0.535857in}}%
\pgfpathlineto{\pgfqpoint{4.930170in}{0.533534in}}%
\pgfpathlineto{\pgfqpoint{4.932742in}{0.541782in}}%
\pgfpathlineto{\pgfqpoint{4.935515in}{0.542041in}}%
\pgfpathlineto{\pgfqpoint{4.938112in}{0.540021in}}%
\pgfpathlineto{\pgfqpoint{4.940881in}{0.543479in}}%
\pgfpathlineto{\pgfqpoint{4.943466in}{0.545357in}}%
\pgfpathlineto{\pgfqpoint{4.946151in}{0.544631in}}%
\pgfpathlineto{\pgfqpoint{4.948827in}{0.545327in}}%
\pgfpathlineto{\pgfqpoint{4.951504in}{0.542652in}}%
\pgfpathlineto{\pgfqpoint{4.954182in}{0.558643in}}%
\pgfpathlineto{\pgfqpoint{4.956862in}{0.579351in}}%
\pgfpathlineto{\pgfqpoint{4.959689in}{0.575308in}}%
\pgfpathlineto{\pgfqpoint{4.962219in}{0.571127in}}%
\pgfpathlineto{\pgfqpoint{4.965002in}{0.553748in}}%
\pgfpathlineto{\pgfqpoint{4.967575in}{0.554248in}}%
\pgfpathlineto{\pgfqpoint{4.970314in}{0.549258in}}%
\pgfpathlineto{\pgfqpoint{4.972933in}{0.545721in}}%
\pgfpathlineto{\pgfqpoint{4.975703in}{0.554551in}}%
\pgfpathlineto{\pgfqpoint{4.978287in}{0.553978in}}%
\pgfpathlineto{\pgfqpoint{4.980967in}{0.545901in}}%
\pgfpathlineto{\pgfqpoint{4.983637in}{0.548303in}}%
\pgfpathlineto{\pgfqpoint{4.986325in}{0.548952in}}%
\pgfpathlineto{\pgfqpoint{4.989001in}{0.545872in}}%
\pgfpathlineto{\pgfqpoint{4.991683in}{0.546355in}}%
\pgfpathlineto{\pgfqpoint{4.994390in}{0.546968in}}%
\pgfpathlineto{\pgfqpoint{4.997028in}{0.552359in}}%
\pgfpathlineto{\pgfqpoint{4.999780in}{0.552838in}}%
\pgfpathlineto{\pgfqpoint{5.002384in}{0.540343in}}%
\pgfpathlineto{\pgfqpoint{5.005178in}{0.547178in}}%
\pgfpathlineto{\pgfqpoint{5.007751in}{0.545108in}}%
\pgfpathlineto{\pgfqpoint{5.010562in}{0.544636in}}%
\pgfpathlineto{\pgfqpoint{5.013104in}{0.544898in}}%
\pgfpathlineto{\pgfqpoint{5.015820in}{0.542053in}}%
\pgfpathlineto{\pgfqpoint{5.018466in}{0.534249in}}%
\pgfpathlineto{\pgfqpoint{5.021147in}{0.542945in}}%
\pgfpathlineto{\pgfqpoint{5.023927in}{0.547571in}}%
\pgfpathlineto{\pgfqpoint{5.026501in}{0.545637in}}%
\pgfpathlineto{\pgfqpoint{5.029275in}{0.547271in}}%
\pgfpathlineto{\pgfqpoint{5.031849in}{0.546929in}}%
\pgfpathlineto{\pgfqpoint{5.034649in}{0.557684in}}%
\pgfpathlineto{\pgfqpoint{5.037214in}{0.553491in}}%
\pgfpathlineto{\pgfqpoint{5.039962in}{0.544423in}}%
\pgfpathlineto{\pgfqpoint{5.042572in}{0.541918in}}%
\pgfpathlineto{\pgfqpoint{5.045249in}{0.547815in}}%
\pgfpathlineto{\pgfqpoint{5.047924in}{0.551430in}}%
\pgfpathlineto{\pgfqpoint{5.050606in}{0.549974in}}%
\pgfpathlineto{\pgfqpoint{5.053284in}{0.548356in}}%
\pgfpathlineto{\pgfqpoint{5.055952in}{0.536075in}}%
\pgfpathlineto{\pgfqpoint{5.058711in}{0.543034in}}%
\pgfpathlineto{\pgfqpoint{5.061315in}{0.551045in}}%
\pgfpathlineto{\pgfqpoint{5.064144in}{0.544064in}}%
\pgfpathlineto{\pgfqpoint{5.066677in}{0.543835in}}%
\pgfpathlineto{\pgfqpoint{5.069463in}{0.545461in}}%
\pgfpathlineto{\pgfqpoint{5.072030in}{0.542694in}}%
\pgfpathlineto{\pgfqpoint{5.074851in}{0.549379in}}%
\pgfpathlineto{\pgfqpoint{5.077390in}{0.550812in}}%
\pgfpathlineto{\pgfqpoint{5.080067in}{0.544276in}}%
\pgfpathlineto{\pgfqpoint{5.082746in}{0.544482in}}%
\pgfpathlineto{\pgfqpoint{5.085426in}{0.546584in}}%
\pgfpathlineto{\pgfqpoint{5.088103in}{0.545955in}}%
\pgfpathlineto{\pgfqpoint{5.090788in}{0.546814in}}%
\pgfpathlineto{\pgfqpoint{5.093579in}{0.548874in}}%
\pgfpathlineto{\pgfqpoint{5.096142in}{0.549130in}}%
\pgfpathlineto{\pgfqpoint{5.098948in}{0.547812in}}%
\pgfpathlineto{\pgfqpoint{5.101496in}{0.547106in}}%
\pgfpathlineto{\pgfqpoint{5.104312in}{0.541981in}}%
\pgfpathlineto{\pgfqpoint{5.106842in}{0.542338in}}%
\pgfpathlineto{\pgfqpoint{5.109530in}{0.549806in}}%
\pgfpathlineto{\pgfqpoint{5.112209in}{0.545833in}}%
\pgfpathlineto{\pgfqpoint{5.114887in}{0.545937in}}%
\pgfpathlineto{\pgfqpoint{5.117550in}{0.553086in}}%
\pgfpathlineto{\pgfqpoint{5.120243in}{0.548377in}}%
\pgfpathlineto{\pgfqpoint{5.123042in}{0.543633in}}%
\pgfpathlineto{\pgfqpoint{5.125599in}{0.547639in}}%
\pgfpathlineto{\pgfqpoint{5.128421in}{0.549887in}}%
\pgfpathlineto{\pgfqpoint{5.130953in}{0.541418in}}%
\pgfpathlineto{\pgfqpoint{5.133716in}{0.541740in}}%
\pgfpathlineto{\pgfqpoint{5.136311in}{0.545877in}}%
\pgfpathlineto{\pgfqpoint{5.139072in}{0.545920in}}%
\pgfpathlineto{\pgfqpoint{5.141660in}{0.545033in}}%
\pgfpathlineto{\pgfqpoint{5.144349in}{0.548662in}}%
\pgfpathlineto{\pgfqpoint{5.147029in}{0.542028in}}%
\pgfpathlineto{\pgfqpoint{5.149734in}{0.543757in}}%
\pgfpathlineto{\pgfqpoint{5.152382in}{0.530558in}}%
\pgfpathlineto{\pgfqpoint{5.155059in}{0.529878in}}%
\pgfpathlineto{\pgfqpoint{5.157815in}{0.539725in}}%
\pgfpathlineto{\pgfqpoint{5.160420in}{0.540848in}}%
\pgfpathlineto{\pgfqpoint{5.163243in}{0.539999in}}%
\pgfpathlineto{\pgfqpoint{5.165775in}{0.542249in}}%
\pgfpathlineto{\pgfqpoint{5.168591in}{0.541221in}}%
\pgfpathlineto{\pgfqpoint{5.171133in}{0.542292in}}%
\pgfpathlineto{\pgfqpoint{5.173925in}{0.544561in}}%
\pgfpathlineto{\pgfqpoint{5.176477in}{0.543285in}}%
\pgfpathlineto{\pgfqpoint{5.179188in}{0.546946in}}%
\pgfpathlineto{\pgfqpoint{5.181848in}{0.594149in}}%
\pgfpathlineto{\pgfqpoint{5.184522in}{0.601735in}}%
\pgfpathlineto{\pgfqpoint{5.187294in}{0.576801in}}%
\pgfpathlineto{\pgfqpoint{5.189880in}{0.565535in}}%
\pgfpathlineto{\pgfqpoint{5.192680in}{0.551545in}}%
\pgfpathlineto{\pgfqpoint{5.195239in}{0.547872in}}%
\pgfpathlineto{\pgfqpoint{5.198008in}{0.544103in}}%
\pgfpathlineto{\pgfqpoint{5.200594in}{0.546506in}}%
\pgfpathlineto{\pgfqpoint{5.203388in}{0.543150in}}%
\pgfpathlineto{\pgfqpoint{5.205952in}{0.547096in}}%
\pgfpathlineto{\pgfqpoint{5.208630in}{0.547856in}}%
\pgfpathlineto{\pgfqpoint{5.211299in}{0.541063in}}%
\pgfpathlineto{\pgfqpoint{5.214027in}{0.544594in}}%
\pgfpathlineto{\pgfqpoint{5.216667in}{0.545764in}}%
\pgfpathlineto{\pgfqpoint{5.219345in}{0.542339in}}%
\pgfpathlineto{\pgfqpoint{5.222151in}{0.545703in}}%
\pgfpathlineto{\pgfqpoint{5.224695in}{0.540368in}}%
\pgfpathlineto{\pgfqpoint{5.227470in}{0.545798in}}%
\pgfpathlineto{\pgfqpoint{5.230059in}{0.540681in}}%
\pgfpathlineto{\pgfqpoint{5.232855in}{0.539957in}}%
\pgfpathlineto{\pgfqpoint{5.235409in}{0.546524in}}%
\pgfpathlineto{\pgfqpoint{5.238173in}{0.539792in}}%
\pgfpathlineto{\pgfqpoint{5.240777in}{0.546143in}}%
\pgfpathlineto{\pgfqpoint{5.243445in}{0.541274in}}%
\pgfpathlineto{\pgfqpoint{5.246130in}{0.539999in}}%
\pgfpathlineto{\pgfqpoint{5.248816in}{0.547121in}}%
\pgfpathlineto{\pgfqpoint{5.251590in}{0.543417in}}%
\pgfpathlineto{\pgfqpoint{5.254236in}{0.547025in}}%
\pgfpathlineto{\pgfqpoint{5.256973in}{0.543530in}}%
\pgfpathlineto{\pgfqpoint{5.259511in}{0.544749in}}%
\pgfpathlineto{\pgfqpoint{5.262264in}{0.542846in}}%
\pgfpathlineto{\pgfqpoint{5.264876in}{0.535499in}}%
\pgfpathlineto{\pgfqpoint{5.267691in}{0.535968in}}%
\pgfpathlineto{\pgfqpoint{5.270238in}{0.545337in}}%
\pgfpathlineto{\pgfqpoint{5.272913in}{0.564122in}}%
\pgfpathlineto{\pgfqpoint{5.275589in}{0.558322in}}%
\pgfpathlineto{\pgfqpoint{5.278322in}{0.545555in}}%
\pgfpathlineto{\pgfqpoint{5.280947in}{0.544571in}}%
\pgfpathlineto{\pgfqpoint{5.283631in}{0.539990in}}%
\pgfpathlineto{\pgfqpoint{5.286436in}{0.542883in}}%
\pgfpathlineto{\pgfqpoint{5.288984in}{0.545778in}}%
\pgfpathlineto{\pgfqpoint{5.291794in}{0.550663in}}%
\pgfpathlineto{\pgfqpoint{5.294339in}{0.544284in}}%
\pgfpathlineto{\pgfqpoint{5.297140in}{0.547188in}}%
\pgfpathlineto{\pgfqpoint{5.299696in}{0.541533in}}%
\pgfpathlineto{\pgfqpoint{5.302443in}{0.540683in}}%
\pgfpathlineto{\pgfqpoint{5.305054in}{0.543336in}}%
\pgfpathlineto{\pgfqpoint{5.307731in}{0.539563in}}%
\pgfpathlineto{\pgfqpoint{5.310411in}{0.541767in}}%
\pgfpathlineto{\pgfqpoint{5.313089in}{0.540929in}}%
\pgfpathlineto{\pgfqpoint{5.315754in}{0.542354in}}%
\pgfpathlineto{\pgfqpoint{5.318430in}{0.543300in}}%
\pgfpathlineto{\pgfqpoint{5.321256in}{0.540097in}}%
\pgfpathlineto{\pgfqpoint{5.323802in}{0.542524in}}%
\pgfpathlineto{\pgfqpoint{5.326564in}{0.545121in}}%
\pgfpathlineto{\pgfqpoint{5.329159in}{0.545631in}}%
\pgfpathlineto{\pgfqpoint{5.331973in}{0.542520in}}%
\pgfpathlineto{\pgfqpoint{5.334510in}{0.537723in}}%
\pgfpathlineto{\pgfqpoint{5.337353in}{0.546343in}}%
\pgfpathlineto{\pgfqpoint{5.339872in}{0.538253in}}%
\pgfpathlineto{\pgfqpoint{5.342549in}{0.537454in}}%
\pgfpathlineto{\pgfqpoint{5.345224in}{0.532247in}}%
\pgfpathlineto{\pgfqpoint{5.347905in}{0.529878in}}%
\pgfpathlineto{\pgfqpoint{5.350723in}{0.529878in}}%
\pgfpathlineto{\pgfqpoint{5.353262in}{0.534616in}}%
\pgfpathlineto{\pgfqpoint{5.356056in}{0.539058in}}%
\pgfpathlineto{\pgfqpoint{5.358612in}{0.543493in}}%
\pgfpathlineto{\pgfqpoint{5.361370in}{0.541002in}}%
\pgfpathlineto{\pgfqpoint{5.363966in}{0.545058in}}%
\pgfpathlineto{\pgfqpoint{5.366727in}{0.539996in}}%
\pgfpathlineto{\pgfqpoint{5.369335in}{0.546980in}}%
\pgfpathlineto{\pgfqpoint{5.372013in}{0.552270in}}%
\pgfpathlineto{\pgfqpoint{5.374692in}{0.547479in}}%
\pgfpathlineto{\pgfqpoint{5.377370in}{0.547833in}}%
\pgfpathlineto{\pgfqpoint{5.380048in}{0.549336in}}%
\pgfpathlineto{\pgfqpoint{5.382725in}{0.549515in}}%
\pgfpathlineto{\pgfqpoint{5.385550in}{0.551396in}}%
\pgfpathlineto{\pgfqpoint{5.388083in}{0.548478in}}%
\pgfpathlineto{\pgfqpoint{5.390900in}{0.547851in}}%
\pgfpathlineto{\pgfqpoint{5.393441in}{0.550365in}}%
\pgfpathlineto{\pgfqpoint{5.396219in}{0.548067in}}%
\pgfpathlineto{\pgfqpoint{5.398784in}{0.543697in}}%
\pgfpathlineto{\pgfqpoint{5.401576in}{0.546674in}}%
\pgfpathlineto{\pgfqpoint{5.404154in}{0.548293in}}%
\pgfpathlineto{\pgfqpoint{5.406832in}{0.545194in}}%
\pgfpathlineto{\pgfqpoint{5.409507in}{0.544125in}}%
\pgfpathlineto{\pgfqpoint{5.412190in}{0.550827in}}%
\pgfpathlineto{\pgfqpoint{5.414954in}{0.548274in}}%
\pgfpathlineto{\pgfqpoint{5.417547in}{0.547714in}}%
\pgfpathlineto{\pgfqpoint{5.420304in}{0.545786in}}%
\pgfpathlineto{\pgfqpoint{5.422897in}{0.550564in}}%
\pgfpathlineto{\pgfqpoint{5.425661in}{0.551462in}}%
\pgfpathlineto{\pgfqpoint{5.428259in}{0.550545in}}%
\pgfpathlineto{\pgfqpoint{5.431015in}{0.553580in}}%
\pgfpathlineto{\pgfqpoint{5.433616in}{0.547533in}}%
\pgfpathlineto{\pgfqpoint{5.436295in}{0.547523in}}%
\pgfpathlineto{\pgfqpoint{5.438974in}{0.549916in}}%
\pgfpathlineto{\pgfqpoint{5.441698in}{0.550522in}}%
\pgfpathlineto{\pgfqpoint{5.444328in}{0.543104in}}%
\pgfpathlineto{\pgfqpoint{5.447021in}{0.543481in}}%
\pgfpathlineto{\pgfqpoint{5.449769in}{0.540661in}}%
\pgfpathlineto{\pgfqpoint{5.452365in}{0.538737in}}%
\pgfpathlineto{\pgfqpoint{5.455168in}{0.545019in}}%
\pgfpathlineto{\pgfqpoint{5.457721in}{0.544011in}}%
\pgfpathlineto{\pgfqpoint{5.460489in}{0.545478in}}%
\pgfpathlineto{\pgfqpoint{5.463079in}{0.542858in}}%
\pgfpathlineto{\pgfqpoint{5.465888in}{0.543801in}}%
\pgfpathlineto{\pgfqpoint{5.468425in}{0.547243in}}%
\pgfpathlineto{\pgfqpoint{5.471113in}{0.550425in}}%
\pgfpathlineto{\pgfqpoint{5.473792in}{0.547712in}}%
\pgfpathlineto{\pgfqpoint{5.476458in}{0.545575in}}%
\pgfpathlineto{\pgfqpoint{5.479152in}{0.551757in}}%
\pgfpathlineto{\pgfqpoint{5.481825in}{0.551008in}}%
\pgfpathlineto{\pgfqpoint{5.484641in}{0.544736in}}%
\pgfpathlineto{\pgfqpoint{5.487176in}{0.547180in}}%
\pgfpathlineto{\pgfqpoint{5.490000in}{0.546596in}}%
\pgfpathlineto{\pgfqpoint{5.492541in}{0.544254in}}%
\pgfpathlineto{\pgfqpoint{5.495346in}{0.546826in}}%
\pgfpathlineto{\pgfqpoint{5.497898in}{0.543395in}}%
\pgfpathlineto{\pgfqpoint{5.500687in}{0.550289in}}%
\pgfpathlineto{\pgfqpoint{5.503255in}{0.547767in}}%
\pgfpathlineto{\pgfqpoint{5.505933in}{0.548203in}}%
\pgfpathlineto{\pgfqpoint{5.508612in}{0.551810in}}%
\pgfpathlineto{\pgfqpoint{5.511290in}{0.544823in}}%
\pgfpathlineto{\pgfqpoint{5.514080in}{0.541084in}}%
\pgfpathlineto{\pgfqpoint{5.516646in}{0.549841in}}%
\pgfpathlineto{\pgfqpoint{5.519433in}{0.554798in}}%
\pgfpathlineto{\pgfqpoint{5.522003in}{0.550706in}}%
\pgfpathlineto{\pgfqpoint{5.524756in}{0.549166in}}%
\pgfpathlineto{\pgfqpoint{5.527360in}{0.547920in}}%
\pgfpathlineto{\pgfqpoint{5.530148in}{0.547380in}}%
\pgfpathlineto{\pgfqpoint{5.532717in}{0.544688in}}%
\pgfpathlineto{\pgfqpoint{5.535395in}{0.545406in}}%
\pgfpathlineto{\pgfqpoint{5.538074in}{0.542201in}}%
\pgfpathlineto{\pgfqpoint{5.540750in}{0.543660in}}%
\pgfpathlineto{\pgfqpoint{5.543421in}{0.545029in}}%
\pgfpathlineto{\pgfqpoint{5.546110in}{0.541961in}}%
\pgfpathlineto{\pgfqpoint{5.548921in}{0.549161in}}%
\pgfpathlineto{\pgfqpoint{5.551457in}{0.544381in}}%
\pgfpathlineto{\pgfqpoint{5.554198in}{0.548872in}}%
\pgfpathlineto{\pgfqpoint{5.556822in}{0.542043in}}%
\pgfpathlineto{\pgfqpoint{5.559612in}{0.540242in}}%
\pgfpathlineto{\pgfqpoint{5.562180in}{0.538600in}}%
\pgfpathlineto{\pgfqpoint{5.564940in}{0.541428in}}%
\pgfpathlineto{\pgfqpoint{5.567536in}{0.541349in}}%
\pgfpathlineto{\pgfqpoint{5.570215in}{0.543237in}}%
\pgfpathlineto{\pgfqpoint{5.572893in}{0.542840in}}%
\pgfpathlineto{\pgfqpoint{5.575596in}{0.546985in}}%
\pgfpathlineto{\pgfqpoint{5.578342in}{0.542902in}}%
\pgfpathlineto{\pgfqpoint{5.580914in}{0.549118in}}%
\pgfpathlineto{\pgfqpoint{5.583709in}{0.556175in}}%
\pgfpathlineto{\pgfqpoint{5.586269in}{0.555634in}}%
\pgfpathlineto{\pgfqpoint{5.589040in}{0.549377in}}%
\pgfpathlineto{\pgfqpoint{5.591641in}{0.546938in}}%
\pgfpathlineto{\pgfqpoint{5.594368in}{0.552135in}}%
\pgfpathlineto{\pgfqpoint{5.596999in}{0.547695in}}%
\pgfpathlineto{\pgfqpoint{5.599674in}{0.547670in}}%
\pgfpathlineto{\pgfqpoint{5.602352in}{0.548597in}}%
\pgfpathlineto{\pgfqpoint{5.605073in}{0.545441in}}%
\pgfpathlineto{\pgfqpoint{5.607698in}{0.547607in}}%
\pgfpathlineto{\pgfqpoint{5.610389in}{0.541218in}}%
\pgfpathlineto{\pgfqpoint{5.613235in}{0.536139in}}%
\pgfpathlineto{\pgfqpoint{5.615743in}{0.546982in}}%
\pgfpathlineto{\pgfqpoint{5.618526in}{0.544346in}}%
\pgfpathlineto{\pgfqpoint{5.621102in}{0.543482in}}%
\pgfpathlineto{\pgfqpoint{5.623868in}{0.548593in}}%
\pgfpathlineto{\pgfqpoint{5.626460in}{0.549682in}}%
\pgfpathlineto{\pgfqpoint{5.629232in}{0.543212in}}%
\pgfpathlineto{\pgfqpoint{5.631815in}{0.546524in}}%
\pgfpathlineto{\pgfqpoint{5.634496in}{0.546114in}}%
\pgfpathlineto{\pgfqpoint{5.637172in}{0.555827in}}%
\pgfpathlineto{\pgfqpoint{5.639852in}{0.564647in}}%
\pgfpathlineto{\pgfqpoint{5.642518in}{0.574793in}}%
\pgfpathlineto{\pgfqpoint{5.645243in}{0.566856in}}%
\pgfpathlineto{\pgfqpoint{5.648008in}{0.574871in}}%
\pgfpathlineto{\pgfqpoint{5.650563in}{0.570135in}}%
\pgfpathlineto{\pgfqpoint{5.653376in}{0.560951in}}%
\pgfpathlineto{\pgfqpoint{5.655919in}{0.550879in}}%
\pgfpathlineto{\pgfqpoint{5.658723in}{0.548143in}}%
\pgfpathlineto{\pgfqpoint{5.661273in}{0.537960in}}%
\pgfpathlineto{\pgfqpoint{5.664099in}{0.541350in}}%
\pgfpathlineto{\pgfqpoint{5.666632in}{0.553711in}}%
\pgfpathlineto{\pgfqpoint{5.669313in}{0.554415in}}%
\pgfpathlineto{\pgfqpoint{5.671991in}{0.542073in}}%
\pgfpathlineto{\pgfqpoint{5.674667in}{0.541539in}}%
\pgfpathlineto{\pgfqpoint{5.677486in}{0.540320in}}%
\pgfpathlineto{\pgfqpoint{5.680027in}{0.540551in}}%
\pgfpathlineto{\pgfqpoint{5.682836in}{0.537988in}}%
\pgfpathlineto{\pgfqpoint{5.685385in}{0.542702in}}%
\pgfpathlineto{\pgfqpoint{5.688159in}{0.548139in}}%
\pgfpathlineto{\pgfqpoint{5.690730in}{0.548824in}}%
\pgfpathlineto{\pgfqpoint{5.693473in}{0.561401in}}%
\pgfpathlineto{\pgfqpoint{5.696101in}{0.559434in}}%
\pgfpathlineto{\pgfqpoint{5.698775in}{0.550639in}}%
\pgfpathlineto{\pgfqpoint{5.701453in}{0.554436in}}%
\pgfpathlineto{\pgfqpoint{5.704130in}{0.545846in}}%
\pgfpathlineto{\pgfqpoint{5.706800in}{0.545403in}}%
\pgfpathlineto{\pgfqpoint{5.709490in}{0.542707in}}%
\pgfpathlineto{\pgfqpoint{5.712291in}{0.544363in}}%
\pgfpathlineto{\pgfqpoint{5.714834in}{0.539284in}}%
\pgfpathlineto{\pgfqpoint{5.717671in}{0.543339in}}%
\pgfpathlineto{\pgfqpoint{5.720201in}{0.540717in}}%
\pgfpathlineto{\pgfqpoint{5.722950in}{0.546776in}}%
\pgfpathlineto{\pgfqpoint{5.725548in}{0.544751in}}%
\pgfpathlineto{\pgfqpoint{5.728339in}{0.543157in}}%
\pgfpathlineto{\pgfqpoint{5.730919in}{0.556168in}}%
\pgfpathlineto{\pgfqpoint{5.733594in}{0.554152in}}%
\pgfpathlineto{\pgfqpoint{5.736276in}{0.548461in}}%
\pgfpathlineto{\pgfqpoint{5.738974in}{0.550391in}}%
\pgfpathlineto{\pgfqpoint{5.741745in}{0.547092in}}%
\pgfpathlineto{\pgfqpoint{5.744310in}{0.543156in}}%
\pgfpathlineto{\pgfqpoint{5.744310in}{0.413320in}}%
\pgfpathlineto{\pgfqpoint{5.744310in}{0.413320in}}%
\pgfpathlineto{\pgfqpoint{5.741745in}{0.413320in}}%
\pgfpathlineto{\pgfqpoint{5.738974in}{0.413320in}}%
\pgfpathlineto{\pgfqpoint{5.736276in}{0.413320in}}%
\pgfpathlineto{\pgfqpoint{5.733594in}{0.413320in}}%
\pgfpathlineto{\pgfqpoint{5.730919in}{0.413320in}}%
\pgfpathlineto{\pgfqpoint{5.728339in}{0.413320in}}%
\pgfpathlineto{\pgfqpoint{5.725548in}{0.413320in}}%
\pgfpathlineto{\pgfqpoint{5.722950in}{0.413320in}}%
\pgfpathlineto{\pgfqpoint{5.720201in}{0.413320in}}%
\pgfpathlineto{\pgfqpoint{5.717671in}{0.413320in}}%
\pgfpathlineto{\pgfqpoint{5.714834in}{0.413320in}}%
\pgfpathlineto{\pgfqpoint{5.712291in}{0.413320in}}%
\pgfpathlineto{\pgfqpoint{5.709490in}{0.413320in}}%
\pgfpathlineto{\pgfqpoint{5.706800in}{0.413320in}}%
\pgfpathlineto{\pgfqpoint{5.704130in}{0.413320in}}%
\pgfpathlineto{\pgfqpoint{5.701453in}{0.413320in}}%
\pgfpathlineto{\pgfqpoint{5.698775in}{0.413320in}}%
\pgfpathlineto{\pgfqpoint{5.696101in}{0.413320in}}%
\pgfpathlineto{\pgfqpoint{5.693473in}{0.413320in}}%
\pgfpathlineto{\pgfqpoint{5.690730in}{0.413320in}}%
\pgfpathlineto{\pgfqpoint{5.688159in}{0.413320in}}%
\pgfpathlineto{\pgfqpoint{5.685385in}{0.413320in}}%
\pgfpathlineto{\pgfqpoint{5.682836in}{0.413320in}}%
\pgfpathlineto{\pgfqpoint{5.680027in}{0.413320in}}%
\pgfpathlineto{\pgfqpoint{5.677486in}{0.413320in}}%
\pgfpathlineto{\pgfqpoint{5.674667in}{0.413320in}}%
\pgfpathlineto{\pgfqpoint{5.671991in}{0.413320in}}%
\pgfpathlineto{\pgfqpoint{5.669313in}{0.413320in}}%
\pgfpathlineto{\pgfqpoint{5.666632in}{0.413320in}}%
\pgfpathlineto{\pgfqpoint{5.664099in}{0.413320in}}%
\pgfpathlineto{\pgfqpoint{5.661273in}{0.413320in}}%
\pgfpathlineto{\pgfqpoint{5.658723in}{0.413320in}}%
\pgfpathlineto{\pgfqpoint{5.655919in}{0.413320in}}%
\pgfpathlineto{\pgfqpoint{5.653376in}{0.413320in}}%
\pgfpathlineto{\pgfqpoint{5.650563in}{0.413320in}}%
\pgfpathlineto{\pgfqpoint{5.648008in}{0.413320in}}%
\pgfpathlineto{\pgfqpoint{5.645243in}{0.413320in}}%
\pgfpathlineto{\pgfqpoint{5.642518in}{0.413320in}}%
\pgfpathlineto{\pgfqpoint{5.639852in}{0.413320in}}%
\pgfpathlineto{\pgfqpoint{5.637172in}{0.413320in}}%
\pgfpathlineto{\pgfqpoint{5.634496in}{0.413320in}}%
\pgfpathlineto{\pgfqpoint{5.631815in}{0.413320in}}%
\pgfpathlineto{\pgfqpoint{5.629232in}{0.413320in}}%
\pgfpathlineto{\pgfqpoint{5.626460in}{0.413320in}}%
\pgfpathlineto{\pgfqpoint{5.623868in}{0.413320in}}%
\pgfpathlineto{\pgfqpoint{5.621102in}{0.413320in}}%
\pgfpathlineto{\pgfqpoint{5.618526in}{0.413320in}}%
\pgfpathlineto{\pgfqpoint{5.615743in}{0.413320in}}%
\pgfpathlineto{\pgfqpoint{5.613235in}{0.413320in}}%
\pgfpathlineto{\pgfqpoint{5.610389in}{0.413320in}}%
\pgfpathlineto{\pgfqpoint{5.607698in}{0.413320in}}%
\pgfpathlineto{\pgfqpoint{5.605073in}{0.413320in}}%
\pgfpathlineto{\pgfqpoint{5.602352in}{0.413320in}}%
\pgfpathlineto{\pgfqpoint{5.599674in}{0.413320in}}%
\pgfpathlineto{\pgfqpoint{5.596999in}{0.413320in}}%
\pgfpathlineto{\pgfqpoint{5.594368in}{0.413320in}}%
\pgfpathlineto{\pgfqpoint{5.591641in}{0.413320in}}%
\pgfpathlineto{\pgfqpoint{5.589040in}{0.413320in}}%
\pgfpathlineto{\pgfqpoint{5.586269in}{0.413320in}}%
\pgfpathlineto{\pgfqpoint{5.583709in}{0.413320in}}%
\pgfpathlineto{\pgfqpoint{5.580914in}{0.413320in}}%
\pgfpathlineto{\pgfqpoint{5.578342in}{0.413320in}}%
\pgfpathlineto{\pgfqpoint{5.575596in}{0.413320in}}%
\pgfpathlineto{\pgfqpoint{5.572893in}{0.413320in}}%
\pgfpathlineto{\pgfqpoint{5.570215in}{0.413320in}}%
\pgfpathlineto{\pgfqpoint{5.567536in}{0.413320in}}%
\pgfpathlineto{\pgfqpoint{5.564940in}{0.413320in}}%
\pgfpathlineto{\pgfqpoint{5.562180in}{0.413320in}}%
\pgfpathlineto{\pgfqpoint{5.559612in}{0.413320in}}%
\pgfpathlineto{\pgfqpoint{5.556822in}{0.413320in}}%
\pgfpathlineto{\pgfqpoint{5.554198in}{0.413320in}}%
\pgfpathlineto{\pgfqpoint{5.551457in}{0.413320in}}%
\pgfpathlineto{\pgfqpoint{5.548921in}{0.413320in}}%
\pgfpathlineto{\pgfqpoint{5.546110in}{0.413320in}}%
\pgfpathlineto{\pgfqpoint{5.543421in}{0.413320in}}%
\pgfpathlineto{\pgfqpoint{5.540750in}{0.413320in}}%
\pgfpathlineto{\pgfqpoint{5.538074in}{0.413320in}}%
\pgfpathlineto{\pgfqpoint{5.535395in}{0.413320in}}%
\pgfpathlineto{\pgfqpoint{5.532717in}{0.413320in}}%
\pgfpathlineto{\pgfqpoint{5.530148in}{0.413320in}}%
\pgfpathlineto{\pgfqpoint{5.527360in}{0.413320in}}%
\pgfpathlineto{\pgfqpoint{5.524756in}{0.413320in}}%
\pgfpathlineto{\pgfqpoint{5.522003in}{0.413320in}}%
\pgfpathlineto{\pgfqpoint{5.519433in}{0.413320in}}%
\pgfpathlineto{\pgfqpoint{5.516646in}{0.413320in}}%
\pgfpathlineto{\pgfqpoint{5.514080in}{0.413320in}}%
\pgfpathlineto{\pgfqpoint{5.511290in}{0.413320in}}%
\pgfpathlineto{\pgfqpoint{5.508612in}{0.413320in}}%
\pgfpathlineto{\pgfqpoint{5.505933in}{0.413320in}}%
\pgfpathlineto{\pgfqpoint{5.503255in}{0.413320in}}%
\pgfpathlineto{\pgfqpoint{5.500687in}{0.413320in}}%
\pgfpathlineto{\pgfqpoint{5.497898in}{0.413320in}}%
\pgfpathlineto{\pgfqpoint{5.495346in}{0.413320in}}%
\pgfpathlineto{\pgfqpoint{5.492541in}{0.413320in}}%
\pgfpathlineto{\pgfqpoint{5.490000in}{0.413320in}}%
\pgfpathlineto{\pgfqpoint{5.487176in}{0.413320in}}%
\pgfpathlineto{\pgfqpoint{5.484641in}{0.413320in}}%
\pgfpathlineto{\pgfqpoint{5.481825in}{0.413320in}}%
\pgfpathlineto{\pgfqpoint{5.479152in}{0.413320in}}%
\pgfpathlineto{\pgfqpoint{5.476458in}{0.413320in}}%
\pgfpathlineto{\pgfqpoint{5.473792in}{0.413320in}}%
\pgfpathlineto{\pgfqpoint{5.471113in}{0.413320in}}%
\pgfpathlineto{\pgfqpoint{5.468425in}{0.413320in}}%
\pgfpathlineto{\pgfqpoint{5.465888in}{0.413320in}}%
\pgfpathlineto{\pgfqpoint{5.463079in}{0.413320in}}%
\pgfpathlineto{\pgfqpoint{5.460489in}{0.413320in}}%
\pgfpathlineto{\pgfqpoint{5.457721in}{0.413320in}}%
\pgfpathlineto{\pgfqpoint{5.455168in}{0.413320in}}%
\pgfpathlineto{\pgfqpoint{5.452365in}{0.413320in}}%
\pgfpathlineto{\pgfqpoint{5.449769in}{0.413320in}}%
\pgfpathlineto{\pgfqpoint{5.447021in}{0.413320in}}%
\pgfpathlineto{\pgfqpoint{5.444328in}{0.413320in}}%
\pgfpathlineto{\pgfqpoint{5.441698in}{0.413320in}}%
\pgfpathlineto{\pgfqpoint{5.438974in}{0.413320in}}%
\pgfpathlineto{\pgfqpoint{5.436295in}{0.413320in}}%
\pgfpathlineto{\pgfqpoint{5.433616in}{0.413320in}}%
\pgfpathlineto{\pgfqpoint{5.431015in}{0.413320in}}%
\pgfpathlineto{\pgfqpoint{5.428259in}{0.413320in}}%
\pgfpathlineto{\pgfqpoint{5.425661in}{0.413320in}}%
\pgfpathlineto{\pgfqpoint{5.422897in}{0.413320in}}%
\pgfpathlineto{\pgfqpoint{5.420304in}{0.413320in}}%
\pgfpathlineto{\pgfqpoint{5.417547in}{0.413320in}}%
\pgfpathlineto{\pgfqpoint{5.414954in}{0.413320in}}%
\pgfpathlineto{\pgfqpoint{5.412190in}{0.413320in}}%
\pgfpathlineto{\pgfqpoint{5.409507in}{0.413320in}}%
\pgfpathlineto{\pgfqpoint{5.406832in}{0.413320in}}%
\pgfpathlineto{\pgfqpoint{5.404154in}{0.413320in}}%
\pgfpathlineto{\pgfqpoint{5.401576in}{0.413320in}}%
\pgfpathlineto{\pgfqpoint{5.398784in}{0.413320in}}%
\pgfpathlineto{\pgfqpoint{5.396219in}{0.413320in}}%
\pgfpathlineto{\pgfqpoint{5.393441in}{0.413320in}}%
\pgfpathlineto{\pgfqpoint{5.390900in}{0.413320in}}%
\pgfpathlineto{\pgfqpoint{5.388083in}{0.413320in}}%
\pgfpathlineto{\pgfqpoint{5.385550in}{0.413320in}}%
\pgfpathlineto{\pgfqpoint{5.382725in}{0.413320in}}%
\pgfpathlineto{\pgfqpoint{5.380048in}{0.413320in}}%
\pgfpathlineto{\pgfqpoint{5.377370in}{0.413320in}}%
\pgfpathlineto{\pgfqpoint{5.374692in}{0.413320in}}%
\pgfpathlineto{\pgfqpoint{5.372013in}{0.413320in}}%
\pgfpathlineto{\pgfqpoint{5.369335in}{0.413320in}}%
\pgfpathlineto{\pgfqpoint{5.366727in}{0.413320in}}%
\pgfpathlineto{\pgfqpoint{5.363966in}{0.413320in}}%
\pgfpathlineto{\pgfqpoint{5.361370in}{0.413320in}}%
\pgfpathlineto{\pgfqpoint{5.358612in}{0.413320in}}%
\pgfpathlineto{\pgfqpoint{5.356056in}{0.413320in}}%
\pgfpathlineto{\pgfqpoint{5.353262in}{0.413320in}}%
\pgfpathlineto{\pgfqpoint{5.350723in}{0.413320in}}%
\pgfpathlineto{\pgfqpoint{5.347905in}{0.413320in}}%
\pgfpathlineto{\pgfqpoint{5.345224in}{0.413320in}}%
\pgfpathlineto{\pgfqpoint{5.342549in}{0.413320in}}%
\pgfpathlineto{\pgfqpoint{5.339872in}{0.413320in}}%
\pgfpathlineto{\pgfqpoint{5.337353in}{0.413320in}}%
\pgfpathlineto{\pgfqpoint{5.334510in}{0.413320in}}%
\pgfpathlineto{\pgfqpoint{5.331973in}{0.413320in}}%
\pgfpathlineto{\pgfqpoint{5.329159in}{0.413320in}}%
\pgfpathlineto{\pgfqpoint{5.326564in}{0.413320in}}%
\pgfpathlineto{\pgfqpoint{5.323802in}{0.413320in}}%
\pgfpathlineto{\pgfqpoint{5.321256in}{0.413320in}}%
\pgfpathlineto{\pgfqpoint{5.318430in}{0.413320in}}%
\pgfpathlineto{\pgfqpoint{5.315754in}{0.413320in}}%
\pgfpathlineto{\pgfqpoint{5.313089in}{0.413320in}}%
\pgfpathlineto{\pgfqpoint{5.310411in}{0.413320in}}%
\pgfpathlineto{\pgfqpoint{5.307731in}{0.413320in}}%
\pgfpathlineto{\pgfqpoint{5.305054in}{0.413320in}}%
\pgfpathlineto{\pgfqpoint{5.302443in}{0.413320in}}%
\pgfpathlineto{\pgfqpoint{5.299696in}{0.413320in}}%
\pgfpathlineto{\pgfqpoint{5.297140in}{0.413320in}}%
\pgfpathlineto{\pgfqpoint{5.294339in}{0.413320in}}%
\pgfpathlineto{\pgfqpoint{5.291794in}{0.413320in}}%
\pgfpathlineto{\pgfqpoint{5.288984in}{0.413320in}}%
\pgfpathlineto{\pgfqpoint{5.286436in}{0.413320in}}%
\pgfpathlineto{\pgfqpoint{5.283631in}{0.413320in}}%
\pgfpathlineto{\pgfqpoint{5.280947in}{0.413320in}}%
\pgfpathlineto{\pgfqpoint{5.278322in}{0.413320in}}%
\pgfpathlineto{\pgfqpoint{5.275589in}{0.413320in}}%
\pgfpathlineto{\pgfqpoint{5.272913in}{0.413320in}}%
\pgfpathlineto{\pgfqpoint{5.270238in}{0.413320in}}%
\pgfpathlineto{\pgfqpoint{5.267691in}{0.413320in}}%
\pgfpathlineto{\pgfqpoint{5.264876in}{0.413320in}}%
\pgfpathlineto{\pgfqpoint{5.262264in}{0.413320in}}%
\pgfpathlineto{\pgfqpoint{5.259511in}{0.413320in}}%
\pgfpathlineto{\pgfqpoint{5.256973in}{0.413320in}}%
\pgfpathlineto{\pgfqpoint{5.254236in}{0.413320in}}%
\pgfpathlineto{\pgfqpoint{5.251590in}{0.413320in}}%
\pgfpathlineto{\pgfqpoint{5.248816in}{0.413320in}}%
\pgfpathlineto{\pgfqpoint{5.246130in}{0.413320in}}%
\pgfpathlineto{\pgfqpoint{5.243445in}{0.413320in}}%
\pgfpathlineto{\pgfqpoint{5.240777in}{0.413320in}}%
\pgfpathlineto{\pgfqpoint{5.238173in}{0.413320in}}%
\pgfpathlineto{\pgfqpoint{5.235409in}{0.413320in}}%
\pgfpathlineto{\pgfqpoint{5.232855in}{0.413320in}}%
\pgfpathlineto{\pgfqpoint{5.230059in}{0.413320in}}%
\pgfpathlineto{\pgfqpoint{5.227470in}{0.413320in}}%
\pgfpathlineto{\pgfqpoint{5.224695in}{0.413320in}}%
\pgfpathlineto{\pgfqpoint{5.222151in}{0.413320in}}%
\pgfpathlineto{\pgfqpoint{5.219345in}{0.413320in}}%
\pgfpathlineto{\pgfqpoint{5.216667in}{0.413320in}}%
\pgfpathlineto{\pgfqpoint{5.214027in}{0.413320in}}%
\pgfpathlineto{\pgfqpoint{5.211299in}{0.413320in}}%
\pgfpathlineto{\pgfqpoint{5.208630in}{0.413320in}}%
\pgfpathlineto{\pgfqpoint{5.205952in}{0.413320in}}%
\pgfpathlineto{\pgfqpoint{5.203388in}{0.413320in}}%
\pgfpathlineto{\pgfqpoint{5.200594in}{0.413320in}}%
\pgfpathlineto{\pgfqpoint{5.198008in}{0.413320in}}%
\pgfpathlineto{\pgfqpoint{5.195239in}{0.413320in}}%
\pgfpathlineto{\pgfqpoint{5.192680in}{0.413320in}}%
\pgfpathlineto{\pgfqpoint{5.189880in}{0.413320in}}%
\pgfpathlineto{\pgfqpoint{5.187294in}{0.413320in}}%
\pgfpathlineto{\pgfqpoint{5.184522in}{0.413320in}}%
\pgfpathlineto{\pgfqpoint{5.181848in}{0.413320in}}%
\pgfpathlineto{\pgfqpoint{5.179188in}{0.413320in}}%
\pgfpathlineto{\pgfqpoint{5.176477in}{0.413320in}}%
\pgfpathlineto{\pgfqpoint{5.173925in}{0.413320in}}%
\pgfpathlineto{\pgfqpoint{5.171133in}{0.413320in}}%
\pgfpathlineto{\pgfqpoint{5.168591in}{0.413320in}}%
\pgfpathlineto{\pgfqpoint{5.165775in}{0.413320in}}%
\pgfpathlineto{\pgfqpoint{5.163243in}{0.413320in}}%
\pgfpathlineto{\pgfqpoint{5.160420in}{0.413320in}}%
\pgfpathlineto{\pgfqpoint{5.157815in}{0.413320in}}%
\pgfpathlineto{\pgfqpoint{5.155059in}{0.413320in}}%
\pgfpathlineto{\pgfqpoint{5.152382in}{0.413320in}}%
\pgfpathlineto{\pgfqpoint{5.149734in}{0.413320in}}%
\pgfpathlineto{\pgfqpoint{5.147029in}{0.413320in}}%
\pgfpathlineto{\pgfqpoint{5.144349in}{0.413320in}}%
\pgfpathlineto{\pgfqpoint{5.141660in}{0.413320in}}%
\pgfpathlineto{\pgfqpoint{5.139072in}{0.413320in}}%
\pgfpathlineto{\pgfqpoint{5.136311in}{0.413320in}}%
\pgfpathlineto{\pgfqpoint{5.133716in}{0.413320in}}%
\pgfpathlineto{\pgfqpoint{5.130953in}{0.413320in}}%
\pgfpathlineto{\pgfqpoint{5.128421in}{0.413320in}}%
\pgfpathlineto{\pgfqpoint{5.125599in}{0.413320in}}%
\pgfpathlineto{\pgfqpoint{5.123042in}{0.413320in}}%
\pgfpathlineto{\pgfqpoint{5.120243in}{0.413320in}}%
\pgfpathlineto{\pgfqpoint{5.117550in}{0.413320in}}%
\pgfpathlineto{\pgfqpoint{5.114887in}{0.413320in}}%
\pgfpathlineto{\pgfqpoint{5.112209in}{0.413320in}}%
\pgfpathlineto{\pgfqpoint{5.109530in}{0.413320in}}%
\pgfpathlineto{\pgfqpoint{5.106842in}{0.413320in}}%
\pgfpathlineto{\pgfqpoint{5.104312in}{0.413320in}}%
\pgfpathlineto{\pgfqpoint{5.101496in}{0.413320in}}%
\pgfpathlineto{\pgfqpoint{5.098948in}{0.413320in}}%
\pgfpathlineto{\pgfqpoint{5.096142in}{0.413320in}}%
\pgfpathlineto{\pgfqpoint{5.093579in}{0.413320in}}%
\pgfpathlineto{\pgfqpoint{5.090788in}{0.413320in}}%
\pgfpathlineto{\pgfqpoint{5.088103in}{0.413320in}}%
\pgfpathlineto{\pgfqpoint{5.085426in}{0.413320in}}%
\pgfpathlineto{\pgfqpoint{5.082746in}{0.413320in}}%
\pgfpathlineto{\pgfqpoint{5.080067in}{0.413320in}}%
\pgfpathlineto{\pgfqpoint{5.077390in}{0.413320in}}%
\pgfpathlineto{\pgfqpoint{5.074851in}{0.413320in}}%
\pgfpathlineto{\pgfqpoint{5.072030in}{0.413320in}}%
\pgfpathlineto{\pgfqpoint{5.069463in}{0.413320in}}%
\pgfpathlineto{\pgfqpoint{5.066677in}{0.413320in}}%
\pgfpathlineto{\pgfqpoint{5.064144in}{0.413320in}}%
\pgfpathlineto{\pgfqpoint{5.061315in}{0.413320in}}%
\pgfpathlineto{\pgfqpoint{5.058711in}{0.413320in}}%
\pgfpathlineto{\pgfqpoint{5.055952in}{0.413320in}}%
\pgfpathlineto{\pgfqpoint{5.053284in}{0.413320in}}%
\pgfpathlineto{\pgfqpoint{5.050606in}{0.413320in}}%
\pgfpathlineto{\pgfqpoint{5.047924in}{0.413320in}}%
\pgfpathlineto{\pgfqpoint{5.045249in}{0.413320in}}%
\pgfpathlineto{\pgfqpoint{5.042572in}{0.413320in}}%
\pgfpathlineto{\pgfqpoint{5.039962in}{0.413320in}}%
\pgfpathlineto{\pgfqpoint{5.037214in}{0.413320in}}%
\pgfpathlineto{\pgfqpoint{5.034649in}{0.413320in}}%
\pgfpathlineto{\pgfqpoint{5.031849in}{0.413320in}}%
\pgfpathlineto{\pgfqpoint{5.029275in}{0.413320in}}%
\pgfpathlineto{\pgfqpoint{5.026501in}{0.413320in}}%
\pgfpathlineto{\pgfqpoint{5.023927in}{0.413320in}}%
\pgfpathlineto{\pgfqpoint{5.021147in}{0.413320in}}%
\pgfpathlineto{\pgfqpoint{5.018466in}{0.413320in}}%
\pgfpathlineto{\pgfqpoint{5.015820in}{0.413320in}}%
\pgfpathlineto{\pgfqpoint{5.013104in}{0.413320in}}%
\pgfpathlineto{\pgfqpoint{5.010562in}{0.413320in}}%
\pgfpathlineto{\pgfqpoint{5.007751in}{0.413320in}}%
\pgfpathlineto{\pgfqpoint{5.005178in}{0.413320in}}%
\pgfpathlineto{\pgfqpoint{5.002384in}{0.413320in}}%
\pgfpathlineto{\pgfqpoint{4.999780in}{0.413320in}}%
\pgfpathlineto{\pgfqpoint{4.997028in}{0.413320in}}%
\pgfpathlineto{\pgfqpoint{4.994390in}{0.413320in}}%
\pgfpathlineto{\pgfqpoint{4.991683in}{0.413320in}}%
\pgfpathlineto{\pgfqpoint{4.989001in}{0.413320in}}%
\pgfpathlineto{\pgfqpoint{4.986325in}{0.413320in}}%
\pgfpathlineto{\pgfqpoint{4.983637in}{0.413320in}}%
\pgfpathlineto{\pgfqpoint{4.980967in}{0.413320in}}%
\pgfpathlineto{\pgfqpoint{4.978287in}{0.413320in}}%
\pgfpathlineto{\pgfqpoint{4.975703in}{0.413320in}}%
\pgfpathlineto{\pgfqpoint{4.972933in}{0.413320in}}%
\pgfpathlineto{\pgfqpoint{4.970314in}{0.413320in}}%
\pgfpathlineto{\pgfqpoint{4.967575in}{0.413320in}}%
\pgfpathlineto{\pgfqpoint{4.965002in}{0.413320in}}%
\pgfpathlineto{\pgfqpoint{4.962219in}{0.413320in}}%
\pgfpathlineto{\pgfqpoint{4.959689in}{0.413320in}}%
\pgfpathlineto{\pgfqpoint{4.956862in}{0.413320in}}%
\pgfpathlineto{\pgfqpoint{4.954182in}{0.413320in}}%
\pgfpathlineto{\pgfqpoint{4.951504in}{0.413320in}}%
\pgfpathlineto{\pgfqpoint{4.948827in}{0.413320in}}%
\pgfpathlineto{\pgfqpoint{4.946151in}{0.413320in}}%
\pgfpathlineto{\pgfqpoint{4.943466in}{0.413320in}}%
\pgfpathlineto{\pgfqpoint{4.940881in}{0.413320in}}%
\pgfpathlineto{\pgfqpoint{4.938112in}{0.413320in}}%
\pgfpathlineto{\pgfqpoint{4.935515in}{0.413320in}}%
\pgfpathlineto{\pgfqpoint{4.932742in}{0.413320in}}%
\pgfpathlineto{\pgfqpoint{4.930170in}{0.413320in}}%
\pgfpathlineto{\pgfqpoint{4.927400in}{0.413320in}}%
\pgfpathlineto{\pgfqpoint{4.924708in}{0.413320in}}%
\pgfpathlineto{\pgfqpoint{4.922041in}{0.413320in}}%
\pgfpathlineto{\pgfqpoint{4.919352in}{0.413320in}}%
\pgfpathlineto{\pgfqpoint{4.916681in}{0.413320in}}%
\pgfpathlineto{\pgfqpoint{4.914009in}{0.413320in}}%
\pgfpathlineto{\pgfqpoint{4.911435in}{0.413320in}}%
\pgfpathlineto{\pgfqpoint{4.908648in}{0.413320in}}%
\pgfpathlineto{\pgfqpoint{4.906096in}{0.413320in}}%
\pgfpathlineto{\pgfqpoint{4.903295in}{0.413320in}}%
\pgfpathlineto{\pgfqpoint{4.900712in}{0.413320in}}%
\pgfpathlineto{\pgfqpoint{4.897938in}{0.413320in}}%
\pgfpathlineto{\pgfqpoint{4.895399in}{0.413320in}}%
\pgfpathlineto{\pgfqpoint{4.892611in}{0.413320in}}%
\pgfpathlineto{\pgfqpoint{4.889902in}{0.413320in}}%
\pgfpathlineto{\pgfqpoint{4.887211in}{0.413320in}}%
\pgfpathlineto{\pgfqpoint{4.884540in}{0.413320in}}%
\pgfpathlineto{\pgfqpoint{4.881864in}{0.413320in}}%
\pgfpathlineto{\pgfqpoint{4.879180in}{0.413320in}}%
\pgfpathlineto{\pgfqpoint{4.876636in}{0.413320in}}%
\pgfpathlineto{\pgfqpoint{4.873832in}{0.413320in}}%
\pgfpathlineto{\pgfqpoint{4.871209in}{0.413320in}}%
\pgfpathlineto{\pgfqpoint{4.868474in}{0.413320in}}%
\pgfpathlineto{\pgfqpoint{4.865910in}{0.413320in}}%
\pgfpathlineto{\pgfqpoint{4.863116in}{0.413320in}}%
\pgfpathlineto{\pgfqpoint{4.860544in}{0.413320in}}%
\pgfpathlineto{\pgfqpoint{4.857807in}{0.413320in}}%
\pgfpathlineto{\pgfqpoint{4.855070in}{0.413320in}}%
\pgfpathlineto{\pgfqpoint{4.852404in}{0.413320in}}%
\pgfpathlineto{\pgfqpoint{4.849715in}{0.413320in}}%
\pgfpathlineto{\pgfqpoint{4.847127in}{0.413320in}}%
\pgfpathlineto{\pgfqpoint{4.844361in}{0.413320in}}%
\pgfpathlineto{\pgfqpoint{4.842380in}{0.413320in}}%
\pgfpathlineto{\pgfqpoint{4.839922in}{0.413320in}}%
\pgfpathlineto{\pgfqpoint{4.837992in}{0.413320in}}%
\pgfpathlineto{\pgfqpoint{4.833657in}{0.413320in}}%
\pgfpathlineto{\pgfqpoint{4.831045in}{0.413320in}}%
\pgfpathlineto{\pgfqpoint{4.828291in}{0.413320in}}%
\pgfpathlineto{\pgfqpoint{4.825619in}{0.413320in}}%
\pgfpathlineto{\pgfqpoint{4.822945in}{0.413320in}}%
\pgfpathlineto{\pgfqpoint{4.820265in}{0.413320in}}%
\pgfpathlineto{\pgfqpoint{4.817587in}{0.413320in}}%
\pgfpathlineto{\pgfqpoint{4.814907in}{0.413320in}}%
\pgfpathlineto{\pgfqpoint{4.812377in}{0.413320in}}%
\pgfpathlineto{\pgfqpoint{4.809538in}{0.413320in}}%
\pgfpathlineto{\pgfqpoint{4.807017in}{0.413320in}}%
\pgfpathlineto{\pgfqpoint{4.804193in}{0.413320in}}%
\pgfpathlineto{\pgfqpoint{4.801586in}{0.413320in}}%
\pgfpathlineto{\pgfqpoint{4.798830in}{0.413320in}}%
\pgfpathlineto{\pgfqpoint{4.796274in}{0.413320in}}%
\pgfpathlineto{\pgfqpoint{4.793512in}{0.413320in}}%
\pgfpathlineto{\pgfqpoint{4.790798in}{0.413320in}}%
\pgfpathlineto{\pgfqpoint{4.788116in}{0.413320in}}%
\pgfpathlineto{\pgfqpoint{4.785445in}{0.413320in}}%
\pgfpathlineto{\pgfqpoint{4.782872in}{0.413320in}}%
\pgfpathlineto{\pgfqpoint{4.780083in}{0.413320in}}%
\pgfpathlineto{\pgfqpoint{4.777535in}{0.413320in}}%
\pgfpathlineto{\pgfqpoint{4.774732in}{0.413320in}}%
\pgfpathlineto{\pgfqpoint{4.772198in}{0.413320in}}%
\pgfpathlineto{\pgfqpoint{4.769367in}{0.413320in}}%
\pgfpathlineto{\pgfqpoint{4.766783in}{0.413320in}}%
\pgfpathlineto{\pgfqpoint{4.764018in}{0.413320in}}%
\pgfpathlineto{\pgfqpoint{4.761337in}{0.413320in}}%
\pgfpathlineto{\pgfqpoint{4.758653in}{0.413320in}}%
\pgfpathlineto{\pgfqpoint{4.755983in}{0.413320in}}%
\pgfpathlineto{\pgfqpoint{4.753298in}{0.413320in}}%
\pgfpathlineto{\pgfqpoint{4.750627in}{0.413320in}}%
\pgfpathlineto{\pgfqpoint{4.748081in}{0.413320in}}%
\pgfpathlineto{\pgfqpoint{4.745256in}{0.413320in}}%
\pgfpathlineto{\pgfqpoint{4.742696in}{0.413320in}}%
\pgfpathlineto{\pgfqpoint{4.739912in}{0.413320in}}%
\pgfpathlineto{\pgfqpoint{4.737348in}{0.413320in}}%
\pgfpathlineto{\pgfqpoint{4.734552in}{0.413320in}}%
\pgfpathlineto{\pgfqpoint{4.731901in}{0.413320in}}%
\pgfpathlineto{\pgfqpoint{4.729233in}{0.413320in}}%
\pgfpathlineto{\pgfqpoint{4.726508in}{0.413320in}}%
\pgfpathlineto{\pgfqpoint{4.723873in}{0.413320in}}%
\pgfpathlineto{\pgfqpoint{4.721160in}{0.413320in}}%
\pgfpathlineto{\pgfqpoint{4.718486in}{0.413320in}}%
\pgfpathlineto{\pgfqpoint{4.715806in}{0.413320in}}%
\pgfpathlineto{\pgfqpoint{4.713275in}{0.413320in}}%
\pgfpathlineto{\pgfqpoint{4.710437in}{0.413320in}}%
\pgfpathlineto{\pgfqpoint{4.707824in}{0.413320in}}%
\pgfpathlineto{\pgfqpoint{4.705094in}{0.413320in}}%
\pgfpathlineto{\pgfqpoint{4.702517in}{0.413320in}}%
\pgfpathlineto{\pgfqpoint{4.699734in}{0.413320in}}%
\pgfpathlineto{\pgfqpoint{4.697170in}{0.413320in}}%
\pgfpathlineto{\pgfqpoint{4.694381in}{0.413320in}}%
\pgfpathlineto{\pgfqpoint{4.691694in}{0.413320in}}%
\pgfpathlineto{\pgfqpoint{4.689051in}{0.413320in}}%
\pgfpathlineto{\pgfqpoint{4.686337in}{0.413320in}}%
\pgfpathlineto{\pgfqpoint{4.683799in}{0.413320in}}%
\pgfpathlineto{\pgfqpoint{4.680988in}{0.413320in}}%
\pgfpathlineto{\pgfqpoint{4.678448in}{0.413320in}}%
\pgfpathlineto{\pgfqpoint{4.675619in}{0.413320in}}%
\pgfpathlineto{\pgfqpoint{4.673068in}{0.413320in}}%
\pgfpathlineto{\pgfqpoint{4.670261in}{0.413320in}}%
\pgfpathlineto{\pgfqpoint{4.667764in}{0.413320in}}%
\pgfpathlineto{\pgfqpoint{4.664923in}{0.413320in}}%
\pgfpathlineto{\pgfqpoint{4.662237in}{0.413320in}}%
\pgfpathlineto{\pgfqpoint{4.659590in}{0.413320in}}%
\pgfpathlineto{\pgfqpoint{4.656873in}{0.413320in}}%
\pgfpathlineto{\pgfqpoint{4.654203in}{0.413320in}}%
\pgfpathlineto{\pgfqpoint{4.651524in}{0.413320in}}%
\pgfpathlineto{\pgfqpoint{4.648922in}{0.413320in}}%
\pgfpathlineto{\pgfqpoint{4.646169in}{0.413320in}}%
\pgfpathlineto{\pgfqpoint{4.643628in}{0.413320in}}%
\pgfpathlineto{\pgfqpoint{4.640809in}{0.413320in}}%
\pgfpathlineto{\pgfqpoint{4.638204in}{0.413320in}}%
\pgfpathlineto{\pgfqpoint{4.635445in}{0.413320in}}%
\pgfpathlineto{\pgfqpoint{4.632902in}{0.413320in}}%
\pgfpathlineto{\pgfqpoint{4.630096in}{0.413320in}}%
\pgfpathlineto{\pgfqpoint{4.627411in}{0.413320in}}%
\pgfpathlineto{\pgfqpoint{4.624741in}{0.413320in}}%
\pgfpathlineto{\pgfqpoint{4.622056in}{0.413320in}}%
\pgfpathlineto{\pgfqpoint{4.619529in}{0.413320in}}%
\pgfpathlineto{\pgfqpoint{4.616702in}{0.413320in}}%
\pgfpathlineto{\pgfqpoint{4.614134in}{0.413320in}}%
\pgfpathlineto{\pgfqpoint{4.611350in}{0.413320in}}%
\pgfpathlineto{\pgfqpoint{4.608808in}{0.413320in}}%
\pgfpathlineto{\pgfqpoint{4.605990in}{0.413320in}}%
\pgfpathlineto{\pgfqpoint{4.603430in}{0.413320in}}%
\pgfpathlineto{\pgfqpoint{4.600633in}{0.413320in}}%
\pgfpathlineto{\pgfqpoint{4.597951in}{0.413320in}}%
\pgfpathlineto{\pgfqpoint{4.595281in}{0.413320in}}%
\pgfpathlineto{\pgfqpoint{4.592589in}{0.413320in}}%
\pgfpathlineto{\pgfqpoint{4.589920in}{0.413320in}}%
\pgfpathlineto{\pgfqpoint{4.587244in}{0.413320in}}%
\pgfpathlineto{\pgfqpoint{4.584672in}{0.413320in}}%
\pgfpathlineto{\pgfqpoint{4.581888in}{0.413320in}}%
\pgfpathlineto{\pgfqpoint{4.579305in}{0.413320in}}%
\pgfpathlineto{\pgfqpoint{4.576531in}{0.413320in}}%
\pgfpathlineto{\pgfqpoint{4.573947in}{0.413320in}}%
\pgfpathlineto{\pgfqpoint{4.571171in}{0.413320in}}%
\pgfpathlineto{\pgfqpoint{4.568612in}{0.413320in}}%
\pgfpathlineto{\pgfqpoint{4.565820in}{0.413320in}}%
\pgfpathlineto{\pgfqpoint{4.563125in}{0.413320in}}%
\pgfpathlineto{\pgfqpoint{4.560448in}{0.413320in}}%
\pgfpathlineto{\pgfqpoint{4.557777in}{0.413320in}}%
\pgfpathlineto{\pgfqpoint{4.555106in}{0.413320in}}%
\pgfpathlineto{\pgfqpoint{4.552425in}{0.413320in}}%
\pgfpathlineto{\pgfqpoint{4.549822in}{0.413320in}}%
\pgfpathlineto{\pgfqpoint{4.547064in}{0.413320in}}%
\pgfpathlineto{\pgfqpoint{4.544464in}{0.413320in}}%
\pgfpathlineto{\pgfqpoint{4.541711in}{0.413320in}}%
\pgfpathlineto{\pgfqpoint{4.539144in}{0.413320in}}%
\pgfpathlineto{\pgfqpoint{4.536400in}{0.413320in}}%
\pgfpathlineto{\pgfqpoint{4.533764in}{0.413320in}}%
\pgfpathlineto{\pgfqpoint{4.530990in}{0.413320in}}%
\pgfpathlineto{\pgfqpoint{4.528307in}{0.413320in}}%
\pgfpathlineto{\pgfqpoint{4.525640in}{0.413320in}}%
\pgfpathlineto{\pgfqpoint{4.522962in}{0.413320in}}%
\pgfpathlineto{\pgfqpoint{4.520345in}{0.413320in}}%
\pgfpathlineto{\pgfqpoint{4.517598in}{0.413320in}}%
\pgfpathlineto{\pgfqpoint{4.515080in}{0.413320in}}%
\pgfpathlineto{\pgfqpoint{4.512246in}{0.413320in}}%
\pgfpathlineto{\pgfqpoint{4.509643in}{0.413320in}}%
\pgfpathlineto{\pgfqpoint{4.506893in}{0.413320in}}%
\pgfpathlineto{\pgfqpoint{4.504305in}{0.413320in}}%
\pgfpathlineto{\pgfqpoint{4.501529in}{0.413320in}}%
\pgfpathlineto{\pgfqpoint{4.498850in}{0.413320in}}%
\pgfpathlineto{\pgfqpoint{4.496167in}{0.413320in}}%
\pgfpathlineto{\pgfqpoint{4.493492in}{0.413320in}}%
\pgfpathlineto{\pgfqpoint{4.490822in}{0.413320in}}%
\pgfpathlineto{\pgfqpoint{4.488130in}{0.413320in}}%
\pgfpathlineto{\pgfqpoint{4.485581in}{0.413320in}}%
\pgfpathlineto{\pgfqpoint{4.482778in}{0.413320in}}%
\pgfpathlineto{\pgfqpoint{4.480201in}{0.413320in}}%
\pgfpathlineto{\pgfqpoint{4.477430in}{0.413320in}}%
\pgfpathlineto{\pgfqpoint{4.474861in}{0.413320in}}%
\pgfpathlineto{\pgfqpoint{4.472059in}{0.413320in}}%
\pgfpathlineto{\pgfqpoint{4.469492in}{0.413320in}}%
\pgfpathlineto{\pgfqpoint{4.466717in}{0.413320in}}%
\pgfpathlineto{\pgfqpoint{4.464029in}{0.413320in}}%
\pgfpathlineto{\pgfqpoint{4.461367in}{0.413320in}}%
\pgfpathlineto{\pgfqpoint{4.458681in}{0.413320in}}%
\pgfpathlineto{\pgfqpoint{4.456138in}{0.413320in}}%
\pgfpathlineto{\pgfqpoint{4.453312in}{0.413320in}}%
\pgfpathlineto{\pgfqpoint{4.450767in}{0.413320in}}%
\pgfpathlineto{\pgfqpoint{4.447965in}{0.413320in}}%
\pgfpathlineto{\pgfqpoint{4.445423in}{0.413320in}}%
\pgfpathlineto{\pgfqpoint{4.442611in}{0.413320in}}%
\pgfpathlineto{\pgfqpoint{4.440041in}{0.413320in}}%
\pgfpathlineto{\pgfqpoint{4.437253in}{0.413320in}}%
\pgfpathlineto{\pgfqpoint{4.434569in}{0.413320in}}%
\pgfpathlineto{\pgfqpoint{4.431901in}{0.413320in}}%
\pgfpathlineto{\pgfqpoint{4.429220in}{0.413320in}}%
\pgfpathlineto{\pgfqpoint{4.426534in}{0.413320in}}%
\pgfpathlineto{\pgfqpoint{4.423863in}{0.413320in}}%
\pgfpathlineto{\pgfqpoint{4.421292in}{0.413320in}}%
\pgfpathlineto{\pgfqpoint{4.418506in}{0.413320in}}%
\pgfpathlineto{\pgfqpoint{4.415932in}{0.413320in}}%
\pgfpathlineto{\pgfqpoint{4.413149in}{0.413320in}}%
\pgfpathlineto{\pgfqpoint{4.410587in}{0.413320in}}%
\pgfpathlineto{\pgfqpoint{4.407788in}{0.413320in}}%
\pgfpathlineto{\pgfqpoint{4.405234in}{0.413320in}}%
\pgfpathlineto{\pgfqpoint{4.402468in}{0.413320in}}%
\pgfpathlineto{\pgfqpoint{4.399745in}{0.413320in}}%
\pgfpathlineto{\pgfqpoint{4.397076in}{0.413320in}}%
\pgfpathlineto{\pgfqpoint{4.394400in}{0.413320in}}%
\pgfpathlineto{\pgfqpoint{4.391721in}{0.413320in}}%
\pgfpathlineto{\pgfqpoint{4.389044in}{0.413320in}}%
\pgfpathlineto{\pgfqpoint{4.386431in}{0.413320in}}%
\pgfpathlineto{\pgfqpoint{4.383674in}{0.413320in}}%
\pgfpathlineto{\pgfqpoint{4.381097in}{0.413320in}}%
\pgfpathlineto{\pgfqpoint{4.378329in}{0.413320in}}%
\pgfpathlineto{\pgfqpoint{4.375761in}{0.413320in}}%
\pgfpathlineto{\pgfqpoint{4.372976in}{0.413320in}}%
\pgfpathlineto{\pgfqpoint{4.370437in}{0.413320in}}%
\pgfpathlineto{\pgfqpoint{4.367646in}{0.413320in}}%
\pgfpathlineto{\pgfqpoint{4.364936in}{0.413320in}}%
\pgfpathlineto{\pgfqpoint{4.362270in}{0.413320in}}%
\pgfpathlineto{\pgfqpoint{4.359582in}{0.413320in}}%
\pgfpathlineto{\pgfqpoint{4.357014in}{0.413320in}}%
\pgfpathlineto{\pgfqpoint{4.354224in}{0.413320in}}%
\pgfpathlineto{\pgfqpoint{4.351645in}{0.413320in}}%
\pgfpathlineto{\pgfqpoint{4.348868in}{0.413320in}}%
\pgfpathlineto{\pgfqpoint{4.346263in}{0.413320in}}%
\pgfpathlineto{\pgfqpoint{4.343510in}{0.413320in}}%
\pgfpathlineto{\pgfqpoint{4.340976in}{0.413320in}}%
\pgfpathlineto{\pgfqpoint{4.338154in}{0.413320in}}%
\pgfpathlineto{\pgfqpoint{4.335463in}{0.413320in}}%
\pgfpathlineto{\pgfqpoint{4.332796in}{0.413320in}}%
\pgfpathlineto{\pgfqpoint{4.330118in}{0.413320in}}%
\pgfpathlineto{\pgfqpoint{4.327440in}{0.413320in}}%
\pgfpathlineto{\pgfqpoint{4.324760in}{0.413320in}}%
\pgfpathlineto{\pgfqpoint{4.322181in}{0.413320in}}%
\pgfpathlineto{\pgfqpoint{4.319405in}{0.413320in}}%
\pgfpathlineto{\pgfqpoint{4.316856in}{0.413320in}}%
\pgfpathlineto{\pgfqpoint{4.314032in}{0.413320in}}%
\pgfpathlineto{\pgfqpoint{4.311494in}{0.413320in}}%
\pgfpathlineto{\pgfqpoint{4.308691in}{0.413320in}}%
\pgfpathlineto{\pgfqpoint{4.306118in}{0.413320in}}%
\pgfpathlineto{\pgfqpoint{4.303357in}{0.413320in}}%
\pgfpathlineto{\pgfqpoint{4.300656in}{0.413320in}}%
\pgfpathlineto{\pgfqpoint{4.297977in}{0.413320in}}%
\pgfpathlineto{\pgfqpoint{4.295299in}{0.413320in}}%
\pgfpathlineto{\pgfqpoint{4.292786in}{0.413320in}}%
\pgfpathlineto{\pgfqpoint{4.289936in}{0.413320in}}%
\pgfpathlineto{\pgfqpoint{4.287399in}{0.413320in}}%
\pgfpathlineto{\pgfqpoint{4.284586in}{0.413320in}}%
\pgfpathlineto{\pgfqpoint{4.282000in}{0.413320in}}%
\pgfpathlineto{\pgfqpoint{4.279212in}{0.413320in}}%
\pgfpathlineto{\pgfqpoint{4.276635in}{0.413320in}}%
\pgfpathlineto{\pgfqpoint{4.273874in}{0.413320in}}%
\pgfpathlineto{\pgfqpoint{4.271187in}{0.413320in}}%
\pgfpathlineto{\pgfqpoint{4.268590in}{0.413320in}}%
\pgfpathlineto{\pgfqpoint{4.265824in}{0.413320in}}%
\pgfpathlineto{\pgfqpoint{4.263157in}{0.413320in}}%
\pgfpathlineto{\pgfqpoint{4.260477in}{0.413320in}}%
\pgfpathlineto{\pgfqpoint{4.257958in}{0.413320in}}%
\pgfpathlineto{\pgfqpoint{4.255120in}{0.413320in}}%
\pgfpathlineto{\pgfqpoint{4.252581in}{0.413320in}}%
\pgfpathlineto{\pgfqpoint{4.249767in}{0.413320in}}%
\pgfpathlineto{\pgfqpoint{4.247225in}{0.413320in}}%
\pgfpathlineto{\pgfqpoint{4.244394in}{0.413320in}}%
\pgfpathlineto{\pgfqpoint{4.241900in}{0.413320in}}%
\pgfpathlineto{\pgfqpoint{4.239084in}{0.413320in}}%
\pgfpathlineto{\pgfqpoint{4.236375in}{0.413320in}}%
\pgfpathlineto{\pgfqpoint{4.233691in}{0.413320in}}%
\pgfpathlineto{\pgfqpoint{4.231013in}{0.413320in}}%
\pgfpathlineto{\pgfqpoint{4.228331in}{0.413320in}}%
\pgfpathlineto{\pgfqpoint{4.225654in}{0.413320in}}%
\pgfpathlineto{\pgfqpoint{4.223082in}{0.413320in}}%
\pgfpathlineto{\pgfqpoint{4.220304in}{0.413320in}}%
\pgfpathlineto{\pgfqpoint{4.217694in}{0.413320in}}%
\pgfpathlineto{\pgfqpoint{4.214948in}{0.413320in}}%
\pgfpathlineto{\pgfqpoint{4.212383in}{0.413320in}}%
\pgfpathlineto{\pgfqpoint{4.209597in}{0.413320in}}%
\pgfpathlineto{\pgfqpoint{4.207076in}{0.413320in}}%
\pgfpathlineto{\pgfqpoint{4.204240in}{0.413320in}}%
\pgfpathlineto{\pgfqpoint{4.201542in}{0.413320in}}%
\pgfpathlineto{\pgfqpoint{4.198878in}{0.413320in}}%
\pgfpathlineto{\pgfqpoint{4.196186in}{0.413320in}}%
\pgfpathlineto{\pgfqpoint{4.193638in}{0.413320in}}%
\pgfpathlineto{\pgfqpoint{4.190842in}{0.413320in}}%
\pgfpathlineto{\pgfqpoint{4.188318in}{0.413320in}}%
\pgfpathlineto{\pgfqpoint{4.185481in}{0.413320in}}%
\pgfpathlineto{\pgfqpoint{4.182899in}{0.413320in}}%
\pgfpathlineto{\pgfqpoint{4.180129in}{0.413320in}}%
\pgfpathlineto{\pgfqpoint{4.177593in}{0.413320in}}%
\pgfpathlineto{\pgfqpoint{4.174770in}{0.413320in}}%
\pgfpathlineto{\pgfqpoint{4.172093in}{0.413320in}}%
\pgfpathlineto{\pgfqpoint{4.169415in}{0.413320in}}%
\pgfpathlineto{\pgfqpoint{4.166737in}{0.413320in}}%
\pgfpathlineto{\pgfqpoint{4.164059in}{0.413320in}}%
\pgfpathlineto{\pgfqpoint{4.161380in}{0.413320in}}%
\pgfpathlineto{\pgfqpoint{4.158806in}{0.413320in}}%
\pgfpathlineto{\pgfqpoint{4.156016in}{0.413320in}}%
\pgfpathlineto{\pgfqpoint{4.153423in}{0.413320in}}%
\pgfpathlineto{\pgfqpoint{4.150665in}{0.413320in}}%
\pgfpathlineto{\pgfqpoint{4.148082in}{0.413320in}}%
\pgfpathlineto{\pgfqpoint{4.145310in}{0.413320in}}%
\pgfpathlineto{\pgfqpoint{4.142713in}{0.413320in}}%
\pgfpathlineto{\pgfqpoint{4.139963in}{0.413320in}}%
\pgfpathlineto{\pgfqpoint{4.137272in}{0.413320in}}%
\pgfpathlineto{\pgfqpoint{4.134615in}{0.413320in}}%
\pgfpathlineto{\pgfqpoint{4.131920in}{0.413320in}}%
\pgfpathlineto{\pgfqpoint{4.129349in}{0.413320in}}%
\pgfpathlineto{\pgfqpoint{4.126553in}{0.413320in}}%
\pgfpathlineto{\pgfqpoint{4.124019in}{0.413320in}}%
\pgfpathlineto{\pgfqpoint{4.121205in}{0.413320in}}%
\pgfpathlineto{\pgfqpoint{4.118554in}{0.413320in}}%
\pgfpathlineto{\pgfqpoint{4.115844in}{0.413320in}}%
\pgfpathlineto{\pgfqpoint{4.113252in}{0.413320in}}%
\pgfpathlineto{\pgfqpoint{4.110488in}{0.413320in}}%
\pgfpathlineto{\pgfqpoint{4.107814in}{0.413320in}}%
\pgfpathlineto{\pgfqpoint{4.105185in}{0.413320in}}%
\pgfpathlineto{\pgfqpoint{4.102456in}{0.413320in}}%
\pgfpathlineto{\pgfqpoint{4.099777in}{0.413320in}}%
\pgfpathlineto{\pgfqpoint{4.097092in}{0.413320in}}%
\pgfpathlineto{\pgfqpoint{4.094527in}{0.413320in}}%
\pgfpathlineto{\pgfqpoint{4.091729in}{0.413320in}}%
\pgfpathlineto{\pgfqpoint{4.089159in}{0.413320in}}%
\pgfpathlineto{\pgfqpoint{4.086385in}{0.413320in}}%
\pgfpathlineto{\pgfqpoint{4.083870in}{0.413320in}}%
\pgfpathlineto{\pgfqpoint{4.081018in}{0.413320in}}%
\pgfpathlineto{\pgfqpoint{4.078471in}{0.413320in}}%
\pgfpathlineto{\pgfqpoint{4.075705in}{0.413320in}}%
\pgfpathlineto{\pgfqpoint{4.072985in}{0.413320in}}%
\pgfpathlineto{\pgfqpoint{4.070313in}{0.413320in}}%
\pgfpathlineto{\pgfqpoint{4.067636in}{0.413320in}}%
\pgfpathlineto{\pgfqpoint{4.064957in}{0.413320in}}%
\pgfpathlineto{\pgfqpoint{4.062266in}{0.413320in}}%
\pgfpathlineto{\pgfqpoint{4.059702in}{0.413320in}}%
\pgfpathlineto{\pgfqpoint{4.056911in}{0.413320in}}%
\pgfpathlineto{\pgfqpoint{4.054326in}{0.413320in}}%
\pgfpathlineto{\pgfqpoint{4.051557in}{0.413320in}}%
\pgfpathlineto{\pgfqpoint{4.049006in}{0.413320in}}%
\pgfpathlineto{\pgfqpoint{4.046210in}{0.413320in}}%
\pgfpathlineto{\pgfqpoint{4.043667in}{0.413320in}}%
\pgfpathlineto{\pgfqpoint{4.040852in}{0.413320in}}%
\pgfpathlineto{\pgfqpoint{4.038174in}{0.413320in}}%
\pgfpathlineto{\pgfqpoint{4.035492in}{0.413320in}}%
\pgfpathlineto{\pgfqpoint{4.032817in}{0.413320in}}%
\pgfpathlineto{\pgfqpoint{4.030229in}{0.413320in}}%
\pgfpathlineto{\pgfqpoint{4.027447in}{0.413320in}}%
\pgfpathlineto{\pgfqpoint{4.024868in}{0.413320in}}%
\pgfpathlineto{\pgfqpoint{4.022097in}{0.413320in}}%
\pgfpathlineto{\pgfqpoint{4.019518in}{0.413320in}}%
\pgfpathlineto{\pgfqpoint{4.016744in}{0.413320in}}%
\pgfpathlineto{\pgfqpoint{4.014186in}{0.413320in}}%
\pgfpathlineto{\pgfqpoint{4.011394in}{0.413320in}}%
\pgfpathlineto{\pgfqpoint{4.008699in}{0.413320in}}%
\pgfpathlineto{\pgfqpoint{4.006034in}{0.413320in}}%
\pgfpathlineto{\pgfqpoint{4.003348in}{0.413320in}}%
\pgfpathlineto{\pgfqpoint{4.000674in}{0.413320in}}%
\pgfpathlineto{\pgfqpoint{3.997990in}{0.413320in}}%
\pgfpathlineto{\pgfqpoint{3.995417in}{0.413320in}}%
\pgfpathlineto{\pgfqpoint{3.992642in}{0.413320in}}%
\pgfpathlineto{\pgfqpoint{3.990055in}{0.413320in}}%
\pgfpathlineto{\pgfqpoint{3.987270in}{0.413320in}}%
\pgfpathlineto{\pgfqpoint{3.984714in}{0.413320in}}%
\pgfpathlineto{\pgfqpoint{3.981929in}{0.413320in}}%
\pgfpathlineto{\pgfqpoint{3.979389in}{0.413320in}}%
\pgfpathlineto{\pgfqpoint{3.976563in}{0.413320in}}%
\pgfpathlineto{\pgfqpoint{3.973885in}{0.413320in}}%
\pgfpathlineto{\pgfqpoint{3.971250in}{0.413320in}}%
\pgfpathlineto{\pgfqpoint{3.968523in}{0.413320in}}%
\pgfpathlineto{\pgfqpoint{3.966013in}{0.413320in}}%
\pgfpathlineto{\pgfqpoint{3.963176in}{0.413320in}}%
\pgfpathlineto{\pgfqpoint{3.960635in}{0.413320in}}%
\pgfpathlineto{\pgfqpoint{3.957823in}{0.413320in}}%
\pgfpathlineto{\pgfqpoint{3.955211in}{0.413320in}}%
\pgfpathlineto{\pgfqpoint{3.952464in}{0.413320in}}%
\pgfpathlineto{\pgfqpoint{3.949894in}{0.413320in}}%
\pgfpathlineto{\pgfqpoint{3.947101in}{0.413320in}}%
\pgfpathlineto{\pgfqpoint{3.944431in}{0.413320in}}%
\pgfpathlineto{\pgfqpoint{3.941778in}{0.413320in}}%
\pgfpathlineto{\pgfqpoint{3.939075in}{0.413320in}}%
\pgfpathlineto{\pgfqpoint{3.936395in}{0.413320in}}%
\pgfpathlineto{\pgfqpoint{3.933714in}{0.413320in}}%
\pgfpathlineto{\pgfqpoint{3.931202in}{0.413320in}}%
\pgfpathlineto{\pgfqpoint{3.928347in}{0.413320in}}%
\pgfpathlineto{\pgfqpoint{3.925778in}{0.413320in}}%
\pgfpathlineto{\pgfqpoint{3.923005in}{0.413320in}}%
\pgfpathlineto{\pgfqpoint{3.920412in}{0.413320in}}%
\pgfpathlineto{\pgfqpoint{3.917646in}{0.413320in}}%
\pgfpathlineto{\pgfqpoint{3.915107in}{0.413320in}}%
\pgfpathlineto{\pgfqpoint{3.912296in}{0.413320in}}%
\pgfpathlineto{\pgfqpoint{3.909602in}{0.413320in}}%
\pgfpathlineto{\pgfqpoint{3.906918in}{0.413320in}}%
\pgfpathlineto{\pgfqpoint{3.904252in}{0.413320in}}%
\pgfpathlineto{\pgfqpoint{3.901573in}{0.413320in}}%
\pgfpathlineto{\pgfqpoint{3.898891in}{0.413320in}}%
\pgfpathlineto{\pgfqpoint{3.896345in}{0.413320in}}%
\pgfpathlineto{\pgfqpoint{3.893541in}{0.413320in}}%
\pgfpathlineto{\pgfqpoint{3.890926in}{0.413320in}}%
\pgfpathlineto{\pgfqpoint{3.888188in}{0.413320in}}%
\pgfpathlineto{\pgfqpoint{3.885621in}{0.413320in}}%
\pgfpathlineto{\pgfqpoint{3.882850in}{0.413320in}}%
\pgfpathlineto{\pgfqpoint{3.880237in}{0.413320in}}%
\pgfpathlineto{\pgfqpoint{3.877466in}{0.413320in}}%
\pgfpathlineto{\pgfqpoint{3.874790in}{0.413320in}}%
\pgfpathlineto{\pgfqpoint{3.872114in}{0.413320in}}%
\pgfpathlineto{\pgfqpoint{3.869435in}{0.413320in}}%
\pgfpathlineto{\pgfqpoint{3.866815in}{0.413320in}}%
\pgfpathlineto{\pgfqpoint{3.864073in}{0.413320in}}%
\pgfpathlineto{\pgfqpoint{3.861561in}{0.413320in}}%
\pgfpathlineto{\pgfqpoint{3.858720in}{0.413320in}}%
\pgfpathlineto{\pgfqpoint{3.856100in}{0.413320in}}%
\pgfpathlineto{\pgfqpoint{3.853358in}{0.413320in}}%
\pgfpathlineto{\pgfqpoint{3.850814in}{0.413320in}}%
\pgfpathlineto{\pgfqpoint{3.848005in}{0.413320in}}%
\pgfpathlineto{\pgfqpoint{3.845329in}{0.413320in}}%
\pgfpathlineto{\pgfqpoint{3.842641in}{0.413320in}}%
\pgfpathlineto{\pgfqpoint{3.839960in}{0.413320in}}%
\pgfpathlineto{\pgfqpoint{3.837286in}{0.413320in}}%
\pgfpathlineto{\pgfqpoint{3.834616in}{0.413320in}}%
\pgfpathlineto{\pgfqpoint{3.832053in}{0.413320in}}%
\pgfpathlineto{\pgfqpoint{3.829252in}{0.413320in}}%
\pgfpathlineto{\pgfqpoint{3.826679in}{0.413320in}}%
\pgfpathlineto{\pgfqpoint{3.823903in}{0.413320in}}%
\pgfpathlineto{\pgfqpoint{3.821315in}{0.413320in}}%
\pgfpathlineto{\pgfqpoint{3.818546in}{0.413320in}}%
\pgfpathlineto{\pgfqpoint{3.815983in}{0.413320in}}%
\pgfpathlineto{\pgfqpoint{3.813172in}{0.413320in}}%
\pgfpathlineto{\pgfqpoint{3.810510in}{0.413320in}}%
\pgfpathlineto{\pgfqpoint{3.807832in}{0.413320in}}%
\pgfpathlineto{\pgfqpoint{3.805145in}{0.413320in}}%
\pgfpathlineto{\pgfqpoint{3.802569in}{0.413320in}}%
\pgfpathlineto{\pgfqpoint{3.799797in}{0.413320in}}%
\pgfpathlineto{\pgfqpoint{3.797265in}{0.413320in}}%
\pgfpathlineto{\pgfqpoint{3.794435in}{0.413320in}}%
\pgfpathlineto{\pgfqpoint{3.791897in}{0.413320in}}%
\pgfpathlineto{\pgfqpoint{3.789084in}{0.413320in}}%
\pgfpathlineto{\pgfqpoint{3.786504in}{0.413320in}}%
\pgfpathlineto{\pgfqpoint{3.783725in}{0.413320in}}%
\pgfpathlineto{\pgfqpoint{3.781046in}{0.413320in}}%
\pgfpathlineto{\pgfqpoint{3.778370in}{0.413320in}}%
\pgfpathlineto{\pgfqpoint{3.775691in}{0.413320in}}%
\pgfpathlineto{\pgfqpoint{3.773014in}{0.413320in}}%
\pgfpathlineto{\pgfqpoint{3.770323in}{0.413320in}}%
\pgfpathlineto{\pgfqpoint{3.767782in}{0.413320in}}%
\pgfpathlineto{\pgfqpoint{3.764966in}{0.413320in}}%
\pgfpathlineto{\pgfqpoint{3.762389in}{0.413320in}}%
\pgfpathlineto{\pgfqpoint{3.759622in}{0.413320in}}%
\pgfpathlineto{\pgfqpoint{3.757065in}{0.413320in}}%
\pgfpathlineto{\pgfqpoint{3.754265in}{0.413320in}}%
\pgfpathlineto{\pgfqpoint{3.751728in}{0.413320in}}%
\pgfpathlineto{\pgfqpoint{3.748903in}{0.413320in}}%
\pgfpathlineto{\pgfqpoint{3.746229in}{0.413320in}}%
\pgfpathlineto{\pgfqpoint{3.743548in}{0.413320in}}%
\pgfpathlineto{\pgfqpoint{3.740874in}{0.413320in}}%
\pgfpathlineto{\pgfqpoint{3.738194in}{0.413320in}}%
\pgfpathlineto{\pgfqpoint{3.735509in}{0.413320in}}%
\pgfpathlineto{\pgfqpoint{3.732950in}{0.413320in}}%
\pgfpathlineto{\pgfqpoint{3.730158in}{0.413320in}}%
\pgfpathlineto{\pgfqpoint{3.727581in}{0.413320in}}%
\pgfpathlineto{\pgfqpoint{3.724804in}{0.413320in}}%
\pgfpathlineto{\pgfqpoint{3.722228in}{0.413320in}}%
\pgfpathlineto{\pgfqpoint{3.719446in}{0.413320in}}%
\pgfpathlineto{\pgfqpoint{3.716875in}{0.413320in}}%
\pgfpathlineto{\pgfqpoint{3.714086in}{0.413320in}}%
\pgfpathlineto{\pgfqpoint{3.711410in}{0.413320in}}%
\pgfpathlineto{\pgfqpoint{3.708729in}{0.413320in}}%
\pgfpathlineto{\pgfqpoint{3.706053in}{0.413320in}}%
\pgfpathlineto{\pgfqpoint{3.703460in}{0.413320in}}%
\pgfpathlineto{\pgfqpoint{3.700684in}{0.413320in}}%
\pgfpathlineto{\pgfqpoint{3.698125in}{0.413320in}}%
\pgfpathlineto{\pgfqpoint{3.695331in}{0.413320in}}%
\pgfpathlineto{\pgfqpoint{3.692765in}{0.413320in}}%
\pgfpathlineto{\pgfqpoint{3.689983in}{0.413320in}}%
\pgfpathlineto{\pgfqpoint{3.687442in}{0.413320in}}%
\pgfpathlineto{\pgfqpoint{3.684620in}{0.413320in}}%
\pgfpathlineto{\pgfqpoint{3.681948in}{0.413320in}}%
\pgfpathlineto{\pgfqpoint{3.679273in}{0.413320in}}%
\pgfpathlineto{\pgfqpoint{3.676591in}{0.413320in}}%
\pgfpathlineto{\pgfqpoint{3.673911in}{0.413320in}}%
\pgfpathlineto{\pgfqpoint{3.671232in}{0.413320in}}%
\pgfpathlineto{\pgfqpoint{3.668665in}{0.413320in}}%
\pgfpathlineto{\pgfqpoint{3.665864in}{0.413320in}}%
\pgfpathlineto{\pgfqpoint{3.663276in}{0.413320in}}%
\pgfpathlineto{\pgfqpoint{3.660515in}{0.413320in}}%
\pgfpathlineto{\pgfqpoint{3.657917in}{0.413320in}}%
\pgfpathlineto{\pgfqpoint{3.655165in}{0.413320in}}%
\pgfpathlineto{\pgfqpoint{3.652628in}{0.413320in}}%
\pgfpathlineto{\pgfqpoint{3.649837in}{0.413320in}}%
\pgfpathlineto{\pgfqpoint{3.647130in}{0.413320in}}%
\pgfpathlineto{\pgfqpoint{3.644452in}{0.413320in}}%
\pgfpathlineto{\pgfqpoint{3.641773in}{0.413320in}}%
\pgfpathlineto{\pgfqpoint{3.639207in}{0.413320in}}%
\pgfpathlineto{\pgfqpoint{3.636413in}{0.413320in}}%
\pgfpathlineto{\pgfqpoint{3.633858in}{0.413320in}}%
\pgfpathlineto{\pgfqpoint{3.631058in}{0.413320in}}%
\pgfpathlineto{\pgfqpoint{3.628460in}{0.413320in}}%
\pgfpathlineto{\pgfqpoint{3.625689in}{0.413320in}}%
\pgfpathlineto{\pgfqpoint{3.623165in}{0.413320in}}%
\pgfpathlineto{\pgfqpoint{3.620345in}{0.413320in}}%
\pgfpathlineto{\pgfqpoint{3.617667in}{0.413320in}}%
\pgfpathlineto{\pgfqpoint{3.614982in}{0.413320in}}%
\pgfpathlineto{\pgfqpoint{3.612311in}{0.413320in}}%
\pgfpathlineto{\pgfqpoint{3.609632in}{0.413320in}}%
\pgfpathlineto{\pgfqpoint{3.606951in}{0.413320in}}%
\pgfpathlineto{\pgfqpoint{3.604387in}{0.413320in}}%
\pgfpathlineto{\pgfqpoint{3.601590in}{0.413320in}}%
\pgfpathlineto{\pgfqpoint{3.598998in}{0.413320in}}%
\pgfpathlineto{\pgfqpoint{3.596240in}{0.413320in}}%
\pgfpathlineto{\pgfqpoint{3.593620in}{0.413320in}}%
\pgfpathlineto{\pgfqpoint{3.590883in}{0.413320in}}%
\pgfpathlineto{\pgfqpoint{3.588258in}{0.413320in}}%
\pgfpathlineto{\pgfqpoint{3.585532in}{0.413320in}}%
\pgfpathlineto{\pgfqpoint{3.582851in}{0.413320in}}%
\pgfpathlineto{\pgfqpoint{3.580191in}{0.413320in}}%
\pgfpathlineto{\pgfqpoint{3.577487in}{0.413320in}}%
\pgfpathlineto{\pgfqpoint{3.574814in}{0.413320in}}%
\pgfpathlineto{\pgfqpoint{3.572126in}{0.413320in}}%
\pgfpathlineto{\pgfqpoint{3.569584in}{0.413320in}}%
\pgfpathlineto{\pgfqpoint{3.566774in}{0.413320in}}%
\pgfpathlineto{\pgfqpoint{3.564188in}{0.413320in}}%
\pgfpathlineto{\pgfqpoint{3.561420in}{0.413320in}}%
\pgfpathlineto{\pgfqpoint{3.558853in}{0.413320in}}%
\pgfpathlineto{\pgfqpoint{3.556061in}{0.413320in}}%
\pgfpathlineto{\pgfqpoint{3.553498in}{0.413320in}}%
\pgfpathlineto{\pgfqpoint{3.550713in}{0.413320in}}%
\pgfpathlineto{\pgfqpoint{3.548029in}{0.413320in}}%
\pgfpathlineto{\pgfqpoint{3.545349in}{0.413320in}}%
\pgfpathlineto{\pgfqpoint{3.542656in}{0.413320in}}%
\pgfpathlineto{\pgfqpoint{3.540093in}{0.413320in}}%
\pgfpathlineto{\pgfqpoint{3.537309in}{0.413320in}}%
\pgfpathlineto{\pgfqpoint{3.534783in}{0.413320in}}%
\pgfpathlineto{\pgfqpoint{3.531955in}{0.413320in}}%
\pgfpathlineto{\pgfqpoint{3.529327in}{0.413320in}}%
\pgfpathlineto{\pgfqpoint{3.526601in}{0.413320in}}%
\pgfpathlineto{\pgfqpoint{3.524041in}{0.413320in}}%
\pgfpathlineto{\pgfqpoint{3.521244in}{0.413320in}}%
\pgfpathlineto{\pgfqpoint{3.518565in}{0.413320in}}%
\pgfpathlineto{\pgfqpoint{3.515884in}{0.413320in}}%
\pgfpathlineto{\pgfqpoint{3.513209in}{0.413320in}}%
\pgfpathlineto{\pgfqpoint{3.510533in}{0.413320in}}%
\pgfpathlineto{\pgfqpoint{3.507840in}{0.413320in}}%
\pgfpathlineto{\pgfqpoint{3.505262in}{0.413320in}}%
\pgfpathlineto{\pgfqpoint{3.502488in}{0.413320in}}%
\pgfpathlineto{\pgfqpoint{3.499909in}{0.413320in}}%
\pgfpathlineto{\pgfqpoint{3.497139in}{0.413320in}}%
\pgfpathlineto{\pgfqpoint{3.494581in}{0.413320in}}%
\pgfpathlineto{\pgfqpoint{3.491783in}{0.413320in}}%
\pgfpathlineto{\pgfqpoint{3.489223in}{0.413320in}}%
\pgfpathlineto{\pgfqpoint{3.486442in}{0.413320in}}%
\pgfpathlineto{\pgfqpoint{3.483744in}{0.413320in}}%
\pgfpathlineto{\pgfqpoint{3.481072in}{0.413320in}}%
\pgfpathlineto{\pgfqpoint{3.478378in}{0.413320in}}%
\pgfpathlineto{\pgfqpoint{3.475821in}{0.413320in}}%
\pgfpathlineto{\pgfqpoint{3.473021in}{0.413320in}}%
\pgfpathlineto{\pgfqpoint{3.470466in}{0.413320in}}%
\pgfpathlineto{\pgfqpoint{3.467678in}{0.413320in}}%
\pgfpathlineto{\pgfqpoint{3.465072in}{0.413320in}}%
\pgfpathlineto{\pgfqpoint{3.462321in}{0.413320in}}%
\pgfpathlineto{\pgfqpoint{3.459695in}{0.413320in}}%
\pgfpathlineto{\pgfqpoint{3.456960in}{0.413320in}}%
\pgfpathlineto{\pgfqpoint{3.454285in}{0.413320in}}%
\pgfpathlineto{\pgfqpoint{3.451597in}{0.413320in}}%
\pgfpathlineto{\pgfqpoint{3.448926in}{0.413320in}}%
\pgfpathlineto{\pgfqpoint{3.446257in}{0.413320in}}%
\pgfpathlineto{\pgfqpoint{3.443574in}{0.413320in}}%
\pgfpathlineto{\pgfqpoint{3.440996in}{0.413320in}}%
\pgfpathlineto{\pgfqpoint{3.438210in}{0.413320in}}%
\pgfpathlineto{\pgfqpoint{3.435635in}{0.413320in}}%
\pgfpathlineto{\pgfqpoint{3.432851in}{0.413320in}}%
\pgfpathlineto{\pgfqpoint{3.430313in}{0.413320in}}%
\pgfpathlineto{\pgfqpoint{3.427501in}{0.413320in}}%
\pgfpathlineto{\pgfqpoint{3.424887in}{0.413320in}}%
\pgfpathlineto{\pgfqpoint{3.422142in}{0.413320in}}%
\pgfpathlineto{\pgfqpoint{3.419455in}{0.413320in}}%
\pgfpathlineto{\pgfqpoint{3.416780in}{0.413320in}}%
\pgfpathlineto{\pgfqpoint{3.414109in}{0.413320in}}%
\pgfpathlineto{\pgfqpoint{3.411431in}{0.413320in}}%
\pgfpathlineto{\pgfqpoint{3.408752in}{0.413320in}}%
\pgfpathlineto{\pgfqpoint{3.406202in}{0.413320in}}%
\pgfpathlineto{\pgfqpoint{3.403394in}{0.413320in}}%
\pgfpathlineto{\pgfqpoint{3.400783in}{0.413320in}}%
\pgfpathlineto{\pgfqpoint{3.398037in}{0.413320in}}%
\pgfpathlineto{\pgfqpoint{3.395461in}{0.413320in}}%
\pgfpathlineto{\pgfqpoint{3.392681in}{0.413320in}}%
\pgfpathlineto{\pgfqpoint{3.390102in}{0.413320in}}%
\pgfpathlineto{\pgfqpoint{3.387309in}{0.413320in}}%
\pgfpathlineto{\pgfqpoint{3.384647in}{0.413320in}}%
\pgfpathlineto{\pgfqpoint{3.381959in}{0.413320in}}%
\pgfpathlineto{\pgfqpoint{3.379290in}{0.413320in}}%
\pgfpathlineto{\pgfqpoint{3.376735in}{0.413320in}}%
\pgfpathlineto{\pgfqpoint{3.373921in}{0.413320in}}%
\pgfpathlineto{\pgfqpoint{3.371357in}{0.413320in}}%
\pgfpathlineto{\pgfqpoint{3.368577in}{0.413320in}}%
\pgfpathlineto{\pgfqpoint{3.365996in}{0.413320in}}%
\pgfpathlineto{\pgfqpoint{3.363221in}{0.413320in}}%
\pgfpathlineto{\pgfqpoint{3.360620in}{0.413320in}}%
\pgfpathlineto{\pgfqpoint{3.357862in}{0.413320in}}%
\pgfpathlineto{\pgfqpoint{3.355177in}{0.413320in}}%
\pgfpathlineto{\pgfqpoint{3.352505in}{0.413320in}}%
\pgfpathlineto{\pgfqpoint{3.349828in}{0.413320in}}%
\pgfpathlineto{\pgfqpoint{3.347139in}{0.413320in}}%
\pgfpathlineto{\pgfqpoint{3.344468in}{0.413320in}}%
\pgfpathlineto{\pgfqpoint{3.341893in}{0.413320in}}%
\pgfpathlineto{\pgfqpoint{3.339101in}{0.413320in}}%
\pgfpathlineto{\pgfqpoint{3.336541in}{0.413320in}}%
\pgfpathlineto{\pgfqpoint{3.333758in}{0.413320in}}%
\pgfpathlineto{\pgfqpoint{3.331183in}{0.413320in}}%
\pgfpathlineto{\pgfqpoint{3.328401in}{0.413320in}}%
\pgfpathlineto{\pgfqpoint{3.325860in}{0.413320in}}%
\pgfpathlineto{\pgfqpoint{3.323049in}{0.413320in}}%
\pgfpathlineto{\pgfqpoint{3.320366in}{0.413320in}}%
\pgfpathlineto{\pgfqpoint{3.317688in}{0.413320in}}%
\pgfpathlineto{\pgfqpoint{3.315008in}{0.413320in}}%
\pgfpathlineto{\pgfqpoint{3.312480in}{0.413320in}}%
\pgfpathlineto{\pgfqpoint{3.309652in}{0.413320in}}%
\pgfpathlineto{\pgfqpoint{3.307104in}{0.413320in}}%
\pgfpathlineto{\pgfqpoint{3.304295in}{0.413320in}}%
\pgfpathlineto{\pgfqpoint{3.301719in}{0.413320in}}%
\pgfpathlineto{\pgfqpoint{3.298937in}{0.413320in}}%
\pgfpathlineto{\pgfqpoint{3.296376in}{0.413320in}}%
\pgfpathlineto{\pgfqpoint{3.293574in}{0.413320in}}%
\pgfpathlineto{\pgfqpoint{3.290890in}{0.413320in}}%
\pgfpathlineto{\pgfqpoint{3.288225in}{0.413320in}}%
\pgfpathlineto{\pgfqpoint{3.285534in}{0.413320in}}%
\pgfpathlineto{\pgfqpoint{3.282870in}{0.413320in}}%
\pgfpathlineto{\pgfqpoint{3.280189in}{0.413320in}}%
\pgfpathlineto{\pgfqpoint{3.277603in}{0.413320in}}%
\pgfpathlineto{\pgfqpoint{3.274831in}{0.413320in}}%
\pgfpathlineto{\pgfqpoint{3.272254in}{0.413320in}}%
\pgfpathlineto{\pgfqpoint{3.269478in}{0.413320in}}%
\pgfpathlineto{\pgfqpoint{3.266849in}{0.413320in}}%
\pgfpathlineto{\pgfqpoint{3.264119in}{0.413320in}}%
\pgfpathlineto{\pgfqpoint{3.261594in}{0.413320in}}%
\pgfpathlineto{\pgfqpoint{3.258784in}{0.413320in}}%
\pgfpathlineto{\pgfqpoint{3.256083in}{0.413320in}}%
\pgfpathlineto{\pgfqpoint{3.253404in}{0.413320in}}%
\pgfpathlineto{\pgfqpoint{3.250716in}{0.413320in}}%
\pgfpathlineto{\pgfqpoint{3.248049in}{0.413320in}}%
\pgfpathlineto{\pgfqpoint{3.245363in}{0.413320in}}%
\pgfpathlineto{\pgfqpoint{3.242807in}{0.413320in}}%
\pgfpathlineto{\pgfqpoint{3.240010in}{0.413320in}}%
\pgfpathlineto{\pgfqpoint{3.237411in}{0.413320in}}%
\pgfpathlineto{\pgfqpoint{3.234658in}{0.413320in}}%
\pgfpathlineto{\pgfqpoint{3.232069in}{0.413320in}}%
\pgfpathlineto{\pgfqpoint{3.229310in}{0.413320in}}%
\pgfpathlineto{\pgfqpoint{3.226609in}{0.413320in}}%
\pgfpathlineto{\pgfqpoint{3.223942in}{0.413320in}}%
\pgfpathlineto{\pgfqpoint{3.221255in}{0.413320in}}%
\pgfpathlineto{\pgfqpoint{3.218586in}{0.413320in}}%
\pgfpathlineto{\pgfqpoint{3.215908in}{0.413320in}}%
\pgfpathlineto{\pgfqpoint{3.213342in}{0.413320in}}%
\pgfpathlineto{\pgfqpoint{3.210545in}{0.413320in}}%
\pgfpathlineto{\pgfqpoint{3.207984in}{0.413320in}}%
\pgfpathlineto{\pgfqpoint{3.205195in}{0.413320in}}%
\pgfpathlineto{\pgfqpoint{3.202562in}{0.413320in}}%
\pgfpathlineto{\pgfqpoint{3.199823in}{0.413320in}}%
\pgfpathlineto{\pgfqpoint{3.197226in}{0.413320in}}%
\pgfpathlineto{\pgfqpoint{3.194508in}{0.413320in}}%
\pgfpathlineto{\pgfqpoint{3.191796in}{0.413320in}}%
\pgfpathlineto{\pgfqpoint{3.189117in}{0.413320in}}%
\pgfpathlineto{\pgfqpoint{3.186440in}{0.413320in}}%
\pgfpathlineto{\pgfqpoint{3.183760in}{0.413320in}}%
\pgfpathlineto{\pgfqpoint{3.181089in}{0.413320in}}%
\pgfpathlineto{\pgfqpoint{3.178525in}{0.413320in}}%
\pgfpathlineto{\pgfqpoint{3.175724in}{0.413320in}}%
\pgfpathlineto{\pgfqpoint{3.173142in}{0.413320in}}%
\pgfpathlineto{\pgfqpoint{3.170375in}{0.413320in}}%
\pgfpathlineto{\pgfqpoint{3.167776in}{0.413320in}}%
\pgfpathlineto{\pgfqpoint{3.165019in}{0.413320in}}%
\pgfpathlineto{\pgfqpoint{3.162474in}{0.413320in}}%
\pgfpathlineto{\pgfqpoint{3.159675in}{0.413320in}}%
\pgfpathlineto{\pgfqpoint{3.156981in}{0.413320in}}%
\pgfpathlineto{\pgfqpoint{3.154327in}{0.413320in}}%
\pgfpathlineto{\pgfqpoint{3.151612in}{0.413320in}}%
\pgfpathlineto{\pgfqpoint{3.149057in}{0.413320in}}%
\pgfpathlineto{\pgfqpoint{3.146271in}{0.413320in}}%
\pgfpathlineto{\pgfqpoint{3.143740in}{0.413320in}}%
\pgfpathlineto{\pgfqpoint{3.140913in}{0.413320in}}%
\pgfpathlineto{\pgfqpoint{3.138375in}{0.413320in}}%
\pgfpathlineto{\pgfqpoint{3.135550in}{0.413320in}}%
\pgfpathlineto{\pgfqpoint{3.132946in}{0.413320in}}%
\pgfpathlineto{\pgfqpoint{3.130199in}{0.413320in}}%
\pgfpathlineto{\pgfqpoint{3.127512in}{0.413320in}}%
\pgfpathlineto{\pgfqpoint{3.124842in}{0.413320in}}%
\pgfpathlineto{\pgfqpoint{3.122163in}{0.413320in}}%
\pgfpathlineto{\pgfqpoint{3.119487in}{0.413320in}}%
\pgfpathlineto{\pgfqpoint{3.116807in}{0.413320in}}%
\pgfpathlineto{\pgfqpoint{3.114242in}{0.413320in}}%
\pgfpathlineto{\pgfqpoint{3.111451in}{0.413320in}}%
\pgfpathlineto{\pgfqpoint{3.108896in}{0.413320in}}%
\pgfpathlineto{\pgfqpoint{3.106094in}{0.413320in}}%
\pgfpathlineto{\pgfqpoint{3.103508in}{0.413320in}}%
\pgfpathlineto{\pgfqpoint{3.100737in}{0.413320in}}%
\pgfpathlineto{\pgfqpoint{3.098163in}{0.413320in}}%
\pgfpathlineto{\pgfqpoint{3.095388in}{0.413320in}}%
\pgfpathlineto{\pgfqpoint{3.092699in}{0.413320in}}%
\pgfpathlineto{\pgfqpoint{3.090023in}{0.413320in}}%
\pgfpathlineto{\pgfqpoint{3.087343in}{0.413320in}}%
\pgfpathlineto{\pgfqpoint{3.084671in}{0.413320in}}%
\pgfpathlineto{\pgfqpoint{3.081990in}{0.413320in}}%
\pgfpathlineto{\pgfqpoint{3.079381in}{0.413320in}}%
\pgfpathlineto{\pgfqpoint{3.076631in}{0.413320in}}%
\pgfpathlineto{\pgfqpoint{3.074056in}{0.413320in}}%
\pgfpathlineto{\pgfqpoint{3.071266in}{0.413320in}}%
\pgfpathlineto{\pgfqpoint{3.068709in}{0.413320in}}%
\pgfpathlineto{\pgfqpoint{3.065916in}{0.413320in}}%
\pgfpathlineto{\pgfqpoint{3.063230in}{0.413320in}}%
\pgfpathlineto{\pgfqpoint{3.060561in}{0.413320in}}%
\pgfpathlineto{\pgfqpoint{3.057884in}{0.413320in}}%
\pgfpathlineto{\pgfqpoint{3.055202in}{0.413320in}}%
\pgfpathlineto{\pgfqpoint{3.052526in}{0.413320in}}%
\pgfpathlineto{\pgfqpoint{3.049988in}{0.413320in}}%
\pgfpathlineto{\pgfqpoint{3.047157in}{0.413320in}}%
\pgfpathlineto{\pgfqpoint{3.044568in}{0.413320in}}%
\pgfpathlineto{\pgfqpoint{3.041813in}{0.413320in}}%
\pgfpathlineto{\pgfqpoint{3.039262in}{0.413320in}}%
\pgfpathlineto{\pgfqpoint{3.036456in}{0.413320in}}%
\pgfpathlineto{\pgfqpoint{3.033921in}{0.413320in}}%
\pgfpathlineto{\pgfqpoint{3.031091in}{0.413320in}}%
\pgfpathlineto{\pgfqpoint{3.028412in}{0.413320in}}%
\pgfpathlineto{\pgfqpoint{3.025803in}{0.413320in}}%
\pgfpathlineto{\pgfqpoint{3.023058in}{0.413320in}}%
\pgfpathlineto{\pgfqpoint{3.020382in}{0.413320in}}%
\pgfpathlineto{\pgfqpoint{3.017707in}{0.413320in}}%
\pgfpathlineto{\pgfqpoint{3.015097in}{0.413320in}}%
\pgfpathlineto{\pgfqpoint{3.012351in}{0.413320in}}%
\pgfpathlineto{\pgfqpoint{3.009784in}{0.413320in}}%
\pgfpathlineto{\pgfqpoint{3.006993in}{0.413320in}}%
\pgfpathlineto{\pgfqpoint{3.004419in}{0.413320in}}%
\pgfpathlineto{\pgfqpoint{3.001635in}{0.413320in}}%
\pgfpathlineto{\pgfqpoint{2.999103in}{0.413320in}}%
\pgfpathlineto{\pgfqpoint{2.996300in}{0.413320in}}%
\pgfpathlineto{\pgfqpoint{2.993595in}{0.413320in}}%
\pgfpathlineto{\pgfqpoint{2.990978in}{0.413320in}}%
\pgfpathlineto{\pgfqpoint{2.988238in}{0.413320in}}%
\pgfpathlineto{\pgfqpoint{2.985666in}{0.413320in}}%
\pgfpathlineto{\pgfqpoint{2.982885in}{0.413320in}}%
\pgfpathlineto{\pgfqpoint{2.980341in}{0.413320in}}%
\pgfpathlineto{\pgfqpoint{2.977517in}{0.413320in}}%
\pgfpathlineto{\pgfqpoint{2.974972in}{0.413320in}}%
\pgfpathlineto{\pgfqpoint{2.972177in}{0.413320in}}%
\pgfpathlineto{\pgfqpoint{2.969599in}{0.413320in}}%
\pgfpathlineto{\pgfqpoint{2.966812in}{0.413320in}}%
\pgfpathlineto{\pgfqpoint{2.964127in}{0.413320in}}%
\pgfpathlineto{\pgfqpoint{2.961460in}{0.413320in}}%
\pgfpathlineto{\pgfqpoint{2.958782in}{0.413320in}}%
\pgfpathlineto{\pgfqpoint{2.956103in}{0.413320in}}%
\pgfpathlineto{\pgfqpoint{2.953422in}{0.413320in}}%
\pgfpathlineto{\pgfqpoint{2.950884in}{0.413320in}}%
\pgfpathlineto{\pgfqpoint{2.948068in}{0.413320in}}%
\pgfpathlineto{\pgfqpoint{2.945461in}{0.413320in}}%
\pgfpathlineto{\pgfqpoint{2.942711in}{0.413320in}}%
\pgfpathlineto{\pgfqpoint{2.940120in}{0.413320in}}%
\pgfpathlineto{\pgfqpoint{2.937352in}{0.413320in}}%
\pgfpathlineto{\pgfqpoint{2.934759in}{0.413320in}}%
\pgfpathlineto{\pgfqpoint{2.932033in}{0.413320in}}%
\pgfpathlineto{\pgfqpoint{2.929321in}{0.413320in}}%
\pgfpathlineto{\pgfqpoint{2.926655in}{0.413320in}}%
\pgfpathlineto{\pgfqpoint{2.923963in}{0.413320in}}%
\pgfpathlineto{\pgfqpoint{2.921363in}{0.413320in}}%
\pgfpathlineto{\pgfqpoint{2.918606in}{0.413320in}}%
\pgfpathlineto{\pgfqpoint{2.916061in}{0.413320in}}%
\pgfpathlineto{\pgfqpoint{2.913243in}{0.413320in}}%
\pgfpathlineto{\pgfqpoint{2.910631in}{0.413320in}}%
\pgfpathlineto{\pgfqpoint{2.907882in}{0.413320in}}%
\pgfpathlineto{\pgfqpoint{2.905341in}{0.413320in}}%
\pgfpathlineto{\pgfqpoint{2.902535in}{0.413320in}}%
\pgfpathlineto{\pgfqpoint{2.899858in}{0.413320in}}%
\pgfpathlineto{\pgfqpoint{2.897179in}{0.413320in}}%
\pgfpathlineto{\pgfqpoint{2.894487in}{0.413320in}}%
\pgfpathlineto{\pgfqpoint{2.891809in}{0.413320in}}%
\pgfpathlineto{\pgfqpoint{2.889145in}{0.413320in}}%
\pgfpathlineto{\pgfqpoint{2.886578in}{0.413320in}}%
\pgfpathlineto{\pgfqpoint{2.883780in}{0.413320in}}%
\pgfpathlineto{\pgfqpoint{2.881254in}{0.413320in}}%
\pgfpathlineto{\pgfqpoint{2.878431in}{0.413320in}}%
\pgfpathlineto{\pgfqpoint{2.875882in}{0.413320in}}%
\pgfpathlineto{\pgfqpoint{2.873074in}{0.413320in}}%
\pgfpathlineto{\pgfqpoint{2.870475in}{0.413320in}}%
\pgfpathlineto{\pgfqpoint{2.867713in}{0.413320in}}%
\pgfpathlineto{\pgfqpoint{2.865031in}{0.413320in}}%
\pgfpathlineto{\pgfqpoint{2.862402in}{0.413320in}}%
\pgfpathlineto{\pgfqpoint{2.859668in}{0.413320in}}%
\pgfpathlineto{\pgfqpoint{2.857003in}{0.413320in}}%
\pgfpathlineto{\pgfqpoint{2.854325in}{0.413320in}}%
\pgfpathlineto{\pgfqpoint{2.851793in}{0.413320in}}%
\pgfpathlineto{\pgfqpoint{2.848960in}{0.413320in}}%
\pgfpathlineto{\pgfqpoint{2.846408in}{0.413320in}}%
\pgfpathlineto{\pgfqpoint{2.843611in}{0.413320in}}%
\pgfpathlineto{\pgfqpoint{2.841055in}{0.413320in}}%
\pgfpathlineto{\pgfqpoint{2.838254in}{0.413320in}}%
\pgfpathlineto{\pgfqpoint{2.835698in}{0.413320in}}%
\pgfpathlineto{\pgfqpoint{2.832894in}{0.413320in}}%
\pgfpathlineto{\pgfqpoint{2.830219in}{0.413320in}}%
\pgfpathlineto{\pgfqpoint{2.827567in}{0.413320in}}%
\pgfpathlineto{\pgfqpoint{2.824851in}{0.413320in}}%
\pgfpathlineto{\pgfqpoint{2.822303in}{0.413320in}}%
\pgfpathlineto{\pgfqpoint{2.819506in}{0.413320in}}%
\pgfpathlineto{\pgfqpoint{2.816867in}{0.413320in}}%
\pgfpathlineto{\pgfqpoint{2.814141in}{0.413320in}}%
\pgfpathlineto{\pgfqpoint{2.811597in}{0.413320in}}%
\pgfpathlineto{\pgfqpoint{2.808792in}{0.413320in}}%
\pgfpathlineto{\pgfqpoint{2.806175in}{0.413320in}}%
\pgfpathlineto{\pgfqpoint{2.803435in}{0.413320in}}%
\pgfpathlineto{\pgfqpoint{2.800756in}{0.413320in}}%
\pgfpathlineto{\pgfqpoint{2.798070in}{0.413320in}}%
\pgfpathlineto{\pgfqpoint{2.795398in}{0.413320in}}%
\pgfpathlineto{\pgfqpoint{2.792721in}{0.413320in}}%
\pgfpathlineto{\pgfqpoint{2.790044in}{0.413320in}}%
\pgfpathlineto{\pgfqpoint{2.787468in}{0.413320in}}%
\pgfpathlineto{\pgfqpoint{2.784687in}{0.413320in}}%
\pgfpathlineto{\pgfqpoint{2.782113in}{0.413320in}}%
\pgfpathlineto{\pgfqpoint{2.779330in}{0.413320in}}%
\pgfpathlineto{\pgfqpoint{2.776767in}{0.413320in}}%
\pgfpathlineto{\pgfqpoint{2.773972in}{0.413320in}}%
\pgfpathlineto{\pgfqpoint{2.771373in}{0.413320in}}%
\pgfpathlineto{\pgfqpoint{2.768617in}{0.413320in}}%
\pgfpathlineto{\pgfqpoint{2.765935in}{0.413320in}}%
\pgfpathlineto{\pgfqpoint{2.763253in}{0.413320in}}%
\pgfpathlineto{\pgfqpoint{2.760581in}{0.413320in}}%
\pgfpathlineto{\pgfqpoint{2.758028in}{0.413320in}}%
\pgfpathlineto{\pgfqpoint{2.755224in}{0.413320in}}%
\pgfpathlineto{\pgfqpoint{2.752614in}{0.413320in}}%
\pgfpathlineto{\pgfqpoint{2.749868in}{0.413320in}}%
\pgfpathlineto{\pgfqpoint{2.747260in}{0.413320in}}%
\pgfpathlineto{\pgfqpoint{2.744510in}{0.413320in}}%
\pgfpathlineto{\pgfqpoint{2.741928in}{0.413320in}}%
\pgfpathlineto{\pgfqpoint{2.739155in}{0.413320in}}%
\pgfpathlineto{\pgfqpoint{2.736476in}{0.413320in}}%
\pgfpathlineto{\pgfqpoint{2.733798in}{0.413320in}}%
\pgfpathlineto{\pgfqpoint{2.731119in}{0.413320in}}%
\pgfpathlineto{\pgfqpoint{2.728439in}{0.413320in}}%
\pgfpathlineto{\pgfqpoint{2.725760in}{0.413320in}}%
\pgfpathlineto{\pgfqpoint{2.723211in}{0.413320in}}%
\pgfpathlineto{\pgfqpoint{2.720404in}{0.413320in}}%
\pgfpathlineto{\pgfqpoint{2.717773in}{0.413320in}}%
\pgfpathlineto{\pgfqpoint{2.715036in}{0.413320in}}%
\pgfpathlineto{\pgfqpoint{2.712477in}{0.413320in}}%
\pgfpathlineto{\pgfqpoint{2.709683in}{0.413320in}}%
\pgfpathlineto{\pgfqpoint{2.707125in}{0.413320in}}%
\pgfpathlineto{\pgfqpoint{2.704326in}{0.413320in}}%
\pgfpathlineto{\pgfqpoint{2.701657in}{0.413320in}}%
\pgfpathlineto{\pgfqpoint{2.698968in}{0.413320in}}%
\pgfpathlineto{\pgfqpoint{2.696293in}{0.413320in}}%
\pgfpathlineto{\pgfqpoint{2.693611in}{0.413320in}}%
\pgfpathlineto{\pgfqpoint{2.690940in}{0.413320in}}%
\pgfpathlineto{\pgfqpoint{2.688328in}{0.413320in}}%
\pgfpathlineto{\pgfqpoint{2.685586in}{0.413320in}}%
\pgfpathlineto{\pgfqpoint{2.683009in}{0.413320in}}%
\pgfpathlineto{\pgfqpoint{2.680224in}{0.413320in}}%
\pgfpathlineto{\pgfqpoint{2.677650in}{0.413320in}}%
\pgfpathlineto{\pgfqpoint{2.674873in}{0.413320in}}%
\pgfpathlineto{\pgfqpoint{2.672301in}{0.413320in}}%
\pgfpathlineto{\pgfqpoint{2.669506in}{0.413320in}}%
\pgfpathlineto{\pgfqpoint{2.666836in}{0.413320in}}%
\pgfpathlineto{\pgfqpoint{2.664151in}{0.413320in}}%
\pgfpathlineto{\pgfqpoint{2.661481in}{0.413320in}}%
\pgfpathlineto{\pgfqpoint{2.658942in}{0.413320in}}%
\pgfpathlineto{\pgfqpoint{2.656124in}{0.413320in}}%
\pgfpathlineto{\pgfqpoint{2.653567in}{0.413320in}}%
\pgfpathlineto{\pgfqpoint{2.650767in}{0.413320in}}%
\pgfpathlineto{\pgfqpoint{2.648196in}{0.413320in}}%
\pgfpathlineto{\pgfqpoint{2.645408in}{0.413320in}}%
\pgfpathlineto{\pgfqpoint{2.642827in}{0.413320in}}%
\pgfpathlineto{\pgfqpoint{2.640053in}{0.413320in}}%
\pgfpathlineto{\pgfqpoint{2.637369in}{0.413320in}}%
\pgfpathlineto{\pgfqpoint{2.634700in}{0.413320in}}%
\pgfpathlineto{\pgfqpoint{2.632018in}{0.413320in}}%
\pgfpathlineto{\pgfqpoint{2.629340in}{0.413320in}}%
\pgfpathlineto{\pgfqpoint{2.626653in}{0.413320in}}%
\pgfpathlineto{\pgfqpoint{2.624077in}{0.413320in}}%
\pgfpathlineto{\pgfqpoint{2.621304in}{0.413320in}}%
\pgfpathlineto{\pgfqpoint{2.618773in}{0.413320in}}%
\pgfpathlineto{\pgfqpoint{2.615934in}{0.413320in}}%
\pgfpathlineto{\pgfqpoint{2.613393in}{0.413320in}}%
\pgfpathlineto{\pgfqpoint{2.610588in}{0.413320in}}%
\pgfpathlineto{\pgfqpoint{2.608004in}{0.413320in}}%
\pgfpathlineto{\pgfqpoint{2.605232in}{0.413320in}}%
\pgfpathlineto{\pgfqpoint{2.602557in}{0.413320in}}%
\pgfpathlineto{\pgfqpoint{2.599920in}{0.413320in}}%
\pgfpathlineto{\pgfqpoint{2.597196in}{0.413320in}}%
\pgfpathlineto{\pgfqpoint{2.594630in}{0.413320in}}%
\pgfpathlineto{\pgfqpoint{2.591842in}{0.413320in}}%
\pgfpathlineto{\pgfqpoint{2.589248in}{0.413320in}}%
\pgfpathlineto{\pgfqpoint{2.586484in}{0.413320in}}%
\pgfpathlineto{\pgfqpoint{2.583913in}{0.413320in}}%
\pgfpathlineto{\pgfqpoint{2.581129in}{0.413320in}}%
\pgfpathlineto{\pgfqpoint{2.578567in}{0.413320in}}%
\pgfpathlineto{\pgfqpoint{2.575779in}{0.413320in}}%
\pgfpathlineto{\pgfqpoint{2.573082in}{0.413320in}}%
\pgfpathlineto{\pgfqpoint{2.570411in}{0.413320in}}%
\pgfpathlineto{\pgfqpoint{2.567730in}{0.413320in}}%
\pgfpathlineto{\pgfqpoint{2.565045in}{0.413320in}}%
\pgfpathlineto{\pgfqpoint{2.562375in}{0.413320in}}%
\pgfpathlineto{\pgfqpoint{2.559790in}{0.413320in}}%
\pgfpathlineto{\pgfqpoint{2.557009in}{0.413320in}}%
\pgfpathlineto{\pgfqpoint{2.554493in}{0.413320in}}%
\pgfpathlineto{\pgfqpoint{2.551664in}{0.413320in}}%
\pgfpathlineto{\pgfqpoint{2.549114in}{0.413320in}}%
\pgfpathlineto{\pgfqpoint{2.546310in}{0.413320in}}%
\pgfpathlineto{\pgfqpoint{2.543765in}{0.413320in}}%
\pgfpathlineto{\pgfqpoint{2.540949in}{0.413320in}}%
\pgfpathlineto{\pgfqpoint{2.538274in}{0.413320in}}%
\pgfpathlineto{\pgfqpoint{2.535624in}{0.413320in}}%
\pgfpathlineto{\pgfqpoint{2.532917in}{0.413320in}}%
\pgfpathlineto{\pgfqpoint{2.530234in}{0.413320in}}%
\pgfpathlineto{\pgfqpoint{2.527560in}{0.413320in}}%
\pgfpathlineto{\pgfqpoint{2.524988in}{0.413320in}}%
\pgfpathlineto{\pgfqpoint{2.522197in}{0.413320in}}%
\pgfpathlineto{\pgfqpoint{2.519607in}{0.413320in}}%
\pgfpathlineto{\pgfqpoint{2.516845in}{0.413320in}}%
\pgfpathlineto{\pgfqpoint{2.514268in}{0.413320in}}%
\pgfpathlineto{\pgfqpoint{2.511478in}{0.413320in}}%
\pgfpathlineto{\pgfqpoint{2.508917in}{0.413320in}}%
\pgfpathlineto{\pgfqpoint{2.506163in}{0.413320in}}%
\pgfpathlineto{\pgfqpoint{2.503454in}{0.413320in}}%
\pgfpathlineto{\pgfqpoint{2.500801in}{0.413320in}}%
\pgfpathlineto{\pgfqpoint{2.498085in}{0.413320in}}%
\pgfpathlineto{\pgfqpoint{2.495542in}{0.413320in}}%
\pgfpathlineto{\pgfqpoint{2.492729in}{0.413320in}}%
\pgfpathlineto{\pgfqpoint{2.490183in}{0.413320in}}%
\pgfpathlineto{\pgfqpoint{2.487384in}{0.413320in}}%
\pgfpathlineto{\pgfqpoint{2.484870in}{0.413320in}}%
\pgfpathlineto{\pgfqpoint{2.482026in}{0.413320in}}%
\pgfpathlineto{\pgfqpoint{2.479420in}{0.413320in}}%
\pgfpathlineto{\pgfqpoint{2.476671in}{0.413320in}}%
\pgfpathlineto{\pgfqpoint{2.473989in}{0.413320in}}%
\pgfpathlineto{\pgfqpoint{2.471311in}{0.413320in}}%
\pgfpathlineto{\pgfqpoint{2.468635in}{0.413320in}}%
\pgfpathlineto{\pgfqpoint{2.465957in}{0.413320in}}%
\pgfpathlineto{\pgfqpoint{2.463280in}{0.413320in}}%
\pgfpathlineto{\pgfqpoint{2.460711in}{0.413320in}}%
\pgfpathlineto{\pgfqpoint{2.457917in}{0.413320in}}%
\pgfpathlineto{\pgfqpoint{2.455353in}{0.413320in}}%
\pgfpathlineto{\pgfqpoint{2.452562in}{0.413320in}}%
\pgfpathlineto{\pgfqpoint{2.450032in}{0.413320in}}%
\pgfpathlineto{\pgfqpoint{2.447209in}{0.413320in}}%
\pgfpathlineto{\pgfqpoint{2.444677in}{0.413320in}}%
\pgfpathlineto{\pgfqpoint{2.441876in}{0.413320in}}%
\pgfpathlineto{\pgfqpoint{2.439167in}{0.413320in}}%
\pgfpathlineto{\pgfqpoint{2.436518in}{0.413320in}}%
\pgfpathlineto{\pgfqpoint{2.433815in}{0.413320in}}%
\pgfpathlineto{\pgfqpoint{2.431251in}{0.413320in}}%
\pgfpathlineto{\pgfqpoint{2.428453in}{0.413320in}}%
\pgfpathlineto{\pgfqpoint{2.425878in}{0.413320in}}%
\pgfpathlineto{\pgfqpoint{2.423098in}{0.413320in}}%
\pgfpathlineto{\pgfqpoint{2.420528in}{0.413320in}}%
\pgfpathlineto{\pgfqpoint{2.417747in}{0.413320in}}%
\pgfpathlineto{\pgfqpoint{2.415184in}{0.413320in}}%
\pgfpathlineto{\pgfqpoint{2.412389in}{0.413320in}}%
\pgfpathlineto{\pgfqpoint{2.409699in}{0.413320in}}%
\pgfpathlineto{\pgfqpoint{2.407024in}{0.413320in}}%
\pgfpathlineto{\pgfqpoint{2.404352in}{0.413320in}}%
\pgfpathlineto{\pgfqpoint{2.401675in}{0.413320in}}%
\pgfpathlineto{\pgfqpoint{2.398995in}{0.413320in}}%
\pgfpathclose%
\pgfusepath{stroke,fill}%
\end{pgfscope}%
\begin{pgfscope}%
\pgfpathrectangle{\pgfqpoint{2.398995in}{0.319877in}}{\pgfqpoint{3.986877in}{1.993438in}} %
\pgfusepath{clip}%
\pgfsetbuttcap%
\pgfsetroundjoin%
\definecolor{currentfill}{rgb}{1.000000,1.000000,1.000000}%
\pgfsetfillcolor{currentfill}%
\pgfsetlinewidth{1.003750pt}%
\definecolor{currentstroke}{rgb}{0.382612,0.684057,0.192575}%
\pgfsetstrokecolor{currentstroke}%
\pgfsetdash{}{0pt}%
\pgfpathmoveto{\pgfqpoint{2.398995in}{0.413320in}}%
\pgfpathlineto{\pgfqpoint{2.398995in}{0.487993in}}%
\pgfpathlineto{\pgfqpoint{2.401675in}{0.482603in}}%
\pgfpathlineto{\pgfqpoint{2.404352in}{0.482223in}}%
\pgfpathlineto{\pgfqpoint{2.407024in}{0.480289in}}%
\pgfpathlineto{\pgfqpoint{2.409699in}{0.473117in}}%
\pgfpathlineto{\pgfqpoint{2.412389in}{0.481737in}}%
\pgfpathlineto{\pgfqpoint{2.415184in}{0.486294in}}%
\pgfpathlineto{\pgfqpoint{2.417747in}{0.482558in}}%
\pgfpathlineto{\pgfqpoint{2.420528in}{0.482850in}}%
\pgfpathlineto{\pgfqpoint{2.423098in}{0.486667in}}%
\pgfpathlineto{\pgfqpoint{2.425878in}{0.480379in}}%
\pgfpathlineto{\pgfqpoint{2.428453in}{0.482017in}}%
\pgfpathlineto{\pgfqpoint{2.431251in}{0.476956in}}%
\pgfpathlineto{\pgfqpoint{2.433815in}{0.473707in}}%
\pgfpathlineto{\pgfqpoint{2.436518in}{0.473926in}}%
\pgfpathlineto{\pgfqpoint{2.439167in}{0.482995in}}%
\pgfpathlineto{\pgfqpoint{2.441876in}{0.479737in}}%
\pgfpathlineto{\pgfqpoint{2.444677in}{0.480080in}}%
\pgfpathlineto{\pgfqpoint{2.447209in}{0.481980in}}%
\pgfpathlineto{\pgfqpoint{2.450032in}{0.487665in}}%
\pgfpathlineto{\pgfqpoint{2.452562in}{0.485551in}}%
\pgfpathlineto{\pgfqpoint{2.455353in}{0.478530in}}%
\pgfpathlineto{\pgfqpoint{2.457917in}{0.480532in}}%
\pgfpathlineto{\pgfqpoint{2.460711in}{0.477292in}}%
\pgfpathlineto{\pgfqpoint{2.463280in}{0.471539in}}%
\pgfpathlineto{\pgfqpoint{2.465957in}{0.476697in}}%
\pgfpathlineto{\pgfqpoint{2.468635in}{0.480618in}}%
\pgfpathlineto{\pgfqpoint{2.471311in}{0.483869in}}%
\pgfpathlineto{\pgfqpoint{2.473989in}{0.483316in}}%
\pgfpathlineto{\pgfqpoint{2.476671in}{0.475888in}}%
\pgfpathlineto{\pgfqpoint{2.479420in}{0.478424in}}%
\pgfpathlineto{\pgfqpoint{2.482026in}{0.473044in}}%
\pgfpathlineto{\pgfqpoint{2.484870in}{0.475929in}}%
\pgfpathlineto{\pgfqpoint{2.487384in}{0.475885in}}%
\pgfpathlineto{\pgfqpoint{2.490183in}{0.474087in}}%
\pgfpathlineto{\pgfqpoint{2.492729in}{0.480525in}}%
\pgfpathlineto{\pgfqpoint{2.495542in}{0.478198in}}%
\pgfpathlineto{\pgfqpoint{2.498085in}{0.481945in}}%
\pgfpathlineto{\pgfqpoint{2.500801in}{0.481310in}}%
\pgfpathlineto{\pgfqpoint{2.503454in}{0.480429in}}%
\pgfpathlineto{\pgfqpoint{2.506163in}{0.481172in}}%
\pgfpathlineto{\pgfqpoint{2.508917in}{0.481507in}}%
\pgfpathlineto{\pgfqpoint{2.511478in}{0.483331in}}%
\pgfpathlineto{\pgfqpoint{2.514268in}{0.477577in}}%
\pgfpathlineto{\pgfqpoint{2.516845in}{0.474689in}}%
\pgfpathlineto{\pgfqpoint{2.519607in}{0.473679in}}%
\pgfpathlineto{\pgfqpoint{2.522197in}{0.473679in}}%
\pgfpathlineto{\pgfqpoint{2.524988in}{0.465847in}}%
\pgfpathlineto{\pgfqpoint{2.527560in}{0.465847in}}%
\pgfpathlineto{\pgfqpoint{2.530234in}{0.469858in}}%
\pgfpathlineto{\pgfqpoint{2.532917in}{0.475238in}}%
\pgfpathlineto{\pgfqpoint{2.535624in}{0.467115in}}%
\pgfpathlineto{\pgfqpoint{2.538274in}{0.473522in}}%
\pgfpathlineto{\pgfqpoint{2.540949in}{0.469007in}}%
\pgfpathlineto{\pgfqpoint{2.543765in}{0.476527in}}%
\pgfpathlineto{\pgfqpoint{2.546310in}{0.484979in}}%
\pgfpathlineto{\pgfqpoint{2.549114in}{0.482091in}}%
\pgfpathlineto{\pgfqpoint{2.551664in}{0.479529in}}%
\pgfpathlineto{\pgfqpoint{2.554493in}{0.478369in}}%
\pgfpathlineto{\pgfqpoint{2.557009in}{0.479657in}}%
\pgfpathlineto{\pgfqpoint{2.559790in}{0.481258in}}%
\pgfpathlineto{\pgfqpoint{2.562375in}{0.480951in}}%
\pgfpathlineto{\pgfqpoint{2.565045in}{0.486274in}}%
\pgfpathlineto{\pgfqpoint{2.567730in}{0.482055in}}%
\pgfpathlineto{\pgfqpoint{2.570411in}{0.482583in}}%
\pgfpathlineto{\pgfqpoint{2.573082in}{0.497338in}}%
\pgfpathlineto{\pgfqpoint{2.575779in}{0.493994in}}%
\pgfpathlineto{\pgfqpoint{2.578567in}{0.487316in}}%
\pgfpathlineto{\pgfqpoint{2.581129in}{0.481202in}}%
\pgfpathlineto{\pgfqpoint{2.583913in}{0.482105in}}%
\pgfpathlineto{\pgfqpoint{2.586484in}{0.484481in}}%
\pgfpathlineto{\pgfqpoint{2.589248in}{0.480358in}}%
\pgfpathlineto{\pgfqpoint{2.591842in}{0.484076in}}%
\pgfpathlineto{\pgfqpoint{2.594630in}{0.486908in}}%
\pgfpathlineto{\pgfqpoint{2.597196in}{0.486051in}}%
\pgfpathlineto{\pgfqpoint{2.599920in}{0.491734in}}%
\pgfpathlineto{\pgfqpoint{2.602557in}{0.492927in}}%
\pgfpathlineto{\pgfqpoint{2.605232in}{0.493780in}}%
\pgfpathlineto{\pgfqpoint{2.608004in}{0.483093in}}%
\pgfpathlineto{\pgfqpoint{2.610588in}{0.487360in}}%
\pgfpathlineto{\pgfqpoint{2.613393in}{0.485167in}}%
\pgfpathlineto{\pgfqpoint{2.615934in}{0.503822in}}%
\pgfpathlineto{\pgfqpoint{2.618773in}{0.499377in}}%
\pgfpathlineto{\pgfqpoint{2.621304in}{0.491754in}}%
\pgfpathlineto{\pgfqpoint{2.624077in}{0.484586in}}%
\pgfpathlineto{\pgfqpoint{2.626653in}{0.484318in}}%
\pgfpathlineto{\pgfqpoint{2.629340in}{0.486000in}}%
\pgfpathlineto{\pgfqpoint{2.632018in}{0.482276in}}%
\pgfpathlineto{\pgfqpoint{2.634700in}{0.481444in}}%
\pgfpathlineto{\pgfqpoint{2.637369in}{0.477362in}}%
\pgfpathlineto{\pgfqpoint{2.640053in}{0.482589in}}%
\pgfpathlineto{\pgfqpoint{2.642827in}{0.484289in}}%
\pgfpathlineto{\pgfqpoint{2.645408in}{0.484859in}}%
\pgfpathlineto{\pgfqpoint{2.648196in}{0.478487in}}%
\pgfpathlineto{\pgfqpoint{2.650767in}{0.477413in}}%
\pgfpathlineto{\pgfqpoint{2.653567in}{0.475827in}}%
\pgfpathlineto{\pgfqpoint{2.656124in}{0.472366in}}%
\pgfpathlineto{\pgfqpoint{2.658942in}{0.471381in}}%
\pgfpathlineto{\pgfqpoint{2.661481in}{0.480812in}}%
\pgfpathlineto{\pgfqpoint{2.664151in}{0.481835in}}%
\pgfpathlineto{\pgfqpoint{2.666836in}{0.484075in}}%
\pgfpathlineto{\pgfqpoint{2.669506in}{0.484405in}}%
\pgfpathlineto{\pgfqpoint{2.672301in}{0.479856in}}%
\pgfpathlineto{\pgfqpoint{2.674873in}{0.474943in}}%
\pgfpathlineto{\pgfqpoint{2.677650in}{0.481469in}}%
\pgfpathlineto{\pgfqpoint{2.680224in}{0.476271in}}%
\pgfpathlineto{\pgfqpoint{2.683009in}{0.478986in}}%
\pgfpathlineto{\pgfqpoint{2.685586in}{0.484132in}}%
\pgfpathlineto{\pgfqpoint{2.688328in}{0.484260in}}%
\pgfpathlineto{\pgfqpoint{2.690940in}{0.487757in}}%
\pgfpathlineto{\pgfqpoint{2.693611in}{0.485121in}}%
\pgfpathlineto{\pgfqpoint{2.696293in}{0.486931in}}%
\pgfpathlineto{\pgfqpoint{2.698968in}{0.489431in}}%
\pgfpathlineto{\pgfqpoint{2.701657in}{0.487129in}}%
\pgfpathlineto{\pgfqpoint{2.704326in}{0.483526in}}%
\pgfpathlineto{\pgfqpoint{2.707125in}{0.494749in}}%
\pgfpathlineto{\pgfqpoint{2.709683in}{0.490061in}}%
\pgfpathlineto{\pgfqpoint{2.712477in}{0.493744in}}%
\pgfpathlineto{\pgfqpoint{2.715036in}{0.486128in}}%
\pgfpathlineto{\pgfqpoint{2.717773in}{0.490091in}}%
\pgfpathlineto{\pgfqpoint{2.720404in}{0.494615in}}%
\pgfpathlineto{\pgfqpoint{2.723211in}{0.506317in}}%
\pgfpathlineto{\pgfqpoint{2.725760in}{0.512434in}}%
\pgfpathlineto{\pgfqpoint{2.728439in}{0.507030in}}%
\pgfpathlineto{\pgfqpoint{2.731119in}{0.518481in}}%
\pgfpathlineto{\pgfqpoint{2.733798in}{0.509638in}}%
\pgfpathlineto{\pgfqpoint{2.736476in}{0.513057in}}%
\pgfpathlineto{\pgfqpoint{2.739155in}{0.498064in}}%
\pgfpathlineto{\pgfqpoint{2.741928in}{0.506094in}}%
\pgfpathlineto{\pgfqpoint{2.744510in}{0.508052in}}%
\pgfpathlineto{\pgfqpoint{2.747260in}{0.519537in}}%
\pgfpathlineto{\pgfqpoint{2.749868in}{0.516878in}}%
\pgfpathlineto{\pgfqpoint{2.752614in}{0.507648in}}%
\pgfpathlineto{\pgfqpoint{2.755224in}{0.515161in}}%
\pgfpathlineto{\pgfqpoint{2.758028in}{0.511000in}}%
\pgfpathlineto{\pgfqpoint{2.760581in}{0.505281in}}%
\pgfpathlineto{\pgfqpoint{2.763253in}{0.500046in}}%
\pgfpathlineto{\pgfqpoint{2.765935in}{0.502812in}}%
\pgfpathlineto{\pgfqpoint{2.768617in}{0.500493in}}%
\pgfpathlineto{\pgfqpoint{2.771373in}{0.494766in}}%
\pgfpathlineto{\pgfqpoint{2.773972in}{0.492641in}}%
\pgfpathlineto{\pgfqpoint{2.776767in}{0.487018in}}%
\pgfpathlineto{\pgfqpoint{2.779330in}{0.489345in}}%
\pgfpathlineto{\pgfqpoint{2.782113in}{0.492834in}}%
\pgfpathlineto{\pgfqpoint{2.784687in}{0.488467in}}%
\pgfpathlineto{\pgfqpoint{2.787468in}{0.484785in}}%
\pgfpathlineto{\pgfqpoint{2.790044in}{0.488489in}}%
\pgfpathlineto{\pgfqpoint{2.792721in}{0.492201in}}%
\pgfpathlineto{\pgfqpoint{2.795398in}{0.486243in}}%
\pgfpathlineto{\pgfqpoint{2.798070in}{0.491368in}}%
\pgfpathlineto{\pgfqpoint{2.800756in}{0.494206in}}%
\pgfpathlineto{\pgfqpoint{2.803435in}{0.490370in}}%
\pgfpathlineto{\pgfqpoint{2.806175in}{0.489743in}}%
\pgfpathlineto{\pgfqpoint{2.808792in}{0.492281in}}%
\pgfpathlineto{\pgfqpoint{2.811597in}{0.488353in}}%
\pgfpathlineto{\pgfqpoint{2.814141in}{0.500105in}}%
\pgfpathlineto{\pgfqpoint{2.816867in}{0.497502in}}%
\pgfpathlineto{\pgfqpoint{2.819506in}{0.488492in}}%
\pgfpathlineto{\pgfqpoint{2.822303in}{0.489261in}}%
\pgfpathlineto{\pgfqpoint{2.824851in}{0.485683in}}%
\pgfpathlineto{\pgfqpoint{2.827567in}{0.484120in}}%
\pgfpathlineto{\pgfqpoint{2.830219in}{0.490529in}}%
\pgfpathlineto{\pgfqpoint{2.832894in}{0.484624in}}%
\pgfpathlineto{\pgfqpoint{2.835698in}{0.482927in}}%
\pgfpathlineto{\pgfqpoint{2.838254in}{0.484411in}}%
\pgfpathlineto{\pgfqpoint{2.841055in}{0.481281in}}%
\pgfpathlineto{\pgfqpoint{2.843611in}{0.474308in}}%
\pgfpathlineto{\pgfqpoint{2.846408in}{0.481785in}}%
\pgfpathlineto{\pgfqpoint{2.848960in}{0.482865in}}%
\pgfpathlineto{\pgfqpoint{2.851793in}{0.481915in}}%
\pgfpathlineto{\pgfqpoint{2.854325in}{0.481346in}}%
\pgfpathlineto{\pgfqpoint{2.857003in}{0.482001in}}%
\pgfpathlineto{\pgfqpoint{2.859668in}{0.486353in}}%
\pgfpathlineto{\pgfqpoint{2.862402in}{0.484736in}}%
\pgfpathlineto{\pgfqpoint{2.865031in}{0.480313in}}%
\pgfpathlineto{\pgfqpoint{2.867713in}{0.482814in}}%
\pgfpathlineto{\pgfqpoint{2.870475in}{0.482749in}}%
\pgfpathlineto{\pgfqpoint{2.873074in}{0.481584in}}%
\pgfpathlineto{\pgfqpoint{2.875882in}{0.485272in}}%
\pgfpathlineto{\pgfqpoint{2.878431in}{0.483992in}}%
\pgfpathlineto{\pgfqpoint{2.881254in}{0.481609in}}%
\pgfpathlineto{\pgfqpoint{2.883780in}{0.477150in}}%
\pgfpathlineto{\pgfqpoint{2.886578in}{0.487350in}}%
\pgfpathlineto{\pgfqpoint{2.889145in}{0.477528in}}%
\pgfpathlineto{\pgfqpoint{2.891809in}{0.476299in}}%
\pgfpathlineto{\pgfqpoint{2.894487in}{0.482055in}}%
\pgfpathlineto{\pgfqpoint{2.897179in}{0.483559in}}%
\pgfpathlineto{\pgfqpoint{2.899858in}{0.481327in}}%
\pgfpathlineto{\pgfqpoint{2.902535in}{0.486596in}}%
\pgfpathlineto{\pgfqpoint{2.905341in}{0.483896in}}%
\pgfpathlineto{\pgfqpoint{2.907882in}{0.485547in}}%
\pgfpathlineto{\pgfqpoint{2.910631in}{0.488166in}}%
\pgfpathlineto{\pgfqpoint{2.913243in}{0.488678in}}%
\pgfpathlineto{\pgfqpoint{2.916061in}{0.481529in}}%
\pgfpathlineto{\pgfqpoint{2.918606in}{0.481618in}}%
\pgfpathlineto{\pgfqpoint{2.921363in}{0.482800in}}%
\pgfpathlineto{\pgfqpoint{2.923963in}{0.484706in}}%
\pgfpathlineto{\pgfqpoint{2.926655in}{0.488860in}}%
\pgfpathlineto{\pgfqpoint{2.929321in}{0.486894in}}%
\pgfpathlineto{\pgfqpoint{2.932033in}{0.487844in}}%
\pgfpathlineto{\pgfqpoint{2.934759in}{0.482272in}}%
\pgfpathlineto{\pgfqpoint{2.937352in}{0.476213in}}%
\pgfpathlineto{\pgfqpoint{2.940120in}{0.480499in}}%
\pgfpathlineto{\pgfqpoint{2.942711in}{0.479243in}}%
\pgfpathlineto{\pgfqpoint{2.945461in}{0.476463in}}%
\pgfpathlineto{\pgfqpoint{2.948068in}{0.482082in}}%
\pgfpathlineto{\pgfqpoint{2.950884in}{0.478288in}}%
\pgfpathlineto{\pgfqpoint{2.953422in}{0.482548in}}%
\pgfpathlineto{\pgfqpoint{2.956103in}{0.486913in}}%
\pgfpathlineto{\pgfqpoint{2.958782in}{0.485316in}}%
\pgfpathlineto{\pgfqpoint{2.961460in}{0.484692in}}%
\pgfpathlineto{\pgfqpoint{2.964127in}{0.487677in}}%
\pgfpathlineto{\pgfqpoint{2.966812in}{0.479865in}}%
\pgfpathlineto{\pgfqpoint{2.969599in}{0.480418in}}%
\pgfpathlineto{\pgfqpoint{2.972177in}{0.486234in}}%
\pgfpathlineto{\pgfqpoint{2.974972in}{0.489370in}}%
\pgfpathlineto{\pgfqpoint{2.977517in}{0.484907in}}%
\pgfpathlineto{\pgfqpoint{2.980341in}{0.483620in}}%
\pgfpathlineto{\pgfqpoint{2.982885in}{0.485057in}}%
\pgfpathlineto{\pgfqpoint{2.985666in}{0.483898in}}%
\pgfpathlineto{\pgfqpoint{2.988238in}{0.484890in}}%
\pgfpathlineto{\pgfqpoint{2.990978in}{0.486639in}}%
\pgfpathlineto{\pgfqpoint{2.993595in}{0.483993in}}%
\pgfpathlineto{\pgfqpoint{2.996300in}{0.485734in}}%
\pgfpathlineto{\pgfqpoint{2.999103in}{0.472687in}}%
\pgfpathlineto{\pgfqpoint{3.001635in}{0.481563in}}%
\pgfpathlineto{\pgfqpoint{3.004419in}{0.483073in}}%
\pgfpathlineto{\pgfqpoint{3.006993in}{0.477372in}}%
\pgfpathlineto{\pgfqpoint{3.009784in}{0.473690in}}%
\pgfpathlineto{\pgfqpoint{3.012351in}{0.472855in}}%
\pgfpathlineto{\pgfqpoint{3.015097in}{0.477031in}}%
\pgfpathlineto{\pgfqpoint{3.017707in}{0.476666in}}%
\pgfpathlineto{\pgfqpoint{3.020382in}{0.477017in}}%
\pgfpathlineto{\pgfqpoint{3.023058in}{0.478656in}}%
\pgfpathlineto{\pgfqpoint{3.025803in}{0.476852in}}%
\pgfpathlineto{\pgfqpoint{3.028412in}{0.480727in}}%
\pgfpathlineto{\pgfqpoint{3.031091in}{0.486324in}}%
\pgfpathlineto{\pgfqpoint{3.033921in}{0.485985in}}%
\pgfpathlineto{\pgfqpoint{3.036456in}{0.489976in}}%
\pgfpathlineto{\pgfqpoint{3.039262in}{0.490293in}}%
\pgfpathlineto{\pgfqpoint{3.041813in}{0.488063in}}%
\pgfpathlineto{\pgfqpoint{3.044568in}{0.486645in}}%
\pgfpathlineto{\pgfqpoint{3.047157in}{0.479989in}}%
\pgfpathlineto{\pgfqpoint{3.049988in}{0.478651in}}%
\pgfpathlineto{\pgfqpoint{3.052526in}{0.484933in}}%
\pgfpathlineto{\pgfqpoint{3.055202in}{0.483967in}}%
\pgfpathlineto{\pgfqpoint{3.057884in}{0.483358in}}%
\pgfpathlineto{\pgfqpoint{3.060561in}{0.489824in}}%
\pgfpathlineto{\pgfqpoint{3.063230in}{0.489864in}}%
\pgfpathlineto{\pgfqpoint{3.065916in}{0.489005in}}%
\pgfpathlineto{\pgfqpoint{3.068709in}{0.481022in}}%
\pgfpathlineto{\pgfqpoint{3.071266in}{0.484277in}}%
\pgfpathlineto{\pgfqpoint{3.074056in}{0.486658in}}%
\pgfpathlineto{\pgfqpoint{3.076631in}{0.490884in}}%
\pgfpathlineto{\pgfqpoint{3.079381in}{0.485223in}}%
\pgfpathlineto{\pgfqpoint{3.081990in}{0.489018in}}%
\pgfpathlineto{\pgfqpoint{3.084671in}{0.484692in}}%
\pgfpathlineto{\pgfqpoint{3.087343in}{0.483354in}}%
\pgfpathlineto{\pgfqpoint{3.090023in}{0.482739in}}%
\pgfpathlineto{\pgfqpoint{3.092699in}{0.478549in}}%
\pgfpathlineto{\pgfqpoint{3.095388in}{0.478731in}}%
\pgfpathlineto{\pgfqpoint{3.098163in}{0.475892in}}%
\pgfpathlineto{\pgfqpoint{3.100737in}{0.476057in}}%
\pgfpathlineto{\pgfqpoint{3.103508in}{0.475847in}}%
\pgfpathlineto{\pgfqpoint{3.106094in}{0.478221in}}%
\pgfpathlineto{\pgfqpoint{3.108896in}{0.486939in}}%
\pgfpathlineto{\pgfqpoint{3.111451in}{0.494543in}}%
\pgfpathlineto{\pgfqpoint{3.114242in}{0.484339in}}%
\pgfpathlineto{\pgfqpoint{3.116807in}{0.479255in}}%
\pgfpathlineto{\pgfqpoint{3.119487in}{0.475191in}}%
\pgfpathlineto{\pgfqpoint{3.122163in}{0.478417in}}%
\pgfpathlineto{\pgfqpoint{3.124842in}{0.488530in}}%
\pgfpathlineto{\pgfqpoint{3.127512in}{0.488047in}}%
\pgfpathlineto{\pgfqpoint{3.130199in}{0.485947in}}%
\pgfpathlineto{\pgfqpoint{3.132946in}{0.484595in}}%
\pgfpathlineto{\pgfqpoint{3.135550in}{0.496994in}}%
\pgfpathlineto{\pgfqpoint{3.138375in}{0.494487in}}%
\pgfpathlineto{\pgfqpoint{3.140913in}{0.487315in}}%
\pgfpathlineto{\pgfqpoint{3.143740in}{0.476597in}}%
\pgfpathlineto{\pgfqpoint{3.146271in}{0.481780in}}%
\pgfpathlineto{\pgfqpoint{3.149057in}{0.479130in}}%
\pgfpathlineto{\pgfqpoint{3.151612in}{0.481470in}}%
\pgfpathlineto{\pgfqpoint{3.154327in}{0.474994in}}%
\pgfpathlineto{\pgfqpoint{3.156981in}{0.479492in}}%
\pgfpathlineto{\pgfqpoint{3.159675in}{0.482979in}}%
\pgfpathlineto{\pgfqpoint{3.162474in}{0.478031in}}%
\pgfpathlineto{\pgfqpoint{3.165019in}{0.477070in}}%
\pgfpathlineto{\pgfqpoint{3.167776in}{0.473358in}}%
\pgfpathlineto{\pgfqpoint{3.170375in}{0.473985in}}%
\pgfpathlineto{\pgfqpoint{3.173142in}{0.473912in}}%
\pgfpathlineto{\pgfqpoint{3.175724in}{0.475139in}}%
\pgfpathlineto{\pgfqpoint{3.178525in}{0.482423in}}%
\pgfpathlineto{\pgfqpoint{3.181089in}{0.474931in}}%
\pgfpathlineto{\pgfqpoint{3.183760in}{0.472464in}}%
\pgfpathlineto{\pgfqpoint{3.186440in}{0.479089in}}%
\pgfpathlineto{\pgfqpoint{3.189117in}{0.472582in}}%
\pgfpathlineto{\pgfqpoint{3.191796in}{0.482058in}}%
\pgfpathlineto{\pgfqpoint{3.194508in}{0.476025in}}%
\pgfpathlineto{\pgfqpoint{3.197226in}{0.473643in}}%
\pgfpathlineto{\pgfqpoint{3.199823in}{0.473097in}}%
\pgfpathlineto{\pgfqpoint{3.202562in}{0.466345in}}%
\pgfpathlineto{\pgfqpoint{3.205195in}{0.478801in}}%
\pgfpathlineto{\pgfqpoint{3.207984in}{0.478805in}}%
\pgfpathlineto{\pgfqpoint{3.210545in}{0.474676in}}%
\pgfpathlineto{\pgfqpoint{3.213342in}{0.477289in}}%
\pgfpathlineto{\pgfqpoint{3.215908in}{0.480088in}}%
\pgfpathlineto{\pgfqpoint{3.218586in}{0.484621in}}%
\pgfpathlineto{\pgfqpoint{3.221255in}{0.481980in}}%
\pgfpathlineto{\pgfqpoint{3.223942in}{0.478736in}}%
\pgfpathlineto{\pgfqpoint{3.226609in}{0.475098in}}%
\pgfpathlineto{\pgfqpoint{3.229310in}{0.477599in}}%
\pgfpathlineto{\pgfqpoint{3.232069in}{0.469135in}}%
\pgfpathlineto{\pgfqpoint{3.234658in}{0.477659in}}%
\pgfpathlineto{\pgfqpoint{3.237411in}{0.476526in}}%
\pgfpathlineto{\pgfqpoint{3.240010in}{0.480073in}}%
\pgfpathlineto{\pgfqpoint{3.242807in}{0.479540in}}%
\pgfpathlineto{\pgfqpoint{3.245363in}{0.480714in}}%
\pgfpathlineto{\pgfqpoint{3.248049in}{0.481204in}}%
\pgfpathlineto{\pgfqpoint{3.250716in}{0.479387in}}%
\pgfpathlineto{\pgfqpoint{3.253404in}{0.485832in}}%
\pgfpathlineto{\pgfqpoint{3.256083in}{0.481440in}}%
\pgfpathlineto{\pgfqpoint{3.258784in}{0.480013in}}%
\pgfpathlineto{\pgfqpoint{3.261594in}{0.481375in}}%
\pgfpathlineto{\pgfqpoint{3.264119in}{0.475807in}}%
\pgfpathlineto{\pgfqpoint{3.266849in}{0.469288in}}%
\pgfpathlineto{\pgfqpoint{3.269478in}{0.478057in}}%
\pgfpathlineto{\pgfqpoint{3.272254in}{0.480005in}}%
\pgfpathlineto{\pgfqpoint{3.274831in}{0.476063in}}%
\pgfpathlineto{\pgfqpoint{3.277603in}{0.479753in}}%
\pgfpathlineto{\pgfqpoint{3.280189in}{0.478339in}}%
\pgfpathlineto{\pgfqpoint{3.282870in}{0.483302in}}%
\pgfpathlineto{\pgfqpoint{3.285534in}{0.481300in}}%
\pgfpathlineto{\pgfqpoint{3.288225in}{0.481747in}}%
\pgfpathlineto{\pgfqpoint{3.290890in}{0.484002in}}%
\pgfpathlineto{\pgfqpoint{3.293574in}{0.480150in}}%
\pgfpathlineto{\pgfqpoint{3.296376in}{0.485668in}}%
\pgfpathlineto{\pgfqpoint{3.298937in}{0.476728in}}%
\pgfpathlineto{\pgfqpoint{3.301719in}{0.480461in}}%
\pgfpathlineto{\pgfqpoint{3.304295in}{0.482821in}}%
\pgfpathlineto{\pgfqpoint{3.307104in}{0.484809in}}%
\pgfpathlineto{\pgfqpoint{3.309652in}{0.487162in}}%
\pgfpathlineto{\pgfqpoint{3.312480in}{0.485863in}}%
\pgfpathlineto{\pgfqpoint{3.315008in}{0.489156in}}%
\pgfpathlineto{\pgfqpoint{3.317688in}{0.485294in}}%
\pgfpathlineto{\pgfqpoint{3.320366in}{0.486627in}}%
\pgfpathlineto{\pgfqpoint{3.323049in}{0.481757in}}%
\pgfpathlineto{\pgfqpoint{3.325860in}{0.488923in}}%
\pgfpathlineto{\pgfqpoint{3.328401in}{0.481618in}}%
\pgfpathlineto{\pgfqpoint{3.331183in}{0.482598in}}%
\pgfpathlineto{\pgfqpoint{3.333758in}{0.487231in}}%
\pgfpathlineto{\pgfqpoint{3.336541in}{0.482175in}}%
\pgfpathlineto{\pgfqpoint{3.339101in}{0.483858in}}%
\pgfpathlineto{\pgfqpoint{3.341893in}{0.482793in}}%
\pgfpathlineto{\pgfqpoint{3.344468in}{0.481919in}}%
\pgfpathlineto{\pgfqpoint{3.347139in}{0.476399in}}%
\pgfpathlineto{\pgfqpoint{3.349828in}{0.479353in}}%
\pgfpathlineto{\pgfqpoint{3.352505in}{0.483395in}}%
\pgfpathlineto{\pgfqpoint{3.355177in}{0.487918in}}%
\pgfpathlineto{\pgfqpoint{3.357862in}{0.488374in}}%
\pgfpathlineto{\pgfqpoint{3.360620in}{0.480004in}}%
\pgfpathlineto{\pgfqpoint{3.363221in}{0.475275in}}%
\pgfpathlineto{\pgfqpoint{3.365996in}{0.486777in}}%
\pgfpathlineto{\pgfqpoint{3.368577in}{0.486196in}}%
\pgfpathlineto{\pgfqpoint{3.371357in}{0.477699in}}%
\pgfpathlineto{\pgfqpoint{3.373921in}{0.479519in}}%
\pgfpathlineto{\pgfqpoint{3.376735in}{0.478704in}}%
\pgfpathlineto{\pgfqpoint{3.379290in}{0.482308in}}%
\pgfpathlineto{\pgfqpoint{3.381959in}{0.481364in}}%
\pgfpathlineto{\pgfqpoint{3.384647in}{0.480651in}}%
\pgfpathlineto{\pgfqpoint{3.387309in}{0.478833in}}%
\pgfpathlineto{\pgfqpoint{3.390102in}{0.480880in}}%
\pgfpathlineto{\pgfqpoint{3.392681in}{0.473580in}}%
\pgfpathlineto{\pgfqpoint{3.395461in}{0.479628in}}%
\pgfpathlineto{\pgfqpoint{3.398037in}{0.481727in}}%
\pgfpathlineto{\pgfqpoint{3.400783in}{0.488818in}}%
\pgfpathlineto{\pgfqpoint{3.403394in}{0.484225in}}%
\pgfpathlineto{\pgfqpoint{3.406202in}{0.487494in}}%
\pgfpathlineto{\pgfqpoint{3.408752in}{0.486175in}}%
\pgfpathlineto{\pgfqpoint{3.411431in}{0.486164in}}%
\pgfpathlineto{\pgfqpoint{3.414109in}{0.488219in}}%
\pgfpathlineto{\pgfqpoint{3.416780in}{0.486330in}}%
\pgfpathlineto{\pgfqpoint{3.419455in}{0.489644in}}%
\pgfpathlineto{\pgfqpoint{3.422142in}{0.491577in}}%
\pgfpathlineto{\pgfqpoint{3.424887in}{0.487298in}}%
\pgfpathlineto{\pgfqpoint{3.427501in}{0.486846in}}%
\pgfpathlineto{\pgfqpoint{3.430313in}{0.488259in}}%
\pgfpathlineto{\pgfqpoint{3.432851in}{0.484915in}}%
\pgfpathlineto{\pgfqpoint{3.435635in}{0.483915in}}%
\pgfpathlineto{\pgfqpoint{3.438210in}{0.486170in}}%
\pgfpathlineto{\pgfqpoint{3.440996in}{0.487165in}}%
\pgfpathlineto{\pgfqpoint{3.443574in}{0.486717in}}%
\pgfpathlineto{\pgfqpoint{3.446257in}{0.485256in}}%
\pgfpathlineto{\pgfqpoint{3.448926in}{0.484914in}}%
\pgfpathlineto{\pgfqpoint{3.451597in}{0.486639in}}%
\pgfpathlineto{\pgfqpoint{3.454285in}{0.483098in}}%
\pgfpathlineto{\pgfqpoint{3.456960in}{0.482433in}}%
\pgfpathlineto{\pgfqpoint{3.459695in}{0.474894in}}%
\pgfpathlineto{\pgfqpoint{3.462321in}{0.476012in}}%
\pgfpathlineto{\pgfqpoint{3.465072in}{0.477581in}}%
\pgfpathlineto{\pgfqpoint{3.467678in}{0.479058in}}%
\pgfpathlineto{\pgfqpoint{3.470466in}{0.481336in}}%
\pgfpathlineto{\pgfqpoint{3.473021in}{0.478947in}}%
\pgfpathlineto{\pgfqpoint{3.475821in}{0.477455in}}%
\pgfpathlineto{\pgfqpoint{3.478378in}{0.483211in}}%
\pgfpathlineto{\pgfqpoint{3.481072in}{0.483092in}}%
\pgfpathlineto{\pgfqpoint{3.483744in}{0.483525in}}%
\pgfpathlineto{\pgfqpoint{3.486442in}{0.489425in}}%
\pgfpathlineto{\pgfqpoint{3.489223in}{0.482591in}}%
\pgfpathlineto{\pgfqpoint{3.491783in}{0.480234in}}%
\pgfpathlineto{\pgfqpoint{3.494581in}{0.479442in}}%
\pgfpathlineto{\pgfqpoint{3.497139in}{0.482380in}}%
\pgfpathlineto{\pgfqpoint{3.499909in}{0.485234in}}%
\pgfpathlineto{\pgfqpoint{3.502488in}{0.480377in}}%
\pgfpathlineto{\pgfqpoint{3.505262in}{0.487835in}}%
\pgfpathlineto{\pgfqpoint{3.507840in}{0.481846in}}%
\pgfpathlineto{\pgfqpoint{3.510533in}{0.486750in}}%
\pgfpathlineto{\pgfqpoint{3.513209in}{0.485060in}}%
\pgfpathlineto{\pgfqpoint{3.515884in}{0.488928in}}%
\pgfpathlineto{\pgfqpoint{3.518565in}{0.479622in}}%
\pgfpathlineto{\pgfqpoint{3.521244in}{0.487444in}}%
\pgfpathlineto{\pgfqpoint{3.524041in}{0.486330in}}%
\pgfpathlineto{\pgfqpoint{3.526601in}{0.485551in}}%
\pgfpathlineto{\pgfqpoint{3.529327in}{0.489730in}}%
\pgfpathlineto{\pgfqpoint{3.531955in}{0.480876in}}%
\pgfpathlineto{\pgfqpoint{3.534783in}{0.478229in}}%
\pgfpathlineto{\pgfqpoint{3.537309in}{0.478348in}}%
\pgfpathlineto{\pgfqpoint{3.540093in}{0.478670in}}%
\pgfpathlineto{\pgfqpoint{3.542656in}{0.485631in}}%
\pgfpathlineto{\pgfqpoint{3.545349in}{0.482107in}}%
\pgfpathlineto{\pgfqpoint{3.548029in}{0.489026in}}%
\pgfpathlineto{\pgfqpoint{3.550713in}{0.481375in}}%
\pgfpathlineto{\pgfqpoint{3.553498in}{0.484495in}}%
\pgfpathlineto{\pgfqpoint{3.556061in}{0.483756in}}%
\pgfpathlineto{\pgfqpoint{3.558853in}{0.486405in}}%
\pgfpathlineto{\pgfqpoint{3.561420in}{0.488908in}}%
\pgfpathlineto{\pgfqpoint{3.564188in}{0.485386in}}%
\pgfpathlineto{\pgfqpoint{3.566774in}{0.485581in}}%
\pgfpathlineto{\pgfqpoint{3.569584in}{0.483042in}}%
\pgfpathlineto{\pgfqpoint{3.572126in}{0.481689in}}%
\pgfpathlineto{\pgfqpoint{3.574814in}{0.484882in}}%
\pgfpathlineto{\pgfqpoint{3.577487in}{0.478046in}}%
\pgfpathlineto{\pgfqpoint{3.580191in}{0.471015in}}%
\pgfpathlineto{\pgfqpoint{3.582851in}{0.481426in}}%
\pgfpathlineto{\pgfqpoint{3.585532in}{0.482223in}}%
\pgfpathlineto{\pgfqpoint{3.588258in}{0.483802in}}%
\pgfpathlineto{\pgfqpoint{3.590883in}{0.488651in}}%
\pgfpathlineto{\pgfqpoint{3.593620in}{0.480929in}}%
\pgfpathlineto{\pgfqpoint{3.596240in}{0.486300in}}%
\pgfpathlineto{\pgfqpoint{3.598998in}{0.487704in}}%
\pgfpathlineto{\pgfqpoint{3.601590in}{0.481694in}}%
\pgfpathlineto{\pgfqpoint{3.604387in}{0.485519in}}%
\pgfpathlineto{\pgfqpoint{3.606951in}{0.487688in}}%
\pgfpathlineto{\pgfqpoint{3.609632in}{0.489432in}}%
\pgfpathlineto{\pgfqpoint{3.612311in}{0.486151in}}%
\pgfpathlineto{\pgfqpoint{3.614982in}{0.479208in}}%
\pgfpathlineto{\pgfqpoint{3.617667in}{0.488038in}}%
\pgfpathlineto{\pgfqpoint{3.620345in}{0.485423in}}%
\pgfpathlineto{\pgfqpoint{3.623165in}{0.481580in}}%
\pgfpathlineto{\pgfqpoint{3.625689in}{0.482284in}}%
\pgfpathlineto{\pgfqpoint{3.628460in}{0.480954in}}%
\pgfpathlineto{\pgfqpoint{3.631058in}{0.482371in}}%
\pgfpathlineto{\pgfqpoint{3.633858in}{0.486057in}}%
\pgfpathlineto{\pgfqpoint{3.636413in}{0.481048in}}%
\pgfpathlineto{\pgfqpoint{3.639207in}{0.479721in}}%
\pgfpathlineto{\pgfqpoint{3.641773in}{0.488882in}}%
\pgfpathlineto{\pgfqpoint{3.644452in}{0.539898in}}%
\pgfpathlineto{\pgfqpoint{3.647130in}{0.623200in}}%
\pgfpathlineto{\pgfqpoint{3.649837in}{0.597031in}}%
\pgfpathlineto{\pgfqpoint{3.652628in}{0.552426in}}%
\pgfpathlineto{\pgfqpoint{3.655165in}{0.536486in}}%
\pgfpathlineto{\pgfqpoint{3.657917in}{0.640479in}}%
\pgfpathlineto{\pgfqpoint{3.660515in}{0.705067in}}%
\pgfpathlineto{\pgfqpoint{3.663276in}{0.738073in}}%
\pgfpathlineto{\pgfqpoint{3.665864in}{0.704301in}}%
\pgfpathlineto{\pgfqpoint{3.668665in}{0.651091in}}%
\pgfpathlineto{\pgfqpoint{3.671232in}{0.616062in}}%
\pgfpathlineto{\pgfqpoint{3.673911in}{0.644391in}}%
\pgfpathlineto{\pgfqpoint{3.676591in}{0.745636in}}%
\pgfpathlineto{\pgfqpoint{3.679273in}{0.761320in}}%
\pgfpathlineto{\pgfqpoint{3.681948in}{0.720137in}}%
\pgfpathlineto{\pgfqpoint{3.684620in}{0.694685in}}%
\pgfpathlineto{\pgfqpoint{3.687442in}{0.655749in}}%
\pgfpathlineto{\pgfqpoint{3.689983in}{0.624446in}}%
\pgfpathlineto{\pgfqpoint{3.692765in}{0.600684in}}%
\pgfpathlineto{\pgfqpoint{3.695331in}{0.581057in}}%
\pgfpathlineto{\pgfqpoint{3.698125in}{0.564039in}}%
\pgfpathlineto{\pgfqpoint{3.700684in}{0.553657in}}%
\pgfpathlineto{\pgfqpoint{3.703460in}{0.534525in}}%
\pgfpathlineto{\pgfqpoint{3.706053in}{0.537433in}}%
\pgfpathlineto{\pgfqpoint{3.708729in}{0.528001in}}%
\pgfpathlineto{\pgfqpoint{3.711410in}{0.522363in}}%
\pgfpathlineto{\pgfqpoint{3.714086in}{0.513559in}}%
\pgfpathlineto{\pgfqpoint{3.716875in}{0.504656in}}%
\pgfpathlineto{\pgfqpoint{3.719446in}{0.490169in}}%
\pgfpathlineto{\pgfqpoint{3.722228in}{0.474024in}}%
\pgfpathlineto{\pgfqpoint{3.724804in}{0.489823in}}%
\pgfpathlineto{\pgfqpoint{3.727581in}{0.489461in}}%
\pgfpathlineto{\pgfqpoint{3.730158in}{0.484186in}}%
\pgfpathlineto{\pgfqpoint{3.732950in}{0.489600in}}%
\pgfpathlineto{\pgfqpoint{3.735509in}{0.493722in}}%
\pgfpathlineto{\pgfqpoint{3.738194in}{0.489932in}}%
\pgfpathlineto{\pgfqpoint{3.740874in}{0.488487in}}%
\pgfpathlineto{\pgfqpoint{3.743548in}{0.488682in}}%
\pgfpathlineto{\pgfqpoint{3.746229in}{0.487345in}}%
\pgfpathlineto{\pgfqpoint{3.748903in}{0.488540in}}%
\pgfpathlineto{\pgfqpoint{3.751728in}{0.484894in}}%
\pgfpathlineto{\pgfqpoint{3.754265in}{0.487002in}}%
\pgfpathlineto{\pgfqpoint{3.757065in}{0.483478in}}%
\pgfpathlineto{\pgfqpoint{3.759622in}{0.486931in}}%
\pgfpathlineto{\pgfqpoint{3.762389in}{0.488795in}}%
\pgfpathlineto{\pgfqpoint{3.764966in}{0.492869in}}%
\pgfpathlineto{\pgfqpoint{3.767782in}{0.488601in}}%
\pgfpathlineto{\pgfqpoint{3.770323in}{0.479913in}}%
\pgfpathlineto{\pgfqpoint{3.773014in}{0.474785in}}%
\pgfpathlineto{\pgfqpoint{3.775691in}{0.478983in}}%
\pgfpathlineto{\pgfqpoint{3.778370in}{0.494775in}}%
\pgfpathlineto{\pgfqpoint{3.781046in}{0.501970in}}%
\pgfpathlineto{\pgfqpoint{3.783725in}{0.498407in}}%
\pgfpathlineto{\pgfqpoint{3.786504in}{0.482455in}}%
\pgfpathlineto{\pgfqpoint{3.789084in}{0.482451in}}%
\pgfpathlineto{\pgfqpoint{3.791897in}{0.480956in}}%
\pgfpathlineto{\pgfqpoint{3.794435in}{0.484829in}}%
\pgfpathlineto{\pgfqpoint{3.797265in}{0.480704in}}%
\pgfpathlineto{\pgfqpoint{3.799797in}{0.483769in}}%
\pgfpathlineto{\pgfqpoint{3.802569in}{0.480526in}}%
\pgfpathlineto{\pgfqpoint{3.805145in}{0.480359in}}%
\pgfpathlineto{\pgfqpoint{3.807832in}{0.479834in}}%
\pgfpathlineto{\pgfqpoint{3.810510in}{0.482160in}}%
\pgfpathlineto{\pgfqpoint{3.813172in}{0.483576in}}%
\pgfpathlineto{\pgfqpoint{3.815983in}{0.482227in}}%
\pgfpathlineto{\pgfqpoint{3.818546in}{0.482938in}}%
\pgfpathlineto{\pgfqpoint{3.821315in}{0.488218in}}%
\pgfpathlineto{\pgfqpoint{3.823903in}{0.487654in}}%
\pgfpathlineto{\pgfqpoint{3.826679in}{0.485506in}}%
\pgfpathlineto{\pgfqpoint{3.829252in}{0.485001in}}%
\pgfpathlineto{\pgfqpoint{3.832053in}{0.484036in}}%
\pgfpathlineto{\pgfqpoint{3.834616in}{0.479994in}}%
\pgfpathlineto{\pgfqpoint{3.837286in}{0.484716in}}%
\pgfpathlineto{\pgfqpoint{3.839960in}{0.487405in}}%
\pgfpathlineto{\pgfqpoint{3.842641in}{0.484465in}}%
\pgfpathlineto{\pgfqpoint{3.845329in}{0.482641in}}%
\pgfpathlineto{\pgfqpoint{3.848005in}{0.484745in}}%
\pgfpathlineto{\pgfqpoint{3.850814in}{0.478914in}}%
\pgfpathlineto{\pgfqpoint{3.853358in}{0.483371in}}%
\pgfpathlineto{\pgfqpoint{3.856100in}{0.486370in}}%
\pgfpathlineto{\pgfqpoint{3.858720in}{0.488514in}}%
\pgfpathlineto{\pgfqpoint{3.861561in}{0.484828in}}%
\pgfpathlineto{\pgfqpoint{3.864073in}{0.482502in}}%
\pgfpathlineto{\pgfqpoint{3.866815in}{0.480323in}}%
\pgfpathlineto{\pgfqpoint{3.869435in}{0.479325in}}%
\pgfpathlineto{\pgfqpoint{3.872114in}{0.479301in}}%
\pgfpathlineto{\pgfqpoint{3.874790in}{0.481800in}}%
\pgfpathlineto{\pgfqpoint{3.877466in}{0.535019in}}%
\pgfpathlineto{\pgfqpoint{3.880237in}{0.570846in}}%
\pgfpathlineto{\pgfqpoint{3.882850in}{0.569002in}}%
\pgfpathlineto{\pgfqpoint{3.885621in}{0.555154in}}%
\pgfpathlineto{\pgfqpoint{3.888188in}{0.557532in}}%
\pgfpathlineto{\pgfqpoint{3.890926in}{0.551097in}}%
\pgfpathlineto{\pgfqpoint{3.893541in}{0.546460in}}%
\pgfpathlineto{\pgfqpoint{3.896345in}{0.528395in}}%
\pgfpathlineto{\pgfqpoint{3.898891in}{0.515652in}}%
\pgfpathlineto{\pgfqpoint{3.901573in}{0.523113in}}%
\pgfpathlineto{\pgfqpoint{3.904252in}{0.516145in}}%
\pgfpathlineto{\pgfqpoint{3.906918in}{0.504391in}}%
\pgfpathlineto{\pgfqpoint{3.909602in}{0.503339in}}%
\pgfpathlineto{\pgfqpoint{3.912296in}{0.497094in}}%
\pgfpathlineto{\pgfqpoint{3.915107in}{0.500753in}}%
\pgfpathlineto{\pgfqpoint{3.917646in}{0.505992in}}%
\pgfpathlineto{\pgfqpoint{3.920412in}{0.506874in}}%
\pgfpathlineto{\pgfqpoint{3.923005in}{0.498442in}}%
\pgfpathlineto{\pgfqpoint{3.925778in}{0.493820in}}%
\pgfpathlineto{\pgfqpoint{3.928347in}{0.491956in}}%
\pgfpathlineto{\pgfqpoint{3.931202in}{0.486640in}}%
\pgfpathlineto{\pgfqpoint{3.933714in}{0.482887in}}%
\pgfpathlineto{\pgfqpoint{3.936395in}{0.483167in}}%
\pgfpathlineto{\pgfqpoint{3.939075in}{0.485769in}}%
\pgfpathlineto{\pgfqpoint{3.941778in}{0.491701in}}%
\pgfpathlineto{\pgfqpoint{3.944431in}{0.478294in}}%
\pgfpathlineto{\pgfqpoint{3.947101in}{0.474957in}}%
\pgfpathlineto{\pgfqpoint{3.949894in}{0.475661in}}%
\pgfpathlineto{\pgfqpoint{3.952464in}{0.471821in}}%
\pgfpathlineto{\pgfqpoint{3.955211in}{0.478759in}}%
\pgfpathlineto{\pgfqpoint{3.957823in}{0.484048in}}%
\pgfpathlineto{\pgfqpoint{3.960635in}{0.483957in}}%
\pgfpathlineto{\pgfqpoint{3.963176in}{0.479805in}}%
\pgfpathlineto{\pgfqpoint{3.966013in}{0.480003in}}%
\pgfpathlineto{\pgfqpoint{3.968523in}{0.482251in}}%
\pgfpathlineto{\pgfqpoint{3.971250in}{0.476280in}}%
\pgfpathlineto{\pgfqpoint{3.973885in}{0.476167in}}%
\pgfpathlineto{\pgfqpoint{3.976563in}{0.475530in}}%
\pgfpathlineto{\pgfqpoint{3.979389in}{0.477586in}}%
\pgfpathlineto{\pgfqpoint{3.981929in}{0.482670in}}%
\pgfpathlineto{\pgfqpoint{3.984714in}{0.477431in}}%
\pgfpathlineto{\pgfqpoint{3.987270in}{0.479497in}}%
\pgfpathlineto{\pgfqpoint{3.990055in}{0.482521in}}%
\pgfpathlineto{\pgfqpoint{3.992642in}{0.474239in}}%
\pgfpathlineto{\pgfqpoint{3.995417in}{0.471448in}}%
\pgfpathlineto{\pgfqpoint{3.997990in}{0.474015in}}%
\pgfpathlineto{\pgfqpoint{4.000674in}{0.473889in}}%
\pgfpathlineto{\pgfqpoint{4.003348in}{0.483032in}}%
\pgfpathlineto{\pgfqpoint{4.006034in}{0.472842in}}%
\pgfpathlineto{\pgfqpoint{4.008699in}{0.472519in}}%
\pgfpathlineto{\pgfqpoint{4.011394in}{0.476787in}}%
\pgfpathlineto{\pgfqpoint{4.014186in}{0.475096in}}%
\pgfpathlineto{\pgfqpoint{4.016744in}{0.479763in}}%
\pgfpathlineto{\pgfqpoint{4.019518in}{0.478746in}}%
\pgfpathlineto{\pgfqpoint{4.022097in}{0.479728in}}%
\pgfpathlineto{\pgfqpoint{4.024868in}{0.477593in}}%
\pgfpathlineto{\pgfqpoint{4.027447in}{0.480973in}}%
\pgfpathlineto{\pgfqpoint{4.030229in}{0.479651in}}%
\pgfpathlineto{\pgfqpoint{4.032817in}{0.484133in}}%
\pgfpathlineto{\pgfqpoint{4.035492in}{0.480015in}}%
\pgfpathlineto{\pgfqpoint{4.038174in}{0.482128in}}%
\pgfpathlineto{\pgfqpoint{4.040852in}{0.478320in}}%
\pgfpathlineto{\pgfqpoint{4.043667in}{0.484983in}}%
\pgfpathlineto{\pgfqpoint{4.046210in}{0.483324in}}%
\pgfpathlineto{\pgfqpoint{4.049006in}{0.483489in}}%
\pgfpathlineto{\pgfqpoint{4.051557in}{0.482047in}}%
\pgfpathlineto{\pgfqpoint{4.054326in}{0.482525in}}%
\pgfpathlineto{\pgfqpoint{4.056911in}{0.483423in}}%
\pgfpathlineto{\pgfqpoint{4.059702in}{0.477672in}}%
\pgfpathlineto{\pgfqpoint{4.062266in}{0.483137in}}%
\pgfpathlineto{\pgfqpoint{4.064957in}{0.484958in}}%
\pgfpathlineto{\pgfqpoint{4.067636in}{0.487409in}}%
\pgfpathlineto{\pgfqpoint{4.070313in}{0.487570in}}%
\pgfpathlineto{\pgfqpoint{4.072985in}{0.484145in}}%
\pgfpathlineto{\pgfqpoint{4.075705in}{0.484635in}}%
\pgfpathlineto{\pgfqpoint{4.078471in}{0.492871in}}%
\pgfpathlineto{\pgfqpoint{4.081018in}{0.498543in}}%
\pgfpathlineto{\pgfqpoint{4.083870in}{0.496133in}}%
\pgfpathlineto{\pgfqpoint{4.086385in}{0.482459in}}%
\pgfpathlineto{\pgfqpoint{4.089159in}{0.480683in}}%
\pgfpathlineto{\pgfqpoint{4.091729in}{0.479319in}}%
\pgfpathlineto{\pgfqpoint{4.094527in}{0.481686in}}%
\pgfpathlineto{\pgfqpoint{4.097092in}{0.474869in}}%
\pgfpathlineto{\pgfqpoint{4.099777in}{0.476517in}}%
\pgfpathlineto{\pgfqpoint{4.102456in}{0.478681in}}%
\pgfpathlineto{\pgfqpoint{4.105185in}{0.481514in}}%
\pgfpathlineto{\pgfqpoint{4.107814in}{0.481076in}}%
\pgfpathlineto{\pgfqpoint{4.110488in}{0.474661in}}%
\pgfpathlineto{\pgfqpoint{4.113252in}{0.477948in}}%
\pgfpathlineto{\pgfqpoint{4.115844in}{0.488794in}}%
\pgfpathlineto{\pgfqpoint{4.118554in}{0.482760in}}%
\pgfpathlineto{\pgfqpoint{4.121205in}{0.491693in}}%
\pgfpathlineto{\pgfqpoint{4.124019in}{0.496469in}}%
\pgfpathlineto{\pgfqpoint{4.126553in}{0.494164in}}%
\pgfpathlineto{\pgfqpoint{4.129349in}{0.497213in}}%
\pgfpathlineto{\pgfqpoint{4.131920in}{0.490857in}}%
\pgfpathlineto{\pgfqpoint{4.134615in}{0.490146in}}%
\pgfpathlineto{\pgfqpoint{4.137272in}{0.493441in}}%
\pgfpathlineto{\pgfqpoint{4.139963in}{0.496521in}}%
\pgfpathlineto{\pgfqpoint{4.142713in}{0.508280in}}%
\pgfpathlineto{\pgfqpoint{4.145310in}{0.514723in}}%
\pgfpathlineto{\pgfqpoint{4.148082in}{0.511369in}}%
\pgfpathlineto{\pgfqpoint{4.150665in}{0.506510in}}%
\pgfpathlineto{\pgfqpoint{4.153423in}{0.504629in}}%
\pgfpathlineto{\pgfqpoint{4.156016in}{0.497606in}}%
\pgfpathlineto{\pgfqpoint{4.158806in}{0.490591in}}%
\pgfpathlineto{\pgfqpoint{4.161380in}{0.496222in}}%
\pgfpathlineto{\pgfqpoint{4.164059in}{0.492358in}}%
\pgfpathlineto{\pgfqpoint{4.166737in}{0.477695in}}%
\pgfpathlineto{\pgfqpoint{4.169415in}{0.471901in}}%
\pgfpathlineto{\pgfqpoint{4.172093in}{0.483970in}}%
\pgfpathlineto{\pgfqpoint{4.174770in}{0.492296in}}%
\pgfpathlineto{\pgfqpoint{4.177593in}{0.496566in}}%
\pgfpathlineto{\pgfqpoint{4.180129in}{0.501384in}}%
\pgfpathlineto{\pgfqpoint{4.182899in}{0.501048in}}%
\pgfpathlineto{\pgfqpoint{4.185481in}{0.496729in}}%
\pgfpathlineto{\pgfqpoint{4.188318in}{0.493087in}}%
\pgfpathlineto{\pgfqpoint{4.190842in}{0.491335in}}%
\pgfpathlineto{\pgfqpoint{4.193638in}{0.496967in}}%
\pgfpathlineto{\pgfqpoint{4.196186in}{0.496240in}}%
\pgfpathlineto{\pgfqpoint{4.198878in}{0.498775in}}%
\pgfpathlineto{\pgfqpoint{4.201542in}{0.494244in}}%
\pgfpathlineto{\pgfqpoint{4.204240in}{0.489706in}}%
\pgfpathlineto{\pgfqpoint{4.207076in}{0.481881in}}%
\pgfpathlineto{\pgfqpoint{4.209597in}{0.476241in}}%
\pgfpathlineto{\pgfqpoint{4.212383in}{0.478531in}}%
\pgfpathlineto{\pgfqpoint{4.214948in}{0.485333in}}%
\pgfpathlineto{\pgfqpoint{4.217694in}{0.490397in}}%
\pgfpathlineto{\pgfqpoint{4.220304in}{0.496494in}}%
\pgfpathlineto{\pgfqpoint{4.223082in}{0.491698in}}%
\pgfpathlineto{\pgfqpoint{4.225654in}{0.483706in}}%
\pgfpathlineto{\pgfqpoint{4.228331in}{0.493074in}}%
\pgfpathlineto{\pgfqpoint{4.231013in}{0.488781in}}%
\pgfpathlineto{\pgfqpoint{4.233691in}{0.483546in}}%
\pgfpathlineto{\pgfqpoint{4.236375in}{0.480347in}}%
\pgfpathlineto{\pgfqpoint{4.239084in}{0.486498in}}%
\pgfpathlineto{\pgfqpoint{4.241900in}{0.481509in}}%
\pgfpathlineto{\pgfqpoint{4.244394in}{0.484725in}}%
\pgfpathlineto{\pgfqpoint{4.247225in}{0.482138in}}%
\pgfpathlineto{\pgfqpoint{4.249767in}{0.479755in}}%
\pgfpathlineto{\pgfqpoint{4.252581in}{0.479267in}}%
\pgfpathlineto{\pgfqpoint{4.255120in}{0.481718in}}%
\pgfpathlineto{\pgfqpoint{4.257958in}{0.476138in}}%
\pgfpathlineto{\pgfqpoint{4.260477in}{0.482701in}}%
\pgfpathlineto{\pgfqpoint{4.263157in}{0.474587in}}%
\pgfpathlineto{\pgfqpoint{4.265824in}{0.475220in}}%
\pgfpathlineto{\pgfqpoint{4.268590in}{0.474064in}}%
\pgfpathlineto{\pgfqpoint{4.271187in}{0.470184in}}%
\pgfpathlineto{\pgfqpoint{4.273874in}{0.470399in}}%
\pgfpathlineto{\pgfqpoint{4.276635in}{0.469093in}}%
\pgfpathlineto{\pgfqpoint{4.279212in}{0.469895in}}%
\pgfpathlineto{\pgfqpoint{4.282000in}{0.480258in}}%
\pgfpathlineto{\pgfqpoint{4.284586in}{0.468890in}}%
\pgfpathlineto{\pgfqpoint{4.287399in}{0.477095in}}%
\pgfpathlineto{\pgfqpoint{4.289936in}{0.478828in}}%
\pgfpathlineto{\pgfqpoint{4.292786in}{0.476667in}}%
\pgfpathlineto{\pgfqpoint{4.295299in}{0.475452in}}%
\pgfpathlineto{\pgfqpoint{4.297977in}{0.475576in}}%
\pgfpathlineto{\pgfqpoint{4.300656in}{0.479251in}}%
\pgfpathlineto{\pgfqpoint{4.303357in}{0.483910in}}%
\pgfpathlineto{\pgfqpoint{4.306118in}{0.482837in}}%
\pgfpathlineto{\pgfqpoint{4.308691in}{0.481476in}}%
\pgfpathlineto{\pgfqpoint{4.311494in}{0.491519in}}%
\pgfpathlineto{\pgfqpoint{4.314032in}{0.491139in}}%
\pgfpathlineto{\pgfqpoint{4.316856in}{0.485120in}}%
\pgfpathlineto{\pgfqpoint{4.319405in}{0.488982in}}%
\pgfpathlineto{\pgfqpoint{4.322181in}{0.487738in}}%
\pgfpathlineto{\pgfqpoint{4.324760in}{0.486292in}}%
\pgfpathlineto{\pgfqpoint{4.327440in}{0.485055in}}%
\pgfpathlineto{\pgfqpoint{4.330118in}{0.485839in}}%
\pgfpathlineto{\pgfqpoint{4.332796in}{0.483757in}}%
\pgfpathlineto{\pgfqpoint{4.335463in}{0.480145in}}%
\pgfpathlineto{\pgfqpoint{4.338154in}{0.483797in}}%
\pgfpathlineto{\pgfqpoint{4.340976in}{0.486058in}}%
\pgfpathlineto{\pgfqpoint{4.343510in}{0.483609in}}%
\pgfpathlineto{\pgfqpoint{4.346263in}{0.487301in}}%
\pgfpathlineto{\pgfqpoint{4.348868in}{0.484413in}}%
\pgfpathlineto{\pgfqpoint{4.351645in}{0.482592in}}%
\pgfpathlineto{\pgfqpoint{4.354224in}{0.481295in}}%
\pgfpathlineto{\pgfqpoint{4.357014in}{0.487272in}}%
\pgfpathlineto{\pgfqpoint{4.359582in}{0.480227in}}%
\pgfpathlineto{\pgfqpoint{4.362270in}{0.484785in}}%
\pgfpathlineto{\pgfqpoint{4.364936in}{0.484364in}}%
\pgfpathlineto{\pgfqpoint{4.367646in}{0.482890in}}%
\pgfpathlineto{\pgfqpoint{4.370437in}{0.486530in}}%
\pgfpathlineto{\pgfqpoint{4.372976in}{0.483383in}}%
\pgfpathlineto{\pgfqpoint{4.375761in}{0.483121in}}%
\pgfpathlineto{\pgfqpoint{4.378329in}{0.480377in}}%
\pgfpathlineto{\pgfqpoint{4.381097in}{0.476781in}}%
\pgfpathlineto{\pgfqpoint{4.383674in}{0.480453in}}%
\pgfpathlineto{\pgfqpoint{4.386431in}{0.483378in}}%
\pgfpathlineto{\pgfqpoint{4.389044in}{0.483543in}}%
\pgfpathlineto{\pgfqpoint{4.391721in}{0.485815in}}%
\pgfpathlineto{\pgfqpoint{4.394400in}{0.472989in}}%
\pgfpathlineto{\pgfqpoint{4.397076in}{0.469783in}}%
\pgfpathlineto{\pgfqpoint{4.399745in}{0.470089in}}%
\pgfpathlineto{\pgfqpoint{4.402468in}{0.478614in}}%
\pgfpathlineto{\pgfqpoint{4.405234in}{0.482819in}}%
\pgfpathlineto{\pgfqpoint{4.407788in}{0.483571in}}%
\pgfpathlineto{\pgfqpoint{4.410587in}{0.477807in}}%
\pgfpathlineto{\pgfqpoint{4.413149in}{0.483318in}}%
\pgfpathlineto{\pgfqpoint{4.415932in}{0.484104in}}%
\pgfpathlineto{\pgfqpoint{4.418506in}{0.484833in}}%
\pgfpathlineto{\pgfqpoint{4.421292in}{0.482512in}}%
\pgfpathlineto{\pgfqpoint{4.423863in}{0.488393in}}%
\pgfpathlineto{\pgfqpoint{4.426534in}{0.490200in}}%
\pgfpathlineto{\pgfqpoint{4.429220in}{0.488802in}}%
\pgfpathlineto{\pgfqpoint{4.431901in}{0.487976in}}%
\pgfpathlineto{\pgfqpoint{4.434569in}{0.485621in}}%
\pgfpathlineto{\pgfqpoint{4.437253in}{0.488718in}}%
\pgfpathlineto{\pgfqpoint{4.440041in}{0.482660in}}%
\pgfpathlineto{\pgfqpoint{4.442611in}{0.486091in}}%
\pgfpathlineto{\pgfqpoint{4.445423in}{0.478200in}}%
\pgfpathlineto{\pgfqpoint{4.447965in}{0.481863in}}%
\pgfpathlineto{\pgfqpoint{4.450767in}{0.483135in}}%
\pgfpathlineto{\pgfqpoint{4.453312in}{0.476189in}}%
\pgfpathlineto{\pgfqpoint{4.456138in}{0.469194in}}%
\pgfpathlineto{\pgfqpoint{4.458681in}{0.468454in}}%
\pgfpathlineto{\pgfqpoint{4.461367in}{0.468461in}}%
\pgfpathlineto{\pgfqpoint{4.464029in}{0.469973in}}%
\pgfpathlineto{\pgfqpoint{4.466717in}{0.472990in}}%
\pgfpathlineto{\pgfqpoint{4.469492in}{0.467232in}}%
\pgfpathlineto{\pgfqpoint{4.472059in}{0.469400in}}%
\pgfpathlineto{\pgfqpoint{4.474861in}{0.470287in}}%
\pgfpathlineto{\pgfqpoint{4.477430in}{0.467405in}}%
\pgfpathlineto{\pgfqpoint{4.480201in}{0.465847in}}%
\pgfpathlineto{\pgfqpoint{4.482778in}{0.473461in}}%
\pgfpathlineto{\pgfqpoint{4.485581in}{0.478653in}}%
\pgfpathlineto{\pgfqpoint{4.488130in}{0.472794in}}%
\pgfpathlineto{\pgfqpoint{4.490822in}{0.476865in}}%
\pgfpathlineto{\pgfqpoint{4.493492in}{0.476946in}}%
\pgfpathlineto{\pgfqpoint{4.496167in}{0.482562in}}%
\pgfpathlineto{\pgfqpoint{4.498850in}{0.481657in}}%
\pgfpathlineto{\pgfqpoint{4.501529in}{0.482627in}}%
\pgfpathlineto{\pgfqpoint{4.504305in}{0.478184in}}%
\pgfpathlineto{\pgfqpoint{4.506893in}{0.479225in}}%
\pgfpathlineto{\pgfqpoint{4.509643in}{0.478622in}}%
\pgfpathlineto{\pgfqpoint{4.512246in}{0.477958in}}%
\pgfpathlineto{\pgfqpoint{4.515080in}{0.484950in}}%
\pgfpathlineto{\pgfqpoint{4.517598in}{0.481612in}}%
\pgfpathlineto{\pgfqpoint{4.520345in}{0.477863in}}%
\pgfpathlineto{\pgfqpoint{4.522962in}{0.484278in}}%
\pgfpathlineto{\pgfqpoint{4.525640in}{0.480591in}}%
\pgfpathlineto{\pgfqpoint{4.528307in}{0.488116in}}%
\pgfpathlineto{\pgfqpoint{4.530990in}{0.488571in}}%
\pgfpathlineto{\pgfqpoint{4.533764in}{0.484382in}}%
\pgfpathlineto{\pgfqpoint{4.536400in}{0.486414in}}%
\pgfpathlineto{\pgfqpoint{4.539144in}{0.487315in}}%
\pgfpathlineto{\pgfqpoint{4.541711in}{0.484571in}}%
\pgfpathlineto{\pgfqpoint{4.544464in}{0.486261in}}%
\pgfpathlineto{\pgfqpoint{4.547064in}{0.484586in}}%
\pgfpathlineto{\pgfqpoint{4.549822in}{0.489247in}}%
\pgfpathlineto{\pgfqpoint{4.552425in}{0.489604in}}%
\pgfpathlineto{\pgfqpoint{4.555106in}{0.486874in}}%
\pgfpathlineto{\pgfqpoint{4.557777in}{0.490455in}}%
\pgfpathlineto{\pgfqpoint{4.560448in}{0.487333in}}%
\pgfpathlineto{\pgfqpoint{4.563125in}{0.487296in}}%
\pgfpathlineto{\pgfqpoint{4.565820in}{0.488067in}}%
\pgfpathlineto{\pgfqpoint{4.568612in}{0.488499in}}%
\pgfpathlineto{\pgfqpoint{4.571171in}{0.486377in}}%
\pgfpathlineto{\pgfqpoint{4.573947in}{0.481899in}}%
\pgfpathlineto{\pgfqpoint{4.576531in}{0.488279in}}%
\pgfpathlineto{\pgfqpoint{4.579305in}{0.486493in}}%
\pgfpathlineto{\pgfqpoint{4.581888in}{0.483698in}}%
\pgfpathlineto{\pgfqpoint{4.584672in}{0.483541in}}%
\pgfpathlineto{\pgfqpoint{4.587244in}{0.477910in}}%
\pgfpathlineto{\pgfqpoint{4.589920in}{0.476948in}}%
\pgfpathlineto{\pgfqpoint{4.592589in}{0.474168in}}%
\pgfpathlineto{\pgfqpoint{4.595281in}{0.479797in}}%
\pgfpathlineto{\pgfqpoint{4.597951in}{0.482352in}}%
\pgfpathlineto{\pgfqpoint{4.600633in}{0.480737in}}%
\pgfpathlineto{\pgfqpoint{4.603430in}{0.481214in}}%
\pgfpathlineto{\pgfqpoint{4.605990in}{0.482940in}}%
\pgfpathlineto{\pgfqpoint{4.608808in}{0.474034in}}%
\pgfpathlineto{\pgfqpoint{4.611350in}{0.479081in}}%
\pgfpathlineto{\pgfqpoint{4.614134in}{0.482979in}}%
\pgfpathlineto{\pgfqpoint{4.616702in}{0.480630in}}%
\pgfpathlineto{\pgfqpoint{4.619529in}{0.483806in}}%
\pgfpathlineto{\pgfqpoint{4.622056in}{0.484257in}}%
\pgfpathlineto{\pgfqpoint{4.624741in}{0.481455in}}%
\pgfpathlineto{\pgfqpoint{4.627411in}{0.482208in}}%
\pgfpathlineto{\pgfqpoint{4.630096in}{0.480152in}}%
\pgfpathlineto{\pgfqpoint{4.632902in}{0.479742in}}%
\pgfpathlineto{\pgfqpoint{4.635445in}{0.481305in}}%
\pgfpathlineto{\pgfqpoint{4.638204in}{0.486588in}}%
\pgfpathlineto{\pgfqpoint{4.640809in}{0.481506in}}%
\pgfpathlineto{\pgfqpoint{4.643628in}{0.486473in}}%
\pgfpathlineto{\pgfqpoint{4.646169in}{0.487487in}}%
\pgfpathlineto{\pgfqpoint{4.648922in}{0.483176in}}%
\pgfpathlineto{\pgfqpoint{4.651524in}{0.482490in}}%
\pgfpathlineto{\pgfqpoint{4.654203in}{0.484028in}}%
\pgfpathlineto{\pgfqpoint{4.656873in}{0.483891in}}%
\pgfpathlineto{\pgfqpoint{4.659590in}{0.484509in}}%
\pgfpathlineto{\pgfqpoint{4.662237in}{0.483524in}}%
\pgfpathlineto{\pgfqpoint{4.664923in}{0.477480in}}%
\pgfpathlineto{\pgfqpoint{4.667764in}{0.479749in}}%
\pgfpathlineto{\pgfqpoint{4.670261in}{0.481599in}}%
\pgfpathlineto{\pgfqpoint{4.673068in}{0.480723in}}%
\pgfpathlineto{\pgfqpoint{4.675619in}{0.483744in}}%
\pgfpathlineto{\pgfqpoint{4.678448in}{0.486152in}}%
\pgfpathlineto{\pgfqpoint{4.680988in}{0.487761in}}%
\pgfpathlineto{\pgfqpoint{4.683799in}{0.486228in}}%
\pgfpathlineto{\pgfqpoint{4.686337in}{0.474739in}}%
\pgfpathlineto{\pgfqpoint{4.689051in}{0.476197in}}%
\pgfpathlineto{\pgfqpoint{4.691694in}{0.487589in}}%
\pgfpathlineto{\pgfqpoint{4.694381in}{0.479356in}}%
\pgfpathlineto{\pgfqpoint{4.697170in}{0.480350in}}%
\pgfpathlineto{\pgfqpoint{4.699734in}{0.487085in}}%
\pgfpathlineto{\pgfqpoint{4.702517in}{0.482520in}}%
\pgfpathlineto{\pgfqpoint{4.705094in}{0.482203in}}%
\pgfpathlineto{\pgfqpoint{4.707824in}{0.484132in}}%
\pgfpathlineto{\pgfqpoint{4.710437in}{0.482469in}}%
\pgfpathlineto{\pgfqpoint{4.713275in}{0.478451in}}%
\pgfpathlineto{\pgfqpoint{4.715806in}{0.478664in}}%
\pgfpathlineto{\pgfqpoint{4.718486in}{0.484310in}}%
\pgfpathlineto{\pgfqpoint{4.721160in}{0.484265in}}%
\pgfpathlineto{\pgfqpoint{4.723873in}{0.483459in}}%
\pgfpathlineto{\pgfqpoint{4.726508in}{0.480376in}}%
\pgfpathlineto{\pgfqpoint{4.729233in}{0.481741in}}%
\pgfpathlineto{\pgfqpoint{4.731901in}{0.479797in}}%
\pgfpathlineto{\pgfqpoint{4.734552in}{0.476367in}}%
\pgfpathlineto{\pgfqpoint{4.737348in}{0.479407in}}%
\pgfpathlineto{\pgfqpoint{4.739912in}{0.485385in}}%
\pgfpathlineto{\pgfqpoint{4.742696in}{0.482225in}}%
\pgfpathlineto{\pgfqpoint{4.745256in}{0.478750in}}%
\pgfpathlineto{\pgfqpoint{4.748081in}{0.475686in}}%
\pgfpathlineto{\pgfqpoint{4.750627in}{0.480315in}}%
\pgfpathlineto{\pgfqpoint{4.753298in}{0.475725in}}%
\pgfpathlineto{\pgfqpoint{4.755983in}{0.467479in}}%
\pgfpathlineto{\pgfqpoint{4.758653in}{0.471848in}}%
\pgfpathlineto{\pgfqpoint{4.761337in}{0.474038in}}%
\pgfpathlineto{\pgfqpoint{4.764018in}{0.470852in}}%
\pgfpathlineto{\pgfqpoint{4.766783in}{0.469754in}}%
\pgfpathlineto{\pgfqpoint{4.769367in}{0.471897in}}%
\pgfpathlineto{\pgfqpoint{4.772198in}{0.476947in}}%
\pgfpathlineto{\pgfqpoint{4.774732in}{0.477454in}}%
\pgfpathlineto{\pgfqpoint{4.777535in}{0.478484in}}%
\pgfpathlineto{\pgfqpoint{4.780083in}{0.481105in}}%
\pgfpathlineto{\pgfqpoint{4.782872in}{0.489412in}}%
\pgfpathlineto{\pgfqpoint{4.785445in}{0.491677in}}%
\pgfpathlineto{\pgfqpoint{4.788116in}{0.477492in}}%
\pgfpathlineto{\pgfqpoint{4.790798in}{0.479247in}}%
\pgfpathlineto{\pgfqpoint{4.793512in}{0.485463in}}%
\pgfpathlineto{\pgfqpoint{4.796274in}{0.481934in}}%
\pgfpathlineto{\pgfqpoint{4.798830in}{0.480790in}}%
\pgfpathlineto{\pgfqpoint{4.801586in}{0.480917in}}%
\pgfpathlineto{\pgfqpoint{4.804193in}{0.485346in}}%
\pgfpathlineto{\pgfqpoint{4.807017in}{0.480160in}}%
\pgfpathlineto{\pgfqpoint{4.809538in}{0.484237in}}%
\pgfpathlineto{\pgfqpoint{4.812377in}{0.485825in}}%
\pgfpathlineto{\pgfqpoint{4.814907in}{0.495359in}}%
\pgfpathlineto{\pgfqpoint{4.817587in}{0.493707in}}%
\pgfpathlineto{\pgfqpoint{4.820265in}{0.501275in}}%
\pgfpathlineto{\pgfqpoint{4.822945in}{0.492303in}}%
\pgfpathlineto{\pgfqpoint{4.825619in}{0.484880in}}%
\pgfpathlineto{\pgfqpoint{4.828291in}{0.484598in}}%
\pgfpathlineto{\pgfqpoint{4.831045in}{0.476447in}}%
\pgfpathlineto{\pgfqpoint{4.833657in}{0.477301in}}%
\pgfpathlineto{\pgfqpoint{4.837992in}{0.478394in}}%
\pgfpathlineto{\pgfqpoint{4.839922in}{0.483720in}}%
\pgfpathlineto{\pgfqpoint{4.842380in}{0.488933in}}%
\pgfpathlineto{\pgfqpoint{4.844361in}{0.492235in}}%
\pgfpathlineto{\pgfqpoint{4.847127in}{0.494488in}}%
\pgfpathlineto{\pgfqpoint{4.849715in}{0.509249in}}%
\pgfpathlineto{\pgfqpoint{4.852404in}{0.512339in}}%
\pgfpathlineto{\pgfqpoint{4.855070in}{0.503269in}}%
\pgfpathlineto{\pgfqpoint{4.857807in}{0.496861in}}%
\pgfpathlineto{\pgfqpoint{4.860544in}{0.496959in}}%
\pgfpathlineto{\pgfqpoint{4.863116in}{0.499471in}}%
\pgfpathlineto{\pgfqpoint{4.865910in}{0.494021in}}%
\pgfpathlineto{\pgfqpoint{4.868474in}{0.490849in}}%
\pgfpathlineto{\pgfqpoint{4.871209in}{0.491904in}}%
\pgfpathlineto{\pgfqpoint{4.873832in}{0.491216in}}%
\pgfpathlineto{\pgfqpoint{4.876636in}{0.486440in}}%
\pgfpathlineto{\pgfqpoint{4.879180in}{0.483165in}}%
\pgfpathlineto{\pgfqpoint{4.881864in}{0.485101in}}%
\pgfpathlineto{\pgfqpoint{4.884540in}{0.486488in}}%
\pgfpathlineto{\pgfqpoint{4.887211in}{0.491078in}}%
\pgfpathlineto{\pgfqpoint{4.889902in}{0.485376in}}%
\pgfpathlineto{\pgfqpoint{4.892611in}{0.487291in}}%
\pgfpathlineto{\pgfqpoint{4.895399in}{0.487894in}}%
\pgfpathlineto{\pgfqpoint{4.897938in}{0.489290in}}%
\pgfpathlineto{\pgfqpoint{4.900712in}{0.488845in}}%
\pgfpathlineto{\pgfqpoint{4.903295in}{0.471274in}}%
\pgfpathlineto{\pgfqpoint{4.906096in}{0.468964in}}%
\pgfpathlineto{\pgfqpoint{4.908648in}{0.473911in}}%
\pgfpathlineto{\pgfqpoint{4.911435in}{0.465847in}}%
\pgfpathlineto{\pgfqpoint{4.914009in}{0.469209in}}%
\pgfpathlineto{\pgfqpoint{4.916681in}{0.468709in}}%
\pgfpathlineto{\pgfqpoint{4.919352in}{0.468179in}}%
\pgfpathlineto{\pgfqpoint{4.922041in}{0.465847in}}%
\pgfpathlineto{\pgfqpoint{4.924708in}{0.465847in}}%
\pgfpathlineto{\pgfqpoint{4.927400in}{0.467811in}}%
\pgfpathlineto{\pgfqpoint{4.930170in}{0.465847in}}%
\pgfpathlineto{\pgfqpoint{4.932742in}{0.465847in}}%
\pgfpathlineto{\pgfqpoint{4.935515in}{0.474363in}}%
\pgfpathlineto{\pgfqpoint{4.938112in}{0.477337in}}%
\pgfpathlineto{\pgfqpoint{4.940881in}{0.478334in}}%
\pgfpathlineto{\pgfqpoint{4.943466in}{0.477950in}}%
\pgfpathlineto{\pgfqpoint{4.946151in}{0.478365in}}%
\pgfpathlineto{\pgfqpoint{4.948827in}{0.486918in}}%
\pgfpathlineto{\pgfqpoint{4.951504in}{0.480390in}}%
\pgfpathlineto{\pgfqpoint{4.954182in}{0.473940in}}%
\pgfpathlineto{\pgfqpoint{4.956862in}{0.479760in}}%
\pgfpathlineto{\pgfqpoint{4.959689in}{0.489879in}}%
\pgfpathlineto{\pgfqpoint{4.962219in}{0.485957in}}%
\pgfpathlineto{\pgfqpoint{4.965002in}{0.485102in}}%
\pgfpathlineto{\pgfqpoint{4.967575in}{0.481377in}}%
\pgfpathlineto{\pgfqpoint{4.970314in}{0.488637in}}%
\pgfpathlineto{\pgfqpoint{4.972933in}{0.484165in}}%
\pgfpathlineto{\pgfqpoint{4.975703in}{0.483102in}}%
\pgfpathlineto{\pgfqpoint{4.978287in}{0.489947in}}%
\pgfpathlineto{\pgfqpoint{4.980967in}{0.489698in}}%
\pgfpathlineto{\pgfqpoint{4.983637in}{0.486519in}}%
\pgfpathlineto{\pgfqpoint{4.986325in}{0.489791in}}%
\pgfpathlineto{\pgfqpoint{4.989001in}{0.497726in}}%
\pgfpathlineto{\pgfqpoint{4.991683in}{0.499635in}}%
\pgfpathlineto{\pgfqpoint{4.994390in}{0.493537in}}%
\pgfpathlineto{\pgfqpoint{4.997028in}{0.489822in}}%
\pgfpathlineto{\pgfqpoint{4.999780in}{0.485648in}}%
\pgfpathlineto{\pgfqpoint{5.002384in}{0.490109in}}%
\pgfpathlineto{\pgfqpoint{5.005178in}{0.494058in}}%
\pgfpathlineto{\pgfqpoint{5.007751in}{0.481611in}}%
\pgfpathlineto{\pgfqpoint{5.010562in}{0.486688in}}%
\pgfpathlineto{\pgfqpoint{5.013104in}{0.486201in}}%
\pgfpathlineto{\pgfqpoint{5.015820in}{0.490024in}}%
\pgfpathlineto{\pgfqpoint{5.018466in}{0.489263in}}%
\pgfpathlineto{\pgfqpoint{5.021147in}{0.481615in}}%
\pgfpathlineto{\pgfqpoint{5.023927in}{0.487131in}}%
\pgfpathlineto{\pgfqpoint{5.026501in}{0.482407in}}%
\pgfpathlineto{\pgfqpoint{5.029275in}{0.484564in}}%
\pgfpathlineto{\pgfqpoint{5.031849in}{0.481120in}}%
\pgfpathlineto{\pgfqpoint{5.034649in}{0.497091in}}%
\pgfpathlineto{\pgfqpoint{5.037214in}{0.487760in}}%
\pgfpathlineto{\pgfqpoint{5.039962in}{0.491178in}}%
\pgfpathlineto{\pgfqpoint{5.042572in}{0.489643in}}%
\pgfpathlineto{\pgfqpoint{5.045249in}{0.490856in}}%
\pgfpathlineto{\pgfqpoint{5.047924in}{0.485073in}}%
\pgfpathlineto{\pgfqpoint{5.050606in}{0.489145in}}%
\pgfpathlineto{\pgfqpoint{5.053284in}{0.486693in}}%
\pgfpathlineto{\pgfqpoint{5.055952in}{0.485411in}}%
\pgfpathlineto{\pgfqpoint{5.058711in}{0.486226in}}%
\pgfpathlineto{\pgfqpoint{5.061315in}{0.485785in}}%
\pgfpathlineto{\pgfqpoint{5.064144in}{0.486715in}}%
\pgfpathlineto{\pgfqpoint{5.066677in}{0.483896in}}%
\pgfpathlineto{\pgfqpoint{5.069463in}{0.480997in}}%
\pgfpathlineto{\pgfqpoint{5.072030in}{0.481330in}}%
\pgfpathlineto{\pgfqpoint{5.074851in}{0.470802in}}%
\pgfpathlineto{\pgfqpoint{5.077390in}{0.468631in}}%
\pgfpathlineto{\pgfqpoint{5.080067in}{0.474158in}}%
\pgfpathlineto{\pgfqpoint{5.082746in}{0.476200in}}%
\pgfpathlineto{\pgfqpoint{5.085426in}{0.481172in}}%
\pgfpathlineto{\pgfqpoint{5.088103in}{0.480160in}}%
\pgfpathlineto{\pgfqpoint{5.090788in}{0.476378in}}%
\pgfpathlineto{\pgfqpoint{5.093579in}{0.480602in}}%
\pgfpathlineto{\pgfqpoint{5.096142in}{0.480376in}}%
\pgfpathlineto{\pgfqpoint{5.098948in}{0.478650in}}%
\pgfpathlineto{\pgfqpoint{5.101496in}{0.485598in}}%
\pgfpathlineto{\pgfqpoint{5.104312in}{0.484651in}}%
\pgfpathlineto{\pgfqpoint{5.106842in}{0.486279in}}%
\pgfpathlineto{\pgfqpoint{5.109530in}{0.482637in}}%
\pgfpathlineto{\pgfqpoint{5.112209in}{0.483581in}}%
\pgfpathlineto{\pgfqpoint{5.114887in}{0.485588in}}%
\pgfpathlineto{\pgfqpoint{5.117550in}{0.483767in}}%
\pgfpathlineto{\pgfqpoint{5.120243in}{0.487295in}}%
\pgfpathlineto{\pgfqpoint{5.123042in}{0.482364in}}%
\pgfpathlineto{\pgfqpoint{5.125599in}{0.480497in}}%
\pgfpathlineto{\pgfqpoint{5.128421in}{0.483201in}}%
\pgfpathlineto{\pgfqpoint{5.130953in}{0.486696in}}%
\pgfpathlineto{\pgfqpoint{5.133716in}{0.484448in}}%
\pgfpathlineto{\pgfqpoint{5.136311in}{0.482856in}}%
\pgfpathlineto{\pgfqpoint{5.139072in}{0.482261in}}%
\pgfpathlineto{\pgfqpoint{5.141660in}{0.481427in}}%
\pgfpathlineto{\pgfqpoint{5.144349in}{0.482268in}}%
\pgfpathlineto{\pgfqpoint{5.147029in}{0.483608in}}%
\pgfpathlineto{\pgfqpoint{5.149734in}{0.477510in}}%
\pgfpathlineto{\pgfqpoint{5.152382in}{0.473663in}}%
\pgfpathlineto{\pgfqpoint{5.155059in}{0.474933in}}%
\pgfpathlineto{\pgfqpoint{5.157815in}{0.477665in}}%
\pgfpathlineto{\pgfqpoint{5.160420in}{0.481557in}}%
\pgfpathlineto{\pgfqpoint{5.163243in}{0.479094in}}%
\pgfpathlineto{\pgfqpoint{5.165775in}{0.485712in}}%
\pgfpathlineto{\pgfqpoint{5.168591in}{0.486077in}}%
\pgfpathlineto{\pgfqpoint{5.171133in}{0.483253in}}%
\pgfpathlineto{\pgfqpoint{5.173925in}{0.484798in}}%
\pgfpathlineto{\pgfqpoint{5.176477in}{0.490195in}}%
\pgfpathlineto{\pgfqpoint{5.179188in}{0.491389in}}%
\pgfpathlineto{\pgfqpoint{5.181848in}{0.490401in}}%
\pgfpathlineto{\pgfqpoint{5.184522in}{0.488785in}}%
\pgfpathlineto{\pgfqpoint{5.187294in}{0.487506in}}%
\pgfpathlineto{\pgfqpoint{5.189880in}{0.483106in}}%
\pgfpathlineto{\pgfqpoint{5.192680in}{0.486471in}}%
\pgfpathlineto{\pgfqpoint{5.195239in}{0.487297in}}%
\pgfpathlineto{\pgfqpoint{5.198008in}{0.477704in}}%
\pgfpathlineto{\pgfqpoint{5.200594in}{0.478952in}}%
\pgfpathlineto{\pgfqpoint{5.203388in}{0.491540in}}%
\pgfpathlineto{\pgfqpoint{5.205952in}{0.494595in}}%
\pgfpathlineto{\pgfqpoint{5.208630in}{0.481331in}}%
\pgfpathlineto{\pgfqpoint{5.211299in}{0.484148in}}%
\pgfpathlineto{\pgfqpoint{5.214027in}{0.478329in}}%
\pgfpathlineto{\pgfqpoint{5.216667in}{0.479255in}}%
\pgfpathlineto{\pgfqpoint{5.219345in}{0.473781in}}%
\pgfpathlineto{\pgfqpoint{5.222151in}{0.474837in}}%
\pgfpathlineto{\pgfqpoint{5.224695in}{0.480503in}}%
\pgfpathlineto{\pgfqpoint{5.227470in}{0.481518in}}%
\pgfpathlineto{\pgfqpoint{5.230059in}{0.478028in}}%
\pgfpathlineto{\pgfqpoint{5.232855in}{0.481368in}}%
\pgfpathlineto{\pgfqpoint{5.235409in}{0.480440in}}%
\pgfpathlineto{\pgfqpoint{5.238173in}{0.484766in}}%
\pgfpathlineto{\pgfqpoint{5.240777in}{0.492771in}}%
\pgfpathlineto{\pgfqpoint{5.243445in}{0.491013in}}%
\pgfpathlineto{\pgfqpoint{5.246130in}{0.482335in}}%
\pgfpathlineto{\pgfqpoint{5.248816in}{0.477654in}}%
\pgfpathlineto{\pgfqpoint{5.251590in}{0.475403in}}%
\pgfpathlineto{\pgfqpoint{5.254236in}{0.475780in}}%
\pgfpathlineto{\pgfqpoint{5.256973in}{0.479895in}}%
\pgfpathlineto{\pgfqpoint{5.259511in}{0.487388in}}%
\pgfpathlineto{\pgfqpoint{5.262264in}{0.480237in}}%
\pgfpathlineto{\pgfqpoint{5.264876in}{0.486129in}}%
\pgfpathlineto{\pgfqpoint{5.267691in}{0.492032in}}%
\pgfpathlineto{\pgfqpoint{5.270238in}{0.492203in}}%
\pgfpathlineto{\pgfqpoint{5.272913in}{0.482013in}}%
\pgfpathlineto{\pgfqpoint{5.275589in}{0.487157in}}%
\pgfpathlineto{\pgfqpoint{5.278322in}{0.480892in}}%
\pgfpathlineto{\pgfqpoint{5.280947in}{0.485504in}}%
\pgfpathlineto{\pgfqpoint{5.283631in}{0.491068in}}%
\pgfpathlineto{\pgfqpoint{5.286436in}{0.494350in}}%
\pgfpathlineto{\pgfqpoint{5.288984in}{0.491342in}}%
\pgfpathlineto{\pgfqpoint{5.291794in}{0.497144in}}%
\pgfpathlineto{\pgfqpoint{5.294339in}{0.498808in}}%
\pgfpathlineto{\pgfqpoint{5.297140in}{0.496146in}}%
\pgfpathlineto{\pgfqpoint{5.299696in}{0.495733in}}%
\pgfpathlineto{\pgfqpoint{5.302443in}{0.492260in}}%
\pgfpathlineto{\pgfqpoint{5.305054in}{0.491339in}}%
\pgfpathlineto{\pgfqpoint{5.307731in}{0.489304in}}%
\pgfpathlineto{\pgfqpoint{5.310411in}{0.489602in}}%
\pgfpathlineto{\pgfqpoint{5.313089in}{0.486333in}}%
\pgfpathlineto{\pgfqpoint{5.315754in}{0.488922in}}%
\pgfpathlineto{\pgfqpoint{5.318430in}{0.484392in}}%
\pgfpathlineto{\pgfqpoint{5.321256in}{0.494184in}}%
\pgfpathlineto{\pgfqpoint{5.323802in}{0.509681in}}%
\pgfpathlineto{\pgfqpoint{5.326564in}{0.502672in}}%
\pgfpathlineto{\pgfqpoint{5.329159in}{0.487269in}}%
\pgfpathlineto{\pgfqpoint{5.331973in}{0.473283in}}%
\pgfpathlineto{\pgfqpoint{5.334510in}{0.480588in}}%
\pgfpathlineto{\pgfqpoint{5.337353in}{0.491468in}}%
\pgfpathlineto{\pgfqpoint{5.339872in}{0.485728in}}%
\pgfpathlineto{\pgfqpoint{5.342549in}{0.493666in}}%
\pgfpathlineto{\pgfqpoint{5.345224in}{0.479486in}}%
\pgfpathlineto{\pgfqpoint{5.347905in}{0.474661in}}%
\pgfpathlineto{\pgfqpoint{5.350723in}{0.466944in}}%
\pgfpathlineto{\pgfqpoint{5.353262in}{0.466634in}}%
\pgfpathlineto{\pgfqpoint{5.356056in}{0.468156in}}%
\pgfpathlineto{\pgfqpoint{5.358612in}{0.465847in}}%
\pgfpathlineto{\pgfqpoint{5.361370in}{0.469547in}}%
\pgfpathlineto{\pgfqpoint{5.363966in}{0.481307in}}%
\pgfpathlineto{\pgfqpoint{5.366727in}{0.486763in}}%
\pgfpathlineto{\pgfqpoint{5.369335in}{0.490480in}}%
\pgfpathlineto{\pgfqpoint{5.372013in}{0.493756in}}%
\pgfpathlineto{\pgfqpoint{5.374692in}{0.490552in}}%
\pgfpathlineto{\pgfqpoint{5.377370in}{0.493654in}}%
\pgfpathlineto{\pgfqpoint{5.380048in}{0.490290in}}%
\pgfpathlineto{\pgfqpoint{5.382725in}{0.491299in}}%
\pgfpathlineto{\pgfqpoint{5.385550in}{0.492734in}}%
\pgfpathlineto{\pgfqpoint{5.388083in}{0.493012in}}%
\pgfpathlineto{\pgfqpoint{5.390900in}{0.500604in}}%
\pgfpathlineto{\pgfqpoint{5.393441in}{0.493400in}}%
\pgfpathlineto{\pgfqpoint{5.396219in}{0.492251in}}%
\pgfpathlineto{\pgfqpoint{5.398784in}{0.496497in}}%
\pgfpathlineto{\pgfqpoint{5.401576in}{0.498159in}}%
\pgfpathlineto{\pgfqpoint{5.404154in}{0.495712in}}%
\pgfpathlineto{\pgfqpoint{5.406832in}{0.489121in}}%
\pgfpathlineto{\pgfqpoint{5.409507in}{0.489140in}}%
\pgfpathlineto{\pgfqpoint{5.412190in}{0.487523in}}%
\pgfpathlineto{\pgfqpoint{5.414954in}{0.488841in}}%
\pgfpathlineto{\pgfqpoint{5.417547in}{0.488158in}}%
\pgfpathlineto{\pgfqpoint{5.420304in}{0.485815in}}%
\pgfpathlineto{\pgfqpoint{5.422897in}{0.490029in}}%
\pgfpathlineto{\pgfqpoint{5.425661in}{0.491427in}}%
\pgfpathlineto{\pgfqpoint{5.428259in}{0.490495in}}%
\pgfpathlineto{\pgfqpoint{5.431015in}{0.494074in}}%
\pgfpathlineto{\pgfqpoint{5.433616in}{0.490485in}}%
\pgfpathlineto{\pgfqpoint{5.436295in}{0.493159in}}%
\pgfpathlineto{\pgfqpoint{5.438974in}{0.488258in}}%
\pgfpathlineto{\pgfqpoint{5.441698in}{0.496323in}}%
\pgfpathlineto{\pgfqpoint{5.444328in}{0.486795in}}%
\pgfpathlineto{\pgfqpoint{5.447021in}{0.486854in}}%
\pgfpathlineto{\pgfqpoint{5.449769in}{0.486326in}}%
\pgfpathlineto{\pgfqpoint{5.452365in}{0.483502in}}%
\pgfpathlineto{\pgfqpoint{5.455168in}{0.486768in}}%
\pgfpathlineto{\pgfqpoint{5.457721in}{0.485066in}}%
\pgfpathlineto{\pgfqpoint{5.460489in}{0.481889in}}%
\pgfpathlineto{\pgfqpoint{5.463079in}{0.485151in}}%
\pgfpathlineto{\pgfqpoint{5.465888in}{0.485897in}}%
\pgfpathlineto{\pgfqpoint{5.468425in}{0.486810in}}%
\pgfpathlineto{\pgfqpoint{5.471113in}{0.490473in}}%
\pgfpathlineto{\pgfqpoint{5.473792in}{0.485854in}}%
\pgfpathlineto{\pgfqpoint{5.476458in}{0.485753in}}%
\pgfpathlineto{\pgfqpoint{5.479152in}{0.489948in}}%
\pgfpathlineto{\pgfqpoint{5.481825in}{0.486330in}}%
\pgfpathlineto{\pgfqpoint{5.484641in}{0.488385in}}%
\pgfpathlineto{\pgfqpoint{5.487176in}{0.488010in}}%
\pgfpathlineto{\pgfqpoint{5.490000in}{0.489083in}}%
\pgfpathlineto{\pgfqpoint{5.492541in}{0.484004in}}%
\pgfpathlineto{\pgfqpoint{5.495346in}{0.487834in}}%
\pgfpathlineto{\pgfqpoint{5.497898in}{0.489239in}}%
\pgfpathlineto{\pgfqpoint{5.500687in}{0.485947in}}%
\pgfpathlineto{\pgfqpoint{5.503255in}{0.490291in}}%
\pgfpathlineto{\pgfqpoint{5.505933in}{0.483144in}}%
\pgfpathlineto{\pgfqpoint{5.508612in}{0.483862in}}%
\pgfpathlineto{\pgfqpoint{5.511290in}{0.485499in}}%
\pgfpathlineto{\pgfqpoint{5.514080in}{0.480792in}}%
\pgfpathlineto{\pgfqpoint{5.516646in}{0.476200in}}%
\pgfpathlineto{\pgfqpoint{5.519433in}{0.483597in}}%
\pgfpathlineto{\pgfqpoint{5.522003in}{0.482124in}}%
\pgfpathlineto{\pgfqpoint{5.524756in}{0.485367in}}%
\pgfpathlineto{\pgfqpoint{5.527360in}{0.485071in}}%
\pgfpathlineto{\pgfqpoint{5.530148in}{0.490102in}}%
\pgfpathlineto{\pgfqpoint{5.532717in}{0.484399in}}%
\pgfpathlineto{\pgfqpoint{5.535395in}{0.481419in}}%
\pgfpathlineto{\pgfqpoint{5.538074in}{0.490714in}}%
\pgfpathlineto{\pgfqpoint{5.540750in}{0.486795in}}%
\pgfpathlineto{\pgfqpoint{5.543421in}{0.488457in}}%
\pgfpathlineto{\pgfqpoint{5.546110in}{0.486687in}}%
\pgfpathlineto{\pgfqpoint{5.548921in}{0.486348in}}%
\pgfpathlineto{\pgfqpoint{5.551457in}{0.491941in}}%
\pgfpathlineto{\pgfqpoint{5.554198in}{0.493276in}}%
\pgfpathlineto{\pgfqpoint{5.556822in}{0.488604in}}%
\pgfpathlineto{\pgfqpoint{5.559612in}{0.482980in}}%
\pgfpathlineto{\pgfqpoint{5.562180in}{0.483038in}}%
\pgfpathlineto{\pgfqpoint{5.564940in}{0.480937in}}%
\pgfpathlineto{\pgfqpoint{5.567536in}{0.484664in}}%
\pgfpathlineto{\pgfqpoint{5.570215in}{0.481779in}}%
\pgfpathlineto{\pgfqpoint{5.572893in}{0.485848in}}%
\pgfpathlineto{\pgfqpoint{5.575596in}{0.480024in}}%
\pgfpathlineto{\pgfqpoint{5.578342in}{0.488727in}}%
\pgfpathlineto{\pgfqpoint{5.580914in}{0.485005in}}%
\pgfpathlineto{\pgfqpoint{5.583709in}{0.483599in}}%
\pgfpathlineto{\pgfqpoint{5.586269in}{0.488322in}}%
\pgfpathlineto{\pgfqpoint{5.589040in}{0.489505in}}%
\pgfpathlineto{\pgfqpoint{5.591641in}{0.490439in}}%
\pgfpathlineto{\pgfqpoint{5.594368in}{0.495132in}}%
\pgfpathlineto{\pgfqpoint{5.596999in}{0.490288in}}%
\pgfpathlineto{\pgfqpoint{5.599674in}{0.490073in}}%
\pgfpathlineto{\pgfqpoint{5.602352in}{0.489823in}}%
\pgfpathlineto{\pgfqpoint{5.605073in}{0.484503in}}%
\pgfpathlineto{\pgfqpoint{5.607698in}{0.474956in}}%
\pgfpathlineto{\pgfqpoint{5.610389in}{0.475619in}}%
\pgfpathlineto{\pgfqpoint{5.613235in}{0.488422in}}%
\pgfpathlineto{\pgfqpoint{5.615743in}{0.491154in}}%
\pgfpathlineto{\pgfqpoint{5.618526in}{0.496646in}}%
\pgfpathlineto{\pgfqpoint{5.621102in}{0.493270in}}%
\pgfpathlineto{\pgfqpoint{5.623868in}{0.496205in}}%
\pgfpathlineto{\pgfqpoint{5.626460in}{0.495342in}}%
\pgfpathlineto{\pgfqpoint{5.629232in}{0.494482in}}%
\pgfpathlineto{\pgfqpoint{5.631815in}{0.496972in}}%
\pgfpathlineto{\pgfqpoint{5.634496in}{0.499334in}}%
\pgfpathlineto{\pgfqpoint{5.637172in}{0.496061in}}%
\pgfpathlineto{\pgfqpoint{5.639852in}{0.495209in}}%
\pgfpathlineto{\pgfqpoint{5.642518in}{0.490000in}}%
\pgfpathlineto{\pgfqpoint{5.645243in}{0.486681in}}%
\pgfpathlineto{\pgfqpoint{5.648008in}{0.483764in}}%
\pgfpathlineto{\pgfqpoint{5.650563in}{0.489859in}}%
\pgfpathlineto{\pgfqpoint{5.653376in}{0.483656in}}%
\pgfpathlineto{\pgfqpoint{5.655919in}{0.482168in}}%
\pgfpathlineto{\pgfqpoint{5.658723in}{0.482510in}}%
\pgfpathlineto{\pgfqpoint{5.661273in}{0.475024in}}%
\pgfpathlineto{\pgfqpoint{5.664099in}{0.470249in}}%
\pgfpathlineto{\pgfqpoint{5.666632in}{0.479602in}}%
\pgfpathlineto{\pgfqpoint{5.669313in}{0.483908in}}%
\pgfpathlineto{\pgfqpoint{5.671991in}{0.477411in}}%
\pgfpathlineto{\pgfqpoint{5.674667in}{0.474109in}}%
\pgfpathlineto{\pgfqpoint{5.677486in}{0.483160in}}%
\pgfpathlineto{\pgfqpoint{5.680027in}{0.483538in}}%
\pgfpathlineto{\pgfqpoint{5.682836in}{0.483715in}}%
\pgfpathlineto{\pgfqpoint{5.685385in}{0.484833in}}%
\pgfpathlineto{\pgfqpoint{5.688159in}{0.483035in}}%
\pgfpathlineto{\pgfqpoint{5.690730in}{0.490327in}}%
\pgfpathlineto{\pgfqpoint{5.693473in}{0.489994in}}%
\pgfpathlineto{\pgfqpoint{5.696101in}{0.485369in}}%
\pgfpathlineto{\pgfqpoint{5.698775in}{0.481578in}}%
\pgfpathlineto{\pgfqpoint{5.701453in}{0.488473in}}%
\pgfpathlineto{\pgfqpoint{5.704130in}{0.490262in}}%
\pgfpathlineto{\pgfqpoint{5.706800in}{0.487887in}}%
\pgfpathlineto{\pgfqpoint{5.709490in}{0.489762in}}%
\pgfpathlineto{\pgfqpoint{5.712291in}{0.480127in}}%
\pgfpathlineto{\pgfqpoint{5.714834in}{0.476734in}}%
\pgfpathlineto{\pgfqpoint{5.717671in}{0.482668in}}%
\pgfpathlineto{\pgfqpoint{5.720201in}{0.484244in}}%
\pgfpathlineto{\pgfqpoint{5.722950in}{0.482067in}}%
\pgfpathlineto{\pgfqpoint{5.725548in}{0.482670in}}%
\pgfpathlineto{\pgfqpoint{5.728339in}{0.478931in}}%
\pgfpathlineto{\pgfqpoint{5.730919in}{0.487147in}}%
\pgfpathlineto{\pgfqpoint{5.733594in}{0.479805in}}%
\pgfpathlineto{\pgfqpoint{5.736276in}{0.485657in}}%
\pgfpathlineto{\pgfqpoint{5.738974in}{0.484063in}}%
\pgfpathlineto{\pgfqpoint{5.741745in}{0.486545in}}%
\pgfpathlineto{\pgfqpoint{5.744310in}{0.480868in}}%
\pgfpathlineto{\pgfqpoint{5.744310in}{0.413320in}}%
\pgfpathlineto{\pgfqpoint{5.744310in}{0.413320in}}%
\pgfpathlineto{\pgfqpoint{5.741745in}{0.413320in}}%
\pgfpathlineto{\pgfqpoint{5.738974in}{0.413320in}}%
\pgfpathlineto{\pgfqpoint{5.736276in}{0.413320in}}%
\pgfpathlineto{\pgfqpoint{5.733594in}{0.413320in}}%
\pgfpathlineto{\pgfqpoint{5.730919in}{0.413320in}}%
\pgfpathlineto{\pgfqpoint{5.728339in}{0.413320in}}%
\pgfpathlineto{\pgfqpoint{5.725548in}{0.413320in}}%
\pgfpathlineto{\pgfqpoint{5.722950in}{0.413320in}}%
\pgfpathlineto{\pgfqpoint{5.720201in}{0.413320in}}%
\pgfpathlineto{\pgfqpoint{5.717671in}{0.413320in}}%
\pgfpathlineto{\pgfqpoint{5.714834in}{0.413320in}}%
\pgfpathlineto{\pgfqpoint{5.712291in}{0.413320in}}%
\pgfpathlineto{\pgfqpoint{5.709490in}{0.413320in}}%
\pgfpathlineto{\pgfqpoint{5.706800in}{0.413320in}}%
\pgfpathlineto{\pgfqpoint{5.704130in}{0.413320in}}%
\pgfpathlineto{\pgfqpoint{5.701453in}{0.413320in}}%
\pgfpathlineto{\pgfqpoint{5.698775in}{0.413320in}}%
\pgfpathlineto{\pgfqpoint{5.696101in}{0.413320in}}%
\pgfpathlineto{\pgfqpoint{5.693473in}{0.413320in}}%
\pgfpathlineto{\pgfqpoint{5.690730in}{0.413320in}}%
\pgfpathlineto{\pgfqpoint{5.688159in}{0.413320in}}%
\pgfpathlineto{\pgfqpoint{5.685385in}{0.413320in}}%
\pgfpathlineto{\pgfqpoint{5.682836in}{0.413320in}}%
\pgfpathlineto{\pgfqpoint{5.680027in}{0.413320in}}%
\pgfpathlineto{\pgfqpoint{5.677486in}{0.413320in}}%
\pgfpathlineto{\pgfqpoint{5.674667in}{0.413320in}}%
\pgfpathlineto{\pgfqpoint{5.671991in}{0.413320in}}%
\pgfpathlineto{\pgfqpoint{5.669313in}{0.413320in}}%
\pgfpathlineto{\pgfqpoint{5.666632in}{0.413320in}}%
\pgfpathlineto{\pgfqpoint{5.664099in}{0.413320in}}%
\pgfpathlineto{\pgfqpoint{5.661273in}{0.413320in}}%
\pgfpathlineto{\pgfqpoint{5.658723in}{0.413320in}}%
\pgfpathlineto{\pgfqpoint{5.655919in}{0.413320in}}%
\pgfpathlineto{\pgfqpoint{5.653376in}{0.413320in}}%
\pgfpathlineto{\pgfqpoint{5.650563in}{0.413320in}}%
\pgfpathlineto{\pgfqpoint{5.648008in}{0.413320in}}%
\pgfpathlineto{\pgfqpoint{5.645243in}{0.413320in}}%
\pgfpathlineto{\pgfqpoint{5.642518in}{0.413320in}}%
\pgfpathlineto{\pgfqpoint{5.639852in}{0.413320in}}%
\pgfpathlineto{\pgfqpoint{5.637172in}{0.413320in}}%
\pgfpathlineto{\pgfqpoint{5.634496in}{0.413320in}}%
\pgfpathlineto{\pgfqpoint{5.631815in}{0.413320in}}%
\pgfpathlineto{\pgfqpoint{5.629232in}{0.413320in}}%
\pgfpathlineto{\pgfqpoint{5.626460in}{0.413320in}}%
\pgfpathlineto{\pgfqpoint{5.623868in}{0.413320in}}%
\pgfpathlineto{\pgfqpoint{5.621102in}{0.413320in}}%
\pgfpathlineto{\pgfqpoint{5.618526in}{0.413320in}}%
\pgfpathlineto{\pgfqpoint{5.615743in}{0.413320in}}%
\pgfpathlineto{\pgfqpoint{5.613235in}{0.413320in}}%
\pgfpathlineto{\pgfqpoint{5.610389in}{0.413320in}}%
\pgfpathlineto{\pgfqpoint{5.607698in}{0.413320in}}%
\pgfpathlineto{\pgfqpoint{5.605073in}{0.413320in}}%
\pgfpathlineto{\pgfqpoint{5.602352in}{0.413320in}}%
\pgfpathlineto{\pgfqpoint{5.599674in}{0.413320in}}%
\pgfpathlineto{\pgfqpoint{5.596999in}{0.413320in}}%
\pgfpathlineto{\pgfqpoint{5.594368in}{0.413320in}}%
\pgfpathlineto{\pgfqpoint{5.591641in}{0.413320in}}%
\pgfpathlineto{\pgfqpoint{5.589040in}{0.413320in}}%
\pgfpathlineto{\pgfqpoint{5.586269in}{0.413320in}}%
\pgfpathlineto{\pgfqpoint{5.583709in}{0.413320in}}%
\pgfpathlineto{\pgfqpoint{5.580914in}{0.413320in}}%
\pgfpathlineto{\pgfqpoint{5.578342in}{0.413320in}}%
\pgfpathlineto{\pgfqpoint{5.575596in}{0.413320in}}%
\pgfpathlineto{\pgfqpoint{5.572893in}{0.413320in}}%
\pgfpathlineto{\pgfqpoint{5.570215in}{0.413320in}}%
\pgfpathlineto{\pgfqpoint{5.567536in}{0.413320in}}%
\pgfpathlineto{\pgfqpoint{5.564940in}{0.413320in}}%
\pgfpathlineto{\pgfqpoint{5.562180in}{0.413320in}}%
\pgfpathlineto{\pgfqpoint{5.559612in}{0.413320in}}%
\pgfpathlineto{\pgfqpoint{5.556822in}{0.413320in}}%
\pgfpathlineto{\pgfqpoint{5.554198in}{0.413320in}}%
\pgfpathlineto{\pgfqpoint{5.551457in}{0.413320in}}%
\pgfpathlineto{\pgfqpoint{5.548921in}{0.413320in}}%
\pgfpathlineto{\pgfqpoint{5.546110in}{0.413320in}}%
\pgfpathlineto{\pgfqpoint{5.543421in}{0.413320in}}%
\pgfpathlineto{\pgfqpoint{5.540750in}{0.413320in}}%
\pgfpathlineto{\pgfqpoint{5.538074in}{0.413320in}}%
\pgfpathlineto{\pgfqpoint{5.535395in}{0.413320in}}%
\pgfpathlineto{\pgfqpoint{5.532717in}{0.413320in}}%
\pgfpathlineto{\pgfqpoint{5.530148in}{0.413320in}}%
\pgfpathlineto{\pgfqpoint{5.527360in}{0.413320in}}%
\pgfpathlineto{\pgfqpoint{5.524756in}{0.413320in}}%
\pgfpathlineto{\pgfqpoint{5.522003in}{0.413320in}}%
\pgfpathlineto{\pgfqpoint{5.519433in}{0.413320in}}%
\pgfpathlineto{\pgfqpoint{5.516646in}{0.413320in}}%
\pgfpathlineto{\pgfqpoint{5.514080in}{0.413320in}}%
\pgfpathlineto{\pgfqpoint{5.511290in}{0.413320in}}%
\pgfpathlineto{\pgfqpoint{5.508612in}{0.413320in}}%
\pgfpathlineto{\pgfqpoint{5.505933in}{0.413320in}}%
\pgfpathlineto{\pgfqpoint{5.503255in}{0.413320in}}%
\pgfpathlineto{\pgfqpoint{5.500687in}{0.413320in}}%
\pgfpathlineto{\pgfqpoint{5.497898in}{0.413320in}}%
\pgfpathlineto{\pgfqpoint{5.495346in}{0.413320in}}%
\pgfpathlineto{\pgfqpoint{5.492541in}{0.413320in}}%
\pgfpathlineto{\pgfqpoint{5.490000in}{0.413320in}}%
\pgfpathlineto{\pgfqpoint{5.487176in}{0.413320in}}%
\pgfpathlineto{\pgfqpoint{5.484641in}{0.413320in}}%
\pgfpathlineto{\pgfqpoint{5.481825in}{0.413320in}}%
\pgfpathlineto{\pgfqpoint{5.479152in}{0.413320in}}%
\pgfpathlineto{\pgfqpoint{5.476458in}{0.413320in}}%
\pgfpathlineto{\pgfqpoint{5.473792in}{0.413320in}}%
\pgfpathlineto{\pgfqpoint{5.471113in}{0.413320in}}%
\pgfpathlineto{\pgfqpoint{5.468425in}{0.413320in}}%
\pgfpathlineto{\pgfqpoint{5.465888in}{0.413320in}}%
\pgfpathlineto{\pgfqpoint{5.463079in}{0.413320in}}%
\pgfpathlineto{\pgfqpoint{5.460489in}{0.413320in}}%
\pgfpathlineto{\pgfqpoint{5.457721in}{0.413320in}}%
\pgfpathlineto{\pgfqpoint{5.455168in}{0.413320in}}%
\pgfpathlineto{\pgfqpoint{5.452365in}{0.413320in}}%
\pgfpathlineto{\pgfqpoint{5.449769in}{0.413320in}}%
\pgfpathlineto{\pgfqpoint{5.447021in}{0.413320in}}%
\pgfpathlineto{\pgfqpoint{5.444328in}{0.413320in}}%
\pgfpathlineto{\pgfqpoint{5.441698in}{0.413320in}}%
\pgfpathlineto{\pgfqpoint{5.438974in}{0.413320in}}%
\pgfpathlineto{\pgfqpoint{5.436295in}{0.413320in}}%
\pgfpathlineto{\pgfqpoint{5.433616in}{0.413320in}}%
\pgfpathlineto{\pgfqpoint{5.431015in}{0.413320in}}%
\pgfpathlineto{\pgfqpoint{5.428259in}{0.413320in}}%
\pgfpathlineto{\pgfqpoint{5.425661in}{0.413320in}}%
\pgfpathlineto{\pgfqpoint{5.422897in}{0.413320in}}%
\pgfpathlineto{\pgfqpoint{5.420304in}{0.413320in}}%
\pgfpathlineto{\pgfqpoint{5.417547in}{0.413320in}}%
\pgfpathlineto{\pgfqpoint{5.414954in}{0.413320in}}%
\pgfpathlineto{\pgfqpoint{5.412190in}{0.413320in}}%
\pgfpathlineto{\pgfqpoint{5.409507in}{0.413320in}}%
\pgfpathlineto{\pgfqpoint{5.406832in}{0.413320in}}%
\pgfpathlineto{\pgfqpoint{5.404154in}{0.413320in}}%
\pgfpathlineto{\pgfqpoint{5.401576in}{0.413320in}}%
\pgfpathlineto{\pgfqpoint{5.398784in}{0.413320in}}%
\pgfpathlineto{\pgfqpoint{5.396219in}{0.413320in}}%
\pgfpathlineto{\pgfqpoint{5.393441in}{0.413320in}}%
\pgfpathlineto{\pgfqpoint{5.390900in}{0.413320in}}%
\pgfpathlineto{\pgfqpoint{5.388083in}{0.413320in}}%
\pgfpathlineto{\pgfqpoint{5.385550in}{0.413320in}}%
\pgfpathlineto{\pgfqpoint{5.382725in}{0.413320in}}%
\pgfpathlineto{\pgfqpoint{5.380048in}{0.413320in}}%
\pgfpathlineto{\pgfqpoint{5.377370in}{0.413320in}}%
\pgfpathlineto{\pgfqpoint{5.374692in}{0.413320in}}%
\pgfpathlineto{\pgfqpoint{5.372013in}{0.413320in}}%
\pgfpathlineto{\pgfqpoint{5.369335in}{0.413320in}}%
\pgfpathlineto{\pgfqpoint{5.366727in}{0.413320in}}%
\pgfpathlineto{\pgfqpoint{5.363966in}{0.413320in}}%
\pgfpathlineto{\pgfqpoint{5.361370in}{0.413320in}}%
\pgfpathlineto{\pgfqpoint{5.358612in}{0.413320in}}%
\pgfpathlineto{\pgfqpoint{5.356056in}{0.413320in}}%
\pgfpathlineto{\pgfqpoint{5.353262in}{0.413320in}}%
\pgfpathlineto{\pgfqpoint{5.350723in}{0.413320in}}%
\pgfpathlineto{\pgfqpoint{5.347905in}{0.413320in}}%
\pgfpathlineto{\pgfqpoint{5.345224in}{0.413320in}}%
\pgfpathlineto{\pgfqpoint{5.342549in}{0.413320in}}%
\pgfpathlineto{\pgfqpoint{5.339872in}{0.413320in}}%
\pgfpathlineto{\pgfqpoint{5.337353in}{0.413320in}}%
\pgfpathlineto{\pgfqpoint{5.334510in}{0.413320in}}%
\pgfpathlineto{\pgfqpoint{5.331973in}{0.413320in}}%
\pgfpathlineto{\pgfqpoint{5.329159in}{0.413320in}}%
\pgfpathlineto{\pgfqpoint{5.326564in}{0.413320in}}%
\pgfpathlineto{\pgfqpoint{5.323802in}{0.413320in}}%
\pgfpathlineto{\pgfqpoint{5.321256in}{0.413320in}}%
\pgfpathlineto{\pgfqpoint{5.318430in}{0.413320in}}%
\pgfpathlineto{\pgfqpoint{5.315754in}{0.413320in}}%
\pgfpathlineto{\pgfqpoint{5.313089in}{0.413320in}}%
\pgfpathlineto{\pgfqpoint{5.310411in}{0.413320in}}%
\pgfpathlineto{\pgfqpoint{5.307731in}{0.413320in}}%
\pgfpathlineto{\pgfqpoint{5.305054in}{0.413320in}}%
\pgfpathlineto{\pgfqpoint{5.302443in}{0.413320in}}%
\pgfpathlineto{\pgfqpoint{5.299696in}{0.413320in}}%
\pgfpathlineto{\pgfqpoint{5.297140in}{0.413320in}}%
\pgfpathlineto{\pgfqpoint{5.294339in}{0.413320in}}%
\pgfpathlineto{\pgfqpoint{5.291794in}{0.413320in}}%
\pgfpathlineto{\pgfqpoint{5.288984in}{0.413320in}}%
\pgfpathlineto{\pgfqpoint{5.286436in}{0.413320in}}%
\pgfpathlineto{\pgfqpoint{5.283631in}{0.413320in}}%
\pgfpathlineto{\pgfqpoint{5.280947in}{0.413320in}}%
\pgfpathlineto{\pgfqpoint{5.278322in}{0.413320in}}%
\pgfpathlineto{\pgfqpoint{5.275589in}{0.413320in}}%
\pgfpathlineto{\pgfqpoint{5.272913in}{0.413320in}}%
\pgfpathlineto{\pgfqpoint{5.270238in}{0.413320in}}%
\pgfpathlineto{\pgfqpoint{5.267691in}{0.413320in}}%
\pgfpathlineto{\pgfqpoint{5.264876in}{0.413320in}}%
\pgfpathlineto{\pgfqpoint{5.262264in}{0.413320in}}%
\pgfpathlineto{\pgfqpoint{5.259511in}{0.413320in}}%
\pgfpathlineto{\pgfqpoint{5.256973in}{0.413320in}}%
\pgfpathlineto{\pgfqpoint{5.254236in}{0.413320in}}%
\pgfpathlineto{\pgfqpoint{5.251590in}{0.413320in}}%
\pgfpathlineto{\pgfqpoint{5.248816in}{0.413320in}}%
\pgfpathlineto{\pgfqpoint{5.246130in}{0.413320in}}%
\pgfpathlineto{\pgfqpoint{5.243445in}{0.413320in}}%
\pgfpathlineto{\pgfqpoint{5.240777in}{0.413320in}}%
\pgfpathlineto{\pgfqpoint{5.238173in}{0.413320in}}%
\pgfpathlineto{\pgfqpoint{5.235409in}{0.413320in}}%
\pgfpathlineto{\pgfqpoint{5.232855in}{0.413320in}}%
\pgfpathlineto{\pgfqpoint{5.230059in}{0.413320in}}%
\pgfpathlineto{\pgfqpoint{5.227470in}{0.413320in}}%
\pgfpathlineto{\pgfqpoint{5.224695in}{0.413320in}}%
\pgfpathlineto{\pgfqpoint{5.222151in}{0.413320in}}%
\pgfpathlineto{\pgfqpoint{5.219345in}{0.413320in}}%
\pgfpathlineto{\pgfqpoint{5.216667in}{0.413320in}}%
\pgfpathlineto{\pgfqpoint{5.214027in}{0.413320in}}%
\pgfpathlineto{\pgfqpoint{5.211299in}{0.413320in}}%
\pgfpathlineto{\pgfqpoint{5.208630in}{0.413320in}}%
\pgfpathlineto{\pgfqpoint{5.205952in}{0.413320in}}%
\pgfpathlineto{\pgfqpoint{5.203388in}{0.413320in}}%
\pgfpathlineto{\pgfqpoint{5.200594in}{0.413320in}}%
\pgfpathlineto{\pgfqpoint{5.198008in}{0.413320in}}%
\pgfpathlineto{\pgfqpoint{5.195239in}{0.413320in}}%
\pgfpathlineto{\pgfqpoint{5.192680in}{0.413320in}}%
\pgfpathlineto{\pgfqpoint{5.189880in}{0.413320in}}%
\pgfpathlineto{\pgfqpoint{5.187294in}{0.413320in}}%
\pgfpathlineto{\pgfqpoint{5.184522in}{0.413320in}}%
\pgfpathlineto{\pgfqpoint{5.181848in}{0.413320in}}%
\pgfpathlineto{\pgfqpoint{5.179188in}{0.413320in}}%
\pgfpathlineto{\pgfqpoint{5.176477in}{0.413320in}}%
\pgfpathlineto{\pgfqpoint{5.173925in}{0.413320in}}%
\pgfpathlineto{\pgfqpoint{5.171133in}{0.413320in}}%
\pgfpathlineto{\pgfqpoint{5.168591in}{0.413320in}}%
\pgfpathlineto{\pgfqpoint{5.165775in}{0.413320in}}%
\pgfpathlineto{\pgfqpoint{5.163243in}{0.413320in}}%
\pgfpathlineto{\pgfqpoint{5.160420in}{0.413320in}}%
\pgfpathlineto{\pgfqpoint{5.157815in}{0.413320in}}%
\pgfpathlineto{\pgfqpoint{5.155059in}{0.413320in}}%
\pgfpathlineto{\pgfqpoint{5.152382in}{0.413320in}}%
\pgfpathlineto{\pgfqpoint{5.149734in}{0.413320in}}%
\pgfpathlineto{\pgfqpoint{5.147029in}{0.413320in}}%
\pgfpathlineto{\pgfqpoint{5.144349in}{0.413320in}}%
\pgfpathlineto{\pgfqpoint{5.141660in}{0.413320in}}%
\pgfpathlineto{\pgfqpoint{5.139072in}{0.413320in}}%
\pgfpathlineto{\pgfqpoint{5.136311in}{0.413320in}}%
\pgfpathlineto{\pgfqpoint{5.133716in}{0.413320in}}%
\pgfpathlineto{\pgfqpoint{5.130953in}{0.413320in}}%
\pgfpathlineto{\pgfqpoint{5.128421in}{0.413320in}}%
\pgfpathlineto{\pgfqpoint{5.125599in}{0.413320in}}%
\pgfpathlineto{\pgfqpoint{5.123042in}{0.413320in}}%
\pgfpathlineto{\pgfqpoint{5.120243in}{0.413320in}}%
\pgfpathlineto{\pgfqpoint{5.117550in}{0.413320in}}%
\pgfpathlineto{\pgfqpoint{5.114887in}{0.413320in}}%
\pgfpathlineto{\pgfqpoint{5.112209in}{0.413320in}}%
\pgfpathlineto{\pgfqpoint{5.109530in}{0.413320in}}%
\pgfpathlineto{\pgfqpoint{5.106842in}{0.413320in}}%
\pgfpathlineto{\pgfqpoint{5.104312in}{0.413320in}}%
\pgfpathlineto{\pgfqpoint{5.101496in}{0.413320in}}%
\pgfpathlineto{\pgfqpoint{5.098948in}{0.413320in}}%
\pgfpathlineto{\pgfqpoint{5.096142in}{0.413320in}}%
\pgfpathlineto{\pgfqpoint{5.093579in}{0.413320in}}%
\pgfpathlineto{\pgfqpoint{5.090788in}{0.413320in}}%
\pgfpathlineto{\pgfqpoint{5.088103in}{0.413320in}}%
\pgfpathlineto{\pgfqpoint{5.085426in}{0.413320in}}%
\pgfpathlineto{\pgfqpoint{5.082746in}{0.413320in}}%
\pgfpathlineto{\pgfqpoint{5.080067in}{0.413320in}}%
\pgfpathlineto{\pgfqpoint{5.077390in}{0.413320in}}%
\pgfpathlineto{\pgfqpoint{5.074851in}{0.413320in}}%
\pgfpathlineto{\pgfqpoint{5.072030in}{0.413320in}}%
\pgfpathlineto{\pgfqpoint{5.069463in}{0.413320in}}%
\pgfpathlineto{\pgfqpoint{5.066677in}{0.413320in}}%
\pgfpathlineto{\pgfqpoint{5.064144in}{0.413320in}}%
\pgfpathlineto{\pgfqpoint{5.061315in}{0.413320in}}%
\pgfpathlineto{\pgfqpoint{5.058711in}{0.413320in}}%
\pgfpathlineto{\pgfqpoint{5.055952in}{0.413320in}}%
\pgfpathlineto{\pgfqpoint{5.053284in}{0.413320in}}%
\pgfpathlineto{\pgfqpoint{5.050606in}{0.413320in}}%
\pgfpathlineto{\pgfqpoint{5.047924in}{0.413320in}}%
\pgfpathlineto{\pgfqpoint{5.045249in}{0.413320in}}%
\pgfpathlineto{\pgfqpoint{5.042572in}{0.413320in}}%
\pgfpathlineto{\pgfqpoint{5.039962in}{0.413320in}}%
\pgfpathlineto{\pgfqpoint{5.037214in}{0.413320in}}%
\pgfpathlineto{\pgfqpoint{5.034649in}{0.413320in}}%
\pgfpathlineto{\pgfqpoint{5.031849in}{0.413320in}}%
\pgfpathlineto{\pgfqpoint{5.029275in}{0.413320in}}%
\pgfpathlineto{\pgfqpoint{5.026501in}{0.413320in}}%
\pgfpathlineto{\pgfqpoint{5.023927in}{0.413320in}}%
\pgfpathlineto{\pgfqpoint{5.021147in}{0.413320in}}%
\pgfpathlineto{\pgfqpoint{5.018466in}{0.413320in}}%
\pgfpathlineto{\pgfqpoint{5.015820in}{0.413320in}}%
\pgfpathlineto{\pgfqpoint{5.013104in}{0.413320in}}%
\pgfpathlineto{\pgfqpoint{5.010562in}{0.413320in}}%
\pgfpathlineto{\pgfqpoint{5.007751in}{0.413320in}}%
\pgfpathlineto{\pgfqpoint{5.005178in}{0.413320in}}%
\pgfpathlineto{\pgfqpoint{5.002384in}{0.413320in}}%
\pgfpathlineto{\pgfqpoint{4.999780in}{0.413320in}}%
\pgfpathlineto{\pgfqpoint{4.997028in}{0.413320in}}%
\pgfpathlineto{\pgfqpoint{4.994390in}{0.413320in}}%
\pgfpathlineto{\pgfqpoint{4.991683in}{0.413320in}}%
\pgfpathlineto{\pgfqpoint{4.989001in}{0.413320in}}%
\pgfpathlineto{\pgfqpoint{4.986325in}{0.413320in}}%
\pgfpathlineto{\pgfqpoint{4.983637in}{0.413320in}}%
\pgfpathlineto{\pgfqpoint{4.980967in}{0.413320in}}%
\pgfpathlineto{\pgfqpoint{4.978287in}{0.413320in}}%
\pgfpathlineto{\pgfqpoint{4.975703in}{0.413320in}}%
\pgfpathlineto{\pgfqpoint{4.972933in}{0.413320in}}%
\pgfpathlineto{\pgfqpoint{4.970314in}{0.413320in}}%
\pgfpathlineto{\pgfqpoint{4.967575in}{0.413320in}}%
\pgfpathlineto{\pgfqpoint{4.965002in}{0.413320in}}%
\pgfpathlineto{\pgfqpoint{4.962219in}{0.413320in}}%
\pgfpathlineto{\pgfqpoint{4.959689in}{0.413320in}}%
\pgfpathlineto{\pgfqpoint{4.956862in}{0.413320in}}%
\pgfpathlineto{\pgfqpoint{4.954182in}{0.413320in}}%
\pgfpathlineto{\pgfqpoint{4.951504in}{0.413320in}}%
\pgfpathlineto{\pgfqpoint{4.948827in}{0.413320in}}%
\pgfpathlineto{\pgfqpoint{4.946151in}{0.413320in}}%
\pgfpathlineto{\pgfqpoint{4.943466in}{0.413320in}}%
\pgfpathlineto{\pgfqpoint{4.940881in}{0.413320in}}%
\pgfpathlineto{\pgfqpoint{4.938112in}{0.413320in}}%
\pgfpathlineto{\pgfqpoint{4.935515in}{0.413320in}}%
\pgfpathlineto{\pgfqpoint{4.932742in}{0.413320in}}%
\pgfpathlineto{\pgfqpoint{4.930170in}{0.413320in}}%
\pgfpathlineto{\pgfqpoint{4.927400in}{0.413320in}}%
\pgfpathlineto{\pgfqpoint{4.924708in}{0.413320in}}%
\pgfpathlineto{\pgfqpoint{4.922041in}{0.413320in}}%
\pgfpathlineto{\pgfqpoint{4.919352in}{0.413320in}}%
\pgfpathlineto{\pgfqpoint{4.916681in}{0.413320in}}%
\pgfpathlineto{\pgfqpoint{4.914009in}{0.413320in}}%
\pgfpathlineto{\pgfqpoint{4.911435in}{0.413320in}}%
\pgfpathlineto{\pgfqpoint{4.908648in}{0.413320in}}%
\pgfpathlineto{\pgfqpoint{4.906096in}{0.413320in}}%
\pgfpathlineto{\pgfqpoint{4.903295in}{0.413320in}}%
\pgfpathlineto{\pgfqpoint{4.900712in}{0.413320in}}%
\pgfpathlineto{\pgfqpoint{4.897938in}{0.413320in}}%
\pgfpathlineto{\pgfqpoint{4.895399in}{0.413320in}}%
\pgfpathlineto{\pgfqpoint{4.892611in}{0.413320in}}%
\pgfpathlineto{\pgfqpoint{4.889902in}{0.413320in}}%
\pgfpathlineto{\pgfqpoint{4.887211in}{0.413320in}}%
\pgfpathlineto{\pgfqpoint{4.884540in}{0.413320in}}%
\pgfpathlineto{\pgfqpoint{4.881864in}{0.413320in}}%
\pgfpathlineto{\pgfqpoint{4.879180in}{0.413320in}}%
\pgfpathlineto{\pgfqpoint{4.876636in}{0.413320in}}%
\pgfpathlineto{\pgfqpoint{4.873832in}{0.413320in}}%
\pgfpathlineto{\pgfqpoint{4.871209in}{0.413320in}}%
\pgfpathlineto{\pgfqpoint{4.868474in}{0.413320in}}%
\pgfpathlineto{\pgfqpoint{4.865910in}{0.413320in}}%
\pgfpathlineto{\pgfqpoint{4.863116in}{0.413320in}}%
\pgfpathlineto{\pgfqpoint{4.860544in}{0.413320in}}%
\pgfpathlineto{\pgfqpoint{4.857807in}{0.413320in}}%
\pgfpathlineto{\pgfqpoint{4.855070in}{0.413320in}}%
\pgfpathlineto{\pgfqpoint{4.852404in}{0.413320in}}%
\pgfpathlineto{\pgfqpoint{4.849715in}{0.413320in}}%
\pgfpathlineto{\pgfqpoint{4.847127in}{0.413320in}}%
\pgfpathlineto{\pgfqpoint{4.844361in}{0.413320in}}%
\pgfpathlineto{\pgfqpoint{4.842380in}{0.413320in}}%
\pgfpathlineto{\pgfqpoint{4.839922in}{0.413320in}}%
\pgfpathlineto{\pgfqpoint{4.837992in}{0.413320in}}%
\pgfpathlineto{\pgfqpoint{4.833657in}{0.413320in}}%
\pgfpathlineto{\pgfqpoint{4.831045in}{0.413320in}}%
\pgfpathlineto{\pgfqpoint{4.828291in}{0.413320in}}%
\pgfpathlineto{\pgfqpoint{4.825619in}{0.413320in}}%
\pgfpathlineto{\pgfqpoint{4.822945in}{0.413320in}}%
\pgfpathlineto{\pgfqpoint{4.820265in}{0.413320in}}%
\pgfpathlineto{\pgfqpoint{4.817587in}{0.413320in}}%
\pgfpathlineto{\pgfqpoint{4.814907in}{0.413320in}}%
\pgfpathlineto{\pgfqpoint{4.812377in}{0.413320in}}%
\pgfpathlineto{\pgfqpoint{4.809538in}{0.413320in}}%
\pgfpathlineto{\pgfqpoint{4.807017in}{0.413320in}}%
\pgfpathlineto{\pgfqpoint{4.804193in}{0.413320in}}%
\pgfpathlineto{\pgfqpoint{4.801586in}{0.413320in}}%
\pgfpathlineto{\pgfqpoint{4.798830in}{0.413320in}}%
\pgfpathlineto{\pgfqpoint{4.796274in}{0.413320in}}%
\pgfpathlineto{\pgfqpoint{4.793512in}{0.413320in}}%
\pgfpathlineto{\pgfqpoint{4.790798in}{0.413320in}}%
\pgfpathlineto{\pgfqpoint{4.788116in}{0.413320in}}%
\pgfpathlineto{\pgfqpoint{4.785445in}{0.413320in}}%
\pgfpathlineto{\pgfqpoint{4.782872in}{0.413320in}}%
\pgfpathlineto{\pgfqpoint{4.780083in}{0.413320in}}%
\pgfpathlineto{\pgfqpoint{4.777535in}{0.413320in}}%
\pgfpathlineto{\pgfqpoint{4.774732in}{0.413320in}}%
\pgfpathlineto{\pgfqpoint{4.772198in}{0.413320in}}%
\pgfpathlineto{\pgfqpoint{4.769367in}{0.413320in}}%
\pgfpathlineto{\pgfqpoint{4.766783in}{0.413320in}}%
\pgfpathlineto{\pgfqpoint{4.764018in}{0.413320in}}%
\pgfpathlineto{\pgfqpoint{4.761337in}{0.413320in}}%
\pgfpathlineto{\pgfqpoint{4.758653in}{0.413320in}}%
\pgfpathlineto{\pgfqpoint{4.755983in}{0.413320in}}%
\pgfpathlineto{\pgfqpoint{4.753298in}{0.413320in}}%
\pgfpathlineto{\pgfqpoint{4.750627in}{0.413320in}}%
\pgfpathlineto{\pgfqpoint{4.748081in}{0.413320in}}%
\pgfpathlineto{\pgfqpoint{4.745256in}{0.413320in}}%
\pgfpathlineto{\pgfqpoint{4.742696in}{0.413320in}}%
\pgfpathlineto{\pgfqpoint{4.739912in}{0.413320in}}%
\pgfpathlineto{\pgfqpoint{4.737348in}{0.413320in}}%
\pgfpathlineto{\pgfqpoint{4.734552in}{0.413320in}}%
\pgfpathlineto{\pgfqpoint{4.731901in}{0.413320in}}%
\pgfpathlineto{\pgfqpoint{4.729233in}{0.413320in}}%
\pgfpathlineto{\pgfqpoint{4.726508in}{0.413320in}}%
\pgfpathlineto{\pgfqpoint{4.723873in}{0.413320in}}%
\pgfpathlineto{\pgfqpoint{4.721160in}{0.413320in}}%
\pgfpathlineto{\pgfqpoint{4.718486in}{0.413320in}}%
\pgfpathlineto{\pgfqpoint{4.715806in}{0.413320in}}%
\pgfpathlineto{\pgfqpoint{4.713275in}{0.413320in}}%
\pgfpathlineto{\pgfqpoint{4.710437in}{0.413320in}}%
\pgfpathlineto{\pgfqpoint{4.707824in}{0.413320in}}%
\pgfpathlineto{\pgfqpoint{4.705094in}{0.413320in}}%
\pgfpathlineto{\pgfqpoint{4.702517in}{0.413320in}}%
\pgfpathlineto{\pgfqpoint{4.699734in}{0.413320in}}%
\pgfpathlineto{\pgfqpoint{4.697170in}{0.413320in}}%
\pgfpathlineto{\pgfqpoint{4.694381in}{0.413320in}}%
\pgfpathlineto{\pgfqpoint{4.691694in}{0.413320in}}%
\pgfpathlineto{\pgfqpoint{4.689051in}{0.413320in}}%
\pgfpathlineto{\pgfqpoint{4.686337in}{0.413320in}}%
\pgfpathlineto{\pgfqpoint{4.683799in}{0.413320in}}%
\pgfpathlineto{\pgfqpoint{4.680988in}{0.413320in}}%
\pgfpathlineto{\pgfqpoint{4.678448in}{0.413320in}}%
\pgfpathlineto{\pgfqpoint{4.675619in}{0.413320in}}%
\pgfpathlineto{\pgfqpoint{4.673068in}{0.413320in}}%
\pgfpathlineto{\pgfqpoint{4.670261in}{0.413320in}}%
\pgfpathlineto{\pgfqpoint{4.667764in}{0.413320in}}%
\pgfpathlineto{\pgfqpoint{4.664923in}{0.413320in}}%
\pgfpathlineto{\pgfqpoint{4.662237in}{0.413320in}}%
\pgfpathlineto{\pgfqpoint{4.659590in}{0.413320in}}%
\pgfpathlineto{\pgfqpoint{4.656873in}{0.413320in}}%
\pgfpathlineto{\pgfqpoint{4.654203in}{0.413320in}}%
\pgfpathlineto{\pgfqpoint{4.651524in}{0.413320in}}%
\pgfpathlineto{\pgfqpoint{4.648922in}{0.413320in}}%
\pgfpathlineto{\pgfqpoint{4.646169in}{0.413320in}}%
\pgfpathlineto{\pgfqpoint{4.643628in}{0.413320in}}%
\pgfpathlineto{\pgfqpoint{4.640809in}{0.413320in}}%
\pgfpathlineto{\pgfqpoint{4.638204in}{0.413320in}}%
\pgfpathlineto{\pgfqpoint{4.635445in}{0.413320in}}%
\pgfpathlineto{\pgfqpoint{4.632902in}{0.413320in}}%
\pgfpathlineto{\pgfqpoint{4.630096in}{0.413320in}}%
\pgfpathlineto{\pgfqpoint{4.627411in}{0.413320in}}%
\pgfpathlineto{\pgfqpoint{4.624741in}{0.413320in}}%
\pgfpathlineto{\pgfqpoint{4.622056in}{0.413320in}}%
\pgfpathlineto{\pgfqpoint{4.619529in}{0.413320in}}%
\pgfpathlineto{\pgfqpoint{4.616702in}{0.413320in}}%
\pgfpathlineto{\pgfqpoint{4.614134in}{0.413320in}}%
\pgfpathlineto{\pgfqpoint{4.611350in}{0.413320in}}%
\pgfpathlineto{\pgfqpoint{4.608808in}{0.413320in}}%
\pgfpathlineto{\pgfqpoint{4.605990in}{0.413320in}}%
\pgfpathlineto{\pgfqpoint{4.603430in}{0.413320in}}%
\pgfpathlineto{\pgfqpoint{4.600633in}{0.413320in}}%
\pgfpathlineto{\pgfqpoint{4.597951in}{0.413320in}}%
\pgfpathlineto{\pgfqpoint{4.595281in}{0.413320in}}%
\pgfpathlineto{\pgfqpoint{4.592589in}{0.413320in}}%
\pgfpathlineto{\pgfqpoint{4.589920in}{0.413320in}}%
\pgfpathlineto{\pgfqpoint{4.587244in}{0.413320in}}%
\pgfpathlineto{\pgfqpoint{4.584672in}{0.413320in}}%
\pgfpathlineto{\pgfqpoint{4.581888in}{0.413320in}}%
\pgfpathlineto{\pgfqpoint{4.579305in}{0.413320in}}%
\pgfpathlineto{\pgfqpoint{4.576531in}{0.413320in}}%
\pgfpathlineto{\pgfqpoint{4.573947in}{0.413320in}}%
\pgfpathlineto{\pgfqpoint{4.571171in}{0.413320in}}%
\pgfpathlineto{\pgfqpoint{4.568612in}{0.413320in}}%
\pgfpathlineto{\pgfqpoint{4.565820in}{0.413320in}}%
\pgfpathlineto{\pgfqpoint{4.563125in}{0.413320in}}%
\pgfpathlineto{\pgfqpoint{4.560448in}{0.413320in}}%
\pgfpathlineto{\pgfqpoint{4.557777in}{0.413320in}}%
\pgfpathlineto{\pgfqpoint{4.555106in}{0.413320in}}%
\pgfpathlineto{\pgfqpoint{4.552425in}{0.413320in}}%
\pgfpathlineto{\pgfqpoint{4.549822in}{0.413320in}}%
\pgfpathlineto{\pgfqpoint{4.547064in}{0.413320in}}%
\pgfpathlineto{\pgfqpoint{4.544464in}{0.413320in}}%
\pgfpathlineto{\pgfqpoint{4.541711in}{0.413320in}}%
\pgfpathlineto{\pgfqpoint{4.539144in}{0.413320in}}%
\pgfpathlineto{\pgfqpoint{4.536400in}{0.413320in}}%
\pgfpathlineto{\pgfqpoint{4.533764in}{0.413320in}}%
\pgfpathlineto{\pgfqpoint{4.530990in}{0.413320in}}%
\pgfpathlineto{\pgfqpoint{4.528307in}{0.413320in}}%
\pgfpathlineto{\pgfqpoint{4.525640in}{0.413320in}}%
\pgfpathlineto{\pgfqpoint{4.522962in}{0.413320in}}%
\pgfpathlineto{\pgfqpoint{4.520345in}{0.413320in}}%
\pgfpathlineto{\pgfqpoint{4.517598in}{0.413320in}}%
\pgfpathlineto{\pgfqpoint{4.515080in}{0.413320in}}%
\pgfpathlineto{\pgfqpoint{4.512246in}{0.413320in}}%
\pgfpathlineto{\pgfqpoint{4.509643in}{0.413320in}}%
\pgfpathlineto{\pgfqpoint{4.506893in}{0.413320in}}%
\pgfpathlineto{\pgfqpoint{4.504305in}{0.413320in}}%
\pgfpathlineto{\pgfqpoint{4.501529in}{0.413320in}}%
\pgfpathlineto{\pgfqpoint{4.498850in}{0.413320in}}%
\pgfpathlineto{\pgfqpoint{4.496167in}{0.413320in}}%
\pgfpathlineto{\pgfqpoint{4.493492in}{0.413320in}}%
\pgfpathlineto{\pgfqpoint{4.490822in}{0.413320in}}%
\pgfpathlineto{\pgfqpoint{4.488130in}{0.413320in}}%
\pgfpathlineto{\pgfqpoint{4.485581in}{0.413320in}}%
\pgfpathlineto{\pgfqpoint{4.482778in}{0.413320in}}%
\pgfpathlineto{\pgfqpoint{4.480201in}{0.413320in}}%
\pgfpathlineto{\pgfqpoint{4.477430in}{0.413320in}}%
\pgfpathlineto{\pgfqpoint{4.474861in}{0.413320in}}%
\pgfpathlineto{\pgfqpoint{4.472059in}{0.413320in}}%
\pgfpathlineto{\pgfqpoint{4.469492in}{0.413320in}}%
\pgfpathlineto{\pgfqpoint{4.466717in}{0.413320in}}%
\pgfpathlineto{\pgfqpoint{4.464029in}{0.413320in}}%
\pgfpathlineto{\pgfqpoint{4.461367in}{0.413320in}}%
\pgfpathlineto{\pgfqpoint{4.458681in}{0.413320in}}%
\pgfpathlineto{\pgfqpoint{4.456138in}{0.413320in}}%
\pgfpathlineto{\pgfqpoint{4.453312in}{0.413320in}}%
\pgfpathlineto{\pgfqpoint{4.450767in}{0.413320in}}%
\pgfpathlineto{\pgfqpoint{4.447965in}{0.413320in}}%
\pgfpathlineto{\pgfqpoint{4.445423in}{0.413320in}}%
\pgfpathlineto{\pgfqpoint{4.442611in}{0.413320in}}%
\pgfpathlineto{\pgfqpoint{4.440041in}{0.413320in}}%
\pgfpathlineto{\pgfqpoint{4.437253in}{0.413320in}}%
\pgfpathlineto{\pgfqpoint{4.434569in}{0.413320in}}%
\pgfpathlineto{\pgfqpoint{4.431901in}{0.413320in}}%
\pgfpathlineto{\pgfqpoint{4.429220in}{0.413320in}}%
\pgfpathlineto{\pgfqpoint{4.426534in}{0.413320in}}%
\pgfpathlineto{\pgfqpoint{4.423863in}{0.413320in}}%
\pgfpathlineto{\pgfqpoint{4.421292in}{0.413320in}}%
\pgfpathlineto{\pgfqpoint{4.418506in}{0.413320in}}%
\pgfpathlineto{\pgfqpoint{4.415932in}{0.413320in}}%
\pgfpathlineto{\pgfqpoint{4.413149in}{0.413320in}}%
\pgfpathlineto{\pgfqpoint{4.410587in}{0.413320in}}%
\pgfpathlineto{\pgfqpoint{4.407788in}{0.413320in}}%
\pgfpathlineto{\pgfqpoint{4.405234in}{0.413320in}}%
\pgfpathlineto{\pgfqpoint{4.402468in}{0.413320in}}%
\pgfpathlineto{\pgfqpoint{4.399745in}{0.413320in}}%
\pgfpathlineto{\pgfqpoint{4.397076in}{0.413320in}}%
\pgfpathlineto{\pgfqpoint{4.394400in}{0.413320in}}%
\pgfpathlineto{\pgfqpoint{4.391721in}{0.413320in}}%
\pgfpathlineto{\pgfqpoint{4.389044in}{0.413320in}}%
\pgfpathlineto{\pgfqpoint{4.386431in}{0.413320in}}%
\pgfpathlineto{\pgfqpoint{4.383674in}{0.413320in}}%
\pgfpathlineto{\pgfqpoint{4.381097in}{0.413320in}}%
\pgfpathlineto{\pgfqpoint{4.378329in}{0.413320in}}%
\pgfpathlineto{\pgfqpoint{4.375761in}{0.413320in}}%
\pgfpathlineto{\pgfqpoint{4.372976in}{0.413320in}}%
\pgfpathlineto{\pgfqpoint{4.370437in}{0.413320in}}%
\pgfpathlineto{\pgfqpoint{4.367646in}{0.413320in}}%
\pgfpathlineto{\pgfqpoint{4.364936in}{0.413320in}}%
\pgfpathlineto{\pgfqpoint{4.362270in}{0.413320in}}%
\pgfpathlineto{\pgfqpoint{4.359582in}{0.413320in}}%
\pgfpathlineto{\pgfqpoint{4.357014in}{0.413320in}}%
\pgfpathlineto{\pgfqpoint{4.354224in}{0.413320in}}%
\pgfpathlineto{\pgfqpoint{4.351645in}{0.413320in}}%
\pgfpathlineto{\pgfqpoint{4.348868in}{0.413320in}}%
\pgfpathlineto{\pgfqpoint{4.346263in}{0.413320in}}%
\pgfpathlineto{\pgfqpoint{4.343510in}{0.413320in}}%
\pgfpathlineto{\pgfqpoint{4.340976in}{0.413320in}}%
\pgfpathlineto{\pgfqpoint{4.338154in}{0.413320in}}%
\pgfpathlineto{\pgfqpoint{4.335463in}{0.413320in}}%
\pgfpathlineto{\pgfqpoint{4.332796in}{0.413320in}}%
\pgfpathlineto{\pgfqpoint{4.330118in}{0.413320in}}%
\pgfpathlineto{\pgfqpoint{4.327440in}{0.413320in}}%
\pgfpathlineto{\pgfqpoint{4.324760in}{0.413320in}}%
\pgfpathlineto{\pgfqpoint{4.322181in}{0.413320in}}%
\pgfpathlineto{\pgfqpoint{4.319405in}{0.413320in}}%
\pgfpathlineto{\pgfqpoint{4.316856in}{0.413320in}}%
\pgfpathlineto{\pgfqpoint{4.314032in}{0.413320in}}%
\pgfpathlineto{\pgfqpoint{4.311494in}{0.413320in}}%
\pgfpathlineto{\pgfqpoint{4.308691in}{0.413320in}}%
\pgfpathlineto{\pgfqpoint{4.306118in}{0.413320in}}%
\pgfpathlineto{\pgfqpoint{4.303357in}{0.413320in}}%
\pgfpathlineto{\pgfqpoint{4.300656in}{0.413320in}}%
\pgfpathlineto{\pgfqpoint{4.297977in}{0.413320in}}%
\pgfpathlineto{\pgfqpoint{4.295299in}{0.413320in}}%
\pgfpathlineto{\pgfqpoint{4.292786in}{0.413320in}}%
\pgfpathlineto{\pgfqpoint{4.289936in}{0.413320in}}%
\pgfpathlineto{\pgfqpoint{4.287399in}{0.413320in}}%
\pgfpathlineto{\pgfqpoint{4.284586in}{0.413320in}}%
\pgfpathlineto{\pgfqpoint{4.282000in}{0.413320in}}%
\pgfpathlineto{\pgfqpoint{4.279212in}{0.413320in}}%
\pgfpathlineto{\pgfqpoint{4.276635in}{0.413320in}}%
\pgfpathlineto{\pgfqpoint{4.273874in}{0.413320in}}%
\pgfpathlineto{\pgfqpoint{4.271187in}{0.413320in}}%
\pgfpathlineto{\pgfqpoint{4.268590in}{0.413320in}}%
\pgfpathlineto{\pgfqpoint{4.265824in}{0.413320in}}%
\pgfpathlineto{\pgfqpoint{4.263157in}{0.413320in}}%
\pgfpathlineto{\pgfqpoint{4.260477in}{0.413320in}}%
\pgfpathlineto{\pgfqpoint{4.257958in}{0.413320in}}%
\pgfpathlineto{\pgfqpoint{4.255120in}{0.413320in}}%
\pgfpathlineto{\pgfqpoint{4.252581in}{0.413320in}}%
\pgfpathlineto{\pgfqpoint{4.249767in}{0.413320in}}%
\pgfpathlineto{\pgfqpoint{4.247225in}{0.413320in}}%
\pgfpathlineto{\pgfqpoint{4.244394in}{0.413320in}}%
\pgfpathlineto{\pgfqpoint{4.241900in}{0.413320in}}%
\pgfpathlineto{\pgfqpoint{4.239084in}{0.413320in}}%
\pgfpathlineto{\pgfqpoint{4.236375in}{0.413320in}}%
\pgfpathlineto{\pgfqpoint{4.233691in}{0.413320in}}%
\pgfpathlineto{\pgfqpoint{4.231013in}{0.413320in}}%
\pgfpathlineto{\pgfqpoint{4.228331in}{0.413320in}}%
\pgfpathlineto{\pgfqpoint{4.225654in}{0.413320in}}%
\pgfpathlineto{\pgfqpoint{4.223082in}{0.413320in}}%
\pgfpathlineto{\pgfqpoint{4.220304in}{0.413320in}}%
\pgfpathlineto{\pgfqpoint{4.217694in}{0.413320in}}%
\pgfpathlineto{\pgfqpoint{4.214948in}{0.413320in}}%
\pgfpathlineto{\pgfqpoint{4.212383in}{0.413320in}}%
\pgfpathlineto{\pgfqpoint{4.209597in}{0.413320in}}%
\pgfpathlineto{\pgfqpoint{4.207076in}{0.413320in}}%
\pgfpathlineto{\pgfqpoint{4.204240in}{0.413320in}}%
\pgfpathlineto{\pgfqpoint{4.201542in}{0.413320in}}%
\pgfpathlineto{\pgfqpoint{4.198878in}{0.413320in}}%
\pgfpathlineto{\pgfqpoint{4.196186in}{0.413320in}}%
\pgfpathlineto{\pgfqpoint{4.193638in}{0.413320in}}%
\pgfpathlineto{\pgfqpoint{4.190842in}{0.413320in}}%
\pgfpathlineto{\pgfqpoint{4.188318in}{0.413320in}}%
\pgfpathlineto{\pgfqpoint{4.185481in}{0.413320in}}%
\pgfpathlineto{\pgfqpoint{4.182899in}{0.413320in}}%
\pgfpathlineto{\pgfqpoint{4.180129in}{0.413320in}}%
\pgfpathlineto{\pgfqpoint{4.177593in}{0.413320in}}%
\pgfpathlineto{\pgfqpoint{4.174770in}{0.413320in}}%
\pgfpathlineto{\pgfqpoint{4.172093in}{0.413320in}}%
\pgfpathlineto{\pgfqpoint{4.169415in}{0.413320in}}%
\pgfpathlineto{\pgfqpoint{4.166737in}{0.413320in}}%
\pgfpathlineto{\pgfqpoint{4.164059in}{0.413320in}}%
\pgfpathlineto{\pgfqpoint{4.161380in}{0.413320in}}%
\pgfpathlineto{\pgfqpoint{4.158806in}{0.413320in}}%
\pgfpathlineto{\pgfqpoint{4.156016in}{0.413320in}}%
\pgfpathlineto{\pgfqpoint{4.153423in}{0.413320in}}%
\pgfpathlineto{\pgfqpoint{4.150665in}{0.413320in}}%
\pgfpathlineto{\pgfqpoint{4.148082in}{0.413320in}}%
\pgfpathlineto{\pgfqpoint{4.145310in}{0.413320in}}%
\pgfpathlineto{\pgfqpoint{4.142713in}{0.413320in}}%
\pgfpathlineto{\pgfqpoint{4.139963in}{0.413320in}}%
\pgfpathlineto{\pgfqpoint{4.137272in}{0.413320in}}%
\pgfpathlineto{\pgfqpoint{4.134615in}{0.413320in}}%
\pgfpathlineto{\pgfqpoint{4.131920in}{0.413320in}}%
\pgfpathlineto{\pgfqpoint{4.129349in}{0.413320in}}%
\pgfpathlineto{\pgfqpoint{4.126553in}{0.413320in}}%
\pgfpathlineto{\pgfqpoint{4.124019in}{0.413320in}}%
\pgfpathlineto{\pgfqpoint{4.121205in}{0.413320in}}%
\pgfpathlineto{\pgfqpoint{4.118554in}{0.413320in}}%
\pgfpathlineto{\pgfqpoint{4.115844in}{0.413320in}}%
\pgfpathlineto{\pgfqpoint{4.113252in}{0.413320in}}%
\pgfpathlineto{\pgfqpoint{4.110488in}{0.413320in}}%
\pgfpathlineto{\pgfqpoint{4.107814in}{0.413320in}}%
\pgfpathlineto{\pgfqpoint{4.105185in}{0.413320in}}%
\pgfpathlineto{\pgfqpoint{4.102456in}{0.413320in}}%
\pgfpathlineto{\pgfqpoint{4.099777in}{0.413320in}}%
\pgfpathlineto{\pgfqpoint{4.097092in}{0.413320in}}%
\pgfpathlineto{\pgfqpoint{4.094527in}{0.413320in}}%
\pgfpathlineto{\pgfqpoint{4.091729in}{0.413320in}}%
\pgfpathlineto{\pgfqpoint{4.089159in}{0.413320in}}%
\pgfpathlineto{\pgfqpoint{4.086385in}{0.413320in}}%
\pgfpathlineto{\pgfqpoint{4.083870in}{0.413320in}}%
\pgfpathlineto{\pgfqpoint{4.081018in}{0.413320in}}%
\pgfpathlineto{\pgfqpoint{4.078471in}{0.413320in}}%
\pgfpathlineto{\pgfqpoint{4.075705in}{0.413320in}}%
\pgfpathlineto{\pgfqpoint{4.072985in}{0.413320in}}%
\pgfpathlineto{\pgfqpoint{4.070313in}{0.413320in}}%
\pgfpathlineto{\pgfqpoint{4.067636in}{0.413320in}}%
\pgfpathlineto{\pgfqpoint{4.064957in}{0.413320in}}%
\pgfpathlineto{\pgfqpoint{4.062266in}{0.413320in}}%
\pgfpathlineto{\pgfqpoint{4.059702in}{0.413320in}}%
\pgfpathlineto{\pgfqpoint{4.056911in}{0.413320in}}%
\pgfpathlineto{\pgfqpoint{4.054326in}{0.413320in}}%
\pgfpathlineto{\pgfqpoint{4.051557in}{0.413320in}}%
\pgfpathlineto{\pgfqpoint{4.049006in}{0.413320in}}%
\pgfpathlineto{\pgfqpoint{4.046210in}{0.413320in}}%
\pgfpathlineto{\pgfqpoint{4.043667in}{0.413320in}}%
\pgfpathlineto{\pgfqpoint{4.040852in}{0.413320in}}%
\pgfpathlineto{\pgfqpoint{4.038174in}{0.413320in}}%
\pgfpathlineto{\pgfqpoint{4.035492in}{0.413320in}}%
\pgfpathlineto{\pgfqpoint{4.032817in}{0.413320in}}%
\pgfpathlineto{\pgfqpoint{4.030229in}{0.413320in}}%
\pgfpathlineto{\pgfqpoint{4.027447in}{0.413320in}}%
\pgfpathlineto{\pgfqpoint{4.024868in}{0.413320in}}%
\pgfpathlineto{\pgfqpoint{4.022097in}{0.413320in}}%
\pgfpathlineto{\pgfqpoint{4.019518in}{0.413320in}}%
\pgfpathlineto{\pgfqpoint{4.016744in}{0.413320in}}%
\pgfpathlineto{\pgfqpoint{4.014186in}{0.413320in}}%
\pgfpathlineto{\pgfqpoint{4.011394in}{0.413320in}}%
\pgfpathlineto{\pgfqpoint{4.008699in}{0.413320in}}%
\pgfpathlineto{\pgfqpoint{4.006034in}{0.413320in}}%
\pgfpathlineto{\pgfqpoint{4.003348in}{0.413320in}}%
\pgfpathlineto{\pgfqpoint{4.000674in}{0.413320in}}%
\pgfpathlineto{\pgfqpoint{3.997990in}{0.413320in}}%
\pgfpathlineto{\pgfqpoint{3.995417in}{0.413320in}}%
\pgfpathlineto{\pgfqpoint{3.992642in}{0.413320in}}%
\pgfpathlineto{\pgfqpoint{3.990055in}{0.413320in}}%
\pgfpathlineto{\pgfqpoint{3.987270in}{0.413320in}}%
\pgfpathlineto{\pgfqpoint{3.984714in}{0.413320in}}%
\pgfpathlineto{\pgfqpoint{3.981929in}{0.413320in}}%
\pgfpathlineto{\pgfqpoint{3.979389in}{0.413320in}}%
\pgfpathlineto{\pgfqpoint{3.976563in}{0.413320in}}%
\pgfpathlineto{\pgfqpoint{3.973885in}{0.413320in}}%
\pgfpathlineto{\pgfqpoint{3.971250in}{0.413320in}}%
\pgfpathlineto{\pgfqpoint{3.968523in}{0.413320in}}%
\pgfpathlineto{\pgfqpoint{3.966013in}{0.413320in}}%
\pgfpathlineto{\pgfqpoint{3.963176in}{0.413320in}}%
\pgfpathlineto{\pgfqpoint{3.960635in}{0.413320in}}%
\pgfpathlineto{\pgfqpoint{3.957823in}{0.413320in}}%
\pgfpathlineto{\pgfqpoint{3.955211in}{0.413320in}}%
\pgfpathlineto{\pgfqpoint{3.952464in}{0.413320in}}%
\pgfpathlineto{\pgfqpoint{3.949894in}{0.413320in}}%
\pgfpathlineto{\pgfqpoint{3.947101in}{0.413320in}}%
\pgfpathlineto{\pgfqpoint{3.944431in}{0.413320in}}%
\pgfpathlineto{\pgfqpoint{3.941778in}{0.413320in}}%
\pgfpathlineto{\pgfqpoint{3.939075in}{0.413320in}}%
\pgfpathlineto{\pgfqpoint{3.936395in}{0.413320in}}%
\pgfpathlineto{\pgfqpoint{3.933714in}{0.413320in}}%
\pgfpathlineto{\pgfqpoint{3.931202in}{0.413320in}}%
\pgfpathlineto{\pgfqpoint{3.928347in}{0.413320in}}%
\pgfpathlineto{\pgfqpoint{3.925778in}{0.413320in}}%
\pgfpathlineto{\pgfqpoint{3.923005in}{0.413320in}}%
\pgfpathlineto{\pgfqpoint{3.920412in}{0.413320in}}%
\pgfpathlineto{\pgfqpoint{3.917646in}{0.413320in}}%
\pgfpathlineto{\pgfqpoint{3.915107in}{0.413320in}}%
\pgfpathlineto{\pgfqpoint{3.912296in}{0.413320in}}%
\pgfpathlineto{\pgfqpoint{3.909602in}{0.413320in}}%
\pgfpathlineto{\pgfqpoint{3.906918in}{0.413320in}}%
\pgfpathlineto{\pgfqpoint{3.904252in}{0.413320in}}%
\pgfpathlineto{\pgfqpoint{3.901573in}{0.413320in}}%
\pgfpathlineto{\pgfqpoint{3.898891in}{0.413320in}}%
\pgfpathlineto{\pgfqpoint{3.896345in}{0.413320in}}%
\pgfpathlineto{\pgfqpoint{3.893541in}{0.413320in}}%
\pgfpathlineto{\pgfqpoint{3.890926in}{0.413320in}}%
\pgfpathlineto{\pgfqpoint{3.888188in}{0.413320in}}%
\pgfpathlineto{\pgfqpoint{3.885621in}{0.413320in}}%
\pgfpathlineto{\pgfqpoint{3.882850in}{0.413320in}}%
\pgfpathlineto{\pgfqpoint{3.880237in}{0.413320in}}%
\pgfpathlineto{\pgfqpoint{3.877466in}{0.413320in}}%
\pgfpathlineto{\pgfqpoint{3.874790in}{0.413320in}}%
\pgfpathlineto{\pgfqpoint{3.872114in}{0.413320in}}%
\pgfpathlineto{\pgfqpoint{3.869435in}{0.413320in}}%
\pgfpathlineto{\pgfqpoint{3.866815in}{0.413320in}}%
\pgfpathlineto{\pgfqpoint{3.864073in}{0.413320in}}%
\pgfpathlineto{\pgfqpoint{3.861561in}{0.413320in}}%
\pgfpathlineto{\pgfqpoint{3.858720in}{0.413320in}}%
\pgfpathlineto{\pgfqpoint{3.856100in}{0.413320in}}%
\pgfpathlineto{\pgfqpoint{3.853358in}{0.413320in}}%
\pgfpathlineto{\pgfqpoint{3.850814in}{0.413320in}}%
\pgfpathlineto{\pgfqpoint{3.848005in}{0.413320in}}%
\pgfpathlineto{\pgfqpoint{3.845329in}{0.413320in}}%
\pgfpathlineto{\pgfqpoint{3.842641in}{0.413320in}}%
\pgfpathlineto{\pgfqpoint{3.839960in}{0.413320in}}%
\pgfpathlineto{\pgfqpoint{3.837286in}{0.413320in}}%
\pgfpathlineto{\pgfqpoint{3.834616in}{0.413320in}}%
\pgfpathlineto{\pgfqpoint{3.832053in}{0.413320in}}%
\pgfpathlineto{\pgfqpoint{3.829252in}{0.413320in}}%
\pgfpathlineto{\pgfqpoint{3.826679in}{0.413320in}}%
\pgfpathlineto{\pgfqpoint{3.823903in}{0.413320in}}%
\pgfpathlineto{\pgfqpoint{3.821315in}{0.413320in}}%
\pgfpathlineto{\pgfqpoint{3.818546in}{0.413320in}}%
\pgfpathlineto{\pgfqpoint{3.815983in}{0.413320in}}%
\pgfpathlineto{\pgfqpoint{3.813172in}{0.413320in}}%
\pgfpathlineto{\pgfqpoint{3.810510in}{0.413320in}}%
\pgfpathlineto{\pgfqpoint{3.807832in}{0.413320in}}%
\pgfpathlineto{\pgfqpoint{3.805145in}{0.413320in}}%
\pgfpathlineto{\pgfqpoint{3.802569in}{0.413320in}}%
\pgfpathlineto{\pgfqpoint{3.799797in}{0.413320in}}%
\pgfpathlineto{\pgfqpoint{3.797265in}{0.413320in}}%
\pgfpathlineto{\pgfqpoint{3.794435in}{0.413320in}}%
\pgfpathlineto{\pgfqpoint{3.791897in}{0.413320in}}%
\pgfpathlineto{\pgfqpoint{3.789084in}{0.413320in}}%
\pgfpathlineto{\pgfqpoint{3.786504in}{0.413320in}}%
\pgfpathlineto{\pgfqpoint{3.783725in}{0.413320in}}%
\pgfpathlineto{\pgfqpoint{3.781046in}{0.413320in}}%
\pgfpathlineto{\pgfqpoint{3.778370in}{0.413320in}}%
\pgfpathlineto{\pgfqpoint{3.775691in}{0.413320in}}%
\pgfpathlineto{\pgfqpoint{3.773014in}{0.413320in}}%
\pgfpathlineto{\pgfqpoint{3.770323in}{0.413320in}}%
\pgfpathlineto{\pgfqpoint{3.767782in}{0.413320in}}%
\pgfpathlineto{\pgfqpoint{3.764966in}{0.413320in}}%
\pgfpathlineto{\pgfqpoint{3.762389in}{0.413320in}}%
\pgfpathlineto{\pgfqpoint{3.759622in}{0.413320in}}%
\pgfpathlineto{\pgfqpoint{3.757065in}{0.413320in}}%
\pgfpathlineto{\pgfqpoint{3.754265in}{0.413320in}}%
\pgfpathlineto{\pgfqpoint{3.751728in}{0.413320in}}%
\pgfpathlineto{\pgfqpoint{3.748903in}{0.413320in}}%
\pgfpathlineto{\pgfqpoint{3.746229in}{0.413320in}}%
\pgfpathlineto{\pgfqpoint{3.743548in}{0.413320in}}%
\pgfpathlineto{\pgfqpoint{3.740874in}{0.413320in}}%
\pgfpathlineto{\pgfqpoint{3.738194in}{0.413320in}}%
\pgfpathlineto{\pgfqpoint{3.735509in}{0.413320in}}%
\pgfpathlineto{\pgfqpoint{3.732950in}{0.413320in}}%
\pgfpathlineto{\pgfqpoint{3.730158in}{0.413320in}}%
\pgfpathlineto{\pgfqpoint{3.727581in}{0.413320in}}%
\pgfpathlineto{\pgfqpoint{3.724804in}{0.413320in}}%
\pgfpathlineto{\pgfqpoint{3.722228in}{0.413320in}}%
\pgfpathlineto{\pgfqpoint{3.719446in}{0.413320in}}%
\pgfpathlineto{\pgfqpoint{3.716875in}{0.413320in}}%
\pgfpathlineto{\pgfqpoint{3.714086in}{0.413320in}}%
\pgfpathlineto{\pgfqpoint{3.711410in}{0.413320in}}%
\pgfpathlineto{\pgfqpoint{3.708729in}{0.413320in}}%
\pgfpathlineto{\pgfqpoint{3.706053in}{0.413320in}}%
\pgfpathlineto{\pgfqpoint{3.703460in}{0.413320in}}%
\pgfpathlineto{\pgfqpoint{3.700684in}{0.413320in}}%
\pgfpathlineto{\pgfqpoint{3.698125in}{0.413320in}}%
\pgfpathlineto{\pgfqpoint{3.695331in}{0.413320in}}%
\pgfpathlineto{\pgfqpoint{3.692765in}{0.413320in}}%
\pgfpathlineto{\pgfqpoint{3.689983in}{0.413320in}}%
\pgfpathlineto{\pgfqpoint{3.687442in}{0.413320in}}%
\pgfpathlineto{\pgfqpoint{3.684620in}{0.413320in}}%
\pgfpathlineto{\pgfqpoint{3.681948in}{0.413320in}}%
\pgfpathlineto{\pgfqpoint{3.679273in}{0.413320in}}%
\pgfpathlineto{\pgfqpoint{3.676591in}{0.413320in}}%
\pgfpathlineto{\pgfqpoint{3.673911in}{0.413320in}}%
\pgfpathlineto{\pgfqpoint{3.671232in}{0.413320in}}%
\pgfpathlineto{\pgfqpoint{3.668665in}{0.413320in}}%
\pgfpathlineto{\pgfqpoint{3.665864in}{0.413320in}}%
\pgfpathlineto{\pgfqpoint{3.663276in}{0.413320in}}%
\pgfpathlineto{\pgfqpoint{3.660515in}{0.413320in}}%
\pgfpathlineto{\pgfqpoint{3.657917in}{0.413320in}}%
\pgfpathlineto{\pgfqpoint{3.655165in}{0.413320in}}%
\pgfpathlineto{\pgfqpoint{3.652628in}{0.413320in}}%
\pgfpathlineto{\pgfqpoint{3.649837in}{0.413320in}}%
\pgfpathlineto{\pgfqpoint{3.647130in}{0.413320in}}%
\pgfpathlineto{\pgfqpoint{3.644452in}{0.413320in}}%
\pgfpathlineto{\pgfqpoint{3.641773in}{0.413320in}}%
\pgfpathlineto{\pgfqpoint{3.639207in}{0.413320in}}%
\pgfpathlineto{\pgfqpoint{3.636413in}{0.413320in}}%
\pgfpathlineto{\pgfqpoint{3.633858in}{0.413320in}}%
\pgfpathlineto{\pgfqpoint{3.631058in}{0.413320in}}%
\pgfpathlineto{\pgfqpoint{3.628460in}{0.413320in}}%
\pgfpathlineto{\pgfqpoint{3.625689in}{0.413320in}}%
\pgfpathlineto{\pgfqpoint{3.623165in}{0.413320in}}%
\pgfpathlineto{\pgfqpoint{3.620345in}{0.413320in}}%
\pgfpathlineto{\pgfqpoint{3.617667in}{0.413320in}}%
\pgfpathlineto{\pgfqpoint{3.614982in}{0.413320in}}%
\pgfpathlineto{\pgfqpoint{3.612311in}{0.413320in}}%
\pgfpathlineto{\pgfqpoint{3.609632in}{0.413320in}}%
\pgfpathlineto{\pgfqpoint{3.606951in}{0.413320in}}%
\pgfpathlineto{\pgfqpoint{3.604387in}{0.413320in}}%
\pgfpathlineto{\pgfqpoint{3.601590in}{0.413320in}}%
\pgfpathlineto{\pgfqpoint{3.598998in}{0.413320in}}%
\pgfpathlineto{\pgfqpoint{3.596240in}{0.413320in}}%
\pgfpathlineto{\pgfqpoint{3.593620in}{0.413320in}}%
\pgfpathlineto{\pgfqpoint{3.590883in}{0.413320in}}%
\pgfpathlineto{\pgfqpoint{3.588258in}{0.413320in}}%
\pgfpathlineto{\pgfqpoint{3.585532in}{0.413320in}}%
\pgfpathlineto{\pgfqpoint{3.582851in}{0.413320in}}%
\pgfpathlineto{\pgfqpoint{3.580191in}{0.413320in}}%
\pgfpathlineto{\pgfqpoint{3.577487in}{0.413320in}}%
\pgfpathlineto{\pgfqpoint{3.574814in}{0.413320in}}%
\pgfpathlineto{\pgfqpoint{3.572126in}{0.413320in}}%
\pgfpathlineto{\pgfqpoint{3.569584in}{0.413320in}}%
\pgfpathlineto{\pgfqpoint{3.566774in}{0.413320in}}%
\pgfpathlineto{\pgfqpoint{3.564188in}{0.413320in}}%
\pgfpathlineto{\pgfqpoint{3.561420in}{0.413320in}}%
\pgfpathlineto{\pgfqpoint{3.558853in}{0.413320in}}%
\pgfpathlineto{\pgfqpoint{3.556061in}{0.413320in}}%
\pgfpathlineto{\pgfqpoint{3.553498in}{0.413320in}}%
\pgfpathlineto{\pgfqpoint{3.550713in}{0.413320in}}%
\pgfpathlineto{\pgfqpoint{3.548029in}{0.413320in}}%
\pgfpathlineto{\pgfqpoint{3.545349in}{0.413320in}}%
\pgfpathlineto{\pgfqpoint{3.542656in}{0.413320in}}%
\pgfpathlineto{\pgfqpoint{3.540093in}{0.413320in}}%
\pgfpathlineto{\pgfqpoint{3.537309in}{0.413320in}}%
\pgfpathlineto{\pgfqpoint{3.534783in}{0.413320in}}%
\pgfpathlineto{\pgfqpoint{3.531955in}{0.413320in}}%
\pgfpathlineto{\pgfqpoint{3.529327in}{0.413320in}}%
\pgfpathlineto{\pgfqpoint{3.526601in}{0.413320in}}%
\pgfpathlineto{\pgfqpoint{3.524041in}{0.413320in}}%
\pgfpathlineto{\pgfqpoint{3.521244in}{0.413320in}}%
\pgfpathlineto{\pgfqpoint{3.518565in}{0.413320in}}%
\pgfpathlineto{\pgfqpoint{3.515884in}{0.413320in}}%
\pgfpathlineto{\pgfqpoint{3.513209in}{0.413320in}}%
\pgfpathlineto{\pgfqpoint{3.510533in}{0.413320in}}%
\pgfpathlineto{\pgfqpoint{3.507840in}{0.413320in}}%
\pgfpathlineto{\pgfqpoint{3.505262in}{0.413320in}}%
\pgfpathlineto{\pgfqpoint{3.502488in}{0.413320in}}%
\pgfpathlineto{\pgfqpoint{3.499909in}{0.413320in}}%
\pgfpathlineto{\pgfqpoint{3.497139in}{0.413320in}}%
\pgfpathlineto{\pgfqpoint{3.494581in}{0.413320in}}%
\pgfpathlineto{\pgfqpoint{3.491783in}{0.413320in}}%
\pgfpathlineto{\pgfqpoint{3.489223in}{0.413320in}}%
\pgfpathlineto{\pgfqpoint{3.486442in}{0.413320in}}%
\pgfpathlineto{\pgfqpoint{3.483744in}{0.413320in}}%
\pgfpathlineto{\pgfqpoint{3.481072in}{0.413320in}}%
\pgfpathlineto{\pgfqpoint{3.478378in}{0.413320in}}%
\pgfpathlineto{\pgfqpoint{3.475821in}{0.413320in}}%
\pgfpathlineto{\pgfqpoint{3.473021in}{0.413320in}}%
\pgfpathlineto{\pgfqpoint{3.470466in}{0.413320in}}%
\pgfpathlineto{\pgfqpoint{3.467678in}{0.413320in}}%
\pgfpathlineto{\pgfqpoint{3.465072in}{0.413320in}}%
\pgfpathlineto{\pgfqpoint{3.462321in}{0.413320in}}%
\pgfpathlineto{\pgfqpoint{3.459695in}{0.413320in}}%
\pgfpathlineto{\pgfqpoint{3.456960in}{0.413320in}}%
\pgfpathlineto{\pgfqpoint{3.454285in}{0.413320in}}%
\pgfpathlineto{\pgfqpoint{3.451597in}{0.413320in}}%
\pgfpathlineto{\pgfqpoint{3.448926in}{0.413320in}}%
\pgfpathlineto{\pgfqpoint{3.446257in}{0.413320in}}%
\pgfpathlineto{\pgfqpoint{3.443574in}{0.413320in}}%
\pgfpathlineto{\pgfqpoint{3.440996in}{0.413320in}}%
\pgfpathlineto{\pgfqpoint{3.438210in}{0.413320in}}%
\pgfpathlineto{\pgfqpoint{3.435635in}{0.413320in}}%
\pgfpathlineto{\pgfqpoint{3.432851in}{0.413320in}}%
\pgfpathlineto{\pgfqpoint{3.430313in}{0.413320in}}%
\pgfpathlineto{\pgfqpoint{3.427501in}{0.413320in}}%
\pgfpathlineto{\pgfqpoint{3.424887in}{0.413320in}}%
\pgfpathlineto{\pgfqpoint{3.422142in}{0.413320in}}%
\pgfpathlineto{\pgfqpoint{3.419455in}{0.413320in}}%
\pgfpathlineto{\pgfqpoint{3.416780in}{0.413320in}}%
\pgfpathlineto{\pgfqpoint{3.414109in}{0.413320in}}%
\pgfpathlineto{\pgfqpoint{3.411431in}{0.413320in}}%
\pgfpathlineto{\pgfqpoint{3.408752in}{0.413320in}}%
\pgfpathlineto{\pgfqpoint{3.406202in}{0.413320in}}%
\pgfpathlineto{\pgfqpoint{3.403394in}{0.413320in}}%
\pgfpathlineto{\pgfqpoint{3.400783in}{0.413320in}}%
\pgfpathlineto{\pgfqpoint{3.398037in}{0.413320in}}%
\pgfpathlineto{\pgfqpoint{3.395461in}{0.413320in}}%
\pgfpathlineto{\pgfqpoint{3.392681in}{0.413320in}}%
\pgfpathlineto{\pgfqpoint{3.390102in}{0.413320in}}%
\pgfpathlineto{\pgfqpoint{3.387309in}{0.413320in}}%
\pgfpathlineto{\pgfqpoint{3.384647in}{0.413320in}}%
\pgfpathlineto{\pgfqpoint{3.381959in}{0.413320in}}%
\pgfpathlineto{\pgfqpoint{3.379290in}{0.413320in}}%
\pgfpathlineto{\pgfqpoint{3.376735in}{0.413320in}}%
\pgfpathlineto{\pgfqpoint{3.373921in}{0.413320in}}%
\pgfpathlineto{\pgfqpoint{3.371357in}{0.413320in}}%
\pgfpathlineto{\pgfqpoint{3.368577in}{0.413320in}}%
\pgfpathlineto{\pgfqpoint{3.365996in}{0.413320in}}%
\pgfpathlineto{\pgfqpoint{3.363221in}{0.413320in}}%
\pgfpathlineto{\pgfqpoint{3.360620in}{0.413320in}}%
\pgfpathlineto{\pgfqpoint{3.357862in}{0.413320in}}%
\pgfpathlineto{\pgfqpoint{3.355177in}{0.413320in}}%
\pgfpathlineto{\pgfqpoint{3.352505in}{0.413320in}}%
\pgfpathlineto{\pgfqpoint{3.349828in}{0.413320in}}%
\pgfpathlineto{\pgfqpoint{3.347139in}{0.413320in}}%
\pgfpathlineto{\pgfqpoint{3.344468in}{0.413320in}}%
\pgfpathlineto{\pgfqpoint{3.341893in}{0.413320in}}%
\pgfpathlineto{\pgfqpoint{3.339101in}{0.413320in}}%
\pgfpathlineto{\pgfqpoint{3.336541in}{0.413320in}}%
\pgfpathlineto{\pgfqpoint{3.333758in}{0.413320in}}%
\pgfpathlineto{\pgfqpoint{3.331183in}{0.413320in}}%
\pgfpathlineto{\pgfqpoint{3.328401in}{0.413320in}}%
\pgfpathlineto{\pgfqpoint{3.325860in}{0.413320in}}%
\pgfpathlineto{\pgfqpoint{3.323049in}{0.413320in}}%
\pgfpathlineto{\pgfqpoint{3.320366in}{0.413320in}}%
\pgfpathlineto{\pgfqpoint{3.317688in}{0.413320in}}%
\pgfpathlineto{\pgfqpoint{3.315008in}{0.413320in}}%
\pgfpathlineto{\pgfqpoint{3.312480in}{0.413320in}}%
\pgfpathlineto{\pgfqpoint{3.309652in}{0.413320in}}%
\pgfpathlineto{\pgfqpoint{3.307104in}{0.413320in}}%
\pgfpathlineto{\pgfqpoint{3.304295in}{0.413320in}}%
\pgfpathlineto{\pgfqpoint{3.301719in}{0.413320in}}%
\pgfpathlineto{\pgfqpoint{3.298937in}{0.413320in}}%
\pgfpathlineto{\pgfqpoint{3.296376in}{0.413320in}}%
\pgfpathlineto{\pgfqpoint{3.293574in}{0.413320in}}%
\pgfpathlineto{\pgfqpoint{3.290890in}{0.413320in}}%
\pgfpathlineto{\pgfqpoint{3.288225in}{0.413320in}}%
\pgfpathlineto{\pgfqpoint{3.285534in}{0.413320in}}%
\pgfpathlineto{\pgfqpoint{3.282870in}{0.413320in}}%
\pgfpathlineto{\pgfqpoint{3.280189in}{0.413320in}}%
\pgfpathlineto{\pgfqpoint{3.277603in}{0.413320in}}%
\pgfpathlineto{\pgfqpoint{3.274831in}{0.413320in}}%
\pgfpathlineto{\pgfqpoint{3.272254in}{0.413320in}}%
\pgfpathlineto{\pgfqpoint{3.269478in}{0.413320in}}%
\pgfpathlineto{\pgfqpoint{3.266849in}{0.413320in}}%
\pgfpathlineto{\pgfqpoint{3.264119in}{0.413320in}}%
\pgfpathlineto{\pgfqpoint{3.261594in}{0.413320in}}%
\pgfpathlineto{\pgfqpoint{3.258784in}{0.413320in}}%
\pgfpathlineto{\pgfqpoint{3.256083in}{0.413320in}}%
\pgfpathlineto{\pgfqpoint{3.253404in}{0.413320in}}%
\pgfpathlineto{\pgfqpoint{3.250716in}{0.413320in}}%
\pgfpathlineto{\pgfqpoint{3.248049in}{0.413320in}}%
\pgfpathlineto{\pgfqpoint{3.245363in}{0.413320in}}%
\pgfpathlineto{\pgfqpoint{3.242807in}{0.413320in}}%
\pgfpathlineto{\pgfqpoint{3.240010in}{0.413320in}}%
\pgfpathlineto{\pgfqpoint{3.237411in}{0.413320in}}%
\pgfpathlineto{\pgfqpoint{3.234658in}{0.413320in}}%
\pgfpathlineto{\pgfqpoint{3.232069in}{0.413320in}}%
\pgfpathlineto{\pgfqpoint{3.229310in}{0.413320in}}%
\pgfpathlineto{\pgfqpoint{3.226609in}{0.413320in}}%
\pgfpathlineto{\pgfqpoint{3.223942in}{0.413320in}}%
\pgfpathlineto{\pgfqpoint{3.221255in}{0.413320in}}%
\pgfpathlineto{\pgfqpoint{3.218586in}{0.413320in}}%
\pgfpathlineto{\pgfqpoint{3.215908in}{0.413320in}}%
\pgfpathlineto{\pgfqpoint{3.213342in}{0.413320in}}%
\pgfpathlineto{\pgfqpoint{3.210545in}{0.413320in}}%
\pgfpathlineto{\pgfqpoint{3.207984in}{0.413320in}}%
\pgfpathlineto{\pgfqpoint{3.205195in}{0.413320in}}%
\pgfpathlineto{\pgfqpoint{3.202562in}{0.413320in}}%
\pgfpathlineto{\pgfqpoint{3.199823in}{0.413320in}}%
\pgfpathlineto{\pgfqpoint{3.197226in}{0.413320in}}%
\pgfpathlineto{\pgfqpoint{3.194508in}{0.413320in}}%
\pgfpathlineto{\pgfqpoint{3.191796in}{0.413320in}}%
\pgfpathlineto{\pgfqpoint{3.189117in}{0.413320in}}%
\pgfpathlineto{\pgfqpoint{3.186440in}{0.413320in}}%
\pgfpathlineto{\pgfqpoint{3.183760in}{0.413320in}}%
\pgfpathlineto{\pgfqpoint{3.181089in}{0.413320in}}%
\pgfpathlineto{\pgfqpoint{3.178525in}{0.413320in}}%
\pgfpathlineto{\pgfqpoint{3.175724in}{0.413320in}}%
\pgfpathlineto{\pgfqpoint{3.173142in}{0.413320in}}%
\pgfpathlineto{\pgfqpoint{3.170375in}{0.413320in}}%
\pgfpathlineto{\pgfqpoint{3.167776in}{0.413320in}}%
\pgfpathlineto{\pgfqpoint{3.165019in}{0.413320in}}%
\pgfpathlineto{\pgfqpoint{3.162474in}{0.413320in}}%
\pgfpathlineto{\pgfqpoint{3.159675in}{0.413320in}}%
\pgfpathlineto{\pgfqpoint{3.156981in}{0.413320in}}%
\pgfpathlineto{\pgfqpoint{3.154327in}{0.413320in}}%
\pgfpathlineto{\pgfqpoint{3.151612in}{0.413320in}}%
\pgfpathlineto{\pgfqpoint{3.149057in}{0.413320in}}%
\pgfpathlineto{\pgfqpoint{3.146271in}{0.413320in}}%
\pgfpathlineto{\pgfqpoint{3.143740in}{0.413320in}}%
\pgfpathlineto{\pgfqpoint{3.140913in}{0.413320in}}%
\pgfpathlineto{\pgfqpoint{3.138375in}{0.413320in}}%
\pgfpathlineto{\pgfqpoint{3.135550in}{0.413320in}}%
\pgfpathlineto{\pgfqpoint{3.132946in}{0.413320in}}%
\pgfpathlineto{\pgfqpoint{3.130199in}{0.413320in}}%
\pgfpathlineto{\pgfqpoint{3.127512in}{0.413320in}}%
\pgfpathlineto{\pgfqpoint{3.124842in}{0.413320in}}%
\pgfpathlineto{\pgfqpoint{3.122163in}{0.413320in}}%
\pgfpathlineto{\pgfqpoint{3.119487in}{0.413320in}}%
\pgfpathlineto{\pgfqpoint{3.116807in}{0.413320in}}%
\pgfpathlineto{\pgfqpoint{3.114242in}{0.413320in}}%
\pgfpathlineto{\pgfqpoint{3.111451in}{0.413320in}}%
\pgfpathlineto{\pgfqpoint{3.108896in}{0.413320in}}%
\pgfpathlineto{\pgfqpoint{3.106094in}{0.413320in}}%
\pgfpathlineto{\pgfqpoint{3.103508in}{0.413320in}}%
\pgfpathlineto{\pgfqpoint{3.100737in}{0.413320in}}%
\pgfpathlineto{\pgfqpoint{3.098163in}{0.413320in}}%
\pgfpathlineto{\pgfqpoint{3.095388in}{0.413320in}}%
\pgfpathlineto{\pgfqpoint{3.092699in}{0.413320in}}%
\pgfpathlineto{\pgfqpoint{3.090023in}{0.413320in}}%
\pgfpathlineto{\pgfqpoint{3.087343in}{0.413320in}}%
\pgfpathlineto{\pgfqpoint{3.084671in}{0.413320in}}%
\pgfpathlineto{\pgfqpoint{3.081990in}{0.413320in}}%
\pgfpathlineto{\pgfqpoint{3.079381in}{0.413320in}}%
\pgfpathlineto{\pgfqpoint{3.076631in}{0.413320in}}%
\pgfpathlineto{\pgfqpoint{3.074056in}{0.413320in}}%
\pgfpathlineto{\pgfqpoint{3.071266in}{0.413320in}}%
\pgfpathlineto{\pgfqpoint{3.068709in}{0.413320in}}%
\pgfpathlineto{\pgfqpoint{3.065916in}{0.413320in}}%
\pgfpathlineto{\pgfqpoint{3.063230in}{0.413320in}}%
\pgfpathlineto{\pgfqpoint{3.060561in}{0.413320in}}%
\pgfpathlineto{\pgfqpoint{3.057884in}{0.413320in}}%
\pgfpathlineto{\pgfqpoint{3.055202in}{0.413320in}}%
\pgfpathlineto{\pgfqpoint{3.052526in}{0.413320in}}%
\pgfpathlineto{\pgfqpoint{3.049988in}{0.413320in}}%
\pgfpathlineto{\pgfqpoint{3.047157in}{0.413320in}}%
\pgfpathlineto{\pgfqpoint{3.044568in}{0.413320in}}%
\pgfpathlineto{\pgfqpoint{3.041813in}{0.413320in}}%
\pgfpathlineto{\pgfqpoint{3.039262in}{0.413320in}}%
\pgfpathlineto{\pgfqpoint{3.036456in}{0.413320in}}%
\pgfpathlineto{\pgfqpoint{3.033921in}{0.413320in}}%
\pgfpathlineto{\pgfqpoint{3.031091in}{0.413320in}}%
\pgfpathlineto{\pgfqpoint{3.028412in}{0.413320in}}%
\pgfpathlineto{\pgfqpoint{3.025803in}{0.413320in}}%
\pgfpathlineto{\pgfqpoint{3.023058in}{0.413320in}}%
\pgfpathlineto{\pgfqpoint{3.020382in}{0.413320in}}%
\pgfpathlineto{\pgfqpoint{3.017707in}{0.413320in}}%
\pgfpathlineto{\pgfqpoint{3.015097in}{0.413320in}}%
\pgfpathlineto{\pgfqpoint{3.012351in}{0.413320in}}%
\pgfpathlineto{\pgfqpoint{3.009784in}{0.413320in}}%
\pgfpathlineto{\pgfqpoint{3.006993in}{0.413320in}}%
\pgfpathlineto{\pgfqpoint{3.004419in}{0.413320in}}%
\pgfpathlineto{\pgfqpoint{3.001635in}{0.413320in}}%
\pgfpathlineto{\pgfqpoint{2.999103in}{0.413320in}}%
\pgfpathlineto{\pgfqpoint{2.996300in}{0.413320in}}%
\pgfpathlineto{\pgfqpoint{2.993595in}{0.413320in}}%
\pgfpathlineto{\pgfqpoint{2.990978in}{0.413320in}}%
\pgfpathlineto{\pgfqpoint{2.988238in}{0.413320in}}%
\pgfpathlineto{\pgfqpoint{2.985666in}{0.413320in}}%
\pgfpathlineto{\pgfqpoint{2.982885in}{0.413320in}}%
\pgfpathlineto{\pgfqpoint{2.980341in}{0.413320in}}%
\pgfpathlineto{\pgfqpoint{2.977517in}{0.413320in}}%
\pgfpathlineto{\pgfqpoint{2.974972in}{0.413320in}}%
\pgfpathlineto{\pgfqpoint{2.972177in}{0.413320in}}%
\pgfpathlineto{\pgfqpoint{2.969599in}{0.413320in}}%
\pgfpathlineto{\pgfqpoint{2.966812in}{0.413320in}}%
\pgfpathlineto{\pgfqpoint{2.964127in}{0.413320in}}%
\pgfpathlineto{\pgfqpoint{2.961460in}{0.413320in}}%
\pgfpathlineto{\pgfqpoint{2.958782in}{0.413320in}}%
\pgfpathlineto{\pgfqpoint{2.956103in}{0.413320in}}%
\pgfpathlineto{\pgfqpoint{2.953422in}{0.413320in}}%
\pgfpathlineto{\pgfqpoint{2.950884in}{0.413320in}}%
\pgfpathlineto{\pgfqpoint{2.948068in}{0.413320in}}%
\pgfpathlineto{\pgfqpoint{2.945461in}{0.413320in}}%
\pgfpathlineto{\pgfqpoint{2.942711in}{0.413320in}}%
\pgfpathlineto{\pgfqpoint{2.940120in}{0.413320in}}%
\pgfpathlineto{\pgfqpoint{2.937352in}{0.413320in}}%
\pgfpathlineto{\pgfqpoint{2.934759in}{0.413320in}}%
\pgfpathlineto{\pgfqpoint{2.932033in}{0.413320in}}%
\pgfpathlineto{\pgfqpoint{2.929321in}{0.413320in}}%
\pgfpathlineto{\pgfqpoint{2.926655in}{0.413320in}}%
\pgfpathlineto{\pgfqpoint{2.923963in}{0.413320in}}%
\pgfpathlineto{\pgfqpoint{2.921363in}{0.413320in}}%
\pgfpathlineto{\pgfqpoint{2.918606in}{0.413320in}}%
\pgfpathlineto{\pgfqpoint{2.916061in}{0.413320in}}%
\pgfpathlineto{\pgfqpoint{2.913243in}{0.413320in}}%
\pgfpathlineto{\pgfqpoint{2.910631in}{0.413320in}}%
\pgfpathlineto{\pgfqpoint{2.907882in}{0.413320in}}%
\pgfpathlineto{\pgfqpoint{2.905341in}{0.413320in}}%
\pgfpathlineto{\pgfqpoint{2.902535in}{0.413320in}}%
\pgfpathlineto{\pgfqpoint{2.899858in}{0.413320in}}%
\pgfpathlineto{\pgfqpoint{2.897179in}{0.413320in}}%
\pgfpathlineto{\pgfqpoint{2.894487in}{0.413320in}}%
\pgfpathlineto{\pgfqpoint{2.891809in}{0.413320in}}%
\pgfpathlineto{\pgfqpoint{2.889145in}{0.413320in}}%
\pgfpathlineto{\pgfqpoint{2.886578in}{0.413320in}}%
\pgfpathlineto{\pgfqpoint{2.883780in}{0.413320in}}%
\pgfpathlineto{\pgfqpoint{2.881254in}{0.413320in}}%
\pgfpathlineto{\pgfqpoint{2.878431in}{0.413320in}}%
\pgfpathlineto{\pgfqpoint{2.875882in}{0.413320in}}%
\pgfpathlineto{\pgfqpoint{2.873074in}{0.413320in}}%
\pgfpathlineto{\pgfqpoint{2.870475in}{0.413320in}}%
\pgfpathlineto{\pgfqpoint{2.867713in}{0.413320in}}%
\pgfpathlineto{\pgfqpoint{2.865031in}{0.413320in}}%
\pgfpathlineto{\pgfqpoint{2.862402in}{0.413320in}}%
\pgfpathlineto{\pgfqpoint{2.859668in}{0.413320in}}%
\pgfpathlineto{\pgfqpoint{2.857003in}{0.413320in}}%
\pgfpathlineto{\pgfqpoint{2.854325in}{0.413320in}}%
\pgfpathlineto{\pgfqpoint{2.851793in}{0.413320in}}%
\pgfpathlineto{\pgfqpoint{2.848960in}{0.413320in}}%
\pgfpathlineto{\pgfqpoint{2.846408in}{0.413320in}}%
\pgfpathlineto{\pgfqpoint{2.843611in}{0.413320in}}%
\pgfpathlineto{\pgfqpoint{2.841055in}{0.413320in}}%
\pgfpathlineto{\pgfqpoint{2.838254in}{0.413320in}}%
\pgfpathlineto{\pgfqpoint{2.835698in}{0.413320in}}%
\pgfpathlineto{\pgfqpoint{2.832894in}{0.413320in}}%
\pgfpathlineto{\pgfqpoint{2.830219in}{0.413320in}}%
\pgfpathlineto{\pgfqpoint{2.827567in}{0.413320in}}%
\pgfpathlineto{\pgfqpoint{2.824851in}{0.413320in}}%
\pgfpathlineto{\pgfqpoint{2.822303in}{0.413320in}}%
\pgfpathlineto{\pgfqpoint{2.819506in}{0.413320in}}%
\pgfpathlineto{\pgfqpoint{2.816867in}{0.413320in}}%
\pgfpathlineto{\pgfqpoint{2.814141in}{0.413320in}}%
\pgfpathlineto{\pgfqpoint{2.811597in}{0.413320in}}%
\pgfpathlineto{\pgfqpoint{2.808792in}{0.413320in}}%
\pgfpathlineto{\pgfqpoint{2.806175in}{0.413320in}}%
\pgfpathlineto{\pgfqpoint{2.803435in}{0.413320in}}%
\pgfpathlineto{\pgfqpoint{2.800756in}{0.413320in}}%
\pgfpathlineto{\pgfqpoint{2.798070in}{0.413320in}}%
\pgfpathlineto{\pgfqpoint{2.795398in}{0.413320in}}%
\pgfpathlineto{\pgfqpoint{2.792721in}{0.413320in}}%
\pgfpathlineto{\pgfqpoint{2.790044in}{0.413320in}}%
\pgfpathlineto{\pgfqpoint{2.787468in}{0.413320in}}%
\pgfpathlineto{\pgfqpoint{2.784687in}{0.413320in}}%
\pgfpathlineto{\pgfqpoint{2.782113in}{0.413320in}}%
\pgfpathlineto{\pgfqpoint{2.779330in}{0.413320in}}%
\pgfpathlineto{\pgfqpoint{2.776767in}{0.413320in}}%
\pgfpathlineto{\pgfqpoint{2.773972in}{0.413320in}}%
\pgfpathlineto{\pgfqpoint{2.771373in}{0.413320in}}%
\pgfpathlineto{\pgfqpoint{2.768617in}{0.413320in}}%
\pgfpathlineto{\pgfqpoint{2.765935in}{0.413320in}}%
\pgfpathlineto{\pgfqpoint{2.763253in}{0.413320in}}%
\pgfpathlineto{\pgfqpoint{2.760581in}{0.413320in}}%
\pgfpathlineto{\pgfqpoint{2.758028in}{0.413320in}}%
\pgfpathlineto{\pgfqpoint{2.755224in}{0.413320in}}%
\pgfpathlineto{\pgfqpoint{2.752614in}{0.413320in}}%
\pgfpathlineto{\pgfqpoint{2.749868in}{0.413320in}}%
\pgfpathlineto{\pgfqpoint{2.747260in}{0.413320in}}%
\pgfpathlineto{\pgfqpoint{2.744510in}{0.413320in}}%
\pgfpathlineto{\pgfqpoint{2.741928in}{0.413320in}}%
\pgfpathlineto{\pgfqpoint{2.739155in}{0.413320in}}%
\pgfpathlineto{\pgfqpoint{2.736476in}{0.413320in}}%
\pgfpathlineto{\pgfqpoint{2.733798in}{0.413320in}}%
\pgfpathlineto{\pgfqpoint{2.731119in}{0.413320in}}%
\pgfpathlineto{\pgfqpoint{2.728439in}{0.413320in}}%
\pgfpathlineto{\pgfqpoint{2.725760in}{0.413320in}}%
\pgfpathlineto{\pgfqpoint{2.723211in}{0.413320in}}%
\pgfpathlineto{\pgfqpoint{2.720404in}{0.413320in}}%
\pgfpathlineto{\pgfqpoint{2.717773in}{0.413320in}}%
\pgfpathlineto{\pgfqpoint{2.715036in}{0.413320in}}%
\pgfpathlineto{\pgfqpoint{2.712477in}{0.413320in}}%
\pgfpathlineto{\pgfqpoint{2.709683in}{0.413320in}}%
\pgfpathlineto{\pgfqpoint{2.707125in}{0.413320in}}%
\pgfpathlineto{\pgfqpoint{2.704326in}{0.413320in}}%
\pgfpathlineto{\pgfqpoint{2.701657in}{0.413320in}}%
\pgfpathlineto{\pgfqpoint{2.698968in}{0.413320in}}%
\pgfpathlineto{\pgfqpoint{2.696293in}{0.413320in}}%
\pgfpathlineto{\pgfqpoint{2.693611in}{0.413320in}}%
\pgfpathlineto{\pgfqpoint{2.690940in}{0.413320in}}%
\pgfpathlineto{\pgfqpoint{2.688328in}{0.413320in}}%
\pgfpathlineto{\pgfqpoint{2.685586in}{0.413320in}}%
\pgfpathlineto{\pgfqpoint{2.683009in}{0.413320in}}%
\pgfpathlineto{\pgfqpoint{2.680224in}{0.413320in}}%
\pgfpathlineto{\pgfqpoint{2.677650in}{0.413320in}}%
\pgfpathlineto{\pgfqpoint{2.674873in}{0.413320in}}%
\pgfpathlineto{\pgfqpoint{2.672301in}{0.413320in}}%
\pgfpathlineto{\pgfqpoint{2.669506in}{0.413320in}}%
\pgfpathlineto{\pgfqpoint{2.666836in}{0.413320in}}%
\pgfpathlineto{\pgfqpoint{2.664151in}{0.413320in}}%
\pgfpathlineto{\pgfqpoint{2.661481in}{0.413320in}}%
\pgfpathlineto{\pgfqpoint{2.658942in}{0.413320in}}%
\pgfpathlineto{\pgfqpoint{2.656124in}{0.413320in}}%
\pgfpathlineto{\pgfqpoint{2.653567in}{0.413320in}}%
\pgfpathlineto{\pgfqpoint{2.650767in}{0.413320in}}%
\pgfpathlineto{\pgfqpoint{2.648196in}{0.413320in}}%
\pgfpathlineto{\pgfqpoint{2.645408in}{0.413320in}}%
\pgfpathlineto{\pgfqpoint{2.642827in}{0.413320in}}%
\pgfpathlineto{\pgfqpoint{2.640053in}{0.413320in}}%
\pgfpathlineto{\pgfqpoint{2.637369in}{0.413320in}}%
\pgfpathlineto{\pgfqpoint{2.634700in}{0.413320in}}%
\pgfpathlineto{\pgfqpoint{2.632018in}{0.413320in}}%
\pgfpathlineto{\pgfqpoint{2.629340in}{0.413320in}}%
\pgfpathlineto{\pgfqpoint{2.626653in}{0.413320in}}%
\pgfpathlineto{\pgfqpoint{2.624077in}{0.413320in}}%
\pgfpathlineto{\pgfqpoint{2.621304in}{0.413320in}}%
\pgfpathlineto{\pgfqpoint{2.618773in}{0.413320in}}%
\pgfpathlineto{\pgfqpoint{2.615934in}{0.413320in}}%
\pgfpathlineto{\pgfqpoint{2.613393in}{0.413320in}}%
\pgfpathlineto{\pgfqpoint{2.610588in}{0.413320in}}%
\pgfpathlineto{\pgfqpoint{2.608004in}{0.413320in}}%
\pgfpathlineto{\pgfqpoint{2.605232in}{0.413320in}}%
\pgfpathlineto{\pgfqpoint{2.602557in}{0.413320in}}%
\pgfpathlineto{\pgfqpoint{2.599920in}{0.413320in}}%
\pgfpathlineto{\pgfqpoint{2.597196in}{0.413320in}}%
\pgfpathlineto{\pgfqpoint{2.594630in}{0.413320in}}%
\pgfpathlineto{\pgfqpoint{2.591842in}{0.413320in}}%
\pgfpathlineto{\pgfqpoint{2.589248in}{0.413320in}}%
\pgfpathlineto{\pgfqpoint{2.586484in}{0.413320in}}%
\pgfpathlineto{\pgfqpoint{2.583913in}{0.413320in}}%
\pgfpathlineto{\pgfqpoint{2.581129in}{0.413320in}}%
\pgfpathlineto{\pgfqpoint{2.578567in}{0.413320in}}%
\pgfpathlineto{\pgfqpoint{2.575779in}{0.413320in}}%
\pgfpathlineto{\pgfqpoint{2.573082in}{0.413320in}}%
\pgfpathlineto{\pgfqpoint{2.570411in}{0.413320in}}%
\pgfpathlineto{\pgfqpoint{2.567730in}{0.413320in}}%
\pgfpathlineto{\pgfqpoint{2.565045in}{0.413320in}}%
\pgfpathlineto{\pgfqpoint{2.562375in}{0.413320in}}%
\pgfpathlineto{\pgfqpoint{2.559790in}{0.413320in}}%
\pgfpathlineto{\pgfqpoint{2.557009in}{0.413320in}}%
\pgfpathlineto{\pgfqpoint{2.554493in}{0.413320in}}%
\pgfpathlineto{\pgfqpoint{2.551664in}{0.413320in}}%
\pgfpathlineto{\pgfqpoint{2.549114in}{0.413320in}}%
\pgfpathlineto{\pgfqpoint{2.546310in}{0.413320in}}%
\pgfpathlineto{\pgfqpoint{2.543765in}{0.413320in}}%
\pgfpathlineto{\pgfqpoint{2.540949in}{0.413320in}}%
\pgfpathlineto{\pgfqpoint{2.538274in}{0.413320in}}%
\pgfpathlineto{\pgfqpoint{2.535624in}{0.413320in}}%
\pgfpathlineto{\pgfqpoint{2.532917in}{0.413320in}}%
\pgfpathlineto{\pgfqpoint{2.530234in}{0.413320in}}%
\pgfpathlineto{\pgfqpoint{2.527560in}{0.413320in}}%
\pgfpathlineto{\pgfqpoint{2.524988in}{0.413320in}}%
\pgfpathlineto{\pgfqpoint{2.522197in}{0.413320in}}%
\pgfpathlineto{\pgfqpoint{2.519607in}{0.413320in}}%
\pgfpathlineto{\pgfqpoint{2.516845in}{0.413320in}}%
\pgfpathlineto{\pgfqpoint{2.514268in}{0.413320in}}%
\pgfpathlineto{\pgfqpoint{2.511478in}{0.413320in}}%
\pgfpathlineto{\pgfqpoint{2.508917in}{0.413320in}}%
\pgfpathlineto{\pgfqpoint{2.506163in}{0.413320in}}%
\pgfpathlineto{\pgfqpoint{2.503454in}{0.413320in}}%
\pgfpathlineto{\pgfqpoint{2.500801in}{0.413320in}}%
\pgfpathlineto{\pgfqpoint{2.498085in}{0.413320in}}%
\pgfpathlineto{\pgfqpoint{2.495542in}{0.413320in}}%
\pgfpathlineto{\pgfqpoint{2.492729in}{0.413320in}}%
\pgfpathlineto{\pgfqpoint{2.490183in}{0.413320in}}%
\pgfpathlineto{\pgfqpoint{2.487384in}{0.413320in}}%
\pgfpathlineto{\pgfqpoint{2.484870in}{0.413320in}}%
\pgfpathlineto{\pgfqpoint{2.482026in}{0.413320in}}%
\pgfpathlineto{\pgfqpoint{2.479420in}{0.413320in}}%
\pgfpathlineto{\pgfqpoint{2.476671in}{0.413320in}}%
\pgfpathlineto{\pgfqpoint{2.473989in}{0.413320in}}%
\pgfpathlineto{\pgfqpoint{2.471311in}{0.413320in}}%
\pgfpathlineto{\pgfqpoint{2.468635in}{0.413320in}}%
\pgfpathlineto{\pgfqpoint{2.465957in}{0.413320in}}%
\pgfpathlineto{\pgfqpoint{2.463280in}{0.413320in}}%
\pgfpathlineto{\pgfqpoint{2.460711in}{0.413320in}}%
\pgfpathlineto{\pgfqpoint{2.457917in}{0.413320in}}%
\pgfpathlineto{\pgfqpoint{2.455353in}{0.413320in}}%
\pgfpathlineto{\pgfqpoint{2.452562in}{0.413320in}}%
\pgfpathlineto{\pgfqpoint{2.450032in}{0.413320in}}%
\pgfpathlineto{\pgfqpoint{2.447209in}{0.413320in}}%
\pgfpathlineto{\pgfqpoint{2.444677in}{0.413320in}}%
\pgfpathlineto{\pgfqpoint{2.441876in}{0.413320in}}%
\pgfpathlineto{\pgfqpoint{2.439167in}{0.413320in}}%
\pgfpathlineto{\pgfqpoint{2.436518in}{0.413320in}}%
\pgfpathlineto{\pgfqpoint{2.433815in}{0.413320in}}%
\pgfpathlineto{\pgfqpoint{2.431251in}{0.413320in}}%
\pgfpathlineto{\pgfqpoint{2.428453in}{0.413320in}}%
\pgfpathlineto{\pgfqpoint{2.425878in}{0.413320in}}%
\pgfpathlineto{\pgfqpoint{2.423098in}{0.413320in}}%
\pgfpathlineto{\pgfqpoint{2.420528in}{0.413320in}}%
\pgfpathlineto{\pgfqpoint{2.417747in}{0.413320in}}%
\pgfpathlineto{\pgfqpoint{2.415184in}{0.413320in}}%
\pgfpathlineto{\pgfqpoint{2.412389in}{0.413320in}}%
\pgfpathlineto{\pgfqpoint{2.409699in}{0.413320in}}%
\pgfpathlineto{\pgfqpoint{2.407024in}{0.413320in}}%
\pgfpathlineto{\pgfqpoint{2.404352in}{0.413320in}}%
\pgfpathlineto{\pgfqpoint{2.401675in}{0.413320in}}%
\pgfpathlineto{\pgfqpoint{2.398995in}{0.413320in}}%
\pgfpathclose%
\pgfusepath{stroke,fill}%
\end{pgfscope}%
\begin{pgfscope}%
\pgfpathrectangle{\pgfqpoint{2.398995in}{0.319877in}}{\pgfqpoint{3.986877in}{1.993438in}} %
\pgfusepath{clip}%
\pgfsetbuttcap%
\pgfsetmiterjoin%
\definecolor{currentfill}{rgb}{1.000000,1.000000,1.000000}%
\pgfsetfillcolor{currentfill}%
\pgfsetlinewidth{0.000000pt}%
\definecolor{currentstroke}{rgb}{1.000000,1.000000,1.000000}%
\pgfsetstrokecolor{currentstroke}%
\pgfsetdash{}{0pt}%
\pgfpathmoveto{\pgfqpoint{5.731021in}{0.394631in}}%
\pgfpathlineto{\pgfqpoint{5.757600in}{0.394631in}}%
\pgfpathlineto{\pgfqpoint{5.757600in}{2.294627in}}%
\pgfpathlineto{\pgfqpoint{5.731021in}{2.294627in}}%
\pgfpathclose%
\pgfusepath{fill}%
\end{pgfscope}%
\begin{pgfscope}%
\pgfpathrectangle{\pgfqpoint{2.398995in}{0.319877in}}{\pgfqpoint{3.986877in}{1.993438in}} %
\pgfusepath{clip}%
\pgfsetbuttcap%
\pgfsetmiterjoin%
\definecolor{currentfill}{rgb}{1.000000,1.000000,1.000000}%
\pgfsetfillcolor{currentfill}%
\pgfsetlinewidth{0.000000pt}%
\definecolor{currentstroke}{rgb}{1.000000,1.000000,1.000000}%
\pgfsetstrokecolor{currentstroke}%
\pgfsetdash{}{0pt}%
\pgfpathmoveto{\pgfqpoint{2.398995in}{0.394631in}}%
\pgfpathlineto{\pgfqpoint{5.744310in}{0.394631in}}%
\pgfpathlineto{\pgfqpoint{5.744310in}{0.456926in}}%
\pgfpathlineto{\pgfqpoint{2.398995in}{0.456926in}}%
\pgfpathclose%
\pgfusepath{fill}%
\end{pgfscope}%
\begin{pgfscope}%
\pgfpathrectangle{\pgfqpoint{2.398995in}{0.319877in}}{\pgfqpoint{3.986877in}{1.993438in}} %
\pgfusepath{clip}%
\pgfsetbuttcap%
\pgfsetroundjoin%
\pgfsetlinewidth{1.756562pt}%
\definecolor{currentstroke}{rgb}{0.000000,0.000000,0.000000}%
\pgfsetstrokecolor{currentstroke}%
\pgfsetdash{}{0pt}%
\pgfpathmoveto{\pgfqpoint{5.624704in}{0.444467in}}%
\pgfpathlineto{\pgfqpoint{5.757600in}{0.444467in}}%
\pgfusepath{stroke}%
\end{pgfscope}%
\begin{pgfscope}%
\pgfpathrectangle{\pgfqpoint{2.398995in}{0.319877in}}{\pgfqpoint{3.986877in}{1.993438in}} %
\pgfusepath{clip}%
\pgfsetbuttcap%
\pgfsetroundjoin%
\pgfsetlinewidth{1.756562pt}%
\definecolor{currentstroke}{rgb}{0.000000,0.000000,0.000000}%
\pgfsetstrokecolor{currentstroke}%
\pgfsetdash{}{0pt}%
\pgfpathmoveto{\pgfqpoint{5.757600in}{0.444467in}}%
\pgfpathlineto{\pgfqpoint{5.757600in}{0.506762in}}%
\pgfusepath{stroke}%
\end{pgfscope}%
\begin{pgfscope}%
\definecolor{textcolor}{rgb}{0.150000,0.150000,0.150000}%
\pgfsetstrokecolor{textcolor}%
\pgfsetfillcolor{textcolor}%
\pgftext[x=5.677863in,y=0.425779in,,top]{\color{textcolor}\rmfamily\fontsize{10.000000}{12.000000}\selectfont 10 s}%
\end{pgfscope}%
\begin{pgfscope}%
\definecolor{textcolor}{rgb}{0.150000,0.150000,0.150000}%
\pgfsetstrokecolor{textcolor}%
\pgfsetfillcolor{textcolor}%
\pgftext[x=5.770890in,y=0.475615in,left,]{\color{textcolor}\rmfamily\fontsize{10.000000}{12.000000}\selectfont 10 \(\displaystyle \sigma\)}%
\end{pgfscope}%
\end{pgfpicture}%
\makeatother%
\endgroup%

    \caption[Sample cell image and calcium transients.]{Sample cell image and calcium transients. Transients are randomly coloured and correspond to cells of the same colour. \label{f.ad.trace}}
\end{figure}


Figure~\ref{f.ad.freezing} shows the percent of time spent freezing during the memory test. A two-way \gls{anova} revealed a significant interaction between \textit{Genotype} and \textit{Treatment} (F\tsb{1,27}=5.45, p=0.027) as well as a significant main effect in \textit{Genotype} (F\tsb{1,27}=12.79, p=0.001).  \textit{Post hoc} tests showed that \gls{tg}-Veh mice had significantly lower freezing than \gls{wt} mice (\gls{wt}-Veh vs \gls{tg}-Veh, T=4.21, p<0.001), and this effect was rescued by \tglu{} treatment (\gls{tg}-\glu{} vs \gls{tg}-Veh, T=2.85, p=0.008; \gls{wt}-Veh vs \gls{tg}-\glu, T=1.12, p=0.27). \tglu{} had no significant effect on \gls{wt} mice (\gls{wt}-Veh vs \gls{wt}-\glu, T=0.355, p=0.72). 

This result is consistent with previous reports \citep{palmer11, zhou16} that \gls{tg} mice have deficits in hippocampal-related memory tasks. It also confirms our hypothesis that the memory deficit of \gls{tg} mice can be rescued by \tglu{} treatment. 

\begin{figure}[h]
    %% Creator: Matplotlib, PGF backend
%%
%% To include the figure in your LaTeX document, write
%%   \input{<filename>.pgf}
%%
%% Make sure the required packages are loaded in your preamble
%%   \usepackage{pgf}
%%
%% Figures using additional raster images can only be included by \input if
%% they are in the same directory as the main LaTeX file. For loading figures
%% from other directories you can use the `import` package
%%   \usepackage{import}
%% and then include the figures with
%%   \import{<path to file>}{<filename>.pgf}
%%
%% Matplotlib used the following preamble
%%   \usepackage[utf8]{inputenc}
%%   \usepackage[T1]{fontenc}
%%   \usepackage{siunitx}
%%
\begingroup%
\makeatletter%
\begin{pgfpicture}%
\pgfpathrectangle{\pgfpointorigin}{\pgfqpoint{6.008587in}{2.614199in}}%
\pgfusepath{use as bounding box, clip}%
\begin{pgfscope}%
\pgfsetbuttcap%
\pgfsetmiterjoin%
\definecolor{currentfill}{rgb}{1.000000,1.000000,1.000000}%
\pgfsetfillcolor{currentfill}%
\pgfsetlinewidth{0.000000pt}%
\definecolor{currentstroke}{rgb}{1.000000,1.000000,1.000000}%
\pgfsetstrokecolor{currentstroke}%
\pgfsetdash{}{0pt}%
\pgfpathmoveto{\pgfqpoint{0.000000in}{0.000000in}}%
\pgfpathlineto{\pgfqpoint{6.008587in}{0.000000in}}%
\pgfpathlineto{\pgfqpoint{6.008587in}{2.614199in}}%
\pgfpathlineto{\pgfqpoint{0.000000in}{2.614199in}}%
\pgfpathclose%
\pgfusepath{fill}%
\end{pgfscope}%
\begin{pgfscope}%
\pgfsetbuttcap%
\pgfsetmiterjoin%
\definecolor{currentfill}{rgb}{1.000000,1.000000,1.000000}%
\pgfsetfillcolor{currentfill}%
\pgfsetlinewidth{0.000000pt}%
\definecolor{currentstroke}{rgb}{0.000000,0.000000,0.000000}%
\pgfsetstrokecolor{currentstroke}%
\pgfsetstrokeopacity{0.000000}%
\pgfsetdash{}{0pt}%
\pgfpathmoveto{\pgfqpoint{0.544411in}{0.161328in}}%
\pgfpathlineto{\pgfqpoint{4.252206in}{0.161328in}}%
\pgfpathlineto{\pgfqpoint{4.252206in}{2.452871in}}%
\pgfpathlineto{\pgfqpoint{0.544411in}{2.452871in}}%
\pgfpathclose%
\pgfusepath{fill}%
\end{pgfscope}%
\begin{pgfscope}%
\pgfsetbuttcap%
\pgfsetroundjoin%
\definecolor{currentfill}{rgb}{0.150000,0.150000,0.150000}%
\pgfsetfillcolor{currentfill}%
\pgfsetlinewidth{1.003750pt}%
\definecolor{currentstroke}{rgb}{0.150000,0.150000,0.150000}%
\pgfsetstrokecolor{currentstroke}%
\pgfsetdash{}{0pt}%
\pgfsys@defobject{currentmarker}{\pgfqpoint{0.000000in}{0.000000in}}{\pgfqpoint{0.041667in}{0.000000in}}{%
\pgfpathmoveto{\pgfqpoint{0.000000in}{0.000000in}}%
\pgfpathlineto{\pgfqpoint{0.041667in}{0.000000in}}%
\pgfusepath{stroke,fill}%
}%
\begin{pgfscope}%
\pgfsys@transformshift{0.544411in}{0.161328in}%
\pgfsys@useobject{currentmarker}{}%
\end{pgfscope}%
\end{pgfscope}%
\begin{pgfscope}%
\definecolor{textcolor}{rgb}{0.150000,0.150000,0.150000}%
\pgfsetstrokecolor{textcolor}%
\pgfsetfillcolor{textcolor}%
\pgftext[x=0.447189in,y=0.161328in,right,]{\color{textcolor}\rmfamily\fontsize{10.000000}{12.000000}\selectfont \(\displaystyle 0\)}%
\end{pgfscope}%
\begin{pgfscope}%
\pgfsetbuttcap%
\pgfsetroundjoin%
\definecolor{currentfill}{rgb}{0.150000,0.150000,0.150000}%
\pgfsetfillcolor{currentfill}%
\pgfsetlinewidth{1.003750pt}%
\definecolor{currentstroke}{rgb}{0.150000,0.150000,0.150000}%
\pgfsetstrokecolor{currentstroke}%
\pgfsetdash{}{0pt}%
\pgfsys@defobject{currentmarker}{\pgfqpoint{0.000000in}{0.000000in}}{\pgfqpoint{0.041667in}{0.000000in}}{%
\pgfpathmoveto{\pgfqpoint{0.000000in}{0.000000in}}%
\pgfpathlineto{\pgfqpoint{0.041667in}{0.000000in}}%
\pgfusepath{stroke,fill}%
}%
\begin{pgfscope}%
\pgfsys@transformshift{0.544411in}{0.488691in}%
\pgfsys@useobject{currentmarker}{}%
\end{pgfscope}%
\end{pgfscope}%
\begin{pgfscope}%
\definecolor{textcolor}{rgb}{0.150000,0.150000,0.150000}%
\pgfsetstrokecolor{textcolor}%
\pgfsetfillcolor{textcolor}%
\pgftext[x=0.447189in,y=0.488691in,right,]{\color{textcolor}\rmfamily\fontsize{10.000000}{12.000000}\selectfont \(\displaystyle 10\)}%
\end{pgfscope}%
\begin{pgfscope}%
\pgfsetbuttcap%
\pgfsetroundjoin%
\definecolor{currentfill}{rgb}{0.150000,0.150000,0.150000}%
\pgfsetfillcolor{currentfill}%
\pgfsetlinewidth{1.003750pt}%
\definecolor{currentstroke}{rgb}{0.150000,0.150000,0.150000}%
\pgfsetstrokecolor{currentstroke}%
\pgfsetdash{}{0pt}%
\pgfsys@defobject{currentmarker}{\pgfqpoint{0.000000in}{0.000000in}}{\pgfqpoint{0.041667in}{0.000000in}}{%
\pgfpathmoveto{\pgfqpoint{0.000000in}{0.000000in}}%
\pgfpathlineto{\pgfqpoint{0.041667in}{0.000000in}}%
\pgfusepath{stroke,fill}%
}%
\begin{pgfscope}%
\pgfsys@transformshift{0.544411in}{0.816054in}%
\pgfsys@useobject{currentmarker}{}%
\end{pgfscope}%
\end{pgfscope}%
\begin{pgfscope}%
\definecolor{textcolor}{rgb}{0.150000,0.150000,0.150000}%
\pgfsetstrokecolor{textcolor}%
\pgfsetfillcolor{textcolor}%
\pgftext[x=0.447189in,y=0.816054in,right,]{\color{textcolor}\rmfamily\fontsize{10.000000}{12.000000}\selectfont \(\displaystyle 20\)}%
\end{pgfscope}%
\begin{pgfscope}%
\pgfsetbuttcap%
\pgfsetroundjoin%
\definecolor{currentfill}{rgb}{0.150000,0.150000,0.150000}%
\pgfsetfillcolor{currentfill}%
\pgfsetlinewidth{1.003750pt}%
\definecolor{currentstroke}{rgb}{0.150000,0.150000,0.150000}%
\pgfsetstrokecolor{currentstroke}%
\pgfsetdash{}{0pt}%
\pgfsys@defobject{currentmarker}{\pgfqpoint{0.000000in}{0.000000in}}{\pgfqpoint{0.041667in}{0.000000in}}{%
\pgfpathmoveto{\pgfqpoint{0.000000in}{0.000000in}}%
\pgfpathlineto{\pgfqpoint{0.041667in}{0.000000in}}%
\pgfusepath{stroke,fill}%
}%
\begin{pgfscope}%
\pgfsys@transformshift{0.544411in}{1.143418in}%
\pgfsys@useobject{currentmarker}{}%
\end{pgfscope}%
\end{pgfscope}%
\begin{pgfscope}%
\definecolor{textcolor}{rgb}{0.150000,0.150000,0.150000}%
\pgfsetstrokecolor{textcolor}%
\pgfsetfillcolor{textcolor}%
\pgftext[x=0.447189in,y=1.143418in,right,]{\color{textcolor}\rmfamily\fontsize{10.000000}{12.000000}\selectfont \(\displaystyle 30\)}%
\end{pgfscope}%
\begin{pgfscope}%
\pgfsetbuttcap%
\pgfsetroundjoin%
\definecolor{currentfill}{rgb}{0.150000,0.150000,0.150000}%
\pgfsetfillcolor{currentfill}%
\pgfsetlinewidth{1.003750pt}%
\definecolor{currentstroke}{rgb}{0.150000,0.150000,0.150000}%
\pgfsetstrokecolor{currentstroke}%
\pgfsetdash{}{0pt}%
\pgfsys@defobject{currentmarker}{\pgfqpoint{0.000000in}{0.000000in}}{\pgfqpoint{0.041667in}{0.000000in}}{%
\pgfpathmoveto{\pgfqpoint{0.000000in}{0.000000in}}%
\pgfpathlineto{\pgfqpoint{0.041667in}{0.000000in}}%
\pgfusepath{stroke,fill}%
}%
\begin{pgfscope}%
\pgfsys@transformshift{0.544411in}{1.470781in}%
\pgfsys@useobject{currentmarker}{}%
\end{pgfscope}%
\end{pgfscope}%
\begin{pgfscope}%
\definecolor{textcolor}{rgb}{0.150000,0.150000,0.150000}%
\pgfsetstrokecolor{textcolor}%
\pgfsetfillcolor{textcolor}%
\pgftext[x=0.447189in,y=1.470781in,right,]{\color{textcolor}\rmfamily\fontsize{10.000000}{12.000000}\selectfont \(\displaystyle 40\)}%
\end{pgfscope}%
\begin{pgfscope}%
\pgfsetbuttcap%
\pgfsetroundjoin%
\definecolor{currentfill}{rgb}{0.150000,0.150000,0.150000}%
\pgfsetfillcolor{currentfill}%
\pgfsetlinewidth{1.003750pt}%
\definecolor{currentstroke}{rgb}{0.150000,0.150000,0.150000}%
\pgfsetstrokecolor{currentstroke}%
\pgfsetdash{}{0pt}%
\pgfsys@defobject{currentmarker}{\pgfqpoint{0.000000in}{0.000000in}}{\pgfqpoint{0.041667in}{0.000000in}}{%
\pgfpathmoveto{\pgfqpoint{0.000000in}{0.000000in}}%
\pgfpathlineto{\pgfqpoint{0.041667in}{0.000000in}}%
\pgfusepath{stroke,fill}%
}%
\begin{pgfscope}%
\pgfsys@transformshift{0.544411in}{1.798144in}%
\pgfsys@useobject{currentmarker}{}%
\end{pgfscope}%
\end{pgfscope}%
\begin{pgfscope}%
\definecolor{textcolor}{rgb}{0.150000,0.150000,0.150000}%
\pgfsetstrokecolor{textcolor}%
\pgfsetfillcolor{textcolor}%
\pgftext[x=0.447189in,y=1.798144in,right,]{\color{textcolor}\rmfamily\fontsize{10.000000}{12.000000}\selectfont \(\displaystyle 50\)}%
\end{pgfscope}%
\begin{pgfscope}%
\pgfsetbuttcap%
\pgfsetroundjoin%
\definecolor{currentfill}{rgb}{0.150000,0.150000,0.150000}%
\pgfsetfillcolor{currentfill}%
\pgfsetlinewidth{1.003750pt}%
\definecolor{currentstroke}{rgb}{0.150000,0.150000,0.150000}%
\pgfsetstrokecolor{currentstroke}%
\pgfsetdash{}{0pt}%
\pgfsys@defobject{currentmarker}{\pgfqpoint{0.000000in}{0.000000in}}{\pgfqpoint{0.041667in}{0.000000in}}{%
\pgfpathmoveto{\pgfqpoint{0.000000in}{0.000000in}}%
\pgfpathlineto{\pgfqpoint{0.041667in}{0.000000in}}%
\pgfusepath{stroke,fill}%
}%
\begin{pgfscope}%
\pgfsys@transformshift{0.544411in}{2.125508in}%
\pgfsys@useobject{currentmarker}{}%
\end{pgfscope}%
\end{pgfscope}%
\begin{pgfscope}%
\definecolor{textcolor}{rgb}{0.150000,0.150000,0.150000}%
\pgfsetstrokecolor{textcolor}%
\pgfsetfillcolor{textcolor}%
\pgftext[x=0.447189in,y=2.125508in,right,]{\color{textcolor}\rmfamily\fontsize{10.000000}{12.000000}\selectfont \(\displaystyle 60\)}%
\end{pgfscope}%
\begin{pgfscope}%
\pgfsetbuttcap%
\pgfsetroundjoin%
\definecolor{currentfill}{rgb}{0.150000,0.150000,0.150000}%
\pgfsetfillcolor{currentfill}%
\pgfsetlinewidth{1.003750pt}%
\definecolor{currentstroke}{rgb}{0.150000,0.150000,0.150000}%
\pgfsetstrokecolor{currentstroke}%
\pgfsetdash{}{0pt}%
\pgfsys@defobject{currentmarker}{\pgfqpoint{0.000000in}{0.000000in}}{\pgfqpoint{0.041667in}{0.000000in}}{%
\pgfpathmoveto{\pgfqpoint{0.000000in}{0.000000in}}%
\pgfpathlineto{\pgfqpoint{0.041667in}{0.000000in}}%
\pgfusepath{stroke,fill}%
}%
\begin{pgfscope}%
\pgfsys@transformshift{0.544411in}{2.452871in}%
\pgfsys@useobject{currentmarker}{}%
\end{pgfscope}%
\end{pgfscope}%
\begin{pgfscope}%
\definecolor{textcolor}{rgb}{0.150000,0.150000,0.150000}%
\pgfsetstrokecolor{textcolor}%
\pgfsetfillcolor{textcolor}%
\pgftext[x=0.447189in,y=2.452871in,right,]{\color{textcolor}\rmfamily\fontsize{10.000000}{12.000000}\selectfont \(\displaystyle 70\)}%
\end{pgfscope}%
\begin{pgfscope}%
\definecolor{textcolor}{rgb}{0.150000,0.150000,0.150000}%
\pgfsetstrokecolor{textcolor}%
\pgfsetfillcolor{textcolor}%
\pgftext[x=0.238855in,y=1.307099in,,bottom,rotate=90.000000]{\color{textcolor}\rmfamily\fontsize{10.000000}{12.000000}\selectfont \textbf{Freezing (\%)}}%
\end{pgfscope}%
\begin{pgfscope}%
\pgfpathrectangle{\pgfqpoint{0.544411in}{0.161328in}}{\pgfqpoint{3.707795in}{2.291544in}} %
\pgfusepath{clip}%
\pgfsetbuttcap%
\pgfsetmiterjoin%
\definecolor{currentfill}{rgb}{0.200000,0.427451,0.650980}%
\pgfsetfillcolor{currentfill}%
\pgfsetlinewidth{1.505625pt}%
\definecolor{currentstroke}{rgb}{0.200000,0.427451,0.650980}%
\pgfsetstrokecolor{currentstroke}%
\pgfsetdash{}{0pt}%
\pgfpathmoveto{\pgfqpoint{0.676832in}{0.161328in}}%
\pgfpathlineto{\pgfqpoint{1.338939in}{0.161328in}}%
\pgfpathlineto{\pgfqpoint{1.338939in}{1.616723in}}%
\pgfpathlineto{\pgfqpoint{0.676832in}{1.616723in}}%
\pgfpathclose%
\pgfusepath{stroke,fill}%
\end{pgfscope}%
\begin{pgfscope}%
\pgfpathrectangle{\pgfqpoint{0.544411in}{0.161328in}}{\pgfqpoint{3.707795in}{2.291544in}} %
\pgfusepath{clip}%
\pgfsetbuttcap%
\pgfsetmiterjoin%
\definecolor{currentfill}{rgb}{0.168627,0.670588,0.494118}%
\pgfsetfillcolor{currentfill}%
\pgfsetlinewidth{1.505625pt}%
\definecolor{currentstroke}{rgb}{0.168627,0.670588,0.494118}%
\pgfsetstrokecolor{currentstroke}%
\pgfsetdash{}{0pt}%
\pgfpathmoveto{\pgfqpoint{1.603781in}{0.161328in}}%
\pgfpathlineto{\pgfqpoint{2.265887in}{0.161328in}}%
\pgfpathlineto{\pgfqpoint{2.265887in}{1.637732in}}%
\pgfpathlineto{\pgfqpoint{1.603781in}{1.637732in}}%
\pgfpathclose%
\pgfusepath{stroke,fill}%
\end{pgfscope}%
\begin{pgfscope}%
\pgfpathrectangle{\pgfqpoint{0.544411in}{0.161328in}}{\pgfqpoint{3.707795in}{2.291544in}} %
\pgfusepath{clip}%
\pgfsetbuttcap%
\pgfsetmiterjoin%
\definecolor{currentfill}{rgb}{1.000000,0.494118,0.250980}%
\pgfsetfillcolor{currentfill}%
\pgfsetlinewidth{1.505625pt}%
\definecolor{currentstroke}{rgb}{1.000000,0.494118,0.250980}%
\pgfsetstrokecolor{currentstroke}%
\pgfsetdash{}{0pt}%
\pgfpathmoveto{\pgfqpoint{2.530730in}{0.161328in}}%
\pgfpathlineto{\pgfqpoint{3.192836in}{0.161328in}}%
\pgfpathlineto{\pgfqpoint{3.192836in}{0.646415in}}%
\pgfpathlineto{\pgfqpoint{2.530730in}{0.646415in}}%
\pgfpathclose%
\pgfusepath{stroke,fill}%
\end{pgfscope}%
\begin{pgfscope}%
\pgfpathrectangle{\pgfqpoint{0.544411in}{0.161328in}}{\pgfqpoint{3.707795in}{2.291544in}} %
\pgfusepath{clip}%
\pgfsetbuttcap%
\pgfsetmiterjoin%
\definecolor{currentfill}{rgb}{1.000000,0.694118,0.250980}%
\pgfsetfillcolor{currentfill}%
\pgfsetlinewidth{1.505625pt}%
\definecolor{currentstroke}{rgb}{1.000000,0.694118,0.250980}%
\pgfsetstrokecolor{currentstroke}%
\pgfsetdash{}{0pt}%
\pgfpathmoveto{\pgfqpoint{3.457679in}{0.161328in}}%
\pgfpathlineto{\pgfqpoint{4.119785in}{0.161328in}}%
\pgfpathlineto{\pgfqpoint{4.119785in}{1.275434in}}%
\pgfpathlineto{\pgfqpoint{3.457679in}{1.275434in}}%
\pgfpathclose%
\pgfusepath{stroke,fill}%
\end{pgfscope}%
\begin{pgfscope}%
\pgfpathrectangle{\pgfqpoint{0.544411in}{0.161328in}}{\pgfqpoint{3.707795in}{2.291544in}} %
\pgfusepath{clip}%
\pgfsetbuttcap%
\pgfsetroundjoin%
\pgfsetlinewidth{1.505625pt}%
\definecolor{currentstroke}{rgb}{0.200000,0.427451,0.650980}%
\pgfsetstrokecolor{currentstroke}%
\pgfsetdash{}{0pt}%
\pgfpathmoveto{\pgfqpoint{1.007885in}{1.616723in}}%
\pgfpathlineto{\pgfqpoint{1.007885in}{1.814743in}}%
\pgfusepath{stroke}%
\end{pgfscope}%
\begin{pgfscope}%
\pgfpathrectangle{\pgfqpoint{0.544411in}{0.161328in}}{\pgfqpoint{3.707795in}{2.291544in}} %
\pgfusepath{clip}%
\pgfsetbuttcap%
\pgfsetroundjoin%
\pgfsetlinewidth{1.505625pt}%
\definecolor{currentstroke}{rgb}{0.168627,0.670588,0.494118}%
\pgfsetstrokecolor{currentstroke}%
\pgfsetdash{}{0pt}%
\pgfpathmoveto{\pgfqpoint{1.934834in}{1.637732in}}%
\pgfpathlineto{\pgfqpoint{1.934834in}{1.891848in}}%
\pgfusepath{stroke}%
\end{pgfscope}%
\begin{pgfscope}%
\pgfpathrectangle{\pgfqpoint{0.544411in}{0.161328in}}{\pgfqpoint{3.707795in}{2.291544in}} %
\pgfusepath{clip}%
\pgfsetbuttcap%
\pgfsetroundjoin%
\pgfsetlinewidth{1.505625pt}%
\definecolor{currentstroke}{rgb}{1.000000,0.494118,0.250980}%
\pgfsetstrokecolor{currentstroke}%
\pgfsetdash{}{0pt}%
\pgfpathmoveto{\pgfqpoint{2.861783in}{0.646415in}}%
\pgfpathlineto{\pgfqpoint{2.861783in}{0.824862in}}%
\pgfusepath{stroke}%
\end{pgfscope}%
\begin{pgfscope}%
\pgfpathrectangle{\pgfqpoint{0.544411in}{0.161328in}}{\pgfqpoint{3.707795in}{2.291544in}} %
\pgfusepath{clip}%
\pgfsetbuttcap%
\pgfsetroundjoin%
\pgfsetlinewidth{1.505625pt}%
\definecolor{currentstroke}{rgb}{1.000000,0.694118,0.250980}%
\pgfsetstrokecolor{currentstroke}%
\pgfsetdash{}{0pt}%
\pgfpathmoveto{\pgfqpoint{3.788732in}{1.275434in}}%
\pgfpathlineto{\pgfqpoint{3.788732in}{1.449126in}}%
\pgfusepath{stroke}%
\end{pgfscope}%
\begin{pgfscope}%
\pgfpathrectangle{\pgfqpoint{0.544411in}{0.161328in}}{\pgfqpoint{3.707795in}{2.291544in}} %
\pgfusepath{clip}%
\pgfsetbuttcap%
\pgfsetroundjoin%
\definecolor{currentfill}{rgb}{0.200000,0.427451,0.650980}%
\pgfsetfillcolor{currentfill}%
\pgfsetlinewidth{1.505625pt}%
\definecolor{currentstroke}{rgb}{0.200000,0.427451,0.650980}%
\pgfsetstrokecolor{currentstroke}%
\pgfsetdash{}{0pt}%
\pgfsys@defobject{currentmarker}{\pgfqpoint{-0.111111in}{-0.000000in}}{\pgfqpoint{0.111111in}{0.000000in}}{%
\pgfpathmoveto{\pgfqpoint{0.111111in}{-0.000000in}}%
\pgfpathlineto{\pgfqpoint{-0.111111in}{0.000000in}}%
\pgfusepath{stroke,fill}%
}%
\begin{pgfscope}%
\pgfsys@transformshift{1.007885in}{1.616723in}%
\pgfsys@useobject{currentmarker}{}%
\end{pgfscope}%
\end{pgfscope}%
\begin{pgfscope}%
\pgfpathrectangle{\pgfqpoint{0.544411in}{0.161328in}}{\pgfqpoint{3.707795in}{2.291544in}} %
\pgfusepath{clip}%
\pgfsetbuttcap%
\pgfsetroundjoin%
\definecolor{currentfill}{rgb}{0.200000,0.427451,0.650980}%
\pgfsetfillcolor{currentfill}%
\pgfsetlinewidth{1.505625pt}%
\definecolor{currentstroke}{rgb}{0.200000,0.427451,0.650980}%
\pgfsetstrokecolor{currentstroke}%
\pgfsetdash{}{0pt}%
\pgfsys@defobject{currentmarker}{\pgfqpoint{-0.111111in}{-0.000000in}}{\pgfqpoint{0.111111in}{0.000000in}}{%
\pgfpathmoveto{\pgfqpoint{0.111111in}{-0.000000in}}%
\pgfpathlineto{\pgfqpoint{-0.111111in}{0.000000in}}%
\pgfusepath{stroke,fill}%
}%
\begin{pgfscope}%
\pgfsys@transformshift{1.007885in}{1.814743in}%
\pgfsys@useobject{currentmarker}{}%
\end{pgfscope}%
\end{pgfscope}%
\begin{pgfscope}%
\pgfpathrectangle{\pgfqpoint{0.544411in}{0.161328in}}{\pgfqpoint{3.707795in}{2.291544in}} %
\pgfusepath{clip}%
\pgfsetbuttcap%
\pgfsetroundjoin%
\definecolor{currentfill}{rgb}{0.168627,0.670588,0.494118}%
\pgfsetfillcolor{currentfill}%
\pgfsetlinewidth{1.505625pt}%
\definecolor{currentstroke}{rgb}{0.168627,0.670588,0.494118}%
\pgfsetstrokecolor{currentstroke}%
\pgfsetdash{}{0pt}%
\pgfsys@defobject{currentmarker}{\pgfqpoint{-0.111111in}{-0.000000in}}{\pgfqpoint{0.111111in}{0.000000in}}{%
\pgfpathmoveto{\pgfqpoint{0.111111in}{-0.000000in}}%
\pgfpathlineto{\pgfqpoint{-0.111111in}{0.000000in}}%
\pgfusepath{stroke,fill}%
}%
\begin{pgfscope}%
\pgfsys@transformshift{1.934834in}{1.637732in}%
\pgfsys@useobject{currentmarker}{}%
\end{pgfscope}%
\end{pgfscope}%
\begin{pgfscope}%
\pgfpathrectangle{\pgfqpoint{0.544411in}{0.161328in}}{\pgfqpoint{3.707795in}{2.291544in}} %
\pgfusepath{clip}%
\pgfsetbuttcap%
\pgfsetroundjoin%
\definecolor{currentfill}{rgb}{0.168627,0.670588,0.494118}%
\pgfsetfillcolor{currentfill}%
\pgfsetlinewidth{1.505625pt}%
\definecolor{currentstroke}{rgb}{0.168627,0.670588,0.494118}%
\pgfsetstrokecolor{currentstroke}%
\pgfsetdash{}{0pt}%
\pgfsys@defobject{currentmarker}{\pgfqpoint{-0.111111in}{-0.000000in}}{\pgfqpoint{0.111111in}{0.000000in}}{%
\pgfpathmoveto{\pgfqpoint{0.111111in}{-0.000000in}}%
\pgfpathlineto{\pgfqpoint{-0.111111in}{0.000000in}}%
\pgfusepath{stroke,fill}%
}%
\begin{pgfscope}%
\pgfsys@transformshift{1.934834in}{1.891848in}%
\pgfsys@useobject{currentmarker}{}%
\end{pgfscope}%
\end{pgfscope}%
\begin{pgfscope}%
\pgfpathrectangle{\pgfqpoint{0.544411in}{0.161328in}}{\pgfqpoint{3.707795in}{2.291544in}} %
\pgfusepath{clip}%
\pgfsetbuttcap%
\pgfsetroundjoin%
\definecolor{currentfill}{rgb}{1.000000,0.494118,0.250980}%
\pgfsetfillcolor{currentfill}%
\pgfsetlinewidth{1.505625pt}%
\definecolor{currentstroke}{rgb}{1.000000,0.494118,0.250980}%
\pgfsetstrokecolor{currentstroke}%
\pgfsetdash{}{0pt}%
\pgfsys@defobject{currentmarker}{\pgfqpoint{-0.111111in}{-0.000000in}}{\pgfqpoint{0.111111in}{0.000000in}}{%
\pgfpathmoveto{\pgfqpoint{0.111111in}{-0.000000in}}%
\pgfpathlineto{\pgfqpoint{-0.111111in}{0.000000in}}%
\pgfusepath{stroke,fill}%
}%
\begin{pgfscope}%
\pgfsys@transformshift{2.861783in}{0.646415in}%
\pgfsys@useobject{currentmarker}{}%
\end{pgfscope}%
\end{pgfscope}%
\begin{pgfscope}%
\pgfpathrectangle{\pgfqpoint{0.544411in}{0.161328in}}{\pgfqpoint{3.707795in}{2.291544in}} %
\pgfusepath{clip}%
\pgfsetbuttcap%
\pgfsetroundjoin%
\definecolor{currentfill}{rgb}{1.000000,0.494118,0.250980}%
\pgfsetfillcolor{currentfill}%
\pgfsetlinewidth{1.505625pt}%
\definecolor{currentstroke}{rgb}{1.000000,0.494118,0.250980}%
\pgfsetstrokecolor{currentstroke}%
\pgfsetdash{}{0pt}%
\pgfsys@defobject{currentmarker}{\pgfqpoint{-0.111111in}{-0.000000in}}{\pgfqpoint{0.111111in}{0.000000in}}{%
\pgfpathmoveto{\pgfqpoint{0.111111in}{-0.000000in}}%
\pgfpathlineto{\pgfqpoint{-0.111111in}{0.000000in}}%
\pgfusepath{stroke,fill}%
}%
\begin{pgfscope}%
\pgfsys@transformshift{2.861783in}{0.824862in}%
\pgfsys@useobject{currentmarker}{}%
\end{pgfscope}%
\end{pgfscope}%
\begin{pgfscope}%
\pgfpathrectangle{\pgfqpoint{0.544411in}{0.161328in}}{\pgfqpoint{3.707795in}{2.291544in}} %
\pgfusepath{clip}%
\pgfsetbuttcap%
\pgfsetroundjoin%
\definecolor{currentfill}{rgb}{1.000000,0.694118,0.250980}%
\pgfsetfillcolor{currentfill}%
\pgfsetlinewidth{1.505625pt}%
\definecolor{currentstroke}{rgb}{1.000000,0.694118,0.250980}%
\pgfsetstrokecolor{currentstroke}%
\pgfsetdash{}{0pt}%
\pgfsys@defobject{currentmarker}{\pgfqpoint{-0.111111in}{-0.000000in}}{\pgfqpoint{0.111111in}{0.000000in}}{%
\pgfpathmoveto{\pgfqpoint{0.111111in}{-0.000000in}}%
\pgfpathlineto{\pgfqpoint{-0.111111in}{0.000000in}}%
\pgfusepath{stroke,fill}%
}%
\begin{pgfscope}%
\pgfsys@transformshift{3.788732in}{1.275434in}%
\pgfsys@useobject{currentmarker}{}%
\end{pgfscope}%
\end{pgfscope}%
\begin{pgfscope}%
\pgfpathrectangle{\pgfqpoint{0.544411in}{0.161328in}}{\pgfqpoint{3.707795in}{2.291544in}} %
\pgfusepath{clip}%
\pgfsetbuttcap%
\pgfsetroundjoin%
\definecolor{currentfill}{rgb}{1.000000,0.694118,0.250980}%
\pgfsetfillcolor{currentfill}%
\pgfsetlinewidth{1.505625pt}%
\definecolor{currentstroke}{rgb}{1.000000,0.694118,0.250980}%
\pgfsetstrokecolor{currentstroke}%
\pgfsetdash{}{0pt}%
\pgfsys@defobject{currentmarker}{\pgfqpoint{-0.111111in}{-0.000000in}}{\pgfqpoint{0.111111in}{0.000000in}}{%
\pgfpathmoveto{\pgfqpoint{0.111111in}{-0.000000in}}%
\pgfpathlineto{\pgfqpoint{-0.111111in}{0.000000in}}%
\pgfusepath{stroke,fill}%
}%
\begin{pgfscope}%
\pgfsys@transformshift{3.788732in}{1.449126in}%
\pgfsys@useobject{currentmarker}{}%
\end{pgfscope}%
\end{pgfscope}%
\begin{pgfscope}%
\pgfpathrectangle{\pgfqpoint{0.544411in}{0.161328in}}{\pgfqpoint{3.707795in}{2.291544in}} %
\pgfusepath{clip}%
\pgfsetroundcap%
\pgfsetroundjoin%
\pgfsetlinewidth{1.756562pt}%
\definecolor{currentstroke}{rgb}{0.627451,0.627451,0.643137}%
\pgfsetstrokecolor{currentstroke}%
\pgfsetdash{}{0pt}%
\pgfpathmoveto{\pgfqpoint{1.007885in}{1.904029in}}%
\pgfpathlineto{\pgfqpoint{1.007885in}{2.129943in}}%
\pgfusepath{stroke}%
\end{pgfscope}%
\begin{pgfscope}%
\pgfpathrectangle{\pgfqpoint{0.544411in}{0.161328in}}{\pgfqpoint{3.707795in}{2.291544in}} %
\pgfusepath{clip}%
\pgfsetroundcap%
\pgfsetroundjoin%
\pgfsetlinewidth{1.756562pt}%
\definecolor{currentstroke}{rgb}{0.627451,0.627451,0.643137}%
\pgfsetstrokecolor{currentstroke}%
\pgfsetdash{}{0pt}%
\pgfpathmoveto{\pgfqpoint{1.007885in}{2.129943in}}%
\pgfpathlineto{\pgfqpoint{2.861783in}{2.129943in}}%
\pgfusepath{stroke}%
\end{pgfscope}%
\begin{pgfscope}%
\pgfpathrectangle{\pgfqpoint{0.544411in}{0.161328in}}{\pgfqpoint{3.707795in}{2.291544in}} %
\pgfusepath{clip}%
\pgfsetroundcap%
\pgfsetroundjoin%
\pgfsetlinewidth{1.756562pt}%
\definecolor{currentstroke}{rgb}{0.627451,0.627451,0.643137}%
\pgfsetstrokecolor{currentstroke}%
\pgfsetdash{}{0pt}%
\pgfpathmoveto{\pgfqpoint{2.861783in}{2.129943in}}%
\pgfpathlineto{\pgfqpoint{2.861783in}{1.003433in}}%
\pgfusepath{stroke}%
\end{pgfscope}%
\begin{pgfscope}%
\pgfpathrectangle{\pgfqpoint{0.544411in}{0.161328in}}{\pgfqpoint{3.707795in}{2.291544in}} %
\pgfusepath{clip}%
\pgfsetroundcap%
\pgfsetroundjoin%
\pgfsetlinewidth{1.756562pt}%
\definecolor{currentstroke}{rgb}{0.627451,0.627451,0.643137}%
\pgfsetstrokecolor{currentstroke}%
\pgfsetdash{}{0pt}%
\pgfpathmoveto{\pgfqpoint{2.861783in}{2.219229in}}%
\pgfpathlineto{\pgfqpoint{2.861783in}{2.368038in}}%
\pgfusepath{stroke}%
\end{pgfscope}%
\begin{pgfscope}%
\pgfpathrectangle{\pgfqpoint{0.544411in}{0.161328in}}{\pgfqpoint{3.707795in}{2.291544in}} %
\pgfusepath{clip}%
\pgfsetroundcap%
\pgfsetroundjoin%
\pgfsetlinewidth{1.756562pt}%
\definecolor{currentstroke}{rgb}{0.627451,0.627451,0.643137}%
\pgfsetstrokecolor{currentstroke}%
\pgfsetdash{}{0pt}%
\pgfpathmoveto{\pgfqpoint{2.861783in}{2.368038in}}%
\pgfpathlineto{\pgfqpoint{3.788732in}{2.368038in}}%
\pgfusepath{stroke}%
\end{pgfscope}%
\begin{pgfscope}%
\pgfpathrectangle{\pgfqpoint{0.544411in}{0.161328in}}{\pgfqpoint{3.707795in}{2.291544in}} %
\pgfusepath{clip}%
\pgfsetroundcap%
\pgfsetroundjoin%
\pgfsetlinewidth{1.756562pt}%
\definecolor{currentstroke}{rgb}{0.627451,0.627451,0.643137}%
\pgfsetstrokecolor{currentstroke}%
\pgfsetdash{}{0pt}%
\pgfpathmoveto{\pgfqpoint{3.788732in}{2.368038in}}%
\pgfpathlineto{\pgfqpoint{3.788732in}{1.627697in}}%
\pgfusepath{stroke}%
\end{pgfscope}%
\begin{pgfscope}%
\pgfsetrectcap%
\pgfsetmiterjoin%
\pgfsetlinewidth{1.254687pt}%
\definecolor{currentstroke}{rgb}{0.150000,0.150000,0.150000}%
\pgfsetstrokecolor{currentstroke}%
\pgfsetdash{}{0pt}%
\pgfpathmoveto{\pgfqpoint{0.544411in}{0.161328in}}%
\pgfpathlineto{\pgfqpoint{0.544411in}{2.452871in}}%
\pgfusepath{stroke}%
\end{pgfscope}%
\begin{pgfscope}%
\pgfsetrectcap%
\pgfsetmiterjoin%
\pgfsetlinewidth{1.254687pt}%
\definecolor{currentstroke}{rgb}{0.150000,0.150000,0.150000}%
\pgfsetstrokecolor{currentstroke}%
\pgfsetdash{}{0pt}%
\pgfpathmoveto{\pgfqpoint{0.544411in}{0.161328in}}%
\pgfpathlineto{\pgfqpoint{4.252206in}{0.161328in}}%
\pgfusepath{stroke}%
\end{pgfscope}%
\begin{pgfscope}%
\definecolor{textcolor}{rgb}{0.150000,0.150000,0.150000}%
\pgfsetstrokecolor{textcolor}%
\pgfsetfillcolor{textcolor}%
\pgftext[x=2.861783in,y=0.880665in,,]{\color{textcolor}\rmfamily\fontsize{15.000000}{18.000000}\selectfont \textbf{*}}%
\end{pgfscope}%
\begin{pgfscope}%
\definecolor{textcolor}{rgb}{0.150000,0.150000,0.150000}%
\pgfsetstrokecolor{textcolor}%
\pgfsetfillcolor{textcolor}%
\pgftext[x=3.788732in,y=1.504930in,,]{\color{textcolor}\rmfamily\fontsize{15.000000}{18.000000}\selectfont \textbf{*}}%
\end{pgfscope}%
\begin{pgfscope}%
\pgfsetbuttcap%
\pgfsetmiterjoin%
\definecolor{currentfill}{rgb}{0.200000,0.427451,0.650980}%
\pgfsetfillcolor{currentfill}%
\pgfsetlinewidth{1.505625pt}%
\definecolor{currentstroke}{rgb}{0.200000,0.427451,0.650980}%
\pgfsetstrokecolor{currentstroke}%
\pgfsetdash{}{0pt}%
\pgfpathmoveto{\pgfqpoint{4.352206in}{2.269558in}}%
\pgfpathlineto{\pgfqpoint{4.463317in}{2.269558in}}%
\pgfpathlineto{\pgfqpoint{4.463317in}{2.347336in}}%
\pgfpathlineto{\pgfqpoint{4.352206in}{2.347336in}}%
\pgfpathclose%
\pgfusepath{stroke,fill}%
\end{pgfscope}%
\begin{pgfscope}%
\definecolor{textcolor}{rgb}{0.150000,0.150000,0.150000}%
\pgfsetstrokecolor{textcolor}%
\pgfsetfillcolor{textcolor}%
\pgftext[x=4.552206in,y=2.269558in,left,base]{\color{textcolor}\rmfamily\fontsize{8.000000}{9.600000}\selectfont WT + Vehicle (10)}%
\end{pgfscope}%
\begin{pgfscope}%
\pgfsetbuttcap%
\pgfsetmiterjoin%
\definecolor{currentfill}{rgb}{0.168627,0.670588,0.494118}%
\pgfsetfillcolor{currentfill}%
\pgfsetlinewidth{1.505625pt}%
\definecolor{currentstroke}{rgb}{0.168627,0.670588,0.494118}%
\pgfsetstrokecolor{currentstroke}%
\pgfsetdash{}{0pt}%
\pgfpathmoveto{\pgfqpoint{4.352206in}{2.102918in}}%
\pgfpathlineto{\pgfqpoint{4.463317in}{2.102918in}}%
\pgfpathlineto{\pgfqpoint{4.463317in}{2.180696in}}%
\pgfpathlineto{\pgfqpoint{4.352206in}{2.180696in}}%
\pgfpathclose%
\pgfusepath{stroke,fill}%
\end{pgfscope}%
\begin{pgfscope}%
\definecolor{textcolor}{rgb}{0.150000,0.150000,0.150000}%
\pgfsetstrokecolor{textcolor}%
\pgfsetfillcolor{textcolor}%
\pgftext[x=4.552206in,y=2.102918in,left,base]{\color{textcolor}\rmfamily\fontsize{8.000000}{9.600000}\selectfont WT + TAT-GluA2\textsubscript{3Y} (7)}%
\end{pgfscope}%
\begin{pgfscope}%
\pgfsetbuttcap%
\pgfsetmiterjoin%
\definecolor{currentfill}{rgb}{1.000000,0.494118,0.250980}%
\pgfsetfillcolor{currentfill}%
\pgfsetlinewidth{1.505625pt}%
\definecolor{currentstroke}{rgb}{1.000000,0.494118,0.250980}%
\pgfsetstrokecolor{currentstroke}%
\pgfsetdash{}{0pt}%
\pgfpathmoveto{\pgfqpoint{4.352206in}{1.936279in}}%
\pgfpathlineto{\pgfqpoint{4.463317in}{1.936279in}}%
\pgfpathlineto{\pgfqpoint{4.463317in}{2.014057in}}%
\pgfpathlineto{\pgfqpoint{4.352206in}{2.014057in}}%
\pgfpathclose%
\pgfusepath{stroke,fill}%
\end{pgfscope}%
\begin{pgfscope}%
\definecolor{textcolor}{rgb}{0.150000,0.150000,0.150000}%
\pgfsetstrokecolor{textcolor}%
\pgfsetfillcolor{textcolor}%
\pgftext[x=4.552206in,y=1.936279in,left,base]{\color{textcolor}\rmfamily\fontsize{8.000000}{9.600000}\selectfont Tg + Vehicle (7)}%
\end{pgfscope}%
\begin{pgfscope}%
\pgfsetbuttcap%
\pgfsetmiterjoin%
\definecolor{currentfill}{rgb}{1.000000,0.694118,0.250980}%
\pgfsetfillcolor{currentfill}%
\pgfsetlinewidth{1.505625pt}%
\definecolor{currentstroke}{rgb}{1.000000,0.694118,0.250980}%
\pgfsetstrokecolor{currentstroke}%
\pgfsetdash{}{0pt}%
\pgfpathmoveto{\pgfqpoint{4.352206in}{1.769639in}}%
\pgfpathlineto{\pgfqpoint{4.463317in}{1.769639in}}%
\pgfpathlineto{\pgfqpoint{4.463317in}{1.847417in}}%
\pgfpathlineto{\pgfqpoint{4.352206in}{1.847417in}}%
\pgfpathclose%
\pgfusepath{stroke,fill}%
\end{pgfscope}%
\begin{pgfscope}%
\definecolor{textcolor}{rgb}{0.150000,0.150000,0.150000}%
\pgfsetstrokecolor{textcolor}%
\pgfsetfillcolor{textcolor}%
\pgftext[x=4.552206in,y=1.769639in,left,base]{\color{textcolor}\rmfamily\fontsize{8.000000}{9.600000}\selectfont Tg + TAT-GluA2\textsubscript{3Y} (7)}%
\end{pgfscope}%
\end{pgfpicture}%
\makeatother%
\endgroup%

    \caption[Percent of freezing during memory test.]{Percent freezing during memory test. \Gls{tg} mice have significant lower freezing, and \tglu{} treatment returns the freezing to wild-type level. \label{f.ad.freezing}}
\end{figure}

\subsection{\Gls{tg} mice can initiate freezing}
We then investigated whether the deficits in \gls{tg} mice were due to less freezing initiation or freezing maintenance. Previous reports suggested that in a mouse model of \gls{ad}, place cells in \gls{ad} are unstable \citep{cheng13}, and it is possible that a similar deficit is present in memory encoding. Here we analyzed the freezing bout length and freezing bout number, where a freezing bout was defined as a continuous period of freezing. We hypothesized that if  \gls{ad} mice had a deficit in memory maintenance, the mice would show significantly reduced freezing bout length. However if the \gls{ad} mice had no deficit in initiating freezing, the mice would have a similar number of freezing bouts.

Figure~\ref{f.ad.freezing_profile} summarizes the number and length of freezing bouts in each group. There is no significant difference between groups in the number of freezing bouts (Figure~\ref{f.ad.freezing_freq}, omnibus F\tsb{3,27}=0.84, p=0.48). We then examined the average length of freezing bouts per mouse, and found a significant interaction between \textit{Genotype} and \textit{Treatment} (F\tsb{1,27}=6.5, p=0.01) and a significant main effect of \textit{genotype} (F\tsb{1,27}=17.7, p<0.001). \textit{Post hoc} tests showed Tg-Veh mice to have significantly shorter freezing bouts (WT-Veh vs Tg-Veh, T=4.75, p<0.001), and this deficit was fully rescued by \tglu{} treatment (Tg-\glu{} vs Tg-Veh, T=3.10, p=0.002; WT-Veh vs Tg-\glu, T=1.66, p=0.10). There was no effect of \tglu{} on \gls{wt} mice (T=0.22, p=0.83). This result suggests that the vehicle treated \gls{tg} mice do not have a deficit in initiating freezing behaviour, however they are unable to maintain this freezing for an extended period. 

\begin{figure}[h]
    \begin{subfigure}[h]{\textwidth}
        %% Creator: Matplotlib, PGF backend
%%
%% To include the figure in your LaTeX document, write
%%   \input{<filename>.pgf}
%%
%% Make sure the required packages are loaded in your preamble
%%   \usepackage{pgf}
%%
%% Figures using additional raster images can only be included by \input if
%% they are in the same directory as the main LaTeX file. For loading figures
%% from other directories you can use the `import` package
%%   \usepackage{import}
%% and then include the figures with
%%   \import{<path to file>}{<filename>.pgf}
%%
%% Matplotlib used the following preamble
%%   \usepackage[utf8]{inputenc}
%%   \usepackage[T1]{fontenc}
%%   \usepackage{siunitx}
%%
\begingroup%
\makeatletter%
\begin{pgfpicture}%
\pgfpathrectangle{\pgfpointorigin}{\pgfqpoint{5.997164in}{2.614199in}}%
\pgfusepath{use as bounding box, clip}%
\begin{pgfscope}%
\pgfsetbuttcap%
\pgfsetmiterjoin%
\definecolor{currentfill}{rgb}{1.000000,1.000000,1.000000}%
\pgfsetfillcolor{currentfill}%
\pgfsetlinewidth{0.000000pt}%
\definecolor{currentstroke}{rgb}{1.000000,1.000000,1.000000}%
\pgfsetstrokecolor{currentstroke}%
\pgfsetdash{}{0pt}%
\pgfpathmoveto{\pgfqpoint{0.000000in}{0.000000in}}%
\pgfpathlineto{\pgfqpoint{5.997164in}{0.000000in}}%
\pgfpathlineto{\pgfqpoint{5.997164in}{2.614199in}}%
\pgfpathlineto{\pgfqpoint{0.000000in}{2.614199in}}%
\pgfpathclose%
\pgfusepath{fill}%
\end{pgfscope}%
\begin{pgfscope}%
\pgfsetbuttcap%
\pgfsetmiterjoin%
\definecolor{currentfill}{rgb}{1.000000,1.000000,1.000000}%
\pgfsetfillcolor{currentfill}%
\pgfsetlinewidth{0.000000pt}%
\definecolor{currentstroke}{rgb}{0.000000,0.000000,0.000000}%
\pgfsetstrokecolor{currentstroke}%
\pgfsetstrokeopacity{0.000000}%
\pgfsetdash{}{0pt}%
\pgfpathmoveto{\pgfqpoint{0.528404in}{0.161328in}}%
\pgfpathlineto{\pgfqpoint{4.236200in}{0.161328in}}%
\pgfpathlineto{\pgfqpoint{4.236200in}{2.452871in}}%
\pgfpathlineto{\pgfqpoint{0.528404in}{2.452871in}}%
\pgfpathclose%
\pgfusepath{fill}%
\end{pgfscope}%
\begin{pgfscope}%
\pgfsetbuttcap%
\pgfsetroundjoin%
\definecolor{currentfill}{rgb}{0.150000,0.150000,0.150000}%
\pgfsetfillcolor{currentfill}%
\pgfsetlinewidth{1.003750pt}%
\definecolor{currentstroke}{rgb}{0.150000,0.150000,0.150000}%
\pgfsetstrokecolor{currentstroke}%
\pgfsetdash{}{0pt}%
\pgfsys@defobject{currentmarker}{\pgfqpoint{0.000000in}{0.000000in}}{\pgfqpoint{0.041667in}{0.000000in}}{%
\pgfpathmoveto{\pgfqpoint{0.000000in}{0.000000in}}%
\pgfpathlineto{\pgfqpoint{0.041667in}{0.000000in}}%
\pgfusepath{stroke,fill}%
}%
\begin{pgfscope}%
\pgfsys@transformshift{0.528404in}{0.161328in}%
\pgfsys@useobject{currentmarker}{}%
\end{pgfscope}%
\end{pgfscope}%
\begin{pgfscope}%
\definecolor{textcolor}{rgb}{0.150000,0.150000,0.150000}%
\pgfsetstrokecolor{textcolor}%
\pgfsetfillcolor{textcolor}%
\pgftext[x=0.431182in,y=0.161328in,right,]{\color{textcolor}\rmfamily\fontsize{10.000000}{12.000000}\selectfont \(\displaystyle 0\)}%
\end{pgfscope}%
\begin{pgfscope}%
\pgfsetbuttcap%
\pgfsetroundjoin%
\definecolor{currentfill}{rgb}{0.150000,0.150000,0.150000}%
\pgfsetfillcolor{currentfill}%
\pgfsetlinewidth{1.003750pt}%
\definecolor{currentstroke}{rgb}{0.150000,0.150000,0.150000}%
\pgfsetstrokecolor{currentstroke}%
\pgfsetdash{}{0pt}%
\pgfsys@defobject{currentmarker}{\pgfqpoint{0.000000in}{0.000000in}}{\pgfqpoint{0.041667in}{0.000000in}}{%
\pgfpathmoveto{\pgfqpoint{0.000000in}{0.000000in}}%
\pgfpathlineto{\pgfqpoint{0.041667in}{0.000000in}}%
\pgfusepath{stroke,fill}%
}%
\begin{pgfscope}%
\pgfsys@transformshift{0.528404in}{0.543251in}%
\pgfsys@useobject{currentmarker}{}%
\end{pgfscope}%
\end{pgfscope}%
\begin{pgfscope}%
\definecolor{textcolor}{rgb}{0.150000,0.150000,0.150000}%
\pgfsetstrokecolor{textcolor}%
\pgfsetfillcolor{textcolor}%
\pgftext[x=0.431182in,y=0.543251in,right,]{\color{textcolor}\rmfamily\fontsize{10.000000}{12.000000}\selectfont \(\displaystyle 10\)}%
\end{pgfscope}%
\begin{pgfscope}%
\pgfsetbuttcap%
\pgfsetroundjoin%
\definecolor{currentfill}{rgb}{0.150000,0.150000,0.150000}%
\pgfsetfillcolor{currentfill}%
\pgfsetlinewidth{1.003750pt}%
\definecolor{currentstroke}{rgb}{0.150000,0.150000,0.150000}%
\pgfsetstrokecolor{currentstroke}%
\pgfsetdash{}{0pt}%
\pgfsys@defobject{currentmarker}{\pgfqpoint{0.000000in}{0.000000in}}{\pgfqpoint{0.041667in}{0.000000in}}{%
\pgfpathmoveto{\pgfqpoint{0.000000in}{0.000000in}}%
\pgfpathlineto{\pgfqpoint{0.041667in}{0.000000in}}%
\pgfusepath{stroke,fill}%
}%
\begin{pgfscope}%
\pgfsys@transformshift{0.528404in}{0.925175in}%
\pgfsys@useobject{currentmarker}{}%
\end{pgfscope}%
\end{pgfscope}%
\begin{pgfscope}%
\definecolor{textcolor}{rgb}{0.150000,0.150000,0.150000}%
\pgfsetstrokecolor{textcolor}%
\pgfsetfillcolor{textcolor}%
\pgftext[x=0.431182in,y=0.925175in,right,]{\color{textcolor}\rmfamily\fontsize{10.000000}{12.000000}\selectfont \(\displaystyle 20\)}%
\end{pgfscope}%
\begin{pgfscope}%
\pgfsetbuttcap%
\pgfsetroundjoin%
\definecolor{currentfill}{rgb}{0.150000,0.150000,0.150000}%
\pgfsetfillcolor{currentfill}%
\pgfsetlinewidth{1.003750pt}%
\definecolor{currentstroke}{rgb}{0.150000,0.150000,0.150000}%
\pgfsetstrokecolor{currentstroke}%
\pgfsetdash{}{0pt}%
\pgfsys@defobject{currentmarker}{\pgfqpoint{0.000000in}{0.000000in}}{\pgfqpoint{0.041667in}{0.000000in}}{%
\pgfpathmoveto{\pgfqpoint{0.000000in}{0.000000in}}%
\pgfpathlineto{\pgfqpoint{0.041667in}{0.000000in}}%
\pgfusepath{stroke,fill}%
}%
\begin{pgfscope}%
\pgfsys@transformshift{0.528404in}{1.307099in}%
\pgfsys@useobject{currentmarker}{}%
\end{pgfscope}%
\end{pgfscope}%
\begin{pgfscope}%
\definecolor{textcolor}{rgb}{0.150000,0.150000,0.150000}%
\pgfsetstrokecolor{textcolor}%
\pgfsetfillcolor{textcolor}%
\pgftext[x=0.431182in,y=1.307099in,right,]{\color{textcolor}\rmfamily\fontsize{10.000000}{12.000000}\selectfont \(\displaystyle 30\)}%
\end{pgfscope}%
\begin{pgfscope}%
\pgfsetbuttcap%
\pgfsetroundjoin%
\definecolor{currentfill}{rgb}{0.150000,0.150000,0.150000}%
\pgfsetfillcolor{currentfill}%
\pgfsetlinewidth{1.003750pt}%
\definecolor{currentstroke}{rgb}{0.150000,0.150000,0.150000}%
\pgfsetstrokecolor{currentstroke}%
\pgfsetdash{}{0pt}%
\pgfsys@defobject{currentmarker}{\pgfqpoint{0.000000in}{0.000000in}}{\pgfqpoint{0.041667in}{0.000000in}}{%
\pgfpathmoveto{\pgfqpoint{0.000000in}{0.000000in}}%
\pgfpathlineto{\pgfqpoint{0.041667in}{0.000000in}}%
\pgfusepath{stroke,fill}%
}%
\begin{pgfscope}%
\pgfsys@transformshift{0.528404in}{1.689023in}%
\pgfsys@useobject{currentmarker}{}%
\end{pgfscope}%
\end{pgfscope}%
\begin{pgfscope}%
\definecolor{textcolor}{rgb}{0.150000,0.150000,0.150000}%
\pgfsetstrokecolor{textcolor}%
\pgfsetfillcolor{textcolor}%
\pgftext[x=0.431182in,y=1.689023in,right,]{\color{textcolor}\rmfamily\fontsize{10.000000}{12.000000}\selectfont \(\displaystyle 40\)}%
\end{pgfscope}%
\begin{pgfscope}%
\pgfsetbuttcap%
\pgfsetroundjoin%
\definecolor{currentfill}{rgb}{0.150000,0.150000,0.150000}%
\pgfsetfillcolor{currentfill}%
\pgfsetlinewidth{1.003750pt}%
\definecolor{currentstroke}{rgb}{0.150000,0.150000,0.150000}%
\pgfsetstrokecolor{currentstroke}%
\pgfsetdash{}{0pt}%
\pgfsys@defobject{currentmarker}{\pgfqpoint{0.000000in}{0.000000in}}{\pgfqpoint{0.041667in}{0.000000in}}{%
\pgfpathmoveto{\pgfqpoint{0.000000in}{0.000000in}}%
\pgfpathlineto{\pgfqpoint{0.041667in}{0.000000in}}%
\pgfusepath{stroke,fill}%
}%
\begin{pgfscope}%
\pgfsys@transformshift{0.528404in}{2.070947in}%
\pgfsys@useobject{currentmarker}{}%
\end{pgfscope}%
\end{pgfscope}%
\begin{pgfscope}%
\definecolor{textcolor}{rgb}{0.150000,0.150000,0.150000}%
\pgfsetstrokecolor{textcolor}%
\pgfsetfillcolor{textcolor}%
\pgftext[x=0.431182in,y=2.070947in,right,]{\color{textcolor}\rmfamily\fontsize{10.000000}{12.000000}\selectfont \(\displaystyle 50\)}%
\end{pgfscope}%
\begin{pgfscope}%
\pgfsetbuttcap%
\pgfsetroundjoin%
\definecolor{currentfill}{rgb}{0.150000,0.150000,0.150000}%
\pgfsetfillcolor{currentfill}%
\pgfsetlinewidth{1.003750pt}%
\definecolor{currentstroke}{rgb}{0.150000,0.150000,0.150000}%
\pgfsetstrokecolor{currentstroke}%
\pgfsetdash{}{0pt}%
\pgfsys@defobject{currentmarker}{\pgfqpoint{0.000000in}{0.000000in}}{\pgfqpoint{0.041667in}{0.000000in}}{%
\pgfpathmoveto{\pgfqpoint{0.000000in}{0.000000in}}%
\pgfpathlineto{\pgfqpoint{0.041667in}{0.000000in}}%
\pgfusepath{stroke,fill}%
}%
\begin{pgfscope}%
\pgfsys@transformshift{0.528404in}{2.452871in}%
\pgfsys@useobject{currentmarker}{}%
\end{pgfscope}%
\end{pgfscope}%
\begin{pgfscope}%
\definecolor{textcolor}{rgb}{0.150000,0.150000,0.150000}%
\pgfsetstrokecolor{textcolor}%
\pgfsetfillcolor{textcolor}%
\pgftext[x=0.431182in,y=2.452871in,right,]{\color{textcolor}\rmfamily\fontsize{10.000000}{12.000000}\selectfont \(\displaystyle 60\)}%
\end{pgfscope}%
\begin{pgfscope}%
\definecolor{textcolor}{rgb}{0.150000,0.150000,0.150000}%
\pgfsetstrokecolor{textcolor}%
\pgfsetfillcolor{textcolor}%
\pgftext[x=0.222848in,y=1.307099in,,bottom,rotate=90.000000]{\color{textcolor}\rmfamily\fontsize{10.000000}{12.000000}\selectfont \textbf{Number of freezing bouts}}%
\end{pgfscope}%
\begin{pgfscope}%
\pgfpathrectangle{\pgfqpoint{0.528404in}{0.161328in}}{\pgfqpoint{3.707795in}{2.291544in}} %
\pgfusepath{clip}%
\pgfsetbuttcap%
\pgfsetmiterjoin%
\definecolor{currentfill}{rgb}{0.200000,0.427451,0.650980}%
\pgfsetfillcolor{currentfill}%
\pgfsetlinewidth{1.505625pt}%
\definecolor{currentstroke}{rgb}{0.200000,0.427451,0.650980}%
\pgfsetstrokecolor{currentstroke}%
\pgfsetdash{}{0pt}%
\pgfpathmoveto{\pgfqpoint{0.660825in}{0.161328in}}%
\pgfpathlineto{\pgfqpoint{1.322932in}{0.161328in}}%
\pgfpathlineto{\pgfqpoint{1.322932in}{2.048032in}}%
\pgfpathlineto{\pgfqpoint{0.660825in}{2.048032in}}%
\pgfpathclose%
\pgfusepath{stroke,fill}%
\end{pgfscope}%
\begin{pgfscope}%
\pgfpathrectangle{\pgfqpoint{0.528404in}{0.161328in}}{\pgfqpoint{3.707795in}{2.291544in}} %
\pgfusepath{clip}%
\pgfsetbuttcap%
\pgfsetmiterjoin%
\definecolor{currentfill}{rgb}{0.168627,0.670588,0.494118}%
\pgfsetfillcolor{currentfill}%
\pgfsetlinewidth{1.505625pt}%
\definecolor{currentstroke}{rgb}{0.168627,0.670588,0.494118}%
\pgfsetstrokecolor{currentstroke}%
\pgfsetdash{}{0pt}%
\pgfpathmoveto{\pgfqpoint{1.587774in}{0.161328in}}%
\pgfpathlineto{\pgfqpoint{2.249881in}{0.161328in}}%
\pgfpathlineto{\pgfqpoint{2.249881in}{2.152788in}}%
\pgfpathlineto{\pgfqpoint{1.587774in}{2.152788in}}%
\pgfpathclose%
\pgfusepath{stroke,fill}%
\end{pgfscope}%
\begin{pgfscope}%
\pgfpathrectangle{\pgfqpoint{0.528404in}{0.161328in}}{\pgfqpoint{3.707795in}{2.291544in}} %
\pgfusepath{clip}%
\pgfsetbuttcap%
\pgfsetmiterjoin%
\definecolor{currentfill}{rgb}{1.000000,0.494118,0.250980}%
\pgfsetfillcolor{currentfill}%
\pgfsetlinewidth{1.505625pt}%
\definecolor{currentstroke}{rgb}{1.000000,0.494118,0.250980}%
\pgfsetstrokecolor{currentstroke}%
\pgfsetdash{}{0pt}%
\pgfpathmoveto{\pgfqpoint{2.514723in}{0.161328in}}%
\pgfpathlineto{\pgfqpoint{3.176829in}{0.161328in}}%
\pgfpathlineto{\pgfqpoint{3.176829in}{1.694479in}}%
\pgfpathlineto{\pgfqpoint{2.514723in}{1.694479in}}%
\pgfpathclose%
\pgfusepath{stroke,fill}%
\end{pgfscope}%
\begin{pgfscope}%
\pgfpathrectangle{\pgfqpoint{0.528404in}{0.161328in}}{\pgfqpoint{3.707795in}{2.291544in}} %
\pgfusepath{clip}%
\pgfsetbuttcap%
\pgfsetmiterjoin%
\definecolor{currentfill}{rgb}{1.000000,0.694118,0.250980}%
\pgfsetfillcolor{currentfill}%
\pgfsetlinewidth{1.505625pt}%
\definecolor{currentstroke}{rgb}{1.000000,0.694118,0.250980}%
\pgfsetstrokecolor{currentstroke}%
\pgfsetdash{}{0pt}%
\pgfpathmoveto{\pgfqpoint{3.441672in}{0.161328in}}%
\pgfpathlineto{\pgfqpoint{4.103778in}{0.161328in}}%
\pgfpathlineto{\pgfqpoint{4.103778in}{2.049123in}}%
\pgfpathlineto{\pgfqpoint{3.441672in}{2.049123in}}%
\pgfpathclose%
\pgfusepath{stroke,fill}%
\end{pgfscope}%
\begin{pgfscope}%
\pgfpathrectangle{\pgfqpoint{0.528404in}{0.161328in}}{\pgfqpoint{3.707795in}{2.291544in}} %
\pgfusepath{clip}%
\pgfsetbuttcap%
\pgfsetroundjoin%
\pgfsetlinewidth{1.505625pt}%
\definecolor{currentstroke}{rgb}{0.200000,0.427451,0.650980}%
\pgfsetstrokecolor{currentstroke}%
\pgfsetdash{}{0pt}%
\pgfpathmoveto{\pgfqpoint{0.991879in}{2.048032in}}%
\pgfpathlineto{\pgfqpoint{0.991879in}{2.310489in}}%
\pgfusepath{stroke}%
\end{pgfscope}%
\begin{pgfscope}%
\pgfpathrectangle{\pgfqpoint{0.528404in}{0.161328in}}{\pgfqpoint{3.707795in}{2.291544in}} %
\pgfusepath{clip}%
\pgfsetbuttcap%
\pgfsetroundjoin%
\pgfsetlinewidth{1.505625pt}%
\definecolor{currentstroke}{rgb}{0.168627,0.670588,0.494118}%
\pgfsetstrokecolor{currentstroke}%
\pgfsetdash{}{0pt}%
\pgfpathmoveto{\pgfqpoint{1.918827in}{2.152788in}}%
\pgfpathlineto{\pgfqpoint{1.918827in}{2.450274in}}%
\pgfusepath{stroke}%
\end{pgfscope}%
\begin{pgfscope}%
\pgfpathrectangle{\pgfqpoint{0.528404in}{0.161328in}}{\pgfqpoint{3.707795in}{2.291544in}} %
\pgfusepath{clip}%
\pgfsetbuttcap%
\pgfsetroundjoin%
\pgfsetlinewidth{1.505625pt}%
\definecolor{currentstroke}{rgb}{1.000000,0.494118,0.250980}%
\pgfsetstrokecolor{currentstroke}%
\pgfsetdash{}{0pt}%
\pgfpathmoveto{\pgfqpoint{2.845776in}{1.694479in}}%
\pgfpathlineto{\pgfqpoint{2.845776in}{2.053953in}}%
\pgfusepath{stroke}%
\end{pgfscope}%
\begin{pgfscope}%
\pgfpathrectangle{\pgfqpoint{0.528404in}{0.161328in}}{\pgfqpoint{3.707795in}{2.291544in}} %
\pgfusepath{clip}%
\pgfsetbuttcap%
\pgfsetroundjoin%
\pgfsetlinewidth{1.505625pt}%
\definecolor{currentstroke}{rgb}{1.000000,0.694118,0.250980}%
\pgfsetstrokecolor{currentstroke}%
\pgfsetdash{}{0pt}%
\pgfpathmoveto{\pgfqpoint{3.772725in}{2.049123in}}%
\pgfpathlineto{\pgfqpoint{3.772725in}{2.289172in}}%
\pgfusepath{stroke}%
\end{pgfscope}%
\begin{pgfscope}%
\pgfpathrectangle{\pgfqpoint{0.528404in}{0.161328in}}{\pgfqpoint{3.707795in}{2.291544in}} %
\pgfusepath{clip}%
\pgfsetbuttcap%
\pgfsetroundjoin%
\definecolor{currentfill}{rgb}{0.200000,0.427451,0.650980}%
\pgfsetfillcolor{currentfill}%
\pgfsetlinewidth{1.505625pt}%
\definecolor{currentstroke}{rgb}{0.200000,0.427451,0.650980}%
\pgfsetstrokecolor{currentstroke}%
\pgfsetdash{}{0pt}%
\pgfsys@defobject{currentmarker}{\pgfqpoint{-0.111111in}{-0.000000in}}{\pgfqpoint{0.111111in}{0.000000in}}{%
\pgfpathmoveto{\pgfqpoint{0.111111in}{-0.000000in}}%
\pgfpathlineto{\pgfqpoint{-0.111111in}{0.000000in}}%
\pgfusepath{stroke,fill}%
}%
\begin{pgfscope}%
\pgfsys@transformshift{0.991879in}{2.048032in}%
\pgfsys@useobject{currentmarker}{}%
\end{pgfscope}%
\end{pgfscope}%
\begin{pgfscope}%
\pgfpathrectangle{\pgfqpoint{0.528404in}{0.161328in}}{\pgfqpoint{3.707795in}{2.291544in}} %
\pgfusepath{clip}%
\pgfsetbuttcap%
\pgfsetroundjoin%
\definecolor{currentfill}{rgb}{0.200000,0.427451,0.650980}%
\pgfsetfillcolor{currentfill}%
\pgfsetlinewidth{1.505625pt}%
\definecolor{currentstroke}{rgb}{0.200000,0.427451,0.650980}%
\pgfsetstrokecolor{currentstroke}%
\pgfsetdash{}{0pt}%
\pgfsys@defobject{currentmarker}{\pgfqpoint{-0.111111in}{-0.000000in}}{\pgfqpoint{0.111111in}{0.000000in}}{%
\pgfpathmoveto{\pgfqpoint{0.111111in}{-0.000000in}}%
\pgfpathlineto{\pgfqpoint{-0.111111in}{0.000000in}}%
\pgfusepath{stroke,fill}%
}%
\begin{pgfscope}%
\pgfsys@transformshift{0.991879in}{2.310489in}%
\pgfsys@useobject{currentmarker}{}%
\end{pgfscope}%
\end{pgfscope}%
\begin{pgfscope}%
\pgfpathrectangle{\pgfqpoint{0.528404in}{0.161328in}}{\pgfqpoint{3.707795in}{2.291544in}} %
\pgfusepath{clip}%
\pgfsetbuttcap%
\pgfsetroundjoin%
\definecolor{currentfill}{rgb}{0.168627,0.670588,0.494118}%
\pgfsetfillcolor{currentfill}%
\pgfsetlinewidth{1.505625pt}%
\definecolor{currentstroke}{rgb}{0.168627,0.670588,0.494118}%
\pgfsetstrokecolor{currentstroke}%
\pgfsetdash{}{0pt}%
\pgfsys@defobject{currentmarker}{\pgfqpoint{-0.111111in}{-0.000000in}}{\pgfqpoint{0.111111in}{0.000000in}}{%
\pgfpathmoveto{\pgfqpoint{0.111111in}{-0.000000in}}%
\pgfpathlineto{\pgfqpoint{-0.111111in}{0.000000in}}%
\pgfusepath{stroke,fill}%
}%
\begin{pgfscope}%
\pgfsys@transformshift{1.918827in}{2.152788in}%
\pgfsys@useobject{currentmarker}{}%
\end{pgfscope}%
\end{pgfscope}%
\begin{pgfscope}%
\pgfpathrectangle{\pgfqpoint{0.528404in}{0.161328in}}{\pgfqpoint{3.707795in}{2.291544in}} %
\pgfusepath{clip}%
\pgfsetbuttcap%
\pgfsetroundjoin%
\definecolor{currentfill}{rgb}{0.168627,0.670588,0.494118}%
\pgfsetfillcolor{currentfill}%
\pgfsetlinewidth{1.505625pt}%
\definecolor{currentstroke}{rgb}{0.168627,0.670588,0.494118}%
\pgfsetstrokecolor{currentstroke}%
\pgfsetdash{}{0pt}%
\pgfsys@defobject{currentmarker}{\pgfqpoint{-0.111111in}{-0.000000in}}{\pgfqpoint{0.111111in}{0.000000in}}{%
\pgfpathmoveto{\pgfqpoint{0.111111in}{-0.000000in}}%
\pgfpathlineto{\pgfqpoint{-0.111111in}{0.000000in}}%
\pgfusepath{stroke,fill}%
}%
\begin{pgfscope}%
\pgfsys@transformshift{1.918827in}{2.450274in}%
\pgfsys@useobject{currentmarker}{}%
\end{pgfscope}%
\end{pgfscope}%
\begin{pgfscope}%
\pgfpathrectangle{\pgfqpoint{0.528404in}{0.161328in}}{\pgfqpoint{3.707795in}{2.291544in}} %
\pgfusepath{clip}%
\pgfsetbuttcap%
\pgfsetroundjoin%
\definecolor{currentfill}{rgb}{1.000000,0.494118,0.250980}%
\pgfsetfillcolor{currentfill}%
\pgfsetlinewidth{1.505625pt}%
\definecolor{currentstroke}{rgb}{1.000000,0.494118,0.250980}%
\pgfsetstrokecolor{currentstroke}%
\pgfsetdash{}{0pt}%
\pgfsys@defobject{currentmarker}{\pgfqpoint{-0.111111in}{-0.000000in}}{\pgfqpoint{0.111111in}{0.000000in}}{%
\pgfpathmoveto{\pgfqpoint{0.111111in}{-0.000000in}}%
\pgfpathlineto{\pgfqpoint{-0.111111in}{0.000000in}}%
\pgfusepath{stroke,fill}%
}%
\begin{pgfscope}%
\pgfsys@transformshift{2.845776in}{1.694479in}%
\pgfsys@useobject{currentmarker}{}%
\end{pgfscope}%
\end{pgfscope}%
\begin{pgfscope}%
\pgfpathrectangle{\pgfqpoint{0.528404in}{0.161328in}}{\pgfqpoint{3.707795in}{2.291544in}} %
\pgfusepath{clip}%
\pgfsetbuttcap%
\pgfsetroundjoin%
\definecolor{currentfill}{rgb}{1.000000,0.494118,0.250980}%
\pgfsetfillcolor{currentfill}%
\pgfsetlinewidth{1.505625pt}%
\definecolor{currentstroke}{rgb}{1.000000,0.494118,0.250980}%
\pgfsetstrokecolor{currentstroke}%
\pgfsetdash{}{0pt}%
\pgfsys@defobject{currentmarker}{\pgfqpoint{-0.111111in}{-0.000000in}}{\pgfqpoint{0.111111in}{0.000000in}}{%
\pgfpathmoveto{\pgfqpoint{0.111111in}{-0.000000in}}%
\pgfpathlineto{\pgfqpoint{-0.111111in}{0.000000in}}%
\pgfusepath{stroke,fill}%
}%
\begin{pgfscope}%
\pgfsys@transformshift{2.845776in}{2.053953in}%
\pgfsys@useobject{currentmarker}{}%
\end{pgfscope}%
\end{pgfscope}%
\begin{pgfscope}%
\pgfpathrectangle{\pgfqpoint{0.528404in}{0.161328in}}{\pgfqpoint{3.707795in}{2.291544in}} %
\pgfusepath{clip}%
\pgfsetbuttcap%
\pgfsetroundjoin%
\definecolor{currentfill}{rgb}{1.000000,0.694118,0.250980}%
\pgfsetfillcolor{currentfill}%
\pgfsetlinewidth{1.505625pt}%
\definecolor{currentstroke}{rgb}{1.000000,0.694118,0.250980}%
\pgfsetstrokecolor{currentstroke}%
\pgfsetdash{}{0pt}%
\pgfsys@defobject{currentmarker}{\pgfqpoint{-0.111111in}{-0.000000in}}{\pgfqpoint{0.111111in}{0.000000in}}{%
\pgfpathmoveto{\pgfqpoint{0.111111in}{-0.000000in}}%
\pgfpathlineto{\pgfqpoint{-0.111111in}{0.000000in}}%
\pgfusepath{stroke,fill}%
}%
\begin{pgfscope}%
\pgfsys@transformshift{3.772725in}{2.049123in}%
\pgfsys@useobject{currentmarker}{}%
\end{pgfscope}%
\end{pgfscope}%
\begin{pgfscope}%
\pgfpathrectangle{\pgfqpoint{0.528404in}{0.161328in}}{\pgfqpoint{3.707795in}{2.291544in}} %
\pgfusepath{clip}%
\pgfsetbuttcap%
\pgfsetroundjoin%
\definecolor{currentfill}{rgb}{1.000000,0.694118,0.250980}%
\pgfsetfillcolor{currentfill}%
\pgfsetlinewidth{1.505625pt}%
\definecolor{currentstroke}{rgb}{1.000000,0.694118,0.250980}%
\pgfsetstrokecolor{currentstroke}%
\pgfsetdash{}{0pt}%
\pgfsys@defobject{currentmarker}{\pgfqpoint{-0.111111in}{-0.000000in}}{\pgfqpoint{0.111111in}{0.000000in}}{%
\pgfpathmoveto{\pgfqpoint{0.111111in}{-0.000000in}}%
\pgfpathlineto{\pgfqpoint{-0.111111in}{0.000000in}}%
\pgfusepath{stroke,fill}%
}%
\begin{pgfscope}%
\pgfsys@transformshift{3.772725in}{2.289172in}%
\pgfsys@useobject{currentmarker}{}%
\end{pgfscope}%
\end{pgfscope}%
\begin{pgfscope}%
\pgfsetrectcap%
\pgfsetmiterjoin%
\pgfsetlinewidth{1.254687pt}%
\definecolor{currentstroke}{rgb}{0.150000,0.150000,0.150000}%
\pgfsetstrokecolor{currentstroke}%
\pgfsetdash{}{0pt}%
\pgfpathmoveto{\pgfqpoint{0.528404in}{0.161328in}}%
\pgfpathlineto{\pgfqpoint{0.528404in}{2.452871in}}%
\pgfusepath{stroke}%
\end{pgfscope}%
\begin{pgfscope}%
\pgfsetrectcap%
\pgfsetmiterjoin%
\pgfsetlinewidth{1.254687pt}%
\definecolor{currentstroke}{rgb}{0.150000,0.150000,0.150000}%
\pgfsetstrokecolor{currentstroke}%
\pgfsetdash{}{0pt}%
\pgfpathmoveto{\pgfqpoint{0.528404in}{0.161328in}}%
\pgfpathlineto{\pgfqpoint{4.236200in}{0.161328in}}%
\pgfusepath{stroke}%
\end{pgfscope}%
\begin{pgfscope}%
\pgfsetbuttcap%
\pgfsetmiterjoin%
\definecolor{currentfill}{rgb}{0.200000,0.427451,0.650980}%
\pgfsetfillcolor{currentfill}%
\pgfsetlinewidth{1.505625pt}%
\definecolor{currentstroke}{rgb}{0.200000,0.427451,0.650980}%
\pgfsetstrokecolor{currentstroke}%
\pgfsetdash{}{0pt}%
\pgfpathmoveto{\pgfqpoint{4.336200in}{2.269558in}}%
\pgfpathlineto{\pgfqpoint{4.447311in}{2.269558in}}%
\pgfpathlineto{\pgfqpoint{4.447311in}{2.347336in}}%
\pgfpathlineto{\pgfqpoint{4.336200in}{2.347336in}}%
\pgfpathclose%
\pgfusepath{stroke,fill}%
\end{pgfscope}%
\begin{pgfscope}%
\definecolor{textcolor}{rgb}{0.150000,0.150000,0.150000}%
\pgfsetstrokecolor{textcolor}%
\pgfsetfillcolor{textcolor}%
\pgftext[x=4.536200in,y=2.269558in,left,base]{\color{textcolor}\rmfamily\fontsize{8.000000}{9.600000}\selectfont WT + Vehicle (10)}%
\end{pgfscope}%
\begin{pgfscope}%
\pgfsetbuttcap%
\pgfsetmiterjoin%
\definecolor{currentfill}{rgb}{0.168627,0.670588,0.494118}%
\pgfsetfillcolor{currentfill}%
\pgfsetlinewidth{1.505625pt}%
\definecolor{currentstroke}{rgb}{0.168627,0.670588,0.494118}%
\pgfsetstrokecolor{currentstroke}%
\pgfsetdash{}{0pt}%
\pgfpathmoveto{\pgfqpoint{4.336200in}{2.102918in}}%
\pgfpathlineto{\pgfqpoint{4.447311in}{2.102918in}}%
\pgfpathlineto{\pgfqpoint{4.447311in}{2.180696in}}%
\pgfpathlineto{\pgfqpoint{4.336200in}{2.180696in}}%
\pgfpathclose%
\pgfusepath{stroke,fill}%
\end{pgfscope}%
\begin{pgfscope}%
\definecolor{textcolor}{rgb}{0.150000,0.150000,0.150000}%
\pgfsetstrokecolor{textcolor}%
\pgfsetfillcolor{textcolor}%
\pgftext[x=4.536200in,y=2.102918in,left,base]{\color{textcolor}\rmfamily\fontsize{8.000000}{9.600000}\selectfont WT + TAT-GluA2\textsubscript{3Y} (7)}%
\end{pgfscope}%
\begin{pgfscope}%
\pgfsetbuttcap%
\pgfsetmiterjoin%
\definecolor{currentfill}{rgb}{1.000000,0.494118,0.250980}%
\pgfsetfillcolor{currentfill}%
\pgfsetlinewidth{1.505625pt}%
\definecolor{currentstroke}{rgb}{1.000000,0.494118,0.250980}%
\pgfsetstrokecolor{currentstroke}%
\pgfsetdash{}{0pt}%
\pgfpathmoveto{\pgfqpoint{4.336200in}{1.936279in}}%
\pgfpathlineto{\pgfqpoint{4.447311in}{1.936279in}}%
\pgfpathlineto{\pgfqpoint{4.447311in}{2.014057in}}%
\pgfpathlineto{\pgfqpoint{4.336200in}{2.014057in}}%
\pgfpathclose%
\pgfusepath{stroke,fill}%
\end{pgfscope}%
\begin{pgfscope}%
\definecolor{textcolor}{rgb}{0.150000,0.150000,0.150000}%
\pgfsetstrokecolor{textcolor}%
\pgfsetfillcolor{textcolor}%
\pgftext[x=4.536200in,y=1.936279in,left,base]{\color{textcolor}\rmfamily\fontsize{8.000000}{9.600000}\selectfont Tg + Vehicle (7)}%
\end{pgfscope}%
\begin{pgfscope}%
\pgfsetbuttcap%
\pgfsetmiterjoin%
\definecolor{currentfill}{rgb}{1.000000,0.694118,0.250980}%
\pgfsetfillcolor{currentfill}%
\pgfsetlinewidth{1.505625pt}%
\definecolor{currentstroke}{rgb}{1.000000,0.694118,0.250980}%
\pgfsetstrokecolor{currentstroke}%
\pgfsetdash{}{0pt}%
\pgfpathmoveto{\pgfqpoint{4.336200in}{1.769639in}}%
\pgfpathlineto{\pgfqpoint{4.447311in}{1.769639in}}%
\pgfpathlineto{\pgfqpoint{4.447311in}{1.847417in}}%
\pgfpathlineto{\pgfqpoint{4.336200in}{1.847417in}}%
\pgfpathclose%
\pgfusepath{stroke,fill}%
\end{pgfscope}%
\begin{pgfscope}%
\definecolor{textcolor}{rgb}{0.150000,0.150000,0.150000}%
\pgfsetstrokecolor{textcolor}%
\pgfsetfillcolor{textcolor}%
\pgftext[x=4.536200in,y=1.769639in,left,base]{\color{textcolor}\rmfamily\fontsize{8.000000}{9.600000}\selectfont Tg + TAT-GluA2\textsubscript{3Y} (7)}%
\end{pgfscope}%
\end{pgfpicture}%
\makeatother%
\endgroup%

        \caption{\label{f.ad.freezing_freq}}
    \end{subfigure}
    \begin{subfigure}[h]{\textwidth}
        %% Creator: Matplotlib, PGF backend
%%
%% To include the figure in your LaTeX document, write
%%   \input{<filename>.pgf}
%%
%% Make sure the required packages are loaded in your preamble
%%   \usepackage{pgf}
%%
%% Figures using additional raster images can only be included by \input if
%% they are in the same directory as the main LaTeX file. For loading figures
%% from other directories you can use the `import` package
%%   \usepackage{import}
%% and then include the figures with
%%   \import{<path to file>}{<filename>.pgf}
%%
%% Matplotlib used the following preamble
%%   \usepackage[utf8]{inputenc}
%%   \usepackage[T1]{fontenc}
%%   \usepackage{siunitx}
%%
\begingroup%
\makeatletter%
\begin{pgfpicture}%
\pgfpathrectangle{\pgfpointorigin}{\pgfqpoint{5.301729in}{3.553934in}}%
\pgfusepath{use as bounding box, clip}%
\begin{pgfscope}%
\pgfsetbuttcap%
\pgfsetmiterjoin%
\definecolor{currentfill}{rgb}{1.000000,1.000000,1.000000}%
\pgfsetfillcolor{currentfill}%
\pgfsetlinewidth{0.000000pt}%
\definecolor{currentstroke}{rgb}{1.000000,1.000000,1.000000}%
\pgfsetstrokecolor{currentstroke}%
\pgfsetdash{}{0pt}%
\pgfpathmoveto{\pgfqpoint{0.000000in}{0.000000in}}%
\pgfpathlineto{\pgfqpoint{5.301729in}{0.000000in}}%
\pgfpathlineto{\pgfqpoint{5.301729in}{3.553934in}}%
\pgfpathlineto{\pgfqpoint{0.000000in}{3.553934in}}%
\pgfpathclose%
\pgfusepath{fill}%
\end{pgfscope}%
\begin{pgfscope}%
\pgfsetbuttcap%
\pgfsetmiterjoin%
\definecolor{currentfill}{rgb}{1.000000,1.000000,1.000000}%
\pgfsetfillcolor{currentfill}%
\pgfsetlinewidth{0.000000pt}%
\definecolor{currentstroke}{rgb}{0.000000,0.000000,0.000000}%
\pgfsetstrokecolor{currentstroke}%
\pgfsetstrokeopacity{0.000000}%
\pgfsetdash{}{0pt}%
\pgfpathmoveto{\pgfqpoint{0.566985in}{0.528177in}}%
\pgfpathlineto{\pgfqpoint{2.673686in}{0.528177in}}%
\pgfpathlineto{\pgfqpoint{2.673686in}{3.392606in}}%
\pgfpathlineto{\pgfqpoint{0.566985in}{3.392606in}}%
\pgfpathclose%
\pgfusepath{fill}%
\end{pgfscope}%
\begin{pgfscope}%
\pgfsetbuttcap%
\pgfsetroundjoin%
\definecolor{currentfill}{rgb}{0.150000,0.150000,0.150000}%
\pgfsetfillcolor{currentfill}%
\pgfsetlinewidth{1.003750pt}%
\definecolor{currentstroke}{rgb}{0.150000,0.150000,0.150000}%
\pgfsetstrokecolor{currentstroke}%
\pgfsetdash{}{0pt}%
\pgfsys@defobject{currentmarker}{\pgfqpoint{0.000000in}{0.000000in}}{\pgfqpoint{0.000000in}{0.041667in}}{%
\pgfpathmoveto{\pgfqpoint{0.000000in}{0.000000in}}%
\pgfpathlineto{\pgfqpoint{0.000000in}{0.041667in}}%
\pgfusepath{stroke,fill}%
}%
\begin{pgfscope}%
\pgfsys@transformshift{0.566985in}{0.528177in}%
\pgfsys@useobject{currentmarker}{}%
\end{pgfscope}%
\end{pgfscope}%
\begin{pgfscope}%
\definecolor{textcolor}{rgb}{0.150000,0.150000,0.150000}%
\pgfsetstrokecolor{textcolor}%
\pgfsetfillcolor{textcolor}%
\pgftext[x=0.566985in,y=0.430955in,,top]{\color{textcolor}\rmfamily\fontsize{10.000000}{12.000000}\selectfont \(\displaystyle 0\)}%
\end{pgfscope}%
\begin{pgfscope}%
\pgfsetbuttcap%
\pgfsetroundjoin%
\definecolor{currentfill}{rgb}{0.150000,0.150000,0.150000}%
\pgfsetfillcolor{currentfill}%
\pgfsetlinewidth{1.003750pt}%
\definecolor{currentstroke}{rgb}{0.150000,0.150000,0.150000}%
\pgfsetstrokecolor{currentstroke}%
\pgfsetdash{}{0pt}%
\pgfsys@defobject{currentmarker}{\pgfqpoint{0.000000in}{0.000000in}}{\pgfqpoint{0.000000in}{0.041667in}}{%
\pgfpathmoveto{\pgfqpoint{0.000000in}{0.000000in}}%
\pgfpathlineto{\pgfqpoint{0.000000in}{0.041667in}}%
\pgfusepath{stroke,fill}%
}%
\begin{pgfscope}%
\pgfsys@transformshift{0.918102in}{0.528177in}%
\pgfsys@useobject{currentmarker}{}%
\end{pgfscope}%
\end{pgfscope}%
\begin{pgfscope}%
\definecolor{textcolor}{rgb}{0.150000,0.150000,0.150000}%
\pgfsetstrokecolor{textcolor}%
\pgfsetfillcolor{textcolor}%
\pgftext[x=0.918102in,y=0.430955in,,top]{\color{textcolor}\rmfamily\fontsize{10.000000}{12.000000}\selectfont \(\displaystyle 20\)}%
\end{pgfscope}%
\begin{pgfscope}%
\pgfsetbuttcap%
\pgfsetroundjoin%
\definecolor{currentfill}{rgb}{0.150000,0.150000,0.150000}%
\pgfsetfillcolor{currentfill}%
\pgfsetlinewidth{1.003750pt}%
\definecolor{currentstroke}{rgb}{0.150000,0.150000,0.150000}%
\pgfsetstrokecolor{currentstroke}%
\pgfsetdash{}{0pt}%
\pgfsys@defobject{currentmarker}{\pgfqpoint{0.000000in}{0.000000in}}{\pgfqpoint{0.000000in}{0.041667in}}{%
\pgfpathmoveto{\pgfqpoint{0.000000in}{0.000000in}}%
\pgfpathlineto{\pgfqpoint{0.000000in}{0.041667in}}%
\pgfusepath{stroke,fill}%
}%
\begin{pgfscope}%
\pgfsys@transformshift{1.269219in}{0.528177in}%
\pgfsys@useobject{currentmarker}{}%
\end{pgfscope}%
\end{pgfscope}%
\begin{pgfscope}%
\definecolor{textcolor}{rgb}{0.150000,0.150000,0.150000}%
\pgfsetstrokecolor{textcolor}%
\pgfsetfillcolor{textcolor}%
\pgftext[x=1.269219in,y=0.430955in,,top]{\color{textcolor}\rmfamily\fontsize{10.000000}{12.000000}\selectfont \(\displaystyle 40\)}%
\end{pgfscope}%
\begin{pgfscope}%
\pgfsetbuttcap%
\pgfsetroundjoin%
\definecolor{currentfill}{rgb}{0.150000,0.150000,0.150000}%
\pgfsetfillcolor{currentfill}%
\pgfsetlinewidth{1.003750pt}%
\definecolor{currentstroke}{rgb}{0.150000,0.150000,0.150000}%
\pgfsetstrokecolor{currentstroke}%
\pgfsetdash{}{0pt}%
\pgfsys@defobject{currentmarker}{\pgfqpoint{0.000000in}{0.000000in}}{\pgfqpoint{0.000000in}{0.041667in}}{%
\pgfpathmoveto{\pgfqpoint{0.000000in}{0.000000in}}%
\pgfpathlineto{\pgfqpoint{0.000000in}{0.041667in}}%
\pgfusepath{stroke,fill}%
}%
\begin{pgfscope}%
\pgfsys@transformshift{1.620336in}{0.528177in}%
\pgfsys@useobject{currentmarker}{}%
\end{pgfscope}%
\end{pgfscope}%
\begin{pgfscope}%
\definecolor{textcolor}{rgb}{0.150000,0.150000,0.150000}%
\pgfsetstrokecolor{textcolor}%
\pgfsetfillcolor{textcolor}%
\pgftext[x=1.620336in,y=0.430955in,,top]{\color{textcolor}\rmfamily\fontsize{10.000000}{12.000000}\selectfont \(\displaystyle 60\)}%
\end{pgfscope}%
\begin{pgfscope}%
\pgfsetbuttcap%
\pgfsetroundjoin%
\definecolor{currentfill}{rgb}{0.150000,0.150000,0.150000}%
\pgfsetfillcolor{currentfill}%
\pgfsetlinewidth{1.003750pt}%
\definecolor{currentstroke}{rgb}{0.150000,0.150000,0.150000}%
\pgfsetstrokecolor{currentstroke}%
\pgfsetdash{}{0pt}%
\pgfsys@defobject{currentmarker}{\pgfqpoint{0.000000in}{0.000000in}}{\pgfqpoint{0.000000in}{0.041667in}}{%
\pgfpathmoveto{\pgfqpoint{0.000000in}{0.000000in}}%
\pgfpathlineto{\pgfqpoint{0.000000in}{0.041667in}}%
\pgfusepath{stroke,fill}%
}%
\begin{pgfscope}%
\pgfsys@transformshift{1.971453in}{0.528177in}%
\pgfsys@useobject{currentmarker}{}%
\end{pgfscope}%
\end{pgfscope}%
\begin{pgfscope}%
\definecolor{textcolor}{rgb}{0.150000,0.150000,0.150000}%
\pgfsetstrokecolor{textcolor}%
\pgfsetfillcolor{textcolor}%
\pgftext[x=1.971453in,y=0.430955in,,top]{\color{textcolor}\rmfamily\fontsize{10.000000}{12.000000}\selectfont \(\displaystyle 80\)}%
\end{pgfscope}%
\begin{pgfscope}%
\pgfsetbuttcap%
\pgfsetroundjoin%
\definecolor{currentfill}{rgb}{0.150000,0.150000,0.150000}%
\pgfsetfillcolor{currentfill}%
\pgfsetlinewidth{1.003750pt}%
\definecolor{currentstroke}{rgb}{0.150000,0.150000,0.150000}%
\pgfsetstrokecolor{currentstroke}%
\pgfsetdash{}{0pt}%
\pgfsys@defobject{currentmarker}{\pgfqpoint{0.000000in}{0.000000in}}{\pgfqpoint{0.000000in}{0.041667in}}{%
\pgfpathmoveto{\pgfqpoint{0.000000in}{0.000000in}}%
\pgfpathlineto{\pgfqpoint{0.000000in}{0.041667in}}%
\pgfusepath{stroke,fill}%
}%
\begin{pgfscope}%
\pgfsys@transformshift{2.322569in}{0.528177in}%
\pgfsys@useobject{currentmarker}{}%
\end{pgfscope}%
\end{pgfscope}%
\begin{pgfscope}%
\definecolor{textcolor}{rgb}{0.150000,0.150000,0.150000}%
\pgfsetstrokecolor{textcolor}%
\pgfsetfillcolor{textcolor}%
\pgftext[x=2.322569in,y=0.430955in,,top]{\color{textcolor}\rmfamily\fontsize{10.000000}{12.000000}\selectfont \(\displaystyle 100\)}%
\end{pgfscope}%
\begin{pgfscope}%
\pgfsetbuttcap%
\pgfsetroundjoin%
\definecolor{currentfill}{rgb}{0.150000,0.150000,0.150000}%
\pgfsetfillcolor{currentfill}%
\pgfsetlinewidth{1.003750pt}%
\definecolor{currentstroke}{rgb}{0.150000,0.150000,0.150000}%
\pgfsetstrokecolor{currentstroke}%
\pgfsetdash{}{0pt}%
\pgfsys@defobject{currentmarker}{\pgfqpoint{0.000000in}{0.000000in}}{\pgfqpoint{0.000000in}{0.041667in}}{%
\pgfpathmoveto{\pgfqpoint{0.000000in}{0.000000in}}%
\pgfpathlineto{\pgfqpoint{0.000000in}{0.041667in}}%
\pgfusepath{stroke,fill}%
}%
\begin{pgfscope}%
\pgfsys@transformshift{2.673686in}{0.528177in}%
\pgfsys@useobject{currentmarker}{}%
\end{pgfscope}%
\end{pgfscope}%
\begin{pgfscope}%
\definecolor{textcolor}{rgb}{0.150000,0.150000,0.150000}%
\pgfsetstrokecolor{textcolor}%
\pgfsetfillcolor{textcolor}%
\pgftext[x=2.673686in,y=0.430955in,,top]{\color{textcolor}\rmfamily\fontsize{10.000000}{12.000000}\selectfont \(\displaystyle 120\)}%
\end{pgfscope}%
\begin{pgfscope}%
\definecolor{textcolor}{rgb}{0.150000,0.150000,0.150000}%
\pgfsetstrokecolor{textcolor}%
\pgfsetfillcolor{textcolor}%
\pgftext[x=1.620336in,y=0.238855in,,top]{\color{textcolor}\rmfamily\fontsize{10.000000}{12.000000}\selectfont \textbf{Length of freezing bouts (s)}}%
\end{pgfscope}%
\begin{pgfscope}%
\pgfsetbuttcap%
\pgfsetroundjoin%
\definecolor{currentfill}{rgb}{0.150000,0.150000,0.150000}%
\pgfsetfillcolor{currentfill}%
\pgfsetlinewidth{1.003750pt}%
\definecolor{currentstroke}{rgb}{0.150000,0.150000,0.150000}%
\pgfsetstrokecolor{currentstroke}%
\pgfsetdash{}{0pt}%
\pgfsys@defobject{currentmarker}{\pgfqpoint{0.000000in}{0.000000in}}{\pgfqpoint{0.041667in}{0.000000in}}{%
\pgfpathmoveto{\pgfqpoint{0.000000in}{0.000000in}}%
\pgfpathlineto{\pgfqpoint{0.041667in}{0.000000in}}%
\pgfusepath{stroke,fill}%
}%
\begin{pgfscope}%
\pgfsys@transformshift{0.566985in}{0.528177in}%
\pgfsys@useobject{currentmarker}{}%
\end{pgfscope}%
\end{pgfscope}%
\begin{pgfscope}%
\definecolor{textcolor}{rgb}{0.150000,0.150000,0.150000}%
\pgfsetstrokecolor{textcolor}%
\pgfsetfillcolor{textcolor}%
\pgftext[x=0.469762in,y=0.528177in,right,]{\color{textcolor}\rmfamily\fontsize{10.000000}{12.000000}\selectfont \(\displaystyle 0.2\)}%
\end{pgfscope}%
\begin{pgfscope}%
\pgfsetbuttcap%
\pgfsetroundjoin%
\definecolor{currentfill}{rgb}{0.150000,0.150000,0.150000}%
\pgfsetfillcolor{currentfill}%
\pgfsetlinewidth{1.003750pt}%
\definecolor{currentstroke}{rgb}{0.150000,0.150000,0.150000}%
\pgfsetstrokecolor{currentstroke}%
\pgfsetdash{}{0pt}%
\pgfsys@defobject{currentmarker}{\pgfqpoint{0.000000in}{0.000000in}}{\pgfqpoint{0.041667in}{0.000000in}}{%
\pgfpathmoveto{\pgfqpoint{0.000000in}{0.000000in}}%
\pgfpathlineto{\pgfqpoint{0.041667in}{0.000000in}}%
\pgfusepath{stroke,fill}%
}%
\begin{pgfscope}%
\pgfsys@transformshift{0.566985in}{0.886230in}%
\pgfsys@useobject{currentmarker}{}%
\end{pgfscope}%
\end{pgfscope}%
\begin{pgfscope}%
\definecolor{textcolor}{rgb}{0.150000,0.150000,0.150000}%
\pgfsetstrokecolor{textcolor}%
\pgfsetfillcolor{textcolor}%
\pgftext[x=0.469762in,y=0.886230in,right,]{\color{textcolor}\rmfamily\fontsize{10.000000}{12.000000}\selectfont \(\displaystyle 0.3\)}%
\end{pgfscope}%
\begin{pgfscope}%
\pgfsetbuttcap%
\pgfsetroundjoin%
\definecolor{currentfill}{rgb}{0.150000,0.150000,0.150000}%
\pgfsetfillcolor{currentfill}%
\pgfsetlinewidth{1.003750pt}%
\definecolor{currentstroke}{rgb}{0.150000,0.150000,0.150000}%
\pgfsetstrokecolor{currentstroke}%
\pgfsetdash{}{0pt}%
\pgfsys@defobject{currentmarker}{\pgfqpoint{0.000000in}{0.000000in}}{\pgfqpoint{0.041667in}{0.000000in}}{%
\pgfpathmoveto{\pgfqpoint{0.000000in}{0.000000in}}%
\pgfpathlineto{\pgfqpoint{0.041667in}{0.000000in}}%
\pgfusepath{stroke,fill}%
}%
\begin{pgfscope}%
\pgfsys@transformshift{0.566985in}{1.244284in}%
\pgfsys@useobject{currentmarker}{}%
\end{pgfscope}%
\end{pgfscope}%
\begin{pgfscope}%
\definecolor{textcolor}{rgb}{0.150000,0.150000,0.150000}%
\pgfsetstrokecolor{textcolor}%
\pgfsetfillcolor{textcolor}%
\pgftext[x=0.469762in,y=1.244284in,right,]{\color{textcolor}\rmfamily\fontsize{10.000000}{12.000000}\selectfont \(\displaystyle 0.4\)}%
\end{pgfscope}%
\begin{pgfscope}%
\pgfsetbuttcap%
\pgfsetroundjoin%
\definecolor{currentfill}{rgb}{0.150000,0.150000,0.150000}%
\pgfsetfillcolor{currentfill}%
\pgfsetlinewidth{1.003750pt}%
\definecolor{currentstroke}{rgb}{0.150000,0.150000,0.150000}%
\pgfsetstrokecolor{currentstroke}%
\pgfsetdash{}{0pt}%
\pgfsys@defobject{currentmarker}{\pgfqpoint{0.000000in}{0.000000in}}{\pgfqpoint{0.041667in}{0.000000in}}{%
\pgfpathmoveto{\pgfqpoint{0.000000in}{0.000000in}}%
\pgfpathlineto{\pgfqpoint{0.041667in}{0.000000in}}%
\pgfusepath{stroke,fill}%
}%
\begin{pgfscope}%
\pgfsys@transformshift{0.566985in}{1.602338in}%
\pgfsys@useobject{currentmarker}{}%
\end{pgfscope}%
\end{pgfscope}%
\begin{pgfscope}%
\definecolor{textcolor}{rgb}{0.150000,0.150000,0.150000}%
\pgfsetstrokecolor{textcolor}%
\pgfsetfillcolor{textcolor}%
\pgftext[x=0.469762in,y=1.602338in,right,]{\color{textcolor}\rmfamily\fontsize{10.000000}{12.000000}\selectfont \(\displaystyle 0.5\)}%
\end{pgfscope}%
\begin{pgfscope}%
\pgfsetbuttcap%
\pgfsetroundjoin%
\definecolor{currentfill}{rgb}{0.150000,0.150000,0.150000}%
\pgfsetfillcolor{currentfill}%
\pgfsetlinewidth{1.003750pt}%
\definecolor{currentstroke}{rgb}{0.150000,0.150000,0.150000}%
\pgfsetstrokecolor{currentstroke}%
\pgfsetdash{}{0pt}%
\pgfsys@defobject{currentmarker}{\pgfqpoint{0.000000in}{0.000000in}}{\pgfqpoint{0.041667in}{0.000000in}}{%
\pgfpathmoveto{\pgfqpoint{0.000000in}{0.000000in}}%
\pgfpathlineto{\pgfqpoint{0.041667in}{0.000000in}}%
\pgfusepath{stroke,fill}%
}%
\begin{pgfscope}%
\pgfsys@transformshift{0.566985in}{1.960392in}%
\pgfsys@useobject{currentmarker}{}%
\end{pgfscope}%
\end{pgfscope}%
\begin{pgfscope}%
\definecolor{textcolor}{rgb}{0.150000,0.150000,0.150000}%
\pgfsetstrokecolor{textcolor}%
\pgfsetfillcolor{textcolor}%
\pgftext[x=0.469762in,y=1.960392in,right,]{\color{textcolor}\rmfamily\fontsize{10.000000}{12.000000}\selectfont \(\displaystyle 0.6\)}%
\end{pgfscope}%
\begin{pgfscope}%
\pgfsetbuttcap%
\pgfsetroundjoin%
\definecolor{currentfill}{rgb}{0.150000,0.150000,0.150000}%
\pgfsetfillcolor{currentfill}%
\pgfsetlinewidth{1.003750pt}%
\definecolor{currentstroke}{rgb}{0.150000,0.150000,0.150000}%
\pgfsetstrokecolor{currentstroke}%
\pgfsetdash{}{0pt}%
\pgfsys@defobject{currentmarker}{\pgfqpoint{0.000000in}{0.000000in}}{\pgfqpoint{0.041667in}{0.000000in}}{%
\pgfpathmoveto{\pgfqpoint{0.000000in}{0.000000in}}%
\pgfpathlineto{\pgfqpoint{0.041667in}{0.000000in}}%
\pgfusepath{stroke,fill}%
}%
\begin{pgfscope}%
\pgfsys@transformshift{0.566985in}{2.318445in}%
\pgfsys@useobject{currentmarker}{}%
\end{pgfscope}%
\end{pgfscope}%
\begin{pgfscope}%
\definecolor{textcolor}{rgb}{0.150000,0.150000,0.150000}%
\pgfsetstrokecolor{textcolor}%
\pgfsetfillcolor{textcolor}%
\pgftext[x=0.469762in,y=2.318445in,right,]{\color{textcolor}\rmfamily\fontsize{10.000000}{12.000000}\selectfont \(\displaystyle 0.7\)}%
\end{pgfscope}%
\begin{pgfscope}%
\pgfsetbuttcap%
\pgfsetroundjoin%
\definecolor{currentfill}{rgb}{0.150000,0.150000,0.150000}%
\pgfsetfillcolor{currentfill}%
\pgfsetlinewidth{1.003750pt}%
\definecolor{currentstroke}{rgb}{0.150000,0.150000,0.150000}%
\pgfsetstrokecolor{currentstroke}%
\pgfsetdash{}{0pt}%
\pgfsys@defobject{currentmarker}{\pgfqpoint{0.000000in}{0.000000in}}{\pgfqpoint{0.041667in}{0.000000in}}{%
\pgfpathmoveto{\pgfqpoint{0.000000in}{0.000000in}}%
\pgfpathlineto{\pgfqpoint{0.041667in}{0.000000in}}%
\pgfusepath{stroke,fill}%
}%
\begin{pgfscope}%
\pgfsys@transformshift{0.566985in}{2.676499in}%
\pgfsys@useobject{currentmarker}{}%
\end{pgfscope}%
\end{pgfscope}%
\begin{pgfscope}%
\definecolor{textcolor}{rgb}{0.150000,0.150000,0.150000}%
\pgfsetstrokecolor{textcolor}%
\pgfsetfillcolor{textcolor}%
\pgftext[x=0.469762in,y=2.676499in,right,]{\color{textcolor}\rmfamily\fontsize{10.000000}{12.000000}\selectfont \(\displaystyle 0.8\)}%
\end{pgfscope}%
\begin{pgfscope}%
\pgfsetbuttcap%
\pgfsetroundjoin%
\definecolor{currentfill}{rgb}{0.150000,0.150000,0.150000}%
\pgfsetfillcolor{currentfill}%
\pgfsetlinewidth{1.003750pt}%
\definecolor{currentstroke}{rgb}{0.150000,0.150000,0.150000}%
\pgfsetstrokecolor{currentstroke}%
\pgfsetdash{}{0pt}%
\pgfsys@defobject{currentmarker}{\pgfqpoint{0.000000in}{0.000000in}}{\pgfqpoint{0.041667in}{0.000000in}}{%
\pgfpathmoveto{\pgfqpoint{0.000000in}{0.000000in}}%
\pgfpathlineto{\pgfqpoint{0.041667in}{0.000000in}}%
\pgfusepath{stroke,fill}%
}%
\begin{pgfscope}%
\pgfsys@transformshift{0.566985in}{3.034553in}%
\pgfsys@useobject{currentmarker}{}%
\end{pgfscope}%
\end{pgfscope}%
\begin{pgfscope}%
\definecolor{textcolor}{rgb}{0.150000,0.150000,0.150000}%
\pgfsetstrokecolor{textcolor}%
\pgfsetfillcolor{textcolor}%
\pgftext[x=0.469762in,y=3.034553in,right,]{\color{textcolor}\rmfamily\fontsize{10.000000}{12.000000}\selectfont \(\displaystyle 0.9\)}%
\end{pgfscope}%
\begin{pgfscope}%
\pgfsetbuttcap%
\pgfsetroundjoin%
\definecolor{currentfill}{rgb}{0.150000,0.150000,0.150000}%
\pgfsetfillcolor{currentfill}%
\pgfsetlinewidth{1.003750pt}%
\definecolor{currentstroke}{rgb}{0.150000,0.150000,0.150000}%
\pgfsetstrokecolor{currentstroke}%
\pgfsetdash{}{0pt}%
\pgfsys@defobject{currentmarker}{\pgfqpoint{0.000000in}{0.000000in}}{\pgfqpoint{0.041667in}{0.000000in}}{%
\pgfpathmoveto{\pgfqpoint{0.000000in}{0.000000in}}%
\pgfpathlineto{\pgfqpoint{0.041667in}{0.000000in}}%
\pgfusepath{stroke,fill}%
}%
\begin{pgfscope}%
\pgfsys@transformshift{0.566985in}{3.392606in}%
\pgfsys@useobject{currentmarker}{}%
\end{pgfscope}%
\end{pgfscope}%
\begin{pgfscope}%
\definecolor{textcolor}{rgb}{0.150000,0.150000,0.150000}%
\pgfsetstrokecolor{textcolor}%
\pgfsetfillcolor{textcolor}%
\pgftext[x=0.469762in,y=3.392606in,right,]{\color{textcolor}\rmfamily\fontsize{10.000000}{12.000000}\selectfont \(\displaystyle 1.0\)}%
\end{pgfscope}%
\begin{pgfscope}%
\definecolor{textcolor}{rgb}{0.150000,0.150000,0.150000}%
\pgfsetstrokecolor{textcolor}%
\pgfsetfillcolor{textcolor}%
\pgftext[x=0.222848in,y=1.960392in,,bottom,rotate=90.000000]{\color{textcolor}\rmfamily\fontsize{10.000000}{12.000000}\selectfont \textbf{Cumulative porportion}}%
\end{pgfscope}%
\begin{pgfscope}%
\pgfpathrectangle{\pgfqpoint{0.566985in}{0.528177in}}{\pgfqpoint{2.106702in}{2.864429in}} %
\pgfusepath{clip}%
\pgfsetroundcap%
\pgfsetroundjoin%
\pgfsetlinewidth{1.003750pt}%
\definecolor{currentstroke}{rgb}{0.200000,0.427451,0.650980}%
\pgfsetstrokecolor{currentstroke}%
\pgfsetdash{}{0pt}%
\pgfpathmoveto{\pgfqpoint{0.570497in}{1.363152in}}%
\pgfpathlineto{\pgfqpoint{0.611255in}{2.189430in}}%
\pgfpathlineto{\pgfqpoint{0.652013in}{2.638809in}}%
\pgfpathlineto{\pgfqpoint{0.692771in}{2.863499in}}%
\pgfpathlineto{\pgfqpoint{0.733529in}{3.022956in}}%
\pgfpathlineto{\pgfqpoint{0.774287in}{3.066444in}}%
\pgfpathlineto{\pgfqpoint{0.815045in}{3.204157in}}%
\pgfpathlineto{\pgfqpoint{0.855803in}{3.254893in}}%
\pgfpathlineto{\pgfqpoint{0.896561in}{3.276637in}}%
\pgfpathlineto{\pgfqpoint{0.937319in}{3.298382in}}%
\pgfpathlineto{\pgfqpoint{0.978077in}{3.312878in}}%
\pgfpathlineto{\pgfqpoint{1.018835in}{3.327374in}}%
\pgfpathlineto{\pgfqpoint{1.059593in}{3.334622in}}%
\pgfpathlineto{\pgfqpoint{1.100350in}{3.349118in}}%
\pgfpathlineto{\pgfqpoint{1.141108in}{3.356366in}}%
\pgfpathlineto{\pgfqpoint{1.181866in}{3.363614in}}%
\pgfpathlineto{\pgfqpoint{1.222624in}{3.370862in}}%
\pgfpathlineto{\pgfqpoint{1.263382in}{3.378110in}}%
\pgfpathlineto{\pgfqpoint{1.304140in}{3.385358in}}%
\pgfpathlineto{\pgfqpoint{1.344898in}{3.385358in}}%
\pgfpathlineto{\pgfqpoint{1.385656in}{3.385358in}}%
\pgfpathlineto{\pgfqpoint{1.426414in}{3.385358in}}%
\pgfpathlineto{\pgfqpoint{1.467172in}{3.385358in}}%
\pgfpathlineto{\pgfqpoint{1.507930in}{3.385358in}}%
\pgfpathlineto{\pgfqpoint{1.548688in}{3.385358in}}%
\pgfpathlineto{\pgfqpoint{1.589446in}{3.385358in}}%
\pgfpathlineto{\pgfqpoint{1.630204in}{3.385358in}}%
\pgfpathlineto{\pgfqpoint{1.670961in}{3.385358in}}%
\pgfpathlineto{\pgfqpoint{1.711719in}{3.385358in}}%
\pgfpathlineto{\pgfqpoint{1.752477in}{3.385358in}}%
\pgfpathlineto{\pgfqpoint{1.793235in}{3.385358in}}%
\pgfpathlineto{\pgfqpoint{1.833993in}{3.385358in}}%
\pgfpathlineto{\pgfqpoint{1.874751in}{3.385358in}}%
\pgfpathlineto{\pgfqpoint{1.915509in}{3.385358in}}%
\pgfpathlineto{\pgfqpoint{1.956267in}{3.385358in}}%
\pgfpathlineto{\pgfqpoint{1.997025in}{3.385358in}}%
\pgfpathlineto{\pgfqpoint{2.037783in}{3.385358in}}%
\pgfpathlineto{\pgfqpoint{2.078541in}{3.385358in}}%
\pgfpathlineto{\pgfqpoint{2.119299in}{3.385358in}}%
\pgfpathlineto{\pgfqpoint{2.160057in}{3.385358in}}%
\pgfpathlineto{\pgfqpoint{2.200815in}{3.385358in}}%
\pgfpathlineto{\pgfqpoint{2.241572in}{3.385358in}}%
\pgfpathlineto{\pgfqpoint{2.282330in}{3.385358in}}%
\pgfpathlineto{\pgfqpoint{2.323088in}{3.385358in}}%
\pgfpathlineto{\pgfqpoint{2.363846in}{3.385358in}}%
\pgfpathlineto{\pgfqpoint{2.404604in}{3.385358in}}%
\pgfpathlineto{\pgfqpoint{2.445362in}{3.385358in}}%
\pgfpathlineto{\pgfqpoint{2.486120in}{3.385358in}}%
\pgfpathlineto{\pgfqpoint{2.526878in}{3.385358in}}%
\pgfpathlineto{\pgfqpoint{2.567636in}{3.392606in}}%
\pgfusepath{stroke}%
\end{pgfscope}%
\begin{pgfscope}%
\pgfpathrectangle{\pgfqpoint{0.566985in}{0.528177in}}{\pgfqpoint{2.106702in}{2.864429in}} %
\pgfusepath{clip}%
\pgfsetroundcap%
\pgfsetroundjoin%
\pgfsetlinewidth{1.003750pt}%
\definecolor{currentstroke}{rgb}{0.168627,0.670588,0.494118}%
\pgfsetstrokecolor{currentstroke}%
\pgfsetdash{}{0pt}%
\pgfpathmoveto{\pgfqpoint{0.571425in}{0.802848in}}%
\pgfpathlineto{\pgfqpoint{0.590645in}{1.450288in}}%
\pgfpathlineto{\pgfqpoint{0.609865in}{1.891724in}}%
\pgfpathlineto{\pgfqpoint{0.629084in}{2.274302in}}%
\pgfpathlineto{\pgfqpoint{0.648304in}{2.470495in}}%
\pgfpathlineto{\pgfqpoint{0.667524in}{2.627450in}}%
\pgfpathlineto{\pgfqpoint{0.686743in}{2.754976in}}%
\pgfpathlineto{\pgfqpoint{0.705963in}{2.862883in}}%
\pgfpathlineto{\pgfqpoint{0.725182in}{2.921741in}}%
\pgfpathlineto{\pgfqpoint{0.744402in}{2.990409in}}%
\pgfpathlineto{\pgfqpoint{0.763622in}{3.010028in}}%
\pgfpathlineto{\pgfqpoint{0.782841in}{3.059077in}}%
\pgfpathlineto{\pgfqpoint{0.802061in}{3.078696in}}%
\pgfpathlineto{\pgfqpoint{0.821281in}{3.137554in}}%
\pgfpathlineto{\pgfqpoint{0.840500in}{3.206222in}}%
\pgfpathlineto{\pgfqpoint{0.859720in}{3.255271in}}%
\pgfpathlineto{\pgfqpoint{0.878939in}{3.274890in}}%
\pgfpathlineto{\pgfqpoint{0.898159in}{3.294509in}}%
\pgfpathlineto{\pgfqpoint{0.917379in}{3.304319in}}%
\pgfpathlineto{\pgfqpoint{0.936598in}{3.323938in}}%
\pgfpathlineto{\pgfqpoint{0.955818in}{3.353367in}}%
\pgfpathlineto{\pgfqpoint{0.975038in}{3.372987in}}%
\pgfpathlineto{\pgfqpoint{0.994257in}{3.372987in}}%
\pgfpathlineto{\pgfqpoint{1.013477in}{3.372987in}}%
\pgfpathlineto{\pgfqpoint{1.032697in}{3.372987in}}%
\pgfpathlineto{\pgfqpoint{1.051916in}{3.382797in}}%
\pgfpathlineto{\pgfqpoint{1.071136in}{3.382797in}}%
\pgfpathlineto{\pgfqpoint{1.090355in}{3.382797in}}%
\pgfpathlineto{\pgfqpoint{1.109575in}{3.382797in}}%
\pgfpathlineto{\pgfqpoint{1.128795in}{3.382797in}}%
\pgfpathlineto{\pgfqpoint{1.148014in}{3.382797in}}%
\pgfpathlineto{\pgfqpoint{1.167234in}{3.382797in}}%
\pgfpathlineto{\pgfqpoint{1.186454in}{3.382797in}}%
\pgfpathlineto{\pgfqpoint{1.205673in}{3.382797in}}%
\pgfpathlineto{\pgfqpoint{1.224893in}{3.382797in}}%
\pgfpathlineto{\pgfqpoint{1.244113in}{3.382797in}}%
\pgfpathlineto{\pgfqpoint{1.263332in}{3.382797in}}%
\pgfpathlineto{\pgfqpoint{1.282552in}{3.382797in}}%
\pgfpathlineto{\pgfqpoint{1.301771in}{3.382797in}}%
\pgfpathlineto{\pgfqpoint{1.320991in}{3.382797in}}%
\pgfpathlineto{\pgfqpoint{1.340211in}{3.382797in}}%
\pgfpathlineto{\pgfqpoint{1.359430in}{3.382797in}}%
\pgfpathlineto{\pgfqpoint{1.378650in}{3.382797in}}%
\pgfpathlineto{\pgfqpoint{1.397870in}{3.382797in}}%
\pgfpathlineto{\pgfqpoint{1.417089in}{3.382797in}}%
\pgfpathlineto{\pgfqpoint{1.436309in}{3.382797in}}%
\pgfpathlineto{\pgfqpoint{1.455529in}{3.382797in}}%
\pgfpathlineto{\pgfqpoint{1.474748in}{3.382797in}}%
\pgfpathlineto{\pgfqpoint{1.493968in}{3.382797in}}%
\pgfpathlineto{\pgfqpoint{1.513187in}{3.392606in}}%
\pgfusepath{stroke}%
\end{pgfscope}%
\begin{pgfscope}%
\pgfpathrectangle{\pgfqpoint{0.566985in}{0.528177in}}{\pgfqpoint{2.106702in}{2.864429in}} %
\pgfusepath{clip}%
\pgfsetroundcap%
\pgfsetroundjoin%
\pgfsetlinewidth{1.003750pt}%
\definecolor{currentstroke}{rgb}{1.000000,0.494118,0.250980}%
\pgfsetstrokecolor{currentstroke}%
\pgfsetdash{}{0pt}%
\pgfpathmoveto{\pgfqpoint{0.571904in}{0.576597in}}%
\pgfpathlineto{\pgfqpoint{0.577963in}{1.251929in}}%
\pgfpathlineto{\pgfqpoint{0.584022in}{1.621451in}}%
\pgfpathlineto{\pgfqpoint{0.590080in}{1.927262in}}%
\pgfpathlineto{\pgfqpoint{0.596139in}{2.156620in}}%
\pgfpathlineto{\pgfqpoint{0.602198in}{2.373236in}}%
\pgfpathlineto{\pgfqpoint{0.608257in}{2.602595in}}%
\pgfpathlineto{\pgfqpoint{0.614315in}{2.730016in}}%
\pgfpathlineto{\pgfqpoint{0.620374in}{2.831953in}}%
\pgfpathlineto{\pgfqpoint{0.626433in}{2.946632in}}%
\pgfpathlineto{\pgfqpoint{0.632491in}{3.010343in}}%
\pgfpathlineto{\pgfqpoint{0.638550in}{3.061311in}}%
\pgfpathlineto{\pgfqpoint{0.644609in}{3.086795in}}%
\pgfpathlineto{\pgfqpoint{0.650668in}{3.112280in}}%
\pgfpathlineto{\pgfqpoint{0.656726in}{3.137764in}}%
\pgfpathlineto{\pgfqpoint{0.662785in}{3.150506in}}%
\pgfpathlineto{\pgfqpoint{0.668844in}{3.150506in}}%
\pgfpathlineto{\pgfqpoint{0.674903in}{3.150506in}}%
\pgfpathlineto{\pgfqpoint{0.680961in}{3.163248in}}%
\pgfpathlineto{\pgfqpoint{0.687020in}{3.188732in}}%
\pgfpathlineto{\pgfqpoint{0.693079in}{3.201474in}}%
\pgfpathlineto{\pgfqpoint{0.699137in}{3.214217in}}%
\pgfpathlineto{\pgfqpoint{0.705196in}{3.239701in}}%
\pgfpathlineto{\pgfqpoint{0.711255in}{3.239701in}}%
\pgfpathlineto{\pgfqpoint{0.717314in}{3.252443in}}%
\pgfpathlineto{\pgfqpoint{0.723372in}{3.252443in}}%
\pgfpathlineto{\pgfqpoint{0.729431in}{3.290669in}}%
\pgfpathlineto{\pgfqpoint{0.735490in}{3.303411in}}%
\pgfpathlineto{\pgfqpoint{0.741548in}{3.316153in}}%
\pgfpathlineto{\pgfqpoint{0.747607in}{3.316153in}}%
\pgfpathlineto{\pgfqpoint{0.753666in}{3.354380in}}%
\pgfpathlineto{\pgfqpoint{0.759725in}{3.354380in}}%
\pgfpathlineto{\pgfqpoint{0.765783in}{3.354380in}}%
\pgfpathlineto{\pgfqpoint{0.771842in}{3.354380in}}%
\pgfpathlineto{\pgfqpoint{0.777901in}{3.354380in}}%
\pgfpathlineto{\pgfqpoint{0.783960in}{3.354380in}}%
\pgfpathlineto{\pgfqpoint{0.790018in}{3.354380in}}%
\pgfpathlineto{\pgfqpoint{0.796077in}{3.354380in}}%
\pgfpathlineto{\pgfqpoint{0.802136in}{3.354380in}}%
\pgfpathlineto{\pgfqpoint{0.808194in}{3.354380in}}%
\pgfpathlineto{\pgfqpoint{0.814253in}{3.354380in}}%
\pgfpathlineto{\pgfqpoint{0.820312in}{3.354380in}}%
\pgfpathlineto{\pgfqpoint{0.826371in}{3.354380in}}%
\pgfpathlineto{\pgfqpoint{0.832429in}{3.354380in}}%
\pgfpathlineto{\pgfqpoint{0.838488in}{3.354380in}}%
\pgfpathlineto{\pgfqpoint{0.844547in}{3.354380in}}%
\pgfpathlineto{\pgfqpoint{0.850606in}{3.367122in}}%
\pgfpathlineto{\pgfqpoint{0.856664in}{3.367122in}}%
\pgfpathlineto{\pgfqpoint{0.862723in}{3.367122in}}%
\pgfpathlineto{\pgfqpoint{0.868782in}{3.392606in}}%
\pgfusepath{stroke}%
\end{pgfscope}%
\begin{pgfscope}%
\pgfpathrectangle{\pgfqpoint{0.566985in}{0.528177in}}{\pgfqpoint{2.106702in}{2.864429in}} %
\pgfusepath{clip}%
\pgfsetroundcap%
\pgfsetroundjoin%
\pgfsetlinewidth{1.003750pt}%
\definecolor{currentstroke}{rgb}{1.000000,0.694118,0.250980}%
\pgfsetstrokecolor{currentstroke}%
\pgfsetdash{}{0pt}%
\pgfpathmoveto{\pgfqpoint{0.571198in}{1.022829in}}%
\pgfpathlineto{\pgfqpoint{0.591761in}{1.747215in}}%
\pgfpathlineto{\pgfqpoint{0.612324in}{2.233589in}}%
\pgfpathlineto{\pgfqpoint{0.632887in}{2.554388in}}%
\pgfpathlineto{\pgfqpoint{0.653450in}{2.813097in}}%
\pgfpathlineto{\pgfqpoint{0.674013in}{2.989020in}}%
\pgfpathlineto{\pgfqpoint{0.694576in}{3.061458in}}%
\pgfpathlineto{\pgfqpoint{0.715139in}{3.133897in}}%
\pgfpathlineto{\pgfqpoint{0.735702in}{3.175290in}}%
\pgfpathlineto{\pgfqpoint{0.756265in}{3.237381in}}%
\pgfpathlineto{\pgfqpoint{0.776829in}{3.237381in}}%
\pgfpathlineto{\pgfqpoint{0.797392in}{3.258077in}}%
\pgfpathlineto{\pgfqpoint{0.817955in}{3.268426in}}%
\pgfpathlineto{\pgfqpoint{0.838518in}{3.268426in}}%
\pgfpathlineto{\pgfqpoint{0.859081in}{3.268426in}}%
\pgfpathlineto{\pgfqpoint{0.879644in}{3.268426in}}%
\pgfpathlineto{\pgfqpoint{0.900207in}{3.289123in}}%
\pgfpathlineto{\pgfqpoint{0.920770in}{3.299471in}}%
\pgfpathlineto{\pgfqpoint{0.941333in}{3.299471in}}%
\pgfpathlineto{\pgfqpoint{0.961896in}{3.320168in}}%
\pgfpathlineto{\pgfqpoint{0.982459in}{3.320168in}}%
\pgfpathlineto{\pgfqpoint{1.003023in}{3.330516in}}%
\pgfpathlineto{\pgfqpoint{1.023586in}{3.340864in}}%
\pgfpathlineto{\pgfqpoint{1.044149in}{3.340864in}}%
\pgfpathlineto{\pgfqpoint{1.064712in}{3.351213in}}%
\pgfpathlineto{\pgfqpoint{1.085275in}{3.361561in}}%
\pgfpathlineto{\pgfqpoint{1.105838in}{3.361561in}}%
\pgfpathlineto{\pgfqpoint{1.126401in}{3.371909in}}%
\pgfpathlineto{\pgfqpoint{1.146964in}{3.382258in}}%
\pgfpathlineto{\pgfqpoint{1.167527in}{3.382258in}}%
\pgfpathlineto{\pgfqpoint{1.188090in}{3.382258in}}%
\pgfpathlineto{\pgfqpoint{1.208653in}{3.382258in}}%
\pgfpathlineto{\pgfqpoint{1.229217in}{3.382258in}}%
\pgfpathlineto{\pgfqpoint{1.249780in}{3.382258in}}%
\pgfpathlineto{\pgfqpoint{1.270343in}{3.382258in}}%
\pgfpathlineto{\pgfqpoint{1.290906in}{3.382258in}}%
\pgfpathlineto{\pgfqpoint{1.311469in}{3.382258in}}%
\pgfpathlineto{\pgfqpoint{1.332032in}{3.382258in}}%
\pgfpathlineto{\pgfqpoint{1.352595in}{3.382258in}}%
\pgfpathlineto{\pgfqpoint{1.373158in}{3.382258in}}%
\pgfpathlineto{\pgfqpoint{1.393721in}{3.382258in}}%
\pgfpathlineto{\pgfqpoint{1.414284in}{3.382258in}}%
\pgfpathlineto{\pgfqpoint{1.434847in}{3.382258in}}%
\pgfpathlineto{\pgfqpoint{1.455411in}{3.382258in}}%
\pgfpathlineto{\pgfqpoint{1.475974in}{3.382258in}}%
\pgfpathlineto{\pgfqpoint{1.496537in}{3.382258in}}%
\pgfpathlineto{\pgfqpoint{1.517100in}{3.382258in}}%
\pgfpathlineto{\pgfqpoint{1.537663in}{3.382258in}}%
\pgfpathlineto{\pgfqpoint{1.558226in}{3.382258in}}%
\pgfpathlineto{\pgfqpoint{1.578789in}{3.392606in}}%
\pgfusepath{stroke}%
\end{pgfscope}%
\begin{pgfscope}%
\pgfsetrectcap%
\pgfsetmiterjoin%
\pgfsetlinewidth{1.254687pt}%
\definecolor{currentstroke}{rgb}{0.150000,0.150000,0.150000}%
\pgfsetstrokecolor{currentstroke}%
\pgfsetdash{}{0pt}%
\pgfpathmoveto{\pgfqpoint{0.566985in}{0.528177in}}%
\pgfpathlineto{\pgfqpoint{0.566985in}{3.392606in}}%
\pgfusepath{stroke}%
\end{pgfscope}%
\begin{pgfscope}%
\pgfsetrectcap%
\pgfsetmiterjoin%
\pgfsetlinewidth{1.254687pt}%
\definecolor{currentstroke}{rgb}{0.150000,0.150000,0.150000}%
\pgfsetstrokecolor{currentstroke}%
\pgfsetdash{}{0pt}%
\pgfpathmoveto{\pgfqpoint{0.566985in}{0.528177in}}%
\pgfpathlineto{\pgfqpoint{2.673686in}{0.528177in}}%
\pgfusepath{stroke}%
\end{pgfscope}%
\begin{pgfscope}%
\pgfsetbuttcap%
\pgfsetmiterjoin%
\definecolor{currentfill}{rgb}{1.000000,1.000000,1.000000}%
\pgfsetfillcolor{currentfill}%
\pgfsetlinewidth{0.000000pt}%
\definecolor{currentstroke}{rgb}{0.000000,0.000000,0.000000}%
\pgfsetstrokecolor{currentstroke}%
\pgfsetstrokeopacity{0.000000}%
\pgfsetdash{}{0pt}%
\pgfpathmoveto{\pgfqpoint{3.095027in}{0.528177in}}%
\pgfpathlineto{\pgfqpoint{5.201729in}{0.528177in}}%
\pgfpathlineto{\pgfqpoint{5.201729in}{2.653399in}}%
\pgfpathlineto{\pgfqpoint{3.095027in}{2.653399in}}%
\pgfpathclose%
\pgfusepath{fill}%
\end{pgfscope}%
\begin{pgfscope}%
\pgfsetroundcap%
\pgfsetroundjoin%
\pgfsetlinewidth{1.003750pt}%
\definecolor{currentstroke}{rgb}{0.200000,0.427451,0.650980}%
\pgfsetstrokecolor{currentstroke}%
\pgfsetdash{}{0pt}%
\pgfpathmoveto{\pgfqpoint{2.984357in}{3.319977in}}%
\pgfpathlineto{\pgfqpoint{3.095468in}{3.319977in}}%
\pgfusepath{stroke}%
\end{pgfscope}%
\begin{pgfscope}%
\definecolor{textcolor}{rgb}{1.000000,1.000000,1.000000}%
\pgfsetstrokecolor{textcolor}%
\pgfsetfillcolor{textcolor}%
\pgftext[x=3.184357in,y=3.281088in,left,base]{\color{textcolor}\rmfamily\fontsize{8.000000}{9.600000}\selectfont WT + Vehicle (494)}%
\end{pgfscope}%
\begin{pgfscope}%
\pgfsetroundcap%
\pgfsetroundjoin%
\pgfsetlinewidth{1.003750pt}%
\definecolor{currentstroke}{rgb}{0.168627,0.670588,0.494118}%
\pgfsetstrokecolor{currentstroke}%
\pgfsetdash{}{0pt}%
\pgfpathmoveto{\pgfqpoint{2.984357in}{3.153338in}}%
\pgfpathlineto{\pgfqpoint{3.095468in}{3.153338in}}%
\pgfusepath{stroke}%
\end{pgfscope}%
\begin{pgfscope}%
\definecolor{textcolor}{rgb}{1.000000,1.000000,1.000000}%
\pgfsetstrokecolor{textcolor}%
\pgfsetfillcolor{textcolor}%
\pgftext[x=3.184357in,y=3.114449in,left,base]{\color{textcolor}\rmfamily\fontsize{8.000000}{9.600000}\selectfont WT + TAT-GluA2\textsubscript{3Y} (365)}%
\end{pgfscope}%
\begin{pgfscope}%
\pgfsetroundcap%
\pgfsetroundjoin%
\pgfsetlinewidth{1.003750pt}%
\definecolor{currentstroke}{rgb}{1.000000,0.494118,0.250980}%
\pgfsetstrokecolor{currentstroke}%
\pgfsetdash{}{0pt}%
\pgfpathmoveto{\pgfqpoint{2.984357in}{2.986698in}}%
\pgfpathlineto{\pgfqpoint{3.095468in}{2.986698in}}%
\pgfusepath{stroke}%
\end{pgfscope}%
\begin{pgfscope}%
\definecolor{textcolor}{rgb}{1.000000,1.000000,1.000000}%
\pgfsetstrokecolor{textcolor}%
\pgfsetfillcolor{textcolor}%
\pgftext[x=3.184357in,y=2.947809in,left,base]{\color{textcolor}\rmfamily\fontsize{8.000000}{9.600000}\selectfont Tg + Vehicle (281)}%
\end{pgfscope}%
\begin{pgfscope}%
\pgfsetroundcap%
\pgfsetroundjoin%
\pgfsetlinewidth{1.003750pt}%
\definecolor{currentstroke}{rgb}{1.000000,0.694118,0.250980}%
\pgfsetstrokecolor{currentstroke}%
\pgfsetdash{}{0pt}%
\pgfpathmoveto{\pgfqpoint{2.984357in}{2.820059in}}%
\pgfpathlineto{\pgfqpoint{3.095468in}{2.820059in}}%
\pgfusepath{stroke}%
\end{pgfscope}%
\begin{pgfscope}%
\definecolor{textcolor}{rgb}{1.000000,1.000000,1.000000}%
\pgfsetstrokecolor{textcolor}%
\pgfsetfillcolor{textcolor}%
\pgftext[x=3.184357in,y=2.781170in,left,base]{\color{textcolor}\rmfamily\fontsize{8.000000}{9.600000}\selectfont Tg + TAT-GluA2\textsubscript{3Y} (346)}%
\end{pgfscope}%
\begin{pgfscope}%
\pgfsetroundcap%
\pgfsetroundjoin%
\pgfsetlinewidth{1.003750pt}%
\definecolor{currentstroke}{rgb}{0.200000,0.427451,0.650980}%
\pgfsetstrokecolor{currentstroke}%
\pgfsetdash{}{0pt}%
\pgfpathmoveto{\pgfqpoint{2.984357in}{3.319977in}}%
\pgfpathlineto{\pgfqpoint{3.095468in}{3.319977in}}%
\pgfusepath{stroke}%
\end{pgfscope}%
\begin{pgfscope}%
\definecolor{textcolor}{rgb}{1.000000,1.000000,1.000000}%
\pgfsetstrokecolor{textcolor}%
\pgfsetfillcolor{textcolor}%
\pgftext[x=3.184357in,y=3.281088in,left,base]{\color{textcolor}\rmfamily\fontsize{8.000000}{9.600000}\selectfont WT + Vehicle (494)}%
\end{pgfscope}%
\begin{pgfscope}%
\pgfsetroundcap%
\pgfsetroundjoin%
\pgfsetlinewidth{1.003750pt}%
\definecolor{currentstroke}{rgb}{0.168627,0.670588,0.494118}%
\pgfsetstrokecolor{currentstroke}%
\pgfsetdash{}{0pt}%
\pgfpathmoveto{\pgfqpoint{2.984357in}{3.153338in}}%
\pgfpathlineto{\pgfqpoint{3.095468in}{3.153338in}}%
\pgfusepath{stroke}%
\end{pgfscope}%
\begin{pgfscope}%
\definecolor{textcolor}{rgb}{1.000000,1.000000,1.000000}%
\pgfsetstrokecolor{textcolor}%
\pgfsetfillcolor{textcolor}%
\pgftext[x=3.184357in,y=3.114449in,left,base]{\color{textcolor}\rmfamily\fontsize{8.000000}{9.600000}\selectfont WT + TAT-GluA2\textsubscript{3Y} (365)}%
\end{pgfscope}%
\begin{pgfscope}%
\pgfsetroundcap%
\pgfsetroundjoin%
\pgfsetlinewidth{1.003750pt}%
\definecolor{currentstroke}{rgb}{1.000000,0.494118,0.250980}%
\pgfsetstrokecolor{currentstroke}%
\pgfsetdash{}{0pt}%
\pgfpathmoveto{\pgfqpoint{2.984357in}{2.986698in}}%
\pgfpathlineto{\pgfqpoint{3.095468in}{2.986698in}}%
\pgfusepath{stroke}%
\end{pgfscope}%
\begin{pgfscope}%
\definecolor{textcolor}{rgb}{1.000000,1.000000,1.000000}%
\pgfsetstrokecolor{textcolor}%
\pgfsetfillcolor{textcolor}%
\pgftext[x=3.184357in,y=2.947809in,left,base]{\color{textcolor}\rmfamily\fontsize{8.000000}{9.600000}\selectfont Tg + Vehicle (281)}%
\end{pgfscope}%
\begin{pgfscope}%
\pgfsetroundcap%
\pgfsetroundjoin%
\pgfsetlinewidth{1.003750pt}%
\definecolor{currentstroke}{rgb}{1.000000,0.694118,0.250980}%
\pgfsetstrokecolor{currentstroke}%
\pgfsetdash{}{0pt}%
\pgfpathmoveto{\pgfqpoint{2.984357in}{2.820059in}}%
\pgfpathlineto{\pgfqpoint{3.095468in}{2.820059in}}%
\pgfusepath{stroke}%
\end{pgfscope}%
\begin{pgfscope}%
\definecolor{textcolor}{rgb}{1.000000,1.000000,1.000000}%
\pgfsetstrokecolor{textcolor}%
\pgfsetfillcolor{textcolor}%
\pgftext[x=3.184357in,y=2.781170in,left,base]{\color{textcolor}\rmfamily\fontsize{8.000000}{9.600000}\selectfont Tg + TAT-GluA2\textsubscript{3Y} (346)}%
\end{pgfscope}%
\begin{pgfscope}%
\pgfsetbuttcap%
\pgfsetroundjoin%
\definecolor{currentfill}{rgb}{0.150000,0.150000,0.150000}%
\pgfsetfillcolor{currentfill}%
\pgfsetlinewidth{1.003750pt}%
\definecolor{currentstroke}{rgb}{0.150000,0.150000,0.150000}%
\pgfsetstrokecolor{currentstroke}%
\pgfsetdash{}{0pt}%
\pgfsys@defobject{currentmarker}{\pgfqpoint{0.000000in}{0.000000in}}{\pgfqpoint{0.041667in}{0.000000in}}{%
\pgfpathmoveto{\pgfqpoint{0.000000in}{0.000000in}}%
\pgfpathlineto{\pgfqpoint{0.041667in}{0.000000in}}%
\pgfusepath{stroke,fill}%
}%
\begin{pgfscope}%
\pgfsys@transformshift{3.095027in}{0.528177in}%
\pgfsys@useobject{currentmarker}{}%
\end{pgfscope}%
\end{pgfscope}%
\begin{pgfscope}%
\definecolor{textcolor}{rgb}{0.150000,0.150000,0.150000}%
\pgfsetstrokecolor{textcolor}%
\pgfsetfillcolor{textcolor}%
\pgftext[x=2.997805in,y=0.528177in,right,]{\color{textcolor}\rmfamily\fontsize{10.000000}{12.000000}\selectfont \(\displaystyle 0\)}%
\end{pgfscope}%
\begin{pgfscope}%
\pgfsetbuttcap%
\pgfsetroundjoin%
\definecolor{currentfill}{rgb}{0.150000,0.150000,0.150000}%
\pgfsetfillcolor{currentfill}%
\pgfsetlinewidth{1.003750pt}%
\definecolor{currentstroke}{rgb}{0.150000,0.150000,0.150000}%
\pgfsetstrokecolor{currentstroke}%
\pgfsetdash{}{0pt}%
\pgfsys@defobject{currentmarker}{\pgfqpoint{0.000000in}{0.000000in}}{\pgfqpoint{0.041667in}{0.000000in}}{%
\pgfpathmoveto{\pgfqpoint{0.000000in}{0.000000in}}%
\pgfpathlineto{\pgfqpoint{0.041667in}{0.000000in}}%
\pgfusepath{stroke,fill}%
}%
\begin{pgfscope}%
\pgfsys@transformshift{3.095027in}{0.793830in}%
\pgfsys@useobject{currentmarker}{}%
\end{pgfscope}%
\end{pgfscope}%
\begin{pgfscope}%
\definecolor{textcolor}{rgb}{0.150000,0.150000,0.150000}%
\pgfsetstrokecolor{textcolor}%
\pgfsetfillcolor{textcolor}%
\pgftext[x=2.997805in,y=0.793830in,right,]{\color{textcolor}\rmfamily\fontsize{10.000000}{12.000000}\selectfont \(\displaystyle 1\)}%
\end{pgfscope}%
\begin{pgfscope}%
\pgfsetbuttcap%
\pgfsetroundjoin%
\definecolor{currentfill}{rgb}{0.150000,0.150000,0.150000}%
\pgfsetfillcolor{currentfill}%
\pgfsetlinewidth{1.003750pt}%
\definecolor{currentstroke}{rgb}{0.150000,0.150000,0.150000}%
\pgfsetstrokecolor{currentstroke}%
\pgfsetdash{}{0pt}%
\pgfsys@defobject{currentmarker}{\pgfqpoint{0.000000in}{0.000000in}}{\pgfqpoint{0.041667in}{0.000000in}}{%
\pgfpathmoveto{\pgfqpoint{0.000000in}{0.000000in}}%
\pgfpathlineto{\pgfqpoint{0.041667in}{0.000000in}}%
\pgfusepath{stroke,fill}%
}%
\begin{pgfscope}%
\pgfsys@transformshift{3.095027in}{1.059482in}%
\pgfsys@useobject{currentmarker}{}%
\end{pgfscope}%
\end{pgfscope}%
\begin{pgfscope}%
\definecolor{textcolor}{rgb}{0.150000,0.150000,0.150000}%
\pgfsetstrokecolor{textcolor}%
\pgfsetfillcolor{textcolor}%
\pgftext[x=2.997805in,y=1.059482in,right,]{\color{textcolor}\rmfamily\fontsize{10.000000}{12.000000}\selectfont \(\displaystyle 2\)}%
\end{pgfscope}%
\begin{pgfscope}%
\pgfsetbuttcap%
\pgfsetroundjoin%
\definecolor{currentfill}{rgb}{0.150000,0.150000,0.150000}%
\pgfsetfillcolor{currentfill}%
\pgfsetlinewidth{1.003750pt}%
\definecolor{currentstroke}{rgb}{0.150000,0.150000,0.150000}%
\pgfsetstrokecolor{currentstroke}%
\pgfsetdash{}{0pt}%
\pgfsys@defobject{currentmarker}{\pgfqpoint{0.000000in}{0.000000in}}{\pgfqpoint{0.041667in}{0.000000in}}{%
\pgfpathmoveto{\pgfqpoint{0.000000in}{0.000000in}}%
\pgfpathlineto{\pgfqpoint{0.041667in}{0.000000in}}%
\pgfusepath{stroke,fill}%
}%
\begin{pgfscope}%
\pgfsys@transformshift{3.095027in}{1.325135in}%
\pgfsys@useobject{currentmarker}{}%
\end{pgfscope}%
\end{pgfscope}%
\begin{pgfscope}%
\definecolor{textcolor}{rgb}{0.150000,0.150000,0.150000}%
\pgfsetstrokecolor{textcolor}%
\pgfsetfillcolor{textcolor}%
\pgftext[x=2.997805in,y=1.325135in,right,]{\color{textcolor}\rmfamily\fontsize{10.000000}{12.000000}\selectfont \(\displaystyle 3\)}%
\end{pgfscope}%
\begin{pgfscope}%
\pgfsetbuttcap%
\pgfsetroundjoin%
\definecolor{currentfill}{rgb}{0.150000,0.150000,0.150000}%
\pgfsetfillcolor{currentfill}%
\pgfsetlinewidth{1.003750pt}%
\definecolor{currentstroke}{rgb}{0.150000,0.150000,0.150000}%
\pgfsetstrokecolor{currentstroke}%
\pgfsetdash{}{0pt}%
\pgfsys@defobject{currentmarker}{\pgfqpoint{0.000000in}{0.000000in}}{\pgfqpoint{0.041667in}{0.000000in}}{%
\pgfpathmoveto{\pgfqpoint{0.000000in}{0.000000in}}%
\pgfpathlineto{\pgfqpoint{0.041667in}{0.000000in}}%
\pgfusepath{stroke,fill}%
}%
\begin{pgfscope}%
\pgfsys@transformshift{3.095027in}{1.590788in}%
\pgfsys@useobject{currentmarker}{}%
\end{pgfscope}%
\end{pgfscope}%
\begin{pgfscope}%
\definecolor{textcolor}{rgb}{0.150000,0.150000,0.150000}%
\pgfsetstrokecolor{textcolor}%
\pgfsetfillcolor{textcolor}%
\pgftext[x=2.997805in,y=1.590788in,right,]{\color{textcolor}\rmfamily\fontsize{10.000000}{12.000000}\selectfont \(\displaystyle 4\)}%
\end{pgfscope}%
\begin{pgfscope}%
\pgfsetbuttcap%
\pgfsetroundjoin%
\definecolor{currentfill}{rgb}{0.150000,0.150000,0.150000}%
\pgfsetfillcolor{currentfill}%
\pgfsetlinewidth{1.003750pt}%
\definecolor{currentstroke}{rgb}{0.150000,0.150000,0.150000}%
\pgfsetstrokecolor{currentstroke}%
\pgfsetdash{}{0pt}%
\pgfsys@defobject{currentmarker}{\pgfqpoint{0.000000in}{0.000000in}}{\pgfqpoint{0.041667in}{0.000000in}}{%
\pgfpathmoveto{\pgfqpoint{0.000000in}{0.000000in}}%
\pgfpathlineto{\pgfqpoint{0.041667in}{0.000000in}}%
\pgfusepath{stroke,fill}%
}%
\begin{pgfscope}%
\pgfsys@transformshift{3.095027in}{1.856440in}%
\pgfsys@useobject{currentmarker}{}%
\end{pgfscope}%
\end{pgfscope}%
\begin{pgfscope}%
\definecolor{textcolor}{rgb}{0.150000,0.150000,0.150000}%
\pgfsetstrokecolor{textcolor}%
\pgfsetfillcolor{textcolor}%
\pgftext[x=2.997805in,y=1.856440in,right,]{\color{textcolor}\rmfamily\fontsize{10.000000}{12.000000}\selectfont \(\displaystyle 5\)}%
\end{pgfscope}%
\begin{pgfscope}%
\pgfsetbuttcap%
\pgfsetroundjoin%
\definecolor{currentfill}{rgb}{0.150000,0.150000,0.150000}%
\pgfsetfillcolor{currentfill}%
\pgfsetlinewidth{1.003750pt}%
\definecolor{currentstroke}{rgb}{0.150000,0.150000,0.150000}%
\pgfsetstrokecolor{currentstroke}%
\pgfsetdash{}{0pt}%
\pgfsys@defobject{currentmarker}{\pgfqpoint{0.000000in}{0.000000in}}{\pgfqpoint{0.041667in}{0.000000in}}{%
\pgfpathmoveto{\pgfqpoint{0.000000in}{0.000000in}}%
\pgfpathlineto{\pgfqpoint{0.041667in}{0.000000in}}%
\pgfusepath{stroke,fill}%
}%
\begin{pgfscope}%
\pgfsys@transformshift{3.095027in}{2.122093in}%
\pgfsys@useobject{currentmarker}{}%
\end{pgfscope}%
\end{pgfscope}%
\begin{pgfscope}%
\definecolor{textcolor}{rgb}{0.150000,0.150000,0.150000}%
\pgfsetstrokecolor{textcolor}%
\pgfsetfillcolor{textcolor}%
\pgftext[x=2.997805in,y=2.122093in,right,]{\color{textcolor}\rmfamily\fontsize{10.000000}{12.000000}\selectfont \(\displaystyle 6\)}%
\end{pgfscope}%
\begin{pgfscope}%
\pgfsetbuttcap%
\pgfsetroundjoin%
\definecolor{currentfill}{rgb}{0.150000,0.150000,0.150000}%
\pgfsetfillcolor{currentfill}%
\pgfsetlinewidth{1.003750pt}%
\definecolor{currentstroke}{rgb}{0.150000,0.150000,0.150000}%
\pgfsetstrokecolor{currentstroke}%
\pgfsetdash{}{0pt}%
\pgfsys@defobject{currentmarker}{\pgfqpoint{0.000000in}{0.000000in}}{\pgfqpoint{0.041667in}{0.000000in}}{%
\pgfpathmoveto{\pgfqpoint{0.000000in}{0.000000in}}%
\pgfpathlineto{\pgfqpoint{0.041667in}{0.000000in}}%
\pgfusepath{stroke,fill}%
}%
\begin{pgfscope}%
\pgfsys@transformshift{3.095027in}{2.387746in}%
\pgfsys@useobject{currentmarker}{}%
\end{pgfscope}%
\end{pgfscope}%
\begin{pgfscope}%
\definecolor{textcolor}{rgb}{0.150000,0.150000,0.150000}%
\pgfsetstrokecolor{textcolor}%
\pgfsetfillcolor{textcolor}%
\pgftext[x=2.997805in,y=2.387746in,right,]{\color{textcolor}\rmfamily\fontsize{10.000000}{12.000000}\selectfont \(\displaystyle 7\)}%
\end{pgfscope}%
\begin{pgfscope}%
\pgfsetbuttcap%
\pgfsetroundjoin%
\definecolor{currentfill}{rgb}{0.150000,0.150000,0.150000}%
\pgfsetfillcolor{currentfill}%
\pgfsetlinewidth{1.003750pt}%
\definecolor{currentstroke}{rgb}{0.150000,0.150000,0.150000}%
\pgfsetstrokecolor{currentstroke}%
\pgfsetdash{}{0pt}%
\pgfsys@defobject{currentmarker}{\pgfqpoint{0.000000in}{0.000000in}}{\pgfqpoint{0.041667in}{0.000000in}}{%
\pgfpathmoveto{\pgfqpoint{0.000000in}{0.000000in}}%
\pgfpathlineto{\pgfqpoint{0.041667in}{0.000000in}}%
\pgfusepath{stroke,fill}%
}%
\begin{pgfscope}%
\pgfsys@transformshift{3.095027in}{2.653399in}%
\pgfsys@useobject{currentmarker}{}%
\end{pgfscope}%
\end{pgfscope}%
\begin{pgfscope}%
\definecolor{textcolor}{rgb}{0.150000,0.150000,0.150000}%
\pgfsetstrokecolor{textcolor}%
\pgfsetfillcolor{textcolor}%
\pgftext[x=2.997805in,y=2.653399in,right,]{\color{textcolor}\rmfamily\fontsize{10.000000}{12.000000}\selectfont \(\displaystyle 8\)}%
\end{pgfscope}%
\begin{pgfscope}%
\definecolor{textcolor}{rgb}{0.150000,0.150000,0.150000}%
\pgfsetstrokecolor{textcolor}%
\pgfsetfillcolor{textcolor}%
\pgftext[x=2.858915in,y=1.590788in,,bottom,rotate=90.000000]{\color{textcolor}\rmfamily\fontsize{10.000000}{12.000000}\selectfont \textbf{Length of freezing bouts (s)}}%
\end{pgfscope}%
\begin{pgfscope}%
\pgfpathrectangle{\pgfqpoint{3.095027in}{0.528177in}}{\pgfqpoint{2.106702in}{2.125222in}} %
\pgfusepath{clip}%
\pgfsetbuttcap%
\pgfsetmiterjoin%
\definecolor{currentfill}{rgb}{0.200000,0.427451,0.650980}%
\pgfsetfillcolor{currentfill}%
\pgfsetlinewidth{1.505625pt}%
\definecolor{currentstroke}{rgb}{0.200000,0.427451,0.650980}%
\pgfsetstrokecolor{currentstroke}%
\pgfsetdash{}{0pt}%
\pgfpathmoveto{\pgfqpoint{3.170266in}{0.528177in}}%
\pgfpathlineto{\pgfqpoint{3.546463in}{0.528177in}}%
\pgfpathlineto{\pgfqpoint{3.546463in}{1.942169in}}%
\pgfpathlineto{\pgfqpoint{3.170266in}{1.942169in}}%
\pgfpathclose%
\pgfusepath{stroke,fill}%
\end{pgfscope}%
\begin{pgfscope}%
\pgfpathrectangle{\pgfqpoint{3.095027in}{0.528177in}}{\pgfqpoint{2.106702in}{2.125222in}} %
\pgfusepath{clip}%
\pgfsetbuttcap%
\pgfsetmiterjoin%
\definecolor{currentfill}{rgb}{0.168627,0.670588,0.494118}%
\pgfsetfillcolor{currentfill}%
\pgfsetlinewidth{1.505625pt}%
\definecolor{currentstroke}{rgb}{0.168627,0.670588,0.494118}%
\pgfsetstrokecolor{currentstroke}%
\pgfsetdash{}{0pt}%
\pgfpathmoveto{\pgfqpoint{3.696942in}{0.528177in}}%
\pgfpathlineto{\pgfqpoint{4.073138in}{0.528177in}}%
\pgfpathlineto{\pgfqpoint{4.073138in}{1.813141in}}%
\pgfpathlineto{\pgfqpoint{3.696942in}{1.813141in}}%
\pgfpathclose%
\pgfusepath{stroke,fill}%
\end{pgfscope}%
\begin{pgfscope}%
\pgfpathrectangle{\pgfqpoint{3.095027in}{0.528177in}}{\pgfqpoint{2.106702in}{2.125222in}} %
\pgfusepath{clip}%
\pgfsetbuttcap%
\pgfsetmiterjoin%
\definecolor{currentfill}{rgb}{1.000000,0.494118,0.250980}%
\pgfsetfillcolor{currentfill}%
\pgfsetlinewidth{1.505625pt}%
\definecolor{currentstroke}{rgb}{1.000000,0.494118,0.250980}%
\pgfsetstrokecolor{currentstroke}%
\pgfsetdash{}{0pt}%
\pgfpathmoveto{\pgfqpoint{4.223617in}{0.528177in}}%
\pgfpathlineto{\pgfqpoint{4.599814in}{0.528177in}}%
\pgfpathlineto{\pgfqpoint{4.599814in}{1.096379in}}%
\pgfpathlineto{\pgfqpoint{4.223617in}{1.096379in}}%
\pgfpathclose%
\pgfusepath{stroke,fill}%
\end{pgfscope}%
\begin{pgfscope}%
\pgfpathrectangle{\pgfqpoint{3.095027in}{0.528177in}}{\pgfqpoint{2.106702in}{2.125222in}} %
\pgfusepath{clip}%
\pgfsetbuttcap%
\pgfsetmiterjoin%
\definecolor{currentfill}{rgb}{1.000000,0.694118,0.250980}%
\pgfsetfillcolor{currentfill}%
\pgfsetlinewidth{1.505625pt}%
\definecolor{currentstroke}{rgb}{1.000000,0.694118,0.250980}%
\pgfsetstrokecolor{currentstroke}%
\pgfsetdash{}{0pt}%
\pgfpathmoveto{\pgfqpoint{4.750293in}{0.528177in}}%
\pgfpathlineto{\pgfqpoint{5.126489in}{0.528177in}}%
\pgfpathlineto{\pgfqpoint{5.126489in}{1.583409in}}%
\pgfpathlineto{\pgfqpoint{4.750293in}{1.583409in}}%
\pgfpathclose%
\pgfusepath{stroke,fill}%
\end{pgfscope}%
\begin{pgfscope}%
\pgfpathrectangle{\pgfqpoint{3.095027in}{0.528177in}}{\pgfqpoint{2.106702in}{2.125222in}} %
\pgfusepath{clip}%
\pgfsetbuttcap%
\pgfsetroundjoin%
\pgfsetlinewidth{1.505625pt}%
\definecolor{currentstroke}{rgb}{0.200000,0.427451,0.650980}%
\pgfsetstrokecolor{currentstroke}%
\pgfsetdash{}{0pt}%
\pgfpathmoveto{\pgfqpoint{3.358365in}{1.942169in}}%
\pgfpathlineto{\pgfqpoint{3.358365in}{2.037176in}}%
\pgfusepath{stroke}%
\end{pgfscope}%
\begin{pgfscope}%
\pgfpathrectangle{\pgfqpoint{3.095027in}{0.528177in}}{\pgfqpoint{2.106702in}{2.125222in}} %
\pgfusepath{clip}%
\pgfsetbuttcap%
\pgfsetroundjoin%
\pgfsetlinewidth{1.505625pt}%
\definecolor{currentstroke}{rgb}{0.168627,0.670588,0.494118}%
\pgfsetstrokecolor{currentstroke}%
\pgfsetdash{}{0pt}%
\pgfpathmoveto{\pgfqpoint{3.885040in}{1.813141in}}%
\pgfpathlineto{\pgfqpoint{3.885040in}{1.894309in}}%
\pgfusepath{stroke}%
\end{pgfscope}%
\begin{pgfscope}%
\pgfpathrectangle{\pgfqpoint{3.095027in}{0.528177in}}{\pgfqpoint{2.106702in}{2.125222in}} %
\pgfusepath{clip}%
\pgfsetbuttcap%
\pgfsetroundjoin%
\pgfsetlinewidth{1.505625pt}%
\definecolor{currentstroke}{rgb}{1.000000,0.494118,0.250980}%
\pgfsetstrokecolor{currentstroke}%
\pgfsetdash{}{0pt}%
\pgfpathmoveto{\pgfqpoint{4.411716in}{1.096379in}}%
\pgfpathlineto{\pgfqpoint{4.411716in}{1.137162in}}%
\pgfusepath{stroke}%
\end{pgfscope}%
\begin{pgfscope}%
\pgfpathrectangle{\pgfqpoint{3.095027in}{0.528177in}}{\pgfqpoint{2.106702in}{2.125222in}} %
\pgfusepath{clip}%
\pgfsetbuttcap%
\pgfsetroundjoin%
\pgfsetlinewidth{1.505625pt}%
\definecolor{currentstroke}{rgb}{1.000000,0.694118,0.250980}%
\pgfsetstrokecolor{currentstroke}%
\pgfsetdash{}{0pt}%
\pgfpathmoveto{\pgfqpoint{4.938391in}{1.583409in}}%
\pgfpathlineto{\pgfqpoint{4.938391in}{1.664874in}}%
\pgfusepath{stroke}%
\end{pgfscope}%
\begin{pgfscope}%
\pgfpathrectangle{\pgfqpoint{3.095027in}{0.528177in}}{\pgfqpoint{2.106702in}{2.125222in}} %
\pgfusepath{clip}%
\pgfsetbuttcap%
\pgfsetroundjoin%
\definecolor{currentfill}{rgb}{0.200000,0.427451,0.650980}%
\pgfsetfillcolor{currentfill}%
\pgfsetlinewidth{1.505625pt}%
\definecolor{currentstroke}{rgb}{0.200000,0.427451,0.650980}%
\pgfsetstrokecolor{currentstroke}%
\pgfsetdash{}{0pt}%
\pgfsys@defobject{currentmarker}{\pgfqpoint{-0.111111in}{-0.000000in}}{\pgfqpoint{0.111111in}{0.000000in}}{%
\pgfpathmoveto{\pgfqpoint{0.111111in}{-0.000000in}}%
\pgfpathlineto{\pgfqpoint{-0.111111in}{0.000000in}}%
\pgfusepath{stroke,fill}%
}%
\begin{pgfscope}%
\pgfsys@transformshift{3.358365in}{1.942169in}%
\pgfsys@useobject{currentmarker}{}%
\end{pgfscope}%
\end{pgfscope}%
\begin{pgfscope}%
\pgfpathrectangle{\pgfqpoint{3.095027in}{0.528177in}}{\pgfqpoint{2.106702in}{2.125222in}} %
\pgfusepath{clip}%
\pgfsetbuttcap%
\pgfsetroundjoin%
\definecolor{currentfill}{rgb}{0.200000,0.427451,0.650980}%
\pgfsetfillcolor{currentfill}%
\pgfsetlinewidth{1.505625pt}%
\definecolor{currentstroke}{rgb}{0.200000,0.427451,0.650980}%
\pgfsetstrokecolor{currentstroke}%
\pgfsetdash{}{0pt}%
\pgfsys@defobject{currentmarker}{\pgfqpoint{-0.111111in}{-0.000000in}}{\pgfqpoint{0.111111in}{0.000000in}}{%
\pgfpathmoveto{\pgfqpoint{0.111111in}{-0.000000in}}%
\pgfpathlineto{\pgfqpoint{-0.111111in}{0.000000in}}%
\pgfusepath{stroke,fill}%
}%
\begin{pgfscope}%
\pgfsys@transformshift{3.358365in}{2.037176in}%
\pgfsys@useobject{currentmarker}{}%
\end{pgfscope}%
\end{pgfscope}%
\begin{pgfscope}%
\pgfpathrectangle{\pgfqpoint{3.095027in}{0.528177in}}{\pgfqpoint{2.106702in}{2.125222in}} %
\pgfusepath{clip}%
\pgfsetbuttcap%
\pgfsetroundjoin%
\definecolor{currentfill}{rgb}{0.168627,0.670588,0.494118}%
\pgfsetfillcolor{currentfill}%
\pgfsetlinewidth{1.505625pt}%
\definecolor{currentstroke}{rgb}{0.168627,0.670588,0.494118}%
\pgfsetstrokecolor{currentstroke}%
\pgfsetdash{}{0pt}%
\pgfsys@defobject{currentmarker}{\pgfqpoint{-0.111111in}{-0.000000in}}{\pgfqpoint{0.111111in}{0.000000in}}{%
\pgfpathmoveto{\pgfqpoint{0.111111in}{-0.000000in}}%
\pgfpathlineto{\pgfqpoint{-0.111111in}{0.000000in}}%
\pgfusepath{stroke,fill}%
}%
\begin{pgfscope}%
\pgfsys@transformshift{3.885040in}{1.813141in}%
\pgfsys@useobject{currentmarker}{}%
\end{pgfscope}%
\end{pgfscope}%
\begin{pgfscope}%
\pgfpathrectangle{\pgfqpoint{3.095027in}{0.528177in}}{\pgfqpoint{2.106702in}{2.125222in}} %
\pgfusepath{clip}%
\pgfsetbuttcap%
\pgfsetroundjoin%
\definecolor{currentfill}{rgb}{0.168627,0.670588,0.494118}%
\pgfsetfillcolor{currentfill}%
\pgfsetlinewidth{1.505625pt}%
\definecolor{currentstroke}{rgb}{0.168627,0.670588,0.494118}%
\pgfsetstrokecolor{currentstroke}%
\pgfsetdash{}{0pt}%
\pgfsys@defobject{currentmarker}{\pgfqpoint{-0.111111in}{-0.000000in}}{\pgfqpoint{0.111111in}{0.000000in}}{%
\pgfpathmoveto{\pgfqpoint{0.111111in}{-0.000000in}}%
\pgfpathlineto{\pgfqpoint{-0.111111in}{0.000000in}}%
\pgfusepath{stroke,fill}%
}%
\begin{pgfscope}%
\pgfsys@transformshift{3.885040in}{1.894309in}%
\pgfsys@useobject{currentmarker}{}%
\end{pgfscope}%
\end{pgfscope}%
\begin{pgfscope}%
\pgfpathrectangle{\pgfqpoint{3.095027in}{0.528177in}}{\pgfqpoint{2.106702in}{2.125222in}} %
\pgfusepath{clip}%
\pgfsetbuttcap%
\pgfsetroundjoin%
\definecolor{currentfill}{rgb}{1.000000,0.494118,0.250980}%
\pgfsetfillcolor{currentfill}%
\pgfsetlinewidth{1.505625pt}%
\definecolor{currentstroke}{rgb}{1.000000,0.494118,0.250980}%
\pgfsetstrokecolor{currentstroke}%
\pgfsetdash{}{0pt}%
\pgfsys@defobject{currentmarker}{\pgfqpoint{-0.111111in}{-0.000000in}}{\pgfqpoint{0.111111in}{0.000000in}}{%
\pgfpathmoveto{\pgfqpoint{0.111111in}{-0.000000in}}%
\pgfpathlineto{\pgfqpoint{-0.111111in}{0.000000in}}%
\pgfusepath{stroke,fill}%
}%
\begin{pgfscope}%
\pgfsys@transformshift{4.411716in}{1.096379in}%
\pgfsys@useobject{currentmarker}{}%
\end{pgfscope}%
\end{pgfscope}%
\begin{pgfscope}%
\pgfpathrectangle{\pgfqpoint{3.095027in}{0.528177in}}{\pgfqpoint{2.106702in}{2.125222in}} %
\pgfusepath{clip}%
\pgfsetbuttcap%
\pgfsetroundjoin%
\definecolor{currentfill}{rgb}{1.000000,0.494118,0.250980}%
\pgfsetfillcolor{currentfill}%
\pgfsetlinewidth{1.505625pt}%
\definecolor{currentstroke}{rgb}{1.000000,0.494118,0.250980}%
\pgfsetstrokecolor{currentstroke}%
\pgfsetdash{}{0pt}%
\pgfsys@defobject{currentmarker}{\pgfqpoint{-0.111111in}{-0.000000in}}{\pgfqpoint{0.111111in}{0.000000in}}{%
\pgfpathmoveto{\pgfqpoint{0.111111in}{-0.000000in}}%
\pgfpathlineto{\pgfqpoint{-0.111111in}{0.000000in}}%
\pgfusepath{stroke,fill}%
}%
\begin{pgfscope}%
\pgfsys@transformshift{4.411716in}{1.137162in}%
\pgfsys@useobject{currentmarker}{}%
\end{pgfscope}%
\end{pgfscope}%
\begin{pgfscope}%
\pgfpathrectangle{\pgfqpoint{3.095027in}{0.528177in}}{\pgfqpoint{2.106702in}{2.125222in}} %
\pgfusepath{clip}%
\pgfsetbuttcap%
\pgfsetroundjoin%
\definecolor{currentfill}{rgb}{1.000000,0.694118,0.250980}%
\pgfsetfillcolor{currentfill}%
\pgfsetlinewidth{1.505625pt}%
\definecolor{currentstroke}{rgb}{1.000000,0.694118,0.250980}%
\pgfsetstrokecolor{currentstroke}%
\pgfsetdash{}{0pt}%
\pgfsys@defobject{currentmarker}{\pgfqpoint{-0.111111in}{-0.000000in}}{\pgfqpoint{0.111111in}{0.000000in}}{%
\pgfpathmoveto{\pgfqpoint{0.111111in}{-0.000000in}}%
\pgfpathlineto{\pgfqpoint{-0.111111in}{0.000000in}}%
\pgfusepath{stroke,fill}%
}%
\begin{pgfscope}%
\pgfsys@transformshift{4.938391in}{1.583409in}%
\pgfsys@useobject{currentmarker}{}%
\end{pgfscope}%
\end{pgfscope}%
\begin{pgfscope}%
\pgfpathrectangle{\pgfqpoint{3.095027in}{0.528177in}}{\pgfqpoint{2.106702in}{2.125222in}} %
\pgfusepath{clip}%
\pgfsetbuttcap%
\pgfsetroundjoin%
\definecolor{currentfill}{rgb}{1.000000,0.694118,0.250980}%
\pgfsetfillcolor{currentfill}%
\pgfsetlinewidth{1.505625pt}%
\definecolor{currentstroke}{rgb}{1.000000,0.694118,0.250980}%
\pgfsetstrokecolor{currentstroke}%
\pgfsetdash{}{0pt}%
\pgfsys@defobject{currentmarker}{\pgfqpoint{-0.111111in}{-0.000000in}}{\pgfqpoint{0.111111in}{0.000000in}}{%
\pgfpathmoveto{\pgfqpoint{0.111111in}{-0.000000in}}%
\pgfpathlineto{\pgfqpoint{-0.111111in}{0.000000in}}%
\pgfusepath{stroke,fill}%
}%
\begin{pgfscope}%
\pgfsys@transformshift{4.938391in}{1.664874in}%
\pgfsys@useobject{currentmarker}{}%
\end{pgfscope}%
\end{pgfscope}%
\begin{pgfscope}%
\pgfpathrectangle{\pgfqpoint{3.095027in}{0.528177in}}{\pgfqpoint{2.106702in}{2.125222in}} %
\pgfusepath{clip}%
\pgfsetroundcap%
\pgfsetroundjoin%
\pgfsetlinewidth{1.756562pt}%
\definecolor{currentstroke}{rgb}{0.627451,0.627451,0.643137}%
\pgfsetstrokecolor{currentstroke}%
\pgfsetdash{}{0pt}%
\pgfpathmoveto{\pgfqpoint{3.358365in}{2.115301in}}%
\pgfpathlineto{\pgfqpoint{3.358365in}{2.245509in}}%
\pgfusepath{stroke}%
\end{pgfscope}%
\begin{pgfscope}%
\pgfpathrectangle{\pgfqpoint{3.095027in}{0.528177in}}{\pgfqpoint{2.106702in}{2.125222in}} %
\pgfusepath{clip}%
\pgfsetroundcap%
\pgfsetroundjoin%
\pgfsetlinewidth{1.756562pt}%
\definecolor{currentstroke}{rgb}{0.627451,0.627451,0.643137}%
\pgfsetstrokecolor{currentstroke}%
\pgfsetdash{}{0pt}%
\pgfpathmoveto{\pgfqpoint{3.358365in}{2.245509in}}%
\pgfpathlineto{\pgfqpoint{4.411716in}{2.245509in}}%
\pgfusepath{stroke}%
\end{pgfscope}%
\begin{pgfscope}%
\pgfpathrectangle{\pgfqpoint{3.095027in}{0.528177in}}{\pgfqpoint{2.106702in}{2.125222in}} %
\pgfusepath{clip}%
\pgfsetroundcap%
\pgfsetroundjoin%
\pgfsetlinewidth{1.756562pt}%
\definecolor{currentstroke}{rgb}{0.627451,0.627451,0.643137}%
\pgfsetstrokecolor{currentstroke}%
\pgfsetdash{}{0pt}%
\pgfpathmoveto{\pgfqpoint{4.411716in}{2.245509in}}%
\pgfpathlineto{\pgfqpoint{4.411716in}{1.293412in}}%
\pgfusepath{stroke}%
\end{pgfscope}%
\begin{pgfscope}%
\pgfpathrectangle{\pgfqpoint{3.095027in}{0.528177in}}{\pgfqpoint{2.106702in}{2.125222in}} %
\pgfusepath{clip}%
\pgfsetroundcap%
\pgfsetroundjoin%
\pgfsetlinewidth{1.756562pt}%
\definecolor{currentstroke}{rgb}{0.627451,0.627451,0.643137}%
\pgfsetstrokecolor{currentstroke}%
\pgfsetdash{}{0pt}%
\pgfpathmoveto{\pgfqpoint{4.411716in}{2.323634in}}%
\pgfpathlineto{\pgfqpoint{4.411716in}{2.453842in}}%
\pgfusepath{stroke}%
\end{pgfscope}%
\begin{pgfscope}%
\pgfpathrectangle{\pgfqpoint{3.095027in}{0.528177in}}{\pgfqpoint{2.106702in}{2.125222in}} %
\pgfusepath{clip}%
\pgfsetroundcap%
\pgfsetroundjoin%
\pgfsetlinewidth{1.756562pt}%
\definecolor{currentstroke}{rgb}{0.627451,0.627451,0.643137}%
\pgfsetstrokecolor{currentstroke}%
\pgfsetdash{}{0pt}%
\pgfpathmoveto{\pgfqpoint{4.411716in}{2.453842in}}%
\pgfpathlineto{\pgfqpoint{4.938391in}{2.453842in}}%
\pgfusepath{stroke}%
\end{pgfscope}%
\begin{pgfscope}%
\pgfpathrectangle{\pgfqpoint{3.095027in}{0.528177in}}{\pgfqpoint{2.106702in}{2.125222in}} %
\pgfusepath{clip}%
\pgfsetroundcap%
\pgfsetroundjoin%
\pgfsetlinewidth{1.756562pt}%
\definecolor{currentstroke}{rgb}{0.627451,0.627451,0.643137}%
\pgfsetstrokecolor{currentstroke}%
\pgfsetdash{}{0pt}%
\pgfpathmoveto{\pgfqpoint{4.938391in}{2.453842in}}%
\pgfpathlineto{\pgfqpoint{4.938391in}{1.821124in}}%
\pgfusepath{stroke}%
\end{pgfscope}%
\begin{pgfscope}%
\pgfsetrectcap%
\pgfsetmiterjoin%
\pgfsetlinewidth{1.254687pt}%
\definecolor{currentstroke}{rgb}{0.150000,0.150000,0.150000}%
\pgfsetstrokecolor{currentstroke}%
\pgfsetdash{}{0pt}%
\pgfpathmoveto{\pgfqpoint{3.095027in}{0.528177in}}%
\pgfpathlineto{\pgfqpoint{3.095027in}{2.653399in}}%
\pgfusepath{stroke}%
\end{pgfscope}%
\begin{pgfscope}%
\pgfsetrectcap%
\pgfsetmiterjoin%
\pgfsetlinewidth{1.254687pt}%
\definecolor{currentstroke}{rgb}{0.150000,0.150000,0.150000}%
\pgfsetstrokecolor{currentstroke}%
\pgfsetdash{}{0pt}%
\pgfpathmoveto{\pgfqpoint{3.095027in}{0.528177in}}%
\pgfpathlineto{\pgfqpoint{5.201729in}{0.528177in}}%
\pgfusepath{stroke}%
\end{pgfscope}%
\begin{pgfscope}%
\definecolor{textcolor}{rgb}{0.150000,0.150000,0.150000}%
\pgfsetstrokecolor{textcolor}%
\pgfsetfillcolor{textcolor}%
\pgftext[x=4.411716in,y=1.185991in,,]{\color{textcolor}\rmfamily\fontsize{15.000000}{18.000000}\selectfont \textbf{*}}%
\end{pgfscope}%
\begin{pgfscope}%
\definecolor{textcolor}{rgb}{0.150000,0.150000,0.150000}%
\pgfsetstrokecolor{textcolor}%
\pgfsetfillcolor{textcolor}%
\pgftext[x=4.938391in,y=1.713702in,,]{\color{textcolor}\rmfamily\fontsize{15.000000}{18.000000}\selectfont \textbf{*}}%
\end{pgfscope}%
\begin{pgfscope}%
\pgfsetbuttcap%
\pgfsetmiterjoin%
\definecolor{currentfill}{rgb}{0.200000,0.427451,0.650980}%
\pgfsetfillcolor{currentfill}%
\pgfsetlinewidth{1.505625pt}%
\definecolor{currentstroke}{rgb}{0.200000,0.427451,0.650980}%
\pgfsetstrokecolor{currentstroke}%
\pgfsetdash{}{0pt}%
\pgfpathmoveto{\pgfqpoint{3.195027in}{3.281088in}}%
\pgfpathlineto{\pgfqpoint{3.306138in}{3.281088in}}%
\pgfpathlineto{\pgfqpoint{3.306138in}{3.358866in}}%
\pgfpathlineto{\pgfqpoint{3.195027in}{3.358866in}}%
\pgfpathclose%
\pgfusepath{stroke,fill}%
\end{pgfscope}%
\begin{pgfscope}%
\definecolor{textcolor}{rgb}{0.150000,0.150000,0.150000}%
\pgfsetstrokecolor{textcolor}%
\pgfsetfillcolor{textcolor}%
\pgftext[x=3.395027in,y=3.281088in,left,base]{\color{textcolor}\rmfamily\fontsize{8.000000}{9.600000}\selectfont WT + Vehicle (494)}%
\end{pgfscope}%
\begin{pgfscope}%
\pgfsetbuttcap%
\pgfsetmiterjoin%
\definecolor{currentfill}{rgb}{0.168627,0.670588,0.494118}%
\pgfsetfillcolor{currentfill}%
\pgfsetlinewidth{1.505625pt}%
\definecolor{currentstroke}{rgb}{0.168627,0.670588,0.494118}%
\pgfsetstrokecolor{currentstroke}%
\pgfsetdash{}{0pt}%
\pgfpathmoveto{\pgfqpoint{3.195027in}{3.114449in}}%
\pgfpathlineto{\pgfqpoint{3.306138in}{3.114449in}}%
\pgfpathlineto{\pgfqpoint{3.306138in}{3.192227in}}%
\pgfpathlineto{\pgfqpoint{3.195027in}{3.192227in}}%
\pgfpathclose%
\pgfusepath{stroke,fill}%
\end{pgfscope}%
\begin{pgfscope}%
\definecolor{textcolor}{rgb}{0.150000,0.150000,0.150000}%
\pgfsetstrokecolor{textcolor}%
\pgfsetfillcolor{textcolor}%
\pgftext[x=3.395027in,y=3.114449in,left,base]{\color{textcolor}\rmfamily\fontsize{8.000000}{9.600000}\selectfont WT + TAT-GluA2\textsubscript{3Y} (365)}%
\end{pgfscope}%
\begin{pgfscope}%
\pgfsetbuttcap%
\pgfsetmiterjoin%
\definecolor{currentfill}{rgb}{1.000000,0.494118,0.250980}%
\pgfsetfillcolor{currentfill}%
\pgfsetlinewidth{1.505625pt}%
\definecolor{currentstroke}{rgb}{1.000000,0.494118,0.250980}%
\pgfsetstrokecolor{currentstroke}%
\pgfsetdash{}{0pt}%
\pgfpathmoveto{\pgfqpoint{3.195027in}{2.947809in}}%
\pgfpathlineto{\pgfqpoint{3.306138in}{2.947809in}}%
\pgfpathlineto{\pgfqpoint{3.306138in}{3.025587in}}%
\pgfpathlineto{\pgfqpoint{3.195027in}{3.025587in}}%
\pgfpathclose%
\pgfusepath{stroke,fill}%
\end{pgfscope}%
\begin{pgfscope}%
\definecolor{textcolor}{rgb}{0.150000,0.150000,0.150000}%
\pgfsetstrokecolor{textcolor}%
\pgfsetfillcolor{textcolor}%
\pgftext[x=3.395027in,y=2.947809in,left,base]{\color{textcolor}\rmfamily\fontsize{8.000000}{9.600000}\selectfont Tg + Vehicle (281)}%
\end{pgfscope}%
\begin{pgfscope}%
\pgfsetbuttcap%
\pgfsetmiterjoin%
\definecolor{currentfill}{rgb}{1.000000,0.694118,0.250980}%
\pgfsetfillcolor{currentfill}%
\pgfsetlinewidth{1.505625pt}%
\definecolor{currentstroke}{rgb}{1.000000,0.694118,0.250980}%
\pgfsetstrokecolor{currentstroke}%
\pgfsetdash{}{0pt}%
\pgfpathmoveto{\pgfqpoint{3.195027in}{2.781170in}}%
\pgfpathlineto{\pgfqpoint{3.306138in}{2.781170in}}%
\pgfpathlineto{\pgfqpoint{3.306138in}{2.858947in}}%
\pgfpathlineto{\pgfqpoint{3.195027in}{2.858947in}}%
\pgfpathclose%
\pgfusepath{stroke,fill}%
\end{pgfscope}%
\begin{pgfscope}%
\definecolor{textcolor}{rgb}{0.150000,0.150000,0.150000}%
\pgfsetstrokecolor{textcolor}%
\pgfsetfillcolor{textcolor}%
\pgftext[x=3.395027in,y=2.781170in,left,base]{\color{textcolor}\rmfamily\fontsize{8.000000}{9.600000}\selectfont Tg + TAT-GluA2\textsubscript{3Y} (346)}%
\end{pgfscope}%
\end{pgfpicture}%
\makeatother%
\endgroup%

        \caption{\label{f.ad.freezing_bouts}}
    \end{subfigure}
    \caption[Freezing lengths and number of freezing bouts.]{Average number of freezing bouts \subref{f.ad.freezing_freq} and length of freezing bouts \subref{f.ad.freezing_bouts}. \Gls{tg} mice freeze as often as \gls{wt} mice, but with shorter duration. This is rescued by \tglu{} treatment. \label{f.ad.freezing_profile}}
\end{figure}


\subsection{\tglu{} rescues hyperactivity in \gls{tg} cells}

It has been previously shown that amygloid plaques in \gls{tg} mice disturb the excitatory-inhibitory balance of cells \citep{palop16}. In the \gls{ca1}, previous reports have found cells to be hyperactive in mouse models of \gls{ad}. Here we asked whether \gls{ca1} cells in \gls{tg} mice show hyperactivity, and hypothesized that \tglu{} treatment would be able to rescue a hyper-excitable phenotype.

Figure~\ref{f.ad.acttrain} shows average cell activity during the training session before foot-shock. During training, two-way \gls{anova} revealed a significant interaction between \textit{Genotype} and \textit{Treatment} (F\tsb{1,3033}=7.7, p=0.006), as well as significant main effects of \textit{Genotype} (F\tsb{1,3033}=6.2, p=0.01) and \textit{Treatment} (F\tsb{1,3033}=5.1, p=0.02). \textit{Post hoc} tests showed that \gls{tg}-Veh animals had significantly higher cell activity (WT-Veh vs Tg-Veh, T=-3.72, p<0.001), and \tglu{} treatment was able to restore average cell activity to \gls{wt} level. (Tg-\glu{} vs Tg-Veh, T=-3.58, p<0.001; WT-Veh vs Tg-\glu, T=0.14, p=0.89). \tglu{} did not have any effect on \gls{wt} mice (WT-Veh vs WT-\glu, T=-0.137, p=0.89) This result is consistent with previous reports in the literature \citep{verret12}, showing cells in Tg mice have increased overall cell activity. Interestingly, while \tglu{} treatment restored the mean cell activity during training, the distribution of cell activity was significantly different from \gls{wt} (WT-Veh vs Tg-\glu, K=0.16, p<0.001). As is shown in the cumulative distribution plot, \gls{tg} mice with \tglu{} treatment still had a higher proportion of highly active cells. 

Similar effect were found during testing (Figure~\ref{f.ad.acttest}). Two-way \gls{anova} revealed significant interaction of \textit{Genotype} and \textit{Treatment}(F\tsb{1,3029}=78.4, p<0.001), as well as major effects of \textit{Genotype} (F\tsb{1,3029}=32.7, p<0.001) and \textit{Treatment} (F\tsb{1,3029}=27.4, p<0.001). \textit{Post hoc} tests showed a significant increase of cell activity in the Tg-Veh group (WT-Veh vs Tg-Veh, T=-10.1, p<0.001), and this effect was corrected by \tglu{} treatment (WT-Veh vs Tg-\glu, T=0.73, p=0.47; Tg-\glu{} vs Tg-Veh, T=-9.97, p<0.001). There was also a trend of increased cell activity after \tglu{} treatment in the \gls{wt} group, however the p-value is close to threshold after correction for multiple comparisons (WT-Veh vs WT-\glu, T=-2.53, p=0.012, threshold = 0.013). Interestingly during testing, \tglu{} was able to fully rescue the hyper-activity in the \gls{tg} group (\gls{kstest}: Veh-WT vs Tg-\glu, K=0.06, p=0.07; Tg-Veh vs Tg-\glu, K=2.29, p<0.001). These results suggest that the effect \tglu{} treatment is long-lasting: treatment during memory encoding is also able to rescue \gls{ca1} hyperactivity in \gls{tg} mice during memory testing \SI{24}{\hour} later.

Comparing the cell activity between testing and training, we found that \gls{tg} mice have significantly increased cell activity during testing,  which was not found in \gls{wt} mice (Figure~\ref{f.ad.actdiff}; Two-way \gls{anova}, \textit{Genotype} $\times$ \textit{Treatment} interaction F\tsb{1,4380}=32.4, p<0.001, \textit{Genotype} main effect F\tsb{1,4380}=23.0, p<0.001, \textit{Treatment} main effect F\tsb{1,4380}=23.6, p<0.001; \textit{post hoc} test WT-Veh vs Tg-Veh, T=-7.4, p<0.001, WT-Veh vs WT-\glu, T=-0.487, p=0.63). \tglu{} treatment blocked increases in cell activity during testing (Tg-\glu{} vs Tg-Veh, T=-7.5, p<0.001). The cumulative plot suggests that \gls{tg} mice had similar proportions of cells with increased activity, however in \gls{tg} mice the cells increased more than those in \gls{wt} mice. This result confirms previous reports that mouse models of \gls{ad} show disrupted excitation-inhibition balance in \gls{ca1} \citep{palop16}.


\begin{figure}[h]
    \begin{subfigure}[h]{0.9\textwidth}
        %% Creator: Matplotlib, PGF backend
%%
%% To include the figure in your LaTeX document, write
%%   \input{<filename>.pgf}
%%
%% Make sure the required packages are loaded in your preamble
%%   \usepackage{pgf}
%%
%% Figures using additional raster images can only be included by \input if
%% they are in the same directory as the main LaTeX file. For loading figures
%% from other directories you can use the `import` package
%%   \usepackage{import}
%% and then include the figures with
%%   \import{<path to file>}{<filename>.pgf}
%%
%% Matplotlib used the following preamble
%%   \usepackage[utf8]{inputenc}
%%   \usepackage[T1]{fontenc}
%%   \usepackage{siunitx}
%%
\begingroup%
\makeatletter%
\begin{pgfpicture}%
\pgfpathrectangle{\pgfpointorigin}{\pgfqpoint{5.301729in}{3.553934in}}%
\pgfusepath{use as bounding box, clip}%
\begin{pgfscope}%
\pgfsetbuttcap%
\pgfsetmiterjoin%
\definecolor{currentfill}{rgb}{1.000000,1.000000,1.000000}%
\pgfsetfillcolor{currentfill}%
\pgfsetlinewidth{0.000000pt}%
\definecolor{currentstroke}{rgb}{1.000000,1.000000,1.000000}%
\pgfsetstrokecolor{currentstroke}%
\pgfsetdash{}{0pt}%
\pgfpathmoveto{\pgfqpoint{0.000000in}{0.000000in}}%
\pgfpathlineto{\pgfqpoint{5.301729in}{0.000000in}}%
\pgfpathlineto{\pgfqpoint{5.301729in}{3.553934in}}%
\pgfpathlineto{\pgfqpoint{0.000000in}{3.553934in}}%
\pgfpathclose%
\pgfusepath{fill}%
\end{pgfscope}%
\begin{pgfscope}%
\pgfsetbuttcap%
\pgfsetmiterjoin%
\definecolor{currentfill}{rgb}{1.000000,1.000000,1.000000}%
\pgfsetfillcolor{currentfill}%
\pgfsetlinewidth{0.000000pt}%
\definecolor{currentstroke}{rgb}{0.000000,0.000000,0.000000}%
\pgfsetstrokecolor{currentstroke}%
\pgfsetstrokeopacity{0.000000}%
\pgfsetdash{}{0pt}%
\pgfpathmoveto{\pgfqpoint{0.566985in}{0.528177in}}%
\pgfpathlineto{\pgfqpoint{2.673686in}{0.528177in}}%
\pgfpathlineto{\pgfqpoint{2.673686in}{3.392606in}}%
\pgfpathlineto{\pgfqpoint{0.566985in}{3.392606in}}%
\pgfpathclose%
\pgfusepath{fill}%
\end{pgfscope}%
\begin{pgfscope}%
\pgfsetbuttcap%
\pgfsetroundjoin%
\definecolor{currentfill}{rgb}{0.150000,0.150000,0.150000}%
\pgfsetfillcolor{currentfill}%
\pgfsetlinewidth{1.003750pt}%
\definecolor{currentstroke}{rgb}{0.150000,0.150000,0.150000}%
\pgfsetstrokecolor{currentstroke}%
\pgfsetdash{}{0pt}%
\pgfsys@defobject{currentmarker}{\pgfqpoint{0.000000in}{0.000000in}}{\pgfqpoint{0.000000in}{0.041667in}}{%
\pgfpathmoveto{\pgfqpoint{0.000000in}{0.000000in}}%
\pgfpathlineto{\pgfqpoint{0.000000in}{0.041667in}}%
\pgfusepath{stroke,fill}%
}%
\begin{pgfscope}%
\pgfsys@transformshift{0.566985in}{0.528177in}%
\pgfsys@useobject{currentmarker}{}%
\end{pgfscope}%
\end{pgfscope}%
\begin{pgfscope}%
\definecolor{textcolor}{rgb}{0.150000,0.150000,0.150000}%
\pgfsetstrokecolor{textcolor}%
\pgfsetfillcolor{textcolor}%
\pgftext[x=0.566985in,y=0.430955in,,top]{\color{textcolor}\rmfamily\fontsize{10.000000}{12.000000}\selectfont \(\displaystyle 0.0\)}%
\end{pgfscope}%
\begin{pgfscope}%
\pgfsetbuttcap%
\pgfsetroundjoin%
\definecolor{currentfill}{rgb}{0.150000,0.150000,0.150000}%
\pgfsetfillcolor{currentfill}%
\pgfsetlinewidth{1.003750pt}%
\definecolor{currentstroke}{rgb}{0.150000,0.150000,0.150000}%
\pgfsetstrokecolor{currentstroke}%
\pgfsetdash{}{0pt}%
\pgfsys@defobject{currentmarker}{\pgfqpoint{0.000000in}{0.000000in}}{\pgfqpoint{0.000000in}{0.041667in}}{%
\pgfpathmoveto{\pgfqpoint{0.000000in}{0.000000in}}%
\pgfpathlineto{\pgfqpoint{0.000000in}{0.041667in}}%
\pgfusepath{stroke,fill}%
}%
\begin{pgfscope}%
\pgfsys@transformshift{0.867942in}{0.528177in}%
\pgfsys@useobject{currentmarker}{}%
\end{pgfscope}%
\end{pgfscope}%
\begin{pgfscope}%
\definecolor{textcolor}{rgb}{0.150000,0.150000,0.150000}%
\pgfsetstrokecolor{textcolor}%
\pgfsetfillcolor{textcolor}%
\pgftext[x=0.867942in,y=0.430955in,,top]{\color{textcolor}\rmfamily\fontsize{10.000000}{12.000000}\selectfont \(\displaystyle 0.2\)}%
\end{pgfscope}%
\begin{pgfscope}%
\pgfsetbuttcap%
\pgfsetroundjoin%
\definecolor{currentfill}{rgb}{0.150000,0.150000,0.150000}%
\pgfsetfillcolor{currentfill}%
\pgfsetlinewidth{1.003750pt}%
\definecolor{currentstroke}{rgb}{0.150000,0.150000,0.150000}%
\pgfsetstrokecolor{currentstroke}%
\pgfsetdash{}{0pt}%
\pgfsys@defobject{currentmarker}{\pgfqpoint{0.000000in}{0.000000in}}{\pgfqpoint{0.000000in}{0.041667in}}{%
\pgfpathmoveto{\pgfqpoint{0.000000in}{0.000000in}}%
\pgfpathlineto{\pgfqpoint{0.000000in}{0.041667in}}%
\pgfusepath{stroke,fill}%
}%
\begin{pgfscope}%
\pgfsys@transformshift{1.168899in}{0.528177in}%
\pgfsys@useobject{currentmarker}{}%
\end{pgfscope}%
\end{pgfscope}%
\begin{pgfscope}%
\definecolor{textcolor}{rgb}{0.150000,0.150000,0.150000}%
\pgfsetstrokecolor{textcolor}%
\pgfsetfillcolor{textcolor}%
\pgftext[x=1.168899in,y=0.430955in,,top]{\color{textcolor}\rmfamily\fontsize{10.000000}{12.000000}\selectfont \(\displaystyle 0.4\)}%
\end{pgfscope}%
\begin{pgfscope}%
\pgfsetbuttcap%
\pgfsetroundjoin%
\definecolor{currentfill}{rgb}{0.150000,0.150000,0.150000}%
\pgfsetfillcolor{currentfill}%
\pgfsetlinewidth{1.003750pt}%
\definecolor{currentstroke}{rgb}{0.150000,0.150000,0.150000}%
\pgfsetstrokecolor{currentstroke}%
\pgfsetdash{}{0pt}%
\pgfsys@defobject{currentmarker}{\pgfqpoint{0.000000in}{0.000000in}}{\pgfqpoint{0.000000in}{0.041667in}}{%
\pgfpathmoveto{\pgfqpoint{0.000000in}{0.000000in}}%
\pgfpathlineto{\pgfqpoint{0.000000in}{0.041667in}}%
\pgfusepath{stroke,fill}%
}%
\begin{pgfscope}%
\pgfsys@transformshift{1.469857in}{0.528177in}%
\pgfsys@useobject{currentmarker}{}%
\end{pgfscope}%
\end{pgfscope}%
\begin{pgfscope}%
\definecolor{textcolor}{rgb}{0.150000,0.150000,0.150000}%
\pgfsetstrokecolor{textcolor}%
\pgfsetfillcolor{textcolor}%
\pgftext[x=1.469857in,y=0.430955in,,top]{\color{textcolor}\rmfamily\fontsize{10.000000}{12.000000}\selectfont \(\displaystyle 0.6\)}%
\end{pgfscope}%
\begin{pgfscope}%
\pgfsetbuttcap%
\pgfsetroundjoin%
\definecolor{currentfill}{rgb}{0.150000,0.150000,0.150000}%
\pgfsetfillcolor{currentfill}%
\pgfsetlinewidth{1.003750pt}%
\definecolor{currentstroke}{rgb}{0.150000,0.150000,0.150000}%
\pgfsetstrokecolor{currentstroke}%
\pgfsetdash{}{0pt}%
\pgfsys@defobject{currentmarker}{\pgfqpoint{0.000000in}{0.000000in}}{\pgfqpoint{0.000000in}{0.041667in}}{%
\pgfpathmoveto{\pgfqpoint{0.000000in}{0.000000in}}%
\pgfpathlineto{\pgfqpoint{0.000000in}{0.041667in}}%
\pgfusepath{stroke,fill}%
}%
\begin{pgfscope}%
\pgfsys@transformshift{1.770814in}{0.528177in}%
\pgfsys@useobject{currentmarker}{}%
\end{pgfscope}%
\end{pgfscope}%
\begin{pgfscope}%
\definecolor{textcolor}{rgb}{0.150000,0.150000,0.150000}%
\pgfsetstrokecolor{textcolor}%
\pgfsetfillcolor{textcolor}%
\pgftext[x=1.770814in,y=0.430955in,,top]{\color{textcolor}\rmfamily\fontsize{10.000000}{12.000000}\selectfont \(\displaystyle 0.8\)}%
\end{pgfscope}%
\begin{pgfscope}%
\pgfsetbuttcap%
\pgfsetroundjoin%
\definecolor{currentfill}{rgb}{0.150000,0.150000,0.150000}%
\pgfsetfillcolor{currentfill}%
\pgfsetlinewidth{1.003750pt}%
\definecolor{currentstroke}{rgb}{0.150000,0.150000,0.150000}%
\pgfsetstrokecolor{currentstroke}%
\pgfsetdash{}{0pt}%
\pgfsys@defobject{currentmarker}{\pgfqpoint{0.000000in}{0.000000in}}{\pgfqpoint{0.000000in}{0.041667in}}{%
\pgfpathmoveto{\pgfqpoint{0.000000in}{0.000000in}}%
\pgfpathlineto{\pgfqpoint{0.000000in}{0.041667in}}%
\pgfusepath{stroke,fill}%
}%
\begin{pgfscope}%
\pgfsys@transformshift{2.071772in}{0.528177in}%
\pgfsys@useobject{currentmarker}{}%
\end{pgfscope}%
\end{pgfscope}%
\begin{pgfscope}%
\definecolor{textcolor}{rgb}{0.150000,0.150000,0.150000}%
\pgfsetstrokecolor{textcolor}%
\pgfsetfillcolor{textcolor}%
\pgftext[x=2.071772in,y=0.430955in,,top]{\color{textcolor}\rmfamily\fontsize{10.000000}{12.000000}\selectfont \(\displaystyle 1.0\)}%
\end{pgfscope}%
\begin{pgfscope}%
\pgfsetbuttcap%
\pgfsetroundjoin%
\definecolor{currentfill}{rgb}{0.150000,0.150000,0.150000}%
\pgfsetfillcolor{currentfill}%
\pgfsetlinewidth{1.003750pt}%
\definecolor{currentstroke}{rgb}{0.150000,0.150000,0.150000}%
\pgfsetstrokecolor{currentstroke}%
\pgfsetdash{}{0pt}%
\pgfsys@defobject{currentmarker}{\pgfqpoint{0.000000in}{0.000000in}}{\pgfqpoint{0.000000in}{0.041667in}}{%
\pgfpathmoveto{\pgfqpoint{0.000000in}{0.000000in}}%
\pgfpathlineto{\pgfqpoint{0.000000in}{0.041667in}}%
\pgfusepath{stroke,fill}%
}%
\begin{pgfscope}%
\pgfsys@transformshift{2.372729in}{0.528177in}%
\pgfsys@useobject{currentmarker}{}%
\end{pgfscope}%
\end{pgfscope}%
\begin{pgfscope}%
\definecolor{textcolor}{rgb}{0.150000,0.150000,0.150000}%
\pgfsetstrokecolor{textcolor}%
\pgfsetfillcolor{textcolor}%
\pgftext[x=2.372729in,y=0.430955in,,top]{\color{textcolor}\rmfamily\fontsize{10.000000}{12.000000}\selectfont \(\displaystyle 1.2\)}%
\end{pgfscope}%
\begin{pgfscope}%
\pgfsetbuttcap%
\pgfsetroundjoin%
\definecolor{currentfill}{rgb}{0.150000,0.150000,0.150000}%
\pgfsetfillcolor{currentfill}%
\pgfsetlinewidth{1.003750pt}%
\definecolor{currentstroke}{rgb}{0.150000,0.150000,0.150000}%
\pgfsetstrokecolor{currentstroke}%
\pgfsetdash{}{0pt}%
\pgfsys@defobject{currentmarker}{\pgfqpoint{0.000000in}{0.000000in}}{\pgfqpoint{0.000000in}{0.041667in}}{%
\pgfpathmoveto{\pgfqpoint{0.000000in}{0.000000in}}%
\pgfpathlineto{\pgfqpoint{0.000000in}{0.041667in}}%
\pgfusepath{stroke,fill}%
}%
\begin{pgfscope}%
\pgfsys@transformshift{2.673686in}{0.528177in}%
\pgfsys@useobject{currentmarker}{}%
\end{pgfscope}%
\end{pgfscope}%
\begin{pgfscope}%
\definecolor{textcolor}{rgb}{0.150000,0.150000,0.150000}%
\pgfsetstrokecolor{textcolor}%
\pgfsetfillcolor{textcolor}%
\pgftext[x=2.673686in,y=0.430955in,,top]{\color{textcolor}\rmfamily\fontsize{10.000000}{12.000000}\selectfont \(\displaystyle 1.4\)}%
\end{pgfscope}%
\begin{pgfscope}%
\definecolor{textcolor}{rgb}{0.150000,0.150000,0.150000}%
\pgfsetstrokecolor{textcolor}%
\pgfsetfillcolor{textcolor}%
\pgftext[x=1.620336in,y=0.238855in,,top]{\color{textcolor}\rmfamily\fontsize{10.000000}{12.000000}\selectfont \textbf{Cell activity (a.u.)}}%
\end{pgfscope}%
\begin{pgfscope}%
\pgfsetbuttcap%
\pgfsetroundjoin%
\definecolor{currentfill}{rgb}{0.150000,0.150000,0.150000}%
\pgfsetfillcolor{currentfill}%
\pgfsetlinewidth{1.003750pt}%
\definecolor{currentstroke}{rgb}{0.150000,0.150000,0.150000}%
\pgfsetstrokecolor{currentstroke}%
\pgfsetdash{}{0pt}%
\pgfsys@defobject{currentmarker}{\pgfqpoint{0.000000in}{0.000000in}}{\pgfqpoint{0.041667in}{0.000000in}}{%
\pgfpathmoveto{\pgfqpoint{0.000000in}{0.000000in}}%
\pgfpathlineto{\pgfqpoint{0.041667in}{0.000000in}}%
\pgfusepath{stroke,fill}%
}%
\begin{pgfscope}%
\pgfsys@transformshift{0.566985in}{0.528177in}%
\pgfsys@useobject{currentmarker}{}%
\end{pgfscope}%
\end{pgfscope}%
\begin{pgfscope}%
\definecolor{textcolor}{rgb}{0.150000,0.150000,0.150000}%
\pgfsetstrokecolor{textcolor}%
\pgfsetfillcolor{textcolor}%
\pgftext[x=0.469762in,y=0.528177in,right,]{\color{textcolor}\rmfamily\fontsize{10.000000}{12.000000}\selectfont \(\displaystyle 0.2\)}%
\end{pgfscope}%
\begin{pgfscope}%
\pgfsetbuttcap%
\pgfsetroundjoin%
\definecolor{currentfill}{rgb}{0.150000,0.150000,0.150000}%
\pgfsetfillcolor{currentfill}%
\pgfsetlinewidth{1.003750pt}%
\definecolor{currentstroke}{rgb}{0.150000,0.150000,0.150000}%
\pgfsetstrokecolor{currentstroke}%
\pgfsetdash{}{0pt}%
\pgfsys@defobject{currentmarker}{\pgfqpoint{0.000000in}{0.000000in}}{\pgfqpoint{0.041667in}{0.000000in}}{%
\pgfpathmoveto{\pgfqpoint{0.000000in}{0.000000in}}%
\pgfpathlineto{\pgfqpoint{0.041667in}{0.000000in}}%
\pgfusepath{stroke,fill}%
}%
\begin{pgfscope}%
\pgfsys@transformshift{0.566985in}{0.886230in}%
\pgfsys@useobject{currentmarker}{}%
\end{pgfscope}%
\end{pgfscope}%
\begin{pgfscope}%
\definecolor{textcolor}{rgb}{0.150000,0.150000,0.150000}%
\pgfsetstrokecolor{textcolor}%
\pgfsetfillcolor{textcolor}%
\pgftext[x=0.469762in,y=0.886230in,right,]{\color{textcolor}\rmfamily\fontsize{10.000000}{12.000000}\selectfont \(\displaystyle 0.3\)}%
\end{pgfscope}%
\begin{pgfscope}%
\pgfsetbuttcap%
\pgfsetroundjoin%
\definecolor{currentfill}{rgb}{0.150000,0.150000,0.150000}%
\pgfsetfillcolor{currentfill}%
\pgfsetlinewidth{1.003750pt}%
\definecolor{currentstroke}{rgb}{0.150000,0.150000,0.150000}%
\pgfsetstrokecolor{currentstroke}%
\pgfsetdash{}{0pt}%
\pgfsys@defobject{currentmarker}{\pgfqpoint{0.000000in}{0.000000in}}{\pgfqpoint{0.041667in}{0.000000in}}{%
\pgfpathmoveto{\pgfqpoint{0.000000in}{0.000000in}}%
\pgfpathlineto{\pgfqpoint{0.041667in}{0.000000in}}%
\pgfusepath{stroke,fill}%
}%
\begin{pgfscope}%
\pgfsys@transformshift{0.566985in}{1.244284in}%
\pgfsys@useobject{currentmarker}{}%
\end{pgfscope}%
\end{pgfscope}%
\begin{pgfscope}%
\definecolor{textcolor}{rgb}{0.150000,0.150000,0.150000}%
\pgfsetstrokecolor{textcolor}%
\pgfsetfillcolor{textcolor}%
\pgftext[x=0.469762in,y=1.244284in,right,]{\color{textcolor}\rmfamily\fontsize{10.000000}{12.000000}\selectfont \(\displaystyle 0.4\)}%
\end{pgfscope}%
\begin{pgfscope}%
\pgfsetbuttcap%
\pgfsetroundjoin%
\definecolor{currentfill}{rgb}{0.150000,0.150000,0.150000}%
\pgfsetfillcolor{currentfill}%
\pgfsetlinewidth{1.003750pt}%
\definecolor{currentstroke}{rgb}{0.150000,0.150000,0.150000}%
\pgfsetstrokecolor{currentstroke}%
\pgfsetdash{}{0pt}%
\pgfsys@defobject{currentmarker}{\pgfqpoint{0.000000in}{0.000000in}}{\pgfqpoint{0.041667in}{0.000000in}}{%
\pgfpathmoveto{\pgfqpoint{0.000000in}{0.000000in}}%
\pgfpathlineto{\pgfqpoint{0.041667in}{0.000000in}}%
\pgfusepath{stroke,fill}%
}%
\begin{pgfscope}%
\pgfsys@transformshift{0.566985in}{1.602338in}%
\pgfsys@useobject{currentmarker}{}%
\end{pgfscope}%
\end{pgfscope}%
\begin{pgfscope}%
\definecolor{textcolor}{rgb}{0.150000,0.150000,0.150000}%
\pgfsetstrokecolor{textcolor}%
\pgfsetfillcolor{textcolor}%
\pgftext[x=0.469762in,y=1.602338in,right,]{\color{textcolor}\rmfamily\fontsize{10.000000}{12.000000}\selectfont \(\displaystyle 0.5\)}%
\end{pgfscope}%
\begin{pgfscope}%
\pgfsetbuttcap%
\pgfsetroundjoin%
\definecolor{currentfill}{rgb}{0.150000,0.150000,0.150000}%
\pgfsetfillcolor{currentfill}%
\pgfsetlinewidth{1.003750pt}%
\definecolor{currentstroke}{rgb}{0.150000,0.150000,0.150000}%
\pgfsetstrokecolor{currentstroke}%
\pgfsetdash{}{0pt}%
\pgfsys@defobject{currentmarker}{\pgfqpoint{0.000000in}{0.000000in}}{\pgfqpoint{0.041667in}{0.000000in}}{%
\pgfpathmoveto{\pgfqpoint{0.000000in}{0.000000in}}%
\pgfpathlineto{\pgfqpoint{0.041667in}{0.000000in}}%
\pgfusepath{stroke,fill}%
}%
\begin{pgfscope}%
\pgfsys@transformshift{0.566985in}{1.960392in}%
\pgfsys@useobject{currentmarker}{}%
\end{pgfscope}%
\end{pgfscope}%
\begin{pgfscope}%
\definecolor{textcolor}{rgb}{0.150000,0.150000,0.150000}%
\pgfsetstrokecolor{textcolor}%
\pgfsetfillcolor{textcolor}%
\pgftext[x=0.469762in,y=1.960392in,right,]{\color{textcolor}\rmfamily\fontsize{10.000000}{12.000000}\selectfont \(\displaystyle 0.6\)}%
\end{pgfscope}%
\begin{pgfscope}%
\pgfsetbuttcap%
\pgfsetroundjoin%
\definecolor{currentfill}{rgb}{0.150000,0.150000,0.150000}%
\pgfsetfillcolor{currentfill}%
\pgfsetlinewidth{1.003750pt}%
\definecolor{currentstroke}{rgb}{0.150000,0.150000,0.150000}%
\pgfsetstrokecolor{currentstroke}%
\pgfsetdash{}{0pt}%
\pgfsys@defobject{currentmarker}{\pgfqpoint{0.000000in}{0.000000in}}{\pgfqpoint{0.041667in}{0.000000in}}{%
\pgfpathmoveto{\pgfqpoint{0.000000in}{0.000000in}}%
\pgfpathlineto{\pgfqpoint{0.041667in}{0.000000in}}%
\pgfusepath{stroke,fill}%
}%
\begin{pgfscope}%
\pgfsys@transformshift{0.566985in}{2.318445in}%
\pgfsys@useobject{currentmarker}{}%
\end{pgfscope}%
\end{pgfscope}%
\begin{pgfscope}%
\definecolor{textcolor}{rgb}{0.150000,0.150000,0.150000}%
\pgfsetstrokecolor{textcolor}%
\pgfsetfillcolor{textcolor}%
\pgftext[x=0.469762in,y=2.318445in,right,]{\color{textcolor}\rmfamily\fontsize{10.000000}{12.000000}\selectfont \(\displaystyle 0.7\)}%
\end{pgfscope}%
\begin{pgfscope}%
\pgfsetbuttcap%
\pgfsetroundjoin%
\definecolor{currentfill}{rgb}{0.150000,0.150000,0.150000}%
\pgfsetfillcolor{currentfill}%
\pgfsetlinewidth{1.003750pt}%
\definecolor{currentstroke}{rgb}{0.150000,0.150000,0.150000}%
\pgfsetstrokecolor{currentstroke}%
\pgfsetdash{}{0pt}%
\pgfsys@defobject{currentmarker}{\pgfqpoint{0.000000in}{0.000000in}}{\pgfqpoint{0.041667in}{0.000000in}}{%
\pgfpathmoveto{\pgfqpoint{0.000000in}{0.000000in}}%
\pgfpathlineto{\pgfqpoint{0.041667in}{0.000000in}}%
\pgfusepath{stroke,fill}%
}%
\begin{pgfscope}%
\pgfsys@transformshift{0.566985in}{2.676499in}%
\pgfsys@useobject{currentmarker}{}%
\end{pgfscope}%
\end{pgfscope}%
\begin{pgfscope}%
\definecolor{textcolor}{rgb}{0.150000,0.150000,0.150000}%
\pgfsetstrokecolor{textcolor}%
\pgfsetfillcolor{textcolor}%
\pgftext[x=0.469762in,y=2.676499in,right,]{\color{textcolor}\rmfamily\fontsize{10.000000}{12.000000}\selectfont \(\displaystyle 0.8\)}%
\end{pgfscope}%
\begin{pgfscope}%
\pgfsetbuttcap%
\pgfsetroundjoin%
\definecolor{currentfill}{rgb}{0.150000,0.150000,0.150000}%
\pgfsetfillcolor{currentfill}%
\pgfsetlinewidth{1.003750pt}%
\definecolor{currentstroke}{rgb}{0.150000,0.150000,0.150000}%
\pgfsetstrokecolor{currentstroke}%
\pgfsetdash{}{0pt}%
\pgfsys@defobject{currentmarker}{\pgfqpoint{0.000000in}{0.000000in}}{\pgfqpoint{0.041667in}{0.000000in}}{%
\pgfpathmoveto{\pgfqpoint{0.000000in}{0.000000in}}%
\pgfpathlineto{\pgfqpoint{0.041667in}{0.000000in}}%
\pgfusepath{stroke,fill}%
}%
\begin{pgfscope}%
\pgfsys@transformshift{0.566985in}{3.034553in}%
\pgfsys@useobject{currentmarker}{}%
\end{pgfscope}%
\end{pgfscope}%
\begin{pgfscope}%
\definecolor{textcolor}{rgb}{0.150000,0.150000,0.150000}%
\pgfsetstrokecolor{textcolor}%
\pgfsetfillcolor{textcolor}%
\pgftext[x=0.469762in,y=3.034553in,right,]{\color{textcolor}\rmfamily\fontsize{10.000000}{12.000000}\selectfont \(\displaystyle 0.9\)}%
\end{pgfscope}%
\begin{pgfscope}%
\pgfsetbuttcap%
\pgfsetroundjoin%
\definecolor{currentfill}{rgb}{0.150000,0.150000,0.150000}%
\pgfsetfillcolor{currentfill}%
\pgfsetlinewidth{1.003750pt}%
\definecolor{currentstroke}{rgb}{0.150000,0.150000,0.150000}%
\pgfsetstrokecolor{currentstroke}%
\pgfsetdash{}{0pt}%
\pgfsys@defobject{currentmarker}{\pgfqpoint{0.000000in}{0.000000in}}{\pgfqpoint{0.041667in}{0.000000in}}{%
\pgfpathmoveto{\pgfqpoint{0.000000in}{0.000000in}}%
\pgfpathlineto{\pgfqpoint{0.041667in}{0.000000in}}%
\pgfusepath{stroke,fill}%
}%
\begin{pgfscope}%
\pgfsys@transformshift{0.566985in}{3.392606in}%
\pgfsys@useobject{currentmarker}{}%
\end{pgfscope}%
\end{pgfscope}%
\begin{pgfscope}%
\definecolor{textcolor}{rgb}{0.150000,0.150000,0.150000}%
\pgfsetstrokecolor{textcolor}%
\pgfsetfillcolor{textcolor}%
\pgftext[x=0.469762in,y=3.392606in,right,]{\color{textcolor}\rmfamily\fontsize{10.000000}{12.000000}\selectfont \(\displaystyle 1.0\)}%
\end{pgfscope}%
\begin{pgfscope}%
\definecolor{textcolor}{rgb}{0.150000,0.150000,0.150000}%
\pgfsetstrokecolor{textcolor}%
\pgfsetfillcolor{textcolor}%
\pgftext[x=0.222848in,y=1.960392in,,bottom,rotate=90.000000]{\color{textcolor}\rmfamily\fontsize{10.000000}{12.000000}\selectfont \textbf{Cumulative porportion}}%
\end{pgfscope}%
\begin{pgfscope}%
\pgfpathrectangle{\pgfqpoint{0.566985in}{0.528177in}}{\pgfqpoint{2.106702in}{2.864429in}} %
\pgfusepath{clip}%
\pgfsetroundcap%
\pgfsetroundjoin%
\pgfsetlinewidth{1.003750pt}%
\definecolor{currentstroke}{rgb}{0.200000,0.427451,0.650980}%
\pgfsetstrokecolor{currentstroke}%
\pgfsetdash{}{0pt}%
\pgfpathmoveto{\pgfqpoint{0.566986in}{0.885615in}}%
\pgfpathlineto{\pgfqpoint{0.592091in}{1.288579in}}%
\pgfpathlineto{\pgfqpoint{0.617196in}{1.568501in}}%
\pgfpathlineto{\pgfqpoint{0.642300in}{1.829966in}}%
\pgfpathlineto{\pgfqpoint{0.667405in}{2.036063in}}%
\pgfpathlineto{\pgfqpoint{0.692509in}{2.199094in}}%
\pgfpathlineto{\pgfqpoint{0.717614in}{2.365201in}}%
\pgfpathlineto{\pgfqpoint{0.742719in}{2.485168in}}%
\pgfpathlineto{\pgfqpoint{0.767823in}{2.617438in}}%
\pgfpathlineto{\pgfqpoint{0.792928in}{2.706644in}}%
\pgfpathlineto{\pgfqpoint{0.818032in}{2.795850in}}%
\pgfpathlineto{\pgfqpoint{0.843137in}{2.860447in}}%
\pgfpathlineto{\pgfqpoint{0.868241in}{2.958881in}}%
\pgfpathlineto{\pgfqpoint{0.893346in}{3.029631in}}%
\pgfpathlineto{\pgfqpoint{0.918451in}{3.057315in}}%
\pgfpathlineto{\pgfqpoint{0.943555in}{3.103456in}}%
\pgfpathlineto{\pgfqpoint{0.968660in}{3.131141in}}%
\pgfpathlineto{\pgfqpoint{0.993764in}{3.158825in}}%
\pgfpathlineto{\pgfqpoint{1.018869in}{3.177282in}}%
\pgfpathlineto{\pgfqpoint{1.043973in}{3.211119in}}%
\pgfpathlineto{\pgfqpoint{1.069078in}{3.241879in}}%
\pgfpathlineto{\pgfqpoint{1.094183in}{3.272640in}}%
\pgfpathlineto{\pgfqpoint{1.119287in}{3.291096in}}%
\pgfpathlineto{\pgfqpoint{1.144392in}{3.300324in}}%
\pgfpathlineto{\pgfqpoint{1.169496in}{3.315705in}}%
\pgfpathlineto{\pgfqpoint{1.194601in}{3.324933in}}%
\pgfpathlineto{\pgfqpoint{1.219705in}{3.328009in}}%
\pgfpathlineto{\pgfqpoint{1.244810in}{3.340313in}}%
\pgfpathlineto{\pgfqpoint{1.269915in}{3.355693in}}%
\pgfpathlineto{\pgfqpoint{1.295019in}{3.358770in}}%
\pgfpathlineto{\pgfqpoint{1.320124in}{3.361846in}}%
\pgfpathlineto{\pgfqpoint{1.345228in}{3.361846in}}%
\pgfpathlineto{\pgfqpoint{1.370333in}{3.364922in}}%
\pgfpathlineto{\pgfqpoint{1.395438in}{3.364922in}}%
\pgfpathlineto{\pgfqpoint{1.420542in}{3.367998in}}%
\pgfpathlineto{\pgfqpoint{1.445647in}{3.367998in}}%
\pgfpathlineto{\pgfqpoint{1.470751in}{3.367998in}}%
\pgfpathlineto{\pgfqpoint{1.495856in}{3.367998in}}%
\pgfpathlineto{\pgfqpoint{1.520960in}{3.371074in}}%
\pgfpathlineto{\pgfqpoint{1.546065in}{3.377226in}}%
\pgfpathlineto{\pgfqpoint{1.571170in}{3.380302in}}%
\pgfpathlineto{\pgfqpoint{1.596274in}{3.386454in}}%
\pgfpathlineto{\pgfqpoint{1.621379in}{3.389530in}}%
\pgfpathlineto{\pgfqpoint{1.646483in}{3.389530in}}%
\pgfpathlineto{\pgfqpoint{1.671588in}{3.389530in}}%
\pgfpathlineto{\pgfqpoint{1.696692in}{3.389530in}}%
\pgfpathlineto{\pgfqpoint{1.721797in}{3.389530in}}%
\pgfpathlineto{\pgfqpoint{1.746902in}{3.389530in}}%
\pgfpathlineto{\pgfqpoint{1.772006in}{3.389530in}}%
\pgfpathlineto{\pgfqpoint{1.797111in}{3.392606in}}%
\pgfusepath{stroke}%
\end{pgfscope}%
\begin{pgfscope}%
\pgfpathrectangle{\pgfqpoint{0.566985in}{0.528177in}}{\pgfqpoint{2.106702in}{2.864429in}} %
\pgfusepath{clip}%
\pgfsetroundcap%
\pgfsetroundjoin%
\pgfsetlinewidth{1.003750pt}%
\definecolor{currentstroke}{rgb}{0.168627,0.670588,0.494118}%
\pgfsetstrokecolor{currentstroke}%
\pgfsetdash{}{0pt}%
\pgfpathmoveto{\pgfqpoint{0.566986in}{1.312024in}}%
\pgfpathlineto{\pgfqpoint{0.606085in}{1.735398in}}%
\pgfpathlineto{\pgfqpoint{0.645184in}{2.098291in}}%
\pgfpathlineto{\pgfqpoint{0.684283in}{2.285785in}}%
\pgfpathlineto{\pgfqpoint{0.723381in}{2.497472in}}%
\pgfpathlineto{\pgfqpoint{0.762480in}{2.678918in}}%
\pgfpathlineto{\pgfqpoint{0.801579in}{2.818027in}}%
\pgfpathlineto{\pgfqpoint{0.840678in}{2.914798in}}%
\pgfpathlineto{\pgfqpoint{0.879777in}{3.017618in}}%
\pgfpathlineto{\pgfqpoint{0.918876in}{3.090196in}}%
\pgfpathlineto{\pgfqpoint{0.957975in}{3.126485in}}%
\pgfpathlineto{\pgfqpoint{0.997074in}{3.150678in}}%
\pgfpathlineto{\pgfqpoint{1.036173in}{3.199064in}}%
\pgfpathlineto{\pgfqpoint{1.075271in}{3.259546in}}%
\pgfpathlineto{\pgfqpoint{1.114370in}{3.283739in}}%
\pgfpathlineto{\pgfqpoint{1.153469in}{3.289787in}}%
\pgfpathlineto{\pgfqpoint{1.192568in}{3.295835in}}%
\pgfpathlineto{\pgfqpoint{1.231667in}{3.301883in}}%
\pgfpathlineto{\pgfqpoint{1.270766in}{3.301883in}}%
\pgfpathlineto{\pgfqpoint{1.309865in}{3.313980in}}%
\pgfpathlineto{\pgfqpoint{1.348964in}{3.326076in}}%
\pgfpathlineto{\pgfqpoint{1.388063in}{3.344221in}}%
\pgfpathlineto{\pgfqpoint{1.427162in}{3.344221in}}%
\pgfpathlineto{\pgfqpoint{1.466260in}{3.356317in}}%
\pgfpathlineto{\pgfqpoint{1.505359in}{3.356317in}}%
\pgfpathlineto{\pgfqpoint{1.544458in}{3.362365in}}%
\pgfpathlineto{\pgfqpoint{1.583557in}{3.362365in}}%
\pgfpathlineto{\pgfqpoint{1.622656in}{3.368413in}}%
\pgfpathlineto{\pgfqpoint{1.661755in}{3.368413in}}%
\pgfpathlineto{\pgfqpoint{1.700854in}{3.368413in}}%
\pgfpathlineto{\pgfqpoint{1.739953in}{3.368413in}}%
\pgfpathlineto{\pgfqpoint{1.779052in}{3.374462in}}%
\pgfpathlineto{\pgfqpoint{1.818151in}{3.374462in}}%
\pgfpathlineto{\pgfqpoint{1.857249in}{3.374462in}}%
\pgfpathlineto{\pgfqpoint{1.896348in}{3.374462in}}%
\pgfpathlineto{\pgfqpoint{1.935447in}{3.374462in}}%
\pgfpathlineto{\pgfqpoint{1.974546in}{3.374462in}}%
\pgfpathlineto{\pgfqpoint{2.013645in}{3.374462in}}%
\pgfpathlineto{\pgfqpoint{2.052744in}{3.374462in}}%
\pgfpathlineto{\pgfqpoint{2.091843in}{3.374462in}}%
\pgfpathlineto{\pgfqpoint{2.130942in}{3.374462in}}%
\pgfpathlineto{\pgfqpoint{2.170041in}{3.374462in}}%
\pgfpathlineto{\pgfqpoint{2.209140in}{3.374462in}}%
\pgfpathlineto{\pgfqpoint{2.248238in}{3.380510in}}%
\pgfpathlineto{\pgfqpoint{2.287337in}{3.380510in}}%
\pgfpathlineto{\pgfqpoint{2.326436in}{3.380510in}}%
\pgfpathlineto{\pgfqpoint{2.365535in}{3.380510in}}%
\pgfpathlineto{\pgfqpoint{2.404634in}{3.380510in}}%
\pgfpathlineto{\pgfqpoint{2.443733in}{3.380510in}}%
\pgfpathlineto{\pgfqpoint{2.482832in}{3.392606in}}%
\pgfusepath{stroke}%
\end{pgfscope}%
\begin{pgfscope}%
\pgfpathrectangle{\pgfqpoint{0.566985in}{0.528177in}}{\pgfqpoint{2.106702in}{2.864429in}} %
\pgfusepath{clip}%
\pgfsetroundcap%
\pgfsetroundjoin%
\pgfsetlinewidth{1.003750pt}%
\definecolor{currentstroke}{rgb}{1.000000,0.494118,0.250980}%
\pgfsetstrokecolor{currentstroke}%
\pgfsetdash{}{0pt}%
\pgfpathmoveto{\pgfqpoint{0.566995in}{1.161875in}}%
\pgfpathlineto{\pgfqpoint{0.609624in}{1.552608in}}%
\pgfpathlineto{\pgfqpoint{0.652254in}{1.850986in}}%
\pgfpathlineto{\pgfqpoint{0.694883in}{2.071218in}}%
\pgfpathlineto{\pgfqpoint{0.737512in}{2.284345in}}%
\pgfpathlineto{\pgfqpoint{0.780142in}{2.433534in}}%
\pgfpathlineto{\pgfqpoint{0.822771in}{2.575619in}}%
\pgfpathlineto{\pgfqpoint{0.865401in}{2.739016in}}%
\pgfpathlineto{\pgfqpoint{0.908030in}{2.838476in}}%
\pgfpathlineto{\pgfqpoint{0.950660in}{2.973456in}}%
\pgfpathlineto{\pgfqpoint{0.993289in}{3.023186in}}%
\pgfpathlineto{\pgfqpoint{1.035919in}{3.087124in}}%
\pgfpathlineto{\pgfqpoint{1.078548in}{3.143958in}}%
\pgfpathlineto{\pgfqpoint{1.121178in}{3.200792in}}%
\pgfpathlineto{\pgfqpoint{1.163807in}{3.236313in}}%
\pgfpathlineto{\pgfqpoint{1.206437in}{3.243417in}}%
\pgfpathlineto{\pgfqpoint{1.249066in}{3.264730in}}%
\pgfpathlineto{\pgfqpoint{1.291695in}{3.286043in}}%
\pgfpathlineto{\pgfqpoint{1.334325in}{3.300251in}}%
\pgfpathlineto{\pgfqpoint{1.376954in}{3.321564in}}%
\pgfpathlineto{\pgfqpoint{1.419584in}{3.328668in}}%
\pgfpathlineto{\pgfqpoint{1.462213in}{3.335772in}}%
\pgfpathlineto{\pgfqpoint{1.504843in}{3.349981in}}%
\pgfpathlineto{\pgfqpoint{1.547472in}{3.364189in}}%
\pgfpathlineto{\pgfqpoint{1.590102in}{3.364189in}}%
\pgfpathlineto{\pgfqpoint{1.632731in}{3.364189in}}%
\pgfpathlineto{\pgfqpoint{1.675361in}{3.364189in}}%
\pgfpathlineto{\pgfqpoint{1.717990in}{3.371294in}}%
\pgfpathlineto{\pgfqpoint{1.760619in}{3.371294in}}%
\pgfpathlineto{\pgfqpoint{1.803249in}{3.371294in}}%
\pgfpathlineto{\pgfqpoint{1.845878in}{3.371294in}}%
\pgfpathlineto{\pgfqpoint{1.888508in}{3.371294in}}%
\pgfpathlineto{\pgfqpoint{1.931137in}{3.378398in}}%
\pgfpathlineto{\pgfqpoint{1.973767in}{3.378398in}}%
\pgfpathlineto{\pgfqpoint{2.016396in}{3.378398in}}%
\pgfpathlineto{\pgfqpoint{2.059026in}{3.378398in}}%
\pgfpathlineto{\pgfqpoint{2.101655in}{3.385502in}}%
\pgfpathlineto{\pgfqpoint{2.144285in}{3.385502in}}%
\pgfpathlineto{\pgfqpoint{2.186914in}{3.385502in}}%
\pgfpathlineto{\pgfqpoint{2.229543in}{3.385502in}}%
\pgfpathlineto{\pgfqpoint{2.272173in}{3.385502in}}%
\pgfpathlineto{\pgfqpoint{2.314802in}{3.385502in}}%
\pgfpathlineto{\pgfqpoint{2.357432in}{3.385502in}}%
\pgfpathlineto{\pgfqpoint{2.400061in}{3.385502in}}%
\pgfpathlineto{\pgfqpoint{2.442691in}{3.385502in}}%
\pgfpathlineto{\pgfqpoint{2.485320in}{3.385502in}}%
\pgfpathlineto{\pgfqpoint{2.527950in}{3.385502in}}%
\pgfpathlineto{\pgfqpoint{2.570579in}{3.385502in}}%
\pgfpathlineto{\pgfqpoint{2.613209in}{3.385502in}}%
\pgfpathlineto{\pgfqpoint{2.655838in}{3.392606in}}%
\pgfusepath{stroke}%
\end{pgfscope}%
\begin{pgfscope}%
\pgfpathrectangle{\pgfqpoint{0.566985in}{0.528177in}}{\pgfqpoint{2.106702in}{2.864429in}} %
\pgfusepath{clip}%
\pgfsetroundcap%
\pgfsetroundjoin%
\pgfsetlinewidth{1.003750pt}%
\definecolor{currentstroke}{rgb}{1.000000,0.694118,0.250980}%
\pgfsetstrokecolor{currentstroke}%
\pgfsetdash{}{0pt}%
\pgfpathmoveto{\pgfqpoint{0.566985in}{1.669156in}}%
\pgfpathlineto{\pgfqpoint{0.606167in}{1.991727in}}%
\pgfpathlineto{\pgfqpoint{0.645350in}{2.231351in}}%
\pgfpathlineto{\pgfqpoint{0.684532in}{2.447934in}}%
\pgfpathlineto{\pgfqpoint{0.723714in}{2.600003in}}%
\pgfpathlineto{\pgfqpoint{0.762896in}{2.729032in}}%
\pgfpathlineto{\pgfqpoint{0.802079in}{2.798154in}}%
\pgfpathlineto{\pgfqpoint{0.841261in}{2.867277in}}%
\pgfpathlineto{\pgfqpoint{0.880443in}{2.927183in}}%
\pgfpathlineto{\pgfqpoint{0.919625in}{2.996305in}}%
\pgfpathlineto{\pgfqpoint{0.958808in}{3.028562in}}%
\pgfpathlineto{\pgfqpoint{0.997990in}{3.079252in}}%
\pgfpathlineto{\pgfqpoint{1.037172in}{3.120725in}}%
\pgfpathlineto{\pgfqpoint{1.076354in}{3.166807in}}%
\pgfpathlineto{\pgfqpoint{1.115537in}{3.203672in}}%
\pgfpathlineto{\pgfqpoint{1.154719in}{3.226713in}}%
\pgfpathlineto{\pgfqpoint{1.193901in}{3.245145in}}%
\pgfpathlineto{\pgfqpoint{1.233083in}{3.258970in}}%
\pgfpathlineto{\pgfqpoint{1.272265in}{3.272794in}}%
\pgfpathlineto{\pgfqpoint{1.311448in}{3.277402in}}%
\pgfpathlineto{\pgfqpoint{1.350630in}{3.282011in}}%
\pgfpathlineto{\pgfqpoint{1.389812in}{3.295835in}}%
\pgfpathlineto{\pgfqpoint{1.428994in}{3.305051in}}%
\pgfpathlineto{\pgfqpoint{1.468177in}{3.328092in}}%
\pgfpathlineto{\pgfqpoint{1.507359in}{3.341917in}}%
\pgfpathlineto{\pgfqpoint{1.546541in}{3.355741in}}%
\pgfpathlineto{\pgfqpoint{1.585723in}{3.360349in}}%
\pgfpathlineto{\pgfqpoint{1.624906in}{3.364957in}}%
\pgfpathlineto{\pgfqpoint{1.664088in}{3.378782in}}%
\pgfpathlineto{\pgfqpoint{1.703270in}{3.383390in}}%
\pgfpathlineto{\pgfqpoint{1.742452in}{3.383390in}}%
\pgfpathlineto{\pgfqpoint{1.781635in}{3.383390in}}%
\pgfpathlineto{\pgfqpoint{1.820817in}{3.383390in}}%
\pgfpathlineto{\pgfqpoint{1.859999in}{3.383390in}}%
\pgfpathlineto{\pgfqpoint{1.899181in}{3.383390in}}%
\pgfpathlineto{\pgfqpoint{1.938364in}{3.387998in}}%
\pgfpathlineto{\pgfqpoint{1.977546in}{3.387998in}}%
\pgfpathlineto{\pgfqpoint{2.016728in}{3.387998in}}%
\pgfpathlineto{\pgfqpoint{2.055910in}{3.387998in}}%
\pgfpathlineto{\pgfqpoint{2.095092in}{3.387998in}}%
\pgfpathlineto{\pgfqpoint{2.134275in}{3.387998in}}%
\pgfpathlineto{\pgfqpoint{2.173457in}{3.387998in}}%
\pgfpathlineto{\pgfqpoint{2.212639in}{3.387998in}}%
\pgfpathlineto{\pgfqpoint{2.251821in}{3.387998in}}%
\pgfpathlineto{\pgfqpoint{2.291004in}{3.387998in}}%
\pgfpathlineto{\pgfqpoint{2.330186in}{3.387998in}}%
\pgfpathlineto{\pgfqpoint{2.369368in}{3.387998in}}%
\pgfpathlineto{\pgfqpoint{2.408550in}{3.387998in}}%
\pgfpathlineto{\pgfqpoint{2.447733in}{3.387998in}}%
\pgfpathlineto{\pgfqpoint{2.486915in}{3.392606in}}%
\pgfusepath{stroke}%
\end{pgfscope}%
\begin{pgfscope}%
\pgfsetrectcap%
\pgfsetmiterjoin%
\pgfsetlinewidth{1.254687pt}%
\definecolor{currentstroke}{rgb}{0.150000,0.150000,0.150000}%
\pgfsetstrokecolor{currentstroke}%
\pgfsetdash{}{0pt}%
\pgfpathmoveto{\pgfqpoint{0.566985in}{0.528177in}}%
\pgfpathlineto{\pgfqpoint{0.566985in}{3.392606in}}%
\pgfusepath{stroke}%
\end{pgfscope}%
\begin{pgfscope}%
\pgfsetrectcap%
\pgfsetmiterjoin%
\pgfsetlinewidth{1.254687pt}%
\definecolor{currentstroke}{rgb}{0.150000,0.150000,0.150000}%
\pgfsetstrokecolor{currentstroke}%
\pgfsetdash{}{0pt}%
\pgfpathmoveto{\pgfqpoint{0.566985in}{0.528177in}}%
\pgfpathlineto{\pgfqpoint{2.673686in}{0.528177in}}%
\pgfusepath{stroke}%
\end{pgfscope}%
\begin{pgfscope}%
\pgfsetbuttcap%
\pgfsetmiterjoin%
\definecolor{currentfill}{rgb}{1.000000,1.000000,1.000000}%
\pgfsetfillcolor{currentfill}%
\pgfsetlinewidth{0.000000pt}%
\definecolor{currentstroke}{rgb}{0.000000,0.000000,0.000000}%
\pgfsetstrokecolor{currentstroke}%
\pgfsetstrokeopacity{0.000000}%
\pgfsetdash{}{0pt}%
\pgfpathmoveto{\pgfqpoint{3.095027in}{0.528177in}}%
\pgfpathlineto{\pgfqpoint{5.201729in}{0.528177in}}%
\pgfpathlineto{\pgfqpoint{5.201729in}{2.653399in}}%
\pgfpathlineto{\pgfqpoint{3.095027in}{2.653399in}}%
\pgfpathclose%
\pgfusepath{fill}%
\end{pgfscope}%
\begin{pgfscope}%
\pgfsetroundcap%
\pgfsetroundjoin%
\pgfsetlinewidth{1.003750pt}%
\definecolor{currentstroke}{rgb}{0.200000,0.427451,0.650980}%
\pgfsetstrokecolor{currentstroke}%
\pgfsetdash{}{0pt}%
\pgfpathmoveto{\pgfqpoint{2.984357in}{3.319977in}}%
\pgfpathlineto{\pgfqpoint{3.095468in}{3.319977in}}%
\pgfusepath{stroke}%
\end{pgfscope}%
\begin{pgfscope}%
\definecolor{textcolor}{rgb}{1.000000,1.000000,1.000000}%
\pgfsetstrokecolor{textcolor}%
\pgfsetfillcolor{textcolor}%
\pgftext[x=3.184357in,y=3.281088in,left,base]{\color{textcolor}\rmfamily\fontsize{8.000000}{9.600000}\selectfont WT + Vehicle (1164)}%
\end{pgfscope}%
\begin{pgfscope}%
\pgfsetroundcap%
\pgfsetroundjoin%
\pgfsetlinewidth{1.003750pt}%
\definecolor{currentstroke}{rgb}{0.168627,0.670588,0.494118}%
\pgfsetstrokecolor{currentstroke}%
\pgfsetdash{}{0pt}%
\pgfpathmoveto{\pgfqpoint{2.984357in}{3.153338in}}%
\pgfpathlineto{\pgfqpoint{3.095468in}{3.153338in}}%
\pgfusepath{stroke}%
\end{pgfscope}%
\begin{pgfscope}%
\definecolor{textcolor}{rgb}{1.000000,1.000000,1.000000}%
\pgfsetstrokecolor{textcolor}%
\pgfsetfillcolor{textcolor}%
\pgftext[x=3.184357in,y=3.114449in,left,base]{\color{textcolor}\rmfamily\fontsize{8.000000}{9.600000}\selectfont WT + TAT-GluA2\textsubscript{3Y} (592)}%
\end{pgfscope}%
\begin{pgfscope}%
\pgfsetroundcap%
\pgfsetroundjoin%
\pgfsetlinewidth{1.003750pt}%
\definecolor{currentstroke}{rgb}{1.000000,0.494118,0.250980}%
\pgfsetstrokecolor{currentstroke}%
\pgfsetdash{}{0pt}%
\pgfpathmoveto{\pgfqpoint{2.984357in}{2.986698in}}%
\pgfpathlineto{\pgfqpoint{3.095468in}{2.986698in}}%
\pgfusepath{stroke}%
\end{pgfscope}%
\begin{pgfscope}%
\definecolor{textcolor}{rgb}{1.000000,1.000000,1.000000}%
\pgfsetstrokecolor{textcolor}%
\pgfsetfillcolor{textcolor}%
\pgftext[x=3.184357in,y=2.947809in,left,base]{\color{textcolor}\rmfamily\fontsize{8.000000}{9.600000}\selectfont Tg + Vehicle (504)}%
\end{pgfscope}%
\begin{pgfscope}%
\pgfsetroundcap%
\pgfsetroundjoin%
\pgfsetlinewidth{1.003750pt}%
\definecolor{currentstroke}{rgb}{1.000000,0.694118,0.250980}%
\pgfsetstrokecolor{currentstroke}%
\pgfsetdash{}{0pt}%
\pgfpathmoveto{\pgfqpoint{2.984357in}{2.820059in}}%
\pgfpathlineto{\pgfqpoint{3.095468in}{2.820059in}}%
\pgfusepath{stroke}%
\end{pgfscope}%
\begin{pgfscope}%
\definecolor{textcolor}{rgb}{1.000000,1.000000,1.000000}%
\pgfsetstrokecolor{textcolor}%
\pgfsetfillcolor{textcolor}%
\pgftext[x=3.184357in,y=2.781170in,left,base]{\color{textcolor}\rmfamily\fontsize{8.000000}{9.600000}\selectfont Tg + TAT-GluA2\textsubscript{3Y} (777)}%
\end{pgfscope}%
\begin{pgfscope}%
\pgfsetroundcap%
\pgfsetroundjoin%
\pgfsetlinewidth{1.003750pt}%
\definecolor{currentstroke}{rgb}{0.200000,0.427451,0.650980}%
\pgfsetstrokecolor{currentstroke}%
\pgfsetdash{}{0pt}%
\pgfpathmoveto{\pgfqpoint{2.984357in}{3.319977in}}%
\pgfpathlineto{\pgfqpoint{3.095468in}{3.319977in}}%
\pgfusepath{stroke}%
\end{pgfscope}%
\begin{pgfscope}%
\definecolor{textcolor}{rgb}{1.000000,1.000000,1.000000}%
\pgfsetstrokecolor{textcolor}%
\pgfsetfillcolor{textcolor}%
\pgftext[x=3.184357in,y=3.281088in,left,base]{\color{textcolor}\rmfamily\fontsize{8.000000}{9.600000}\selectfont WT + Vehicle (1164)}%
\end{pgfscope}%
\begin{pgfscope}%
\pgfsetroundcap%
\pgfsetroundjoin%
\pgfsetlinewidth{1.003750pt}%
\definecolor{currentstroke}{rgb}{0.168627,0.670588,0.494118}%
\pgfsetstrokecolor{currentstroke}%
\pgfsetdash{}{0pt}%
\pgfpathmoveto{\pgfqpoint{2.984357in}{3.153338in}}%
\pgfpathlineto{\pgfqpoint{3.095468in}{3.153338in}}%
\pgfusepath{stroke}%
\end{pgfscope}%
\begin{pgfscope}%
\definecolor{textcolor}{rgb}{1.000000,1.000000,1.000000}%
\pgfsetstrokecolor{textcolor}%
\pgfsetfillcolor{textcolor}%
\pgftext[x=3.184357in,y=3.114449in,left,base]{\color{textcolor}\rmfamily\fontsize{8.000000}{9.600000}\selectfont WT + TAT-GluA2\textsubscript{3Y} (592)}%
\end{pgfscope}%
\begin{pgfscope}%
\pgfsetroundcap%
\pgfsetroundjoin%
\pgfsetlinewidth{1.003750pt}%
\definecolor{currentstroke}{rgb}{1.000000,0.494118,0.250980}%
\pgfsetstrokecolor{currentstroke}%
\pgfsetdash{}{0pt}%
\pgfpathmoveto{\pgfqpoint{2.984357in}{2.986698in}}%
\pgfpathlineto{\pgfqpoint{3.095468in}{2.986698in}}%
\pgfusepath{stroke}%
\end{pgfscope}%
\begin{pgfscope}%
\definecolor{textcolor}{rgb}{1.000000,1.000000,1.000000}%
\pgfsetstrokecolor{textcolor}%
\pgfsetfillcolor{textcolor}%
\pgftext[x=3.184357in,y=2.947809in,left,base]{\color{textcolor}\rmfamily\fontsize{8.000000}{9.600000}\selectfont Tg + Vehicle (504)}%
\end{pgfscope}%
\begin{pgfscope}%
\pgfsetroundcap%
\pgfsetroundjoin%
\pgfsetlinewidth{1.003750pt}%
\definecolor{currentstroke}{rgb}{1.000000,0.694118,0.250980}%
\pgfsetstrokecolor{currentstroke}%
\pgfsetdash{}{0pt}%
\pgfpathmoveto{\pgfqpoint{2.984357in}{2.820059in}}%
\pgfpathlineto{\pgfqpoint{3.095468in}{2.820059in}}%
\pgfusepath{stroke}%
\end{pgfscope}%
\begin{pgfscope}%
\definecolor{textcolor}{rgb}{1.000000,1.000000,1.000000}%
\pgfsetstrokecolor{textcolor}%
\pgfsetfillcolor{textcolor}%
\pgftext[x=3.184357in,y=2.781170in,left,base]{\color{textcolor}\rmfamily\fontsize{8.000000}{9.600000}\selectfont Tg + TAT-GluA2\textsubscript{3Y} (777)}%
\end{pgfscope}%
\begin{pgfscope}%
\pgfsetbuttcap%
\pgfsetroundjoin%
\definecolor{currentfill}{rgb}{0.150000,0.150000,0.150000}%
\pgfsetfillcolor{currentfill}%
\pgfsetlinewidth{1.003750pt}%
\definecolor{currentstroke}{rgb}{0.150000,0.150000,0.150000}%
\pgfsetstrokecolor{currentstroke}%
\pgfsetdash{}{0pt}%
\pgfsys@defobject{currentmarker}{\pgfqpoint{0.000000in}{0.000000in}}{\pgfqpoint{0.041667in}{0.000000in}}{%
\pgfpathmoveto{\pgfqpoint{0.000000in}{0.000000in}}%
\pgfpathlineto{\pgfqpoint{0.041667in}{0.000000in}}%
\pgfusepath{stroke,fill}%
}%
\begin{pgfscope}%
\pgfsys@transformshift{3.095027in}{0.528177in}%
\pgfsys@useobject{currentmarker}{}%
\end{pgfscope}%
\end{pgfscope}%
\begin{pgfscope}%
\definecolor{textcolor}{rgb}{0.150000,0.150000,0.150000}%
\pgfsetstrokecolor{textcolor}%
\pgfsetfillcolor{textcolor}%
\pgftext[x=2.997805in,y=0.528177in,right,]{\color{textcolor}\rmfamily\fontsize{10.000000}{12.000000}\selectfont \(\displaystyle 0.00\)}%
\end{pgfscope}%
\begin{pgfscope}%
\pgfsetbuttcap%
\pgfsetroundjoin%
\definecolor{currentfill}{rgb}{0.150000,0.150000,0.150000}%
\pgfsetfillcolor{currentfill}%
\pgfsetlinewidth{1.003750pt}%
\definecolor{currentstroke}{rgb}{0.150000,0.150000,0.150000}%
\pgfsetstrokecolor{currentstroke}%
\pgfsetdash{}{0pt}%
\pgfsys@defobject{currentmarker}{\pgfqpoint{0.000000in}{0.000000in}}{\pgfqpoint{0.041667in}{0.000000in}}{%
\pgfpathmoveto{\pgfqpoint{0.000000in}{0.000000in}}%
\pgfpathlineto{\pgfqpoint{0.041667in}{0.000000in}}%
\pgfusepath{stroke,fill}%
}%
\begin{pgfscope}%
\pgfsys@transformshift{3.095027in}{0.764313in}%
\pgfsys@useobject{currentmarker}{}%
\end{pgfscope}%
\end{pgfscope}%
\begin{pgfscope}%
\definecolor{textcolor}{rgb}{0.150000,0.150000,0.150000}%
\pgfsetstrokecolor{textcolor}%
\pgfsetfillcolor{textcolor}%
\pgftext[x=2.997805in,y=0.764313in,right,]{\color{textcolor}\rmfamily\fontsize{10.000000}{12.000000}\selectfont \(\displaystyle 0.02\)}%
\end{pgfscope}%
\begin{pgfscope}%
\pgfsetbuttcap%
\pgfsetroundjoin%
\definecolor{currentfill}{rgb}{0.150000,0.150000,0.150000}%
\pgfsetfillcolor{currentfill}%
\pgfsetlinewidth{1.003750pt}%
\definecolor{currentstroke}{rgb}{0.150000,0.150000,0.150000}%
\pgfsetstrokecolor{currentstroke}%
\pgfsetdash{}{0pt}%
\pgfsys@defobject{currentmarker}{\pgfqpoint{0.000000in}{0.000000in}}{\pgfqpoint{0.041667in}{0.000000in}}{%
\pgfpathmoveto{\pgfqpoint{0.000000in}{0.000000in}}%
\pgfpathlineto{\pgfqpoint{0.041667in}{0.000000in}}%
\pgfusepath{stroke,fill}%
}%
\begin{pgfscope}%
\pgfsys@transformshift{3.095027in}{1.000448in}%
\pgfsys@useobject{currentmarker}{}%
\end{pgfscope}%
\end{pgfscope}%
\begin{pgfscope}%
\definecolor{textcolor}{rgb}{0.150000,0.150000,0.150000}%
\pgfsetstrokecolor{textcolor}%
\pgfsetfillcolor{textcolor}%
\pgftext[x=2.997805in,y=1.000448in,right,]{\color{textcolor}\rmfamily\fontsize{10.000000}{12.000000}\selectfont \(\displaystyle 0.04\)}%
\end{pgfscope}%
\begin{pgfscope}%
\pgfsetbuttcap%
\pgfsetroundjoin%
\definecolor{currentfill}{rgb}{0.150000,0.150000,0.150000}%
\pgfsetfillcolor{currentfill}%
\pgfsetlinewidth{1.003750pt}%
\definecolor{currentstroke}{rgb}{0.150000,0.150000,0.150000}%
\pgfsetstrokecolor{currentstroke}%
\pgfsetdash{}{0pt}%
\pgfsys@defobject{currentmarker}{\pgfqpoint{0.000000in}{0.000000in}}{\pgfqpoint{0.041667in}{0.000000in}}{%
\pgfpathmoveto{\pgfqpoint{0.000000in}{0.000000in}}%
\pgfpathlineto{\pgfqpoint{0.041667in}{0.000000in}}%
\pgfusepath{stroke,fill}%
}%
\begin{pgfscope}%
\pgfsys@transformshift{3.095027in}{1.236584in}%
\pgfsys@useobject{currentmarker}{}%
\end{pgfscope}%
\end{pgfscope}%
\begin{pgfscope}%
\definecolor{textcolor}{rgb}{0.150000,0.150000,0.150000}%
\pgfsetstrokecolor{textcolor}%
\pgfsetfillcolor{textcolor}%
\pgftext[x=2.997805in,y=1.236584in,right,]{\color{textcolor}\rmfamily\fontsize{10.000000}{12.000000}\selectfont \(\displaystyle 0.06\)}%
\end{pgfscope}%
\begin{pgfscope}%
\pgfsetbuttcap%
\pgfsetroundjoin%
\definecolor{currentfill}{rgb}{0.150000,0.150000,0.150000}%
\pgfsetfillcolor{currentfill}%
\pgfsetlinewidth{1.003750pt}%
\definecolor{currentstroke}{rgb}{0.150000,0.150000,0.150000}%
\pgfsetstrokecolor{currentstroke}%
\pgfsetdash{}{0pt}%
\pgfsys@defobject{currentmarker}{\pgfqpoint{0.000000in}{0.000000in}}{\pgfqpoint{0.041667in}{0.000000in}}{%
\pgfpathmoveto{\pgfqpoint{0.000000in}{0.000000in}}%
\pgfpathlineto{\pgfqpoint{0.041667in}{0.000000in}}%
\pgfusepath{stroke,fill}%
}%
\begin{pgfscope}%
\pgfsys@transformshift{3.095027in}{1.472720in}%
\pgfsys@useobject{currentmarker}{}%
\end{pgfscope}%
\end{pgfscope}%
\begin{pgfscope}%
\definecolor{textcolor}{rgb}{0.150000,0.150000,0.150000}%
\pgfsetstrokecolor{textcolor}%
\pgfsetfillcolor{textcolor}%
\pgftext[x=2.997805in,y=1.472720in,right,]{\color{textcolor}\rmfamily\fontsize{10.000000}{12.000000}\selectfont \(\displaystyle 0.08\)}%
\end{pgfscope}%
\begin{pgfscope}%
\pgfsetbuttcap%
\pgfsetroundjoin%
\definecolor{currentfill}{rgb}{0.150000,0.150000,0.150000}%
\pgfsetfillcolor{currentfill}%
\pgfsetlinewidth{1.003750pt}%
\definecolor{currentstroke}{rgb}{0.150000,0.150000,0.150000}%
\pgfsetstrokecolor{currentstroke}%
\pgfsetdash{}{0pt}%
\pgfsys@defobject{currentmarker}{\pgfqpoint{0.000000in}{0.000000in}}{\pgfqpoint{0.041667in}{0.000000in}}{%
\pgfpathmoveto{\pgfqpoint{0.000000in}{0.000000in}}%
\pgfpathlineto{\pgfqpoint{0.041667in}{0.000000in}}%
\pgfusepath{stroke,fill}%
}%
\begin{pgfscope}%
\pgfsys@transformshift{3.095027in}{1.708856in}%
\pgfsys@useobject{currentmarker}{}%
\end{pgfscope}%
\end{pgfscope}%
\begin{pgfscope}%
\definecolor{textcolor}{rgb}{0.150000,0.150000,0.150000}%
\pgfsetstrokecolor{textcolor}%
\pgfsetfillcolor{textcolor}%
\pgftext[x=2.997805in,y=1.708856in,right,]{\color{textcolor}\rmfamily\fontsize{10.000000}{12.000000}\selectfont \(\displaystyle 0.10\)}%
\end{pgfscope}%
\begin{pgfscope}%
\pgfsetbuttcap%
\pgfsetroundjoin%
\definecolor{currentfill}{rgb}{0.150000,0.150000,0.150000}%
\pgfsetfillcolor{currentfill}%
\pgfsetlinewidth{1.003750pt}%
\definecolor{currentstroke}{rgb}{0.150000,0.150000,0.150000}%
\pgfsetstrokecolor{currentstroke}%
\pgfsetdash{}{0pt}%
\pgfsys@defobject{currentmarker}{\pgfqpoint{0.000000in}{0.000000in}}{\pgfqpoint{0.041667in}{0.000000in}}{%
\pgfpathmoveto{\pgfqpoint{0.000000in}{0.000000in}}%
\pgfpathlineto{\pgfqpoint{0.041667in}{0.000000in}}%
\pgfusepath{stroke,fill}%
}%
\begin{pgfscope}%
\pgfsys@transformshift{3.095027in}{1.944991in}%
\pgfsys@useobject{currentmarker}{}%
\end{pgfscope}%
\end{pgfscope}%
\begin{pgfscope}%
\definecolor{textcolor}{rgb}{0.150000,0.150000,0.150000}%
\pgfsetstrokecolor{textcolor}%
\pgfsetfillcolor{textcolor}%
\pgftext[x=2.997805in,y=1.944991in,right,]{\color{textcolor}\rmfamily\fontsize{10.000000}{12.000000}\selectfont \(\displaystyle 0.12\)}%
\end{pgfscope}%
\begin{pgfscope}%
\pgfsetbuttcap%
\pgfsetroundjoin%
\definecolor{currentfill}{rgb}{0.150000,0.150000,0.150000}%
\pgfsetfillcolor{currentfill}%
\pgfsetlinewidth{1.003750pt}%
\definecolor{currentstroke}{rgb}{0.150000,0.150000,0.150000}%
\pgfsetstrokecolor{currentstroke}%
\pgfsetdash{}{0pt}%
\pgfsys@defobject{currentmarker}{\pgfqpoint{0.000000in}{0.000000in}}{\pgfqpoint{0.041667in}{0.000000in}}{%
\pgfpathmoveto{\pgfqpoint{0.000000in}{0.000000in}}%
\pgfpathlineto{\pgfqpoint{0.041667in}{0.000000in}}%
\pgfusepath{stroke,fill}%
}%
\begin{pgfscope}%
\pgfsys@transformshift{3.095027in}{2.181127in}%
\pgfsys@useobject{currentmarker}{}%
\end{pgfscope}%
\end{pgfscope}%
\begin{pgfscope}%
\definecolor{textcolor}{rgb}{0.150000,0.150000,0.150000}%
\pgfsetstrokecolor{textcolor}%
\pgfsetfillcolor{textcolor}%
\pgftext[x=2.997805in,y=2.181127in,right,]{\color{textcolor}\rmfamily\fontsize{10.000000}{12.000000}\selectfont \(\displaystyle 0.14\)}%
\end{pgfscope}%
\begin{pgfscope}%
\pgfsetbuttcap%
\pgfsetroundjoin%
\definecolor{currentfill}{rgb}{0.150000,0.150000,0.150000}%
\pgfsetfillcolor{currentfill}%
\pgfsetlinewidth{1.003750pt}%
\definecolor{currentstroke}{rgb}{0.150000,0.150000,0.150000}%
\pgfsetstrokecolor{currentstroke}%
\pgfsetdash{}{0pt}%
\pgfsys@defobject{currentmarker}{\pgfqpoint{0.000000in}{0.000000in}}{\pgfqpoint{0.041667in}{0.000000in}}{%
\pgfpathmoveto{\pgfqpoint{0.000000in}{0.000000in}}%
\pgfpathlineto{\pgfqpoint{0.041667in}{0.000000in}}%
\pgfusepath{stroke,fill}%
}%
\begin{pgfscope}%
\pgfsys@transformshift{3.095027in}{2.417263in}%
\pgfsys@useobject{currentmarker}{}%
\end{pgfscope}%
\end{pgfscope}%
\begin{pgfscope}%
\definecolor{textcolor}{rgb}{0.150000,0.150000,0.150000}%
\pgfsetstrokecolor{textcolor}%
\pgfsetfillcolor{textcolor}%
\pgftext[x=2.997805in,y=2.417263in,right,]{\color{textcolor}\rmfamily\fontsize{10.000000}{12.000000}\selectfont \(\displaystyle 0.16\)}%
\end{pgfscope}%
\begin{pgfscope}%
\pgfsetbuttcap%
\pgfsetroundjoin%
\definecolor{currentfill}{rgb}{0.150000,0.150000,0.150000}%
\pgfsetfillcolor{currentfill}%
\pgfsetlinewidth{1.003750pt}%
\definecolor{currentstroke}{rgb}{0.150000,0.150000,0.150000}%
\pgfsetstrokecolor{currentstroke}%
\pgfsetdash{}{0pt}%
\pgfsys@defobject{currentmarker}{\pgfqpoint{0.000000in}{0.000000in}}{\pgfqpoint{0.041667in}{0.000000in}}{%
\pgfpathmoveto{\pgfqpoint{0.000000in}{0.000000in}}%
\pgfpathlineto{\pgfqpoint{0.041667in}{0.000000in}}%
\pgfusepath{stroke,fill}%
}%
\begin{pgfscope}%
\pgfsys@transformshift{3.095027in}{2.653399in}%
\pgfsys@useobject{currentmarker}{}%
\end{pgfscope}%
\end{pgfscope}%
\begin{pgfscope}%
\definecolor{textcolor}{rgb}{0.150000,0.150000,0.150000}%
\pgfsetstrokecolor{textcolor}%
\pgfsetfillcolor{textcolor}%
\pgftext[x=2.997805in,y=2.653399in,right,]{\color{textcolor}\rmfamily\fontsize{10.000000}{12.000000}\selectfont \(\displaystyle 0.18\)}%
\end{pgfscope}%
\begin{pgfscope}%
\definecolor{textcolor}{rgb}{0.150000,0.150000,0.150000}%
\pgfsetstrokecolor{textcolor}%
\pgfsetfillcolor{textcolor}%
\pgftext[x=2.681446in,y=1.590788in,,bottom,rotate=90.000000]{\color{textcolor}\rmfamily\fontsize{10.000000}{12.000000}\selectfont \textbf{Cell activity (a.u.)}}%
\end{pgfscope}%
\begin{pgfscope}%
\pgfpathrectangle{\pgfqpoint{3.095027in}{0.528177in}}{\pgfqpoint{2.106702in}{2.125222in}} %
\pgfusepath{clip}%
\pgfsetbuttcap%
\pgfsetmiterjoin%
\definecolor{currentfill}{rgb}{0.200000,0.427451,0.650980}%
\pgfsetfillcolor{currentfill}%
\pgfsetlinewidth{1.505625pt}%
\definecolor{currentstroke}{rgb}{0.200000,0.427451,0.650980}%
\pgfsetstrokecolor{currentstroke}%
\pgfsetdash{}{0pt}%
\pgfpathmoveto{\pgfqpoint{3.170266in}{0.528177in}}%
\pgfpathlineto{\pgfqpoint{3.546463in}{0.528177in}}%
\pgfpathlineto{\pgfqpoint{3.546463in}{1.631369in}}%
\pgfpathlineto{\pgfqpoint{3.170266in}{1.631369in}}%
\pgfpathclose%
\pgfusepath{stroke,fill}%
\end{pgfscope}%
\begin{pgfscope}%
\pgfpathrectangle{\pgfqpoint{3.095027in}{0.528177in}}{\pgfqpoint{2.106702in}{2.125222in}} %
\pgfusepath{clip}%
\pgfsetbuttcap%
\pgfsetmiterjoin%
\definecolor{currentfill}{rgb}{0.168627,0.670588,0.494118}%
\pgfsetfillcolor{currentfill}%
\pgfsetlinewidth{1.505625pt}%
\definecolor{currentstroke}{rgb}{0.168627,0.670588,0.494118}%
\pgfsetstrokecolor{currentstroke}%
\pgfsetdash{}{0pt}%
\pgfpathmoveto{\pgfqpoint{3.696942in}{0.528177in}}%
\pgfpathlineto{\pgfqpoint{4.073138in}{0.528177in}}%
\pgfpathlineto{\pgfqpoint{4.073138in}{1.642665in}}%
\pgfpathlineto{\pgfqpoint{3.696942in}{1.642665in}}%
\pgfpathclose%
\pgfusepath{stroke,fill}%
\end{pgfscope}%
\begin{pgfscope}%
\pgfpathrectangle{\pgfqpoint{3.095027in}{0.528177in}}{\pgfqpoint{2.106702in}{2.125222in}} %
\pgfusepath{clip}%
\pgfsetbuttcap%
\pgfsetmiterjoin%
\definecolor{currentfill}{rgb}{1.000000,0.494118,0.250980}%
\pgfsetfillcolor{currentfill}%
\pgfsetlinewidth{1.505625pt}%
\definecolor{currentstroke}{rgb}{1.000000,0.494118,0.250980}%
\pgfsetstrokecolor{currentstroke}%
\pgfsetdash{}{0pt}%
\pgfpathmoveto{\pgfqpoint{4.223617in}{0.528177in}}%
\pgfpathlineto{\pgfqpoint{4.599814in}{0.528177in}}%
\pgfpathlineto{\pgfqpoint{4.599814in}{1.955281in}}%
\pgfpathlineto{\pgfqpoint{4.223617in}{1.955281in}}%
\pgfpathclose%
\pgfusepath{stroke,fill}%
\end{pgfscope}%
\begin{pgfscope}%
\pgfpathrectangle{\pgfqpoint{3.095027in}{0.528177in}}{\pgfqpoint{2.106702in}{2.125222in}} %
\pgfusepath{clip}%
\pgfsetbuttcap%
\pgfsetmiterjoin%
\definecolor{currentfill}{rgb}{1.000000,0.694118,0.250980}%
\pgfsetfillcolor{currentfill}%
\pgfsetlinewidth{1.505625pt}%
\definecolor{currentstroke}{rgb}{1.000000,0.694118,0.250980}%
\pgfsetstrokecolor{currentstroke}%
\pgfsetdash{}{0pt}%
\pgfpathmoveto{\pgfqpoint{4.750293in}{0.528177in}}%
\pgfpathlineto{\pgfqpoint{5.126489in}{0.528177in}}%
\pgfpathlineto{\pgfqpoint{5.126489in}{1.620824in}}%
\pgfpathlineto{\pgfqpoint{4.750293in}{1.620824in}}%
\pgfpathclose%
\pgfusepath{stroke,fill}%
\end{pgfscope}%
\begin{pgfscope}%
\pgfpathrectangle{\pgfqpoint{3.095027in}{0.528177in}}{\pgfqpoint{2.106702in}{2.125222in}} %
\pgfusepath{clip}%
\pgfsetbuttcap%
\pgfsetroundjoin%
\pgfsetlinewidth{1.505625pt}%
\definecolor{currentstroke}{rgb}{0.200000,0.427451,0.650980}%
\pgfsetstrokecolor{currentstroke}%
\pgfsetdash{}{0pt}%
\pgfpathmoveto{\pgfqpoint{3.358365in}{1.631369in}}%
\pgfpathlineto{\pgfqpoint{3.358365in}{1.670997in}}%
\pgfusepath{stroke}%
\end{pgfscope}%
\begin{pgfscope}%
\pgfpathrectangle{\pgfqpoint{3.095027in}{0.528177in}}{\pgfqpoint{2.106702in}{2.125222in}} %
\pgfusepath{clip}%
\pgfsetbuttcap%
\pgfsetroundjoin%
\pgfsetlinewidth{1.505625pt}%
\definecolor{currentstroke}{rgb}{0.168627,0.670588,0.494118}%
\pgfsetstrokecolor{currentstroke}%
\pgfsetdash{}{0pt}%
\pgfpathmoveto{\pgfqpoint{3.885040in}{1.642665in}}%
\pgfpathlineto{\pgfqpoint{3.885040in}{1.712792in}}%
\pgfusepath{stroke}%
\end{pgfscope}%
\begin{pgfscope}%
\pgfpathrectangle{\pgfqpoint{3.095027in}{0.528177in}}{\pgfqpoint{2.106702in}{2.125222in}} %
\pgfusepath{clip}%
\pgfsetbuttcap%
\pgfsetroundjoin%
\pgfsetlinewidth{1.505625pt}%
\definecolor{currentstroke}{rgb}{1.000000,0.494118,0.250980}%
\pgfsetstrokecolor{currentstroke}%
\pgfsetdash{}{0pt}%
\pgfpathmoveto{\pgfqpoint{4.411716in}{1.955281in}}%
\pgfpathlineto{\pgfqpoint{4.411716in}{2.038826in}}%
\pgfusepath{stroke}%
\end{pgfscope}%
\begin{pgfscope}%
\pgfpathrectangle{\pgfqpoint{3.095027in}{0.528177in}}{\pgfqpoint{2.106702in}{2.125222in}} %
\pgfusepath{clip}%
\pgfsetbuttcap%
\pgfsetroundjoin%
\pgfsetlinewidth{1.505625pt}%
\definecolor{currentstroke}{rgb}{1.000000,0.694118,0.250980}%
\pgfsetstrokecolor{currentstroke}%
\pgfsetdash{}{0pt}%
\pgfpathmoveto{\pgfqpoint{4.938391in}{1.620824in}}%
\pgfpathlineto{\pgfqpoint{4.938391in}{1.684762in}}%
\pgfusepath{stroke}%
\end{pgfscope}%
\begin{pgfscope}%
\pgfpathrectangle{\pgfqpoint{3.095027in}{0.528177in}}{\pgfqpoint{2.106702in}{2.125222in}} %
\pgfusepath{clip}%
\pgfsetbuttcap%
\pgfsetroundjoin%
\definecolor{currentfill}{rgb}{0.200000,0.427451,0.650980}%
\pgfsetfillcolor{currentfill}%
\pgfsetlinewidth{1.505625pt}%
\definecolor{currentstroke}{rgb}{0.200000,0.427451,0.650980}%
\pgfsetstrokecolor{currentstroke}%
\pgfsetdash{}{0pt}%
\pgfsys@defobject{currentmarker}{\pgfqpoint{-0.111111in}{-0.000000in}}{\pgfqpoint{0.111111in}{0.000000in}}{%
\pgfpathmoveto{\pgfqpoint{0.111111in}{-0.000000in}}%
\pgfpathlineto{\pgfqpoint{-0.111111in}{0.000000in}}%
\pgfusepath{stroke,fill}%
}%
\begin{pgfscope}%
\pgfsys@transformshift{3.358365in}{1.631369in}%
\pgfsys@useobject{currentmarker}{}%
\end{pgfscope}%
\end{pgfscope}%
\begin{pgfscope}%
\pgfpathrectangle{\pgfqpoint{3.095027in}{0.528177in}}{\pgfqpoint{2.106702in}{2.125222in}} %
\pgfusepath{clip}%
\pgfsetbuttcap%
\pgfsetroundjoin%
\definecolor{currentfill}{rgb}{0.200000,0.427451,0.650980}%
\pgfsetfillcolor{currentfill}%
\pgfsetlinewidth{1.505625pt}%
\definecolor{currentstroke}{rgb}{0.200000,0.427451,0.650980}%
\pgfsetstrokecolor{currentstroke}%
\pgfsetdash{}{0pt}%
\pgfsys@defobject{currentmarker}{\pgfqpoint{-0.111111in}{-0.000000in}}{\pgfqpoint{0.111111in}{0.000000in}}{%
\pgfpathmoveto{\pgfqpoint{0.111111in}{-0.000000in}}%
\pgfpathlineto{\pgfqpoint{-0.111111in}{0.000000in}}%
\pgfusepath{stroke,fill}%
}%
\begin{pgfscope}%
\pgfsys@transformshift{3.358365in}{1.670997in}%
\pgfsys@useobject{currentmarker}{}%
\end{pgfscope}%
\end{pgfscope}%
\begin{pgfscope}%
\pgfpathrectangle{\pgfqpoint{3.095027in}{0.528177in}}{\pgfqpoint{2.106702in}{2.125222in}} %
\pgfusepath{clip}%
\pgfsetbuttcap%
\pgfsetroundjoin%
\definecolor{currentfill}{rgb}{0.168627,0.670588,0.494118}%
\pgfsetfillcolor{currentfill}%
\pgfsetlinewidth{1.505625pt}%
\definecolor{currentstroke}{rgb}{0.168627,0.670588,0.494118}%
\pgfsetstrokecolor{currentstroke}%
\pgfsetdash{}{0pt}%
\pgfsys@defobject{currentmarker}{\pgfqpoint{-0.111111in}{-0.000000in}}{\pgfqpoint{0.111111in}{0.000000in}}{%
\pgfpathmoveto{\pgfqpoint{0.111111in}{-0.000000in}}%
\pgfpathlineto{\pgfqpoint{-0.111111in}{0.000000in}}%
\pgfusepath{stroke,fill}%
}%
\begin{pgfscope}%
\pgfsys@transformshift{3.885040in}{1.642665in}%
\pgfsys@useobject{currentmarker}{}%
\end{pgfscope}%
\end{pgfscope}%
\begin{pgfscope}%
\pgfpathrectangle{\pgfqpoint{3.095027in}{0.528177in}}{\pgfqpoint{2.106702in}{2.125222in}} %
\pgfusepath{clip}%
\pgfsetbuttcap%
\pgfsetroundjoin%
\definecolor{currentfill}{rgb}{0.168627,0.670588,0.494118}%
\pgfsetfillcolor{currentfill}%
\pgfsetlinewidth{1.505625pt}%
\definecolor{currentstroke}{rgb}{0.168627,0.670588,0.494118}%
\pgfsetstrokecolor{currentstroke}%
\pgfsetdash{}{0pt}%
\pgfsys@defobject{currentmarker}{\pgfqpoint{-0.111111in}{-0.000000in}}{\pgfqpoint{0.111111in}{0.000000in}}{%
\pgfpathmoveto{\pgfqpoint{0.111111in}{-0.000000in}}%
\pgfpathlineto{\pgfqpoint{-0.111111in}{0.000000in}}%
\pgfusepath{stroke,fill}%
}%
\begin{pgfscope}%
\pgfsys@transformshift{3.885040in}{1.712792in}%
\pgfsys@useobject{currentmarker}{}%
\end{pgfscope}%
\end{pgfscope}%
\begin{pgfscope}%
\pgfpathrectangle{\pgfqpoint{3.095027in}{0.528177in}}{\pgfqpoint{2.106702in}{2.125222in}} %
\pgfusepath{clip}%
\pgfsetbuttcap%
\pgfsetroundjoin%
\definecolor{currentfill}{rgb}{1.000000,0.494118,0.250980}%
\pgfsetfillcolor{currentfill}%
\pgfsetlinewidth{1.505625pt}%
\definecolor{currentstroke}{rgb}{1.000000,0.494118,0.250980}%
\pgfsetstrokecolor{currentstroke}%
\pgfsetdash{}{0pt}%
\pgfsys@defobject{currentmarker}{\pgfqpoint{-0.111111in}{-0.000000in}}{\pgfqpoint{0.111111in}{0.000000in}}{%
\pgfpathmoveto{\pgfqpoint{0.111111in}{-0.000000in}}%
\pgfpathlineto{\pgfqpoint{-0.111111in}{0.000000in}}%
\pgfusepath{stroke,fill}%
}%
\begin{pgfscope}%
\pgfsys@transformshift{4.411716in}{1.955281in}%
\pgfsys@useobject{currentmarker}{}%
\end{pgfscope}%
\end{pgfscope}%
\begin{pgfscope}%
\pgfpathrectangle{\pgfqpoint{3.095027in}{0.528177in}}{\pgfqpoint{2.106702in}{2.125222in}} %
\pgfusepath{clip}%
\pgfsetbuttcap%
\pgfsetroundjoin%
\definecolor{currentfill}{rgb}{1.000000,0.494118,0.250980}%
\pgfsetfillcolor{currentfill}%
\pgfsetlinewidth{1.505625pt}%
\definecolor{currentstroke}{rgb}{1.000000,0.494118,0.250980}%
\pgfsetstrokecolor{currentstroke}%
\pgfsetdash{}{0pt}%
\pgfsys@defobject{currentmarker}{\pgfqpoint{-0.111111in}{-0.000000in}}{\pgfqpoint{0.111111in}{0.000000in}}{%
\pgfpathmoveto{\pgfqpoint{0.111111in}{-0.000000in}}%
\pgfpathlineto{\pgfqpoint{-0.111111in}{0.000000in}}%
\pgfusepath{stroke,fill}%
}%
\begin{pgfscope}%
\pgfsys@transformshift{4.411716in}{2.038826in}%
\pgfsys@useobject{currentmarker}{}%
\end{pgfscope}%
\end{pgfscope}%
\begin{pgfscope}%
\pgfpathrectangle{\pgfqpoint{3.095027in}{0.528177in}}{\pgfqpoint{2.106702in}{2.125222in}} %
\pgfusepath{clip}%
\pgfsetbuttcap%
\pgfsetroundjoin%
\definecolor{currentfill}{rgb}{1.000000,0.694118,0.250980}%
\pgfsetfillcolor{currentfill}%
\pgfsetlinewidth{1.505625pt}%
\definecolor{currentstroke}{rgb}{1.000000,0.694118,0.250980}%
\pgfsetstrokecolor{currentstroke}%
\pgfsetdash{}{0pt}%
\pgfsys@defobject{currentmarker}{\pgfqpoint{-0.111111in}{-0.000000in}}{\pgfqpoint{0.111111in}{0.000000in}}{%
\pgfpathmoveto{\pgfqpoint{0.111111in}{-0.000000in}}%
\pgfpathlineto{\pgfqpoint{-0.111111in}{0.000000in}}%
\pgfusepath{stroke,fill}%
}%
\begin{pgfscope}%
\pgfsys@transformshift{4.938391in}{1.620824in}%
\pgfsys@useobject{currentmarker}{}%
\end{pgfscope}%
\end{pgfscope}%
\begin{pgfscope}%
\pgfpathrectangle{\pgfqpoint{3.095027in}{0.528177in}}{\pgfqpoint{2.106702in}{2.125222in}} %
\pgfusepath{clip}%
\pgfsetbuttcap%
\pgfsetroundjoin%
\definecolor{currentfill}{rgb}{1.000000,0.694118,0.250980}%
\pgfsetfillcolor{currentfill}%
\pgfsetlinewidth{1.505625pt}%
\definecolor{currentstroke}{rgb}{1.000000,0.694118,0.250980}%
\pgfsetstrokecolor{currentstroke}%
\pgfsetdash{}{0pt}%
\pgfsys@defobject{currentmarker}{\pgfqpoint{-0.111111in}{-0.000000in}}{\pgfqpoint{0.111111in}{0.000000in}}{%
\pgfpathmoveto{\pgfqpoint{0.111111in}{-0.000000in}}%
\pgfpathlineto{\pgfqpoint{-0.111111in}{0.000000in}}%
\pgfusepath{stroke,fill}%
}%
\begin{pgfscope}%
\pgfsys@transformshift{4.938391in}{1.684762in}%
\pgfsys@useobject{currentmarker}{}%
\end{pgfscope}%
\end{pgfscope}%
\begin{pgfscope}%
\pgfpathrectangle{\pgfqpoint{3.095027in}{0.528177in}}{\pgfqpoint{2.106702in}{2.125222in}} %
\pgfusepath{clip}%
\pgfsetroundcap%
\pgfsetroundjoin%
\pgfsetlinewidth{1.756562pt}%
\definecolor{currentstroke}{rgb}{0.627451,0.627451,0.643137}%
\pgfsetstrokecolor{currentstroke}%
\pgfsetdash{}{0pt}%
\pgfpathmoveto{\pgfqpoint{3.358365in}{1.752015in}}%
\pgfpathlineto{\pgfqpoint{3.358365in}{2.254876in}}%
\pgfusepath{stroke}%
\end{pgfscope}%
\begin{pgfscope}%
\pgfpathrectangle{\pgfqpoint{3.095027in}{0.528177in}}{\pgfqpoint{2.106702in}{2.125222in}} %
\pgfusepath{clip}%
\pgfsetroundcap%
\pgfsetroundjoin%
\pgfsetlinewidth{1.756562pt}%
\definecolor{currentstroke}{rgb}{0.627451,0.627451,0.643137}%
\pgfsetstrokecolor{currentstroke}%
\pgfsetdash{}{0pt}%
\pgfpathmoveto{\pgfqpoint{3.358365in}{2.254876in}}%
\pgfpathlineto{\pgfqpoint{4.411716in}{2.254876in}}%
\pgfusepath{stroke}%
\end{pgfscope}%
\begin{pgfscope}%
\pgfpathrectangle{\pgfqpoint{3.095027in}{0.528177in}}{\pgfqpoint{2.106702in}{2.125222in}} %
\pgfusepath{clip}%
\pgfsetroundcap%
\pgfsetroundjoin%
\pgfsetlinewidth{1.756562pt}%
\definecolor{currentstroke}{rgb}{0.627451,0.627451,0.643137}%
\pgfsetstrokecolor{currentstroke}%
\pgfsetdash{}{0pt}%
\pgfpathmoveto{\pgfqpoint{4.411716in}{2.254876in}}%
\pgfpathlineto{\pgfqpoint{4.411716in}{2.200863in}}%
\pgfusepath{stroke}%
\end{pgfscope}%
\begin{pgfscope}%
\pgfpathrectangle{\pgfqpoint{3.095027in}{0.528177in}}{\pgfqpoint{2.106702in}{2.125222in}} %
\pgfusepath{clip}%
\pgfsetroundcap%
\pgfsetroundjoin%
\pgfsetlinewidth{1.756562pt}%
\definecolor{currentstroke}{rgb}{0.627451,0.627451,0.643137}%
\pgfsetstrokecolor{currentstroke}%
\pgfsetdash{}{0pt}%
\pgfpathmoveto{\pgfqpoint{4.411716in}{2.335894in}}%
\pgfpathlineto{\pgfqpoint{4.411716in}{2.470925in}}%
\pgfusepath{stroke}%
\end{pgfscope}%
\begin{pgfscope}%
\pgfpathrectangle{\pgfqpoint{3.095027in}{0.528177in}}{\pgfqpoint{2.106702in}{2.125222in}} %
\pgfusepath{clip}%
\pgfsetroundcap%
\pgfsetroundjoin%
\pgfsetlinewidth{1.756562pt}%
\definecolor{currentstroke}{rgb}{0.627451,0.627451,0.643137}%
\pgfsetstrokecolor{currentstroke}%
\pgfsetdash{}{0pt}%
\pgfpathmoveto{\pgfqpoint{4.411716in}{2.470925in}}%
\pgfpathlineto{\pgfqpoint{4.938391in}{2.470925in}}%
\pgfusepath{stroke}%
\end{pgfscope}%
\begin{pgfscope}%
\pgfpathrectangle{\pgfqpoint{3.095027in}{0.528177in}}{\pgfqpoint{2.106702in}{2.125222in}} %
\pgfusepath{clip}%
\pgfsetroundcap%
\pgfsetroundjoin%
\pgfsetlinewidth{1.756562pt}%
\definecolor{currentstroke}{rgb}{0.627451,0.627451,0.643137}%
\pgfsetstrokecolor{currentstroke}%
\pgfsetdash{}{0pt}%
\pgfpathmoveto{\pgfqpoint{4.938391in}{2.470925in}}%
\pgfpathlineto{\pgfqpoint{4.938391in}{1.846799in}}%
\pgfusepath{stroke}%
\end{pgfscope}%
\begin{pgfscope}%
\pgfsetrectcap%
\pgfsetmiterjoin%
\pgfsetlinewidth{1.254687pt}%
\definecolor{currentstroke}{rgb}{0.150000,0.150000,0.150000}%
\pgfsetstrokecolor{currentstroke}%
\pgfsetdash{}{0pt}%
\pgfpathmoveto{\pgfqpoint{3.095027in}{0.528177in}}%
\pgfpathlineto{\pgfqpoint{3.095027in}{2.653399in}}%
\pgfusepath{stroke}%
\end{pgfscope}%
\begin{pgfscope}%
\pgfsetrectcap%
\pgfsetmiterjoin%
\pgfsetlinewidth{1.254687pt}%
\definecolor{currentstroke}{rgb}{0.150000,0.150000,0.150000}%
\pgfsetstrokecolor{currentstroke}%
\pgfsetdash{}{0pt}%
\pgfpathmoveto{\pgfqpoint{3.095027in}{0.528177in}}%
\pgfpathlineto{\pgfqpoint{5.201729in}{0.528177in}}%
\pgfusepath{stroke}%
\end{pgfscope}%
\begin{pgfscope}%
\definecolor{textcolor}{rgb}{0.150000,0.150000,0.150000}%
\pgfsetstrokecolor{textcolor}%
\pgfsetfillcolor{textcolor}%
\pgftext[x=4.411716in,y=2.089463in,,]{\color{textcolor}\rmfamily\fontsize{15.000000}{18.000000}\selectfont \textbf{*}}%
\end{pgfscope}%
\begin{pgfscope}%
\definecolor{textcolor}{rgb}{0.150000,0.150000,0.150000}%
\pgfsetstrokecolor{textcolor}%
\pgfsetfillcolor{textcolor}%
\pgftext[x=4.938391in,y=1.735399in,,]{\color{textcolor}\rmfamily\fontsize{15.000000}{18.000000}\selectfont \textbf{*}}%
\end{pgfscope}%
\begin{pgfscope}%
\pgfsetbuttcap%
\pgfsetmiterjoin%
\definecolor{currentfill}{rgb}{0.200000,0.427451,0.650980}%
\pgfsetfillcolor{currentfill}%
\pgfsetlinewidth{1.505625pt}%
\definecolor{currentstroke}{rgb}{0.200000,0.427451,0.650980}%
\pgfsetstrokecolor{currentstroke}%
\pgfsetdash{}{0pt}%
\pgfpathmoveto{\pgfqpoint{3.195027in}{3.281088in}}%
\pgfpathlineto{\pgfqpoint{3.306138in}{3.281088in}}%
\pgfpathlineto{\pgfqpoint{3.306138in}{3.358866in}}%
\pgfpathlineto{\pgfqpoint{3.195027in}{3.358866in}}%
\pgfpathclose%
\pgfusepath{stroke,fill}%
\end{pgfscope}%
\begin{pgfscope}%
\definecolor{textcolor}{rgb}{0.150000,0.150000,0.150000}%
\pgfsetstrokecolor{textcolor}%
\pgfsetfillcolor{textcolor}%
\pgftext[x=3.395027in,y=3.281088in,left,base]{\color{textcolor}\rmfamily\fontsize{8.000000}{9.600000}\selectfont WT + Vehicle (1164)}%
\end{pgfscope}%
\begin{pgfscope}%
\pgfsetbuttcap%
\pgfsetmiterjoin%
\definecolor{currentfill}{rgb}{0.168627,0.670588,0.494118}%
\pgfsetfillcolor{currentfill}%
\pgfsetlinewidth{1.505625pt}%
\definecolor{currentstroke}{rgb}{0.168627,0.670588,0.494118}%
\pgfsetstrokecolor{currentstroke}%
\pgfsetdash{}{0pt}%
\pgfpathmoveto{\pgfqpoint{3.195027in}{3.114449in}}%
\pgfpathlineto{\pgfqpoint{3.306138in}{3.114449in}}%
\pgfpathlineto{\pgfqpoint{3.306138in}{3.192227in}}%
\pgfpathlineto{\pgfqpoint{3.195027in}{3.192227in}}%
\pgfpathclose%
\pgfusepath{stroke,fill}%
\end{pgfscope}%
\begin{pgfscope}%
\definecolor{textcolor}{rgb}{0.150000,0.150000,0.150000}%
\pgfsetstrokecolor{textcolor}%
\pgfsetfillcolor{textcolor}%
\pgftext[x=3.395027in,y=3.114449in,left,base]{\color{textcolor}\rmfamily\fontsize{8.000000}{9.600000}\selectfont WT + TAT-GluA2\textsubscript{3Y} (592)}%
\end{pgfscope}%
\begin{pgfscope}%
\pgfsetbuttcap%
\pgfsetmiterjoin%
\definecolor{currentfill}{rgb}{1.000000,0.494118,0.250980}%
\pgfsetfillcolor{currentfill}%
\pgfsetlinewidth{1.505625pt}%
\definecolor{currentstroke}{rgb}{1.000000,0.494118,0.250980}%
\pgfsetstrokecolor{currentstroke}%
\pgfsetdash{}{0pt}%
\pgfpathmoveto{\pgfqpoint{3.195027in}{2.947809in}}%
\pgfpathlineto{\pgfqpoint{3.306138in}{2.947809in}}%
\pgfpathlineto{\pgfqpoint{3.306138in}{3.025587in}}%
\pgfpathlineto{\pgfqpoint{3.195027in}{3.025587in}}%
\pgfpathclose%
\pgfusepath{stroke,fill}%
\end{pgfscope}%
\begin{pgfscope}%
\definecolor{textcolor}{rgb}{0.150000,0.150000,0.150000}%
\pgfsetstrokecolor{textcolor}%
\pgfsetfillcolor{textcolor}%
\pgftext[x=3.395027in,y=2.947809in,left,base]{\color{textcolor}\rmfamily\fontsize{8.000000}{9.600000}\selectfont Tg + Vehicle (504)}%
\end{pgfscope}%
\begin{pgfscope}%
\pgfsetbuttcap%
\pgfsetmiterjoin%
\definecolor{currentfill}{rgb}{1.000000,0.694118,0.250980}%
\pgfsetfillcolor{currentfill}%
\pgfsetlinewidth{1.505625pt}%
\definecolor{currentstroke}{rgb}{1.000000,0.694118,0.250980}%
\pgfsetstrokecolor{currentstroke}%
\pgfsetdash{}{0pt}%
\pgfpathmoveto{\pgfqpoint{3.195027in}{2.781170in}}%
\pgfpathlineto{\pgfqpoint{3.306138in}{2.781170in}}%
\pgfpathlineto{\pgfqpoint{3.306138in}{2.858947in}}%
\pgfpathlineto{\pgfqpoint{3.195027in}{2.858947in}}%
\pgfpathclose%
\pgfusepath{stroke,fill}%
\end{pgfscope}%
\begin{pgfscope}%
\definecolor{textcolor}{rgb}{0.150000,0.150000,0.150000}%
\pgfsetstrokecolor{textcolor}%
\pgfsetfillcolor{textcolor}%
\pgftext[x=3.395027in,y=2.781170in,left,base]{\color{textcolor}\rmfamily\fontsize{8.000000}{9.600000}\selectfont Tg + TAT-GluA2\textsubscript{3Y} (777)}%
\end{pgfscope}%
\end{pgfpicture}%
\makeatother%
\endgroup%

        \caption{\label{f.ad.acttrain}}
    \end{subfigure}
    \begin{subfigure}[h]{0.9\textwidth}
        %% Creator: Matplotlib, PGF backend
%%
%% To include the figure in your LaTeX document, write
%%   \input{<filename>.pgf}
%%
%% Make sure the required packages are loaded in your preamble
%%   \usepackage{pgf}
%%
%% Figures using additional raster images can only be included by \input if
%% they are in the same directory as the main LaTeX file. For loading figures
%% from other directories you can use the `import` package
%%   \usepackage{import}
%% and then include the figures with
%%   \import{<path to file>}{<filename>.pgf}
%%
%% Matplotlib used the following preamble
%%   \usepackage[utf8]{inputenc}
%%   \usepackage[T1]{fontenc}
%%   \usepackage{siunitx}
%%
\begingroup%
\makeatletter%
\begin{pgfpicture}%
\pgfpathrectangle{\pgfpointorigin}{\pgfqpoint{5.301729in}{3.553934in}}%
\pgfusepath{use as bounding box, clip}%
\begin{pgfscope}%
\pgfsetbuttcap%
\pgfsetmiterjoin%
\definecolor{currentfill}{rgb}{1.000000,1.000000,1.000000}%
\pgfsetfillcolor{currentfill}%
\pgfsetlinewidth{0.000000pt}%
\definecolor{currentstroke}{rgb}{1.000000,1.000000,1.000000}%
\pgfsetstrokecolor{currentstroke}%
\pgfsetdash{}{0pt}%
\pgfpathmoveto{\pgfqpoint{0.000000in}{0.000000in}}%
\pgfpathlineto{\pgfqpoint{5.301729in}{0.000000in}}%
\pgfpathlineto{\pgfqpoint{5.301729in}{3.553934in}}%
\pgfpathlineto{\pgfqpoint{0.000000in}{3.553934in}}%
\pgfpathclose%
\pgfusepath{fill}%
\end{pgfscope}%
\begin{pgfscope}%
\pgfsetbuttcap%
\pgfsetmiterjoin%
\definecolor{currentfill}{rgb}{1.000000,1.000000,1.000000}%
\pgfsetfillcolor{currentfill}%
\pgfsetlinewidth{0.000000pt}%
\definecolor{currentstroke}{rgb}{0.000000,0.000000,0.000000}%
\pgfsetstrokecolor{currentstroke}%
\pgfsetstrokeopacity{0.000000}%
\pgfsetdash{}{0pt}%
\pgfpathmoveto{\pgfqpoint{0.566985in}{0.528177in}}%
\pgfpathlineto{\pgfqpoint{2.673686in}{0.528177in}}%
\pgfpathlineto{\pgfqpoint{2.673686in}{3.392606in}}%
\pgfpathlineto{\pgfqpoint{0.566985in}{3.392606in}}%
\pgfpathclose%
\pgfusepath{fill}%
\end{pgfscope}%
\begin{pgfscope}%
\pgfsetbuttcap%
\pgfsetroundjoin%
\definecolor{currentfill}{rgb}{0.150000,0.150000,0.150000}%
\pgfsetfillcolor{currentfill}%
\pgfsetlinewidth{1.003750pt}%
\definecolor{currentstroke}{rgb}{0.150000,0.150000,0.150000}%
\pgfsetstrokecolor{currentstroke}%
\pgfsetdash{}{0pt}%
\pgfsys@defobject{currentmarker}{\pgfqpoint{0.000000in}{0.000000in}}{\pgfqpoint{0.000000in}{0.041667in}}{%
\pgfpathmoveto{\pgfqpoint{0.000000in}{0.000000in}}%
\pgfpathlineto{\pgfqpoint{0.000000in}{0.041667in}}%
\pgfusepath{stroke,fill}%
}%
\begin{pgfscope}%
\pgfsys@transformshift{0.566985in}{0.528177in}%
\pgfsys@useobject{currentmarker}{}%
\end{pgfscope}%
\end{pgfscope}%
\begin{pgfscope}%
\definecolor{textcolor}{rgb}{0.150000,0.150000,0.150000}%
\pgfsetstrokecolor{textcolor}%
\pgfsetfillcolor{textcolor}%
\pgftext[x=0.566985in,y=0.430955in,,top]{\color{textcolor}\rmfamily\fontsize{10.000000}{12.000000}\selectfont \(\displaystyle 0.0\)}%
\end{pgfscope}%
\begin{pgfscope}%
\pgfsetbuttcap%
\pgfsetroundjoin%
\definecolor{currentfill}{rgb}{0.150000,0.150000,0.150000}%
\pgfsetfillcolor{currentfill}%
\pgfsetlinewidth{1.003750pt}%
\definecolor{currentstroke}{rgb}{0.150000,0.150000,0.150000}%
\pgfsetstrokecolor{currentstroke}%
\pgfsetdash{}{0pt}%
\pgfsys@defobject{currentmarker}{\pgfqpoint{0.000000in}{0.000000in}}{\pgfqpoint{0.000000in}{0.041667in}}{%
\pgfpathmoveto{\pgfqpoint{0.000000in}{0.000000in}}%
\pgfpathlineto{\pgfqpoint{0.000000in}{0.041667in}}%
\pgfusepath{stroke,fill}%
}%
\begin{pgfscope}%
\pgfsys@transformshift{0.918102in}{0.528177in}%
\pgfsys@useobject{currentmarker}{}%
\end{pgfscope}%
\end{pgfscope}%
\begin{pgfscope}%
\definecolor{textcolor}{rgb}{0.150000,0.150000,0.150000}%
\pgfsetstrokecolor{textcolor}%
\pgfsetfillcolor{textcolor}%
\pgftext[x=0.918102in,y=0.430955in,,top]{\color{textcolor}\rmfamily\fontsize{10.000000}{12.000000}\selectfont \(\displaystyle 0.2\)}%
\end{pgfscope}%
\begin{pgfscope}%
\pgfsetbuttcap%
\pgfsetroundjoin%
\definecolor{currentfill}{rgb}{0.150000,0.150000,0.150000}%
\pgfsetfillcolor{currentfill}%
\pgfsetlinewidth{1.003750pt}%
\definecolor{currentstroke}{rgb}{0.150000,0.150000,0.150000}%
\pgfsetstrokecolor{currentstroke}%
\pgfsetdash{}{0pt}%
\pgfsys@defobject{currentmarker}{\pgfqpoint{0.000000in}{0.000000in}}{\pgfqpoint{0.000000in}{0.041667in}}{%
\pgfpathmoveto{\pgfqpoint{0.000000in}{0.000000in}}%
\pgfpathlineto{\pgfqpoint{0.000000in}{0.041667in}}%
\pgfusepath{stroke,fill}%
}%
\begin{pgfscope}%
\pgfsys@transformshift{1.269219in}{0.528177in}%
\pgfsys@useobject{currentmarker}{}%
\end{pgfscope}%
\end{pgfscope}%
\begin{pgfscope}%
\definecolor{textcolor}{rgb}{0.150000,0.150000,0.150000}%
\pgfsetstrokecolor{textcolor}%
\pgfsetfillcolor{textcolor}%
\pgftext[x=1.269219in,y=0.430955in,,top]{\color{textcolor}\rmfamily\fontsize{10.000000}{12.000000}\selectfont \(\displaystyle 0.4\)}%
\end{pgfscope}%
\begin{pgfscope}%
\pgfsetbuttcap%
\pgfsetroundjoin%
\definecolor{currentfill}{rgb}{0.150000,0.150000,0.150000}%
\pgfsetfillcolor{currentfill}%
\pgfsetlinewidth{1.003750pt}%
\definecolor{currentstroke}{rgb}{0.150000,0.150000,0.150000}%
\pgfsetstrokecolor{currentstroke}%
\pgfsetdash{}{0pt}%
\pgfsys@defobject{currentmarker}{\pgfqpoint{0.000000in}{0.000000in}}{\pgfqpoint{0.000000in}{0.041667in}}{%
\pgfpathmoveto{\pgfqpoint{0.000000in}{0.000000in}}%
\pgfpathlineto{\pgfqpoint{0.000000in}{0.041667in}}%
\pgfusepath{stroke,fill}%
}%
\begin{pgfscope}%
\pgfsys@transformshift{1.620336in}{0.528177in}%
\pgfsys@useobject{currentmarker}{}%
\end{pgfscope}%
\end{pgfscope}%
\begin{pgfscope}%
\definecolor{textcolor}{rgb}{0.150000,0.150000,0.150000}%
\pgfsetstrokecolor{textcolor}%
\pgfsetfillcolor{textcolor}%
\pgftext[x=1.620336in,y=0.430955in,,top]{\color{textcolor}\rmfamily\fontsize{10.000000}{12.000000}\selectfont \(\displaystyle 0.6\)}%
\end{pgfscope}%
\begin{pgfscope}%
\pgfsetbuttcap%
\pgfsetroundjoin%
\definecolor{currentfill}{rgb}{0.150000,0.150000,0.150000}%
\pgfsetfillcolor{currentfill}%
\pgfsetlinewidth{1.003750pt}%
\definecolor{currentstroke}{rgb}{0.150000,0.150000,0.150000}%
\pgfsetstrokecolor{currentstroke}%
\pgfsetdash{}{0pt}%
\pgfsys@defobject{currentmarker}{\pgfqpoint{0.000000in}{0.000000in}}{\pgfqpoint{0.000000in}{0.041667in}}{%
\pgfpathmoveto{\pgfqpoint{0.000000in}{0.000000in}}%
\pgfpathlineto{\pgfqpoint{0.000000in}{0.041667in}}%
\pgfusepath{stroke,fill}%
}%
\begin{pgfscope}%
\pgfsys@transformshift{1.971453in}{0.528177in}%
\pgfsys@useobject{currentmarker}{}%
\end{pgfscope}%
\end{pgfscope}%
\begin{pgfscope}%
\definecolor{textcolor}{rgb}{0.150000,0.150000,0.150000}%
\pgfsetstrokecolor{textcolor}%
\pgfsetfillcolor{textcolor}%
\pgftext[x=1.971453in,y=0.430955in,,top]{\color{textcolor}\rmfamily\fontsize{10.000000}{12.000000}\selectfont \(\displaystyle 0.8\)}%
\end{pgfscope}%
\begin{pgfscope}%
\pgfsetbuttcap%
\pgfsetroundjoin%
\definecolor{currentfill}{rgb}{0.150000,0.150000,0.150000}%
\pgfsetfillcolor{currentfill}%
\pgfsetlinewidth{1.003750pt}%
\definecolor{currentstroke}{rgb}{0.150000,0.150000,0.150000}%
\pgfsetstrokecolor{currentstroke}%
\pgfsetdash{}{0pt}%
\pgfsys@defobject{currentmarker}{\pgfqpoint{0.000000in}{0.000000in}}{\pgfqpoint{0.000000in}{0.041667in}}{%
\pgfpathmoveto{\pgfqpoint{0.000000in}{0.000000in}}%
\pgfpathlineto{\pgfqpoint{0.000000in}{0.041667in}}%
\pgfusepath{stroke,fill}%
}%
\begin{pgfscope}%
\pgfsys@transformshift{2.322569in}{0.528177in}%
\pgfsys@useobject{currentmarker}{}%
\end{pgfscope}%
\end{pgfscope}%
\begin{pgfscope}%
\definecolor{textcolor}{rgb}{0.150000,0.150000,0.150000}%
\pgfsetstrokecolor{textcolor}%
\pgfsetfillcolor{textcolor}%
\pgftext[x=2.322569in,y=0.430955in,,top]{\color{textcolor}\rmfamily\fontsize{10.000000}{12.000000}\selectfont \(\displaystyle 1.0\)}%
\end{pgfscope}%
\begin{pgfscope}%
\pgfsetbuttcap%
\pgfsetroundjoin%
\definecolor{currentfill}{rgb}{0.150000,0.150000,0.150000}%
\pgfsetfillcolor{currentfill}%
\pgfsetlinewidth{1.003750pt}%
\definecolor{currentstroke}{rgb}{0.150000,0.150000,0.150000}%
\pgfsetstrokecolor{currentstroke}%
\pgfsetdash{}{0pt}%
\pgfsys@defobject{currentmarker}{\pgfqpoint{0.000000in}{0.000000in}}{\pgfqpoint{0.000000in}{0.041667in}}{%
\pgfpathmoveto{\pgfqpoint{0.000000in}{0.000000in}}%
\pgfpathlineto{\pgfqpoint{0.000000in}{0.041667in}}%
\pgfusepath{stroke,fill}%
}%
\begin{pgfscope}%
\pgfsys@transformshift{2.673686in}{0.528177in}%
\pgfsys@useobject{currentmarker}{}%
\end{pgfscope}%
\end{pgfscope}%
\begin{pgfscope}%
\definecolor{textcolor}{rgb}{0.150000,0.150000,0.150000}%
\pgfsetstrokecolor{textcolor}%
\pgfsetfillcolor{textcolor}%
\pgftext[x=2.673686in,y=0.430955in,,top]{\color{textcolor}\rmfamily\fontsize{10.000000}{12.000000}\selectfont \(\displaystyle 1.2\)}%
\end{pgfscope}%
\begin{pgfscope}%
\definecolor{textcolor}{rgb}{0.150000,0.150000,0.150000}%
\pgfsetstrokecolor{textcolor}%
\pgfsetfillcolor{textcolor}%
\pgftext[x=1.620336in,y=0.238855in,,top]{\color{textcolor}\rmfamily\fontsize{10.000000}{12.000000}\selectfont \textbf{Cell activity (a.u.)}}%
\end{pgfscope}%
\begin{pgfscope}%
\pgfsetbuttcap%
\pgfsetroundjoin%
\definecolor{currentfill}{rgb}{0.150000,0.150000,0.150000}%
\pgfsetfillcolor{currentfill}%
\pgfsetlinewidth{1.003750pt}%
\definecolor{currentstroke}{rgb}{0.150000,0.150000,0.150000}%
\pgfsetstrokecolor{currentstroke}%
\pgfsetdash{}{0pt}%
\pgfsys@defobject{currentmarker}{\pgfqpoint{0.000000in}{0.000000in}}{\pgfqpoint{0.041667in}{0.000000in}}{%
\pgfpathmoveto{\pgfqpoint{0.000000in}{0.000000in}}%
\pgfpathlineto{\pgfqpoint{0.041667in}{0.000000in}}%
\pgfusepath{stroke,fill}%
}%
\begin{pgfscope}%
\pgfsys@transformshift{0.566985in}{0.528177in}%
\pgfsys@useobject{currentmarker}{}%
\end{pgfscope}%
\end{pgfscope}%
\begin{pgfscope}%
\definecolor{textcolor}{rgb}{0.150000,0.150000,0.150000}%
\pgfsetstrokecolor{textcolor}%
\pgfsetfillcolor{textcolor}%
\pgftext[x=0.469762in,y=0.528177in,right,]{\color{textcolor}\rmfamily\fontsize{10.000000}{12.000000}\selectfont \(\displaystyle 0.1\)}%
\end{pgfscope}%
\begin{pgfscope}%
\pgfsetbuttcap%
\pgfsetroundjoin%
\definecolor{currentfill}{rgb}{0.150000,0.150000,0.150000}%
\pgfsetfillcolor{currentfill}%
\pgfsetlinewidth{1.003750pt}%
\definecolor{currentstroke}{rgb}{0.150000,0.150000,0.150000}%
\pgfsetstrokecolor{currentstroke}%
\pgfsetdash{}{0pt}%
\pgfsys@defobject{currentmarker}{\pgfqpoint{0.000000in}{0.000000in}}{\pgfqpoint{0.041667in}{0.000000in}}{%
\pgfpathmoveto{\pgfqpoint{0.000000in}{0.000000in}}%
\pgfpathlineto{\pgfqpoint{0.041667in}{0.000000in}}%
\pgfusepath{stroke,fill}%
}%
\begin{pgfscope}%
\pgfsys@transformshift{0.566985in}{0.846447in}%
\pgfsys@useobject{currentmarker}{}%
\end{pgfscope}%
\end{pgfscope}%
\begin{pgfscope}%
\definecolor{textcolor}{rgb}{0.150000,0.150000,0.150000}%
\pgfsetstrokecolor{textcolor}%
\pgfsetfillcolor{textcolor}%
\pgftext[x=0.469762in,y=0.846447in,right,]{\color{textcolor}\rmfamily\fontsize{10.000000}{12.000000}\selectfont \(\displaystyle 0.2\)}%
\end{pgfscope}%
\begin{pgfscope}%
\pgfsetbuttcap%
\pgfsetroundjoin%
\definecolor{currentfill}{rgb}{0.150000,0.150000,0.150000}%
\pgfsetfillcolor{currentfill}%
\pgfsetlinewidth{1.003750pt}%
\definecolor{currentstroke}{rgb}{0.150000,0.150000,0.150000}%
\pgfsetstrokecolor{currentstroke}%
\pgfsetdash{}{0pt}%
\pgfsys@defobject{currentmarker}{\pgfqpoint{0.000000in}{0.000000in}}{\pgfqpoint{0.041667in}{0.000000in}}{%
\pgfpathmoveto{\pgfqpoint{0.000000in}{0.000000in}}%
\pgfpathlineto{\pgfqpoint{0.041667in}{0.000000in}}%
\pgfusepath{stroke,fill}%
}%
\begin{pgfscope}%
\pgfsys@transformshift{0.566985in}{1.164717in}%
\pgfsys@useobject{currentmarker}{}%
\end{pgfscope}%
\end{pgfscope}%
\begin{pgfscope}%
\definecolor{textcolor}{rgb}{0.150000,0.150000,0.150000}%
\pgfsetstrokecolor{textcolor}%
\pgfsetfillcolor{textcolor}%
\pgftext[x=0.469762in,y=1.164717in,right,]{\color{textcolor}\rmfamily\fontsize{10.000000}{12.000000}\selectfont \(\displaystyle 0.3\)}%
\end{pgfscope}%
\begin{pgfscope}%
\pgfsetbuttcap%
\pgfsetroundjoin%
\definecolor{currentfill}{rgb}{0.150000,0.150000,0.150000}%
\pgfsetfillcolor{currentfill}%
\pgfsetlinewidth{1.003750pt}%
\definecolor{currentstroke}{rgb}{0.150000,0.150000,0.150000}%
\pgfsetstrokecolor{currentstroke}%
\pgfsetdash{}{0pt}%
\pgfsys@defobject{currentmarker}{\pgfqpoint{0.000000in}{0.000000in}}{\pgfqpoint{0.041667in}{0.000000in}}{%
\pgfpathmoveto{\pgfqpoint{0.000000in}{0.000000in}}%
\pgfpathlineto{\pgfqpoint{0.041667in}{0.000000in}}%
\pgfusepath{stroke,fill}%
}%
\begin{pgfscope}%
\pgfsys@transformshift{0.566985in}{1.482987in}%
\pgfsys@useobject{currentmarker}{}%
\end{pgfscope}%
\end{pgfscope}%
\begin{pgfscope}%
\definecolor{textcolor}{rgb}{0.150000,0.150000,0.150000}%
\pgfsetstrokecolor{textcolor}%
\pgfsetfillcolor{textcolor}%
\pgftext[x=0.469762in,y=1.482987in,right,]{\color{textcolor}\rmfamily\fontsize{10.000000}{12.000000}\selectfont \(\displaystyle 0.4\)}%
\end{pgfscope}%
\begin{pgfscope}%
\pgfsetbuttcap%
\pgfsetroundjoin%
\definecolor{currentfill}{rgb}{0.150000,0.150000,0.150000}%
\pgfsetfillcolor{currentfill}%
\pgfsetlinewidth{1.003750pt}%
\definecolor{currentstroke}{rgb}{0.150000,0.150000,0.150000}%
\pgfsetstrokecolor{currentstroke}%
\pgfsetdash{}{0pt}%
\pgfsys@defobject{currentmarker}{\pgfqpoint{0.000000in}{0.000000in}}{\pgfqpoint{0.041667in}{0.000000in}}{%
\pgfpathmoveto{\pgfqpoint{0.000000in}{0.000000in}}%
\pgfpathlineto{\pgfqpoint{0.041667in}{0.000000in}}%
\pgfusepath{stroke,fill}%
}%
\begin{pgfscope}%
\pgfsys@transformshift{0.566985in}{1.801257in}%
\pgfsys@useobject{currentmarker}{}%
\end{pgfscope}%
\end{pgfscope}%
\begin{pgfscope}%
\definecolor{textcolor}{rgb}{0.150000,0.150000,0.150000}%
\pgfsetstrokecolor{textcolor}%
\pgfsetfillcolor{textcolor}%
\pgftext[x=0.469762in,y=1.801257in,right,]{\color{textcolor}\rmfamily\fontsize{10.000000}{12.000000}\selectfont \(\displaystyle 0.5\)}%
\end{pgfscope}%
\begin{pgfscope}%
\pgfsetbuttcap%
\pgfsetroundjoin%
\definecolor{currentfill}{rgb}{0.150000,0.150000,0.150000}%
\pgfsetfillcolor{currentfill}%
\pgfsetlinewidth{1.003750pt}%
\definecolor{currentstroke}{rgb}{0.150000,0.150000,0.150000}%
\pgfsetstrokecolor{currentstroke}%
\pgfsetdash{}{0pt}%
\pgfsys@defobject{currentmarker}{\pgfqpoint{0.000000in}{0.000000in}}{\pgfqpoint{0.041667in}{0.000000in}}{%
\pgfpathmoveto{\pgfqpoint{0.000000in}{0.000000in}}%
\pgfpathlineto{\pgfqpoint{0.041667in}{0.000000in}}%
\pgfusepath{stroke,fill}%
}%
\begin{pgfscope}%
\pgfsys@transformshift{0.566985in}{2.119526in}%
\pgfsys@useobject{currentmarker}{}%
\end{pgfscope}%
\end{pgfscope}%
\begin{pgfscope}%
\definecolor{textcolor}{rgb}{0.150000,0.150000,0.150000}%
\pgfsetstrokecolor{textcolor}%
\pgfsetfillcolor{textcolor}%
\pgftext[x=0.469762in,y=2.119526in,right,]{\color{textcolor}\rmfamily\fontsize{10.000000}{12.000000}\selectfont \(\displaystyle 0.6\)}%
\end{pgfscope}%
\begin{pgfscope}%
\pgfsetbuttcap%
\pgfsetroundjoin%
\definecolor{currentfill}{rgb}{0.150000,0.150000,0.150000}%
\pgfsetfillcolor{currentfill}%
\pgfsetlinewidth{1.003750pt}%
\definecolor{currentstroke}{rgb}{0.150000,0.150000,0.150000}%
\pgfsetstrokecolor{currentstroke}%
\pgfsetdash{}{0pt}%
\pgfsys@defobject{currentmarker}{\pgfqpoint{0.000000in}{0.000000in}}{\pgfqpoint{0.041667in}{0.000000in}}{%
\pgfpathmoveto{\pgfqpoint{0.000000in}{0.000000in}}%
\pgfpathlineto{\pgfqpoint{0.041667in}{0.000000in}}%
\pgfusepath{stroke,fill}%
}%
\begin{pgfscope}%
\pgfsys@transformshift{0.566985in}{2.437796in}%
\pgfsys@useobject{currentmarker}{}%
\end{pgfscope}%
\end{pgfscope}%
\begin{pgfscope}%
\definecolor{textcolor}{rgb}{0.150000,0.150000,0.150000}%
\pgfsetstrokecolor{textcolor}%
\pgfsetfillcolor{textcolor}%
\pgftext[x=0.469762in,y=2.437796in,right,]{\color{textcolor}\rmfamily\fontsize{10.000000}{12.000000}\selectfont \(\displaystyle 0.7\)}%
\end{pgfscope}%
\begin{pgfscope}%
\pgfsetbuttcap%
\pgfsetroundjoin%
\definecolor{currentfill}{rgb}{0.150000,0.150000,0.150000}%
\pgfsetfillcolor{currentfill}%
\pgfsetlinewidth{1.003750pt}%
\definecolor{currentstroke}{rgb}{0.150000,0.150000,0.150000}%
\pgfsetstrokecolor{currentstroke}%
\pgfsetdash{}{0pt}%
\pgfsys@defobject{currentmarker}{\pgfqpoint{0.000000in}{0.000000in}}{\pgfqpoint{0.041667in}{0.000000in}}{%
\pgfpathmoveto{\pgfqpoint{0.000000in}{0.000000in}}%
\pgfpathlineto{\pgfqpoint{0.041667in}{0.000000in}}%
\pgfusepath{stroke,fill}%
}%
\begin{pgfscope}%
\pgfsys@transformshift{0.566985in}{2.756066in}%
\pgfsys@useobject{currentmarker}{}%
\end{pgfscope}%
\end{pgfscope}%
\begin{pgfscope}%
\definecolor{textcolor}{rgb}{0.150000,0.150000,0.150000}%
\pgfsetstrokecolor{textcolor}%
\pgfsetfillcolor{textcolor}%
\pgftext[x=0.469762in,y=2.756066in,right,]{\color{textcolor}\rmfamily\fontsize{10.000000}{12.000000}\selectfont \(\displaystyle 0.8\)}%
\end{pgfscope}%
\begin{pgfscope}%
\pgfsetbuttcap%
\pgfsetroundjoin%
\definecolor{currentfill}{rgb}{0.150000,0.150000,0.150000}%
\pgfsetfillcolor{currentfill}%
\pgfsetlinewidth{1.003750pt}%
\definecolor{currentstroke}{rgb}{0.150000,0.150000,0.150000}%
\pgfsetstrokecolor{currentstroke}%
\pgfsetdash{}{0pt}%
\pgfsys@defobject{currentmarker}{\pgfqpoint{0.000000in}{0.000000in}}{\pgfqpoint{0.041667in}{0.000000in}}{%
\pgfpathmoveto{\pgfqpoint{0.000000in}{0.000000in}}%
\pgfpathlineto{\pgfqpoint{0.041667in}{0.000000in}}%
\pgfusepath{stroke,fill}%
}%
\begin{pgfscope}%
\pgfsys@transformshift{0.566985in}{3.074336in}%
\pgfsys@useobject{currentmarker}{}%
\end{pgfscope}%
\end{pgfscope}%
\begin{pgfscope}%
\definecolor{textcolor}{rgb}{0.150000,0.150000,0.150000}%
\pgfsetstrokecolor{textcolor}%
\pgfsetfillcolor{textcolor}%
\pgftext[x=0.469762in,y=3.074336in,right,]{\color{textcolor}\rmfamily\fontsize{10.000000}{12.000000}\selectfont \(\displaystyle 0.9\)}%
\end{pgfscope}%
\begin{pgfscope}%
\pgfsetbuttcap%
\pgfsetroundjoin%
\definecolor{currentfill}{rgb}{0.150000,0.150000,0.150000}%
\pgfsetfillcolor{currentfill}%
\pgfsetlinewidth{1.003750pt}%
\definecolor{currentstroke}{rgb}{0.150000,0.150000,0.150000}%
\pgfsetstrokecolor{currentstroke}%
\pgfsetdash{}{0pt}%
\pgfsys@defobject{currentmarker}{\pgfqpoint{0.000000in}{0.000000in}}{\pgfqpoint{0.041667in}{0.000000in}}{%
\pgfpathmoveto{\pgfqpoint{0.000000in}{0.000000in}}%
\pgfpathlineto{\pgfqpoint{0.041667in}{0.000000in}}%
\pgfusepath{stroke,fill}%
}%
\begin{pgfscope}%
\pgfsys@transformshift{0.566985in}{3.392606in}%
\pgfsys@useobject{currentmarker}{}%
\end{pgfscope}%
\end{pgfscope}%
\begin{pgfscope}%
\definecolor{textcolor}{rgb}{0.150000,0.150000,0.150000}%
\pgfsetstrokecolor{textcolor}%
\pgfsetfillcolor{textcolor}%
\pgftext[x=0.469762in,y=3.392606in,right,]{\color{textcolor}\rmfamily\fontsize{10.000000}{12.000000}\selectfont \(\displaystyle 1.0\)}%
\end{pgfscope}%
\begin{pgfscope}%
\definecolor{textcolor}{rgb}{0.150000,0.150000,0.150000}%
\pgfsetstrokecolor{textcolor}%
\pgfsetfillcolor{textcolor}%
\pgftext[x=0.222848in,y=1.960392in,,bottom,rotate=90.000000]{\color{textcolor}\rmfamily\fontsize{10.000000}{12.000000}\selectfont \textbf{Cumulative porportion}}%
\end{pgfscope}%
\begin{pgfscope}%
\pgfpathrectangle{\pgfqpoint{0.566985in}{0.528177in}}{\pgfqpoint{2.106702in}{2.864429in}} %
\pgfusepath{clip}%
\pgfsetroundcap%
\pgfsetroundjoin%
\pgfsetlinewidth{1.003750pt}%
\definecolor{currentstroke}{rgb}{0.200000,0.427451,0.650980}%
\pgfsetstrokecolor{currentstroke}%
\pgfsetdash{}{0pt}%
\pgfpathmoveto{\pgfqpoint{0.567200in}{0.596116in}}%
\pgfpathlineto{\pgfqpoint{0.605129in}{0.911849in}}%
\pgfpathlineto{\pgfqpoint{0.643057in}{1.247315in}}%
\pgfpathlineto{\pgfqpoint{0.680986in}{1.596876in}}%
\pgfpathlineto{\pgfqpoint{0.718914in}{1.946437in}}%
\pgfpathlineto{\pgfqpoint{0.756843in}{2.135313in}}%
\pgfpathlineto{\pgfqpoint{0.794771in}{2.332646in}}%
\pgfpathlineto{\pgfqpoint{0.832699in}{2.538436in}}%
\pgfpathlineto{\pgfqpoint{0.870628in}{2.718855in}}%
\pgfpathlineto{\pgfqpoint{0.908556in}{2.851350in}}%
\pgfpathlineto{\pgfqpoint{0.946485in}{2.964112in}}%
\pgfpathlineto{\pgfqpoint{0.984413in}{3.034588in}}%
\pgfpathlineto{\pgfqpoint{1.022342in}{3.113521in}}%
\pgfpathlineto{\pgfqpoint{1.060270in}{3.161445in}}%
\pgfpathlineto{\pgfqpoint{1.098199in}{3.223464in}}%
\pgfpathlineto{\pgfqpoint{1.136127in}{3.254473in}}%
\pgfpathlineto{\pgfqpoint{1.174056in}{3.274206in}}%
\pgfpathlineto{\pgfqpoint{1.211984in}{3.313673in}}%
\pgfpathlineto{\pgfqpoint{1.249913in}{3.333406in}}%
\pgfpathlineto{\pgfqpoint{1.287841in}{3.344683in}}%
\pgfpathlineto{\pgfqpoint{1.325770in}{3.350321in}}%
\pgfpathlineto{\pgfqpoint{1.363698in}{3.358778in}}%
\pgfpathlineto{\pgfqpoint{1.401627in}{3.367235in}}%
\pgfpathlineto{\pgfqpoint{1.439555in}{3.370054in}}%
\pgfpathlineto{\pgfqpoint{1.477483in}{3.370054in}}%
\pgfpathlineto{\pgfqpoint{1.515412in}{3.372873in}}%
\pgfpathlineto{\pgfqpoint{1.553340in}{3.372873in}}%
\pgfpathlineto{\pgfqpoint{1.591269in}{3.378511in}}%
\pgfpathlineto{\pgfqpoint{1.629197in}{3.378511in}}%
\pgfpathlineto{\pgfqpoint{1.667126in}{3.381330in}}%
\pgfpathlineto{\pgfqpoint{1.705054in}{3.384149in}}%
\pgfpathlineto{\pgfqpoint{1.742983in}{3.384149in}}%
\pgfpathlineto{\pgfqpoint{1.780911in}{3.384149in}}%
\pgfpathlineto{\pgfqpoint{1.818840in}{3.384149in}}%
\pgfpathlineto{\pgfqpoint{1.856768in}{3.384149in}}%
\pgfpathlineto{\pgfqpoint{1.894697in}{3.384149in}}%
\pgfpathlineto{\pgfqpoint{1.932625in}{3.386968in}}%
\pgfpathlineto{\pgfqpoint{1.970554in}{3.386968in}}%
\pgfpathlineto{\pgfqpoint{2.008482in}{3.389787in}}%
\pgfpathlineto{\pgfqpoint{2.046411in}{3.389787in}}%
\pgfpathlineto{\pgfqpoint{2.084339in}{3.389787in}}%
\pgfpathlineto{\pgfqpoint{2.122267in}{3.389787in}}%
\pgfpathlineto{\pgfqpoint{2.160196in}{3.389787in}}%
\pgfpathlineto{\pgfqpoint{2.198124in}{3.389787in}}%
\pgfpathlineto{\pgfqpoint{2.236053in}{3.389787in}}%
\pgfpathlineto{\pgfqpoint{2.273981in}{3.389787in}}%
\pgfpathlineto{\pgfqpoint{2.311910in}{3.389787in}}%
\pgfpathlineto{\pgfqpoint{2.349838in}{3.389787in}}%
\pgfpathlineto{\pgfqpoint{2.387767in}{3.389787in}}%
\pgfpathlineto{\pgfqpoint{2.425695in}{3.392606in}}%
\pgfusepath{stroke}%
\end{pgfscope}%
\begin{pgfscope}%
\pgfpathrectangle{\pgfqpoint{0.566985in}{0.528177in}}{\pgfqpoint{2.106702in}{2.864429in}} %
\pgfusepath{clip}%
\pgfsetroundcap%
\pgfsetroundjoin%
\pgfsetlinewidth{1.003750pt}%
\definecolor{currentstroke}{rgb}{0.168627,0.670588,0.494118}%
\pgfsetstrokecolor{currentstroke}%
\pgfsetdash{}{0pt}%
\pgfpathmoveto{\pgfqpoint{0.567188in}{0.843960in}}%
\pgfpathlineto{\pgfqpoint{0.604207in}{1.179636in}}%
\pgfpathlineto{\pgfqpoint{0.641225in}{1.428284in}}%
\pgfpathlineto{\pgfqpoint{0.678244in}{1.633419in}}%
\pgfpathlineto{\pgfqpoint{0.715263in}{1.857202in}}%
\pgfpathlineto{\pgfqpoint{0.752281in}{2.180445in}}%
\pgfpathlineto{\pgfqpoint{0.789300in}{2.385580in}}%
\pgfpathlineto{\pgfqpoint{0.826319in}{2.522337in}}%
\pgfpathlineto{\pgfqpoint{0.863337in}{2.727472in}}%
\pgfpathlineto{\pgfqpoint{0.900356in}{2.901526in}}%
\pgfpathlineto{\pgfqpoint{0.937375in}{2.951255in}}%
\pgfpathlineto{\pgfqpoint{0.974393in}{3.025850in}}%
\pgfpathlineto{\pgfqpoint{1.011412in}{3.069363in}}%
\pgfpathlineto{\pgfqpoint{1.048431in}{3.100444in}}%
\pgfpathlineto{\pgfqpoint{1.085449in}{3.143958in}}%
\pgfpathlineto{\pgfqpoint{1.122468in}{3.175039in}}%
\pgfpathlineto{\pgfqpoint{1.159487in}{3.193688in}}%
\pgfpathlineto{\pgfqpoint{1.196505in}{3.212336in}}%
\pgfpathlineto{\pgfqpoint{1.233524in}{3.224769in}}%
\pgfpathlineto{\pgfqpoint{1.270543in}{3.249633in}}%
\pgfpathlineto{\pgfqpoint{1.307561in}{3.280714in}}%
\pgfpathlineto{\pgfqpoint{1.344580in}{3.293147in}}%
\pgfpathlineto{\pgfqpoint{1.381599in}{3.305579in}}%
\pgfpathlineto{\pgfqpoint{1.418617in}{3.318012in}}%
\pgfpathlineto{\pgfqpoint{1.455636in}{3.318012in}}%
\pgfpathlineto{\pgfqpoint{1.492655in}{3.318012in}}%
\pgfpathlineto{\pgfqpoint{1.529673in}{3.330444in}}%
\pgfpathlineto{\pgfqpoint{1.566692in}{3.330444in}}%
\pgfpathlineto{\pgfqpoint{1.603711in}{3.349093in}}%
\pgfpathlineto{\pgfqpoint{1.640729in}{3.355309in}}%
\pgfpathlineto{\pgfqpoint{1.677748in}{3.361525in}}%
\pgfpathlineto{\pgfqpoint{1.714767in}{3.361525in}}%
\pgfpathlineto{\pgfqpoint{1.751785in}{3.367741in}}%
\pgfpathlineto{\pgfqpoint{1.788804in}{3.367741in}}%
\pgfpathlineto{\pgfqpoint{1.825823in}{3.373958in}}%
\pgfpathlineto{\pgfqpoint{1.862841in}{3.380174in}}%
\pgfpathlineto{\pgfqpoint{1.899860in}{3.380174in}}%
\pgfpathlineto{\pgfqpoint{1.936879in}{3.380174in}}%
\pgfpathlineto{\pgfqpoint{1.973897in}{3.380174in}}%
\pgfpathlineto{\pgfqpoint{2.010916in}{3.380174in}}%
\pgfpathlineto{\pgfqpoint{2.047935in}{3.386390in}}%
\pgfpathlineto{\pgfqpoint{2.084953in}{3.386390in}}%
\pgfpathlineto{\pgfqpoint{2.121972in}{3.386390in}}%
\pgfpathlineto{\pgfqpoint{2.158990in}{3.386390in}}%
\pgfpathlineto{\pgfqpoint{2.196009in}{3.386390in}}%
\pgfpathlineto{\pgfqpoint{2.233028in}{3.386390in}}%
\pgfpathlineto{\pgfqpoint{2.270046in}{3.386390in}}%
\pgfpathlineto{\pgfqpoint{2.307065in}{3.386390in}}%
\pgfpathlineto{\pgfqpoint{2.344084in}{3.386390in}}%
\pgfpathlineto{\pgfqpoint{2.381102in}{3.392606in}}%
\pgfusepath{stroke}%
\end{pgfscope}%
\begin{pgfscope}%
\pgfpathrectangle{\pgfqpoint{0.566985in}{0.528177in}}{\pgfqpoint{2.106702in}{2.864429in}} %
\pgfusepath{clip}%
\pgfsetroundcap%
\pgfsetroundjoin%
\pgfsetlinewidth{1.003750pt}%
\definecolor{currentstroke}{rgb}{1.000000,0.494118,0.250980}%
\pgfsetstrokecolor{currentstroke}%
\pgfsetdash{}{0pt}%
\pgfpathmoveto{\pgfqpoint{0.567154in}{0.638923in}}%
\pgfpathlineto{\pgfqpoint{0.597078in}{0.828488in}}%
\pgfpathlineto{\pgfqpoint{0.627002in}{1.038007in}}%
\pgfpathlineto{\pgfqpoint{0.656926in}{1.207618in}}%
\pgfpathlineto{\pgfqpoint{0.686850in}{1.372241in}}%
\pgfpathlineto{\pgfqpoint{0.716774in}{1.566794in}}%
\pgfpathlineto{\pgfqpoint{0.746698in}{1.711463in}}%
\pgfpathlineto{\pgfqpoint{0.776622in}{1.846154in}}%
\pgfpathlineto{\pgfqpoint{0.806546in}{1.980845in}}%
\pgfpathlineto{\pgfqpoint{0.836471in}{2.085604in}}%
\pgfpathlineto{\pgfqpoint{0.866395in}{2.205330in}}%
\pgfpathlineto{\pgfqpoint{0.896319in}{2.320066in}}%
\pgfpathlineto{\pgfqpoint{0.926243in}{2.404872in}}%
\pgfpathlineto{\pgfqpoint{0.956167in}{2.509632in}}%
\pgfpathlineto{\pgfqpoint{0.986091in}{2.614391in}}%
\pgfpathlineto{\pgfqpoint{1.016015in}{2.699197in}}%
\pgfpathlineto{\pgfqpoint{1.045939in}{2.779014in}}%
\pgfpathlineto{\pgfqpoint{1.075863in}{2.833888in}}%
\pgfpathlineto{\pgfqpoint{1.105787in}{2.913705in}}%
\pgfpathlineto{\pgfqpoint{1.135711in}{2.938648in}}%
\pgfpathlineto{\pgfqpoint{1.165635in}{2.983545in}}%
\pgfpathlineto{\pgfqpoint{1.195559in}{3.053384in}}%
\pgfpathlineto{\pgfqpoint{1.225483in}{3.073339in}}%
\pgfpathlineto{\pgfqpoint{1.255407in}{3.113247in}}%
\pgfpathlineto{\pgfqpoint{1.285331in}{3.158144in}}%
\pgfpathlineto{\pgfqpoint{1.315255in}{3.193064in}}%
\pgfpathlineto{\pgfqpoint{1.345180in}{3.218007in}}%
\pgfpathlineto{\pgfqpoint{1.375104in}{3.247938in}}%
\pgfpathlineto{\pgfqpoint{1.405028in}{3.257915in}}%
\pgfpathlineto{\pgfqpoint{1.434952in}{3.287847in}}%
\pgfpathlineto{\pgfqpoint{1.464876in}{3.307801in}}%
\pgfpathlineto{\pgfqpoint{1.494800in}{3.327755in}}%
\pgfpathlineto{\pgfqpoint{1.524724in}{3.337732in}}%
\pgfpathlineto{\pgfqpoint{1.554648in}{3.347709in}}%
\pgfpathlineto{\pgfqpoint{1.584572in}{3.357686in}}%
\pgfpathlineto{\pgfqpoint{1.614496in}{3.367663in}}%
\pgfpathlineto{\pgfqpoint{1.644420in}{3.367663in}}%
\pgfpathlineto{\pgfqpoint{1.674344in}{3.372652in}}%
\pgfpathlineto{\pgfqpoint{1.704268in}{3.377641in}}%
\pgfpathlineto{\pgfqpoint{1.734192in}{3.377641in}}%
\pgfpathlineto{\pgfqpoint{1.764116in}{3.377641in}}%
\pgfpathlineto{\pgfqpoint{1.794040in}{3.377641in}}%
\pgfpathlineto{\pgfqpoint{1.823964in}{3.382629in}}%
\pgfpathlineto{\pgfqpoint{1.853889in}{3.382629in}}%
\pgfpathlineto{\pgfqpoint{1.883813in}{3.382629in}}%
\pgfpathlineto{\pgfqpoint{1.913737in}{3.387618in}}%
\pgfpathlineto{\pgfqpoint{1.943661in}{3.387618in}}%
\pgfpathlineto{\pgfqpoint{1.973585in}{3.387618in}}%
\pgfpathlineto{\pgfqpoint{2.003509in}{3.387618in}}%
\pgfpathlineto{\pgfqpoint{2.033433in}{3.392606in}}%
\pgfusepath{stroke}%
\end{pgfscope}%
\begin{pgfscope}%
\pgfpathrectangle{\pgfqpoint{0.566985in}{0.528177in}}{\pgfqpoint{2.106702in}{2.864429in}} %
\pgfusepath{clip}%
\pgfsetroundcap%
\pgfsetroundjoin%
\pgfsetlinewidth{1.003750pt}%
\definecolor{currentstroke}{rgb}{1.000000,0.694118,0.250980}%
\pgfsetstrokecolor{currentstroke}%
\pgfsetdash{}{0pt}%
\pgfpathmoveto{\pgfqpoint{0.567659in}{0.591495in}}%
\pgfpathlineto{\pgfqpoint{0.593884in}{0.947924in}}%
\pgfpathlineto{\pgfqpoint{0.620109in}{1.300160in}}%
\pgfpathlineto{\pgfqpoint{0.646334in}{1.597882in}}%
\pgfpathlineto{\pgfqpoint{0.672559in}{1.853673in}}%
\pgfpathlineto{\pgfqpoint{0.698783in}{2.046563in}}%
\pgfpathlineto{\pgfqpoint{0.725008in}{2.189135in}}%
\pgfpathlineto{\pgfqpoint{0.751233in}{2.319127in}}%
\pgfpathlineto{\pgfqpoint{0.777458in}{2.453312in}}%
\pgfpathlineto{\pgfqpoint{0.803683in}{2.545564in}}%
\pgfpathlineto{\pgfqpoint{0.829907in}{2.683942in}}%
\pgfpathlineto{\pgfqpoint{0.856132in}{2.759421in}}%
\pgfpathlineto{\pgfqpoint{0.882357in}{2.851673in}}%
\pgfpathlineto{\pgfqpoint{0.908582in}{2.931345in}}%
\pgfpathlineto{\pgfqpoint{0.934807in}{3.006824in}}%
\pgfpathlineto{\pgfqpoint{0.961031in}{3.031984in}}%
\pgfpathlineto{\pgfqpoint{0.987256in}{3.099077in}}%
\pgfpathlineto{\pgfqpoint{1.013481in}{3.128430in}}%
\pgfpathlineto{\pgfqpoint{1.039706in}{3.145203in}}%
\pgfpathlineto{\pgfqpoint{1.065931in}{3.178749in}}%
\pgfpathlineto{\pgfqpoint{1.092155in}{3.212295in}}%
\pgfpathlineto{\pgfqpoint{1.118380in}{3.262615in}}%
\pgfpathlineto{\pgfqpoint{1.144605in}{3.275194in}}%
\pgfpathlineto{\pgfqpoint{1.170830in}{3.300354in}}%
\pgfpathlineto{\pgfqpoint{1.197055in}{3.304547in}}%
\pgfpathlineto{\pgfqpoint{1.223279in}{3.321320in}}%
\pgfpathlineto{\pgfqpoint{1.249504in}{3.329707in}}%
\pgfpathlineto{\pgfqpoint{1.275729in}{3.338094in}}%
\pgfpathlineto{\pgfqpoint{1.301954in}{3.342287in}}%
\pgfpathlineto{\pgfqpoint{1.328179in}{3.346480in}}%
\pgfpathlineto{\pgfqpoint{1.354403in}{3.350673in}}%
\pgfpathlineto{\pgfqpoint{1.380628in}{3.363253in}}%
\pgfpathlineto{\pgfqpoint{1.406853in}{3.367447in}}%
\pgfpathlineto{\pgfqpoint{1.433078in}{3.371640in}}%
\pgfpathlineto{\pgfqpoint{1.459303in}{3.371640in}}%
\pgfpathlineto{\pgfqpoint{1.485527in}{3.371640in}}%
\pgfpathlineto{\pgfqpoint{1.511752in}{3.371640in}}%
\pgfpathlineto{\pgfqpoint{1.537977in}{3.380026in}}%
\pgfpathlineto{\pgfqpoint{1.564202in}{3.384220in}}%
\pgfpathlineto{\pgfqpoint{1.590427in}{3.388413in}}%
\pgfpathlineto{\pgfqpoint{1.616651in}{3.388413in}}%
\pgfpathlineto{\pgfqpoint{1.642876in}{3.388413in}}%
\pgfpathlineto{\pgfqpoint{1.669101in}{3.388413in}}%
\pgfpathlineto{\pgfqpoint{1.695326in}{3.388413in}}%
\pgfpathlineto{\pgfqpoint{1.721551in}{3.388413in}}%
\pgfpathlineto{\pgfqpoint{1.747775in}{3.388413in}}%
\pgfpathlineto{\pgfqpoint{1.774000in}{3.388413in}}%
\pgfpathlineto{\pgfqpoint{1.800225in}{3.388413in}}%
\pgfpathlineto{\pgfqpoint{1.826450in}{3.388413in}}%
\pgfpathlineto{\pgfqpoint{1.852675in}{3.392606in}}%
\pgfusepath{stroke}%
\end{pgfscope}%
\begin{pgfscope}%
\pgfsetrectcap%
\pgfsetmiterjoin%
\pgfsetlinewidth{1.254687pt}%
\definecolor{currentstroke}{rgb}{0.150000,0.150000,0.150000}%
\pgfsetstrokecolor{currentstroke}%
\pgfsetdash{}{0pt}%
\pgfpathmoveto{\pgfqpoint{0.566985in}{0.528177in}}%
\pgfpathlineto{\pgfqpoint{0.566985in}{3.392606in}}%
\pgfusepath{stroke}%
\end{pgfscope}%
\begin{pgfscope}%
\pgfsetrectcap%
\pgfsetmiterjoin%
\pgfsetlinewidth{1.254687pt}%
\definecolor{currentstroke}{rgb}{0.150000,0.150000,0.150000}%
\pgfsetstrokecolor{currentstroke}%
\pgfsetdash{}{0pt}%
\pgfpathmoveto{\pgfqpoint{0.566985in}{0.528177in}}%
\pgfpathlineto{\pgfqpoint{2.673686in}{0.528177in}}%
\pgfusepath{stroke}%
\end{pgfscope}%
\begin{pgfscope}%
\pgfsetbuttcap%
\pgfsetmiterjoin%
\definecolor{currentfill}{rgb}{1.000000,1.000000,1.000000}%
\pgfsetfillcolor{currentfill}%
\pgfsetlinewidth{0.000000pt}%
\definecolor{currentstroke}{rgb}{0.000000,0.000000,0.000000}%
\pgfsetstrokecolor{currentstroke}%
\pgfsetstrokeopacity{0.000000}%
\pgfsetdash{}{0pt}%
\pgfpathmoveto{\pgfqpoint{3.095027in}{0.528177in}}%
\pgfpathlineto{\pgfqpoint{5.201729in}{0.528177in}}%
\pgfpathlineto{\pgfqpoint{5.201729in}{2.653399in}}%
\pgfpathlineto{\pgfqpoint{3.095027in}{2.653399in}}%
\pgfpathclose%
\pgfusepath{fill}%
\end{pgfscope}%
\begin{pgfscope}%
\pgfsetroundcap%
\pgfsetroundjoin%
\pgfsetlinewidth{1.003750pt}%
\definecolor{currentstroke}{rgb}{0.200000,0.427451,0.650980}%
\pgfsetstrokecolor{currentstroke}%
\pgfsetdash{}{0pt}%
\pgfpathmoveto{\pgfqpoint{2.984357in}{3.319977in}}%
\pgfpathlineto{\pgfqpoint{3.095468in}{3.319977in}}%
\pgfusepath{stroke}%
\end{pgfscope}%
\begin{pgfscope}%
\definecolor{textcolor}{rgb}{1.000000,1.000000,1.000000}%
\pgfsetstrokecolor{textcolor}%
\pgfsetfillcolor{textcolor}%
\pgftext[x=3.184357in,y=3.281088in,left,base]{\color{textcolor}\rmfamily\fontsize{8.000000}{9.600000}\selectfont WT + Vehicle (1129)}%
\end{pgfscope}%
\begin{pgfscope}%
\pgfsetroundcap%
\pgfsetroundjoin%
\pgfsetlinewidth{1.003750pt}%
\definecolor{currentstroke}{rgb}{0.168627,0.670588,0.494118}%
\pgfsetstrokecolor{currentstroke}%
\pgfsetdash{}{0pt}%
\pgfpathmoveto{\pgfqpoint{2.984357in}{3.153338in}}%
\pgfpathlineto{\pgfqpoint{3.095468in}{3.153338in}}%
\pgfusepath{stroke}%
\end{pgfscope}%
\begin{pgfscope}%
\definecolor{textcolor}{rgb}{1.000000,1.000000,1.000000}%
\pgfsetstrokecolor{textcolor}%
\pgfsetfillcolor{textcolor}%
\pgftext[x=3.184357in,y=3.114449in,left,base]{\color{textcolor}\rmfamily\fontsize{8.000000}{9.600000}\selectfont WT + TAT-GluA2\textsubscript{3Y} (512)}%
\end{pgfscope}%
\begin{pgfscope}%
\pgfsetroundcap%
\pgfsetroundjoin%
\pgfsetlinewidth{1.003750pt}%
\definecolor{currentstroke}{rgb}{1.000000,0.494118,0.250980}%
\pgfsetstrokecolor{currentstroke}%
\pgfsetdash{}{0pt}%
\pgfpathmoveto{\pgfqpoint{2.984357in}{2.986698in}}%
\pgfpathlineto{\pgfqpoint{3.095468in}{2.986698in}}%
\pgfusepath{stroke}%
\end{pgfscope}%
\begin{pgfscope}%
\definecolor{textcolor}{rgb}{1.000000,1.000000,1.000000}%
\pgfsetstrokecolor{textcolor}%
\pgfsetfillcolor{textcolor}%
\pgftext[x=3.184357in,y=2.947809in,left,base]{\color{textcolor}\rmfamily\fontsize{8.000000}{9.600000}\selectfont Tg + Vehicle (638)}%
\end{pgfscope}%
\begin{pgfscope}%
\pgfsetroundcap%
\pgfsetroundjoin%
\pgfsetlinewidth{1.003750pt}%
\definecolor{currentstroke}{rgb}{1.000000,0.694118,0.250980}%
\pgfsetstrokecolor{currentstroke}%
\pgfsetdash{}{0pt}%
\pgfpathmoveto{\pgfqpoint{2.984357in}{2.820059in}}%
\pgfpathlineto{\pgfqpoint{3.095468in}{2.820059in}}%
\pgfusepath{stroke}%
\end{pgfscope}%
\begin{pgfscope}%
\definecolor{textcolor}{rgb}{1.000000,1.000000,1.000000}%
\pgfsetstrokecolor{textcolor}%
\pgfsetfillcolor{textcolor}%
\pgftext[x=3.184357in,y=2.781170in,left,base]{\color{textcolor}\rmfamily\fontsize{8.000000}{9.600000}\selectfont Tg + TAT-GluA2\textsubscript{3Y} (759)}%
\end{pgfscope}%
\begin{pgfscope}%
\pgfsetroundcap%
\pgfsetroundjoin%
\pgfsetlinewidth{1.003750pt}%
\definecolor{currentstroke}{rgb}{0.200000,0.427451,0.650980}%
\pgfsetstrokecolor{currentstroke}%
\pgfsetdash{}{0pt}%
\pgfpathmoveto{\pgfqpoint{2.984357in}{3.319977in}}%
\pgfpathlineto{\pgfqpoint{3.095468in}{3.319977in}}%
\pgfusepath{stroke}%
\end{pgfscope}%
\begin{pgfscope}%
\definecolor{textcolor}{rgb}{1.000000,1.000000,1.000000}%
\pgfsetstrokecolor{textcolor}%
\pgfsetfillcolor{textcolor}%
\pgftext[x=3.184357in,y=3.281088in,left,base]{\color{textcolor}\rmfamily\fontsize{8.000000}{9.600000}\selectfont WT + Vehicle (1129)}%
\end{pgfscope}%
\begin{pgfscope}%
\pgfsetroundcap%
\pgfsetroundjoin%
\pgfsetlinewidth{1.003750pt}%
\definecolor{currentstroke}{rgb}{0.168627,0.670588,0.494118}%
\pgfsetstrokecolor{currentstroke}%
\pgfsetdash{}{0pt}%
\pgfpathmoveto{\pgfqpoint{2.984357in}{3.153338in}}%
\pgfpathlineto{\pgfqpoint{3.095468in}{3.153338in}}%
\pgfusepath{stroke}%
\end{pgfscope}%
\begin{pgfscope}%
\definecolor{textcolor}{rgb}{1.000000,1.000000,1.000000}%
\pgfsetstrokecolor{textcolor}%
\pgfsetfillcolor{textcolor}%
\pgftext[x=3.184357in,y=3.114449in,left,base]{\color{textcolor}\rmfamily\fontsize{8.000000}{9.600000}\selectfont WT + TAT-GluA2\textsubscript{3Y} (512)}%
\end{pgfscope}%
\begin{pgfscope}%
\pgfsetroundcap%
\pgfsetroundjoin%
\pgfsetlinewidth{1.003750pt}%
\definecolor{currentstroke}{rgb}{1.000000,0.494118,0.250980}%
\pgfsetstrokecolor{currentstroke}%
\pgfsetdash{}{0pt}%
\pgfpathmoveto{\pgfqpoint{2.984357in}{2.986698in}}%
\pgfpathlineto{\pgfqpoint{3.095468in}{2.986698in}}%
\pgfusepath{stroke}%
\end{pgfscope}%
\begin{pgfscope}%
\definecolor{textcolor}{rgb}{1.000000,1.000000,1.000000}%
\pgfsetstrokecolor{textcolor}%
\pgfsetfillcolor{textcolor}%
\pgftext[x=3.184357in,y=2.947809in,left,base]{\color{textcolor}\rmfamily\fontsize{8.000000}{9.600000}\selectfont Tg + Vehicle (638)}%
\end{pgfscope}%
\begin{pgfscope}%
\pgfsetroundcap%
\pgfsetroundjoin%
\pgfsetlinewidth{1.003750pt}%
\definecolor{currentstroke}{rgb}{1.000000,0.694118,0.250980}%
\pgfsetstrokecolor{currentstroke}%
\pgfsetdash{}{0pt}%
\pgfpathmoveto{\pgfqpoint{2.984357in}{2.820059in}}%
\pgfpathlineto{\pgfqpoint{3.095468in}{2.820059in}}%
\pgfusepath{stroke}%
\end{pgfscope}%
\begin{pgfscope}%
\definecolor{textcolor}{rgb}{1.000000,1.000000,1.000000}%
\pgfsetstrokecolor{textcolor}%
\pgfsetfillcolor{textcolor}%
\pgftext[x=3.184357in,y=2.781170in,left,base]{\color{textcolor}\rmfamily\fontsize{8.000000}{9.600000}\selectfont Tg + TAT-GluA2\textsubscript{3Y} (759)}%
\end{pgfscope}%
\begin{pgfscope}%
\pgfsetbuttcap%
\pgfsetroundjoin%
\definecolor{currentfill}{rgb}{0.150000,0.150000,0.150000}%
\pgfsetfillcolor{currentfill}%
\pgfsetlinewidth{1.003750pt}%
\definecolor{currentstroke}{rgb}{0.150000,0.150000,0.150000}%
\pgfsetstrokecolor{currentstroke}%
\pgfsetdash{}{0pt}%
\pgfsys@defobject{currentmarker}{\pgfqpoint{0.000000in}{0.000000in}}{\pgfqpoint{0.041667in}{0.000000in}}{%
\pgfpathmoveto{\pgfqpoint{0.000000in}{0.000000in}}%
\pgfpathlineto{\pgfqpoint{0.041667in}{0.000000in}}%
\pgfusepath{stroke,fill}%
}%
\begin{pgfscope}%
\pgfsys@transformshift{3.095027in}{0.528177in}%
\pgfsys@useobject{currentmarker}{}%
\end{pgfscope}%
\end{pgfscope}%
\begin{pgfscope}%
\definecolor{textcolor}{rgb}{0.150000,0.150000,0.150000}%
\pgfsetstrokecolor{textcolor}%
\pgfsetfillcolor{textcolor}%
\pgftext[x=2.997805in,y=0.528177in,right,]{\color{textcolor}\rmfamily\fontsize{10.000000}{12.000000}\selectfont \(\displaystyle 0.00\)}%
\end{pgfscope}%
\begin{pgfscope}%
\pgfsetbuttcap%
\pgfsetroundjoin%
\definecolor{currentfill}{rgb}{0.150000,0.150000,0.150000}%
\pgfsetfillcolor{currentfill}%
\pgfsetlinewidth{1.003750pt}%
\definecolor{currentstroke}{rgb}{0.150000,0.150000,0.150000}%
\pgfsetstrokecolor{currentstroke}%
\pgfsetdash{}{0pt}%
\pgfsys@defobject{currentmarker}{\pgfqpoint{0.000000in}{0.000000in}}{\pgfqpoint{0.041667in}{0.000000in}}{%
\pgfpathmoveto{\pgfqpoint{0.000000in}{0.000000in}}%
\pgfpathlineto{\pgfqpoint{0.041667in}{0.000000in}}%
\pgfusepath{stroke,fill}%
}%
\begin{pgfscope}%
\pgfsys@transformshift{3.095027in}{0.953221in}%
\pgfsys@useobject{currentmarker}{}%
\end{pgfscope}%
\end{pgfscope}%
\begin{pgfscope}%
\definecolor{textcolor}{rgb}{0.150000,0.150000,0.150000}%
\pgfsetstrokecolor{textcolor}%
\pgfsetfillcolor{textcolor}%
\pgftext[x=2.997805in,y=0.953221in,right,]{\color{textcolor}\rmfamily\fontsize{10.000000}{12.000000}\selectfont \(\displaystyle 0.05\)}%
\end{pgfscope}%
\begin{pgfscope}%
\pgfsetbuttcap%
\pgfsetroundjoin%
\definecolor{currentfill}{rgb}{0.150000,0.150000,0.150000}%
\pgfsetfillcolor{currentfill}%
\pgfsetlinewidth{1.003750pt}%
\definecolor{currentstroke}{rgb}{0.150000,0.150000,0.150000}%
\pgfsetstrokecolor{currentstroke}%
\pgfsetdash{}{0pt}%
\pgfsys@defobject{currentmarker}{\pgfqpoint{0.000000in}{0.000000in}}{\pgfqpoint{0.041667in}{0.000000in}}{%
\pgfpathmoveto{\pgfqpoint{0.000000in}{0.000000in}}%
\pgfpathlineto{\pgfqpoint{0.041667in}{0.000000in}}%
\pgfusepath{stroke,fill}%
}%
\begin{pgfscope}%
\pgfsys@transformshift{3.095027in}{1.378266in}%
\pgfsys@useobject{currentmarker}{}%
\end{pgfscope}%
\end{pgfscope}%
\begin{pgfscope}%
\definecolor{textcolor}{rgb}{0.150000,0.150000,0.150000}%
\pgfsetstrokecolor{textcolor}%
\pgfsetfillcolor{textcolor}%
\pgftext[x=2.997805in,y=1.378266in,right,]{\color{textcolor}\rmfamily\fontsize{10.000000}{12.000000}\selectfont \(\displaystyle 0.10\)}%
\end{pgfscope}%
\begin{pgfscope}%
\pgfsetbuttcap%
\pgfsetroundjoin%
\definecolor{currentfill}{rgb}{0.150000,0.150000,0.150000}%
\pgfsetfillcolor{currentfill}%
\pgfsetlinewidth{1.003750pt}%
\definecolor{currentstroke}{rgb}{0.150000,0.150000,0.150000}%
\pgfsetstrokecolor{currentstroke}%
\pgfsetdash{}{0pt}%
\pgfsys@defobject{currentmarker}{\pgfqpoint{0.000000in}{0.000000in}}{\pgfqpoint{0.041667in}{0.000000in}}{%
\pgfpathmoveto{\pgfqpoint{0.000000in}{0.000000in}}%
\pgfpathlineto{\pgfqpoint{0.041667in}{0.000000in}}%
\pgfusepath{stroke,fill}%
}%
\begin{pgfscope}%
\pgfsys@transformshift{3.095027in}{1.803310in}%
\pgfsys@useobject{currentmarker}{}%
\end{pgfscope}%
\end{pgfscope}%
\begin{pgfscope}%
\definecolor{textcolor}{rgb}{0.150000,0.150000,0.150000}%
\pgfsetstrokecolor{textcolor}%
\pgfsetfillcolor{textcolor}%
\pgftext[x=2.997805in,y=1.803310in,right,]{\color{textcolor}\rmfamily\fontsize{10.000000}{12.000000}\selectfont \(\displaystyle 0.15\)}%
\end{pgfscope}%
\begin{pgfscope}%
\pgfsetbuttcap%
\pgfsetroundjoin%
\definecolor{currentfill}{rgb}{0.150000,0.150000,0.150000}%
\pgfsetfillcolor{currentfill}%
\pgfsetlinewidth{1.003750pt}%
\definecolor{currentstroke}{rgb}{0.150000,0.150000,0.150000}%
\pgfsetstrokecolor{currentstroke}%
\pgfsetdash{}{0pt}%
\pgfsys@defobject{currentmarker}{\pgfqpoint{0.000000in}{0.000000in}}{\pgfqpoint{0.041667in}{0.000000in}}{%
\pgfpathmoveto{\pgfqpoint{0.000000in}{0.000000in}}%
\pgfpathlineto{\pgfqpoint{0.041667in}{0.000000in}}%
\pgfusepath{stroke,fill}%
}%
\begin{pgfscope}%
\pgfsys@transformshift{3.095027in}{2.228354in}%
\pgfsys@useobject{currentmarker}{}%
\end{pgfscope}%
\end{pgfscope}%
\begin{pgfscope}%
\definecolor{textcolor}{rgb}{0.150000,0.150000,0.150000}%
\pgfsetstrokecolor{textcolor}%
\pgfsetfillcolor{textcolor}%
\pgftext[x=2.997805in,y=2.228354in,right,]{\color{textcolor}\rmfamily\fontsize{10.000000}{12.000000}\selectfont \(\displaystyle 0.20\)}%
\end{pgfscope}%
\begin{pgfscope}%
\pgfsetbuttcap%
\pgfsetroundjoin%
\definecolor{currentfill}{rgb}{0.150000,0.150000,0.150000}%
\pgfsetfillcolor{currentfill}%
\pgfsetlinewidth{1.003750pt}%
\definecolor{currentstroke}{rgb}{0.150000,0.150000,0.150000}%
\pgfsetstrokecolor{currentstroke}%
\pgfsetdash{}{0pt}%
\pgfsys@defobject{currentmarker}{\pgfqpoint{0.000000in}{0.000000in}}{\pgfqpoint{0.041667in}{0.000000in}}{%
\pgfpathmoveto{\pgfqpoint{0.000000in}{0.000000in}}%
\pgfpathlineto{\pgfqpoint{0.041667in}{0.000000in}}%
\pgfusepath{stroke,fill}%
}%
\begin{pgfscope}%
\pgfsys@transformshift{3.095027in}{2.653399in}%
\pgfsys@useobject{currentmarker}{}%
\end{pgfscope}%
\end{pgfscope}%
\begin{pgfscope}%
\definecolor{textcolor}{rgb}{0.150000,0.150000,0.150000}%
\pgfsetstrokecolor{textcolor}%
\pgfsetfillcolor{textcolor}%
\pgftext[x=2.997805in,y=2.653399in,right,]{\color{textcolor}\rmfamily\fontsize{10.000000}{12.000000}\selectfont \(\displaystyle 0.25\)}%
\end{pgfscope}%
\begin{pgfscope}%
\definecolor{textcolor}{rgb}{0.150000,0.150000,0.150000}%
\pgfsetstrokecolor{textcolor}%
\pgfsetfillcolor{textcolor}%
\pgftext[x=2.681446in,y=1.590788in,,bottom,rotate=90.000000]{\color{textcolor}\rmfamily\fontsize{10.000000}{12.000000}\selectfont \textbf{Cell activity (a.u.)}}%
\end{pgfscope}%
\begin{pgfscope}%
\pgfpathrectangle{\pgfqpoint{3.095027in}{0.528177in}}{\pgfqpoint{2.106702in}{2.125222in}} %
\pgfusepath{clip}%
\pgfsetbuttcap%
\pgfsetmiterjoin%
\definecolor{currentfill}{rgb}{0.200000,0.427451,0.650980}%
\pgfsetfillcolor{currentfill}%
\pgfsetlinewidth{1.505625pt}%
\definecolor{currentstroke}{rgb}{0.200000,0.427451,0.650980}%
\pgfsetstrokecolor{currentstroke}%
\pgfsetdash{}{0pt}%
\pgfpathmoveto{\pgfqpoint{3.170266in}{0.528177in}}%
\pgfpathlineto{\pgfqpoint{3.546463in}{0.528177in}}%
\pgfpathlineto{\pgfqpoint{3.546463in}{1.613136in}}%
\pgfpathlineto{\pgfqpoint{3.170266in}{1.613136in}}%
\pgfpathclose%
\pgfusepath{stroke,fill}%
\end{pgfscope}%
\begin{pgfscope}%
\pgfpathrectangle{\pgfqpoint{3.095027in}{0.528177in}}{\pgfqpoint{2.106702in}{2.125222in}} %
\pgfusepath{clip}%
\pgfsetbuttcap%
\pgfsetmiterjoin%
\definecolor{currentfill}{rgb}{0.168627,0.670588,0.494118}%
\pgfsetfillcolor{currentfill}%
\pgfsetlinewidth{1.505625pt}%
\definecolor{currentstroke}{rgb}{0.168627,0.670588,0.494118}%
\pgfsetstrokecolor{currentstroke}%
\pgfsetdash{}{0pt}%
\pgfpathmoveto{\pgfqpoint{3.696942in}{0.528177in}}%
\pgfpathlineto{\pgfqpoint{4.073138in}{0.528177in}}%
\pgfpathlineto{\pgfqpoint{4.073138in}{1.615039in}}%
\pgfpathlineto{\pgfqpoint{3.696942in}{1.615039in}}%
\pgfpathclose%
\pgfusepath{stroke,fill}%
\end{pgfscope}%
\begin{pgfscope}%
\pgfpathrectangle{\pgfqpoint{3.095027in}{0.528177in}}{\pgfqpoint{2.106702in}{2.125222in}} %
\pgfusepath{clip}%
\pgfsetbuttcap%
\pgfsetmiterjoin%
\definecolor{currentfill}{rgb}{1.000000,0.494118,0.250980}%
\pgfsetfillcolor{currentfill}%
\pgfsetlinewidth{1.505625pt}%
\definecolor{currentstroke}{rgb}{1.000000,0.494118,0.250980}%
\pgfsetstrokecolor{currentstroke}%
\pgfsetdash{}{0pt}%
\pgfpathmoveto{\pgfqpoint{4.223617in}{0.528177in}}%
\pgfpathlineto{\pgfqpoint{4.599814in}{0.528177in}}%
\pgfpathlineto{\pgfqpoint{4.599814in}{1.969462in}}%
\pgfpathlineto{\pgfqpoint{4.223617in}{1.969462in}}%
\pgfpathclose%
\pgfusepath{stroke,fill}%
\end{pgfscope}%
\begin{pgfscope}%
\pgfpathrectangle{\pgfqpoint{3.095027in}{0.528177in}}{\pgfqpoint{2.106702in}{2.125222in}} %
\pgfusepath{clip}%
\pgfsetbuttcap%
\pgfsetmiterjoin%
\definecolor{currentfill}{rgb}{1.000000,0.694118,0.250980}%
\pgfsetfillcolor{currentfill}%
\pgfsetlinewidth{1.505625pt}%
\definecolor{currentstroke}{rgb}{1.000000,0.694118,0.250980}%
\pgfsetstrokecolor{currentstroke}%
\pgfsetdash{}{0pt}%
\pgfpathmoveto{\pgfqpoint{4.750293in}{0.528177in}}%
\pgfpathlineto{\pgfqpoint{5.126489in}{0.528177in}}%
\pgfpathlineto{\pgfqpoint{5.126489in}{1.443146in}}%
\pgfpathlineto{\pgfqpoint{4.750293in}{1.443146in}}%
\pgfpathclose%
\pgfusepath{stroke,fill}%
\end{pgfscope}%
\begin{pgfscope}%
\pgfpathrectangle{\pgfqpoint{3.095027in}{0.528177in}}{\pgfqpoint{2.106702in}{2.125222in}} %
\pgfusepath{clip}%
\pgfsetbuttcap%
\pgfsetroundjoin%
\pgfsetlinewidth{1.505625pt}%
\definecolor{currentstroke}{rgb}{0.200000,0.427451,0.650980}%
\pgfsetstrokecolor{currentstroke}%
\pgfsetdash{}{0pt}%
\pgfpathmoveto{\pgfqpoint{3.358365in}{1.613136in}}%
\pgfpathlineto{\pgfqpoint{3.358365in}{1.641193in}}%
\pgfusepath{stroke}%
\end{pgfscope}%
\begin{pgfscope}%
\pgfpathrectangle{\pgfqpoint{3.095027in}{0.528177in}}{\pgfqpoint{2.106702in}{2.125222in}} %
\pgfusepath{clip}%
\pgfsetbuttcap%
\pgfsetroundjoin%
\pgfsetlinewidth{1.505625pt}%
\definecolor{currentstroke}{rgb}{0.168627,0.670588,0.494118}%
\pgfsetstrokecolor{currentstroke}%
\pgfsetdash{}{0pt}%
\pgfpathmoveto{\pgfqpoint{3.885040in}{1.615039in}}%
\pgfpathlineto{\pgfqpoint{3.885040in}{1.666109in}}%
\pgfusepath{stroke}%
\end{pgfscope}%
\begin{pgfscope}%
\pgfpathrectangle{\pgfqpoint{3.095027in}{0.528177in}}{\pgfqpoint{2.106702in}{2.125222in}} %
\pgfusepath{clip}%
\pgfsetbuttcap%
\pgfsetroundjoin%
\pgfsetlinewidth{1.505625pt}%
\definecolor{currentstroke}{rgb}{1.000000,0.494118,0.250980}%
\pgfsetstrokecolor{currentstroke}%
\pgfsetdash{}{0pt}%
\pgfpathmoveto{\pgfqpoint{4.411716in}{1.969462in}}%
\pgfpathlineto{\pgfqpoint{4.411716in}{2.020055in}}%
\pgfusepath{stroke}%
\end{pgfscope}%
\begin{pgfscope}%
\pgfpathrectangle{\pgfqpoint{3.095027in}{0.528177in}}{\pgfqpoint{2.106702in}{2.125222in}} %
\pgfusepath{clip}%
\pgfsetbuttcap%
\pgfsetroundjoin%
\pgfsetlinewidth{1.505625pt}%
\definecolor{currentstroke}{rgb}{1.000000,0.694118,0.250980}%
\pgfsetstrokecolor{currentstroke}%
\pgfsetdash{}{0pt}%
\pgfpathmoveto{\pgfqpoint{4.938391in}{1.443146in}}%
\pgfpathlineto{\pgfqpoint{4.938391in}{1.475526in}}%
\pgfusepath{stroke}%
\end{pgfscope}%
\begin{pgfscope}%
\pgfpathrectangle{\pgfqpoint{3.095027in}{0.528177in}}{\pgfqpoint{2.106702in}{2.125222in}} %
\pgfusepath{clip}%
\pgfsetbuttcap%
\pgfsetroundjoin%
\definecolor{currentfill}{rgb}{0.200000,0.427451,0.650980}%
\pgfsetfillcolor{currentfill}%
\pgfsetlinewidth{1.505625pt}%
\definecolor{currentstroke}{rgb}{0.200000,0.427451,0.650980}%
\pgfsetstrokecolor{currentstroke}%
\pgfsetdash{}{0pt}%
\pgfsys@defobject{currentmarker}{\pgfqpoint{-0.111111in}{-0.000000in}}{\pgfqpoint{0.111111in}{0.000000in}}{%
\pgfpathmoveto{\pgfqpoint{0.111111in}{-0.000000in}}%
\pgfpathlineto{\pgfqpoint{-0.111111in}{0.000000in}}%
\pgfusepath{stroke,fill}%
}%
\begin{pgfscope}%
\pgfsys@transformshift{3.358365in}{1.613136in}%
\pgfsys@useobject{currentmarker}{}%
\end{pgfscope}%
\end{pgfscope}%
\begin{pgfscope}%
\pgfpathrectangle{\pgfqpoint{3.095027in}{0.528177in}}{\pgfqpoint{2.106702in}{2.125222in}} %
\pgfusepath{clip}%
\pgfsetbuttcap%
\pgfsetroundjoin%
\definecolor{currentfill}{rgb}{0.200000,0.427451,0.650980}%
\pgfsetfillcolor{currentfill}%
\pgfsetlinewidth{1.505625pt}%
\definecolor{currentstroke}{rgb}{0.200000,0.427451,0.650980}%
\pgfsetstrokecolor{currentstroke}%
\pgfsetdash{}{0pt}%
\pgfsys@defobject{currentmarker}{\pgfqpoint{-0.111111in}{-0.000000in}}{\pgfqpoint{0.111111in}{0.000000in}}{%
\pgfpathmoveto{\pgfqpoint{0.111111in}{-0.000000in}}%
\pgfpathlineto{\pgfqpoint{-0.111111in}{0.000000in}}%
\pgfusepath{stroke,fill}%
}%
\begin{pgfscope}%
\pgfsys@transformshift{3.358365in}{1.641193in}%
\pgfsys@useobject{currentmarker}{}%
\end{pgfscope}%
\end{pgfscope}%
\begin{pgfscope}%
\pgfpathrectangle{\pgfqpoint{3.095027in}{0.528177in}}{\pgfqpoint{2.106702in}{2.125222in}} %
\pgfusepath{clip}%
\pgfsetbuttcap%
\pgfsetroundjoin%
\definecolor{currentfill}{rgb}{0.168627,0.670588,0.494118}%
\pgfsetfillcolor{currentfill}%
\pgfsetlinewidth{1.505625pt}%
\definecolor{currentstroke}{rgb}{0.168627,0.670588,0.494118}%
\pgfsetstrokecolor{currentstroke}%
\pgfsetdash{}{0pt}%
\pgfsys@defobject{currentmarker}{\pgfqpoint{-0.111111in}{-0.000000in}}{\pgfqpoint{0.111111in}{0.000000in}}{%
\pgfpathmoveto{\pgfqpoint{0.111111in}{-0.000000in}}%
\pgfpathlineto{\pgfqpoint{-0.111111in}{0.000000in}}%
\pgfusepath{stroke,fill}%
}%
\begin{pgfscope}%
\pgfsys@transformshift{3.885040in}{1.615039in}%
\pgfsys@useobject{currentmarker}{}%
\end{pgfscope}%
\end{pgfscope}%
\begin{pgfscope}%
\pgfpathrectangle{\pgfqpoint{3.095027in}{0.528177in}}{\pgfqpoint{2.106702in}{2.125222in}} %
\pgfusepath{clip}%
\pgfsetbuttcap%
\pgfsetroundjoin%
\definecolor{currentfill}{rgb}{0.168627,0.670588,0.494118}%
\pgfsetfillcolor{currentfill}%
\pgfsetlinewidth{1.505625pt}%
\definecolor{currentstroke}{rgb}{0.168627,0.670588,0.494118}%
\pgfsetstrokecolor{currentstroke}%
\pgfsetdash{}{0pt}%
\pgfsys@defobject{currentmarker}{\pgfqpoint{-0.111111in}{-0.000000in}}{\pgfqpoint{0.111111in}{0.000000in}}{%
\pgfpathmoveto{\pgfqpoint{0.111111in}{-0.000000in}}%
\pgfpathlineto{\pgfqpoint{-0.111111in}{0.000000in}}%
\pgfusepath{stroke,fill}%
}%
\begin{pgfscope}%
\pgfsys@transformshift{3.885040in}{1.666109in}%
\pgfsys@useobject{currentmarker}{}%
\end{pgfscope}%
\end{pgfscope}%
\begin{pgfscope}%
\pgfpathrectangle{\pgfqpoint{3.095027in}{0.528177in}}{\pgfqpoint{2.106702in}{2.125222in}} %
\pgfusepath{clip}%
\pgfsetbuttcap%
\pgfsetroundjoin%
\definecolor{currentfill}{rgb}{1.000000,0.494118,0.250980}%
\pgfsetfillcolor{currentfill}%
\pgfsetlinewidth{1.505625pt}%
\definecolor{currentstroke}{rgb}{1.000000,0.494118,0.250980}%
\pgfsetstrokecolor{currentstroke}%
\pgfsetdash{}{0pt}%
\pgfsys@defobject{currentmarker}{\pgfqpoint{-0.111111in}{-0.000000in}}{\pgfqpoint{0.111111in}{0.000000in}}{%
\pgfpathmoveto{\pgfqpoint{0.111111in}{-0.000000in}}%
\pgfpathlineto{\pgfqpoint{-0.111111in}{0.000000in}}%
\pgfusepath{stroke,fill}%
}%
\begin{pgfscope}%
\pgfsys@transformshift{4.411716in}{1.969462in}%
\pgfsys@useobject{currentmarker}{}%
\end{pgfscope}%
\end{pgfscope}%
\begin{pgfscope}%
\pgfpathrectangle{\pgfqpoint{3.095027in}{0.528177in}}{\pgfqpoint{2.106702in}{2.125222in}} %
\pgfusepath{clip}%
\pgfsetbuttcap%
\pgfsetroundjoin%
\definecolor{currentfill}{rgb}{1.000000,0.494118,0.250980}%
\pgfsetfillcolor{currentfill}%
\pgfsetlinewidth{1.505625pt}%
\definecolor{currentstroke}{rgb}{1.000000,0.494118,0.250980}%
\pgfsetstrokecolor{currentstroke}%
\pgfsetdash{}{0pt}%
\pgfsys@defobject{currentmarker}{\pgfqpoint{-0.111111in}{-0.000000in}}{\pgfqpoint{0.111111in}{0.000000in}}{%
\pgfpathmoveto{\pgfqpoint{0.111111in}{-0.000000in}}%
\pgfpathlineto{\pgfqpoint{-0.111111in}{0.000000in}}%
\pgfusepath{stroke,fill}%
}%
\begin{pgfscope}%
\pgfsys@transformshift{4.411716in}{2.020055in}%
\pgfsys@useobject{currentmarker}{}%
\end{pgfscope}%
\end{pgfscope}%
\begin{pgfscope}%
\pgfpathrectangle{\pgfqpoint{3.095027in}{0.528177in}}{\pgfqpoint{2.106702in}{2.125222in}} %
\pgfusepath{clip}%
\pgfsetbuttcap%
\pgfsetroundjoin%
\definecolor{currentfill}{rgb}{1.000000,0.694118,0.250980}%
\pgfsetfillcolor{currentfill}%
\pgfsetlinewidth{1.505625pt}%
\definecolor{currentstroke}{rgb}{1.000000,0.694118,0.250980}%
\pgfsetstrokecolor{currentstroke}%
\pgfsetdash{}{0pt}%
\pgfsys@defobject{currentmarker}{\pgfqpoint{-0.111111in}{-0.000000in}}{\pgfqpoint{0.111111in}{0.000000in}}{%
\pgfpathmoveto{\pgfqpoint{0.111111in}{-0.000000in}}%
\pgfpathlineto{\pgfqpoint{-0.111111in}{0.000000in}}%
\pgfusepath{stroke,fill}%
}%
\begin{pgfscope}%
\pgfsys@transformshift{4.938391in}{1.443146in}%
\pgfsys@useobject{currentmarker}{}%
\end{pgfscope}%
\end{pgfscope}%
\begin{pgfscope}%
\pgfpathrectangle{\pgfqpoint{3.095027in}{0.528177in}}{\pgfqpoint{2.106702in}{2.125222in}} %
\pgfusepath{clip}%
\pgfsetbuttcap%
\pgfsetroundjoin%
\definecolor{currentfill}{rgb}{1.000000,0.694118,0.250980}%
\pgfsetfillcolor{currentfill}%
\pgfsetlinewidth{1.505625pt}%
\definecolor{currentstroke}{rgb}{1.000000,0.694118,0.250980}%
\pgfsetstrokecolor{currentstroke}%
\pgfsetdash{}{0pt}%
\pgfsys@defobject{currentmarker}{\pgfqpoint{-0.111111in}{-0.000000in}}{\pgfqpoint{0.111111in}{0.000000in}}{%
\pgfpathmoveto{\pgfqpoint{0.111111in}{-0.000000in}}%
\pgfpathlineto{\pgfqpoint{-0.111111in}{0.000000in}}%
\pgfusepath{stroke,fill}%
}%
\begin{pgfscope}%
\pgfsys@transformshift{4.938391in}{1.475526in}%
\pgfsys@useobject{currentmarker}{}%
\end{pgfscope}%
\end{pgfscope}%
\begin{pgfscope}%
\pgfpathrectangle{\pgfqpoint{3.095027in}{0.528177in}}{\pgfqpoint{2.106702in}{2.125222in}} %
\pgfusepath{clip}%
\pgfsetroundcap%
\pgfsetroundjoin%
\pgfsetlinewidth{1.756562pt}%
\definecolor{currentstroke}{rgb}{0.627451,0.627451,0.643137}%
\pgfsetstrokecolor{currentstroke}%
\pgfsetdash{}{0pt}%
\pgfpathmoveto{\pgfqpoint{3.358365in}{1.716193in}}%
\pgfpathlineto{\pgfqpoint{3.358365in}{2.220055in}}%
\pgfusepath{stroke}%
\end{pgfscope}%
\begin{pgfscope}%
\pgfpathrectangle{\pgfqpoint{3.095027in}{0.528177in}}{\pgfqpoint{2.106702in}{2.125222in}} %
\pgfusepath{clip}%
\pgfsetroundcap%
\pgfsetroundjoin%
\pgfsetlinewidth{1.756562pt}%
\definecolor{currentstroke}{rgb}{0.627451,0.627451,0.643137}%
\pgfsetstrokecolor{currentstroke}%
\pgfsetdash{}{0pt}%
\pgfpathmoveto{\pgfqpoint{3.358365in}{2.220055in}}%
\pgfpathlineto{\pgfqpoint{4.411716in}{2.220055in}}%
\pgfusepath{stroke}%
\end{pgfscope}%
\begin{pgfscope}%
\pgfpathrectangle{\pgfqpoint{3.095027in}{0.528177in}}{\pgfqpoint{2.106702in}{2.125222in}} %
\pgfusepath{clip}%
\pgfsetroundcap%
\pgfsetroundjoin%
\pgfsetlinewidth{1.756562pt}%
\definecolor{currentstroke}{rgb}{0.627451,0.627451,0.643137}%
\pgfsetstrokecolor{currentstroke}%
\pgfsetdash{}{0pt}%
\pgfpathmoveto{\pgfqpoint{4.411716in}{2.220055in}}%
\pgfpathlineto{\pgfqpoint{4.411716in}{2.170055in}}%
\pgfusepath{stroke}%
\end{pgfscope}%
\begin{pgfscope}%
\pgfpathrectangle{\pgfqpoint{3.095027in}{0.528177in}}{\pgfqpoint{2.106702in}{2.125222in}} %
\pgfusepath{clip}%
\pgfsetroundcap%
\pgfsetroundjoin%
\pgfsetlinewidth{1.756562pt}%
\definecolor{currentstroke}{rgb}{0.627451,0.627451,0.643137}%
\pgfsetstrokecolor{currentstroke}%
\pgfsetdash{}{0pt}%
\pgfpathmoveto{\pgfqpoint{4.411716in}{2.295055in}}%
\pgfpathlineto{\pgfqpoint{4.411716in}{2.420055in}}%
\pgfusepath{stroke}%
\end{pgfscope}%
\begin{pgfscope}%
\pgfpathrectangle{\pgfqpoint{3.095027in}{0.528177in}}{\pgfqpoint{2.106702in}{2.125222in}} %
\pgfusepath{clip}%
\pgfsetroundcap%
\pgfsetroundjoin%
\pgfsetlinewidth{1.756562pt}%
\definecolor{currentstroke}{rgb}{0.627451,0.627451,0.643137}%
\pgfsetstrokecolor{currentstroke}%
\pgfsetdash{}{0pt}%
\pgfpathmoveto{\pgfqpoint{4.411716in}{2.420055in}}%
\pgfpathlineto{\pgfqpoint{4.938391in}{2.420055in}}%
\pgfusepath{stroke}%
\end{pgfscope}%
\begin{pgfscope}%
\pgfpathrectangle{\pgfqpoint{3.095027in}{0.528177in}}{\pgfqpoint{2.106702in}{2.125222in}} %
\pgfusepath{clip}%
\pgfsetroundcap%
\pgfsetroundjoin%
\pgfsetlinewidth{1.756562pt}%
\definecolor{currentstroke}{rgb}{0.627451,0.627451,0.643137}%
\pgfsetstrokecolor{currentstroke}%
\pgfsetdash{}{0pt}%
\pgfpathmoveto{\pgfqpoint{4.938391in}{2.420055in}}%
\pgfpathlineto{\pgfqpoint{4.938391in}{1.625526in}}%
\pgfusepath{stroke}%
\end{pgfscope}%
\begin{pgfscope}%
\pgfsetrectcap%
\pgfsetmiterjoin%
\pgfsetlinewidth{1.254687pt}%
\definecolor{currentstroke}{rgb}{0.150000,0.150000,0.150000}%
\pgfsetstrokecolor{currentstroke}%
\pgfsetdash{}{0pt}%
\pgfpathmoveto{\pgfqpoint{3.095027in}{0.528177in}}%
\pgfpathlineto{\pgfqpoint{3.095027in}{2.653399in}}%
\pgfusepath{stroke}%
\end{pgfscope}%
\begin{pgfscope}%
\pgfsetrectcap%
\pgfsetmiterjoin%
\pgfsetlinewidth{1.254687pt}%
\definecolor{currentstroke}{rgb}{0.150000,0.150000,0.150000}%
\pgfsetstrokecolor{currentstroke}%
\pgfsetdash{}{0pt}%
\pgfpathmoveto{\pgfqpoint{3.095027in}{0.528177in}}%
\pgfpathlineto{\pgfqpoint{5.201729in}{0.528177in}}%
\pgfusepath{stroke}%
\end{pgfscope}%
\begin{pgfscope}%
\definecolor{textcolor}{rgb}{0.150000,0.150000,0.150000}%
\pgfsetstrokecolor{textcolor}%
\pgfsetfillcolor{textcolor}%
\pgftext[x=4.411716in,y=2.066930in,,]{\color{textcolor}\rmfamily\fontsize{15.000000}{18.000000}\selectfont \textbf{*}}%
\end{pgfscope}%
\begin{pgfscope}%
\definecolor{textcolor}{rgb}{0.150000,0.150000,0.150000}%
\pgfsetstrokecolor{textcolor}%
\pgfsetfillcolor{textcolor}%
\pgftext[x=4.938391in,y=1.522401in,,]{\color{textcolor}\rmfamily\fontsize{15.000000}{18.000000}\selectfont \textbf{*}}%
\end{pgfscope}%
\begin{pgfscope}%
\pgfsetbuttcap%
\pgfsetmiterjoin%
\definecolor{currentfill}{rgb}{0.200000,0.427451,0.650980}%
\pgfsetfillcolor{currentfill}%
\pgfsetlinewidth{1.505625pt}%
\definecolor{currentstroke}{rgb}{0.200000,0.427451,0.650980}%
\pgfsetstrokecolor{currentstroke}%
\pgfsetdash{}{0pt}%
\pgfpathmoveto{\pgfqpoint{3.195027in}{3.281088in}}%
\pgfpathlineto{\pgfqpoint{3.306138in}{3.281088in}}%
\pgfpathlineto{\pgfqpoint{3.306138in}{3.358866in}}%
\pgfpathlineto{\pgfqpoint{3.195027in}{3.358866in}}%
\pgfpathclose%
\pgfusepath{stroke,fill}%
\end{pgfscope}%
\begin{pgfscope}%
\definecolor{textcolor}{rgb}{0.150000,0.150000,0.150000}%
\pgfsetstrokecolor{textcolor}%
\pgfsetfillcolor{textcolor}%
\pgftext[x=3.395027in,y=3.281088in,left,base]{\color{textcolor}\rmfamily\fontsize{8.000000}{9.600000}\selectfont WT + Vehicle (1129)}%
\end{pgfscope}%
\begin{pgfscope}%
\pgfsetbuttcap%
\pgfsetmiterjoin%
\definecolor{currentfill}{rgb}{0.168627,0.670588,0.494118}%
\pgfsetfillcolor{currentfill}%
\pgfsetlinewidth{1.505625pt}%
\definecolor{currentstroke}{rgb}{0.168627,0.670588,0.494118}%
\pgfsetstrokecolor{currentstroke}%
\pgfsetdash{}{0pt}%
\pgfpathmoveto{\pgfqpoint{3.195027in}{3.114449in}}%
\pgfpathlineto{\pgfqpoint{3.306138in}{3.114449in}}%
\pgfpathlineto{\pgfqpoint{3.306138in}{3.192227in}}%
\pgfpathlineto{\pgfqpoint{3.195027in}{3.192227in}}%
\pgfpathclose%
\pgfusepath{stroke,fill}%
\end{pgfscope}%
\begin{pgfscope}%
\definecolor{textcolor}{rgb}{0.150000,0.150000,0.150000}%
\pgfsetstrokecolor{textcolor}%
\pgfsetfillcolor{textcolor}%
\pgftext[x=3.395027in,y=3.114449in,left,base]{\color{textcolor}\rmfamily\fontsize{8.000000}{9.600000}\selectfont WT + TAT-GluA2\textsubscript{3Y} (512)}%
\end{pgfscope}%
\begin{pgfscope}%
\pgfsetbuttcap%
\pgfsetmiterjoin%
\definecolor{currentfill}{rgb}{1.000000,0.494118,0.250980}%
\pgfsetfillcolor{currentfill}%
\pgfsetlinewidth{1.505625pt}%
\definecolor{currentstroke}{rgb}{1.000000,0.494118,0.250980}%
\pgfsetstrokecolor{currentstroke}%
\pgfsetdash{}{0pt}%
\pgfpathmoveto{\pgfqpoint{3.195027in}{2.947809in}}%
\pgfpathlineto{\pgfqpoint{3.306138in}{2.947809in}}%
\pgfpathlineto{\pgfqpoint{3.306138in}{3.025587in}}%
\pgfpathlineto{\pgfqpoint{3.195027in}{3.025587in}}%
\pgfpathclose%
\pgfusepath{stroke,fill}%
\end{pgfscope}%
\begin{pgfscope}%
\definecolor{textcolor}{rgb}{0.150000,0.150000,0.150000}%
\pgfsetstrokecolor{textcolor}%
\pgfsetfillcolor{textcolor}%
\pgftext[x=3.395027in,y=2.947809in,left,base]{\color{textcolor}\rmfamily\fontsize{8.000000}{9.600000}\selectfont Tg + Vehicle (638)}%
\end{pgfscope}%
\begin{pgfscope}%
\pgfsetbuttcap%
\pgfsetmiterjoin%
\definecolor{currentfill}{rgb}{1.000000,0.694118,0.250980}%
\pgfsetfillcolor{currentfill}%
\pgfsetlinewidth{1.505625pt}%
\definecolor{currentstroke}{rgb}{1.000000,0.694118,0.250980}%
\pgfsetstrokecolor{currentstroke}%
\pgfsetdash{}{0pt}%
\pgfpathmoveto{\pgfqpoint{3.195027in}{2.781170in}}%
\pgfpathlineto{\pgfqpoint{3.306138in}{2.781170in}}%
\pgfpathlineto{\pgfqpoint{3.306138in}{2.858947in}}%
\pgfpathlineto{\pgfqpoint{3.195027in}{2.858947in}}%
\pgfpathclose%
\pgfusepath{stroke,fill}%
\end{pgfscope}%
\begin{pgfscope}%
\definecolor{textcolor}{rgb}{0.150000,0.150000,0.150000}%
\pgfsetstrokecolor{textcolor}%
\pgfsetfillcolor{textcolor}%
\pgftext[x=3.395027in,y=2.781170in,left,base]{\color{textcolor}\rmfamily\fontsize{8.000000}{9.600000}\selectfont Tg + TAT-GluA2\textsubscript{3Y} (759)}%
\end{pgfscope}%
\end{pgfpicture}%
\makeatother%
\endgroup%

        \caption{\label{f.ad.acttest}}
    \end{subfigure}
    \caption[Cell activity during training and memory test.]{Distribution and mean of average cell activity during \subref{f.ad.acttrain} training before foot shock and \subref{f.ad.acttest} memory test. Cells in the Tg mice are significantly more active, and this is rescued by \tglu{} treatment. Cell numbers are listed in the legend with parenthesis. The cell activity are measured in \gls{au}. \label{f.ad.activity}}
\end{figure}

\begin{figure}[h]
    %% Creator: Matplotlib, PGF backend
%%
%% To include the figure in your LaTeX document, write
%%   \input{<filename>.pgf}
%%
%% Make sure the required packages are loaded in your preamble
%%   \usepackage{pgf}
%%
%% Figures using additional raster images can only be included by \input if
%% they are in the same directory as the main LaTeX file. For loading figures
%% from other directories you can use the `import` package
%%   \usepackage{import}
%% and then include the figures with
%%   \import{<path to file>}{<filename>.pgf}
%%
%% Matplotlib used the following preamble
%%   \usepackage[utf8]{inputenc}
%%   \usepackage[T1]{fontenc}
%%   \usepackage{siunitx}
%%
\begingroup%
\makeatletter%
\begin{pgfpicture}%
\pgfpathrectangle{\pgfpointorigin}{\pgfqpoint{5.301729in}{3.689896in}}%
\pgfusepath{use as bounding box, clip}%
\begin{pgfscope}%
\pgfsetbuttcap%
\pgfsetmiterjoin%
\definecolor{currentfill}{rgb}{1.000000,1.000000,1.000000}%
\pgfsetfillcolor{currentfill}%
\pgfsetlinewidth{0.000000pt}%
\definecolor{currentstroke}{rgb}{1.000000,1.000000,1.000000}%
\pgfsetstrokecolor{currentstroke}%
\pgfsetdash{}{0pt}%
\pgfpathmoveto{\pgfqpoint{0.000000in}{0.000000in}}%
\pgfpathlineto{\pgfqpoint{5.301729in}{0.000000in}}%
\pgfpathlineto{\pgfqpoint{5.301729in}{3.689896in}}%
\pgfpathlineto{\pgfqpoint{0.000000in}{3.689896in}}%
\pgfpathclose%
\pgfusepath{fill}%
\end{pgfscope}%
\begin{pgfscope}%
\pgfsetbuttcap%
\pgfsetmiterjoin%
\definecolor{currentfill}{rgb}{1.000000,1.000000,1.000000}%
\pgfsetfillcolor{currentfill}%
\pgfsetlinewidth{0.000000pt}%
\definecolor{currentstroke}{rgb}{0.000000,0.000000,0.000000}%
\pgfsetstrokecolor{currentstroke}%
\pgfsetstrokeopacity{0.000000}%
\pgfsetdash{}{0pt}%
\pgfpathmoveto{\pgfqpoint{0.566985in}{0.664139in}}%
\pgfpathlineto{\pgfqpoint{2.582091in}{0.664139in}}%
\pgfpathlineto{\pgfqpoint{2.582091in}{3.528569in}}%
\pgfpathlineto{\pgfqpoint{0.566985in}{3.528569in}}%
\pgfpathclose%
\pgfusepath{fill}%
\end{pgfscope}%
\begin{pgfscope}%
\pgfsetbuttcap%
\pgfsetroundjoin%
\definecolor{currentfill}{rgb}{0.150000,0.150000,0.150000}%
\pgfsetfillcolor{currentfill}%
\pgfsetlinewidth{1.003750pt}%
\definecolor{currentstroke}{rgb}{0.150000,0.150000,0.150000}%
\pgfsetstrokecolor{currentstroke}%
\pgfsetdash{}{0pt}%
\pgfsys@defobject{currentmarker}{\pgfqpoint{0.000000in}{0.000000in}}{\pgfqpoint{0.000000in}{0.041667in}}{%
\pgfpathmoveto{\pgfqpoint{0.000000in}{0.000000in}}%
\pgfpathlineto{\pgfqpoint{0.000000in}{0.041667in}}%
\pgfusepath{stroke,fill}%
}%
\begin{pgfscope}%
\pgfsys@transformshift{0.566985in}{0.664139in}%
\pgfsys@useobject{currentmarker}{}%
\end{pgfscope}%
\end{pgfscope}%
\begin{pgfscope}%
\definecolor{textcolor}{rgb}{0.150000,0.150000,0.150000}%
\pgfsetstrokecolor{textcolor}%
\pgfsetfillcolor{textcolor}%
\pgftext[x=0.566985in,y=0.566917in,,top]{\color{textcolor}\rmfamily\fontsize{10.000000}{12.000000}\selectfont \(\displaystyle -1.5\)}%
\end{pgfscope}%
\begin{pgfscope}%
\pgfsetbuttcap%
\pgfsetroundjoin%
\definecolor{currentfill}{rgb}{0.150000,0.150000,0.150000}%
\pgfsetfillcolor{currentfill}%
\pgfsetlinewidth{1.003750pt}%
\definecolor{currentstroke}{rgb}{0.150000,0.150000,0.150000}%
\pgfsetstrokecolor{currentstroke}%
\pgfsetdash{}{0pt}%
\pgfsys@defobject{currentmarker}{\pgfqpoint{0.000000in}{0.000000in}}{\pgfqpoint{0.000000in}{0.041667in}}{%
\pgfpathmoveto{\pgfqpoint{0.000000in}{0.000000in}}%
\pgfpathlineto{\pgfqpoint{0.000000in}{0.041667in}}%
\pgfusepath{stroke,fill}%
}%
\begin{pgfscope}%
\pgfsys@transformshift{0.902836in}{0.664139in}%
\pgfsys@useobject{currentmarker}{}%
\end{pgfscope}%
\end{pgfscope}%
\begin{pgfscope}%
\definecolor{textcolor}{rgb}{0.150000,0.150000,0.150000}%
\pgfsetstrokecolor{textcolor}%
\pgfsetfillcolor{textcolor}%
\pgftext[x=0.902836in,y=0.566917in,,top]{\color{textcolor}\rmfamily\fontsize{10.000000}{12.000000}\selectfont \(\displaystyle -1.0\)}%
\end{pgfscope}%
\begin{pgfscope}%
\pgfsetbuttcap%
\pgfsetroundjoin%
\definecolor{currentfill}{rgb}{0.150000,0.150000,0.150000}%
\pgfsetfillcolor{currentfill}%
\pgfsetlinewidth{1.003750pt}%
\definecolor{currentstroke}{rgb}{0.150000,0.150000,0.150000}%
\pgfsetstrokecolor{currentstroke}%
\pgfsetdash{}{0pt}%
\pgfsys@defobject{currentmarker}{\pgfqpoint{0.000000in}{0.000000in}}{\pgfqpoint{0.000000in}{0.041667in}}{%
\pgfpathmoveto{\pgfqpoint{0.000000in}{0.000000in}}%
\pgfpathlineto{\pgfqpoint{0.000000in}{0.041667in}}%
\pgfusepath{stroke,fill}%
}%
\begin{pgfscope}%
\pgfsys@transformshift{1.238687in}{0.664139in}%
\pgfsys@useobject{currentmarker}{}%
\end{pgfscope}%
\end{pgfscope}%
\begin{pgfscope}%
\definecolor{textcolor}{rgb}{0.150000,0.150000,0.150000}%
\pgfsetstrokecolor{textcolor}%
\pgfsetfillcolor{textcolor}%
\pgftext[x=1.238687in,y=0.566917in,,top]{\color{textcolor}\rmfamily\fontsize{10.000000}{12.000000}\selectfont \(\displaystyle -0.5\)}%
\end{pgfscope}%
\begin{pgfscope}%
\pgfsetbuttcap%
\pgfsetroundjoin%
\definecolor{currentfill}{rgb}{0.150000,0.150000,0.150000}%
\pgfsetfillcolor{currentfill}%
\pgfsetlinewidth{1.003750pt}%
\definecolor{currentstroke}{rgb}{0.150000,0.150000,0.150000}%
\pgfsetstrokecolor{currentstroke}%
\pgfsetdash{}{0pt}%
\pgfsys@defobject{currentmarker}{\pgfqpoint{0.000000in}{0.000000in}}{\pgfqpoint{0.000000in}{0.041667in}}{%
\pgfpathmoveto{\pgfqpoint{0.000000in}{0.000000in}}%
\pgfpathlineto{\pgfqpoint{0.000000in}{0.041667in}}%
\pgfusepath{stroke,fill}%
}%
\begin{pgfscope}%
\pgfsys@transformshift{1.574538in}{0.664139in}%
\pgfsys@useobject{currentmarker}{}%
\end{pgfscope}%
\end{pgfscope}%
\begin{pgfscope}%
\definecolor{textcolor}{rgb}{0.150000,0.150000,0.150000}%
\pgfsetstrokecolor{textcolor}%
\pgfsetfillcolor{textcolor}%
\pgftext[x=1.574538in,y=0.566917in,,top]{\color{textcolor}\rmfamily\fontsize{10.000000}{12.000000}\selectfont \(\displaystyle 0.0\)}%
\end{pgfscope}%
\begin{pgfscope}%
\pgfsetbuttcap%
\pgfsetroundjoin%
\definecolor{currentfill}{rgb}{0.150000,0.150000,0.150000}%
\pgfsetfillcolor{currentfill}%
\pgfsetlinewidth{1.003750pt}%
\definecolor{currentstroke}{rgb}{0.150000,0.150000,0.150000}%
\pgfsetstrokecolor{currentstroke}%
\pgfsetdash{}{0pt}%
\pgfsys@defobject{currentmarker}{\pgfqpoint{0.000000in}{0.000000in}}{\pgfqpoint{0.000000in}{0.041667in}}{%
\pgfpathmoveto{\pgfqpoint{0.000000in}{0.000000in}}%
\pgfpathlineto{\pgfqpoint{0.000000in}{0.041667in}}%
\pgfusepath{stroke,fill}%
}%
\begin{pgfscope}%
\pgfsys@transformshift{1.910389in}{0.664139in}%
\pgfsys@useobject{currentmarker}{}%
\end{pgfscope}%
\end{pgfscope}%
\begin{pgfscope}%
\definecolor{textcolor}{rgb}{0.150000,0.150000,0.150000}%
\pgfsetstrokecolor{textcolor}%
\pgfsetfillcolor{textcolor}%
\pgftext[x=1.910389in,y=0.566917in,,top]{\color{textcolor}\rmfamily\fontsize{10.000000}{12.000000}\selectfont \(\displaystyle 0.5\)}%
\end{pgfscope}%
\begin{pgfscope}%
\pgfsetbuttcap%
\pgfsetroundjoin%
\definecolor{currentfill}{rgb}{0.150000,0.150000,0.150000}%
\pgfsetfillcolor{currentfill}%
\pgfsetlinewidth{1.003750pt}%
\definecolor{currentstroke}{rgb}{0.150000,0.150000,0.150000}%
\pgfsetstrokecolor{currentstroke}%
\pgfsetdash{}{0pt}%
\pgfsys@defobject{currentmarker}{\pgfqpoint{0.000000in}{0.000000in}}{\pgfqpoint{0.000000in}{0.041667in}}{%
\pgfpathmoveto{\pgfqpoint{0.000000in}{0.000000in}}%
\pgfpathlineto{\pgfqpoint{0.000000in}{0.041667in}}%
\pgfusepath{stroke,fill}%
}%
\begin{pgfscope}%
\pgfsys@transformshift{2.246240in}{0.664139in}%
\pgfsys@useobject{currentmarker}{}%
\end{pgfscope}%
\end{pgfscope}%
\begin{pgfscope}%
\definecolor{textcolor}{rgb}{0.150000,0.150000,0.150000}%
\pgfsetstrokecolor{textcolor}%
\pgfsetfillcolor{textcolor}%
\pgftext[x=2.246240in,y=0.566917in,,top]{\color{textcolor}\rmfamily\fontsize{10.000000}{12.000000}\selectfont \(\displaystyle 1.0\)}%
\end{pgfscope}%
\begin{pgfscope}%
\pgfsetbuttcap%
\pgfsetroundjoin%
\definecolor{currentfill}{rgb}{0.150000,0.150000,0.150000}%
\pgfsetfillcolor{currentfill}%
\pgfsetlinewidth{1.003750pt}%
\definecolor{currentstroke}{rgb}{0.150000,0.150000,0.150000}%
\pgfsetstrokecolor{currentstroke}%
\pgfsetdash{}{0pt}%
\pgfsys@defobject{currentmarker}{\pgfqpoint{0.000000in}{0.000000in}}{\pgfqpoint{0.000000in}{0.041667in}}{%
\pgfpathmoveto{\pgfqpoint{0.000000in}{0.000000in}}%
\pgfpathlineto{\pgfqpoint{0.000000in}{0.041667in}}%
\pgfusepath{stroke,fill}%
}%
\begin{pgfscope}%
\pgfsys@transformshift{2.582091in}{0.664139in}%
\pgfsys@useobject{currentmarker}{}%
\end{pgfscope}%
\end{pgfscope}%
\begin{pgfscope}%
\definecolor{textcolor}{rgb}{0.150000,0.150000,0.150000}%
\pgfsetstrokecolor{textcolor}%
\pgfsetfillcolor{textcolor}%
\pgftext[x=2.582091in,y=0.566917in,,top]{\color{textcolor}\rmfamily\fontsize{10.000000}{12.000000}\selectfont \(\displaystyle 1.5\)}%
\end{pgfscope}%
\begin{pgfscope}%
\definecolor{textcolor}{rgb}{0.150000,0.150000,0.150000}%
\pgfsetstrokecolor{textcolor}%
\pgfsetfillcolor{textcolor}%
\pgftext[x=0.695316in,y=0.286683in,left,base]{\color{textcolor}\rmfamily\fontsize{10.000000}{12.000000}\selectfont \textbf{Difference in cell activity}}%
\end{pgfscope}%
\begin{pgfscope}%
\definecolor{textcolor}{rgb}{0.150000,0.150000,0.150000}%
\pgfsetstrokecolor{textcolor}%
\pgfsetfillcolor{textcolor}%
\pgftext[x=1.384914in,y=0.134714in,left,base]{\color{textcolor}\rmfamily\fontsize{10.000000}{12.000000}\selectfont \textbf{(a.u.)}}%
\end{pgfscope}%
\begin{pgfscope}%
\pgfsetbuttcap%
\pgfsetroundjoin%
\definecolor{currentfill}{rgb}{0.150000,0.150000,0.150000}%
\pgfsetfillcolor{currentfill}%
\pgfsetlinewidth{1.003750pt}%
\definecolor{currentstroke}{rgb}{0.150000,0.150000,0.150000}%
\pgfsetstrokecolor{currentstroke}%
\pgfsetdash{}{0pt}%
\pgfsys@defobject{currentmarker}{\pgfqpoint{0.000000in}{0.000000in}}{\pgfqpoint{0.041667in}{0.000000in}}{%
\pgfpathmoveto{\pgfqpoint{0.000000in}{0.000000in}}%
\pgfpathlineto{\pgfqpoint{0.041667in}{0.000000in}}%
\pgfusepath{stroke,fill}%
}%
\begin{pgfscope}%
\pgfsys@transformshift{0.566985in}{0.664139in}%
\pgfsys@useobject{currentmarker}{}%
\end{pgfscope}%
\end{pgfscope}%
\begin{pgfscope}%
\definecolor{textcolor}{rgb}{0.150000,0.150000,0.150000}%
\pgfsetstrokecolor{textcolor}%
\pgfsetfillcolor{textcolor}%
\pgftext[x=0.469762in,y=0.664139in,right,]{\color{textcolor}\rmfamily\fontsize{10.000000}{12.000000}\selectfont \(\displaystyle 0.0\)}%
\end{pgfscope}%
\begin{pgfscope}%
\pgfsetbuttcap%
\pgfsetroundjoin%
\definecolor{currentfill}{rgb}{0.150000,0.150000,0.150000}%
\pgfsetfillcolor{currentfill}%
\pgfsetlinewidth{1.003750pt}%
\definecolor{currentstroke}{rgb}{0.150000,0.150000,0.150000}%
\pgfsetstrokecolor{currentstroke}%
\pgfsetdash{}{0pt}%
\pgfsys@defobject{currentmarker}{\pgfqpoint{0.000000in}{0.000000in}}{\pgfqpoint{0.041667in}{0.000000in}}{%
\pgfpathmoveto{\pgfqpoint{0.000000in}{0.000000in}}%
\pgfpathlineto{\pgfqpoint{0.041667in}{0.000000in}}%
\pgfusepath{stroke,fill}%
}%
\begin{pgfscope}%
\pgfsys@transformshift{0.566985in}{1.237025in}%
\pgfsys@useobject{currentmarker}{}%
\end{pgfscope}%
\end{pgfscope}%
\begin{pgfscope}%
\definecolor{textcolor}{rgb}{0.150000,0.150000,0.150000}%
\pgfsetstrokecolor{textcolor}%
\pgfsetfillcolor{textcolor}%
\pgftext[x=0.469762in,y=1.237025in,right,]{\color{textcolor}\rmfamily\fontsize{10.000000}{12.000000}\selectfont \(\displaystyle 0.2\)}%
\end{pgfscope}%
\begin{pgfscope}%
\pgfsetbuttcap%
\pgfsetroundjoin%
\definecolor{currentfill}{rgb}{0.150000,0.150000,0.150000}%
\pgfsetfillcolor{currentfill}%
\pgfsetlinewidth{1.003750pt}%
\definecolor{currentstroke}{rgb}{0.150000,0.150000,0.150000}%
\pgfsetstrokecolor{currentstroke}%
\pgfsetdash{}{0pt}%
\pgfsys@defobject{currentmarker}{\pgfqpoint{0.000000in}{0.000000in}}{\pgfqpoint{0.041667in}{0.000000in}}{%
\pgfpathmoveto{\pgfqpoint{0.000000in}{0.000000in}}%
\pgfpathlineto{\pgfqpoint{0.041667in}{0.000000in}}%
\pgfusepath{stroke,fill}%
}%
\begin{pgfscope}%
\pgfsys@transformshift{0.566985in}{1.809911in}%
\pgfsys@useobject{currentmarker}{}%
\end{pgfscope}%
\end{pgfscope}%
\begin{pgfscope}%
\definecolor{textcolor}{rgb}{0.150000,0.150000,0.150000}%
\pgfsetstrokecolor{textcolor}%
\pgfsetfillcolor{textcolor}%
\pgftext[x=0.469762in,y=1.809911in,right,]{\color{textcolor}\rmfamily\fontsize{10.000000}{12.000000}\selectfont \(\displaystyle 0.4\)}%
\end{pgfscope}%
\begin{pgfscope}%
\pgfsetbuttcap%
\pgfsetroundjoin%
\definecolor{currentfill}{rgb}{0.150000,0.150000,0.150000}%
\pgfsetfillcolor{currentfill}%
\pgfsetlinewidth{1.003750pt}%
\definecolor{currentstroke}{rgb}{0.150000,0.150000,0.150000}%
\pgfsetstrokecolor{currentstroke}%
\pgfsetdash{}{0pt}%
\pgfsys@defobject{currentmarker}{\pgfqpoint{0.000000in}{0.000000in}}{\pgfqpoint{0.041667in}{0.000000in}}{%
\pgfpathmoveto{\pgfqpoint{0.000000in}{0.000000in}}%
\pgfpathlineto{\pgfqpoint{0.041667in}{0.000000in}}%
\pgfusepath{stroke,fill}%
}%
\begin{pgfscope}%
\pgfsys@transformshift{0.566985in}{2.382797in}%
\pgfsys@useobject{currentmarker}{}%
\end{pgfscope}%
\end{pgfscope}%
\begin{pgfscope}%
\definecolor{textcolor}{rgb}{0.150000,0.150000,0.150000}%
\pgfsetstrokecolor{textcolor}%
\pgfsetfillcolor{textcolor}%
\pgftext[x=0.469762in,y=2.382797in,right,]{\color{textcolor}\rmfamily\fontsize{10.000000}{12.000000}\selectfont \(\displaystyle 0.6\)}%
\end{pgfscope}%
\begin{pgfscope}%
\pgfsetbuttcap%
\pgfsetroundjoin%
\definecolor{currentfill}{rgb}{0.150000,0.150000,0.150000}%
\pgfsetfillcolor{currentfill}%
\pgfsetlinewidth{1.003750pt}%
\definecolor{currentstroke}{rgb}{0.150000,0.150000,0.150000}%
\pgfsetstrokecolor{currentstroke}%
\pgfsetdash{}{0pt}%
\pgfsys@defobject{currentmarker}{\pgfqpoint{0.000000in}{0.000000in}}{\pgfqpoint{0.041667in}{0.000000in}}{%
\pgfpathmoveto{\pgfqpoint{0.000000in}{0.000000in}}%
\pgfpathlineto{\pgfqpoint{0.041667in}{0.000000in}}%
\pgfusepath{stroke,fill}%
}%
\begin{pgfscope}%
\pgfsys@transformshift{0.566985in}{2.955683in}%
\pgfsys@useobject{currentmarker}{}%
\end{pgfscope}%
\end{pgfscope}%
\begin{pgfscope}%
\definecolor{textcolor}{rgb}{0.150000,0.150000,0.150000}%
\pgfsetstrokecolor{textcolor}%
\pgfsetfillcolor{textcolor}%
\pgftext[x=0.469762in,y=2.955683in,right,]{\color{textcolor}\rmfamily\fontsize{10.000000}{12.000000}\selectfont \(\displaystyle 0.8\)}%
\end{pgfscope}%
\begin{pgfscope}%
\pgfsetbuttcap%
\pgfsetroundjoin%
\definecolor{currentfill}{rgb}{0.150000,0.150000,0.150000}%
\pgfsetfillcolor{currentfill}%
\pgfsetlinewidth{1.003750pt}%
\definecolor{currentstroke}{rgb}{0.150000,0.150000,0.150000}%
\pgfsetstrokecolor{currentstroke}%
\pgfsetdash{}{0pt}%
\pgfsys@defobject{currentmarker}{\pgfqpoint{0.000000in}{0.000000in}}{\pgfqpoint{0.041667in}{0.000000in}}{%
\pgfpathmoveto{\pgfqpoint{0.000000in}{0.000000in}}%
\pgfpathlineto{\pgfqpoint{0.041667in}{0.000000in}}%
\pgfusepath{stroke,fill}%
}%
\begin{pgfscope}%
\pgfsys@transformshift{0.566985in}{3.528569in}%
\pgfsys@useobject{currentmarker}{}%
\end{pgfscope}%
\end{pgfscope}%
\begin{pgfscope}%
\definecolor{textcolor}{rgb}{0.150000,0.150000,0.150000}%
\pgfsetstrokecolor{textcolor}%
\pgfsetfillcolor{textcolor}%
\pgftext[x=0.469762in,y=3.528569in,right,]{\color{textcolor}\rmfamily\fontsize{10.000000}{12.000000}\selectfont \(\displaystyle 1.0\)}%
\end{pgfscope}%
\begin{pgfscope}%
\definecolor{textcolor}{rgb}{0.150000,0.150000,0.150000}%
\pgfsetstrokecolor{textcolor}%
\pgfsetfillcolor{textcolor}%
\pgftext[x=0.222848in,y=2.096354in,,bottom,rotate=90.000000]{\color{textcolor}\rmfamily\fontsize{10.000000}{12.000000}\selectfont \textbf{Cumulative porportion}}%
\end{pgfscope}%
\begin{pgfscope}%
\pgfpathrectangle{\pgfqpoint{0.566985in}{0.664139in}}{\pgfqpoint{2.015106in}{2.864429in}} %
\pgfusepath{clip}%
\pgfsetroundcap%
\pgfsetroundjoin%
\pgfsetlinewidth{1.003750pt}%
\definecolor{currentstroke}{rgb}{0.200000,0.427451,0.650980}%
\pgfsetstrokecolor{currentstroke}%
\pgfsetdash{}{0pt}%
\pgfpathmoveto{\pgfqpoint{1.014232in}{0.666032in}}%
\pgfpathlineto{\pgfqpoint{1.046086in}{0.666032in}}%
\pgfpathlineto{\pgfqpoint{1.077939in}{0.666032in}}%
\pgfpathlineto{\pgfqpoint{1.109793in}{0.669819in}}%
\pgfpathlineto{\pgfqpoint{1.141647in}{0.675498in}}%
\pgfpathlineto{\pgfqpoint{1.173501in}{0.679285in}}%
\pgfpathlineto{\pgfqpoint{1.205354in}{0.679285in}}%
\pgfpathlineto{\pgfqpoint{1.237208in}{0.688751in}}%
\pgfpathlineto{\pgfqpoint{1.269062in}{0.698217in}}%
\pgfpathlineto{\pgfqpoint{1.300916in}{0.709576in}}%
\pgfpathlineto{\pgfqpoint{1.332770in}{0.741761in}}%
\pgfpathlineto{\pgfqpoint{1.364623in}{0.766373in}}%
\pgfpathlineto{\pgfqpoint{1.396477in}{0.819382in}}%
\pgfpathlineto{\pgfqpoint{1.428331in}{0.895111in}}%
\pgfpathlineto{\pgfqpoint{1.460185in}{1.037102in}}%
\pgfpathlineto{\pgfqpoint{1.492038in}{1.215064in}}%
\pgfpathlineto{\pgfqpoint{1.523892in}{1.542589in}}%
\pgfpathlineto{\pgfqpoint{1.555746in}{2.553564in}}%
\pgfpathlineto{\pgfqpoint{1.587600in}{2.875410in}}%
\pgfpathlineto{\pgfqpoint{1.619453in}{3.091237in}}%
\pgfpathlineto{\pgfqpoint{1.651307in}{3.221868in}}%
\pgfpathlineto{\pgfqpoint{1.683161in}{3.325995in}}%
\pgfpathlineto{\pgfqpoint{1.715015in}{3.377112in}}%
\pgfpathlineto{\pgfqpoint{1.746869in}{3.422549in}}%
\pgfpathlineto{\pgfqpoint{1.778722in}{3.441481in}}%
\pgfpathlineto{\pgfqpoint{1.810576in}{3.458520in}}%
\pgfpathlineto{\pgfqpoint{1.842430in}{3.479345in}}%
\pgfpathlineto{\pgfqpoint{1.874284in}{3.486918in}}%
\pgfpathlineto{\pgfqpoint{1.906137in}{3.496384in}}%
\pgfpathlineto{\pgfqpoint{1.937991in}{3.505850in}}%
\pgfpathlineto{\pgfqpoint{1.969845in}{3.509636in}}%
\pgfpathlineto{\pgfqpoint{2.001699in}{3.513423in}}%
\pgfpathlineto{\pgfqpoint{2.033553in}{3.513423in}}%
\pgfpathlineto{\pgfqpoint{2.065406in}{3.515316in}}%
\pgfpathlineto{\pgfqpoint{2.097260in}{3.524782in}}%
\pgfpathlineto{\pgfqpoint{2.129114in}{3.524782in}}%
\pgfpathlineto{\pgfqpoint{2.160968in}{3.526675in}}%
\pgfpathlineto{\pgfqpoint{2.192821in}{3.526675in}}%
\pgfpathlineto{\pgfqpoint{2.224675in}{3.526675in}}%
\pgfpathlineto{\pgfqpoint{2.256529in}{3.526675in}}%
\pgfpathlineto{\pgfqpoint{2.288383in}{3.526675in}}%
\pgfpathlineto{\pgfqpoint{2.320237in}{3.526675in}}%
\pgfpathlineto{\pgfqpoint{2.352090in}{3.526675in}}%
\pgfpathlineto{\pgfqpoint{2.383944in}{3.526675in}}%
\pgfpathlineto{\pgfqpoint{2.415798in}{3.526675in}}%
\pgfpathlineto{\pgfqpoint{2.447652in}{3.526675in}}%
\pgfpathlineto{\pgfqpoint{2.479505in}{3.526675in}}%
\pgfpathlineto{\pgfqpoint{2.511359in}{3.526675in}}%
\pgfpathlineto{\pgfqpoint{2.543213in}{3.526675in}}%
\pgfpathlineto{\pgfqpoint{2.575067in}{3.528569in}}%
\pgfusepath{stroke}%
\end{pgfscope}%
\begin{pgfscope}%
\pgfpathrectangle{\pgfqpoint{0.566985in}{0.664139in}}{\pgfqpoint{2.015106in}{2.864429in}} %
\pgfusepath{clip}%
\pgfsetroundcap%
\pgfsetroundjoin%
\pgfsetlinewidth{1.003750pt}%
\definecolor{currentstroke}{rgb}{0.168627,0.670588,0.494118}%
\pgfsetstrokecolor{currentstroke}%
\pgfsetdash{}{0pt}%
\pgfpathmoveto{\pgfqpoint{0.701895in}{0.667654in}}%
\pgfpathlineto{\pgfqpoint{0.735843in}{0.667654in}}%
\pgfpathlineto{\pgfqpoint{0.769791in}{0.667654in}}%
\pgfpathlineto{\pgfqpoint{0.803739in}{0.671168in}}%
\pgfpathlineto{\pgfqpoint{0.837687in}{0.674683in}}%
\pgfpathlineto{\pgfqpoint{0.871635in}{0.674683in}}%
\pgfpathlineto{\pgfqpoint{0.905583in}{0.674683in}}%
\pgfpathlineto{\pgfqpoint{0.939531in}{0.674683in}}%
\pgfpathlineto{\pgfqpoint{0.973479in}{0.674683in}}%
\pgfpathlineto{\pgfqpoint{1.007427in}{0.674683in}}%
\pgfpathlineto{\pgfqpoint{1.041375in}{0.674683in}}%
\pgfpathlineto{\pgfqpoint{1.075323in}{0.678198in}}%
\pgfpathlineto{\pgfqpoint{1.109271in}{0.681712in}}%
\pgfpathlineto{\pgfqpoint{1.143219in}{0.685227in}}%
\pgfpathlineto{\pgfqpoint{1.177167in}{0.695771in}}%
\pgfpathlineto{\pgfqpoint{1.211116in}{0.702800in}}%
\pgfpathlineto{\pgfqpoint{1.245064in}{0.706315in}}%
\pgfpathlineto{\pgfqpoint{1.279012in}{0.709829in}}%
\pgfpathlineto{\pgfqpoint{1.312960in}{0.737946in}}%
\pgfpathlineto{\pgfqpoint{1.346908in}{0.776607in}}%
\pgfpathlineto{\pgfqpoint{1.380856in}{0.815268in}}%
\pgfpathlineto{\pgfqpoint{1.414804in}{0.903134in}}%
\pgfpathlineto{\pgfqpoint{1.448752in}{1.008574in}}%
\pgfpathlineto{\pgfqpoint{1.482700in}{1.142130in}}%
\pgfpathlineto{\pgfqpoint{1.516648in}{1.461962in}}%
\pgfpathlineto{\pgfqpoint{1.550596in}{2.551499in}}%
\pgfpathlineto{\pgfqpoint{1.584544in}{2.878361in}}%
\pgfpathlineto{\pgfqpoint{1.618492in}{3.050578in}}%
\pgfpathlineto{\pgfqpoint{1.652440in}{3.180619in}}%
\pgfpathlineto{\pgfqpoint{1.686388in}{3.272000in}}%
\pgfpathlineto{\pgfqpoint{1.720336in}{3.352837in}}%
\pgfpathlineto{\pgfqpoint{1.754284in}{3.377439in}}%
\pgfpathlineto{\pgfqpoint{1.788232in}{3.395012in}}%
\pgfpathlineto{\pgfqpoint{1.822180in}{3.426644in}}%
\pgfpathlineto{\pgfqpoint{1.856128in}{3.454761in}}%
\pgfpathlineto{\pgfqpoint{1.890076in}{3.461790in}}%
\pgfpathlineto{\pgfqpoint{1.924024in}{3.479364in}}%
\pgfpathlineto{\pgfqpoint{1.957972in}{3.493422in}}%
\pgfpathlineto{\pgfqpoint{1.991920in}{3.503966in}}%
\pgfpathlineto{\pgfqpoint{2.025868in}{3.507481in}}%
\pgfpathlineto{\pgfqpoint{2.059816in}{3.510995in}}%
\pgfpathlineto{\pgfqpoint{2.093764in}{3.510995in}}%
\pgfpathlineto{\pgfqpoint{2.127712in}{3.514510in}}%
\pgfpathlineto{\pgfqpoint{2.161660in}{3.514510in}}%
\pgfpathlineto{\pgfqpoint{2.195608in}{3.521539in}}%
\pgfpathlineto{\pgfqpoint{2.229556in}{3.521539in}}%
\pgfpathlineto{\pgfqpoint{2.263504in}{3.521539in}}%
\pgfpathlineto{\pgfqpoint{2.297453in}{3.521539in}}%
\pgfpathlineto{\pgfqpoint{2.331401in}{3.525054in}}%
\pgfpathlineto{\pgfqpoint{2.365349in}{3.528569in}}%
\pgfusepath{stroke}%
\end{pgfscope}%
\begin{pgfscope}%
\pgfpathrectangle{\pgfqpoint{0.566985in}{0.664139in}}{\pgfqpoint{2.015106in}{2.864429in}} %
\pgfusepath{clip}%
\pgfsetroundcap%
\pgfsetroundjoin%
\pgfsetlinewidth{1.003750pt}%
\definecolor{currentstroke}{rgb}{1.000000,0.494118,0.250980}%
\pgfsetstrokecolor{currentstroke}%
\pgfsetdash{}{0pt}%
\pgfpathmoveto{\pgfqpoint{0.623093in}{0.667443in}}%
\pgfpathlineto{\pgfqpoint{0.659006in}{0.667443in}}%
\pgfpathlineto{\pgfqpoint{0.694919in}{0.667443in}}%
\pgfpathlineto{\pgfqpoint{0.730832in}{0.667443in}}%
\pgfpathlineto{\pgfqpoint{0.766745in}{0.667443in}}%
\pgfpathlineto{\pgfqpoint{0.802658in}{0.667443in}}%
\pgfpathlineto{\pgfqpoint{0.838571in}{0.667443in}}%
\pgfpathlineto{\pgfqpoint{0.874483in}{0.670747in}}%
\pgfpathlineto{\pgfqpoint{0.910396in}{0.670747in}}%
\pgfpathlineto{\pgfqpoint{0.946309in}{0.670747in}}%
\pgfpathlineto{\pgfqpoint{0.982222in}{0.670747in}}%
\pgfpathlineto{\pgfqpoint{1.018135in}{0.674051in}}%
\pgfpathlineto{\pgfqpoint{1.054048in}{0.674051in}}%
\pgfpathlineto{\pgfqpoint{1.089961in}{0.674051in}}%
\pgfpathlineto{\pgfqpoint{1.125874in}{0.677354in}}%
\pgfpathlineto{\pgfqpoint{1.161787in}{0.680658in}}%
\pgfpathlineto{\pgfqpoint{1.197700in}{0.697177in}}%
\pgfpathlineto{\pgfqpoint{1.233612in}{0.713697in}}%
\pgfpathlineto{\pgfqpoint{1.269525in}{0.723608in}}%
\pgfpathlineto{\pgfqpoint{1.305438in}{0.756647in}}%
\pgfpathlineto{\pgfqpoint{1.341351in}{0.786381in}}%
\pgfpathlineto{\pgfqpoint{1.377264in}{0.855762in}}%
\pgfpathlineto{\pgfqpoint{1.413177in}{0.921839in}}%
\pgfpathlineto{\pgfqpoint{1.449090in}{1.024258in}}%
\pgfpathlineto{\pgfqpoint{1.485003in}{1.139892in}}%
\pgfpathlineto{\pgfqpoint{1.520916in}{1.341426in}}%
\pgfpathlineto{\pgfqpoint{1.556829in}{2.183906in}}%
\pgfpathlineto{\pgfqpoint{1.592741in}{2.494467in}}%
\pgfpathlineto{\pgfqpoint{1.628654in}{2.712520in}}%
\pgfpathlineto{\pgfqpoint{1.664567in}{2.867800in}}%
\pgfpathlineto{\pgfqpoint{1.700480in}{3.003258in}}%
\pgfpathlineto{\pgfqpoint{1.736393in}{3.099069in}}%
\pgfpathlineto{\pgfqpoint{1.772306in}{3.221311in}}%
\pgfpathlineto{\pgfqpoint{1.808219in}{3.293996in}}%
\pgfpathlineto{\pgfqpoint{1.844132in}{3.350161in}}%
\pgfpathlineto{\pgfqpoint{1.880045in}{3.403023in}}%
\pgfpathlineto{\pgfqpoint{1.915958in}{3.426149in}}%
\pgfpathlineto{\pgfqpoint{1.951870in}{3.455884in}}%
\pgfpathlineto{\pgfqpoint{1.987783in}{3.475707in}}%
\pgfpathlineto{\pgfqpoint{2.023696in}{3.485619in}}%
\pgfpathlineto{\pgfqpoint{2.059609in}{3.502138in}}%
\pgfpathlineto{\pgfqpoint{2.095522in}{3.508745in}}%
\pgfpathlineto{\pgfqpoint{2.131435in}{3.508745in}}%
\pgfpathlineto{\pgfqpoint{2.167348in}{3.515353in}}%
\pgfpathlineto{\pgfqpoint{2.203261in}{3.518657in}}%
\pgfpathlineto{\pgfqpoint{2.239174in}{3.518657in}}%
\pgfpathlineto{\pgfqpoint{2.275087in}{3.521961in}}%
\pgfpathlineto{\pgfqpoint{2.310999in}{3.521961in}}%
\pgfpathlineto{\pgfqpoint{2.346912in}{3.525265in}}%
\pgfpathlineto{\pgfqpoint{2.382825in}{3.528569in}}%
\pgfusepath{stroke}%
\end{pgfscope}%
\begin{pgfscope}%
\pgfpathrectangle{\pgfqpoint{0.566985in}{0.664139in}}{\pgfqpoint{2.015106in}{2.864429in}} %
\pgfusepath{clip}%
\pgfsetroundcap%
\pgfsetroundjoin%
\pgfsetlinewidth{1.003750pt}%
\definecolor{currentstroke}{rgb}{1.000000,0.694118,0.250980}%
\pgfsetstrokecolor{currentstroke}%
\pgfsetdash{}{0pt}%
\pgfpathmoveto{\pgfqpoint{0.700035in}{0.666548in}}%
\pgfpathlineto{\pgfqpoint{0.730931in}{0.666548in}}%
\pgfpathlineto{\pgfqpoint{0.761827in}{0.666548in}}%
\pgfpathlineto{\pgfqpoint{0.792722in}{0.666548in}}%
\pgfpathlineto{\pgfqpoint{0.823618in}{0.666548in}}%
\pgfpathlineto{\pgfqpoint{0.854514in}{0.666548in}}%
\pgfpathlineto{\pgfqpoint{0.885409in}{0.666548in}}%
\pgfpathlineto{\pgfqpoint{0.916305in}{0.666548in}}%
\pgfpathlineto{\pgfqpoint{0.947200in}{0.668957in}}%
\pgfpathlineto{\pgfqpoint{0.978096in}{0.668957in}}%
\pgfpathlineto{\pgfqpoint{1.008992in}{0.668957in}}%
\pgfpathlineto{\pgfqpoint{1.039887in}{0.671366in}}%
\pgfpathlineto{\pgfqpoint{1.070783in}{0.678594in}}%
\pgfpathlineto{\pgfqpoint{1.101679in}{0.681003in}}%
\pgfpathlineto{\pgfqpoint{1.132574in}{0.690639in}}%
\pgfpathlineto{\pgfqpoint{1.163470in}{0.700276in}}%
\pgfpathlineto{\pgfqpoint{1.194366in}{0.705094in}}%
\pgfpathlineto{\pgfqpoint{1.225261in}{0.717139in}}%
\pgfpathlineto{\pgfqpoint{1.256157in}{0.726776in}}%
\pgfpathlineto{\pgfqpoint{1.287053in}{0.746049in}}%
\pgfpathlineto{\pgfqpoint{1.317948in}{0.774958in}}%
\pgfpathlineto{\pgfqpoint{1.348844in}{0.796640in}}%
\pgfpathlineto{\pgfqpoint{1.379740in}{0.840004in}}%
\pgfpathlineto{\pgfqpoint{1.410635in}{0.895413in}}%
\pgfpathlineto{\pgfqpoint{1.441531in}{0.955641in}}%
\pgfpathlineto{\pgfqpoint{1.472427in}{1.076097in}}%
\pgfpathlineto{\pgfqpoint{1.503322in}{1.225461in}}%
\pgfpathlineto{\pgfqpoint{1.534218in}{1.548282in}}%
\pgfpathlineto{\pgfqpoint{1.565114in}{2.531198in}}%
\pgfpathlineto{\pgfqpoint{1.596009in}{2.878109in}}%
\pgfpathlineto{\pgfqpoint{1.626905in}{3.109384in}}%
\pgfpathlineto{\pgfqpoint{1.657801in}{3.253930in}}%
\pgfpathlineto{\pgfqpoint{1.688696in}{3.335840in}}%
\pgfpathlineto{\pgfqpoint{1.719592in}{3.384022in}}%
\pgfpathlineto{\pgfqpoint{1.750487in}{3.420159in}}%
\pgfpathlineto{\pgfqpoint{1.781383in}{3.453886in}}%
\pgfpathlineto{\pgfqpoint{1.812279in}{3.473159in}}%
\pgfpathlineto{\pgfqpoint{1.843174in}{3.490023in}}%
\pgfpathlineto{\pgfqpoint{1.874070in}{3.502068in}}%
\pgfpathlineto{\pgfqpoint{1.904966in}{3.514114in}}%
\pgfpathlineto{\pgfqpoint{1.935861in}{3.518932in}}%
\pgfpathlineto{\pgfqpoint{1.966757in}{3.523750in}}%
\pgfpathlineto{\pgfqpoint{1.997653in}{3.526159in}}%
\pgfpathlineto{\pgfqpoint{2.028548in}{3.526159in}}%
\pgfpathlineto{\pgfqpoint{2.059444in}{3.526159in}}%
\pgfpathlineto{\pgfqpoint{2.090340in}{3.526159in}}%
\pgfpathlineto{\pgfqpoint{2.121235in}{3.526159in}}%
\pgfpathlineto{\pgfqpoint{2.152131in}{3.526159in}}%
\pgfpathlineto{\pgfqpoint{2.183027in}{3.526159in}}%
\pgfpathlineto{\pgfqpoint{2.213922in}{3.528569in}}%
\pgfusepath{stroke}%
\end{pgfscope}%
\begin{pgfscope}%
\pgfsetrectcap%
\pgfsetmiterjoin%
\pgfsetlinewidth{1.254687pt}%
\definecolor{currentstroke}{rgb}{0.150000,0.150000,0.150000}%
\pgfsetstrokecolor{currentstroke}%
\pgfsetdash{}{0pt}%
\pgfpathmoveto{\pgfqpoint{0.566985in}{0.664139in}}%
\pgfpathlineto{\pgfqpoint{0.566985in}{3.528569in}}%
\pgfusepath{stroke}%
\end{pgfscope}%
\begin{pgfscope}%
\pgfsetrectcap%
\pgfsetmiterjoin%
\pgfsetlinewidth{1.254687pt}%
\definecolor{currentstroke}{rgb}{0.150000,0.150000,0.150000}%
\pgfsetstrokecolor{currentstroke}%
\pgfsetdash{}{0pt}%
\pgfpathmoveto{\pgfqpoint{0.566985in}{0.664139in}}%
\pgfpathlineto{\pgfqpoint{2.582091in}{0.664139in}}%
\pgfusepath{stroke}%
\end{pgfscope}%
\begin{pgfscope}%
\pgfsetbuttcap%
\pgfsetmiterjoin%
\definecolor{currentfill}{rgb}{1.000000,1.000000,1.000000}%
\pgfsetfillcolor{currentfill}%
\pgfsetlinewidth{0.000000pt}%
\definecolor{currentstroke}{rgb}{0.000000,0.000000,0.000000}%
\pgfsetstrokecolor{currentstroke}%
\pgfsetstrokeopacity{0.000000}%
\pgfsetdash{}{0pt}%
\pgfpathmoveto{\pgfqpoint{3.186623in}{0.664139in}}%
\pgfpathlineto{\pgfqpoint{5.201729in}{0.664139in}}%
\pgfpathlineto{\pgfqpoint{5.201729in}{2.789361in}}%
\pgfpathlineto{\pgfqpoint{3.186623in}{2.789361in}}%
\pgfpathclose%
\pgfusepath{fill}%
\end{pgfscope}%
\begin{pgfscope}%
\pgfsetroundcap%
\pgfsetroundjoin%
\pgfsetlinewidth{1.003750pt}%
\definecolor{currentstroke}{rgb}{0.200000,0.427451,0.650980}%
\pgfsetstrokecolor{currentstroke}%
\pgfsetdash{}{0pt}%
\pgfpathmoveto{\pgfqpoint{3.085112in}{3.455940in}}%
\pgfpathlineto{\pgfqpoint{3.196223in}{3.455940in}}%
\pgfusepath{stroke}%
\end{pgfscope}%
\begin{pgfscope}%
\definecolor{textcolor}{rgb}{1.000000,1.000000,1.000000}%
\pgfsetstrokecolor{textcolor}%
\pgfsetfillcolor{textcolor}%
\pgftext[x=3.285112in,y=3.417051in,left,base]{\color{textcolor}\rmfamily\fontsize{8.000000}{9.600000}\selectfont WT + Vehicle (1513)}%
\end{pgfscope}%
\begin{pgfscope}%
\pgfsetroundcap%
\pgfsetroundjoin%
\pgfsetlinewidth{1.003750pt}%
\definecolor{currentstroke}{rgb}{0.168627,0.670588,0.494118}%
\pgfsetstrokecolor{currentstroke}%
\pgfsetdash{}{0pt}%
\pgfpathmoveto{\pgfqpoint{3.085112in}{3.289300in}}%
\pgfpathlineto{\pgfqpoint{3.196223in}{3.289300in}}%
\pgfusepath{stroke}%
\end{pgfscope}%
\begin{pgfscope}%
\definecolor{textcolor}{rgb}{1.000000,1.000000,1.000000}%
\pgfsetstrokecolor{textcolor}%
\pgfsetfillcolor{textcolor}%
\pgftext[x=3.285112in,y=3.250411in,left,base]{\color{textcolor}\rmfamily\fontsize{8.000000}{9.600000}\selectfont WT + TAT-GluA2\textsubscript{3Y} (815)}%
\end{pgfscope}%
\begin{pgfscope}%
\pgfsetroundcap%
\pgfsetroundjoin%
\pgfsetlinewidth{1.003750pt}%
\definecolor{currentstroke}{rgb}{1.000000,0.494118,0.250980}%
\pgfsetstrokecolor{currentstroke}%
\pgfsetdash{}{0pt}%
\pgfpathmoveto{\pgfqpoint{3.085112in}{3.122660in}}%
\pgfpathlineto{\pgfqpoint{3.196223in}{3.122660in}}%
\pgfusepath{stroke}%
\end{pgfscope}%
\begin{pgfscope}%
\definecolor{textcolor}{rgb}{1.000000,1.000000,1.000000}%
\pgfsetstrokecolor{textcolor}%
\pgfsetfillcolor{textcolor}%
\pgftext[x=3.285112in,y=3.083771in,left,base]{\color{textcolor}\rmfamily\fontsize{8.000000}{9.600000}\selectfont Tg + Vehicle (867)}%
\end{pgfscope}%
\begin{pgfscope}%
\pgfsetroundcap%
\pgfsetroundjoin%
\pgfsetlinewidth{1.003750pt}%
\definecolor{currentstroke}{rgb}{1.000000,0.694118,0.250980}%
\pgfsetstrokecolor{currentstroke}%
\pgfsetdash{}{0pt}%
\pgfpathmoveto{\pgfqpoint{3.085112in}{2.956021in}}%
\pgfpathlineto{\pgfqpoint{3.196223in}{2.956021in}}%
\pgfusepath{stroke}%
\end{pgfscope}%
\begin{pgfscope}%
\definecolor{textcolor}{rgb}{1.000000,1.000000,1.000000}%
\pgfsetstrokecolor{textcolor}%
\pgfsetfillcolor{textcolor}%
\pgftext[x=3.285112in,y=2.917132in,left,base]{\color{textcolor}\rmfamily\fontsize{8.000000}{9.600000}\selectfont Tg + TAT-GluA2\textsubscript{3Y} (1189)}%
\end{pgfscope}%
\begin{pgfscope}%
\pgfsetroundcap%
\pgfsetroundjoin%
\pgfsetlinewidth{1.003750pt}%
\definecolor{currentstroke}{rgb}{0.200000,0.427451,0.650980}%
\pgfsetstrokecolor{currentstroke}%
\pgfsetdash{}{0pt}%
\pgfpathmoveto{\pgfqpoint{3.085112in}{3.455940in}}%
\pgfpathlineto{\pgfqpoint{3.196223in}{3.455940in}}%
\pgfusepath{stroke}%
\end{pgfscope}%
\begin{pgfscope}%
\definecolor{textcolor}{rgb}{1.000000,1.000000,1.000000}%
\pgfsetstrokecolor{textcolor}%
\pgfsetfillcolor{textcolor}%
\pgftext[x=3.285112in,y=3.417051in,left,base]{\color{textcolor}\rmfamily\fontsize{8.000000}{9.600000}\selectfont WT + Vehicle (1513)}%
\end{pgfscope}%
\begin{pgfscope}%
\pgfsetroundcap%
\pgfsetroundjoin%
\pgfsetlinewidth{1.003750pt}%
\definecolor{currentstroke}{rgb}{0.168627,0.670588,0.494118}%
\pgfsetstrokecolor{currentstroke}%
\pgfsetdash{}{0pt}%
\pgfpathmoveto{\pgfqpoint{3.085112in}{3.289300in}}%
\pgfpathlineto{\pgfqpoint{3.196223in}{3.289300in}}%
\pgfusepath{stroke}%
\end{pgfscope}%
\begin{pgfscope}%
\definecolor{textcolor}{rgb}{1.000000,1.000000,1.000000}%
\pgfsetstrokecolor{textcolor}%
\pgfsetfillcolor{textcolor}%
\pgftext[x=3.285112in,y=3.250411in,left,base]{\color{textcolor}\rmfamily\fontsize{8.000000}{9.600000}\selectfont WT + TAT-GluA2\textsubscript{3Y} (815)}%
\end{pgfscope}%
\begin{pgfscope}%
\pgfsetroundcap%
\pgfsetroundjoin%
\pgfsetlinewidth{1.003750pt}%
\definecolor{currentstroke}{rgb}{1.000000,0.494118,0.250980}%
\pgfsetstrokecolor{currentstroke}%
\pgfsetdash{}{0pt}%
\pgfpathmoveto{\pgfqpoint{3.085112in}{3.122660in}}%
\pgfpathlineto{\pgfqpoint{3.196223in}{3.122660in}}%
\pgfusepath{stroke}%
\end{pgfscope}%
\begin{pgfscope}%
\definecolor{textcolor}{rgb}{1.000000,1.000000,1.000000}%
\pgfsetstrokecolor{textcolor}%
\pgfsetfillcolor{textcolor}%
\pgftext[x=3.285112in,y=3.083771in,left,base]{\color{textcolor}\rmfamily\fontsize{8.000000}{9.600000}\selectfont Tg + Vehicle (867)}%
\end{pgfscope}%
\begin{pgfscope}%
\pgfsetroundcap%
\pgfsetroundjoin%
\pgfsetlinewidth{1.003750pt}%
\definecolor{currentstroke}{rgb}{1.000000,0.694118,0.250980}%
\pgfsetstrokecolor{currentstroke}%
\pgfsetdash{}{0pt}%
\pgfpathmoveto{\pgfqpoint{3.085112in}{2.956021in}}%
\pgfpathlineto{\pgfqpoint{3.196223in}{2.956021in}}%
\pgfusepath{stroke}%
\end{pgfscope}%
\begin{pgfscope}%
\definecolor{textcolor}{rgb}{1.000000,1.000000,1.000000}%
\pgfsetstrokecolor{textcolor}%
\pgfsetfillcolor{textcolor}%
\pgftext[x=3.285112in,y=2.917132in,left,base]{\color{textcolor}\rmfamily\fontsize{8.000000}{9.600000}\selectfont Tg + TAT-GluA2\textsubscript{3Y} (1189)}%
\end{pgfscope}%
\begin{pgfscope}%
\pgfsetbuttcap%
\pgfsetroundjoin%
\definecolor{currentfill}{rgb}{0.150000,0.150000,0.150000}%
\pgfsetfillcolor{currentfill}%
\pgfsetlinewidth{1.003750pt}%
\definecolor{currentstroke}{rgb}{0.150000,0.150000,0.150000}%
\pgfsetstrokecolor{currentstroke}%
\pgfsetdash{}{0pt}%
\pgfsys@defobject{currentmarker}{\pgfqpoint{0.000000in}{0.000000in}}{\pgfqpoint{0.041667in}{0.000000in}}{%
\pgfpathmoveto{\pgfqpoint{0.000000in}{0.000000in}}%
\pgfpathlineto{\pgfqpoint{0.041667in}{0.000000in}}%
\pgfusepath{stroke,fill}%
}%
\begin{pgfscope}%
\pgfsys@transformshift{3.186623in}{0.664139in}%
\pgfsys@useobject{currentmarker}{}%
\end{pgfscope}%
\end{pgfscope}%
\begin{pgfscope}%
\definecolor{textcolor}{rgb}{0.150000,0.150000,0.150000}%
\pgfsetstrokecolor{textcolor}%
\pgfsetfillcolor{textcolor}%
\pgftext[x=3.089400in,y=0.664139in,right,]{\color{textcolor}\rmfamily\fontsize{10.000000}{12.000000}\selectfont \(\displaystyle 0.00\)}%
\end{pgfscope}%
\begin{pgfscope}%
\pgfsetbuttcap%
\pgfsetroundjoin%
\definecolor{currentfill}{rgb}{0.150000,0.150000,0.150000}%
\pgfsetfillcolor{currentfill}%
\pgfsetlinewidth{1.003750pt}%
\definecolor{currentstroke}{rgb}{0.150000,0.150000,0.150000}%
\pgfsetstrokecolor{currentstroke}%
\pgfsetdash{}{0pt}%
\pgfsys@defobject{currentmarker}{\pgfqpoint{0.000000in}{0.000000in}}{\pgfqpoint{0.041667in}{0.000000in}}{%
\pgfpathmoveto{\pgfqpoint{0.000000in}{0.000000in}}%
\pgfpathlineto{\pgfqpoint{0.041667in}{0.000000in}}%
\pgfusepath{stroke,fill}%
}%
\begin{pgfscope}%
\pgfsys@transformshift{3.186623in}{1.089183in}%
\pgfsys@useobject{currentmarker}{}%
\end{pgfscope}%
\end{pgfscope}%
\begin{pgfscope}%
\definecolor{textcolor}{rgb}{0.150000,0.150000,0.150000}%
\pgfsetstrokecolor{textcolor}%
\pgfsetfillcolor{textcolor}%
\pgftext[x=3.089400in,y=1.089183in,right,]{\color{textcolor}\rmfamily\fontsize{10.000000}{12.000000}\selectfont \(\displaystyle 0.02\)}%
\end{pgfscope}%
\begin{pgfscope}%
\pgfsetbuttcap%
\pgfsetroundjoin%
\definecolor{currentfill}{rgb}{0.150000,0.150000,0.150000}%
\pgfsetfillcolor{currentfill}%
\pgfsetlinewidth{1.003750pt}%
\definecolor{currentstroke}{rgb}{0.150000,0.150000,0.150000}%
\pgfsetstrokecolor{currentstroke}%
\pgfsetdash{}{0pt}%
\pgfsys@defobject{currentmarker}{\pgfqpoint{0.000000in}{0.000000in}}{\pgfqpoint{0.041667in}{0.000000in}}{%
\pgfpathmoveto{\pgfqpoint{0.000000in}{0.000000in}}%
\pgfpathlineto{\pgfqpoint{0.041667in}{0.000000in}}%
\pgfusepath{stroke,fill}%
}%
\begin{pgfscope}%
\pgfsys@transformshift{3.186623in}{1.514228in}%
\pgfsys@useobject{currentmarker}{}%
\end{pgfscope}%
\end{pgfscope}%
\begin{pgfscope}%
\definecolor{textcolor}{rgb}{0.150000,0.150000,0.150000}%
\pgfsetstrokecolor{textcolor}%
\pgfsetfillcolor{textcolor}%
\pgftext[x=3.089400in,y=1.514228in,right,]{\color{textcolor}\rmfamily\fontsize{10.000000}{12.000000}\selectfont \(\displaystyle 0.04\)}%
\end{pgfscope}%
\begin{pgfscope}%
\pgfsetbuttcap%
\pgfsetroundjoin%
\definecolor{currentfill}{rgb}{0.150000,0.150000,0.150000}%
\pgfsetfillcolor{currentfill}%
\pgfsetlinewidth{1.003750pt}%
\definecolor{currentstroke}{rgb}{0.150000,0.150000,0.150000}%
\pgfsetstrokecolor{currentstroke}%
\pgfsetdash{}{0pt}%
\pgfsys@defobject{currentmarker}{\pgfqpoint{0.000000in}{0.000000in}}{\pgfqpoint{0.041667in}{0.000000in}}{%
\pgfpathmoveto{\pgfqpoint{0.000000in}{0.000000in}}%
\pgfpathlineto{\pgfqpoint{0.041667in}{0.000000in}}%
\pgfusepath{stroke,fill}%
}%
\begin{pgfscope}%
\pgfsys@transformshift{3.186623in}{1.939272in}%
\pgfsys@useobject{currentmarker}{}%
\end{pgfscope}%
\end{pgfscope}%
\begin{pgfscope}%
\definecolor{textcolor}{rgb}{0.150000,0.150000,0.150000}%
\pgfsetstrokecolor{textcolor}%
\pgfsetfillcolor{textcolor}%
\pgftext[x=3.089400in,y=1.939272in,right,]{\color{textcolor}\rmfamily\fontsize{10.000000}{12.000000}\selectfont \(\displaystyle 0.06\)}%
\end{pgfscope}%
\begin{pgfscope}%
\pgfsetbuttcap%
\pgfsetroundjoin%
\definecolor{currentfill}{rgb}{0.150000,0.150000,0.150000}%
\pgfsetfillcolor{currentfill}%
\pgfsetlinewidth{1.003750pt}%
\definecolor{currentstroke}{rgb}{0.150000,0.150000,0.150000}%
\pgfsetstrokecolor{currentstroke}%
\pgfsetdash{}{0pt}%
\pgfsys@defobject{currentmarker}{\pgfqpoint{0.000000in}{0.000000in}}{\pgfqpoint{0.041667in}{0.000000in}}{%
\pgfpathmoveto{\pgfqpoint{0.000000in}{0.000000in}}%
\pgfpathlineto{\pgfqpoint{0.041667in}{0.000000in}}%
\pgfusepath{stroke,fill}%
}%
\begin{pgfscope}%
\pgfsys@transformshift{3.186623in}{2.364317in}%
\pgfsys@useobject{currentmarker}{}%
\end{pgfscope}%
\end{pgfscope}%
\begin{pgfscope}%
\definecolor{textcolor}{rgb}{0.150000,0.150000,0.150000}%
\pgfsetstrokecolor{textcolor}%
\pgfsetfillcolor{textcolor}%
\pgftext[x=3.089400in,y=2.364317in,right,]{\color{textcolor}\rmfamily\fontsize{10.000000}{12.000000}\selectfont \(\displaystyle 0.08\)}%
\end{pgfscope}%
\begin{pgfscope}%
\pgfsetbuttcap%
\pgfsetroundjoin%
\definecolor{currentfill}{rgb}{0.150000,0.150000,0.150000}%
\pgfsetfillcolor{currentfill}%
\pgfsetlinewidth{1.003750pt}%
\definecolor{currentstroke}{rgb}{0.150000,0.150000,0.150000}%
\pgfsetstrokecolor{currentstroke}%
\pgfsetdash{}{0pt}%
\pgfsys@defobject{currentmarker}{\pgfqpoint{0.000000in}{0.000000in}}{\pgfqpoint{0.041667in}{0.000000in}}{%
\pgfpathmoveto{\pgfqpoint{0.000000in}{0.000000in}}%
\pgfpathlineto{\pgfqpoint{0.041667in}{0.000000in}}%
\pgfusepath{stroke,fill}%
}%
\begin{pgfscope}%
\pgfsys@transformshift{3.186623in}{2.789361in}%
\pgfsys@useobject{currentmarker}{}%
\end{pgfscope}%
\end{pgfscope}%
\begin{pgfscope}%
\definecolor{textcolor}{rgb}{0.150000,0.150000,0.150000}%
\pgfsetstrokecolor{textcolor}%
\pgfsetfillcolor{textcolor}%
\pgftext[x=3.089400in,y=2.789361in,right,]{\color{textcolor}\rmfamily\fontsize{10.000000}{12.000000}\selectfont \(\displaystyle 0.10\)}%
\end{pgfscope}%
\begin{pgfscope}%
\definecolor{textcolor}{rgb}{0.150000,0.150000,0.150000}%
\pgfsetstrokecolor{textcolor}%
\pgfsetfillcolor{textcolor}%
\pgftext[x=2.586359in,y=0.847528in,left,base,rotate=90.000000]{\color{textcolor}\rmfamily\fontsize{10.000000}{12.000000}\selectfont \textbf{Difference in cell activity}}%
\end{pgfscope}%
\begin{pgfscope}%
\definecolor{textcolor}{rgb}{0.150000,0.150000,0.150000}%
\pgfsetstrokecolor{textcolor}%
\pgfsetfillcolor{textcolor}%
\pgftext[x=2.738328in,y=1.537126in,left,base,rotate=90.000000]{\color{textcolor}\rmfamily\fontsize{10.000000}{12.000000}\selectfont \textbf{(a.u.)}}%
\end{pgfscope}%
\begin{pgfscope}%
\pgfpathrectangle{\pgfqpoint{3.186623in}{0.664139in}}{\pgfqpoint{2.015106in}{2.125222in}} %
\pgfusepath{clip}%
\pgfsetbuttcap%
\pgfsetmiterjoin%
\definecolor{currentfill}{rgb}{0.200000,0.427451,0.650980}%
\pgfsetfillcolor{currentfill}%
\pgfsetlinewidth{1.505625pt}%
\definecolor{currentstroke}{rgb}{0.200000,0.427451,0.650980}%
\pgfsetstrokecolor{currentstroke}%
\pgfsetdash{}{0pt}%
\pgfpathmoveto{\pgfqpoint{3.258591in}{0.664139in}}%
\pgfpathlineto{\pgfqpoint{3.618431in}{0.664139in}}%
\pgfpathlineto{\pgfqpoint{3.618431in}{0.833058in}}%
\pgfpathlineto{\pgfqpoint{3.258591in}{0.833058in}}%
\pgfpathclose%
\pgfusepath{stroke,fill}%
\end{pgfscope}%
\begin{pgfscope}%
\pgfpathrectangle{\pgfqpoint{3.186623in}{0.664139in}}{\pgfqpoint{2.015106in}{2.125222in}} %
\pgfusepath{clip}%
\pgfsetbuttcap%
\pgfsetmiterjoin%
\definecolor{currentfill}{rgb}{0.168627,0.670588,0.494118}%
\pgfsetfillcolor{currentfill}%
\pgfsetlinewidth{1.505625pt}%
\definecolor{currentstroke}{rgb}{0.168627,0.670588,0.494118}%
\pgfsetstrokecolor{currentstroke}%
\pgfsetdash{}{0pt}%
\pgfpathmoveto{\pgfqpoint{3.762367in}{0.664139in}}%
\pgfpathlineto{\pgfqpoint{4.122208in}{0.664139in}}%
\pgfpathlineto{\pgfqpoint{4.122208in}{0.917566in}}%
\pgfpathlineto{\pgfqpoint{3.762367in}{0.917566in}}%
\pgfpathclose%
\pgfusepath{stroke,fill}%
\end{pgfscope}%
\begin{pgfscope}%
\pgfpathrectangle{\pgfqpoint{3.186623in}{0.664139in}}{\pgfqpoint{2.015106in}{2.125222in}} %
\pgfusepath{clip}%
\pgfsetbuttcap%
\pgfsetmiterjoin%
\definecolor{currentfill}{rgb}{1.000000,0.494118,0.250980}%
\pgfsetfillcolor{currentfill}%
\pgfsetlinewidth{1.505625pt}%
\definecolor{currentstroke}{rgb}{1.000000,0.494118,0.250980}%
\pgfsetstrokecolor{currentstroke}%
\pgfsetdash{}{0pt}%
\pgfpathmoveto{\pgfqpoint{4.266144in}{0.664139in}}%
\pgfpathlineto{\pgfqpoint{4.625984in}{0.664139in}}%
\pgfpathlineto{\pgfqpoint{4.625984in}{2.090351in}}%
\pgfpathlineto{\pgfqpoint{4.266144in}{2.090351in}}%
\pgfpathclose%
\pgfusepath{stroke,fill}%
\end{pgfscope}%
\begin{pgfscope}%
\pgfpathrectangle{\pgfqpoint{3.186623in}{0.664139in}}{\pgfqpoint{2.015106in}{2.125222in}} %
\pgfusepath{clip}%
\pgfsetbuttcap%
\pgfsetmiterjoin%
\definecolor{currentfill}{rgb}{1.000000,0.694118,0.250980}%
\pgfsetfillcolor{currentfill}%
\pgfsetlinewidth{1.505625pt}%
\definecolor{currentstroke}{rgb}{1.000000,0.694118,0.250980}%
\pgfsetstrokecolor{currentstroke}%
\pgfsetdash{}{0pt}%
\pgfpathmoveto{\pgfqpoint{4.769920in}{0.664139in}}%
\pgfpathlineto{\pgfqpoint{5.129761in}{0.664139in}}%
\pgfpathlineto{\pgfqpoint{5.129761in}{0.758839in}}%
\pgfpathlineto{\pgfqpoint{4.769920in}{0.758839in}}%
\pgfpathclose%
\pgfusepath{stroke,fill}%
\end{pgfscope}%
\begin{pgfscope}%
\pgfpathrectangle{\pgfqpoint{3.186623in}{0.664139in}}{\pgfqpoint{2.015106in}{2.125222in}} %
\pgfusepath{clip}%
\pgfsetbuttcap%
\pgfsetroundjoin%
\pgfsetlinewidth{1.505625pt}%
\definecolor{currentstroke}{rgb}{0.200000,0.427451,0.650980}%
\pgfsetstrokecolor{currentstroke}%
\pgfsetdash{}{0pt}%
\pgfpathmoveto{\pgfqpoint{3.438511in}{0.833058in}}%
\pgfpathlineto{\pgfqpoint{3.438511in}{0.921639in}}%
\pgfusepath{stroke}%
\end{pgfscope}%
\begin{pgfscope}%
\pgfpathrectangle{\pgfqpoint{3.186623in}{0.664139in}}{\pgfqpoint{2.015106in}{2.125222in}} %
\pgfusepath{clip}%
\pgfsetbuttcap%
\pgfsetroundjoin%
\pgfsetlinewidth{1.505625pt}%
\definecolor{currentstroke}{rgb}{0.168627,0.670588,0.494118}%
\pgfsetstrokecolor{currentstroke}%
\pgfsetdash{}{0pt}%
\pgfpathmoveto{\pgfqpoint{3.942287in}{0.917566in}}%
\pgfpathlineto{\pgfqpoint{3.942287in}{1.064349in}}%
\pgfusepath{stroke}%
\end{pgfscope}%
\begin{pgfscope}%
\pgfpathrectangle{\pgfqpoint{3.186623in}{0.664139in}}{\pgfqpoint{2.015106in}{2.125222in}} %
\pgfusepath{clip}%
\pgfsetbuttcap%
\pgfsetroundjoin%
\pgfsetlinewidth{1.505625pt}%
\definecolor{currentstroke}{rgb}{1.000000,0.494118,0.250980}%
\pgfsetstrokecolor{currentstroke}%
\pgfsetdash{}{0pt}%
\pgfpathmoveto{\pgfqpoint{4.446064in}{2.090351in}}%
\pgfpathlineto{\pgfqpoint{4.446064in}{2.261611in}}%
\pgfusepath{stroke}%
\end{pgfscope}%
\begin{pgfscope}%
\pgfpathrectangle{\pgfqpoint{3.186623in}{0.664139in}}{\pgfqpoint{2.015106in}{2.125222in}} %
\pgfusepath{clip}%
\pgfsetbuttcap%
\pgfsetroundjoin%
\pgfsetlinewidth{1.505625pt}%
\definecolor{currentstroke}{rgb}{1.000000,0.694118,0.250980}%
\pgfsetstrokecolor{currentstroke}%
\pgfsetdash{}{0pt}%
\pgfpathmoveto{\pgfqpoint{4.949840in}{0.758839in}}%
\pgfpathlineto{\pgfqpoint{4.949840in}{0.863636in}}%
\pgfusepath{stroke}%
\end{pgfscope}%
\begin{pgfscope}%
\pgfpathrectangle{\pgfqpoint{3.186623in}{0.664139in}}{\pgfqpoint{2.015106in}{2.125222in}} %
\pgfusepath{clip}%
\pgfsetbuttcap%
\pgfsetroundjoin%
\definecolor{currentfill}{rgb}{0.200000,0.427451,0.650980}%
\pgfsetfillcolor{currentfill}%
\pgfsetlinewidth{1.505625pt}%
\definecolor{currentstroke}{rgb}{0.200000,0.427451,0.650980}%
\pgfsetstrokecolor{currentstroke}%
\pgfsetdash{}{0pt}%
\pgfsys@defobject{currentmarker}{\pgfqpoint{-0.111111in}{-0.000000in}}{\pgfqpoint{0.111111in}{0.000000in}}{%
\pgfpathmoveto{\pgfqpoint{0.111111in}{-0.000000in}}%
\pgfpathlineto{\pgfqpoint{-0.111111in}{0.000000in}}%
\pgfusepath{stroke,fill}%
}%
\begin{pgfscope}%
\pgfsys@transformshift{3.438511in}{0.833058in}%
\pgfsys@useobject{currentmarker}{}%
\end{pgfscope}%
\end{pgfscope}%
\begin{pgfscope}%
\pgfpathrectangle{\pgfqpoint{3.186623in}{0.664139in}}{\pgfqpoint{2.015106in}{2.125222in}} %
\pgfusepath{clip}%
\pgfsetbuttcap%
\pgfsetroundjoin%
\definecolor{currentfill}{rgb}{0.200000,0.427451,0.650980}%
\pgfsetfillcolor{currentfill}%
\pgfsetlinewidth{1.505625pt}%
\definecolor{currentstroke}{rgb}{0.200000,0.427451,0.650980}%
\pgfsetstrokecolor{currentstroke}%
\pgfsetdash{}{0pt}%
\pgfsys@defobject{currentmarker}{\pgfqpoint{-0.111111in}{-0.000000in}}{\pgfqpoint{0.111111in}{0.000000in}}{%
\pgfpathmoveto{\pgfqpoint{0.111111in}{-0.000000in}}%
\pgfpathlineto{\pgfqpoint{-0.111111in}{0.000000in}}%
\pgfusepath{stroke,fill}%
}%
\begin{pgfscope}%
\pgfsys@transformshift{3.438511in}{0.921639in}%
\pgfsys@useobject{currentmarker}{}%
\end{pgfscope}%
\end{pgfscope}%
\begin{pgfscope}%
\pgfpathrectangle{\pgfqpoint{3.186623in}{0.664139in}}{\pgfqpoint{2.015106in}{2.125222in}} %
\pgfusepath{clip}%
\pgfsetbuttcap%
\pgfsetroundjoin%
\definecolor{currentfill}{rgb}{0.168627,0.670588,0.494118}%
\pgfsetfillcolor{currentfill}%
\pgfsetlinewidth{1.505625pt}%
\definecolor{currentstroke}{rgb}{0.168627,0.670588,0.494118}%
\pgfsetstrokecolor{currentstroke}%
\pgfsetdash{}{0pt}%
\pgfsys@defobject{currentmarker}{\pgfqpoint{-0.111111in}{-0.000000in}}{\pgfqpoint{0.111111in}{0.000000in}}{%
\pgfpathmoveto{\pgfqpoint{0.111111in}{-0.000000in}}%
\pgfpathlineto{\pgfqpoint{-0.111111in}{0.000000in}}%
\pgfusepath{stroke,fill}%
}%
\begin{pgfscope}%
\pgfsys@transformshift{3.942287in}{0.917566in}%
\pgfsys@useobject{currentmarker}{}%
\end{pgfscope}%
\end{pgfscope}%
\begin{pgfscope}%
\pgfpathrectangle{\pgfqpoint{3.186623in}{0.664139in}}{\pgfqpoint{2.015106in}{2.125222in}} %
\pgfusepath{clip}%
\pgfsetbuttcap%
\pgfsetroundjoin%
\definecolor{currentfill}{rgb}{0.168627,0.670588,0.494118}%
\pgfsetfillcolor{currentfill}%
\pgfsetlinewidth{1.505625pt}%
\definecolor{currentstroke}{rgb}{0.168627,0.670588,0.494118}%
\pgfsetstrokecolor{currentstroke}%
\pgfsetdash{}{0pt}%
\pgfsys@defobject{currentmarker}{\pgfqpoint{-0.111111in}{-0.000000in}}{\pgfqpoint{0.111111in}{0.000000in}}{%
\pgfpathmoveto{\pgfqpoint{0.111111in}{-0.000000in}}%
\pgfpathlineto{\pgfqpoint{-0.111111in}{0.000000in}}%
\pgfusepath{stroke,fill}%
}%
\begin{pgfscope}%
\pgfsys@transformshift{3.942287in}{1.064349in}%
\pgfsys@useobject{currentmarker}{}%
\end{pgfscope}%
\end{pgfscope}%
\begin{pgfscope}%
\pgfpathrectangle{\pgfqpoint{3.186623in}{0.664139in}}{\pgfqpoint{2.015106in}{2.125222in}} %
\pgfusepath{clip}%
\pgfsetbuttcap%
\pgfsetroundjoin%
\definecolor{currentfill}{rgb}{1.000000,0.494118,0.250980}%
\pgfsetfillcolor{currentfill}%
\pgfsetlinewidth{1.505625pt}%
\definecolor{currentstroke}{rgb}{1.000000,0.494118,0.250980}%
\pgfsetstrokecolor{currentstroke}%
\pgfsetdash{}{0pt}%
\pgfsys@defobject{currentmarker}{\pgfqpoint{-0.111111in}{-0.000000in}}{\pgfqpoint{0.111111in}{0.000000in}}{%
\pgfpathmoveto{\pgfqpoint{0.111111in}{-0.000000in}}%
\pgfpathlineto{\pgfqpoint{-0.111111in}{0.000000in}}%
\pgfusepath{stroke,fill}%
}%
\begin{pgfscope}%
\pgfsys@transformshift{4.446064in}{2.090351in}%
\pgfsys@useobject{currentmarker}{}%
\end{pgfscope}%
\end{pgfscope}%
\begin{pgfscope}%
\pgfpathrectangle{\pgfqpoint{3.186623in}{0.664139in}}{\pgfqpoint{2.015106in}{2.125222in}} %
\pgfusepath{clip}%
\pgfsetbuttcap%
\pgfsetroundjoin%
\definecolor{currentfill}{rgb}{1.000000,0.494118,0.250980}%
\pgfsetfillcolor{currentfill}%
\pgfsetlinewidth{1.505625pt}%
\definecolor{currentstroke}{rgb}{1.000000,0.494118,0.250980}%
\pgfsetstrokecolor{currentstroke}%
\pgfsetdash{}{0pt}%
\pgfsys@defobject{currentmarker}{\pgfqpoint{-0.111111in}{-0.000000in}}{\pgfqpoint{0.111111in}{0.000000in}}{%
\pgfpathmoveto{\pgfqpoint{0.111111in}{-0.000000in}}%
\pgfpathlineto{\pgfqpoint{-0.111111in}{0.000000in}}%
\pgfusepath{stroke,fill}%
}%
\begin{pgfscope}%
\pgfsys@transformshift{4.446064in}{2.261611in}%
\pgfsys@useobject{currentmarker}{}%
\end{pgfscope}%
\end{pgfscope}%
\begin{pgfscope}%
\pgfpathrectangle{\pgfqpoint{3.186623in}{0.664139in}}{\pgfqpoint{2.015106in}{2.125222in}} %
\pgfusepath{clip}%
\pgfsetbuttcap%
\pgfsetroundjoin%
\definecolor{currentfill}{rgb}{1.000000,0.694118,0.250980}%
\pgfsetfillcolor{currentfill}%
\pgfsetlinewidth{1.505625pt}%
\definecolor{currentstroke}{rgb}{1.000000,0.694118,0.250980}%
\pgfsetstrokecolor{currentstroke}%
\pgfsetdash{}{0pt}%
\pgfsys@defobject{currentmarker}{\pgfqpoint{-0.111111in}{-0.000000in}}{\pgfqpoint{0.111111in}{0.000000in}}{%
\pgfpathmoveto{\pgfqpoint{0.111111in}{-0.000000in}}%
\pgfpathlineto{\pgfqpoint{-0.111111in}{0.000000in}}%
\pgfusepath{stroke,fill}%
}%
\begin{pgfscope}%
\pgfsys@transformshift{4.949840in}{0.758839in}%
\pgfsys@useobject{currentmarker}{}%
\end{pgfscope}%
\end{pgfscope}%
\begin{pgfscope}%
\pgfpathrectangle{\pgfqpoint{3.186623in}{0.664139in}}{\pgfqpoint{2.015106in}{2.125222in}} %
\pgfusepath{clip}%
\pgfsetbuttcap%
\pgfsetroundjoin%
\definecolor{currentfill}{rgb}{1.000000,0.694118,0.250980}%
\pgfsetfillcolor{currentfill}%
\pgfsetlinewidth{1.505625pt}%
\definecolor{currentstroke}{rgb}{1.000000,0.694118,0.250980}%
\pgfsetstrokecolor{currentstroke}%
\pgfsetdash{}{0pt}%
\pgfsys@defobject{currentmarker}{\pgfqpoint{-0.111111in}{-0.000000in}}{\pgfqpoint{0.111111in}{0.000000in}}{%
\pgfpathmoveto{\pgfqpoint{0.111111in}{-0.000000in}}%
\pgfpathlineto{\pgfqpoint{-0.111111in}{0.000000in}}%
\pgfusepath{stroke,fill}%
}%
\begin{pgfscope}%
\pgfsys@transformshift{4.949840in}{0.863636in}%
\pgfsys@useobject{currentmarker}{}%
\end{pgfscope}%
\end{pgfscope}%
\begin{pgfscope}%
\pgfpathrectangle{\pgfqpoint{3.186623in}{0.664139in}}{\pgfqpoint{2.015106in}{2.125222in}} %
\pgfusepath{clip}%
\pgfsetroundcap%
\pgfsetroundjoin%
\pgfsetlinewidth{1.756562pt}%
\definecolor{currentstroke}{rgb}{0.627451,0.627451,0.643137}%
\pgfsetstrokecolor{currentstroke}%
\pgfsetdash{}{0pt}%
\pgfpathmoveto{\pgfqpoint{3.438511in}{1.004972in}}%
\pgfpathlineto{\pgfqpoint{3.438511in}{2.483834in}}%
\pgfusepath{stroke}%
\end{pgfscope}%
\begin{pgfscope}%
\pgfpathrectangle{\pgfqpoint{3.186623in}{0.664139in}}{\pgfqpoint{2.015106in}{2.125222in}} %
\pgfusepath{clip}%
\pgfsetroundcap%
\pgfsetroundjoin%
\pgfsetlinewidth{1.756562pt}%
\definecolor{currentstroke}{rgb}{0.627451,0.627451,0.643137}%
\pgfsetstrokecolor{currentstroke}%
\pgfsetdash{}{0pt}%
\pgfpathmoveto{\pgfqpoint{3.438511in}{2.483834in}}%
\pgfpathlineto{\pgfqpoint{4.446064in}{2.483834in}}%
\pgfusepath{stroke}%
\end{pgfscope}%
\begin{pgfscope}%
\pgfpathrectangle{\pgfqpoint{3.186623in}{0.664139in}}{\pgfqpoint{2.015106in}{2.125222in}} %
\pgfusepath{clip}%
\pgfsetroundcap%
\pgfsetroundjoin%
\pgfsetlinewidth{1.756562pt}%
\definecolor{currentstroke}{rgb}{0.627451,0.627451,0.643137}%
\pgfsetstrokecolor{currentstroke}%
\pgfsetdash{}{0pt}%
\pgfpathmoveto{\pgfqpoint{4.446064in}{2.483834in}}%
\pgfpathlineto{\pgfqpoint{4.446064in}{2.428278in}}%
\pgfusepath{stroke}%
\end{pgfscope}%
\begin{pgfscope}%
\pgfpathrectangle{\pgfqpoint{3.186623in}{0.664139in}}{\pgfqpoint{2.015106in}{2.125222in}} %
\pgfusepath{clip}%
\pgfsetroundcap%
\pgfsetroundjoin%
\pgfsetlinewidth{1.756562pt}%
\definecolor{currentstroke}{rgb}{0.627451,0.627451,0.643137}%
\pgfsetstrokecolor{currentstroke}%
\pgfsetdash{}{0pt}%
\pgfpathmoveto{\pgfqpoint{4.446064in}{2.567167in}}%
\pgfpathlineto{\pgfqpoint{4.446064in}{2.706056in}}%
\pgfusepath{stroke}%
\end{pgfscope}%
\begin{pgfscope}%
\pgfpathrectangle{\pgfqpoint{3.186623in}{0.664139in}}{\pgfqpoint{2.015106in}{2.125222in}} %
\pgfusepath{clip}%
\pgfsetroundcap%
\pgfsetroundjoin%
\pgfsetlinewidth{1.756562pt}%
\definecolor{currentstroke}{rgb}{0.627451,0.627451,0.643137}%
\pgfsetstrokecolor{currentstroke}%
\pgfsetdash{}{0pt}%
\pgfpathmoveto{\pgfqpoint{4.446064in}{2.706056in}}%
\pgfpathlineto{\pgfqpoint{4.949840in}{2.706056in}}%
\pgfusepath{stroke}%
\end{pgfscope}%
\begin{pgfscope}%
\pgfpathrectangle{\pgfqpoint{3.186623in}{0.664139in}}{\pgfqpoint{2.015106in}{2.125222in}} %
\pgfusepath{clip}%
\pgfsetroundcap%
\pgfsetroundjoin%
\pgfsetlinewidth{1.756562pt}%
\definecolor{currentstroke}{rgb}{0.627451,0.627451,0.643137}%
\pgfsetstrokecolor{currentstroke}%
\pgfsetdash{}{0pt}%
\pgfpathmoveto{\pgfqpoint{4.949840in}{2.706056in}}%
\pgfpathlineto{\pgfqpoint{4.949840in}{1.030303in}}%
\pgfusepath{stroke}%
\end{pgfscope}%
\begin{pgfscope}%
\pgfsetrectcap%
\pgfsetmiterjoin%
\pgfsetlinewidth{1.254687pt}%
\definecolor{currentstroke}{rgb}{0.150000,0.150000,0.150000}%
\pgfsetstrokecolor{currentstroke}%
\pgfsetdash{}{0pt}%
\pgfpathmoveto{\pgfqpoint{3.186623in}{0.664139in}}%
\pgfpathlineto{\pgfqpoint{3.186623in}{2.789361in}}%
\pgfusepath{stroke}%
\end{pgfscope}%
\begin{pgfscope}%
\pgfsetrectcap%
\pgfsetmiterjoin%
\pgfsetlinewidth{1.254687pt}%
\definecolor{currentstroke}{rgb}{0.150000,0.150000,0.150000}%
\pgfsetstrokecolor{currentstroke}%
\pgfsetdash{}{0pt}%
\pgfpathmoveto{\pgfqpoint{3.186623in}{0.664139in}}%
\pgfpathlineto{\pgfqpoint{5.201729in}{0.664139in}}%
\pgfusepath{stroke}%
\end{pgfscope}%
\begin{pgfscope}%
\definecolor{textcolor}{rgb}{0.150000,0.150000,0.150000}%
\pgfsetstrokecolor{textcolor}%
\pgfsetfillcolor{textcolor}%
\pgftext[x=4.446064in,y=2.313695in,,]{\color{textcolor}\rmfamily\fontsize{15.000000}{18.000000}\selectfont \textbf{*}}%
\end{pgfscope}%
\begin{pgfscope}%
\definecolor{textcolor}{rgb}{0.150000,0.150000,0.150000}%
\pgfsetstrokecolor{textcolor}%
\pgfsetfillcolor{textcolor}%
\pgftext[x=4.949840in,y=0.915720in,,]{\color{textcolor}\rmfamily\fontsize{15.000000}{18.000000}\selectfont \textbf{*}}%
\end{pgfscope}%
\begin{pgfscope}%
\pgfsetbuttcap%
\pgfsetmiterjoin%
\definecolor{currentfill}{rgb}{0.200000,0.427451,0.650980}%
\pgfsetfillcolor{currentfill}%
\pgfsetlinewidth{1.505625pt}%
\definecolor{currentstroke}{rgb}{0.200000,0.427451,0.650980}%
\pgfsetstrokecolor{currentstroke}%
\pgfsetdash{}{0pt}%
\pgfpathmoveto{\pgfqpoint{3.286623in}{3.417051in}}%
\pgfpathlineto{\pgfqpoint{3.397734in}{3.417051in}}%
\pgfpathlineto{\pgfqpoint{3.397734in}{3.494828in}}%
\pgfpathlineto{\pgfqpoint{3.286623in}{3.494828in}}%
\pgfpathclose%
\pgfusepath{stroke,fill}%
\end{pgfscope}%
\begin{pgfscope}%
\definecolor{textcolor}{rgb}{0.150000,0.150000,0.150000}%
\pgfsetstrokecolor{textcolor}%
\pgfsetfillcolor{textcolor}%
\pgftext[x=3.486623in,y=3.417051in,left,base]{\color{textcolor}\rmfamily\fontsize{8.000000}{9.600000}\selectfont WT + Vehicle (1513)}%
\end{pgfscope}%
\begin{pgfscope}%
\pgfsetbuttcap%
\pgfsetmiterjoin%
\definecolor{currentfill}{rgb}{0.168627,0.670588,0.494118}%
\pgfsetfillcolor{currentfill}%
\pgfsetlinewidth{1.505625pt}%
\definecolor{currentstroke}{rgb}{0.168627,0.670588,0.494118}%
\pgfsetstrokecolor{currentstroke}%
\pgfsetdash{}{0pt}%
\pgfpathmoveto{\pgfqpoint{3.286623in}{3.250411in}}%
\pgfpathlineto{\pgfqpoint{3.397734in}{3.250411in}}%
\pgfpathlineto{\pgfqpoint{3.397734in}{3.328189in}}%
\pgfpathlineto{\pgfqpoint{3.286623in}{3.328189in}}%
\pgfpathclose%
\pgfusepath{stroke,fill}%
\end{pgfscope}%
\begin{pgfscope}%
\definecolor{textcolor}{rgb}{0.150000,0.150000,0.150000}%
\pgfsetstrokecolor{textcolor}%
\pgfsetfillcolor{textcolor}%
\pgftext[x=3.486623in,y=3.250411in,left,base]{\color{textcolor}\rmfamily\fontsize{8.000000}{9.600000}\selectfont WT + TAT-GluA2\textsubscript{3Y} (815)}%
\end{pgfscope}%
\begin{pgfscope}%
\pgfsetbuttcap%
\pgfsetmiterjoin%
\definecolor{currentfill}{rgb}{1.000000,0.494118,0.250980}%
\pgfsetfillcolor{currentfill}%
\pgfsetlinewidth{1.505625pt}%
\definecolor{currentstroke}{rgb}{1.000000,0.494118,0.250980}%
\pgfsetstrokecolor{currentstroke}%
\pgfsetdash{}{0pt}%
\pgfpathmoveto{\pgfqpoint{3.286623in}{3.083771in}}%
\pgfpathlineto{\pgfqpoint{3.397734in}{3.083771in}}%
\pgfpathlineto{\pgfqpoint{3.397734in}{3.161549in}}%
\pgfpathlineto{\pgfqpoint{3.286623in}{3.161549in}}%
\pgfpathclose%
\pgfusepath{stroke,fill}%
\end{pgfscope}%
\begin{pgfscope}%
\definecolor{textcolor}{rgb}{0.150000,0.150000,0.150000}%
\pgfsetstrokecolor{textcolor}%
\pgfsetfillcolor{textcolor}%
\pgftext[x=3.486623in,y=3.083771in,left,base]{\color{textcolor}\rmfamily\fontsize{8.000000}{9.600000}\selectfont Tg + Vehicle (867)}%
\end{pgfscope}%
\begin{pgfscope}%
\pgfsetbuttcap%
\pgfsetmiterjoin%
\definecolor{currentfill}{rgb}{1.000000,0.694118,0.250980}%
\pgfsetfillcolor{currentfill}%
\pgfsetlinewidth{1.505625pt}%
\definecolor{currentstroke}{rgb}{1.000000,0.694118,0.250980}%
\pgfsetstrokecolor{currentstroke}%
\pgfsetdash{}{0pt}%
\pgfpathmoveto{\pgfqpoint{3.286623in}{2.917132in}}%
\pgfpathlineto{\pgfqpoint{3.397734in}{2.917132in}}%
\pgfpathlineto{\pgfqpoint{3.397734in}{2.994910in}}%
\pgfpathlineto{\pgfqpoint{3.286623in}{2.994910in}}%
\pgfpathclose%
\pgfusepath{stroke,fill}%
\end{pgfscope}%
\begin{pgfscope}%
\definecolor{textcolor}{rgb}{0.150000,0.150000,0.150000}%
\pgfsetstrokecolor{textcolor}%
\pgfsetfillcolor{textcolor}%
\pgftext[x=3.486623in,y=2.917132in,left,base]{\color{textcolor}\rmfamily\fontsize{8.000000}{9.600000}\selectfont Tg + TAT-GluA2\textsubscript{3Y} (1189)}%
\end{pgfscope}%
\end{pgfpicture}%
\makeatother%
\endgroup%

    \caption[Cell activity difference between training and memory test.]{Cell activity difference between training and testing sessions. \gls{tg} mice significantly increased their cell activity during testing session, and this is blocked by \tglu{} treatment. \label{f.ad.actdiff}}
\end{figure}


\subsection{\tglu{} rescues contextual fear memory recall by decreasing activity}

Given that the \gls{tg} mice and \gls{wt} mice showed different amounts of freezing, it is possible that the observed hyperactivity is a result of these behavioural differences. To control for behavioural state, we investigated the average cell activity when the mice were freezing or not freezing during the 5-min context memory test. The results are summarized in figure~\ref{f.ad.activity_freezing}.  

A two-way \gls{anova} on the average activity when mice are freezing showed an insignificant interaction between \textit{Genotype} and \textit{Treatment} (F\tsb{1,3034}=-0.03, p=0.97). There was a significant main effect of both \textit{Genotype} (F\tsb{1,3034}=6.9, p=0.008) and \textit{Treatment} (F\tsb{1,3034}=18.7, p<0.001). \textit{Post hoc} tests showed that cells in the vehicle-treated \gls{tg} mice had significantly higher activity than those in \gls{wt} mice (WT-Veh vs Tg-Veh, T=-2.0, p=0.04), and this deficit was fully rescued by \tglu{} treatment (Tg-\glu{} vs Tg-Veh, T=-3.1, p=0.002; WT-Veh vs Tg-\glu{}, T=1.4, p=0.16). Moreover, \tglu{} treatment also significantly decreased cell activity in \gls{wt} mice (WT-Veh vs WT-\glu{}, T=3.0, p=0.002). These results show that neurons in \Gls{tg} mice have significantly increased activity during freezing, and \tglu{} treatment has a significant inhibitory effect on cell activity both in \gls{wt} and \gls{tg} mice. 

For the cell activity when the mice did not show freezing behaviour, we performed a two-way \gls{anova}, and found a significant \textit{Genotype} $\times$ \textit{Treatment} interaction (F\tsb{1,3029}=13.6, p<0.001), as well as a significant main effect of \textit{Treatment} (F\tsb{1,3029}=12.3, p<0.001). \tglu{} treatment only significantly decreased activity in the \gls{tg} mice (Tg-\glu{} vs Tg-Veh, T=-5.4, p<0.001), and had no effect on \gls{wt} mice (WT-\glu{} vs WT-Veh, T=-0.56, p=0.58). These results suggest \tglu{} may rescue the \gls{tg} phenotype by globally decreasing background cell activity.These results suggest \tglu{} treatment is able to decrease cell activity during freezing for both \gls{tg} and \gls{wt} mice, however it also decrease cell activity in \gls{tg} when the mouse is not freezing.


\begin{figure}[h]
    \begin{subfigure}[h]{\textwidth}
        %% Creator: Matplotlib, PGF backend
%%
%% To include the figure in your LaTeX document, write
%%   \input{<filename>.pgf}
%%
%% Make sure the required packages are loaded in your preamble
%%   \usepackage{pgf}
%%
%% Figures using additional raster images can only be included by \input if
%% they are in the same directory as the main LaTeX file. For loading figures
%% from other directories you can use the `import` package
%%   \usepackage{import}
%% and then include the figures with
%%   \import{<path to file>}{<filename>.pgf}
%%
%% Matplotlib used the following preamble
%%   \usepackage[utf8]{inputenc}
%%   \usepackage[T1]{fontenc}
%%   \usepackage{siunitx}
%%
\begingroup%
\makeatletter%
\begin{pgfpicture}%
\pgfpathrectangle{\pgfpointorigin}{\pgfqpoint{5.301729in}{3.553934in}}%
\pgfusepath{use as bounding box, clip}%
\begin{pgfscope}%
\pgfsetbuttcap%
\pgfsetmiterjoin%
\definecolor{currentfill}{rgb}{1.000000,1.000000,1.000000}%
\pgfsetfillcolor{currentfill}%
\pgfsetlinewidth{0.000000pt}%
\definecolor{currentstroke}{rgb}{1.000000,1.000000,1.000000}%
\pgfsetstrokecolor{currentstroke}%
\pgfsetdash{}{0pt}%
\pgfpathmoveto{\pgfqpoint{0.000000in}{0.000000in}}%
\pgfpathlineto{\pgfqpoint{5.301729in}{0.000000in}}%
\pgfpathlineto{\pgfqpoint{5.301729in}{3.553934in}}%
\pgfpathlineto{\pgfqpoint{0.000000in}{3.553934in}}%
\pgfpathclose%
\pgfusepath{fill}%
\end{pgfscope}%
\begin{pgfscope}%
\pgfsetbuttcap%
\pgfsetmiterjoin%
\definecolor{currentfill}{rgb}{1.000000,1.000000,1.000000}%
\pgfsetfillcolor{currentfill}%
\pgfsetlinewidth{0.000000pt}%
\definecolor{currentstroke}{rgb}{0.000000,0.000000,0.000000}%
\pgfsetstrokecolor{currentstroke}%
\pgfsetstrokeopacity{0.000000}%
\pgfsetdash{}{0pt}%
\pgfpathmoveto{\pgfqpoint{0.566985in}{0.528177in}}%
\pgfpathlineto{\pgfqpoint{2.582091in}{0.528177in}}%
\pgfpathlineto{\pgfqpoint{2.582091in}{3.392606in}}%
\pgfpathlineto{\pgfqpoint{0.566985in}{3.392606in}}%
\pgfpathclose%
\pgfusepath{fill}%
\end{pgfscope}%
\begin{pgfscope}%
\pgfsetbuttcap%
\pgfsetroundjoin%
\definecolor{currentfill}{rgb}{0.150000,0.150000,0.150000}%
\pgfsetfillcolor{currentfill}%
\pgfsetlinewidth{1.003750pt}%
\definecolor{currentstroke}{rgb}{0.150000,0.150000,0.150000}%
\pgfsetstrokecolor{currentstroke}%
\pgfsetdash{}{0pt}%
\pgfsys@defobject{currentmarker}{\pgfqpoint{0.000000in}{0.000000in}}{\pgfqpoint{0.000000in}{0.041667in}}{%
\pgfpathmoveto{\pgfqpoint{0.000000in}{0.000000in}}%
\pgfpathlineto{\pgfqpoint{0.000000in}{0.041667in}}%
\pgfusepath{stroke,fill}%
}%
\begin{pgfscope}%
\pgfsys@transformshift{0.566985in}{0.528177in}%
\pgfsys@useobject{currentmarker}{}%
\end{pgfscope}%
\end{pgfscope}%
\begin{pgfscope}%
\definecolor{textcolor}{rgb}{0.150000,0.150000,0.150000}%
\pgfsetstrokecolor{textcolor}%
\pgfsetfillcolor{textcolor}%
\pgftext[x=0.566985in,y=0.430955in,,top]{\color{textcolor}\rmfamily\fontsize{10.000000}{12.000000}\selectfont \(\displaystyle 0\)}%
\end{pgfscope}%
\begin{pgfscope}%
\pgfsetbuttcap%
\pgfsetroundjoin%
\definecolor{currentfill}{rgb}{0.150000,0.150000,0.150000}%
\pgfsetfillcolor{currentfill}%
\pgfsetlinewidth{1.003750pt}%
\definecolor{currentstroke}{rgb}{0.150000,0.150000,0.150000}%
\pgfsetstrokecolor{currentstroke}%
\pgfsetdash{}{0pt}%
\pgfsys@defobject{currentmarker}{\pgfqpoint{0.000000in}{0.000000in}}{\pgfqpoint{0.000000in}{0.041667in}}{%
\pgfpathmoveto{\pgfqpoint{0.000000in}{0.000000in}}%
\pgfpathlineto{\pgfqpoint{0.000000in}{0.041667in}}%
\pgfusepath{stroke,fill}%
}%
\begin{pgfscope}%
\pgfsys@transformshift{0.970006in}{0.528177in}%
\pgfsys@useobject{currentmarker}{}%
\end{pgfscope}%
\end{pgfscope}%
\begin{pgfscope}%
\definecolor{textcolor}{rgb}{0.150000,0.150000,0.150000}%
\pgfsetstrokecolor{textcolor}%
\pgfsetfillcolor{textcolor}%
\pgftext[x=0.970006in,y=0.430955in,,top]{\color{textcolor}\rmfamily\fontsize{10.000000}{12.000000}\selectfont \(\displaystyle 10\)}%
\end{pgfscope}%
\begin{pgfscope}%
\pgfsetbuttcap%
\pgfsetroundjoin%
\definecolor{currentfill}{rgb}{0.150000,0.150000,0.150000}%
\pgfsetfillcolor{currentfill}%
\pgfsetlinewidth{1.003750pt}%
\definecolor{currentstroke}{rgb}{0.150000,0.150000,0.150000}%
\pgfsetstrokecolor{currentstroke}%
\pgfsetdash{}{0pt}%
\pgfsys@defobject{currentmarker}{\pgfqpoint{0.000000in}{0.000000in}}{\pgfqpoint{0.000000in}{0.041667in}}{%
\pgfpathmoveto{\pgfqpoint{0.000000in}{0.000000in}}%
\pgfpathlineto{\pgfqpoint{0.000000in}{0.041667in}}%
\pgfusepath{stroke,fill}%
}%
\begin{pgfscope}%
\pgfsys@transformshift{1.373027in}{0.528177in}%
\pgfsys@useobject{currentmarker}{}%
\end{pgfscope}%
\end{pgfscope}%
\begin{pgfscope}%
\definecolor{textcolor}{rgb}{0.150000,0.150000,0.150000}%
\pgfsetstrokecolor{textcolor}%
\pgfsetfillcolor{textcolor}%
\pgftext[x=1.373027in,y=0.430955in,,top]{\color{textcolor}\rmfamily\fontsize{10.000000}{12.000000}\selectfont \(\displaystyle 20\)}%
\end{pgfscope}%
\begin{pgfscope}%
\pgfsetbuttcap%
\pgfsetroundjoin%
\definecolor{currentfill}{rgb}{0.150000,0.150000,0.150000}%
\pgfsetfillcolor{currentfill}%
\pgfsetlinewidth{1.003750pt}%
\definecolor{currentstroke}{rgb}{0.150000,0.150000,0.150000}%
\pgfsetstrokecolor{currentstroke}%
\pgfsetdash{}{0pt}%
\pgfsys@defobject{currentmarker}{\pgfqpoint{0.000000in}{0.000000in}}{\pgfqpoint{0.000000in}{0.041667in}}{%
\pgfpathmoveto{\pgfqpoint{0.000000in}{0.000000in}}%
\pgfpathlineto{\pgfqpoint{0.000000in}{0.041667in}}%
\pgfusepath{stroke,fill}%
}%
\begin{pgfscope}%
\pgfsys@transformshift{1.776048in}{0.528177in}%
\pgfsys@useobject{currentmarker}{}%
\end{pgfscope}%
\end{pgfscope}%
\begin{pgfscope}%
\definecolor{textcolor}{rgb}{0.150000,0.150000,0.150000}%
\pgfsetstrokecolor{textcolor}%
\pgfsetfillcolor{textcolor}%
\pgftext[x=1.776048in,y=0.430955in,,top]{\color{textcolor}\rmfamily\fontsize{10.000000}{12.000000}\selectfont \(\displaystyle 30\)}%
\end{pgfscope}%
\begin{pgfscope}%
\pgfsetbuttcap%
\pgfsetroundjoin%
\definecolor{currentfill}{rgb}{0.150000,0.150000,0.150000}%
\pgfsetfillcolor{currentfill}%
\pgfsetlinewidth{1.003750pt}%
\definecolor{currentstroke}{rgb}{0.150000,0.150000,0.150000}%
\pgfsetstrokecolor{currentstroke}%
\pgfsetdash{}{0pt}%
\pgfsys@defobject{currentmarker}{\pgfqpoint{0.000000in}{0.000000in}}{\pgfqpoint{0.000000in}{0.041667in}}{%
\pgfpathmoveto{\pgfqpoint{0.000000in}{0.000000in}}%
\pgfpathlineto{\pgfqpoint{0.000000in}{0.041667in}}%
\pgfusepath{stroke,fill}%
}%
\begin{pgfscope}%
\pgfsys@transformshift{2.179070in}{0.528177in}%
\pgfsys@useobject{currentmarker}{}%
\end{pgfscope}%
\end{pgfscope}%
\begin{pgfscope}%
\definecolor{textcolor}{rgb}{0.150000,0.150000,0.150000}%
\pgfsetstrokecolor{textcolor}%
\pgfsetfillcolor{textcolor}%
\pgftext[x=2.179070in,y=0.430955in,,top]{\color{textcolor}\rmfamily\fontsize{10.000000}{12.000000}\selectfont \(\displaystyle 40\)}%
\end{pgfscope}%
\begin{pgfscope}%
\pgfsetbuttcap%
\pgfsetroundjoin%
\definecolor{currentfill}{rgb}{0.150000,0.150000,0.150000}%
\pgfsetfillcolor{currentfill}%
\pgfsetlinewidth{1.003750pt}%
\definecolor{currentstroke}{rgb}{0.150000,0.150000,0.150000}%
\pgfsetstrokecolor{currentstroke}%
\pgfsetdash{}{0pt}%
\pgfsys@defobject{currentmarker}{\pgfqpoint{0.000000in}{0.000000in}}{\pgfqpoint{0.000000in}{0.041667in}}{%
\pgfpathmoveto{\pgfqpoint{0.000000in}{0.000000in}}%
\pgfpathlineto{\pgfqpoint{0.000000in}{0.041667in}}%
\pgfusepath{stroke,fill}%
}%
\begin{pgfscope}%
\pgfsys@transformshift{2.582091in}{0.528177in}%
\pgfsys@useobject{currentmarker}{}%
\end{pgfscope}%
\end{pgfscope}%
\begin{pgfscope}%
\definecolor{textcolor}{rgb}{0.150000,0.150000,0.150000}%
\pgfsetstrokecolor{textcolor}%
\pgfsetfillcolor{textcolor}%
\pgftext[x=2.582091in,y=0.430955in,,top]{\color{textcolor}\rmfamily\fontsize{10.000000}{12.000000}\selectfont \(\displaystyle 50\)}%
\end{pgfscope}%
\begin{pgfscope}%
\definecolor{textcolor}{rgb}{0.150000,0.150000,0.150000}%
\pgfsetstrokecolor{textcolor}%
\pgfsetfillcolor{textcolor}%
\pgftext[x=1.574538in,y=0.238855in,,top]{\color{textcolor}\rmfamily\fontsize{10.000000}{12.000000}\selectfont \textbf{Cell activity (freezing, a.u.)}}%
\end{pgfscope}%
\begin{pgfscope}%
\pgfsetbuttcap%
\pgfsetroundjoin%
\definecolor{currentfill}{rgb}{0.150000,0.150000,0.150000}%
\pgfsetfillcolor{currentfill}%
\pgfsetlinewidth{1.003750pt}%
\definecolor{currentstroke}{rgb}{0.150000,0.150000,0.150000}%
\pgfsetstrokecolor{currentstroke}%
\pgfsetdash{}{0pt}%
\pgfsys@defobject{currentmarker}{\pgfqpoint{0.000000in}{0.000000in}}{\pgfqpoint{0.041667in}{0.000000in}}{%
\pgfpathmoveto{\pgfqpoint{0.000000in}{0.000000in}}%
\pgfpathlineto{\pgfqpoint{0.041667in}{0.000000in}}%
\pgfusepath{stroke,fill}%
}%
\begin{pgfscope}%
\pgfsys@transformshift{0.566985in}{0.528177in}%
\pgfsys@useobject{currentmarker}{}%
\end{pgfscope}%
\end{pgfscope}%
\begin{pgfscope}%
\definecolor{textcolor}{rgb}{0.150000,0.150000,0.150000}%
\pgfsetstrokecolor{textcolor}%
\pgfsetfillcolor{textcolor}%
\pgftext[x=0.469762in,y=0.528177in,right,]{\color{textcolor}\rmfamily\fontsize{10.000000}{12.000000}\selectfont \(\displaystyle 0.4\)}%
\end{pgfscope}%
\begin{pgfscope}%
\pgfsetbuttcap%
\pgfsetroundjoin%
\definecolor{currentfill}{rgb}{0.150000,0.150000,0.150000}%
\pgfsetfillcolor{currentfill}%
\pgfsetlinewidth{1.003750pt}%
\definecolor{currentstroke}{rgb}{0.150000,0.150000,0.150000}%
\pgfsetstrokecolor{currentstroke}%
\pgfsetdash{}{0pt}%
\pgfsys@defobject{currentmarker}{\pgfqpoint{0.000000in}{0.000000in}}{\pgfqpoint{0.041667in}{0.000000in}}{%
\pgfpathmoveto{\pgfqpoint{0.000000in}{0.000000in}}%
\pgfpathlineto{\pgfqpoint{0.041667in}{0.000000in}}%
\pgfusepath{stroke,fill}%
}%
\begin{pgfscope}%
\pgfsys@transformshift{0.566985in}{1.005582in}%
\pgfsys@useobject{currentmarker}{}%
\end{pgfscope}%
\end{pgfscope}%
\begin{pgfscope}%
\definecolor{textcolor}{rgb}{0.150000,0.150000,0.150000}%
\pgfsetstrokecolor{textcolor}%
\pgfsetfillcolor{textcolor}%
\pgftext[x=0.469762in,y=1.005582in,right,]{\color{textcolor}\rmfamily\fontsize{10.000000}{12.000000}\selectfont \(\displaystyle 0.5\)}%
\end{pgfscope}%
\begin{pgfscope}%
\pgfsetbuttcap%
\pgfsetroundjoin%
\definecolor{currentfill}{rgb}{0.150000,0.150000,0.150000}%
\pgfsetfillcolor{currentfill}%
\pgfsetlinewidth{1.003750pt}%
\definecolor{currentstroke}{rgb}{0.150000,0.150000,0.150000}%
\pgfsetstrokecolor{currentstroke}%
\pgfsetdash{}{0pt}%
\pgfsys@defobject{currentmarker}{\pgfqpoint{0.000000in}{0.000000in}}{\pgfqpoint{0.041667in}{0.000000in}}{%
\pgfpathmoveto{\pgfqpoint{0.000000in}{0.000000in}}%
\pgfpathlineto{\pgfqpoint{0.041667in}{0.000000in}}%
\pgfusepath{stroke,fill}%
}%
\begin{pgfscope}%
\pgfsys@transformshift{0.566985in}{1.482987in}%
\pgfsys@useobject{currentmarker}{}%
\end{pgfscope}%
\end{pgfscope}%
\begin{pgfscope}%
\definecolor{textcolor}{rgb}{0.150000,0.150000,0.150000}%
\pgfsetstrokecolor{textcolor}%
\pgfsetfillcolor{textcolor}%
\pgftext[x=0.469762in,y=1.482987in,right,]{\color{textcolor}\rmfamily\fontsize{10.000000}{12.000000}\selectfont \(\displaystyle 0.6\)}%
\end{pgfscope}%
\begin{pgfscope}%
\pgfsetbuttcap%
\pgfsetroundjoin%
\definecolor{currentfill}{rgb}{0.150000,0.150000,0.150000}%
\pgfsetfillcolor{currentfill}%
\pgfsetlinewidth{1.003750pt}%
\definecolor{currentstroke}{rgb}{0.150000,0.150000,0.150000}%
\pgfsetstrokecolor{currentstroke}%
\pgfsetdash{}{0pt}%
\pgfsys@defobject{currentmarker}{\pgfqpoint{0.000000in}{0.000000in}}{\pgfqpoint{0.041667in}{0.000000in}}{%
\pgfpathmoveto{\pgfqpoint{0.000000in}{0.000000in}}%
\pgfpathlineto{\pgfqpoint{0.041667in}{0.000000in}}%
\pgfusepath{stroke,fill}%
}%
\begin{pgfscope}%
\pgfsys@transformshift{0.566985in}{1.960392in}%
\pgfsys@useobject{currentmarker}{}%
\end{pgfscope}%
\end{pgfscope}%
\begin{pgfscope}%
\definecolor{textcolor}{rgb}{0.150000,0.150000,0.150000}%
\pgfsetstrokecolor{textcolor}%
\pgfsetfillcolor{textcolor}%
\pgftext[x=0.469762in,y=1.960392in,right,]{\color{textcolor}\rmfamily\fontsize{10.000000}{12.000000}\selectfont \(\displaystyle 0.7\)}%
\end{pgfscope}%
\begin{pgfscope}%
\pgfsetbuttcap%
\pgfsetroundjoin%
\definecolor{currentfill}{rgb}{0.150000,0.150000,0.150000}%
\pgfsetfillcolor{currentfill}%
\pgfsetlinewidth{1.003750pt}%
\definecolor{currentstroke}{rgb}{0.150000,0.150000,0.150000}%
\pgfsetstrokecolor{currentstroke}%
\pgfsetdash{}{0pt}%
\pgfsys@defobject{currentmarker}{\pgfqpoint{0.000000in}{0.000000in}}{\pgfqpoint{0.041667in}{0.000000in}}{%
\pgfpathmoveto{\pgfqpoint{0.000000in}{0.000000in}}%
\pgfpathlineto{\pgfqpoint{0.041667in}{0.000000in}}%
\pgfusepath{stroke,fill}%
}%
\begin{pgfscope}%
\pgfsys@transformshift{0.566985in}{2.437796in}%
\pgfsys@useobject{currentmarker}{}%
\end{pgfscope}%
\end{pgfscope}%
\begin{pgfscope}%
\definecolor{textcolor}{rgb}{0.150000,0.150000,0.150000}%
\pgfsetstrokecolor{textcolor}%
\pgfsetfillcolor{textcolor}%
\pgftext[x=0.469762in,y=2.437796in,right,]{\color{textcolor}\rmfamily\fontsize{10.000000}{12.000000}\selectfont \(\displaystyle 0.8\)}%
\end{pgfscope}%
\begin{pgfscope}%
\pgfsetbuttcap%
\pgfsetroundjoin%
\definecolor{currentfill}{rgb}{0.150000,0.150000,0.150000}%
\pgfsetfillcolor{currentfill}%
\pgfsetlinewidth{1.003750pt}%
\definecolor{currentstroke}{rgb}{0.150000,0.150000,0.150000}%
\pgfsetstrokecolor{currentstroke}%
\pgfsetdash{}{0pt}%
\pgfsys@defobject{currentmarker}{\pgfqpoint{0.000000in}{0.000000in}}{\pgfqpoint{0.041667in}{0.000000in}}{%
\pgfpathmoveto{\pgfqpoint{0.000000in}{0.000000in}}%
\pgfpathlineto{\pgfqpoint{0.041667in}{0.000000in}}%
\pgfusepath{stroke,fill}%
}%
\begin{pgfscope}%
\pgfsys@transformshift{0.566985in}{2.915201in}%
\pgfsys@useobject{currentmarker}{}%
\end{pgfscope}%
\end{pgfscope}%
\begin{pgfscope}%
\definecolor{textcolor}{rgb}{0.150000,0.150000,0.150000}%
\pgfsetstrokecolor{textcolor}%
\pgfsetfillcolor{textcolor}%
\pgftext[x=0.469762in,y=2.915201in,right,]{\color{textcolor}\rmfamily\fontsize{10.000000}{12.000000}\selectfont \(\displaystyle 0.9\)}%
\end{pgfscope}%
\begin{pgfscope}%
\pgfsetbuttcap%
\pgfsetroundjoin%
\definecolor{currentfill}{rgb}{0.150000,0.150000,0.150000}%
\pgfsetfillcolor{currentfill}%
\pgfsetlinewidth{1.003750pt}%
\definecolor{currentstroke}{rgb}{0.150000,0.150000,0.150000}%
\pgfsetstrokecolor{currentstroke}%
\pgfsetdash{}{0pt}%
\pgfsys@defobject{currentmarker}{\pgfqpoint{0.000000in}{0.000000in}}{\pgfqpoint{0.041667in}{0.000000in}}{%
\pgfpathmoveto{\pgfqpoint{0.000000in}{0.000000in}}%
\pgfpathlineto{\pgfqpoint{0.041667in}{0.000000in}}%
\pgfusepath{stroke,fill}%
}%
\begin{pgfscope}%
\pgfsys@transformshift{0.566985in}{3.392606in}%
\pgfsys@useobject{currentmarker}{}%
\end{pgfscope}%
\end{pgfscope}%
\begin{pgfscope}%
\definecolor{textcolor}{rgb}{0.150000,0.150000,0.150000}%
\pgfsetstrokecolor{textcolor}%
\pgfsetfillcolor{textcolor}%
\pgftext[x=0.469762in,y=3.392606in,right,]{\color{textcolor}\rmfamily\fontsize{10.000000}{12.000000}\selectfont \(\displaystyle 1.0\)}%
\end{pgfscope}%
\begin{pgfscope}%
\definecolor{textcolor}{rgb}{0.150000,0.150000,0.150000}%
\pgfsetstrokecolor{textcolor}%
\pgfsetfillcolor{textcolor}%
\pgftext[x=0.222848in,y=1.960392in,,bottom,rotate=90.000000]{\color{textcolor}\rmfamily\fontsize{10.000000}{12.000000}\selectfont \textbf{Cumulative porportion}}%
\end{pgfscope}%
\begin{pgfscope}%
\pgfpathrectangle{\pgfqpoint{0.566985in}{0.528177in}}{\pgfqpoint{2.015106in}{2.864429in}} %
\pgfusepath{clip}%
\pgfsetroundcap%
\pgfsetroundjoin%
\pgfsetlinewidth{1.003750pt}%
\definecolor{currentstroke}{rgb}{0.200000,0.427451,0.650980}%
\pgfsetstrokecolor{currentstroke}%
\pgfsetdash{}{0pt}%
\pgfpathmoveto{\pgfqpoint{0.566985in}{0.694782in}}%
\pgfpathlineto{\pgfqpoint{0.584860in}{1.388267in}}%
\pgfpathlineto{\pgfqpoint{0.602736in}{1.853409in}}%
\pgfpathlineto{\pgfqpoint{0.620612in}{2.149408in}}%
\pgfpathlineto{\pgfqpoint{0.638488in}{2.335465in}}%
\pgfpathlineto{\pgfqpoint{0.656363in}{2.475008in}}%
\pgfpathlineto{\pgfqpoint{0.674239in}{2.606093in}}%
\pgfpathlineto{\pgfqpoint{0.692115in}{2.716036in}}%
\pgfpathlineto{\pgfqpoint{0.709991in}{2.830207in}}%
\pgfpathlineto{\pgfqpoint{0.727866in}{2.931693in}}%
\pgfpathlineto{\pgfqpoint{0.745742in}{3.020493in}}%
\pgfpathlineto{\pgfqpoint{0.763618in}{3.058550in}}%
\pgfpathlineto{\pgfqpoint{0.781494in}{3.121978in}}%
\pgfpathlineto{\pgfqpoint{0.799369in}{3.160035in}}%
\pgfpathlineto{\pgfqpoint{0.817245in}{3.185407in}}%
\pgfpathlineto{\pgfqpoint{0.835121in}{3.202321in}}%
\pgfpathlineto{\pgfqpoint{0.852997in}{3.223464in}}%
\pgfpathlineto{\pgfqpoint{0.870872in}{3.240378in}}%
\pgfpathlineto{\pgfqpoint{0.888748in}{3.261521in}}%
\pgfpathlineto{\pgfqpoint{0.906624in}{3.274206in}}%
\pgfpathlineto{\pgfqpoint{0.924500in}{3.295349in}}%
\pgfpathlineto{\pgfqpoint{0.942375in}{3.312264in}}%
\pgfpathlineto{\pgfqpoint{0.960251in}{3.312264in}}%
\pgfpathlineto{\pgfqpoint{0.978127in}{3.312264in}}%
\pgfpathlineto{\pgfqpoint{0.996003in}{3.316492in}}%
\pgfpathlineto{\pgfqpoint{1.013878in}{3.324949in}}%
\pgfpathlineto{\pgfqpoint{1.031754in}{3.333406in}}%
\pgfpathlineto{\pgfqpoint{1.049630in}{3.337635in}}%
\pgfpathlineto{\pgfqpoint{1.067506in}{3.354549in}}%
\pgfpathlineto{\pgfqpoint{1.085381in}{3.363006in}}%
\pgfpathlineto{\pgfqpoint{1.103257in}{3.367235in}}%
\pgfpathlineto{\pgfqpoint{1.121133in}{3.371463in}}%
\pgfpathlineto{\pgfqpoint{1.139009in}{3.371463in}}%
\pgfpathlineto{\pgfqpoint{1.156884in}{3.379921in}}%
\pgfpathlineto{\pgfqpoint{1.174760in}{3.379921in}}%
\pgfpathlineto{\pgfqpoint{1.192636in}{3.379921in}}%
\pgfpathlineto{\pgfqpoint{1.210512in}{3.379921in}}%
\pgfpathlineto{\pgfqpoint{1.228387in}{3.384149in}}%
\pgfpathlineto{\pgfqpoint{1.246263in}{3.384149in}}%
\pgfpathlineto{\pgfqpoint{1.264139in}{3.384149in}}%
\pgfpathlineto{\pgfqpoint{1.282015in}{3.384149in}}%
\pgfpathlineto{\pgfqpoint{1.299890in}{3.384149in}}%
\pgfpathlineto{\pgfqpoint{1.317766in}{3.384149in}}%
\pgfpathlineto{\pgfqpoint{1.335642in}{3.384149in}}%
\pgfpathlineto{\pgfqpoint{1.353518in}{3.384149in}}%
\pgfpathlineto{\pgfqpoint{1.371393in}{3.384149in}}%
\pgfpathlineto{\pgfqpoint{1.389269in}{3.384149in}}%
\pgfpathlineto{\pgfqpoint{1.407145in}{3.384149in}}%
\pgfpathlineto{\pgfqpoint{1.425021in}{3.388378in}}%
\pgfpathlineto{\pgfqpoint{1.442896in}{3.392606in}}%
\pgfusepath{stroke}%
\end{pgfscope}%
\begin{pgfscope}%
\pgfpathrectangle{\pgfqpoint{0.566985in}{0.528177in}}{\pgfqpoint{2.015106in}{2.864429in}} %
\pgfusepath{clip}%
\pgfsetroundcap%
\pgfsetroundjoin%
\pgfsetlinewidth{1.003750pt}%
\definecolor{currentstroke}{rgb}{0.168627,0.670588,0.494118}%
\pgfsetstrokecolor{currentstroke}%
\pgfsetdash{}{0pt}%
\pgfpathmoveto{\pgfqpoint{0.566985in}{0.669906in}}%
\pgfpathlineto{\pgfqpoint{0.576849in}{1.145446in}}%
\pgfpathlineto{\pgfqpoint{0.586714in}{1.574365in}}%
\pgfpathlineto{\pgfqpoint{0.596579in}{1.872743in}}%
\pgfpathlineto{\pgfqpoint{0.606444in}{2.096527in}}%
\pgfpathlineto{\pgfqpoint{0.616309in}{2.301661in}}%
\pgfpathlineto{\pgfqpoint{0.626174in}{2.404229in}}%
\pgfpathlineto{\pgfqpoint{0.636038in}{2.534769in}}%
\pgfpathlineto{\pgfqpoint{0.645903in}{2.637337in}}%
\pgfpathlineto{\pgfqpoint{0.655768in}{2.730580in}}%
\pgfpathlineto{\pgfqpoint{0.665633in}{2.861120in}}%
\pgfpathlineto{\pgfqpoint{0.675498in}{2.926391in}}%
\pgfpathlineto{\pgfqpoint{0.685362in}{2.991661in}}%
\pgfpathlineto{\pgfqpoint{0.695227in}{3.000985in}}%
\pgfpathlineto{\pgfqpoint{0.705092in}{3.028958in}}%
\pgfpathlineto{\pgfqpoint{0.714957in}{3.066255in}}%
\pgfpathlineto{\pgfqpoint{0.724822in}{3.112877in}}%
\pgfpathlineto{\pgfqpoint{0.734687in}{3.122201in}}%
\pgfpathlineto{\pgfqpoint{0.744551in}{3.150174in}}%
\pgfpathlineto{\pgfqpoint{0.754416in}{3.178147in}}%
\pgfpathlineto{\pgfqpoint{0.764281in}{3.196796in}}%
\pgfpathlineto{\pgfqpoint{0.774146in}{3.206120in}}%
\pgfpathlineto{\pgfqpoint{0.784011in}{3.215444in}}%
\pgfpathlineto{\pgfqpoint{0.793875in}{3.215444in}}%
\pgfpathlineto{\pgfqpoint{0.803740in}{3.243417in}}%
\pgfpathlineto{\pgfqpoint{0.813605in}{3.262066in}}%
\pgfpathlineto{\pgfqpoint{0.823470in}{3.271390in}}%
\pgfpathlineto{\pgfqpoint{0.833335in}{3.280714in}}%
\pgfpathlineto{\pgfqpoint{0.843200in}{3.290039in}}%
\pgfpathlineto{\pgfqpoint{0.853064in}{3.290039in}}%
\pgfpathlineto{\pgfqpoint{0.862929in}{3.290039in}}%
\pgfpathlineto{\pgfqpoint{0.872794in}{3.308687in}}%
\pgfpathlineto{\pgfqpoint{0.882659in}{3.318012in}}%
\pgfpathlineto{\pgfqpoint{0.892524in}{3.327336in}}%
\pgfpathlineto{\pgfqpoint{0.902389in}{3.336660in}}%
\pgfpathlineto{\pgfqpoint{0.912253in}{3.336660in}}%
\pgfpathlineto{\pgfqpoint{0.922118in}{3.336660in}}%
\pgfpathlineto{\pgfqpoint{0.931983in}{3.336660in}}%
\pgfpathlineto{\pgfqpoint{0.941848in}{3.345985in}}%
\pgfpathlineto{\pgfqpoint{0.951713in}{3.345985in}}%
\pgfpathlineto{\pgfqpoint{0.961577in}{3.355309in}}%
\pgfpathlineto{\pgfqpoint{0.971442in}{3.355309in}}%
\pgfpathlineto{\pgfqpoint{0.981307in}{3.355309in}}%
\pgfpathlineto{\pgfqpoint{0.991172in}{3.364633in}}%
\pgfpathlineto{\pgfqpoint{1.001037in}{3.373958in}}%
\pgfpathlineto{\pgfqpoint{1.010902in}{3.373958in}}%
\pgfpathlineto{\pgfqpoint{1.020766in}{3.373958in}}%
\pgfpathlineto{\pgfqpoint{1.030631in}{3.383282in}}%
\pgfpathlineto{\pgfqpoint{1.040496in}{3.383282in}}%
\pgfpathlineto{\pgfqpoint{1.050361in}{3.392606in}}%
\pgfusepath{stroke}%
\end{pgfscope}%
\begin{pgfscope}%
\pgfpathrectangle{\pgfqpoint{0.566985in}{0.528177in}}{\pgfqpoint{2.015106in}{2.864429in}} %
\pgfusepath{clip}%
\pgfsetroundcap%
\pgfsetroundjoin%
\pgfsetlinewidth{1.003750pt}%
\definecolor{currentstroke}{rgb}{1.000000,0.494118,0.250980}%
\pgfsetstrokecolor{currentstroke}%
\pgfsetdash{}{0pt}%
\pgfpathmoveto{\pgfqpoint{0.566985in}{1.738900in}}%
\pgfpathlineto{\pgfqpoint{0.606375in}{2.292629in}}%
\pgfpathlineto{\pgfqpoint{0.645765in}{2.591943in}}%
\pgfpathlineto{\pgfqpoint{0.685156in}{2.779014in}}%
\pgfpathlineto{\pgfqpoint{0.724546in}{2.891256in}}%
\pgfpathlineto{\pgfqpoint{0.763936in}{3.003499in}}%
\pgfpathlineto{\pgfqpoint{0.803327in}{3.078327in}}%
\pgfpathlineto{\pgfqpoint{0.842717in}{3.138190in}}%
\pgfpathlineto{\pgfqpoint{0.882107in}{3.175604in}}%
\pgfpathlineto{\pgfqpoint{0.921498in}{3.183087in}}%
\pgfpathlineto{\pgfqpoint{0.960888in}{3.190570in}}%
\pgfpathlineto{\pgfqpoint{1.000278in}{3.213018in}}%
\pgfpathlineto{\pgfqpoint{1.039669in}{3.227984in}}%
\pgfpathlineto{\pgfqpoint{1.079059in}{3.257915in}}%
\pgfpathlineto{\pgfqpoint{1.118449in}{3.280364in}}%
\pgfpathlineto{\pgfqpoint{1.157839in}{3.302812in}}%
\pgfpathlineto{\pgfqpoint{1.197230in}{3.317778in}}%
\pgfpathlineto{\pgfqpoint{1.236620in}{3.317778in}}%
\pgfpathlineto{\pgfqpoint{1.276010in}{3.325261in}}%
\pgfpathlineto{\pgfqpoint{1.315401in}{3.332744in}}%
\pgfpathlineto{\pgfqpoint{1.354791in}{3.340226in}}%
\pgfpathlineto{\pgfqpoint{1.394181in}{3.355192in}}%
\pgfpathlineto{\pgfqpoint{1.433572in}{3.355192in}}%
\pgfpathlineto{\pgfqpoint{1.472962in}{3.362675in}}%
\pgfpathlineto{\pgfqpoint{1.512352in}{3.362675in}}%
\pgfpathlineto{\pgfqpoint{1.551743in}{3.370158in}}%
\pgfpathlineto{\pgfqpoint{1.591133in}{3.370158in}}%
\pgfpathlineto{\pgfqpoint{1.630523in}{3.370158in}}%
\pgfpathlineto{\pgfqpoint{1.669914in}{3.370158in}}%
\pgfpathlineto{\pgfqpoint{1.709304in}{3.377641in}}%
\pgfpathlineto{\pgfqpoint{1.748694in}{3.377641in}}%
\pgfpathlineto{\pgfqpoint{1.788085in}{3.377641in}}%
\pgfpathlineto{\pgfqpoint{1.827475in}{3.377641in}}%
\pgfpathlineto{\pgfqpoint{1.866865in}{3.385123in}}%
\pgfpathlineto{\pgfqpoint{1.906256in}{3.385123in}}%
\pgfpathlineto{\pgfqpoint{1.945646in}{3.385123in}}%
\pgfpathlineto{\pgfqpoint{1.985036in}{3.385123in}}%
\pgfpathlineto{\pgfqpoint{2.024427in}{3.385123in}}%
\pgfpathlineto{\pgfqpoint{2.063817in}{3.385123in}}%
\pgfpathlineto{\pgfqpoint{2.103207in}{3.385123in}}%
\pgfpathlineto{\pgfqpoint{2.142598in}{3.385123in}}%
\pgfpathlineto{\pgfqpoint{2.181988in}{3.385123in}}%
\pgfpathlineto{\pgfqpoint{2.221378in}{3.385123in}}%
\pgfpathlineto{\pgfqpoint{2.260769in}{3.385123in}}%
\pgfpathlineto{\pgfqpoint{2.300159in}{3.385123in}}%
\pgfpathlineto{\pgfqpoint{2.339549in}{3.385123in}}%
\pgfpathlineto{\pgfqpoint{2.378940in}{3.385123in}}%
\pgfpathlineto{\pgfqpoint{2.418330in}{3.385123in}}%
\pgfpathlineto{\pgfqpoint{2.457720in}{3.385123in}}%
\pgfpathlineto{\pgfqpoint{2.497111in}{3.392606in}}%
\pgfusepath{stroke}%
\end{pgfscope}%
\begin{pgfscope}%
\pgfpathrectangle{\pgfqpoint{0.566985in}{0.528177in}}{\pgfqpoint{2.015106in}{2.864429in}} %
\pgfusepath{clip}%
\pgfsetroundcap%
\pgfsetroundjoin%
\pgfsetlinewidth{1.003750pt}%
\definecolor{currentstroke}{rgb}{1.000000,0.694118,0.250980}%
\pgfsetstrokecolor{currentstroke}%
\pgfsetdash{}{0pt}%
\pgfpathmoveto{\pgfqpoint{0.566985in}{1.084206in}}%
\pgfpathlineto{\pgfqpoint{0.583553in}{1.706908in}}%
\pgfpathlineto{\pgfqpoint{0.600120in}{2.040274in}}%
\pgfpathlineto{\pgfqpoint{0.616688in}{2.298160in}}%
\pgfpathlineto{\pgfqpoint{0.633256in}{2.455408in}}%
\pgfpathlineto{\pgfqpoint{0.649824in}{2.593786in}}%
\pgfpathlineto{\pgfqpoint{0.666392in}{2.662976in}}%
\pgfpathlineto{\pgfqpoint{0.682960in}{2.757324in}}%
\pgfpathlineto{\pgfqpoint{0.699528in}{2.839093in}}%
\pgfpathlineto{\pgfqpoint{0.716096in}{2.895703in}}%
\pgfpathlineto{\pgfqpoint{0.732664in}{2.946022in}}%
\pgfpathlineto{\pgfqpoint{0.749232in}{3.015211in}}%
\pgfpathlineto{\pgfqpoint{0.765800in}{3.078110in}}%
\pgfpathlineto{\pgfqpoint{0.782368in}{3.122140in}}%
\pgfpathlineto{\pgfqpoint{0.798936in}{3.159879in}}%
\pgfpathlineto{\pgfqpoint{0.815504in}{3.166169in}}%
\pgfpathlineto{\pgfqpoint{0.832072in}{3.178749in}}%
\pgfpathlineto{\pgfqpoint{0.848640in}{3.210199in}}%
\pgfpathlineto{\pgfqpoint{0.865208in}{3.235358in}}%
\pgfpathlineto{\pgfqpoint{0.881776in}{3.247938in}}%
\pgfpathlineto{\pgfqpoint{0.898344in}{3.266808in}}%
\pgfpathlineto{\pgfqpoint{0.914912in}{3.273098in}}%
\pgfpathlineto{\pgfqpoint{0.931479in}{3.285678in}}%
\pgfpathlineto{\pgfqpoint{0.948047in}{3.291968in}}%
\pgfpathlineto{\pgfqpoint{0.964615in}{3.304547in}}%
\pgfpathlineto{\pgfqpoint{0.981183in}{3.310837in}}%
\pgfpathlineto{\pgfqpoint{0.997751in}{3.335997in}}%
\pgfpathlineto{\pgfqpoint{1.014319in}{3.342287in}}%
\pgfpathlineto{\pgfqpoint{1.030887in}{3.354867in}}%
\pgfpathlineto{\pgfqpoint{1.047455in}{3.354867in}}%
\pgfpathlineto{\pgfqpoint{1.064023in}{3.354867in}}%
\pgfpathlineto{\pgfqpoint{1.080591in}{3.361157in}}%
\pgfpathlineto{\pgfqpoint{1.097159in}{3.367447in}}%
\pgfpathlineto{\pgfqpoint{1.113727in}{3.367447in}}%
\pgfpathlineto{\pgfqpoint{1.130295in}{3.367447in}}%
\pgfpathlineto{\pgfqpoint{1.146863in}{3.367447in}}%
\pgfpathlineto{\pgfqpoint{1.163431in}{3.373736in}}%
\pgfpathlineto{\pgfqpoint{1.179999in}{3.380026in}}%
\pgfpathlineto{\pgfqpoint{1.196567in}{3.380026in}}%
\pgfpathlineto{\pgfqpoint{1.213135in}{3.380026in}}%
\pgfpathlineto{\pgfqpoint{1.229703in}{3.380026in}}%
\pgfpathlineto{\pgfqpoint{1.246270in}{3.386316in}}%
\pgfpathlineto{\pgfqpoint{1.262838in}{3.386316in}}%
\pgfpathlineto{\pgfqpoint{1.279406in}{3.386316in}}%
\pgfpathlineto{\pgfqpoint{1.295974in}{3.386316in}}%
\pgfpathlineto{\pgfqpoint{1.312542in}{3.386316in}}%
\pgfpathlineto{\pgfqpoint{1.329110in}{3.386316in}}%
\pgfpathlineto{\pgfqpoint{1.345678in}{3.386316in}}%
\pgfpathlineto{\pgfqpoint{1.362246in}{3.386316in}}%
\pgfpathlineto{\pgfqpoint{1.378814in}{3.392606in}}%
\pgfusepath{stroke}%
\end{pgfscope}%
\begin{pgfscope}%
\pgfsetrectcap%
\pgfsetmiterjoin%
\pgfsetlinewidth{1.254687pt}%
\definecolor{currentstroke}{rgb}{0.150000,0.150000,0.150000}%
\pgfsetstrokecolor{currentstroke}%
\pgfsetdash{}{0pt}%
\pgfpathmoveto{\pgfqpoint{0.566985in}{0.528177in}}%
\pgfpathlineto{\pgfqpoint{0.566985in}{3.392606in}}%
\pgfusepath{stroke}%
\end{pgfscope}%
\begin{pgfscope}%
\pgfsetrectcap%
\pgfsetmiterjoin%
\pgfsetlinewidth{1.254687pt}%
\definecolor{currentstroke}{rgb}{0.150000,0.150000,0.150000}%
\pgfsetstrokecolor{currentstroke}%
\pgfsetdash{}{0pt}%
\pgfpathmoveto{\pgfqpoint{0.566985in}{0.528177in}}%
\pgfpathlineto{\pgfqpoint{2.582091in}{0.528177in}}%
\pgfusepath{stroke}%
\end{pgfscope}%
\begin{pgfscope}%
\pgfsetbuttcap%
\pgfsetmiterjoin%
\definecolor{currentfill}{rgb}{1.000000,1.000000,1.000000}%
\pgfsetfillcolor{currentfill}%
\pgfsetlinewidth{0.000000pt}%
\definecolor{currentstroke}{rgb}{0.000000,0.000000,0.000000}%
\pgfsetstrokecolor{currentstroke}%
\pgfsetstrokeopacity{0.000000}%
\pgfsetdash{}{0pt}%
\pgfpathmoveto{\pgfqpoint{3.186623in}{0.528177in}}%
\pgfpathlineto{\pgfqpoint{5.201729in}{0.528177in}}%
\pgfpathlineto{\pgfqpoint{5.201729in}{2.653399in}}%
\pgfpathlineto{\pgfqpoint{3.186623in}{2.653399in}}%
\pgfpathclose%
\pgfusepath{fill}%
\end{pgfscope}%
\begin{pgfscope}%
\pgfsetroundcap%
\pgfsetroundjoin%
\pgfsetlinewidth{1.003750pt}%
\definecolor{currentstroke}{rgb}{0.200000,0.427451,0.650980}%
\pgfsetstrokecolor{currentstroke}%
\pgfsetdash{}{0pt}%
\pgfpathmoveto{\pgfqpoint{3.085112in}{3.319977in}}%
\pgfpathlineto{\pgfqpoint{3.196223in}{3.319977in}}%
\pgfusepath{stroke}%
\end{pgfscope}%
\begin{pgfscope}%
\definecolor{textcolor}{rgb}{1.000000,1.000000,1.000000}%
\pgfsetstrokecolor{textcolor}%
\pgfsetfillcolor{textcolor}%
\pgftext[x=3.285112in,y=3.281088in,left,base]{\color{textcolor}\rmfamily\fontsize{8.000000}{9.600000}\selectfont WT + Vehicle (1129)}%
\end{pgfscope}%
\begin{pgfscope}%
\pgfsetroundcap%
\pgfsetroundjoin%
\pgfsetlinewidth{1.003750pt}%
\definecolor{currentstroke}{rgb}{0.168627,0.670588,0.494118}%
\pgfsetstrokecolor{currentstroke}%
\pgfsetdash{}{0pt}%
\pgfpathmoveto{\pgfqpoint{3.085112in}{3.153338in}}%
\pgfpathlineto{\pgfqpoint{3.196223in}{3.153338in}}%
\pgfusepath{stroke}%
\end{pgfscope}%
\begin{pgfscope}%
\definecolor{textcolor}{rgb}{1.000000,1.000000,1.000000}%
\pgfsetstrokecolor{textcolor}%
\pgfsetfillcolor{textcolor}%
\pgftext[x=3.285112in,y=3.114449in,left,base]{\color{textcolor}\rmfamily\fontsize{8.000000}{9.600000}\selectfont WT + TAT-GluA2\textsubscript{3Y} (512)}%
\end{pgfscope}%
\begin{pgfscope}%
\pgfsetroundcap%
\pgfsetroundjoin%
\pgfsetlinewidth{1.003750pt}%
\definecolor{currentstroke}{rgb}{1.000000,0.494118,0.250980}%
\pgfsetstrokecolor{currentstroke}%
\pgfsetdash{}{0pt}%
\pgfpathmoveto{\pgfqpoint{3.085112in}{2.986698in}}%
\pgfpathlineto{\pgfqpoint{3.196223in}{2.986698in}}%
\pgfusepath{stroke}%
\end{pgfscope}%
\begin{pgfscope}%
\definecolor{textcolor}{rgb}{1.000000,1.000000,1.000000}%
\pgfsetstrokecolor{textcolor}%
\pgfsetfillcolor{textcolor}%
\pgftext[x=3.285112in,y=2.947809in,left,base]{\color{textcolor}\rmfamily\fontsize{8.000000}{9.600000}\selectfont Tg + Vehicle (638)}%
\end{pgfscope}%
\begin{pgfscope}%
\pgfsetroundcap%
\pgfsetroundjoin%
\pgfsetlinewidth{1.003750pt}%
\definecolor{currentstroke}{rgb}{1.000000,0.694118,0.250980}%
\pgfsetstrokecolor{currentstroke}%
\pgfsetdash{}{0pt}%
\pgfpathmoveto{\pgfqpoint{3.085112in}{2.820059in}}%
\pgfpathlineto{\pgfqpoint{3.196223in}{2.820059in}}%
\pgfusepath{stroke}%
\end{pgfscope}%
\begin{pgfscope}%
\definecolor{textcolor}{rgb}{1.000000,1.000000,1.000000}%
\pgfsetstrokecolor{textcolor}%
\pgfsetfillcolor{textcolor}%
\pgftext[x=3.285112in,y=2.781170in,left,base]{\color{textcolor}\rmfamily\fontsize{8.000000}{9.600000}\selectfont Tg + TAT-GluA2\textsubscript{3Y} (759)}%
\end{pgfscope}%
\begin{pgfscope}%
\pgfsetroundcap%
\pgfsetroundjoin%
\pgfsetlinewidth{1.003750pt}%
\definecolor{currentstroke}{rgb}{0.200000,0.427451,0.650980}%
\pgfsetstrokecolor{currentstroke}%
\pgfsetdash{}{0pt}%
\pgfpathmoveto{\pgfqpoint{3.085112in}{3.319977in}}%
\pgfpathlineto{\pgfqpoint{3.196223in}{3.319977in}}%
\pgfusepath{stroke}%
\end{pgfscope}%
\begin{pgfscope}%
\definecolor{textcolor}{rgb}{1.000000,1.000000,1.000000}%
\pgfsetstrokecolor{textcolor}%
\pgfsetfillcolor{textcolor}%
\pgftext[x=3.285112in,y=3.281088in,left,base]{\color{textcolor}\rmfamily\fontsize{8.000000}{9.600000}\selectfont WT + Vehicle (1129)}%
\end{pgfscope}%
\begin{pgfscope}%
\pgfsetroundcap%
\pgfsetroundjoin%
\pgfsetlinewidth{1.003750pt}%
\definecolor{currentstroke}{rgb}{0.168627,0.670588,0.494118}%
\pgfsetstrokecolor{currentstroke}%
\pgfsetdash{}{0pt}%
\pgfpathmoveto{\pgfqpoint{3.085112in}{3.153338in}}%
\pgfpathlineto{\pgfqpoint{3.196223in}{3.153338in}}%
\pgfusepath{stroke}%
\end{pgfscope}%
\begin{pgfscope}%
\definecolor{textcolor}{rgb}{1.000000,1.000000,1.000000}%
\pgfsetstrokecolor{textcolor}%
\pgfsetfillcolor{textcolor}%
\pgftext[x=3.285112in,y=3.114449in,left,base]{\color{textcolor}\rmfamily\fontsize{8.000000}{9.600000}\selectfont WT + TAT-GluA2\textsubscript{3Y} (512)}%
\end{pgfscope}%
\begin{pgfscope}%
\pgfsetroundcap%
\pgfsetroundjoin%
\pgfsetlinewidth{1.003750pt}%
\definecolor{currentstroke}{rgb}{1.000000,0.494118,0.250980}%
\pgfsetstrokecolor{currentstroke}%
\pgfsetdash{}{0pt}%
\pgfpathmoveto{\pgfqpoint{3.085112in}{2.986698in}}%
\pgfpathlineto{\pgfqpoint{3.196223in}{2.986698in}}%
\pgfusepath{stroke}%
\end{pgfscope}%
\begin{pgfscope}%
\definecolor{textcolor}{rgb}{1.000000,1.000000,1.000000}%
\pgfsetstrokecolor{textcolor}%
\pgfsetfillcolor{textcolor}%
\pgftext[x=3.285112in,y=2.947809in,left,base]{\color{textcolor}\rmfamily\fontsize{8.000000}{9.600000}\selectfont Tg + Vehicle (638)}%
\end{pgfscope}%
\begin{pgfscope}%
\pgfsetroundcap%
\pgfsetroundjoin%
\pgfsetlinewidth{1.003750pt}%
\definecolor{currentstroke}{rgb}{1.000000,0.694118,0.250980}%
\pgfsetstrokecolor{currentstroke}%
\pgfsetdash{}{0pt}%
\pgfpathmoveto{\pgfqpoint{3.085112in}{2.820059in}}%
\pgfpathlineto{\pgfqpoint{3.196223in}{2.820059in}}%
\pgfusepath{stroke}%
\end{pgfscope}%
\begin{pgfscope}%
\definecolor{textcolor}{rgb}{1.000000,1.000000,1.000000}%
\pgfsetstrokecolor{textcolor}%
\pgfsetfillcolor{textcolor}%
\pgftext[x=3.285112in,y=2.781170in,left,base]{\color{textcolor}\rmfamily\fontsize{8.000000}{9.600000}\selectfont Tg + TAT-GluA2\textsubscript{3Y} (759)}%
\end{pgfscope}%
\begin{pgfscope}%
\pgfsetbuttcap%
\pgfsetroundjoin%
\definecolor{currentfill}{rgb}{0.150000,0.150000,0.150000}%
\pgfsetfillcolor{currentfill}%
\pgfsetlinewidth{1.003750pt}%
\definecolor{currentstroke}{rgb}{0.150000,0.150000,0.150000}%
\pgfsetstrokecolor{currentstroke}%
\pgfsetdash{}{0pt}%
\pgfsys@defobject{currentmarker}{\pgfqpoint{0.000000in}{0.000000in}}{\pgfqpoint{0.041667in}{0.000000in}}{%
\pgfpathmoveto{\pgfqpoint{0.000000in}{0.000000in}}%
\pgfpathlineto{\pgfqpoint{0.041667in}{0.000000in}}%
\pgfusepath{stroke,fill}%
}%
\begin{pgfscope}%
\pgfsys@transformshift{3.186623in}{0.528177in}%
\pgfsys@useobject{currentmarker}{}%
\end{pgfscope}%
\end{pgfscope}%
\begin{pgfscope}%
\definecolor{textcolor}{rgb}{0.150000,0.150000,0.150000}%
\pgfsetstrokecolor{textcolor}%
\pgfsetfillcolor{textcolor}%
\pgftext[x=3.089400in,y=0.528177in,right,]{\color{textcolor}\rmfamily\fontsize{10.000000}{12.000000}\selectfont \(\displaystyle 0.0\)}%
\end{pgfscope}%
\begin{pgfscope}%
\pgfsetbuttcap%
\pgfsetroundjoin%
\definecolor{currentfill}{rgb}{0.150000,0.150000,0.150000}%
\pgfsetfillcolor{currentfill}%
\pgfsetlinewidth{1.003750pt}%
\definecolor{currentstroke}{rgb}{0.150000,0.150000,0.150000}%
\pgfsetstrokecolor{currentstroke}%
\pgfsetdash{}{0pt}%
\pgfsys@defobject{currentmarker}{\pgfqpoint{0.000000in}{0.000000in}}{\pgfqpoint{0.041667in}{0.000000in}}{%
\pgfpathmoveto{\pgfqpoint{0.000000in}{0.000000in}}%
\pgfpathlineto{\pgfqpoint{0.041667in}{0.000000in}}%
\pgfusepath{stroke,fill}%
}%
\begin{pgfscope}%
\pgfsys@transformshift{3.186623in}{0.953221in}%
\pgfsys@useobject{currentmarker}{}%
\end{pgfscope}%
\end{pgfscope}%
\begin{pgfscope}%
\definecolor{textcolor}{rgb}{0.150000,0.150000,0.150000}%
\pgfsetstrokecolor{textcolor}%
\pgfsetfillcolor{textcolor}%
\pgftext[x=3.089400in,y=0.953221in,right,]{\color{textcolor}\rmfamily\fontsize{10.000000}{12.000000}\selectfont \(\displaystyle 0.5\)}%
\end{pgfscope}%
\begin{pgfscope}%
\pgfsetbuttcap%
\pgfsetroundjoin%
\definecolor{currentfill}{rgb}{0.150000,0.150000,0.150000}%
\pgfsetfillcolor{currentfill}%
\pgfsetlinewidth{1.003750pt}%
\definecolor{currentstroke}{rgb}{0.150000,0.150000,0.150000}%
\pgfsetstrokecolor{currentstroke}%
\pgfsetdash{}{0pt}%
\pgfsys@defobject{currentmarker}{\pgfqpoint{0.000000in}{0.000000in}}{\pgfqpoint{0.041667in}{0.000000in}}{%
\pgfpathmoveto{\pgfqpoint{0.000000in}{0.000000in}}%
\pgfpathlineto{\pgfqpoint{0.041667in}{0.000000in}}%
\pgfusepath{stroke,fill}%
}%
\begin{pgfscope}%
\pgfsys@transformshift{3.186623in}{1.378266in}%
\pgfsys@useobject{currentmarker}{}%
\end{pgfscope}%
\end{pgfscope}%
\begin{pgfscope}%
\definecolor{textcolor}{rgb}{0.150000,0.150000,0.150000}%
\pgfsetstrokecolor{textcolor}%
\pgfsetfillcolor{textcolor}%
\pgftext[x=3.089400in,y=1.378266in,right,]{\color{textcolor}\rmfamily\fontsize{10.000000}{12.000000}\selectfont \(\displaystyle 1.0\)}%
\end{pgfscope}%
\begin{pgfscope}%
\pgfsetbuttcap%
\pgfsetroundjoin%
\definecolor{currentfill}{rgb}{0.150000,0.150000,0.150000}%
\pgfsetfillcolor{currentfill}%
\pgfsetlinewidth{1.003750pt}%
\definecolor{currentstroke}{rgb}{0.150000,0.150000,0.150000}%
\pgfsetstrokecolor{currentstroke}%
\pgfsetdash{}{0pt}%
\pgfsys@defobject{currentmarker}{\pgfqpoint{0.000000in}{0.000000in}}{\pgfqpoint{0.041667in}{0.000000in}}{%
\pgfpathmoveto{\pgfqpoint{0.000000in}{0.000000in}}%
\pgfpathlineto{\pgfqpoint{0.041667in}{0.000000in}}%
\pgfusepath{stroke,fill}%
}%
\begin{pgfscope}%
\pgfsys@transformshift{3.186623in}{1.803310in}%
\pgfsys@useobject{currentmarker}{}%
\end{pgfscope}%
\end{pgfscope}%
\begin{pgfscope}%
\definecolor{textcolor}{rgb}{0.150000,0.150000,0.150000}%
\pgfsetstrokecolor{textcolor}%
\pgfsetfillcolor{textcolor}%
\pgftext[x=3.089400in,y=1.803310in,right,]{\color{textcolor}\rmfamily\fontsize{10.000000}{12.000000}\selectfont \(\displaystyle 1.5\)}%
\end{pgfscope}%
\begin{pgfscope}%
\pgfsetbuttcap%
\pgfsetroundjoin%
\definecolor{currentfill}{rgb}{0.150000,0.150000,0.150000}%
\pgfsetfillcolor{currentfill}%
\pgfsetlinewidth{1.003750pt}%
\definecolor{currentstroke}{rgb}{0.150000,0.150000,0.150000}%
\pgfsetstrokecolor{currentstroke}%
\pgfsetdash{}{0pt}%
\pgfsys@defobject{currentmarker}{\pgfqpoint{0.000000in}{0.000000in}}{\pgfqpoint{0.041667in}{0.000000in}}{%
\pgfpathmoveto{\pgfqpoint{0.000000in}{0.000000in}}%
\pgfpathlineto{\pgfqpoint{0.041667in}{0.000000in}}%
\pgfusepath{stroke,fill}%
}%
\begin{pgfscope}%
\pgfsys@transformshift{3.186623in}{2.228354in}%
\pgfsys@useobject{currentmarker}{}%
\end{pgfscope}%
\end{pgfscope}%
\begin{pgfscope}%
\definecolor{textcolor}{rgb}{0.150000,0.150000,0.150000}%
\pgfsetstrokecolor{textcolor}%
\pgfsetfillcolor{textcolor}%
\pgftext[x=3.089400in,y=2.228354in,right,]{\color{textcolor}\rmfamily\fontsize{10.000000}{12.000000}\selectfont \(\displaystyle 2.0\)}%
\end{pgfscope}%
\begin{pgfscope}%
\pgfsetbuttcap%
\pgfsetroundjoin%
\definecolor{currentfill}{rgb}{0.150000,0.150000,0.150000}%
\pgfsetfillcolor{currentfill}%
\pgfsetlinewidth{1.003750pt}%
\definecolor{currentstroke}{rgb}{0.150000,0.150000,0.150000}%
\pgfsetstrokecolor{currentstroke}%
\pgfsetdash{}{0pt}%
\pgfsys@defobject{currentmarker}{\pgfqpoint{0.000000in}{0.000000in}}{\pgfqpoint{0.041667in}{0.000000in}}{%
\pgfpathmoveto{\pgfqpoint{0.000000in}{0.000000in}}%
\pgfpathlineto{\pgfqpoint{0.041667in}{0.000000in}}%
\pgfusepath{stroke,fill}%
}%
\begin{pgfscope}%
\pgfsys@transformshift{3.186623in}{2.653399in}%
\pgfsys@useobject{currentmarker}{}%
\end{pgfscope}%
\end{pgfscope}%
\begin{pgfscope}%
\definecolor{textcolor}{rgb}{0.150000,0.150000,0.150000}%
\pgfsetstrokecolor{textcolor}%
\pgfsetfillcolor{textcolor}%
\pgftext[x=3.089400in,y=2.653399in,right,]{\color{textcolor}\rmfamily\fontsize{10.000000}{12.000000}\selectfont \(\displaystyle 2.5\)}%
\end{pgfscope}%
\begin{pgfscope}%
\definecolor{textcolor}{rgb}{0.150000,0.150000,0.150000}%
\pgfsetstrokecolor{textcolor}%
\pgfsetfillcolor{textcolor}%
\pgftext[x=2.842486in,y=1.590788in,,bottom,rotate=90.000000]{\color{textcolor}\rmfamily\fontsize{10.000000}{12.000000}\selectfont \textbf{Cell activity (freezing, a.u.)}}%
\end{pgfscope}%
\begin{pgfscope}%
\pgfpathrectangle{\pgfqpoint{3.186623in}{0.528177in}}{\pgfqpoint{2.015106in}{2.125222in}} %
\pgfusepath{clip}%
\pgfsetbuttcap%
\pgfsetmiterjoin%
\definecolor{currentfill}{rgb}{0.200000,0.427451,0.650980}%
\pgfsetfillcolor{currentfill}%
\pgfsetlinewidth{1.505625pt}%
\definecolor{currentstroke}{rgb}{0.200000,0.427451,0.650980}%
\pgfsetstrokecolor{currentstroke}%
\pgfsetdash{}{0pt}%
\pgfpathmoveto{\pgfqpoint{3.258591in}{0.528177in}}%
\pgfpathlineto{\pgfqpoint{3.618431in}{0.528177in}}%
\pgfpathlineto{\pgfqpoint{3.618431in}{1.858224in}}%
\pgfpathlineto{\pgfqpoint{3.258591in}{1.858224in}}%
\pgfpathclose%
\pgfusepath{stroke,fill}%
\end{pgfscope}%
\begin{pgfscope}%
\pgfpathrectangle{\pgfqpoint{3.186623in}{0.528177in}}{\pgfqpoint{2.015106in}{2.125222in}} %
\pgfusepath{clip}%
\pgfsetbuttcap%
\pgfsetmiterjoin%
\definecolor{currentfill}{rgb}{0.168627,0.670588,0.494118}%
\pgfsetfillcolor{currentfill}%
\pgfsetlinewidth{1.505625pt}%
\definecolor{currentstroke}{rgb}{0.168627,0.670588,0.494118}%
\pgfsetstrokecolor{currentstroke}%
\pgfsetdash{}{0pt}%
\pgfpathmoveto{\pgfqpoint{3.762367in}{0.528177in}}%
\pgfpathlineto{\pgfqpoint{4.122208in}{0.528177in}}%
\pgfpathlineto{\pgfqpoint{4.122208in}{1.467376in}}%
\pgfpathlineto{\pgfqpoint{3.762367in}{1.467376in}}%
\pgfpathclose%
\pgfusepath{stroke,fill}%
\end{pgfscope}%
\begin{pgfscope}%
\pgfpathrectangle{\pgfqpoint{3.186623in}{0.528177in}}{\pgfqpoint{2.015106in}{2.125222in}} %
\pgfusepath{clip}%
\pgfsetbuttcap%
\pgfsetmiterjoin%
\definecolor{currentfill}{rgb}{1.000000,0.494118,0.250980}%
\pgfsetfillcolor{currentfill}%
\pgfsetlinewidth{1.505625pt}%
\definecolor{currentstroke}{rgb}{1.000000,0.494118,0.250980}%
\pgfsetstrokecolor{currentstroke}%
\pgfsetdash{}{0pt}%
\pgfpathmoveto{\pgfqpoint{4.266144in}{0.528177in}}%
\pgfpathlineto{\pgfqpoint{4.625984in}{0.528177in}}%
\pgfpathlineto{\pgfqpoint{4.625984in}{2.097693in}}%
\pgfpathlineto{\pgfqpoint{4.266144in}{2.097693in}}%
\pgfpathclose%
\pgfusepath{stroke,fill}%
\end{pgfscope}%
\begin{pgfscope}%
\pgfpathrectangle{\pgfqpoint{3.186623in}{0.528177in}}{\pgfqpoint{2.015106in}{2.125222in}} %
\pgfusepath{clip}%
\pgfsetbuttcap%
\pgfsetmiterjoin%
\definecolor{currentfill}{rgb}{1.000000,0.694118,0.250980}%
\pgfsetfillcolor{currentfill}%
\pgfsetlinewidth{1.505625pt}%
\definecolor{currentstroke}{rgb}{1.000000,0.694118,0.250980}%
\pgfsetstrokecolor{currentstroke}%
\pgfsetdash{}{0pt}%
\pgfpathmoveto{\pgfqpoint{4.769920in}{0.528177in}}%
\pgfpathlineto{\pgfqpoint{5.129761in}{0.528177in}}%
\pgfpathlineto{\pgfqpoint{5.129761in}{1.700329in}}%
\pgfpathlineto{\pgfqpoint{4.769920in}{1.700329in}}%
\pgfpathclose%
\pgfusepath{stroke,fill}%
\end{pgfscope}%
\begin{pgfscope}%
\pgfpathrectangle{\pgfqpoint{3.186623in}{0.528177in}}{\pgfqpoint{2.015106in}{2.125222in}} %
\pgfusepath{clip}%
\pgfsetbuttcap%
\pgfsetroundjoin%
\pgfsetlinewidth{1.505625pt}%
\definecolor{currentstroke}{rgb}{0.200000,0.427451,0.650980}%
\pgfsetstrokecolor{currentstroke}%
\pgfsetdash{}{0pt}%
\pgfpathmoveto{\pgfqpoint{3.438511in}{1.858224in}}%
\pgfpathlineto{\pgfqpoint{3.438511in}{1.920823in}}%
\pgfusepath{stroke}%
\end{pgfscope}%
\begin{pgfscope}%
\pgfpathrectangle{\pgfqpoint{3.186623in}{0.528177in}}{\pgfqpoint{2.015106in}{2.125222in}} %
\pgfusepath{clip}%
\pgfsetbuttcap%
\pgfsetroundjoin%
\pgfsetlinewidth{1.505625pt}%
\definecolor{currentstroke}{rgb}{0.168627,0.670588,0.494118}%
\pgfsetstrokecolor{currentstroke}%
\pgfsetdash{}{0pt}%
\pgfpathmoveto{\pgfqpoint{3.942287in}{1.467376in}}%
\pgfpathlineto{\pgfqpoint{3.942287in}{1.535346in}}%
\pgfusepath{stroke}%
\end{pgfscope}%
\begin{pgfscope}%
\pgfpathrectangle{\pgfqpoint{3.186623in}{0.528177in}}{\pgfqpoint{2.015106in}{2.125222in}} %
\pgfusepath{clip}%
\pgfsetbuttcap%
\pgfsetroundjoin%
\pgfsetlinewidth{1.505625pt}%
\definecolor{currentstroke}{rgb}{1.000000,0.494118,0.250980}%
\pgfsetstrokecolor{currentstroke}%
\pgfsetdash{}{0pt}%
\pgfpathmoveto{\pgfqpoint{4.446064in}{2.097693in}}%
\pgfpathlineto{\pgfqpoint{4.446064in}{2.238634in}}%
\pgfusepath{stroke}%
\end{pgfscope}%
\begin{pgfscope}%
\pgfpathrectangle{\pgfqpoint{3.186623in}{0.528177in}}{\pgfqpoint{2.015106in}{2.125222in}} %
\pgfusepath{clip}%
\pgfsetbuttcap%
\pgfsetroundjoin%
\pgfsetlinewidth{1.505625pt}%
\definecolor{currentstroke}{rgb}{1.000000,0.694118,0.250980}%
\pgfsetstrokecolor{currentstroke}%
\pgfsetdash{}{0pt}%
\pgfpathmoveto{\pgfqpoint{4.949840in}{1.700329in}}%
\pgfpathlineto{\pgfqpoint{4.949840in}{1.775744in}}%
\pgfusepath{stroke}%
\end{pgfscope}%
\begin{pgfscope}%
\pgfpathrectangle{\pgfqpoint{3.186623in}{0.528177in}}{\pgfqpoint{2.015106in}{2.125222in}} %
\pgfusepath{clip}%
\pgfsetbuttcap%
\pgfsetroundjoin%
\definecolor{currentfill}{rgb}{0.200000,0.427451,0.650980}%
\pgfsetfillcolor{currentfill}%
\pgfsetlinewidth{1.505625pt}%
\definecolor{currentstroke}{rgb}{0.200000,0.427451,0.650980}%
\pgfsetstrokecolor{currentstroke}%
\pgfsetdash{}{0pt}%
\pgfsys@defobject{currentmarker}{\pgfqpoint{-0.111111in}{-0.000000in}}{\pgfqpoint{0.111111in}{0.000000in}}{%
\pgfpathmoveto{\pgfqpoint{0.111111in}{-0.000000in}}%
\pgfpathlineto{\pgfqpoint{-0.111111in}{0.000000in}}%
\pgfusepath{stroke,fill}%
}%
\begin{pgfscope}%
\pgfsys@transformshift{3.438511in}{1.858224in}%
\pgfsys@useobject{currentmarker}{}%
\end{pgfscope}%
\end{pgfscope}%
\begin{pgfscope}%
\pgfpathrectangle{\pgfqpoint{3.186623in}{0.528177in}}{\pgfqpoint{2.015106in}{2.125222in}} %
\pgfusepath{clip}%
\pgfsetbuttcap%
\pgfsetroundjoin%
\definecolor{currentfill}{rgb}{0.200000,0.427451,0.650980}%
\pgfsetfillcolor{currentfill}%
\pgfsetlinewidth{1.505625pt}%
\definecolor{currentstroke}{rgb}{0.200000,0.427451,0.650980}%
\pgfsetstrokecolor{currentstroke}%
\pgfsetdash{}{0pt}%
\pgfsys@defobject{currentmarker}{\pgfqpoint{-0.111111in}{-0.000000in}}{\pgfqpoint{0.111111in}{0.000000in}}{%
\pgfpathmoveto{\pgfqpoint{0.111111in}{-0.000000in}}%
\pgfpathlineto{\pgfqpoint{-0.111111in}{0.000000in}}%
\pgfusepath{stroke,fill}%
}%
\begin{pgfscope}%
\pgfsys@transformshift{3.438511in}{1.920823in}%
\pgfsys@useobject{currentmarker}{}%
\end{pgfscope}%
\end{pgfscope}%
\begin{pgfscope}%
\pgfpathrectangle{\pgfqpoint{3.186623in}{0.528177in}}{\pgfqpoint{2.015106in}{2.125222in}} %
\pgfusepath{clip}%
\pgfsetbuttcap%
\pgfsetroundjoin%
\definecolor{currentfill}{rgb}{0.168627,0.670588,0.494118}%
\pgfsetfillcolor{currentfill}%
\pgfsetlinewidth{1.505625pt}%
\definecolor{currentstroke}{rgb}{0.168627,0.670588,0.494118}%
\pgfsetstrokecolor{currentstroke}%
\pgfsetdash{}{0pt}%
\pgfsys@defobject{currentmarker}{\pgfqpoint{-0.111111in}{-0.000000in}}{\pgfqpoint{0.111111in}{0.000000in}}{%
\pgfpathmoveto{\pgfqpoint{0.111111in}{-0.000000in}}%
\pgfpathlineto{\pgfqpoint{-0.111111in}{0.000000in}}%
\pgfusepath{stroke,fill}%
}%
\begin{pgfscope}%
\pgfsys@transformshift{3.942287in}{1.467376in}%
\pgfsys@useobject{currentmarker}{}%
\end{pgfscope}%
\end{pgfscope}%
\begin{pgfscope}%
\pgfpathrectangle{\pgfqpoint{3.186623in}{0.528177in}}{\pgfqpoint{2.015106in}{2.125222in}} %
\pgfusepath{clip}%
\pgfsetbuttcap%
\pgfsetroundjoin%
\definecolor{currentfill}{rgb}{0.168627,0.670588,0.494118}%
\pgfsetfillcolor{currentfill}%
\pgfsetlinewidth{1.505625pt}%
\definecolor{currentstroke}{rgb}{0.168627,0.670588,0.494118}%
\pgfsetstrokecolor{currentstroke}%
\pgfsetdash{}{0pt}%
\pgfsys@defobject{currentmarker}{\pgfqpoint{-0.111111in}{-0.000000in}}{\pgfqpoint{0.111111in}{0.000000in}}{%
\pgfpathmoveto{\pgfqpoint{0.111111in}{-0.000000in}}%
\pgfpathlineto{\pgfqpoint{-0.111111in}{0.000000in}}%
\pgfusepath{stroke,fill}%
}%
\begin{pgfscope}%
\pgfsys@transformshift{3.942287in}{1.535346in}%
\pgfsys@useobject{currentmarker}{}%
\end{pgfscope}%
\end{pgfscope}%
\begin{pgfscope}%
\pgfpathrectangle{\pgfqpoint{3.186623in}{0.528177in}}{\pgfqpoint{2.015106in}{2.125222in}} %
\pgfusepath{clip}%
\pgfsetbuttcap%
\pgfsetroundjoin%
\definecolor{currentfill}{rgb}{1.000000,0.494118,0.250980}%
\pgfsetfillcolor{currentfill}%
\pgfsetlinewidth{1.505625pt}%
\definecolor{currentstroke}{rgb}{1.000000,0.494118,0.250980}%
\pgfsetstrokecolor{currentstroke}%
\pgfsetdash{}{0pt}%
\pgfsys@defobject{currentmarker}{\pgfqpoint{-0.111111in}{-0.000000in}}{\pgfqpoint{0.111111in}{0.000000in}}{%
\pgfpathmoveto{\pgfqpoint{0.111111in}{-0.000000in}}%
\pgfpathlineto{\pgfqpoint{-0.111111in}{0.000000in}}%
\pgfusepath{stroke,fill}%
}%
\begin{pgfscope}%
\pgfsys@transformshift{4.446064in}{2.097693in}%
\pgfsys@useobject{currentmarker}{}%
\end{pgfscope}%
\end{pgfscope}%
\begin{pgfscope}%
\pgfpathrectangle{\pgfqpoint{3.186623in}{0.528177in}}{\pgfqpoint{2.015106in}{2.125222in}} %
\pgfusepath{clip}%
\pgfsetbuttcap%
\pgfsetroundjoin%
\definecolor{currentfill}{rgb}{1.000000,0.494118,0.250980}%
\pgfsetfillcolor{currentfill}%
\pgfsetlinewidth{1.505625pt}%
\definecolor{currentstroke}{rgb}{1.000000,0.494118,0.250980}%
\pgfsetstrokecolor{currentstroke}%
\pgfsetdash{}{0pt}%
\pgfsys@defobject{currentmarker}{\pgfqpoint{-0.111111in}{-0.000000in}}{\pgfqpoint{0.111111in}{0.000000in}}{%
\pgfpathmoveto{\pgfqpoint{0.111111in}{-0.000000in}}%
\pgfpathlineto{\pgfqpoint{-0.111111in}{0.000000in}}%
\pgfusepath{stroke,fill}%
}%
\begin{pgfscope}%
\pgfsys@transformshift{4.446064in}{2.238634in}%
\pgfsys@useobject{currentmarker}{}%
\end{pgfscope}%
\end{pgfscope}%
\begin{pgfscope}%
\pgfpathrectangle{\pgfqpoint{3.186623in}{0.528177in}}{\pgfqpoint{2.015106in}{2.125222in}} %
\pgfusepath{clip}%
\pgfsetbuttcap%
\pgfsetroundjoin%
\definecolor{currentfill}{rgb}{1.000000,0.694118,0.250980}%
\pgfsetfillcolor{currentfill}%
\pgfsetlinewidth{1.505625pt}%
\definecolor{currentstroke}{rgb}{1.000000,0.694118,0.250980}%
\pgfsetstrokecolor{currentstroke}%
\pgfsetdash{}{0pt}%
\pgfsys@defobject{currentmarker}{\pgfqpoint{-0.111111in}{-0.000000in}}{\pgfqpoint{0.111111in}{0.000000in}}{%
\pgfpathmoveto{\pgfqpoint{0.111111in}{-0.000000in}}%
\pgfpathlineto{\pgfqpoint{-0.111111in}{0.000000in}}%
\pgfusepath{stroke,fill}%
}%
\begin{pgfscope}%
\pgfsys@transformshift{4.949840in}{1.700329in}%
\pgfsys@useobject{currentmarker}{}%
\end{pgfscope}%
\end{pgfscope}%
\begin{pgfscope}%
\pgfpathrectangle{\pgfqpoint{3.186623in}{0.528177in}}{\pgfqpoint{2.015106in}{2.125222in}} %
\pgfusepath{clip}%
\pgfsetbuttcap%
\pgfsetroundjoin%
\definecolor{currentfill}{rgb}{1.000000,0.694118,0.250980}%
\pgfsetfillcolor{currentfill}%
\pgfsetlinewidth{1.505625pt}%
\definecolor{currentstroke}{rgb}{1.000000,0.694118,0.250980}%
\pgfsetstrokecolor{currentstroke}%
\pgfsetdash{}{0pt}%
\pgfsys@defobject{currentmarker}{\pgfqpoint{-0.111111in}{-0.000000in}}{\pgfqpoint{0.111111in}{0.000000in}}{%
\pgfpathmoveto{\pgfqpoint{0.111111in}{-0.000000in}}%
\pgfpathlineto{\pgfqpoint{-0.111111in}{0.000000in}}%
\pgfusepath{stroke,fill}%
}%
\begin{pgfscope}%
\pgfsys@transformshift{4.949840in}{1.775744in}%
\pgfsys@useobject{currentmarker}{}%
\end{pgfscope}%
\end{pgfscope}%
\begin{pgfscope}%
\pgfpathrectangle{\pgfqpoint{3.186623in}{0.528177in}}{\pgfqpoint{2.015106in}{2.125222in}} %
\pgfusepath{clip}%
\pgfsetroundcap%
\pgfsetroundjoin%
\pgfsetlinewidth{1.756562pt}%
\definecolor{currentstroke}{rgb}{0.627451,0.627451,0.643137}%
\pgfsetstrokecolor{currentstroke}%
\pgfsetdash{}{0pt}%
\pgfpathmoveto{\pgfqpoint{4.446064in}{2.342800in}}%
\pgfpathlineto{\pgfqpoint{4.446064in}{2.516412in}}%
\pgfusepath{stroke}%
\end{pgfscope}%
\begin{pgfscope}%
\pgfpathrectangle{\pgfqpoint{3.186623in}{0.528177in}}{\pgfqpoint{2.015106in}{2.125222in}} %
\pgfusepath{clip}%
\pgfsetroundcap%
\pgfsetroundjoin%
\pgfsetlinewidth{1.756562pt}%
\definecolor{currentstroke}{rgb}{0.627451,0.627451,0.643137}%
\pgfsetstrokecolor{currentstroke}%
\pgfsetdash{}{0pt}%
\pgfpathmoveto{\pgfqpoint{4.446064in}{2.516412in}}%
\pgfpathlineto{\pgfqpoint{4.949840in}{2.516412in}}%
\pgfusepath{stroke}%
\end{pgfscope}%
\begin{pgfscope}%
\pgfpathrectangle{\pgfqpoint{3.186623in}{0.528177in}}{\pgfqpoint{2.015106in}{2.125222in}} %
\pgfusepath{clip}%
\pgfsetroundcap%
\pgfsetroundjoin%
\pgfsetlinewidth{1.756562pt}%
\definecolor{currentstroke}{rgb}{0.627451,0.627451,0.643137}%
\pgfsetstrokecolor{currentstroke}%
\pgfsetdash{}{0pt}%
\pgfpathmoveto{\pgfqpoint{4.949840in}{2.516412in}}%
\pgfpathlineto{\pgfqpoint{4.949840in}{1.984078in}}%
\pgfusepath{stroke}%
\end{pgfscope}%
\begin{pgfscope}%
\pgfsetrectcap%
\pgfsetmiterjoin%
\pgfsetlinewidth{1.254687pt}%
\definecolor{currentstroke}{rgb}{0.150000,0.150000,0.150000}%
\pgfsetstrokecolor{currentstroke}%
\pgfsetdash{}{0pt}%
\pgfpathmoveto{\pgfqpoint{3.186623in}{0.528177in}}%
\pgfpathlineto{\pgfqpoint{3.186623in}{2.653399in}}%
\pgfusepath{stroke}%
\end{pgfscope}%
\begin{pgfscope}%
\pgfsetrectcap%
\pgfsetmiterjoin%
\pgfsetlinewidth{1.254687pt}%
\definecolor{currentstroke}{rgb}{0.150000,0.150000,0.150000}%
\pgfsetstrokecolor{currentstroke}%
\pgfsetdash{}{0pt}%
\pgfpathmoveto{\pgfqpoint{3.186623in}{0.528177in}}%
\pgfpathlineto{\pgfqpoint{5.201729in}{0.528177in}}%
\pgfusepath{stroke}%
\end{pgfscope}%
\begin{pgfscope}%
\definecolor{textcolor}{rgb}{0.150000,0.150000,0.150000}%
\pgfsetstrokecolor{textcolor}%
\pgfsetfillcolor{textcolor}%
\pgftext[x=4.949840in,y=1.840848in,,]{\color{textcolor}\rmfamily\fontsize{15.000000}{18.000000}\selectfont \textbf{*}}%
\end{pgfscope}%
\begin{pgfscope}%
\pgfsetbuttcap%
\pgfsetmiterjoin%
\definecolor{currentfill}{rgb}{0.200000,0.427451,0.650980}%
\pgfsetfillcolor{currentfill}%
\pgfsetlinewidth{1.505625pt}%
\definecolor{currentstroke}{rgb}{0.200000,0.427451,0.650980}%
\pgfsetstrokecolor{currentstroke}%
\pgfsetdash{}{0pt}%
\pgfpathmoveto{\pgfqpoint{3.286623in}{3.281088in}}%
\pgfpathlineto{\pgfqpoint{3.397734in}{3.281088in}}%
\pgfpathlineto{\pgfqpoint{3.397734in}{3.358866in}}%
\pgfpathlineto{\pgfqpoint{3.286623in}{3.358866in}}%
\pgfpathclose%
\pgfusepath{stroke,fill}%
\end{pgfscope}%
\begin{pgfscope}%
\definecolor{textcolor}{rgb}{0.150000,0.150000,0.150000}%
\pgfsetstrokecolor{textcolor}%
\pgfsetfillcolor{textcolor}%
\pgftext[x=3.486623in,y=3.281088in,left,base]{\color{textcolor}\rmfamily\fontsize{8.000000}{9.600000}\selectfont WT + Vehicle (1129)}%
\end{pgfscope}%
\begin{pgfscope}%
\pgfsetbuttcap%
\pgfsetmiterjoin%
\definecolor{currentfill}{rgb}{0.168627,0.670588,0.494118}%
\pgfsetfillcolor{currentfill}%
\pgfsetlinewidth{1.505625pt}%
\definecolor{currentstroke}{rgb}{0.168627,0.670588,0.494118}%
\pgfsetstrokecolor{currentstroke}%
\pgfsetdash{}{0pt}%
\pgfpathmoveto{\pgfqpoint{3.286623in}{3.114449in}}%
\pgfpathlineto{\pgfqpoint{3.397734in}{3.114449in}}%
\pgfpathlineto{\pgfqpoint{3.397734in}{3.192227in}}%
\pgfpathlineto{\pgfqpoint{3.286623in}{3.192227in}}%
\pgfpathclose%
\pgfusepath{stroke,fill}%
\end{pgfscope}%
\begin{pgfscope}%
\definecolor{textcolor}{rgb}{0.150000,0.150000,0.150000}%
\pgfsetstrokecolor{textcolor}%
\pgfsetfillcolor{textcolor}%
\pgftext[x=3.486623in,y=3.114449in,left,base]{\color{textcolor}\rmfamily\fontsize{8.000000}{9.600000}\selectfont WT + TAT-GluA2\textsubscript{3Y} (512)}%
\end{pgfscope}%
\begin{pgfscope}%
\pgfsetbuttcap%
\pgfsetmiterjoin%
\definecolor{currentfill}{rgb}{1.000000,0.494118,0.250980}%
\pgfsetfillcolor{currentfill}%
\pgfsetlinewidth{1.505625pt}%
\definecolor{currentstroke}{rgb}{1.000000,0.494118,0.250980}%
\pgfsetstrokecolor{currentstroke}%
\pgfsetdash{}{0pt}%
\pgfpathmoveto{\pgfqpoint{3.286623in}{2.947809in}}%
\pgfpathlineto{\pgfqpoint{3.397734in}{2.947809in}}%
\pgfpathlineto{\pgfqpoint{3.397734in}{3.025587in}}%
\pgfpathlineto{\pgfqpoint{3.286623in}{3.025587in}}%
\pgfpathclose%
\pgfusepath{stroke,fill}%
\end{pgfscope}%
\begin{pgfscope}%
\definecolor{textcolor}{rgb}{0.150000,0.150000,0.150000}%
\pgfsetstrokecolor{textcolor}%
\pgfsetfillcolor{textcolor}%
\pgftext[x=3.486623in,y=2.947809in,left,base]{\color{textcolor}\rmfamily\fontsize{8.000000}{9.600000}\selectfont Tg + Vehicle (638)}%
\end{pgfscope}%
\begin{pgfscope}%
\pgfsetbuttcap%
\pgfsetmiterjoin%
\definecolor{currentfill}{rgb}{1.000000,0.694118,0.250980}%
\pgfsetfillcolor{currentfill}%
\pgfsetlinewidth{1.505625pt}%
\definecolor{currentstroke}{rgb}{1.000000,0.694118,0.250980}%
\pgfsetstrokecolor{currentstroke}%
\pgfsetdash{}{0pt}%
\pgfpathmoveto{\pgfqpoint{3.286623in}{2.781170in}}%
\pgfpathlineto{\pgfqpoint{3.397734in}{2.781170in}}%
\pgfpathlineto{\pgfqpoint{3.397734in}{2.858947in}}%
\pgfpathlineto{\pgfqpoint{3.286623in}{2.858947in}}%
\pgfpathclose%
\pgfusepath{stroke,fill}%
\end{pgfscope}%
\begin{pgfscope}%
\definecolor{textcolor}{rgb}{0.150000,0.150000,0.150000}%
\pgfsetstrokecolor{textcolor}%
\pgfsetfillcolor{textcolor}%
\pgftext[x=3.486623in,y=2.781170in,left,base]{\color{textcolor}\rmfamily\fontsize{8.000000}{9.600000}\selectfont Tg + TAT-GluA2\textsubscript{3Y} (759)}%
\end{pgfscope}%
\end{pgfpicture}%
\makeatother%
\endgroup%

        \caption{\label{f.ad.actf}}
    \end{subfigure}
    \begin{subfigure}[h]{\textwidth}
        %% Creator: Matplotlib, PGF backend
%%
%% To include the figure in your LaTeX document, write
%%   \input{<filename>.pgf}
%%
%% Make sure the required packages are loaded in your preamble
%%   \usepackage{pgf}
%%
%% Figures using additional raster images can only be included by \input if
%% they are in the same directory as the main LaTeX file. For loading figures
%% from other directories you can use the `import` package
%%   \usepackage{import}
%% and then include the figures with
%%   \import{<path to file>}{<filename>.pgf}
%%
%% Matplotlib used the following preamble
%%   \usepackage[utf8]{inputenc}
%%   \usepackage[T1]{fontenc}
%%   \usepackage{siunitx}
%%
\begingroup%
\makeatletter%
\begin{pgfpicture}%
\pgfpathrectangle{\pgfpointorigin}{\pgfqpoint{5.301729in}{3.553934in}}%
\pgfusepath{use as bounding box, clip}%
\begin{pgfscope}%
\pgfsetbuttcap%
\pgfsetmiterjoin%
\definecolor{currentfill}{rgb}{1.000000,1.000000,1.000000}%
\pgfsetfillcolor{currentfill}%
\pgfsetlinewidth{0.000000pt}%
\definecolor{currentstroke}{rgb}{1.000000,1.000000,1.000000}%
\pgfsetstrokecolor{currentstroke}%
\pgfsetdash{}{0pt}%
\pgfpathmoveto{\pgfqpoint{0.000000in}{0.000000in}}%
\pgfpathlineto{\pgfqpoint{5.301729in}{0.000000in}}%
\pgfpathlineto{\pgfqpoint{5.301729in}{3.553934in}}%
\pgfpathlineto{\pgfqpoint{0.000000in}{3.553934in}}%
\pgfpathclose%
\pgfusepath{fill}%
\end{pgfscope}%
\begin{pgfscope}%
\pgfsetbuttcap%
\pgfsetmiterjoin%
\definecolor{currentfill}{rgb}{1.000000,1.000000,1.000000}%
\pgfsetfillcolor{currentfill}%
\pgfsetlinewidth{0.000000pt}%
\definecolor{currentstroke}{rgb}{0.000000,0.000000,0.000000}%
\pgfsetstrokecolor{currentstroke}%
\pgfsetstrokeopacity{0.000000}%
\pgfsetdash{}{0pt}%
\pgfpathmoveto{\pgfqpoint{0.566985in}{0.528177in}}%
\pgfpathlineto{\pgfqpoint{2.582091in}{0.528177in}}%
\pgfpathlineto{\pgfqpoint{2.582091in}{3.392606in}}%
\pgfpathlineto{\pgfqpoint{0.566985in}{3.392606in}}%
\pgfpathclose%
\pgfusepath{fill}%
\end{pgfscope}%
\begin{pgfscope}%
\pgfsetbuttcap%
\pgfsetroundjoin%
\definecolor{currentfill}{rgb}{0.150000,0.150000,0.150000}%
\pgfsetfillcolor{currentfill}%
\pgfsetlinewidth{1.003750pt}%
\definecolor{currentstroke}{rgb}{0.150000,0.150000,0.150000}%
\pgfsetstrokecolor{currentstroke}%
\pgfsetdash{}{0pt}%
\pgfsys@defobject{currentmarker}{\pgfqpoint{0.000000in}{0.000000in}}{\pgfqpoint{0.000000in}{0.041667in}}{%
\pgfpathmoveto{\pgfqpoint{0.000000in}{0.000000in}}%
\pgfpathlineto{\pgfqpoint{0.000000in}{0.041667in}}%
\pgfusepath{stroke,fill}%
}%
\begin{pgfscope}%
\pgfsys@transformshift{0.566985in}{0.528177in}%
\pgfsys@useobject{currentmarker}{}%
\end{pgfscope}%
\end{pgfscope}%
\begin{pgfscope}%
\definecolor{textcolor}{rgb}{0.150000,0.150000,0.150000}%
\pgfsetstrokecolor{textcolor}%
\pgfsetfillcolor{textcolor}%
\pgftext[x=0.566985in,y=0.430955in,,top]{\color{textcolor}\rmfamily\fontsize{10.000000}{12.000000}\selectfont \(\displaystyle 0\)}%
\end{pgfscope}%
\begin{pgfscope}%
\pgfsetbuttcap%
\pgfsetroundjoin%
\definecolor{currentfill}{rgb}{0.150000,0.150000,0.150000}%
\pgfsetfillcolor{currentfill}%
\pgfsetlinewidth{1.003750pt}%
\definecolor{currentstroke}{rgb}{0.150000,0.150000,0.150000}%
\pgfsetstrokecolor{currentstroke}%
\pgfsetdash{}{0pt}%
\pgfsys@defobject{currentmarker}{\pgfqpoint{0.000000in}{0.000000in}}{\pgfqpoint{0.000000in}{0.041667in}}{%
\pgfpathmoveto{\pgfqpoint{0.000000in}{0.000000in}}%
\pgfpathlineto{\pgfqpoint{0.000000in}{0.041667in}}%
\pgfusepath{stroke,fill}%
}%
\begin{pgfscope}%
\pgfsys@transformshift{0.854857in}{0.528177in}%
\pgfsys@useobject{currentmarker}{}%
\end{pgfscope}%
\end{pgfscope}%
\begin{pgfscope}%
\definecolor{textcolor}{rgb}{0.150000,0.150000,0.150000}%
\pgfsetstrokecolor{textcolor}%
\pgfsetfillcolor{textcolor}%
\pgftext[x=0.854857in,y=0.430955in,,top]{\color{textcolor}\rmfamily\fontsize{10.000000}{12.000000}\selectfont \(\displaystyle 5\)}%
\end{pgfscope}%
\begin{pgfscope}%
\pgfsetbuttcap%
\pgfsetroundjoin%
\definecolor{currentfill}{rgb}{0.150000,0.150000,0.150000}%
\pgfsetfillcolor{currentfill}%
\pgfsetlinewidth{1.003750pt}%
\definecolor{currentstroke}{rgb}{0.150000,0.150000,0.150000}%
\pgfsetstrokecolor{currentstroke}%
\pgfsetdash{}{0pt}%
\pgfsys@defobject{currentmarker}{\pgfqpoint{0.000000in}{0.000000in}}{\pgfqpoint{0.000000in}{0.041667in}}{%
\pgfpathmoveto{\pgfqpoint{0.000000in}{0.000000in}}%
\pgfpathlineto{\pgfqpoint{0.000000in}{0.041667in}}%
\pgfusepath{stroke,fill}%
}%
\begin{pgfscope}%
\pgfsys@transformshift{1.142729in}{0.528177in}%
\pgfsys@useobject{currentmarker}{}%
\end{pgfscope}%
\end{pgfscope}%
\begin{pgfscope}%
\definecolor{textcolor}{rgb}{0.150000,0.150000,0.150000}%
\pgfsetstrokecolor{textcolor}%
\pgfsetfillcolor{textcolor}%
\pgftext[x=1.142729in,y=0.430955in,,top]{\color{textcolor}\rmfamily\fontsize{10.000000}{12.000000}\selectfont \(\displaystyle 10\)}%
\end{pgfscope}%
\begin{pgfscope}%
\pgfsetbuttcap%
\pgfsetroundjoin%
\definecolor{currentfill}{rgb}{0.150000,0.150000,0.150000}%
\pgfsetfillcolor{currentfill}%
\pgfsetlinewidth{1.003750pt}%
\definecolor{currentstroke}{rgb}{0.150000,0.150000,0.150000}%
\pgfsetstrokecolor{currentstroke}%
\pgfsetdash{}{0pt}%
\pgfsys@defobject{currentmarker}{\pgfqpoint{0.000000in}{0.000000in}}{\pgfqpoint{0.000000in}{0.041667in}}{%
\pgfpathmoveto{\pgfqpoint{0.000000in}{0.000000in}}%
\pgfpathlineto{\pgfqpoint{0.000000in}{0.041667in}}%
\pgfusepath{stroke,fill}%
}%
\begin{pgfscope}%
\pgfsys@transformshift{1.430602in}{0.528177in}%
\pgfsys@useobject{currentmarker}{}%
\end{pgfscope}%
\end{pgfscope}%
\begin{pgfscope}%
\definecolor{textcolor}{rgb}{0.150000,0.150000,0.150000}%
\pgfsetstrokecolor{textcolor}%
\pgfsetfillcolor{textcolor}%
\pgftext[x=1.430602in,y=0.430955in,,top]{\color{textcolor}\rmfamily\fontsize{10.000000}{12.000000}\selectfont \(\displaystyle 15\)}%
\end{pgfscope}%
\begin{pgfscope}%
\pgfsetbuttcap%
\pgfsetroundjoin%
\definecolor{currentfill}{rgb}{0.150000,0.150000,0.150000}%
\pgfsetfillcolor{currentfill}%
\pgfsetlinewidth{1.003750pt}%
\definecolor{currentstroke}{rgb}{0.150000,0.150000,0.150000}%
\pgfsetstrokecolor{currentstroke}%
\pgfsetdash{}{0pt}%
\pgfsys@defobject{currentmarker}{\pgfqpoint{0.000000in}{0.000000in}}{\pgfqpoint{0.000000in}{0.041667in}}{%
\pgfpathmoveto{\pgfqpoint{0.000000in}{0.000000in}}%
\pgfpathlineto{\pgfqpoint{0.000000in}{0.041667in}}%
\pgfusepath{stroke,fill}%
}%
\begin{pgfscope}%
\pgfsys@transformshift{1.718474in}{0.528177in}%
\pgfsys@useobject{currentmarker}{}%
\end{pgfscope}%
\end{pgfscope}%
\begin{pgfscope}%
\definecolor{textcolor}{rgb}{0.150000,0.150000,0.150000}%
\pgfsetstrokecolor{textcolor}%
\pgfsetfillcolor{textcolor}%
\pgftext[x=1.718474in,y=0.430955in,,top]{\color{textcolor}\rmfamily\fontsize{10.000000}{12.000000}\selectfont \(\displaystyle 20\)}%
\end{pgfscope}%
\begin{pgfscope}%
\pgfsetbuttcap%
\pgfsetroundjoin%
\definecolor{currentfill}{rgb}{0.150000,0.150000,0.150000}%
\pgfsetfillcolor{currentfill}%
\pgfsetlinewidth{1.003750pt}%
\definecolor{currentstroke}{rgb}{0.150000,0.150000,0.150000}%
\pgfsetstrokecolor{currentstroke}%
\pgfsetdash{}{0pt}%
\pgfsys@defobject{currentmarker}{\pgfqpoint{0.000000in}{0.000000in}}{\pgfqpoint{0.000000in}{0.041667in}}{%
\pgfpathmoveto{\pgfqpoint{0.000000in}{0.000000in}}%
\pgfpathlineto{\pgfqpoint{0.000000in}{0.041667in}}%
\pgfusepath{stroke,fill}%
}%
\begin{pgfscope}%
\pgfsys@transformshift{2.006346in}{0.528177in}%
\pgfsys@useobject{currentmarker}{}%
\end{pgfscope}%
\end{pgfscope}%
\begin{pgfscope}%
\definecolor{textcolor}{rgb}{0.150000,0.150000,0.150000}%
\pgfsetstrokecolor{textcolor}%
\pgfsetfillcolor{textcolor}%
\pgftext[x=2.006346in,y=0.430955in,,top]{\color{textcolor}\rmfamily\fontsize{10.000000}{12.000000}\selectfont \(\displaystyle 25\)}%
\end{pgfscope}%
\begin{pgfscope}%
\pgfsetbuttcap%
\pgfsetroundjoin%
\definecolor{currentfill}{rgb}{0.150000,0.150000,0.150000}%
\pgfsetfillcolor{currentfill}%
\pgfsetlinewidth{1.003750pt}%
\definecolor{currentstroke}{rgb}{0.150000,0.150000,0.150000}%
\pgfsetstrokecolor{currentstroke}%
\pgfsetdash{}{0pt}%
\pgfsys@defobject{currentmarker}{\pgfqpoint{0.000000in}{0.000000in}}{\pgfqpoint{0.000000in}{0.041667in}}{%
\pgfpathmoveto{\pgfqpoint{0.000000in}{0.000000in}}%
\pgfpathlineto{\pgfqpoint{0.000000in}{0.041667in}}%
\pgfusepath{stroke,fill}%
}%
\begin{pgfscope}%
\pgfsys@transformshift{2.294218in}{0.528177in}%
\pgfsys@useobject{currentmarker}{}%
\end{pgfscope}%
\end{pgfscope}%
\begin{pgfscope}%
\definecolor{textcolor}{rgb}{0.150000,0.150000,0.150000}%
\pgfsetstrokecolor{textcolor}%
\pgfsetfillcolor{textcolor}%
\pgftext[x=2.294218in,y=0.430955in,,top]{\color{textcolor}\rmfamily\fontsize{10.000000}{12.000000}\selectfont \(\displaystyle 30\)}%
\end{pgfscope}%
\begin{pgfscope}%
\pgfsetbuttcap%
\pgfsetroundjoin%
\definecolor{currentfill}{rgb}{0.150000,0.150000,0.150000}%
\pgfsetfillcolor{currentfill}%
\pgfsetlinewidth{1.003750pt}%
\definecolor{currentstroke}{rgb}{0.150000,0.150000,0.150000}%
\pgfsetstrokecolor{currentstroke}%
\pgfsetdash{}{0pt}%
\pgfsys@defobject{currentmarker}{\pgfqpoint{0.000000in}{0.000000in}}{\pgfqpoint{0.000000in}{0.041667in}}{%
\pgfpathmoveto{\pgfqpoint{0.000000in}{0.000000in}}%
\pgfpathlineto{\pgfqpoint{0.000000in}{0.041667in}}%
\pgfusepath{stroke,fill}%
}%
\begin{pgfscope}%
\pgfsys@transformshift{2.582091in}{0.528177in}%
\pgfsys@useobject{currentmarker}{}%
\end{pgfscope}%
\end{pgfscope}%
\begin{pgfscope}%
\definecolor{textcolor}{rgb}{0.150000,0.150000,0.150000}%
\pgfsetstrokecolor{textcolor}%
\pgfsetfillcolor{textcolor}%
\pgftext[x=2.582091in,y=0.430955in,,top]{\color{textcolor}\rmfamily\fontsize{10.000000}{12.000000}\selectfont \(\displaystyle 35\)}%
\end{pgfscope}%
\begin{pgfscope}%
\definecolor{textcolor}{rgb}{0.150000,0.150000,0.150000}%
\pgfsetstrokecolor{textcolor}%
\pgfsetfillcolor{textcolor}%
\pgftext[x=1.574538in,y=0.238855in,,top]{\color{textcolor}\rmfamily\fontsize{10.000000}{12.000000}\selectfont \textbf{Cell activity (not freezing, a.u.)}}%
\end{pgfscope}%
\begin{pgfscope}%
\pgfsetbuttcap%
\pgfsetroundjoin%
\definecolor{currentfill}{rgb}{0.150000,0.150000,0.150000}%
\pgfsetfillcolor{currentfill}%
\pgfsetlinewidth{1.003750pt}%
\definecolor{currentstroke}{rgb}{0.150000,0.150000,0.150000}%
\pgfsetstrokecolor{currentstroke}%
\pgfsetdash{}{0pt}%
\pgfsys@defobject{currentmarker}{\pgfqpoint{0.000000in}{0.000000in}}{\pgfqpoint{0.041667in}{0.000000in}}{%
\pgfpathmoveto{\pgfqpoint{0.000000in}{0.000000in}}%
\pgfpathlineto{\pgfqpoint{0.041667in}{0.000000in}}%
\pgfusepath{stroke,fill}%
}%
\begin{pgfscope}%
\pgfsys@transformshift{0.566985in}{0.528177in}%
\pgfsys@useobject{currentmarker}{}%
\end{pgfscope}%
\end{pgfscope}%
\begin{pgfscope}%
\definecolor{textcolor}{rgb}{0.150000,0.150000,0.150000}%
\pgfsetstrokecolor{textcolor}%
\pgfsetfillcolor{textcolor}%
\pgftext[x=0.469762in,y=0.528177in,right,]{\color{textcolor}\rmfamily\fontsize{10.000000}{12.000000}\selectfont \(\displaystyle 0.1\)}%
\end{pgfscope}%
\begin{pgfscope}%
\pgfsetbuttcap%
\pgfsetroundjoin%
\definecolor{currentfill}{rgb}{0.150000,0.150000,0.150000}%
\pgfsetfillcolor{currentfill}%
\pgfsetlinewidth{1.003750pt}%
\definecolor{currentstroke}{rgb}{0.150000,0.150000,0.150000}%
\pgfsetstrokecolor{currentstroke}%
\pgfsetdash{}{0pt}%
\pgfsys@defobject{currentmarker}{\pgfqpoint{0.000000in}{0.000000in}}{\pgfqpoint{0.041667in}{0.000000in}}{%
\pgfpathmoveto{\pgfqpoint{0.000000in}{0.000000in}}%
\pgfpathlineto{\pgfqpoint{0.041667in}{0.000000in}}%
\pgfusepath{stroke,fill}%
}%
\begin{pgfscope}%
\pgfsys@transformshift{0.566985in}{0.846447in}%
\pgfsys@useobject{currentmarker}{}%
\end{pgfscope}%
\end{pgfscope}%
\begin{pgfscope}%
\definecolor{textcolor}{rgb}{0.150000,0.150000,0.150000}%
\pgfsetstrokecolor{textcolor}%
\pgfsetfillcolor{textcolor}%
\pgftext[x=0.469762in,y=0.846447in,right,]{\color{textcolor}\rmfamily\fontsize{10.000000}{12.000000}\selectfont \(\displaystyle 0.2\)}%
\end{pgfscope}%
\begin{pgfscope}%
\pgfsetbuttcap%
\pgfsetroundjoin%
\definecolor{currentfill}{rgb}{0.150000,0.150000,0.150000}%
\pgfsetfillcolor{currentfill}%
\pgfsetlinewidth{1.003750pt}%
\definecolor{currentstroke}{rgb}{0.150000,0.150000,0.150000}%
\pgfsetstrokecolor{currentstroke}%
\pgfsetdash{}{0pt}%
\pgfsys@defobject{currentmarker}{\pgfqpoint{0.000000in}{0.000000in}}{\pgfqpoint{0.041667in}{0.000000in}}{%
\pgfpathmoveto{\pgfqpoint{0.000000in}{0.000000in}}%
\pgfpathlineto{\pgfqpoint{0.041667in}{0.000000in}}%
\pgfusepath{stroke,fill}%
}%
\begin{pgfscope}%
\pgfsys@transformshift{0.566985in}{1.164717in}%
\pgfsys@useobject{currentmarker}{}%
\end{pgfscope}%
\end{pgfscope}%
\begin{pgfscope}%
\definecolor{textcolor}{rgb}{0.150000,0.150000,0.150000}%
\pgfsetstrokecolor{textcolor}%
\pgfsetfillcolor{textcolor}%
\pgftext[x=0.469762in,y=1.164717in,right,]{\color{textcolor}\rmfamily\fontsize{10.000000}{12.000000}\selectfont \(\displaystyle 0.3\)}%
\end{pgfscope}%
\begin{pgfscope}%
\pgfsetbuttcap%
\pgfsetroundjoin%
\definecolor{currentfill}{rgb}{0.150000,0.150000,0.150000}%
\pgfsetfillcolor{currentfill}%
\pgfsetlinewidth{1.003750pt}%
\definecolor{currentstroke}{rgb}{0.150000,0.150000,0.150000}%
\pgfsetstrokecolor{currentstroke}%
\pgfsetdash{}{0pt}%
\pgfsys@defobject{currentmarker}{\pgfqpoint{0.000000in}{0.000000in}}{\pgfqpoint{0.041667in}{0.000000in}}{%
\pgfpathmoveto{\pgfqpoint{0.000000in}{0.000000in}}%
\pgfpathlineto{\pgfqpoint{0.041667in}{0.000000in}}%
\pgfusepath{stroke,fill}%
}%
\begin{pgfscope}%
\pgfsys@transformshift{0.566985in}{1.482987in}%
\pgfsys@useobject{currentmarker}{}%
\end{pgfscope}%
\end{pgfscope}%
\begin{pgfscope}%
\definecolor{textcolor}{rgb}{0.150000,0.150000,0.150000}%
\pgfsetstrokecolor{textcolor}%
\pgfsetfillcolor{textcolor}%
\pgftext[x=0.469762in,y=1.482987in,right,]{\color{textcolor}\rmfamily\fontsize{10.000000}{12.000000}\selectfont \(\displaystyle 0.4\)}%
\end{pgfscope}%
\begin{pgfscope}%
\pgfsetbuttcap%
\pgfsetroundjoin%
\definecolor{currentfill}{rgb}{0.150000,0.150000,0.150000}%
\pgfsetfillcolor{currentfill}%
\pgfsetlinewidth{1.003750pt}%
\definecolor{currentstroke}{rgb}{0.150000,0.150000,0.150000}%
\pgfsetstrokecolor{currentstroke}%
\pgfsetdash{}{0pt}%
\pgfsys@defobject{currentmarker}{\pgfqpoint{0.000000in}{0.000000in}}{\pgfqpoint{0.041667in}{0.000000in}}{%
\pgfpathmoveto{\pgfqpoint{0.000000in}{0.000000in}}%
\pgfpathlineto{\pgfqpoint{0.041667in}{0.000000in}}%
\pgfusepath{stroke,fill}%
}%
\begin{pgfscope}%
\pgfsys@transformshift{0.566985in}{1.801257in}%
\pgfsys@useobject{currentmarker}{}%
\end{pgfscope}%
\end{pgfscope}%
\begin{pgfscope}%
\definecolor{textcolor}{rgb}{0.150000,0.150000,0.150000}%
\pgfsetstrokecolor{textcolor}%
\pgfsetfillcolor{textcolor}%
\pgftext[x=0.469762in,y=1.801257in,right,]{\color{textcolor}\rmfamily\fontsize{10.000000}{12.000000}\selectfont \(\displaystyle 0.5\)}%
\end{pgfscope}%
\begin{pgfscope}%
\pgfsetbuttcap%
\pgfsetroundjoin%
\definecolor{currentfill}{rgb}{0.150000,0.150000,0.150000}%
\pgfsetfillcolor{currentfill}%
\pgfsetlinewidth{1.003750pt}%
\definecolor{currentstroke}{rgb}{0.150000,0.150000,0.150000}%
\pgfsetstrokecolor{currentstroke}%
\pgfsetdash{}{0pt}%
\pgfsys@defobject{currentmarker}{\pgfqpoint{0.000000in}{0.000000in}}{\pgfqpoint{0.041667in}{0.000000in}}{%
\pgfpathmoveto{\pgfqpoint{0.000000in}{0.000000in}}%
\pgfpathlineto{\pgfqpoint{0.041667in}{0.000000in}}%
\pgfusepath{stroke,fill}%
}%
\begin{pgfscope}%
\pgfsys@transformshift{0.566985in}{2.119526in}%
\pgfsys@useobject{currentmarker}{}%
\end{pgfscope}%
\end{pgfscope}%
\begin{pgfscope}%
\definecolor{textcolor}{rgb}{0.150000,0.150000,0.150000}%
\pgfsetstrokecolor{textcolor}%
\pgfsetfillcolor{textcolor}%
\pgftext[x=0.469762in,y=2.119526in,right,]{\color{textcolor}\rmfamily\fontsize{10.000000}{12.000000}\selectfont \(\displaystyle 0.6\)}%
\end{pgfscope}%
\begin{pgfscope}%
\pgfsetbuttcap%
\pgfsetroundjoin%
\definecolor{currentfill}{rgb}{0.150000,0.150000,0.150000}%
\pgfsetfillcolor{currentfill}%
\pgfsetlinewidth{1.003750pt}%
\definecolor{currentstroke}{rgb}{0.150000,0.150000,0.150000}%
\pgfsetstrokecolor{currentstroke}%
\pgfsetdash{}{0pt}%
\pgfsys@defobject{currentmarker}{\pgfqpoint{0.000000in}{0.000000in}}{\pgfqpoint{0.041667in}{0.000000in}}{%
\pgfpathmoveto{\pgfqpoint{0.000000in}{0.000000in}}%
\pgfpathlineto{\pgfqpoint{0.041667in}{0.000000in}}%
\pgfusepath{stroke,fill}%
}%
\begin{pgfscope}%
\pgfsys@transformshift{0.566985in}{2.437796in}%
\pgfsys@useobject{currentmarker}{}%
\end{pgfscope}%
\end{pgfscope}%
\begin{pgfscope}%
\definecolor{textcolor}{rgb}{0.150000,0.150000,0.150000}%
\pgfsetstrokecolor{textcolor}%
\pgfsetfillcolor{textcolor}%
\pgftext[x=0.469762in,y=2.437796in,right,]{\color{textcolor}\rmfamily\fontsize{10.000000}{12.000000}\selectfont \(\displaystyle 0.7\)}%
\end{pgfscope}%
\begin{pgfscope}%
\pgfsetbuttcap%
\pgfsetroundjoin%
\definecolor{currentfill}{rgb}{0.150000,0.150000,0.150000}%
\pgfsetfillcolor{currentfill}%
\pgfsetlinewidth{1.003750pt}%
\definecolor{currentstroke}{rgb}{0.150000,0.150000,0.150000}%
\pgfsetstrokecolor{currentstroke}%
\pgfsetdash{}{0pt}%
\pgfsys@defobject{currentmarker}{\pgfqpoint{0.000000in}{0.000000in}}{\pgfqpoint{0.041667in}{0.000000in}}{%
\pgfpathmoveto{\pgfqpoint{0.000000in}{0.000000in}}%
\pgfpathlineto{\pgfqpoint{0.041667in}{0.000000in}}%
\pgfusepath{stroke,fill}%
}%
\begin{pgfscope}%
\pgfsys@transformshift{0.566985in}{2.756066in}%
\pgfsys@useobject{currentmarker}{}%
\end{pgfscope}%
\end{pgfscope}%
\begin{pgfscope}%
\definecolor{textcolor}{rgb}{0.150000,0.150000,0.150000}%
\pgfsetstrokecolor{textcolor}%
\pgfsetfillcolor{textcolor}%
\pgftext[x=0.469762in,y=2.756066in,right,]{\color{textcolor}\rmfamily\fontsize{10.000000}{12.000000}\selectfont \(\displaystyle 0.8\)}%
\end{pgfscope}%
\begin{pgfscope}%
\pgfsetbuttcap%
\pgfsetroundjoin%
\definecolor{currentfill}{rgb}{0.150000,0.150000,0.150000}%
\pgfsetfillcolor{currentfill}%
\pgfsetlinewidth{1.003750pt}%
\definecolor{currentstroke}{rgb}{0.150000,0.150000,0.150000}%
\pgfsetstrokecolor{currentstroke}%
\pgfsetdash{}{0pt}%
\pgfsys@defobject{currentmarker}{\pgfqpoint{0.000000in}{0.000000in}}{\pgfqpoint{0.041667in}{0.000000in}}{%
\pgfpathmoveto{\pgfqpoint{0.000000in}{0.000000in}}%
\pgfpathlineto{\pgfqpoint{0.041667in}{0.000000in}}%
\pgfusepath{stroke,fill}%
}%
\begin{pgfscope}%
\pgfsys@transformshift{0.566985in}{3.074336in}%
\pgfsys@useobject{currentmarker}{}%
\end{pgfscope}%
\end{pgfscope}%
\begin{pgfscope}%
\definecolor{textcolor}{rgb}{0.150000,0.150000,0.150000}%
\pgfsetstrokecolor{textcolor}%
\pgfsetfillcolor{textcolor}%
\pgftext[x=0.469762in,y=3.074336in,right,]{\color{textcolor}\rmfamily\fontsize{10.000000}{12.000000}\selectfont \(\displaystyle 0.9\)}%
\end{pgfscope}%
\begin{pgfscope}%
\pgfsetbuttcap%
\pgfsetroundjoin%
\definecolor{currentfill}{rgb}{0.150000,0.150000,0.150000}%
\pgfsetfillcolor{currentfill}%
\pgfsetlinewidth{1.003750pt}%
\definecolor{currentstroke}{rgb}{0.150000,0.150000,0.150000}%
\pgfsetstrokecolor{currentstroke}%
\pgfsetdash{}{0pt}%
\pgfsys@defobject{currentmarker}{\pgfqpoint{0.000000in}{0.000000in}}{\pgfqpoint{0.041667in}{0.000000in}}{%
\pgfpathmoveto{\pgfqpoint{0.000000in}{0.000000in}}%
\pgfpathlineto{\pgfqpoint{0.041667in}{0.000000in}}%
\pgfusepath{stroke,fill}%
}%
\begin{pgfscope}%
\pgfsys@transformshift{0.566985in}{3.392606in}%
\pgfsys@useobject{currentmarker}{}%
\end{pgfscope}%
\end{pgfscope}%
\begin{pgfscope}%
\definecolor{textcolor}{rgb}{0.150000,0.150000,0.150000}%
\pgfsetstrokecolor{textcolor}%
\pgfsetfillcolor{textcolor}%
\pgftext[x=0.469762in,y=3.392606in,right,]{\color{textcolor}\rmfamily\fontsize{10.000000}{12.000000}\selectfont \(\displaystyle 1.0\)}%
\end{pgfscope}%
\begin{pgfscope}%
\definecolor{textcolor}{rgb}{0.150000,0.150000,0.150000}%
\pgfsetstrokecolor{textcolor}%
\pgfsetfillcolor{textcolor}%
\pgftext[x=0.222848in,y=1.960392in,,bottom,rotate=90.000000]{\color{textcolor}\rmfamily\fontsize{10.000000}{12.000000}\selectfont \textbf{Cumulative porportion}}%
\end{pgfscope}%
\begin{pgfscope}%
\pgfpathrectangle{\pgfqpoint{0.566985in}{0.528177in}}{\pgfqpoint{2.015106in}{2.864429in}} %
\pgfusepath{clip}%
\pgfsetroundcap%
\pgfsetroundjoin%
\pgfsetlinewidth{1.003750pt}%
\definecolor{currentstroke}{rgb}{0.200000,0.427451,0.650980}%
\pgfsetstrokecolor{currentstroke}%
\pgfsetdash{}{0pt}%
\pgfpathmoveto{\pgfqpoint{0.567029in}{0.644039in}}%
\pgfpathlineto{\pgfqpoint{0.598687in}{0.951315in}}%
\pgfpathlineto{\pgfqpoint{0.630345in}{1.252953in}}%
\pgfpathlineto{\pgfqpoint{0.662003in}{1.501028in}}%
\pgfpathlineto{\pgfqpoint{0.693661in}{1.802666in}}%
\pgfpathlineto{\pgfqpoint{0.725318in}{2.031009in}}%
\pgfpathlineto{\pgfqpoint{0.756976in}{2.259351in}}%
\pgfpathlineto{\pgfqpoint{0.788634in}{2.420036in}}%
\pgfpathlineto{\pgfqpoint{0.820292in}{2.580722in}}%
\pgfpathlineto{\pgfqpoint{0.851950in}{2.665293in}}%
\pgfpathlineto{\pgfqpoint{0.883608in}{2.786512in}}%
\pgfpathlineto{\pgfqpoint{0.915266in}{2.890817in}}%
\pgfpathlineto{\pgfqpoint{0.946924in}{2.981026in}}%
\pgfpathlineto{\pgfqpoint{0.978582in}{3.068416in}}%
\pgfpathlineto{\pgfqpoint{1.010240in}{3.133254in}}%
\pgfpathlineto{\pgfqpoint{1.041898in}{3.172721in}}%
\pgfpathlineto{\pgfqpoint{1.073555in}{3.212187in}}%
\pgfpathlineto{\pgfqpoint{1.105213in}{3.231921in}}%
\pgfpathlineto{\pgfqpoint{1.136871in}{3.251654in}}%
\pgfpathlineto{\pgfqpoint{1.168529in}{3.260111in}}%
\pgfpathlineto{\pgfqpoint{1.200187in}{3.277025in}}%
\pgfpathlineto{\pgfqpoint{1.231845in}{3.302397in}}%
\pgfpathlineto{\pgfqpoint{1.263503in}{3.316492in}}%
\pgfpathlineto{\pgfqpoint{1.295161in}{3.324949in}}%
\pgfpathlineto{\pgfqpoint{1.326819in}{3.327768in}}%
\pgfpathlineto{\pgfqpoint{1.358477in}{3.341863in}}%
\pgfpathlineto{\pgfqpoint{1.390135in}{3.344683in}}%
\pgfpathlineto{\pgfqpoint{1.421793in}{3.355959in}}%
\pgfpathlineto{\pgfqpoint{1.453450in}{3.358778in}}%
\pgfpathlineto{\pgfqpoint{1.485108in}{3.364416in}}%
\pgfpathlineto{\pgfqpoint{1.516766in}{3.364416in}}%
\pgfpathlineto{\pgfqpoint{1.548424in}{3.372873in}}%
\pgfpathlineto{\pgfqpoint{1.580082in}{3.378511in}}%
\pgfpathlineto{\pgfqpoint{1.611740in}{3.378511in}}%
\pgfpathlineto{\pgfqpoint{1.643398in}{3.378511in}}%
\pgfpathlineto{\pgfqpoint{1.675056in}{3.384149in}}%
\pgfpathlineto{\pgfqpoint{1.706714in}{3.384149in}}%
\pgfpathlineto{\pgfqpoint{1.738372in}{3.384149in}}%
\pgfpathlineto{\pgfqpoint{1.770030in}{3.386968in}}%
\pgfpathlineto{\pgfqpoint{1.801688in}{3.386968in}}%
\pgfpathlineto{\pgfqpoint{1.833345in}{3.389787in}}%
\pgfpathlineto{\pgfqpoint{1.865003in}{3.389787in}}%
\pgfpathlineto{\pgfqpoint{1.896661in}{3.389787in}}%
\pgfpathlineto{\pgfqpoint{1.928319in}{3.389787in}}%
\pgfpathlineto{\pgfqpoint{1.959977in}{3.389787in}}%
\pgfpathlineto{\pgfqpoint{1.991635in}{3.389787in}}%
\pgfpathlineto{\pgfqpoint{2.023293in}{3.389787in}}%
\pgfpathlineto{\pgfqpoint{2.054951in}{3.389787in}}%
\pgfpathlineto{\pgfqpoint{2.086609in}{3.389787in}}%
\pgfpathlineto{\pgfqpoint{2.118267in}{3.392606in}}%
\pgfusepath{stroke}%
\end{pgfscope}%
\begin{pgfscope}%
\pgfpathrectangle{\pgfqpoint{0.566985in}{0.528177in}}{\pgfqpoint{2.015106in}{2.864429in}} %
\pgfusepath{clip}%
\pgfsetroundcap%
\pgfsetroundjoin%
\pgfsetlinewidth{1.003750pt}%
\definecolor{currentstroke}{rgb}{0.168627,0.670588,0.494118}%
\pgfsetstrokecolor{currentstroke}%
\pgfsetdash{}{0pt}%
\pgfpathmoveto{\pgfqpoint{0.567122in}{0.868825in}}%
\pgfpathlineto{\pgfqpoint{0.603616in}{1.241798in}}%
\pgfpathlineto{\pgfqpoint{0.640109in}{1.502878in}}%
\pgfpathlineto{\pgfqpoint{0.676603in}{1.751527in}}%
\pgfpathlineto{\pgfqpoint{0.713096in}{2.049905in}}%
\pgfpathlineto{\pgfqpoint{0.749590in}{2.286121in}}%
\pgfpathlineto{\pgfqpoint{0.786083in}{2.472607in}}%
\pgfpathlineto{\pgfqpoint{0.822577in}{2.628012in}}%
\pgfpathlineto{\pgfqpoint{0.859070in}{2.770985in}}%
\pgfpathlineto{\pgfqpoint{0.895564in}{2.882877in}}%
\pgfpathlineto{\pgfqpoint{0.932057in}{2.945039in}}%
\pgfpathlineto{\pgfqpoint{0.968551in}{3.032066in}}%
\pgfpathlineto{\pgfqpoint{1.005044in}{3.112877in}}%
\pgfpathlineto{\pgfqpoint{1.041538in}{3.150174in}}%
\pgfpathlineto{\pgfqpoint{1.078031in}{3.168823in}}%
\pgfpathlineto{\pgfqpoint{1.114525in}{3.181255in}}%
\pgfpathlineto{\pgfqpoint{1.151018in}{3.193688in}}%
\pgfpathlineto{\pgfqpoint{1.187512in}{3.224769in}}%
\pgfpathlineto{\pgfqpoint{1.224005in}{3.255850in}}%
\pgfpathlineto{\pgfqpoint{1.260499in}{3.293147in}}%
\pgfpathlineto{\pgfqpoint{1.296992in}{3.318012in}}%
\pgfpathlineto{\pgfqpoint{1.333486in}{3.330444in}}%
\pgfpathlineto{\pgfqpoint{1.369979in}{3.336660in}}%
\pgfpathlineto{\pgfqpoint{1.406473in}{3.349093in}}%
\pgfpathlineto{\pgfqpoint{1.442966in}{3.355309in}}%
\pgfpathlineto{\pgfqpoint{1.479460in}{3.361525in}}%
\pgfpathlineto{\pgfqpoint{1.515953in}{3.361525in}}%
\pgfpathlineto{\pgfqpoint{1.552447in}{3.367741in}}%
\pgfpathlineto{\pgfqpoint{1.588940in}{3.367741in}}%
\pgfpathlineto{\pgfqpoint{1.625434in}{3.367741in}}%
\pgfpathlineto{\pgfqpoint{1.661927in}{3.367741in}}%
\pgfpathlineto{\pgfqpoint{1.698421in}{3.367741in}}%
\pgfpathlineto{\pgfqpoint{1.734914in}{3.380174in}}%
\pgfpathlineto{\pgfqpoint{1.771408in}{3.380174in}}%
\pgfpathlineto{\pgfqpoint{1.807901in}{3.386390in}}%
\pgfpathlineto{\pgfqpoint{1.844395in}{3.386390in}}%
\pgfpathlineto{\pgfqpoint{1.880888in}{3.386390in}}%
\pgfpathlineto{\pgfqpoint{1.917382in}{3.386390in}}%
\pgfpathlineto{\pgfqpoint{1.953875in}{3.386390in}}%
\pgfpathlineto{\pgfqpoint{1.990369in}{3.386390in}}%
\pgfpathlineto{\pgfqpoint{2.026862in}{3.386390in}}%
\pgfpathlineto{\pgfqpoint{2.063356in}{3.386390in}}%
\pgfpathlineto{\pgfqpoint{2.099849in}{3.386390in}}%
\pgfpathlineto{\pgfqpoint{2.136343in}{3.386390in}}%
\pgfpathlineto{\pgfqpoint{2.172836in}{3.386390in}}%
\pgfpathlineto{\pgfqpoint{2.209330in}{3.386390in}}%
\pgfpathlineto{\pgfqpoint{2.245823in}{3.386390in}}%
\pgfpathlineto{\pgfqpoint{2.282317in}{3.386390in}}%
\pgfpathlineto{\pgfqpoint{2.318810in}{3.386390in}}%
\pgfpathlineto{\pgfqpoint{2.355304in}{3.392606in}}%
\pgfusepath{stroke}%
\end{pgfscope}%
\begin{pgfscope}%
\pgfpathrectangle{\pgfqpoint{0.566985in}{0.528177in}}{\pgfqpoint{2.015106in}{2.864429in}} %
\pgfusepath{clip}%
\pgfsetroundcap%
\pgfsetroundjoin%
\pgfsetlinewidth{1.003750pt}%
\definecolor{currentstroke}{rgb}{1.000000,0.494118,0.250980}%
\pgfsetstrokecolor{currentstroke}%
\pgfsetdash{}{0pt}%
\pgfpathmoveto{\pgfqpoint{0.567067in}{0.683820in}}%
\pgfpathlineto{\pgfqpoint{0.590343in}{0.873385in}}%
\pgfpathlineto{\pgfqpoint{0.613618in}{1.102859in}}%
\pgfpathlineto{\pgfqpoint{0.636893in}{1.297412in}}%
\pgfpathlineto{\pgfqpoint{0.660169in}{1.501943in}}%
\pgfpathlineto{\pgfqpoint{0.683444in}{1.711463in}}%
\pgfpathlineto{\pgfqpoint{0.706720in}{1.831188in}}%
\pgfpathlineto{\pgfqpoint{0.729995in}{1.960890in}}%
\pgfpathlineto{\pgfqpoint{0.753271in}{2.100570in}}%
\pgfpathlineto{\pgfqpoint{0.776546in}{2.245238in}}%
\pgfpathlineto{\pgfqpoint{0.799821in}{2.359975in}}%
\pgfpathlineto{\pgfqpoint{0.823097in}{2.469723in}}%
\pgfpathlineto{\pgfqpoint{0.846372in}{2.584460in}}%
\pgfpathlineto{\pgfqpoint{0.869648in}{2.679243in}}%
\pgfpathlineto{\pgfqpoint{0.892923in}{2.739105in}}%
\pgfpathlineto{\pgfqpoint{0.916199in}{2.823911in}}%
\pgfpathlineto{\pgfqpoint{0.939474in}{2.898739in}}%
\pgfpathlineto{\pgfqpoint{0.962749in}{2.923682in}}%
\pgfpathlineto{\pgfqpoint{0.986025in}{2.988533in}}%
\pgfpathlineto{\pgfqpoint{1.009300in}{3.058373in}}%
\pgfpathlineto{\pgfqpoint{1.032576in}{3.073339in}}%
\pgfpathlineto{\pgfqpoint{1.055851in}{3.143178in}}%
\pgfpathlineto{\pgfqpoint{1.079127in}{3.188075in}}%
\pgfpathlineto{\pgfqpoint{1.102402in}{3.222995in}}%
\pgfpathlineto{\pgfqpoint{1.125678in}{3.242950in}}%
\pgfpathlineto{\pgfqpoint{1.148953in}{3.272881in}}%
\pgfpathlineto{\pgfqpoint{1.172228in}{3.312789in}}%
\pgfpathlineto{\pgfqpoint{1.195504in}{3.317778in}}%
\pgfpathlineto{\pgfqpoint{1.218779in}{3.322766in}}%
\pgfpathlineto{\pgfqpoint{1.242055in}{3.342721in}}%
\pgfpathlineto{\pgfqpoint{1.265330in}{3.352698in}}%
\pgfpathlineto{\pgfqpoint{1.288606in}{3.357686in}}%
\pgfpathlineto{\pgfqpoint{1.311881in}{3.362675in}}%
\pgfpathlineto{\pgfqpoint{1.335156in}{3.372652in}}%
\pgfpathlineto{\pgfqpoint{1.358432in}{3.372652in}}%
\pgfpathlineto{\pgfqpoint{1.381707in}{3.372652in}}%
\pgfpathlineto{\pgfqpoint{1.404983in}{3.372652in}}%
\pgfpathlineto{\pgfqpoint{1.428258in}{3.372652in}}%
\pgfpathlineto{\pgfqpoint{1.451534in}{3.377641in}}%
\pgfpathlineto{\pgfqpoint{1.474809in}{3.377641in}}%
\pgfpathlineto{\pgfqpoint{1.498084in}{3.377641in}}%
\pgfpathlineto{\pgfqpoint{1.521360in}{3.382629in}}%
\pgfpathlineto{\pgfqpoint{1.544635in}{3.382629in}}%
\pgfpathlineto{\pgfqpoint{1.567911in}{3.382629in}}%
\pgfpathlineto{\pgfqpoint{1.591186in}{3.387618in}}%
\pgfpathlineto{\pgfqpoint{1.614462in}{3.387618in}}%
\pgfpathlineto{\pgfqpoint{1.637737in}{3.387618in}}%
\pgfpathlineto{\pgfqpoint{1.661013in}{3.387618in}}%
\pgfpathlineto{\pgfqpoint{1.684288in}{3.387618in}}%
\pgfpathlineto{\pgfqpoint{1.707563in}{3.392606in}}%
\pgfusepath{stroke}%
\end{pgfscope}%
\begin{pgfscope}%
\pgfpathrectangle{\pgfqpoint{0.566985in}{0.528177in}}{\pgfqpoint{2.015106in}{2.864429in}} %
\pgfusepath{clip}%
\pgfsetroundcap%
\pgfsetroundjoin%
\pgfsetlinewidth{1.003750pt}%
\definecolor{currentstroke}{rgb}{1.000000,0.694118,0.250980}%
\pgfsetstrokecolor{currentstroke}%
\pgfsetdash{}{0pt}%
\pgfpathmoveto{\pgfqpoint{0.567186in}{0.625042in}}%
\pgfpathlineto{\pgfqpoint{0.590220in}{1.048563in}}%
\pgfpathlineto{\pgfqpoint{0.613253in}{1.430151in}}%
\pgfpathlineto{\pgfqpoint{0.636287in}{1.690135in}}%
\pgfpathlineto{\pgfqpoint{0.659321in}{1.962698in}}%
\pgfpathlineto{\pgfqpoint{0.682354in}{2.134622in}}%
\pgfpathlineto{\pgfqpoint{0.705388in}{2.289774in}}%
\pgfpathlineto{\pgfqpoint{0.728422in}{2.453312in}}%
\pgfpathlineto{\pgfqpoint{0.751455in}{2.612656in}}%
\pgfpathlineto{\pgfqpoint{0.774489in}{2.713295in}}%
\pgfpathlineto{\pgfqpoint{0.797523in}{2.805547in}}%
\pgfpathlineto{\pgfqpoint{0.820556in}{2.893606in}}%
\pgfpathlineto{\pgfqpoint{0.843590in}{2.964892in}}%
\pgfpathlineto{\pgfqpoint{0.866624in}{3.015211in}}%
\pgfpathlineto{\pgfqpoint{0.889657in}{3.090690in}}%
\pgfpathlineto{\pgfqpoint{0.912691in}{3.157783in}}%
\pgfpathlineto{\pgfqpoint{0.935725in}{3.182942in}}%
\pgfpathlineto{\pgfqpoint{0.958758in}{3.195522in}}%
\pgfpathlineto{\pgfqpoint{0.981792in}{3.241648in}}%
\pgfpathlineto{\pgfqpoint{1.004826in}{3.262615in}}%
\pgfpathlineto{\pgfqpoint{1.027860in}{3.283581in}}%
\pgfpathlineto{\pgfqpoint{1.050893in}{3.287774in}}%
\pgfpathlineto{\pgfqpoint{1.073927in}{3.304547in}}%
\pgfpathlineto{\pgfqpoint{1.096961in}{3.338094in}}%
\pgfpathlineto{\pgfqpoint{1.119994in}{3.346480in}}%
\pgfpathlineto{\pgfqpoint{1.143028in}{3.354867in}}%
\pgfpathlineto{\pgfqpoint{1.166062in}{3.359060in}}%
\pgfpathlineto{\pgfqpoint{1.189095in}{3.363253in}}%
\pgfpathlineto{\pgfqpoint{1.212129in}{3.363253in}}%
\pgfpathlineto{\pgfqpoint{1.235163in}{3.367447in}}%
\pgfpathlineto{\pgfqpoint{1.258196in}{3.367447in}}%
\pgfpathlineto{\pgfqpoint{1.281230in}{3.375833in}}%
\pgfpathlineto{\pgfqpoint{1.304264in}{3.375833in}}%
\pgfpathlineto{\pgfqpoint{1.327297in}{3.375833in}}%
\pgfpathlineto{\pgfqpoint{1.350331in}{3.380026in}}%
\pgfpathlineto{\pgfqpoint{1.373365in}{3.380026in}}%
\pgfpathlineto{\pgfqpoint{1.396398in}{3.380026in}}%
\pgfpathlineto{\pgfqpoint{1.419432in}{3.380026in}}%
\pgfpathlineto{\pgfqpoint{1.442466in}{3.380026in}}%
\pgfpathlineto{\pgfqpoint{1.465499in}{3.380026in}}%
\pgfpathlineto{\pgfqpoint{1.488533in}{3.380026in}}%
\pgfpathlineto{\pgfqpoint{1.511567in}{3.384220in}}%
\pgfpathlineto{\pgfqpoint{1.534600in}{3.384220in}}%
\pgfpathlineto{\pgfqpoint{1.557634in}{3.384220in}}%
\pgfpathlineto{\pgfqpoint{1.580668in}{3.388413in}}%
\pgfpathlineto{\pgfqpoint{1.603701in}{3.388413in}}%
\pgfpathlineto{\pgfqpoint{1.626735in}{3.388413in}}%
\pgfpathlineto{\pgfqpoint{1.649769in}{3.388413in}}%
\pgfpathlineto{\pgfqpoint{1.672803in}{3.388413in}}%
\pgfpathlineto{\pgfqpoint{1.695836in}{3.392606in}}%
\pgfusepath{stroke}%
\end{pgfscope}%
\begin{pgfscope}%
\pgfsetrectcap%
\pgfsetmiterjoin%
\pgfsetlinewidth{1.254687pt}%
\definecolor{currentstroke}{rgb}{0.150000,0.150000,0.150000}%
\pgfsetstrokecolor{currentstroke}%
\pgfsetdash{}{0pt}%
\pgfpathmoveto{\pgfqpoint{0.566985in}{0.528177in}}%
\pgfpathlineto{\pgfqpoint{0.566985in}{3.392606in}}%
\pgfusepath{stroke}%
\end{pgfscope}%
\begin{pgfscope}%
\pgfsetrectcap%
\pgfsetmiterjoin%
\pgfsetlinewidth{1.254687pt}%
\definecolor{currentstroke}{rgb}{0.150000,0.150000,0.150000}%
\pgfsetstrokecolor{currentstroke}%
\pgfsetdash{}{0pt}%
\pgfpathmoveto{\pgfqpoint{0.566985in}{0.528177in}}%
\pgfpathlineto{\pgfqpoint{2.582091in}{0.528177in}}%
\pgfusepath{stroke}%
\end{pgfscope}%
\begin{pgfscope}%
\pgfsetbuttcap%
\pgfsetmiterjoin%
\definecolor{currentfill}{rgb}{1.000000,1.000000,1.000000}%
\pgfsetfillcolor{currentfill}%
\pgfsetlinewidth{0.000000pt}%
\definecolor{currentstroke}{rgb}{0.000000,0.000000,0.000000}%
\pgfsetstrokecolor{currentstroke}%
\pgfsetstrokeopacity{0.000000}%
\pgfsetdash{}{0pt}%
\pgfpathmoveto{\pgfqpoint{3.186623in}{0.528177in}}%
\pgfpathlineto{\pgfqpoint{5.201729in}{0.528177in}}%
\pgfpathlineto{\pgfqpoint{5.201729in}{2.653399in}}%
\pgfpathlineto{\pgfqpoint{3.186623in}{2.653399in}}%
\pgfpathclose%
\pgfusepath{fill}%
\end{pgfscope}%
\begin{pgfscope}%
\pgfsetroundcap%
\pgfsetroundjoin%
\pgfsetlinewidth{1.003750pt}%
\definecolor{currentstroke}{rgb}{0.200000,0.427451,0.650980}%
\pgfsetstrokecolor{currentstroke}%
\pgfsetdash{}{0pt}%
\pgfpathmoveto{\pgfqpoint{3.085112in}{3.319977in}}%
\pgfpathlineto{\pgfqpoint{3.196223in}{3.319977in}}%
\pgfusepath{stroke}%
\end{pgfscope}%
\begin{pgfscope}%
\definecolor{textcolor}{rgb}{1.000000,1.000000,1.000000}%
\pgfsetstrokecolor{textcolor}%
\pgfsetfillcolor{textcolor}%
\pgftext[x=3.285112in,y=3.281088in,left,base]{\color{textcolor}\rmfamily\fontsize{8.000000}{9.600000}\selectfont WT + Vehicle (1129)}%
\end{pgfscope}%
\begin{pgfscope}%
\pgfsetroundcap%
\pgfsetroundjoin%
\pgfsetlinewidth{1.003750pt}%
\definecolor{currentstroke}{rgb}{0.168627,0.670588,0.494118}%
\pgfsetstrokecolor{currentstroke}%
\pgfsetdash{}{0pt}%
\pgfpathmoveto{\pgfqpoint{3.085112in}{3.153338in}}%
\pgfpathlineto{\pgfqpoint{3.196223in}{3.153338in}}%
\pgfusepath{stroke}%
\end{pgfscope}%
\begin{pgfscope}%
\definecolor{textcolor}{rgb}{1.000000,1.000000,1.000000}%
\pgfsetstrokecolor{textcolor}%
\pgfsetfillcolor{textcolor}%
\pgftext[x=3.285112in,y=3.114449in,left,base]{\color{textcolor}\rmfamily\fontsize{8.000000}{9.600000}\selectfont WT + TAT-GluA2\textsubscript{3Y} (512)}%
\end{pgfscope}%
\begin{pgfscope}%
\pgfsetroundcap%
\pgfsetroundjoin%
\pgfsetlinewidth{1.003750pt}%
\definecolor{currentstroke}{rgb}{1.000000,0.494118,0.250980}%
\pgfsetstrokecolor{currentstroke}%
\pgfsetdash{}{0pt}%
\pgfpathmoveto{\pgfqpoint{3.085112in}{2.986698in}}%
\pgfpathlineto{\pgfqpoint{3.196223in}{2.986698in}}%
\pgfusepath{stroke}%
\end{pgfscope}%
\begin{pgfscope}%
\definecolor{textcolor}{rgb}{1.000000,1.000000,1.000000}%
\pgfsetstrokecolor{textcolor}%
\pgfsetfillcolor{textcolor}%
\pgftext[x=3.285112in,y=2.947809in,left,base]{\color{textcolor}\rmfamily\fontsize{8.000000}{9.600000}\selectfont Tg + Vehicle (638)}%
\end{pgfscope}%
\begin{pgfscope}%
\pgfsetroundcap%
\pgfsetroundjoin%
\pgfsetlinewidth{1.003750pt}%
\definecolor{currentstroke}{rgb}{1.000000,0.694118,0.250980}%
\pgfsetstrokecolor{currentstroke}%
\pgfsetdash{}{0pt}%
\pgfpathmoveto{\pgfqpoint{3.085112in}{2.820059in}}%
\pgfpathlineto{\pgfqpoint{3.196223in}{2.820059in}}%
\pgfusepath{stroke}%
\end{pgfscope}%
\begin{pgfscope}%
\definecolor{textcolor}{rgb}{1.000000,1.000000,1.000000}%
\pgfsetstrokecolor{textcolor}%
\pgfsetfillcolor{textcolor}%
\pgftext[x=3.285112in,y=2.781170in,left,base]{\color{textcolor}\rmfamily\fontsize{8.000000}{9.600000}\selectfont Tg + TAT-GluA2\textsubscript{3Y} (759)}%
\end{pgfscope}%
\begin{pgfscope}%
\pgfsetroundcap%
\pgfsetroundjoin%
\pgfsetlinewidth{1.003750pt}%
\definecolor{currentstroke}{rgb}{0.200000,0.427451,0.650980}%
\pgfsetstrokecolor{currentstroke}%
\pgfsetdash{}{0pt}%
\pgfpathmoveto{\pgfqpoint{3.085112in}{3.319977in}}%
\pgfpathlineto{\pgfqpoint{3.196223in}{3.319977in}}%
\pgfusepath{stroke}%
\end{pgfscope}%
\begin{pgfscope}%
\definecolor{textcolor}{rgb}{1.000000,1.000000,1.000000}%
\pgfsetstrokecolor{textcolor}%
\pgfsetfillcolor{textcolor}%
\pgftext[x=3.285112in,y=3.281088in,left,base]{\color{textcolor}\rmfamily\fontsize{8.000000}{9.600000}\selectfont WT + Vehicle (1129)}%
\end{pgfscope}%
\begin{pgfscope}%
\pgfsetroundcap%
\pgfsetroundjoin%
\pgfsetlinewidth{1.003750pt}%
\definecolor{currentstroke}{rgb}{0.168627,0.670588,0.494118}%
\pgfsetstrokecolor{currentstroke}%
\pgfsetdash{}{0pt}%
\pgfpathmoveto{\pgfqpoint{3.085112in}{3.153338in}}%
\pgfpathlineto{\pgfqpoint{3.196223in}{3.153338in}}%
\pgfusepath{stroke}%
\end{pgfscope}%
\begin{pgfscope}%
\definecolor{textcolor}{rgb}{1.000000,1.000000,1.000000}%
\pgfsetstrokecolor{textcolor}%
\pgfsetfillcolor{textcolor}%
\pgftext[x=3.285112in,y=3.114449in,left,base]{\color{textcolor}\rmfamily\fontsize{8.000000}{9.600000}\selectfont WT + TAT-GluA2\textsubscript{3Y} (512)}%
\end{pgfscope}%
\begin{pgfscope}%
\pgfsetroundcap%
\pgfsetroundjoin%
\pgfsetlinewidth{1.003750pt}%
\definecolor{currentstroke}{rgb}{1.000000,0.494118,0.250980}%
\pgfsetstrokecolor{currentstroke}%
\pgfsetdash{}{0pt}%
\pgfpathmoveto{\pgfqpoint{3.085112in}{2.986698in}}%
\pgfpathlineto{\pgfqpoint{3.196223in}{2.986698in}}%
\pgfusepath{stroke}%
\end{pgfscope}%
\begin{pgfscope}%
\definecolor{textcolor}{rgb}{1.000000,1.000000,1.000000}%
\pgfsetstrokecolor{textcolor}%
\pgfsetfillcolor{textcolor}%
\pgftext[x=3.285112in,y=2.947809in,left,base]{\color{textcolor}\rmfamily\fontsize{8.000000}{9.600000}\selectfont Tg + Vehicle (638)}%
\end{pgfscope}%
\begin{pgfscope}%
\pgfsetroundcap%
\pgfsetroundjoin%
\pgfsetlinewidth{1.003750pt}%
\definecolor{currentstroke}{rgb}{1.000000,0.694118,0.250980}%
\pgfsetstrokecolor{currentstroke}%
\pgfsetdash{}{0pt}%
\pgfpathmoveto{\pgfqpoint{3.085112in}{2.820059in}}%
\pgfpathlineto{\pgfqpoint{3.196223in}{2.820059in}}%
\pgfusepath{stroke}%
\end{pgfscope}%
\begin{pgfscope}%
\definecolor{textcolor}{rgb}{1.000000,1.000000,1.000000}%
\pgfsetstrokecolor{textcolor}%
\pgfsetfillcolor{textcolor}%
\pgftext[x=3.285112in,y=2.781170in,left,base]{\color{textcolor}\rmfamily\fontsize{8.000000}{9.600000}\selectfont Tg + TAT-GluA2\textsubscript{3Y} (759)}%
\end{pgfscope}%
\begin{pgfscope}%
\pgfsetbuttcap%
\pgfsetroundjoin%
\definecolor{currentfill}{rgb}{0.150000,0.150000,0.150000}%
\pgfsetfillcolor{currentfill}%
\pgfsetlinewidth{1.003750pt}%
\definecolor{currentstroke}{rgb}{0.150000,0.150000,0.150000}%
\pgfsetstrokecolor{currentstroke}%
\pgfsetdash{}{0pt}%
\pgfsys@defobject{currentmarker}{\pgfqpoint{0.000000in}{0.000000in}}{\pgfqpoint{0.041667in}{0.000000in}}{%
\pgfpathmoveto{\pgfqpoint{0.000000in}{0.000000in}}%
\pgfpathlineto{\pgfqpoint{0.041667in}{0.000000in}}%
\pgfusepath{stroke,fill}%
}%
\begin{pgfscope}%
\pgfsys@transformshift{3.186623in}{0.528177in}%
\pgfsys@useobject{currentmarker}{}%
\end{pgfscope}%
\end{pgfscope}%
\begin{pgfscope}%
\definecolor{textcolor}{rgb}{0.150000,0.150000,0.150000}%
\pgfsetstrokecolor{textcolor}%
\pgfsetfillcolor{textcolor}%
\pgftext[x=3.089400in,y=0.528177in,right,]{\color{textcolor}\rmfamily\fontsize{10.000000}{12.000000}\selectfont \(\displaystyle 0.0\)}%
\end{pgfscope}%
\begin{pgfscope}%
\pgfsetbuttcap%
\pgfsetroundjoin%
\definecolor{currentfill}{rgb}{0.150000,0.150000,0.150000}%
\pgfsetfillcolor{currentfill}%
\pgfsetlinewidth{1.003750pt}%
\definecolor{currentstroke}{rgb}{0.150000,0.150000,0.150000}%
\pgfsetstrokecolor{currentstroke}%
\pgfsetdash{}{0pt}%
\pgfsys@defobject{currentmarker}{\pgfqpoint{0.000000in}{0.000000in}}{\pgfqpoint{0.041667in}{0.000000in}}{%
\pgfpathmoveto{\pgfqpoint{0.000000in}{0.000000in}}%
\pgfpathlineto{\pgfqpoint{0.041667in}{0.000000in}}%
\pgfusepath{stroke,fill}%
}%
\begin{pgfscope}%
\pgfsys@transformshift{3.186623in}{0.764313in}%
\pgfsys@useobject{currentmarker}{}%
\end{pgfscope}%
\end{pgfscope}%
\begin{pgfscope}%
\definecolor{textcolor}{rgb}{0.150000,0.150000,0.150000}%
\pgfsetstrokecolor{textcolor}%
\pgfsetfillcolor{textcolor}%
\pgftext[x=3.089400in,y=0.764313in,right,]{\color{textcolor}\rmfamily\fontsize{10.000000}{12.000000}\selectfont \(\displaystyle 0.5\)}%
\end{pgfscope}%
\begin{pgfscope}%
\pgfsetbuttcap%
\pgfsetroundjoin%
\definecolor{currentfill}{rgb}{0.150000,0.150000,0.150000}%
\pgfsetfillcolor{currentfill}%
\pgfsetlinewidth{1.003750pt}%
\definecolor{currentstroke}{rgb}{0.150000,0.150000,0.150000}%
\pgfsetstrokecolor{currentstroke}%
\pgfsetdash{}{0pt}%
\pgfsys@defobject{currentmarker}{\pgfqpoint{0.000000in}{0.000000in}}{\pgfqpoint{0.041667in}{0.000000in}}{%
\pgfpathmoveto{\pgfqpoint{0.000000in}{0.000000in}}%
\pgfpathlineto{\pgfqpoint{0.041667in}{0.000000in}}%
\pgfusepath{stroke,fill}%
}%
\begin{pgfscope}%
\pgfsys@transformshift{3.186623in}{1.000448in}%
\pgfsys@useobject{currentmarker}{}%
\end{pgfscope}%
\end{pgfscope}%
\begin{pgfscope}%
\definecolor{textcolor}{rgb}{0.150000,0.150000,0.150000}%
\pgfsetstrokecolor{textcolor}%
\pgfsetfillcolor{textcolor}%
\pgftext[x=3.089400in,y=1.000448in,right,]{\color{textcolor}\rmfamily\fontsize{10.000000}{12.000000}\selectfont \(\displaystyle 1.0\)}%
\end{pgfscope}%
\begin{pgfscope}%
\pgfsetbuttcap%
\pgfsetroundjoin%
\definecolor{currentfill}{rgb}{0.150000,0.150000,0.150000}%
\pgfsetfillcolor{currentfill}%
\pgfsetlinewidth{1.003750pt}%
\definecolor{currentstroke}{rgb}{0.150000,0.150000,0.150000}%
\pgfsetstrokecolor{currentstroke}%
\pgfsetdash{}{0pt}%
\pgfsys@defobject{currentmarker}{\pgfqpoint{0.000000in}{0.000000in}}{\pgfqpoint{0.041667in}{0.000000in}}{%
\pgfpathmoveto{\pgfqpoint{0.000000in}{0.000000in}}%
\pgfpathlineto{\pgfqpoint{0.041667in}{0.000000in}}%
\pgfusepath{stroke,fill}%
}%
\begin{pgfscope}%
\pgfsys@transformshift{3.186623in}{1.236584in}%
\pgfsys@useobject{currentmarker}{}%
\end{pgfscope}%
\end{pgfscope}%
\begin{pgfscope}%
\definecolor{textcolor}{rgb}{0.150000,0.150000,0.150000}%
\pgfsetstrokecolor{textcolor}%
\pgfsetfillcolor{textcolor}%
\pgftext[x=3.089400in,y=1.236584in,right,]{\color{textcolor}\rmfamily\fontsize{10.000000}{12.000000}\selectfont \(\displaystyle 1.5\)}%
\end{pgfscope}%
\begin{pgfscope}%
\pgfsetbuttcap%
\pgfsetroundjoin%
\definecolor{currentfill}{rgb}{0.150000,0.150000,0.150000}%
\pgfsetfillcolor{currentfill}%
\pgfsetlinewidth{1.003750pt}%
\definecolor{currentstroke}{rgb}{0.150000,0.150000,0.150000}%
\pgfsetstrokecolor{currentstroke}%
\pgfsetdash{}{0pt}%
\pgfsys@defobject{currentmarker}{\pgfqpoint{0.000000in}{0.000000in}}{\pgfqpoint{0.041667in}{0.000000in}}{%
\pgfpathmoveto{\pgfqpoint{0.000000in}{0.000000in}}%
\pgfpathlineto{\pgfqpoint{0.041667in}{0.000000in}}%
\pgfusepath{stroke,fill}%
}%
\begin{pgfscope}%
\pgfsys@transformshift{3.186623in}{1.472720in}%
\pgfsys@useobject{currentmarker}{}%
\end{pgfscope}%
\end{pgfscope}%
\begin{pgfscope}%
\definecolor{textcolor}{rgb}{0.150000,0.150000,0.150000}%
\pgfsetstrokecolor{textcolor}%
\pgfsetfillcolor{textcolor}%
\pgftext[x=3.089400in,y=1.472720in,right,]{\color{textcolor}\rmfamily\fontsize{10.000000}{12.000000}\selectfont \(\displaystyle 2.0\)}%
\end{pgfscope}%
\begin{pgfscope}%
\pgfsetbuttcap%
\pgfsetroundjoin%
\definecolor{currentfill}{rgb}{0.150000,0.150000,0.150000}%
\pgfsetfillcolor{currentfill}%
\pgfsetlinewidth{1.003750pt}%
\definecolor{currentstroke}{rgb}{0.150000,0.150000,0.150000}%
\pgfsetstrokecolor{currentstroke}%
\pgfsetdash{}{0pt}%
\pgfsys@defobject{currentmarker}{\pgfqpoint{0.000000in}{0.000000in}}{\pgfqpoint{0.041667in}{0.000000in}}{%
\pgfpathmoveto{\pgfqpoint{0.000000in}{0.000000in}}%
\pgfpathlineto{\pgfqpoint{0.041667in}{0.000000in}}%
\pgfusepath{stroke,fill}%
}%
\begin{pgfscope}%
\pgfsys@transformshift{3.186623in}{1.708856in}%
\pgfsys@useobject{currentmarker}{}%
\end{pgfscope}%
\end{pgfscope}%
\begin{pgfscope}%
\definecolor{textcolor}{rgb}{0.150000,0.150000,0.150000}%
\pgfsetstrokecolor{textcolor}%
\pgfsetfillcolor{textcolor}%
\pgftext[x=3.089400in,y=1.708856in,right,]{\color{textcolor}\rmfamily\fontsize{10.000000}{12.000000}\selectfont \(\displaystyle 2.5\)}%
\end{pgfscope}%
\begin{pgfscope}%
\pgfsetbuttcap%
\pgfsetroundjoin%
\definecolor{currentfill}{rgb}{0.150000,0.150000,0.150000}%
\pgfsetfillcolor{currentfill}%
\pgfsetlinewidth{1.003750pt}%
\definecolor{currentstroke}{rgb}{0.150000,0.150000,0.150000}%
\pgfsetstrokecolor{currentstroke}%
\pgfsetdash{}{0pt}%
\pgfsys@defobject{currentmarker}{\pgfqpoint{0.000000in}{0.000000in}}{\pgfqpoint{0.041667in}{0.000000in}}{%
\pgfpathmoveto{\pgfqpoint{0.000000in}{0.000000in}}%
\pgfpathlineto{\pgfqpoint{0.041667in}{0.000000in}}%
\pgfusepath{stroke,fill}%
}%
\begin{pgfscope}%
\pgfsys@transformshift{3.186623in}{1.944991in}%
\pgfsys@useobject{currentmarker}{}%
\end{pgfscope}%
\end{pgfscope}%
\begin{pgfscope}%
\definecolor{textcolor}{rgb}{0.150000,0.150000,0.150000}%
\pgfsetstrokecolor{textcolor}%
\pgfsetfillcolor{textcolor}%
\pgftext[x=3.089400in,y=1.944991in,right,]{\color{textcolor}\rmfamily\fontsize{10.000000}{12.000000}\selectfont \(\displaystyle 3.0\)}%
\end{pgfscope}%
\begin{pgfscope}%
\pgfsetbuttcap%
\pgfsetroundjoin%
\definecolor{currentfill}{rgb}{0.150000,0.150000,0.150000}%
\pgfsetfillcolor{currentfill}%
\pgfsetlinewidth{1.003750pt}%
\definecolor{currentstroke}{rgb}{0.150000,0.150000,0.150000}%
\pgfsetstrokecolor{currentstroke}%
\pgfsetdash{}{0pt}%
\pgfsys@defobject{currentmarker}{\pgfqpoint{0.000000in}{0.000000in}}{\pgfqpoint{0.041667in}{0.000000in}}{%
\pgfpathmoveto{\pgfqpoint{0.000000in}{0.000000in}}%
\pgfpathlineto{\pgfqpoint{0.041667in}{0.000000in}}%
\pgfusepath{stroke,fill}%
}%
\begin{pgfscope}%
\pgfsys@transformshift{3.186623in}{2.181127in}%
\pgfsys@useobject{currentmarker}{}%
\end{pgfscope}%
\end{pgfscope}%
\begin{pgfscope}%
\definecolor{textcolor}{rgb}{0.150000,0.150000,0.150000}%
\pgfsetstrokecolor{textcolor}%
\pgfsetfillcolor{textcolor}%
\pgftext[x=3.089400in,y=2.181127in,right,]{\color{textcolor}\rmfamily\fontsize{10.000000}{12.000000}\selectfont \(\displaystyle 3.5\)}%
\end{pgfscope}%
\begin{pgfscope}%
\pgfsetbuttcap%
\pgfsetroundjoin%
\definecolor{currentfill}{rgb}{0.150000,0.150000,0.150000}%
\pgfsetfillcolor{currentfill}%
\pgfsetlinewidth{1.003750pt}%
\definecolor{currentstroke}{rgb}{0.150000,0.150000,0.150000}%
\pgfsetstrokecolor{currentstroke}%
\pgfsetdash{}{0pt}%
\pgfsys@defobject{currentmarker}{\pgfqpoint{0.000000in}{0.000000in}}{\pgfqpoint{0.041667in}{0.000000in}}{%
\pgfpathmoveto{\pgfqpoint{0.000000in}{0.000000in}}%
\pgfpathlineto{\pgfqpoint{0.041667in}{0.000000in}}%
\pgfusepath{stroke,fill}%
}%
\begin{pgfscope}%
\pgfsys@transformshift{3.186623in}{2.417263in}%
\pgfsys@useobject{currentmarker}{}%
\end{pgfscope}%
\end{pgfscope}%
\begin{pgfscope}%
\definecolor{textcolor}{rgb}{0.150000,0.150000,0.150000}%
\pgfsetstrokecolor{textcolor}%
\pgfsetfillcolor{textcolor}%
\pgftext[x=3.089400in,y=2.417263in,right,]{\color{textcolor}\rmfamily\fontsize{10.000000}{12.000000}\selectfont \(\displaystyle 4.0\)}%
\end{pgfscope}%
\begin{pgfscope}%
\pgfsetbuttcap%
\pgfsetroundjoin%
\definecolor{currentfill}{rgb}{0.150000,0.150000,0.150000}%
\pgfsetfillcolor{currentfill}%
\pgfsetlinewidth{1.003750pt}%
\definecolor{currentstroke}{rgb}{0.150000,0.150000,0.150000}%
\pgfsetstrokecolor{currentstroke}%
\pgfsetdash{}{0pt}%
\pgfsys@defobject{currentmarker}{\pgfqpoint{0.000000in}{0.000000in}}{\pgfqpoint{0.041667in}{0.000000in}}{%
\pgfpathmoveto{\pgfqpoint{0.000000in}{0.000000in}}%
\pgfpathlineto{\pgfqpoint{0.041667in}{0.000000in}}%
\pgfusepath{stroke,fill}%
}%
\begin{pgfscope}%
\pgfsys@transformshift{3.186623in}{2.653399in}%
\pgfsys@useobject{currentmarker}{}%
\end{pgfscope}%
\end{pgfscope}%
\begin{pgfscope}%
\definecolor{textcolor}{rgb}{0.150000,0.150000,0.150000}%
\pgfsetstrokecolor{textcolor}%
\pgfsetfillcolor{textcolor}%
\pgftext[x=3.089400in,y=2.653399in,right,]{\color{textcolor}\rmfamily\fontsize{10.000000}{12.000000}\selectfont \(\displaystyle 4.5\)}%
\end{pgfscope}%
\begin{pgfscope}%
\definecolor{textcolor}{rgb}{0.150000,0.150000,0.150000}%
\pgfsetstrokecolor{textcolor}%
\pgfsetfillcolor{textcolor}%
\pgftext[x=2.842486in,y=1.590788in,,bottom,rotate=90.000000]{\color{textcolor}\rmfamily\fontsize{10.000000}{12.000000}\selectfont \textbf{Cell activity (not freezing, a.u.)}}%
\end{pgfscope}%
\begin{pgfscope}%
\pgfpathrectangle{\pgfqpoint{3.186623in}{0.528177in}}{\pgfqpoint{2.015106in}{2.125222in}} %
\pgfusepath{clip}%
\pgfsetbuttcap%
\pgfsetmiterjoin%
\definecolor{currentfill}{rgb}{0.200000,0.427451,0.650980}%
\pgfsetfillcolor{currentfill}%
\pgfsetlinewidth{1.505625pt}%
\definecolor{currentstroke}{rgb}{0.200000,0.427451,0.650980}%
\pgfsetstrokecolor{currentstroke}%
\pgfsetdash{}{0pt}%
\pgfpathmoveto{\pgfqpoint{3.258591in}{0.528177in}}%
\pgfpathlineto{\pgfqpoint{3.618431in}{0.528177in}}%
\pgfpathlineto{\pgfqpoint{3.618431in}{2.248845in}}%
\pgfpathlineto{\pgfqpoint{3.258591in}{2.248845in}}%
\pgfpathclose%
\pgfusepath{stroke,fill}%
\end{pgfscope}%
\begin{pgfscope}%
\pgfpathrectangle{\pgfqpoint{3.186623in}{0.528177in}}{\pgfqpoint{2.015106in}{2.125222in}} %
\pgfusepath{clip}%
\pgfsetbuttcap%
\pgfsetmiterjoin%
\definecolor{currentfill}{rgb}{0.168627,0.670588,0.494118}%
\pgfsetfillcolor{currentfill}%
\pgfsetlinewidth{1.505625pt}%
\definecolor{currentstroke}{rgb}{0.168627,0.670588,0.494118}%
\pgfsetstrokecolor{currentstroke}%
\pgfsetdash{}{0pt}%
\pgfpathmoveto{\pgfqpoint{3.762367in}{0.528177in}}%
\pgfpathlineto{\pgfqpoint{4.122208in}{0.528177in}}%
\pgfpathlineto{\pgfqpoint{4.122208in}{2.211001in}}%
\pgfpathlineto{\pgfqpoint{3.762367in}{2.211001in}}%
\pgfpathclose%
\pgfusepath{stroke,fill}%
\end{pgfscope}%
\begin{pgfscope}%
\pgfpathrectangle{\pgfqpoint{3.186623in}{0.528177in}}{\pgfqpoint{2.015106in}{2.125222in}} %
\pgfusepath{clip}%
\pgfsetbuttcap%
\pgfsetmiterjoin%
\definecolor{currentfill}{rgb}{1.000000,0.494118,0.250980}%
\pgfsetfillcolor{currentfill}%
\pgfsetlinewidth{1.505625pt}%
\definecolor{currentstroke}{rgb}{1.000000,0.494118,0.250980}%
\pgfsetstrokecolor{currentstroke}%
\pgfsetdash{}{0pt}%
\pgfpathmoveto{\pgfqpoint{4.266144in}{0.528177in}}%
\pgfpathlineto{\pgfqpoint{4.625984in}{0.528177in}}%
\pgfpathlineto{\pgfqpoint{4.625984in}{2.226502in}}%
\pgfpathlineto{\pgfqpoint{4.266144in}{2.226502in}}%
\pgfpathclose%
\pgfusepath{stroke,fill}%
\end{pgfscope}%
\begin{pgfscope}%
\pgfpathrectangle{\pgfqpoint{3.186623in}{0.528177in}}{\pgfqpoint{2.015106in}{2.125222in}} %
\pgfusepath{clip}%
\pgfsetbuttcap%
\pgfsetmiterjoin%
\definecolor{currentfill}{rgb}{1.000000,0.694118,0.250980}%
\pgfsetfillcolor{currentfill}%
\pgfsetlinewidth{1.505625pt}%
\definecolor{currentstroke}{rgb}{1.000000,0.694118,0.250980}%
\pgfsetstrokecolor{currentstroke}%
\pgfsetdash{}{0pt}%
\pgfpathmoveto{\pgfqpoint{4.769920in}{0.528177in}}%
\pgfpathlineto{\pgfqpoint{5.129761in}{0.528177in}}%
\pgfpathlineto{\pgfqpoint{5.129761in}{1.740963in}}%
\pgfpathlineto{\pgfqpoint{4.769920in}{1.740963in}}%
\pgfpathclose%
\pgfusepath{stroke,fill}%
\end{pgfscope}%
\begin{pgfscope}%
\pgfpathrectangle{\pgfqpoint{3.186623in}{0.528177in}}{\pgfqpoint{2.015106in}{2.125222in}} %
\pgfusepath{clip}%
\pgfsetbuttcap%
\pgfsetroundjoin%
\pgfsetlinewidth{1.505625pt}%
\definecolor{currentstroke}{rgb}{0.200000,0.427451,0.650980}%
\pgfsetstrokecolor{currentstroke}%
\pgfsetdash{}{0pt}%
\pgfpathmoveto{\pgfqpoint{3.438511in}{2.248845in}}%
\pgfpathlineto{\pgfqpoint{3.438511in}{2.296498in}}%
\pgfusepath{stroke}%
\end{pgfscope}%
\begin{pgfscope}%
\pgfpathrectangle{\pgfqpoint{3.186623in}{0.528177in}}{\pgfqpoint{2.015106in}{2.125222in}} %
\pgfusepath{clip}%
\pgfsetbuttcap%
\pgfsetroundjoin%
\pgfsetlinewidth{1.505625pt}%
\definecolor{currentstroke}{rgb}{0.168627,0.670588,0.494118}%
\pgfsetstrokecolor{currentstroke}%
\pgfsetdash{}{0pt}%
\pgfpathmoveto{\pgfqpoint{3.942287in}{2.211001in}}%
\pgfpathlineto{\pgfqpoint{3.942287in}{2.289657in}}%
\pgfusepath{stroke}%
\end{pgfscope}%
\begin{pgfscope}%
\pgfpathrectangle{\pgfqpoint{3.186623in}{0.528177in}}{\pgfqpoint{2.015106in}{2.125222in}} %
\pgfusepath{clip}%
\pgfsetbuttcap%
\pgfsetroundjoin%
\pgfsetlinewidth{1.505625pt}%
\definecolor{currentstroke}{rgb}{1.000000,0.494118,0.250980}%
\pgfsetstrokecolor{currentstroke}%
\pgfsetdash{}{0pt}%
\pgfpathmoveto{\pgfqpoint{4.446064in}{2.226502in}}%
\pgfpathlineto{\pgfqpoint{4.446064in}{2.287160in}}%
\pgfusepath{stroke}%
\end{pgfscope}%
\begin{pgfscope}%
\pgfpathrectangle{\pgfqpoint{3.186623in}{0.528177in}}{\pgfqpoint{2.015106in}{2.125222in}} %
\pgfusepath{clip}%
\pgfsetbuttcap%
\pgfsetroundjoin%
\pgfsetlinewidth{1.505625pt}%
\definecolor{currentstroke}{rgb}{1.000000,0.694118,0.250980}%
\pgfsetstrokecolor{currentstroke}%
\pgfsetdash{}{0pt}%
\pgfpathmoveto{\pgfqpoint{4.949840in}{1.740963in}}%
\pgfpathlineto{\pgfqpoint{4.949840in}{1.784519in}}%
\pgfusepath{stroke}%
\end{pgfscope}%
\begin{pgfscope}%
\pgfpathrectangle{\pgfqpoint{3.186623in}{0.528177in}}{\pgfqpoint{2.015106in}{2.125222in}} %
\pgfusepath{clip}%
\pgfsetbuttcap%
\pgfsetroundjoin%
\definecolor{currentfill}{rgb}{0.200000,0.427451,0.650980}%
\pgfsetfillcolor{currentfill}%
\pgfsetlinewidth{1.505625pt}%
\definecolor{currentstroke}{rgb}{0.200000,0.427451,0.650980}%
\pgfsetstrokecolor{currentstroke}%
\pgfsetdash{}{0pt}%
\pgfsys@defobject{currentmarker}{\pgfqpoint{-0.111111in}{-0.000000in}}{\pgfqpoint{0.111111in}{0.000000in}}{%
\pgfpathmoveto{\pgfqpoint{0.111111in}{-0.000000in}}%
\pgfpathlineto{\pgfqpoint{-0.111111in}{0.000000in}}%
\pgfusepath{stroke,fill}%
}%
\begin{pgfscope}%
\pgfsys@transformshift{3.438511in}{2.248845in}%
\pgfsys@useobject{currentmarker}{}%
\end{pgfscope}%
\end{pgfscope}%
\begin{pgfscope}%
\pgfpathrectangle{\pgfqpoint{3.186623in}{0.528177in}}{\pgfqpoint{2.015106in}{2.125222in}} %
\pgfusepath{clip}%
\pgfsetbuttcap%
\pgfsetroundjoin%
\definecolor{currentfill}{rgb}{0.200000,0.427451,0.650980}%
\pgfsetfillcolor{currentfill}%
\pgfsetlinewidth{1.505625pt}%
\definecolor{currentstroke}{rgb}{0.200000,0.427451,0.650980}%
\pgfsetstrokecolor{currentstroke}%
\pgfsetdash{}{0pt}%
\pgfsys@defobject{currentmarker}{\pgfqpoint{-0.111111in}{-0.000000in}}{\pgfqpoint{0.111111in}{0.000000in}}{%
\pgfpathmoveto{\pgfqpoint{0.111111in}{-0.000000in}}%
\pgfpathlineto{\pgfqpoint{-0.111111in}{0.000000in}}%
\pgfusepath{stroke,fill}%
}%
\begin{pgfscope}%
\pgfsys@transformshift{3.438511in}{2.296498in}%
\pgfsys@useobject{currentmarker}{}%
\end{pgfscope}%
\end{pgfscope}%
\begin{pgfscope}%
\pgfpathrectangle{\pgfqpoint{3.186623in}{0.528177in}}{\pgfqpoint{2.015106in}{2.125222in}} %
\pgfusepath{clip}%
\pgfsetbuttcap%
\pgfsetroundjoin%
\definecolor{currentfill}{rgb}{0.168627,0.670588,0.494118}%
\pgfsetfillcolor{currentfill}%
\pgfsetlinewidth{1.505625pt}%
\definecolor{currentstroke}{rgb}{0.168627,0.670588,0.494118}%
\pgfsetstrokecolor{currentstroke}%
\pgfsetdash{}{0pt}%
\pgfsys@defobject{currentmarker}{\pgfqpoint{-0.111111in}{-0.000000in}}{\pgfqpoint{0.111111in}{0.000000in}}{%
\pgfpathmoveto{\pgfqpoint{0.111111in}{-0.000000in}}%
\pgfpathlineto{\pgfqpoint{-0.111111in}{0.000000in}}%
\pgfusepath{stroke,fill}%
}%
\begin{pgfscope}%
\pgfsys@transformshift{3.942287in}{2.211001in}%
\pgfsys@useobject{currentmarker}{}%
\end{pgfscope}%
\end{pgfscope}%
\begin{pgfscope}%
\pgfpathrectangle{\pgfqpoint{3.186623in}{0.528177in}}{\pgfqpoint{2.015106in}{2.125222in}} %
\pgfusepath{clip}%
\pgfsetbuttcap%
\pgfsetroundjoin%
\definecolor{currentfill}{rgb}{0.168627,0.670588,0.494118}%
\pgfsetfillcolor{currentfill}%
\pgfsetlinewidth{1.505625pt}%
\definecolor{currentstroke}{rgb}{0.168627,0.670588,0.494118}%
\pgfsetstrokecolor{currentstroke}%
\pgfsetdash{}{0pt}%
\pgfsys@defobject{currentmarker}{\pgfqpoint{-0.111111in}{-0.000000in}}{\pgfqpoint{0.111111in}{0.000000in}}{%
\pgfpathmoveto{\pgfqpoint{0.111111in}{-0.000000in}}%
\pgfpathlineto{\pgfqpoint{-0.111111in}{0.000000in}}%
\pgfusepath{stroke,fill}%
}%
\begin{pgfscope}%
\pgfsys@transformshift{3.942287in}{2.289657in}%
\pgfsys@useobject{currentmarker}{}%
\end{pgfscope}%
\end{pgfscope}%
\begin{pgfscope}%
\pgfpathrectangle{\pgfqpoint{3.186623in}{0.528177in}}{\pgfqpoint{2.015106in}{2.125222in}} %
\pgfusepath{clip}%
\pgfsetbuttcap%
\pgfsetroundjoin%
\definecolor{currentfill}{rgb}{1.000000,0.494118,0.250980}%
\pgfsetfillcolor{currentfill}%
\pgfsetlinewidth{1.505625pt}%
\definecolor{currentstroke}{rgb}{1.000000,0.494118,0.250980}%
\pgfsetstrokecolor{currentstroke}%
\pgfsetdash{}{0pt}%
\pgfsys@defobject{currentmarker}{\pgfqpoint{-0.111111in}{-0.000000in}}{\pgfqpoint{0.111111in}{0.000000in}}{%
\pgfpathmoveto{\pgfqpoint{0.111111in}{-0.000000in}}%
\pgfpathlineto{\pgfqpoint{-0.111111in}{0.000000in}}%
\pgfusepath{stroke,fill}%
}%
\begin{pgfscope}%
\pgfsys@transformshift{4.446064in}{2.226502in}%
\pgfsys@useobject{currentmarker}{}%
\end{pgfscope}%
\end{pgfscope}%
\begin{pgfscope}%
\pgfpathrectangle{\pgfqpoint{3.186623in}{0.528177in}}{\pgfqpoint{2.015106in}{2.125222in}} %
\pgfusepath{clip}%
\pgfsetbuttcap%
\pgfsetroundjoin%
\definecolor{currentfill}{rgb}{1.000000,0.494118,0.250980}%
\pgfsetfillcolor{currentfill}%
\pgfsetlinewidth{1.505625pt}%
\definecolor{currentstroke}{rgb}{1.000000,0.494118,0.250980}%
\pgfsetstrokecolor{currentstroke}%
\pgfsetdash{}{0pt}%
\pgfsys@defobject{currentmarker}{\pgfqpoint{-0.111111in}{-0.000000in}}{\pgfqpoint{0.111111in}{0.000000in}}{%
\pgfpathmoveto{\pgfqpoint{0.111111in}{-0.000000in}}%
\pgfpathlineto{\pgfqpoint{-0.111111in}{0.000000in}}%
\pgfusepath{stroke,fill}%
}%
\begin{pgfscope}%
\pgfsys@transformshift{4.446064in}{2.287160in}%
\pgfsys@useobject{currentmarker}{}%
\end{pgfscope}%
\end{pgfscope}%
\begin{pgfscope}%
\pgfpathrectangle{\pgfqpoint{3.186623in}{0.528177in}}{\pgfqpoint{2.015106in}{2.125222in}} %
\pgfusepath{clip}%
\pgfsetbuttcap%
\pgfsetroundjoin%
\definecolor{currentfill}{rgb}{1.000000,0.694118,0.250980}%
\pgfsetfillcolor{currentfill}%
\pgfsetlinewidth{1.505625pt}%
\definecolor{currentstroke}{rgb}{1.000000,0.694118,0.250980}%
\pgfsetstrokecolor{currentstroke}%
\pgfsetdash{}{0pt}%
\pgfsys@defobject{currentmarker}{\pgfqpoint{-0.111111in}{-0.000000in}}{\pgfqpoint{0.111111in}{0.000000in}}{%
\pgfpathmoveto{\pgfqpoint{0.111111in}{-0.000000in}}%
\pgfpathlineto{\pgfqpoint{-0.111111in}{0.000000in}}%
\pgfusepath{stroke,fill}%
}%
\begin{pgfscope}%
\pgfsys@transformshift{4.949840in}{1.740963in}%
\pgfsys@useobject{currentmarker}{}%
\end{pgfscope}%
\end{pgfscope}%
\begin{pgfscope}%
\pgfpathrectangle{\pgfqpoint{3.186623in}{0.528177in}}{\pgfqpoint{2.015106in}{2.125222in}} %
\pgfusepath{clip}%
\pgfsetbuttcap%
\pgfsetroundjoin%
\definecolor{currentfill}{rgb}{1.000000,0.694118,0.250980}%
\pgfsetfillcolor{currentfill}%
\pgfsetlinewidth{1.505625pt}%
\definecolor{currentstroke}{rgb}{1.000000,0.694118,0.250980}%
\pgfsetstrokecolor{currentstroke}%
\pgfsetdash{}{0pt}%
\pgfsys@defobject{currentmarker}{\pgfqpoint{-0.111111in}{-0.000000in}}{\pgfqpoint{0.111111in}{0.000000in}}{%
\pgfpathmoveto{\pgfqpoint{0.111111in}{-0.000000in}}%
\pgfpathlineto{\pgfqpoint{-0.111111in}{0.000000in}}%
\pgfusepath{stroke,fill}%
}%
\begin{pgfscope}%
\pgfsys@transformshift{4.949840in}{1.784519in}%
\pgfsys@useobject{currentmarker}{}%
\end{pgfscope}%
\end{pgfscope}%
\begin{pgfscope}%
\pgfpathrectangle{\pgfqpoint{3.186623in}{0.528177in}}{\pgfqpoint{2.015106in}{2.125222in}} %
\pgfusepath{clip}%
\pgfsetroundcap%
\pgfsetroundjoin%
\pgfsetlinewidth{1.756562pt}%
\definecolor{currentstroke}{rgb}{0.627451,0.627451,0.643137}%
\pgfsetstrokecolor{currentstroke}%
\pgfsetdash{}{0pt}%
\pgfpathmoveto{\pgfqpoint{4.446064in}{2.379753in}}%
\pgfpathlineto{\pgfqpoint{4.446064in}{2.534074in}}%
\pgfusepath{stroke}%
\end{pgfscope}%
\begin{pgfscope}%
\pgfpathrectangle{\pgfqpoint{3.186623in}{0.528177in}}{\pgfqpoint{2.015106in}{2.125222in}} %
\pgfusepath{clip}%
\pgfsetroundcap%
\pgfsetroundjoin%
\pgfsetlinewidth{1.756562pt}%
\definecolor{currentstroke}{rgb}{0.627451,0.627451,0.643137}%
\pgfsetstrokecolor{currentstroke}%
\pgfsetdash{}{0pt}%
\pgfpathmoveto{\pgfqpoint{4.446064in}{2.534074in}}%
\pgfpathlineto{\pgfqpoint{4.949840in}{2.534074in}}%
\pgfusepath{stroke}%
\end{pgfscope}%
\begin{pgfscope}%
\pgfpathrectangle{\pgfqpoint{3.186623in}{0.528177in}}{\pgfqpoint{2.015106in}{2.125222in}} %
\pgfusepath{clip}%
\pgfsetroundcap%
\pgfsetroundjoin%
\pgfsetlinewidth{1.756562pt}%
\definecolor{currentstroke}{rgb}{0.627451,0.627451,0.643137}%
\pgfsetstrokecolor{currentstroke}%
\pgfsetdash{}{0pt}%
\pgfpathmoveto{\pgfqpoint{4.949840in}{2.534074in}}%
\pgfpathlineto{\pgfqpoint{4.949840in}{1.969704in}}%
\pgfusepath{stroke}%
\end{pgfscope}%
\begin{pgfscope}%
\pgfsetrectcap%
\pgfsetmiterjoin%
\pgfsetlinewidth{1.254687pt}%
\definecolor{currentstroke}{rgb}{0.150000,0.150000,0.150000}%
\pgfsetstrokecolor{currentstroke}%
\pgfsetdash{}{0pt}%
\pgfpathmoveto{\pgfqpoint{3.186623in}{0.528177in}}%
\pgfpathlineto{\pgfqpoint{3.186623in}{2.653399in}}%
\pgfusepath{stroke}%
\end{pgfscope}%
\begin{pgfscope}%
\pgfsetrectcap%
\pgfsetmiterjoin%
\pgfsetlinewidth{1.254687pt}%
\definecolor{currentstroke}{rgb}{0.150000,0.150000,0.150000}%
\pgfsetstrokecolor{currentstroke}%
\pgfsetdash{}{0pt}%
\pgfpathmoveto{\pgfqpoint{3.186623in}{0.528177in}}%
\pgfpathlineto{\pgfqpoint{5.201729in}{0.528177in}}%
\pgfusepath{stroke}%
\end{pgfscope}%
\begin{pgfscope}%
\definecolor{textcolor}{rgb}{0.150000,0.150000,0.150000}%
\pgfsetstrokecolor{textcolor}%
\pgfsetfillcolor{textcolor}%
\pgftext[x=4.949840in,y=1.842390in,,]{\color{textcolor}\rmfamily\fontsize{15.000000}{18.000000}\selectfont \textbf{*}}%
\end{pgfscope}%
\begin{pgfscope}%
\pgfsetbuttcap%
\pgfsetmiterjoin%
\definecolor{currentfill}{rgb}{0.200000,0.427451,0.650980}%
\pgfsetfillcolor{currentfill}%
\pgfsetlinewidth{1.505625pt}%
\definecolor{currentstroke}{rgb}{0.200000,0.427451,0.650980}%
\pgfsetstrokecolor{currentstroke}%
\pgfsetdash{}{0pt}%
\pgfpathmoveto{\pgfqpoint{3.286623in}{3.281088in}}%
\pgfpathlineto{\pgfqpoint{3.397734in}{3.281088in}}%
\pgfpathlineto{\pgfqpoint{3.397734in}{3.358866in}}%
\pgfpathlineto{\pgfqpoint{3.286623in}{3.358866in}}%
\pgfpathclose%
\pgfusepath{stroke,fill}%
\end{pgfscope}%
\begin{pgfscope}%
\definecolor{textcolor}{rgb}{0.150000,0.150000,0.150000}%
\pgfsetstrokecolor{textcolor}%
\pgfsetfillcolor{textcolor}%
\pgftext[x=3.486623in,y=3.281088in,left,base]{\color{textcolor}\rmfamily\fontsize{8.000000}{9.600000}\selectfont WT + Vehicle (1129)}%
\end{pgfscope}%
\begin{pgfscope}%
\pgfsetbuttcap%
\pgfsetmiterjoin%
\definecolor{currentfill}{rgb}{0.168627,0.670588,0.494118}%
\pgfsetfillcolor{currentfill}%
\pgfsetlinewidth{1.505625pt}%
\definecolor{currentstroke}{rgb}{0.168627,0.670588,0.494118}%
\pgfsetstrokecolor{currentstroke}%
\pgfsetdash{}{0pt}%
\pgfpathmoveto{\pgfqpoint{3.286623in}{3.114449in}}%
\pgfpathlineto{\pgfqpoint{3.397734in}{3.114449in}}%
\pgfpathlineto{\pgfqpoint{3.397734in}{3.192227in}}%
\pgfpathlineto{\pgfqpoint{3.286623in}{3.192227in}}%
\pgfpathclose%
\pgfusepath{stroke,fill}%
\end{pgfscope}%
\begin{pgfscope}%
\definecolor{textcolor}{rgb}{0.150000,0.150000,0.150000}%
\pgfsetstrokecolor{textcolor}%
\pgfsetfillcolor{textcolor}%
\pgftext[x=3.486623in,y=3.114449in,left,base]{\color{textcolor}\rmfamily\fontsize{8.000000}{9.600000}\selectfont WT + TAT-GluA2\textsubscript{3Y} (512)}%
\end{pgfscope}%
\begin{pgfscope}%
\pgfsetbuttcap%
\pgfsetmiterjoin%
\definecolor{currentfill}{rgb}{1.000000,0.494118,0.250980}%
\pgfsetfillcolor{currentfill}%
\pgfsetlinewidth{1.505625pt}%
\definecolor{currentstroke}{rgb}{1.000000,0.494118,0.250980}%
\pgfsetstrokecolor{currentstroke}%
\pgfsetdash{}{0pt}%
\pgfpathmoveto{\pgfqpoint{3.286623in}{2.947809in}}%
\pgfpathlineto{\pgfqpoint{3.397734in}{2.947809in}}%
\pgfpathlineto{\pgfqpoint{3.397734in}{3.025587in}}%
\pgfpathlineto{\pgfqpoint{3.286623in}{3.025587in}}%
\pgfpathclose%
\pgfusepath{stroke,fill}%
\end{pgfscope}%
\begin{pgfscope}%
\definecolor{textcolor}{rgb}{0.150000,0.150000,0.150000}%
\pgfsetstrokecolor{textcolor}%
\pgfsetfillcolor{textcolor}%
\pgftext[x=3.486623in,y=2.947809in,left,base]{\color{textcolor}\rmfamily\fontsize{8.000000}{9.600000}\selectfont Tg + Vehicle (638)}%
\end{pgfscope}%
\begin{pgfscope}%
\pgfsetbuttcap%
\pgfsetmiterjoin%
\definecolor{currentfill}{rgb}{1.000000,0.694118,0.250980}%
\pgfsetfillcolor{currentfill}%
\pgfsetlinewidth{1.505625pt}%
\definecolor{currentstroke}{rgb}{1.000000,0.694118,0.250980}%
\pgfsetstrokecolor{currentstroke}%
\pgfsetdash{}{0pt}%
\pgfpathmoveto{\pgfqpoint{3.286623in}{2.781170in}}%
\pgfpathlineto{\pgfqpoint{3.397734in}{2.781170in}}%
\pgfpathlineto{\pgfqpoint{3.397734in}{2.858947in}}%
\pgfpathlineto{\pgfqpoint{3.286623in}{2.858947in}}%
\pgfpathclose%
\pgfusepath{stroke,fill}%
\end{pgfscope}%
\begin{pgfscope}%
\definecolor{textcolor}{rgb}{0.150000,0.150000,0.150000}%
\pgfsetstrokecolor{textcolor}%
\pgfsetfillcolor{textcolor}%
\pgftext[x=3.486623in,y=2.781170in,left,base]{\color{textcolor}\rmfamily\fontsize{8.000000}{9.600000}\selectfont Tg + TAT-GluA2\textsubscript{3Y} (759)}%
\end{pgfscope}%
\end{pgfpicture}%
\makeatother%
\endgroup%

        \caption{\label{f.ad.actnf}}
    \end{subfigure}
    \caption[Cell activity during freezing.]{Average cell activity when mice were \subref{f.ad.actf} freezing and \subref{f.ad.actnf} not freezing during the 5-min contextual memory test. Cells in \gls{tg} mice have significantly higher activity during freezing, and \tglu{} treatment decreases cell activity in both \gls{wt} and \gls{tg} groups. When the mice were not freezing, \tglu{} treatment decreases cell activity only in \gls{tg} mice, but has no effect in \gls{wt} mice. This result suggests that the effect of \tglu{} in \gls{tg} mice is a global decrease of cell activity. \label{f.ad.activity_freezing}}
\end{figure}

\subsection{\tglu{} rescues freezing encoding deficit in \gls{tg} cells}

Previous reports have suggested that \gls{ca1} neurons in \gls{tg} mice have deficits in encoding spatial information \citep{cheng13}. It is possible that such encoding deficits in \gls{ad} is a general phenomenon, and also affects contextual fear memory. Therefore we investigated how \gls{wt} and \gls{tg} cells encode freezing. 

First we examined cells individually, and calculated the mutual information between the calcium transients and freezing behaviour \citep{ross14, victor02}. Mutual information captures both linear and non-linear relationships between two random variables and reflects how much predictive power one variable has for another. Following the naming from \citet{skaggs93}, where the mutual information between cell activity and animal's position is dubbed ``spatial information'', here we refer to the mutual information between cell activity and mouse's freezing behaviour as ``freezing information''. 

The group differences in freezing information were compared using a two-way \gls{anova}. There was a significant \textit{Genotype} $\times$ \textit{Treatment} interaction (F\tsb{1,4380}=126.7, p<0.001), as well as main effects in both \textit{Genotype} (F\tsb{1,4380}=254.0, p<0.001) and \textit{Treatment} (F\tsb{1,4380}=54.7, p<0.001). \textit{Post hoc} tests showed the Tg-Veh group to have significantly less freezing information (WT-Veh vs Tg-Veh, T=19.3, p<0.001), and that this deficit was partially rescued by \tglu{} treatment (Tg-\glu{} vs Tg-Veh, T=13.2, p<0.001), as the Tg-\glu{} group had significantly less freezing information than WT-Veh group (WT-Veh vs Tg-\glu, T=6.0, p<0.001; Figure~\ref{f.ad.freeze_info}). WT-\glu{} had significantly less freezing information than WT-Veh, although the significance was close to threshold (WT-Veh vs WT-\glu, T=2.5, p=0.011, threshold=0.013). This result suggests that in the \gls{tg} group, individual cell activity in hippocampus \gls{ca1} is not a good predictor of the mouse's freezing behaviour. This is partially recovered by \tglu{} treatment. 
\begin{figure}[h]
    %% Creator: Matplotlib, PGF backend
%%
%% To include the figure in your LaTeX document, write
%%   \input{<filename>.pgf}
%%
%% Make sure the required packages are loaded in your preamble
%%   \usepackage{pgf}
%%
%% Figures using additional raster images can only be included by \input if
%% they are in the same directory as the main LaTeX file. For loading figures
%% from other directories you can use the `import` package
%%   \usepackage{import}
%% and then include the figures with
%%   \import{<path to file>}{<filename>.pgf}
%%
%% Matplotlib used the following preamble
%%   \usepackage[utf8]{inputenc}
%%   \usepackage[T1]{fontenc}
%%   \usepackage{siunitx}
%%
\begingroup%
\makeatletter%
\begin{pgfpicture}%
\pgfpathrectangle{\pgfpointorigin}{\pgfqpoint{5.301729in}{3.689896in}}%
\pgfusepath{use as bounding box, clip}%
\begin{pgfscope}%
\pgfsetbuttcap%
\pgfsetmiterjoin%
\definecolor{currentfill}{rgb}{1.000000,1.000000,1.000000}%
\pgfsetfillcolor{currentfill}%
\pgfsetlinewidth{0.000000pt}%
\definecolor{currentstroke}{rgb}{1.000000,1.000000,1.000000}%
\pgfsetstrokecolor{currentstroke}%
\pgfsetdash{}{0pt}%
\pgfpathmoveto{\pgfqpoint{0.000000in}{0.000000in}}%
\pgfpathlineto{\pgfqpoint{5.301729in}{0.000000in}}%
\pgfpathlineto{\pgfqpoint{5.301729in}{3.689896in}}%
\pgfpathlineto{\pgfqpoint{0.000000in}{3.689896in}}%
\pgfpathclose%
\pgfusepath{fill}%
\end{pgfscope}%
\begin{pgfscope}%
\pgfsetbuttcap%
\pgfsetmiterjoin%
\definecolor{currentfill}{rgb}{1.000000,1.000000,1.000000}%
\pgfsetfillcolor{currentfill}%
\pgfsetlinewidth{0.000000pt}%
\definecolor{currentstroke}{rgb}{0.000000,0.000000,0.000000}%
\pgfsetstrokecolor{currentstroke}%
\pgfsetstrokeopacity{0.000000}%
\pgfsetdash{}{0pt}%
\pgfpathmoveto{\pgfqpoint{0.566985in}{0.664139in}}%
\pgfpathlineto{\pgfqpoint{2.582091in}{0.664139in}}%
\pgfpathlineto{\pgfqpoint{2.582091in}{3.528569in}}%
\pgfpathlineto{\pgfqpoint{0.566985in}{3.528569in}}%
\pgfpathclose%
\pgfusepath{fill}%
\end{pgfscope}%
\begin{pgfscope}%
\pgfsetbuttcap%
\pgfsetroundjoin%
\definecolor{currentfill}{rgb}{0.150000,0.150000,0.150000}%
\pgfsetfillcolor{currentfill}%
\pgfsetlinewidth{1.003750pt}%
\definecolor{currentstroke}{rgb}{0.150000,0.150000,0.150000}%
\pgfsetstrokecolor{currentstroke}%
\pgfsetdash{}{0pt}%
\pgfsys@defobject{currentmarker}{\pgfqpoint{0.000000in}{0.000000in}}{\pgfqpoint{0.000000in}{0.041667in}}{%
\pgfpathmoveto{\pgfqpoint{0.000000in}{0.000000in}}%
\pgfpathlineto{\pgfqpoint{0.000000in}{0.041667in}}%
\pgfusepath{stroke,fill}%
}%
\begin{pgfscope}%
\pgfsys@transformshift{0.566985in}{0.664139in}%
\pgfsys@useobject{currentmarker}{}%
\end{pgfscope}%
\end{pgfscope}%
\begin{pgfscope}%
\definecolor{textcolor}{rgb}{0.150000,0.150000,0.150000}%
\pgfsetstrokecolor{textcolor}%
\pgfsetfillcolor{textcolor}%
\pgftext[x=0.566985in,y=0.566917in,,top]{\color{textcolor}\rmfamily\fontsize{10.000000}{12.000000}\selectfont \(\displaystyle 0.00\)}%
\end{pgfscope}%
\begin{pgfscope}%
\pgfsetbuttcap%
\pgfsetroundjoin%
\definecolor{currentfill}{rgb}{0.150000,0.150000,0.150000}%
\pgfsetfillcolor{currentfill}%
\pgfsetlinewidth{1.003750pt}%
\definecolor{currentstroke}{rgb}{0.150000,0.150000,0.150000}%
\pgfsetstrokecolor{currentstroke}%
\pgfsetdash{}{0pt}%
\pgfsys@defobject{currentmarker}{\pgfqpoint{0.000000in}{0.000000in}}{\pgfqpoint{0.000000in}{0.041667in}}{%
\pgfpathmoveto{\pgfqpoint{0.000000in}{0.000000in}}%
\pgfpathlineto{\pgfqpoint{0.000000in}{0.041667in}}%
\pgfusepath{stroke,fill}%
}%
\begin{pgfscope}%
\pgfsys@transformshift{0.970006in}{0.664139in}%
\pgfsys@useobject{currentmarker}{}%
\end{pgfscope}%
\end{pgfscope}%
\begin{pgfscope}%
\definecolor{textcolor}{rgb}{0.150000,0.150000,0.150000}%
\pgfsetstrokecolor{textcolor}%
\pgfsetfillcolor{textcolor}%
\pgftext[x=0.970006in,y=0.566917in,,top]{\color{textcolor}\rmfamily\fontsize{10.000000}{12.000000}\selectfont \(\displaystyle 0.05\)}%
\end{pgfscope}%
\begin{pgfscope}%
\pgfsetbuttcap%
\pgfsetroundjoin%
\definecolor{currentfill}{rgb}{0.150000,0.150000,0.150000}%
\pgfsetfillcolor{currentfill}%
\pgfsetlinewidth{1.003750pt}%
\definecolor{currentstroke}{rgb}{0.150000,0.150000,0.150000}%
\pgfsetstrokecolor{currentstroke}%
\pgfsetdash{}{0pt}%
\pgfsys@defobject{currentmarker}{\pgfqpoint{0.000000in}{0.000000in}}{\pgfqpoint{0.000000in}{0.041667in}}{%
\pgfpathmoveto{\pgfqpoint{0.000000in}{0.000000in}}%
\pgfpathlineto{\pgfqpoint{0.000000in}{0.041667in}}%
\pgfusepath{stroke,fill}%
}%
\begin{pgfscope}%
\pgfsys@transformshift{1.373027in}{0.664139in}%
\pgfsys@useobject{currentmarker}{}%
\end{pgfscope}%
\end{pgfscope}%
\begin{pgfscope}%
\definecolor{textcolor}{rgb}{0.150000,0.150000,0.150000}%
\pgfsetstrokecolor{textcolor}%
\pgfsetfillcolor{textcolor}%
\pgftext[x=1.373027in,y=0.566917in,,top]{\color{textcolor}\rmfamily\fontsize{10.000000}{12.000000}\selectfont \(\displaystyle 0.10\)}%
\end{pgfscope}%
\begin{pgfscope}%
\pgfsetbuttcap%
\pgfsetroundjoin%
\definecolor{currentfill}{rgb}{0.150000,0.150000,0.150000}%
\pgfsetfillcolor{currentfill}%
\pgfsetlinewidth{1.003750pt}%
\definecolor{currentstroke}{rgb}{0.150000,0.150000,0.150000}%
\pgfsetstrokecolor{currentstroke}%
\pgfsetdash{}{0pt}%
\pgfsys@defobject{currentmarker}{\pgfqpoint{0.000000in}{0.000000in}}{\pgfqpoint{0.000000in}{0.041667in}}{%
\pgfpathmoveto{\pgfqpoint{0.000000in}{0.000000in}}%
\pgfpathlineto{\pgfqpoint{0.000000in}{0.041667in}}%
\pgfusepath{stroke,fill}%
}%
\begin{pgfscope}%
\pgfsys@transformshift{1.776048in}{0.664139in}%
\pgfsys@useobject{currentmarker}{}%
\end{pgfscope}%
\end{pgfscope}%
\begin{pgfscope}%
\definecolor{textcolor}{rgb}{0.150000,0.150000,0.150000}%
\pgfsetstrokecolor{textcolor}%
\pgfsetfillcolor{textcolor}%
\pgftext[x=1.776048in,y=0.566917in,,top]{\color{textcolor}\rmfamily\fontsize{10.000000}{12.000000}\selectfont \(\displaystyle 0.15\)}%
\end{pgfscope}%
\begin{pgfscope}%
\pgfsetbuttcap%
\pgfsetroundjoin%
\definecolor{currentfill}{rgb}{0.150000,0.150000,0.150000}%
\pgfsetfillcolor{currentfill}%
\pgfsetlinewidth{1.003750pt}%
\definecolor{currentstroke}{rgb}{0.150000,0.150000,0.150000}%
\pgfsetstrokecolor{currentstroke}%
\pgfsetdash{}{0pt}%
\pgfsys@defobject{currentmarker}{\pgfqpoint{0.000000in}{0.000000in}}{\pgfqpoint{0.000000in}{0.041667in}}{%
\pgfpathmoveto{\pgfqpoint{0.000000in}{0.000000in}}%
\pgfpathlineto{\pgfqpoint{0.000000in}{0.041667in}}%
\pgfusepath{stroke,fill}%
}%
\begin{pgfscope}%
\pgfsys@transformshift{2.179070in}{0.664139in}%
\pgfsys@useobject{currentmarker}{}%
\end{pgfscope}%
\end{pgfscope}%
\begin{pgfscope}%
\definecolor{textcolor}{rgb}{0.150000,0.150000,0.150000}%
\pgfsetstrokecolor{textcolor}%
\pgfsetfillcolor{textcolor}%
\pgftext[x=2.179070in,y=0.566917in,,top]{\color{textcolor}\rmfamily\fontsize{10.000000}{12.000000}\selectfont \(\displaystyle 0.20\)}%
\end{pgfscope}%
\begin{pgfscope}%
\pgfsetbuttcap%
\pgfsetroundjoin%
\definecolor{currentfill}{rgb}{0.150000,0.150000,0.150000}%
\pgfsetfillcolor{currentfill}%
\pgfsetlinewidth{1.003750pt}%
\definecolor{currentstroke}{rgb}{0.150000,0.150000,0.150000}%
\pgfsetstrokecolor{currentstroke}%
\pgfsetdash{}{0pt}%
\pgfsys@defobject{currentmarker}{\pgfqpoint{0.000000in}{0.000000in}}{\pgfqpoint{0.000000in}{0.041667in}}{%
\pgfpathmoveto{\pgfqpoint{0.000000in}{0.000000in}}%
\pgfpathlineto{\pgfqpoint{0.000000in}{0.041667in}}%
\pgfusepath{stroke,fill}%
}%
\begin{pgfscope}%
\pgfsys@transformshift{2.582091in}{0.664139in}%
\pgfsys@useobject{currentmarker}{}%
\end{pgfscope}%
\end{pgfscope}%
\begin{pgfscope}%
\definecolor{textcolor}{rgb}{0.150000,0.150000,0.150000}%
\pgfsetstrokecolor{textcolor}%
\pgfsetfillcolor{textcolor}%
\pgftext[x=2.582091in,y=0.566917in,,top]{\color{textcolor}\rmfamily\fontsize{10.000000}{12.000000}\selectfont \(\displaystyle 0.25\)}%
\end{pgfscope}%
\begin{pgfscope}%
\definecolor{textcolor}{rgb}{0.150000,0.150000,0.150000}%
\pgfsetstrokecolor{textcolor}%
\pgfsetfillcolor{textcolor}%
\pgftext[x=0.849744in,y=0.286683in,left,base]{\color{textcolor}\rmfamily\fontsize{10.000000}{12.000000}\selectfont \textbf{Freezing information}}%
\end{pgfscope}%
\begin{pgfscope}%
\definecolor{textcolor}{rgb}{0.150000,0.150000,0.150000}%
\pgfsetstrokecolor{textcolor}%
\pgfsetfillcolor{textcolor}%
\pgftext[x=1.144222in,y=0.134714in,left,base]{\color{textcolor}\rmfamily\fontsize{10.000000}{12.000000}\selectfont \textbf{(bits/frame)}}%
\end{pgfscope}%
\begin{pgfscope}%
\pgfsetbuttcap%
\pgfsetroundjoin%
\definecolor{currentfill}{rgb}{0.150000,0.150000,0.150000}%
\pgfsetfillcolor{currentfill}%
\pgfsetlinewidth{1.003750pt}%
\definecolor{currentstroke}{rgb}{0.150000,0.150000,0.150000}%
\pgfsetstrokecolor{currentstroke}%
\pgfsetdash{}{0pt}%
\pgfsys@defobject{currentmarker}{\pgfqpoint{0.000000in}{0.000000in}}{\pgfqpoint{0.041667in}{0.000000in}}{%
\pgfpathmoveto{\pgfqpoint{0.000000in}{0.000000in}}%
\pgfpathlineto{\pgfqpoint{0.041667in}{0.000000in}}%
\pgfusepath{stroke,fill}%
}%
\begin{pgfscope}%
\pgfsys@transformshift{0.566985in}{0.664139in}%
\pgfsys@useobject{currentmarker}{}%
\end{pgfscope}%
\end{pgfscope}%
\begin{pgfscope}%
\definecolor{textcolor}{rgb}{0.150000,0.150000,0.150000}%
\pgfsetstrokecolor{textcolor}%
\pgfsetfillcolor{textcolor}%
\pgftext[x=0.469762in,y=0.664139in,right,]{\color{textcolor}\rmfamily\fontsize{10.000000}{12.000000}\selectfont \(\displaystyle 0.2\)}%
\end{pgfscope}%
\begin{pgfscope}%
\pgfsetbuttcap%
\pgfsetroundjoin%
\definecolor{currentfill}{rgb}{0.150000,0.150000,0.150000}%
\pgfsetfillcolor{currentfill}%
\pgfsetlinewidth{1.003750pt}%
\definecolor{currentstroke}{rgb}{0.150000,0.150000,0.150000}%
\pgfsetstrokecolor{currentstroke}%
\pgfsetdash{}{0pt}%
\pgfsys@defobject{currentmarker}{\pgfqpoint{0.000000in}{0.000000in}}{\pgfqpoint{0.041667in}{0.000000in}}{%
\pgfpathmoveto{\pgfqpoint{0.000000in}{0.000000in}}%
\pgfpathlineto{\pgfqpoint{0.041667in}{0.000000in}}%
\pgfusepath{stroke,fill}%
}%
\begin{pgfscope}%
\pgfsys@transformshift{0.566985in}{1.022193in}%
\pgfsys@useobject{currentmarker}{}%
\end{pgfscope}%
\end{pgfscope}%
\begin{pgfscope}%
\definecolor{textcolor}{rgb}{0.150000,0.150000,0.150000}%
\pgfsetstrokecolor{textcolor}%
\pgfsetfillcolor{textcolor}%
\pgftext[x=0.469762in,y=1.022193in,right,]{\color{textcolor}\rmfamily\fontsize{10.000000}{12.000000}\selectfont \(\displaystyle 0.3\)}%
\end{pgfscope}%
\begin{pgfscope}%
\pgfsetbuttcap%
\pgfsetroundjoin%
\definecolor{currentfill}{rgb}{0.150000,0.150000,0.150000}%
\pgfsetfillcolor{currentfill}%
\pgfsetlinewidth{1.003750pt}%
\definecolor{currentstroke}{rgb}{0.150000,0.150000,0.150000}%
\pgfsetstrokecolor{currentstroke}%
\pgfsetdash{}{0pt}%
\pgfsys@defobject{currentmarker}{\pgfqpoint{0.000000in}{0.000000in}}{\pgfqpoint{0.041667in}{0.000000in}}{%
\pgfpathmoveto{\pgfqpoint{0.000000in}{0.000000in}}%
\pgfpathlineto{\pgfqpoint{0.041667in}{0.000000in}}%
\pgfusepath{stroke,fill}%
}%
\begin{pgfscope}%
\pgfsys@transformshift{0.566985in}{1.380246in}%
\pgfsys@useobject{currentmarker}{}%
\end{pgfscope}%
\end{pgfscope}%
\begin{pgfscope}%
\definecolor{textcolor}{rgb}{0.150000,0.150000,0.150000}%
\pgfsetstrokecolor{textcolor}%
\pgfsetfillcolor{textcolor}%
\pgftext[x=0.469762in,y=1.380246in,right,]{\color{textcolor}\rmfamily\fontsize{10.000000}{12.000000}\selectfont \(\displaystyle 0.4\)}%
\end{pgfscope}%
\begin{pgfscope}%
\pgfsetbuttcap%
\pgfsetroundjoin%
\definecolor{currentfill}{rgb}{0.150000,0.150000,0.150000}%
\pgfsetfillcolor{currentfill}%
\pgfsetlinewidth{1.003750pt}%
\definecolor{currentstroke}{rgb}{0.150000,0.150000,0.150000}%
\pgfsetstrokecolor{currentstroke}%
\pgfsetdash{}{0pt}%
\pgfsys@defobject{currentmarker}{\pgfqpoint{0.000000in}{0.000000in}}{\pgfqpoint{0.041667in}{0.000000in}}{%
\pgfpathmoveto{\pgfqpoint{0.000000in}{0.000000in}}%
\pgfpathlineto{\pgfqpoint{0.041667in}{0.000000in}}%
\pgfusepath{stroke,fill}%
}%
\begin{pgfscope}%
\pgfsys@transformshift{0.566985in}{1.738300in}%
\pgfsys@useobject{currentmarker}{}%
\end{pgfscope}%
\end{pgfscope}%
\begin{pgfscope}%
\definecolor{textcolor}{rgb}{0.150000,0.150000,0.150000}%
\pgfsetstrokecolor{textcolor}%
\pgfsetfillcolor{textcolor}%
\pgftext[x=0.469762in,y=1.738300in,right,]{\color{textcolor}\rmfamily\fontsize{10.000000}{12.000000}\selectfont \(\displaystyle 0.5\)}%
\end{pgfscope}%
\begin{pgfscope}%
\pgfsetbuttcap%
\pgfsetroundjoin%
\definecolor{currentfill}{rgb}{0.150000,0.150000,0.150000}%
\pgfsetfillcolor{currentfill}%
\pgfsetlinewidth{1.003750pt}%
\definecolor{currentstroke}{rgb}{0.150000,0.150000,0.150000}%
\pgfsetstrokecolor{currentstroke}%
\pgfsetdash{}{0pt}%
\pgfsys@defobject{currentmarker}{\pgfqpoint{0.000000in}{0.000000in}}{\pgfqpoint{0.041667in}{0.000000in}}{%
\pgfpathmoveto{\pgfqpoint{0.000000in}{0.000000in}}%
\pgfpathlineto{\pgfqpoint{0.041667in}{0.000000in}}%
\pgfusepath{stroke,fill}%
}%
\begin{pgfscope}%
\pgfsys@transformshift{0.566985in}{2.096354in}%
\pgfsys@useobject{currentmarker}{}%
\end{pgfscope}%
\end{pgfscope}%
\begin{pgfscope}%
\definecolor{textcolor}{rgb}{0.150000,0.150000,0.150000}%
\pgfsetstrokecolor{textcolor}%
\pgfsetfillcolor{textcolor}%
\pgftext[x=0.469762in,y=2.096354in,right,]{\color{textcolor}\rmfamily\fontsize{10.000000}{12.000000}\selectfont \(\displaystyle 0.6\)}%
\end{pgfscope}%
\begin{pgfscope}%
\pgfsetbuttcap%
\pgfsetroundjoin%
\definecolor{currentfill}{rgb}{0.150000,0.150000,0.150000}%
\pgfsetfillcolor{currentfill}%
\pgfsetlinewidth{1.003750pt}%
\definecolor{currentstroke}{rgb}{0.150000,0.150000,0.150000}%
\pgfsetstrokecolor{currentstroke}%
\pgfsetdash{}{0pt}%
\pgfsys@defobject{currentmarker}{\pgfqpoint{0.000000in}{0.000000in}}{\pgfqpoint{0.041667in}{0.000000in}}{%
\pgfpathmoveto{\pgfqpoint{0.000000in}{0.000000in}}%
\pgfpathlineto{\pgfqpoint{0.041667in}{0.000000in}}%
\pgfusepath{stroke,fill}%
}%
\begin{pgfscope}%
\pgfsys@transformshift{0.566985in}{2.454407in}%
\pgfsys@useobject{currentmarker}{}%
\end{pgfscope}%
\end{pgfscope}%
\begin{pgfscope}%
\definecolor{textcolor}{rgb}{0.150000,0.150000,0.150000}%
\pgfsetstrokecolor{textcolor}%
\pgfsetfillcolor{textcolor}%
\pgftext[x=0.469762in,y=2.454407in,right,]{\color{textcolor}\rmfamily\fontsize{10.000000}{12.000000}\selectfont \(\displaystyle 0.7\)}%
\end{pgfscope}%
\begin{pgfscope}%
\pgfsetbuttcap%
\pgfsetroundjoin%
\definecolor{currentfill}{rgb}{0.150000,0.150000,0.150000}%
\pgfsetfillcolor{currentfill}%
\pgfsetlinewidth{1.003750pt}%
\definecolor{currentstroke}{rgb}{0.150000,0.150000,0.150000}%
\pgfsetstrokecolor{currentstroke}%
\pgfsetdash{}{0pt}%
\pgfsys@defobject{currentmarker}{\pgfqpoint{0.000000in}{0.000000in}}{\pgfqpoint{0.041667in}{0.000000in}}{%
\pgfpathmoveto{\pgfqpoint{0.000000in}{0.000000in}}%
\pgfpathlineto{\pgfqpoint{0.041667in}{0.000000in}}%
\pgfusepath{stroke,fill}%
}%
\begin{pgfscope}%
\pgfsys@transformshift{0.566985in}{2.812461in}%
\pgfsys@useobject{currentmarker}{}%
\end{pgfscope}%
\end{pgfscope}%
\begin{pgfscope}%
\definecolor{textcolor}{rgb}{0.150000,0.150000,0.150000}%
\pgfsetstrokecolor{textcolor}%
\pgfsetfillcolor{textcolor}%
\pgftext[x=0.469762in,y=2.812461in,right,]{\color{textcolor}\rmfamily\fontsize{10.000000}{12.000000}\selectfont \(\displaystyle 0.8\)}%
\end{pgfscope}%
\begin{pgfscope}%
\pgfsetbuttcap%
\pgfsetroundjoin%
\definecolor{currentfill}{rgb}{0.150000,0.150000,0.150000}%
\pgfsetfillcolor{currentfill}%
\pgfsetlinewidth{1.003750pt}%
\definecolor{currentstroke}{rgb}{0.150000,0.150000,0.150000}%
\pgfsetstrokecolor{currentstroke}%
\pgfsetdash{}{0pt}%
\pgfsys@defobject{currentmarker}{\pgfqpoint{0.000000in}{0.000000in}}{\pgfqpoint{0.041667in}{0.000000in}}{%
\pgfpathmoveto{\pgfqpoint{0.000000in}{0.000000in}}%
\pgfpathlineto{\pgfqpoint{0.041667in}{0.000000in}}%
\pgfusepath{stroke,fill}%
}%
\begin{pgfscope}%
\pgfsys@transformshift{0.566985in}{3.170515in}%
\pgfsys@useobject{currentmarker}{}%
\end{pgfscope}%
\end{pgfscope}%
\begin{pgfscope}%
\definecolor{textcolor}{rgb}{0.150000,0.150000,0.150000}%
\pgfsetstrokecolor{textcolor}%
\pgfsetfillcolor{textcolor}%
\pgftext[x=0.469762in,y=3.170515in,right,]{\color{textcolor}\rmfamily\fontsize{10.000000}{12.000000}\selectfont \(\displaystyle 0.9\)}%
\end{pgfscope}%
\begin{pgfscope}%
\pgfsetbuttcap%
\pgfsetroundjoin%
\definecolor{currentfill}{rgb}{0.150000,0.150000,0.150000}%
\pgfsetfillcolor{currentfill}%
\pgfsetlinewidth{1.003750pt}%
\definecolor{currentstroke}{rgb}{0.150000,0.150000,0.150000}%
\pgfsetstrokecolor{currentstroke}%
\pgfsetdash{}{0pt}%
\pgfsys@defobject{currentmarker}{\pgfqpoint{0.000000in}{0.000000in}}{\pgfqpoint{0.041667in}{0.000000in}}{%
\pgfpathmoveto{\pgfqpoint{0.000000in}{0.000000in}}%
\pgfpathlineto{\pgfqpoint{0.041667in}{0.000000in}}%
\pgfusepath{stroke,fill}%
}%
\begin{pgfscope}%
\pgfsys@transformshift{0.566985in}{3.528569in}%
\pgfsys@useobject{currentmarker}{}%
\end{pgfscope}%
\end{pgfscope}%
\begin{pgfscope}%
\definecolor{textcolor}{rgb}{0.150000,0.150000,0.150000}%
\pgfsetstrokecolor{textcolor}%
\pgfsetfillcolor{textcolor}%
\pgftext[x=0.469762in,y=3.528569in,right,]{\color{textcolor}\rmfamily\fontsize{10.000000}{12.000000}\selectfont \(\displaystyle 1.0\)}%
\end{pgfscope}%
\begin{pgfscope}%
\definecolor{textcolor}{rgb}{0.150000,0.150000,0.150000}%
\pgfsetstrokecolor{textcolor}%
\pgfsetfillcolor{textcolor}%
\pgftext[x=0.222848in,y=2.096354in,,bottom,rotate=90.000000]{\color{textcolor}\rmfamily\fontsize{10.000000}{12.000000}\selectfont \textbf{Cumulative porportion}}%
\end{pgfscope}%
\begin{pgfscope}%
\pgfpathrectangle{\pgfqpoint{0.566985in}{0.664139in}}{\pgfqpoint{2.015106in}{2.864429in}} %
\pgfusepath{clip}%
\pgfsetroundcap%
\pgfsetroundjoin%
\pgfsetlinewidth{1.003750pt}%
\definecolor{currentstroke}{rgb}{0.200000,0.427451,0.650980}%
\pgfsetstrokecolor{currentstroke}%
\pgfsetdash{}{0pt}%
\pgfpathmoveto{\pgfqpoint{0.566985in}{1.027162in}}%
\pgfpathlineto{\pgfqpoint{0.601742in}{1.358475in}}%
\pgfpathlineto{\pgfqpoint{0.636499in}{1.632990in}}%
\pgfpathlineto{\pgfqpoint{0.671256in}{1.796280in}}%
\pgfpathlineto{\pgfqpoint{0.706013in}{1.995067in}}%
\pgfpathlineto{\pgfqpoint{0.740770in}{2.134691in}}%
\pgfpathlineto{\pgfqpoint{0.775527in}{2.269583in}}%
\pgfpathlineto{\pgfqpoint{0.810284in}{2.383175in}}%
\pgfpathlineto{\pgfqpoint{0.845041in}{2.494402in}}%
\pgfpathlineto{\pgfqpoint{0.879798in}{2.605628in}}%
\pgfpathlineto{\pgfqpoint{0.914555in}{2.686089in}}%
\pgfpathlineto{\pgfqpoint{0.949312in}{2.752352in}}%
\pgfpathlineto{\pgfqpoint{0.984069in}{2.842279in}}%
\pgfpathlineto{\pgfqpoint{1.018826in}{2.913275in}}%
\pgfpathlineto{\pgfqpoint{1.053583in}{2.981904in}}%
\pgfpathlineto{\pgfqpoint{1.088340in}{3.057632in}}%
\pgfpathlineto{\pgfqpoint{1.123097in}{3.097863in}}%
\pgfpathlineto{\pgfqpoint{1.157854in}{3.133361in}}%
\pgfpathlineto{\pgfqpoint{1.192611in}{3.194890in}}%
\pgfpathlineto{\pgfqpoint{1.227368in}{3.235121in}}%
\pgfpathlineto{\pgfqpoint{1.262125in}{3.287184in}}%
\pgfpathlineto{\pgfqpoint{1.296882in}{3.322682in}}%
\pgfpathlineto{\pgfqpoint{1.331639in}{3.341614in}}%
\pgfpathlineto{\pgfqpoint{1.366396in}{3.372379in}}%
\pgfpathlineto{\pgfqpoint{1.401153in}{3.391311in}}%
\pgfpathlineto{\pgfqpoint{1.435910in}{3.414976in}}%
\pgfpathlineto{\pgfqpoint{1.470667in}{3.422075in}}%
\pgfpathlineto{\pgfqpoint{1.505424in}{3.443374in}}%
\pgfpathlineto{\pgfqpoint{1.540182in}{3.450474in}}%
\pgfpathlineto{\pgfqpoint{1.574939in}{3.464673in}}%
\pgfpathlineto{\pgfqpoint{1.609696in}{3.478872in}}%
\pgfpathlineto{\pgfqpoint{1.644453in}{3.488338in}}%
\pgfpathlineto{\pgfqpoint{1.679210in}{3.495437in}}%
\pgfpathlineto{\pgfqpoint{1.713967in}{3.497804in}}%
\pgfpathlineto{\pgfqpoint{1.748724in}{3.500170in}}%
\pgfpathlineto{\pgfqpoint{1.783481in}{3.504903in}}%
\pgfpathlineto{\pgfqpoint{1.818238in}{3.509636in}}%
\pgfpathlineto{\pgfqpoint{1.852995in}{3.509636in}}%
\pgfpathlineto{\pgfqpoint{1.887752in}{3.514369in}}%
\pgfpathlineto{\pgfqpoint{1.922509in}{3.521469in}}%
\pgfpathlineto{\pgfqpoint{1.957266in}{3.521469in}}%
\pgfpathlineto{\pgfqpoint{1.992023in}{3.521469in}}%
\pgfpathlineto{\pgfqpoint{2.026780in}{3.523835in}}%
\pgfpathlineto{\pgfqpoint{2.061537in}{3.523835in}}%
\pgfpathlineto{\pgfqpoint{2.096294in}{3.523835in}}%
\pgfpathlineto{\pgfqpoint{2.131051in}{3.523835in}}%
\pgfpathlineto{\pgfqpoint{2.165808in}{3.526202in}}%
\pgfpathlineto{\pgfqpoint{2.200565in}{3.526202in}}%
\pgfpathlineto{\pgfqpoint{2.235322in}{3.526202in}}%
\pgfpathlineto{\pgfqpoint{2.270079in}{3.528569in}}%
\pgfusepath{stroke}%
\end{pgfscope}%
\begin{pgfscope}%
\pgfpathrectangle{\pgfqpoint{0.566985in}{0.664139in}}{\pgfqpoint{2.015106in}{2.864429in}} %
\pgfusepath{clip}%
\pgfsetroundcap%
\pgfsetroundjoin%
\pgfsetlinewidth{1.003750pt}%
\definecolor{currentstroke}{rgb}{0.168627,0.670588,0.494118}%
\pgfsetstrokecolor{currentstroke}%
\pgfsetdash{}{0pt}%
\pgfpathmoveto{\pgfqpoint{0.566985in}{1.367067in}}%
\pgfpathlineto{\pgfqpoint{0.602258in}{1.661417in}}%
\pgfpathlineto{\pgfqpoint{0.637531in}{1.815183in}}%
\pgfpathlineto{\pgfqpoint{0.672804in}{1.986521in}}%
\pgfpathlineto{\pgfqpoint{0.708078in}{2.149073in}}%
\pgfpathlineto{\pgfqpoint{0.743351in}{2.298445in}}%
\pgfpathlineto{\pgfqpoint{0.778624in}{2.465391in}}%
\pgfpathlineto{\pgfqpoint{0.813897in}{2.601583in}}%
\pgfpathlineto{\pgfqpoint{0.849171in}{2.685056in}}%
\pgfpathlineto{\pgfqpoint{0.884444in}{2.772921in}}%
\pgfpathlineto{\pgfqpoint{0.919717in}{2.891540in}}%
\pgfpathlineto{\pgfqpoint{0.954990in}{2.953047in}}%
\pgfpathlineto{\pgfqpoint{0.990264in}{3.010160in}}%
\pgfpathlineto{\pgfqpoint{1.025537in}{3.084846in}}%
\pgfpathlineto{\pgfqpoint{1.060810in}{3.146352in}}%
\pgfpathlineto{\pgfqpoint{1.096083in}{3.168318in}}%
\pgfpathlineto{\pgfqpoint{1.131357in}{3.207858in}}%
\pgfpathlineto{\pgfqpoint{1.166630in}{3.260577in}}%
\pgfpathlineto{\pgfqpoint{1.201903in}{3.304510in}}%
\pgfpathlineto{\pgfqpoint{1.237176in}{3.330870in}}%
\pgfpathlineto{\pgfqpoint{1.272450in}{3.361623in}}%
\pgfpathlineto{\pgfqpoint{1.307723in}{3.405556in}}%
\pgfpathlineto{\pgfqpoint{1.342996in}{3.427523in}}%
\pgfpathlineto{\pgfqpoint{1.378270in}{3.449489in}}%
\pgfpathlineto{\pgfqpoint{1.413543in}{3.453882in}}%
\pgfpathlineto{\pgfqpoint{1.448816in}{3.462669in}}%
\pgfpathlineto{\pgfqpoint{1.484089in}{3.480242in}}%
\pgfpathlineto{\pgfqpoint{1.519363in}{3.489029in}}%
\pgfpathlineto{\pgfqpoint{1.554636in}{3.489029in}}%
\pgfpathlineto{\pgfqpoint{1.589909in}{3.497815in}}%
\pgfpathlineto{\pgfqpoint{1.625182in}{3.506602in}}%
\pgfpathlineto{\pgfqpoint{1.660456in}{3.506602in}}%
\pgfpathlineto{\pgfqpoint{1.695729in}{3.506602in}}%
\pgfpathlineto{\pgfqpoint{1.731002in}{3.506602in}}%
\pgfpathlineto{\pgfqpoint{1.766275in}{3.506602in}}%
\pgfpathlineto{\pgfqpoint{1.801549in}{3.510995in}}%
\pgfpathlineto{\pgfqpoint{1.836822in}{3.510995in}}%
\pgfpathlineto{\pgfqpoint{1.872095in}{3.519782in}}%
\pgfpathlineto{\pgfqpoint{1.907368in}{3.519782in}}%
\pgfpathlineto{\pgfqpoint{1.942642in}{3.519782in}}%
\pgfpathlineto{\pgfqpoint{1.977915in}{3.519782in}}%
\pgfpathlineto{\pgfqpoint{2.013188in}{3.524175in}}%
\pgfpathlineto{\pgfqpoint{2.048461in}{3.524175in}}%
\pgfpathlineto{\pgfqpoint{2.083735in}{3.524175in}}%
\pgfpathlineto{\pgfqpoint{2.119008in}{3.524175in}}%
\pgfpathlineto{\pgfqpoint{2.154281in}{3.524175in}}%
\pgfpathlineto{\pgfqpoint{2.189554in}{3.524175in}}%
\pgfpathlineto{\pgfqpoint{2.224828in}{3.524175in}}%
\pgfpathlineto{\pgfqpoint{2.260101in}{3.524175in}}%
\pgfpathlineto{\pgfqpoint{2.295374in}{3.528569in}}%
\pgfusepath{stroke}%
\end{pgfscope}%
\begin{pgfscope}%
\pgfpathrectangle{\pgfqpoint{0.566985in}{0.664139in}}{\pgfqpoint{2.015106in}{2.864429in}} %
\pgfusepath{clip}%
\pgfsetroundcap%
\pgfsetroundjoin%
\pgfsetlinewidth{1.003750pt}%
\definecolor{currentstroke}{rgb}{1.000000,0.494118,0.250980}%
\pgfsetstrokecolor{currentstroke}%
\pgfsetdash{}{0pt}%
\pgfpathmoveto{\pgfqpoint{0.566985in}{0.984612in}}%
\pgfpathlineto{\pgfqpoint{0.578925in}{1.327385in}}%
\pgfpathlineto{\pgfqpoint{0.590866in}{1.641250in}}%
\pgfpathlineto{\pgfqpoint{0.602807in}{1.839480in}}%
\pgfpathlineto{\pgfqpoint{0.614748in}{2.021191in}}%
\pgfpathlineto{\pgfqpoint{0.626688in}{2.273109in}}%
\pgfpathlineto{\pgfqpoint{0.638629in}{2.425912in}}%
\pgfpathlineto{\pgfqpoint{0.650570in}{2.603493in}}%
\pgfpathlineto{\pgfqpoint{0.662511in}{2.702608in}}%
\pgfpathlineto{\pgfqpoint{0.674452in}{2.834762in}}%
\pgfpathlineto{\pgfqpoint{0.686392in}{2.942137in}}%
\pgfpathlineto{\pgfqpoint{0.698333in}{3.032992in}}%
\pgfpathlineto{\pgfqpoint{0.710274in}{3.103199in}}%
\pgfpathlineto{\pgfqpoint{0.722215in}{3.148627in}}%
\pgfpathlineto{\pgfqpoint{0.734155in}{3.202314in}}%
\pgfpathlineto{\pgfqpoint{0.746096in}{3.235353in}}%
\pgfpathlineto{\pgfqpoint{0.758037in}{3.276651in}}%
\pgfpathlineto{\pgfqpoint{0.769978in}{3.330338in}}%
\pgfpathlineto{\pgfqpoint{0.781918in}{3.375766in}}%
\pgfpathlineto{\pgfqpoint{0.793859in}{3.388155in}}%
\pgfpathlineto{\pgfqpoint{0.805800in}{3.408804in}}%
\pgfpathlineto{\pgfqpoint{0.817741in}{3.429453in}}%
\pgfpathlineto{\pgfqpoint{0.829682in}{3.437713in}}%
\pgfpathlineto{\pgfqpoint{0.841622in}{3.445973in}}%
\pgfpathlineto{\pgfqpoint{0.853563in}{3.445973in}}%
\pgfpathlineto{\pgfqpoint{0.865504in}{3.450102in}}%
\pgfpathlineto{\pgfqpoint{0.877445in}{3.458362in}}%
\pgfpathlineto{\pgfqpoint{0.889385in}{3.466622in}}%
\pgfpathlineto{\pgfqpoint{0.901326in}{3.479011in}}%
\pgfpathlineto{\pgfqpoint{0.913267in}{3.483141in}}%
\pgfpathlineto{\pgfqpoint{0.925208in}{3.491400in}}%
\pgfpathlineto{\pgfqpoint{0.937148in}{3.495530in}}%
\pgfpathlineto{\pgfqpoint{0.949089in}{3.499660in}}%
\pgfpathlineto{\pgfqpoint{0.961030in}{3.499660in}}%
\pgfpathlineto{\pgfqpoint{0.972971in}{3.507920in}}%
\pgfpathlineto{\pgfqpoint{0.984912in}{3.507920in}}%
\pgfpathlineto{\pgfqpoint{0.996852in}{3.507920in}}%
\pgfpathlineto{\pgfqpoint{1.008793in}{3.512049in}}%
\pgfpathlineto{\pgfqpoint{1.020734in}{3.512049in}}%
\pgfpathlineto{\pgfqpoint{1.032675in}{3.512049in}}%
\pgfpathlineto{\pgfqpoint{1.044615in}{3.512049in}}%
\pgfpathlineto{\pgfqpoint{1.056556in}{3.516179in}}%
\pgfpathlineto{\pgfqpoint{1.068497in}{3.516179in}}%
\pgfpathlineto{\pgfqpoint{1.080438in}{3.516179in}}%
\pgfpathlineto{\pgfqpoint{1.092379in}{3.516179in}}%
\pgfpathlineto{\pgfqpoint{1.104319in}{3.520309in}}%
\pgfpathlineto{\pgfqpoint{1.116260in}{3.520309in}}%
\pgfpathlineto{\pgfqpoint{1.128201in}{3.520309in}}%
\pgfpathlineto{\pgfqpoint{1.140142in}{3.524439in}}%
\pgfpathlineto{\pgfqpoint{1.152082in}{3.528569in}}%
\pgfusepath{stroke}%
\end{pgfscope}%
\begin{pgfscope}%
\pgfpathrectangle{\pgfqpoint{0.566985in}{0.664139in}}{\pgfqpoint{2.015106in}{2.864429in}} %
\pgfusepath{clip}%
\pgfsetroundcap%
\pgfsetroundjoin%
\pgfsetlinewidth{1.003750pt}%
\definecolor{currentstroke}{rgb}{1.000000,0.694118,0.250980}%
\pgfsetstrokecolor{currentstroke}%
\pgfsetdash{}{0pt}%
\pgfpathmoveto{\pgfqpoint{0.566985in}{1.149574in}}%
\pgfpathlineto{\pgfqpoint{0.595082in}{1.519975in}}%
\pgfpathlineto{\pgfqpoint{0.623179in}{1.812079in}}%
\pgfpathlineto{\pgfqpoint{0.651275in}{2.043956in}}%
\pgfpathlineto{\pgfqpoint{0.679372in}{2.215605in}}%
\pgfpathlineto{\pgfqpoint{0.707469in}{2.390265in}}%
\pgfpathlineto{\pgfqpoint{0.735566in}{2.531800in}}%
\pgfpathlineto{\pgfqpoint{0.763663in}{2.670324in}}%
\pgfpathlineto{\pgfqpoint{0.791760in}{2.805836in}}%
\pgfpathlineto{\pgfqpoint{0.819857in}{2.905212in}}%
\pgfpathlineto{\pgfqpoint{0.847954in}{3.001576in}}%
\pgfpathlineto{\pgfqpoint{0.876051in}{3.091918in}}%
\pgfpathlineto{\pgfqpoint{0.904148in}{3.161180in}}%
\pgfpathlineto{\pgfqpoint{0.932245in}{3.218396in}}%
\pgfpathlineto{\pgfqpoint{0.960342in}{3.254532in}}%
\pgfpathlineto{\pgfqpoint{0.988439in}{3.302715in}}%
\pgfpathlineto{\pgfqpoint{1.016536in}{3.335840in}}%
\pgfpathlineto{\pgfqpoint{1.044633in}{3.359931in}}%
\pgfpathlineto{\pgfqpoint{1.072730in}{3.387033in}}%
\pgfpathlineto{\pgfqpoint{1.100827in}{3.399079in}}%
\pgfpathlineto{\pgfqpoint{1.128924in}{3.408113in}}%
\pgfpathlineto{\pgfqpoint{1.157021in}{3.420159in}}%
\pgfpathlineto{\pgfqpoint{1.185118in}{3.429193in}}%
\pgfpathlineto{\pgfqpoint{1.213215in}{3.444250in}}%
\pgfpathlineto{\pgfqpoint{1.241312in}{3.456295in}}%
\pgfpathlineto{\pgfqpoint{1.269409in}{3.459307in}}%
\pgfpathlineto{\pgfqpoint{1.297506in}{3.465329in}}%
\pgfpathlineto{\pgfqpoint{1.325603in}{3.480386in}}%
\pgfpathlineto{\pgfqpoint{1.353700in}{3.489421in}}%
\pgfpathlineto{\pgfqpoint{1.381797in}{3.489421in}}%
\pgfpathlineto{\pgfqpoint{1.409894in}{3.495443in}}%
\pgfpathlineto{\pgfqpoint{1.437991in}{3.495443in}}%
\pgfpathlineto{\pgfqpoint{1.466088in}{3.495443in}}%
\pgfpathlineto{\pgfqpoint{1.494185in}{3.498455in}}%
\pgfpathlineto{\pgfqpoint{1.522281in}{3.501466in}}%
\pgfpathlineto{\pgfqpoint{1.550378in}{3.501466in}}%
\pgfpathlineto{\pgfqpoint{1.578475in}{3.504477in}}%
\pgfpathlineto{\pgfqpoint{1.606572in}{3.504477in}}%
\pgfpathlineto{\pgfqpoint{1.634669in}{3.510500in}}%
\pgfpathlineto{\pgfqpoint{1.662766in}{3.510500in}}%
\pgfpathlineto{\pgfqpoint{1.690863in}{3.510500in}}%
\pgfpathlineto{\pgfqpoint{1.718960in}{3.510500in}}%
\pgfpathlineto{\pgfqpoint{1.747057in}{3.510500in}}%
\pgfpathlineto{\pgfqpoint{1.775154in}{3.513512in}}%
\pgfpathlineto{\pgfqpoint{1.803251in}{3.516523in}}%
\pgfpathlineto{\pgfqpoint{1.831348in}{3.522546in}}%
\pgfpathlineto{\pgfqpoint{1.859445in}{3.525557in}}%
\pgfpathlineto{\pgfqpoint{1.887542in}{3.525557in}}%
\pgfpathlineto{\pgfqpoint{1.915639in}{3.525557in}}%
\pgfpathlineto{\pgfqpoint{1.943736in}{3.528569in}}%
\pgfusepath{stroke}%
\end{pgfscope}%
\begin{pgfscope}%
\pgfsetrectcap%
\pgfsetmiterjoin%
\pgfsetlinewidth{1.254687pt}%
\definecolor{currentstroke}{rgb}{0.150000,0.150000,0.150000}%
\pgfsetstrokecolor{currentstroke}%
\pgfsetdash{}{0pt}%
\pgfpathmoveto{\pgfqpoint{0.566985in}{0.664139in}}%
\pgfpathlineto{\pgfqpoint{0.566985in}{3.528569in}}%
\pgfusepath{stroke}%
\end{pgfscope}%
\begin{pgfscope}%
\pgfsetrectcap%
\pgfsetmiterjoin%
\pgfsetlinewidth{1.254687pt}%
\definecolor{currentstroke}{rgb}{0.150000,0.150000,0.150000}%
\pgfsetstrokecolor{currentstroke}%
\pgfsetdash{}{0pt}%
\pgfpathmoveto{\pgfqpoint{0.566985in}{0.664139in}}%
\pgfpathlineto{\pgfqpoint{2.582091in}{0.664139in}}%
\pgfusepath{stroke}%
\end{pgfscope}%
\begin{pgfscope}%
\pgfsetbuttcap%
\pgfsetmiterjoin%
\definecolor{currentfill}{rgb}{1.000000,1.000000,1.000000}%
\pgfsetfillcolor{currentfill}%
\pgfsetlinewidth{0.000000pt}%
\definecolor{currentstroke}{rgb}{0.000000,0.000000,0.000000}%
\pgfsetstrokecolor{currentstroke}%
\pgfsetstrokeopacity{0.000000}%
\pgfsetdash{}{0pt}%
\pgfpathmoveto{\pgfqpoint{3.186623in}{0.664139in}}%
\pgfpathlineto{\pgfqpoint{5.201729in}{0.664139in}}%
\pgfpathlineto{\pgfqpoint{5.201729in}{2.789361in}}%
\pgfpathlineto{\pgfqpoint{3.186623in}{2.789361in}}%
\pgfpathclose%
\pgfusepath{fill}%
\end{pgfscope}%
\begin{pgfscope}%
\pgfsetroundcap%
\pgfsetroundjoin%
\pgfsetlinewidth{1.003750pt}%
\definecolor{currentstroke}{rgb}{0.200000,0.427451,0.650980}%
\pgfsetstrokecolor{currentstroke}%
\pgfsetdash{}{0pt}%
\pgfpathmoveto{\pgfqpoint{3.085112in}{3.455940in}}%
\pgfpathlineto{\pgfqpoint{3.196223in}{3.455940in}}%
\pgfusepath{stroke}%
\end{pgfscope}%
\begin{pgfscope}%
\definecolor{textcolor}{rgb}{1.000000,1.000000,1.000000}%
\pgfsetstrokecolor{textcolor}%
\pgfsetfillcolor{textcolor}%
\pgftext[x=3.285112in,y=3.417051in,left,base]{\color{textcolor}\rmfamily\fontsize{8.000000}{9.600000}\selectfont WT + Vehicle (1513)}%
\end{pgfscope}%
\begin{pgfscope}%
\pgfsetroundcap%
\pgfsetroundjoin%
\pgfsetlinewidth{1.003750pt}%
\definecolor{currentstroke}{rgb}{0.168627,0.670588,0.494118}%
\pgfsetstrokecolor{currentstroke}%
\pgfsetdash{}{0pt}%
\pgfpathmoveto{\pgfqpoint{3.085112in}{3.289300in}}%
\pgfpathlineto{\pgfqpoint{3.196223in}{3.289300in}}%
\pgfusepath{stroke}%
\end{pgfscope}%
\begin{pgfscope}%
\definecolor{textcolor}{rgb}{1.000000,1.000000,1.000000}%
\pgfsetstrokecolor{textcolor}%
\pgfsetfillcolor{textcolor}%
\pgftext[x=3.285112in,y=3.250411in,left,base]{\color{textcolor}\rmfamily\fontsize{8.000000}{9.600000}\selectfont WT + TAT-GluA2\textsubscript{3Y} (815)}%
\end{pgfscope}%
\begin{pgfscope}%
\pgfsetroundcap%
\pgfsetroundjoin%
\pgfsetlinewidth{1.003750pt}%
\definecolor{currentstroke}{rgb}{1.000000,0.494118,0.250980}%
\pgfsetstrokecolor{currentstroke}%
\pgfsetdash{}{0pt}%
\pgfpathmoveto{\pgfqpoint{3.085112in}{3.122660in}}%
\pgfpathlineto{\pgfqpoint{3.196223in}{3.122660in}}%
\pgfusepath{stroke}%
\end{pgfscope}%
\begin{pgfscope}%
\definecolor{textcolor}{rgb}{1.000000,1.000000,1.000000}%
\pgfsetstrokecolor{textcolor}%
\pgfsetfillcolor{textcolor}%
\pgftext[x=3.285112in,y=3.083771in,left,base]{\color{textcolor}\rmfamily\fontsize{8.000000}{9.600000}\selectfont Tg + Vehicle (867)}%
\end{pgfscope}%
\begin{pgfscope}%
\pgfsetroundcap%
\pgfsetroundjoin%
\pgfsetlinewidth{1.003750pt}%
\definecolor{currentstroke}{rgb}{1.000000,0.694118,0.250980}%
\pgfsetstrokecolor{currentstroke}%
\pgfsetdash{}{0pt}%
\pgfpathmoveto{\pgfqpoint{3.085112in}{2.956021in}}%
\pgfpathlineto{\pgfqpoint{3.196223in}{2.956021in}}%
\pgfusepath{stroke}%
\end{pgfscope}%
\begin{pgfscope}%
\definecolor{textcolor}{rgb}{1.000000,1.000000,1.000000}%
\pgfsetstrokecolor{textcolor}%
\pgfsetfillcolor{textcolor}%
\pgftext[x=3.285112in,y=2.917132in,left,base]{\color{textcolor}\rmfamily\fontsize{8.000000}{9.600000}\selectfont Tg + TAT-GluA2\textsubscript{3Y} (1189)}%
\end{pgfscope}%
\begin{pgfscope}%
\pgfsetroundcap%
\pgfsetroundjoin%
\pgfsetlinewidth{1.003750pt}%
\definecolor{currentstroke}{rgb}{0.200000,0.427451,0.650980}%
\pgfsetstrokecolor{currentstroke}%
\pgfsetdash{}{0pt}%
\pgfpathmoveto{\pgfqpoint{3.085112in}{3.455940in}}%
\pgfpathlineto{\pgfqpoint{3.196223in}{3.455940in}}%
\pgfusepath{stroke}%
\end{pgfscope}%
\begin{pgfscope}%
\definecolor{textcolor}{rgb}{1.000000,1.000000,1.000000}%
\pgfsetstrokecolor{textcolor}%
\pgfsetfillcolor{textcolor}%
\pgftext[x=3.285112in,y=3.417051in,left,base]{\color{textcolor}\rmfamily\fontsize{8.000000}{9.600000}\selectfont WT + Vehicle (1513)}%
\end{pgfscope}%
\begin{pgfscope}%
\pgfsetroundcap%
\pgfsetroundjoin%
\pgfsetlinewidth{1.003750pt}%
\definecolor{currentstroke}{rgb}{0.168627,0.670588,0.494118}%
\pgfsetstrokecolor{currentstroke}%
\pgfsetdash{}{0pt}%
\pgfpathmoveto{\pgfqpoint{3.085112in}{3.289300in}}%
\pgfpathlineto{\pgfqpoint{3.196223in}{3.289300in}}%
\pgfusepath{stroke}%
\end{pgfscope}%
\begin{pgfscope}%
\definecolor{textcolor}{rgb}{1.000000,1.000000,1.000000}%
\pgfsetstrokecolor{textcolor}%
\pgfsetfillcolor{textcolor}%
\pgftext[x=3.285112in,y=3.250411in,left,base]{\color{textcolor}\rmfamily\fontsize{8.000000}{9.600000}\selectfont WT + TAT-GluA2\textsubscript{3Y} (815)}%
\end{pgfscope}%
\begin{pgfscope}%
\pgfsetroundcap%
\pgfsetroundjoin%
\pgfsetlinewidth{1.003750pt}%
\definecolor{currentstroke}{rgb}{1.000000,0.494118,0.250980}%
\pgfsetstrokecolor{currentstroke}%
\pgfsetdash{}{0pt}%
\pgfpathmoveto{\pgfqpoint{3.085112in}{3.122660in}}%
\pgfpathlineto{\pgfqpoint{3.196223in}{3.122660in}}%
\pgfusepath{stroke}%
\end{pgfscope}%
\begin{pgfscope}%
\definecolor{textcolor}{rgb}{1.000000,1.000000,1.000000}%
\pgfsetstrokecolor{textcolor}%
\pgfsetfillcolor{textcolor}%
\pgftext[x=3.285112in,y=3.083771in,left,base]{\color{textcolor}\rmfamily\fontsize{8.000000}{9.600000}\selectfont Tg + Vehicle (867)}%
\end{pgfscope}%
\begin{pgfscope}%
\pgfsetroundcap%
\pgfsetroundjoin%
\pgfsetlinewidth{1.003750pt}%
\definecolor{currentstroke}{rgb}{1.000000,0.694118,0.250980}%
\pgfsetstrokecolor{currentstroke}%
\pgfsetdash{}{0pt}%
\pgfpathmoveto{\pgfqpoint{3.085112in}{2.956021in}}%
\pgfpathlineto{\pgfqpoint{3.196223in}{2.956021in}}%
\pgfusepath{stroke}%
\end{pgfscope}%
\begin{pgfscope}%
\definecolor{textcolor}{rgb}{1.000000,1.000000,1.000000}%
\pgfsetstrokecolor{textcolor}%
\pgfsetfillcolor{textcolor}%
\pgftext[x=3.285112in,y=2.917132in,left,base]{\color{textcolor}\rmfamily\fontsize{8.000000}{9.600000}\selectfont Tg + TAT-GluA2\textsubscript{3Y} (1189)}%
\end{pgfscope}%
\begin{pgfscope}%
\pgfsetbuttcap%
\pgfsetroundjoin%
\definecolor{currentfill}{rgb}{0.150000,0.150000,0.150000}%
\pgfsetfillcolor{currentfill}%
\pgfsetlinewidth{1.003750pt}%
\definecolor{currentstroke}{rgb}{0.150000,0.150000,0.150000}%
\pgfsetstrokecolor{currentstroke}%
\pgfsetdash{}{0pt}%
\pgfsys@defobject{currentmarker}{\pgfqpoint{0.000000in}{0.000000in}}{\pgfqpoint{0.041667in}{0.000000in}}{%
\pgfpathmoveto{\pgfqpoint{0.000000in}{0.000000in}}%
\pgfpathlineto{\pgfqpoint{0.041667in}{0.000000in}}%
\pgfusepath{stroke,fill}%
}%
\begin{pgfscope}%
\pgfsys@transformshift{3.186623in}{0.664139in}%
\pgfsys@useobject{currentmarker}{}%
\end{pgfscope}%
\end{pgfscope}%
\begin{pgfscope}%
\definecolor{textcolor}{rgb}{0.150000,0.150000,0.150000}%
\pgfsetstrokecolor{textcolor}%
\pgfsetfillcolor{textcolor}%
\pgftext[x=3.089400in,y=0.664139in,right,]{\color{textcolor}\rmfamily\fontsize{10.000000}{12.000000}\selectfont \(\displaystyle 0.000\)}%
\end{pgfscope}%
\begin{pgfscope}%
\pgfsetbuttcap%
\pgfsetroundjoin%
\definecolor{currentfill}{rgb}{0.150000,0.150000,0.150000}%
\pgfsetfillcolor{currentfill}%
\pgfsetlinewidth{1.003750pt}%
\definecolor{currentstroke}{rgb}{0.150000,0.150000,0.150000}%
\pgfsetstrokecolor{currentstroke}%
\pgfsetdash{}{0pt}%
\pgfsys@defobject{currentmarker}{\pgfqpoint{0.000000in}{0.000000in}}{\pgfqpoint{0.041667in}{0.000000in}}{%
\pgfpathmoveto{\pgfqpoint{0.000000in}{0.000000in}}%
\pgfpathlineto{\pgfqpoint{0.041667in}{0.000000in}}%
\pgfusepath{stroke,fill}%
}%
\begin{pgfscope}%
\pgfsys@transformshift{3.186623in}{0.929792in}%
\pgfsys@useobject{currentmarker}{}%
\end{pgfscope}%
\end{pgfscope}%
\begin{pgfscope}%
\definecolor{textcolor}{rgb}{0.150000,0.150000,0.150000}%
\pgfsetstrokecolor{textcolor}%
\pgfsetfillcolor{textcolor}%
\pgftext[x=3.089400in,y=0.929792in,right,]{\color{textcolor}\rmfamily\fontsize{10.000000}{12.000000}\selectfont \(\displaystyle 0.005\)}%
\end{pgfscope}%
\begin{pgfscope}%
\pgfsetbuttcap%
\pgfsetroundjoin%
\definecolor{currentfill}{rgb}{0.150000,0.150000,0.150000}%
\pgfsetfillcolor{currentfill}%
\pgfsetlinewidth{1.003750pt}%
\definecolor{currentstroke}{rgb}{0.150000,0.150000,0.150000}%
\pgfsetstrokecolor{currentstroke}%
\pgfsetdash{}{0pt}%
\pgfsys@defobject{currentmarker}{\pgfqpoint{0.000000in}{0.000000in}}{\pgfqpoint{0.041667in}{0.000000in}}{%
\pgfpathmoveto{\pgfqpoint{0.000000in}{0.000000in}}%
\pgfpathlineto{\pgfqpoint{0.041667in}{0.000000in}}%
\pgfusepath{stroke,fill}%
}%
\begin{pgfscope}%
\pgfsys@transformshift{3.186623in}{1.195445in}%
\pgfsys@useobject{currentmarker}{}%
\end{pgfscope}%
\end{pgfscope}%
\begin{pgfscope}%
\definecolor{textcolor}{rgb}{0.150000,0.150000,0.150000}%
\pgfsetstrokecolor{textcolor}%
\pgfsetfillcolor{textcolor}%
\pgftext[x=3.089400in,y=1.195445in,right,]{\color{textcolor}\rmfamily\fontsize{10.000000}{12.000000}\selectfont \(\displaystyle 0.010\)}%
\end{pgfscope}%
\begin{pgfscope}%
\pgfsetbuttcap%
\pgfsetroundjoin%
\definecolor{currentfill}{rgb}{0.150000,0.150000,0.150000}%
\pgfsetfillcolor{currentfill}%
\pgfsetlinewidth{1.003750pt}%
\definecolor{currentstroke}{rgb}{0.150000,0.150000,0.150000}%
\pgfsetstrokecolor{currentstroke}%
\pgfsetdash{}{0pt}%
\pgfsys@defobject{currentmarker}{\pgfqpoint{0.000000in}{0.000000in}}{\pgfqpoint{0.041667in}{0.000000in}}{%
\pgfpathmoveto{\pgfqpoint{0.000000in}{0.000000in}}%
\pgfpathlineto{\pgfqpoint{0.041667in}{0.000000in}}%
\pgfusepath{stroke,fill}%
}%
\begin{pgfscope}%
\pgfsys@transformshift{3.186623in}{1.461097in}%
\pgfsys@useobject{currentmarker}{}%
\end{pgfscope}%
\end{pgfscope}%
\begin{pgfscope}%
\definecolor{textcolor}{rgb}{0.150000,0.150000,0.150000}%
\pgfsetstrokecolor{textcolor}%
\pgfsetfillcolor{textcolor}%
\pgftext[x=3.089400in,y=1.461097in,right,]{\color{textcolor}\rmfamily\fontsize{10.000000}{12.000000}\selectfont \(\displaystyle 0.015\)}%
\end{pgfscope}%
\begin{pgfscope}%
\pgfsetbuttcap%
\pgfsetroundjoin%
\definecolor{currentfill}{rgb}{0.150000,0.150000,0.150000}%
\pgfsetfillcolor{currentfill}%
\pgfsetlinewidth{1.003750pt}%
\definecolor{currentstroke}{rgb}{0.150000,0.150000,0.150000}%
\pgfsetstrokecolor{currentstroke}%
\pgfsetdash{}{0pt}%
\pgfsys@defobject{currentmarker}{\pgfqpoint{0.000000in}{0.000000in}}{\pgfqpoint{0.041667in}{0.000000in}}{%
\pgfpathmoveto{\pgfqpoint{0.000000in}{0.000000in}}%
\pgfpathlineto{\pgfqpoint{0.041667in}{0.000000in}}%
\pgfusepath{stroke,fill}%
}%
\begin{pgfscope}%
\pgfsys@transformshift{3.186623in}{1.726750in}%
\pgfsys@useobject{currentmarker}{}%
\end{pgfscope}%
\end{pgfscope}%
\begin{pgfscope}%
\definecolor{textcolor}{rgb}{0.150000,0.150000,0.150000}%
\pgfsetstrokecolor{textcolor}%
\pgfsetfillcolor{textcolor}%
\pgftext[x=3.089400in,y=1.726750in,right,]{\color{textcolor}\rmfamily\fontsize{10.000000}{12.000000}\selectfont \(\displaystyle 0.020\)}%
\end{pgfscope}%
\begin{pgfscope}%
\pgfsetbuttcap%
\pgfsetroundjoin%
\definecolor{currentfill}{rgb}{0.150000,0.150000,0.150000}%
\pgfsetfillcolor{currentfill}%
\pgfsetlinewidth{1.003750pt}%
\definecolor{currentstroke}{rgb}{0.150000,0.150000,0.150000}%
\pgfsetstrokecolor{currentstroke}%
\pgfsetdash{}{0pt}%
\pgfsys@defobject{currentmarker}{\pgfqpoint{0.000000in}{0.000000in}}{\pgfqpoint{0.041667in}{0.000000in}}{%
\pgfpathmoveto{\pgfqpoint{0.000000in}{0.000000in}}%
\pgfpathlineto{\pgfqpoint{0.041667in}{0.000000in}}%
\pgfusepath{stroke,fill}%
}%
\begin{pgfscope}%
\pgfsys@transformshift{3.186623in}{1.992403in}%
\pgfsys@useobject{currentmarker}{}%
\end{pgfscope}%
\end{pgfscope}%
\begin{pgfscope}%
\definecolor{textcolor}{rgb}{0.150000,0.150000,0.150000}%
\pgfsetstrokecolor{textcolor}%
\pgfsetfillcolor{textcolor}%
\pgftext[x=3.089400in,y=1.992403in,right,]{\color{textcolor}\rmfamily\fontsize{10.000000}{12.000000}\selectfont \(\displaystyle 0.025\)}%
\end{pgfscope}%
\begin{pgfscope}%
\pgfsetbuttcap%
\pgfsetroundjoin%
\definecolor{currentfill}{rgb}{0.150000,0.150000,0.150000}%
\pgfsetfillcolor{currentfill}%
\pgfsetlinewidth{1.003750pt}%
\definecolor{currentstroke}{rgb}{0.150000,0.150000,0.150000}%
\pgfsetstrokecolor{currentstroke}%
\pgfsetdash{}{0pt}%
\pgfsys@defobject{currentmarker}{\pgfqpoint{0.000000in}{0.000000in}}{\pgfqpoint{0.041667in}{0.000000in}}{%
\pgfpathmoveto{\pgfqpoint{0.000000in}{0.000000in}}%
\pgfpathlineto{\pgfqpoint{0.041667in}{0.000000in}}%
\pgfusepath{stroke,fill}%
}%
\begin{pgfscope}%
\pgfsys@transformshift{3.186623in}{2.258055in}%
\pgfsys@useobject{currentmarker}{}%
\end{pgfscope}%
\end{pgfscope}%
\begin{pgfscope}%
\definecolor{textcolor}{rgb}{0.150000,0.150000,0.150000}%
\pgfsetstrokecolor{textcolor}%
\pgfsetfillcolor{textcolor}%
\pgftext[x=3.089400in,y=2.258055in,right,]{\color{textcolor}\rmfamily\fontsize{10.000000}{12.000000}\selectfont \(\displaystyle 0.030\)}%
\end{pgfscope}%
\begin{pgfscope}%
\pgfsetbuttcap%
\pgfsetroundjoin%
\definecolor{currentfill}{rgb}{0.150000,0.150000,0.150000}%
\pgfsetfillcolor{currentfill}%
\pgfsetlinewidth{1.003750pt}%
\definecolor{currentstroke}{rgb}{0.150000,0.150000,0.150000}%
\pgfsetstrokecolor{currentstroke}%
\pgfsetdash{}{0pt}%
\pgfsys@defobject{currentmarker}{\pgfqpoint{0.000000in}{0.000000in}}{\pgfqpoint{0.041667in}{0.000000in}}{%
\pgfpathmoveto{\pgfqpoint{0.000000in}{0.000000in}}%
\pgfpathlineto{\pgfqpoint{0.041667in}{0.000000in}}%
\pgfusepath{stroke,fill}%
}%
\begin{pgfscope}%
\pgfsys@transformshift{3.186623in}{2.523708in}%
\pgfsys@useobject{currentmarker}{}%
\end{pgfscope}%
\end{pgfscope}%
\begin{pgfscope}%
\definecolor{textcolor}{rgb}{0.150000,0.150000,0.150000}%
\pgfsetstrokecolor{textcolor}%
\pgfsetfillcolor{textcolor}%
\pgftext[x=3.089400in,y=2.523708in,right,]{\color{textcolor}\rmfamily\fontsize{10.000000}{12.000000}\selectfont \(\displaystyle 0.035\)}%
\end{pgfscope}%
\begin{pgfscope}%
\pgfsetbuttcap%
\pgfsetroundjoin%
\definecolor{currentfill}{rgb}{0.150000,0.150000,0.150000}%
\pgfsetfillcolor{currentfill}%
\pgfsetlinewidth{1.003750pt}%
\definecolor{currentstroke}{rgb}{0.150000,0.150000,0.150000}%
\pgfsetstrokecolor{currentstroke}%
\pgfsetdash{}{0pt}%
\pgfsys@defobject{currentmarker}{\pgfqpoint{0.000000in}{0.000000in}}{\pgfqpoint{0.041667in}{0.000000in}}{%
\pgfpathmoveto{\pgfqpoint{0.000000in}{0.000000in}}%
\pgfpathlineto{\pgfqpoint{0.041667in}{0.000000in}}%
\pgfusepath{stroke,fill}%
}%
\begin{pgfscope}%
\pgfsys@transformshift{3.186623in}{2.789361in}%
\pgfsys@useobject{currentmarker}{}%
\end{pgfscope}%
\end{pgfscope}%
\begin{pgfscope}%
\definecolor{textcolor}{rgb}{0.150000,0.150000,0.150000}%
\pgfsetstrokecolor{textcolor}%
\pgfsetfillcolor{textcolor}%
\pgftext[x=3.089400in,y=2.789361in,right,]{\color{textcolor}\rmfamily\fontsize{10.000000}{12.000000}\selectfont \(\displaystyle 0.040\)}%
\end{pgfscope}%
\begin{pgfscope}%
\definecolor{textcolor}{rgb}{0.150000,0.150000,0.150000}%
\pgfsetstrokecolor{textcolor}%
\pgfsetfillcolor{textcolor}%
\pgftext[x=2.516914in,y=1.001956in,left,base,rotate=90.000000]{\color{textcolor}\rmfamily\fontsize{10.000000}{12.000000}\selectfont \textbf{Freezing information}}%
\end{pgfscope}%
\begin{pgfscope}%
\definecolor{textcolor}{rgb}{0.150000,0.150000,0.150000}%
\pgfsetstrokecolor{textcolor}%
\pgfsetfillcolor{textcolor}%
\pgftext[x=2.668883in,y=1.296435in,left,base,rotate=90.000000]{\color{textcolor}\rmfamily\fontsize{10.000000}{12.000000}\selectfont \textbf{(bits/frame)}}%
\end{pgfscope}%
\begin{pgfscope}%
\pgfpathrectangle{\pgfqpoint{3.186623in}{0.664139in}}{\pgfqpoint{2.015106in}{2.125222in}} %
\pgfusepath{clip}%
\pgfsetbuttcap%
\pgfsetmiterjoin%
\definecolor{currentfill}{rgb}{0.200000,0.427451,0.650980}%
\pgfsetfillcolor{currentfill}%
\pgfsetlinewidth{1.505625pt}%
\definecolor{currentstroke}{rgb}{0.200000,0.427451,0.650980}%
\pgfsetstrokecolor{currentstroke}%
\pgfsetdash{}{0pt}%
\pgfpathmoveto{\pgfqpoint{3.258591in}{0.664139in}}%
\pgfpathlineto{\pgfqpoint{3.618431in}{0.664139in}}%
\pgfpathlineto{\pgfqpoint{3.618431in}{2.208045in}}%
\pgfpathlineto{\pgfqpoint{3.258591in}{2.208045in}}%
\pgfpathclose%
\pgfusepath{stroke,fill}%
\end{pgfscope}%
\begin{pgfscope}%
\pgfpathrectangle{\pgfqpoint{3.186623in}{0.664139in}}{\pgfqpoint{2.015106in}{2.125222in}} %
\pgfusepath{clip}%
\pgfsetbuttcap%
\pgfsetmiterjoin%
\definecolor{currentfill}{rgb}{0.168627,0.670588,0.494118}%
\pgfsetfillcolor{currentfill}%
\pgfsetlinewidth{1.505625pt}%
\definecolor{currentstroke}{rgb}{0.168627,0.670588,0.494118}%
\pgfsetstrokecolor{currentstroke}%
\pgfsetdash{}{0pt}%
\pgfpathmoveto{\pgfqpoint{3.762367in}{0.664139in}}%
\pgfpathlineto{\pgfqpoint{4.122208in}{0.664139in}}%
\pgfpathlineto{\pgfqpoint{4.122208in}{1.943529in}}%
\pgfpathlineto{\pgfqpoint{3.762367in}{1.943529in}}%
\pgfpathclose%
\pgfusepath{stroke,fill}%
\end{pgfscope}%
\begin{pgfscope}%
\pgfpathrectangle{\pgfqpoint{3.186623in}{0.664139in}}{\pgfqpoint{2.015106in}{2.125222in}} %
\pgfusepath{clip}%
\pgfsetbuttcap%
\pgfsetmiterjoin%
\definecolor{currentfill}{rgb}{1.000000,0.494118,0.250980}%
\pgfsetfillcolor{currentfill}%
\pgfsetlinewidth{1.505625pt}%
\definecolor{currentstroke}{rgb}{1.000000,0.494118,0.250980}%
\pgfsetstrokecolor{currentstroke}%
\pgfsetdash{}{0pt}%
\pgfpathmoveto{\pgfqpoint{4.266144in}{0.664139in}}%
\pgfpathlineto{\pgfqpoint{4.625984in}{0.664139in}}%
\pgfpathlineto{\pgfqpoint{4.625984in}{1.113750in}}%
\pgfpathlineto{\pgfqpoint{4.266144in}{1.113750in}}%
\pgfpathclose%
\pgfusepath{stroke,fill}%
\end{pgfscope}%
\begin{pgfscope}%
\pgfpathrectangle{\pgfqpoint{3.186623in}{0.664139in}}{\pgfqpoint{2.015106in}{2.125222in}} %
\pgfusepath{clip}%
\pgfsetbuttcap%
\pgfsetmiterjoin%
\definecolor{currentfill}{rgb}{1.000000,0.694118,0.250980}%
\pgfsetfillcolor{currentfill}%
\pgfsetlinewidth{1.505625pt}%
\definecolor{currentstroke}{rgb}{1.000000,0.694118,0.250980}%
\pgfsetstrokecolor{currentstroke}%
\pgfsetdash{}{0pt}%
\pgfpathmoveto{\pgfqpoint{4.769920in}{0.664139in}}%
\pgfpathlineto{\pgfqpoint{5.129761in}{0.664139in}}%
\pgfpathlineto{\pgfqpoint{5.129761in}{1.620455in}}%
\pgfpathlineto{\pgfqpoint{4.769920in}{1.620455in}}%
\pgfpathclose%
\pgfusepath{stroke,fill}%
\end{pgfscope}%
\begin{pgfscope}%
\pgfpathrectangle{\pgfqpoint{3.186623in}{0.664139in}}{\pgfqpoint{2.015106in}{2.125222in}} %
\pgfusepath{clip}%
\pgfsetbuttcap%
\pgfsetroundjoin%
\pgfsetlinewidth{1.505625pt}%
\definecolor{currentstroke}{rgb}{0.200000,0.427451,0.650980}%
\pgfsetstrokecolor{currentstroke}%
\pgfsetdash{}{0pt}%
\pgfpathmoveto{\pgfqpoint{3.438511in}{2.208045in}}%
\pgfpathlineto{\pgfqpoint{3.438511in}{2.254914in}}%
\pgfusepath{stroke}%
\end{pgfscope}%
\begin{pgfscope}%
\pgfpathrectangle{\pgfqpoint{3.186623in}{0.664139in}}{\pgfqpoint{2.015106in}{2.125222in}} %
\pgfusepath{clip}%
\pgfsetbuttcap%
\pgfsetroundjoin%
\pgfsetlinewidth{1.505625pt}%
\definecolor{currentstroke}{rgb}{0.168627,0.670588,0.494118}%
\pgfsetstrokecolor{currentstroke}%
\pgfsetdash{}{0pt}%
\pgfpathmoveto{\pgfqpoint{3.942287in}{1.943529in}}%
\pgfpathlineto{\pgfqpoint{3.942287in}{2.001479in}}%
\pgfusepath{stroke}%
\end{pgfscope}%
\begin{pgfscope}%
\pgfpathrectangle{\pgfqpoint{3.186623in}{0.664139in}}{\pgfqpoint{2.015106in}{2.125222in}} %
\pgfusepath{clip}%
\pgfsetbuttcap%
\pgfsetroundjoin%
\pgfsetlinewidth{1.505625pt}%
\definecolor{currentstroke}{rgb}{1.000000,0.494118,0.250980}%
\pgfsetstrokecolor{currentstroke}%
\pgfsetdash{}{0pt}%
\pgfpathmoveto{\pgfqpoint{4.446064in}{1.113750in}}%
\pgfpathlineto{\pgfqpoint{4.446064in}{1.132099in}}%
\pgfusepath{stroke}%
\end{pgfscope}%
\begin{pgfscope}%
\pgfpathrectangle{\pgfqpoint{3.186623in}{0.664139in}}{\pgfqpoint{2.015106in}{2.125222in}} %
\pgfusepath{clip}%
\pgfsetbuttcap%
\pgfsetroundjoin%
\pgfsetlinewidth{1.505625pt}%
\definecolor{currentstroke}{rgb}{1.000000,0.694118,0.250980}%
\pgfsetstrokecolor{currentstroke}%
\pgfsetdash{}{0pt}%
\pgfpathmoveto{\pgfqpoint{4.949840in}{1.620455in}}%
\pgfpathlineto{\pgfqpoint{4.949840in}{1.656494in}}%
\pgfusepath{stroke}%
\end{pgfscope}%
\begin{pgfscope}%
\pgfpathrectangle{\pgfqpoint{3.186623in}{0.664139in}}{\pgfqpoint{2.015106in}{2.125222in}} %
\pgfusepath{clip}%
\pgfsetbuttcap%
\pgfsetroundjoin%
\definecolor{currentfill}{rgb}{0.200000,0.427451,0.650980}%
\pgfsetfillcolor{currentfill}%
\pgfsetlinewidth{1.505625pt}%
\definecolor{currentstroke}{rgb}{0.200000,0.427451,0.650980}%
\pgfsetstrokecolor{currentstroke}%
\pgfsetdash{}{0pt}%
\pgfsys@defobject{currentmarker}{\pgfqpoint{-0.111111in}{-0.000000in}}{\pgfqpoint{0.111111in}{0.000000in}}{%
\pgfpathmoveto{\pgfqpoint{0.111111in}{-0.000000in}}%
\pgfpathlineto{\pgfqpoint{-0.111111in}{0.000000in}}%
\pgfusepath{stroke,fill}%
}%
\begin{pgfscope}%
\pgfsys@transformshift{3.438511in}{2.208045in}%
\pgfsys@useobject{currentmarker}{}%
\end{pgfscope}%
\end{pgfscope}%
\begin{pgfscope}%
\pgfpathrectangle{\pgfqpoint{3.186623in}{0.664139in}}{\pgfqpoint{2.015106in}{2.125222in}} %
\pgfusepath{clip}%
\pgfsetbuttcap%
\pgfsetroundjoin%
\definecolor{currentfill}{rgb}{0.200000,0.427451,0.650980}%
\pgfsetfillcolor{currentfill}%
\pgfsetlinewidth{1.505625pt}%
\definecolor{currentstroke}{rgb}{0.200000,0.427451,0.650980}%
\pgfsetstrokecolor{currentstroke}%
\pgfsetdash{}{0pt}%
\pgfsys@defobject{currentmarker}{\pgfqpoint{-0.111111in}{-0.000000in}}{\pgfqpoint{0.111111in}{0.000000in}}{%
\pgfpathmoveto{\pgfqpoint{0.111111in}{-0.000000in}}%
\pgfpathlineto{\pgfqpoint{-0.111111in}{0.000000in}}%
\pgfusepath{stroke,fill}%
}%
\begin{pgfscope}%
\pgfsys@transformshift{3.438511in}{2.254914in}%
\pgfsys@useobject{currentmarker}{}%
\end{pgfscope}%
\end{pgfscope}%
\begin{pgfscope}%
\pgfpathrectangle{\pgfqpoint{3.186623in}{0.664139in}}{\pgfqpoint{2.015106in}{2.125222in}} %
\pgfusepath{clip}%
\pgfsetbuttcap%
\pgfsetroundjoin%
\definecolor{currentfill}{rgb}{0.168627,0.670588,0.494118}%
\pgfsetfillcolor{currentfill}%
\pgfsetlinewidth{1.505625pt}%
\definecolor{currentstroke}{rgb}{0.168627,0.670588,0.494118}%
\pgfsetstrokecolor{currentstroke}%
\pgfsetdash{}{0pt}%
\pgfsys@defobject{currentmarker}{\pgfqpoint{-0.111111in}{-0.000000in}}{\pgfqpoint{0.111111in}{0.000000in}}{%
\pgfpathmoveto{\pgfqpoint{0.111111in}{-0.000000in}}%
\pgfpathlineto{\pgfqpoint{-0.111111in}{0.000000in}}%
\pgfusepath{stroke,fill}%
}%
\begin{pgfscope}%
\pgfsys@transformshift{3.942287in}{1.943529in}%
\pgfsys@useobject{currentmarker}{}%
\end{pgfscope}%
\end{pgfscope}%
\begin{pgfscope}%
\pgfpathrectangle{\pgfqpoint{3.186623in}{0.664139in}}{\pgfqpoint{2.015106in}{2.125222in}} %
\pgfusepath{clip}%
\pgfsetbuttcap%
\pgfsetroundjoin%
\definecolor{currentfill}{rgb}{0.168627,0.670588,0.494118}%
\pgfsetfillcolor{currentfill}%
\pgfsetlinewidth{1.505625pt}%
\definecolor{currentstroke}{rgb}{0.168627,0.670588,0.494118}%
\pgfsetstrokecolor{currentstroke}%
\pgfsetdash{}{0pt}%
\pgfsys@defobject{currentmarker}{\pgfqpoint{-0.111111in}{-0.000000in}}{\pgfqpoint{0.111111in}{0.000000in}}{%
\pgfpathmoveto{\pgfqpoint{0.111111in}{-0.000000in}}%
\pgfpathlineto{\pgfqpoint{-0.111111in}{0.000000in}}%
\pgfusepath{stroke,fill}%
}%
\begin{pgfscope}%
\pgfsys@transformshift{3.942287in}{2.001479in}%
\pgfsys@useobject{currentmarker}{}%
\end{pgfscope}%
\end{pgfscope}%
\begin{pgfscope}%
\pgfpathrectangle{\pgfqpoint{3.186623in}{0.664139in}}{\pgfqpoint{2.015106in}{2.125222in}} %
\pgfusepath{clip}%
\pgfsetbuttcap%
\pgfsetroundjoin%
\definecolor{currentfill}{rgb}{1.000000,0.494118,0.250980}%
\pgfsetfillcolor{currentfill}%
\pgfsetlinewidth{1.505625pt}%
\definecolor{currentstroke}{rgb}{1.000000,0.494118,0.250980}%
\pgfsetstrokecolor{currentstroke}%
\pgfsetdash{}{0pt}%
\pgfsys@defobject{currentmarker}{\pgfqpoint{-0.111111in}{-0.000000in}}{\pgfqpoint{0.111111in}{0.000000in}}{%
\pgfpathmoveto{\pgfqpoint{0.111111in}{-0.000000in}}%
\pgfpathlineto{\pgfqpoint{-0.111111in}{0.000000in}}%
\pgfusepath{stroke,fill}%
}%
\begin{pgfscope}%
\pgfsys@transformshift{4.446064in}{1.113750in}%
\pgfsys@useobject{currentmarker}{}%
\end{pgfscope}%
\end{pgfscope}%
\begin{pgfscope}%
\pgfpathrectangle{\pgfqpoint{3.186623in}{0.664139in}}{\pgfqpoint{2.015106in}{2.125222in}} %
\pgfusepath{clip}%
\pgfsetbuttcap%
\pgfsetroundjoin%
\definecolor{currentfill}{rgb}{1.000000,0.494118,0.250980}%
\pgfsetfillcolor{currentfill}%
\pgfsetlinewidth{1.505625pt}%
\definecolor{currentstroke}{rgb}{1.000000,0.494118,0.250980}%
\pgfsetstrokecolor{currentstroke}%
\pgfsetdash{}{0pt}%
\pgfsys@defobject{currentmarker}{\pgfqpoint{-0.111111in}{-0.000000in}}{\pgfqpoint{0.111111in}{0.000000in}}{%
\pgfpathmoveto{\pgfqpoint{0.111111in}{-0.000000in}}%
\pgfpathlineto{\pgfqpoint{-0.111111in}{0.000000in}}%
\pgfusepath{stroke,fill}%
}%
\begin{pgfscope}%
\pgfsys@transformshift{4.446064in}{1.132099in}%
\pgfsys@useobject{currentmarker}{}%
\end{pgfscope}%
\end{pgfscope}%
\begin{pgfscope}%
\pgfpathrectangle{\pgfqpoint{3.186623in}{0.664139in}}{\pgfqpoint{2.015106in}{2.125222in}} %
\pgfusepath{clip}%
\pgfsetbuttcap%
\pgfsetroundjoin%
\definecolor{currentfill}{rgb}{1.000000,0.694118,0.250980}%
\pgfsetfillcolor{currentfill}%
\pgfsetlinewidth{1.505625pt}%
\definecolor{currentstroke}{rgb}{1.000000,0.694118,0.250980}%
\pgfsetstrokecolor{currentstroke}%
\pgfsetdash{}{0pt}%
\pgfsys@defobject{currentmarker}{\pgfqpoint{-0.111111in}{-0.000000in}}{\pgfqpoint{0.111111in}{0.000000in}}{%
\pgfpathmoveto{\pgfqpoint{0.111111in}{-0.000000in}}%
\pgfpathlineto{\pgfqpoint{-0.111111in}{0.000000in}}%
\pgfusepath{stroke,fill}%
}%
\begin{pgfscope}%
\pgfsys@transformshift{4.949840in}{1.620455in}%
\pgfsys@useobject{currentmarker}{}%
\end{pgfscope}%
\end{pgfscope}%
\begin{pgfscope}%
\pgfpathrectangle{\pgfqpoint{3.186623in}{0.664139in}}{\pgfqpoint{2.015106in}{2.125222in}} %
\pgfusepath{clip}%
\pgfsetbuttcap%
\pgfsetroundjoin%
\definecolor{currentfill}{rgb}{1.000000,0.694118,0.250980}%
\pgfsetfillcolor{currentfill}%
\pgfsetlinewidth{1.505625pt}%
\definecolor{currentstroke}{rgb}{1.000000,0.694118,0.250980}%
\pgfsetstrokecolor{currentstroke}%
\pgfsetdash{}{0pt}%
\pgfsys@defobject{currentmarker}{\pgfqpoint{-0.111111in}{-0.000000in}}{\pgfqpoint{0.111111in}{0.000000in}}{%
\pgfpathmoveto{\pgfqpoint{0.111111in}{-0.000000in}}%
\pgfpathlineto{\pgfqpoint{-0.111111in}{0.000000in}}%
\pgfusepath{stroke,fill}%
}%
\begin{pgfscope}%
\pgfsys@transformshift{4.949840in}{1.656494in}%
\pgfsys@useobject{currentmarker}{}%
\end{pgfscope}%
\end{pgfscope}%
\begin{pgfscope}%
\pgfpathrectangle{\pgfqpoint{3.186623in}{0.664139in}}{\pgfqpoint{2.015106in}{2.125222in}} %
\pgfusepath{clip}%
\pgfsetroundcap%
\pgfsetroundjoin%
\pgfsetlinewidth{1.756562pt}%
\definecolor{currentstroke}{rgb}{0.627451,0.627451,0.643137}%
\pgfsetstrokecolor{currentstroke}%
\pgfsetdash{}{0pt}%
\pgfpathmoveto{\pgfqpoint{3.438511in}{2.333039in}}%
\pgfpathlineto{\pgfqpoint{3.438511in}{2.463248in}}%
\pgfusepath{stroke}%
\end{pgfscope}%
\begin{pgfscope}%
\pgfpathrectangle{\pgfqpoint{3.186623in}{0.664139in}}{\pgfqpoint{2.015106in}{2.125222in}} %
\pgfusepath{clip}%
\pgfsetroundcap%
\pgfsetroundjoin%
\pgfsetlinewidth{1.756562pt}%
\definecolor{currentstroke}{rgb}{0.627451,0.627451,0.643137}%
\pgfsetstrokecolor{currentstroke}%
\pgfsetdash{}{0pt}%
\pgfpathmoveto{\pgfqpoint{3.438511in}{2.463248in}}%
\pgfpathlineto{\pgfqpoint{4.446064in}{2.463248in}}%
\pgfusepath{stroke}%
\end{pgfscope}%
\begin{pgfscope}%
\pgfpathrectangle{\pgfqpoint{3.186623in}{0.664139in}}{\pgfqpoint{2.015106in}{2.125222in}} %
\pgfusepath{clip}%
\pgfsetroundcap%
\pgfsetroundjoin%
\pgfsetlinewidth{1.756562pt}%
\definecolor{currentstroke}{rgb}{0.627451,0.627451,0.643137}%
\pgfsetstrokecolor{currentstroke}%
\pgfsetdash{}{0pt}%
\pgfpathmoveto{\pgfqpoint{4.446064in}{2.463248in}}%
\pgfpathlineto{\pgfqpoint{4.446064in}{1.288349in}}%
\pgfusepath{stroke}%
\end{pgfscope}%
\begin{pgfscope}%
\pgfpathrectangle{\pgfqpoint{3.186623in}{0.664139in}}{\pgfqpoint{2.015106in}{2.125222in}} %
\pgfusepath{clip}%
\pgfsetroundcap%
\pgfsetroundjoin%
\pgfsetlinewidth{1.756562pt}%
\definecolor{currentstroke}{rgb}{0.627451,0.627451,0.643137}%
\pgfsetstrokecolor{currentstroke}%
\pgfsetdash{}{0pt}%
\pgfpathmoveto{\pgfqpoint{4.446064in}{2.541373in}}%
\pgfpathlineto{\pgfqpoint{4.446064in}{2.671581in}}%
\pgfusepath{stroke}%
\end{pgfscope}%
\begin{pgfscope}%
\pgfpathrectangle{\pgfqpoint{3.186623in}{0.664139in}}{\pgfqpoint{2.015106in}{2.125222in}} %
\pgfusepath{clip}%
\pgfsetroundcap%
\pgfsetroundjoin%
\pgfsetlinewidth{1.756562pt}%
\definecolor{currentstroke}{rgb}{0.627451,0.627451,0.643137}%
\pgfsetstrokecolor{currentstroke}%
\pgfsetdash{}{0pt}%
\pgfpathmoveto{\pgfqpoint{4.446064in}{2.671581in}}%
\pgfpathlineto{\pgfqpoint{4.949840in}{2.671581in}}%
\pgfusepath{stroke}%
\end{pgfscope}%
\begin{pgfscope}%
\pgfpathrectangle{\pgfqpoint{3.186623in}{0.664139in}}{\pgfqpoint{2.015106in}{2.125222in}} %
\pgfusepath{clip}%
\pgfsetroundcap%
\pgfsetroundjoin%
\pgfsetlinewidth{1.756562pt}%
\definecolor{currentstroke}{rgb}{0.627451,0.627451,0.643137}%
\pgfsetstrokecolor{currentstroke}%
\pgfsetdash{}{0pt}%
\pgfpathmoveto{\pgfqpoint{4.949840in}{2.671581in}}%
\pgfpathlineto{\pgfqpoint{4.949840in}{1.812744in}}%
\pgfusepath{stroke}%
\end{pgfscope}%
\begin{pgfscope}%
\pgfsetrectcap%
\pgfsetmiterjoin%
\pgfsetlinewidth{1.254687pt}%
\definecolor{currentstroke}{rgb}{0.150000,0.150000,0.150000}%
\pgfsetstrokecolor{currentstroke}%
\pgfsetdash{}{0pt}%
\pgfpathmoveto{\pgfqpoint{3.186623in}{0.664139in}}%
\pgfpathlineto{\pgfqpoint{3.186623in}{2.789361in}}%
\pgfusepath{stroke}%
\end{pgfscope}%
\begin{pgfscope}%
\pgfsetrectcap%
\pgfsetmiterjoin%
\pgfsetlinewidth{1.254687pt}%
\definecolor{currentstroke}{rgb}{0.150000,0.150000,0.150000}%
\pgfsetstrokecolor{currentstroke}%
\pgfsetdash{}{0pt}%
\pgfpathmoveto{\pgfqpoint{3.186623in}{0.664139in}}%
\pgfpathlineto{\pgfqpoint{5.201729in}{0.664139in}}%
\pgfusepath{stroke}%
\end{pgfscope}%
\begin{pgfscope}%
\definecolor{textcolor}{rgb}{0.150000,0.150000,0.150000}%
\pgfsetstrokecolor{textcolor}%
\pgfsetfillcolor{textcolor}%
\pgftext[x=4.446064in,y=1.180927in,,]{\color{textcolor}\rmfamily\fontsize{15.000000}{18.000000}\selectfont \textbf{*}}%
\end{pgfscope}%
\begin{pgfscope}%
\definecolor{textcolor}{rgb}{0.150000,0.150000,0.150000}%
\pgfsetstrokecolor{textcolor}%
\pgfsetfillcolor{textcolor}%
\pgftext[x=4.949840in,y=1.705322in,,]{\color{textcolor}\rmfamily\fontsize{15.000000}{18.000000}\selectfont \textbf{*}}%
\end{pgfscope}%
\begin{pgfscope}%
\pgfsetbuttcap%
\pgfsetmiterjoin%
\definecolor{currentfill}{rgb}{0.200000,0.427451,0.650980}%
\pgfsetfillcolor{currentfill}%
\pgfsetlinewidth{1.505625pt}%
\definecolor{currentstroke}{rgb}{0.200000,0.427451,0.650980}%
\pgfsetstrokecolor{currentstroke}%
\pgfsetdash{}{0pt}%
\pgfpathmoveto{\pgfqpoint{3.286623in}{3.417051in}}%
\pgfpathlineto{\pgfqpoint{3.397734in}{3.417051in}}%
\pgfpathlineto{\pgfqpoint{3.397734in}{3.494828in}}%
\pgfpathlineto{\pgfqpoint{3.286623in}{3.494828in}}%
\pgfpathclose%
\pgfusepath{stroke,fill}%
\end{pgfscope}%
\begin{pgfscope}%
\definecolor{textcolor}{rgb}{0.150000,0.150000,0.150000}%
\pgfsetstrokecolor{textcolor}%
\pgfsetfillcolor{textcolor}%
\pgftext[x=3.486623in,y=3.417051in,left,base]{\color{textcolor}\rmfamily\fontsize{8.000000}{9.600000}\selectfont WT + Vehicle (1513)}%
\end{pgfscope}%
\begin{pgfscope}%
\pgfsetbuttcap%
\pgfsetmiterjoin%
\definecolor{currentfill}{rgb}{0.168627,0.670588,0.494118}%
\pgfsetfillcolor{currentfill}%
\pgfsetlinewidth{1.505625pt}%
\definecolor{currentstroke}{rgb}{0.168627,0.670588,0.494118}%
\pgfsetstrokecolor{currentstroke}%
\pgfsetdash{}{0pt}%
\pgfpathmoveto{\pgfqpoint{3.286623in}{3.250411in}}%
\pgfpathlineto{\pgfqpoint{3.397734in}{3.250411in}}%
\pgfpathlineto{\pgfqpoint{3.397734in}{3.328189in}}%
\pgfpathlineto{\pgfqpoint{3.286623in}{3.328189in}}%
\pgfpathclose%
\pgfusepath{stroke,fill}%
\end{pgfscope}%
\begin{pgfscope}%
\definecolor{textcolor}{rgb}{0.150000,0.150000,0.150000}%
\pgfsetstrokecolor{textcolor}%
\pgfsetfillcolor{textcolor}%
\pgftext[x=3.486623in,y=3.250411in,left,base]{\color{textcolor}\rmfamily\fontsize{8.000000}{9.600000}\selectfont WT + TAT-GluA2\textsubscript{3Y} (815)}%
\end{pgfscope}%
\begin{pgfscope}%
\pgfsetbuttcap%
\pgfsetmiterjoin%
\definecolor{currentfill}{rgb}{1.000000,0.494118,0.250980}%
\pgfsetfillcolor{currentfill}%
\pgfsetlinewidth{1.505625pt}%
\definecolor{currentstroke}{rgb}{1.000000,0.494118,0.250980}%
\pgfsetstrokecolor{currentstroke}%
\pgfsetdash{}{0pt}%
\pgfpathmoveto{\pgfqpoint{3.286623in}{3.083771in}}%
\pgfpathlineto{\pgfqpoint{3.397734in}{3.083771in}}%
\pgfpathlineto{\pgfqpoint{3.397734in}{3.161549in}}%
\pgfpathlineto{\pgfqpoint{3.286623in}{3.161549in}}%
\pgfpathclose%
\pgfusepath{stroke,fill}%
\end{pgfscope}%
\begin{pgfscope}%
\definecolor{textcolor}{rgb}{0.150000,0.150000,0.150000}%
\pgfsetstrokecolor{textcolor}%
\pgfsetfillcolor{textcolor}%
\pgftext[x=3.486623in,y=3.083771in,left,base]{\color{textcolor}\rmfamily\fontsize{8.000000}{9.600000}\selectfont Tg + Vehicle (867)}%
\end{pgfscope}%
\begin{pgfscope}%
\pgfsetbuttcap%
\pgfsetmiterjoin%
\definecolor{currentfill}{rgb}{1.000000,0.694118,0.250980}%
\pgfsetfillcolor{currentfill}%
\pgfsetlinewidth{1.505625pt}%
\definecolor{currentstroke}{rgb}{1.000000,0.694118,0.250980}%
\pgfsetstrokecolor{currentstroke}%
\pgfsetdash{}{0pt}%
\pgfpathmoveto{\pgfqpoint{3.286623in}{2.917132in}}%
\pgfpathlineto{\pgfqpoint{3.397734in}{2.917132in}}%
\pgfpathlineto{\pgfqpoint{3.397734in}{2.994910in}}%
\pgfpathlineto{\pgfqpoint{3.286623in}{2.994910in}}%
\pgfpathclose%
\pgfusepath{stroke,fill}%
\end{pgfscope}%
\begin{pgfscope}%
\definecolor{textcolor}{rgb}{0.150000,0.150000,0.150000}%
\pgfsetstrokecolor{textcolor}%
\pgfsetfillcolor{textcolor}%
\pgftext[x=3.486623in,y=2.917132in,left,base]{\color{textcolor}\rmfamily\fontsize{8.000000}{9.600000}\selectfont Tg + TAT-GluA2\textsubscript{3Y} (1189)}%
\end{pgfscope}%
\end{pgfpicture}%
\makeatother%
\endgroup%

    \caption[Freezing information during memory test.]{Freezing information during the context memory test. This measurement represents how much information a particular cell has in a period of time about whether the mouse is freezing. Cells in \gls{tg} mice encode significantly less freezing information than the \gls{wt} groups, and \tglu{} treatment can only partially rescue the effect. \label{f.ad.freeze_info}}
\end{figure}
    

%\subsection{Deficit of freezing encoding in \gls{tg} animals is not a result of animal's position or freezing levels}

Given that CA1 cells are known to encode place, and \gls{tg} mice have deficits in spatial encoding, it is possible the differences in  spatial encoding confounded the freezing information measurement. Moreover, \gls{tg} mice have low levels of freezing during the memory test. It is possible that mice with low levels of freezing, therefore weaker memory, have deficits in freezing encoding in the cells. That is, the difference of freezing information in \gls{tg} may be the result of weaker memory instead of a specific deficit due to the expression of the transgene. To eliminate the effect of the mouse's position, we calculated the average freezing information for all mouse positions, therefore eliminating the contribution of place. To control for the potential effect of freezing level, we have included it as a covariate in the two-way \gls{anova}. 

The result is similar to the freezing information observed in Figure~\ref{f.ad.freeze_info} (Figure~\ref{f.ad.freeze_ctrl}). There was a significant interaction between \textit{Genotype} and \textit{Treatment} (F\tsb{1,4380}=97.9, p<0.001), as well as significant main effects in \textit{Genotype} (F\tsb{1,4380}=139.4, p<0.001) and \textit{Treatment} (F\tsb{1,4380}=100.3, p<0.001). Furthermore, this test revealed that the freezing level was not a significant confounding factor (T=-0.65, p=0.50) for freezing information. Again, \textit{post hoc} tests showed that Tg-Veh cells had significantly lower freezing information with position controlled (WT-Veh vs Tg-Veh, T=15.1, p<0.001), and this deficit was fully rescued by \tglu{} (Tg-\glu{} vs Tg-Veh, t=13.9, p<0.001; WT-Veh vs Tg-\glu, t=1.9, p=0.06, threshold=0.013). This result suggests that the deficit of freezing encoding in \gls{tg} mice is independent of the mouse's position in space and freezing levels during the memory test session.
\begin{figure}[h]
    %% Creator: Matplotlib, PGF backend
%%
%% To include the figure in your LaTeX document, write
%%   \input{<filename>.pgf}
%%
%% Make sure the required packages are loaded in your preamble
%%   \usepackage{pgf}
%%
%% Figures using additional raster images can only be included by \input if
%% they are in the same directory as the main LaTeX file. For loading figures
%% from other directories you can use the `import` package
%%   \usepackage{import}
%% and then include the figures with
%%   \import{<path to file>}{<filename>.pgf}
%%
%% Matplotlib used the following preamble
%%   \usepackage[utf8]{inputenc}
%%   \usepackage[T1]{fontenc}
%%   \usepackage{siunitx}
%%
\begingroup%
\makeatletter%
\begin{pgfpicture}%
\pgfpathrectangle{\pgfpointorigin}{\pgfqpoint{5.301729in}{3.713659in}}%
\pgfusepath{use as bounding box, clip}%
\begin{pgfscope}%
\pgfsetbuttcap%
\pgfsetmiterjoin%
\definecolor{currentfill}{rgb}{1.000000,1.000000,1.000000}%
\pgfsetfillcolor{currentfill}%
\pgfsetlinewidth{0.000000pt}%
\definecolor{currentstroke}{rgb}{1.000000,1.000000,1.000000}%
\pgfsetstrokecolor{currentstroke}%
\pgfsetdash{}{0pt}%
\pgfpathmoveto{\pgfqpoint{0.000000in}{0.000000in}}%
\pgfpathlineto{\pgfqpoint{5.301729in}{0.000000in}}%
\pgfpathlineto{\pgfqpoint{5.301729in}{3.713659in}}%
\pgfpathlineto{\pgfqpoint{0.000000in}{3.713659in}}%
\pgfpathclose%
\pgfusepath{fill}%
\end{pgfscope}%
\begin{pgfscope}%
\pgfsetbuttcap%
\pgfsetmiterjoin%
\definecolor{currentfill}{rgb}{1.000000,1.000000,1.000000}%
\pgfsetfillcolor{currentfill}%
\pgfsetlinewidth{0.000000pt}%
\definecolor{currentstroke}{rgb}{0.000000,0.000000,0.000000}%
\pgfsetstrokecolor{currentstroke}%
\pgfsetstrokeopacity{0.000000}%
\pgfsetdash{}{0pt}%
\pgfpathmoveto{\pgfqpoint{0.566985in}{0.687902in}}%
\pgfpathlineto{\pgfqpoint{2.582091in}{0.687902in}}%
\pgfpathlineto{\pgfqpoint{2.582091in}{3.552332in}}%
\pgfpathlineto{\pgfqpoint{0.566985in}{3.552332in}}%
\pgfpathclose%
\pgfusepath{fill}%
\end{pgfscope}%
\begin{pgfscope}%
\pgfsetbuttcap%
\pgfsetroundjoin%
\definecolor{currentfill}{rgb}{0.150000,0.150000,0.150000}%
\pgfsetfillcolor{currentfill}%
\pgfsetlinewidth{1.003750pt}%
\definecolor{currentstroke}{rgb}{0.150000,0.150000,0.150000}%
\pgfsetstrokecolor{currentstroke}%
\pgfsetdash{}{0pt}%
\pgfsys@defobject{currentmarker}{\pgfqpoint{0.000000in}{0.000000in}}{\pgfqpoint{0.000000in}{0.041667in}}{%
\pgfpathmoveto{\pgfqpoint{0.000000in}{0.000000in}}%
\pgfpathlineto{\pgfqpoint{0.000000in}{0.041667in}}%
\pgfusepath{stroke,fill}%
}%
\begin{pgfscope}%
\pgfsys@transformshift{0.566985in}{0.687902in}%
\pgfsys@useobject{currentmarker}{}%
\end{pgfscope}%
\end{pgfscope}%
\begin{pgfscope}%
\definecolor{textcolor}{rgb}{0.150000,0.150000,0.150000}%
\pgfsetstrokecolor{textcolor}%
\pgfsetfillcolor{textcolor}%
\pgftext[x=0.566985in,y=0.590680in,,top]{\color{textcolor}\rmfamily\fontsize{10.000000}{12.000000}\selectfont \(\displaystyle -0.05\)}%
\end{pgfscope}%
\begin{pgfscope}%
\pgfsetbuttcap%
\pgfsetroundjoin%
\definecolor{currentfill}{rgb}{0.150000,0.150000,0.150000}%
\pgfsetfillcolor{currentfill}%
\pgfsetlinewidth{1.003750pt}%
\definecolor{currentstroke}{rgb}{0.150000,0.150000,0.150000}%
\pgfsetstrokecolor{currentstroke}%
\pgfsetdash{}{0pt}%
\pgfsys@defobject{currentmarker}{\pgfqpoint{0.000000in}{0.000000in}}{\pgfqpoint{0.000000in}{0.041667in}}{%
\pgfpathmoveto{\pgfqpoint{0.000000in}{0.000000in}}%
\pgfpathlineto{\pgfqpoint{0.000000in}{0.041667in}}%
\pgfusepath{stroke,fill}%
}%
\begin{pgfscope}%
\pgfsys@transformshift{0.970006in}{0.687902in}%
\pgfsys@useobject{currentmarker}{}%
\end{pgfscope}%
\end{pgfscope}%
\begin{pgfscope}%
\definecolor{textcolor}{rgb}{0.150000,0.150000,0.150000}%
\pgfsetstrokecolor{textcolor}%
\pgfsetfillcolor{textcolor}%
\pgftext[x=0.970006in,y=0.590680in,,top]{\color{textcolor}\rmfamily\fontsize{10.000000}{12.000000}\selectfont \(\displaystyle 0.00\)}%
\end{pgfscope}%
\begin{pgfscope}%
\pgfsetbuttcap%
\pgfsetroundjoin%
\definecolor{currentfill}{rgb}{0.150000,0.150000,0.150000}%
\pgfsetfillcolor{currentfill}%
\pgfsetlinewidth{1.003750pt}%
\definecolor{currentstroke}{rgb}{0.150000,0.150000,0.150000}%
\pgfsetstrokecolor{currentstroke}%
\pgfsetdash{}{0pt}%
\pgfsys@defobject{currentmarker}{\pgfqpoint{0.000000in}{0.000000in}}{\pgfqpoint{0.000000in}{0.041667in}}{%
\pgfpathmoveto{\pgfqpoint{0.000000in}{0.000000in}}%
\pgfpathlineto{\pgfqpoint{0.000000in}{0.041667in}}%
\pgfusepath{stroke,fill}%
}%
\begin{pgfscope}%
\pgfsys@transformshift{1.373027in}{0.687902in}%
\pgfsys@useobject{currentmarker}{}%
\end{pgfscope}%
\end{pgfscope}%
\begin{pgfscope}%
\definecolor{textcolor}{rgb}{0.150000,0.150000,0.150000}%
\pgfsetstrokecolor{textcolor}%
\pgfsetfillcolor{textcolor}%
\pgftext[x=1.373027in,y=0.590680in,,top]{\color{textcolor}\rmfamily\fontsize{10.000000}{12.000000}\selectfont \(\displaystyle 0.05\)}%
\end{pgfscope}%
\begin{pgfscope}%
\pgfsetbuttcap%
\pgfsetroundjoin%
\definecolor{currentfill}{rgb}{0.150000,0.150000,0.150000}%
\pgfsetfillcolor{currentfill}%
\pgfsetlinewidth{1.003750pt}%
\definecolor{currentstroke}{rgb}{0.150000,0.150000,0.150000}%
\pgfsetstrokecolor{currentstroke}%
\pgfsetdash{}{0pt}%
\pgfsys@defobject{currentmarker}{\pgfqpoint{0.000000in}{0.000000in}}{\pgfqpoint{0.000000in}{0.041667in}}{%
\pgfpathmoveto{\pgfqpoint{0.000000in}{0.000000in}}%
\pgfpathlineto{\pgfqpoint{0.000000in}{0.041667in}}%
\pgfusepath{stroke,fill}%
}%
\begin{pgfscope}%
\pgfsys@transformshift{1.776048in}{0.687902in}%
\pgfsys@useobject{currentmarker}{}%
\end{pgfscope}%
\end{pgfscope}%
\begin{pgfscope}%
\definecolor{textcolor}{rgb}{0.150000,0.150000,0.150000}%
\pgfsetstrokecolor{textcolor}%
\pgfsetfillcolor{textcolor}%
\pgftext[x=1.776048in,y=0.590680in,,top]{\color{textcolor}\rmfamily\fontsize{10.000000}{12.000000}\selectfont \(\displaystyle 0.10\)}%
\end{pgfscope}%
\begin{pgfscope}%
\pgfsetbuttcap%
\pgfsetroundjoin%
\definecolor{currentfill}{rgb}{0.150000,0.150000,0.150000}%
\pgfsetfillcolor{currentfill}%
\pgfsetlinewidth{1.003750pt}%
\definecolor{currentstroke}{rgb}{0.150000,0.150000,0.150000}%
\pgfsetstrokecolor{currentstroke}%
\pgfsetdash{}{0pt}%
\pgfsys@defobject{currentmarker}{\pgfqpoint{0.000000in}{0.000000in}}{\pgfqpoint{0.000000in}{0.041667in}}{%
\pgfpathmoveto{\pgfqpoint{0.000000in}{0.000000in}}%
\pgfpathlineto{\pgfqpoint{0.000000in}{0.041667in}}%
\pgfusepath{stroke,fill}%
}%
\begin{pgfscope}%
\pgfsys@transformshift{2.179070in}{0.687902in}%
\pgfsys@useobject{currentmarker}{}%
\end{pgfscope}%
\end{pgfscope}%
\begin{pgfscope}%
\definecolor{textcolor}{rgb}{0.150000,0.150000,0.150000}%
\pgfsetstrokecolor{textcolor}%
\pgfsetfillcolor{textcolor}%
\pgftext[x=2.179070in,y=0.590680in,,top]{\color{textcolor}\rmfamily\fontsize{10.000000}{12.000000}\selectfont \(\displaystyle 0.15\)}%
\end{pgfscope}%
\begin{pgfscope}%
\pgfsetbuttcap%
\pgfsetroundjoin%
\definecolor{currentfill}{rgb}{0.150000,0.150000,0.150000}%
\pgfsetfillcolor{currentfill}%
\pgfsetlinewidth{1.003750pt}%
\definecolor{currentstroke}{rgb}{0.150000,0.150000,0.150000}%
\pgfsetstrokecolor{currentstroke}%
\pgfsetdash{}{0pt}%
\pgfsys@defobject{currentmarker}{\pgfqpoint{0.000000in}{0.000000in}}{\pgfqpoint{0.000000in}{0.041667in}}{%
\pgfpathmoveto{\pgfqpoint{0.000000in}{0.000000in}}%
\pgfpathlineto{\pgfqpoint{0.000000in}{0.041667in}}%
\pgfusepath{stroke,fill}%
}%
\begin{pgfscope}%
\pgfsys@transformshift{2.582091in}{0.687902in}%
\pgfsys@useobject{currentmarker}{}%
\end{pgfscope}%
\end{pgfscope}%
\begin{pgfscope}%
\definecolor{textcolor}{rgb}{0.150000,0.150000,0.150000}%
\pgfsetstrokecolor{textcolor}%
\pgfsetfillcolor{textcolor}%
\pgftext[x=2.582091in,y=0.590680in,,top]{\color{textcolor}\rmfamily\fontsize{10.000000}{12.000000}\selectfont \(\displaystyle 0.20\)}%
\end{pgfscope}%
\begin{pgfscope}%
\definecolor{textcolor}{rgb}{0.150000,0.150000,0.150000}%
\pgfsetstrokecolor{textcolor}%
\pgfsetfillcolor{textcolor}%
\pgftext[x=0.720710in,y=0.294405in,left,base]{\color{textcolor}\rmfamily\fontsize{10.000000}{12.000000}\selectfont \textbf{Freezing info.~\(\displaystyle |\) Position }}%
\end{pgfscope}%
\begin{pgfscope}%
\definecolor{textcolor}{rgb}{0.150000,0.150000,0.150000}%
\pgfsetstrokecolor{textcolor}%
\pgfsetfillcolor{textcolor}%
\pgftext[x=1.117608in,y=0.134714in,left,base]{\color{textcolor}\rmfamily\fontsize{10.000000}{12.000000}\selectfont \textbf{ (bits/frame)}}%
\end{pgfscope}%
\begin{pgfscope}%
\pgfsetbuttcap%
\pgfsetroundjoin%
\definecolor{currentfill}{rgb}{0.150000,0.150000,0.150000}%
\pgfsetfillcolor{currentfill}%
\pgfsetlinewidth{1.003750pt}%
\definecolor{currentstroke}{rgb}{0.150000,0.150000,0.150000}%
\pgfsetstrokecolor{currentstroke}%
\pgfsetdash{}{0pt}%
\pgfsys@defobject{currentmarker}{\pgfqpoint{0.000000in}{0.000000in}}{\pgfqpoint{0.041667in}{0.000000in}}{%
\pgfpathmoveto{\pgfqpoint{0.000000in}{0.000000in}}%
\pgfpathlineto{\pgfqpoint{0.041667in}{0.000000in}}%
\pgfusepath{stroke,fill}%
}%
\begin{pgfscope}%
\pgfsys@transformshift{0.566985in}{0.687902in}%
\pgfsys@useobject{currentmarker}{}%
\end{pgfscope}%
\end{pgfscope}%
\begin{pgfscope}%
\definecolor{textcolor}{rgb}{0.150000,0.150000,0.150000}%
\pgfsetstrokecolor{textcolor}%
\pgfsetfillcolor{textcolor}%
\pgftext[x=0.469762in,y=0.687902in,right,]{\color{textcolor}\rmfamily\fontsize{10.000000}{12.000000}\selectfont \(\displaystyle 0.0\)}%
\end{pgfscope}%
\begin{pgfscope}%
\pgfsetbuttcap%
\pgfsetroundjoin%
\definecolor{currentfill}{rgb}{0.150000,0.150000,0.150000}%
\pgfsetfillcolor{currentfill}%
\pgfsetlinewidth{1.003750pt}%
\definecolor{currentstroke}{rgb}{0.150000,0.150000,0.150000}%
\pgfsetstrokecolor{currentstroke}%
\pgfsetdash{}{0pt}%
\pgfsys@defobject{currentmarker}{\pgfqpoint{0.000000in}{0.000000in}}{\pgfqpoint{0.041667in}{0.000000in}}{%
\pgfpathmoveto{\pgfqpoint{0.000000in}{0.000000in}}%
\pgfpathlineto{\pgfqpoint{0.041667in}{0.000000in}}%
\pgfusepath{stroke,fill}%
}%
\begin{pgfscope}%
\pgfsys@transformshift{0.566985in}{1.260788in}%
\pgfsys@useobject{currentmarker}{}%
\end{pgfscope}%
\end{pgfscope}%
\begin{pgfscope}%
\definecolor{textcolor}{rgb}{0.150000,0.150000,0.150000}%
\pgfsetstrokecolor{textcolor}%
\pgfsetfillcolor{textcolor}%
\pgftext[x=0.469762in,y=1.260788in,right,]{\color{textcolor}\rmfamily\fontsize{10.000000}{12.000000}\selectfont \(\displaystyle 0.2\)}%
\end{pgfscope}%
\begin{pgfscope}%
\pgfsetbuttcap%
\pgfsetroundjoin%
\definecolor{currentfill}{rgb}{0.150000,0.150000,0.150000}%
\pgfsetfillcolor{currentfill}%
\pgfsetlinewidth{1.003750pt}%
\definecolor{currentstroke}{rgb}{0.150000,0.150000,0.150000}%
\pgfsetstrokecolor{currentstroke}%
\pgfsetdash{}{0pt}%
\pgfsys@defobject{currentmarker}{\pgfqpoint{0.000000in}{0.000000in}}{\pgfqpoint{0.041667in}{0.000000in}}{%
\pgfpathmoveto{\pgfqpoint{0.000000in}{0.000000in}}%
\pgfpathlineto{\pgfqpoint{0.041667in}{0.000000in}}%
\pgfusepath{stroke,fill}%
}%
\begin{pgfscope}%
\pgfsys@transformshift{0.566985in}{1.833674in}%
\pgfsys@useobject{currentmarker}{}%
\end{pgfscope}%
\end{pgfscope}%
\begin{pgfscope}%
\definecolor{textcolor}{rgb}{0.150000,0.150000,0.150000}%
\pgfsetstrokecolor{textcolor}%
\pgfsetfillcolor{textcolor}%
\pgftext[x=0.469762in,y=1.833674in,right,]{\color{textcolor}\rmfamily\fontsize{10.000000}{12.000000}\selectfont \(\displaystyle 0.4\)}%
\end{pgfscope}%
\begin{pgfscope}%
\pgfsetbuttcap%
\pgfsetroundjoin%
\definecolor{currentfill}{rgb}{0.150000,0.150000,0.150000}%
\pgfsetfillcolor{currentfill}%
\pgfsetlinewidth{1.003750pt}%
\definecolor{currentstroke}{rgb}{0.150000,0.150000,0.150000}%
\pgfsetstrokecolor{currentstroke}%
\pgfsetdash{}{0pt}%
\pgfsys@defobject{currentmarker}{\pgfqpoint{0.000000in}{0.000000in}}{\pgfqpoint{0.041667in}{0.000000in}}{%
\pgfpathmoveto{\pgfqpoint{0.000000in}{0.000000in}}%
\pgfpathlineto{\pgfqpoint{0.041667in}{0.000000in}}%
\pgfusepath{stroke,fill}%
}%
\begin{pgfscope}%
\pgfsys@transformshift{0.566985in}{2.406560in}%
\pgfsys@useobject{currentmarker}{}%
\end{pgfscope}%
\end{pgfscope}%
\begin{pgfscope}%
\definecolor{textcolor}{rgb}{0.150000,0.150000,0.150000}%
\pgfsetstrokecolor{textcolor}%
\pgfsetfillcolor{textcolor}%
\pgftext[x=0.469762in,y=2.406560in,right,]{\color{textcolor}\rmfamily\fontsize{10.000000}{12.000000}\selectfont \(\displaystyle 0.6\)}%
\end{pgfscope}%
\begin{pgfscope}%
\pgfsetbuttcap%
\pgfsetroundjoin%
\definecolor{currentfill}{rgb}{0.150000,0.150000,0.150000}%
\pgfsetfillcolor{currentfill}%
\pgfsetlinewidth{1.003750pt}%
\definecolor{currentstroke}{rgb}{0.150000,0.150000,0.150000}%
\pgfsetstrokecolor{currentstroke}%
\pgfsetdash{}{0pt}%
\pgfsys@defobject{currentmarker}{\pgfqpoint{0.000000in}{0.000000in}}{\pgfqpoint{0.041667in}{0.000000in}}{%
\pgfpathmoveto{\pgfqpoint{0.000000in}{0.000000in}}%
\pgfpathlineto{\pgfqpoint{0.041667in}{0.000000in}}%
\pgfusepath{stroke,fill}%
}%
\begin{pgfscope}%
\pgfsys@transformshift{0.566985in}{2.979446in}%
\pgfsys@useobject{currentmarker}{}%
\end{pgfscope}%
\end{pgfscope}%
\begin{pgfscope}%
\definecolor{textcolor}{rgb}{0.150000,0.150000,0.150000}%
\pgfsetstrokecolor{textcolor}%
\pgfsetfillcolor{textcolor}%
\pgftext[x=0.469762in,y=2.979446in,right,]{\color{textcolor}\rmfamily\fontsize{10.000000}{12.000000}\selectfont \(\displaystyle 0.8\)}%
\end{pgfscope}%
\begin{pgfscope}%
\pgfsetbuttcap%
\pgfsetroundjoin%
\definecolor{currentfill}{rgb}{0.150000,0.150000,0.150000}%
\pgfsetfillcolor{currentfill}%
\pgfsetlinewidth{1.003750pt}%
\definecolor{currentstroke}{rgb}{0.150000,0.150000,0.150000}%
\pgfsetstrokecolor{currentstroke}%
\pgfsetdash{}{0pt}%
\pgfsys@defobject{currentmarker}{\pgfqpoint{0.000000in}{0.000000in}}{\pgfqpoint{0.041667in}{0.000000in}}{%
\pgfpathmoveto{\pgfqpoint{0.000000in}{0.000000in}}%
\pgfpathlineto{\pgfqpoint{0.041667in}{0.000000in}}%
\pgfusepath{stroke,fill}%
}%
\begin{pgfscope}%
\pgfsys@transformshift{0.566985in}{3.552332in}%
\pgfsys@useobject{currentmarker}{}%
\end{pgfscope}%
\end{pgfscope}%
\begin{pgfscope}%
\definecolor{textcolor}{rgb}{0.150000,0.150000,0.150000}%
\pgfsetstrokecolor{textcolor}%
\pgfsetfillcolor{textcolor}%
\pgftext[x=0.469762in,y=3.552332in,right,]{\color{textcolor}\rmfamily\fontsize{10.000000}{12.000000}\selectfont \(\displaystyle 1.0\)}%
\end{pgfscope}%
\begin{pgfscope}%
\definecolor{textcolor}{rgb}{0.150000,0.150000,0.150000}%
\pgfsetstrokecolor{textcolor}%
\pgfsetfillcolor{textcolor}%
\pgftext[x=0.222848in,y=2.120117in,,bottom,rotate=90.000000]{\color{textcolor}\rmfamily\fontsize{10.000000}{12.000000}\selectfont \textbf{Cumulative porportion}}%
\end{pgfscope}%
\begin{pgfscope}%
\pgfpathrectangle{\pgfqpoint{0.566985in}{0.687902in}}{\pgfqpoint{2.015106in}{2.864429in}} %
\pgfusepath{clip}%
\pgfsetroundcap%
\pgfsetroundjoin%
\pgfsetlinewidth{1.003750pt}%
\definecolor{currentstroke}{rgb}{0.200000,0.427451,0.650980}%
\pgfsetstrokecolor{currentstroke}%
\pgfsetdash{}{0pt}%
\pgfpathmoveto{\pgfqpoint{0.765447in}{0.689796in}}%
\pgfpathlineto{\pgfqpoint{0.797513in}{0.693582in}}%
\pgfpathlineto{\pgfqpoint{0.829579in}{0.703048in}}%
\pgfpathlineto{\pgfqpoint{0.861645in}{0.729553in}}%
\pgfpathlineto{\pgfqpoint{0.893711in}{0.818534in}}%
\pgfpathlineto{\pgfqpoint{0.925777in}{0.907515in}}%
\pgfpathlineto{\pgfqpoint{0.957843in}{1.087370in}}%
\pgfpathlineto{\pgfqpoint{0.989910in}{1.235041in}}%
\pgfpathlineto{\pgfqpoint{1.021976in}{1.399750in}}%
\pgfpathlineto{\pgfqpoint{1.054042in}{1.551207in}}%
\pgfpathlineto{\pgfqpoint{1.086108in}{1.708344in}}%
\pgfpathlineto{\pgfqpoint{1.118174in}{1.886305in}}%
\pgfpathlineto{\pgfqpoint{1.150240in}{2.058588in}}%
\pgfpathlineto{\pgfqpoint{1.182306in}{2.183540in}}%
\pgfpathlineto{\pgfqpoint{1.214372in}{2.327424in}}%
\pgfpathlineto{\pgfqpoint{1.246438in}{2.469415in}}%
\pgfpathlineto{\pgfqpoint{1.278504in}{2.618978in}}%
\pgfpathlineto{\pgfqpoint{1.310570in}{2.766649in}}%
\pgfpathlineto{\pgfqpoint{1.342636in}{2.882135in}}%
\pgfpathlineto{\pgfqpoint{1.374702in}{2.986262in}}%
\pgfpathlineto{\pgfqpoint{1.406768in}{3.071456in}}%
\pgfpathlineto{\pgfqpoint{1.438834in}{3.132039in}}%
\pgfpathlineto{\pgfqpoint{1.470901in}{3.183156in}}%
\pgfpathlineto{\pgfqpoint{1.502967in}{3.232379in}}%
\pgfpathlineto{\pgfqpoint{1.535033in}{3.275923in}}%
\pgfpathlineto{\pgfqpoint{1.567099in}{3.302428in}}%
\pgfpathlineto{\pgfqpoint{1.599165in}{3.332719in}}%
\pgfpathlineto{\pgfqpoint{1.631231in}{3.370584in}}%
\pgfpathlineto{\pgfqpoint{1.663297in}{3.389516in}}%
\pgfpathlineto{\pgfqpoint{1.695363in}{3.408448in}}%
\pgfpathlineto{\pgfqpoint{1.727429in}{3.436846in}}%
\pgfpathlineto{\pgfqpoint{1.759495in}{3.465244in}}%
\pgfpathlineto{\pgfqpoint{1.791561in}{3.478497in}}%
\pgfpathlineto{\pgfqpoint{1.823627in}{3.491749in}}%
\pgfpathlineto{\pgfqpoint{1.855693in}{3.493642in}}%
\pgfpathlineto{\pgfqpoint{1.887759in}{3.497429in}}%
\pgfpathlineto{\pgfqpoint{1.919825in}{3.516361in}}%
\pgfpathlineto{\pgfqpoint{1.951891in}{3.525827in}}%
\pgfpathlineto{\pgfqpoint{1.983958in}{3.529613in}}%
\pgfpathlineto{\pgfqpoint{2.016024in}{3.535293in}}%
\pgfpathlineto{\pgfqpoint{2.048090in}{3.540973in}}%
\pgfpathlineto{\pgfqpoint{2.080156in}{3.542866in}}%
\pgfpathlineto{\pgfqpoint{2.112222in}{3.542866in}}%
\pgfpathlineto{\pgfqpoint{2.144288in}{3.544759in}}%
\pgfpathlineto{\pgfqpoint{2.176354in}{3.548545in}}%
\pgfpathlineto{\pgfqpoint{2.208420in}{3.550439in}}%
\pgfpathlineto{\pgfqpoint{2.240486in}{3.550439in}}%
\pgfpathlineto{\pgfqpoint{2.272552in}{3.550439in}}%
\pgfpathlineto{\pgfqpoint{2.304618in}{3.550439in}}%
\pgfpathlineto{\pgfqpoint{2.336684in}{3.552332in}}%
\pgfusepath{stroke}%
\end{pgfscope}%
\begin{pgfscope}%
\pgfpathrectangle{\pgfqpoint{0.566985in}{0.687902in}}{\pgfqpoint{2.015106in}{2.864429in}} %
\pgfusepath{clip}%
\pgfsetroundcap%
\pgfsetroundjoin%
\pgfsetlinewidth{1.003750pt}%
\definecolor{currentstroke}{rgb}{0.168627,0.670588,0.494118}%
\pgfsetstrokecolor{currentstroke}%
\pgfsetdash{}{0pt}%
\pgfpathmoveto{\pgfqpoint{0.763592in}{0.691417in}}%
\pgfpathlineto{\pgfqpoint{0.793269in}{0.691417in}}%
\pgfpathlineto{\pgfqpoint{0.822946in}{0.705476in}}%
\pgfpathlineto{\pgfqpoint{0.852624in}{0.747651in}}%
\pgfpathlineto{\pgfqpoint{0.882301in}{0.800371in}}%
\pgfpathlineto{\pgfqpoint{0.911978in}{0.877693in}}%
\pgfpathlineto{\pgfqpoint{0.941655in}{1.042881in}}%
\pgfpathlineto{\pgfqpoint{0.971332in}{1.208069in}}%
\pgfpathlineto{\pgfqpoint{1.001009in}{1.401374in}}%
\pgfpathlineto{\pgfqpoint{1.030686in}{1.650913in}}%
\pgfpathlineto{\pgfqpoint{1.060363in}{1.833674in}}%
\pgfpathlineto{\pgfqpoint{1.090040in}{1.995348in}}%
\pgfpathlineto{\pgfqpoint{1.119718in}{2.121874in}}%
\pgfpathlineto{\pgfqpoint{1.149395in}{2.209740in}}%
\pgfpathlineto{\pgfqpoint{1.179072in}{2.308150in}}%
\pgfpathlineto{\pgfqpoint{1.208749in}{2.403045in}}%
\pgfpathlineto{\pgfqpoint{1.238426in}{2.508485in}}%
\pgfpathlineto{\pgfqpoint{1.268103in}{2.585807in}}%
\pgfpathlineto{\pgfqpoint{1.297780in}{2.677187in}}%
\pgfpathlineto{\pgfqpoint{1.327457in}{2.782626in}}%
\pgfpathlineto{\pgfqpoint{1.357134in}{2.870492in}}%
\pgfpathlineto{\pgfqpoint{1.386812in}{2.958358in}}%
\pgfpathlineto{\pgfqpoint{1.416489in}{3.025136in}}%
\pgfpathlineto{\pgfqpoint{1.446166in}{3.084885in}}%
\pgfpathlineto{\pgfqpoint{1.475843in}{3.151663in}}%
\pgfpathlineto{\pgfqpoint{1.505520in}{3.211412in}}%
\pgfpathlineto{\pgfqpoint{1.535197in}{3.250073in}}%
\pgfpathlineto{\pgfqpoint{1.564874in}{3.313337in}}%
\pgfpathlineto{\pgfqpoint{1.594551in}{3.337939in}}%
\pgfpathlineto{\pgfqpoint{1.624229in}{3.404717in}}%
\pgfpathlineto{\pgfqpoint{1.653906in}{3.422290in}}%
\pgfpathlineto{\pgfqpoint{1.683583in}{3.453922in}}%
\pgfpathlineto{\pgfqpoint{1.713260in}{3.467981in}}%
\pgfpathlineto{\pgfqpoint{1.742937in}{3.482039in}}%
\pgfpathlineto{\pgfqpoint{1.772614in}{3.496098in}}%
\pgfpathlineto{\pgfqpoint{1.802291in}{3.506642in}}%
\pgfpathlineto{\pgfqpoint{1.831968in}{3.513671in}}%
\pgfpathlineto{\pgfqpoint{1.861645in}{3.517185in}}%
\pgfpathlineto{\pgfqpoint{1.891323in}{3.524215in}}%
\pgfpathlineto{\pgfqpoint{1.921000in}{3.527729in}}%
\pgfpathlineto{\pgfqpoint{1.950677in}{3.534759in}}%
\pgfpathlineto{\pgfqpoint{1.980354in}{3.538273in}}%
\pgfpathlineto{\pgfqpoint{2.010031in}{3.538273in}}%
\pgfpathlineto{\pgfqpoint{2.039708in}{3.545303in}}%
\pgfpathlineto{\pgfqpoint{2.069385in}{3.545303in}}%
\pgfpathlineto{\pgfqpoint{2.099062in}{3.545303in}}%
\pgfpathlineto{\pgfqpoint{2.128740in}{3.545303in}}%
\pgfpathlineto{\pgfqpoint{2.158417in}{3.545303in}}%
\pgfpathlineto{\pgfqpoint{2.188094in}{3.548817in}}%
\pgfpathlineto{\pgfqpoint{2.217771in}{3.552332in}}%
\pgfusepath{stroke}%
\end{pgfscope}%
\begin{pgfscope}%
\pgfpathrectangle{\pgfqpoint{0.566985in}{0.687902in}}{\pgfqpoint{2.015106in}{2.864429in}} %
\pgfusepath{clip}%
\pgfsetroundcap%
\pgfsetroundjoin%
\pgfsetlinewidth{1.003750pt}%
\definecolor{currentstroke}{rgb}{1.000000,0.494118,0.250980}%
\pgfsetstrokecolor{currentstroke}%
\pgfsetdash{}{0pt}%
\pgfpathmoveto{\pgfqpoint{0.746993in}{0.694510in}}%
\pgfpathlineto{\pgfqpoint{0.765671in}{0.694510in}}%
\pgfpathlineto{\pgfqpoint{0.784349in}{0.697814in}}%
\pgfpathlineto{\pgfqpoint{0.803028in}{0.697814in}}%
\pgfpathlineto{\pgfqpoint{0.821706in}{0.704422in}}%
\pgfpathlineto{\pgfqpoint{0.840384in}{0.717637in}}%
\pgfpathlineto{\pgfqpoint{0.859063in}{0.737460in}}%
\pgfpathlineto{\pgfqpoint{0.877741in}{0.763891in}}%
\pgfpathlineto{\pgfqpoint{0.896420in}{0.783714in}}%
\pgfpathlineto{\pgfqpoint{0.915098in}{0.796929in}}%
\pgfpathlineto{\pgfqpoint{0.933776in}{0.886133in}}%
\pgfpathlineto{\pgfqpoint{0.952455in}{1.048021in}}%
\pgfpathlineto{\pgfqpoint{0.971133in}{1.147136in}}%
\pgfpathlineto{\pgfqpoint{0.989811in}{1.246251in}}%
\pgfpathlineto{\pgfqpoint{1.008490in}{1.414747in}}%
\pgfpathlineto{\pgfqpoint{1.027168in}{1.741827in}}%
\pgfpathlineto{\pgfqpoint{1.045846in}{1.996223in}}%
\pgfpathlineto{\pgfqpoint{1.064525in}{2.224188in}}%
\pgfpathlineto{\pgfqpoint{1.083203in}{2.488495in}}%
\pgfpathlineto{\pgfqpoint{1.101881in}{2.706549in}}%
\pgfpathlineto{\pgfqpoint{1.120560in}{2.881652in}}%
\pgfpathlineto{\pgfqpoint{1.139238in}{3.020414in}}%
\pgfpathlineto{\pgfqpoint{1.157916in}{3.126136in}}%
\pgfpathlineto{\pgfqpoint{1.176595in}{3.271505in}}%
\pgfpathlineto{\pgfqpoint{1.195273in}{3.321063in}}%
\pgfpathlineto{\pgfqpoint{1.213951in}{3.367317in}}%
\pgfpathlineto{\pgfqpoint{1.232630in}{3.426786in}}%
\pgfpathlineto{\pgfqpoint{1.251308in}{3.446609in}}%
\pgfpathlineto{\pgfqpoint{1.269987in}{3.473040in}}%
\pgfpathlineto{\pgfqpoint{1.288665in}{3.482951in}}%
\pgfpathlineto{\pgfqpoint{1.307343in}{3.496167in}}%
\pgfpathlineto{\pgfqpoint{1.326022in}{3.502774in}}%
\pgfpathlineto{\pgfqpoint{1.344700in}{3.515990in}}%
\pgfpathlineto{\pgfqpoint{1.363378in}{3.519293in}}%
\pgfpathlineto{\pgfqpoint{1.382057in}{3.529205in}}%
\pgfpathlineto{\pgfqpoint{1.400735in}{3.539116in}}%
\pgfpathlineto{\pgfqpoint{1.419413in}{3.539116in}}%
\pgfpathlineto{\pgfqpoint{1.438092in}{3.542420in}}%
\pgfpathlineto{\pgfqpoint{1.456770in}{3.542420in}}%
\pgfpathlineto{\pgfqpoint{1.475448in}{3.542420in}}%
\pgfpathlineto{\pgfqpoint{1.494127in}{3.542420in}}%
\pgfpathlineto{\pgfqpoint{1.512805in}{3.542420in}}%
\pgfpathlineto{\pgfqpoint{1.531483in}{3.542420in}}%
\pgfpathlineto{\pgfqpoint{1.550162in}{3.545724in}}%
\pgfpathlineto{\pgfqpoint{1.568840in}{3.545724in}}%
\pgfpathlineto{\pgfqpoint{1.587518in}{3.545724in}}%
\pgfpathlineto{\pgfqpoint{1.606197in}{3.545724in}}%
\pgfpathlineto{\pgfqpoint{1.624875in}{3.549028in}}%
\pgfpathlineto{\pgfqpoint{1.643554in}{3.549028in}}%
\pgfpathlineto{\pgfqpoint{1.662232in}{3.552332in}}%
\pgfusepath{stroke}%
\end{pgfscope}%
\begin{pgfscope}%
\pgfpathrectangle{\pgfqpoint{0.566985in}{0.687902in}}{\pgfqpoint{2.015106in}{2.864429in}} %
\pgfusepath{clip}%
\pgfsetroundcap%
\pgfsetroundjoin%
\pgfsetlinewidth{1.003750pt}%
\definecolor{currentstroke}{rgb}{1.000000,0.694118,0.250980}%
\pgfsetstrokecolor{currentstroke}%
\pgfsetdash{}{0pt}%
\pgfpathmoveto{\pgfqpoint{0.779181in}{0.697539in}}%
\pgfpathlineto{\pgfqpoint{0.806883in}{0.707175in}}%
\pgfpathlineto{\pgfqpoint{0.834585in}{0.719221in}}%
\pgfpathlineto{\pgfqpoint{0.862287in}{0.764994in}}%
\pgfpathlineto{\pgfqpoint{0.889989in}{0.856540in}}%
\pgfpathlineto{\pgfqpoint{0.917691in}{0.996268in}}%
\pgfpathlineto{\pgfqpoint{0.945393in}{1.157678in}}%
\pgfpathlineto{\pgfqpoint{0.973095in}{1.307043in}}%
\pgfpathlineto{\pgfqpoint{1.000796in}{1.466044in}}%
\pgfpathlineto{\pgfqpoint{1.028498in}{1.605773in}}%
\pgfpathlineto{\pgfqpoint{1.056200in}{1.735864in}}%
\pgfpathlineto{\pgfqpoint{1.083902in}{1.894866in}}%
\pgfpathlineto{\pgfqpoint{1.111604in}{2.015321in}}%
\pgfpathlineto{\pgfqpoint{1.139306in}{2.200822in}}%
\pgfpathlineto{\pgfqpoint{1.167008in}{2.318869in}}%
\pgfpathlineto{\pgfqpoint{1.194710in}{2.410415in}}%
\pgfpathlineto{\pgfqpoint{1.222412in}{2.523643in}}%
\pgfpathlineto{\pgfqpoint{1.250114in}{2.622416in}}%
\pgfpathlineto{\pgfqpoint{1.277816in}{2.706735in}}%
\pgfpathlineto{\pgfqpoint{1.305518in}{2.754917in}}%
\pgfpathlineto{\pgfqpoint{1.333220in}{2.841645in}}%
\pgfpathlineto{\pgfqpoint{1.360922in}{2.930782in}}%
\pgfpathlineto{\pgfqpoint{1.388624in}{3.003055in}}%
\pgfpathlineto{\pgfqpoint{1.416325in}{3.060874in}}%
\pgfpathlineto{\pgfqpoint{1.444027in}{3.128329in}}%
\pgfpathlineto{\pgfqpoint{1.471729in}{3.200602in}}%
\pgfpathlineto{\pgfqpoint{1.499431in}{3.248784in}}%
\pgfpathlineto{\pgfqpoint{1.527133in}{3.277694in}}%
\pgfpathlineto{\pgfqpoint{1.554835in}{3.316239in}}%
\pgfpathlineto{\pgfqpoint{1.582537in}{3.364421in}}%
\pgfpathlineto{\pgfqpoint{1.610239in}{3.390922in}}%
\pgfpathlineto{\pgfqpoint{1.637941in}{3.434286in}}%
\pgfpathlineto{\pgfqpoint{1.665643in}{3.451149in}}%
\pgfpathlineto{\pgfqpoint{1.693345in}{3.465604in}}%
\pgfpathlineto{\pgfqpoint{1.721047in}{3.480059in}}%
\pgfpathlineto{\pgfqpoint{1.748749in}{3.487286in}}%
\pgfpathlineto{\pgfqpoint{1.776451in}{3.508968in}}%
\pgfpathlineto{\pgfqpoint{1.804153in}{3.516195in}}%
\pgfpathlineto{\pgfqpoint{1.831854in}{3.525832in}}%
\pgfpathlineto{\pgfqpoint{1.859556in}{3.528241in}}%
\pgfpathlineto{\pgfqpoint{1.887258in}{3.530650in}}%
\pgfpathlineto{\pgfqpoint{1.914960in}{3.540286in}}%
\pgfpathlineto{\pgfqpoint{1.942662in}{3.540286in}}%
\pgfpathlineto{\pgfqpoint{1.970364in}{3.545105in}}%
\pgfpathlineto{\pgfqpoint{1.998066in}{3.545105in}}%
\pgfpathlineto{\pgfqpoint{2.025768in}{3.545105in}}%
\pgfpathlineto{\pgfqpoint{2.053470in}{3.549923in}}%
\pgfpathlineto{\pgfqpoint{2.081172in}{3.549923in}}%
\pgfpathlineto{\pgfqpoint{2.108874in}{3.549923in}}%
\pgfpathlineto{\pgfqpoint{2.136576in}{3.552332in}}%
\pgfusepath{stroke}%
\end{pgfscope}%
\begin{pgfscope}%
\pgfsetrectcap%
\pgfsetmiterjoin%
\pgfsetlinewidth{1.254687pt}%
\definecolor{currentstroke}{rgb}{0.150000,0.150000,0.150000}%
\pgfsetstrokecolor{currentstroke}%
\pgfsetdash{}{0pt}%
\pgfpathmoveto{\pgfqpoint{0.566985in}{0.687902in}}%
\pgfpathlineto{\pgfqpoint{0.566985in}{3.552332in}}%
\pgfusepath{stroke}%
\end{pgfscope}%
\begin{pgfscope}%
\pgfsetrectcap%
\pgfsetmiterjoin%
\pgfsetlinewidth{1.254687pt}%
\definecolor{currentstroke}{rgb}{0.150000,0.150000,0.150000}%
\pgfsetstrokecolor{currentstroke}%
\pgfsetdash{}{0pt}%
\pgfpathmoveto{\pgfqpoint{0.566985in}{0.687902in}}%
\pgfpathlineto{\pgfqpoint{2.582091in}{0.687902in}}%
\pgfusepath{stroke}%
\end{pgfscope}%
\begin{pgfscope}%
\pgfsetbuttcap%
\pgfsetmiterjoin%
\definecolor{currentfill}{rgb}{1.000000,1.000000,1.000000}%
\pgfsetfillcolor{currentfill}%
\pgfsetlinewidth{0.000000pt}%
\definecolor{currentstroke}{rgb}{0.000000,0.000000,0.000000}%
\pgfsetstrokecolor{currentstroke}%
\pgfsetstrokeopacity{0.000000}%
\pgfsetdash{}{0pt}%
\pgfpathmoveto{\pgfqpoint{3.186623in}{0.687902in}}%
\pgfpathlineto{\pgfqpoint{5.201729in}{0.687902in}}%
\pgfpathlineto{\pgfqpoint{5.201729in}{2.813124in}}%
\pgfpathlineto{\pgfqpoint{3.186623in}{2.813124in}}%
\pgfpathclose%
\pgfusepath{fill}%
\end{pgfscope}%
\begin{pgfscope}%
\pgfsetroundcap%
\pgfsetroundjoin%
\pgfsetlinewidth{1.003750pt}%
\definecolor{currentstroke}{rgb}{0.200000,0.427451,0.650980}%
\pgfsetstrokecolor{currentstroke}%
\pgfsetdash{}{0pt}%
\pgfpathmoveto{\pgfqpoint{3.085112in}{3.479703in}}%
\pgfpathlineto{\pgfqpoint{3.196223in}{3.479703in}}%
\pgfusepath{stroke}%
\end{pgfscope}%
\begin{pgfscope}%
\definecolor{textcolor}{rgb}{1.000000,1.000000,1.000000}%
\pgfsetstrokecolor{textcolor}%
\pgfsetfillcolor{textcolor}%
\pgftext[x=3.285112in,y=3.440814in,left,base]{\color{textcolor}\rmfamily\fontsize{8.000000}{9.600000}\selectfont WT + Vehicle (1513)}%
\end{pgfscope}%
\begin{pgfscope}%
\pgfsetroundcap%
\pgfsetroundjoin%
\pgfsetlinewidth{1.003750pt}%
\definecolor{currentstroke}{rgb}{0.168627,0.670588,0.494118}%
\pgfsetstrokecolor{currentstroke}%
\pgfsetdash{}{0pt}%
\pgfpathmoveto{\pgfqpoint{3.085112in}{3.313063in}}%
\pgfpathlineto{\pgfqpoint{3.196223in}{3.313063in}}%
\pgfusepath{stroke}%
\end{pgfscope}%
\begin{pgfscope}%
\definecolor{textcolor}{rgb}{1.000000,1.000000,1.000000}%
\pgfsetstrokecolor{textcolor}%
\pgfsetfillcolor{textcolor}%
\pgftext[x=3.285112in,y=3.274174in,left,base]{\color{textcolor}\rmfamily\fontsize{8.000000}{9.600000}\selectfont WT + TAT-GluA2\textsubscript{3Y} (815)}%
\end{pgfscope}%
\begin{pgfscope}%
\pgfsetroundcap%
\pgfsetroundjoin%
\pgfsetlinewidth{1.003750pt}%
\definecolor{currentstroke}{rgb}{1.000000,0.494118,0.250980}%
\pgfsetstrokecolor{currentstroke}%
\pgfsetdash{}{0pt}%
\pgfpathmoveto{\pgfqpoint{3.085112in}{3.146424in}}%
\pgfpathlineto{\pgfqpoint{3.196223in}{3.146424in}}%
\pgfusepath{stroke}%
\end{pgfscope}%
\begin{pgfscope}%
\definecolor{textcolor}{rgb}{1.000000,1.000000,1.000000}%
\pgfsetstrokecolor{textcolor}%
\pgfsetfillcolor{textcolor}%
\pgftext[x=3.285112in,y=3.107535in,left,base]{\color{textcolor}\rmfamily\fontsize{8.000000}{9.600000}\selectfont Tg + Vehicle (867)}%
\end{pgfscope}%
\begin{pgfscope}%
\pgfsetroundcap%
\pgfsetroundjoin%
\pgfsetlinewidth{1.003750pt}%
\definecolor{currentstroke}{rgb}{1.000000,0.694118,0.250980}%
\pgfsetstrokecolor{currentstroke}%
\pgfsetdash{}{0pt}%
\pgfpathmoveto{\pgfqpoint{3.085112in}{2.979784in}}%
\pgfpathlineto{\pgfqpoint{3.196223in}{2.979784in}}%
\pgfusepath{stroke}%
\end{pgfscope}%
\begin{pgfscope}%
\definecolor{textcolor}{rgb}{1.000000,1.000000,1.000000}%
\pgfsetstrokecolor{textcolor}%
\pgfsetfillcolor{textcolor}%
\pgftext[x=3.285112in,y=2.940895in,left,base]{\color{textcolor}\rmfamily\fontsize{8.000000}{9.600000}\selectfont Tg + TAT-GluA2\textsubscript{3Y} (1189)}%
\end{pgfscope}%
\begin{pgfscope}%
\pgfsetroundcap%
\pgfsetroundjoin%
\pgfsetlinewidth{1.003750pt}%
\definecolor{currentstroke}{rgb}{0.200000,0.427451,0.650980}%
\pgfsetstrokecolor{currentstroke}%
\pgfsetdash{}{0pt}%
\pgfpathmoveto{\pgfqpoint{3.085112in}{3.479703in}}%
\pgfpathlineto{\pgfqpoint{3.196223in}{3.479703in}}%
\pgfusepath{stroke}%
\end{pgfscope}%
\begin{pgfscope}%
\definecolor{textcolor}{rgb}{1.000000,1.000000,1.000000}%
\pgfsetstrokecolor{textcolor}%
\pgfsetfillcolor{textcolor}%
\pgftext[x=3.285112in,y=3.440814in,left,base]{\color{textcolor}\rmfamily\fontsize{8.000000}{9.600000}\selectfont WT + Vehicle (1513)}%
\end{pgfscope}%
\begin{pgfscope}%
\pgfsetroundcap%
\pgfsetroundjoin%
\pgfsetlinewidth{1.003750pt}%
\definecolor{currentstroke}{rgb}{0.168627,0.670588,0.494118}%
\pgfsetstrokecolor{currentstroke}%
\pgfsetdash{}{0pt}%
\pgfpathmoveto{\pgfqpoint{3.085112in}{3.313063in}}%
\pgfpathlineto{\pgfqpoint{3.196223in}{3.313063in}}%
\pgfusepath{stroke}%
\end{pgfscope}%
\begin{pgfscope}%
\definecolor{textcolor}{rgb}{1.000000,1.000000,1.000000}%
\pgfsetstrokecolor{textcolor}%
\pgfsetfillcolor{textcolor}%
\pgftext[x=3.285112in,y=3.274174in,left,base]{\color{textcolor}\rmfamily\fontsize{8.000000}{9.600000}\selectfont WT + TAT-GluA2\textsubscript{3Y} (815)}%
\end{pgfscope}%
\begin{pgfscope}%
\pgfsetroundcap%
\pgfsetroundjoin%
\pgfsetlinewidth{1.003750pt}%
\definecolor{currentstroke}{rgb}{1.000000,0.494118,0.250980}%
\pgfsetstrokecolor{currentstroke}%
\pgfsetdash{}{0pt}%
\pgfpathmoveto{\pgfqpoint{3.085112in}{3.146424in}}%
\pgfpathlineto{\pgfqpoint{3.196223in}{3.146424in}}%
\pgfusepath{stroke}%
\end{pgfscope}%
\begin{pgfscope}%
\definecolor{textcolor}{rgb}{1.000000,1.000000,1.000000}%
\pgfsetstrokecolor{textcolor}%
\pgfsetfillcolor{textcolor}%
\pgftext[x=3.285112in,y=3.107535in,left,base]{\color{textcolor}\rmfamily\fontsize{8.000000}{9.600000}\selectfont Tg + Vehicle (867)}%
\end{pgfscope}%
\begin{pgfscope}%
\pgfsetroundcap%
\pgfsetroundjoin%
\pgfsetlinewidth{1.003750pt}%
\definecolor{currentstroke}{rgb}{1.000000,0.694118,0.250980}%
\pgfsetstrokecolor{currentstroke}%
\pgfsetdash{}{0pt}%
\pgfpathmoveto{\pgfqpoint{3.085112in}{2.979784in}}%
\pgfpathlineto{\pgfqpoint{3.196223in}{2.979784in}}%
\pgfusepath{stroke}%
\end{pgfscope}%
\begin{pgfscope}%
\definecolor{textcolor}{rgb}{1.000000,1.000000,1.000000}%
\pgfsetstrokecolor{textcolor}%
\pgfsetfillcolor{textcolor}%
\pgftext[x=3.285112in,y=2.940895in,left,base]{\color{textcolor}\rmfamily\fontsize{8.000000}{9.600000}\selectfont Tg + TAT-GluA2\textsubscript{3Y} (1189)}%
\end{pgfscope}%
\begin{pgfscope}%
\pgfsetbuttcap%
\pgfsetroundjoin%
\definecolor{currentfill}{rgb}{0.150000,0.150000,0.150000}%
\pgfsetfillcolor{currentfill}%
\pgfsetlinewidth{1.003750pt}%
\definecolor{currentstroke}{rgb}{0.150000,0.150000,0.150000}%
\pgfsetstrokecolor{currentstroke}%
\pgfsetdash{}{0pt}%
\pgfsys@defobject{currentmarker}{\pgfqpoint{0.000000in}{0.000000in}}{\pgfqpoint{0.041667in}{0.000000in}}{%
\pgfpathmoveto{\pgfqpoint{0.000000in}{0.000000in}}%
\pgfpathlineto{\pgfqpoint{0.041667in}{0.000000in}}%
\pgfusepath{stroke,fill}%
}%
\begin{pgfscope}%
\pgfsys@transformshift{3.186623in}{0.687902in}%
\pgfsys@useobject{currentmarker}{}%
\end{pgfscope}%
\end{pgfscope}%
\begin{pgfscope}%
\definecolor{textcolor}{rgb}{0.150000,0.150000,0.150000}%
\pgfsetstrokecolor{textcolor}%
\pgfsetfillcolor{textcolor}%
\pgftext[x=3.089400in,y=0.687902in,right,]{\color{textcolor}\rmfamily\fontsize{10.000000}{12.000000}\selectfont \(\displaystyle 0.000\)}%
\end{pgfscope}%
\begin{pgfscope}%
\pgfsetbuttcap%
\pgfsetroundjoin%
\definecolor{currentfill}{rgb}{0.150000,0.150000,0.150000}%
\pgfsetfillcolor{currentfill}%
\pgfsetlinewidth{1.003750pt}%
\definecolor{currentstroke}{rgb}{0.150000,0.150000,0.150000}%
\pgfsetstrokecolor{currentstroke}%
\pgfsetdash{}{0pt}%
\pgfsys@defobject{currentmarker}{\pgfqpoint{0.000000in}{0.000000in}}{\pgfqpoint{0.041667in}{0.000000in}}{%
\pgfpathmoveto{\pgfqpoint{0.000000in}{0.000000in}}%
\pgfpathlineto{\pgfqpoint{0.041667in}{0.000000in}}%
\pgfusepath{stroke,fill}%
}%
\begin{pgfscope}%
\pgfsys@transformshift{3.186623in}{0.924038in}%
\pgfsys@useobject{currentmarker}{}%
\end{pgfscope}%
\end{pgfscope}%
\begin{pgfscope}%
\definecolor{textcolor}{rgb}{0.150000,0.150000,0.150000}%
\pgfsetstrokecolor{textcolor}%
\pgfsetfillcolor{textcolor}%
\pgftext[x=3.089400in,y=0.924038in,right,]{\color{textcolor}\rmfamily\fontsize{10.000000}{12.000000}\selectfont \(\displaystyle 0.005\)}%
\end{pgfscope}%
\begin{pgfscope}%
\pgfsetbuttcap%
\pgfsetroundjoin%
\definecolor{currentfill}{rgb}{0.150000,0.150000,0.150000}%
\pgfsetfillcolor{currentfill}%
\pgfsetlinewidth{1.003750pt}%
\definecolor{currentstroke}{rgb}{0.150000,0.150000,0.150000}%
\pgfsetstrokecolor{currentstroke}%
\pgfsetdash{}{0pt}%
\pgfsys@defobject{currentmarker}{\pgfqpoint{0.000000in}{0.000000in}}{\pgfqpoint{0.041667in}{0.000000in}}{%
\pgfpathmoveto{\pgfqpoint{0.000000in}{0.000000in}}%
\pgfpathlineto{\pgfqpoint{0.041667in}{0.000000in}}%
\pgfusepath{stroke,fill}%
}%
\begin{pgfscope}%
\pgfsys@transformshift{3.186623in}{1.160174in}%
\pgfsys@useobject{currentmarker}{}%
\end{pgfscope}%
\end{pgfscope}%
\begin{pgfscope}%
\definecolor{textcolor}{rgb}{0.150000,0.150000,0.150000}%
\pgfsetstrokecolor{textcolor}%
\pgfsetfillcolor{textcolor}%
\pgftext[x=3.089400in,y=1.160174in,right,]{\color{textcolor}\rmfamily\fontsize{10.000000}{12.000000}\selectfont \(\displaystyle 0.010\)}%
\end{pgfscope}%
\begin{pgfscope}%
\pgfsetbuttcap%
\pgfsetroundjoin%
\definecolor{currentfill}{rgb}{0.150000,0.150000,0.150000}%
\pgfsetfillcolor{currentfill}%
\pgfsetlinewidth{1.003750pt}%
\definecolor{currentstroke}{rgb}{0.150000,0.150000,0.150000}%
\pgfsetstrokecolor{currentstroke}%
\pgfsetdash{}{0pt}%
\pgfsys@defobject{currentmarker}{\pgfqpoint{0.000000in}{0.000000in}}{\pgfqpoint{0.041667in}{0.000000in}}{%
\pgfpathmoveto{\pgfqpoint{0.000000in}{0.000000in}}%
\pgfpathlineto{\pgfqpoint{0.041667in}{0.000000in}}%
\pgfusepath{stroke,fill}%
}%
\begin{pgfscope}%
\pgfsys@transformshift{3.186623in}{1.396310in}%
\pgfsys@useobject{currentmarker}{}%
\end{pgfscope}%
\end{pgfscope}%
\begin{pgfscope}%
\definecolor{textcolor}{rgb}{0.150000,0.150000,0.150000}%
\pgfsetstrokecolor{textcolor}%
\pgfsetfillcolor{textcolor}%
\pgftext[x=3.089400in,y=1.396310in,right,]{\color{textcolor}\rmfamily\fontsize{10.000000}{12.000000}\selectfont \(\displaystyle 0.015\)}%
\end{pgfscope}%
\begin{pgfscope}%
\pgfsetbuttcap%
\pgfsetroundjoin%
\definecolor{currentfill}{rgb}{0.150000,0.150000,0.150000}%
\pgfsetfillcolor{currentfill}%
\pgfsetlinewidth{1.003750pt}%
\definecolor{currentstroke}{rgb}{0.150000,0.150000,0.150000}%
\pgfsetstrokecolor{currentstroke}%
\pgfsetdash{}{0pt}%
\pgfsys@defobject{currentmarker}{\pgfqpoint{0.000000in}{0.000000in}}{\pgfqpoint{0.041667in}{0.000000in}}{%
\pgfpathmoveto{\pgfqpoint{0.000000in}{0.000000in}}%
\pgfpathlineto{\pgfqpoint{0.041667in}{0.000000in}}%
\pgfusepath{stroke,fill}%
}%
\begin{pgfscope}%
\pgfsys@transformshift{3.186623in}{1.632445in}%
\pgfsys@useobject{currentmarker}{}%
\end{pgfscope}%
\end{pgfscope}%
\begin{pgfscope}%
\definecolor{textcolor}{rgb}{0.150000,0.150000,0.150000}%
\pgfsetstrokecolor{textcolor}%
\pgfsetfillcolor{textcolor}%
\pgftext[x=3.089400in,y=1.632445in,right,]{\color{textcolor}\rmfamily\fontsize{10.000000}{12.000000}\selectfont \(\displaystyle 0.020\)}%
\end{pgfscope}%
\begin{pgfscope}%
\pgfsetbuttcap%
\pgfsetroundjoin%
\definecolor{currentfill}{rgb}{0.150000,0.150000,0.150000}%
\pgfsetfillcolor{currentfill}%
\pgfsetlinewidth{1.003750pt}%
\definecolor{currentstroke}{rgb}{0.150000,0.150000,0.150000}%
\pgfsetstrokecolor{currentstroke}%
\pgfsetdash{}{0pt}%
\pgfsys@defobject{currentmarker}{\pgfqpoint{0.000000in}{0.000000in}}{\pgfqpoint{0.041667in}{0.000000in}}{%
\pgfpathmoveto{\pgfqpoint{0.000000in}{0.000000in}}%
\pgfpathlineto{\pgfqpoint{0.041667in}{0.000000in}}%
\pgfusepath{stroke,fill}%
}%
\begin{pgfscope}%
\pgfsys@transformshift{3.186623in}{1.868581in}%
\pgfsys@useobject{currentmarker}{}%
\end{pgfscope}%
\end{pgfscope}%
\begin{pgfscope}%
\definecolor{textcolor}{rgb}{0.150000,0.150000,0.150000}%
\pgfsetstrokecolor{textcolor}%
\pgfsetfillcolor{textcolor}%
\pgftext[x=3.089400in,y=1.868581in,right,]{\color{textcolor}\rmfamily\fontsize{10.000000}{12.000000}\selectfont \(\displaystyle 0.025\)}%
\end{pgfscope}%
\begin{pgfscope}%
\pgfsetbuttcap%
\pgfsetroundjoin%
\definecolor{currentfill}{rgb}{0.150000,0.150000,0.150000}%
\pgfsetfillcolor{currentfill}%
\pgfsetlinewidth{1.003750pt}%
\definecolor{currentstroke}{rgb}{0.150000,0.150000,0.150000}%
\pgfsetstrokecolor{currentstroke}%
\pgfsetdash{}{0pt}%
\pgfsys@defobject{currentmarker}{\pgfqpoint{0.000000in}{0.000000in}}{\pgfqpoint{0.041667in}{0.000000in}}{%
\pgfpathmoveto{\pgfqpoint{0.000000in}{0.000000in}}%
\pgfpathlineto{\pgfqpoint{0.041667in}{0.000000in}}%
\pgfusepath{stroke,fill}%
}%
\begin{pgfscope}%
\pgfsys@transformshift{3.186623in}{2.104717in}%
\pgfsys@useobject{currentmarker}{}%
\end{pgfscope}%
\end{pgfscope}%
\begin{pgfscope}%
\definecolor{textcolor}{rgb}{0.150000,0.150000,0.150000}%
\pgfsetstrokecolor{textcolor}%
\pgfsetfillcolor{textcolor}%
\pgftext[x=3.089400in,y=2.104717in,right,]{\color{textcolor}\rmfamily\fontsize{10.000000}{12.000000}\selectfont \(\displaystyle 0.030\)}%
\end{pgfscope}%
\begin{pgfscope}%
\pgfsetbuttcap%
\pgfsetroundjoin%
\definecolor{currentfill}{rgb}{0.150000,0.150000,0.150000}%
\pgfsetfillcolor{currentfill}%
\pgfsetlinewidth{1.003750pt}%
\definecolor{currentstroke}{rgb}{0.150000,0.150000,0.150000}%
\pgfsetstrokecolor{currentstroke}%
\pgfsetdash{}{0pt}%
\pgfsys@defobject{currentmarker}{\pgfqpoint{0.000000in}{0.000000in}}{\pgfqpoint{0.041667in}{0.000000in}}{%
\pgfpathmoveto{\pgfqpoint{0.000000in}{0.000000in}}%
\pgfpathlineto{\pgfqpoint{0.041667in}{0.000000in}}%
\pgfusepath{stroke,fill}%
}%
\begin{pgfscope}%
\pgfsys@transformshift{3.186623in}{2.340853in}%
\pgfsys@useobject{currentmarker}{}%
\end{pgfscope}%
\end{pgfscope}%
\begin{pgfscope}%
\definecolor{textcolor}{rgb}{0.150000,0.150000,0.150000}%
\pgfsetstrokecolor{textcolor}%
\pgfsetfillcolor{textcolor}%
\pgftext[x=3.089400in,y=2.340853in,right,]{\color{textcolor}\rmfamily\fontsize{10.000000}{12.000000}\selectfont \(\displaystyle 0.035\)}%
\end{pgfscope}%
\begin{pgfscope}%
\pgfsetbuttcap%
\pgfsetroundjoin%
\definecolor{currentfill}{rgb}{0.150000,0.150000,0.150000}%
\pgfsetfillcolor{currentfill}%
\pgfsetlinewidth{1.003750pt}%
\definecolor{currentstroke}{rgb}{0.150000,0.150000,0.150000}%
\pgfsetstrokecolor{currentstroke}%
\pgfsetdash{}{0pt}%
\pgfsys@defobject{currentmarker}{\pgfqpoint{0.000000in}{0.000000in}}{\pgfqpoint{0.041667in}{0.000000in}}{%
\pgfpathmoveto{\pgfqpoint{0.000000in}{0.000000in}}%
\pgfpathlineto{\pgfqpoint{0.041667in}{0.000000in}}%
\pgfusepath{stroke,fill}%
}%
\begin{pgfscope}%
\pgfsys@transformshift{3.186623in}{2.576988in}%
\pgfsys@useobject{currentmarker}{}%
\end{pgfscope}%
\end{pgfscope}%
\begin{pgfscope}%
\definecolor{textcolor}{rgb}{0.150000,0.150000,0.150000}%
\pgfsetstrokecolor{textcolor}%
\pgfsetfillcolor{textcolor}%
\pgftext[x=3.089400in,y=2.576988in,right,]{\color{textcolor}\rmfamily\fontsize{10.000000}{12.000000}\selectfont \(\displaystyle 0.040\)}%
\end{pgfscope}%
\begin{pgfscope}%
\pgfsetbuttcap%
\pgfsetroundjoin%
\definecolor{currentfill}{rgb}{0.150000,0.150000,0.150000}%
\pgfsetfillcolor{currentfill}%
\pgfsetlinewidth{1.003750pt}%
\definecolor{currentstroke}{rgb}{0.150000,0.150000,0.150000}%
\pgfsetstrokecolor{currentstroke}%
\pgfsetdash{}{0pt}%
\pgfsys@defobject{currentmarker}{\pgfqpoint{0.000000in}{0.000000in}}{\pgfqpoint{0.041667in}{0.000000in}}{%
\pgfpathmoveto{\pgfqpoint{0.000000in}{0.000000in}}%
\pgfpathlineto{\pgfqpoint{0.041667in}{0.000000in}}%
\pgfusepath{stroke,fill}%
}%
\begin{pgfscope}%
\pgfsys@transformshift{3.186623in}{2.813124in}%
\pgfsys@useobject{currentmarker}{}%
\end{pgfscope}%
\end{pgfscope}%
\begin{pgfscope}%
\definecolor{textcolor}{rgb}{0.150000,0.150000,0.150000}%
\pgfsetstrokecolor{textcolor}%
\pgfsetfillcolor{textcolor}%
\pgftext[x=3.089400in,y=2.813124in,right,]{\color{textcolor}\rmfamily\fontsize{10.000000}{12.000000}\selectfont \(\displaystyle 0.045\)}%
\end{pgfscope}%
\begin{pgfscope}%
\definecolor{textcolor}{rgb}{0.150000,0.150000,0.150000}%
\pgfsetstrokecolor{textcolor}%
\pgfsetfillcolor{textcolor}%
\pgftext[x=2.509191in,y=0.896685in,left,base,rotate=90.000000]{\color{textcolor}\rmfamily\fontsize{10.000000}{12.000000}\selectfont \textbf{Freezing info.~\(\displaystyle |\) Position }}%
\end{pgfscope}%
\begin{pgfscope}%
\definecolor{textcolor}{rgb}{0.150000,0.150000,0.150000}%
\pgfsetstrokecolor{textcolor}%
\pgfsetfillcolor{textcolor}%
\pgftext[x=2.668883in,y=1.293584in,left,base,rotate=90.000000]{\color{textcolor}\rmfamily\fontsize{10.000000}{12.000000}\selectfont \textbf{ (bits/frame)}}%
\end{pgfscope}%
\begin{pgfscope}%
\pgfpathrectangle{\pgfqpoint{3.186623in}{0.687902in}}{\pgfqpoint{2.015106in}{2.125222in}} %
\pgfusepath{clip}%
\pgfsetbuttcap%
\pgfsetmiterjoin%
\definecolor{currentfill}{rgb}{0.200000,0.427451,0.650980}%
\pgfsetfillcolor{currentfill}%
\pgfsetlinewidth{1.505625pt}%
\definecolor{currentstroke}{rgb}{0.200000,0.427451,0.650980}%
\pgfsetstrokecolor{currentstroke}%
\pgfsetdash{}{0pt}%
\pgfpathmoveto{\pgfqpoint{3.258591in}{0.687902in}}%
\pgfpathlineto{\pgfqpoint{3.618431in}{0.687902in}}%
\pgfpathlineto{\pgfqpoint{3.618431in}{2.256357in}}%
\pgfpathlineto{\pgfqpoint{3.258591in}{2.256357in}}%
\pgfpathclose%
\pgfusepath{stroke,fill}%
\end{pgfscope}%
\begin{pgfscope}%
\pgfpathrectangle{\pgfqpoint{3.186623in}{0.687902in}}{\pgfqpoint{2.015106in}{2.125222in}} %
\pgfusepath{clip}%
\pgfsetbuttcap%
\pgfsetmiterjoin%
\definecolor{currentfill}{rgb}{0.168627,0.670588,0.494118}%
\pgfsetfillcolor{currentfill}%
\pgfsetlinewidth{1.505625pt}%
\definecolor{currentstroke}{rgb}{0.168627,0.670588,0.494118}%
\pgfsetstrokecolor{currentstroke}%
\pgfsetdash{}{0pt}%
\pgfpathmoveto{\pgfqpoint{3.762367in}{0.687902in}}%
\pgfpathlineto{\pgfqpoint{4.122208in}{0.687902in}}%
\pgfpathlineto{\pgfqpoint{4.122208in}{2.140435in}}%
\pgfpathlineto{\pgfqpoint{3.762367in}{2.140435in}}%
\pgfpathclose%
\pgfusepath{stroke,fill}%
\end{pgfscope}%
\begin{pgfscope}%
\pgfpathrectangle{\pgfqpoint{3.186623in}{0.687902in}}{\pgfqpoint{2.015106in}{2.125222in}} %
\pgfusepath{clip}%
\pgfsetbuttcap%
\pgfsetmiterjoin%
\definecolor{currentfill}{rgb}{1.000000,0.494118,0.250980}%
\pgfsetfillcolor{currentfill}%
\pgfsetlinewidth{1.505625pt}%
\definecolor{currentstroke}{rgb}{1.000000,0.494118,0.250980}%
\pgfsetstrokecolor{currentstroke}%
\pgfsetdash{}{0pt}%
\pgfpathmoveto{\pgfqpoint{4.266144in}{0.687902in}}%
\pgfpathlineto{\pgfqpoint{4.625984in}{0.687902in}}%
\pgfpathlineto{\pgfqpoint{4.625984in}{1.341582in}}%
\pgfpathlineto{\pgfqpoint{4.266144in}{1.341582in}}%
\pgfpathclose%
\pgfusepath{stroke,fill}%
\end{pgfscope}%
\begin{pgfscope}%
\pgfpathrectangle{\pgfqpoint{3.186623in}{0.687902in}}{\pgfqpoint{2.015106in}{2.125222in}} %
\pgfusepath{clip}%
\pgfsetbuttcap%
\pgfsetmiterjoin%
\definecolor{currentfill}{rgb}{1.000000,0.694118,0.250980}%
\pgfsetfillcolor{currentfill}%
\pgfsetlinewidth{1.505625pt}%
\definecolor{currentstroke}{rgb}{1.000000,0.694118,0.250980}%
\pgfsetstrokecolor{currentstroke}%
\pgfsetdash{}{0pt}%
\pgfpathmoveto{\pgfqpoint{4.769920in}{0.687902in}}%
\pgfpathlineto{\pgfqpoint{5.129761in}{0.687902in}}%
\pgfpathlineto{\pgfqpoint{5.129761in}{2.062992in}}%
\pgfpathlineto{\pgfqpoint{4.769920in}{2.062992in}}%
\pgfpathclose%
\pgfusepath{stroke,fill}%
\end{pgfscope}%
\begin{pgfscope}%
\pgfpathrectangle{\pgfqpoint{3.186623in}{0.687902in}}{\pgfqpoint{2.015106in}{2.125222in}} %
\pgfusepath{clip}%
\pgfsetbuttcap%
\pgfsetroundjoin%
\pgfsetlinewidth{1.505625pt}%
\definecolor{currentstroke}{rgb}{0.200000,0.427451,0.650980}%
\pgfsetstrokecolor{currentstroke}%
\pgfsetdash{}{0pt}%
\pgfpathmoveto{\pgfqpoint{3.438511in}{2.256357in}}%
\pgfpathlineto{\pgfqpoint{3.438511in}{2.292713in}}%
\pgfusepath{stroke}%
\end{pgfscope}%
\begin{pgfscope}%
\pgfpathrectangle{\pgfqpoint{3.186623in}{0.687902in}}{\pgfqpoint{2.015106in}{2.125222in}} %
\pgfusepath{clip}%
\pgfsetbuttcap%
\pgfsetroundjoin%
\pgfsetlinewidth{1.505625pt}%
\definecolor{currentstroke}{rgb}{0.168627,0.670588,0.494118}%
\pgfsetstrokecolor{currentstroke}%
\pgfsetdash{}{0pt}%
\pgfpathmoveto{\pgfqpoint{3.942287in}{2.140435in}}%
\pgfpathlineto{\pgfqpoint{3.942287in}{2.190465in}}%
\pgfusepath{stroke}%
\end{pgfscope}%
\begin{pgfscope}%
\pgfpathrectangle{\pgfqpoint{3.186623in}{0.687902in}}{\pgfqpoint{2.015106in}{2.125222in}} %
\pgfusepath{clip}%
\pgfsetbuttcap%
\pgfsetroundjoin%
\pgfsetlinewidth{1.505625pt}%
\definecolor{currentstroke}{rgb}{1.000000,0.494118,0.250980}%
\pgfsetstrokecolor{currentstroke}%
\pgfsetdash{}{0pt}%
\pgfpathmoveto{\pgfqpoint{4.446064in}{1.341582in}}%
\pgfpathlineto{\pgfqpoint{4.446064in}{1.361550in}}%
\pgfusepath{stroke}%
\end{pgfscope}%
\begin{pgfscope}%
\pgfpathrectangle{\pgfqpoint{3.186623in}{0.687902in}}{\pgfqpoint{2.015106in}{2.125222in}} %
\pgfusepath{clip}%
\pgfsetbuttcap%
\pgfsetroundjoin%
\pgfsetlinewidth{1.505625pt}%
\definecolor{currentstroke}{rgb}{1.000000,0.694118,0.250980}%
\pgfsetstrokecolor{currentstroke}%
\pgfsetdash{}{0pt}%
\pgfpathmoveto{\pgfqpoint{4.949840in}{2.062992in}}%
\pgfpathlineto{\pgfqpoint{4.949840in}{2.103047in}}%
\pgfusepath{stroke}%
\end{pgfscope}%
\begin{pgfscope}%
\pgfpathrectangle{\pgfqpoint{3.186623in}{0.687902in}}{\pgfqpoint{2.015106in}{2.125222in}} %
\pgfusepath{clip}%
\pgfsetbuttcap%
\pgfsetroundjoin%
\definecolor{currentfill}{rgb}{0.200000,0.427451,0.650980}%
\pgfsetfillcolor{currentfill}%
\pgfsetlinewidth{1.505625pt}%
\definecolor{currentstroke}{rgb}{0.200000,0.427451,0.650980}%
\pgfsetstrokecolor{currentstroke}%
\pgfsetdash{}{0pt}%
\pgfsys@defobject{currentmarker}{\pgfqpoint{-0.111111in}{-0.000000in}}{\pgfqpoint{0.111111in}{0.000000in}}{%
\pgfpathmoveto{\pgfqpoint{0.111111in}{-0.000000in}}%
\pgfpathlineto{\pgfqpoint{-0.111111in}{0.000000in}}%
\pgfusepath{stroke,fill}%
}%
\begin{pgfscope}%
\pgfsys@transformshift{3.438511in}{2.256357in}%
\pgfsys@useobject{currentmarker}{}%
\end{pgfscope}%
\end{pgfscope}%
\begin{pgfscope}%
\pgfpathrectangle{\pgfqpoint{3.186623in}{0.687902in}}{\pgfqpoint{2.015106in}{2.125222in}} %
\pgfusepath{clip}%
\pgfsetbuttcap%
\pgfsetroundjoin%
\definecolor{currentfill}{rgb}{0.200000,0.427451,0.650980}%
\pgfsetfillcolor{currentfill}%
\pgfsetlinewidth{1.505625pt}%
\definecolor{currentstroke}{rgb}{0.200000,0.427451,0.650980}%
\pgfsetstrokecolor{currentstroke}%
\pgfsetdash{}{0pt}%
\pgfsys@defobject{currentmarker}{\pgfqpoint{-0.111111in}{-0.000000in}}{\pgfqpoint{0.111111in}{0.000000in}}{%
\pgfpathmoveto{\pgfqpoint{0.111111in}{-0.000000in}}%
\pgfpathlineto{\pgfqpoint{-0.111111in}{0.000000in}}%
\pgfusepath{stroke,fill}%
}%
\begin{pgfscope}%
\pgfsys@transformshift{3.438511in}{2.292713in}%
\pgfsys@useobject{currentmarker}{}%
\end{pgfscope}%
\end{pgfscope}%
\begin{pgfscope}%
\pgfpathrectangle{\pgfqpoint{3.186623in}{0.687902in}}{\pgfqpoint{2.015106in}{2.125222in}} %
\pgfusepath{clip}%
\pgfsetbuttcap%
\pgfsetroundjoin%
\definecolor{currentfill}{rgb}{0.168627,0.670588,0.494118}%
\pgfsetfillcolor{currentfill}%
\pgfsetlinewidth{1.505625pt}%
\definecolor{currentstroke}{rgb}{0.168627,0.670588,0.494118}%
\pgfsetstrokecolor{currentstroke}%
\pgfsetdash{}{0pt}%
\pgfsys@defobject{currentmarker}{\pgfqpoint{-0.111111in}{-0.000000in}}{\pgfqpoint{0.111111in}{0.000000in}}{%
\pgfpathmoveto{\pgfqpoint{0.111111in}{-0.000000in}}%
\pgfpathlineto{\pgfqpoint{-0.111111in}{0.000000in}}%
\pgfusepath{stroke,fill}%
}%
\begin{pgfscope}%
\pgfsys@transformshift{3.942287in}{2.140435in}%
\pgfsys@useobject{currentmarker}{}%
\end{pgfscope}%
\end{pgfscope}%
\begin{pgfscope}%
\pgfpathrectangle{\pgfqpoint{3.186623in}{0.687902in}}{\pgfqpoint{2.015106in}{2.125222in}} %
\pgfusepath{clip}%
\pgfsetbuttcap%
\pgfsetroundjoin%
\definecolor{currentfill}{rgb}{0.168627,0.670588,0.494118}%
\pgfsetfillcolor{currentfill}%
\pgfsetlinewidth{1.505625pt}%
\definecolor{currentstroke}{rgb}{0.168627,0.670588,0.494118}%
\pgfsetstrokecolor{currentstroke}%
\pgfsetdash{}{0pt}%
\pgfsys@defobject{currentmarker}{\pgfqpoint{-0.111111in}{-0.000000in}}{\pgfqpoint{0.111111in}{0.000000in}}{%
\pgfpathmoveto{\pgfqpoint{0.111111in}{-0.000000in}}%
\pgfpathlineto{\pgfqpoint{-0.111111in}{0.000000in}}%
\pgfusepath{stroke,fill}%
}%
\begin{pgfscope}%
\pgfsys@transformshift{3.942287in}{2.190465in}%
\pgfsys@useobject{currentmarker}{}%
\end{pgfscope}%
\end{pgfscope}%
\begin{pgfscope}%
\pgfpathrectangle{\pgfqpoint{3.186623in}{0.687902in}}{\pgfqpoint{2.015106in}{2.125222in}} %
\pgfusepath{clip}%
\pgfsetbuttcap%
\pgfsetroundjoin%
\definecolor{currentfill}{rgb}{1.000000,0.494118,0.250980}%
\pgfsetfillcolor{currentfill}%
\pgfsetlinewidth{1.505625pt}%
\definecolor{currentstroke}{rgb}{1.000000,0.494118,0.250980}%
\pgfsetstrokecolor{currentstroke}%
\pgfsetdash{}{0pt}%
\pgfsys@defobject{currentmarker}{\pgfqpoint{-0.111111in}{-0.000000in}}{\pgfqpoint{0.111111in}{0.000000in}}{%
\pgfpathmoveto{\pgfqpoint{0.111111in}{-0.000000in}}%
\pgfpathlineto{\pgfqpoint{-0.111111in}{0.000000in}}%
\pgfusepath{stroke,fill}%
}%
\begin{pgfscope}%
\pgfsys@transformshift{4.446064in}{1.341582in}%
\pgfsys@useobject{currentmarker}{}%
\end{pgfscope}%
\end{pgfscope}%
\begin{pgfscope}%
\pgfpathrectangle{\pgfqpoint{3.186623in}{0.687902in}}{\pgfqpoint{2.015106in}{2.125222in}} %
\pgfusepath{clip}%
\pgfsetbuttcap%
\pgfsetroundjoin%
\definecolor{currentfill}{rgb}{1.000000,0.494118,0.250980}%
\pgfsetfillcolor{currentfill}%
\pgfsetlinewidth{1.505625pt}%
\definecolor{currentstroke}{rgb}{1.000000,0.494118,0.250980}%
\pgfsetstrokecolor{currentstroke}%
\pgfsetdash{}{0pt}%
\pgfsys@defobject{currentmarker}{\pgfqpoint{-0.111111in}{-0.000000in}}{\pgfqpoint{0.111111in}{0.000000in}}{%
\pgfpathmoveto{\pgfqpoint{0.111111in}{-0.000000in}}%
\pgfpathlineto{\pgfqpoint{-0.111111in}{0.000000in}}%
\pgfusepath{stroke,fill}%
}%
\begin{pgfscope}%
\pgfsys@transformshift{4.446064in}{1.361550in}%
\pgfsys@useobject{currentmarker}{}%
\end{pgfscope}%
\end{pgfscope}%
\begin{pgfscope}%
\pgfpathrectangle{\pgfqpoint{3.186623in}{0.687902in}}{\pgfqpoint{2.015106in}{2.125222in}} %
\pgfusepath{clip}%
\pgfsetbuttcap%
\pgfsetroundjoin%
\definecolor{currentfill}{rgb}{1.000000,0.694118,0.250980}%
\pgfsetfillcolor{currentfill}%
\pgfsetlinewidth{1.505625pt}%
\definecolor{currentstroke}{rgb}{1.000000,0.694118,0.250980}%
\pgfsetstrokecolor{currentstroke}%
\pgfsetdash{}{0pt}%
\pgfsys@defobject{currentmarker}{\pgfqpoint{-0.111111in}{-0.000000in}}{\pgfqpoint{0.111111in}{0.000000in}}{%
\pgfpathmoveto{\pgfqpoint{0.111111in}{-0.000000in}}%
\pgfpathlineto{\pgfqpoint{-0.111111in}{0.000000in}}%
\pgfusepath{stroke,fill}%
}%
\begin{pgfscope}%
\pgfsys@transformshift{4.949840in}{2.062992in}%
\pgfsys@useobject{currentmarker}{}%
\end{pgfscope}%
\end{pgfscope}%
\begin{pgfscope}%
\pgfpathrectangle{\pgfqpoint{3.186623in}{0.687902in}}{\pgfqpoint{2.015106in}{2.125222in}} %
\pgfusepath{clip}%
\pgfsetbuttcap%
\pgfsetroundjoin%
\definecolor{currentfill}{rgb}{1.000000,0.694118,0.250980}%
\pgfsetfillcolor{currentfill}%
\pgfsetlinewidth{1.505625pt}%
\definecolor{currentstroke}{rgb}{1.000000,0.694118,0.250980}%
\pgfsetstrokecolor{currentstroke}%
\pgfsetdash{}{0pt}%
\pgfsys@defobject{currentmarker}{\pgfqpoint{-0.111111in}{-0.000000in}}{\pgfqpoint{0.111111in}{0.000000in}}{%
\pgfpathmoveto{\pgfqpoint{0.111111in}{-0.000000in}}%
\pgfpathlineto{\pgfqpoint{-0.111111in}{0.000000in}}%
\pgfusepath{stroke,fill}%
}%
\begin{pgfscope}%
\pgfsys@transformshift{4.949840in}{2.103047in}%
\pgfsys@useobject{currentmarker}{}%
\end{pgfscope}%
\end{pgfscope}%
\begin{pgfscope}%
\pgfpathrectangle{\pgfqpoint{3.186623in}{0.687902in}}{\pgfqpoint{2.015106in}{2.125222in}} %
\pgfusepath{clip}%
\pgfsetroundcap%
\pgfsetroundjoin%
\pgfsetlinewidth{1.756562pt}%
\definecolor{currentstroke}{rgb}{0.627451,0.627451,0.643137}%
\pgfsetstrokecolor{currentstroke}%
\pgfsetdash{}{0pt}%
\pgfpathmoveto{\pgfqpoint{3.438511in}{2.373731in}}%
\pgfpathlineto{\pgfqpoint{3.438511in}{2.508762in}}%
\pgfusepath{stroke}%
\end{pgfscope}%
\begin{pgfscope}%
\pgfpathrectangle{\pgfqpoint{3.186623in}{0.687902in}}{\pgfqpoint{2.015106in}{2.125222in}} %
\pgfusepath{clip}%
\pgfsetroundcap%
\pgfsetroundjoin%
\pgfsetlinewidth{1.756562pt}%
\definecolor{currentstroke}{rgb}{0.627451,0.627451,0.643137}%
\pgfsetstrokecolor{currentstroke}%
\pgfsetdash{}{0pt}%
\pgfpathmoveto{\pgfqpoint{3.438511in}{2.508762in}}%
\pgfpathlineto{\pgfqpoint{4.446064in}{2.508762in}}%
\pgfusepath{stroke}%
\end{pgfscope}%
\begin{pgfscope}%
\pgfpathrectangle{\pgfqpoint{3.186623in}{0.687902in}}{\pgfqpoint{2.015106in}{2.125222in}} %
\pgfusepath{clip}%
\pgfsetroundcap%
\pgfsetroundjoin%
\pgfsetlinewidth{1.756562pt}%
\definecolor{currentstroke}{rgb}{0.627451,0.627451,0.643137}%
\pgfsetstrokecolor{currentstroke}%
\pgfsetdash{}{0pt}%
\pgfpathmoveto{\pgfqpoint{4.446064in}{2.508762in}}%
\pgfpathlineto{\pgfqpoint{4.446064in}{1.523587in}}%
\pgfusepath{stroke}%
\end{pgfscope}%
\begin{pgfscope}%
\pgfpathrectangle{\pgfqpoint{3.186623in}{0.687902in}}{\pgfqpoint{2.015106in}{2.125222in}} %
\pgfusepath{clip}%
\pgfsetroundcap%
\pgfsetroundjoin%
\pgfsetlinewidth{1.756562pt}%
\definecolor{currentstroke}{rgb}{0.627451,0.627451,0.643137}%
\pgfsetstrokecolor{currentstroke}%
\pgfsetdash{}{0pt}%
\pgfpathmoveto{\pgfqpoint{4.446064in}{2.589780in}}%
\pgfpathlineto{\pgfqpoint{4.446064in}{2.724811in}}%
\pgfusepath{stroke}%
\end{pgfscope}%
\begin{pgfscope}%
\pgfpathrectangle{\pgfqpoint{3.186623in}{0.687902in}}{\pgfqpoint{2.015106in}{2.125222in}} %
\pgfusepath{clip}%
\pgfsetroundcap%
\pgfsetroundjoin%
\pgfsetlinewidth{1.756562pt}%
\definecolor{currentstroke}{rgb}{0.627451,0.627451,0.643137}%
\pgfsetstrokecolor{currentstroke}%
\pgfsetdash{}{0pt}%
\pgfpathmoveto{\pgfqpoint{4.446064in}{2.724811in}}%
\pgfpathlineto{\pgfqpoint{4.949840in}{2.724811in}}%
\pgfusepath{stroke}%
\end{pgfscope}%
\begin{pgfscope}%
\pgfpathrectangle{\pgfqpoint{3.186623in}{0.687902in}}{\pgfqpoint{2.015106in}{2.125222in}} %
\pgfusepath{clip}%
\pgfsetroundcap%
\pgfsetroundjoin%
\pgfsetlinewidth{1.756562pt}%
\definecolor{currentstroke}{rgb}{0.627451,0.627451,0.643137}%
\pgfsetstrokecolor{currentstroke}%
\pgfsetdash{}{0pt}%
\pgfpathmoveto{\pgfqpoint{4.949840in}{2.724811in}}%
\pgfpathlineto{\pgfqpoint{4.949840in}{2.265084in}}%
\pgfusepath{stroke}%
\end{pgfscope}%
\begin{pgfscope}%
\pgfsetrectcap%
\pgfsetmiterjoin%
\pgfsetlinewidth{1.254687pt}%
\definecolor{currentstroke}{rgb}{0.150000,0.150000,0.150000}%
\pgfsetstrokecolor{currentstroke}%
\pgfsetdash{}{0pt}%
\pgfpathmoveto{\pgfqpoint{3.186623in}{0.687902in}}%
\pgfpathlineto{\pgfqpoint{3.186623in}{2.813124in}}%
\pgfusepath{stroke}%
\end{pgfscope}%
\begin{pgfscope}%
\pgfsetrectcap%
\pgfsetmiterjoin%
\pgfsetlinewidth{1.254687pt}%
\definecolor{currentstroke}{rgb}{0.150000,0.150000,0.150000}%
\pgfsetstrokecolor{currentstroke}%
\pgfsetdash{}{0pt}%
\pgfpathmoveto{\pgfqpoint{3.186623in}{0.687902in}}%
\pgfpathlineto{\pgfqpoint{5.201729in}{0.687902in}}%
\pgfusepath{stroke}%
\end{pgfscope}%
\begin{pgfscope}%
\definecolor{textcolor}{rgb}{0.150000,0.150000,0.150000}%
\pgfsetstrokecolor{textcolor}%
\pgfsetfillcolor{textcolor}%
\pgftext[x=4.446064in,y=1.412187in,,]{\color{textcolor}\rmfamily\fontsize{15.000000}{18.000000}\selectfont \textbf{*}}%
\end{pgfscope}%
\begin{pgfscope}%
\definecolor{textcolor}{rgb}{0.150000,0.150000,0.150000}%
\pgfsetstrokecolor{textcolor}%
\pgfsetfillcolor{textcolor}%
\pgftext[x=4.949840in,y=2.153683in,,]{\color{textcolor}\rmfamily\fontsize{15.000000}{18.000000}\selectfont \textbf{*}}%
\end{pgfscope}%
\begin{pgfscope}%
\pgfsetbuttcap%
\pgfsetmiterjoin%
\definecolor{currentfill}{rgb}{0.200000,0.427451,0.650980}%
\pgfsetfillcolor{currentfill}%
\pgfsetlinewidth{1.505625pt}%
\definecolor{currentstroke}{rgb}{0.200000,0.427451,0.650980}%
\pgfsetstrokecolor{currentstroke}%
\pgfsetdash{}{0pt}%
\pgfpathmoveto{\pgfqpoint{3.286623in}{3.440814in}}%
\pgfpathlineto{\pgfqpoint{3.397734in}{3.440814in}}%
\pgfpathlineto{\pgfqpoint{3.397734in}{3.518592in}}%
\pgfpathlineto{\pgfqpoint{3.286623in}{3.518592in}}%
\pgfpathclose%
\pgfusepath{stroke,fill}%
\end{pgfscope}%
\begin{pgfscope}%
\definecolor{textcolor}{rgb}{0.150000,0.150000,0.150000}%
\pgfsetstrokecolor{textcolor}%
\pgfsetfillcolor{textcolor}%
\pgftext[x=3.486623in,y=3.440814in,left,base]{\color{textcolor}\rmfamily\fontsize{8.000000}{9.600000}\selectfont WT + Vehicle (1513)}%
\end{pgfscope}%
\begin{pgfscope}%
\pgfsetbuttcap%
\pgfsetmiterjoin%
\definecolor{currentfill}{rgb}{0.168627,0.670588,0.494118}%
\pgfsetfillcolor{currentfill}%
\pgfsetlinewidth{1.505625pt}%
\definecolor{currentstroke}{rgb}{0.168627,0.670588,0.494118}%
\pgfsetstrokecolor{currentstroke}%
\pgfsetdash{}{0pt}%
\pgfpathmoveto{\pgfqpoint{3.286623in}{3.274174in}}%
\pgfpathlineto{\pgfqpoint{3.397734in}{3.274174in}}%
\pgfpathlineto{\pgfqpoint{3.397734in}{3.351952in}}%
\pgfpathlineto{\pgfqpoint{3.286623in}{3.351952in}}%
\pgfpathclose%
\pgfusepath{stroke,fill}%
\end{pgfscope}%
\begin{pgfscope}%
\definecolor{textcolor}{rgb}{0.150000,0.150000,0.150000}%
\pgfsetstrokecolor{textcolor}%
\pgfsetfillcolor{textcolor}%
\pgftext[x=3.486623in,y=3.274174in,left,base]{\color{textcolor}\rmfamily\fontsize{8.000000}{9.600000}\selectfont WT + TAT-GluA2\textsubscript{3Y} (815)}%
\end{pgfscope}%
\begin{pgfscope}%
\pgfsetbuttcap%
\pgfsetmiterjoin%
\definecolor{currentfill}{rgb}{1.000000,0.494118,0.250980}%
\pgfsetfillcolor{currentfill}%
\pgfsetlinewidth{1.505625pt}%
\definecolor{currentstroke}{rgb}{1.000000,0.494118,0.250980}%
\pgfsetstrokecolor{currentstroke}%
\pgfsetdash{}{0pt}%
\pgfpathmoveto{\pgfqpoint{3.286623in}{3.107535in}}%
\pgfpathlineto{\pgfqpoint{3.397734in}{3.107535in}}%
\pgfpathlineto{\pgfqpoint{3.397734in}{3.185313in}}%
\pgfpathlineto{\pgfqpoint{3.286623in}{3.185313in}}%
\pgfpathclose%
\pgfusepath{stroke,fill}%
\end{pgfscope}%
\begin{pgfscope}%
\definecolor{textcolor}{rgb}{0.150000,0.150000,0.150000}%
\pgfsetstrokecolor{textcolor}%
\pgfsetfillcolor{textcolor}%
\pgftext[x=3.486623in,y=3.107535in,left,base]{\color{textcolor}\rmfamily\fontsize{8.000000}{9.600000}\selectfont Tg + Vehicle (867)}%
\end{pgfscope}%
\begin{pgfscope}%
\pgfsetbuttcap%
\pgfsetmiterjoin%
\definecolor{currentfill}{rgb}{1.000000,0.694118,0.250980}%
\pgfsetfillcolor{currentfill}%
\pgfsetlinewidth{1.505625pt}%
\definecolor{currentstroke}{rgb}{1.000000,0.694118,0.250980}%
\pgfsetstrokecolor{currentstroke}%
\pgfsetdash{}{0pt}%
\pgfpathmoveto{\pgfqpoint{3.286623in}{2.940895in}}%
\pgfpathlineto{\pgfqpoint{3.397734in}{2.940895in}}%
\pgfpathlineto{\pgfqpoint{3.397734in}{3.018673in}}%
\pgfpathlineto{\pgfqpoint{3.286623in}{3.018673in}}%
\pgfpathclose%
\pgfusepath{stroke,fill}%
\end{pgfscope}%
\begin{pgfscope}%
\definecolor{textcolor}{rgb}{0.150000,0.150000,0.150000}%
\pgfsetstrokecolor{textcolor}%
\pgfsetfillcolor{textcolor}%
\pgftext[x=3.486623in,y=2.940895in,left,base]{\color{textcolor}\rmfamily\fontsize{8.000000}{9.600000}\selectfont Tg + TAT-GluA2\textsubscript{3Y} (1189)}%
\end{pgfscope}%
\end{pgfpicture}%
\makeatother%
\endgroup%

    \caption[Position-controlled freezing information.]{Freezing information encoded in cells is not reliant on mouse position or freezing level. This measurement removes the effect of position from our freezing information measurement. This result is similar to Figure~\ref{f.ad.freeze_info}. This suggests that the position of the mouse is not a confounding factor for our freezing information measurement. \label{f.ad.freeze_ctrl}}
\end{figure}



\subsection{Network encoding of freezing behaviour}

While all the measurements we have performed thus far consider each cell individually, we must also ask whether the activity of these cells are independent of each other, or if some degree of information is encoded in the coordination between them. To answer this question, we took advantage of machine learning algorithms. A \gls{nbc} models the cells as independent of each other, and therefore uses the neural signatures in individual cells for prediction. A general classifier such as a \gls{gsvm} on the other hand, is able to take advantage of information encoded by individual cells as well as the ensemble of cells. To investigate whether the neural signature of recalling a contextual fear memory (as measured by freezing behaviour) is encoded by the cells individually or at the network, we trained both an \gls{nbc} and a \gls{gsvm} to predict freezing behaviour from calcium transients at each time point. The performance of \gls{nbc} would represent prediction power encoded independently in cells, and the performance of \gls{gsvm} represents total prediction power of the network.

The result is shown in Figure~\ref{f.ad.classifier}. A three-way \gls{anova} of \textit{Genotype} $\times$ \textit{Treatment} $\times$ \textit{Classifier} was carried out. We found a significant main effect of \textit{Classifier} (F\tsb{1,51}=60.8, p<0.001), and no other significant interactions or main effects. This result shows that \gls{gsvm} significantly outperforms the \gls{nbc} classifier across all treatment groups, suggesting that the network encodes more information about freezing than the cells individually. Moreover, unlike the information content for individual cells, both \gls{nbc} and \gls{gsvm} have the same performance across all genotype and treatment groups. This suggests that with the activity of multiple cells, the deficits in information content in \gls{tg} mice are compensated. 

\begin{figure}[h]
    \begin{subfigure}[b]{0.5\textwidth}
        %% Creator: Matplotlib, PGF backend
%%
%% To include the figure in your LaTeX document, write
%%   \input{<filename>.pgf}
%%
%% Make sure the required packages are loaded in your preamble
%%   \usepackage{pgf}
%%
%% Figures using additional raster images can only be included by \input if
%% they are in the same directory as the main LaTeX file. For loading figures
%% from other directories you can use the `import` package
%%   \usepackage{import}
%% and then include the figures with
%%   \import{<path to file>}{<filename>.pgf}
%%
%% Matplotlib used the following preamble
%%   \usepackage[utf8]{inputenc}
%%   \usepackage[T1]{fontenc}
%%   \usepackage{siunitx}
%%
\begingroup%
\makeatletter%
\begin{pgfpicture}%
\pgfpathrectangle{\pgfpointorigin}{\pgfqpoint{2.984357in}{1.754870in}}%
\pgfusepath{use as bounding box, clip}%
\begin{pgfscope}%
\pgfsetbuttcap%
\pgfsetmiterjoin%
\definecolor{currentfill}{rgb}{1.000000,1.000000,1.000000}%
\pgfsetfillcolor{currentfill}%
\pgfsetlinewidth{0.000000pt}%
\definecolor{currentstroke}{rgb}{1.000000,1.000000,1.000000}%
\pgfsetstrokecolor{currentstroke}%
\pgfsetdash{}{0pt}%
\pgfpathmoveto{\pgfqpoint{0.000000in}{0.000000in}}%
\pgfpathlineto{\pgfqpoint{2.984357in}{0.000000in}}%
\pgfpathlineto{\pgfqpoint{2.984357in}{1.754870in}}%
\pgfpathlineto{\pgfqpoint{0.000000in}{1.754870in}}%
\pgfpathclose%
\pgfusepath{fill}%
\end{pgfscope}%
\begin{pgfscope}%
\pgfsetbuttcap%
\pgfsetmiterjoin%
\definecolor{currentfill}{rgb}{1.000000,1.000000,1.000000}%
\pgfsetfillcolor{currentfill}%
\pgfsetlinewidth{0.000000pt}%
\definecolor{currentstroke}{rgb}{0.000000,0.000000,0.000000}%
\pgfsetstrokecolor{currentstroke}%
\pgfsetstrokeopacity{0.000000}%
\pgfsetdash{}{0pt}%
\pgfpathmoveto{\pgfqpoint{0.566985in}{0.161328in}}%
\pgfpathlineto{\pgfqpoint{2.884357in}{0.161328in}}%
\pgfpathlineto{\pgfqpoint{2.884357in}{1.593542in}}%
\pgfpathlineto{\pgfqpoint{0.566985in}{1.593542in}}%
\pgfpathclose%
\pgfusepath{fill}%
\end{pgfscope}%
\begin{pgfscope}%
\pgfsetbuttcap%
\pgfsetroundjoin%
\definecolor{currentfill}{rgb}{0.150000,0.150000,0.150000}%
\pgfsetfillcolor{currentfill}%
\pgfsetlinewidth{1.003750pt}%
\definecolor{currentstroke}{rgb}{0.150000,0.150000,0.150000}%
\pgfsetstrokecolor{currentstroke}%
\pgfsetdash{}{0pt}%
\pgfsys@defobject{currentmarker}{\pgfqpoint{0.000000in}{0.000000in}}{\pgfqpoint{0.041667in}{0.000000in}}{%
\pgfpathmoveto{\pgfqpoint{0.000000in}{0.000000in}}%
\pgfpathlineto{\pgfqpoint{0.041667in}{0.000000in}}%
\pgfusepath{stroke,fill}%
}%
\begin{pgfscope}%
\pgfsys@transformshift{0.566985in}{0.161328in}%
\pgfsys@useobject{currentmarker}{}%
\end{pgfscope}%
\end{pgfscope}%
\begin{pgfscope}%
\definecolor{textcolor}{rgb}{0.150000,0.150000,0.150000}%
\pgfsetstrokecolor{textcolor}%
\pgfsetfillcolor{textcolor}%
\pgftext[x=0.469762in,y=0.161328in,right,]{\color{textcolor}\rmfamily\fontsize{10.000000}{12.000000}\selectfont \(\displaystyle 0.0\)}%
\end{pgfscope}%
\begin{pgfscope}%
\pgfsetbuttcap%
\pgfsetroundjoin%
\definecolor{currentfill}{rgb}{0.150000,0.150000,0.150000}%
\pgfsetfillcolor{currentfill}%
\pgfsetlinewidth{1.003750pt}%
\definecolor{currentstroke}{rgb}{0.150000,0.150000,0.150000}%
\pgfsetstrokecolor{currentstroke}%
\pgfsetdash{}{0pt}%
\pgfsys@defobject{currentmarker}{\pgfqpoint{0.000000in}{0.000000in}}{\pgfqpoint{0.041667in}{0.000000in}}{%
\pgfpathmoveto{\pgfqpoint{0.000000in}{0.000000in}}%
\pgfpathlineto{\pgfqpoint{0.041667in}{0.000000in}}%
\pgfusepath{stroke,fill}%
}%
\begin{pgfscope}%
\pgfsys@transformshift{0.566985in}{0.447771in}%
\pgfsys@useobject{currentmarker}{}%
\end{pgfscope}%
\end{pgfscope}%
\begin{pgfscope}%
\definecolor{textcolor}{rgb}{0.150000,0.150000,0.150000}%
\pgfsetstrokecolor{textcolor}%
\pgfsetfillcolor{textcolor}%
\pgftext[x=0.469762in,y=0.447771in,right,]{\color{textcolor}\rmfamily\fontsize{10.000000}{12.000000}\selectfont \(\displaystyle 0.2\)}%
\end{pgfscope}%
\begin{pgfscope}%
\pgfsetbuttcap%
\pgfsetroundjoin%
\definecolor{currentfill}{rgb}{0.150000,0.150000,0.150000}%
\pgfsetfillcolor{currentfill}%
\pgfsetlinewidth{1.003750pt}%
\definecolor{currentstroke}{rgb}{0.150000,0.150000,0.150000}%
\pgfsetstrokecolor{currentstroke}%
\pgfsetdash{}{0pt}%
\pgfsys@defobject{currentmarker}{\pgfqpoint{0.000000in}{0.000000in}}{\pgfqpoint{0.041667in}{0.000000in}}{%
\pgfpathmoveto{\pgfqpoint{0.000000in}{0.000000in}}%
\pgfpathlineto{\pgfqpoint{0.041667in}{0.000000in}}%
\pgfusepath{stroke,fill}%
}%
\begin{pgfscope}%
\pgfsys@transformshift{0.566985in}{0.734213in}%
\pgfsys@useobject{currentmarker}{}%
\end{pgfscope}%
\end{pgfscope}%
\begin{pgfscope}%
\definecolor{textcolor}{rgb}{0.150000,0.150000,0.150000}%
\pgfsetstrokecolor{textcolor}%
\pgfsetfillcolor{textcolor}%
\pgftext[x=0.469762in,y=0.734213in,right,]{\color{textcolor}\rmfamily\fontsize{10.000000}{12.000000}\selectfont \(\displaystyle 0.4\)}%
\end{pgfscope}%
\begin{pgfscope}%
\pgfsetbuttcap%
\pgfsetroundjoin%
\definecolor{currentfill}{rgb}{0.150000,0.150000,0.150000}%
\pgfsetfillcolor{currentfill}%
\pgfsetlinewidth{1.003750pt}%
\definecolor{currentstroke}{rgb}{0.150000,0.150000,0.150000}%
\pgfsetstrokecolor{currentstroke}%
\pgfsetdash{}{0pt}%
\pgfsys@defobject{currentmarker}{\pgfqpoint{0.000000in}{0.000000in}}{\pgfqpoint{0.041667in}{0.000000in}}{%
\pgfpathmoveto{\pgfqpoint{0.000000in}{0.000000in}}%
\pgfpathlineto{\pgfqpoint{0.041667in}{0.000000in}}%
\pgfusepath{stroke,fill}%
}%
\begin{pgfscope}%
\pgfsys@transformshift{0.566985in}{1.020656in}%
\pgfsys@useobject{currentmarker}{}%
\end{pgfscope}%
\end{pgfscope}%
\begin{pgfscope}%
\definecolor{textcolor}{rgb}{0.150000,0.150000,0.150000}%
\pgfsetstrokecolor{textcolor}%
\pgfsetfillcolor{textcolor}%
\pgftext[x=0.469762in,y=1.020656in,right,]{\color{textcolor}\rmfamily\fontsize{10.000000}{12.000000}\selectfont \(\displaystyle 0.6\)}%
\end{pgfscope}%
\begin{pgfscope}%
\pgfsetbuttcap%
\pgfsetroundjoin%
\definecolor{currentfill}{rgb}{0.150000,0.150000,0.150000}%
\pgfsetfillcolor{currentfill}%
\pgfsetlinewidth{1.003750pt}%
\definecolor{currentstroke}{rgb}{0.150000,0.150000,0.150000}%
\pgfsetstrokecolor{currentstroke}%
\pgfsetdash{}{0pt}%
\pgfsys@defobject{currentmarker}{\pgfqpoint{0.000000in}{0.000000in}}{\pgfqpoint{0.041667in}{0.000000in}}{%
\pgfpathmoveto{\pgfqpoint{0.000000in}{0.000000in}}%
\pgfpathlineto{\pgfqpoint{0.041667in}{0.000000in}}%
\pgfusepath{stroke,fill}%
}%
\begin{pgfscope}%
\pgfsys@transformshift{0.566985in}{1.307099in}%
\pgfsys@useobject{currentmarker}{}%
\end{pgfscope}%
\end{pgfscope}%
\begin{pgfscope}%
\definecolor{textcolor}{rgb}{0.150000,0.150000,0.150000}%
\pgfsetstrokecolor{textcolor}%
\pgfsetfillcolor{textcolor}%
\pgftext[x=0.469762in,y=1.307099in,right,]{\color{textcolor}\rmfamily\fontsize{10.000000}{12.000000}\selectfont \(\displaystyle 0.8\)}%
\end{pgfscope}%
\begin{pgfscope}%
\pgfsetbuttcap%
\pgfsetroundjoin%
\definecolor{currentfill}{rgb}{0.150000,0.150000,0.150000}%
\pgfsetfillcolor{currentfill}%
\pgfsetlinewidth{1.003750pt}%
\definecolor{currentstroke}{rgb}{0.150000,0.150000,0.150000}%
\pgfsetstrokecolor{currentstroke}%
\pgfsetdash{}{0pt}%
\pgfsys@defobject{currentmarker}{\pgfqpoint{0.000000in}{0.000000in}}{\pgfqpoint{0.041667in}{0.000000in}}{%
\pgfpathmoveto{\pgfqpoint{0.000000in}{0.000000in}}%
\pgfpathlineto{\pgfqpoint{0.041667in}{0.000000in}}%
\pgfusepath{stroke,fill}%
}%
\begin{pgfscope}%
\pgfsys@transformshift{0.566985in}{1.593542in}%
\pgfsys@useobject{currentmarker}{}%
\end{pgfscope}%
\end{pgfscope}%
\begin{pgfscope}%
\definecolor{textcolor}{rgb}{0.150000,0.150000,0.150000}%
\pgfsetstrokecolor{textcolor}%
\pgfsetfillcolor{textcolor}%
\pgftext[x=0.469762in,y=1.593542in,right,]{\color{textcolor}\rmfamily\fontsize{10.000000}{12.000000}\selectfont \(\displaystyle 1.0\)}%
\end{pgfscope}%
\begin{pgfscope}%
\definecolor{textcolor}{rgb}{0.150000,0.150000,0.150000}%
\pgfsetstrokecolor{textcolor}%
\pgfsetfillcolor{textcolor}%
\pgftext[x=0.222848in,y=0.877435in,,bottom,rotate=90.000000]{\color{textcolor}\rmfamily\fontsize{10.000000}{12.000000}\selectfont \textbf{NBC Accuracy}}%
\end{pgfscope}%
\begin{pgfscope}%
\pgfpathrectangle{\pgfqpoint{0.566985in}{0.161328in}}{\pgfqpoint{2.317372in}{1.432215in}} %
\pgfusepath{clip}%
\pgfsetbuttcap%
\pgfsetmiterjoin%
\definecolor{currentfill}{rgb}{0.200000,0.427451,0.650980}%
\pgfsetfillcolor{currentfill}%
\pgfsetlinewidth{1.505625pt}%
\definecolor{currentstroke}{rgb}{0.200000,0.427451,0.650980}%
\pgfsetstrokecolor{currentstroke}%
\pgfsetdash{}{0pt}%
\pgfpathmoveto{\pgfqpoint{0.649748in}{0.161328in}}%
\pgfpathlineto{\pgfqpoint{1.063564in}{0.161328in}}%
\pgfpathlineto{\pgfqpoint{1.063564in}{1.242928in}}%
\pgfpathlineto{\pgfqpoint{0.649748in}{1.242928in}}%
\pgfpathclose%
\pgfusepath{stroke,fill}%
\end{pgfscope}%
\begin{pgfscope}%
\pgfpathrectangle{\pgfqpoint{0.566985in}{0.161328in}}{\pgfqpoint{2.317372in}{1.432215in}} %
\pgfusepath{clip}%
\pgfsetbuttcap%
\pgfsetmiterjoin%
\definecolor{currentfill}{rgb}{0.168627,0.670588,0.494118}%
\pgfsetfillcolor{currentfill}%
\pgfsetlinewidth{1.505625pt}%
\definecolor{currentstroke}{rgb}{0.168627,0.670588,0.494118}%
\pgfsetstrokecolor{currentstroke}%
\pgfsetdash{}{0pt}%
\pgfpathmoveto{\pgfqpoint{1.229091in}{0.161328in}}%
\pgfpathlineto{\pgfqpoint{1.642907in}{0.161328in}}%
\pgfpathlineto{\pgfqpoint{1.642907in}{1.195970in}}%
\pgfpathlineto{\pgfqpoint{1.229091in}{1.195970in}}%
\pgfpathclose%
\pgfusepath{stroke,fill}%
\end{pgfscope}%
\begin{pgfscope}%
\pgfpathrectangle{\pgfqpoint{0.566985in}{0.161328in}}{\pgfqpoint{2.317372in}{1.432215in}} %
\pgfusepath{clip}%
\pgfsetbuttcap%
\pgfsetmiterjoin%
\definecolor{currentfill}{rgb}{1.000000,0.494118,0.250980}%
\pgfsetfillcolor{currentfill}%
\pgfsetlinewidth{1.505625pt}%
\definecolor{currentstroke}{rgb}{1.000000,0.494118,0.250980}%
\pgfsetstrokecolor{currentstroke}%
\pgfsetdash{}{0pt}%
\pgfpathmoveto{\pgfqpoint{1.808434in}{0.161328in}}%
\pgfpathlineto{\pgfqpoint{2.222250in}{0.161328in}}%
\pgfpathlineto{\pgfqpoint{2.222250in}{1.190497in}}%
\pgfpathlineto{\pgfqpoint{1.808434in}{1.190497in}}%
\pgfpathclose%
\pgfusepath{stroke,fill}%
\end{pgfscope}%
\begin{pgfscope}%
\pgfpathrectangle{\pgfqpoint{0.566985in}{0.161328in}}{\pgfqpoint{2.317372in}{1.432215in}} %
\pgfusepath{clip}%
\pgfsetbuttcap%
\pgfsetmiterjoin%
\definecolor{currentfill}{rgb}{1.000000,0.694118,0.250980}%
\pgfsetfillcolor{currentfill}%
\pgfsetlinewidth{1.505625pt}%
\definecolor{currentstroke}{rgb}{1.000000,0.694118,0.250980}%
\pgfsetstrokecolor{currentstroke}%
\pgfsetdash{}{0pt}%
\pgfpathmoveto{\pgfqpoint{2.387777in}{0.161328in}}%
\pgfpathlineto{\pgfqpoint{2.801593in}{0.161328in}}%
\pgfpathlineto{\pgfqpoint{2.801593in}{1.130103in}}%
\pgfpathlineto{\pgfqpoint{2.387777in}{1.130103in}}%
\pgfpathclose%
\pgfusepath{stroke,fill}%
\end{pgfscope}%
\begin{pgfscope}%
\pgfpathrectangle{\pgfqpoint{0.566985in}{0.161328in}}{\pgfqpoint{2.317372in}{1.432215in}} %
\pgfusepath{clip}%
\pgfsetbuttcap%
\pgfsetroundjoin%
\pgfsetlinewidth{1.505625pt}%
\definecolor{currentstroke}{rgb}{0.200000,0.427451,0.650980}%
\pgfsetstrokecolor{currentstroke}%
\pgfsetdash{}{0pt}%
\pgfpathmoveto{\pgfqpoint{0.856656in}{1.242928in}}%
\pgfpathlineto{\pgfqpoint{0.856656in}{1.276489in}}%
\pgfusepath{stroke}%
\end{pgfscope}%
\begin{pgfscope}%
\pgfpathrectangle{\pgfqpoint{0.566985in}{0.161328in}}{\pgfqpoint{2.317372in}{1.432215in}} %
\pgfusepath{clip}%
\pgfsetbuttcap%
\pgfsetroundjoin%
\pgfsetlinewidth{1.505625pt}%
\definecolor{currentstroke}{rgb}{0.168627,0.670588,0.494118}%
\pgfsetstrokecolor{currentstroke}%
\pgfsetdash{}{0pt}%
\pgfpathmoveto{\pgfqpoint{1.435999in}{1.195970in}}%
\pgfpathlineto{\pgfqpoint{1.435999in}{1.248860in}}%
\pgfusepath{stroke}%
\end{pgfscope}%
\begin{pgfscope}%
\pgfpathrectangle{\pgfqpoint{0.566985in}{0.161328in}}{\pgfqpoint{2.317372in}{1.432215in}} %
\pgfusepath{clip}%
\pgfsetbuttcap%
\pgfsetroundjoin%
\pgfsetlinewidth{1.505625pt}%
\definecolor{currentstroke}{rgb}{1.000000,0.494118,0.250980}%
\pgfsetstrokecolor{currentstroke}%
\pgfsetdash{}{0pt}%
\pgfpathmoveto{\pgfqpoint{2.015342in}{1.190497in}}%
\pgfpathlineto{\pgfqpoint{2.015342in}{1.238442in}}%
\pgfusepath{stroke}%
\end{pgfscope}%
\begin{pgfscope}%
\pgfpathrectangle{\pgfqpoint{0.566985in}{0.161328in}}{\pgfqpoint{2.317372in}{1.432215in}} %
\pgfusepath{clip}%
\pgfsetbuttcap%
\pgfsetroundjoin%
\pgfsetlinewidth{1.505625pt}%
\definecolor{currentstroke}{rgb}{1.000000,0.694118,0.250980}%
\pgfsetstrokecolor{currentstroke}%
\pgfsetdash{}{0pt}%
\pgfpathmoveto{\pgfqpoint{2.594685in}{1.130103in}}%
\pgfpathlineto{\pgfqpoint{2.594685in}{1.163239in}}%
\pgfusepath{stroke}%
\end{pgfscope}%
\begin{pgfscope}%
\pgfpathrectangle{\pgfqpoint{0.566985in}{0.161328in}}{\pgfqpoint{2.317372in}{1.432215in}} %
\pgfusepath{clip}%
\pgfsetbuttcap%
\pgfsetroundjoin%
\definecolor{currentfill}{rgb}{0.200000,0.427451,0.650980}%
\pgfsetfillcolor{currentfill}%
\pgfsetlinewidth{1.505625pt}%
\definecolor{currentstroke}{rgb}{0.200000,0.427451,0.650980}%
\pgfsetstrokecolor{currentstroke}%
\pgfsetdash{}{0pt}%
\pgfsys@defobject{currentmarker}{\pgfqpoint{-0.111111in}{-0.000000in}}{\pgfqpoint{0.111111in}{0.000000in}}{%
\pgfpathmoveto{\pgfqpoint{0.111111in}{-0.000000in}}%
\pgfpathlineto{\pgfqpoint{-0.111111in}{0.000000in}}%
\pgfusepath{stroke,fill}%
}%
\begin{pgfscope}%
\pgfsys@transformshift{0.856656in}{1.242928in}%
\pgfsys@useobject{currentmarker}{}%
\end{pgfscope}%
\end{pgfscope}%
\begin{pgfscope}%
\pgfpathrectangle{\pgfqpoint{0.566985in}{0.161328in}}{\pgfqpoint{2.317372in}{1.432215in}} %
\pgfusepath{clip}%
\pgfsetbuttcap%
\pgfsetroundjoin%
\definecolor{currentfill}{rgb}{0.200000,0.427451,0.650980}%
\pgfsetfillcolor{currentfill}%
\pgfsetlinewidth{1.505625pt}%
\definecolor{currentstroke}{rgb}{0.200000,0.427451,0.650980}%
\pgfsetstrokecolor{currentstroke}%
\pgfsetdash{}{0pt}%
\pgfsys@defobject{currentmarker}{\pgfqpoint{-0.111111in}{-0.000000in}}{\pgfqpoint{0.111111in}{0.000000in}}{%
\pgfpathmoveto{\pgfqpoint{0.111111in}{-0.000000in}}%
\pgfpathlineto{\pgfqpoint{-0.111111in}{0.000000in}}%
\pgfusepath{stroke,fill}%
}%
\begin{pgfscope}%
\pgfsys@transformshift{0.856656in}{1.276489in}%
\pgfsys@useobject{currentmarker}{}%
\end{pgfscope}%
\end{pgfscope}%
\begin{pgfscope}%
\pgfpathrectangle{\pgfqpoint{0.566985in}{0.161328in}}{\pgfqpoint{2.317372in}{1.432215in}} %
\pgfusepath{clip}%
\pgfsetbuttcap%
\pgfsetroundjoin%
\definecolor{currentfill}{rgb}{0.168627,0.670588,0.494118}%
\pgfsetfillcolor{currentfill}%
\pgfsetlinewidth{1.505625pt}%
\definecolor{currentstroke}{rgb}{0.168627,0.670588,0.494118}%
\pgfsetstrokecolor{currentstroke}%
\pgfsetdash{}{0pt}%
\pgfsys@defobject{currentmarker}{\pgfqpoint{-0.111111in}{-0.000000in}}{\pgfqpoint{0.111111in}{0.000000in}}{%
\pgfpathmoveto{\pgfqpoint{0.111111in}{-0.000000in}}%
\pgfpathlineto{\pgfqpoint{-0.111111in}{0.000000in}}%
\pgfusepath{stroke,fill}%
}%
\begin{pgfscope}%
\pgfsys@transformshift{1.435999in}{1.195970in}%
\pgfsys@useobject{currentmarker}{}%
\end{pgfscope}%
\end{pgfscope}%
\begin{pgfscope}%
\pgfpathrectangle{\pgfqpoint{0.566985in}{0.161328in}}{\pgfqpoint{2.317372in}{1.432215in}} %
\pgfusepath{clip}%
\pgfsetbuttcap%
\pgfsetroundjoin%
\definecolor{currentfill}{rgb}{0.168627,0.670588,0.494118}%
\pgfsetfillcolor{currentfill}%
\pgfsetlinewidth{1.505625pt}%
\definecolor{currentstroke}{rgb}{0.168627,0.670588,0.494118}%
\pgfsetstrokecolor{currentstroke}%
\pgfsetdash{}{0pt}%
\pgfsys@defobject{currentmarker}{\pgfqpoint{-0.111111in}{-0.000000in}}{\pgfqpoint{0.111111in}{0.000000in}}{%
\pgfpathmoveto{\pgfqpoint{0.111111in}{-0.000000in}}%
\pgfpathlineto{\pgfqpoint{-0.111111in}{0.000000in}}%
\pgfusepath{stroke,fill}%
}%
\begin{pgfscope}%
\pgfsys@transformshift{1.435999in}{1.248860in}%
\pgfsys@useobject{currentmarker}{}%
\end{pgfscope}%
\end{pgfscope}%
\begin{pgfscope}%
\pgfpathrectangle{\pgfqpoint{0.566985in}{0.161328in}}{\pgfqpoint{2.317372in}{1.432215in}} %
\pgfusepath{clip}%
\pgfsetbuttcap%
\pgfsetroundjoin%
\definecolor{currentfill}{rgb}{1.000000,0.494118,0.250980}%
\pgfsetfillcolor{currentfill}%
\pgfsetlinewidth{1.505625pt}%
\definecolor{currentstroke}{rgb}{1.000000,0.494118,0.250980}%
\pgfsetstrokecolor{currentstroke}%
\pgfsetdash{}{0pt}%
\pgfsys@defobject{currentmarker}{\pgfqpoint{-0.111111in}{-0.000000in}}{\pgfqpoint{0.111111in}{0.000000in}}{%
\pgfpathmoveto{\pgfqpoint{0.111111in}{-0.000000in}}%
\pgfpathlineto{\pgfqpoint{-0.111111in}{0.000000in}}%
\pgfusepath{stroke,fill}%
}%
\begin{pgfscope}%
\pgfsys@transformshift{2.015342in}{1.190497in}%
\pgfsys@useobject{currentmarker}{}%
\end{pgfscope}%
\end{pgfscope}%
\begin{pgfscope}%
\pgfpathrectangle{\pgfqpoint{0.566985in}{0.161328in}}{\pgfqpoint{2.317372in}{1.432215in}} %
\pgfusepath{clip}%
\pgfsetbuttcap%
\pgfsetroundjoin%
\definecolor{currentfill}{rgb}{1.000000,0.494118,0.250980}%
\pgfsetfillcolor{currentfill}%
\pgfsetlinewidth{1.505625pt}%
\definecolor{currentstroke}{rgb}{1.000000,0.494118,0.250980}%
\pgfsetstrokecolor{currentstroke}%
\pgfsetdash{}{0pt}%
\pgfsys@defobject{currentmarker}{\pgfqpoint{-0.111111in}{-0.000000in}}{\pgfqpoint{0.111111in}{0.000000in}}{%
\pgfpathmoveto{\pgfqpoint{0.111111in}{-0.000000in}}%
\pgfpathlineto{\pgfqpoint{-0.111111in}{0.000000in}}%
\pgfusepath{stroke,fill}%
}%
\begin{pgfscope}%
\pgfsys@transformshift{2.015342in}{1.238442in}%
\pgfsys@useobject{currentmarker}{}%
\end{pgfscope}%
\end{pgfscope}%
\begin{pgfscope}%
\pgfpathrectangle{\pgfqpoint{0.566985in}{0.161328in}}{\pgfqpoint{2.317372in}{1.432215in}} %
\pgfusepath{clip}%
\pgfsetbuttcap%
\pgfsetroundjoin%
\definecolor{currentfill}{rgb}{1.000000,0.694118,0.250980}%
\pgfsetfillcolor{currentfill}%
\pgfsetlinewidth{1.505625pt}%
\definecolor{currentstroke}{rgb}{1.000000,0.694118,0.250980}%
\pgfsetstrokecolor{currentstroke}%
\pgfsetdash{}{0pt}%
\pgfsys@defobject{currentmarker}{\pgfqpoint{-0.111111in}{-0.000000in}}{\pgfqpoint{0.111111in}{0.000000in}}{%
\pgfpathmoveto{\pgfqpoint{0.111111in}{-0.000000in}}%
\pgfpathlineto{\pgfqpoint{-0.111111in}{0.000000in}}%
\pgfusepath{stroke,fill}%
}%
\begin{pgfscope}%
\pgfsys@transformshift{2.594685in}{1.130103in}%
\pgfsys@useobject{currentmarker}{}%
\end{pgfscope}%
\end{pgfscope}%
\begin{pgfscope}%
\pgfpathrectangle{\pgfqpoint{0.566985in}{0.161328in}}{\pgfqpoint{2.317372in}{1.432215in}} %
\pgfusepath{clip}%
\pgfsetbuttcap%
\pgfsetroundjoin%
\definecolor{currentfill}{rgb}{1.000000,0.694118,0.250980}%
\pgfsetfillcolor{currentfill}%
\pgfsetlinewidth{1.505625pt}%
\definecolor{currentstroke}{rgb}{1.000000,0.694118,0.250980}%
\pgfsetstrokecolor{currentstroke}%
\pgfsetdash{}{0pt}%
\pgfsys@defobject{currentmarker}{\pgfqpoint{-0.111111in}{-0.000000in}}{\pgfqpoint{0.111111in}{0.000000in}}{%
\pgfpathmoveto{\pgfqpoint{0.111111in}{-0.000000in}}%
\pgfpathlineto{\pgfqpoint{-0.111111in}{0.000000in}}%
\pgfusepath{stroke,fill}%
}%
\begin{pgfscope}%
\pgfsys@transformshift{2.594685in}{1.163239in}%
\pgfsys@useobject{currentmarker}{}%
\end{pgfscope}%
\end{pgfscope}%
\begin{pgfscope}%
\pgfsetrectcap%
\pgfsetmiterjoin%
\pgfsetlinewidth{1.254687pt}%
\definecolor{currentstroke}{rgb}{0.150000,0.150000,0.150000}%
\pgfsetstrokecolor{currentstroke}%
\pgfsetdash{}{0pt}%
\pgfpathmoveto{\pgfqpoint{0.566985in}{0.161328in}}%
\pgfpathlineto{\pgfqpoint{0.566985in}{1.593542in}}%
\pgfusepath{stroke}%
\end{pgfscope}%
\begin{pgfscope}%
\pgfsetrectcap%
\pgfsetmiterjoin%
\pgfsetlinewidth{1.254687pt}%
\definecolor{currentstroke}{rgb}{0.150000,0.150000,0.150000}%
\pgfsetstrokecolor{currentstroke}%
\pgfsetdash{}{0pt}%
\pgfpathmoveto{\pgfqpoint{0.566985in}{0.161328in}}%
\pgfpathlineto{\pgfqpoint{2.884357in}{0.161328in}}%
\pgfusepath{stroke}%
\end{pgfscope}%
\begin{pgfscope}%
\pgfsetbuttcap%
\pgfsetmiterjoin%
\definecolor{currentfill}{rgb}{0.200000,0.427451,0.650980}%
\pgfsetfillcolor{currentfill}%
\pgfsetlinewidth{1.505625pt}%
\definecolor{currentstroke}{rgb}{0.200000,0.427451,0.650980}%
\pgfsetstrokecolor{currentstroke}%
\pgfsetdash{}{0pt}%
\pgfpathmoveto{\pgfqpoint{0.435247in}{2.221232in}}%
\pgfpathlineto{\pgfqpoint{0.546358in}{2.221232in}}%
\pgfpathlineto{\pgfqpoint{0.546358in}{2.299010in}}%
\pgfpathlineto{\pgfqpoint{0.435247in}{2.299010in}}%
\pgfpathclose%
\pgfusepath{stroke,fill}%
\end{pgfscope}%
\begin{pgfscope}%
\definecolor{textcolor}{rgb}{0.150000,0.150000,0.150000}%
\pgfsetstrokecolor{textcolor}%
\pgfsetfillcolor{textcolor}%
\pgftext[x=0.635247in,y=2.221232in,left,base]{\color{textcolor}\rmfamily\fontsize{8.000000}{9.600000}\selectfont WT + Vehicle (10)}%
\end{pgfscope}%
\begin{pgfscope}%
\pgfsetbuttcap%
\pgfsetmiterjoin%
\definecolor{currentfill}{rgb}{1.000000,0.494118,0.250980}%
\pgfsetfillcolor{currentfill}%
\pgfsetlinewidth{1.505625pt}%
\definecolor{currentstroke}{rgb}{1.000000,0.494118,0.250980}%
\pgfsetstrokecolor{currentstroke}%
\pgfsetdash{}{0pt}%
\pgfpathmoveto{\pgfqpoint{0.435247in}{1.887953in}}%
\pgfpathlineto{\pgfqpoint{0.546358in}{1.887953in}}%
\pgfpathlineto{\pgfqpoint{0.546358in}{1.965731in}}%
\pgfpathlineto{\pgfqpoint{0.435247in}{1.965731in}}%
\pgfpathclose%
\pgfusepath{stroke,fill}%
\end{pgfscope}%
\begin{pgfscope}%
\definecolor{textcolor}{rgb}{0.150000,0.150000,0.150000}%
\pgfsetstrokecolor{textcolor}%
\pgfsetfillcolor{textcolor}%
\pgftext[x=0.635247in,y=1.887953in,left,base]{\color{textcolor}\rmfamily\fontsize{8.000000}{9.600000}\selectfont Tg + Vehicle (6)}%
\end{pgfscope}%
\end{pgfpicture}%
\makeatother%
\endgroup%

        \caption{\label{f.ad.nb}}
    \end{subfigure}
    \begin{subfigure}[b]{0.5\textwidth}
        %% Creator: Matplotlib, PGF backend
%%
%% To include the figure in your LaTeX document, write
%%   \input{<filename>.pgf}
%%
%% Make sure the required packages are loaded in your preamble
%%   \usepackage{pgf}
%%
%% Figures using additional raster images can only be included by \input if
%% they are in the same directory as the main LaTeX file. For loading figures
%% from other directories you can use the `import` package
%%   \usepackage{import}
%% and then include the figures with
%%   \import{<path to file>}{<filename>.pgf}
%%
%% Matplotlib used the following preamble
%%   \usepackage[utf8]{inputenc}
%%   \usepackage[T1]{fontenc}
%%   \usepackage{siunitx}
%%
\begingroup%
\makeatletter%
\begin{pgfpicture}%
\pgfpathrectangle{\pgfpointorigin}{\pgfqpoint{2.984357in}{2.447513in}}%
\pgfusepath{use as bounding box, clip}%
\begin{pgfscope}%
\pgfsetbuttcap%
\pgfsetmiterjoin%
\definecolor{currentfill}{rgb}{1.000000,1.000000,1.000000}%
\pgfsetfillcolor{currentfill}%
\pgfsetlinewidth{0.000000pt}%
\definecolor{currentstroke}{rgb}{1.000000,1.000000,1.000000}%
\pgfsetstrokecolor{currentstroke}%
\pgfsetdash{}{0pt}%
\pgfpathmoveto{\pgfqpoint{0.000000in}{0.000000in}}%
\pgfpathlineto{\pgfqpoint{2.984357in}{0.000000in}}%
\pgfpathlineto{\pgfqpoint{2.984357in}{2.447513in}}%
\pgfpathlineto{\pgfqpoint{0.000000in}{2.447513in}}%
\pgfpathclose%
\pgfusepath{fill}%
\end{pgfscope}%
\begin{pgfscope}%
\pgfsetbuttcap%
\pgfsetmiterjoin%
\definecolor{currentfill}{rgb}{1.000000,1.000000,1.000000}%
\pgfsetfillcolor{currentfill}%
\pgfsetlinewidth{0.000000pt}%
\definecolor{currentstroke}{rgb}{0.000000,0.000000,0.000000}%
\pgfsetstrokecolor{currentstroke}%
\pgfsetstrokeopacity{0.000000}%
\pgfsetdash{}{0pt}%
\pgfpathmoveto{\pgfqpoint{0.566985in}{0.161328in}}%
\pgfpathlineto{\pgfqpoint{2.884357in}{0.161328in}}%
\pgfpathlineto{\pgfqpoint{2.884357in}{1.593542in}}%
\pgfpathlineto{\pgfqpoint{0.566985in}{1.593542in}}%
\pgfpathclose%
\pgfusepath{fill}%
\end{pgfscope}%
\begin{pgfscope}%
\pgfsetbuttcap%
\pgfsetroundjoin%
\definecolor{currentfill}{rgb}{0.150000,0.150000,0.150000}%
\pgfsetfillcolor{currentfill}%
\pgfsetlinewidth{1.003750pt}%
\definecolor{currentstroke}{rgb}{0.150000,0.150000,0.150000}%
\pgfsetstrokecolor{currentstroke}%
\pgfsetdash{}{0pt}%
\pgfsys@defobject{currentmarker}{\pgfqpoint{0.000000in}{0.000000in}}{\pgfqpoint{0.041667in}{0.000000in}}{%
\pgfpathmoveto{\pgfqpoint{0.000000in}{0.000000in}}%
\pgfpathlineto{\pgfqpoint{0.041667in}{0.000000in}}%
\pgfusepath{stroke,fill}%
}%
\begin{pgfscope}%
\pgfsys@transformshift{0.566985in}{0.161328in}%
\pgfsys@useobject{currentmarker}{}%
\end{pgfscope}%
\end{pgfscope}%
\begin{pgfscope}%
\definecolor{textcolor}{rgb}{0.150000,0.150000,0.150000}%
\pgfsetstrokecolor{textcolor}%
\pgfsetfillcolor{textcolor}%
\pgftext[x=0.469762in,y=0.161328in,right,]{\color{textcolor}\rmfamily\fontsize{10.000000}{12.000000}\selectfont \(\displaystyle 0.0\)}%
\end{pgfscope}%
\begin{pgfscope}%
\pgfsetbuttcap%
\pgfsetroundjoin%
\definecolor{currentfill}{rgb}{0.150000,0.150000,0.150000}%
\pgfsetfillcolor{currentfill}%
\pgfsetlinewidth{1.003750pt}%
\definecolor{currentstroke}{rgb}{0.150000,0.150000,0.150000}%
\pgfsetstrokecolor{currentstroke}%
\pgfsetdash{}{0pt}%
\pgfsys@defobject{currentmarker}{\pgfqpoint{0.000000in}{0.000000in}}{\pgfqpoint{0.041667in}{0.000000in}}{%
\pgfpathmoveto{\pgfqpoint{0.000000in}{0.000000in}}%
\pgfpathlineto{\pgfqpoint{0.041667in}{0.000000in}}%
\pgfusepath{stroke,fill}%
}%
\begin{pgfscope}%
\pgfsys@transformshift{0.566985in}{0.447771in}%
\pgfsys@useobject{currentmarker}{}%
\end{pgfscope}%
\end{pgfscope}%
\begin{pgfscope}%
\definecolor{textcolor}{rgb}{0.150000,0.150000,0.150000}%
\pgfsetstrokecolor{textcolor}%
\pgfsetfillcolor{textcolor}%
\pgftext[x=0.469762in,y=0.447771in,right,]{\color{textcolor}\rmfamily\fontsize{10.000000}{12.000000}\selectfont \(\displaystyle 0.2\)}%
\end{pgfscope}%
\begin{pgfscope}%
\pgfsetbuttcap%
\pgfsetroundjoin%
\definecolor{currentfill}{rgb}{0.150000,0.150000,0.150000}%
\pgfsetfillcolor{currentfill}%
\pgfsetlinewidth{1.003750pt}%
\definecolor{currentstroke}{rgb}{0.150000,0.150000,0.150000}%
\pgfsetstrokecolor{currentstroke}%
\pgfsetdash{}{0pt}%
\pgfsys@defobject{currentmarker}{\pgfqpoint{0.000000in}{0.000000in}}{\pgfqpoint{0.041667in}{0.000000in}}{%
\pgfpathmoveto{\pgfqpoint{0.000000in}{0.000000in}}%
\pgfpathlineto{\pgfqpoint{0.041667in}{0.000000in}}%
\pgfusepath{stroke,fill}%
}%
\begin{pgfscope}%
\pgfsys@transformshift{0.566985in}{0.734213in}%
\pgfsys@useobject{currentmarker}{}%
\end{pgfscope}%
\end{pgfscope}%
\begin{pgfscope}%
\definecolor{textcolor}{rgb}{0.150000,0.150000,0.150000}%
\pgfsetstrokecolor{textcolor}%
\pgfsetfillcolor{textcolor}%
\pgftext[x=0.469762in,y=0.734213in,right,]{\color{textcolor}\rmfamily\fontsize{10.000000}{12.000000}\selectfont \(\displaystyle 0.4\)}%
\end{pgfscope}%
\begin{pgfscope}%
\pgfsetbuttcap%
\pgfsetroundjoin%
\definecolor{currentfill}{rgb}{0.150000,0.150000,0.150000}%
\pgfsetfillcolor{currentfill}%
\pgfsetlinewidth{1.003750pt}%
\definecolor{currentstroke}{rgb}{0.150000,0.150000,0.150000}%
\pgfsetstrokecolor{currentstroke}%
\pgfsetdash{}{0pt}%
\pgfsys@defobject{currentmarker}{\pgfqpoint{0.000000in}{0.000000in}}{\pgfqpoint{0.041667in}{0.000000in}}{%
\pgfpathmoveto{\pgfqpoint{0.000000in}{0.000000in}}%
\pgfpathlineto{\pgfqpoint{0.041667in}{0.000000in}}%
\pgfusepath{stroke,fill}%
}%
\begin{pgfscope}%
\pgfsys@transformshift{0.566985in}{1.020656in}%
\pgfsys@useobject{currentmarker}{}%
\end{pgfscope}%
\end{pgfscope}%
\begin{pgfscope}%
\definecolor{textcolor}{rgb}{0.150000,0.150000,0.150000}%
\pgfsetstrokecolor{textcolor}%
\pgfsetfillcolor{textcolor}%
\pgftext[x=0.469762in,y=1.020656in,right,]{\color{textcolor}\rmfamily\fontsize{10.000000}{12.000000}\selectfont \(\displaystyle 0.6\)}%
\end{pgfscope}%
\begin{pgfscope}%
\pgfsetbuttcap%
\pgfsetroundjoin%
\definecolor{currentfill}{rgb}{0.150000,0.150000,0.150000}%
\pgfsetfillcolor{currentfill}%
\pgfsetlinewidth{1.003750pt}%
\definecolor{currentstroke}{rgb}{0.150000,0.150000,0.150000}%
\pgfsetstrokecolor{currentstroke}%
\pgfsetdash{}{0pt}%
\pgfsys@defobject{currentmarker}{\pgfqpoint{0.000000in}{0.000000in}}{\pgfqpoint{0.041667in}{0.000000in}}{%
\pgfpathmoveto{\pgfqpoint{0.000000in}{0.000000in}}%
\pgfpathlineto{\pgfqpoint{0.041667in}{0.000000in}}%
\pgfusepath{stroke,fill}%
}%
\begin{pgfscope}%
\pgfsys@transformshift{0.566985in}{1.307099in}%
\pgfsys@useobject{currentmarker}{}%
\end{pgfscope}%
\end{pgfscope}%
\begin{pgfscope}%
\definecolor{textcolor}{rgb}{0.150000,0.150000,0.150000}%
\pgfsetstrokecolor{textcolor}%
\pgfsetfillcolor{textcolor}%
\pgftext[x=0.469762in,y=1.307099in,right,]{\color{textcolor}\rmfamily\fontsize{10.000000}{12.000000}\selectfont \(\displaystyle 0.8\)}%
\end{pgfscope}%
\begin{pgfscope}%
\pgfsetbuttcap%
\pgfsetroundjoin%
\definecolor{currentfill}{rgb}{0.150000,0.150000,0.150000}%
\pgfsetfillcolor{currentfill}%
\pgfsetlinewidth{1.003750pt}%
\definecolor{currentstroke}{rgb}{0.150000,0.150000,0.150000}%
\pgfsetstrokecolor{currentstroke}%
\pgfsetdash{}{0pt}%
\pgfsys@defobject{currentmarker}{\pgfqpoint{0.000000in}{0.000000in}}{\pgfqpoint{0.041667in}{0.000000in}}{%
\pgfpathmoveto{\pgfqpoint{0.000000in}{0.000000in}}%
\pgfpathlineto{\pgfqpoint{0.041667in}{0.000000in}}%
\pgfusepath{stroke,fill}%
}%
\begin{pgfscope}%
\pgfsys@transformshift{0.566985in}{1.593542in}%
\pgfsys@useobject{currentmarker}{}%
\end{pgfscope}%
\end{pgfscope}%
\begin{pgfscope}%
\definecolor{textcolor}{rgb}{0.150000,0.150000,0.150000}%
\pgfsetstrokecolor{textcolor}%
\pgfsetfillcolor{textcolor}%
\pgftext[x=0.469762in,y=1.593542in,right,]{\color{textcolor}\rmfamily\fontsize{10.000000}{12.000000}\selectfont \(\displaystyle 1.0\)}%
\end{pgfscope}%
\begin{pgfscope}%
\definecolor{textcolor}{rgb}{0.150000,0.150000,0.150000}%
\pgfsetstrokecolor{textcolor}%
\pgfsetfillcolor{textcolor}%
\pgftext[x=0.222848in,y=0.877435in,,bottom,rotate=90.000000]{\color{textcolor}\rmfamily\fontsize{10.000000}{12.000000}\selectfont \textbf{gSVM accuracy}}%
\end{pgfscope}%
\begin{pgfscope}%
\pgfpathrectangle{\pgfqpoint{0.566985in}{0.161328in}}{\pgfqpoint{2.317372in}{1.432215in}} %
\pgfusepath{clip}%
\pgfsetbuttcap%
\pgfsetmiterjoin%
\definecolor{currentfill}{rgb}{0.200000,0.427451,0.650980}%
\pgfsetfillcolor{currentfill}%
\pgfsetlinewidth{1.505625pt}%
\definecolor{currentstroke}{rgb}{0.200000,0.427451,0.650980}%
\pgfsetstrokecolor{currentstroke}%
\pgfsetdash{}{0pt}%
\pgfpathmoveto{\pgfqpoint{0.649748in}{0.161328in}}%
\pgfpathlineto{\pgfqpoint{1.063564in}{0.161328in}}%
\pgfpathlineto{\pgfqpoint{1.063564in}{1.517204in}}%
\pgfpathlineto{\pgfqpoint{0.649748in}{1.517204in}}%
\pgfpathclose%
\pgfusepath{stroke,fill}%
\end{pgfscope}%
\begin{pgfscope}%
\pgfpathrectangle{\pgfqpoint{0.566985in}{0.161328in}}{\pgfqpoint{2.317372in}{1.432215in}} %
\pgfusepath{clip}%
\pgfsetbuttcap%
\pgfsetmiterjoin%
\definecolor{currentfill}{rgb}{0.168627,0.670588,0.494118}%
\pgfsetfillcolor{currentfill}%
\pgfsetlinewidth{1.505625pt}%
\definecolor{currentstroke}{rgb}{0.168627,0.670588,0.494118}%
\pgfsetstrokecolor{currentstroke}%
\pgfsetdash{}{0pt}%
\pgfpathmoveto{\pgfqpoint{1.229091in}{0.161328in}}%
\pgfpathlineto{\pgfqpoint{1.642907in}{0.161328in}}%
\pgfpathlineto{\pgfqpoint{1.642907in}{1.466077in}}%
\pgfpathlineto{\pgfqpoint{1.229091in}{1.466077in}}%
\pgfpathclose%
\pgfusepath{stroke,fill}%
\end{pgfscope}%
\begin{pgfscope}%
\pgfpathrectangle{\pgfqpoint{0.566985in}{0.161328in}}{\pgfqpoint{2.317372in}{1.432215in}} %
\pgfusepath{clip}%
\pgfsetbuttcap%
\pgfsetmiterjoin%
\definecolor{currentfill}{rgb}{1.000000,0.494118,0.250980}%
\pgfsetfillcolor{currentfill}%
\pgfsetlinewidth{1.505625pt}%
\definecolor{currentstroke}{rgb}{1.000000,0.494118,0.250980}%
\pgfsetstrokecolor{currentstroke}%
\pgfsetdash{}{0pt}%
\pgfpathmoveto{\pgfqpoint{1.808434in}{0.161328in}}%
\pgfpathlineto{\pgfqpoint{2.222250in}{0.161328in}}%
\pgfpathlineto{\pgfqpoint{2.222250in}{1.461698in}}%
\pgfpathlineto{\pgfqpoint{1.808434in}{1.461698in}}%
\pgfpathclose%
\pgfusepath{stroke,fill}%
\end{pgfscope}%
\begin{pgfscope}%
\pgfpathrectangle{\pgfqpoint{0.566985in}{0.161328in}}{\pgfqpoint{2.317372in}{1.432215in}} %
\pgfusepath{clip}%
\pgfsetbuttcap%
\pgfsetmiterjoin%
\definecolor{currentfill}{rgb}{1.000000,0.694118,0.250980}%
\pgfsetfillcolor{currentfill}%
\pgfsetlinewidth{1.505625pt}%
\definecolor{currentstroke}{rgb}{1.000000,0.694118,0.250980}%
\pgfsetstrokecolor{currentstroke}%
\pgfsetdash{}{0pt}%
\pgfpathmoveto{\pgfqpoint{2.387777in}{0.161328in}}%
\pgfpathlineto{\pgfqpoint{2.801593in}{0.161328in}}%
\pgfpathlineto{\pgfqpoint{2.801593in}{1.424681in}}%
\pgfpathlineto{\pgfqpoint{2.387777in}{1.424681in}}%
\pgfpathclose%
\pgfusepath{stroke,fill}%
\end{pgfscope}%
\begin{pgfscope}%
\pgfpathrectangle{\pgfqpoint{0.566985in}{0.161328in}}{\pgfqpoint{2.317372in}{1.432215in}} %
\pgfusepath{clip}%
\pgfsetbuttcap%
\pgfsetroundjoin%
\pgfsetlinewidth{1.505625pt}%
\definecolor{currentstroke}{rgb}{0.200000,0.427451,0.650980}%
\pgfsetstrokecolor{currentstroke}%
\pgfsetdash{}{0pt}%
\pgfpathmoveto{\pgfqpoint{0.856656in}{1.517204in}}%
\pgfpathlineto{\pgfqpoint{0.856656in}{1.537558in}}%
\pgfusepath{stroke}%
\end{pgfscope}%
\begin{pgfscope}%
\pgfpathrectangle{\pgfqpoint{0.566985in}{0.161328in}}{\pgfqpoint{2.317372in}{1.432215in}} %
\pgfusepath{clip}%
\pgfsetbuttcap%
\pgfsetroundjoin%
\pgfsetlinewidth{1.505625pt}%
\definecolor{currentstroke}{rgb}{0.168627,0.670588,0.494118}%
\pgfsetstrokecolor{currentstroke}%
\pgfsetdash{}{0pt}%
\pgfpathmoveto{\pgfqpoint{1.435999in}{1.466077in}}%
\pgfpathlineto{\pgfqpoint{1.435999in}{1.503432in}}%
\pgfusepath{stroke}%
\end{pgfscope}%
\begin{pgfscope}%
\pgfpathrectangle{\pgfqpoint{0.566985in}{0.161328in}}{\pgfqpoint{2.317372in}{1.432215in}} %
\pgfusepath{clip}%
\pgfsetbuttcap%
\pgfsetroundjoin%
\pgfsetlinewidth{1.505625pt}%
\definecolor{currentstroke}{rgb}{1.000000,0.494118,0.250980}%
\pgfsetstrokecolor{currentstroke}%
\pgfsetdash{}{0pt}%
\pgfpathmoveto{\pgfqpoint{2.015342in}{1.461698in}}%
\pgfpathlineto{\pgfqpoint{2.015342in}{1.503630in}}%
\pgfusepath{stroke}%
\end{pgfscope}%
\begin{pgfscope}%
\pgfpathrectangle{\pgfqpoint{0.566985in}{0.161328in}}{\pgfqpoint{2.317372in}{1.432215in}} %
\pgfusepath{clip}%
\pgfsetbuttcap%
\pgfsetroundjoin%
\pgfsetlinewidth{1.505625pt}%
\definecolor{currentstroke}{rgb}{1.000000,0.694118,0.250980}%
\pgfsetstrokecolor{currentstroke}%
\pgfsetdash{}{0pt}%
\pgfpathmoveto{\pgfqpoint{2.594685in}{1.424681in}}%
\pgfpathlineto{\pgfqpoint{2.594685in}{1.467436in}}%
\pgfusepath{stroke}%
\end{pgfscope}%
\begin{pgfscope}%
\pgfpathrectangle{\pgfqpoint{0.566985in}{0.161328in}}{\pgfqpoint{2.317372in}{1.432215in}} %
\pgfusepath{clip}%
\pgfsetbuttcap%
\pgfsetroundjoin%
\definecolor{currentfill}{rgb}{0.200000,0.427451,0.650980}%
\pgfsetfillcolor{currentfill}%
\pgfsetlinewidth{1.505625pt}%
\definecolor{currentstroke}{rgb}{0.200000,0.427451,0.650980}%
\pgfsetstrokecolor{currentstroke}%
\pgfsetdash{}{0pt}%
\pgfsys@defobject{currentmarker}{\pgfqpoint{-0.111111in}{-0.000000in}}{\pgfqpoint{0.111111in}{0.000000in}}{%
\pgfpathmoveto{\pgfqpoint{0.111111in}{-0.000000in}}%
\pgfpathlineto{\pgfqpoint{-0.111111in}{0.000000in}}%
\pgfusepath{stroke,fill}%
}%
\begin{pgfscope}%
\pgfsys@transformshift{0.856656in}{1.517204in}%
\pgfsys@useobject{currentmarker}{}%
\end{pgfscope}%
\end{pgfscope}%
\begin{pgfscope}%
\pgfpathrectangle{\pgfqpoint{0.566985in}{0.161328in}}{\pgfqpoint{2.317372in}{1.432215in}} %
\pgfusepath{clip}%
\pgfsetbuttcap%
\pgfsetroundjoin%
\definecolor{currentfill}{rgb}{0.200000,0.427451,0.650980}%
\pgfsetfillcolor{currentfill}%
\pgfsetlinewidth{1.505625pt}%
\definecolor{currentstroke}{rgb}{0.200000,0.427451,0.650980}%
\pgfsetstrokecolor{currentstroke}%
\pgfsetdash{}{0pt}%
\pgfsys@defobject{currentmarker}{\pgfqpoint{-0.111111in}{-0.000000in}}{\pgfqpoint{0.111111in}{0.000000in}}{%
\pgfpathmoveto{\pgfqpoint{0.111111in}{-0.000000in}}%
\pgfpathlineto{\pgfqpoint{-0.111111in}{0.000000in}}%
\pgfusepath{stroke,fill}%
}%
\begin{pgfscope}%
\pgfsys@transformshift{0.856656in}{1.537558in}%
\pgfsys@useobject{currentmarker}{}%
\end{pgfscope}%
\end{pgfscope}%
\begin{pgfscope}%
\pgfpathrectangle{\pgfqpoint{0.566985in}{0.161328in}}{\pgfqpoint{2.317372in}{1.432215in}} %
\pgfusepath{clip}%
\pgfsetbuttcap%
\pgfsetroundjoin%
\definecolor{currentfill}{rgb}{0.168627,0.670588,0.494118}%
\pgfsetfillcolor{currentfill}%
\pgfsetlinewidth{1.505625pt}%
\definecolor{currentstroke}{rgb}{0.168627,0.670588,0.494118}%
\pgfsetstrokecolor{currentstroke}%
\pgfsetdash{}{0pt}%
\pgfsys@defobject{currentmarker}{\pgfqpoint{-0.111111in}{-0.000000in}}{\pgfqpoint{0.111111in}{0.000000in}}{%
\pgfpathmoveto{\pgfqpoint{0.111111in}{-0.000000in}}%
\pgfpathlineto{\pgfqpoint{-0.111111in}{0.000000in}}%
\pgfusepath{stroke,fill}%
}%
\begin{pgfscope}%
\pgfsys@transformshift{1.435999in}{1.466077in}%
\pgfsys@useobject{currentmarker}{}%
\end{pgfscope}%
\end{pgfscope}%
\begin{pgfscope}%
\pgfpathrectangle{\pgfqpoint{0.566985in}{0.161328in}}{\pgfqpoint{2.317372in}{1.432215in}} %
\pgfusepath{clip}%
\pgfsetbuttcap%
\pgfsetroundjoin%
\definecolor{currentfill}{rgb}{0.168627,0.670588,0.494118}%
\pgfsetfillcolor{currentfill}%
\pgfsetlinewidth{1.505625pt}%
\definecolor{currentstroke}{rgb}{0.168627,0.670588,0.494118}%
\pgfsetstrokecolor{currentstroke}%
\pgfsetdash{}{0pt}%
\pgfsys@defobject{currentmarker}{\pgfqpoint{-0.111111in}{-0.000000in}}{\pgfqpoint{0.111111in}{0.000000in}}{%
\pgfpathmoveto{\pgfqpoint{0.111111in}{-0.000000in}}%
\pgfpathlineto{\pgfqpoint{-0.111111in}{0.000000in}}%
\pgfusepath{stroke,fill}%
}%
\begin{pgfscope}%
\pgfsys@transformshift{1.435999in}{1.503432in}%
\pgfsys@useobject{currentmarker}{}%
\end{pgfscope}%
\end{pgfscope}%
\begin{pgfscope}%
\pgfpathrectangle{\pgfqpoint{0.566985in}{0.161328in}}{\pgfqpoint{2.317372in}{1.432215in}} %
\pgfusepath{clip}%
\pgfsetbuttcap%
\pgfsetroundjoin%
\definecolor{currentfill}{rgb}{1.000000,0.494118,0.250980}%
\pgfsetfillcolor{currentfill}%
\pgfsetlinewidth{1.505625pt}%
\definecolor{currentstroke}{rgb}{1.000000,0.494118,0.250980}%
\pgfsetstrokecolor{currentstroke}%
\pgfsetdash{}{0pt}%
\pgfsys@defobject{currentmarker}{\pgfqpoint{-0.111111in}{-0.000000in}}{\pgfqpoint{0.111111in}{0.000000in}}{%
\pgfpathmoveto{\pgfqpoint{0.111111in}{-0.000000in}}%
\pgfpathlineto{\pgfqpoint{-0.111111in}{0.000000in}}%
\pgfusepath{stroke,fill}%
}%
\begin{pgfscope}%
\pgfsys@transformshift{2.015342in}{1.461698in}%
\pgfsys@useobject{currentmarker}{}%
\end{pgfscope}%
\end{pgfscope}%
\begin{pgfscope}%
\pgfpathrectangle{\pgfqpoint{0.566985in}{0.161328in}}{\pgfqpoint{2.317372in}{1.432215in}} %
\pgfusepath{clip}%
\pgfsetbuttcap%
\pgfsetroundjoin%
\definecolor{currentfill}{rgb}{1.000000,0.494118,0.250980}%
\pgfsetfillcolor{currentfill}%
\pgfsetlinewidth{1.505625pt}%
\definecolor{currentstroke}{rgb}{1.000000,0.494118,0.250980}%
\pgfsetstrokecolor{currentstroke}%
\pgfsetdash{}{0pt}%
\pgfsys@defobject{currentmarker}{\pgfqpoint{-0.111111in}{-0.000000in}}{\pgfqpoint{0.111111in}{0.000000in}}{%
\pgfpathmoveto{\pgfqpoint{0.111111in}{-0.000000in}}%
\pgfpathlineto{\pgfqpoint{-0.111111in}{0.000000in}}%
\pgfusepath{stroke,fill}%
}%
\begin{pgfscope}%
\pgfsys@transformshift{2.015342in}{1.503630in}%
\pgfsys@useobject{currentmarker}{}%
\end{pgfscope}%
\end{pgfscope}%
\begin{pgfscope}%
\pgfpathrectangle{\pgfqpoint{0.566985in}{0.161328in}}{\pgfqpoint{2.317372in}{1.432215in}} %
\pgfusepath{clip}%
\pgfsetbuttcap%
\pgfsetroundjoin%
\definecolor{currentfill}{rgb}{1.000000,0.694118,0.250980}%
\pgfsetfillcolor{currentfill}%
\pgfsetlinewidth{1.505625pt}%
\definecolor{currentstroke}{rgb}{1.000000,0.694118,0.250980}%
\pgfsetstrokecolor{currentstroke}%
\pgfsetdash{}{0pt}%
\pgfsys@defobject{currentmarker}{\pgfqpoint{-0.111111in}{-0.000000in}}{\pgfqpoint{0.111111in}{0.000000in}}{%
\pgfpathmoveto{\pgfqpoint{0.111111in}{-0.000000in}}%
\pgfpathlineto{\pgfqpoint{-0.111111in}{0.000000in}}%
\pgfusepath{stroke,fill}%
}%
\begin{pgfscope}%
\pgfsys@transformshift{2.594685in}{1.424681in}%
\pgfsys@useobject{currentmarker}{}%
\end{pgfscope}%
\end{pgfscope}%
\begin{pgfscope}%
\pgfpathrectangle{\pgfqpoint{0.566985in}{0.161328in}}{\pgfqpoint{2.317372in}{1.432215in}} %
\pgfusepath{clip}%
\pgfsetbuttcap%
\pgfsetroundjoin%
\definecolor{currentfill}{rgb}{1.000000,0.694118,0.250980}%
\pgfsetfillcolor{currentfill}%
\pgfsetlinewidth{1.505625pt}%
\definecolor{currentstroke}{rgb}{1.000000,0.694118,0.250980}%
\pgfsetstrokecolor{currentstroke}%
\pgfsetdash{}{0pt}%
\pgfsys@defobject{currentmarker}{\pgfqpoint{-0.111111in}{-0.000000in}}{\pgfqpoint{0.111111in}{0.000000in}}{%
\pgfpathmoveto{\pgfqpoint{0.111111in}{-0.000000in}}%
\pgfpathlineto{\pgfqpoint{-0.111111in}{0.000000in}}%
\pgfusepath{stroke,fill}%
}%
\begin{pgfscope}%
\pgfsys@transformshift{2.594685in}{1.467436in}%
\pgfsys@useobject{currentmarker}{}%
\end{pgfscope}%
\end{pgfscope}%
\begin{pgfscope}%
\pgfsetrectcap%
\pgfsetmiterjoin%
\pgfsetlinewidth{1.254687pt}%
\definecolor{currentstroke}{rgb}{0.150000,0.150000,0.150000}%
\pgfsetstrokecolor{currentstroke}%
\pgfsetdash{}{0pt}%
\pgfpathmoveto{\pgfqpoint{0.566985in}{0.161328in}}%
\pgfpathlineto{\pgfqpoint{0.566985in}{1.593542in}}%
\pgfusepath{stroke}%
\end{pgfscope}%
\begin{pgfscope}%
\pgfsetrectcap%
\pgfsetmiterjoin%
\pgfsetlinewidth{1.254687pt}%
\definecolor{currentstroke}{rgb}{0.150000,0.150000,0.150000}%
\pgfsetstrokecolor{currentstroke}%
\pgfsetdash{}{0pt}%
\pgfpathmoveto{\pgfqpoint{0.566985in}{0.161328in}}%
\pgfpathlineto{\pgfqpoint{2.884357in}{0.161328in}}%
\pgfusepath{stroke}%
\end{pgfscope}%
\begin{pgfscope}%
\pgfsetbuttcap%
\pgfsetmiterjoin%
\definecolor{currentfill}{rgb}{0.200000,0.427451,0.650980}%
\pgfsetfillcolor{currentfill}%
\pgfsetlinewidth{1.505625pt}%
\definecolor{currentstroke}{rgb}{0.200000,0.427451,0.650980}%
\pgfsetstrokecolor{currentstroke}%
\pgfsetdash{}{0pt}%
\pgfpathmoveto{\pgfqpoint{1.267836in}{2.221232in}}%
\pgfpathlineto{\pgfqpoint{1.378948in}{2.221232in}}%
\pgfpathlineto{\pgfqpoint{1.378948in}{2.299010in}}%
\pgfpathlineto{\pgfqpoint{1.267836in}{2.299010in}}%
\pgfpathclose%
\pgfusepath{stroke,fill}%
\end{pgfscope}%
\begin{pgfscope}%
\definecolor{textcolor}{rgb}{0.150000,0.150000,0.150000}%
\pgfsetstrokecolor{textcolor}%
\pgfsetfillcolor{textcolor}%
\pgftext[x=1.467836in,y=2.221232in,left,base]{\color{textcolor}\rmfamily\fontsize{8.000000}{9.600000}\selectfont WT + Vehicle (10)}%
\end{pgfscope}%
\begin{pgfscope}%
\pgfsetbuttcap%
\pgfsetmiterjoin%
\definecolor{currentfill}{rgb}{0.168627,0.670588,0.494118}%
\pgfsetfillcolor{currentfill}%
\pgfsetlinewidth{1.505625pt}%
\definecolor{currentstroke}{rgb}{0.168627,0.670588,0.494118}%
\pgfsetstrokecolor{currentstroke}%
\pgfsetdash{}{0pt}%
\pgfpathmoveto{\pgfqpoint{1.267836in}{2.054592in}}%
\pgfpathlineto{\pgfqpoint{1.378948in}{2.054592in}}%
\pgfpathlineto{\pgfqpoint{1.378948in}{2.132370in}}%
\pgfpathlineto{\pgfqpoint{1.267836in}{2.132370in}}%
\pgfpathclose%
\pgfusepath{stroke,fill}%
\end{pgfscope}%
\begin{pgfscope}%
\definecolor{textcolor}{rgb}{0.150000,0.150000,0.150000}%
\pgfsetstrokecolor{textcolor}%
\pgfsetfillcolor{textcolor}%
\pgftext[x=1.467836in,y=2.054592in,left,base]{\color{textcolor}\rmfamily\fontsize{8.000000}{9.600000}\selectfont WT + TAT-GluA2\textsubscript{3Y} (7)}%
\end{pgfscope}%
\begin{pgfscope}%
\pgfsetbuttcap%
\pgfsetmiterjoin%
\definecolor{currentfill}{rgb}{1.000000,0.494118,0.250980}%
\pgfsetfillcolor{currentfill}%
\pgfsetlinewidth{1.505625pt}%
\definecolor{currentstroke}{rgb}{1.000000,0.494118,0.250980}%
\pgfsetstrokecolor{currentstroke}%
\pgfsetdash{}{0pt}%
\pgfpathmoveto{\pgfqpoint{1.267836in}{1.887953in}}%
\pgfpathlineto{\pgfqpoint{1.378948in}{1.887953in}}%
\pgfpathlineto{\pgfqpoint{1.378948in}{1.965731in}}%
\pgfpathlineto{\pgfqpoint{1.267836in}{1.965731in}}%
\pgfpathclose%
\pgfusepath{stroke,fill}%
\end{pgfscope}%
\begin{pgfscope}%
\definecolor{textcolor}{rgb}{0.150000,0.150000,0.150000}%
\pgfsetstrokecolor{textcolor}%
\pgfsetfillcolor{textcolor}%
\pgftext[x=1.467836in,y=1.887953in,left,base]{\color{textcolor}\rmfamily\fontsize{8.000000}{9.600000}\selectfont Tg + Vehicle (6)}%
\end{pgfscope}%
\begin{pgfscope}%
\pgfsetbuttcap%
\pgfsetmiterjoin%
\definecolor{currentfill}{rgb}{1.000000,0.694118,0.250980}%
\pgfsetfillcolor{currentfill}%
\pgfsetlinewidth{1.505625pt}%
\definecolor{currentstroke}{rgb}{1.000000,0.694118,0.250980}%
\pgfsetstrokecolor{currentstroke}%
\pgfsetdash{}{0pt}%
\pgfpathmoveto{\pgfqpoint{1.267836in}{1.721313in}}%
\pgfpathlineto{\pgfqpoint{1.378948in}{1.721313in}}%
\pgfpathlineto{\pgfqpoint{1.378948in}{1.799091in}}%
\pgfpathlineto{\pgfqpoint{1.267836in}{1.799091in}}%
\pgfpathclose%
\pgfusepath{stroke,fill}%
\end{pgfscope}%
\begin{pgfscope}%
\definecolor{textcolor}{rgb}{0.150000,0.150000,0.150000}%
\pgfsetstrokecolor{textcolor}%
\pgfsetfillcolor{textcolor}%
\pgftext[x=1.467836in,y=1.721313in,left,base]{\color{textcolor}\rmfamily\fontsize{8.000000}{9.600000}\selectfont Tg + TAT-GluA2\textsubscript{3Y} (7)}%
\end{pgfscope}%
\end{pgfpicture}%
\makeatother%
\endgroup%

        \caption{\label{f.ad.svm}}
    \end{subfigure}
    \caption[Accuracy of machine learning classifier in predicting freezing.]{Performance of \subref{f.ad.nb} \gls{nbc} and \subref{f.ad.svm} \gls{gsvm} in predicting freezing from cell activity. Since \gls{nbc} assumes the cells are independent and \gls{gsvm} is more general, the performance difference between the two suggests a portion of freezing information is encoded at the network level. Both \gls{nbc} and \gls{gsvm} perform similarly across all groups, suggesting that the deficit in information content in \gls{tg} mice can be compensated by the network activity. \label{f.ad.classifier}}
\end{figure}

\begin{comment}
To show that the machine learning classifiers are indeed detecting freezing instead of a lack of movement of the mice, we investigated timepoints where the mice was not moving, but also not freezing. If the classifiers were detecting a lack of movement, this subset of data will be misclassified as freezing, and result in a significant worse performance. The accuracy of the classifiers are shown in Figure~\ref{f.ad.accu-freezing-no-moving}. We have found that while \gls{nbc} shows significantly worse performance in this subset of data, the performance is significantly better than if it were detecting a lack of movement. \Gls{gsvm} on the other hand is able to completely distinguish a lack of movement and freezing, and have similar performance as the rest of the time points. 
\todo{stats}

\begin{figure}[h]
    \begin{subfigure}[h]{\textwidth}
        \input{../figs/ad/nb-freezing-no-moving.pgf}
        \caption{\label{f.ad.nb-freezing-no-moving}}
    \end{subfigure}
    \begin{subfigure}[h]{\textwidth}
        \input{../figs/ad/svm-freezing-no-moving.pgf}
        \caption{\label{f.ad.svm-freezing-no-moving}}
    \end{subfigure}
    \caption[Machine learning classifiers predictions are not based on a lack of movement.]{Performance of \subref{f.ad.nb-freezing-no-moving} \gls{nbc} and \subref{f.ad.svm-freezing-no-moving} \gls{gsvm} in predicting freezing from cell activity, at timepoints where the mice was not moving horizontally, but also not freezing. We found both \gls{nbc} and \gls{gsvm} perform significantly better than if they were detecting a lack of movement instead of freezing. However, while the \gls{gsvm} can completely distinguish lack of movement with freezing, the \gls{nbc} shows some confusion between lack of movement and freezing. \label{f.ad.accu-freezing-no-moving}}

\end{figure}
\end{comment}

\subsection{Freezing encoding precedes freezing behaviour}


\begin{figure}[h]
    \centering
    \begin{minipage}[b]{0.45\linewidth}
        
\begin{tikzpicture}
    \node[obs] (acc) {acc};
    \node[latent, left= of acc] (sig) {$\sigma$};
    \node[det, above= of acc] (mu) {$\mu$};     
    \node[obs, above=of mu] (t) {$t$};
    \node[latent, left=of t] (mu0) {$\mu_0$};
    \node[latent, right= of mu] (s) {$s$};
    \node[latent, left=of mu] (tau) {$\tau$};
    \node[latent, right=of t] (a) {$a$};

    \edge {tau,s,t,mu0,a} {mu};
    \edge {mu, sig} {acc};
   
    \plate{tplate} {(t) (mu) (acc)} {$\forall t \in [T_{min}, 0)$};
\end{tikzpicture}

    \end{minipage}
    \begin{minipage}[b]{0.45\linewidth}
        \begin{align*}
            \mu_0 &\sim \operatorname{Gaussian}(0, 1000) \\
            a &\sim \operatorname{Gaussian}(0, 1000) \\
            a &> 0 \\
            \tau &\sim \operatorname{DiscreteUniform}(T_{min}, 0) \\
            s &\sim \operatorname{Bernoulli}(0.5) \\
            acc &\sim \operatorname{Gaussian}(\mu, \sigma) \\
            \mu &=
                \begin{cases}
                    \mu_0 & \text{if }s=0\text{ or }t>\tau \\
                    \mu_0 - a(t-\tau) & \text{otherwise}
                \end{cases}
        \end{align*}
    \end{minipage}
    \caption[Bayes model for change point detection.]{Bayes model for change point detection. The accuracy is modelled as a Gaussian distribution with mean as a function of time and constant variance. In the null hypothesis, the mean is constant and estimated from the data. In the alternative hypothesis, the mean is modelled as constant up to the change point, then as a linear function of time with a negative slope. A Bernoulli variable $selector$ is estimated to choose from each of the hypothesis. All variables have non-informative priors \label{f.ad.bayesmodel}}
\end{figure}

We next examined whether freezing encoding in the network precedes freezing onset. We plotted average prediction accuracy for the classifiers at the time mice transition into freezing. We hypothesized that if the neural signature for freezing behaviour in the network precedes the behaviour, the classifiers will predict freezing earlier than behavioural onset, which will result in a reduction in prediction accuracy before behavioural onset.

We used Bayes modeling (Figure \ref{f.ad.bayesmodel}) to detect whether the prediction accuracy changes before behaviour change, and compared change points between groups. The result is summarized in Figure~\ref{f.ad.into_f}. We found very strong evidence for a change point in \gls{nbc} performance in every group (all Bayes factors BF\tsb{10} > \num{5e4}). The change point appeared $3.4_{-0.1}^{+0.1}$\SI{}{\s} in WT-Veh, $2.8_{-0.1}^{+0.1}$\SI{}{\second} in WT-\glu, $2.8_{-0.3}^{+0.2}$\SI{}{\second} in Tg-Veh, and $2.6_{-0.1}^{+0.1}$\SI{}{\second} in Tg-\glu{} before freezing onset (uncertainties are \SI{95}{\percent} credible intervals). Pairwise comparisons showed very strong evidence that the change point in WT-Veh occurred earlier than the other three groups, while the other three groups occurred at the same time (WT-Veh vs WT-\glu, BF\tsb{10} > \num{250}; WT-Veh vs Tg-Veh, BF\tsb{10} = \num{7.5}; WT-Veh vs Tg-\glu, BF\tsb{10} > \num{250}; all other comparisons BF\tsb{10} < \num{0.2}). Since \gls{nbc} regards each cell independently, this result suggests that the freezing signal in individual cells occurs earlier than the onset of freezing behaviour.

On the other hand,  we found very strong evidence for a change point in \gls{gsvm} performance in WT-Veh (BF\tsb{10} > \num{5e4}), WT-\glu{} (BF\tsb{10} > \num{5e4}) and Tg-\glu{} (BF\tsb{10} > \num{5e4}), and also strong evidence for the lack of a change point in Tg-Veh mice (BF\tsb{10} < \num{0.02}). The change points were estimated to be $0.6_{-0.1}^{+0.0}$\SI{}{\s} in WT-Veh, $2.6_{-0.3}^{+0.2}$\SI{}{\second} in WT-\glu, and $0.3_{-0.2}^{+0.1}$\SI{}{\s} in Tg-\glu, before the freezing onset. Pairwise comparisons showed very strong evidence of an earlier change point in WT-\glu{} than the other groups (WT-\glu{} vs WT-Veh, BF\tsb{10} > \num{250}; WT-\glu{} vs Tg-\glu, BF\tsb{10} > \num{250}). There is minimal evidence that the change point in WT-Veh was earlier than Tg-\glu{} (BF\tsb{10} = \num{2.5}). These results suggest that the network neural signature for freezing occurs earlier than freezing behaviour in \gls{wt} groups. However this neural code is not detected in the \gls{tg} group. \tglu{} treatment of the \gls{tg} mice is able to partially rescue this effect. In addition, the network freezing signal occurs earlier in WT-\glu{} than the other groups. These results suggest that \tglu{} treatment helps strengthen the network signal, both in \gls{wt} and \gls{tg} mice. 


\begin{figure}[h]
    \begin{subfigure}[h]{\textwidth}
        %% Creator: Matplotlib, PGF backend
%%
%% To include the figure in your LaTeX document, write
%%   \input{<filename>.pgf}
%%
%% Make sure the required packages are loaded in your preamble
%%   \usepackage{pgf}
%%
%% Figures using additional raster images can only be included by \input if
%% they are in the same directory as the main LaTeX file. For loading figures
%% from other directories you can use the `import` package
%%   \usepackage{import}
%% and then include the figures with
%%   \import{<path to file>}{<filename>.pgf}
%%
%% Matplotlib used the following preamble
%%   \usepackage[utf8]{inputenc}
%%   \usepackage[T1]{fontenc}
%%   \usepackage{siunitx}
%%
\begingroup%
\makeatletter%
\begin{pgfpicture}%
\pgfpathrectangle{\pgfpointorigin}{\pgfqpoint{5.301729in}{3.553934in}}%
\pgfusepath{use as bounding box, clip}%
\begin{pgfscope}%
\pgfsetbuttcap%
\pgfsetmiterjoin%
\definecolor{currentfill}{rgb}{1.000000,1.000000,1.000000}%
\pgfsetfillcolor{currentfill}%
\pgfsetlinewidth{0.000000pt}%
\definecolor{currentstroke}{rgb}{1.000000,1.000000,1.000000}%
\pgfsetstrokecolor{currentstroke}%
\pgfsetdash{}{0pt}%
\pgfpathmoveto{\pgfqpoint{0.000000in}{0.000000in}}%
\pgfpathlineto{\pgfqpoint{5.301729in}{0.000000in}}%
\pgfpathlineto{\pgfqpoint{5.301729in}{3.553934in}}%
\pgfpathlineto{\pgfqpoint{0.000000in}{3.553934in}}%
\pgfpathclose%
\pgfusepath{fill}%
\end{pgfscope}%
\begin{pgfscope}%
\pgfsetbuttcap%
\pgfsetmiterjoin%
\definecolor{currentfill}{rgb}{1.000000,1.000000,1.000000}%
\pgfsetfillcolor{currentfill}%
\pgfsetlinewidth{0.000000pt}%
\definecolor{currentstroke}{rgb}{0.000000,0.000000,0.000000}%
\pgfsetstrokecolor{currentstroke}%
\pgfsetstrokeopacity{0.000000}%
\pgfsetdash{}{0pt}%
\pgfpathmoveto{\pgfqpoint{0.566985in}{0.528177in}}%
\pgfpathlineto{\pgfqpoint{3.253793in}{0.528177in}}%
\pgfpathlineto{\pgfqpoint{3.253793in}{3.392606in}}%
\pgfpathlineto{\pgfqpoint{0.566985in}{3.392606in}}%
\pgfpathclose%
\pgfusepath{fill}%
\end{pgfscope}%
\begin{pgfscope}%
\pgfsetbuttcap%
\pgfsetroundjoin%
\definecolor{currentfill}{rgb}{0.150000,0.150000,0.150000}%
\pgfsetfillcolor{currentfill}%
\pgfsetlinewidth{1.003750pt}%
\definecolor{currentstroke}{rgb}{0.150000,0.150000,0.150000}%
\pgfsetstrokecolor{currentstroke}%
\pgfsetdash{}{0pt}%
\pgfsys@defobject{currentmarker}{\pgfqpoint{0.000000in}{0.000000in}}{\pgfqpoint{0.000000in}{0.041667in}}{%
\pgfpathmoveto{\pgfqpoint{0.000000in}{0.000000in}}%
\pgfpathlineto{\pgfqpoint{0.000000in}{0.041667in}}%
\pgfusepath{stroke,fill}%
}%
\begin{pgfscope}%
\pgfsys@transformshift{0.566985in}{0.528177in}%
\pgfsys@useobject{currentmarker}{}%
\end{pgfscope}%
\end{pgfscope}%
\begin{pgfscope}%
\definecolor{textcolor}{rgb}{0.150000,0.150000,0.150000}%
\pgfsetstrokecolor{textcolor}%
\pgfsetfillcolor{textcolor}%
\pgftext[x=0.566985in,y=0.430955in,,top]{\color{textcolor}\rmfamily\fontsize{10.000000}{12.000000}\selectfont \(\displaystyle -10\)}%
\end{pgfscope}%
\begin{pgfscope}%
\pgfsetbuttcap%
\pgfsetroundjoin%
\definecolor{currentfill}{rgb}{0.150000,0.150000,0.150000}%
\pgfsetfillcolor{currentfill}%
\pgfsetlinewidth{1.003750pt}%
\definecolor{currentstroke}{rgb}{0.150000,0.150000,0.150000}%
\pgfsetstrokecolor{currentstroke}%
\pgfsetdash{}{0pt}%
\pgfsys@defobject{currentmarker}{\pgfqpoint{0.000000in}{0.000000in}}{\pgfqpoint{0.000000in}{0.041667in}}{%
\pgfpathmoveto{\pgfqpoint{0.000000in}{0.000000in}}%
\pgfpathlineto{\pgfqpoint{0.000000in}{0.041667in}}%
\pgfusepath{stroke,fill}%
}%
\begin{pgfscope}%
\pgfsys@transformshift{1.055495in}{0.528177in}%
\pgfsys@useobject{currentmarker}{}%
\end{pgfscope}%
\end{pgfscope}%
\begin{pgfscope}%
\definecolor{textcolor}{rgb}{0.150000,0.150000,0.150000}%
\pgfsetstrokecolor{textcolor}%
\pgfsetfillcolor{textcolor}%
\pgftext[x=1.055495in,y=0.430955in,,top]{\color{textcolor}\rmfamily\fontsize{10.000000}{12.000000}\selectfont \(\displaystyle -8\)}%
\end{pgfscope}%
\begin{pgfscope}%
\pgfsetbuttcap%
\pgfsetroundjoin%
\definecolor{currentfill}{rgb}{0.150000,0.150000,0.150000}%
\pgfsetfillcolor{currentfill}%
\pgfsetlinewidth{1.003750pt}%
\definecolor{currentstroke}{rgb}{0.150000,0.150000,0.150000}%
\pgfsetstrokecolor{currentstroke}%
\pgfsetdash{}{0pt}%
\pgfsys@defobject{currentmarker}{\pgfqpoint{0.000000in}{0.000000in}}{\pgfqpoint{0.000000in}{0.041667in}}{%
\pgfpathmoveto{\pgfqpoint{0.000000in}{0.000000in}}%
\pgfpathlineto{\pgfqpoint{0.000000in}{0.041667in}}%
\pgfusepath{stroke,fill}%
}%
\begin{pgfscope}%
\pgfsys@transformshift{1.544006in}{0.528177in}%
\pgfsys@useobject{currentmarker}{}%
\end{pgfscope}%
\end{pgfscope}%
\begin{pgfscope}%
\definecolor{textcolor}{rgb}{0.150000,0.150000,0.150000}%
\pgfsetstrokecolor{textcolor}%
\pgfsetfillcolor{textcolor}%
\pgftext[x=1.544006in,y=0.430955in,,top]{\color{textcolor}\rmfamily\fontsize{10.000000}{12.000000}\selectfont \(\displaystyle -6\)}%
\end{pgfscope}%
\begin{pgfscope}%
\pgfsetbuttcap%
\pgfsetroundjoin%
\definecolor{currentfill}{rgb}{0.150000,0.150000,0.150000}%
\pgfsetfillcolor{currentfill}%
\pgfsetlinewidth{1.003750pt}%
\definecolor{currentstroke}{rgb}{0.150000,0.150000,0.150000}%
\pgfsetstrokecolor{currentstroke}%
\pgfsetdash{}{0pt}%
\pgfsys@defobject{currentmarker}{\pgfqpoint{0.000000in}{0.000000in}}{\pgfqpoint{0.000000in}{0.041667in}}{%
\pgfpathmoveto{\pgfqpoint{0.000000in}{0.000000in}}%
\pgfpathlineto{\pgfqpoint{0.000000in}{0.041667in}}%
\pgfusepath{stroke,fill}%
}%
\begin{pgfscope}%
\pgfsys@transformshift{2.032516in}{0.528177in}%
\pgfsys@useobject{currentmarker}{}%
\end{pgfscope}%
\end{pgfscope}%
\begin{pgfscope}%
\definecolor{textcolor}{rgb}{0.150000,0.150000,0.150000}%
\pgfsetstrokecolor{textcolor}%
\pgfsetfillcolor{textcolor}%
\pgftext[x=2.032516in,y=0.430955in,,top]{\color{textcolor}\rmfamily\fontsize{10.000000}{12.000000}\selectfont \(\displaystyle -4\)}%
\end{pgfscope}%
\begin{pgfscope}%
\pgfsetbuttcap%
\pgfsetroundjoin%
\definecolor{currentfill}{rgb}{0.150000,0.150000,0.150000}%
\pgfsetfillcolor{currentfill}%
\pgfsetlinewidth{1.003750pt}%
\definecolor{currentstroke}{rgb}{0.150000,0.150000,0.150000}%
\pgfsetstrokecolor{currentstroke}%
\pgfsetdash{}{0pt}%
\pgfsys@defobject{currentmarker}{\pgfqpoint{0.000000in}{0.000000in}}{\pgfqpoint{0.000000in}{0.041667in}}{%
\pgfpathmoveto{\pgfqpoint{0.000000in}{0.000000in}}%
\pgfpathlineto{\pgfqpoint{0.000000in}{0.041667in}}%
\pgfusepath{stroke,fill}%
}%
\begin{pgfscope}%
\pgfsys@transformshift{2.521027in}{0.528177in}%
\pgfsys@useobject{currentmarker}{}%
\end{pgfscope}%
\end{pgfscope}%
\begin{pgfscope}%
\definecolor{textcolor}{rgb}{0.150000,0.150000,0.150000}%
\pgfsetstrokecolor{textcolor}%
\pgfsetfillcolor{textcolor}%
\pgftext[x=2.521027in,y=0.430955in,,top]{\color{textcolor}\rmfamily\fontsize{10.000000}{12.000000}\selectfont \(\displaystyle -2\)}%
\end{pgfscope}%
\begin{pgfscope}%
\pgfsetbuttcap%
\pgfsetroundjoin%
\definecolor{currentfill}{rgb}{0.150000,0.150000,0.150000}%
\pgfsetfillcolor{currentfill}%
\pgfsetlinewidth{1.003750pt}%
\definecolor{currentstroke}{rgb}{0.150000,0.150000,0.150000}%
\pgfsetstrokecolor{currentstroke}%
\pgfsetdash{}{0pt}%
\pgfsys@defobject{currentmarker}{\pgfqpoint{0.000000in}{0.000000in}}{\pgfqpoint{0.000000in}{0.041667in}}{%
\pgfpathmoveto{\pgfqpoint{0.000000in}{0.000000in}}%
\pgfpathlineto{\pgfqpoint{0.000000in}{0.041667in}}%
\pgfusepath{stroke,fill}%
}%
\begin{pgfscope}%
\pgfsys@transformshift{3.009538in}{0.528177in}%
\pgfsys@useobject{currentmarker}{}%
\end{pgfscope}%
\end{pgfscope}%
\begin{pgfscope}%
\definecolor{textcolor}{rgb}{0.150000,0.150000,0.150000}%
\pgfsetstrokecolor{textcolor}%
\pgfsetfillcolor{textcolor}%
\pgftext[x=3.009538in,y=0.430955in,,top]{\color{textcolor}\rmfamily\fontsize{10.000000}{12.000000}\selectfont \(\displaystyle 0\)}%
\end{pgfscope}%
\begin{pgfscope}%
\definecolor{textcolor}{rgb}{0.150000,0.150000,0.150000}%
\pgfsetstrokecolor{textcolor}%
\pgfsetfillcolor{textcolor}%
\pgftext[x=1.910389in,y=0.238855in,,top]{\color{textcolor}\rmfamily\fontsize{10.000000}{12.000000}\selectfont \textbf{Time from freezing (s)}}%
\end{pgfscope}%
\begin{pgfscope}%
\pgfsetbuttcap%
\pgfsetroundjoin%
\definecolor{currentfill}{rgb}{0.150000,0.150000,0.150000}%
\pgfsetfillcolor{currentfill}%
\pgfsetlinewidth{1.003750pt}%
\definecolor{currentstroke}{rgb}{0.150000,0.150000,0.150000}%
\pgfsetstrokecolor{currentstroke}%
\pgfsetdash{}{0pt}%
\pgfsys@defobject{currentmarker}{\pgfqpoint{0.000000in}{0.000000in}}{\pgfqpoint{0.041667in}{0.000000in}}{%
\pgfpathmoveto{\pgfqpoint{0.000000in}{0.000000in}}%
\pgfpathlineto{\pgfqpoint{0.041667in}{0.000000in}}%
\pgfusepath{stroke,fill}%
}%
\begin{pgfscope}%
\pgfsys@transformshift{0.566985in}{0.528177in}%
\pgfsys@useobject{currentmarker}{}%
\end{pgfscope}%
\end{pgfscope}%
\begin{pgfscope}%
\definecolor{textcolor}{rgb}{0.150000,0.150000,0.150000}%
\pgfsetstrokecolor{textcolor}%
\pgfsetfillcolor{textcolor}%
\pgftext[x=0.469762in,y=0.528177in,right,]{\color{textcolor}\rmfamily\fontsize{10.000000}{12.000000}\selectfont \(\displaystyle 0.2\)}%
\end{pgfscope}%
\begin{pgfscope}%
\pgfsetbuttcap%
\pgfsetroundjoin%
\definecolor{currentfill}{rgb}{0.150000,0.150000,0.150000}%
\pgfsetfillcolor{currentfill}%
\pgfsetlinewidth{1.003750pt}%
\definecolor{currentstroke}{rgb}{0.150000,0.150000,0.150000}%
\pgfsetstrokecolor{currentstroke}%
\pgfsetdash{}{0pt}%
\pgfsys@defobject{currentmarker}{\pgfqpoint{0.000000in}{0.000000in}}{\pgfqpoint{0.041667in}{0.000000in}}{%
\pgfpathmoveto{\pgfqpoint{0.000000in}{0.000000in}}%
\pgfpathlineto{\pgfqpoint{0.041667in}{0.000000in}}%
\pgfusepath{stroke,fill}%
}%
\begin{pgfscope}%
\pgfsys@transformshift{0.566985in}{1.005582in}%
\pgfsys@useobject{currentmarker}{}%
\end{pgfscope}%
\end{pgfscope}%
\begin{pgfscope}%
\definecolor{textcolor}{rgb}{0.150000,0.150000,0.150000}%
\pgfsetstrokecolor{textcolor}%
\pgfsetfillcolor{textcolor}%
\pgftext[x=0.469762in,y=1.005582in,right,]{\color{textcolor}\rmfamily\fontsize{10.000000}{12.000000}\selectfont \(\displaystyle 0.3\)}%
\end{pgfscope}%
\begin{pgfscope}%
\pgfsetbuttcap%
\pgfsetroundjoin%
\definecolor{currentfill}{rgb}{0.150000,0.150000,0.150000}%
\pgfsetfillcolor{currentfill}%
\pgfsetlinewidth{1.003750pt}%
\definecolor{currentstroke}{rgb}{0.150000,0.150000,0.150000}%
\pgfsetstrokecolor{currentstroke}%
\pgfsetdash{}{0pt}%
\pgfsys@defobject{currentmarker}{\pgfqpoint{0.000000in}{0.000000in}}{\pgfqpoint{0.041667in}{0.000000in}}{%
\pgfpathmoveto{\pgfqpoint{0.000000in}{0.000000in}}%
\pgfpathlineto{\pgfqpoint{0.041667in}{0.000000in}}%
\pgfusepath{stroke,fill}%
}%
\begin{pgfscope}%
\pgfsys@transformshift{0.566985in}{1.482987in}%
\pgfsys@useobject{currentmarker}{}%
\end{pgfscope}%
\end{pgfscope}%
\begin{pgfscope}%
\definecolor{textcolor}{rgb}{0.150000,0.150000,0.150000}%
\pgfsetstrokecolor{textcolor}%
\pgfsetfillcolor{textcolor}%
\pgftext[x=0.469762in,y=1.482987in,right,]{\color{textcolor}\rmfamily\fontsize{10.000000}{12.000000}\selectfont \(\displaystyle 0.4\)}%
\end{pgfscope}%
\begin{pgfscope}%
\pgfsetbuttcap%
\pgfsetroundjoin%
\definecolor{currentfill}{rgb}{0.150000,0.150000,0.150000}%
\pgfsetfillcolor{currentfill}%
\pgfsetlinewidth{1.003750pt}%
\definecolor{currentstroke}{rgb}{0.150000,0.150000,0.150000}%
\pgfsetstrokecolor{currentstroke}%
\pgfsetdash{}{0pt}%
\pgfsys@defobject{currentmarker}{\pgfqpoint{0.000000in}{0.000000in}}{\pgfqpoint{0.041667in}{0.000000in}}{%
\pgfpathmoveto{\pgfqpoint{0.000000in}{0.000000in}}%
\pgfpathlineto{\pgfqpoint{0.041667in}{0.000000in}}%
\pgfusepath{stroke,fill}%
}%
\begin{pgfscope}%
\pgfsys@transformshift{0.566985in}{1.960392in}%
\pgfsys@useobject{currentmarker}{}%
\end{pgfscope}%
\end{pgfscope}%
\begin{pgfscope}%
\definecolor{textcolor}{rgb}{0.150000,0.150000,0.150000}%
\pgfsetstrokecolor{textcolor}%
\pgfsetfillcolor{textcolor}%
\pgftext[x=0.469762in,y=1.960392in,right,]{\color{textcolor}\rmfamily\fontsize{10.000000}{12.000000}\selectfont \(\displaystyle 0.5\)}%
\end{pgfscope}%
\begin{pgfscope}%
\pgfsetbuttcap%
\pgfsetroundjoin%
\definecolor{currentfill}{rgb}{0.150000,0.150000,0.150000}%
\pgfsetfillcolor{currentfill}%
\pgfsetlinewidth{1.003750pt}%
\definecolor{currentstroke}{rgb}{0.150000,0.150000,0.150000}%
\pgfsetstrokecolor{currentstroke}%
\pgfsetdash{}{0pt}%
\pgfsys@defobject{currentmarker}{\pgfqpoint{0.000000in}{0.000000in}}{\pgfqpoint{0.041667in}{0.000000in}}{%
\pgfpathmoveto{\pgfqpoint{0.000000in}{0.000000in}}%
\pgfpathlineto{\pgfqpoint{0.041667in}{0.000000in}}%
\pgfusepath{stroke,fill}%
}%
\begin{pgfscope}%
\pgfsys@transformshift{0.566985in}{2.437796in}%
\pgfsys@useobject{currentmarker}{}%
\end{pgfscope}%
\end{pgfscope}%
\begin{pgfscope}%
\definecolor{textcolor}{rgb}{0.150000,0.150000,0.150000}%
\pgfsetstrokecolor{textcolor}%
\pgfsetfillcolor{textcolor}%
\pgftext[x=0.469762in,y=2.437796in,right,]{\color{textcolor}\rmfamily\fontsize{10.000000}{12.000000}\selectfont \(\displaystyle 0.6\)}%
\end{pgfscope}%
\begin{pgfscope}%
\pgfsetbuttcap%
\pgfsetroundjoin%
\definecolor{currentfill}{rgb}{0.150000,0.150000,0.150000}%
\pgfsetfillcolor{currentfill}%
\pgfsetlinewidth{1.003750pt}%
\definecolor{currentstroke}{rgb}{0.150000,0.150000,0.150000}%
\pgfsetstrokecolor{currentstroke}%
\pgfsetdash{}{0pt}%
\pgfsys@defobject{currentmarker}{\pgfqpoint{0.000000in}{0.000000in}}{\pgfqpoint{0.041667in}{0.000000in}}{%
\pgfpathmoveto{\pgfqpoint{0.000000in}{0.000000in}}%
\pgfpathlineto{\pgfqpoint{0.041667in}{0.000000in}}%
\pgfusepath{stroke,fill}%
}%
\begin{pgfscope}%
\pgfsys@transformshift{0.566985in}{2.915201in}%
\pgfsys@useobject{currentmarker}{}%
\end{pgfscope}%
\end{pgfscope}%
\begin{pgfscope}%
\definecolor{textcolor}{rgb}{0.150000,0.150000,0.150000}%
\pgfsetstrokecolor{textcolor}%
\pgfsetfillcolor{textcolor}%
\pgftext[x=0.469762in,y=2.915201in,right,]{\color{textcolor}\rmfamily\fontsize{10.000000}{12.000000}\selectfont \(\displaystyle 0.7\)}%
\end{pgfscope}%
\begin{pgfscope}%
\pgfsetbuttcap%
\pgfsetroundjoin%
\definecolor{currentfill}{rgb}{0.150000,0.150000,0.150000}%
\pgfsetfillcolor{currentfill}%
\pgfsetlinewidth{1.003750pt}%
\definecolor{currentstroke}{rgb}{0.150000,0.150000,0.150000}%
\pgfsetstrokecolor{currentstroke}%
\pgfsetdash{}{0pt}%
\pgfsys@defobject{currentmarker}{\pgfqpoint{0.000000in}{0.000000in}}{\pgfqpoint{0.041667in}{0.000000in}}{%
\pgfpathmoveto{\pgfqpoint{0.000000in}{0.000000in}}%
\pgfpathlineto{\pgfqpoint{0.041667in}{0.000000in}}%
\pgfusepath{stroke,fill}%
}%
\begin{pgfscope}%
\pgfsys@transformshift{0.566985in}{3.392606in}%
\pgfsys@useobject{currentmarker}{}%
\end{pgfscope}%
\end{pgfscope}%
\begin{pgfscope}%
\definecolor{textcolor}{rgb}{0.150000,0.150000,0.150000}%
\pgfsetstrokecolor{textcolor}%
\pgfsetfillcolor{textcolor}%
\pgftext[x=0.469762in,y=3.392606in,right,]{\color{textcolor}\rmfamily\fontsize{10.000000}{12.000000}\selectfont \(\displaystyle 0.8\)}%
\end{pgfscope}%
\begin{pgfscope}%
\definecolor{textcolor}{rgb}{0.150000,0.150000,0.150000}%
\pgfsetstrokecolor{textcolor}%
\pgfsetfillcolor{textcolor}%
\pgftext[x=0.222848in,y=1.960392in,,bottom,rotate=90.000000]{\color{textcolor}\rmfamily\fontsize{10.000000}{12.000000}\selectfont \textbf{Accuracy}}%
\end{pgfscope}%
\begin{pgfscope}%
\pgfpathrectangle{\pgfqpoint{0.566985in}{0.528177in}}{\pgfqpoint{2.686808in}{2.864429in}} %
\pgfusepath{clip}%
\pgfsetroundcap%
\pgfsetroundjoin%
\pgfsetlinewidth{1.756562pt}%
\definecolor{currentstroke}{rgb}{0.200000,0.427451,0.650980}%
\pgfsetstrokecolor{currentstroke}%
\pgfsetstrokeopacity{0.800000}%
\pgfsetdash{}{0pt}%
\pgfpathmoveto{\pgfqpoint{0.566985in}{3.013491in}}%
\pgfpathlineto{\pgfqpoint{0.579197in}{2.866160in}}%
\pgfpathlineto{\pgfqpoint{0.591410in}{2.946228in}}%
\pgfpathlineto{\pgfqpoint{0.603623in}{2.826126in}}%
\pgfpathlineto{\pgfqpoint{0.615836in}{2.876168in}}%
\pgfpathlineto{\pgfqpoint{0.628048in}{2.816117in}}%
\pgfpathlineto{\pgfqpoint{0.640261in}{2.766075in}}%
\pgfpathlineto{\pgfqpoint{0.652474in}{2.756066in}}%
\pgfpathlineto{\pgfqpoint{0.664687in}{2.856151in}}%
\pgfpathlineto{\pgfqpoint{0.676899in}{2.896185in}}%
\pgfpathlineto{\pgfqpoint{0.689112in}{2.946228in}}%
\pgfpathlineto{\pgfqpoint{0.701325in}{2.946228in}}%
\pgfpathlineto{\pgfqpoint{0.713538in}{2.866160in}}%
\pgfpathlineto{\pgfqpoint{0.725751in}{2.846143in}}%
\pgfpathlineto{\pgfqpoint{0.737963in}{2.856151in}}%
\pgfpathlineto{\pgfqpoint{0.750176in}{2.906194in}}%
\pgfpathlineto{\pgfqpoint{0.762389in}{2.876168in}}%
\pgfpathlineto{\pgfqpoint{0.774602in}{2.786092in}}%
\pgfpathlineto{\pgfqpoint{0.786814in}{2.796100in}}%
\pgfpathlineto{\pgfqpoint{0.799027in}{2.836134in}}%
\pgfpathlineto{\pgfqpoint{0.811240in}{2.806109in}}%
\pgfpathlineto{\pgfqpoint{0.823453in}{2.826126in}}%
\pgfpathlineto{\pgfqpoint{0.835665in}{2.896185in}}%
\pgfpathlineto{\pgfqpoint{0.847878in}{2.926211in}}%
\pgfpathlineto{\pgfqpoint{0.860091in}{2.886177in}}%
\pgfpathlineto{\pgfqpoint{0.872304in}{2.926211in}}%
\pgfpathlineto{\pgfqpoint{0.884516in}{2.996270in}}%
\pgfpathlineto{\pgfqpoint{0.896729in}{2.936219in}}%
\pgfpathlineto{\pgfqpoint{0.921155in}{3.016287in}}%
\pgfpathlineto{\pgfqpoint{0.933368in}{3.036304in}}%
\pgfpathlineto{\pgfqpoint{0.945580in}{2.936219in}}%
\pgfpathlineto{\pgfqpoint{0.957793in}{3.016287in}}%
\pgfpathlineto{\pgfqpoint{0.970006in}{2.966245in}}%
\pgfpathlineto{\pgfqpoint{0.982219in}{2.886177in}}%
\pgfpathlineto{\pgfqpoint{0.994431in}{2.856151in}}%
\pgfpathlineto{\pgfqpoint{1.006644in}{2.919196in}}%
\pgfpathlineto{\pgfqpoint{1.018857in}{2.839296in}}%
\pgfpathlineto{\pgfqpoint{1.031070in}{2.822511in}}%
\pgfpathlineto{\pgfqpoint{1.043282in}{2.842444in}}%
\pgfpathlineto{\pgfqpoint{1.067708in}{2.962045in}}%
\pgfpathlineto{\pgfqpoint{1.079921in}{2.872345in}}%
\pgfpathlineto{\pgfqpoint{1.092133in}{2.892278in}}%
\pgfpathlineto{\pgfqpoint{1.104346in}{2.862378in}}%
\pgfpathlineto{\pgfqpoint{1.116559in}{2.852411in}}%
\pgfpathlineto{\pgfqpoint{1.128772in}{2.822511in}}%
\pgfpathlineto{\pgfqpoint{1.140985in}{2.892278in}}%
\pgfpathlineto{\pgfqpoint{1.153197in}{2.892278in}}%
\pgfpathlineto{\pgfqpoint{1.165410in}{2.962045in}}%
\pgfpathlineto{\pgfqpoint{1.177623in}{2.952078in}}%
\pgfpathlineto{\pgfqpoint{1.189836in}{3.051745in}}%
\pgfpathlineto{\pgfqpoint{1.202048in}{3.011878in}}%
\pgfpathlineto{\pgfqpoint{1.214261in}{3.071679in}}%
\pgfpathlineto{\pgfqpoint{1.226474in}{2.991945in}}%
\pgfpathlineto{\pgfqpoint{1.238687in}{3.064390in}}%
\pgfpathlineto{\pgfqpoint{1.250899in}{3.014661in}}%
\pgfpathlineto{\pgfqpoint{1.263112in}{2.994769in}}%
\pgfpathlineto{\pgfqpoint{1.275325in}{2.915201in}}%
\pgfpathlineto{\pgfqpoint{1.287538in}{2.865472in}}%
\pgfpathlineto{\pgfqpoint{1.299750in}{2.845580in}}%
\pgfpathlineto{\pgfqpoint{1.311963in}{2.875418in}}%
\pgfpathlineto{\pgfqpoint{1.324176in}{2.945039in}}%
\pgfpathlineto{\pgfqpoint{1.336389in}{2.865472in}}%
\pgfpathlineto{\pgfqpoint{1.348602in}{2.935093in}}%
\pgfpathlineto{\pgfqpoint{1.360814in}{2.885364in}}%
\pgfpathlineto{\pgfqpoint{1.373027in}{2.954985in}}%
\pgfpathlineto{\pgfqpoint{1.385240in}{2.885364in}}%
\pgfpathlineto{\pgfqpoint{1.409665in}{3.024607in}}%
\pgfpathlineto{\pgfqpoint{1.421878in}{2.935093in}}%
\pgfpathlineto{\pgfqpoint{1.434091in}{3.004715in}}%
\pgfpathlineto{\pgfqpoint{1.446304in}{2.954985in}}%
\pgfpathlineto{\pgfqpoint{1.458516in}{2.945039in}}%
\pgfpathlineto{\pgfqpoint{1.470729in}{2.964931in}}%
\pgfpathlineto{\pgfqpoint{1.482942in}{2.895309in}}%
\pgfpathlineto{\pgfqpoint{1.495155in}{3.034553in}}%
\pgfpathlineto{\pgfqpoint{1.507367in}{2.945039in}}%
\pgfpathlineto{\pgfqpoint{1.519580in}{2.925147in}}%
\pgfpathlineto{\pgfqpoint{1.531793in}{2.915201in}}%
\pgfpathlineto{\pgfqpoint{1.544006in}{2.994769in}}%
\pgfpathlineto{\pgfqpoint{1.556219in}{2.964931in}}%
\pgfpathlineto{\pgfqpoint{1.568431in}{2.895309in}}%
\pgfpathlineto{\pgfqpoint{1.580644in}{2.908254in}}%
\pgfpathlineto{\pgfqpoint{1.592857in}{2.828852in}}%
\pgfpathlineto{\pgfqpoint{1.605070in}{2.898328in}}%
\pgfpathlineto{\pgfqpoint{1.617282in}{2.868553in}}%
\pgfpathlineto{\pgfqpoint{1.629495in}{2.871621in}}%
\pgfpathlineto{\pgfqpoint{1.653921in}{2.931049in}}%
\pgfpathlineto{\pgfqpoint{1.666133in}{2.940953in}}%
\pgfpathlineto{\pgfqpoint{1.678346in}{2.871621in}}%
\pgfpathlineto{\pgfqpoint{1.690559in}{2.901335in}}%
\pgfpathlineto{\pgfqpoint{1.702772in}{2.871621in}}%
\pgfpathlineto{\pgfqpoint{1.714984in}{2.802288in}}%
\pgfpathlineto{\pgfqpoint{1.727197in}{2.841907in}}%
\pgfpathlineto{\pgfqpoint{1.739410in}{2.822097in}}%
\pgfpathlineto{\pgfqpoint{1.751623in}{2.782479in}}%
\pgfpathlineto{\pgfqpoint{1.763836in}{2.871621in}}%
\pgfpathlineto{\pgfqpoint{1.776048in}{2.812193in}}%
\pgfpathlineto{\pgfqpoint{1.788261in}{2.802288in}}%
\pgfpathlineto{\pgfqpoint{1.800474in}{2.812193in}}%
\pgfpathlineto{\pgfqpoint{1.812687in}{2.911239in}}%
\pgfpathlineto{\pgfqpoint{1.837112in}{2.792383in}}%
\pgfpathlineto{\pgfqpoint{1.849325in}{2.752765in}}%
\pgfpathlineto{\pgfqpoint{1.861538in}{2.841907in}}%
\pgfpathlineto{\pgfqpoint{1.873750in}{2.703241in}}%
\pgfpathlineto{\pgfqpoint{1.885963in}{2.742860in}}%
\pgfpathlineto{\pgfqpoint{1.898176in}{2.713146in}}%
\pgfpathlineto{\pgfqpoint{1.910389in}{2.703241in}}%
\pgfpathlineto{\pgfqpoint{1.922601in}{2.683432in}}%
\pgfpathlineto{\pgfqpoint{1.934814in}{2.703241in}}%
\pgfpathlineto{\pgfqpoint{1.947027in}{2.643813in}}%
\pgfpathlineto{\pgfqpoint{1.959240in}{2.732955in}}%
\pgfpathlineto{\pgfqpoint{1.971453in}{2.732955in}}%
\pgfpathlineto{\pgfqpoint{1.983665in}{2.703241in}}%
\pgfpathlineto{\pgfqpoint{2.008091in}{2.683432in}}%
\pgfpathlineto{\pgfqpoint{2.020304in}{2.851811in}}%
\pgfpathlineto{\pgfqpoint{2.032516in}{2.742860in}}%
\pgfpathlineto{\pgfqpoint{2.044729in}{2.663623in}}%
\pgfpathlineto{\pgfqpoint{2.056942in}{2.723051in}}%
\pgfpathlineto{\pgfqpoint{2.069155in}{2.792383in}}%
\pgfpathlineto{\pgfqpoint{2.081367in}{2.792383in}}%
\pgfpathlineto{\pgfqpoint{2.105793in}{2.854908in}}%
\pgfpathlineto{\pgfqpoint{2.118006in}{2.808673in}}%
\pgfpathlineto{\pgfqpoint{2.130218in}{2.927038in}}%
\pgfpathlineto{\pgfqpoint{2.142431in}{2.857992in}}%
\pgfpathlineto{\pgfqpoint{2.154644in}{2.897447in}}%
\pgfpathlineto{\pgfqpoint{2.166857in}{2.867855in}}%
\pgfpathlineto{\pgfqpoint{2.179070in}{2.828400in}}%
\pgfpathlineto{\pgfqpoint{2.191282in}{2.798809in}}%
\pgfpathlineto{\pgfqpoint{2.203495in}{2.710036in}}%
\pgfpathlineto{\pgfqpoint{2.215708in}{2.719899in}}%
\pgfpathlineto{\pgfqpoint{2.227921in}{2.690308in}}%
\pgfpathlineto{\pgfqpoint{2.240133in}{2.571943in}}%
\pgfpathlineto{\pgfqpoint{2.252346in}{2.621262in}}%
\pgfpathlineto{\pgfqpoint{2.264559in}{2.565761in}}%
\pgfpathlineto{\pgfqpoint{2.276772in}{2.575604in}}%
\pgfpathlineto{\pgfqpoint{2.288984in}{2.546074in}}%
\pgfpathlineto{\pgfqpoint{2.301197in}{2.614978in}}%
\pgfpathlineto{\pgfqpoint{2.313410in}{2.559603in}}%
\pgfpathlineto{\pgfqpoint{2.325623in}{2.608719in}}%
\pgfpathlineto{\pgfqpoint{2.337835in}{2.549780in}}%
\pgfpathlineto{\pgfqpoint{2.350048in}{2.441726in}}%
\pgfpathlineto{\pgfqpoint{2.362261in}{2.435836in}}%
\pgfpathlineto{\pgfqpoint{2.374474in}{2.475048in}}%
\pgfpathlineto{\pgfqpoint{2.386687in}{2.435836in}}%
\pgfpathlineto{\pgfqpoint{2.398899in}{2.367215in}}%
\pgfpathlineto{\pgfqpoint{2.411112in}{2.416230in}}%
\pgfpathlineto{\pgfqpoint{2.423325in}{2.357412in}}%
\pgfpathlineto{\pgfqpoint{2.435538in}{2.318200in}}%
\pgfpathlineto{\pgfqpoint{2.447750in}{2.386821in}}%
\pgfpathlineto{\pgfqpoint{2.459963in}{2.308397in}}%
\pgfpathlineto{\pgfqpoint{2.472176in}{2.249579in}}%
\pgfpathlineto{\pgfqpoint{2.484389in}{2.347609in}}%
\pgfpathlineto{\pgfqpoint{2.496601in}{2.328003in}}%
\pgfpathlineto{\pgfqpoint{2.508814in}{2.377018in}}%
\pgfpathlineto{\pgfqpoint{2.521027in}{2.341924in}}%
\pgfpathlineto{\pgfqpoint{2.533240in}{2.459319in}}%
\pgfpathlineto{\pgfqpoint{2.545452in}{2.253878in}}%
\pgfpathlineto{\pgfqpoint{2.557665in}{2.214747in}}%
\pgfpathlineto{\pgfqpoint{2.569878in}{2.302793in}}%
\pgfpathlineto{\pgfqpoint{2.582091in}{2.175615in}}%
\pgfpathlineto{\pgfqpoint{2.594304in}{2.224529in}}%
\pgfpathlineto{\pgfqpoint{2.606516in}{2.136483in}}%
\pgfpathlineto{\pgfqpoint{2.618729in}{2.097352in}}%
\pgfpathlineto{\pgfqpoint{2.630942in}{2.214747in}}%
\pgfpathlineto{\pgfqpoint{2.643155in}{2.165832in}}%
\pgfpathlineto{\pgfqpoint{2.655367in}{2.107135in}}%
\pgfpathlineto{\pgfqpoint{2.667580in}{2.097352in}}%
\pgfpathlineto{\pgfqpoint{2.679793in}{2.028872in}}%
\pgfpathlineto{\pgfqpoint{2.692006in}{1.896933in}}%
\pgfpathlineto{\pgfqpoint{2.704218in}{1.848118in}}%
\pgfpathlineto{\pgfqpoint{2.716431in}{1.809067in}}%
\pgfpathlineto{\pgfqpoint{2.728644in}{1.750490in}}%
\pgfpathlineto{\pgfqpoint{2.740857in}{1.799304in}}%
\pgfpathlineto{\pgfqpoint{2.753069in}{1.730964in}}%
\pgfpathlineto{\pgfqpoint{2.765282in}{1.730964in}}%
\pgfpathlineto{\pgfqpoint{2.777495in}{1.652861in}}%
\pgfpathlineto{\pgfqpoint{2.789708in}{1.672387in}}%
\pgfpathlineto{\pgfqpoint{2.801921in}{1.682149in}}%
\pgfpathlineto{\pgfqpoint{2.814133in}{1.701675in}}%
\pgfpathlineto{\pgfqpoint{2.826346in}{1.643098in}}%
\pgfpathlineto{\pgfqpoint{2.850772in}{1.506418in}}%
\pgfpathlineto{\pgfqpoint{2.862984in}{1.447840in}}%
\pgfpathlineto{\pgfqpoint{2.887410in}{1.428314in}}%
\pgfpathlineto{\pgfqpoint{2.899623in}{1.356328in}}%
\pgfpathlineto{\pgfqpoint{2.911835in}{1.239413in}}%
\pgfpathlineto{\pgfqpoint{2.924048in}{1.190698in}}%
\pgfpathlineto{\pgfqpoint{2.936261in}{1.151726in}}%
\pgfpathlineto{\pgfqpoint{2.948474in}{1.151726in}}%
\pgfpathlineto{\pgfqpoint{2.960686in}{1.190698in}}%
\pgfpathlineto{\pgfqpoint{2.972899in}{1.200441in}}%
\pgfpathlineto{\pgfqpoint{2.985112in}{1.122497in}}%
\pgfpathlineto{\pgfqpoint{2.997325in}{1.087090in}}%
\pgfpathlineto{\pgfqpoint{3.009538in}{2.950133in}}%
\pgfpathlineto{\pgfqpoint{3.021750in}{2.959837in}}%
\pgfpathlineto{\pgfqpoint{3.033963in}{2.911320in}}%
\pgfpathlineto{\pgfqpoint{3.046176in}{3.008354in}}%
\pgfpathlineto{\pgfqpoint{3.058389in}{3.153904in}}%
\pgfpathlineto{\pgfqpoint{3.070601in}{3.134497in}}%
\pgfpathlineto{\pgfqpoint{3.082814in}{3.183014in}}%
\pgfpathlineto{\pgfqpoint{3.095027in}{3.173310in}}%
\pgfpathlineto{\pgfqpoint{3.119452in}{3.115090in}}%
\pgfpathlineto{\pgfqpoint{3.131665in}{3.115090in}}%
\pgfpathlineto{\pgfqpoint{3.143878in}{3.144200in}}%
\pgfpathlineto{\pgfqpoint{3.156091in}{3.221827in}}%
\pgfpathlineto{\pgfqpoint{3.168303in}{3.250937in}}%
\pgfpathlineto{\pgfqpoint{3.180516in}{3.212124in}}%
\pgfpathlineto{\pgfqpoint{3.192729in}{3.241234in}}%
\pgfpathlineto{\pgfqpoint{3.204942in}{3.221827in}}%
\pgfpathlineto{\pgfqpoint{3.217155in}{3.183014in}}%
\pgfpathlineto{\pgfqpoint{3.229367in}{3.085980in}}%
\pgfpathlineto{\pgfqpoint{3.241580in}{3.095684in}}%
\pgfpathlineto{\pgfqpoint{3.253793in}{3.102857in}}%
\pgfpathlineto{\pgfqpoint{3.266006in}{3.190365in}}%
\pgfpathlineto{\pgfqpoint{3.267682in}{3.183693in}}%
\pgfpathlineto{\pgfqpoint{3.267682in}{3.183693in}}%
\pgfusepath{stroke}%
\end{pgfscope}%
\begin{pgfscope}%
\pgfpathrectangle{\pgfqpoint{0.566985in}{0.528177in}}{\pgfqpoint{2.686808in}{2.864429in}} %
\pgfusepath{clip}%
\pgfsetroundcap%
\pgfsetroundjoin%
\pgfsetlinewidth{1.756562pt}%
\definecolor{currentstroke}{rgb}{0.168627,0.670588,0.494118}%
\pgfsetstrokecolor{currentstroke}%
\pgfsetstrokeopacity{0.800000}%
\pgfsetdash{}{0pt}%
\pgfpathmoveto{\pgfqpoint{0.566985in}{2.791827in}}%
\pgfpathlineto{\pgfqpoint{0.603623in}{2.832058in}}%
\pgfpathlineto{\pgfqpoint{0.615836in}{2.858878in}}%
\pgfpathlineto{\pgfqpoint{0.628048in}{2.858878in}}%
\pgfpathlineto{\pgfqpoint{0.640261in}{2.818648in}}%
\pgfpathlineto{\pgfqpoint{0.652474in}{2.885699in}}%
\pgfpathlineto{\pgfqpoint{0.664687in}{2.912519in}}%
\pgfpathlineto{\pgfqpoint{0.676899in}{2.912519in}}%
\pgfpathlineto{\pgfqpoint{0.689112in}{2.858878in}}%
\pgfpathlineto{\pgfqpoint{0.701325in}{2.939340in}}%
\pgfpathlineto{\pgfqpoint{0.713538in}{2.858878in}}%
\pgfpathlineto{\pgfqpoint{0.725751in}{2.899109in}}%
\pgfpathlineto{\pgfqpoint{0.737963in}{2.845468in}}%
\pgfpathlineto{\pgfqpoint{0.750176in}{2.832058in}}%
\pgfpathlineto{\pgfqpoint{0.762389in}{2.899109in}}%
\pgfpathlineto{\pgfqpoint{0.774602in}{2.872289in}}%
\pgfpathlineto{\pgfqpoint{0.786814in}{2.858878in}}%
\pgfpathlineto{\pgfqpoint{0.799027in}{2.885699in}}%
\pgfpathlineto{\pgfqpoint{0.811240in}{2.899109in}}%
\pgfpathlineto{\pgfqpoint{0.823453in}{3.019801in}}%
\pgfpathlineto{\pgfqpoint{0.835665in}{3.033212in}}%
\pgfpathlineto{\pgfqpoint{0.847878in}{2.966160in}}%
\pgfpathlineto{\pgfqpoint{0.860091in}{2.925930in}}%
\pgfpathlineto{\pgfqpoint{0.872304in}{2.912519in}}%
\pgfpathlineto{\pgfqpoint{0.884516in}{2.952750in}}%
\pgfpathlineto{\pgfqpoint{0.896729in}{2.979571in}}%
\pgfpathlineto{\pgfqpoint{0.908942in}{2.925930in}}%
\pgfpathlineto{\pgfqpoint{0.921155in}{2.979571in}}%
\pgfpathlineto{\pgfqpoint{0.933368in}{3.060032in}}%
\pgfpathlineto{\pgfqpoint{0.945580in}{3.006391in}}%
\pgfpathlineto{\pgfqpoint{0.957793in}{2.912519in}}%
\pgfpathlineto{\pgfqpoint{0.970006in}{2.858878in}}%
\pgfpathlineto{\pgfqpoint{0.982219in}{2.916539in}}%
\pgfpathlineto{\pgfqpoint{0.994431in}{2.929911in}}%
\pgfpathlineto{\pgfqpoint{1.006644in}{2.970029in}}%
\pgfpathlineto{\pgfqpoint{1.018857in}{2.943284in}}%
\pgfpathlineto{\pgfqpoint{1.031070in}{2.996775in}}%
\pgfpathlineto{\pgfqpoint{1.043282in}{2.943284in}}%
\pgfpathlineto{\pgfqpoint{1.055495in}{3.090384in}}%
\pgfpathlineto{\pgfqpoint{1.067708in}{3.036893in}}%
\pgfpathlineto{\pgfqpoint{1.079921in}{3.077011in}}%
\pgfpathlineto{\pgfqpoint{1.092133in}{3.143874in}}%
\pgfpathlineto{\pgfqpoint{1.116559in}{3.090384in}}%
\pgfpathlineto{\pgfqpoint{1.128772in}{3.036893in}}%
\pgfpathlineto{\pgfqpoint{1.140985in}{2.822930in}}%
\pgfpathlineto{\pgfqpoint{1.153197in}{2.836302in}}%
\pgfpathlineto{\pgfqpoint{1.165410in}{2.836302in}}%
\pgfpathlineto{\pgfqpoint{1.177623in}{2.809557in}}%
\pgfpathlineto{\pgfqpoint{1.189836in}{2.836302in}}%
\pgfpathlineto{\pgfqpoint{1.202048in}{2.840523in}}%
\pgfpathlineto{\pgfqpoint{1.214261in}{2.840523in}}%
\pgfpathlineto{\pgfqpoint{1.226474in}{2.853859in}}%
\pgfpathlineto{\pgfqpoint{1.238687in}{2.880529in}}%
\pgfpathlineto{\pgfqpoint{1.250899in}{2.960541in}}%
\pgfpathlineto{\pgfqpoint{1.263112in}{2.933871in}}%
\pgfpathlineto{\pgfqpoint{1.275325in}{2.920535in}}%
\pgfpathlineto{\pgfqpoint{1.287538in}{2.853859in}}%
\pgfpathlineto{\pgfqpoint{1.299750in}{2.893865in}}%
\pgfpathlineto{\pgfqpoint{1.311963in}{2.787182in}}%
\pgfpathlineto{\pgfqpoint{1.324176in}{2.760511in}}%
\pgfpathlineto{\pgfqpoint{1.336389in}{2.880529in}}%
\pgfpathlineto{\pgfqpoint{1.348602in}{2.813853in}}%
\pgfpathlineto{\pgfqpoint{1.360814in}{2.773847in}}%
\pgfpathlineto{\pgfqpoint{1.373027in}{2.747176in}}%
\pgfpathlineto{\pgfqpoint{1.385240in}{2.787182in}}%
\pgfpathlineto{\pgfqpoint{1.397453in}{2.800517in}}%
\pgfpathlineto{\pgfqpoint{1.409665in}{2.853859in}}%
\pgfpathlineto{\pgfqpoint{1.421878in}{2.920535in}}%
\pgfpathlineto{\pgfqpoint{1.434091in}{2.933871in}}%
\pgfpathlineto{\pgfqpoint{1.446304in}{2.964405in}}%
\pgfpathlineto{\pgfqpoint{1.458516in}{3.017597in}}%
\pgfpathlineto{\pgfqpoint{1.482942in}{2.751634in}}%
\pgfpathlineto{\pgfqpoint{1.495155in}{2.911212in}}%
\pgfpathlineto{\pgfqpoint{1.507367in}{2.858019in}}%
\pgfpathlineto{\pgfqpoint{1.519580in}{2.951106in}}%
\pgfpathlineto{\pgfqpoint{1.531793in}{2.964405in}}%
\pgfpathlineto{\pgfqpoint{1.544006in}{2.911212in}}%
\pgfpathlineto{\pgfqpoint{1.556219in}{2.924510in}}%
\pgfpathlineto{\pgfqpoint{1.568431in}{2.951106in}}%
\pgfpathlineto{\pgfqpoint{1.580644in}{2.924510in}}%
\pgfpathlineto{\pgfqpoint{1.592857in}{3.030896in}}%
\pgfpathlineto{\pgfqpoint{1.605070in}{3.004299in}}%
\pgfpathlineto{\pgfqpoint{1.617282in}{2.951106in}}%
\pgfpathlineto{\pgfqpoint{1.629495in}{2.991001in}}%
\pgfpathlineto{\pgfqpoint{1.641708in}{3.004299in}}%
\pgfpathlineto{\pgfqpoint{1.653921in}{2.977703in}}%
\pgfpathlineto{\pgfqpoint{1.666133in}{2.977703in}}%
\pgfpathlineto{\pgfqpoint{1.678346in}{2.941724in}}%
\pgfpathlineto{\pgfqpoint{1.690559in}{2.888679in}}%
\pgfpathlineto{\pgfqpoint{1.702772in}{2.941724in}}%
\pgfpathlineto{\pgfqpoint{1.714984in}{2.822373in}}%
\pgfpathlineto{\pgfqpoint{1.727197in}{2.901940in}}%
\pgfpathlineto{\pgfqpoint{1.739410in}{2.848895in}}%
\pgfpathlineto{\pgfqpoint{1.751623in}{2.809111in}}%
\pgfpathlineto{\pgfqpoint{1.763836in}{2.795850in}}%
\pgfpathlineto{\pgfqpoint{1.776048in}{2.729544in}}%
\pgfpathlineto{\pgfqpoint{1.788261in}{2.742805in}}%
\pgfpathlineto{\pgfqpoint{1.800474in}{2.782589in}}%
\pgfpathlineto{\pgfqpoint{1.812687in}{2.649976in}}%
\pgfpathlineto{\pgfqpoint{1.824899in}{2.610193in}}%
\pgfpathlineto{\pgfqpoint{1.837112in}{2.716283in}}%
\pgfpathlineto{\pgfqpoint{1.849325in}{2.729544in}}%
\pgfpathlineto{\pgfqpoint{1.861538in}{2.557148in}}%
\pgfpathlineto{\pgfqpoint{1.873750in}{2.610193in}}%
\pgfpathlineto{\pgfqpoint{1.885963in}{2.623454in}}%
\pgfpathlineto{\pgfqpoint{1.898176in}{2.583670in}}%
\pgfpathlineto{\pgfqpoint{1.910389in}{2.636715in}}%
\pgfpathlineto{\pgfqpoint{1.922601in}{2.615005in}}%
\pgfpathlineto{\pgfqpoint{1.934814in}{2.707576in}}%
\pgfpathlineto{\pgfqpoint{1.959240in}{2.681127in}}%
\pgfpathlineto{\pgfqpoint{1.971453in}{2.786924in}}%
\pgfpathlineto{\pgfqpoint{1.983665in}{2.681127in}}%
\pgfpathlineto{\pgfqpoint{1.995878in}{2.720801in}}%
\pgfpathlineto{\pgfqpoint{2.008091in}{2.786924in}}%
\pgfpathlineto{\pgfqpoint{2.020304in}{2.773699in}}%
\pgfpathlineto{\pgfqpoint{2.032516in}{2.773699in}}%
\pgfpathlineto{\pgfqpoint{2.044729in}{2.800148in}}%
\pgfpathlineto{\pgfqpoint{2.056942in}{2.747250in}}%
\pgfpathlineto{\pgfqpoint{2.069155in}{2.681127in}}%
\pgfpathlineto{\pgfqpoint{2.081367in}{2.892720in}}%
\pgfpathlineto{\pgfqpoint{2.093580in}{2.853046in}}%
\pgfpathlineto{\pgfqpoint{2.105793in}{2.892720in}}%
\pgfpathlineto{\pgfqpoint{2.118006in}{2.883550in}}%
\pgfpathlineto{\pgfqpoint{2.130218in}{2.817610in}}%
\pgfpathlineto{\pgfqpoint{2.142431in}{2.764858in}}%
\pgfpathlineto{\pgfqpoint{2.154644in}{2.606603in}}%
\pgfpathlineto{\pgfqpoint{2.166857in}{2.778046in}}%
\pgfpathlineto{\pgfqpoint{2.179070in}{2.712106in}}%
\pgfpathlineto{\pgfqpoint{2.203495in}{2.791234in}}%
\pgfpathlineto{\pgfqpoint{2.215708in}{2.764858in}}%
\pgfpathlineto{\pgfqpoint{2.227921in}{2.685730in}}%
\pgfpathlineto{\pgfqpoint{2.240133in}{2.751670in}}%
\pgfpathlineto{\pgfqpoint{2.252346in}{2.769218in}}%
\pgfpathlineto{\pgfqpoint{2.276772in}{2.821825in}}%
\pgfpathlineto{\pgfqpoint{2.288984in}{2.703460in}}%
\pgfpathlineto{\pgfqpoint{2.301197in}{2.716611in}}%
\pgfpathlineto{\pgfqpoint{2.313410in}{2.677156in}}%
\pgfpathlineto{\pgfqpoint{2.325623in}{2.677156in}}%
\pgfpathlineto{\pgfqpoint{2.337835in}{2.624550in}}%
\pgfpathlineto{\pgfqpoint{2.350048in}{2.650853in}}%
\pgfpathlineto{\pgfqpoint{2.362261in}{2.729763in}}%
\pgfpathlineto{\pgfqpoint{2.374474in}{2.742915in}}%
\pgfpathlineto{\pgfqpoint{2.386687in}{2.650853in}}%
\pgfpathlineto{\pgfqpoint{2.398899in}{2.519337in}}%
\pgfpathlineto{\pgfqpoint{2.411112in}{2.519337in}}%
\pgfpathlineto{\pgfqpoint{2.423325in}{2.598247in}}%
\pgfpathlineto{\pgfqpoint{2.435538in}{2.611398in}}%
\pgfpathlineto{\pgfqpoint{2.447750in}{2.479882in}}%
\pgfpathlineto{\pgfqpoint{2.459963in}{2.532488in}}%
\pgfpathlineto{\pgfqpoint{2.472176in}{2.598247in}}%
\pgfpathlineto{\pgfqpoint{2.484389in}{2.506185in}}%
\pgfpathlineto{\pgfqpoint{2.496601in}{2.519337in}}%
\pgfpathlineto{\pgfqpoint{2.508814in}{2.440427in}}%
\pgfpathlineto{\pgfqpoint{2.521027in}{2.387820in}}%
\pgfpathlineto{\pgfqpoint{2.533240in}{2.374669in}}%
\pgfpathlineto{\pgfqpoint{2.545452in}{2.308910in}}%
\pgfpathlineto{\pgfqpoint{2.557665in}{2.177394in}}%
\pgfpathlineto{\pgfqpoint{2.569878in}{2.124787in}}%
\pgfpathlineto{\pgfqpoint{2.582091in}{2.059029in}}%
\pgfpathlineto{\pgfqpoint{2.594304in}{2.098484in}}%
\pgfpathlineto{\pgfqpoint{2.606516in}{2.045877in}}%
\pgfpathlineto{\pgfqpoint{2.630942in}{2.072181in}}%
\pgfpathlineto{\pgfqpoint{2.643155in}{2.059029in}}%
\pgfpathlineto{\pgfqpoint{2.655367in}{1.914361in}}%
\pgfpathlineto{\pgfqpoint{2.667580in}{1.888057in}}%
\pgfpathlineto{\pgfqpoint{2.679793in}{1.953816in}}%
\pgfpathlineto{\pgfqpoint{2.692006in}{1.822299in}}%
\pgfpathlineto{\pgfqpoint{2.704218in}{1.822299in}}%
\pgfpathlineto{\pgfqpoint{2.716431in}{1.809148in}}%
\pgfpathlineto{\pgfqpoint{2.728644in}{1.809148in}}%
\pgfpathlineto{\pgfqpoint{2.740857in}{1.730238in}}%
\pgfpathlineto{\pgfqpoint{2.753069in}{1.717086in}}%
\pgfpathlineto{\pgfqpoint{2.765282in}{1.690783in}}%
\pgfpathlineto{\pgfqpoint{2.777495in}{1.651328in}}%
\pgfpathlineto{\pgfqpoint{2.789708in}{1.677631in}}%
\pgfpathlineto{\pgfqpoint{2.801921in}{1.598721in}}%
\pgfpathlineto{\pgfqpoint{2.814133in}{1.454053in}}%
\pgfpathlineto{\pgfqpoint{2.826346in}{1.414598in}}%
\pgfpathlineto{\pgfqpoint{2.850772in}{1.309385in}}%
\pgfpathlineto{\pgfqpoint{2.862984in}{1.269930in}}%
\pgfpathlineto{\pgfqpoint{2.875197in}{1.204172in}}%
\pgfpathlineto{\pgfqpoint{2.887410in}{1.204172in}}%
\pgfpathlineto{\pgfqpoint{2.899623in}{1.191020in}}%
\pgfpathlineto{\pgfqpoint{2.911835in}{1.164717in}}%
\pgfpathlineto{\pgfqpoint{2.924048in}{1.204172in}}%
\pgfpathlineto{\pgfqpoint{2.936261in}{1.125262in}}%
\pgfpathlineto{\pgfqpoint{2.948474in}{1.112110in}}%
\pgfpathlineto{\pgfqpoint{2.960686in}{1.085807in}}%
\pgfpathlineto{\pgfqpoint{2.972899in}{1.033200in}}%
\pgfpathlineto{\pgfqpoint{2.985112in}{1.085807in}}%
\pgfpathlineto{\pgfqpoint{2.997325in}{1.029190in}}%
\pgfpathlineto{\pgfqpoint{3.009538in}{2.891593in}}%
\pgfpathlineto{\pgfqpoint{3.021750in}{2.970287in}}%
\pgfpathlineto{\pgfqpoint{3.046176in}{3.101442in}}%
\pgfpathlineto{\pgfqpoint{3.058389in}{3.075211in}}%
\pgfpathlineto{\pgfqpoint{3.070601in}{3.062095in}}%
\pgfpathlineto{\pgfqpoint{3.082814in}{3.167019in}}%
\pgfpathlineto{\pgfqpoint{3.095027in}{3.180135in}}%
\pgfpathlineto{\pgfqpoint{3.107240in}{3.048980in}}%
\pgfpathlineto{\pgfqpoint{3.119452in}{3.009633in}}%
\pgfpathlineto{\pgfqpoint{3.131665in}{3.101442in}}%
\pgfpathlineto{\pgfqpoint{3.143878in}{3.019099in}}%
\pgfpathlineto{\pgfqpoint{3.156091in}{2.962678in}}%
\pgfpathlineto{\pgfqpoint{3.168303in}{3.015430in}}%
\pgfpathlineto{\pgfqpoint{3.180516in}{3.041806in}}%
\pgfpathlineto{\pgfqpoint{3.192729in}{3.028618in}}%
\pgfpathlineto{\pgfqpoint{3.204942in}{2.975866in}}%
\pgfpathlineto{\pgfqpoint{3.217155in}{2.936302in}}%
\pgfpathlineto{\pgfqpoint{3.229367in}{3.120934in}}%
\pgfpathlineto{\pgfqpoint{3.241580in}{3.120934in}}%
\pgfpathlineto{\pgfqpoint{3.253793in}{3.213250in}}%
\pgfpathlineto{\pgfqpoint{3.267682in}{3.138260in}}%
\pgfpathlineto{\pgfqpoint{3.267682in}{3.138260in}}%
\pgfusepath{stroke}%
\end{pgfscope}%
\begin{pgfscope}%
\pgfpathrectangle{\pgfqpoint{0.566985in}{0.528177in}}{\pgfqpoint{2.686808in}{2.864429in}} %
\pgfusepath{clip}%
\pgfsetroundcap%
\pgfsetroundjoin%
\pgfsetlinewidth{1.756562pt}%
\definecolor{currentstroke}{rgb}{1.000000,0.494118,0.250980}%
\pgfsetstrokecolor{currentstroke}%
\pgfsetstrokeopacity{0.800000}%
\pgfsetdash{}{0pt}%
\pgfpathmoveto{\pgfqpoint{0.566985in}{2.818361in}}%
\pgfpathlineto{\pgfqpoint{0.579197in}{2.495561in}}%
\pgfpathlineto{\pgfqpoint{0.591410in}{2.767393in}}%
\pgfpathlineto{\pgfqpoint{0.603623in}{2.750403in}}%
\pgfpathlineto{\pgfqpoint{0.615836in}{2.835351in}}%
\pgfpathlineto{\pgfqpoint{0.640261in}{2.903309in}}%
\pgfpathlineto{\pgfqpoint{0.652474in}{2.852340in}}%
\pgfpathlineto{\pgfqpoint{0.664687in}{2.903309in}}%
\pgfpathlineto{\pgfqpoint{0.676899in}{2.631477in}}%
\pgfpathlineto{\pgfqpoint{0.689112in}{2.563519in}}%
\pgfpathlineto{\pgfqpoint{0.701325in}{2.563519in}}%
\pgfpathlineto{\pgfqpoint{0.713538in}{2.580508in}}%
\pgfpathlineto{\pgfqpoint{0.725751in}{2.495561in}}%
\pgfpathlineto{\pgfqpoint{0.737963in}{2.784382in}}%
\pgfpathlineto{\pgfqpoint{0.750176in}{2.716424in}}%
\pgfpathlineto{\pgfqpoint{0.762389in}{2.801372in}}%
\pgfpathlineto{\pgfqpoint{0.774602in}{2.682445in}}%
\pgfpathlineto{\pgfqpoint{0.786814in}{2.682445in}}%
\pgfpathlineto{\pgfqpoint{0.799027in}{2.767393in}}%
\pgfpathlineto{\pgfqpoint{0.811240in}{2.716424in}}%
\pgfpathlineto{\pgfqpoint{0.823453in}{2.648466in}}%
\pgfpathlineto{\pgfqpoint{0.835665in}{2.699435in}}%
\pgfpathlineto{\pgfqpoint{0.847878in}{2.801372in}}%
\pgfpathlineto{\pgfqpoint{0.860091in}{2.767393in}}%
\pgfpathlineto{\pgfqpoint{0.872304in}{2.716424in}}%
\pgfpathlineto{\pgfqpoint{0.884516in}{2.852340in}}%
\pgfpathlineto{\pgfqpoint{0.896729in}{2.699435in}}%
\pgfpathlineto{\pgfqpoint{0.908942in}{2.648466in}}%
\pgfpathlineto{\pgfqpoint{0.921155in}{2.665456in}}%
\pgfpathlineto{\pgfqpoint{0.933368in}{2.716424in}}%
\pgfpathlineto{\pgfqpoint{0.945580in}{2.733414in}}%
\pgfpathlineto{\pgfqpoint{0.957793in}{2.614487in}}%
\pgfpathlineto{\pgfqpoint{0.970006in}{2.580508in}}%
\pgfpathlineto{\pgfqpoint{0.982219in}{2.682445in}}%
\pgfpathlineto{\pgfqpoint{0.994431in}{2.580508in}}%
\pgfpathlineto{\pgfqpoint{1.006644in}{2.631477in}}%
\pgfpathlineto{\pgfqpoint{1.018857in}{2.631477in}}%
\pgfpathlineto{\pgfqpoint{1.031070in}{2.529540in}}%
\pgfpathlineto{\pgfqpoint{1.043282in}{2.614487in}}%
\pgfpathlineto{\pgfqpoint{1.055495in}{2.665456in}}%
\pgfpathlineto{\pgfqpoint{1.067708in}{2.512550in}}%
\pgfpathlineto{\pgfqpoint{1.079921in}{2.665456in}}%
\pgfpathlineto{\pgfqpoint{1.092133in}{2.614487in}}%
\pgfpathlineto{\pgfqpoint{1.104346in}{2.648466in}}%
\pgfpathlineto{\pgfqpoint{1.116559in}{2.614487in}}%
\pgfpathlineto{\pgfqpoint{1.128772in}{2.648466in}}%
\pgfpathlineto{\pgfqpoint{1.140985in}{2.631477in}}%
\pgfpathlineto{\pgfqpoint{1.153197in}{2.767393in}}%
\pgfpathlineto{\pgfqpoint{1.165410in}{2.818361in}}%
\pgfpathlineto{\pgfqpoint{1.177623in}{2.648466in}}%
\pgfpathlineto{\pgfqpoint{1.189836in}{2.886319in}}%
\pgfpathlineto{\pgfqpoint{1.202048in}{2.835351in}}%
\pgfpathlineto{\pgfqpoint{1.214261in}{2.699435in}}%
\pgfpathlineto{\pgfqpoint{1.226474in}{2.597498in}}%
\pgfpathlineto{\pgfqpoint{1.238687in}{2.750403in}}%
\pgfpathlineto{\pgfqpoint{1.250899in}{2.665456in}}%
\pgfpathlineto{\pgfqpoint{1.263112in}{2.767393in}}%
\pgfpathlineto{\pgfqpoint{1.275325in}{2.716424in}}%
\pgfpathlineto{\pgfqpoint{1.287538in}{2.648466in}}%
\pgfpathlineto{\pgfqpoint{1.299750in}{2.597498in}}%
\pgfpathlineto{\pgfqpoint{1.311963in}{2.733414in}}%
\pgfpathlineto{\pgfqpoint{1.324176in}{2.682445in}}%
\pgfpathlineto{\pgfqpoint{1.348602in}{2.614487in}}%
\pgfpathlineto{\pgfqpoint{1.360814in}{2.529540in}}%
\pgfpathlineto{\pgfqpoint{1.373027in}{2.495561in}}%
\pgfpathlineto{\pgfqpoint{1.385240in}{2.359645in}}%
\pgfpathlineto{\pgfqpoint{1.397453in}{2.546529in}}%
\pgfpathlineto{\pgfqpoint{1.409665in}{2.563519in}}%
\pgfpathlineto{\pgfqpoint{1.421878in}{2.546529in}}%
\pgfpathlineto{\pgfqpoint{1.434091in}{2.631477in}}%
\pgfpathlineto{\pgfqpoint{1.446304in}{2.597498in}}%
\pgfpathlineto{\pgfqpoint{1.458516in}{2.461582in}}%
\pgfpathlineto{\pgfqpoint{1.470729in}{2.478571in}}%
\pgfpathlineto{\pgfqpoint{1.482942in}{2.461582in}}%
\pgfpathlineto{\pgfqpoint{1.495155in}{2.682445in}}%
\pgfpathlineto{\pgfqpoint{1.507367in}{2.495561in}}%
\pgfpathlineto{\pgfqpoint{1.519580in}{2.478571in}}%
\pgfpathlineto{\pgfqpoint{1.531793in}{2.529540in}}%
\pgfpathlineto{\pgfqpoint{1.544006in}{2.444592in}}%
\pgfpathlineto{\pgfqpoint{1.556219in}{2.597498in}}%
\pgfpathlineto{\pgfqpoint{1.568431in}{2.478571in}}%
\pgfpathlineto{\pgfqpoint{1.580644in}{2.563519in}}%
\pgfpathlineto{\pgfqpoint{1.592857in}{2.614487in}}%
\pgfpathlineto{\pgfqpoint{1.605070in}{2.529540in}}%
\pgfpathlineto{\pgfqpoint{1.617282in}{2.665456in}}%
\pgfpathlineto{\pgfqpoint{1.629495in}{2.614487in}}%
\pgfpathlineto{\pgfqpoint{1.641708in}{2.767393in}}%
\pgfpathlineto{\pgfqpoint{1.653921in}{2.529540in}}%
\pgfpathlineto{\pgfqpoint{1.666133in}{2.563519in}}%
\pgfpathlineto{\pgfqpoint{1.678346in}{2.835351in}}%
\pgfpathlineto{\pgfqpoint{1.690559in}{2.665456in}}%
\pgfpathlineto{\pgfqpoint{1.702772in}{2.750403in}}%
\pgfpathlineto{\pgfqpoint{1.714984in}{2.852340in}}%
\pgfpathlineto{\pgfqpoint{1.727197in}{2.852340in}}%
\pgfpathlineto{\pgfqpoint{1.739410in}{2.767393in}}%
\pgfpathlineto{\pgfqpoint{1.751623in}{2.784382in}}%
\pgfpathlineto{\pgfqpoint{1.763836in}{2.733414in}}%
\pgfpathlineto{\pgfqpoint{1.776048in}{2.852340in}}%
\pgfpathlineto{\pgfqpoint{1.788261in}{2.835351in}}%
\pgfpathlineto{\pgfqpoint{1.800474in}{2.614487in}}%
\pgfpathlineto{\pgfqpoint{1.812687in}{2.818361in}}%
\pgfpathlineto{\pgfqpoint{1.824899in}{2.580508in}}%
\pgfpathlineto{\pgfqpoint{1.837112in}{2.665456in}}%
\pgfpathlineto{\pgfqpoint{1.849325in}{2.512550in}}%
\pgfpathlineto{\pgfqpoint{1.861538in}{2.631477in}}%
\pgfpathlineto{\pgfqpoint{1.873750in}{2.733414in}}%
\pgfpathlineto{\pgfqpoint{1.885963in}{2.614487in}}%
\pgfpathlineto{\pgfqpoint{1.898176in}{2.682445in}}%
\pgfpathlineto{\pgfqpoint{1.910389in}{2.716424in}}%
\pgfpathlineto{\pgfqpoint{1.922601in}{2.631477in}}%
\pgfpathlineto{\pgfqpoint{1.934814in}{2.631477in}}%
\pgfpathlineto{\pgfqpoint{1.947027in}{2.648466in}}%
\pgfpathlineto{\pgfqpoint{1.959240in}{2.529540in}}%
\pgfpathlineto{\pgfqpoint{1.971453in}{2.478571in}}%
\pgfpathlineto{\pgfqpoint{1.983665in}{2.512550in}}%
\pgfpathlineto{\pgfqpoint{1.995878in}{2.512550in}}%
\pgfpathlineto{\pgfqpoint{2.008091in}{2.597498in}}%
\pgfpathlineto{\pgfqpoint{2.020304in}{2.444592in}}%
\pgfpathlineto{\pgfqpoint{2.032516in}{2.427603in}}%
\pgfpathlineto{\pgfqpoint{2.044729in}{2.393624in}}%
\pgfpathlineto{\pgfqpoint{2.056942in}{2.699435in}}%
\pgfpathlineto{\pgfqpoint{2.069155in}{2.359645in}}%
\pgfpathlineto{\pgfqpoint{2.081367in}{2.478571in}}%
\pgfpathlineto{\pgfqpoint{2.093580in}{2.580508in}}%
\pgfpathlineto{\pgfqpoint{2.105793in}{2.580508in}}%
\pgfpathlineto{\pgfqpoint{2.118006in}{2.529540in}}%
\pgfpathlineto{\pgfqpoint{2.130218in}{2.376634in}}%
\pgfpathlineto{\pgfqpoint{2.142431in}{2.257708in}}%
\pgfpathlineto{\pgfqpoint{2.154644in}{2.427603in}}%
\pgfpathlineto{\pgfqpoint{2.166857in}{2.291687in}}%
\pgfpathlineto{\pgfqpoint{2.179070in}{2.376634in}}%
\pgfpathlineto{\pgfqpoint{2.191282in}{2.393624in}}%
\pgfpathlineto{\pgfqpoint{2.203495in}{2.393624in}}%
\pgfpathlineto{\pgfqpoint{2.215708in}{2.291687in}}%
\pgfpathlineto{\pgfqpoint{2.227921in}{2.376634in}}%
\pgfpathlineto{\pgfqpoint{2.240133in}{2.393624in}}%
\pgfpathlineto{\pgfqpoint{2.252346in}{2.478571in}}%
\pgfpathlineto{\pgfqpoint{2.264559in}{2.495561in}}%
\pgfpathlineto{\pgfqpoint{2.276772in}{2.410613in}}%
\pgfpathlineto{\pgfqpoint{2.288984in}{2.393624in}}%
\pgfpathlineto{\pgfqpoint{2.301197in}{2.393624in}}%
\pgfpathlineto{\pgfqpoint{2.313410in}{2.546529in}}%
\pgfpathlineto{\pgfqpoint{2.325623in}{2.546529in}}%
\pgfpathlineto{\pgfqpoint{2.337835in}{2.495561in}}%
\pgfpathlineto{\pgfqpoint{2.350048in}{2.580508in}}%
\pgfpathlineto{\pgfqpoint{2.362261in}{2.682445in}}%
\pgfpathlineto{\pgfqpoint{2.374474in}{2.614487in}}%
\pgfpathlineto{\pgfqpoint{2.386687in}{2.512550in}}%
\pgfpathlineto{\pgfqpoint{2.398899in}{2.376634in}}%
\pgfpathlineto{\pgfqpoint{2.411112in}{2.410613in}}%
\pgfpathlineto{\pgfqpoint{2.423325in}{2.291687in}}%
\pgfpathlineto{\pgfqpoint{2.435538in}{2.393624in}}%
\pgfpathlineto{\pgfqpoint{2.459963in}{2.461582in}}%
\pgfpathlineto{\pgfqpoint{2.472176in}{2.478571in}}%
\pgfpathlineto{\pgfqpoint{2.484389in}{2.427603in}}%
\pgfpathlineto{\pgfqpoint{2.496601in}{2.393624in}}%
\pgfpathlineto{\pgfqpoint{2.508814in}{2.223729in}}%
\pgfpathlineto{\pgfqpoint{2.521027in}{2.410613in}}%
\pgfpathlineto{\pgfqpoint{2.545452in}{2.240718in}}%
\pgfpathlineto{\pgfqpoint{2.557665in}{2.138781in}}%
\pgfpathlineto{\pgfqpoint{2.569878in}{2.121792in}}%
\pgfpathlineto{\pgfqpoint{2.582091in}{2.036844in}}%
\pgfpathlineto{\pgfqpoint{2.594304in}{2.070823in}}%
\pgfpathlineto{\pgfqpoint{2.606516in}{2.189750in}}%
\pgfpathlineto{\pgfqpoint{2.618729in}{2.053834in}}%
\pgfpathlineto{\pgfqpoint{2.630942in}{2.087813in}}%
\pgfpathlineto{\pgfqpoint{2.643155in}{1.883939in}}%
\pgfpathlineto{\pgfqpoint{2.655367in}{1.934907in}}%
\pgfpathlineto{\pgfqpoint{2.667580in}{2.070823in}}%
\pgfpathlineto{\pgfqpoint{2.679793in}{1.765012in}}%
\pgfpathlineto{\pgfqpoint{2.692006in}{1.866949in}}%
\pgfpathlineto{\pgfqpoint{2.704218in}{1.951897in}}%
\pgfpathlineto{\pgfqpoint{2.716431in}{1.748023in}}%
\pgfpathlineto{\pgfqpoint{2.728644in}{1.815981in}}%
\pgfpathlineto{\pgfqpoint{2.740857in}{1.782002in}}%
\pgfpathlineto{\pgfqpoint{2.753069in}{1.815981in}}%
\pgfpathlineto{\pgfqpoint{2.765282in}{1.782002in}}%
\pgfpathlineto{\pgfqpoint{2.777495in}{1.629096in}}%
\pgfpathlineto{\pgfqpoint{2.789708in}{1.714044in}}%
\pgfpathlineto{\pgfqpoint{2.801921in}{1.595117in}}%
\pgfpathlineto{\pgfqpoint{2.814133in}{1.629096in}}%
\pgfpathlineto{\pgfqpoint{2.826346in}{1.697054in}}%
\pgfpathlineto{\pgfqpoint{2.838559in}{1.731033in}}%
\pgfpathlineto{\pgfqpoint{2.850772in}{1.629096in}}%
\pgfpathlineto{\pgfqpoint{2.862984in}{1.612107in}}%
\pgfpathlineto{\pgfqpoint{2.875197in}{1.612107in}}%
\pgfpathlineto{\pgfqpoint{2.887410in}{1.493180in}}%
\pgfpathlineto{\pgfqpoint{2.899623in}{1.731033in}}%
\pgfpathlineto{\pgfqpoint{2.911835in}{1.731033in}}%
\pgfpathlineto{\pgfqpoint{2.924048in}{1.578128in}}%
\pgfpathlineto{\pgfqpoint{2.936261in}{1.595117in}}%
\pgfpathlineto{\pgfqpoint{2.948474in}{1.510170in}}%
\pgfpathlineto{\pgfqpoint{2.960686in}{1.561138in}}%
\pgfpathlineto{\pgfqpoint{2.972899in}{1.544149in}}%
\pgfpathlineto{\pgfqpoint{2.985112in}{1.476191in}}%
\pgfpathlineto{\pgfqpoint{2.997325in}{1.289306in}}%
\pgfpathlineto{\pgfqpoint{3.009538in}{2.631477in}}%
\pgfpathlineto{\pgfqpoint{3.021750in}{2.563519in}}%
\pgfpathlineto{\pgfqpoint{3.033963in}{2.512550in}}%
\pgfpathlineto{\pgfqpoint{3.046176in}{2.495561in}}%
\pgfpathlineto{\pgfqpoint{3.058389in}{2.784382in}}%
\pgfpathlineto{\pgfqpoint{3.070601in}{2.665456in}}%
\pgfpathlineto{\pgfqpoint{3.082814in}{2.665456in}}%
\pgfpathlineto{\pgfqpoint{3.095027in}{2.835351in}}%
\pgfpathlineto{\pgfqpoint{3.107240in}{2.750403in}}%
\pgfpathlineto{\pgfqpoint{3.119452in}{2.886319in}}%
\pgfpathlineto{\pgfqpoint{3.131665in}{2.835351in}}%
\pgfpathlineto{\pgfqpoint{3.143878in}{3.039225in}}%
\pgfpathlineto{\pgfqpoint{3.156091in}{2.665456in}}%
\pgfpathlineto{\pgfqpoint{3.168303in}{2.733414in}}%
\pgfpathlineto{\pgfqpoint{3.180516in}{2.682445in}}%
\pgfpathlineto{\pgfqpoint{3.192729in}{2.750403in}}%
\pgfpathlineto{\pgfqpoint{3.204942in}{2.682445in}}%
\pgfpathlineto{\pgfqpoint{3.217155in}{2.835351in}}%
\pgfpathlineto{\pgfqpoint{3.229367in}{2.563519in}}%
\pgfpathlineto{\pgfqpoint{3.241580in}{2.699435in}}%
\pgfpathlineto{\pgfqpoint{3.253793in}{2.801372in}}%
\pgfpathlineto{\pgfqpoint{3.267682in}{2.702434in}}%
\pgfpathlineto{\pgfqpoint{3.267682in}{2.702434in}}%
\pgfusepath{stroke}%
\end{pgfscope}%
\begin{pgfscope}%
\pgfpathrectangle{\pgfqpoint{0.566985in}{0.528177in}}{\pgfqpoint{2.686808in}{2.864429in}} %
\pgfusepath{clip}%
\pgfsetroundcap%
\pgfsetroundjoin%
\pgfsetlinewidth{1.756562pt}%
\definecolor{currentstroke}{rgb}{1.000000,0.694118,0.250980}%
\pgfsetstrokecolor{currentstroke}%
\pgfsetstrokeopacity{0.800000}%
\pgfsetdash{}{0pt}%
\pgfpathmoveto{\pgfqpoint{0.566985in}{2.704582in}}%
\pgfpathlineto{\pgfqpoint{0.579197in}{2.676499in}}%
\pgfpathlineto{\pgfqpoint{0.591410in}{2.592251in}}%
\pgfpathlineto{\pgfqpoint{0.603623in}{2.606292in}}%
\pgfpathlineto{\pgfqpoint{0.615836in}{2.550127in}}%
\pgfpathlineto{\pgfqpoint{0.628048in}{2.620334in}}%
\pgfpathlineto{\pgfqpoint{0.640261in}{2.718623in}}%
\pgfpathlineto{\pgfqpoint{0.652474in}{2.578210in}}%
\pgfpathlineto{\pgfqpoint{0.664687in}{2.620334in}}%
\pgfpathlineto{\pgfqpoint{0.676899in}{2.634375in}}%
\pgfpathlineto{\pgfqpoint{0.689112in}{2.606292in}}%
\pgfpathlineto{\pgfqpoint{0.701325in}{2.634375in}}%
\pgfpathlineto{\pgfqpoint{0.713538in}{2.704582in}}%
\pgfpathlineto{\pgfqpoint{0.725751in}{2.592251in}}%
\pgfpathlineto{\pgfqpoint{0.737963in}{2.634375in}}%
\pgfpathlineto{\pgfqpoint{0.750176in}{2.662458in}}%
\pgfpathlineto{\pgfqpoint{0.762389in}{2.620334in}}%
\pgfpathlineto{\pgfqpoint{0.774602in}{2.479920in}}%
\pgfpathlineto{\pgfqpoint{0.786814in}{2.522044in}}%
\pgfpathlineto{\pgfqpoint{0.799027in}{2.718623in}}%
\pgfpathlineto{\pgfqpoint{0.811240in}{2.634375in}}%
\pgfpathlineto{\pgfqpoint{0.823453in}{2.690540in}}%
\pgfpathlineto{\pgfqpoint{0.835665in}{2.709399in}}%
\pgfpathlineto{\pgfqpoint{0.847878in}{2.658352in}}%
\pgfpathlineto{\pgfqpoint{0.860091in}{2.742107in}}%
\pgfpathlineto{\pgfqpoint{0.872304in}{2.714189in}}%
\pgfpathlineto{\pgfqpoint{0.884516in}{2.644393in}}%
\pgfpathlineto{\pgfqpoint{0.896729in}{2.644393in}}%
\pgfpathlineto{\pgfqpoint{0.908942in}{2.728148in}}%
\pgfpathlineto{\pgfqpoint{0.921155in}{2.644393in}}%
\pgfpathlineto{\pgfqpoint{0.933368in}{2.686270in}}%
\pgfpathlineto{\pgfqpoint{0.945580in}{2.630433in}}%
\pgfpathlineto{\pgfqpoint{0.957793in}{2.756066in}}%
\pgfpathlineto{\pgfqpoint{0.970006in}{2.658352in}}%
\pgfpathlineto{\pgfqpoint{0.982219in}{2.714189in}}%
\pgfpathlineto{\pgfqpoint{0.994431in}{2.965454in}}%
\pgfpathlineto{\pgfqpoint{1.006644in}{2.742107in}}%
\pgfpathlineto{\pgfqpoint{1.018857in}{2.797944in}}%
\pgfpathlineto{\pgfqpoint{1.031070in}{2.672311in}}%
\pgfpathlineto{\pgfqpoint{1.043282in}{2.616474in}}%
\pgfpathlineto{\pgfqpoint{1.055495in}{2.658352in}}%
\pgfpathlineto{\pgfqpoint{1.067708in}{2.672311in}}%
\pgfpathlineto{\pgfqpoint{1.079921in}{2.644393in}}%
\pgfpathlineto{\pgfqpoint{1.092133in}{2.770026in}}%
\pgfpathlineto{\pgfqpoint{1.104346in}{2.672311in}}%
\pgfpathlineto{\pgfqpoint{1.116559in}{2.630433in}}%
\pgfpathlineto{\pgfqpoint{1.128772in}{2.630433in}}%
\pgfpathlineto{\pgfqpoint{1.140985in}{2.644393in}}%
\pgfpathlineto{\pgfqpoint{1.153197in}{2.616474in}}%
\pgfpathlineto{\pgfqpoint{1.165410in}{2.658352in}}%
\pgfpathlineto{\pgfqpoint{1.189836in}{2.658352in}}%
\pgfpathlineto{\pgfqpoint{1.202048in}{2.616474in}}%
\pgfpathlineto{\pgfqpoint{1.214261in}{2.644393in}}%
\pgfpathlineto{\pgfqpoint{1.226474in}{2.686270in}}%
\pgfpathlineto{\pgfqpoint{1.238687in}{2.686270in}}%
\pgfpathlineto{\pgfqpoint{1.250899in}{2.616474in}}%
\pgfpathlineto{\pgfqpoint{1.263112in}{2.714189in}}%
\pgfpathlineto{\pgfqpoint{1.275325in}{2.672311in}}%
\pgfpathlineto{\pgfqpoint{1.287538in}{2.714189in}}%
\pgfpathlineto{\pgfqpoint{1.299750in}{2.742107in}}%
\pgfpathlineto{\pgfqpoint{1.311963in}{2.630433in}}%
\pgfpathlineto{\pgfqpoint{1.324176in}{2.630433in}}%
\pgfpathlineto{\pgfqpoint{1.336389in}{2.714189in}}%
\pgfpathlineto{\pgfqpoint{1.348602in}{2.574597in}}%
\pgfpathlineto{\pgfqpoint{1.360814in}{2.672311in}}%
\pgfpathlineto{\pgfqpoint{1.373027in}{2.797944in}}%
\pgfpathlineto{\pgfqpoint{1.385240in}{2.546678in}}%
\pgfpathlineto{\pgfqpoint{1.397453in}{2.546678in}}%
\pgfpathlineto{\pgfqpoint{1.409665in}{2.504801in}}%
\pgfpathlineto{\pgfqpoint{1.434091in}{2.560637in}}%
\pgfpathlineto{\pgfqpoint{1.446304in}{2.574597in}}%
\pgfpathlineto{\pgfqpoint{1.458516in}{2.658352in}}%
\pgfpathlineto{\pgfqpoint{1.470729in}{2.635439in}}%
\pgfpathlineto{\pgfqpoint{1.482942in}{2.649358in}}%
\pgfpathlineto{\pgfqpoint{1.495155in}{2.607602in}}%
\pgfpathlineto{\pgfqpoint{1.507367in}{2.635439in}}%
\pgfpathlineto{\pgfqpoint{1.531793in}{2.718950in}}%
\pgfpathlineto{\pgfqpoint{1.544006in}{2.774624in}}%
\pgfpathlineto{\pgfqpoint{1.556219in}{2.746787in}}%
\pgfpathlineto{\pgfqpoint{1.568431in}{2.844217in}}%
\pgfpathlineto{\pgfqpoint{1.580644in}{2.816380in}}%
\pgfpathlineto{\pgfqpoint{1.592857in}{2.760706in}}%
\pgfpathlineto{\pgfqpoint{1.605070in}{2.732869in}}%
\pgfpathlineto{\pgfqpoint{1.617282in}{2.635439in}}%
\pgfpathlineto{\pgfqpoint{1.629495in}{2.565847in}}%
\pgfpathlineto{\pgfqpoint{1.641708in}{2.621521in}}%
\pgfpathlineto{\pgfqpoint{1.653921in}{2.607602in}}%
\pgfpathlineto{\pgfqpoint{1.666133in}{2.607602in}}%
\pgfpathlineto{\pgfqpoint{1.678346in}{2.510173in}}%
\pgfpathlineto{\pgfqpoint{1.690559in}{2.482336in}}%
\pgfpathlineto{\pgfqpoint{1.702772in}{2.482336in}}%
\pgfpathlineto{\pgfqpoint{1.714984in}{2.579765in}}%
\pgfpathlineto{\pgfqpoint{1.727197in}{2.593684in}}%
\pgfpathlineto{\pgfqpoint{1.739410in}{2.551928in}}%
\pgfpathlineto{\pgfqpoint{1.751623in}{2.538010in}}%
\pgfpathlineto{\pgfqpoint{1.763836in}{2.691113in}}%
\pgfpathlineto{\pgfqpoint{1.776048in}{2.579765in}}%
\pgfpathlineto{\pgfqpoint{1.788261in}{2.649358in}}%
\pgfpathlineto{\pgfqpoint{1.800474in}{2.705032in}}%
\pgfpathlineto{\pgfqpoint{1.812687in}{2.709806in}}%
\pgfpathlineto{\pgfqpoint{1.824899in}{2.640416in}}%
\pgfpathlineto{\pgfqpoint{1.837112in}{2.584904in}}%
\pgfpathlineto{\pgfqpoint{1.849325in}{2.487757in}}%
\pgfpathlineto{\pgfqpoint{1.861538in}{2.487757in}}%
\pgfpathlineto{\pgfqpoint{1.873750in}{2.612660in}}%
\pgfpathlineto{\pgfqpoint{1.885963in}{2.571026in}}%
\pgfpathlineto{\pgfqpoint{1.898176in}{2.626538in}}%
\pgfpathlineto{\pgfqpoint{1.910389in}{2.584904in}}%
\pgfpathlineto{\pgfqpoint{1.922601in}{2.557148in}}%
\pgfpathlineto{\pgfqpoint{1.934814in}{2.515514in}}%
\pgfpathlineto{\pgfqpoint{1.947027in}{2.446123in}}%
\pgfpathlineto{\pgfqpoint{1.959240in}{2.571026in}}%
\pgfpathlineto{\pgfqpoint{1.971453in}{2.584904in}}%
\pgfpathlineto{\pgfqpoint{1.983665in}{2.543270in}}%
\pgfpathlineto{\pgfqpoint{1.995878in}{2.487757in}}%
\pgfpathlineto{\pgfqpoint{2.008091in}{2.515514in}}%
\pgfpathlineto{\pgfqpoint{2.020304in}{2.598782in}}%
\pgfpathlineto{\pgfqpoint{2.032516in}{2.584904in}}%
\pgfpathlineto{\pgfqpoint{2.044729in}{2.598782in}}%
\pgfpathlineto{\pgfqpoint{2.056942in}{2.557148in}}%
\pgfpathlineto{\pgfqpoint{2.069155in}{2.584904in}}%
\pgfpathlineto{\pgfqpoint{2.081367in}{2.418367in}}%
\pgfpathlineto{\pgfqpoint{2.093580in}{2.571026in}}%
\pgfpathlineto{\pgfqpoint{2.105793in}{2.487757in}}%
\pgfpathlineto{\pgfqpoint{2.118006in}{2.487757in}}%
\pgfpathlineto{\pgfqpoint{2.130218in}{2.418367in}}%
\pgfpathlineto{\pgfqpoint{2.142431in}{2.584904in}}%
\pgfpathlineto{\pgfqpoint{2.154644in}{2.432245in}}%
\pgfpathlineto{\pgfqpoint{2.166857in}{2.473879in}}%
\pgfpathlineto{\pgfqpoint{2.179070in}{2.571026in}}%
\pgfpathlineto{\pgfqpoint{2.191282in}{2.432245in}}%
\pgfpathlineto{\pgfqpoint{2.203495in}{2.557148in}}%
\pgfpathlineto{\pgfqpoint{2.215708in}{2.557148in}}%
\pgfpathlineto{\pgfqpoint{2.227921in}{2.460001in}}%
\pgfpathlineto{\pgfqpoint{2.240133in}{2.571026in}}%
\pgfpathlineto{\pgfqpoint{2.252346in}{2.612660in}}%
\pgfpathlineto{\pgfqpoint{2.264559in}{2.571026in}}%
\pgfpathlineto{\pgfqpoint{2.276772in}{2.571026in}}%
\pgfpathlineto{\pgfqpoint{2.288984in}{2.515514in}}%
\pgfpathlineto{\pgfqpoint{2.301197in}{2.626538in}}%
\pgfpathlineto{\pgfqpoint{2.313410in}{2.598782in}}%
\pgfpathlineto{\pgfqpoint{2.325623in}{2.584904in}}%
\pgfpathlineto{\pgfqpoint{2.337835in}{2.584904in}}%
\pgfpathlineto{\pgfqpoint{2.350048in}{2.390611in}}%
\pgfpathlineto{\pgfqpoint{2.362261in}{2.432245in}}%
\pgfpathlineto{\pgfqpoint{2.374474in}{2.515514in}}%
\pgfpathlineto{\pgfqpoint{2.386687in}{2.501635in}}%
\pgfpathlineto{\pgfqpoint{2.398899in}{2.460001in}}%
\pgfpathlineto{\pgfqpoint{2.411112in}{2.404489in}}%
\pgfpathlineto{\pgfqpoint{2.435538in}{2.196318in}}%
\pgfpathlineto{\pgfqpoint{2.447750in}{2.279587in}}%
\pgfpathlineto{\pgfqpoint{2.459963in}{2.376733in}}%
\pgfpathlineto{\pgfqpoint{2.472176in}{2.348977in}}%
\pgfpathlineto{\pgfqpoint{2.484389in}{2.362855in}}%
\pgfpathlineto{\pgfqpoint{2.496601in}{2.362855in}}%
\pgfpathlineto{\pgfqpoint{2.508814in}{2.460001in}}%
\pgfpathlineto{\pgfqpoint{2.521027in}{2.279587in}}%
\pgfpathlineto{\pgfqpoint{2.533240in}{2.140806in}}%
\pgfpathlineto{\pgfqpoint{2.545452in}{2.182440in}}%
\pgfpathlineto{\pgfqpoint{2.557665in}{2.126928in}}%
\pgfpathlineto{\pgfqpoint{2.569878in}{2.154684in}}%
\pgfpathlineto{\pgfqpoint{2.582091in}{2.154684in}}%
\pgfpathlineto{\pgfqpoint{2.594304in}{2.057538in}}%
\pgfpathlineto{\pgfqpoint{2.606516in}{1.988148in}}%
\pgfpathlineto{\pgfqpoint{2.618729in}{1.974270in}}%
\pgfpathlineto{\pgfqpoint{2.630942in}{2.071416in}}%
\pgfpathlineto{\pgfqpoint{2.643155in}{2.071416in}}%
\pgfpathlineto{\pgfqpoint{2.655367in}{2.126928in}}%
\pgfpathlineto{\pgfqpoint{2.667580in}{2.029782in}}%
\pgfpathlineto{\pgfqpoint{2.679793in}{1.960392in}}%
\pgfpathlineto{\pgfqpoint{2.692006in}{2.015904in}}%
\pgfpathlineto{\pgfqpoint{2.716431in}{1.779977in}}%
\pgfpathlineto{\pgfqpoint{2.728644in}{1.655074in}}%
\pgfpathlineto{\pgfqpoint{2.740857in}{1.738343in}}%
\pgfpathlineto{\pgfqpoint{2.753069in}{1.766099in}}%
\pgfpathlineto{\pgfqpoint{2.765282in}{1.613440in}}%
\pgfpathlineto{\pgfqpoint{2.777495in}{1.724465in}}%
\pgfpathlineto{\pgfqpoint{2.789708in}{1.641196in}}%
\pgfpathlineto{\pgfqpoint{2.814133in}{1.557928in}}%
\pgfpathlineto{\pgfqpoint{2.826346in}{1.446904in}}%
\pgfpathlineto{\pgfqpoint{2.838559in}{1.363635in}}%
\pgfpathlineto{\pgfqpoint{2.850772in}{1.252611in}}%
\pgfpathlineto{\pgfqpoint{2.862984in}{1.252611in}}%
\pgfpathlineto{\pgfqpoint{2.875197in}{1.183221in}}%
\pgfpathlineto{\pgfqpoint{2.899623in}{1.072196in}}%
\pgfpathlineto{\pgfqpoint{2.911835in}{1.183221in}}%
\pgfpathlineto{\pgfqpoint{2.924048in}{1.169343in}}%
\pgfpathlineto{\pgfqpoint{2.948474in}{1.169343in}}%
\pgfpathlineto{\pgfqpoint{2.960686in}{1.127709in}}%
\pgfpathlineto{\pgfqpoint{2.985112in}{1.099952in}}%
\pgfpathlineto{\pgfqpoint{2.997325in}{0.874122in}}%
\pgfpathlineto{\pgfqpoint{3.009538in}{2.977472in}}%
\pgfpathlineto{\pgfqpoint{3.021750in}{3.102012in}}%
\pgfpathlineto{\pgfqpoint{3.033963in}{3.088174in}}%
\pgfpathlineto{\pgfqpoint{3.046176in}{3.143525in}}%
\pgfpathlineto{\pgfqpoint{3.058389in}{3.074336in}}%
\pgfpathlineto{\pgfqpoint{3.070601in}{3.018985in}}%
\pgfpathlineto{\pgfqpoint{3.082814in}{3.212715in}}%
\pgfpathlineto{\pgfqpoint{3.095027in}{3.129688in}}%
\pgfpathlineto{\pgfqpoint{3.119452in}{3.323417in}}%
\pgfpathlineto{\pgfqpoint{3.131665in}{3.281904in}}%
\pgfpathlineto{\pgfqpoint{3.143878in}{3.281904in}}%
\pgfpathlineto{\pgfqpoint{3.156091in}{3.102012in}}%
\pgfpathlineto{\pgfqpoint{3.168303in}{3.140026in}}%
\pgfpathlineto{\pgfqpoint{3.180516in}{3.112270in}}%
\pgfpathlineto{\pgfqpoint{3.192729in}{3.223294in}}%
\pgfpathlineto{\pgfqpoint{3.204942in}{3.140026in}}%
\pgfpathlineto{\pgfqpoint{3.217155in}{3.070635in}}%
\pgfpathlineto{\pgfqpoint{3.229367in}{3.181660in}}%
\pgfpathlineto{\pgfqpoint{3.241580in}{3.251050in}}%
\pgfpathlineto{\pgfqpoint{3.253793in}{3.056757in}}%
\pgfpathlineto{\pgfqpoint{3.266006in}{3.080832in}}%
\pgfpathlineto{\pgfqpoint{3.267682in}{3.059819in}}%
\pgfpathlineto{\pgfqpoint{3.267682in}{3.059819in}}%
\pgfusepath{stroke}%
\end{pgfscope}%
\begin{pgfscope}%
\pgfpathrectangle{\pgfqpoint{0.566985in}{0.528177in}}{\pgfqpoint{2.686808in}{2.864429in}} %
\pgfusepath{clip}%
\pgfsetbuttcap%
\pgfsetroundjoin%
\pgfsetlinewidth{1.756562pt}%
\definecolor{currentstroke}{rgb}{0.501961,0.501961,0.501961}%
\pgfsetstrokecolor{currentstroke}%
\pgfsetdash{{6.000000pt}{6.000000pt}}{0.000000pt}%
\pgfpathmoveto{\pgfqpoint{3.009538in}{0.528177in}}%
\pgfpathlineto{\pgfqpoint{3.009538in}{3.392606in}}%
\pgfusepath{stroke}%
\end{pgfscope}%
\begin{pgfscope}%
\pgfsetrectcap%
\pgfsetmiterjoin%
\pgfsetlinewidth{1.254687pt}%
\definecolor{currentstroke}{rgb}{0.150000,0.150000,0.150000}%
\pgfsetstrokecolor{currentstroke}%
\pgfsetdash{}{0pt}%
\pgfpathmoveto{\pgfqpoint{0.566985in}{0.528177in}}%
\pgfpathlineto{\pgfqpoint{0.566985in}{3.392606in}}%
\pgfusepath{stroke}%
\end{pgfscope}%
\begin{pgfscope}%
\pgfsetrectcap%
\pgfsetmiterjoin%
\pgfsetlinewidth{1.254687pt}%
\definecolor{currentstroke}{rgb}{0.150000,0.150000,0.150000}%
\pgfsetstrokecolor{currentstroke}%
\pgfsetdash{}{0pt}%
\pgfpathmoveto{\pgfqpoint{0.566985in}{0.528177in}}%
\pgfpathlineto{\pgfqpoint{3.253793in}{0.528177in}}%
\pgfusepath{stroke}%
\end{pgfscope}%
\begin{pgfscope}%
\pgfsetbuttcap%
\pgfsetmiterjoin%
\definecolor{currentfill}{rgb}{1.000000,1.000000,1.000000}%
\pgfsetfillcolor{currentfill}%
\pgfsetlinewidth{0.000000pt}%
\definecolor{currentstroke}{rgb}{0.000000,0.000000,0.000000}%
\pgfsetstrokecolor{currentstroke}%
\pgfsetstrokeopacity{0.000000}%
\pgfsetdash{}{0pt}%
\pgfpathmoveto{\pgfqpoint{3.858325in}{0.528177in}}%
\pgfpathlineto{\pgfqpoint{5.201729in}{0.528177in}}%
\pgfpathlineto{\pgfqpoint{5.201729in}{2.653399in}}%
\pgfpathlineto{\pgfqpoint{3.858325in}{2.653399in}}%
\pgfpathclose%
\pgfusepath{fill}%
\end{pgfscope}%
\begin{pgfscope}%
\pgfsetroundcap%
\pgfsetroundjoin%
\pgfsetlinewidth{1.756562pt}%
\definecolor{currentstroke}{rgb}{0.200000,0.427451,0.650980}%
\pgfsetstrokecolor{currentstroke}%
\pgfsetstrokeopacity{0.800000}%
\pgfsetdash{}{0pt}%
\pgfpathmoveto{\pgfqpoint{3.689644in}{3.278686in}}%
\pgfpathlineto{\pgfqpoint{3.800755in}{3.278686in}}%
\pgfusepath{stroke}%
\end{pgfscope}%
\begin{pgfscope}%
\definecolor{textcolor}{rgb}{1.000000,1.000000,1.000000}%
\pgfsetstrokecolor{textcolor}%
\pgfsetfillcolor{textcolor}%
\pgftext[x=3.889644in,y=3.239798in,left,base]{\color{textcolor}\rmfamily\fontsize{8.000000}{9.600000}\selectfont WT + Vehicle}%
\end{pgfscope}%
\begin{pgfscope}%
\pgfsetroundcap%
\pgfsetroundjoin%
\pgfsetlinewidth{1.756562pt}%
\definecolor{currentstroke}{rgb}{0.168627,0.670588,0.494118}%
\pgfsetstrokecolor{currentstroke}%
\pgfsetstrokeopacity{0.800000}%
\pgfsetdash{}{0pt}%
\pgfpathmoveto{\pgfqpoint{3.689644in}{3.123753in}}%
\pgfpathlineto{\pgfqpoint{3.800755in}{3.123753in}}%
\pgfusepath{stroke}%
\end{pgfscope}%
\begin{pgfscope}%
\definecolor{textcolor}{rgb}{1.000000,1.000000,1.000000}%
\pgfsetstrokecolor{textcolor}%
\pgfsetfillcolor{textcolor}%
\pgftext[x=3.889644in,y=3.084864in,left,base]{\color{textcolor}\rmfamily\fontsize{8.000000}{9.600000}\selectfont WT + TAT-GluA2\textsubscript{3Y}}%
\end{pgfscope}%
\begin{pgfscope}%
\pgfsetroundcap%
\pgfsetroundjoin%
\pgfsetlinewidth{1.756562pt}%
\definecolor{currentstroke}{rgb}{1.000000,0.494118,0.250980}%
\pgfsetstrokecolor{currentstroke}%
\pgfsetstrokeopacity{0.800000}%
\pgfsetdash{}{0pt}%
\pgfpathmoveto{\pgfqpoint{3.689644in}{2.968820in}}%
\pgfpathlineto{\pgfqpoint{3.800755in}{2.968820in}}%
\pgfusepath{stroke}%
\end{pgfscope}%
\begin{pgfscope}%
\definecolor{textcolor}{rgb}{1.000000,1.000000,1.000000}%
\pgfsetstrokecolor{textcolor}%
\pgfsetfillcolor{textcolor}%
\pgftext[x=3.889644in,y=2.929931in,left,base]{\color{textcolor}\rmfamily\fontsize{8.000000}{9.600000}\selectfont Tg + Vehicle}%
\end{pgfscope}%
\begin{pgfscope}%
\pgfsetroundcap%
\pgfsetroundjoin%
\pgfsetlinewidth{1.756562pt}%
\definecolor{currentstroke}{rgb}{1.000000,0.694118,0.250980}%
\pgfsetstrokecolor{currentstroke}%
\pgfsetstrokeopacity{0.800000}%
\pgfsetdash{}{0pt}%
\pgfpathmoveto{\pgfqpoint{3.689644in}{2.813887in}}%
\pgfpathlineto{\pgfqpoint{3.800755in}{2.813887in}}%
\pgfusepath{stroke}%
\end{pgfscope}%
\begin{pgfscope}%
\definecolor{textcolor}{rgb}{1.000000,1.000000,1.000000}%
\pgfsetstrokecolor{textcolor}%
\pgfsetfillcolor{textcolor}%
\pgftext[x=3.889644in,y=2.774998in,left,base]{\color{textcolor}\rmfamily\fontsize{8.000000}{9.600000}\selectfont Tg + TAT-GluA2\textsubscript{3Y}}%
\end{pgfscope}%
\begin{pgfscope}%
\pgfsetroundcap%
\pgfsetroundjoin%
\pgfsetlinewidth{1.756562pt}%
\definecolor{currentstroke}{rgb}{0.200000,0.427451,0.650980}%
\pgfsetstrokecolor{currentstroke}%
\pgfsetstrokeopacity{0.800000}%
\pgfsetdash{}{0pt}%
\pgfpathmoveto{\pgfqpoint{3.689644in}{3.278686in}}%
\pgfpathlineto{\pgfqpoint{3.800755in}{3.278686in}}%
\pgfusepath{stroke}%
\end{pgfscope}%
\begin{pgfscope}%
\definecolor{textcolor}{rgb}{1.000000,1.000000,1.000000}%
\pgfsetstrokecolor{textcolor}%
\pgfsetfillcolor{textcolor}%
\pgftext[x=3.889644in,y=3.239798in,left,base]{\color{textcolor}\rmfamily\fontsize{8.000000}{9.600000}\selectfont WT + Vehicle}%
\end{pgfscope}%
\begin{pgfscope}%
\pgfsetroundcap%
\pgfsetroundjoin%
\pgfsetlinewidth{1.756562pt}%
\definecolor{currentstroke}{rgb}{0.168627,0.670588,0.494118}%
\pgfsetstrokecolor{currentstroke}%
\pgfsetstrokeopacity{0.800000}%
\pgfsetdash{}{0pt}%
\pgfpathmoveto{\pgfqpoint{3.689644in}{3.123753in}}%
\pgfpathlineto{\pgfqpoint{3.800755in}{3.123753in}}%
\pgfusepath{stroke}%
\end{pgfscope}%
\begin{pgfscope}%
\definecolor{textcolor}{rgb}{1.000000,1.000000,1.000000}%
\pgfsetstrokecolor{textcolor}%
\pgfsetfillcolor{textcolor}%
\pgftext[x=3.889644in,y=3.084864in,left,base]{\color{textcolor}\rmfamily\fontsize{8.000000}{9.600000}\selectfont WT + TAT-GluA2\textsubscript{3Y}}%
\end{pgfscope}%
\begin{pgfscope}%
\pgfsetroundcap%
\pgfsetroundjoin%
\pgfsetlinewidth{1.756562pt}%
\definecolor{currentstroke}{rgb}{1.000000,0.494118,0.250980}%
\pgfsetstrokecolor{currentstroke}%
\pgfsetstrokeopacity{0.800000}%
\pgfsetdash{}{0pt}%
\pgfpathmoveto{\pgfqpoint{3.689644in}{2.968820in}}%
\pgfpathlineto{\pgfqpoint{3.800755in}{2.968820in}}%
\pgfusepath{stroke}%
\end{pgfscope}%
\begin{pgfscope}%
\definecolor{textcolor}{rgb}{1.000000,1.000000,1.000000}%
\pgfsetstrokecolor{textcolor}%
\pgfsetfillcolor{textcolor}%
\pgftext[x=3.889644in,y=2.929931in,left,base]{\color{textcolor}\rmfamily\fontsize{8.000000}{9.600000}\selectfont Tg + Vehicle}%
\end{pgfscope}%
\begin{pgfscope}%
\pgfsetroundcap%
\pgfsetroundjoin%
\pgfsetlinewidth{1.756562pt}%
\definecolor{currentstroke}{rgb}{1.000000,0.694118,0.250980}%
\pgfsetstrokecolor{currentstroke}%
\pgfsetstrokeopacity{0.800000}%
\pgfsetdash{}{0pt}%
\pgfpathmoveto{\pgfqpoint{3.689644in}{2.813887in}}%
\pgfpathlineto{\pgfqpoint{3.800755in}{2.813887in}}%
\pgfusepath{stroke}%
\end{pgfscope}%
\begin{pgfscope}%
\definecolor{textcolor}{rgb}{1.000000,1.000000,1.000000}%
\pgfsetstrokecolor{textcolor}%
\pgfsetfillcolor{textcolor}%
\pgftext[x=3.889644in,y=2.774998in,left,base]{\color{textcolor}\rmfamily\fontsize{8.000000}{9.600000}\selectfont Tg + TAT-GluA2\textsubscript{3Y}}%
\end{pgfscope}%
\begin{pgfscope}%
\pgfsetbuttcap%
\pgfsetroundjoin%
\definecolor{currentfill}{rgb}{0.150000,0.150000,0.150000}%
\pgfsetfillcolor{currentfill}%
\pgfsetlinewidth{1.003750pt}%
\definecolor{currentstroke}{rgb}{0.150000,0.150000,0.150000}%
\pgfsetstrokecolor{currentstroke}%
\pgfsetdash{}{0pt}%
\pgfsys@defobject{currentmarker}{\pgfqpoint{0.000000in}{0.000000in}}{\pgfqpoint{0.041667in}{0.000000in}}{%
\pgfpathmoveto{\pgfqpoint{0.000000in}{0.000000in}}%
\pgfpathlineto{\pgfqpoint{0.041667in}{0.000000in}}%
\pgfusepath{stroke,fill}%
}%
\begin{pgfscope}%
\pgfsys@transformshift{3.858325in}{0.528177in}%
\pgfsys@useobject{currentmarker}{}%
\end{pgfscope}%
\end{pgfscope}%
\begin{pgfscope}%
\definecolor{textcolor}{rgb}{0.150000,0.150000,0.150000}%
\pgfsetstrokecolor{textcolor}%
\pgfsetfillcolor{textcolor}%
\pgftext[x=3.761102in,y=0.528177in,right,]{\color{textcolor}\rmfamily\fontsize{10.000000}{12.000000}\selectfont \(\displaystyle 0.0\)}%
\end{pgfscope}%
\begin{pgfscope}%
\pgfsetbuttcap%
\pgfsetroundjoin%
\definecolor{currentfill}{rgb}{0.150000,0.150000,0.150000}%
\pgfsetfillcolor{currentfill}%
\pgfsetlinewidth{1.003750pt}%
\definecolor{currentstroke}{rgb}{0.150000,0.150000,0.150000}%
\pgfsetstrokecolor{currentstroke}%
\pgfsetdash{}{0pt}%
\pgfsys@defobject{currentmarker}{\pgfqpoint{0.000000in}{0.000000in}}{\pgfqpoint{0.041667in}{0.000000in}}{%
\pgfpathmoveto{\pgfqpoint{0.000000in}{0.000000in}}%
\pgfpathlineto{\pgfqpoint{0.041667in}{0.000000in}}%
\pgfusepath{stroke,fill}%
}%
\begin{pgfscope}%
\pgfsys@transformshift{3.858325in}{0.831780in}%
\pgfsys@useobject{currentmarker}{}%
\end{pgfscope}%
\end{pgfscope}%
\begin{pgfscope}%
\definecolor{textcolor}{rgb}{0.150000,0.150000,0.150000}%
\pgfsetstrokecolor{textcolor}%
\pgfsetfillcolor{textcolor}%
\pgftext[x=3.761102in,y=0.831780in,right,]{\color{textcolor}\rmfamily\fontsize{10.000000}{12.000000}\selectfont \(\displaystyle 0.5\)}%
\end{pgfscope}%
\begin{pgfscope}%
\pgfsetbuttcap%
\pgfsetroundjoin%
\definecolor{currentfill}{rgb}{0.150000,0.150000,0.150000}%
\pgfsetfillcolor{currentfill}%
\pgfsetlinewidth{1.003750pt}%
\definecolor{currentstroke}{rgb}{0.150000,0.150000,0.150000}%
\pgfsetstrokecolor{currentstroke}%
\pgfsetdash{}{0pt}%
\pgfsys@defobject{currentmarker}{\pgfqpoint{0.000000in}{0.000000in}}{\pgfqpoint{0.041667in}{0.000000in}}{%
\pgfpathmoveto{\pgfqpoint{0.000000in}{0.000000in}}%
\pgfpathlineto{\pgfqpoint{0.041667in}{0.000000in}}%
\pgfusepath{stroke,fill}%
}%
\begin{pgfscope}%
\pgfsys@transformshift{3.858325in}{1.135383in}%
\pgfsys@useobject{currentmarker}{}%
\end{pgfscope}%
\end{pgfscope}%
\begin{pgfscope}%
\definecolor{textcolor}{rgb}{0.150000,0.150000,0.150000}%
\pgfsetstrokecolor{textcolor}%
\pgfsetfillcolor{textcolor}%
\pgftext[x=3.761102in,y=1.135383in,right,]{\color{textcolor}\rmfamily\fontsize{10.000000}{12.000000}\selectfont \(\displaystyle 1.0\)}%
\end{pgfscope}%
\begin{pgfscope}%
\pgfsetbuttcap%
\pgfsetroundjoin%
\definecolor{currentfill}{rgb}{0.150000,0.150000,0.150000}%
\pgfsetfillcolor{currentfill}%
\pgfsetlinewidth{1.003750pt}%
\definecolor{currentstroke}{rgb}{0.150000,0.150000,0.150000}%
\pgfsetstrokecolor{currentstroke}%
\pgfsetdash{}{0pt}%
\pgfsys@defobject{currentmarker}{\pgfqpoint{0.000000in}{0.000000in}}{\pgfqpoint{0.041667in}{0.000000in}}{%
\pgfpathmoveto{\pgfqpoint{0.000000in}{0.000000in}}%
\pgfpathlineto{\pgfqpoint{0.041667in}{0.000000in}}%
\pgfusepath{stroke,fill}%
}%
\begin{pgfscope}%
\pgfsys@transformshift{3.858325in}{1.438986in}%
\pgfsys@useobject{currentmarker}{}%
\end{pgfscope}%
\end{pgfscope}%
\begin{pgfscope}%
\definecolor{textcolor}{rgb}{0.150000,0.150000,0.150000}%
\pgfsetstrokecolor{textcolor}%
\pgfsetfillcolor{textcolor}%
\pgftext[x=3.761102in,y=1.438986in,right,]{\color{textcolor}\rmfamily\fontsize{10.000000}{12.000000}\selectfont \(\displaystyle 1.5\)}%
\end{pgfscope}%
\begin{pgfscope}%
\pgfsetbuttcap%
\pgfsetroundjoin%
\definecolor{currentfill}{rgb}{0.150000,0.150000,0.150000}%
\pgfsetfillcolor{currentfill}%
\pgfsetlinewidth{1.003750pt}%
\definecolor{currentstroke}{rgb}{0.150000,0.150000,0.150000}%
\pgfsetstrokecolor{currentstroke}%
\pgfsetdash{}{0pt}%
\pgfsys@defobject{currentmarker}{\pgfqpoint{0.000000in}{0.000000in}}{\pgfqpoint{0.041667in}{0.000000in}}{%
\pgfpathmoveto{\pgfqpoint{0.000000in}{0.000000in}}%
\pgfpathlineto{\pgfqpoint{0.041667in}{0.000000in}}%
\pgfusepath{stroke,fill}%
}%
\begin{pgfscope}%
\pgfsys@transformshift{3.858325in}{1.742589in}%
\pgfsys@useobject{currentmarker}{}%
\end{pgfscope}%
\end{pgfscope}%
\begin{pgfscope}%
\definecolor{textcolor}{rgb}{0.150000,0.150000,0.150000}%
\pgfsetstrokecolor{textcolor}%
\pgfsetfillcolor{textcolor}%
\pgftext[x=3.761102in,y=1.742589in,right,]{\color{textcolor}\rmfamily\fontsize{10.000000}{12.000000}\selectfont \(\displaystyle 2.0\)}%
\end{pgfscope}%
\begin{pgfscope}%
\pgfsetbuttcap%
\pgfsetroundjoin%
\definecolor{currentfill}{rgb}{0.150000,0.150000,0.150000}%
\pgfsetfillcolor{currentfill}%
\pgfsetlinewidth{1.003750pt}%
\definecolor{currentstroke}{rgb}{0.150000,0.150000,0.150000}%
\pgfsetstrokecolor{currentstroke}%
\pgfsetdash{}{0pt}%
\pgfsys@defobject{currentmarker}{\pgfqpoint{0.000000in}{0.000000in}}{\pgfqpoint{0.041667in}{0.000000in}}{%
\pgfpathmoveto{\pgfqpoint{0.000000in}{0.000000in}}%
\pgfpathlineto{\pgfqpoint{0.041667in}{0.000000in}}%
\pgfusepath{stroke,fill}%
}%
\begin{pgfscope}%
\pgfsys@transformshift{3.858325in}{2.046192in}%
\pgfsys@useobject{currentmarker}{}%
\end{pgfscope}%
\end{pgfscope}%
\begin{pgfscope}%
\definecolor{textcolor}{rgb}{0.150000,0.150000,0.150000}%
\pgfsetstrokecolor{textcolor}%
\pgfsetfillcolor{textcolor}%
\pgftext[x=3.761102in,y=2.046192in,right,]{\color{textcolor}\rmfamily\fontsize{10.000000}{12.000000}\selectfont \(\displaystyle 2.5\)}%
\end{pgfscope}%
\begin{pgfscope}%
\pgfsetbuttcap%
\pgfsetroundjoin%
\definecolor{currentfill}{rgb}{0.150000,0.150000,0.150000}%
\pgfsetfillcolor{currentfill}%
\pgfsetlinewidth{1.003750pt}%
\definecolor{currentstroke}{rgb}{0.150000,0.150000,0.150000}%
\pgfsetstrokecolor{currentstroke}%
\pgfsetdash{}{0pt}%
\pgfsys@defobject{currentmarker}{\pgfqpoint{0.000000in}{0.000000in}}{\pgfqpoint{0.041667in}{0.000000in}}{%
\pgfpathmoveto{\pgfqpoint{0.000000in}{0.000000in}}%
\pgfpathlineto{\pgfqpoint{0.041667in}{0.000000in}}%
\pgfusepath{stroke,fill}%
}%
\begin{pgfscope}%
\pgfsys@transformshift{3.858325in}{2.349796in}%
\pgfsys@useobject{currentmarker}{}%
\end{pgfscope}%
\end{pgfscope}%
\begin{pgfscope}%
\definecolor{textcolor}{rgb}{0.150000,0.150000,0.150000}%
\pgfsetstrokecolor{textcolor}%
\pgfsetfillcolor{textcolor}%
\pgftext[x=3.761102in,y=2.349796in,right,]{\color{textcolor}\rmfamily\fontsize{10.000000}{12.000000}\selectfont \(\displaystyle 3.0\)}%
\end{pgfscope}%
\begin{pgfscope}%
\pgfsetbuttcap%
\pgfsetroundjoin%
\definecolor{currentfill}{rgb}{0.150000,0.150000,0.150000}%
\pgfsetfillcolor{currentfill}%
\pgfsetlinewidth{1.003750pt}%
\definecolor{currentstroke}{rgb}{0.150000,0.150000,0.150000}%
\pgfsetstrokecolor{currentstroke}%
\pgfsetdash{}{0pt}%
\pgfsys@defobject{currentmarker}{\pgfqpoint{0.000000in}{0.000000in}}{\pgfqpoint{0.041667in}{0.000000in}}{%
\pgfpathmoveto{\pgfqpoint{0.000000in}{0.000000in}}%
\pgfpathlineto{\pgfqpoint{0.041667in}{0.000000in}}%
\pgfusepath{stroke,fill}%
}%
\begin{pgfscope}%
\pgfsys@transformshift{3.858325in}{2.653399in}%
\pgfsys@useobject{currentmarker}{}%
\end{pgfscope}%
\end{pgfscope}%
\begin{pgfscope}%
\definecolor{textcolor}{rgb}{0.150000,0.150000,0.150000}%
\pgfsetstrokecolor{textcolor}%
\pgfsetfillcolor{textcolor}%
\pgftext[x=3.761102in,y=2.653399in,right,]{\color{textcolor}\rmfamily\fontsize{10.000000}{12.000000}\selectfont \(\displaystyle 3.5\)}%
\end{pgfscope}%
\begin{pgfscope}%
\definecolor{textcolor}{rgb}{0.150000,0.150000,0.150000}%
\pgfsetstrokecolor{textcolor}%
\pgfsetfillcolor{textcolor}%
\pgftext[x=3.514188in,y=1.590788in,,bottom,rotate=90.000000]{\color{textcolor}\rmfamily\fontsize{10.000000}{12.000000}\selectfont \textbf{Time to freezing (s)}}%
\end{pgfscope}%
\begin{pgfscope}%
\pgfpathrectangle{\pgfqpoint{3.858325in}{0.528177in}}{\pgfqpoint{1.343404in}{2.125222in}} %
\pgfusepath{clip}%
\pgfsetbuttcap%
\pgfsetmiterjoin%
\definecolor{currentfill}{rgb}{0.200000,0.427451,0.650980}%
\pgfsetfillcolor{currentfill}%
\pgfsetlinewidth{1.505625pt}%
\definecolor{currentstroke}{rgb}{0.200000,0.427451,0.650980}%
\pgfsetstrokecolor{currentstroke}%
\pgfsetdash{}{0pt}%
\pgfpathmoveto{\pgfqpoint{3.906303in}{0.528177in}}%
\pgfpathlineto{\pgfqpoint{4.146197in}{0.528177in}}%
\pgfpathlineto{\pgfqpoint{4.146197in}{2.582818in}}%
\pgfpathlineto{\pgfqpoint{3.906303in}{2.582818in}}%
\pgfpathclose%
\pgfusepath{stroke,fill}%
\end{pgfscope}%
\begin{pgfscope}%
\pgfpathrectangle{\pgfqpoint{3.858325in}{0.528177in}}{\pgfqpoint{1.343404in}{2.125222in}} %
\pgfusepath{clip}%
\pgfsetbuttcap%
\pgfsetmiterjoin%
\definecolor{currentfill}{rgb}{0.168627,0.670588,0.494118}%
\pgfsetfillcolor{currentfill}%
\pgfsetlinewidth{1.505625pt}%
\definecolor{currentstroke}{rgb}{0.168627,0.670588,0.494118}%
\pgfsetstrokecolor{currentstroke}%
\pgfsetdash{}{0pt}%
\pgfpathmoveto{\pgfqpoint{4.242154in}{0.528177in}}%
\pgfpathlineto{\pgfqpoint{4.482048in}{0.528177in}}%
\pgfpathlineto{\pgfqpoint{4.482048in}{2.227674in}}%
\pgfpathlineto{\pgfqpoint{4.242154in}{2.227674in}}%
\pgfpathclose%
\pgfusepath{stroke,fill}%
\end{pgfscope}%
\begin{pgfscope}%
\pgfpathrectangle{\pgfqpoint{3.858325in}{0.528177in}}{\pgfqpoint{1.343404in}{2.125222in}} %
\pgfusepath{clip}%
\pgfsetbuttcap%
\pgfsetmiterjoin%
\definecolor{currentfill}{rgb}{1.000000,0.494118,0.250980}%
\pgfsetfillcolor{currentfill}%
\pgfsetlinewidth{1.505625pt}%
\definecolor{currentstroke}{rgb}{1.000000,0.494118,0.250980}%
\pgfsetstrokecolor{currentstroke}%
\pgfsetdash{}{0pt}%
\pgfpathmoveto{\pgfqpoint{4.578005in}{0.528177in}}%
\pgfpathlineto{\pgfqpoint{4.817899in}{0.528177in}}%
\pgfpathlineto{\pgfqpoint{4.817899in}{2.282166in}}%
\pgfpathlineto{\pgfqpoint{4.578005in}{2.282166in}}%
\pgfpathclose%
\pgfusepath{stroke,fill}%
\end{pgfscope}%
\begin{pgfscope}%
\pgfpathrectangle{\pgfqpoint{3.858325in}{0.528177in}}{\pgfqpoint{1.343404in}{2.125222in}} %
\pgfusepath{clip}%
\pgfsetbuttcap%
\pgfsetmiterjoin%
\definecolor{currentfill}{rgb}{1.000000,0.694118,0.250980}%
\pgfsetfillcolor{currentfill}%
\pgfsetlinewidth{1.505625pt}%
\definecolor{currentstroke}{rgb}{1.000000,0.694118,0.250980}%
\pgfsetstrokecolor{currentstroke}%
\pgfsetdash{}{0pt}%
\pgfpathmoveto{\pgfqpoint{4.913856in}{0.528177in}}%
\pgfpathlineto{\pgfqpoint{5.153750in}{0.528177in}}%
\pgfpathlineto{\pgfqpoint{5.153750in}{2.125053in}}%
\pgfpathlineto{\pgfqpoint{4.913856in}{2.125053in}}%
\pgfpathclose%
\pgfusepath{stroke,fill}%
\end{pgfscope}%
\begin{pgfscope}%
\pgfpathrectangle{\pgfqpoint{3.858325in}{0.528177in}}{\pgfqpoint{1.343404in}{2.125222in}} %
\pgfusepath{clip}%
\pgfsetbuttcap%
\pgfsetroundjoin%
\pgfsetlinewidth{1.505625pt}%
\definecolor{currentstroke}{rgb}{0.200000,0.427451,0.650980}%
\pgfsetstrokecolor{currentstroke}%
\pgfsetdash{}{0pt}%
\pgfpathmoveto{\pgfqpoint{4.026250in}{2.653399in}}%
\pgfpathlineto{\pgfqpoint{4.026250in}{2.501597in}}%
\pgfusepath{stroke}%
\end{pgfscope}%
\begin{pgfscope}%
\pgfpathrectangle{\pgfqpoint{3.858325in}{0.528177in}}{\pgfqpoint{1.343404in}{2.125222in}} %
\pgfusepath{clip}%
\pgfsetbuttcap%
\pgfsetroundjoin%
\pgfsetlinewidth{1.505625pt}%
\definecolor{currentstroke}{rgb}{0.168627,0.670588,0.494118}%
\pgfsetstrokecolor{currentstroke}%
\pgfsetdash{}{0pt}%
\pgfpathmoveto{\pgfqpoint{4.362101in}{2.289075in}}%
\pgfpathlineto{\pgfqpoint{4.362101in}{2.167634in}}%
\pgfusepath{stroke}%
\end{pgfscope}%
\begin{pgfscope}%
\pgfpathrectangle{\pgfqpoint{3.858325in}{0.528177in}}{\pgfqpoint{1.343404in}{2.125222in}} %
\pgfusepath{clip}%
\pgfsetbuttcap%
\pgfsetroundjoin%
\pgfsetlinewidth{1.505625pt}%
\definecolor{currentstroke}{rgb}{1.000000,0.494118,0.250980}%
\pgfsetstrokecolor{currentstroke}%
\pgfsetdash{}{0pt}%
\pgfpathmoveto{\pgfqpoint{4.697952in}{2.471237in}}%
\pgfpathlineto{\pgfqpoint{4.697952in}{2.137273in}}%
\pgfusepath{stroke}%
\end{pgfscope}%
\begin{pgfscope}%
\pgfpathrectangle{\pgfqpoint{3.858325in}{0.528177in}}{\pgfqpoint{1.343404in}{2.125222in}} %
\pgfusepath{clip}%
\pgfsetbuttcap%
\pgfsetroundjoin%
\pgfsetlinewidth{1.505625pt}%
\definecolor{currentstroke}{rgb}{1.000000,0.694118,0.250980}%
\pgfsetstrokecolor{currentstroke}%
\pgfsetdash{}{0pt}%
\pgfpathmoveto{\pgfqpoint{5.033803in}{2.197994in}}%
\pgfpathlineto{\pgfqpoint{5.033803in}{2.046192in}}%
\pgfusepath{stroke}%
\end{pgfscope}%
\begin{pgfscope}%
\pgfpathrectangle{\pgfqpoint{3.858325in}{0.528177in}}{\pgfqpoint{1.343404in}{2.125222in}} %
\pgfusepath{clip}%
\pgfsetbuttcap%
\pgfsetroundjoin%
\definecolor{currentfill}{rgb}{0.200000,0.427451,0.650980}%
\pgfsetfillcolor{currentfill}%
\pgfsetlinewidth{1.505625pt}%
\definecolor{currentstroke}{rgb}{0.200000,0.427451,0.650980}%
\pgfsetstrokecolor{currentstroke}%
\pgfsetdash{}{0pt}%
\pgfsys@defobject{currentmarker}{\pgfqpoint{-0.111111in}{-0.000000in}}{\pgfqpoint{0.111111in}{0.000000in}}{%
\pgfpathmoveto{\pgfqpoint{0.111111in}{-0.000000in}}%
\pgfpathlineto{\pgfqpoint{-0.111111in}{0.000000in}}%
\pgfusepath{stroke,fill}%
}%
\begin{pgfscope}%
\pgfsys@transformshift{4.026250in}{2.653399in}%
\pgfsys@useobject{currentmarker}{}%
\end{pgfscope}%
\end{pgfscope}%
\begin{pgfscope}%
\pgfpathrectangle{\pgfqpoint{3.858325in}{0.528177in}}{\pgfqpoint{1.343404in}{2.125222in}} %
\pgfusepath{clip}%
\pgfsetbuttcap%
\pgfsetroundjoin%
\definecolor{currentfill}{rgb}{0.200000,0.427451,0.650980}%
\pgfsetfillcolor{currentfill}%
\pgfsetlinewidth{1.505625pt}%
\definecolor{currentstroke}{rgb}{0.200000,0.427451,0.650980}%
\pgfsetstrokecolor{currentstroke}%
\pgfsetdash{}{0pt}%
\pgfsys@defobject{currentmarker}{\pgfqpoint{-0.111111in}{-0.000000in}}{\pgfqpoint{0.111111in}{0.000000in}}{%
\pgfpathmoveto{\pgfqpoint{0.111111in}{-0.000000in}}%
\pgfpathlineto{\pgfqpoint{-0.111111in}{0.000000in}}%
\pgfusepath{stroke,fill}%
}%
\begin{pgfscope}%
\pgfsys@transformshift{4.026250in}{2.501597in}%
\pgfsys@useobject{currentmarker}{}%
\end{pgfscope}%
\end{pgfscope}%
\begin{pgfscope}%
\pgfpathrectangle{\pgfqpoint{3.858325in}{0.528177in}}{\pgfqpoint{1.343404in}{2.125222in}} %
\pgfusepath{clip}%
\pgfsetbuttcap%
\pgfsetroundjoin%
\definecolor{currentfill}{rgb}{0.168627,0.670588,0.494118}%
\pgfsetfillcolor{currentfill}%
\pgfsetlinewidth{1.505625pt}%
\definecolor{currentstroke}{rgb}{0.168627,0.670588,0.494118}%
\pgfsetstrokecolor{currentstroke}%
\pgfsetdash{}{0pt}%
\pgfsys@defobject{currentmarker}{\pgfqpoint{-0.111111in}{-0.000000in}}{\pgfqpoint{0.111111in}{0.000000in}}{%
\pgfpathmoveto{\pgfqpoint{0.111111in}{-0.000000in}}%
\pgfpathlineto{\pgfqpoint{-0.111111in}{0.000000in}}%
\pgfusepath{stroke,fill}%
}%
\begin{pgfscope}%
\pgfsys@transformshift{4.362101in}{2.289075in}%
\pgfsys@useobject{currentmarker}{}%
\end{pgfscope}%
\end{pgfscope}%
\begin{pgfscope}%
\pgfpathrectangle{\pgfqpoint{3.858325in}{0.528177in}}{\pgfqpoint{1.343404in}{2.125222in}} %
\pgfusepath{clip}%
\pgfsetbuttcap%
\pgfsetroundjoin%
\definecolor{currentfill}{rgb}{0.168627,0.670588,0.494118}%
\pgfsetfillcolor{currentfill}%
\pgfsetlinewidth{1.505625pt}%
\definecolor{currentstroke}{rgb}{0.168627,0.670588,0.494118}%
\pgfsetstrokecolor{currentstroke}%
\pgfsetdash{}{0pt}%
\pgfsys@defobject{currentmarker}{\pgfqpoint{-0.111111in}{-0.000000in}}{\pgfqpoint{0.111111in}{0.000000in}}{%
\pgfpathmoveto{\pgfqpoint{0.111111in}{-0.000000in}}%
\pgfpathlineto{\pgfqpoint{-0.111111in}{0.000000in}}%
\pgfusepath{stroke,fill}%
}%
\begin{pgfscope}%
\pgfsys@transformshift{4.362101in}{2.167634in}%
\pgfsys@useobject{currentmarker}{}%
\end{pgfscope}%
\end{pgfscope}%
\begin{pgfscope}%
\pgfpathrectangle{\pgfqpoint{3.858325in}{0.528177in}}{\pgfqpoint{1.343404in}{2.125222in}} %
\pgfusepath{clip}%
\pgfsetbuttcap%
\pgfsetroundjoin%
\definecolor{currentfill}{rgb}{1.000000,0.494118,0.250980}%
\pgfsetfillcolor{currentfill}%
\pgfsetlinewidth{1.505625pt}%
\definecolor{currentstroke}{rgb}{1.000000,0.494118,0.250980}%
\pgfsetstrokecolor{currentstroke}%
\pgfsetdash{}{0pt}%
\pgfsys@defobject{currentmarker}{\pgfqpoint{-0.111111in}{-0.000000in}}{\pgfqpoint{0.111111in}{0.000000in}}{%
\pgfpathmoveto{\pgfqpoint{0.111111in}{-0.000000in}}%
\pgfpathlineto{\pgfqpoint{-0.111111in}{0.000000in}}%
\pgfusepath{stroke,fill}%
}%
\begin{pgfscope}%
\pgfsys@transformshift{4.697952in}{2.471237in}%
\pgfsys@useobject{currentmarker}{}%
\end{pgfscope}%
\end{pgfscope}%
\begin{pgfscope}%
\pgfpathrectangle{\pgfqpoint{3.858325in}{0.528177in}}{\pgfqpoint{1.343404in}{2.125222in}} %
\pgfusepath{clip}%
\pgfsetbuttcap%
\pgfsetroundjoin%
\definecolor{currentfill}{rgb}{1.000000,0.494118,0.250980}%
\pgfsetfillcolor{currentfill}%
\pgfsetlinewidth{1.505625pt}%
\definecolor{currentstroke}{rgb}{1.000000,0.494118,0.250980}%
\pgfsetstrokecolor{currentstroke}%
\pgfsetdash{}{0pt}%
\pgfsys@defobject{currentmarker}{\pgfqpoint{-0.111111in}{-0.000000in}}{\pgfqpoint{0.111111in}{0.000000in}}{%
\pgfpathmoveto{\pgfqpoint{0.111111in}{-0.000000in}}%
\pgfpathlineto{\pgfqpoint{-0.111111in}{0.000000in}}%
\pgfusepath{stroke,fill}%
}%
\begin{pgfscope}%
\pgfsys@transformshift{4.697952in}{2.137273in}%
\pgfsys@useobject{currentmarker}{}%
\end{pgfscope}%
\end{pgfscope}%
\begin{pgfscope}%
\pgfpathrectangle{\pgfqpoint{3.858325in}{0.528177in}}{\pgfqpoint{1.343404in}{2.125222in}} %
\pgfusepath{clip}%
\pgfsetbuttcap%
\pgfsetroundjoin%
\definecolor{currentfill}{rgb}{1.000000,0.694118,0.250980}%
\pgfsetfillcolor{currentfill}%
\pgfsetlinewidth{1.505625pt}%
\definecolor{currentstroke}{rgb}{1.000000,0.694118,0.250980}%
\pgfsetstrokecolor{currentstroke}%
\pgfsetdash{}{0pt}%
\pgfsys@defobject{currentmarker}{\pgfqpoint{-0.111111in}{-0.000000in}}{\pgfqpoint{0.111111in}{0.000000in}}{%
\pgfpathmoveto{\pgfqpoint{0.111111in}{-0.000000in}}%
\pgfpathlineto{\pgfqpoint{-0.111111in}{0.000000in}}%
\pgfusepath{stroke,fill}%
}%
\begin{pgfscope}%
\pgfsys@transformshift{5.033803in}{2.197994in}%
\pgfsys@useobject{currentmarker}{}%
\end{pgfscope}%
\end{pgfscope}%
\begin{pgfscope}%
\pgfpathrectangle{\pgfqpoint{3.858325in}{0.528177in}}{\pgfqpoint{1.343404in}{2.125222in}} %
\pgfusepath{clip}%
\pgfsetbuttcap%
\pgfsetroundjoin%
\definecolor{currentfill}{rgb}{1.000000,0.694118,0.250980}%
\pgfsetfillcolor{currentfill}%
\pgfsetlinewidth{1.505625pt}%
\definecolor{currentstroke}{rgb}{1.000000,0.694118,0.250980}%
\pgfsetstrokecolor{currentstroke}%
\pgfsetdash{}{0pt}%
\pgfsys@defobject{currentmarker}{\pgfqpoint{-0.111111in}{-0.000000in}}{\pgfqpoint{0.111111in}{0.000000in}}{%
\pgfpathmoveto{\pgfqpoint{0.111111in}{-0.000000in}}%
\pgfpathlineto{\pgfqpoint{-0.111111in}{0.000000in}}%
\pgfusepath{stroke,fill}%
}%
\begin{pgfscope}%
\pgfsys@transformshift{5.033803in}{2.046192in}%
\pgfsys@useobject{currentmarker}{}%
\end{pgfscope}%
\end{pgfscope}%
\begin{pgfscope}%
\pgfsetrectcap%
\pgfsetmiterjoin%
\pgfsetlinewidth{1.254687pt}%
\definecolor{currentstroke}{rgb}{0.150000,0.150000,0.150000}%
\pgfsetstrokecolor{currentstroke}%
\pgfsetdash{}{0pt}%
\pgfpathmoveto{\pgfqpoint{3.858325in}{0.528177in}}%
\pgfpathlineto{\pgfqpoint{3.858325in}{2.653399in}}%
\pgfusepath{stroke}%
\end{pgfscope}%
\begin{pgfscope}%
\pgfsetrectcap%
\pgfsetmiterjoin%
\pgfsetlinewidth{1.254687pt}%
\definecolor{currentstroke}{rgb}{0.150000,0.150000,0.150000}%
\pgfsetstrokecolor{currentstroke}%
\pgfsetdash{}{0pt}%
\pgfpathmoveto{\pgfqpoint{3.858325in}{0.528177in}}%
\pgfpathlineto{\pgfqpoint{5.201729in}{0.528177in}}%
\pgfusepath{stroke}%
\end{pgfscope}%
\begin{pgfscope}%
\pgfsetbuttcap%
\pgfsetmiterjoin%
\definecolor{currentfill}{rgb}{0.200000,0.427451,0.650980}%
\pgfsetfillcolor{currentfill}%
\pgfsetlinewidth{1.505625pt}%
\definecolor{currentstroke}{rgb}{0.200000,0.427451,0.650980}%
\pgfsetstrokecolor{currentstroke}%
\pgfsetdash{}{0pt}%
\pgfpathmoveto{\pgfqpoint{3.891154in}{3.239798in}}%
\pgfpathlineto{\pgfqpoint{4.002266in}{3.239798in}}%
\pgfpathlineto{\pgfqpoint{4.002266in}{3.317575in}}%
\pgfpathlineto{\pgfqpoint{3.891154in}{3.317575in}}%
\pgfpathclose%
\pgfusepath{stroke,fill}%
\end{pgfscope}%
\begin{pgfscope}%
\definecolor{textcolor}{rgb}{0.150000,0.150000,0.150000}%
\pgfsetstrokecolor{textcolor}%
\pgfsetfillcolor{textcolor}%
\pgftext[x=4.091154in,y=3.239798in,left,base]{\color{textcolor}\rmfamily\fontsize{8.000000}{9.600000}\selectfont WT + Vehicle}%
\end{pgfscope}%
\begin{pgfscope}%
\pgfsetbuttcap%
\pgfsetmiterjoin%
\definecolor{currentfill}{rgb}{0.168627,0.670588,0.494118}%
\pgfsetfillcolor{currentfill}%
\pgfsetlinewidth{1.505625pt}%
\definecolor{currentstroke}{rgb}{0.168627,0.670588,0.494118}%
\pgfsetstrokecolor{currentstroke}%
\pgfsetdash{}{0pt}%
\pgfpathmoveto{\pgfqpoint{3.891154in}{3.084864in}}%
\pgfpathlineto{\pgfqpoint{4.002266in}{3.084864in}}%
\pgfpathlineto{\pgfqpoint{4.002266in}{3.162642in}}%
\pgfpathlineto{\pgfqpoint{3.891154in}{3.162642in}}%
\pgfpathclose%
\pgfusepath{stroke,fill}%
\end{pgfscope}%
\begin{pgfscope}%
\definecolor{textcolor}{rgb}{0.150000,0.150000,0.150000}%
\pgfsetstrokecolor{textcolor}%
\pgfsetfillcolor{textcolor}%
\pgftext[x=4.091154in,y=3.084864in,left,base]{\color{textcolor}\rmfamily\fontsize{8.000000}{9.600000}\selectfont WT + TAT-GluA2\textsubscript{3Y}}%
\end{pgfscope}%
\begin{pgfscope}%
\pgfsetbuttcap%
\pgfsetmiterjoin%
\definecolor{currentfill}{rgb}{1.000000,0.494118,0.250980}%
\pgfsetfillcolor{currentfill}%
\pgfsetlinewidth{1.505625pt}%
\definecolor{currentstroke}{rgb}{1.000000,0.494118,0.250980}%
\pgfsetstrokecolor{currentstroke}%
\pgfsetdash{}{0pt}%
\pgfpathmoveto{\pgfqpoint{3.891154in}{2.929931in}}%
\pgfpathlineto{\pgfqpoint{4.002266in}{2.929931in}}%
\pgfpathlineto{\pgfqpoint{4.002266in}{3.007709in}}%
\pgfpathlineto{\pgfqpoint{3.891154in}{3.007709in}}%
\pgfpathclose%
\pgfusepath{stroke,fill}%
\end{pgfscope}%
\begin{pgfscope}%
\definecolor{textcolor}{rgb}{0.150000,0.150000,0.150000}%
\pgfsetstrokecolor{textcolor}%
\pgfsetfillcolor{textcolor}%
\pgftext[x=4.091154in,y=2.929931in,left,base]{\color{textcolor}\rmfamily\fontsize{8.000000}{9.600000}\selectfont Tg + Vehicle}%
\end{pgfscope}%
\begin{pgfscope}%
\pgfsetbuttcap%
\pgfsetmiterjoin%
\definecolor{currentfill}{rgb}{1.000000,0.694118,0.250980}%
\pgfsetfillcolor{currentfill}%
\pgfsetlinewidth{1.505625pt}%
\definecolor{currentstroke}{rgb}{1.000000,0.694118,0.250980}%
\pgfsetstrokecolor{currentstroke}%
\pgfsetdash{}{0pt}%
\pgfpathmoveto{\pgfqpoint{3.891154in}{2.774998in}}%
\pgfpathlineto{\pgfqpoint{4.002266in}{2.774998in}}%
\pgfpathlineto{\pgfqpoint{4.002266in}{2.852776in}}%
\pgfpathlineto{\pgfqpoint{3.891154in}{2.852776in}}%
\pgfpathclose%
\pgfusepath{stroke,fill}%
\end{pgfscope}%
\begin{pgfscope}%
\definecolor{textcolor}{rgb}{0.150000,0.150000,0.150000}%
\pgfsetstrokecolor{textcolor}%
\pgfsetfillcolor{textcolor}%
\pgftext[x=4.091154in,y=2.774998in,left,base]{\color{textcolor}\rmfamily\fontsize{8.000000}{9.600000}\selectfont Tg + TAT-GluA2\textsubscript{3Y}}%
\end{pgfscope}%
\end{pgfpicture}%
\makeatother%
\endgroup%

        \caption{\label{f.ad.nb_into_f}}
    \end{subfigure}
    \begin{subfigure}[h]{\textwidth}
        %% Creator: Matplotlib, PGF backend
%%
%% To include the figure in your LaTeX document, write
%%   \input{<filename>.pgf}
%%
%% Make sure the required packages are loaded in your preamble
%%   \usepackage{pgf}
%%
%% Figures using additional raster images can only be included by \input if
%% they are in the same directory as the main LaTeX file. For loading figures
%% from other directories you can use the `import` package
%%   \usepackage{import}
%% and then include the figures with
%%   \import{<path to file>}{<filename>.pgf}
%%
%% Matplotlib used the following preamble
%%   \usepackage[utf8]{inputenc}
%%   \usepackage[T1]{fontenc}
%%   \usepackage{siunitx}
%%
\begingroup%
\makeatletter%
\begin{pgfpicture}%
\pgfpathrectangle{\pgfpointorigin}{\pgfqpoint{5.301729in}{3.553934in}}%
\pgfusepath{use as bounding box, clip}%
\begin{pgfscope}%
\pgfsetbuttcap%
\pgfsetmiterjoin%
\definecolor{currentfill}{rgb}{1.000000,1.000000,1.000000}%
\pgfsetfillcolor{currentfill}%
\pgfsetlinewidth{0.000000pt}%
\definecolor{currentstroke}{rgb}{1.000000,1.000000,1.000000}%
\pgfsetstrokecolor{currentstroke}%
\pgfsetdash{}{0pt}%
\pgfpathmoveto{\pgfqpoint{0.000000in}{0.000000in}}%
\pgfpathlineto{\pgfqpoint{5.301729in}{0.000000in}}%
\pgfpathlineto{\pgfqpoint{5.301729in}{3.553934in}}%
\pgfpathlineto{\pgfqpoint{0.000000in}{3.553934in}}%
\pgfpathclose%
\pgfusepath{fill}%
\end{pgfscope}%
\begin{pgfscope}%
\pgfsetbuttcap%
\pgfsetmiterjoin%
\definecolor{currentfill}{rgb}{1.000000,1.000000,1.000000}%
\pgfsetfillcolor{currentfill}%
\pgfsetlinewidth{0.000000pt}%
\definecolor{currentstroke}{rgb}{0.000000,0.000000,0.000000}%
\pgfsetstrokecolor{currentstroke}%
\pgfsetstrokeopacity{0.000000}%
\pgfsetdash{}{0pt}%
\pgfpathmoveto{\pgfqpoint{0.566985in}{0.528177in}}%
\pgfpathlineto{\pgfqpoint{3.253793in}{0.528177in}}%
\pgfpathlineto{\pgfqpoint{3.253793in}{3.392606in}}%
\pgfpathlineto{\pgfqpoint{0.566985in}{3.392606in}}%
\pgfpathclose%
\pgfusepath{fill}%
\end{pgfscope}%
\begin{pgfscope}%
\pgfsetbuttcap%
\pgfsetroundjoin%
\definecolor{currentfill}{rgb}{0.150000,0.150000,0.150000}%
\pgfsetfillcolor{currentfill}%
\pgfsetlinewidth{1.003750pt}%
\definecolor{currentstroke}{rgb}{0.150000,0.150000,0.150000}%
\pgfsetstrokecolor{currentstroke}%
\pgfsetdash{}{0pt}%
\pgfsys@defobject{currentmarker}{\pgfqpoint{0.000000in}{0.000000in}}{\pgfqpoint{0.000000in}{0.041667in}}{%
\pgfpathmoveto{\pgfqpoint{0.000000in}{0.000000in}}%
\pgfpathlineto{\pgfqpoint{0.000000in}{0.041667in}}%
\pgfusepath{stroke,fill}%
}%
\begin{pgfscope}%
\pgfsys@transformshift{0.566985in}{0.528177in}%
\pgfsys@useobject{currentmarker}{}%
\end{pgfscope}%
\end{pgfscope}%
\begin{pgfscope}%
\definecolor{textcolor}{rgb}{0.150000,0.150000,0.150000}%
\pgfsetstrokecolor{textcolor}%
\pgfsetfillcolor{textcolor}%
\pgftext[x=0.566985in,y=0.430955in,,top]{\color{textcolor}\rmfamily\fontsize{10.000000}{12.000000}\selectfont \(\displaystyle -10\)}%
\end{pgfscope}%
\begin{pgfscope}%
\pgfsetbuttcap%
\pgfsetroundjoin%
\definecolor{currentfill}{rgb}{0.150000,0.150000,0.150000}%
\pgfsetfillcolor{currentfill}%
\pgfsetlinewidth{1.003750pt}%
\definecolor{currentstroke}{rgb}{0.150000,0.150000,0.150000}%
\pgfsetstrokecolor{currentstroke}%
\pgfsetdash{}{0pt}%
\pgfsys@defobject{currentmarker}{\pgfqpoint{0.000000in}{0.000000in}}{\pgfqpoint{0.000000in}{0.041667in}}{%
\pgfpathmoveto{\pgfqpoint{0.000000in}{0.000000in}}%
\pgfpathlineto{\pgfqpoint{0.000000in}{0.041667in}}%
\pgfusepath{stroke,fill}%
}%
\begin{pgfscope}%
\pgfsys@transformshift{1.055495in}{0.528177in}%
\pgfsys@useobject{currentmarker}{}%
\end{pgfscope}%
\end{pgfscope}%
\begin{pgfscope}%
\definecolor{textcolor}{rgb}{0.150000,0.150000,0.150000}%
\pgfsetstrokecolor{textcolor}%
\pgfsetfillcolor{textcolor}%
\pgftext[x=1.055495in,y=0.430955in,,top]{\color{textcolor}\rmfamily\fontsize{10.000000}{12.000000}\selectfont \(\displaystyle -8\)}%
\end{pgfscope}%
\begin{pgfscope}%
\pgfsetbuttcap%
\pgfsetroundjoin%
\definecolor{currentfill}{rgb}{0.150000,0.150000,0.150000}%
\pgfsetfillcolor{currentfill}%
\pgfsetlinewidth{1.003750pt}%
\definecolor{currentstroke}{rgb}{0.150000,0.150000,0.150000}%
\pgfsetstrokecolor{currentstroke}%
\pgfsetdash{}{0pt}%
\pgfsys@defobject{currentmarker}{\pgfqpoint{0.000000in}{0.000000in}}{\pgfqpoint{0.000000in}{0.041667in}}{%
\pgfpathmoveto{\pgfqpoint{0.000000in}{0.000000in}}%
\pgfpathlineto{\pgfqpoint{0.000000in}{0.041667in}}%
\pgfusepath{stroke,fill}%
}%
\begin{pgfscope}%
\pgfsys@transformshift{1.544006in}{0.528177in}%
\pgfsys@useobject{currentmarker}{}%
\end{pgfscope}%
\end{pgfscope}%
\begin{pgfscope}%
\definecolor{textcolor}{rgb}{0.150000,0.150000,0.150000}%
\pgfsetstrokecolor{textcolor}%
\pgfsetfillcolor{textcolor}%
\pgftext[x=1.544006in,y=0.430955in,,top]{\color{textcolor}\rmfamily\fontsize{10.000000}{12.000000}\selectfont \(\displaystyle -6\)}%
\end{pgfscope}%
\begin{pgfscope}%
\pgfsetbuttcap%
\pgfsetroundjoin%
\definecolor{currentfill}{rgb}{0.150000,0.150000,0.150000}%
\pgfsetfillcolor{currentfill}%
\pgfsetlinewidth{1.003750pt}%
\definecolor{currentstroke}{rgb}{0.150000,0.150000,0.150000}%
\pgfsetstrokecolor{currentstroke}%
\pgfsetdash{}{0pt}%
\pgfsys@defobject{currentmarker}{\pgfqpoint{0.000000in}{0.000000in}}{\pgfqpoint{0.000000in}{0.041667in}}{%
\pgfpathmoveto{\pgfqpoint{0.000000in}{0.000000in}}%
\pgfpathlineto{\pgfqpoint{0.000000in}{0.041667in}}%
\pgfusepath{stroke,fill}%
}%
\begin{pgfscope}%
\pgfsys@transformshift{2.032516in}{0.528177in}%
\pgfsys@useobject{currentmarker}{}%
\end{pgfscope}%
\end{pgfscope}%
\begin{pgfscope}%
\definecolor{textcolor}{rgb}{0.150000,0.150000,0.150000}%
\pgfsetstrokecolor{textcolor}%
\pgfsetfillcolor{textcolor}%
\pgftext[x=2.032516in,y=0.430955in,,top]{\color{textcolor}\rmfamily\fontsize{10.000000}{12.000000}\selectfont \(\displaystyle -4\)}%
\end{pgfscope}%
\begin{pgfscope}%
\pgfsetbuttcap%
\pgfsetroundjoin%
\definecolor{currentfill}{rgb}{0.150000,0.150000,0.150000}%
\pgfsetfillcolor{currentfill}%
\pgfsetlinewidth{1.003750pt}%
\definecolor{currentstroke}{rgb}{0.150000,0.150000,0.150000}%
\pgfsetstrokecolor{currentstroke}%
\pgfsetdash{}{0pt}%
\pgfsys@defobject{currentmarker}{\pgfqpoint{0.000000in}{0.000000in}}{\pgfqpoint{0.000000in}{0.041667in}}{%
\pgfpathmoveto{\pgfqpoint{0.000000in}{0.000000in}}%
\pgfpathlineto{\pgfqpoint{0.000000in}{0.041667in}}%
\pgfusepath{stroke,fill}%
}%
\begin{pgfscope}%
\pgfsys@transformshift{2.521027in}{0.528177in}%
\pgfsys@useobject{currentmarker}{}%
\end{pgfscope}%
\end{pgfscope}%
\begin{pgfscope}%
\definecolor{textcolor}{rgb}{0.150000,0.150000,0.150000}%
\pgfsetstrokecolor{textcolor}%
\pgfsetfillcolor{textcolor}%
\pgftext[x=2.521027in,y=0.430955in,,top]{\color{textcolor}\rmfamily\fontsize{10.000000}{12.000000}\selectfont \(\displaystyle -2\)}%
\end{pgfscope}%
\begin{pgfscope}%
\pgfsetbuttcap%
\pgfsetroundjoin%
\definecolor{currentfill}{rgb}{0.150000,0.150000,0.150000}%
\pgfsetfillcolor{currentfill}%
\pgfsetlinewidth{1.003750pt}%
\definecolor{currentstroke}{rgb}{0.150000,0.150000,0.150000}%
\pgfsetstrokecolor{currentstroke}%
\pgfsetdash{}{0pt}%
\pgfsys@defobject{currentmarker}{\pgfqpoint{0.000000in}{0.000000in}}{\pgfqpoint{0.000000in}{0.041667in}}{%
\pgfpathmoveto{\pgfqpoint{0.000000in}{0.000000in}}%
\pgfpathlineto{\pgfqpoint{0.000000in}{0.041667in}}%
\pgfusepath{stroke,fill}%
}%
\begin{pgfscope}%
\pgfsys@transformshift{3.009538in}{0.528177in}%
\pgfsys@useobject{currentmarker}{}%
\end{pgfscope}%
\end{pgfscope}%
\begin{pgfscope}%
\definecolor{textcolor}{rgb}{0.150000,0.150000,0.150000}%
\pgfsetstrokecolor{textcolor}%
\pgfsetfillcolor{textcolor}%
\pgftext[x=3.009538in,y=0.430955in,,top]{\color{textcolor}\rmfamily\fontsize{10.000000}{12.000000}\selectfont \(\displaystyle 0\)}%
\end{pgfscope}%
\begin{pgfscope}%
\definecolor{textcolor}{rgb}{0.150000,0.150000,0.150000}%
\pgfsetstrokecolor{textcolor}%
\pgfsetfillcolor{textcolor}%
\pgftext[x=1.910389in,y=0.238855in,,top]{\color{textcolor}\rmfamily\fontsize{10.000000}{12.000000}\selectfont \textbf{Time from freezing (s)}}%
\end{pgfscope}%
\begin{pgfscope}%
\pgfsetbuttcap%
\pgfsetroundjoin%
\definecolor{currentfill}{rgb}{0.150000,0.150000,0.150000}%
\pgfsetfillcolor{currentfill}%
\pgfsetlinewidth{1.003750pt}%
\definecolor{currentstroke}{rgb}{0.150000,0.150000,0.150000}%
\pgfsetstrokecolor{currentstroke}%
\pgfsetdash{}{0pt}%
\pgfsys@defobject{currentmarker}{\pgfqpoint{0.000000in}{0.000000in}}{\pgfqpoint{0.041667in}{0.000000in}}{%
\pgfpathmoveto{\pgfqpoint{0.000000in}{0.000000in}}%
\pgfpathlineto{\pgfqpoint{0.041667in}{0.000000in}}%
\pgfusepath{stroke,fill}%
}%
\begin{pgfscope}%
\pgfsys@transformshift{0.566985in}{0.528177in}%
\pgfsys@useobject{currentmarker}{}%
\end{pgfscope}%
\end{pgfscope}%
\begin{pgfscope}%
\definecolor{textcolor}{rgb}{0.150000,0.150000,0.150000}%
\pgfsetstrokecolor{textcolor}%
\pgfsetfillcolor{textcolor}%
\pgftext[x=0.469762in,y=0.528177in,right,]{\color{textcolor}\rmfamily\fontsize{10.000000}{12.000000}\selectfont \(\displaystyle 0.2\)}%
\end{pgfscope}%
\begin{pgfscope}%
\pgfsetbuttcap%
\pgfsetroundjoin%
\definecolor{currentfill}{rgb}{0.150000,0.150000,0.150000}%
\pgfsetfillcolor{currentfill}%
\pgfsetlinewidth{1.003750pt}%
\definecolor{currentstroke}{rgb}{0.150000,0.150000,0.150000}%
\pgfsetstrokecolor{currentstroke}%
\pgfsetdash{}{0pt}%
\pgfsys@defobject{currentmarker}{\pgfqpoint{0.000000in}{0.000000in}}{\pgfqpoint{0.041667in}{0.000000in}}{%
\pgfpathmoveto{\pgfqpoint{0.000000in}{0.000000in}}%
\pgfpathlineto{\pgfqpoint{0.041667in}{0.000000in}}%
\pgfusepath{stroke,fill}%
}%
\begin{pgfscope}%
\pgfsys@transformshift{0.566985in}{0.886230in}%
\pgfsys@useobject{currentmarker}{}%
\end{pgfscope}%
\end{pgfscope}%
\begin{pgfscope}%
\definecolor{textcolor}{rgb}{0.150000,0.150000,0.150000}%
\pgfsetstrokecolor{textcolor}%
\pgfsetfillcolor{textcolor}%
\pgftext[x=0.469762in,y=0.886230in,right,]{\color{textcolor}\rmfamily\fontsize{10.000000}{12.000000}\selectfont \(\displaystyle 0.3\)}%
\end{pgfscope}%
\begin{pgfscope}%
\pgfsetbuttcap%
\pgfsetroundjoin%
\definecolor{currentfill}{rgb}{0.150000,0.150000,0.150000}%
\pgfsetfillcolor{currentfill}%
\pgfsetlinewidth{1.003750pt}%
\definecolor{currentstroke}{rgb}{0.150000,0.150000,0.150000}%
\pgfsetstrokecolor{currentstroke}%
\pgfsetdash{}{0pt}%
\pgfsys@defobject{currentmarker}{\pgfqpoint{0.000000in}{0.000000in}}{\pgfqpoint{0.041667in}{0.000000in}}{%
\pgfpathmoveto{\pgfqpoint{0.000000in}{0.000000in}}%
\pgfpathlineto{\pgfqpoint{0.041667in}{0.000000in}}%
\pgfusepath{stroke,fill}%
}%
\begin{pgfscope}%
\pgfsys@transformshift{0.566985in}{1.244284in}%
\pgfsys@useobject{currentmarker}{}%
\end{pgfscope}%
\end{pgfscope}%
\begin{pgfscope}%
\definecolor{textcolor}{rgb}{0.150000,0.150000,0.150000}%
\pgfsetstrokecolor{textcolor}%
\pgfsetfillcolor{textcolor}%
\pgftext[x=0.469762in,y=1.244284in,right,]{\color{textcolor}\rmfamily\fontsize{10.000000}{12.000000}\selectfont \(\displaystyle 0.4\)}%
\end{pgfscope}%
\begin{pgfscope}%
\pgfsetbuttcap%
\pgfsetroundjoin%
\definecolor{currentfill}{rgb}{0.150000,0.150000,0.150000}%
\pgfsetfillcolor{currentfill}%
\pgfsetlinewidth{1.003750pt}%
\definecolor{currentstroke}{rgb}{0.150000,0.150000,0.150000}%
\pgfsetstrokecolor{currentstroke}%
\pgfsetdash{}{0pt}%
\pgfsys@defobject{currentmarker}{\pgfqpoint{0.000000in}{0.000000in}}{\pgfqpoint{0.041667in}{0.000000in}}{%
\pgfpathmoveto{\pgfqpoint{0.000000in}{0.000000in}}%
\pgfpathlineto{\pgfqpoint{0.041667in}{0.000000in}}%
\pgfusepath{stroke,fill}%
}%
\begin{pgfscope}%
\pgfsys@transformshift{0.566985in}{1.602338in}%
\pgfsys@useobject{currentmarker}{}%
\end{pgfscope}%
\end{pgfscope}%
\begin{pgfscope}%
\definecolor{textcolor}{rgb}{0.150000,0.150000,0.150000}%
\pgfsetstrokecolor{textcolor}%
\pgfsetfillcolor{textcolor}%
\pgftext[x=0.469762in,y=1.602338in,right,]{\color{textcolor}\rmfamily\fontsize{10.000000}{12.000000}\selectfont \(\displaystyle 0.5\)}%
\end{pgfscope}%
\begin{pgfscope}%
\pgfsetbuttcap%
\pgfsetroundjoin%
\definecolor{currentfill}{rgb}{0.150000,0.150000,0.150000}%
\pgfsetfillcolor{currentfill}%
\pgfsetlinewidth{1.003750pt}%
\definecolor{currentstroke}{rgb}{0.150000,0.150000,0.150000}%
\pgfsetstrokecolor{currentstroke}%
\pgfsetdash{}{0pt}%
\pgfsys@defobject{currentmarker}{\pgfqpoint{0.000000in}{0.000000in}}{\pgfqpoint{0.041667in}{0.000000in}}{%
\pgfpathmoveto{\pgfqpoint{0.000000in}{0.000000in}}%
\pgfpathlineto{\pgfqpoint{0.041667in}{0.000000in}}%
\pgfusepath{stroke,fill}%
}%
\begin{pgfscope}%
\pgfsys@transformshift{0.566985in}{1.960392in}%
\pgfsys@useobject{currentmarker}{}%
\end{pgfscope}%
\end{pgfscope}%
\begin{pgfscope}%
\definecolor{textcolor}{rgb}{0.150000,0.150000,0.150000}%
\pgfsetstrokecolor{textcolor}%
\pgfsetfillcolor{textcolor}%
\pgftext[x=0.469762in,y=1.960392in,right,]{\color{textcolor}\rmfamily\fontsize{10.000000}{12.000000}\selectfont \(\displaystyle 0.6\)}%
\end{pgfscope}%
\begin{pgfscope}%
\pgfsetbuttcap%
\pgfsetroundjoin%
\definecolor{currentfill}{rgb}{0.150000,0.150000,0.150000}%
\pgfsetfillcolor{currentfill}%
\pgfsetlinewidth{1.003750pt}%
\definecolor{currentstroke}{rgb}{0.150000,0.150000,0.150000}%
\pgfsetstrokecolor{currentstroke}%
\pgfsetdash{}{0pt}%
\pgfsys@defobject{currentmarker}{\pgfqpoint{0.000000in}{0.000000in}}{\pgfqpoint{0.041667in}{0.000000in}}{%
\pgfpathmoveto{\pgfqpoint{0.000000in}{0.000000in}}%
\pgfpathlineto{\pgfqpoint{0.041667in}{0.000000in}}%
\pgfusepath{stroke,fill}%
}%
\begin{pgfscope}%
\pgfsys@transformshift{0.566985in}{2.318445in}%
\pgfsys@useobject{currentmarker}{}%
\end{pgfscope}%
\end{pgfscope}%
\begin{pgfscope}%
\definecolor{textcolor}{rgb}{0.150000,0.150000,0.150000}%
\pgfsetstrokecolor{textcolor}%
\pgfsetfillcolor{textcolor}%
\pgftext[x=0.469762in,y=2.318445in,right,]{\color{textcolor}\rmfamily\fontsize{10.000000}{12.000000}\selectfont \(\displaystyle 0.7\)}%
\end{pgfscope}%
\begin{pgfscope}%
\pgfsetbuttcap%
\pgfsetroundjoin%
\definecolor{currentfill}{rgb}{0.150000,0.150000,0.150000}%
\pgfsetfillcolor{currentfill}%
\pgfsetlinewidth{1.003750pt}%
\definecolor{currentstroke}{rgb}{0.150000,0.150000,0.150000}%
\pgfsetstrokecolor{currentstroke}%
\pgfsetdash{}{0pt}%
\pgfsys@defobject{currentmarker}{\pgfqpoint{0.000000in}{0.000000in}}{\pgfqpoint{0.041667in}{0.000000in}}{%
\pgfpathmoveto{\pgfqpoint{0.000000in}{0.000000in}}%
\pgfpathlineto{\pgfqpoint{0.041667in}{0.000000in}}%
\pgfusepath{stroke,fill}%
}%
\begin{pgfscope}%
\pgfsys@transformshift{0.566985in}{2.676499in}%
\pgfsys@useobject{currentmarker}{}%
\end{pgfscope}%
\end{pgfscope}%
\begin{pgfscope}%
\definecolor{textcolor}{rgb}{0.150000,0.150000,0.150000}%
\pgfsetstrokecolor{textcolor}%
\pgfsetfillcolor{textcolor}%
\pgftext[x=0.469762in,y=2.676499in,right,]{\color{textcolor}\rmfamily\fontsize{10.000000}{12.000000}\selectfont \(\displaystyle 0.8\)}%
\end{pgfscope}%
\begin{pgfscope}%
\pgfsetbuttcap%
\pgfsetroundjoin%
\definecolor{currentfill}{rgb}{0.150000,0.150000,0.150000}%
\pgfsetfillcolor{currentfill}%
\pgfsetlinewidth{1.003750pt}%
\definecolor{currentstroke}{rgb}{0.150000,0.150000,0.150000}%
\pgfsetstrokecolor{currentstroke}%
\pgfsetdash{}{0pt}%
\pgfsys@defobject{currentmarker}{\pgfqpoint{0.000000in}{0.000000in}}{\pgfqpoint{0.041667in}{0.000000in}}{%
\pgfpathmoveto{\pgfqpoint{0.000000in}{0.000000in}}%
\pgfpathlineto{\pgfqpoint{0.041667in}{0.000000in}}%
\pgfusepath{stroke,fill}%
}%
\begin{pgfscope}%
\pgfsys@transformshift{0.566985in}{3.034553in}%
\pgfsys@useobject{currentmarker}{}%
\end{pgfscope}%
\end{pgfscope}%
\begin{pgfscope}%
\definecolor{textcolor}{rgb}{0.150000,0.150000,0.150000}%
\pgfsetstrokecolor{textcolor}%
\pgfsetfillcolor{textcolor}%
\pgftext[x=0.469762in,y=3.034553in,right,]{\color{textcolor}\rmfamily\fontsize{10.000000}{12.000000}\selectfont \(\displaystyle 0.9\)}%
\end{pgfscope}%
\begin{pgfscope}%
\pgfsetbuttcap%
\pgfsetroundjoin%
\definecolor{currentfill}{rgb}{0.150000,0.150000,0.150000}%
\pgfsetfillcolor{currentfill}%
\pgfsetlinewidth{1.003750pt}%
\definecolor{currentstroke}{rgb}{0.150000,0.150000,0.150000}%
\pgfsetstrokecolor{currentstroke}%
\pgfsetdash{}{0pt}%
\pgfsys@defobject{currentmarker}{\pgfqpoint{0.000000in}{0.000000in}}{\pgfqpoint{0.041667in}{0.000000in}}{%
\pgfpathmoveto{\pgfqpoint{0.000000in}{0.000000in}}%
\pgfpathlineto{\pgfqpoint{0.041667in}{0.000000in}}%
\pgfusepath{stroke,fill}%
}%
\begin{pgfscope}%
\pgfsys@transformshift{0.566985in}{3.392606in}%
\pgfsys@useobject{currentmarker}{}%
\end{pgfscope}%
\end{pgfscope}%
\begin{pgfscope}%
\definecolor{textcolor}{rgb}{0.150000,0.150000,0.150000}%
\pgfsetstrokecolor{textcolor}%
\pgfsetfillcolor{textcolor}%
\pgftext[x=0.469762in,y=3.392606in,right,]{\color{textcolor}\rmfamily\fontsize{10.000000}{12.000000}\selectfont \(\displaystyle 1.0\)}%
\end{pgfscope}%
\begin{pgfscope}%
\definecolor{textcolor}{rgb}{0.150000,0.150000,0.150000}%
\pgfsetstrokecolor{textcolor}%
\pgfsetfillcolor{textcolor}%
\pgftext[x=0.222848in,y=1.960392in,,bottom,rotate=90.000000]{\color{textcolor}\rmfamily\fontsize{10.000000}{12.000000}\selectfont \textbf{Accuracy}}%
\end{pgfscope}%
\begin{pgfscope}%
\pgfpathrectangle{\pgfqpoint{0.566985in}{0.528177in}}{\pgfqpoint{2.686808in}{2.864429in}} %
\pgfusepath{clip}%
\pgfsetroundcap%
\pgfsetroundjoin%
\pgfsetlinewidth{1.756562pt}%
\definecolor{currentstroke}{rgb}{0.200000,0.427451,0.650980}%
\pgfsetstrokecolor{currentstroke}%
\pgfsetstrokeopacity{0.800000}%
\pgfsetdash{}{0pt}%
\pgfpathmoveto{\pgfqpoint{0.566985in}{2.993933in}}%
\pgfpathlineto{\pgfqpoint{0.579197in}{3.092352in}}%
\pgfpathlineto{\pgfqpoint{0.591410in}{3.099858in}}%
\pgfpathlineto{\pgfqpoint{0.603623in}{3.137390in}}%
\pgfpathlineto{\pgfqpoint{0.615836in}{3.024794in}}%
\pgfpathlineto{\pgfqpoint{0.628048in}{3.047313in}}%
\pgfpathlineto{\pgfqpoint{0.640261in}{3.009782in}}%
\pgfpathlineto{\pgfqpoint{0.652474in}{3.009782in}}%
\pgfpathlineto{\pgfqpoint{0.664687in}{3.077339in}}%
\pgfpathlineto{\pgfqpoint{0.676899in}{3.099858in}}%
\pgfpathlineto{\pgfqpoint{0.689112in}{3.047313in}}%
\pgfpathlineto{\pgfqpoint{0.701325in}{3.032301in}}%
\pgfpathlineto{\pgfqpoint{0.713538in}{3.047313in}}%
\pgfpathlineto{\pgfqpoint{0.725751in}{2.927212in}}%
\pgfpathlineto{\pgfqpoint{0.737963in}{3.009782in}}%
\pgfpathlineto{\pgfqpoint{0.750176in}{3.002275in}}%
\pgfpathlineto{\pgfqpoint{0.762389in}{3.024794in}}%
\pgfpathlineto{\pgfqpoint{0.774602in}{3.017288in}}%
\pgfpathlineto{\pgfqpoint{0.786814in}{3.024794in}}%
\pgfpathlineto{\pgfqpoint{0.799027in}{2.972250in}}%
\pgfpathlineto{\pgfqpoint{0.811240in}{3.032301in}}%
\pgfpathlineto{\pgfqpoint{0.823453in}{3.032301in}}%
\pgfpathlineto{\pgfqpoint{0.835665in}{2.994769in}}%
\pgfpathlineto{\pgfqpoint{0.847878in}{3.069832in}}%
\pgfpathlineto{\pgfqpoint{0.860091in}{2.994769in}}%
\pgfpathlineto{\pgfqpoint{0.872304in}{3.054820in}}%
\pgfpathlineto{\pgfqpoint{0.884516in}{3.099858in}}%
\pgfpathlineto{\pgfqpoint{0.896729in}{3.002275in}}%
\pgfpathlineto{\pgfqpoint{0.908942in}{3.032301in}}%
\pgfpathlineto{\pgfqpoint{0.921155in}{3.009782in}}%
\pgfpathlineto{\pgfqpoint{0.933368in}{2.964743in}}%
\pgfpathlineto{\pgfqpoint{0.945580in}{2.927212in}}%
\pgfpathlineto{\pgfqpoint{0.957793in}{2.994769in}}%
\pgfpathlineto{\pgfqpoint{0.970006in}{3.017288in}}%
\pgfpathlineto{\pgfqpoint{0.982219in}{2.994769in}}%
\pgfpathlineto{\pgfqpoint{0.994431in}{3.084845in}}%
\pgfpathlineto{\pgfqpoint{1.006644in}{2.988110in}}%
\pgfpathlineto{\pgfqpoint{1.018857in}{2.973129in}}%
\pgfpathlineto{\pgfqpoint{1.031070in}{3.011380in}}%
\pgfpathlineto{\pgfqpoint{1.043282in}{3.003905in}}%
\pgfpathlineto{\pgfqpoint{1.055495in}{3.018855in}}%
\pgfpathlineto{\pgfqpoint{1.067708in}{3.026330in}}%
\pgfpathlineto{\pgfqpoint{1.079921in}{3.011380in}}%
\pgfpathlineto{\pgfqpoint{1.092133in}{2.936630in}}%
\pgfpathlineto{\pgfqpoint{1.104346in}{2.921680in}}%
\pgfpathlineto{\pgfqpoint{1.116559in}{2.914205in}}%
\pgfpathlineto{\pgfqpoint{1.128772in}{2.966530in}}%
\pgfpathlineto{\pgfqpoint{1.140985in}{2.929155in}}%
\pgfpathlineto{\pgfqpoint{1.165410in}{2.988955in}}%
\pgfpathlineto{\pgfqpoint{1.177623in}{3.011380in}}%
\pgfpathlineto{\pgfqpoint{1.189836in}{2.988955in}}%
\pgfpathlineto{\pgfqpoint{1.202048in}{2.981480in}}%
\pgfpathlineto{\pgfqpoint{1.214261in}{2.966530in}}%
\pgfpathlineto{\pgfqpoint{1.226474in}{2.944105in}}%
\pgfpathlineto{\pgfqpoint{1.238687in}{2.974877in}}%
\pgfpathlineto{\pgfqpoint{1.250899in}{2.989796in}}%
\pgfpathlineto{\pgfqpoint{1.263112in}{2.959958in}}%
\pgfpathlineto{\pgfqpoint{1.275325in}{2.952499in}}%
\pgfpathlineto{\pgfqpoint{1.287538in}{2.952499in}}%
\pgfpathlineto{\pgfqpoint{1.299750in}{3.004715in}}%
\pgfpathlineto{\pgfqpoint{1.311963in}{2.989796in}}%
\pgfpathlineto{\pgfqpoint{1.336389in}{2.900282in}}%
\pgfpathlineto{\pgfqpoint{1.348602in}{2.922661in}}%
\pgfpathlineto{\pgfqpoint{1.360814in}{2.937580in}}%
\pgfpathlineto{\pgfqpoint{1.373027in}{2.937580in}}%
\pgfpathlineto{\pgfqpoint{1.385240in}{2.922661in}}%
\pgfpathlineto{\pgfqpoint{1.397453in}{3.019634in}}%
\pgfpathlineto{\pgfqpoint{1.409665in}{3.012174in}}%
\pgfpathlineto{\pgfqpoint{1.421878in}{2.959958in}}%
\pgfpathlineto{\pgfqpoint{1.434091in}{2.952499in}}%
\pgfpathlineto{\pgfqpoint{1.446304in}{2.952499in}}%
\pgfpathlineto{\pgfqpoint{1.458516in}{2.959958in}}%
\pgfpathlineto{\pgfqpoint{1.470729in}{2.952499in}}%
\pgfpathlineto{\pgfqpoint{1.482942in}{2.907742in}}%
\pgfpathlineto{\pgfqpoint{1.507367in}{2.997255in}}%
\pgfpathlineto{\pgfqpoint{1.519580in}{2.937580in}}%
\pgfpathlineto{\pgfqpoint{1.544006in}{2.967417in}}%
\pgfpathlineto{\pgfqpoint{1.556219in}{3.042012in}}%
\pgfpathlineto{\pgfqpoint{1.568431in}{2.997255in}}%
\pgfpathlineto{\pgfqpoint{1.580644in}{3.012965in}}%
\pgfpathlineto{\pgfqpoint{1.592857in}{3.035297in}}%
\pgfpathlineto{\pgfqpoint{1.605070in}{2.983189in}}%
\pgfpathlineto{\pgfqpoint{1.617282in}{2.945970in}}%
\pgfpathlineto{\pgfqpoint{1.641708in}{2.946896in}}%
\pgfpathlineto{\pgfqpoint{1.653921in}{2.991467in}}%
\pgfpathlineto{\pgfqpoint{1.666133in}{2.984039in}}%
\pgfpathlineto{\pgfqpoint{1.678346in}{3.021181in}}%
\pgfpathlineto{\pgfqpoint{1.690559in}{2.991467in}}%
\pgfpathlineto{\pgfqpoint{1.702772in}{2.917182in}}%
\pgfpathlineto{\pgfqpoint{1.727197in}{2.976610in}}%
\pgfpathlineto{\pgfqpoint{1.739410in}{2.909754in}}%
\pgfpathlineto{\pgfqpoint{1.751623in}{2.969182in}}%
\pgfpathlineto{\pgfqpoint{1.763836in}{2.939468in}}%
\pgfpathlineto{\pgfqpoint{1.776048in}{2.969182in}}%
\pgfpathlineto{\pgfqpoint{1.788261in}{3.028610in}}%
\pgfpathlineto{\pgfqpoint{1.800474in}{2.998896in}}%
\pgfpathlineto{\pgfqpoint{1.812687in}{2.932039in}}%
\pgfpathlineto{\pgfqpoint{1.824899in}{2.969182in}}%
\pgfpathlineto{\pgfqpoint{1.837112in}{2.984039in}}%
\pgfpathlineto{\pgfqpoint{1.849325in}{2.961753in}}%
\pgfpathlineto{\pgfqpoint{1.861538in}{2.924611in}}%
\pgfpathlineto{\pgfqpoint{1.885963in}{2.939468in}}%
\pgfpathlineto{\pgfqpoint{1.898176in}{2.894897in}}%
\pgfpathlineto{\pgfqpoint{1.910389in}{2.939468in}}%
\pgfpathlineto{\pgfqpoint{1.922601in}{2.961753in}}%
\pgfpathlineto{\pgfqpoint{1.934814in}{2.894897in}}%
\pgfpathlineto{\pgfqpoint{1.947027in}{2.917182in}}%
\pgfpathlineto{\pgfqpoint{1.959240in}{2.932039in}}%
\pgfpathlineto{\pgfqpoint{1.971453in}{2.984039in}}%
\pgfpathlineto{\pgfqpoint{1.983665in}{2.961753in}}%
\pgfpathlineto{\pgfqpoint{1.995878in}{2.991467in}}%
\pgfpathlineto{\pgfqpoint{2.008091in}{2.984039in}}%
\pgfpathlineto{\pgfqpoint{2.020304in}{2.984039in}}%
\pgfpathlineto{\pgfqpoint{2.044729in}{2.954325in}}%
\pgfpathlineto{\pgfqpoint{2.056942in}{3.050895in}}%
\pgfpathlineto{\pgfqpoint{2.069155in}{3.013753in}}%
\pgfpathlineto{\pgfqpoint{2.081367in}{3.028610in}}%
\pgfpathlineto{\pgfqpoint{2.093580in}{3.029363in}}%
\pgfpathlineto{\pgfqpoint{2.105793in}{2.984885in}}%
\pgfpathlineto{\pgfqpoint{2.118006in}{2.933942in}}%
\pgfpathlineto{\pgfqpoint{2.130218in}{2.985727in}}%
\pgfpathlineto{\pgfqpoint{2.142431in}{2.978329in}}%
\pgfpathlineto{\pgfqpoint{2.154644in}{3.037512in}}%
\pgfpathlineto{\pgfqpoint{2.166857in}{2.978329in}}%
\pgfpathlineto{\pgfqpoint{2.179070in}{2.889556in}}%
\pgfpathlineto{\pgfqpoint{2.191282in}{2.948738in}}%
\pgfpathlineto{\pgfqpoint{2.215708in}{2.963534in}}%
\pgfpathlineto{\pgfqpoint{2.227921in}{2.941340in}}%
\pgfpathlineto{\pgfqpoint{2.240133in}{2.926545in}}%
\pgfpathlineto{\pgfqpoint{2.252346in}{2.904351in}}%
\pgfpathlineto{\pgfqpoint{2.264559in}{2.890593in}}%
\pgfpathlineto{\pgfqpoint{2.276772in}{2.927506in}}%
\pgfpathlineto{\pgfqpoint{2.288984in}{2.905358in}}%
\pgfpathlineto{\pgfqpoint{2.301197in}{2.957036in}}%
\pgfpathlineto{\pgfqpoint{2.313410in}{2.906360in}}%
\pgfpathlineto{\pgfqpoint{2.325623in}{2.965299in}}%
\pgfpathlineto{\pgfqpoint{2.337835in}{2.913728in}}%
\pgfpathlineto{\pgfqpoint{2.350048in}{2.957932in}}%
\pgfpathlineto{\pgfqpoint{2.362261in}{2.914711in}}%
\pgfpathlineto{\pgfqpoint{2.374474in}{2.936768in}}%
\pgfpathlineto{\pgfqpoint{2.386687in}{2.944120in}}%
\pgfpathlineto{\pgfqpoint{2.398899in}{2.936768in}}%
\pgfpathlineto{\pgfqpoint{2.411112in}{2.885302in}}%
\pgfpathlineto{\pgfqpoint{2.423325in}{2.966177in}}%
\pgfpathlineto{\pgfqpoint{2.435538in}{2.944120in}}%
\pgfpathlineto{\pgfqpoint{2.447750in}{2.966177in}}%
\pgfpathlineto{\pgfqpoint{2.459963in}{2.951472in}}%
\pgfpathlineto{\pgfqpoint{2.472176in}{2.892654in}}%
\pgfpathlineto{\pgfqpoint{2.484389in}{2.944120in}}%
\pgfpathlineto{\pgfqpoint{2.496601in}{2.988233in}}%
\pgfpathlineto{\pgfqpoint{2.508814in}{2.944120in}}%
\pgfpathlineto{\pgfqpoint{2.521027in}{2.945039in}}%
\pgfpathlineto{\pgfqpoint{2.533240in}{2.967051in}}%
\pgfpathlineto{\pgfqpoint{2.545452in}{2.930365in}}%
\pgfpathlineto{\pgfqpoint{2.557665in}{2.930365in}}%
\pgfpathlineto{\pgfqpoint{2.569878in}{2.871667in}}%
\pgfpathlineto{\pgfqpoint{2.594304in}{2.915690in}}%
\pgfpathlineto{\pgfqpoint{2.606516in}{2.864330in}}%
\pgfpathlineto{\pgfqpoint{2.618729in}{2.879005in}}%
\pgfpathlineto{\pgfqpoint{2.630942in}{2.952376in}}%
\pgfpathlineto{\pgfqpoint{2.643155in}{2.856993in}}%
\pgfpathlineto{\pgfqpoint{2.655367in}{2.908353in}}%
\pgfpathlineto{\pgfqpoint{2.667580in}{2.893679in}}%
\pgfpathlineto{\pgfqpoint{2.679793in}{2.842319in}}%
\pgfpathlineto{\pgfqpoint{2.692006in}{2.865411in}}%
\pgfpathlineto{\pgfqpoint{2.704218in}{2.821478in}}%
\pgfpathlineto{\pgfqpoint{2.716431in}{2.865411in}}%
\pgfpathlineto{\pgfqpoint{2.728644in}{2.887377in}}%
\pgfpathlineto{\pgfqpoint{2.740857in}{2.887377in}}%
\pgfpathlineto{\pgfqpoint{2.753069in}{2.902021in}}%
\pgfpathlineto{\pgfqpoint{2.789708in}{2.880055in}}%
\pgfpathlineto{\pgfqpoint{2.801921in}{2.931310in}}%
\pgfpathlineto{\pgfqpoint{2.814133in}{2.938632in}}%
\pgfpathlineto{\pgfqpoint{2.826346in}{2.872733in}}%
\pgfpathlineto{\pgfqpoint{2.838559in}{2.880055in}}%
\pgfpathlineto{\pgfqpoint{2.850772in}{2.872733in}}%
\pgfpathlineto{\pgfqpoint{2.862984in}{2.872733in}}%
\pgfpathlineto{\pgfqpoint{2.875197in}{2.762900in}}%
\pgfpathlineto{\pgfqpoint{2.887410in}{2.667712in}}%
\pgfpathlineto{\pgfqpoint{2.899623in}{2.683806in}}%
\pgfpathlineto{\pgfqpoint{2.911835in}{2.625348in}}%
\pgfpathlineto{\pgfqpoint{2.924048in}{2.603427in}}%
\pgfpathlineto{\pgfqpoint{2.936261in}{2.501126in}}%
\pgfpathlineto{\pgfqpoint{2.948474in}{2.493818in}}%
\pgfpathlineto{\pgfqpoint{2.960686in}{2.435361in}}%
\pgfpathlineto{\pgfqpoint{2.972899in}{2.245373in}}%
\pgfpathlineto{\pgfqpoint{2.985112in}{1.982313in}}%
\pgfpathlineto{\pgfqpoint{2.997325in}{1.507730in}}%
\pgfpathlineto{\pgfqpoint{3.009538in}{1.609615in}}%
\pgfpathlineto{\pgfqpoint{3.021750in}{1.995324in}}%
\pgfpathlineto{\pgfqpoint{3.033963in}{2.322812in}}%
\pgfpathlineto{\pgfqpoint{3.058389in}{2.519305in}}%
\pgfpathlineto{\pgfqpoint{3.070601in}{2.562970in}}%
\pgfpathlineto{\pgfqpoint{3.082814in}{2.562970in}}%
\pgfpathlineto{\pgfqpoint{3.095027in}{2.628467in}}%
\pgfpathlineto{\pgfqpoint{3.107240in}{2.679410in}}%
\pgfpathlineto{\pgfqpoint{3.119452in}{2.686687in}}%
\pgfpathlineto{\pgfqpoint{3.131665in}{2.723075in}}%
\pgfpathlineto{\pgfqpoint{3.143878in}{2.752185in}}%
\pgfpathlineto{\pgfqpoint{3.156091in}{2.759463in}}%
\pgfpathlineto{\pgfqpoint{3.168303in}{2.708520in}}%
\pgfpathlineto{\pgfqpoint{3.180516in}{2.817683in}}%
\pgfpathlineto{\pgfqpoint{3.192729in}{2.832238in}}%
\pgfpathlineto{\pgfqpoint{3.204942in}{2.803128in}}%
\pgfpathlineto{\pgfqpoint{3.217155in}{2.861348in}}%
\pgfpathlineto{\pgfqpoint{3.229367in}{2.875903in}}%
\pgfpathlineto{\pgfqpoint{3.241580in}{2.854070in}}%
\pgfpathlineto{\pgfqpoint{3.253793in}{2.874850in}}%
\pgfpathlineto{\pgfqpoint{3.266006in}{2.860266in}}%
\pgfpathlineto{\pgfqpoint{3.267682in}{2.867272in}}%
\pgfpathlineto{\pgfqpoint{3.267682in}{2.867272in}}%
\pgfusepath{stroke}%
\end{pgfscope}%
\begin{pgfscope}%
\pgfpathrectangle{\pgfqpoint{0.566985in}{0.528177in}}{\pgfqpoint{2.686808in}{2.864429in}} %
\pgfusepath{clip}%
\pgfsetroundcap%
\pgfsetroundjoin%
\pgfsetlinewidth{1.756562pt}%
\definecolor{currentstroke}{rgb}{0.168627,0.670588,0.494118}%
\pgfsetstrokecolor{currentstroke}%
\pgfsetstrokeopacity{0.800000}%
\pgfsetdash{}{0pt}%
\pgfpathmoveto{\pgfqpoint{0.566985in}{2.688568in}}%
\pgfpathlineto{\pgfqpoint{0.579197in}{2.748914in}}%
\pgfpathlineto{\pgfqpoint{0.591410in}{2.849491in}}%
\pgfpathlineto{\pgfqpoint{0.603623in}{2.859549in}}%
\pgfpathlineto{\pgfqpoint{0.615836in}{2.819318in}}%
\pgfpathlineto{\pgfqpoint{0.628048in}{2.829376in}}%
\pgfpathlineto{\pgfqpoint{0.640261in}{2.889722in}}%
\pgfpathlineto{\pgfqpoint{0.652474in}{2.869606in}}%
\pgfpathlineto{\pgfqpoint{0.664687in}{2.819318in}}%
\pgfpathlineto{\pgfqpoint{0.676899in}{2.789145in}}%
\pgfpathlineto{\pgfqpoint{0.689112in}{2.799203in}}%
\pgfpathlineto{\pgfqpoint{0.701325in}{2.879664in}}%
\pgfpathlineto{\pgfqpoint{0.713538in}{2.799203in}}%
\pgfpathlineto{\pgfqpoint{0.725751in}{2.819318in}}%
\pgfpathlineto{\pgfqpoint{0.737963in}{2.879664in}}%
\pgfpathlineto{\pgfqpoint{0.750176in}{2.849491in}}%
\pgfpathlineto{\pgfqpoint{0.762389in}{2.859549in}}%
\pgfpathlineto{\pgfqpoint{0.774602in}{2.929953in}}%
\pgfpathlineto{\pgfqpoint{0.786814in}{2.929953in}}%
\pgfpathlineto{\pgfqpoint{0.811240in}{2.869606in}}%
\pgfpathlineto{\pgfqpoint{0.823453in}{2.889722in}}%
\pgfpathlineto{\pgfqpoint{0.835665in}{2.899780in}}%
\pgfpathlineto{\pgfqpoint{0.847878in}{2.899780in}}%
\pgfpathlineto{\pgfqpoint{0.872304in}{2.879664in}}%
\pgfpathlineto{\pgfqpoint{0.884516in}{2.859549in}}%
\pgfpathlineto{\pgfqpoint{0.896729in}{2.869606in}}%
\pgfpathlineto{\pgfqpoint{0.908942in}{2.789145in}}%
\pgfpathlineto{\pgfqpoint{0.921155in}{2.879664in}}%
\pgfpathlineto{\pgfqpoint{0.933368in}{2.839433in}}%
\pgfpathlineto{\pgfqpoint{0.945580in}{2.859549in}}%
\pgfpathlineto{\pgfqpoint{0.957793in}{2.789145in}}%
\pgfpathlineto{\pgfqpoint{0.970006in}{2.789145in}}%
\pgfpathlineto{\pgfqpoint{0.982219in}{2.700570in}}%
\pgfpathlineto{\pgfqpoint{0.994431in}{2.760747in}}%
\pgfpathlineto{\pgfqpoint{1.006644in}{2.830953in}}%
\pgfpathlineto{\pgfqpoint{1.018857in}{2.830953in}}%
\pgfpathlineto{\pgfqpoint{1.031070in}{2.820924in}}%
\pgfpathlineto{\pgfqpoint{1.043282in}{2.891130in}}%
\pgfpathlineto{\pgfqpoint{1.055495in}{2.891130in}}%
\pgfpathlineto{\pgfqpoint{1.067708in}{2.820924in}}%
\pgfpathlineto{\pgfqpoint{1.079921in}{2.861042in}}%
\pgfpathlineto{\pgfqpoint{1.092133in}{2.891130in}}%
\pgfpathlineto{\pgfqpoint{1.104346in}{2.871071in}}%
\pgfpathlineto{\pgfqpoint{1.116559in}{2.820924in}}%
\pgfpathlineto{\pgfqpoint{1.128772in}{2.830953in}}%
\pgfpathlineto{\pgfqpoint{1.140985in}{2.851012in}}%
\pgfpathlineto{\pgfqpoint{1.153197in}{2.770776in}}%
\pgfpathlineto{\pgfqpoint{1.165410in}{2.820924in}}%
\pgfpathlineto{\pgfqpoint{1.177623in}{2.790835in}}%
\pgfpathlineto{\pgfqpoint{1.189836in}{2.830953in}}%
\pgfpathlineto{\pgfqpoint{1.202048in}{2.892531in}}%
\pgfpathlineto{\pgfqpoint{1.214261in}{2.902533in}}%
\pgfpathlineto{\pgfqpoint{1.226474in}{2.892531in}}%
\pgfpathlineto{\pgfqpoint{1.238687in}{2.872528in}}%
\pgfpathlineto{\pgfqpoint{1.250899in}{2.862527in}}%
\pgfpathlineto{\pgfqpoint{1.263112in}{2.902533in}}%
\pgfpathlineto{\pgfqpoint{1.275325in}{2.892531in}}%
\pgfpathlineto{\pgfqpoint{1.287538in}{2.952540in}}%
\pgfpathlineto{\pgfqpoint{1.299750in}{2.892531in}}%
\pgfpathlineto{\pgfqpoint{1.311963in}{2.902533in}}%
\pgfpathlineto{\pgfqpoint{1.336389in}{2.862527in}}%
\pgfpathlineto{\pgfqpoint{1.360814in}{2.672498in}}%
\pgfpathlineto{\pgfqpoint{1.385240in}{2.772513in}}%
\pgfpathlineto{\pgfqpoint{1.397453in}{2.732507in}}%
\pgfpathlineto{\pgfqpoint{1.409665in}{2.772513in}}%
\pgfpathlineto{\pgfqpoint{1.421878in}{2.762512in}}%
\pgfpathlineto{\pgfqpoint{1.434091in}{2.822521in}}%
\pgfpathlineto{\pgfqpoint{1.458516in}{2.724372in}}%
\pgfpathlineto{\pgfqpoint{1.470729in}{2.814135in}}%
\pgfpathlineto{\pgfqpoint{1.482942in}{2.824109in}}%
\pgfpathlineto{\pgfqpoint{1.495155in}{2.844056in}}%
\pgfpathlineto{\pgfqpoint{1.507367in}{2.804161in}}%
\pgfpathlineto{\pgfqpoint{1.519580in}{2.834082in}}%
\pgfpathlineto{\pgfqpoint{1.531793in}{2.744320in}}%
\pgfpathlineto{\pgfqpoint{1.544006in}{2.804161in}}%
\pgfpathlineto{\pgfqpoint{1.556219in}{2.804161in}}%
\pgfpathlineto{\pgfqpoint{1.568431in}{2.854030in}}%
\pgfpathlineto{\pgfqpoint{1.580644in}{2.814135in}}%
\pgfpathlineto{\pgfqpoint{1.592857in}{2.784214in}}%
\pgfpathlineto{\pgfqpoint{1.605070in}{2.804161in}}%
\pgfpathlineto{\pgfqpoint{1.617282in}{2.804161in}}%
\pgfpathlineto{\pgfqpoint{1.629495in}{2.893924in}}%
\pgfpathlineto{\pgfqpoint{1.641708in}{2.893924in}}%
\pgfpathlineto{\pgfqpoint{1.653921in}{2.834082in}}%
\pgfpathlineto{\pgfqpoint{1.666133in}{2.864003in}}%
\pgfpathlineto{\pgfqpoint{1.678346in}{2.855526in}}%
\pgfpathlineto{\pgfqpoint{1.690559in}{2.865472in}}%
\pgfpathlineto{\pgfqpoint{1.702772in}{2.835634in}}%
\pgfpathlineto{\pgfqpoint{1.714984in}{2.855526in}}%
\pgfpathlineto{\pgfqpoint{1.727197in}{2.835634in}}%
\pgfpathlineto{\pgfqpoint{1.739410in}{2.746120in}}%
\pgfpathlineto{\pgfqpoint{1.751623in}{2.795850in}}%
\pgfpathlineto{\pgfqpoint{1.763836in}{2.795850in}}%
\pgfpathlineto{\pgfqpoint{1.776048in}{2.746120in}}%
\pgfpathlineto{\pgfqpoint{1.788261in}{2.815742in}}%
\pgfpathlineto{\pgfqpoint{1.800474in}{2.775958in}}%
\pgfpathlineto{\pgfqpoint{1.812687in}{2.795850in}}%
\pgfpathlineto{\pgfqpoint{1.824899in}{2.736174in}}%
\pgfpathlineto{\pgfqpoint{1.837112in}{2.736174in}}%
\pgfpathlineto{\pgfqpoint{1.849325in}{2.726229in}}%
\pgfpathlineto{\pgfqpoint{1.873750in}{2.586985in}}%
\pgfpathlineto{\pgfqpoint{1.885963in}{2.537256in}}%
\pgfpathlineto{\pgfqpoint{1.898176in}{2.507418in}}%
\pgfpathlineto{\pgfqpoint{1.910389in}{2.567094in}}%
\pgfpathlineto{\pgfqpoint{1.922601in}{2.668564in}}%
\pgfpathlineto{\pgfqpoint{1.934814in}{2.658646in}}%
\pgfpathlineto{\pgfqpoint{1.947027in}{2.668564in}}%
\pgfpathlineto{\pgfqpoint{1.959240in}{2.737993in}}%
\pgfpathlineto{\pgfqpoint{1.971453in}{2.757830in}}%
\pgfpathlineto{\pgfqpoint{1.983665in}{2.658646in}}%
\pgfpathlineto{\pgfqpoint{1.995878in}{2.658646in}}%
\pgfpathlineto{\pgfqpoint{2.008091in}{2.737993in}}%
\pgfpathlineto{\pgfqpoint{2.020304in}{2.837177in}}%
\pgfpathlineto{\pgfqpoint{2.032516in}{2.817340in}}%
\pgfpathlineto{\pgfqpoint{2.044729in}{2.837177in}}%
\pgfpathlineto{\pgfqpoint{2.056942in}{2.757830in}}%
\pgfpathlineto{\pgfqpoint{2.069155in}{2.787585in}}%
\pgfpathlineto{\pgfqpoint{2.081367in}{2.737993in}}%
\pgfpathlineto{\pgfqpoint{2.093580in}{2.757830in}}%
\pgfpathlineto{\pgfqpoint{2.105793in}{2.857013in}}%
\pgfpathlineto{\pgfqpoint{2.118006in}{2.838711in}}%
\pgfpathlineto{\pgfqpoint{2.130218in}{2.868384in}}%
\pgfpathlineto{\pgfqpoint{2.142431in}{2.809038in}}%
\pgfpathlineto{\pgfqpoint{2.154644in}{2.720019in}}%
\pgfpathlineto{\pgfqpoint{2.166857in}{2.680455in}}%
\pgfpathlineto{\pgfqpoint{2.179070in}{2.759583in}}%
\pgfpathlineto{\pgfqpoint{2.191282in}{2.710128in}}%
\pgfpathlineto{\pgfqpoint{2.203495in}{2.640891in}}%
\pgfpathlineto{\pgfqpoint{2.215708in}{2.680455in}}%
\pgfpathlineto{\pgfqpoint{2.227921in}{2.749692in}}%
\pgfpathlineto{\pgfqpoint{2.240133in}{2.749692in}}%
\pgfpathlineto{\pgfqpoint{2.252346in}{2.800782in}}%
\pgfpathlineto{\pgfqpoint{2.264559in}{2.810646in}}%
\pgfpathlineto{\pgfqpoint{2.276772in}{2.830373in}}%
\pgfpathlineto{\pgfqpoint{2.301197in}{2.810646in}}%
\pgfpathlineto{\pgfqpoint{2.313410in}{2.810646in}}%
\pgfpathlineto{\pgfqpoint{2.325623in}{2.692281in}}%
\pgfpathlineto{\pgfqpoint{2.337835in}{2.810646in}}%
\pgfpathlineto{\pgfqpoint{2.350048in}{2.741600in}}%
\pgfpathlineto{\pgfqpoint{2.362261in}{2.810646in}}%
\pgfpathlineto{\pgfqpoint{2.374474in}{2.790918in}}%
\pgfpathlineto{\pgfqpoint{2.386687in}{2.850101in}}%
\pgfpathlineto{\pgfqpoint{2.398899in}{2.751463in}}%
\pgfpathlineto{\pgfqpoint{2.411112in}{2.790918in}}%
\pgfpathlineto{\pgfqpoint{2.423325in}{2.761327in}}%
\pgfpathlineto{\pgfqpoint{2.435538in}{2.672553in}}%
\pgfpathlineto{\pgfqpoint{2.447750in}{2.702145in}}%
\pgfpathlineto{\pgfqpoint{2.459963in}{2.652826in}}%
\pgfpathlineto{\pgfqpoint{2.472176in}{2.633098in}}%
\pgfpathlineto{\pgfqpoint{2.484389in}{2.712008in}}%
\pgfpathlineto{\pgfqpoint{2.496601in}{2.613371in}}%
\pgfpathlineto{\pgfqpoint{2.508814in}{2.642962in}}%
\pgfpathlineto{\pgfqpoint{2.521027in}{2.564052in}}%
\pgfpathlineto{\pgfqpoint{2.533240in}{2.583780in}}%
\pgfpathlineto{\pgfqpoint{2.545452in}{2.593643in}}%
\pgfpathlineto{\pgfqpoint{2.557665in}{2.564052in}}%
\pgfpathlineto{\pgfqpoint{2.569878in}{2.524597in}}%
\pgfpathlineto{\pgfqpoint{2.582091in}{2.524597in}}%
\pgfpathlineto{\pgfqpoint{2.594304in}{2.573916in}}%
\pgfpathlineto{\pgfqpoint{2.606516in}{2.514734in}}%
\pgfpathlineto{\pgfqpoint{2.618729in}{2.544325in}}%
\pgfpathlineto{\pgfqpoint{2.630942in}{2.425960in}}%
\pgfpathlineto{\pgfqpoint{2.643155in}{2.534461in}}%
\pgfpathlineto{\pgfqpoint{2.667580in}{2.455551in}}%
\pgfpathlineto{\pgfqpoint{2.692006in}{2.554189in}}%
\pgfpathlineto{\pgfqpoint{2.716431in}{2.514734in}}%
\pgfpathlineto{\pgfqpoint{2.728644in}{2.603507in}}%
\pgfpathlineto{\pgfqpoint{2.740857in}{2.662690in}}%
\pgfpathlineto{\pgfqpoint{2.753069in}{2.642962in}}%
\pgfpathlineto{\pgfqpoint{2.765282in}{2.633098in}}%
\pgfpathlineto{\pgfqpoint{2.777495in}{2.603507in}}%
\pgfpathlineto{\pgfqpoint{2.789708in}{2.564052in}}%
\pgfpathlineto{\pgfqpoint{2.801921in}{2.564052in}}%
\pgfpathlineto{\pgfqpoint{2.814133in}{2.593643in}}%
\pgfpathlineto{\pgfqpoint{2.826346in}{2.485142in}}%
\pgfpathlineto{\pgfqpoint{2.838559in}{2.495006in}}%
\pgfpathlineto{\pgfqpoint{2.850772in}{2.445687in}}%
\pgfpathlineto{\pgfqpoint{2.862984in}{2.406232in}}%
\pgfpathlineto{\pgfqpoint{2.875197in}{2.317459in}}%
\pgfpathlineto{\pgfqpoint{2.887410in}{2.327323in}}%
\pgfpathlineto{\pgfqpoint{2.899623in}{2.238549in}}%
\pgfpathlineto{\pgfqpoint{2.911835in}{2.258276in}}%
\pgfpathlineto{\pgfqpoint{2.924048in}{2.179366in}}%
\pgfpathlineto{\pgfqpoint{2.936261in}{2.149775in}}%
\pgfpathlineto{\pgfqpoint{2.960686in}{2.169503in}}%
\pgfpathlineto{\pgfqpoint{2.972899in}{2.061002in}}%
\pgfpathlineto{\pgfqpoint{2.985112in}{1.903182in}}%
\pgfpathlineto{\pgfqpoint{2.997325in}{1.671194in}}%
\pgfpathlineto{\pgfqpoint{3.009538in}{1.622011in}}%
\pgfpathlineto{\pgfqpoint{3.021750in}{1.926947in}}%
\pgfpathlineto{\pgfqpoint{3.033963in}{2.182700in}}%
\pgfpathlineto{\pgfqpoint{3.046176in}{2.349922in}}%
\pgfpathlineto{\pgfqpoint{3.058389in}{2.379432in}}%
\pgfpathlineto{\pgfqpoint{3.082814in}{2.458125in}}%
\pgfpathlineto{\pgfqpoint{3.095027in}{2.428616in}}%
\pgfpathlineto{\pgfqpoint{3.119452in}{2.615512in}}%
\pgfpathlineto{\pgfqpoint{3.131665in}{2.546655in}}%
\pgfpathlineto{\pgfqpoint{3.143878in}{2.642962in}}%
\pgfpathlineto{\pgfqpoint{3.156091in}{2.670564in}}%
\pgfpathlineto{\pgfqpoint{3.168303in}{2.670564in}}%
\pgfpathlineto{\pgfqpoint{3.180516in}{2.680455in}}%
\pgfpathlineto{\pgfqpoint{3.192729in}{2.759583in}}%
\pgfpathlineto{\pgfqpoint{3.204942in}{2.779365in}}%
\pgfpathlineto{\pgfqpoint{3.217155in}{2.769474in}}%
\pgfpathlineto{\pgfqpoint{3.229367in}{2.799147in}}%
\pgfpathlineto{\pgfqpoint{3.241580in}{2.799147in}}%
\pgfpathlineto{\pgfqpoint{3.253793in}{2.848602in}}%
\pgfpathlineto{\pgfqpoint{3.266006in}{2.878275in}}%
\pgfpathlineto{\pgfqpoint{3.267682in}{2.875560in}}%
\pgfpathlineto{\pgfqpoint{3.267682in}{2.875560in}}%
\pgfusepath{stroke}%
\end{pgfscope}%
\begin{pgfscope}%
\pgfpathrectangle{\pgfqpoint{0.566985in}{0.528177in}}{\pgfqpoint{2.686808in}{2.864429in}} %
\pgfusepath{clip}%
\pgfsetroundcap%
\pgfsetroundjoin%
\pgfsetlinewidth{1.756562pt}%
\definecolor{currentstroke}{rgb}{1.000000,0.494118,0.250980}%
\pgfsetstrokecolor{currentstroke}%
\pgfsetstrokeopacity{0.800000}%
\pgfsetdash{}{0pt}%
\pgfpathmoveto{\pgfqpoint{0.566985in}{2.831251in}}%
\pgfpathlineto{\pgfqpoint{0.579197in}{2.846423in}}%
\pgfpathlineto{\pgfqpoint{0.603623in}{2.846423in}}%
\pgfpathlineto{\pgfqpoint{0.615836in}{2.740220in}}%
\pgfpathlineto{\pgfqpoint{0.628048in}{2.725049in}}%
\pgfpathlineto{\pgfqpoint{0.640261in}{2.725049in}}%
\pgfpathlineto{\pgfqpoint{0.652474in}{2.785736in}}%
\pgfpathlineto{\pgfqpoint{0.664687in}{2.709877in}}%
\pgfpathlineto{\pgfqpoint{0.676899in}{2.725049in}}%
\pgfpathlineto{\pgfqpoint{0.689112in}{2.694705in}}%
\pgfpathlineto{\pgfqpoint{0.701325in}{2.725049in}}%
\pgfpathlineto{\pgfqpoint{0.713538in}{2.770564in}}%
\pgfpathlineto{\pgfqpoint{0.725751in}{2.709877in}}%
\pgfpathlineto{\pgfqpoint{0.737963in}{2.755392in}}%
\pgfpathlineto{\pgfqpoint{0.750176in}{2.649190in}}%
\pgfpathlineto{\pgfqpoint{0.762389in}{2.618846in}}%
\pgfpathlineto{\pgfqpoint{0.774602in}{2.634018in}}%
\pgfpathlineto{\pgfqpoint{0.786814in}{2.603674in}}%
\pgfpathlineto{\pgfqpoint{0.799027in}{2.679533in}}%
\pgfpathlineto{\pgfqpoint{0.811240in}{2.740220in}}%
\pgfpathlineto{\pgfqpoint{0.823453in}{2.664361in}}%
\pgfpathlineto{\pgfqpoint{0.835665in}{2.679533in}}%
\pgfpathlineto{\pgfqpoint{0.847878in}{2.725049in}}%
\pgfpathlineto{\pgfqpoint{0.860091in}{2.694705in}}%
\pgfpathlineto{\pgfqpoint{0.872304in}{2.679533in}}%
\pgfpathlineto{\pgfqpoint{0.884516in}{2.709877in}}%
\pgfpathlineto{\pgfqpoint{0.908942in}{2.709877in}}%
\pgfpathlineto{\pgfqpoint{0.921155in}{2.740220in}}%
\pgfpathlineto{\pgfqpoint{0.933368in}{2.755392in}}%
\pgfpathlineto{\pgfqpoint{0.945580in}{2.816079in}}%
\pgfpathlineto{\pgfqpoint{0.957793in}{2.831251in}}%
\pgfpathlineto{\pgfqpoint{0.970006in}{2.861594in}}%
\pgfpathlineto{\pgfqpoint{0.982219in}{2.816079in}}%
\pgfpathlineto{\pgfqpoint{0.994431in}{2.846423in}}%
\pgfpathlineto{\pgfqpoint{1.018857in}{2.876766in}}%
\pgfpathlineto{\pgfqpoint{1.031070in}{2.937453in}}%
\pgfpathlineto{\pgfqpoint{1.043282in}{2.952625in}}%
\pgfpathlineto{\pgfqpoint{1.055495in}{2.937453in}}%
\pgfpathlineto{\pgfqpoint{1.067708in}{2.891938in}}%
\pgfpathlineto{\pgfqpoint{1.079921in}{2.891938in}}%
\pgfpathlineto{\pgfqpoint{1.092133in}{2.937453in}}%
\pgfpathlineto{\pgfqpoint{1.104346in}{2.952625in}}%
\pgfpathlineto{\pgfqpoint{1.116559in}{2.876766in}}%
\pgfpathlineto{\pgfqpoint{1.128772in}{2.922281in}}%
\pgfpathlineto{\pgfqpoint{1.140985in}{2.800907in}}%
\pgfpathlineto{\pgfqpoint{1.153197in}{2.846423in}}%
\pgfpathlineto{\pgfqpoint{1.165410in}{2.831251in}}%
\pgfpathlineto{\pgfqpoint{1.177623in}{2.846423in}}%
\pgfpathlineto{\pgfqpoint{1.189836in}{2.876766in}}%
\pgfpathlineto{\pgfqpoint{1.202048in}{2.846423in}}%
\pgfpathlineto{\pgfqpoint{1.214261in}{2.861594in}}%
\pgfpathlineto{\pgfqpoint{1.226474in}{2.816079in}}%
\pgfpathlineto{\pgfqpoint{1.238687in}{2.785736in}}%
\pgfpathlineto{\pgfqpoint{1.250899in}{2.740220in}}%
\pgfpathlineto{\pgfqpoint{1.263112in}{2.725049in}}%
\pgfpathlineto{\pgfqpoint{1.299750in}{2.770564in}}%
\pgfpathlineto{\pgfqpoint{1.311963in}{2.725049in}}%
\pgfpathlineto{\pgfqpoint{1.324176in}{2.664361in}}%
\pgfpathlineto{\pgfqpoint{1.336389in}{2.664361in}}%
\pgfpathlineto{\pgfqpoint{1.348602in}{2.634018in}}%
\pgfpathlineto{\pgfqpoint{1.360814in}{2.634018in}}%
\pgfpathlineto{\pgfqpoint{1.373027in}{2.649190in}}%
\pgfpathlineto{\pgfqpoint{1.397453in}{2.618846in}}%
\pgfpathlineto{\pgfqpoint{1.409665in}{2.649190in}}%
\pgfpathlineto{\pgfqpoint{1.421878in}{2.588503in}}%
\pgfpathlineto{\pgfqpoint{1.434091in}{2.634018in}}%
\pgfpathlineto{\pgfqpoint{1.446304in}{2.649190in}}%
\pgfpathlineto{\pgfqpoint{1.458516in}{2.740220in}}%
\pgfpathlineto{\pgfqpoint{1.470729in}{2.634018in}}%
\pgfpathlineto{\pgfqpoint{1.482942in}{2.725049in}}%
\pgfpathlineto{\pgfqpoint{1.495155in}{2.725049in}}%
\pgfpathlineto{\pgfqpoint{1.507367in}{2.755392in}}%
\pgfpathlineto{\pgfqpoint{1.519580in}{2.725049in}}%
\pgfpathlineto{\pgfqpoint{1.531793in}{2.755392in}}%
\pgfpathlineto{\pgfqpoint{1.544006in}{2.634018in}}%
\pgfpathlineto{\pgfqpoint{1.556219in}{2.755392in}}%
\pgfpathlineto{\pgfqpoint{1.568431in}{2.785736in}}%
\pgfpathlineto{\pgfqpoint{1.580644in}{2.846423in}}%
\pgfpathlineto{\pgfqpoint{1.617282in}{2.891938in}}%
\pgfpathlineto{\pgfqpoint{1.629495in}{2.891938in}}%
\pgfpathlineto{\pgfqpoint{1.641708in}{2.861594in}}%
\pgfpathlineto{\pgfqpoint{1.653921in}{2.876766in}}%
\pgfpathlineto{\pgfqpoint{1.678346in}{2.846423in}}%
\pgfpathlineto{\pgfqpoint{1.690559in}{2.861594in}}%
\pgfpathlineto{\pgfqpoint{1.702772in}{2.831251in}}%
\pgfpathlineto{\pgfqpoint{1.714984in}{2.755392in}}%
\pgfpathlineto{\pgfqpoint{1.727197in}{2.709877in}}%
\pgfpathlineto{\pgfqpoint{1.739410in}{2.694705in}}%
\pgfpathlineto{\pgfqpoint{1.751623in}{2.603674in}}%
\pgfpathlineto{\pgfqpoint{1.776048in}{2.603674in}}%
\pgfpathlineto{\pgfqpoint{1.788261in}{2.588503in}}%
\pgfpathlineto{\pgfqpoint{1.800474in}{2.558159in}}%
\pgfpathlineto{\pgfqpoint{1.812687in}{2.618846in}}%
\pgfpathlineto{\pgfqpoint{1.824899in}{2.618846in}}%
\pgfpathlineto{\pgfqpoint{1.837112in}{2.664361in}}%
\pgfpathlineto{\pgfqpoint{1.849325in}{2.649190in}}%
\pgfpathlineto{\pgfqpoint{1.861538in}{2.603674in}}%
\pgfpathlineto{\pgfqpoint{1.873750in}{2.634018in}}%
\pgfpathlineto{\pgfqpoint{1.885963in}{2.618846in}}%
\pgfpathlineto{\pgfqpoint{1.898176in}{2.649190in}}%
\pgfpathlineto{\pgfqpoint{1.910389in}{2.573331in}}%
\pgfpathlineto{\pgfqpoint{1.922601in}{2.558159in}}%
\pgfpathlineto{\pgfqpoint{1.934814in}{2.649190in}}%
\pgfpathlineto{\pgfqpoint{1.947027in}{2.603674in}}%
\pgfpathlineto{\pgfqpoint{1.959240in}{2.664361in}}%
\pgfpathlineto{\pgfqpoint{1.971453in}{2.679533in}}%
\pgfpathlineto{\pgfqpoint{1.983665in}{2.709877in}}%
\pgfpathlineto{\pgfqpoint{1.995878in}{2.770564in}}%
\pgfpathlineto{\pgfqpoint{2.020304in}{2.740220in}}%
\pgfpathlineto{\pgfqpoint{2.032516in}{2.709877in}}%
\pgfpathlineto{\pgfqpoint{2.044729in}{2.694705in}}%
\pgfpathlineto{\pgfqpoint{2.056942in}{2.740220in}}%
\pgfpathlineto{\pgfqpoint{2.069155in}{2.755392in}}%
\pgfpathlineto{\pgfqpoint{2.081367in}{2.740220in}}%
\pgfpathlineto{\pgfqpoint{2.093580in}{2.618846in}}%
\pgfpathlineto{\pgfqpoint{2.105793in}{2.694705in}}%
\pgfpathlineto{\pgfqpoint{2.118006in}{2.649190in}}%
\pgfpathlineto{\pgfqpoint{2.130218in}{2.725049in}}%
\pgfpathlineto{\pgfqpoint{2.142431in}{2.755392in}}%
\pgfpathlineto{\pgfqpoint{2.154644in}{2.755392in}}%
\pgfpathlineto{\pgfqpoint{2.166857in}{2.740220in}}%
\pgfpathlineto{\pgfqpoint{2.179070in}{2.679533in}}%
\pgfpathlineto{\pgfqpoint{2.191282in}{2.846423in}}%
\pgfpathlineto{\pgfqpoint{2.203495in}{2.740220in}}%
\pgfpathlineto{\pgfqpoint{2.227921in}{2.709877in}}%
\pgfpathlineto{\pgfqpoint{2.240133in}{2.755392in}}%
\pgfpathlineto{\pgfqpoint{2.264559in}{2.876766in}}%
\pgfpathlineto{\pgfqpoint{2.276772in}{2.891938in}}%
\pgfpathlineto{\pgfqpoint{2.288984in}{2.861594in}}%
\pgfpathlineto{\pgfqpoint{2.301197in}{2.891938in}}%
\pgfpathlineto{\pgfqpoint{2.313410in}{2.861594in}}%
\pgfpathlineto{\pgfqpoint{2.325623in}{2.846423in}}%
\pgfpathlineto{\pgfqpoint{2.337835in}{2.846423in}}%
\pgfpathlineto{\pgfqpoint{2.350048in}{2.816079in}}%
\pgfpathlineto{\pgfqpoint{2.362261in}{2.770564in}}%
\pgfpathlineto{\pgfqpoint{2.374474in}{2.785736in}}%
\pgfpathlineto{\pgfqpoint{2.386687in}{2.785736in}}%
\pgfpathlineto{\pgfqpoint{2.398899in}{2.816079in}}%
\pgfpathlineto{\pgfqpoint{2.411112in}{2.891938in}}%
\pgfpathlineto{\pgfqpoint{2.423325in}{2.816079in}}%
\pgfpathlineto{\pgfqpoint{2.435538in}{2.861594in}}%
\pgfpathlineto{\pgfqpoint{2.447750in}{2.770564in}}%
\pgfpathlineto{\pgfqpoint{2.459963in}{2.755392in}}%
\pgfpathlineto{\pgfqpoint{2.472176in}{2.755392in}}%
\pgfpathlineto{\pgfqpoint{2.484389in}{2.740220in}}%
\pgfpathlineto{\pgfqpoint{2.496601in}{2.755392in}}%
\pgfpathlineto{\pgfqpoint{2.521027in}{2.725049in}}%
\pgfpathlineto{\pgfqpoint{2.533240in}{2.740220in}}%
\pgfpathlineto{\pgfqpoint{2.545452in}{2.664361in}}%
\pgfpathlineto{\pgfqpoint{2.557665in}{2.679533in}}%
\pgfpathlineto{\pgfqpoint{2.569878in}{2.725049in}}%
\pgfpathlineto{\pgfqpoint{2.582091in}{2.709877in}}%
\pgfpathlineto{\pgfqpoint{2.594304in}{2.755392in}}%
\pgfpathlineto{\pgfqpoint{2.606516in}{2.785736in}}%
\pgfpathlineto{\pgfqpoint{2.618729in}{2.740220in}}%
\pgfpathlineto{\pgfqpoint{2.630942in}{2.755392in}}%
\pgfpathlineto{\pgfqpoint{2.643155in}{2.831251in}}%
\pgfpathlineto{\pgfqpoint{2.655367in}{2.785736in}}%
\pgfpathlineto{\pgfqpoint{2.667580in}{2.816079in}}%
\pgfpathlineto{\pgfqpoint{2.679793in}{2.876766in}}%
\pgfpathlineto{\pgfqpoint{2.692006in}{2.876766in}}%
\pgfpathlineto{\pgfqpoint{2.704218in}{2.937453in}}%
\pgfpathlineto{\pgfqpoint{2.716431in}{2.982969in}}%
\pgfpathlineto{\pgfqpoint{2.728644in}{3.073999in}}%
\pgfpathlineto{\pgfqpoint{2.740857in}{3.119514in}}%
\pgfpathlineto{\pgfqpoint{2.753069in}{3.104343in}}%
\pgfpathlineto{\pgfqpoint{2.777495in}{3.043656in}}%
\pgfpathlineto{\pgfqpoint{2.789708in}{3.058827in}}%
\pgfpathlineto{\pgfqpoint{2.801921in}{3.104343in}}%
\pgfpathlineto{\pgfqpoint{2.814133in}{3.165030in}}%
\pgfpathlineto{\pgfqpoint{2.826346in}{3.104343in}}%
\pgfpathlineto{\pgfqpoint{2.838559in}{3.134686in}}%
\pgfpathlineto{\pgfqpoint{2.850772in}{3.119514in}}%
\pgfpathlineto{\pgfqpoint{2.862984in}{3.119514in}}%
\pgfpathlineto{\pgfqpoint{2.875197in}{3.104343in}}%
\pgfpathlineto{\pgfqpoint{2.887410in}{3.119514in}}%
\pgfpathlineto{\pgfqpoint{2.899623in}{3.104343in}}%
\pgfpathlineto{\pgfqpoint{2.911835in}{3.073999in}}%
\pgfpathlineto{\pgfqpoint{2.924048in}{3.119514in}}%
\pgfpathlineto{\pgfqpoint{2.936261in}{3.073999in}}%
\pgfpathlineto{\pgfqpoint{2.948474in}{2.967797in}}%
\pgfpathlineto{\pgfqpoint{2.960686in}{3.028484in}}%
\pgfpathlineto{\pgfqpoint{2.972899in}{3.028484in}}%
\pgfpathlineto{\pgfqpoint{2.985112in}{2.785736in}}%
\pgfpathlineto{\pgfqpoint{2.997325in}{2.497472in}}%
\pgfpathlineto{\pgfqpoint{3.009538in}{0.692032in}}%
\pgfpathlineto{\pgfqpoint{3.021750in}{0.828578in}}%
\pgfpathlineto{\pgfqpoint{3.033963in}{1.056154in}}%
\pgfpathlineto{\pgfqpoint{3.046176in}{1.101670in}}%
\pgfpathlineto{\pgfqpoint{3.058389in}{1.116841in}}%
\pgfpathlineto{\pgfqpoint{3.070601in}{1.268559in}}%
\pgfpathlineto{\pgfqpoint{3.082814in}{1.238215in}}%
\pgfpathlineto{\pgfqpoint{3.095027in}{1.253387in}}%
\pgfpathlineto{\pgfqpoint{3.107240in}{1.405105in}}%
\pgfpathlineto{\pgfqpoint{3.119452in}{1.526479in}}%
\pgfpathlineto{\pgfqpoint{3.131665in}{1.617510in}}%
\pgfpathlineto{\pgfqpoint{3.143878in}{1.617510in}}%
\pgfpathlineto{\pgfqpoint{3.156091in}{1.693368in}}%
\pgfpathlineto{\pgfqpoint{3.168303in}{1.875430in}}%
\pgfpathlineto{\pgfqpoint{3.180516in}{1.845086in}}%
\pgfpathlineto{\pgfqpoint{3.192729in}{1.829914in}}%
\pgfpathlineto{\pgfqpoint{3.204942in}{1.905773in}}%
\pgfpathlineto{\pgfqpoint{3.217155in}{1.920945in}}%
\pgfpathlineto{\pgfqpoint{3.229367in}{2.087834in}}%
\pgfpathlineto{\pgfqpoint{3.241580in}{2.163693in}}%
\pgfpathlineto{\pgfqpoint{3.266006in}{2.254724in}}%
\pgfpathlineto{\pgfqpoint{3.267682in}{2.248477in}}%
\pgfpathlineto{\pgfqpoint{3.267682in}{2.248477in}}%
\pgfusepath{stroke}%
\end{pgfscope}%
\begin{pgfscope}%
\pgfpathrectangle{\pgfqpoint{0.566985in}{0.528177in}}{\pgfqpoint{2.686808in}{2.864429in}} %
\pgfusepath{clip}%
\pgfsetroundcap%
\pgfsetroundjoin%
\pgfsetlinewidth{1.756562pt}%
\definecolor{currentstroke}{rgb}{1.000000,0.694118,0.250980}%
\pgfsetstrokecolor{currentstroke}%
\pgfsetstrokeopacity{0.800000}%
\pgfsetdash{}{0pt}%
\pgfpathmoveto{\pgfqpoint{0.566985in}{2.813402in}}%
\pgfpathlineto{\pgfqpoint{0.579197in}{2.739685in}}%
\pgfpathlineto{\pgfqpoint{0.591410in}{2.802871in}}%
\pgfpathlineto{\pgfqpoint{0.603623in}{2.844995in}}%
\pgfpathlineto{\pgfqpoint{0.615836in}{2.792340in}}%
\pgfpathlineto{\pgfqpoint{0.628048in}{2.760747in}}%
\pgfpathlineto{\pgfqpoint{0.640261in}{2.750216in}}%
\pgfpathlineto{\pgfqpoint{0.652474in}{2.771278in}}%
\pgfpathlineto{\pgfqpoint{0.664687in}{2.771278in}}%
\pgfpathlineto{\pgfqpoint{0.676899in}{2.802871in}}%
\pgfpathlineto{\pgfqpoint{0.689112in}{2.655437in}}%
\pgfpathlineto{\pgfqpoint{0.701325in}{2.781809in}}%
\pgfpathlineto{\pgfqpoint{0.713538in}{2.708092in}}%
\pgfpathlineto{\pgfqpoint{0.725751in}{2.813402in}}%
\pgfpathlineto{\pgfqpoint{0.737963in}{2.792340in}}%
\pgfpathlineto{\pgfqpoint{0.750176in}{2.739685in}}%
\pgfpathlineto{\pgfqpoint{0.762389in}{2.792340in}}%
\pgfpathlineto{\pgfqpoint{0.774602in}{2.729154in}}%
\pgfpathlineto{\pgfqpoint{0.786814in}{2.760747in}}%
\pgfpathlineto{\pgfqpoint{0.799027in}{2.760747in}}%
\pgfpathlineto{\pgfqpoint{0.811240in}{2.792340in}}%
\pgfpathlineto{\pgfqpoint{0.823453in}{2.844995in}}%
\pgfpathlineto{\pgfqpoint{0.835665in}{2.867601in}}%
\pgfpathlineto{\pgfqpoint{0.847878in}{2.869136in}}%
\pgfpathlineto{\pgfqpoint{0.860091in}{2.900544in}}%
\pgfpathlineto{\pgfqpoint{0.872304in}{2.911014in}}%
\pgfpathlineto{\pgfqpoint{0.884516in}{2.806320in}}%
\pgfpathlineto{\pgfqpoint{0.896729in}{2.806320in}}%
\pgfpathlineto{\pgfqpoint{0.908942in}{2.837728in}}%
\pgfpathlineto{\pgfqpoint{0.921155in}{2.816789in}}%
\pgfpathlineto{\pgfqpoint{0.933368in}{2.879605in}}%
\pgfpathlineto{\pgfqpoint{0.945580in}{2.816789in}}%
\pgfpathlineto{\pgfqpoint{0.957793in}{2.827258in}}%
\pgfpathlineto{\pgfqpoint{0.970006in}{2.670217in}}%
\pgfpathlineto{\pgfqpoint{0.982219in}{2.816789in}}%
\pgfpathlineto{\pgfqpoint{0.994431in}{2.879605in}}%
\pgfpathlineto{\pgfqpoint{1.006644in}{2.753972in}}%
\pgfpathlineto{\pgfqpoint{1.018857in}{2.743503in}}%
\pgfpathlineto{\pgfqpoint{1.031070in}{2.848197in}}%
\pgfpathlineto{\pgfqpoint{1.043282in}{2.816789in}}%
\pgfpathlineto{\pgfqpoint{1.055495in}{2.753972in}}%
\pgfpathlineto{\pgfqpoint{1.067708in}{2.712095in}}%
\pgfpathlineto{\pgfqpoint{1.079921in}{2.743503in}}%
\pgfpathlineto{\pgfqpoint{1.092133in}{2.795850in}}%
\pgfpathlineto{\pgfqpoint{1.104346in}{2.827258in}}%
\pgfpathlineto{\pgfqpoint{1.116559in}{2.670217in}}%
\pgfpathlineto{\pgfqpoint{1.128772in}{2.816789in}}%
\pgfpathlineto{\pgfqpoint{1.140985in}{2.733034in}}%
\pgfpathlineto{\pgfqpoint{1.153197in}{2.701625in}}%
\pgfpathlineto{\pgfqpoint{1.165410in}{2.722564in}}%
\pgfpathlineto{\pgfqpoint{1.177623in}{2.607401in}}%
\pgfpathlineto{\pgfqpoint{1.189836in}{2.701625in}}%
\pgfpathlineto{\pgfqpoint{1.202048in}{2.607401in}}%
\pgfpathlineto{\pgfqpoint{1.214261in}{2.670217in}}%
\pgfpathlineto{\pgfqpoint{1.226474in}{2.607401in}}%
\pgfpathlineto{\pgfqpoint{1.238687in}{2.596931in}}%
\pgfpathlineto{\pgfqpoint{1.250899in}{2.575993in}}%
\pgfpathlineto{\pgfqpoint{1.263112in}{2.617870in}}%
\pgfpathlineto{\pgfqpoint{1.275325in}{2.628340in}}%
\pgfpathlineto{\pgfqpoint{1.287538in}{2.628340in}}%
\pgfpathlineto{\pgfqpoint{1.299750in}{2.575993in}}%
\pgfpathlineto{\pgfqpoint{1.311963in}{2.575993in}}%
\pgfpathlineto{\pgfqpoint{1.324176in}{2.733034in}}%
\pgfpathlineto{\pgfqpoint{1.336389in}{2.743503in}}%
\pgfpathlineto{\pgfqpoint{1.348602in}{2.649278in}}%
\pgfpathlineto{\pgfqpoint{1.360814in}{2.670217in}}%
\pgfpathlineto{\pgfqpoint{1.373027in}{2.575993in}}%
\pgfpathlineto{\pgfqpoint{1.385240in}{2.649278in}}%
\pgfpathlineto{\pgfqpoint{1.397453in}{2.659748in}}%
\pgfpathlineto{\pgfqpoint{1.409665in}{2.733034in}}%
\pgfpathlineto{\pgfqpoint{1.421878in}{2.753972in}}%
\pgfpathlineto{\pgfqpoint{1.434091in}{2.659748in}}%
\pgfpathlineto{\pgfqpoint{1.446304in}{2.712095in}}%
\pgfpathlineto{\pgfqpoint{1.458516in}{2.617870in}}%
\pgfpathlineto{\pgfqpoint{1.470729in}{2.620129in}}%
\pgfpathlineto{\pgfqpoint{1.482942in}{2.724518in}}%
\pgfpathlineto{\pgfqpoint{1.495155in}{2.766273in}}%
\pgfpathlineto{\pgfqpoint{1.507367in}{2.766273in}}%
\pgfpathlineto{\pgfqpoint{1.519580in}{2.651446in}}%
\pgfpathlineto{\pgfqpoint{1.531793in}{2.745396in}}%
\pgfpathlineto{\pgfqpoint{1.544006in}{2.693201in}}%
\pgfpathlineto{\pgfqpoint{1.556219in}{2.766273in}}%
\pgfpathlineto{\pgfqpoint{1.568431in}{2.797590in}}%
\pgfpathlineto{\pgfqpoint{1.580644in}{2.808029in}}%
\pgfpathlineto{\pgfqpoint{1.592857in}{2.901979in}}%
\pgfpathlineto{\pgfqpoint{1.605070in}{2.839345in}}%
\pgfpathlineto{\pgfqpoint{1.617282in}{2.849784in}}%
\pgfpathlineto{\pgfqpoint{1.629495in}{2.818468in}}%
\pgfpathlineto{\pgfqpoint{1.641708in}{2.808029in}}%
\pgfpathlineto{\pgfqpoint{1.653921in}{2.693201in}}%
\pgfpathlineto{\pgfqpoint{1.666133in}{2.734957in}}%
\pgfpathlineto{\pgfqpoint{1.678346in}{2.714079in}}%
\pgfpathlineto{\pgfqpoint{1.690559in}{2.641007in}}%
\pgfpathlineto{\pgfqpoint{1.702772in}{2.620129in}}%
\pgfpathlineto{\pgfqpoint{1.714984in}{2.818468in}}%
\pgfpathlineto{\pgfqpoint{1.739410in}{2.641007in}}%
\pgfpathlineto{\pgfqpoint{1.751623in}{2.641007in}}%
\pgfpathlineto{\pgfqpoint{1.763836in}{2.682762in}}%
\pgfpathlineto{\pgfqpoint{1.776048in}{2.672323in}}%
\pgfpathlineto{\pgfqpoint{1.788261in}{2.703640in}}%
\pgfpathlineto{\pgfqpoint{1.800474in}{2.714079in}}%
\pgfpathlineto{\pgfqpoint{1.812687in}{2.611966in}}%
\pgfpathlineto{\pgfqpoint{1.824899in}{2.788911in}}%
\pgfpathlineto{\pgfqpoint{1.837112in}{2.632783in}}%
\pgfpathlineto{\pgfqpoint{1.849325in}{2.674417in}}%
\pgfpathlineto{\pgfqpoint{1.861538in}{2.684826in}}%
\pgfpathlineto{\pgfqpoint{1.873750in}{2.716051in}}%
\pgfpathlineto{\pgfqpoint{1.885963in}{2.778503in}}%
\pgfpathlineto{\pgfqpoint{1.898176in}{2.747277in}}%
\pgfpathlineto{\pgfqpoint{1.910389in}{2.684826in}}%
\pgfpathlineto{\pgfqpoint{1.922601in}{2.601557in}}%
\pgfpathlineto{\pgfqpoint{1.934814in}{2.705643in}}%
\pgfpathlineto{\pgfqpoint{1.947027in}{2.716051in}}%
\pgfpathlineto{\pgfqpoint{1.959240in}{2.736868in}}%
\pgfpathlineto{\pgfqpoint{1.971453in}{2.674417in}}%
\pgfpathlineto{\pgfqpoint{1.983665in}{2.757685in}}%
\pgfpathlineto{\pgfqpoint{1.995878in}{2.778503in}}%
\pgfpathlineto{\pgfqpoint{2.008091in}{2.768094in}}%
\pgfpathlineto{\pgfqpoint{2.020304in}{2.716051in}}%
\pgfpathlineto{\pgfqpoint{2.032516in}{2.601557in}}%
\pgfpathlineto{\pgfqpoint{2.044729in}{2.705643in}}%
\pgfpathlineto{\pgfqpoint{2.056942in}{2.695234in}}%
\pgfpathlineto{\pgfqpoint{2.069155in}{2.570332in}}%
\pgfpathlineto{\pgfqpoint{2.081367in}{2.778503in}}%
\pgfpathlineto{\pgfqpoint{2.093580in}{2.716051in}}%
\pgfpathlineto{\pgfqpoint{2.105793in}{2.684826in}}%
\pgfpathlineto{\pgfqpoint{2.118006in}{2.747277in}}%
\pgfpathlineto{\pgfqpoint{2.130218in}{2.643192in}}%
\pgfpathlineto{\pgfqpoint{2.142431in}{2.757685in}}%
\pgfpathlineto{\pgfqpoint{2.154644in}{2.601557in}}%
\pgfpathlineto{\pgfqpoint{2.166857in}{2.653600in}}%
\pgfpathlineto{\pgfqpoint{2.179070in}{2.736868in}}%
\pgfpathlineto{\pgfqpoint{2.191282in}{2.674417in}}%
\pgfpathlineto{\pgfqpoint{2.203495in}{2.736868in}}%
\pgfpathlineto{\pgfqpoint{2.227921in}{2.695234in}}%
\pgfpathlineto{\pgfqpoint{2.240133in}{2.799320in}}%
\pgfpathlineto{\pgfqpoint{2.252346in}{2.799320in}}%
\pgfpathlineto{\pgfqpoint{2.264559in}{2.726460in}}%
\pgfpathlineto{\pgfqpoint{2.276772in}{2.726460in}}%
\pgfpathlineto{\pgfqpoint{2.288984in}{2.757685in}}%
\pgfpathlineto{\pgfqpoint{2.301197in}{2.653600in}}%
\pgfpathlineto{\pgfqpoint{2.313410in}{2.664009in}}%
\pgfpathlineto{\pgfqpoint{2.325623in}{2.716051in}}%
\pgfpathlineto{\pgfqpoint{2.350048in}{2.736868in}}%
\pgfpathlineto{\pgfqpoint{2.362261in}{2.736868in}}%
\pgfpathlineto{\pgfqpoint{2.374474in}{2.747277in}}%
\pgfpathlineto{\pgfqpoint{2.386687in}{2.747277in}}%
\pgfpathlineto{\pgfqpoint{2.398899in}{2.788911in}}%
\pgfpathlineto{\pgfqpoint{2.411112in}{2.705643in}}%
\pgfpathlineto{\pgfqpoint{2.423325in}{2.716051in}}%
\pgfpathlineto{\pgfqpoint{2.435538in}{2.674417in}}%
\pgfpathlineto{\pgfqpoint{2.447750in}{2.747277in}}%
\pgfpathlineto{\pgfqpoint{2.459963in}{2.695234in}}%
\pgfpathlineto{\pgfqpoint{2.472176in}{2.705643in}}%
\pgfpathlineto{\pgfqpoint{2.484389in}{2.684826in}}%
\pgfpathlineto{\pgfqpoint{2.496601in}{2.705643in}}%
\pgfpathlineto{\pgfqpoint{2.508814in}{2.788911in}}%
\pgfpathlineto{\pgfqpoint{2.521027in}{2.622374in}}%
\pgfpathlineto{\pgfqpoint{2.533240in}{2.747277in}}%
\pgfpathlineto{\pgfqpoint{2.545452in}{2.726460in}}%
\pgfpathlineto{\pgfqpoint{2.557665in}{2.643192in}}%
\pgfpathlineto{\pgfqpoint{2.569878in}{2.705643in}}%
\pgfpathlineto{\pgfqpoint{2.582091in}{2.643192in}}%
\pgfpathlineto{\pgfqpoint{2.594304in}{2.757685in}}%
\pgfpathlineto{\pgfqpoint{2.606516in}{2.664009in}}%
\pgfpathlineto{\pgfqpoint{2.618729in}{2.830545in}}%
\pgfpathlineto{\pgfqpoint{2.630942in}{2.799320in}}%
\pgfpathlineto{\pgfqpoint{2.643155in}{2.799320in}}%
\pgfpathlineto{\pgfqpoint{2.655367in}{2.736868in}}%
\pgfpathlineto{\pgfqpoint{2.667580in}{2.747277in}}%
\pgfpathlineto{\pgfqpoint{2.679793in}{2.726460in}}%
\pgfpathlineto{\pgfqpoint{2.692006in}{2.778503in}}%
\pgfpathlineto{\pgfqpoint{2.704218in}{2.788911in}}%
\pgfpathlineto{\pgfqpoint{2.716431in}{2.736868in}}%
\pgfpathlineto{\pgfqpoint{2.728644in}{2.768094in}}%
\pgfpathlineto{\pgfqpoint{2.740857in}{2.736868in}}%
\pgfpathlineto{\pgfqpoint{2.753069in}{2.747277in}}%
\pgfpathlineto{\pgfqpoint{2.765282in}{2.653600in}}%
\pgfpathlineto{\pgfqpoint{2.777495in}{2.736868in}}%
\pgfpathlineto{\pgfqpoint{2.789708in}{2.736868in}}%
\pgfpathlineto{\pgfqpoint{2.801921in}{2.674417in}}%
\pgfpathlineto{\pgfqpoint{2.814133in}{2.684826in}}%
\pgfpathlineto{\pgfqpoint{2.826346in}{2.778503in}}%
\pgfpathlineto{\pgfqpoint{2.838559in}{2.705643in}}%
\pgfpathlineto{\pgfqpoint{2.862984in}{2.664009in}}%
\pgfpathlineto{\pgfqpoint{2.875197in}{2.716051in}}%
\pgfpathlineto{\pgfqpoint{2.887410in}{2.716051in}}%
\pgfpathlineto{\pgfqpoint{2.899623in}{2.611966in}}%
\pgfpathlineto{\pgfqpoint{2.911835in}{2.601557in}}%
\pgfpathlineto{\pgfqpoint{2.924048in}{2.559923in}}%
\pgfpathlineto{\pgfqpoint{2.936261in}{2.601557in}}%
\pgfpathlineto{\pgfqpoint{2.948474in}{2.601557in}}%
\pgfpathlineto{\pgfqpoint{2.960686in}{2.549515in}}%
\pgfpathlineto{\pgfqpoint{2.972899in}{2.518289in}}%
\pgfpathlineto{\pgfqpoint{2.985112in}{2.164399in}}%
\pgfpathlineto{\pgfqpoint{2.997325in}{2.064175in}}%
\pgfpathlineto{\pgfqpoint{3.009538in}{1.202771in}}%
\pgfpathlineto{\pgfqpoint{3.033963in}{1.669797in}}%
\pgfpathlineto{\pgfqpoint{3.046176in}{1.815094in}}%
\pgfpathlineto{\pgfqpoint{3.058389in}{1.908500in}}%
\pgfpathlineto{\pgfqpoint{3.070601in}{1.825473in}}%
\pgfpathlineto{\pgfqpoint{3.082814in}{1.970770in}}%
\pgfpathlineto{\pgfqpoint{3.095027in}{2.053797in}}%
\pgfpathlineto{\pgfqpoint{3.107240in}{2.240607in}}%
\pgfpathlineto{\pgfqpoint{3.119452in}{2.084932in}}%
\pgfpathlineto{\pgfqpoint{3.131665in}{2.084932in}}%
\pgfpathlineto{\pgfqpoint{3.143878in}{2.240607in}}%
\pgfpathlineto{\pgfqpoint{3.156091in}{2.147202in}}%
\pgfpathlineto{\pgfqpoint{3.168303in}{2.310118in}}%
\pgfpathlineto{\pgfqpoint{3.180516in}{2.289301in}}%
\pgfpathlineto{\pgfqpoint{3.192729in}{2.289301in}}%
\pgfpathlineto{\pgfqpoint{3.204942in}{2.393387in}}%
\pgfpathlineto{\pgfqpoint{3.217155in}{2.455838in}}%
\pgfpathlineto{\pgfqpoint{3.229367in}{2.539106in}}%
\pgfpathlineto{\pgfqpoint{3.241580in}{2.393387in}}%
\pgfpathlineto{\pgfqpoint{3.253793in}{2.341344in}}%
\pgfpathlineto{\pgfqpoint{3.266006in}{2.557496in}}%
\pgfpathlineto{\pgfqpoint{3.267682in}{2.550332in}}%
\pgfpathlineto{\pgfqpoint{3.267682in}{2.550332in}}%
\pgfusepath{stroke}%
\end{pgfscope}%
\begin{pgfscope}%
\pgfpathrectangle{\pgfqpoint{0.566985in}{0.528177in}}{\pgfqpoint{2.686808in}{2.864429in}} %
\pgfusepath{clip}%
\pgfsetbuttcap%
\pgfsetroundjoin%
\pgfsetlinewidth{1.756562pt}%
\definecolor{currentstroke}{rgb}{0.501961,0.501961,0.501961}%
\pgfsetstrokecolor{currentstroke}%
\pgfsetdash{{6.000000pt}{6.000000pt}}{0.000000pt}%
\pgfpathmoveto{\pgfqpoint{3.009538in}{0.528177in}}%
\pgfpathlineto{\pgfqpoint{3.009538in}{3.392606in}}%
\pgfusepath{stroke}%
\end{pgfscope}%
\begin{pgfscope}%
\pgfsetrectcap%
\pgfsetmiterjoin%
\pgfsetlinewidth{1.254687pt}%
\definecolor{currentstroke}{rgb}{0.150000,0.150000,0.150000}%
\pgfsetstrokecolor{currentstroke}%
\pgfsetdash{}{0pt}%
\pgfpathmoveto{\pgfqpoint{0.566985in}{0.528177in}}%
\pgfpathlineto{\pgfqpoint{0.566985in}{3.392606in}}%
\pgfusepath{stroke}%
\end{pgfscope}%
\begin{pgfscope}%
\pgfsetrectcap%
\pgfsetmiterjoin%
\pgfsetlinewidth{1.254687pt}%
\definecolor{currentstroke}{rgb}{0.150000,0.150000,0.150000}%
\pgfsetstrokecolor{currentstroke}%
\pgfsetdash{}{0pt}%
\pgfpathmoveto{\pgfqpoint{0.566985in}{0.528177in}}%
\pgfpathlineto{\pgfqpoint{3.253793in}{0.528177in}}%
\pgfusepath{stroke}%
\end{pgfscope}%
\begin{pgfscope}%
\pgfsetbuttcap%
\pgfsetmiterjoin%
\definecolor{currentfill}{rgb}{1.000000,1.000000,1.000000}%
\pgfsetfillcolor{currentfill}%
\pgfsetlinewidth{0.000000pt}%
\definecolor{currentstroke}{rgb}{0.000000,0.000000,0.000000}%
\pgfsetstrokecolor{currentstroke}%
\pgfsetstrokeopacity{0.000000}%
\pgfsetdash{}{0pt}%
\pgfpathmoveto{\pgfqpoint{3.858325in}{0.528177in}}%
\pgfpathlineto{\pgfqpoint{5.201729in}{0.528177in}}%
\pgfpathlineto{\pgfqpoint{5.201729in}{2.653399in}}%
\pgfpathlineto{\pgfqpoint{3.858325in}{2.653399in}}%
\pgfpathclose%
\pgfusepath{fill}%
\end{pgfscope}%
\begin{pgfscope}%
\pgfsetroundcap%
\pgfsetroundjoin%
\pgfsetlinewidth{1.756562pt}%
\definecolor{currentstroke}{rgb}{0.200000,0.427451,0.650980}%
\pgfsetstrokecolor{currentstroke}%
\pgfsetstrokeopacity{0.800000}%
\pgfsetdash{}{0pt}%
\pgfpathmoveto{\pgfqpoint{3.689644in}{3.278686in}}%
\pgfpathlineto{\pgfqpoint{3.800755in}{3.278686in}}%
\pgfusepath{stroke}%
\end{pgfscope}%
\begin{pgfscope}%
\definecolor{textcolor}{rgb}{1.000000,1.000000,1.000000}%
\pgfsetstrokecolor{textcolor}%
\pgfsetfillcolor{textcolor}%
\pgftext[x=3.889644in,y=3.239798in,left,base]{\color{textcolor}\rmfamily\fontsize{8.000000}{9.600000}\selectfont WT + Vehicle}%
\end{pgfscope}%
\begin{pgfscope}%
\pgfsetroundcap%
\pgfsetroundjoin%
\pgfsetlinewidth{1.756562pt}%
\definecolor{currentstroke}{rgb}{0.168627,0.670588,0.494118}%
\pgfsetstrokecolor{currentstroke}%
\pgfsetstrokeopacity{0.800000}%
\pgfsetdash{}{0pt}%
\pgfpathmoveto{\pgfqpoint{3.689644in}{3.123753in}}%
\pgfpathlineto{\pgfqpoint{3.800755in}{3.123753in}}%
\pgfusepath{stroke}%
\end{pgfscope}%
\begin{pgfscope}%
\definecolor{textcolor}{rgb}{1.000000,1.000000,1.000000}%
\pgfsetstrokecolor{textcolor}%
\pgfsetfillcolor{textcolor}%
\pgftext[x=3.889644in,y=3.084864in,left,base]{\color{textcolor}\rmfamily\fontsize{8.000000}{9.600000}\selectfont WT + TAT-GluA2\textsubscript{3Y}}%
\end{pgfscope}%
\begin{pgfscope}%
\pgfsetroundcap%
\pgfsetroundjoin%
\pgfsetlinewidth{1.756562pt}%
\definecolor{currentstroke}{rgb}{1.000000,0.494118,0.250980}%
\pgfsetstrokecolor{currentstroke}%
\pgfsetstrokeopacity{0.800000}%
\pgfsetdash{}{0pt}%
\pgfpathmoveto{\pgfqpoint{3.689644in}{2.968820in}}%
\pgfpathlineto{\pgfqpoint{3.800755in}{2.968820in}}%
\pgfusepath{stroke}%
\end{pgfscope}%
\begin{pgfscope}%
\definecolor{textcolor}{rgb}{1.000000,1.000000,1.000000}%
\pgfsetstrokecolor{textcolor}%
\pgfsetfillcolor{textcolor}%
\pgftext[x=3.889644in,y=2.929931in,left,base]{\color{textcolor}\rmfamily\fontsize{8.000000}{9.600000}\selectfont Tg + Vehicle}%
\end{pgfscope}%
\begin{pgfscope}%
\pgfsetroundcap%
\pgfsetroundjoin%
\pgfsetlinewidth{1.756562pt}%
\definecolor{currentstroke}{rgb}{1.000000,0.694118,0.250980}%
\pgfsetstrokecolor{currentstroke}%
\pgfsetstrokeopacity{0.800000}%
\pgfsetdash{}{0pt}%
\pgfpathmoveto{\pgfqpoint{3.689644in}{2.813887in}}%
\pgfpathlineto{\pgfqpoint{3.800755in}{2.813887in}}%
\pgfusepath{stroke}%
\end{pgfscope}%
\begin{pgfscope}%
\definecolor{textcolor}{rgb}{1.000000,1.000000,1.000000}%
\pgfsetstrokecolor{textcolor}%
\pgfsetfillcolor{textcolor}%
\pgftext[x=3.889644in,y=2.774998in,left,base]{\color{textcolor}\rmfamily\fontsize{8.000000}{9.600000}\selectfont Tg + TAT-GluA2\textsubscript{3Y}}%
\end{pgfscope}%
\begin{pgfscope}%
\pgfsetroundcap%
\pgfsetroundjoin%
\pgfsetlinewidth{1.756562pt}%
\definecolor{currentstroke}{rgb}{0.200000,0.427451,0.650980}%
\pgfsetstrokecolor{currentstroke}%
\pgfsetstrokeopacity{0.800000}%
\pgfsetdash{}{0pt}%
\pgfpathmoveto{\pgfqpoint{3.689644in}{3.278686in}}%
\pgfpathlineto{\pgfqpoint{3.800755in}{3.278686in}}%
\pgfusepath{stroke}%
\end{pgfscope}%
\begin{pgfscope}%
\definecolor{textcolor}{rgb}{1.000000,1.000000,1.000000}%
\pgfsetstrokecolor{textcolor}%
\pgfsetfillcolor{textcolor}%
\pgftext[x=3.889644in,y=3.239798in,left,base]{\color{textcolor}\rmfamily\fontsize{8.000000}{9.600000}\selectfont WT + Vehicle}%
\end{pgfscope}%
\begin{pgfscope}%
\pgfsetroundcap%
\pgfsetroundjoin%
\pgfsetlinewidth{1.756562pt}%
\definecolor{currentstroke}{rgb}{0.168627,0.670588,0.494118}%
\pgfsetstrokecolor{currentstroke}%
\pgfsetstrokeopacity{0.800000}%
\pgfsetdash{}{0pt}%
\pgfpathmoveto{\pgfqpoint{3.689644in}{3.123753in}}%
\pgfpathlineto{\pgfqpoint{3.800755in}{3.123753in}}%
\pgfusepath{stroke}%
\end{pgfscope}%
\begin{pgfscope}%
\definecolor{textcolor}{rgb}{1.000000,1.000000,1.000000}%
\pgfsetstrokecolor{textcolor}%
\pgfsetfillcolor{textcolor}%
\pgftext[x=3.889644in,y=3.084864in,left,base]{\color{textcolor}\rmfamily\fontsize{8.000000}{9.600000}\selectfont WT + TAT-GluA2\textsubscript{3Y}}%
\end{pgfscope}%
\begin{pgfscope}%
\pgfsetroundcap%
\pgfsetroundjoin%
\pgfsetlinewidth{1.756562pt}%
\definecolor{currentstroke}{rgb}{1.000000,0.494118,0.250980}%
\pgfsetstrokecolor{currentstroke}%
\pgfsetstrokeopacity{0.800000}%
\pgfsetdash{}{0pt}%
\pgfpathmoveto{\pgfqpoint{3.689644in}{2.968820in}}%
\pgfpathlineto{\pgfqpoint{3.800755in}{2.968820in}}%
\pgfusepath{stroke}%
\end{pgfscope}%
\begin{pgfscope}%
\definecolor{textcolor}{rgb}{1.000000,1.000000,1.000000}%
\pgfsetstrokecolor{textcolor}%
\pgfsetfillcolor{textcolor}%
\pgftext[x=3.889644in,y=2.929931in,left,base]{\color{textcolor}\rmfamily\fontsize{8.000000}{9.600000}\selectfont Tg + Vehicle}%
\end{pgfscope}%
\begin{pgfscope}%
\pgfsetroundcap%
\pgfsetroundjoin%
\pgfsetlinewidth{1.756562pt}%
\definecolor{currentstroke}{rgb}{1.000000,0.694118,0.250980}%
\pgfsetstrokecolor{currentstroke}%
\pgfsetstrokeopacity{0.800000}%
\pgfsetdash{}{0pt}%
\pgfpathmoveto{\pgfqpoint{3.689644in}{2.813887in}}%
\pgfpathlineto{\pgfqpoint{3.800755in}{2.813887in}}%
\pgfusepath{stroke}%
\end{pgfscope}%
\begin{pgfscope}%
\definecolor{textcolor}{rgb}{1.000000,1.000000,1.000000}%
\pgfsetstrokecolor{textcolor}%
\pgfsetfillcolor{textcolor}%
\pgftext[x=3.889644in,y=2.774998in,left,base]{\color{textcolor}\rmfamily\fontsize{8.000000}{9.600000}\selectfont Tg + TAT-GluA2\textsubscript{3Y}}%
\end{pgfscope}%
\begin{pgfscope}%
\pgfsetbuttcap%
\pgfsetroundjoin%
\definecolor{currentfill}{rgb}{0.150000,0.150000,0.150000}%
\pgfsetfillcolor{currentfill}%
\pgfsetlinewidth{1.003750pt}%
\definecolor{currentstroke}{rgb}{0.150000,0.150000,0.150000}%
\pgfsetstrokecolor{currentstroke}%
\pgfsetdash{}{0pt}%
\pgfsys@defobject{currentmarker}{\pgfqpoint{0.000000in}{0.000000in}}{\pgfqpoint{0.041667in}{0.000000in}}{%
\pgfpathmoveto{\pgfqpoint{0.000000in}{0.000000in}}%
\pgfpathlineto{\pgfqpoint{0.041667in}{0.000000in}}%
\pgfusepath{stroke,fill}%
}%
\begin{pgfscope}%
\pgfsys@transformshift{3.858325in}{0.528177in}%
\pgfsys@useobject{currentmarker}{}%
\end{pgfscope}%
\end{pgfscope}%
\begin{pgfscope}%
\definecolor{textcolor}{rgb}{0.150000,0.150000,0.150000}%
\pgfsetstrokecolor{textcolor}%
\pgfsetfillcolor{textcolor}%
\pgftext[x=3.761102in,y=0.528177in,right,]{\color{textcolor}\rmfamily\fontsize{10.000000}{12.000000}\selectfont \(\displaystyle 0.0\)}%
\end{pgfscope}%
\begin{pgfscope}%
\pgfsetbuttcap%
\pgfsetroundjoin%
\definecolor{currentfill}{rgb}{0.150000,0.150000,0.150000}%
\pgfsetfillcolor{currentfill}%
\pgfsetlinewidth{1.003750pt}%
\definecolor{currentstroke}{rgb}{0.150000,0.150000,0.150000}%
\pgfsetstrokecolor{currentstroke}%
\pgfsetdash{}{0pt}%
\pgfsys@defobject{currentmarker}{\pgfqpoint{0.000000in}{0.000000in}}{\pgfqpoint{0.041667in}{0.000000in}}{%
\pgfpathmoveto{\pgfqpoint{0.000000in}{0.000000in}}%
\pgfpathlineto{\pgfqpoint{0.041667in}{0.000000in}}%
\pgfusepath{stroke,fill}%
}%
\begin{pgfscope}%
\pgfsys@transformshift{3.858325in}{0.831780in}%
\pgfsys@useobject{currentmarker}{}%
\end{pgfscope}%
\end{pgfscope}%
\begin{pgfscope}%
\definecolor{textcolor}{rgb}{0.150000,0.150000,0.150000}%
\pgfsetstrokecolor{textcolor}%
\pgfsetfillcolor{textcolor}%
\pgftext[x=3.761102in,y=0.831780in,right,]{\color{textcolor}\rmfamily\fontsize{10.000000}{12.000000}\selectfont \(\displaystyle 0.5\)}%
\end{pgfscope}%
\begin{pgfscope}%
\pgfsetbuttcap%
\pgfsetroundjoin%
\definecolor{currentfill}{rgb}{0.150000,0.150000,0.150000}%
\pgfsetfillcolor{currentfill}%
\pgfsetlinewidth{1.003750pt}%
\definecolor{currentstroke}{rgb}{0.150000,0.150000,0.150000}%
\pgfsetstrokecolor{currentstroke}%
\pgfsetdash{}{0pt}%
\pgfsys@defobject{currentmarker}{\pgfqpoint{0.000000in}{0.000000in}}{\pgfqpoint{0.041667in}{0.000000in}}{%
\pgfpathmoveto{\pgfqpoint{0.000000in}{0.000000in}}%
\pgfpathlineto{\pgfqpoint{0.041667in}{0.000000in}}%
\pgfusepath{stroke,fill}%
}%
\begin{pgfscope}%
\pgfsys@transformshift{3.858325in}{1.135383in}%
\pgfsys@useobject{currentmarker}{}%
\end{pgfscope}%
\end{pgfscope}%
\begin{pgfscope}%
\definecolor{textcolor}{rgb}{0.150000,0.150000,0.150000}%
\pgfsetstrokecolor{textcolor}%
\pgfsetfillcolor{textcolor}%
\pgftext[x=3.761102in,y=1.135383in,right,]{\color{textcolor}\rmfamily\fontsize{10.000000}{12.000000}\selectfont \(\displaystyle 1.0\)}%
\end{pgfscope}%
\begin{pgfscope}%
\pgfsetbuttcap%
\pgfsetroundjoin%
\definecolor{currentfill}{rgb}{0.150000,0.150000,0.150000}%
\pgfsetfillcolor{currentfill}%
\pgfsetlinewidth{1.003750pt}%
\definecolor{currentstroke}{rgb}{0.150000,0.150000,0.150000}%
\pgfsetstrokecolor{currentstroke}%
\pgfsetdash{}{0pt}%
\pgfsys@defobject{currentmarker}{\pgfqpoint{0.000000in}{0.000000in}}{\pgfqpoint{0.041667in}{0.000000in}}{%
\pgfpathmoveto{\pgfqpoint{0.000000in}{0.000000in}}%
\pgfpathlineto{\pgfqpoint{0.041667in}{0.000000in}}%
\pgfusepath{stroke,fill}%
}%
\begin{pgfscope}%
\pgfsys@transformshift{3.858325in}{1.438986in}%
\pgfsys@useobject{currentmarker}{}%
\end{pgfscope}%
\end{pgfscope}%
\begin{pgfscope}%
\definecolor{textcolor}{rgb}{0.150000,0.150000,0.150000}%
\pgfsetstrokecolor{textcolor}%
\pgfsetfillcolor{textcolor}%
\pgftext[x=3.761102in,y=1.438986in,right,]{\color{textcolor}\rmfamily\fontsize{10.000000}{12.000000}\selectfont \(\displaystyle 1.5\)}%
\end{pgfscope}%
\begin{pgfscope}%
\pgfsetbuttcap%
\pgfsetroundjoin%
\definecolor{currentfill}{rgb}{0.150000,0.150000,0.150000}%
\pgfsetfillcolor{currentfill}%
\pgfsetlinewidth{1.003750pt}%
\definecolor{currentstroke}{rgb}{0.150000,0.150000,0.150000}%
\pgfsetstrokecolor{currentstroke}%
\pgfsetdash{}{0pt}%
\pgfsys@defobject{currentmarker}{\pgfqpoint{0.000000in}{0.000000in}}{\pgfqpoint{0.041667in}{0.000000in}}{%
\pgfpathmoveto{\pgfqpoint{0.000000in}{0.000000in}}%
\pgfpathlineto{\pgfqpoint{0.041667in}{0.000000in}}%
\pgfusepath{stroke,fill}%
}%
\begin{pgfscope}%
\pgfsys@transformshift{3.858325in}{1.742589in}%
\pgfsys@useobject{currentmarker}{}%
\end{pgfscope}%
\end{pgfscope}%
\begin{pgfscope}%
\definecolor{textcolor}{rgb}{0.150000,0.150000,0.150000}%
\pgfsetstrokecolor{textcolor}%
\pgfsetfillcolor{textcolor}%
\pgftext[x=3.761102in,y=1.742589in,right,]{\color{textcolor}\rmfamily\fontsize{10.000000}{12.000000}\selectfont \(\displaystyle 2.0\)}%
\end{pgfscope}%
\begin{pgfscope}%
\pgfsetbuttcap%
\pgfsetroundjoin%
\definecolor{currentfill}{rgb}{0.150000,0.150000,0.150000}%
\pgfsetfillcolor{currentfill}%
\pgfsetlinewidth{1.003750pt}%
\definecolor{currentstroke}{rgb}{0.150000,0.150000,0.150000}%
\pgfsetstrokecolor{currentstroke}%
\pgfsetdash{}{0pt}%
\pgfsys@defobject{currentmarker}{\pgfqpoint{0.000000in}{0.000000in}}{\pgfqpoint{0.041667in}{0.000000in}}{%
\pgfpathmoveto{\pgfqpoint{0.000000in}{0.000000in}}%
\pgfpathlineto{\pgfqpoint{0.041667in}{0.000000in}}%
\pgfusepath{stroke,fill}%
}%
\begin{pgfscope}%
\pgfsys@transformshift{3.858325in}{2.046192in}%
\pgfsys@useobject{currentmarker}{}%
\end{pgfscope}%
\end{pgfscope}%
\begin{pgfscope}%
\definecolor{textcolor}{rgb}{0.150000,0.150000,0.150000}%
\pgfsetstrokecolor{textcolor}%
\pgfsetfillcolor{textcolor}%
\pgftext[x=3.761102in,y=2.046192in,right,]{\color{textcolor}\rmfamily\fontsize{10.000000}{12.000000}\selectfont \(\displaystyle 2.5\)}%
\end{pgfscope}%
\begin{pgfscope}%
\pgfsetbuttcap%
\pgfsetroundjoin%
\definecolor{currentfill}{rgb}{0.150000,0.150000,0.150000}%
\pgfsetfillcolor{currentfill}%
\pgfsetlinewidth{1.003750pt}%
\definecolor{currentstroke}{rgb}{0.150000,0.150000,0.150000}%
\pgfsetstrokecolor{currentstroke}%
\pgfsetdash{}{0pt}%
\pgfsys@defobject{currentmarker}{\pgfqpoint{0.000000in}{0.000000in}}{\pgfqpoint{0.041667in}{0.000000in}}{%
\pgfpathmoveto{\pgfqpoint{0.000000in}{0.000000in}}%
\pgfpathlineto{\pgfqpoint{0.041667in}{0.000000in}}%
\pgfusepath{stroke,fill}%
}%
\begin{pgfscope}%
\pgfsys@transformshift{3.858325in}{2.349796in}%
\pgfsys@useobject{currentmarker}{}%
\end{pgfscope}%
\end{pgfscope}%
\begin{pgfscope}%
\definecolor{textcolor}{rgb}{0.150000,0.150000,0.150000}%
\pgfsetstrokecolor{textcolor}%
\pgfsetfillcolor{textcolor}%
\pgftext[x=3.761102in,y=2.349796in,right,]{\color{textcolor}\rmfamily\fontsize{10.000000}{12.000000}\selectfont \(\displaystyle 3.0\)}%
\end{pgfscope}%
\begin{pgfscope}%
\pgfsetbuttcap%
\pgfsetroundjoin%
\definecolor{currentfill}{rgb}{0.150000,0.150000,0.150000}%
\pgfsetfillcolor{currentfill}%
\pgfsetlinewidth{1.003750pt}%
\definecolor{currentstroke}{rgb}{0.150000,0.150000,0.150000}%
\pgfsetstrokecolor{currentstroke}%
\pgfsetdash{}{0pt}%
\pgfsys@defobject{currentmarker}{\pgfqpoint{0.000000in}{0.000000in}}{\pgfqpoint{0.041667in}{0.000000in}}{%
\pgfpathmoveto{\pgfqpoint{0.000000in}{0.000000in}}%
\pgfpathlineto{\pgfqpoint{0.041667in}{0.000000in}}%
\pgfusepath{stroke,fill}%
}%
\begin{pgfscope}%
\pgfsys@transformshift{3.858325in}{2.653399in}%
\pgfsys@useobject{currentmarker}{}%
\end{pgfscope}%
\end{pgfscope}%
\begin{pgfscope}%
\definecolor{textcolor}{rgb}{0.150000,0.150000,0.150000}%
\pgfsetstrokecolor{textcolor}%
\pgfsetfillcolor{textcolor}%
\pgftext[x=3.761102in,y=2.653399in,right,]{\color{textcolor}\rmfamily\fontsize{10.000000}{12.000000}\selectfont \(\displaystyle 3.5\)}%
\end{pgfscope}%
\begin{pgfscope}%
\definecolor{textcolor}{rgb}{0.150000,0.150000,0.150000}%
\pgfsetstrokecolor{textcolor}%
\pgfsetfillcolor{textcolor}%
\pgftext[x=3.514188in,y=1.590788in,,bottom,rotate=90.000000]{\color{textcolor}\rmfamily\fontsize{10.000000}{12.000000}\selectfont \textbf{Time to freezing (s)}}%
\end{pgfscope}%
\begin{pgfscope}%
\pgfpathrectangle{\pgfqpoint{3.858325in}{0.528177in}}{\pgfqpoint{1.343404in}{2.125222in}} %
\pgfusepath{clip}%
\pgfsetbuttcap%
\pgfsetmiterjoin%
\definecolor{currentfill}{rgb}{0.200000,0.427451,0.650980}%
\pgfsetfillcolor{currentfill}%
\pgfsetlinewidth{1.505625pt}%
\definecolor{currentstroke}{rgb}{0.200000,0.427451,0.650980}%
\pgfsetstrokecolor{currentstroke}%
\pgfsetdash{}{0pt}%
\pgfpathmoveto{\pgfqpoint{3.906303in}{0.528177in}}%
\pgfpathlineto{\pgfqpoint{4.146197in}{0.528177in}}%
\pgfpathlineto{\pgfqpoint{4.146197in}{0.884744in}}%
\pgfpathlineto{\pgfqpoint{3.906303in}{0.884744in}}%
\pgfpathclose%
\pgfusepath{stroke,fill}%
\end{pgfscope}%
\begin{pgfscope}%
\pgfpathrectangle{\pgfqpoint{3.858325in}{0.528177in}}{\pgfqpoint{1.343404in}{2.125222in}} %
\pgfusepath{clip}%
\pgfsetbuttcap%
\pgfsetmiterjoin%
\definecolor{currentfill}{rgb}{0.168627,0.670588,0.494118}%
\pgfsetfillcolor{currentfill}%
\pgfsetlinewidth{1.505625pt}%
\definecolor{currentstroke}{rgb}{0.168627,0.670588,0.494118}%
\pgfsetstrokecolor{currentstroke}%
\pgfsetdash{}{0pt}%
\pgfpathmoveto{\pgfqpoint{4.242154in}{0.528177in}}%
\pgfpathlineto{\pgfqpoint{4.482048in}{0.528177in}}%
\pgfpathlineto{\pgfqpoint{4.482048in}{2.094858in}}%
\pgfpathlineto{\pgfqpoint{4.242154in}{2.094858in}}%
\pgfpathclose%
\pgfusepath{stroke,fill}%
\end{pgfscope}%
\begin{pgfscope}%
\pgfpathrectangle{\pgfqpoint{3.858325in}{0.528177in}}{\pgfqpoint{1.343404in}{2.125222in}} %
\pgfusepath{clip}%
\pgfsetbuttcap%
\pgfsetmiterjoin%
\definecolor{currentfill}{rgb}{1.000000,0.494118,0.250980}%
\pgfsetfillcolor{currentfill}%
\pgfsetlinewidth{1.505625pt}%
\definecolor{currentstroke}{rgb}{1.000000,0.494118,0.250980}%
\pgfsetstrokecolor{currentstroke}%
\pgfsetdash{}{0pt}%
\pgfpathclose%
\pgfusepath{stroke,fill}%
\end{pgfscope}%
\begin{pgfscope}%
\pgfpathrectangle{\pgfqpoint{3.858325in}{0.528177in}}{\pgfqpoint{1.343404in}{2.125222in}} %
\pgfusepath{clip}%
\pgfsetbuttcap%
\pgfsetmiterjoin%
\definecolor{currentfill}{rgb}{1.000000,0.694118,0.250980}%
\pgfsetfillcolor{currentfill}%
\pgfsetlinewidth{1.505625pt}%
\definecolor{currentstroke}{rgb}{1.000000,0.694118,0.250980}%
\pgfsetstrokecolor{currentstroke}%
\pgfsetdash{}{0pt}%
\pgfpathmoveto{\pgfqpoint{4.913856in}{0.528177in}}%
\pgfpathlineto{\pgfqpoint{5.153750in}{0.528177in}}%
\pgfpathlineto{\pgfqpoint{5.153750in}{0.700367in}}%
\pgfpathlineto{\pgfqpoint{4.913856in}{0.700367in}}%
\pgfpathclose%
\pgfusepath{stroke,fill}%
\end{pgfscope}%
\begin{pgfscope}%
\pgfpathrectangle{\pgfqpoint{3.858325in}{0.528177in}}{\pgfqpoint{1.343404in}{2.125222in}} %
\pgfusepath{clip}%
\pgfsetbuttcap%
\pgfsetroundjoin%
\pgfsetlinewidth{1.505625pt}%
\definecolor{currentstroke}{rgb}{0.200000,0.427451,0.650980}%
\pgfsetstrokecolor{currentstroke}%
\pgfsetdash{}{0pt}%
\pgfpathmoveto{\pgfqpoint{4.026250in}{0.922861in}}%
\pgfpathlineto{\pgfqpoint{4.026250in}{0.862140in}}%
\pgfusepath{stroke}%
\end{pgfscope}%
\begin{pgfscope}%
\pgfpathrectangle{\pgfqpoint{3.858325in}{0.528177in}}{\pgfqpoint{1.343404in}{2.125222in}} %
\pgfusepath{clip}%
\pgfsetbuttcap%
\pgfsetroundjoin%
\pgfsetlinewidth{1.505625pt}%
\definecolor{currentstroke}{rgb}{0.168627,0.670588,0.494118}%
\pgfsetstrokecolor{currentstroke}%
\pgfsetdash{}{0pt}%
\pgfpathmoveto{\pgfqpoint{4.362101in}{2.258715in}}%
\pgfpathlineto{\pgfqpoint{4.362101in}{1.955111in}}%
\pgfusepath{stroke}%
\end{pgfscope}%
\begin{pgfscope}%
\pgfpathrectangle{\pgfqpoint{3.858325in}{0.528177in}}{\pgfqpoint{1.343404in}{2.125222in}} %
\pgfusepath{clip}%
\pgfsetbuttcap%
\pgfsetroundjoin%
\pgfsetlinewidth{1.505625pt}%
\definecolor{currentstroke}{rgb}{1.000000,0.494118,0.250980}%
\pgfsetstrokecolor{currentstroke}%
\pgfsetdash{}{0pt}%
\pgfusepath{stroke}%
\end{pgfscope}%
\begin{pgfscope}%
\pgfpathrectangle{\pgfqpoint{3.858325in}{0.528177in}}{\pgfqpoint{1.343404in}{2.125222in}} %
\pgfusepath{clip}%
\pgfsetbuttcap%
\pgfsetroundjoin%
\pgfsetlinewidth{1.505625pt}%
\definecolor{currentstroke}{rgb}{1.000000,0.694118,0.250980}%
\pgfsetstrokecolor{currentstroke}%
\pgfsetdash{}{0pt}%
\pgfpathmoveto{\pgfqpoint{5.033803in}{0.801420in}}%
\pgfpathlineto{\pgfqpoint{5.033803in}{0.649618in}}%
\pgfusepath{stroke}%
\end{pgfscope}%
\begin{pgfscope}%
\pgfpathrectangle{\pgfqpoint{3.858325in}{0.528177in}}{\pgfqpoint{1.343404in}{2.125222in}} %
\pgfusepath{clip}%
\pgfsetbuttcap%
\pgfsetroundjoin%
\definecolor{currentfill}{rgb}{0.200000,0.427451,0.650980}%
\pgfsetfillcolor{currentfill}%
\pgfsetlinewidth{1.505625pt}%
\definecolor{currentstroke}{rgb}{0.200000,0.427451,0.650980}%
\pgfsetstrokecolor{currentstroke}%
\pgfsetdash{}{0pt}%
\pgfsys@defobject{currentmarker}{\pgfqpoint{-0.111111in}{-0.000000in}}{\pgfqpoint{0.111111in}{0.000000in}}{%
\pgfpathmoveto{\pgfqpoint{0.111111in}{-0.000000in}}%
\pgfpathlineto{\pgfqpoint{-0.111111in}{0.000000in}}%
\pgfusepath{stroke,fill}%
}%
\begin{pgfscope}%
\pgfsys@transformshift{4.026250in}{0.922861in}%
\pgfsys@useobject{currentmarker}{}%
\end{pgfscope}%
\end{pgfscope}%
\begin{pgfscope}%
\pgfpathrectangle{\pgfqpoint{3.858325in}{0.528177in}}{\pgfqpoint{1.343404in}{2.125222in}} %
\pgfusepath{clip}%
\pgfsetbuttcap%
\pgfsetroundjoin%
\definecolor{currentfill}{rgb}{0.200000,0.427451,0.650980}%
\pgfsetfillcolor{currentfill}%
\pgfsetlinewidth{1.505625pt}%
\definecolor{currentstroke}{rgb}{0.200000,0.427451,0.650980}%
\pgfsetstrokecolor{currentstroke}%
\pgfsetdash{}{0pt}%
\pgfsys@defobject{currentmarker}{\pgfqpoint{-0.111111in}{-0.000000in}}{\pgfqpoint{0.111111in}{0.000000in}}{%
\pgfpathmoveto{\pgfqpoint{0.111111in}{-0.000000in}}%
\pgfpathlineto{\pgfqpoint{-0.111111in}{0.000000in}}%
\pgfusepath{stroke,fill}%
}%
\begin{pgfscope}%
\pgfsys@transformshift{4.026250in}{0.862140in}%
\pgfsys@useobject{currentmarker}{}%
\end{pgfscope}%
\end{pgfscope}%
\begin{pgfscope}%
\pgfpathrectangle{\pgfqpoint{3.858325in}{0.528177in}}{\pgfqpoint{1.343404in}{2.125222in}} %
\pgfusepath{clip}%
\pgfsetbuttcap%
\pgfsetroundjoin%
\definecolor{currentfill}{rgb}{0.168627,0.670588,0.494118}%
\pgfsetfillcolor{currentfill}%
\pgfsetlinewidth{1.505625pt}%
\definecolor{currentstroke}{rgb}{0.168627,0.670588,0.494118}%
\pgfsetstrokecolor{currentstroke}%
\pgfsetdash{}{0pt}%
\pgfsys@defobject{currentmarker}{\pgfqpoint{-0.111111in}{-0.000000in}}{\pgfqpoint{0.111111in}{0.000000in}}{%
\pgfpathmoveto{\pgfqpoint{0.111111in}{-0.000000in}}%
\pgfpathlineto{\pgfqpoint{-0.111111in}{0.000000in}}%
\pgfusepath{stroke,fill}%
}%
\begin{pgfscope}%
\pgfsys@transformshift{4.362101in}{2.258715in}%
\pgfsys@useobject{currentmarker}{}%
\end{pgfscope}%
\end{pgfscope}%
\begin{pgfscope}%
\pgfpathrectangle{\pgfqpoint{3.858325in}{0.528177in}}{\pgfqpoint{1.343404in}{2.125222in}} %
\pgfusepath{clip}%
\pgfsetbuttcap%
\pgfsetroundjoin%
\definecolor{currentfill}{rgb}{0.168627,0.670588,0.494118}%
\pgfsetfillcolor{currentfill}%
\pgfsetlinewidth{1.505625pt}%
\definecolor{currentstroke}{rgb}{0.168627,0.670588,0.494118}%
\pgfsetstrokecolor{currentstroke}%
\pgfsetdash{}{0pt}%
\pgfsys@defobject{currentmarker}{\pgfqpoint{-0.111111in}{-0.000000in}}{\pgfqpoint{0.111111in}{0.000000in}}{%
\pgfpathmoveto{\pgfqpoint{0.111111in}{-0.000000in}}%
\pgfpathlineto{\pgfqpoint{-0.111111in}{0.000000in}}%
\pgfusepath{stroke,fill}%
}%
\begin{pgfscope}%
\pgfsys@transformshift{4.362101in}{1.955111in}%
\pgfsys@useobject{currentmarker}{}%
\end{pgfscope}%
\end{pgfscope}%
\begin{pgfscope}%
\pgfpathrectangle{\pgfqpoint{3.858325in}{0.528177in}}{\pgfqpoint{1.343404in}{2.125222in}} %
\pgfusepath{clip}%
\pgfsetbuttcap%
\pgfsetroundjoin%
\definecolor{currentfill}{rgb}{1.000000,0.494118,0.250980}%
\pgfsetfillcolor{currentfill}%
\pgfsetlinewidth{1.505625pt}%
\definecolor{currentstroke}{rgb}{1.000000,0.494118,0.250980}%
\pgfsetstrokecolor{currentstroke}%
\pgfsetdash{}{0pt}%
\pgfsys@defobject{currentmarker}{\pgfqpoint{-0.111111in}{-0.000000in}}{\pgfqpoint{0.111111in}{0.000000in}}{%
\pgfpathmoveto{\pgfqpoint{0.111111in}{-0.000000in}}%
\pgfpathlineto{\pgfqpoint{-0.111111in}{0.000000in}}%
\pgfusepath{stroke,fill}%
}%
\end{pgfscope}%
\begin{pgfscope}%
\pgfpathrectangle{\pgfqpoint{3.858325in}{0.528177in}}{\pgfqpoint{1.343404in}{2.125222in}} %
\pgfusepath{clip}%
\pgfsetbuttcap%
\pgfsetroundjoin%
\definecolor{currentfill}{rgb}{1.000000,0.494118,0.250980}%
\pgfsetfillcolor{currentfill}%
\pgfsetlinewidth{1.505625pt}%
\definecolor{currentstroke}{rgb}{1.000000,0.494118,0.250980}%
\pgfsetstrokecolor{currentstroke}%
\pgfsetdash{}{0pt}%
\pgfsys@defobject{currentmarker}{\pgfqpoint{-0.111111in}{-0.000000in}}{\pgfqpoint{0.111111in}{0.000000in}}{%
\pgfpathmoveto{\pgfqpoint{0.111111in}{-0.000000in}}%
\pgfpathlineto{\pgfqpoint{-0.111111in}{0.000000in}}%
\pgfusepath{stroke,fill}%
}%
\end{pgfscope}%
\begin{pgfscope}%
\pgfpathrectangle{\pgfqpoint{3.858325in}{0.528177in}}{\pgfqpoint{1.343404in}{2.125222in}} %
\pgfusepath{clip}%
\pgfsetbuttcap%
\pgfsetroundjoin%
\definecolor{currentfill}{rgb}{1.000000,0.694118,0.250980}%
\pgfsetfillcolor{currentfill}%
\pgfsetlinewidth{1.505625pt}%
\definecolor{currentstroke}{rgb}{1.000000,0.694118,0.250980}%
\pgfsetstrokecolor{currentstroke}%
\pgfsetdash{}{0pt}%
\pgfsys@defobject{currentmarker}{\pgfqpoint{-0.111111in}{-0.000000in}}{\pgfqpoint{0.111111in}{0.000000in}}{%
\pgfpathmoveto{\pgfqpoint{0.111111in}{-0.000000in}}%
\pgfpathlineto{\pgfqpoint{-0.111111in}{0.000000in}}%
\pgfusepath{stroke,fill}%
}%
\begin{pgfscope}%
\pgfsys@transformshift{5.033803in}{0.801420in}%
\pgfsys@useobject{currentmarker}{}%
\end{pgfscope}%
\end{pgfscope}%
\begin{pgfscope}%
\pgfpathrectangle{\pgfqpoint{3.858325in}{0.528177in}}{\pgfqpoint{1.343404in}{2.125222in}} %
\pgfusepath{clip}%
\pgfsetbuttcap%
\pgfsetroundjoin%
\definecolor{currentfill}{rgb}{1.000000,0.694118,0.250980}%
\pgfsetfillcolor{currentfill}%
\pgfsetlinewidth{1.505625pt}%
\definecolor{currentstroke}{rgb}{1.000000,0.694118,0.250980}%
\pgfsetstrokecolor{currentstroke}%
\pgfsetdash{}{0pt}%
\pgfsys@defobject{currentmarker}{\pgfqpoint{-0.111111in}{-0.000000in}}{\pgfqpoint{0.111111in}{0.000000in}}{%
\pgfpathmoveto{\pgfqpoint{0.111111in}{-0.000000in}}%
\pgfpathlineto{\pgfqpoint{-0.111111in}{0.000000in}}%
\pgfusepath{stroke,fill}%
}%
\begin{pgfscope}%
\pgfsys@transformshift{5.033803in}{0.649618in}%
\pgfsys@useobject{currentmarker}{}%
\end{pgfscope}%
\end{pgfscope}%
\begin{pgfscope}%
\pgfsetrectcap%
\pgfsetmiterjoin%
\pgfsetlinewidth{1.254687pt}%
\definecolor{currentstroke}{rgb}{0.150000,0.150000,0.150000}%
\pgfsetstrokecolor{currentstroke}%
\pgfsetdash{}{0pt}%
\pgfpathmoveto{\pgfqpoint{3.858325in}{0.528177in}}%
\pgfpathlineto{\pgfqpoint{3.858325in}{2.653399in}}%
\pgfusepath{stroke}%
\end{pgfscope}%
\begin{pgfscope}%
\pgfsetrectcap%
\pgfsetmiterjoin%
\pgfsetlinewidth{1.254687pt}%
\definecolor{currentstroke}{rgb}{0.150000,0.150000,0.150000}%
\pgfsetstrokecolor{currentstroke}%
\pgfsetdash{}{0pt}%
\pgfpathmoveto{\pgfqpoint{3.858325in}{0.528177in}}%
\pgfpathlineto{\pgfqpoint{5.201729in}{0.528177in}}%
\pgfusepath{stroke}%
\end{pgfscope}%
\begin{pgfscope}%
\pgfsetbuttcap%
\pgfsetmiterjoin%
\definecolor{currentfill}{rgb}{0.200000,0.427451,0.650980}%
\pgfsetfillcolor{currentfill}%
\pgfsetlinewidth{1.505625pt}%
\definecolor{currentstroke}{rgb}{0.200000,0.427451,0.650980}%
\pgfsetstrokecolor{currentstroke}%
\pgfsetdash{}{0pt}%
\pgfpathmoveto{\pgfqpoint{3.891154in}{3.239798in}}%
\pgfpathlineto{\pgfqpoint{4.002266in}{3.239798in}}%
\pgfpathlineto{\pgfqpoint{4.002266in}{3.317575in}}%
\pgfpathlineto{\pgfqpoint{3.891154in}{3.317575in}}%
\pgfpathclose%
\pgfusepath{stroke,fill}%
\end{pgfscope}%
\begin{pgfscope}%
\definecolor{textcolor}{rgb}{0.150000,0.150000,0.150000}%
\pgfsetstrokecolor{textcolor}%
\pgfsetfillcolor{textcolor}%
\pgftext[x=4.091154in,y=3.239798in,left,base]{\color{textcolor}\rmfamily\fontsize{8.000000}{9.600000}\selectfont WT + Vehicle}%
\end{pgfscope}%
\begin{pgfscope}%
\pgfsetbuttcap%
\pgfsetmiterjoin%
\definecolor{currentfill}{rgb}{0.168627,0.670588,0.494118}%
\pgfsetfillcolor{currentfill}%
\pgfsetlinewidth{1.505625pt}%
\definecolor{currentstroke}{rgb}{0.168627,0.670588,0.494118}%
\pgfsetstrokecolor{currentstroke}%
\pgfsetdash{}{0pt}%
\pgfpathmoveto{\pgfqpoint{3.891154in}{3.084864in}}%
\pgfpathlineto{\pgfqpoint{4.002266in}{3.084864in}}%
\pgfpathlineto{\pgfqpoint{4.002266in}{3.162642in}}%
\pgfpathlineto{\pgfqpoint{3.891154in}{3.162642in}}%
\pgfpathclose%
\pgfusepath{stroke,fill}%
\end{pgfscope}%
\begin{pgfscope}%
\definecolor{textcolor}{rgb}{0.150000,0.150000,0.150000}%
\pgfsetstrokecolor{textcolor}%
\pgfsetfillcolor{textcolor}%
\pgftext[x=4.091154in,y=3.084864in,left,base]{\color{textcolor}\rmfamily\fontsize{8.000000}{9.600000}\selectfont WT + TAT-GluA2\textsubscript{3Y}}%
\end{pgfscope}%
\begin{pgfscope}%
\pgfsetbuttcap%
\pgfsetmiterjoin%
\definecolor{currentfill}{rgb}{1.000000,0.494118,0.250980}%
\pgfsetfillcolor{currentfill}%
\pgfsetlinewidth{1.505625pt}%
\definecolor{currentstroke}{rgb}{1.000000,0.494118,0.250980}%
\pgfsetstrokecolor{currentstroke}%
\pgfsetdash{}{0pt}%
\pgfpathmoveto{\pgfqpoint{3.891154in}{2.929931in}}%
\pgfpathlineto{\pgfqpoint{4.002266in}{2.929931in}}%
\pgfpathlineto{\pgfqpoint{4.002266in}{3.007709in}}%
\pgfpathlineto{\pgfqpoint{3.891154in}{3.007709in}}%
\pgfpathclose%
\pgfusepath{stroke,fill}%
\end{pgfscope}%
\begin{pgfscope}%
\definecolor{textcolor}{rgb}{0.150000,0.150000,0.150000}%
\pgfsetstrokecolor{textcolor}%
\pgfsetfillcolor{textcolor}%
\pgftext[x=4.091154in,y=2.929931in,left,base]{\color{textcolor}\rmfamily\fontsize{8.000000}{9.600000}\selectfont Tg + Vehicle}%
\end{pgfscope}%
\begin{pgfscope}%
\pgfsetbuttcap%
\pgfsetmiterjoin%
\definecolor{currentfill}{rgb}{1.000000,0.694118,0.250980}%
\pgfsetfillcolor{currentfill}%
\pgfsetlinewidth{1.505625pt}%
\definecolor{currentstroke}{rgb}{1.000000,0.694118,0.250980}%
\pgfsetstrokecolor{currentstroke}%
\pgfsetdash{}{0pt}%
\pgfpathmoveto{\pgfqpoint{3.891154in}{2.774998in}}%
\pgfpathlineto{\pgfqpoint{4.002266in}{2.774998in}}%
\pgfpathlineto{\pgfqpoint{4.002266in}{2.852776in}}%
\pgfpathlineto{\pgfqpoint{3.891154in}{2.852776in}}%
\pgfpathclose%
\pgfusepath{stroke,fill}%
\end{pgfscope}%
\begin{pgfscope}%
\definecolor{textcolor}{rgb}{0.150000,0.150000,0.150000}%
\pgfsetstrokecolor{textcolor}%
\pgfsetfillcolor{textcolor}%
\pgftext[x=4.091154in,y=2.774998in,left,base]{\color{textcolor}\rmfamily\fontsize{8.000000}{9.600000}\selectfont Tg + TAT-GluA2\textsubscript{3Y}}%
\end{pgfscope}%
\end{pgfpicture}%
\makeatother%
\endgroup%

        \caption{\label{f.ad.svm_into_f}}
    \end{subfigure}
    \caption[Average accuracy of classifiers at initiation of freezing behaviour.]{Average accuracy of classifiers at the beginning of freezing behaviour in mice during the context memory test. For \gls{wt} and \gls{tg}-\glu{} groups, both \gls{nbc} and \gls{gsvm} shows significantly decreased performance just before the mice show freezing behaviour. This suggests that \gls{ca1} neural activity drives freezing behaviour. The error bars on the top shows \SI{95}{\percent} credible interval of the change point, if the alternative hypothesis is favoured. \label{f.ad.into_f}}
\end{figure}

\subsection{Memory recall in \gls{tg} mice is unstable}
Our results show that \gls{tg} mice have shorter freezing bouts, and suggest that \gls{tg} mice may have an unstable representation of contextual fear memory, and may be unable to maintain the expression of this fear memory. Here we directly test the robustness of the representation of fear memory expression in \gls{ca1}. Using the \gls{gsvm} classifier, we calculated the distance of the brain state to the \gls{gsvm} classification boundary. We hypothesized that if \gls{tg} mice have an unstable representation of fear memory, their brain state during freezing will be closer to the classification boundary than the \gls{wt} mice. 

\begin{figure}[h]
    %% Creator: Matplotlib, PGF backend
%%
%% To include the figure in your LaTeX document, write
%%   \input{<filename>.pgf}
%%
%% Make sure the required packages are loaded in your preamble
%%   \usepackage{pgf}
%%
%% Figures using additional raster images can only be included by \input if
%% they are in the same directory as the main LaTeX file. For loading figures
%% from other directories you can use the `import` package
%%   \usepackage{import}
%% and then include the figures with
%%   \import{<path to file>}{<filename>.pgf}
%%
%% Matplotlib used the following preamble
%%   \usepackage[utf8]{inputenc}
%%   \usepackage[T1]{fontenc}
%%   \usepackage{siunitx}
%%
\begingroup%
\makeatletter%
\begin{pgfpicture}%
\pgfpathrectangle{\pgfpointorigin}{\pgfqpoint{6.597765in}{4.500299in}}%
\pgfusepath{use as bounding box, clip}%
\begin{pgfscope}%
\pgfsetbuttcap%
\pgfsetmiterjoin%
\definecolor{currentfill}{rgb}{1.000000,1.000000,1.000000}%
\pgfsetfillcolor{currentfill}%
\pgfsetlinewidth{0.000000pt}%
\definecolor{currentstroke}{rgb}{1.000000,1.000000,1.000000}%
\pgfsetstrokecolor{currentstroke}%
\pgfsetdash{}{0pt}%
\pgfpathmoveto{\pgfqpoint{0.000000in}{0.000000in}}%
\pgfpathlineto{\pgfqpoint{6.597765in}{0.000000in}}%
\pgfpathlineto{\pgfqpoint{6.597765in}{4.500299in}}%
\pgfpathlineto{\pgfqpoint{0.000000in}{4.500299in}}%
\pgfpathclose%
\pgfusepath{fill}%
\end{pgfscope}%
\begin{pgfscope}%
\pgfsetbuttcap%
\pgfsetmiterjoin%
\definecolor{currentfill}{rgb}{1.000000,1.000000,1.000000}%
\pgfsetfillcolor{currentfill}%
\pgfsetlinewidth{0.000000pt}%
\definecolor{currentstroke}{rgb}{0.000000,0.000000,0.000000}%
\pgfsetstrokecolor{currentstroke}%
\pgfsetstrokeopacity{0.000000}%
\pgfsetdash{}{0pt}%
\pgfpathmoveto{\pgfqpoint{0.610762in}{0.961156in}}%
\pgfpathlineto{\pgfqpoint{4.782032in}{0.961156in}}%
\pgfpathlineto{\pgfqpoint{4.782032in}{3.539143in}}%
\pgfpathlineto{\pgfqpoint{0.610762in}{3.539143in}}%
\pgfpathclose%
\pgfusepath{fill}%
\end{pgfscope}%
\begin{pgfscope}%
\pgfsetbuttcap%
\pgfsetroundjoin%
\definecolor{currentfill}{rgb}{0.150000,0.150000,0.150000}%
\pgfsetfillcolor{currentfill}%
\pgfsetlinewidth{1.003750pt}%
\definecolor{currentstroke}{rgb}{0.150000,0.150000,0.150000}%
\pgfsetstrokecolor{currentstroke}%
\pgfsetdash{}{0pt}%
\pgfsys@defobject{currentmarker}{\pgfqpoint{0.000000in}{0.000000in}}{\pgfqpoint{0.000000in}{0.041667in}}{%
\pgfpathmoveto{\pgfqpoint{0.000000in}{0.000000in}}%
\pgfpathlineto{\pgfqpoint{0.000000in}{0.041667in}}%
\pgfusepath{stroke,fill}%
}%
\begin{pgfscope}%
\pgfsys@transformshift{0.610762in}{0.961156in}%
\pgfsys@useobject{currentmarker}{}%
\end{pgfscope}%
\end{pgfscope}%
\begin{pgfscope}%
\definecolor{textcolor}{rgb}{0.150000,0.150000,0.150000}%
\pgfsetstrokecolor{textcolor}%
\pgfsetfillcolor{textcolor}%
\pgftext[x=0.610762in,y=0.863934in,,top]{\color{textcolor}\rmfamily\fontsize{10.000000}{12.000000}\selectfont \(\displaystyle -10\)}%
\end{pgfscope}%
\begin{pgfscope}%
\pgfsetbuttcap%
\pgfsetroundjoin%
\definecolor{currentfill}{rgb}{0.150000,0.150000,0.150000}%
\pgfsetfillcolor{currentfill}%
\pgfsetlinewidth{1.003750pt}%
\definecolor{currentstroke}{rgb}{0.150000,0.150000,0.150000}%
\pgfsetstrokecolor{currentstroke}%
\pgfsetdash{}{0pt}%
\pgfsys@defobject{currentmarker}{\pgfqpoint{0.000000in}{0.000000in}}{\pgfqpoint{0.000000in}{0.041667in}}{%
\pgfpathmoveto{\pgfqpoint{0.000000in}{0.000000in}}%
\pgfpathlineto{\pgfqpoint{0.000000in}{0.041667in}}%
\pgfusepath{stroke,fill}%
}%
\begin{pgfscope}%
\pgfsys@transformshift{1.653580in}{0.961156in}%
\pgfsys@useobject{currentmarker}{}%
\end{pgfscope}%
\end{pgfscope}%
\begin{pgfscope}%
\definecolor{textcolor}{rgb}{0.150000,0.150000,0.150000}%
\pgfsetstrokecolor{textcolor}%
\pgfsetfillcolor{textcolor}%
\pgftext[x=1.653580in,y=0.863934in,,top]{\color{textcolor}\rmfamily\fontsize{10.000000}{12.000000}\selectfont \(\displaystyle -5\)}%
\end{pgfscope}%
\begin{pgfscope}%
\pgfsetbuttcap%
\pgfsetroundjoin%
\definecolor{currentfill}{rgb}{0.150000,0.150000,0.150000}%
\pgfsetfillcolor{currentfill}%
\pgfsetlinewidth{1.003750pt}%
\definecolor{currentstroke}{rgb}{0.150000,0.150000,0.150000}%
\pgfsetstrokecolor{currentstroke}%
\pgfsetdash{}{0pt}%
\pgfsys@defobject{currentmarker}{\pgfqpoint{0.000000in}{0.000000in}}{\pgfqpoint{0.000000in}{0.041667in}}{%
\pgfpathmoveto{\pgfqpoint{0.000000in}{0.000000in}}%
\pgfpathlineto{\pgfqpoint{0.000000in}{0.041667in}}%
\pgfusepath{stroke,fill}%
}%
\begin{pgfscope}%
\pgfsys@transformshift{2.696397in}{0.961156in}%
\pgfsys@useobject{currentmarker}{}%
\end{pgfscope}%
\end{pgfscope}%
\begin{pgfscope}%
\definecolor{textcolor}{rgb}{0.150000,0.150000,0.150000}%
\pgfsetstrokecolor{textcolor}%
\pgfsetfillcolor{textcolor}%
\pgftext[x=2.696397in,y=0.863934in,,top]{\color{textcolor}\rmfamily\fontsize{10.000000}{12.000000}\selectfont \(\displaystyle 0\)}%
\end{pgfscope}%
\begin{pgfscope}%
\pgfsetbuttcap%
\pgfsetroundjoin%
\definecolor{currentfill}{rgb}{0.150000,0.150000,0.150000}%
\pgfsetfillcolor{currentfill}%
\pgfsetlinewidth{1.003750pt}%
\definecolor{currentstroke}{rgb}{0.150000,0.150000,0.150000}%
\pgfsetstrokecolor{currentstroke}%
\pgfsetdash{}{0pt}%
\pgfsys@defobject{currentmarker}{\pgfqpoint{0.000000in}{0.000000in}}{\pgfqpoint{0.000000in}{0.041667in}}{%
\pgfpathmoveto{\pgfqpoint{0.000000in}{0.000000in}}%
\pgfpathlineto{\pgfqpoint{0.000000in}{0.041667in}}%
\pgfusepath{stroke,fill}%
}%
\begin{pgfscope}%
\pgfsys@transformshift{3.739215in}{0.961156in}%
\pgfsys@useobject{currentmarker}{}%
\end{pgfscope}%
\end{pgfscope}%
\begin{pgfscope}%
\definecolor{textcolor}{rgb}{0.150000,0.150000,0.150000}%
\pgfsetstrokecolor{textcolor}%
\pgfsetfillcolor{textcolor}%
\pgftext[x=3.739215in,y=0.863934in,,top]{\color{textcolor}\rmfamily\fontsize{10.000000}{12.000000}\selectfont \(\displaystyle 5\)}%
\end{pgfscope}%
\begin{pgfscope}%
\pgfsetbuttcap%
\pgfsetroundjoin%
\definecolor{currentfill}{rgb}{0.150000,0.150000,0.150000}%
\pgfsetfillcolor{currentfill}%
\pgfsetlinewidth{1.003750pt}%
\definecolor{currentstroke}{rgb}{0.150000,0.150000,0.150000}%
\pgfsetstrokecolor{currentstroke}%
\pgfsetdash{}{0pt}%
\pgfsys@defobject{currentmarker}{\pgfqpoint{0.000000in}{0.000000in}}{\pgfqpoint{0.000000in}{0.041667in}}{%
\pgfpathmoveto{\pgfqpoint{0.000000in}{0.000000in}}%
\pgfpathlineto{\pgfqpoint{0.000000in}{0.041667in}}%
\pgfusepath{stroke,fill}%
}%
\begin{pgfscope}%
\pgfsys@transformshift{4.782032in}{0.961156in}%
\pgfsys@useobject{currentmarker}{}%
\end{pgfscope}%
\end{pgfscope}%
\begin{pgfscope}%
\definecolor{textcolor}{rgb}{0.150000,0.150000,0.150000}%
\pgfsetstrokecolor{textcolor}%
\pgfsetfillcolor{textcolor}%
\pgftext[x=4.782032in,y=0.863934in,,top]{\color{textcolor}\rmfamily\fontsize{10.000000}{12.000000}\selectfont \(\displaystyle 10\)}%
\end{pgfscope}%
\begin{pgfscope}%
\definecolor{textcolor}{rgb}{0.150000,0.150000,0.150000}%
\pgfsetstrokecolor{textcolor}%
\pgfsetfillcolor{textcolor}%
\pgftext[x=2.696397in,y=0.671835in,,top]{\color{textcolor}\rmfamily\fontsize{12.000000}{14.400000}\selectfont \textbf{Time from freezing (s)}}%
\end{pgfscope}%
\begin{pgfscope}%
\pgfsetbuttcap%
\pgfsetroundjoin%
\definecolor{currentfill}{rgb}{0.150000,0.150000,0.150000}%
\pgfsetfillcolor{currentfill}%
\pgfsetlinewidth{1.003750pt}%
\definecolor{currentstroke}{rgb}{0.150000,0.150000,0.150000}%
\pgfsetstrokecolor{currentstroke}%
\pgfsetdash{}{0pt}%
\pgfsys@defobject{currentmarker}{\pgfqpoint{0.000000in}{0.000000in}}{\pgfqpoint{0.041667in}{0.000000in}}{%
\pgfpathmoveto{\pgfqpoint{0.000000in}{0.000000in}}%
\pgfpathlineto{\pgfqpoint{0.041667in}{0.000000in}}%
\pgfusepath{stroke,fill}%
}%
\begin{pgfscope}%
\pgfsys@transformshift{0.610762in}{0.961156in}%
\pgfsys@useobject{currentmarker}{}%
\end{pgfscope}%
\end{pgfscope}%
\begin{pgfscope}%
\definecolor{textcolor}{rgb}{0.150000,0.150000,0.150000}%
\pgfsetstrokecolor{textcolor}%
\pgfsetfillcolor{textcolor}%
\pgftext[x=0.513540in,y=0.961156in,right,]{\color{textcolor}\rmfamily\fontsize{10.000000}{12.000000}\selectfont \(\displaystyle -3\)}%
\end{pgfscope}%
\begin{pgfscope}%
\pgfsetbuttcap%
\pgfsetroundjoin%
\definecolor{currentfill}{rgb}{0.150000,0.150000,0.150000}%
\pgfsetfillcolor{currentfill}%
\pgfsetlinewidth{1.003750pt}%
\definecolor{currentstroke}{rgb}{0.150000,0.150000,0.150000}%
\pgfsetstrokecolor{currentstroke}%
\pgfsetdash{}{0pt}%
\pgfsys@defobject{currentmarker}{\pgfqpoint{0.000000in}{0.000000in}}{\pgfqpoint{0.041667in}{0.000000in}}{%
\pgfpathmoveto{\pgfqpoint{0.000000in}{0.000000in}}%
\pgfpathlineto{\pgfqpoint{0.041667in}{0.000000in}}%
\pgfusepath{stroke,fill}%
}%
\begin{pgfscope}%
\pgfsys@transformshift{0.610762in}{1.476754in}%
\pgfsys@useobject{currentmarker}{}%
\end{pgfscope}%
\end{pgfscope}%
\begin{pgfscope}%
\definecolor{textcolor}{rgb}{0.150000,0.150000,0.150000}%
\pgfsetstrokecolor{textcolor}%
\pgfsetfillcolor{textcolor}%
\pgftext[x=0.513540in,y=1.476754in,right,]{\color{textcolor}\rmfamily\fontsize{10.000000}{12.000000}\selectfont \(\displaystyle -2\)}%
\end{pgfscope}%
\begin{pgfscope}%
\pgfsetbuttcap%
\pgfsetroundjoin%
\definecolor{currentfill}{rgb}{0.150000,0.150000,0.150000}%
\pgfsetfillcolor{currentfill}%
\pgfsetlinewidth{1.003750pt}%
\definecolor{currentstroke}{rgb}{0.150000,0.150000,0.150000}%
\pgfsetstrokecolor{currentstroke}%
\pgfsetdash{}{0pt}%
\pgfsys@defobject{currentmarker}{\pgfqpoint{0.000000in}{0.000000in}}{\pgfqpoint{0.041667in}{0.000000in}}{%
\pgfpathmoveto{\pgfqpoint{0.000000in}{0.000000in}}%
\pgfpathlineto{\pgfqpoint{0.041667in}{0.000000in}}%
\pgfusepath{stroke,fill}%
}%
\begin{pgfscope}%
\pgfsys@transformshift{0.610762in}{1.992351in}%
\pgfsys@useobject{currentmarker}{}%
\end{pgfscope}%
\end{pgfscope}%
\begin{pgfscope}%
\definecolor{textcolor}{rgb}{0.150000,0.150000,0.150000}%
\pgfsetstrokecolor{textcolor}%
\pgfsetfillcolor{textcolor}%
\pgftext[x=0.513540in,y=1.992351in,right,]{\color{textcolor}\rmfamily\fontsize{10.000000}{12.000000}\selectfont \(\displaystyle -1\)}%
\end{pgfscope}%
\begin{pgfscope}%
\pgfsetbuttcap%
\pgfsetroundjoin%
\definecolor{currentfill}{rgb}{0.150000,0.150000,0.150000}%
\pgfsetfillcolor{currentfill}%
\pgfsetlinewidth{1.003750pt}%
\definecolor{currentstroke}{rgb}{0.150000,0.150000,0.150000}%
\pgfsetstrokecolor{currentstroke}%
\pgfsetdash{}{0pt}%
\pgfsys@defobject{currentmarker}{\pgfqpoint{0.000000in}{0.000000in}}{\pgfqpoint{0.041667in}{0.000000in}}{%
\pgfpathmoveto{\pgfqpoint{0.000000in}{0.000000in}}%
\pgfpathlineto{\pgfqpoint{0.041667in}{0.000000in}}%
\pgfusepath{stroke,fill}%
}%
\begin{pgfscope}%
\pgfsys@transformshift{0.610762in}{2.507948in}%
\pgfsys@useobject{currentmarker}{}%
\end{pgfscope}%
\end{pgfscope}%
\begin{pgfscope}%
\definecolor{textcolor}{rgb}{0.150000,0.150000,0.150000}%
\pgfsetstrokecolor{textcolor}%
\pgfsetfillcolor{textcolor}%
\pgftext[x=0.513540in,y=2.507948in,right,]{\color{textcolor}\rmfamily\fontsize{10.000000}{12.000000}\selectfont \(\displaystyle 0\)}%
\end{pgfscope}%
\begin{pgfscope}%
\pgfsetbuttcap%
\pgfsetroundjoin%
\definecolor{currentfill}{rgb}{0.150000,0.150000,0.150000}%
\pgfsetfillcolor{currentfill}%
\pgfsetlinewidth{1.003750pt}%
\definecolor{currentstroke}{rgb}{0.150000,0.150000,0.150000}%
\pgfsetstrokecolor{currentstroke}%
\pgfsetdash{}{0pt}%
\pgfsys@defobject{currentmarker}{\pgfqpoint{0.000000in}{0.000000in}}{\pgfqpoint{0.041667in}{0.000000in}}{%
\pgfpathmoveto{\pgfqpoint{0.000000in}{0.000000in}}%
\pgfpathlineto{\pgfqpoint{0.041667in}{0.000000in}}%
\pgfusepath{stroke,fill}%
}%
\begin{pgfscope}%
\pgfsys@transformshift{0.610762in}{3.023546in}%
\pgfsys@useobject{currentmarker}{}%
\end{pgfscope}%
\end{pgfscope}%
\begin{pgfscope}%
\definecolor{textcolor}{rgb}{0.150000,0.150000,0.150000}%
\pgfsetstrokecolor{textcolor}%
\pgfsetfillcolor{textcolor}%
\pgftext[x=0.513540in,y=3.023546in,right,]{\color{textcolor}\rmfamily\fontsize{10.000000}{12.000000}\selectfont \(\displaystyle 1\)}%
\end{pgfscope}%
\begin{pgfscope}%
\pgfsetbuttcap%
\pgfsetroundjoin%
\definecolor{currentfill}{rgb}{0.150000,0.150000,0.150000}%
\pgfsetfillcolor{currentfill}%
\pgfsetlinewidth{1.003750pt}%
\definecolor{currentstroke}{rgb}{0.150000,0.150000,0.150000}%
\pgfsetstrokecolor{currentstroke}%
\pgfsetdash{}{0pt}%
\pgfsys@defobject{currentmarker}{\pgfqpoint{0.000000in}{0.000000in}}{\pgfqpoint{0.041667in}{0.000000in}}{%
\pgfpathmoveto{\pgfqpoint{0.000000in}{0.000000in}}%
\pgfpathlineto{\pgfqpoint{0.041667in}{0.000000in}}%
\pgfusepath{stroke,fill}%
}%
\begin{pgfscope}%
\pgfsys@transformshift{0.610762in}{3.539143in}%
\pgfsys@useobject{currentmarker}{}%
\end{pgfscope}%
\end{pgfscope}%
\begin{pgfscope}%
\definecolor{textcolor}{rgb}{0.150000,0.150000,0.150000}%
\pgfsetstrokecolor{textcolor}%
\pgfsetfillcolor{textcolor}%
\pgftext[x=0.513540in,y=3.539143in,right,]{\color{textcolor}\rmfamily\fontsize{10.000000}{12.000000}\selectfont \(\displaystyle 2\)}%
\end{pgfscope}%
\begin{pgfscope}%
\definecolor{textcolor}{rgb}{0.150000,0.150000,0.150000}%
\pgfsetstrokecolor{textcolor}%
\pgfsetfillcolor{textcolor}%
\pgftext[x=0.266626in,y=2.250150in,,bottom,rotate=90.000000]{\color{textcolor}\rmfamily\fontsize{12.000000}{14.400000}\selectfont \textbf{normalized distance to classification boundary (a.u.)}}%
\end{pgfscope}%
\begin{pgfscope}%
\pgfpathrectangle{\pgfqpoint{0.610762in}{0.961156in}}{\pgfqpoint{4.171270in}{2.577986in}} %
\pgfusepath{clip}%
\pgfsetbuttcap%
\pgfsetroundjoin%
\definecolor{currentfill}{rgb}{0.200000,0.427451,0.650980}%
\pgfsetfillcolor{currentfill}%
\pgfsetfillopacity{0.200000}%
\pgfsetlinewidth{0.301125pt}%
\definecolor{currentstroke}{rgb}{0.000000,0.000000,0.000000}%
\pgfsetstrokecolor{currentstroke}%
\pgfsetstrokeopacity{0.200000}%
\pgfsetdash{}{0pt}%
\pgfpathmoveto{\pgfqpoint{0.610762in}{3.037590in}}%
\pgfpathlineto{\pgfqpoint{0.610762in}{3.089606in}}%
\pgfpathlineto{\pgfqpoint{0.621191in}{3.107071in}}%
\pgfpathlineto{\pgfqpoint{0.631619in}{3.114397in}}%
\pgfpathlineto{\pgfqpoint{0.642047in}{3.117379in}}%
\pgfpathlineto{\pgfqpoint{0.652475in}{3.114822in}}%
\pgfpathlineto{\pgfqpoint{0.662903in}{3.149365in}}%
\pgfpathlineto{\pgfqpoint{0.673331in}{3.114399in}}%
\pgfpathlineto{\pgfqpoint{0.683760in}{3.130521in}}%
\pgfpathlineto{\pgfqpoint{0.694188in}{3.102292in}}%
\pgfpathlineto{\pgfqpoint{0.704616in}{3.120156in}}%
\pgfpathlineto{\pgfqpoint{0.715044in}{3.151460in}}%
\pgfpathlineto{\pgfqpoint{0.725472in}{3.118728in}}%
\pgfpathlineto{\pgfqpoint{0.735900in}{3.170207in}}%
\pgfpathlineto{\pgfqpoint{0.746329in}{3.117284in}}%
\pgfpathlineto{\pgfqpoint{0.756757in}{3.113281in}}%
\pgfpathlineto{\pgfqpoint{0.767185in}{3.094453in}}%
\pgfpathlineto{\pgfqpoint{0.777613in}{3.088006in}}%
\pgfpathlineto{\pgfqpoint{0.788041in}{3.106899in}}%
\pgfpathlineto{\pgfqpoint{0.798470in}{3.089078in}}%
\pgfpathlineto{\pgfqpoint{0.808898in}{3.105460in}}%
\pgfpathlineto{\pgfqpoint{0.819326in}{3.072793in}}%
\pgfpathlineto{\pgfqpoint{0.829754in}{3.113964in}}%
\pgfpathlineto{\pgfqpoint{0.840182in}{3.108152in}}%
\pgfpathlineto{\pgfqpoint{0.850610in}{3.128963in}}%
\pgfpathlineto{\pgfqpoint{0.861039in}{3.045333in}}%
\pgfpathlineto{\pgfqpoint{0.871467in}{3.084250in}}%
\pgfpathlineto{\pgfqpoint{0.881895in}{3.058025in}}%
\pgfpathlineto{\pgfqpoint{0.892323in}{3.052883in}}%
\pgfpathlineto{\pgfqpoint{0.902751in}{3.073013in}}%
\pgfpathlineto{\pgfqpoint{0.913179in}{3.072075in}}%
\pgfpathlineto{\pgfqpoint{0.923608in}{3.073374in}}%
\pgfpathlineto{\pgfqpoint{0.934036in}{3.059335in}}%
\pgfpathlineto{\pgfqpoint{0.944464in}{3.084397in}}%
\pgfpathlineto{\pgfqpoint{0.954892in}{3.082847in}}%
\pgfpathlineto{\pgfqpoint{0.965320in}{3.085534in}}%
\pgfpathlineto{\pgfqpoint{0.975748in}{3.115598in}}%
\pgfpathlineto{\pgfqpoint{0.986177in}{3.140767in}}%
\pgfpathlineto{\pgfqpoint{0.996605in}{3.141576in}}%
\pgfpathlineto{\pgfqpoint{1.007033in}{3.153956in}}%
\pgfpathlineto{\pgfqpoint{1.017461in}{3.127965in}}%
\pgfpathlineto{\pgfqpoint{1.027889in}{3.104873in}}%
\pgfpathlineto{\pgfqpoint{1.038318in}{3.088913in}}%
\pgfpathlineto{\pgfqpoint{1.048746in}{3.075388in}}%
\pgfpathlineto{\pgfqpoint{1.059174in}{3.105074in}}%
\pgfpathlineto{\pgfqpoint{1.069602in}{3.080389in}}%
\pgfpathlineto{\pgfqpoint{1.080030in}{3.060728in}}%
\pgfpathlineto{\pgfqpoint{1.090458in}{3.084290in}}%
\pgfpathlineto{\pgfqpoint{1.100887in}{3.045524in}}%
\pgfpathlineto{\pgfqpoint{1.111315in}{3.059803in}}%
\pgfpathlineto{\pgfqpoint{1.121743in}{3.058793in}}%
\pgfpathlineto{\pgfqpoint{1.132171in}{3.093832in}}%
\pgfpathlineto{\pgfqpoint{1.142599in}{3.093540in}}%
\pgfpathlineto{\pgfqpoint{1.153027in}{3.100428in}}%
\pgfpathlineto{\pgfqpoint{1.163456in}{3.075167in}}%
\pgfpathlineto{\pgfqpoint{1.173884in}{3.065023in}}%
\pgfpathlineto{\pgfqpoint{1.184312in}{3.066371in}}%
\pgfpathlineto{\pgfqpoint{1.194740in}{3.044745in}}%
\pgfpathlineto{\pgfqpoint{1.205168in}{3.060487in}}%
\pgfpathlineto{\pgfqpoint{1.215596in}{3.053148in}}%
\pgfpathlineto{\pgfqpoint{1.226025in}{3.033570in}}%
\pgfpathlineto{\pgfqpoint{1.236453in}{3.053765in}}%
\pgfpathlineto{\pgfqpoint{1.246881in}{3.080452in}}%
\pgfpathlineto{\pgfqpoint{1.257309in}{3.073719in}}%
\pgfpathlineto{\pgfqpoint{1.267737in}{3.079475in}}%
\pgfpathlineto{\pgfqpoint{1.278166in}{3.056364in}}%
\pgfpathlineto{\pgfqpoint{1.288594in}{3.060279in}}%
\pgfpathlineto{\pgfqpoint{1.299022in}{3.077776in}}%
\pgfpathlineto{\pgfqpoint{1.309450in}{3.094995in}}%
\pgfpathlineto{\pgfqpoint{1.319878in}{3.089745in}}%
\pgfpathlineto{\pgfqpoint{1.330306in}{3.102463in}}%
\pgfpathlineto{\pgfqpoint{1.340735in}{3.076963in}}%
\pgfpathlineto{\pgfqpoint{1.351163in}{3.088871in}}%
\pgfpathlineto{\pgfqpoint{1.361591in}{3.099640in}}%
\pgfpathlineto{\pgfqpoint{1.372019in}{3.088594in}}%
\pgfpathlineto{\pgfqpoint{1.382447in}{3.095870in}}%
\pgfpathlineto{\pgfqpoint{1.392875in}{3.081508in}}%
\pgfpathlineto{\pgfqpoint{1.403304in}{3.071109in}}%
\pgfpathlineto{\pgfqpoint{1.413732in}{3.081802in}}%
\pgfpathlineto{\pgfqpoint{1.424160in}{3.074262in}}%
\pgfpathlineto{\pgfqpoint{1.434588in}{3.083735in}}%
\pgfpathlineto{\pgfqpoint{1.445016in}{3.097797in}}%
\pgfpathlineto{\pgfqpoint{1.455444in}{3.108042in}}%
\pgfpathlineto{\pgfqpoint{1.465873in}{3.092120in}}%
\pgfpathlineto{\pgfqpoint{1.476301in}{3.053709in}}%
\pgfpathlineto{\pgfqpoint{1.486729in}{3.049013in}}%
\pgfpathlineto{\pgfqpoint{1.497157in}{3.057209in}}%
\pgfpathlineto{\pgfqpoint{1.507585in}{3.030791in}}%
\pgfpathlineto{\pgfqpoint{1.518014in}{3.003097in}}%
\pgfpathlineto{\pgfqpoint{1.528442in}{3.016523in}}%
\pgfpathlineto{\pgfqpoint{1.538870in}{3.007441in}}%
\pgfpathlineto{\pgfqpoint{1.549298in}{3.025123in}}%
\pgfpathlineto{\pgfqpoint{1.559726in}{3.030976in}}%
\pgfpathlineto{\pgfqpoint{1.570154in}{3.049290in}}%
\pgfpathlineto{\pgfqpoint{1.580583in}{3.043683in}}%
\pgfpathlineto{\pgfqpoint{1.591011in}{3.033429in}}%
\pgfpathlineto{\pgfqpoint{1.601439in}{3.011070in}}%
\pgfpathlineto{\pgfqpoint{1.611867in}{3.018664in}}%
\pgfpathlineto{\pgfqpoint{1.622295in}{3.013540in}}%
\pgfpathlineto{\pgfqpoint{1.632723in}{3.011138in}}%
\pgfpathlineto{\pgfqpoint{1.643152in}{3.032167in}}%
\pgfpathlineto{\pgfqpoint{1.653580in}{3.018669in}}%
\pgfpathlineto{\pgfqpoint{1.664008in}{3.046874in}}%
\pgfpathlineto{\pgfqpoint{1.674436in}{3.028475in}}%
\pgfpathlineto{\pgfqpoint{1.684864in}{3.052685in}}%
\pgfpathlineto{\pgfqpoint{1.695292in}{3.054056in}}%
\pgfpathlineto{\pgfqpoint{1.705721in}{3.053256in}}%
\pgfpathlineto{\pgfqpoint{1.716149in}{3.066118in}}%
\pgfpathlineto{\pgfqpoint{1.726577in}{3.028401in}}%
\pgfpathlineto{\pgfqpoint{1.737005in}{3.033559in}}%
\pgfpathlineto{\pgfqpoint{1.747433in}{3.039993in}}%
\pgfpathlineto{\pgfqpoint{1.757862in}{3.049900in}}%
\pgfpathlineto{\pgfqpoint{1.768290in}{3.046739in}}%
\pgfpathlineto{\pgfqpoint{1.778718in}{3.033732in}}%
\pgfpathlineto{\pgfqpoint{1.789146in}{3.048724in}}%
\pgfpathlineto{\pgfqpoint{1.799574in}{3.037148in}}%
\pgfpathlineto{\pgfqpoint{1.810002in}{3.043793in}}%
\pgfpathlineto{\pgfqpoint{1.820431in}{3.034163in}}%
\pgfpathlineto{\pgfqpoint{1.830859in}{3.045569in}}%
\pgfpathlineto{\pgfqpoint{1.841287in}{3.054711in}}%
\pgfpathlineto{\pgfqpoint{1.851715in}{3.087874in}}%
\pgfpathlineto{\pgfqpoint{1.862143in}{3.072683in}}%
\pgfpathlineto{\pgfqpoint{1.872571in}{3.058220in}}%
\pgfpathlineto{\pgfqpoint{1.883000in}{3.066163in}}%
\pgfpathlineto{\pgfqpoint{1.893428in}{3.074138in}}%
\pgfpathlineto{\pgfqpoint{1.903856in}{3.080513in}}%
\pgfpathlineto{\pgfqpoint{1.914284in}{3.105453in}}%
\pgfpathlineto{\pgfqpoint{1.924712in}{3.094545in}}%
\pgfpathlineto{\pgfqpoint{1.935141in}{3.075517in}}%
\pgfpathlineto{\pgfqpoint{1.945569in}{3.083063in}}%
\pgfpathlineto{\pgfqpoint{1.955997in}{3.092375in}}%
\pgfpathlineto{\pgfqpoint{1.966425in}{3.097630in}}%
\pgfpathlineto{\pgfqpoint{1.976853in}{3.100519in}}%
\pgfpathlineto{\pgfqpoint{1.987281in}{3.070941in}}%
\pgfpathlineto{\pgfqpoint{1.997710in}{3.076823in}}%
\pgfpathlineto{\pgfqpoint{2.008138in}{3.077506in}}%
\pgfpathlineto{\pgfqpoint{2.018566in}{3.060520in}}%
\pgfpathlineto{\pgfqpoint{2.028994in}{3.066003in}}%
\pgfpathlineto{\pgfqpoint{2.039422in}{3.087847in}}%
\pgfpathlineto{\pgfqpoint{2.049850in}{3.046676in}}%
\pgfpathlineto{\pgfqpoint{2.060279in}{3.053065in}}%
\pgfpathlineto{\pgfqpoint{2.070707in}{3.039165in}}%
\pgfpathlineto{\pgfqpoint{2.081135in}{3.053147in}}%
\pgfpathlineto{\pgfqpoint{2.091563in}{3.067591in}}%
\pgfpathlineto{\pgfqpoint{2.101991in}{3.048143in}}%
\pgfpathlineto{\pgfqpoint{2.112419in}{3.058363in}}%
\pgfpathlineto{\pgfqpoint{2.122848in}{3.028505in}}%
\pgfpathlineto{\pgfqpoint{2.133276in}{3.029504in}}%
\pgfpathlineto{\pgfqpoint{2.143704in}{3.043863in}}%
\pgfpathlineto{\pgfqpoint{2.154132in}{3.054071in}}%
\pgfpathlineto{\pgfqpoint{2.164560in}{3.063902in}}%
\pgfpathlineto{\pgfqpoint{2.174989in}{3.011515in}}%
\pgfpathlineto{\pgfqpoint{2.185417in}{3.045785in}}%
\pgfpathlineto{\pgfqpoint{2.195845in}{3.041764in}}%
\pgfpathlineto{\pgfqpoint{2.206273in}{3.042527in}}%
\pgfpathlineto{\pgfqpoint{2.216701in}{3.033006in}}%
\pgfpathlineto{\pgfqpoint{2.227129in}{3.021658in}}%
\pgfpathlineto{\pgfqpoint{2.237558in}{3.039773in}}%
\pgfpathlineto{\pgfqpoint{2.247986in}{3.048728in}}%
\pgfpathlineto{\pgfqpoint{2.258414in}{3.045590in}}%
\pgfpathlineto{\pgfqpoint{2.268842in}{3.066822in}}%
\pgfpathlineto{\pgfqpoint{2.279270in}{3.051574in}}%
\pgfpathlineto{\pgfqpoint{2.289698in}{3.064003in}}%
\pgfpathlineto{\pgfqpoint{2.300127in}{3.064154in}}%
\pgfpathlineto{\pgfqpoint{2.310555in}{3.072570in}}%
\pgfpathlineto{\pgfqpoint{2.320983in}{3.035406in}}%
\pgfpathlineto{\pgfqpoint{2.331411in}{3.057177in}}%
\pgfpathlineto{\pgfqpoint{2.341839in}{3.049040in}}%
\pgfpathlineto{\pgfqpoint{2.352267in}{3.060015in}}%
\pgfpathlineto{\pgfqpoint{2.362696in}{3.052596in}}%
\pgfpathlineto{\pgfqpoint{2.373124in}{3.042531in}}%
\pgfpathlineto{\pgfqpoint{2.383552in}{3.047329in}}%
\pgfpathlineto{\pgfqpoint{2.393980in}{3.046782in}}%
\pgfpathlineto{\pgfqpoint{2.404408in}{3.042481in}}%
\pgfpathlineto{\pgfqpoint{2.414837in}{3.034980in}}%
\pgfpathlineto{\pgfqpoint{2.425265in}{3.038144in}}%
\pgfpathlineto{\pgfqpoint{2.435693in}{3.013739in}}%
\pgfpathlineto{\pgfqpoint{2.446121in}{3.034239in}}%
\pgfpathlineto{\pgfqpoint{2.456549in}{3.028567in}}%
\pgfpathlineto{\pgfqpoint{2.466977in}{3.028862in}}%
\pgfpathlineto{\pgfqpoint{2.477406in}{3.048663in}}%
\pgfpathlineto{\pgfqpoint{2.487834in}{3.043521in}}%
\pgfpathlineto{\pgfqpoint{2.498262in}{3.040892in}}%
\pgfpathlineto{\pgfqpoint{2.508690in}{3.042210in}}%
\pgfpathlineto{\pgfqpoint{2.519118in}{3.040202in}}%
\pgfpathlineto{\pgfqpoint{2.529546in}{3.016583in}}%
\pgfpathlineto{\pgfqpoint{2.539975in}{3.007176in}}%
\pgfpathlineto{\pgfqpoint{2.550403in}{3.011368in}}%
\pgfpathlineto{\pgfqpoint{2.560831in}{2.987964in}}%
\pgfpathlineto{\pgfqpoint{2.571259in}{2.996787in}}%
\pgfpathlineto{\pgfqpoint{2.581687in}{2.954179in}}%
\pgfpathlineto{\pgfqpoint{2.592115in}{2.933615in}}%
\pgfpathlineto{\pgfqpoint{2.602544in}{2.924266in}}%
\pgfpathlineto{\pgfqpoint{2.612972in}{2.890548in}}%
\pgfpathlineto{\pgfqpoint{2.623400in}{2.870113in}}%
\pgfpathlineto{\pgfqpoint{2.633828in}{2.845384in}}%
\pgfpathlineto{\pgfqpoint{2.644256in}{2.803565in}}%
\pgfpathlineto{\pgfqpoint{2.654685in}{2.767693in}}%
\pgfpathlineto{\pgfqpoint{2.665113in}{2.716390in}}%
\pgfpathlineto{\pgfqpoint{2.675541in}{2.619585in}}%
\pgfpathlineto{\pgfqpoint{2.685969in}{2.482265in}}%
\pgfpathlineto{\pgfqpoint{2.696397in}{2.483513in}}%
\pgfpathlineto{\pgfqpoint{2.706825in}{2.355416in}}%
\pgfpathlineto{\pgfqpoint{2.717254in}{2.275427in}}%
\pgfpathlineto{\pgfqpoint{2.727682in}{2.222922in}}%
\pgfpathlineto{\pgfqpoint{2.738110in}{2.199223in}}%
\pgfpathlineto{\pgfqpoint{2.748538in}{2.143041in}}%
\pgfpathlineto{\pgfqpoint{2.758966in}{2.145061in}}%
\pgfpathlineto{\pgfqpoint{2.769394in}{2.116907in}}%
\pgfpathlineto{\pgfqpoint{2.779823in}{2.110421in}}%
\pgfpathlineto{\pgfqpoint{2.790251in}{2.102002in}}%
\pgfpathlineto{\pgfqpoint{2.800679in}{2.077637in}}%
\pgfpathlineto{\pgfqpoint{2.811107in}{2.068387in}}%
\pgfpathlineto{\pgfqpoint{2.821535in}{2.052913in}}%
\pgfpathlineto{\pgfqpoint{2.831964in}{2.034503in}}%
\pgfpathlineto{\pgfqpoint{2.842392in}{2.023170in}}%
\pgfpathlineto{\pgfqpoint{2.852820in}{2.024762in}}%
\pgfpathlineto{\pgfqpoint{2.863248in}{2.015977in}}%
\pgfpathlineto{\pgfqpoint{2.873676in}{1.999574in}}%
\pgfpathlineto{\pgfqpoint{2.884104in}{1.999732in}}%
\pgfpathlineto{\pgfqpoint{2.894533in}{1.999135in}}%
\pgfpathlineto{\pgfqpoint{2.904961in}{1.967943in}}%
\pgfpathlineto{\pgfqpoint{2.915389in}{1.982883in}}%
\pgfpathlineto{\pgfqpoint{2.925817in}{1.954264in}}%
\pgfpathlineto{\pgfqpoint{2.936245in}{1.959973in}}%
\pgfpathlineto{\pgfqpoint{2.946673in}{1.959409in}}%
\pgfpathlineto{\pgfqpoint{2.957102in}{1.938270in}}%
\pgfpathlineto{\pgfqpoint{2.967530in}{1.934127in}}%
\pgfpathlineto{\pgfqpoint{2.977958in}{1.925024in}}%
\pgfpathlineto{\pgfqpoint{2.988386in}{1.929507in}}%
\pgfpathlineto{\pgfqpoint{2.998814in}{1.915341in}}%
\pgfpathlineto{\pgfqpoint{3.009242in}{1.897877in}}%
\pgfpathlineto{\pgfqpoint{3.019671in}{1.940366in}}%
\pgfpathlineto{\pgfqpoint{3.030099in}{1.903402in}}%
\pgfpathlineto{\pgfqpoint{3.040527in}{1.902263in}}%
\pgfpathlineto{\pgfqpoint{3.050955in}{1.933861in}}%
\pgfpathlineto{\pgfqpoint{3.061383in}{1.908670in}}%
\pgfpathlineto{\pgfqpoint{3.071812in}{1.892114in}}%
\pgfpathlineto{\pgfqpoint{3.082240in}{1.888281in}}%
\pgfpathlineto{\pgfqpoint{3.092668in}{1.879629in}}%
\pgfpathlineto{\pgfqpoint{3.103096in}{1.885188in}}%
\pgfpathlineto{\pgfqpoint{3.113524in}{1.863638in}}%
\pgfpathlineto{\pgfqpoint{3.123952in}{1.862024in}}%
\pgfpathlineto{\pgfqpoint{3.134381in}{1.854855in}}%
\pgfpathlineto{\pgfqpoint{3.144809in}{1.866020in}}%
\pgfpathlineto{\pgfqpoint{3.155237in}{1.860753in}}%
\pgfpathlineto{\pgfqpoint{3.165665in}{1.881012in}}%
\pgfpathlineto{\pgfqpoint{3.176093in}{1.862558in}}%
\pgfpathlineto{\pgfqpoint{3.186521in}{1.860310in}}%
\pgfpathlineto{\pgfqpoint{3.196950in}{1.844458in}}%
\pgfpathlineto{\pgfqpoint{3.207378in}{1.831452in}}%
\pgfpathlineto{\pgfqpoint{3.217806in}{1.834149in}}%
\pgfpathlineto{\pgfqpoint{3.228234in}{1.843577in}}%
\pgfpathlineto{\pgfqpoint{3.238662in}{1.847851in}}%
\pgfpathlineto{\pgfqpoint{3.249090in}{1.831313in}}%
\pgfpathlineto{\pgfqpoint{3.259519in}{1.831493in}}%
\pgfpathlineto{\pgfqpoint{3.269947in}{1.805249in}}%
\pgfpathlineto{\pgfqpoint{3.280375in}{1.809985in}}%
\pgfpathlineto{\pgfqpoint{3.290803in}{1.823797in}}%
\pgfpathlineto{\pgfqpoint{3.301231in}{1.820075in}}%
\pgfpathlineto{\pgfqpoint{3.311660in}{1.821540in}}%
\pgfpathlineto{\pgfqpoint{3.322088in}{1.823293in}}%
\pgfpathlineto{\pgfqpoint{3.332516in}{1.815097in}}%
\pgfpathlineto{\pgfqpoint{3.342944in}{1.814640in}}%
\pgfpathlineto{\pgfqpoint{3.353372in}{1.803253in}}%
\pgfpathlineto{\pgfqpoint{3.363800in}{1.800097in}}%
\pgfpathlineto{\pgfqpoint{3.374229in}{1.819942in}}%
\pgfpathlineto{\pgfqpoint{3.384657in}{1.810647in}}%
\pgfpathlineto{\pgfqpoint{3.395085in}{1.811834in}}%
\pgfpathlineto{\pgfqpoint{3.405513in}{1.816036in}}%
\pgfpathlineto{\pgfqpoint{3.415941in}{1.800418in}}%
\pgfpathlineto{\pgfqpoint{3.426369in}{1.808101in}}%
\pgfpathlineto{\pgfqpoint{3.436798in}{1.828915in}}%
\pgfpathlineto{\pgfqpoint{3.447226in}{1.815961in}}%
\pgfpathlineto{\pgfqpoint{3.457654in}{1.791191in}}%
\pgfpathlineto{\pgfqpoint{3.468082in}{1.795290in}}%
\pgfpathlineto{\pgfqpoint{3.478510in}{1.786028in}}%
\pgfpathlineto{\pgfqpoint{3.488938in}{1.785738in}}%
\pgfpathlineto{\pgfqpoint{3.499367in}{1.784297in}}%
\pgfpathlineto{\pgfqpoint{3.509795in}{1.818438in}}%
\pgfpathlineto{\pgfqpoint{3.520223in}{1.815771in}}%
\pgfpathlineto{\pgfqpoint{3.530651in}{1.807848in}}%
\pgfpathlineto{\pgfqpoint{3.541079in}{1.778603in}}%
\pgfpathlineto{\pgfqpoint{3.551508in}{1.774004in}}%
\pgfpathlineto{\pgfqpoint{3.561936in}{1.770069in}}%
\pgfpathlineto{\pgfqpoint{3.572364in}{1.787580in}}%
\pgfpathlineto{\pgfqpoint{3.582792in}{1.797193in}}%
\pgfpathlineto{\pgfqpoint{3.593220in}{1.794626in}}%
\pgfpathlineto{\pgfqpoint{3.603648in}{1.799381in}}%
\pgfpathlineto{\pgfqpoint{3.614077in}{1.804422in}}%
\pgfpathlineto{\pgfqpoint{3.624505in}{1.779117in}}%
\pgfpathlineto{\pgfqpoint{3.634933in}{1.759972in}}%
\pgfpathlineto{\pgfqpoint{3.645361in}{1.768258in}}%
\pgfpathlineto{\pgfqpoint{3.655789in}{1.763531in}}%
\pgfpathlineto{\pgfqpoint{3.666217in}{1.760969in}}%
\pgfpathlineto{\pgfqpoint{3.676646in}{1.757342in}}%
\pgfpathlineto{\pgfqpoint{3.687074in}{1.780221in}}%
\pgfpathlineto{\pgfqpoint{3.697502in}{1.761702in}}%
\pgfpathlineto{\pgfqpoint{3.707930in}{1.761221in}}%
\pgfpathlineto{\pgfqpoint{3.718358in}{1.752499in}}%
\pgfpathlineto{\pgfqpoint{3.728787in}{1.775861in}}%
\pgfpathlineto{\pgfqpoint{3.739215in}{1.744984in}}%
\pgfpathlineto{\pgfqpoint{3.749643in}{1.751652in}}%
\pgfpathlineto{\pgfqpoint{3.760071in}{1.748933in}}%
\pgfpathlineto{\pgfqpoint{3.770499in}{1.717538in}}%
\pgfpathlineto{\pgfqpoint{3.780927in}{1.748959in}}%
\pgfpathlineto{\pgfqpoint{3.791356in}{1.739588in}}%
\pgfpathlineto{\pgfqpoint{3.801784in}{1.781170in}}%
\pgfpathlineto{\pgfqpoint{3.812212in}{1.763366in}}%
\pgfpathlineto{\pgfqpoint{3.822640in}{1.774125in}}%
\pgfpathlineto{\pgfqpoint{3.833068in}{1.750055in}}%
\pgfpathlineto{\pgfqpoint{3.843496in}{1.729453in}}%
\pgfpathlineto{\pgfqpoint{3.853925in}{1.722451in}}%
\pgfpathlineto{\pgfqpoint{3.864353in}{1.709473in}}%
\pgfpathlineto{\pgfqpoint{3.874781in}{1.680018in}}%
\pgfpathlineto{\pgfqpoint{3.885209in}{1.701325in}}%
\pgfpathlineto{\pgfqpoint{3.895637in}{1.708822in}}%
\pgfpathlineto{\pgfqpoint{3.906065in}{1.741998in}}%
\pgfpathlineto{\pgfqpoint{3.916494in}{1.710821in}}%
\pgfpathlineto{\pgfqpoint{3.926922in}{1.733406in}}%
\pgfpathlineto{\pgfqpoint{3.937350in}{1.760395in}}%
\pgfpathlineto{\pgfqpoint{3.947778in}{1.746588in}}%
\pgfpathlineto{\pgfqpoint{3.958206in}{1.762278in}}%
\pgfpathlineto{\pgfqpoint{3.968635in}{1.753381in}}%
\pgfpathlineto{\pgfqpoint{3.979063in}{1.724706in}}%
\pgfpathlineto{\pgfqpoint{3.989491in}{1.715625in}}%
\pgfpathlineto{\pgfqpoint{3.999919in}{1.730803in}}%
\pgfpathlineto{\pgfqpoint{4.010347in}{1.750733in}}%
\pgfpathlineto{\pgfqpoint{4.020775in}{1.753844in}}%
\pgfpathlineto{\pgfqpoint{4.031204in}{1.729368in}}%
\pgfpathlineto{\pgfqpoint{4.041632in}{1.716341in}}%
\pgfpathlineto{\pgfqpoint{4.052060in}{1.690511in}}%
\pgfpathlineto{\pgfqpoint{4.062488in}{1.731597in}}%
\pgfpathlineto{\pgfqpoint{4.072916in}{1.738858in}}%
\pgfpathlineto{\pgfqpoint{4.083344in}{1.750302in}}%
\pgfpathlineto{\pgfqpoint{4.093773in}{1.716941in}}%
\pgfpathlineto{\pgfqpoint{4.104201in}{1.716168in}}%
\pgfpathlineto{\pgfqpoint{4.114629in}{1.701623in}}%
\pgfpathlineto{\pgfqpoint{4.125057in}{1.724041in}}%
\pgfpathlineto{\pgfqpoint{4.135485in}{1.706506in}}%
\pgfpathlineto{\pgfqpoint{4.145913in}{1.735951in}}%
\pgfpathlineto{\pgfqpoint{4.156342in}{1.755267in}}%
\pgfpathlineto{\pgfqpoint{4.166770in}{1.738240in}}%
\pgfpathlineto{\pgfqpoint{4.177198in}{1.726084in}}%
\pgfpathlineto{\pgfqpoint{4.187626in}{1.766295in}}%
\pgfpathlineto{\pgfqpoint{4.198054in}{1.754870in}}%
\pgfpathlineto{\pgfqpoint{4.208483in}{1.759714in}}%
\pgfpathlineto{\pgfqpoint{4.218911in}{1.717168in}}%
\pgfpathlineto{\pgfqpoint{4.229339in}{1.772918in}}%
\pgfpathlineto{\pgfqpoint{4.239767in}{1.747000in}}%
\pgfpathlineto{\pgfqpoint{4.250195in}{1.732801in}}%
\pgfpathlineto{\pgfqpoint{4.260623in}{1.753646in}}%
\pgfpathlineto{\pgfqpoint{4.271052in}{1.755470in}}%
\pgfpathlineto{\pgfqpoint{4.281480in}{1.727547in}}%
\pgfpathlineto{\pgfqpoint{4.291908in}{1.753153in}}%
\pgfpathlineto{\pgfqpoint{4.302336in}{1.714828in}}%
\pgfpathlineto{\pgfqpoint{4.312764in}{1.748225in}}%
\pgfpathlineto{\pgfqpoint{4.323192in}{1.727245in}}%
\pgfpathlineto{\pgfqpoint{4.333621in}{1.716719in}}%
\pgfpathlineto{\pgfqpoint{4.344049in}{1.713540in}}%
\pgfpathlineto{\pgfqpoint{4.354477in}{1.736661in}}%
\pgfpathlineto{\pgfqpoint{4.364905in}{1.729037in}}%
\pgfpathlineto{\pgfqpoint{4.375333in}{1.748631in}}%
\pgfpathlineto{\pgfqpoint{4.385761in}{1.704238in}}%
\pgfpathlineto{\pgfqpoint{4.396190in}{1.761211in}}%
\pgfpathlineto{\pgfqpoint{4.406618in}{1.724972in}}%
\pgfpathlineto{\pgfqpoint{4.417046in}{1.727122in}}%
\pgfpathlineto{\pgfqpoint{4.427474in}{1.728254in}}%
\pgfpathlineto{\pgfqpoint{4.437902in}{1.673939in}}%
\pgfpathlineto{\pgfqpoint{4.448331in}{1.649472in}}%
\pgfpathlineto{\pgfqpoint{4.458759in}{1.672008in}}%
\pgfpathlineto{\pgfqpoint{4.469187in}{1.697909in}}%
\pgfpathlineto{\pgfqpoint{4.479615in}{1.666263in}}%
\pgfpathlineto{\pgfqpoint{4.490043in}{1.673829in}}%
\pgfpathlineto{\pgfqpoint{4.500471in}{1.659310in}}%
\pgfpathlineto{\pgfqpoint{4.510900in}{1.702738in}}%
\pgfpathlineto{\pgfqpoint{4.521328in}{1.698866in}}%
\pgfpathlineto{\pgfqpoint{4.531756in}{1.697982in}}%
\pgfpathlineto{\pgfqpoint{4.542184in}{1.761020in}}%
\pgfpathlineto{\pgfqpoint{4.552612in}{1.747927in}}%
\pgfpathlineto{\pgfqpoint{4.563040in}{1.730210in}}%
\pgfpathlineto{\pgfqpoint{4.573469in}{1.710801in}}%
\pgfpathlineto{\pgfqpoint{4.583897in}{1.709867in}}%
\pgfpathlineto{\pgfqpoint{4.594325in}{1.679690in}}%
\pgfpathlineto{\pgfqpoint{4.604753in}{1.697994in}}%
\pgfpathlineto{\pgfqpoint{4.615181in}{1.674997in}}%
\pgfpathlineto{\pgfqpoint{4.625610in}{1.708258in}}%
\pgfpathlineto{\pgfqpoint{4.636038in}{1.691409in}}%
\pgfpathlineto{\pgfqpoint{4.646466in}{1.724736in}}%
\pgfpathlineto{\pgfqpoint{4.656894in}{1.724517in}}%
\pgfpathlineto{\pgfqpoint{4.667322in}{1.708473in}}%
\pgfpathlineto{\pgfqpoint{4.677750in}{1.720424in}}%
\pgfpathlineto{\pgfqpoint{4.688179in}{1.756255in}}%
\pgfpathlineto{\pgfqpoint{4.698607in}{1.728893in}}%
\pgfpathlineto{\pgfqpoint{4.709035in}{1.719193in}}%
\pgfpathlineto{\pgfqpoint{4.719463in}{1.763708in}}%
\pgfpathlineto{\pgfqpoint{4.729891in}{1.687015in}}%
\pgfpathlineto{\pgfqpoint{4.740319in}{1.669425in}}%
\pgfpathlineto{\pgfqpoint{4.750748in}{1.727028in}}%
\pgfpathlineto{\pgfqpoint{4.761176in}{1.714789in}}%
\pgfpathlineto{\pgfqpoint{4.771604in}{1.687532in}}%
\pgfpathlineto{\pgfqpoint{4.771604in}{1.595545in}}%
\pgfpathlineto{\pgfqpoint{4.771604in}{1.595545in}}%
\pgfpathlineto{\pgfqpoint{4.761176in}{1.629566in}}%
\pgfpathlineto{\pgfqpoint{4.750748in}{1.636819in}}%
\pgfpathlineto{\pgfqpoint{4.740319in}{1.576800in}}%
\pgfpathlineto{\pgfqpoint{4.729891in}{1.602281in}}%
\pgfpathlineto{\pgfqpoint{4.719463in}{1.681677in}}%
\pgfpathlineto{\pgfqpoint{4.709035in}{1.641005in}}%
\pgfpathlineto{\pgfqpoint{4.698607in}{1.634843in}}%
\pgfpathlineto{\pgfqpoint{4.688179in}{1.663525in}}%
\pgfpathlineto{\pgfqpoint{4.677750in}{1.636861in}}%
\pgfpathlineto{\pgfqpoint{4.667322in}{1.620631in}}%
\pgfpathlineto{\pgfqpoint{4.656894in}{1.658739in}}%
\pgfpathlineto{\pgfqpoint{4.646466in}{1.640912in}}%
\pgfpathlineto{\pgfqpoint{4.636038in}{1.612076in}}%
\pgfpathlineto{\pgfqpoint{4.625610in}{1.631972in}}%
\pgfpathlineto{\pgfqpoint{4.615181in}{1.594142in}}%
\pgfpathlineto{\pgfqpoint{4.604753in}{1.611708in}}%
\pgfpathlineto{\pgfqpoint{4.594325in}{1.605262in}}%
\pgfpathlineto{\pgfqpoint{4.583897in}{1.630296in}}%
\pgfpathlineto{\pgfqpoint{4.573469in}{1.643253in}}%
\pgfpathlineto{\pgfqpoint{4.563040in}{1.656809in}}%
\pgfpathlineto{\pgfqpoint{4.552612in}{1.682432in}}%
\pgfpathlineto{\pgfqpoint{4.542184in}{1.688001in}}%
\pgfpathlineto{\pgfqpoint{4.531756in}{1.630384in}}%
\pgfpathlineto{\pgfqpoint{4.521328in}{1.622050in}}%
\pgfpathlineto{\pgfqpoint{4.510900in}{1.623915in}}%
\pgfpathlineto{\pgfqpoint{4.500471in}{1.572646in}}%
\pgfpathlineto{\pgfqpoint{4.490043in}{1.584457in}}%
\pgfpathlineto{\pgfqpoint{4.479615in}{1.582554in}}%
\pgfpathlineto{\pgfqpoint{4.469187in}{1.615645in}}%
\pgfpathlineto{\pgfqpoint{4.458759in}{1.597475in}}%
\pgfpathlineto{\pgfqpoint{4.448331in}{1.565390in}}%
\pgfpathlineto{\pgfqpoint{4.437902in}{1.596975in}}%
\pgfpathlineto{\pgfqpoint{4.427474in}{1.659590in}}%
\pgfpathlineto{\pgfqpoint{4.417046in}{1.650022in}}%
\pgfpathlineto{\pgfqpoint{4.406618in}{1.652838in}}%
\pgfpathlineto{\pgfqpoint{4.396190in}{1.692393in}}%
\pgfpathlineto{\pgfqpoint{4.385761in}{1.626515in}}%
\pgfpathlineto{\pgfqpoint{4.375333in}{1.675116in}}%
\pgfpathlineto{\pgfqpoint{4.364905in}{1.655166in}}%
\pgfpathlineto{\pgfqpoint{4.354477in}{1.658293in}}%
\pgfpathlineto{\pgfqpoint{4.344049in}{1.628568in}}%
\pgfpathlineto{\pgfqpoint{4.333621in}{1.629318in}}%
\pgfpathlineto{\pgfqpoint{4.323192in}{1.641034in}}%
\pgfpathlineto{\pgfqpoint{4.312764in}{1.666175in}}%
\pgfpathlineto{\pgfqpoint{4.302336in}{1.636825in}}%
\pgfpathlineto{\pgfqpoint{4.291908in}{1.674889in}}%
\pgfpathlineto{\pgfqpoint{4.281480in}{1.648633in}}%
\pgfpathlineto{\pgfqpoint{4.271052in}{1.670111in}}%
\pgfpathlineto{\pgfqpoint{4.260623in}{1.672306in}}%
\pgfpathlineto{\pgfqpoint{4.250195in}{1.653577in}}%
\pgfpathlineto{\pgfqpoint{4.239767in}{1.672966in}}%
\pgfpathlineto{\pgfqpoint{4.229339in}{1.702687in}}%
\pgfpathlineto{\pgfqpoint{4.218911in}{1.635819in}}%
\pgfpathlineto{\pgfqpoint{4.208483in}{1.688938in}}%
\pgfpathlineto{\pgfqpoint{4.198054in}{1.677633in}}%
\pgfpathlineto{\pgfqpoint{4.187626in}{1.689354in}}%
\pgfpathlineto{\pgfqpoint{4.177198in}{1.659352in}}%
\pgfpathlineto{\pgfqpoint{4.166770in}{1.673699in}}%
\pgfpathlineto{\pgfqpoint{4.156342in}{1.678397in}}%
\pgfpathlineto{\pgfqpoint{4.145913in}{1.665492in}}%
\pgfpathlineto{\pgfqpoint{4.135485in}{1.637612in}}%
\pgfpathlineto{\pgfqpoint{4.125057in}{1.662287in}}%
\pgfpathlineto{\pgfqpoint{4.114629in}{1.638417in}}%
\pgfpathlineto{\pgfqpoint{4.104201in}{1.655933in}}%
\pgfpathlineto{\pgfqpoint{4.093773in}{1.655598in}}%
\pgfpathlineto{\pgfqpoint{4.083344in}{1.682504in}}%
\pgfpathlineto{\pgfqpoint{4.072916in}{1.669251in}}%
\pgfpathlineto{\pgfqpoint{4.062488in}{1.668869in}}%
\pgfpathlineto{\pgfqpoint{4.052060in}{1.621744in}}%
\pgfpathlineto{\pgfqpoint{4.041632in}{1.652175in}}%
\pgfpathlineto{\pgfqpoint{4.031204in}{1.662848in}}%
\pgfpathlineto{\pgfqpoint{4.020775in}{1.692420in}}%
\pgfpathlineto{\pgfqpoint{4.010347in}{1.685975in}}%
\pgfpathlineto{\pgfqpoint{3.999919in}{1.664877in}}%
\pgfpathlineto{\pgfqpoint{3.989491in}{1.653640in}}%
\pgfpathlineto{\pgfqpoint{3.979063in}{1.662137in}}%
\pgfpathlineto{\pgfqpoint{3.968635in}{1.687206in}}%
\pgfpathlineto{\pgfqpoint{3.958206in}{1.699327in}}%
\pgfpathlineto{\pgfqpoint{3.947778in}{1.688662in}}%
\pgfpathlineto{\pgfqpoint{3.937350in}{1.697438in}}%
\pgfpathlineto{\pgfqpoint{3.926922in}{1.670205in}}%
\pgfpathlineto{\pgfqpoint{3.916494in}{1.643319in}}%
\pgfpathlineto{\pgfqpoint{3.906065in}{1.680378in}}%
\pgfpathlineto{\pgfqpoint{3.895637in}{1.648878in}}%
\pgfpathlineto{\pgfqpoint{3.885209in}{1.637749in}}%
\pgfpathlineto{\pgfqpoint{3.874781in}{1.615552in}}%
\pgfpathlineto{\pgfqpoint{3.864353in}{1.644015in}}%
\pgfpathlineto{\pgfqpoint{3.853925in}{1.653142in}}%
\pgfpathlineto{\pgfqpoint{3.843496in}{1.660517in}}%
\pgfpathlineto{\pgfqpoint{3.833068in}{1.686447in}}%
\pgfpathlineto{\pgfqpoint{3.822640in}{1.713995in}}%
\pgfpathlineto{\pgfqpoint{3.812212in}{1.691847in}}%
\pgfpathlineto{\pgfqpoint{3.801784in}{1.718414in}}%
\pgfpathlineto{\pgfqpoint{3.791356in}{1.673812in}}%
\pgfpathlineto{\pgfqpoint{3.780927in}{1.683259in}}%
\pgfpathlineto{\pgfqpoint{3.770499in}{1.643043in}}%
\pgfpathlineto{\pgfqpoint{3.760071in}{1.682691in}}%
\pgfpathlineto{\pgfqpoint{3.749643in}{1.686276in}}%
\pgfpathlineto{\pgfqpoint{3.739215in}{1.679030in}}%
\pgfpathlineto{\pgfqpoint{3.728787in}{1.707236in}}%
\pgfpathlineto{\pgfqpoint{3.718358in}{1.690959in}}%
\pgfpathlineto{\pgfqpoint{3.707930in}{1.701963in}}%
\pgfpathlineto{\pgfqpoint{3.697502in}{1.700787in}}%
\pgfpathlineto{\pgfqpoint{3.687074in}{1.719237in}}%
\pgfpathlineto{\pgfqpoint{3.676646in}{1.697934in}}%
\pgfpathlineto{\pgfqpoint{3.666217in}{1.697139in}}%
\pgfpathlineto{\pgfqpoint{3.655789in}{1.701997in}}%
\pgfpathlineto{\pgfqpoint{3.645361in}{1.706862in}}%
\pgfpathlineto{\pgfqpoint{3.634933in}{1.699158in}}%
\pgfpathlineto{\pgfqpoint{3.624505in}{1.723994in}}%
\pgfpathlineto{\pgfqpoint{3.614077in}{1.738725in}}%
\pgfpathlineto{\pgfqpoint{3.603648in}{1.739611in}}%
\pgfpathlineto{\pgfqpoint{3.593220in}{1.738364in}}%
\pgfpathlineto{\pgfqpoint{3.582792in}{1.740638in}}%
\pgfpathlineto{\pgfqpoint{3.572364in}{1.733672in}}%
\pgfpathlineto{\pgfqpoint{3.561936in}{1.712851in}}%
\pgfpathlineto{\pgfqpoint{3.551508in}{1.716101in}}%
\pgfpathlineto{\pgfqpoint{3.541079in}{1.726480in}}%
\pgfpathlineto{\pgfqpoint{3.530651in}{1.750683in}}%
\pgfpathlineto{\pgfqpoint{3.520223in}{1.759648in}}%
\pgfpathlineto{\pgfqpoint{3.509795in}{1.763445in}}%
\pgfpathlineto{\pgfqpoint{3.499367in}{1.731379in}}%
\pgfpathlineto{\pgfqpoint{3.488938in}{1.729918in}}%
\pgfpathlineto{\pgfqpoint{3.478510in}{1.727322in}}%
\pgfpathlineto{\pgfqpoint{3.468082in}{1.740321in}}%
\pgfpathlineto{\pgfqpoint{3.457654in}{1.737090in}}%
\pgfpathlineto{\pgfqpoint{3.447226in}{1.759507in}}%
\pgfpathlineto{\pgfqpoint{3.436798in}{1.775401in}}%
\pgfpathlineto{\pgfqpoint{3.426369in}{1.758374in}}%
\pgfpathlineto{\pgfqpoint{3.415941in}{1.746040in}}%
\pgfpathlineto{\pgfqpoint{3.405513in}{1.768270in}}%
\pgfpathlineto{\pgfqpoint{3.395085in}{1.761862in}}%
\pgfpathlineto{\pgfqpoint{3.384657in}{1.761636in}}%
\pgfpathlineto{\pgfqpoint{3.374229in}{1.770601in}}%
\pgfpathlineto{\pgfqpoint{3.363800in}{1.747738in}}%
\pgfpathlineto{\pgfqpoint{3.353372in}{1.751171in}}%
\pgfpathlineto{\pgfqpoint{3.342944in}{1.762564in}}%
\pgfpathlineto{\pgfqpoint{3.332516in}{1.761015in}}%
\pgfpathlineto{\pgfqpoint{3.322088in}{1.774865in}}%
\pgfpathlineto{\pgfqpoint{3.311660in}{1.774160in}}%
\pgfpathlineto{\pgfqpoint{3.301231in}{1.771455in}}%
\pgfpathlineto{\pgfqpoint{3.290803in}{1.776276in}}%
\pgfpathlineto{\pgfqpoint{3.280375in}{1.759670in}}%
\pgfpathlineto{\pgfqpoint{3.269947in}{1.756535in}}%
\pgfpathlineto{\pgfqpoint{3.259519in}{1.781679in}}%
\pgfpathlineto{\pgfqpoint{3.249090in}{1.784342in}}%
\pgfpathlineto{\pgfqpoint{3.238662in}{1.803357in}}%
\pgfpathlineto{\pgfqpoint{3.228234in}{1.797546in}}%
\pgfpathlineto{\pgfqpoint{3.217806in}{1.786586in}}%
\pgfpathlineto{\pgfqpoint{3.207378in}{1.784522in}}%
\pgfpathlineto{\pgfqpoint{3.196950in}{1.798258in}}%
\pgfpathlineto{\pgfqpoint{3.186521in}{1.811298in}}%
\pgfpathlineto{\pgfqpoint{3.176093in}{1.813528in}}%
\pgfpathlineto{\pgfqpoint{3.165665in}{1.833565in}}%
\pgfpathlineto{\pgfqpoint{3.155237in}{1.812895in}}%
\pgfpathlineto{\pgfqpoint{3.144809in}{1.820646in}}%
\pgfpathlineto{\pgfqpoint{3.134381in}{1.807620in}}%
\pgfpathlineto{\pgfqpoint{3.123952in}{1.815999in}}%
\pgfpathlineto{\pgfqpoint{3.113524in}{1.815678in}}%
\pgfpathlineto{\pgfqpoint{3.103096in}{1.838763in}}%
\pgfpathlineto{\pgfqpoint{3.092668in}{1.835166in}}%
\pgfpathlineto{\pgfqpoint{3.082240in}{1.844022in}}%
\pgfpathlineto{\pgfqpoint{3.071812in}{1.847108in}}%
\pgfpathlineto{\pgfqpoint{3.061383in}{1.865739in}}%
\pgfpathlineto{\pgfqpoint{3.050955in}{1.892470in}}%
\pgfpathlineto{\pgfqpoint{3.040527in}{1.859650in}}%
\pgfpathlineto{\pgfqpoint{3.030099in}{1.860972in}}%
\pgfpathlineto{\pgfqpoint{3.019671in}{1.897465in}}%
\pgfpathlineto{\pgfqpoint{3.009242in}{1.854766in}}%
\pgfpathlineto{\pgfqpoint{2.998814in}{1.869115in}}%
\pgfpathlineto{\pgfqpoint{2.988386in}{1.885640in}}%
\pgfpathlineto{\pgfqpoint{2.977958in}{1.881369in}}%
\pgfpathlineto{\pgfqpoint{2.967530in}{1.891498in}}%
\pgfpathlineto{\pgfqpoint{2.957102in}{1.896152in}}%
\pgfpathlineto{\pgfqpoint{2.946673in}{1.919207in}}%
\pgfpathlineto{\pgfqpoint{2.936245in}{1.919063in}}%
\pgfpathlineto{\pgfqpoint{2.925817in}{1.913533in}}%
\pgfpathlineto{\pgfqpoint{2.915389in}{1.940819in}}%
\pgfpathlineto{\pgfqpoint{2.904961in}{1.926503in}}%
\pgfpathlineto{\pgfqpoint{2.894533in}{1.961575in}}%
\pgfpathlineto{\pgfqpoint{2.884104in}{1.962932in}}%
\pgfpathlineto{\pgfqpoint{2.873676in}{1.960019in}}%
\pgfpathlineto{\pgfqpoint{2.863248in}{1.979091in}}%
\pgfpathlineto{\pgfqpoint{2.852820in}{1.986164in}}%
\pgfpathlineto{\pgfqpoint{2.842392in}{1.982545in}}%
\pgfpathlineto{\pgfqpoint{2.831964in}{1.994396in}}%
\pgfpathlineto{\pgfqpoint{2.821535in}{2.014159in}}%
\pgfpathlineto{\pgfqpoint{2.811107in}{2.030570in}}%
\pgfpathlineto{\pgfqpoint{2.800679in}{2.038460in}}%
\pgfpathlineto{\pgfqpoint{2.790251in}{2.063006in}}%
\pgfpathlineto{\pgfqpoint{2.779823in}{2.071496in}}%
\pgfpathlineto{\pgfqpoint{2.769394in}{2.077917in}}%
\pgfpathlineto{\pgfqpoint{2.758966in}{2.106100in}}%
\pgfpathlineto{\pgfqpoint{2.748538in}{2.104958in}}%
\pgfpathlineto{\pgfqpoint{2.738110in}{2.162863in}}%
\pgfpathlineto{\pgfqpoint{2.727682in}{2.187125in}}%
\pgfpathlineto{\pgfqpoint{2.717254in}{2.240109in}}%
\pgfpathlineto{\pgfqpoint{2.706825in}{2.317928in}}%
\pgfpathlineto{\pgfqpoint{2.696397in}{2.447338in}}%
\pgfpathlineto{\pgfqpoint{2.685969in}{2.446296in}}%
\pgfpathlineto{\pgfqpoint{2.675541in}{2.580574in}}%
\pgfpathlineto{\pgfqpoint{2.665113in}{2.677604in}}%
\pgfpathlineto{\pgfqpoint{2.654685in}{2.731048in}}%
\pgfpathlineto{\pgfqpoint{2.644256in}{2.765269in}}%
\pgfpathlineto{\pgfqpoint{2.633828in}{2.807612in}}%
\pgfpathlineto{\pgfqpoint{2.623400in}{2.833016in}}%
\pgfpathlineto{\pgfqpoint{2.612972in}{2.852334in}}%
\pgfpathlineto{\pgfqpoint{2.602544in}{2.888745in}}%
\pgfpathlineto{\pgfqpoint{2.592115in}{2.896671in}}%
\pgfpathlineto{\pgfqpoint{2.581687in}{2.916885in}}%
\pgfpathlineto{\pgfqpoint{2.571259in}{2.960784in}}%
\pgfpathlineto{\pgfqpoint{2.560831in}{2.950027in}}%
\pgfpathlineto{\pgfqpoint{2.550403in}{2.973528in}}%
\pgfpathlineto{\pgfqpoint{2.539975in}{2.969567in}}%
\pgfpathlineto{\pgfqpoint{2.529546in}{2.980418in}}%
\pgfpathlineto{\pgfqpoint{2.519118in}{3.001838in}}%
\pgfpathlineto{\pgfqpoint{2.508690in}{3.006587in}}%
\pgfpathlineto{\pgfqpoint{2.498262in}{3.003617in}}%
\pgfpathlineto{\pgfqpoint{2.487834in}{3.006632in}}%
\pgfpathlineto{\pgfqpoint{2.477406in}{3.013012in}}%
\pgfpathlineto{\pgfqpoint{2.466977in}{2.992698in}}%
\pgfpathlineto{\pgfqpoint{2.456549in}{2.988856in}}%
\pgfpathlineto{\pgfqpoint{2.446121in}{2.995102in}}%
\pgfpathlineto{\pgfqpoint{2.435693in}{2.971696in}}%
\pgfpathlineto{\pgfqpoint{2.425265in}{2.999276in}}%
\pgfpathlineto{\pgfqpoint{2.414837in}{2.995390in}}%
\pgfpathlineto{\pgfqpoint{2.404408in}{3.005183in}}%
\pgfpathlineto{\pgfqpoint{2.393980in}{3.008148in}}%
\pgfpathlineto{\pgfqpoint{2.383552in}{3.006719in}}%
\pgfpathlineto{\pgfqpoint{2.373124in}{3.002052in}}%
\pgfpathlineto{\pgfqpoint{2.362696in}{3.013042in}}%
\pgfpathlineto{\pgfqpoint{2.352267in}{3.022196in}}%
\pgfpathlineto{\pgfqpoint{2.341839in}{3.011573in}}%
\pgfpathlineto{\pgfqpoint{2.331411in}{3.017413in}}%
\pgfpathlineto{\pgfqpoint{2.320983in}{2.994822in}}%
\pgfpathlineto{\pgfqpoint{2.310555in}{3.037524in}}%
\pgfpathlineto{\pgfqpoint{2.300127in}{3.029396in}}%
\pgfpathlineto{\pgfqpoint{2.289698in}{3.026885in}}%
\pgfpathlineto{\pgfqpoint{2.279270in}{3.014941in}}%
\pgfpathlineto{\pgfqpoint{2.268842in}{3.028758in}}%
\pgfpathlineto{\pgfqpoint{2.258414in}{3.003379in}}%
\pgfpathlineto{\pgfqpoint{2.247986in}{3.008009in}}%
\pgfpathlineto{\pgfqpoint{2.237558in}{2.998483in}}%
\pgfpathlineto{\pgfqpoint{2.227129in}{2.979564in}}%
\pgfpathlineto{\pgfqpoint{2.216701in}{2.987276in}}%
\pgfpathlineto{\pgfqpoint{2.206273in}{2.999789in}}%
\pgfpathlineto{\pgfqpoint{2.195845in}{2.995477in}}%
\pgfpathlineto{\pgfqpoint{2.185417in}{3.002729in}}%
\pgfpathlineto{\pgfqpoint{2.174989in}{2.968849in}}%
\pgfpathlineto{\pgfqpoint{2.164560in}{3.020861in}}%
\pgfpathlineto{\pgfqpoint{2.154132in}{3.011779in}}%
\pgfpathlineto{\pgfqpoint{2.143704in}{3.001653in}}%
\pgfpathlineto{\pgfqpoint{2.133276in}{2.987957in}}%
\pgfpathlineto{\pgfqpoint{2.122848in}{2.981381in}}%
\pgfpathlineto{\pgfqpoint{2.112419in}{3.017125in}}%
\pgfpathlineto{\pgfqpoint{2.101991in}{3.006298in}}%
\pgfpathlineto{\pgfqpoint{2.091563in}{3.030061in}}%
\pgfpathlineto{\pgfqpoint{2.081135in}{3.010701in}}%
\pgfpathlineto{\pgfqpoint{2.070707in}{2.994336in}}%
\pgfpathlineto{\pgfqpoint{2.060279in}{3.007562in}}%
\pgfpathlineto{\pgfqpoint{2.049850in}{2.997519in}}%
\pgfpathlineto{\pgfqpoint{2.039422in}{3.041540in}}%
\pgfpathlineto{\pgfqpoint{2.028994in}{3.019111in}}%
\pgfpathlineto{\pgfqpoint{2.018566in}{3.013676in}}%
\pgfpathlineto{\pgfqpoint{2.008138in}{3.035941in}}%
\pgfpathlineto{\pgfqpoint{1.997710in}{3.028388in}}%
\pgfpathlineto{\pgfqpoint{1.987281in}{3.021593in}}%
\pgfpathlineto{\pgfqpoint{1.976853in}{3.058249in}}%
\pgfpathlineto{\pgfqpoint{1.966425in}{3.058996in}}%
\pgfpathlineto{\pgfqpoint{1.955997in}{3.051311in}}%
\pgfpathlineto{\pgfqpoint{1.945569in}{3.034935in}}%
\pgfpathlineto{\pgfqpoint{1.935141in}{3.023723in}}%
\pgfpathlineto{\pgfqpoint{1.924712in}{3.047475in}}%
\pgfpathlineto{\pgfqpoint{1.914284in}{3.052433in}}%
\pgfpathlineto{\pgfqpoint{1.903856in}{3.028511in}}%
\pgfpathlineto{\pgfqpoint{1.893428in}{3.025634in}}%
\pgfpathlineto{\pgfqpoint{1.883000in}{3.017468in}}%
\pgfpathlineto{\pgfqpoint{1.872571in}{3.007256in}}%
\pgfpathlineto{\pgfqpoint{1.862143in}{3.022822in}}%
\pgfpathlineto{\pgfqpoint{1.851715in}{3.040582in}}%
\pgfpathlineto{\pgfqpoint{1.841287in}{3.010165in}}%
\pgfpathlineto{\pgfqpoint{1.830859in}{2.995818in}}%
\pgfpathlineto{\pgfqpoint{1.820431in}{2.978622in}}%
\pgfpathlineto{\pgfqpoint{1.810002in}{2.995113in}}%
\pgfpathlineto{\pgfqpoint{1.799574in}{2.988085in}}%
\pgfpathlineto{\pgfqpoint{1.789146in}{3.001046in}}%
\pgfpathlineto{\pgfqpoint{1.778718in}{2.985549in}}%
\pgfpathlineto{\pgfqpoint{1.768290in}{3.000429in}}%
\pgfpathlineto{\pgfqpoint{1.757862in}{3.000011in}}%
\pgfpathlineto{\pgfqpoint{1.747433in}{2.984302in}}%
\pgfpathlineto{\pgfqpoint{1.737005in}{2.979601in}}%
\pgfpathlineto{\pgfqpoint{1.726577in}{2.973288in}}%
\pgfpathlineto{\pgfqpoint{1.716149in}{3.015810in}}%
\pgfpathlineto{\pgfqpoint{1.705721in}{3.007752in}}%
\pgfpathlineto{\pgfqpoint{1.695292in}{3.000838in}}%
\pgfpathlineto{\pgfqpoint{1.684864in}{2.998943in}}%
\pgfpathlineto{\pgfqpoint{1.674436in}{2.972145in}}%
\pgfpathlineto{\pgfqpoint{1.664008in}{2.988621in}}%
\pgfpathlineto{\pgfqpoint{1.653580in}{2.958416in}}%
\pgfpathlineto{\pgfqpoint{1.643152in}{2.968474in}}%
\pgfpathlineto{\pgfqpoint{1.632723in}{2.949020in}}%
\pgfpathlineto{\pgfqpoint{1.622295in}{2.955345in}}%
\pgfpathlineto{\pgfqpoint{1.611867in}{2.958956in}}%
\pgfpathlineto{\pgfqpoint{1.601439in}{2.947072in}}%
\pgfpathlineto{\pgfqpoint{1.591011in}{2.980224in}}%
\pgfpathlineto{\pgfqpoint{1.580583in}{2.983309in}}%
\pgfpathlineto{\pgfqpoint{1.570154in}{2.987746in}}%
\pgfpathlineto{\pgfqpoint{1.559726in}{2.967240in}}%
\pgfpathlineto{\pgfqpoint{1.549298in}{2.957515in}}%
\pgfpathlineto{\pgfqpoint{1.538870in}{2.946524in}}%
\pgfpathlineto{\pgfqpoint{1.528442in}{2.952692in}}%
\pgfpathlineto{\pgfqpoint{1.518014in}{2.937823in}}%
\pgfpathlineto{\pgfqpoint{1.507585in}{2.965661in}}%
\pgfpathlineto{\pgfqpoint{1.497157in}{2.995809in}}%
\pgfpathlineto{\pgfqpoint{1.486729in}{2.978956in}}%
\pgfpathlineto{\pgfqpoint{1.476301in}{2.974924in}}%
\pgfpathlineto{\pgfqpoint{1.465873in}{3.033937in}}%
\pgfpathlineto{\pgfqpoint{1.455444in}{3.065653in}}%
\pgfpathlineto{\pgfqpoint{1.445016in}{3.050942in}}%
\pgfpathlineto{\pgfqpoint{1.434588in}{3.040917in}}%
\pgfpathlineto{\pgfqpoint{1.424160in}{3.023699in}}%
\pgfpathlineto{\pgfqpoint{1.413732in}{3.034477in}}%
\pgfpathlineto{\pgfqpoint{1.403304in}{3.025984in}}%
\pgfpathlineto{\pgfqpoint{1.392875in}{3.035613in}}%
\pgfpathlineto{\pgfqpoint{1.382447in}{3.049383in}}%
\pgfpathlineto{\pgfqpoint{1.372019in}{3.044149in}}%
\pgfpathlineto{\pgfqpoint{1.361591in}{3.037579in}}%
\pgfpathlineto{\pgfqpoint{1.351163in}{3.034572in}}%
\pgfpathlineto{\pgfqpoint{1.340735in}{3.020069in}}%
\pgfpathlineto{\pgfqpoint{1.330306in}{3.046496in}}%
\pgfpathlineto{\pgfqpoint{1.319878in}{3.038072in}}%
\pgfpathlineto{\pgfqpoint{1.309450in}{3.038199in}}%
\pgfpathlineto{\pgfqpoint{1.299022in}{3.027569in}}%
\pgfpathlineto{\pgfqpoint{1.288594in}{3.000335in}}%
\pgfpathlineto{\pgfqpoint{1.278166in}{2.991109in}}%
\pgfpathlineto{\pgfqpoint{1.267737in}{3.023457in}}%
\pgfpathlineto{\pgfqpoint{1.257309in}{3.010976in}}%
\pgfpathlineto{\pgfqpoint{1.246881in}{3.016277in}}%
\pgfpathlineto{\pgfqpoint{1.236453in}{2.990768in}}%
\pgfpathlineto{\pgfqpoint{1.226025in}{2.957325in}}%
\pgfpathlineto{\pgfqpoint{1.215596in}{2.992012in}}%
\pgfpathlineto{\pgfqpoint{1.205168in}{2.990642in}}%
\pgfpathlineto{\pgfqpoint{1.194740in}{2.977270in}}%
\pgfpathlineto{\pgfqpoint{1.184312in}{3.000179in}}%
\pgfpathlineto{\pgfqpoint{1.173884in}{2.996835in}}%
\pgfpathlineto{\pgfqpoint{1.163456in}{3.014888in}}%
\pgfpathlineto{\pgfqpoint{1.153027in}{3.040908in}}%
\pgfpathlineto{\pgfqpoint{1.142599in}{3.030490in}}%
\pgfpathlineto{\pgfqpoint{1.132171in}{3.034354in}}%
\pgfpathlineto{\pgfqpoint{1.121743in}{2.990526in}}%
\pgfpathlineto{\pgfqpoint{1.111315in}{2.987119in}}%
\pgfpathlineto{\pgfqpoint{1.100887in}{2.984529in}}%
\pgfpathlineto{\pgfqpoint{1.090458in}{3.019625in}}%
\pgfpathlineto{\pgfqpoint{1.080030in}{2.995955in}}%
\pgfpathlineto{\pgfqpoint{1.069602in}{3.026871in}}%
\pgfpathlineto{\pgfqpoint{1.059174in}{3.042839in}}%
\pgfpathlineto{\pgfqpoint{1.048746in}{3.016588in}}%
\pgfpathlineto{\pgfqpoint{1.038318in}{3.028677in}}%
\pgfpathlineto{\pgfqpoint{1.027889in}{3.047867in}}%
\pgfpathlineto{\pgfqpoint{1.017461in}{3.068893in}}%
\pgfpathlineto{\pgfqpoint{1.007033in}{3.097477in}}%
\pgfpathlineto{\pgfqpoint{0.996605in}{3.084077in}}%
\pgfpathlineto{\pgfqpoint{0.986177in}{3.084008in}}%
\pgfpathlineto{\pgfqpoint{0.975748in}{3.050743in}}%
\pgfpathlineto{\pgfqpoint{0.965320in}{3.024083in}}%
\pgfpathlineto{\pgfqpoint{0.954892in}{3.015504in}}%
\pgfpathlineto{\pgfqpoint{0.944464in}{3.018219in}}%
\pgfpathlineto{\pgfqpoint{0.934036in}{2.988707in}}%
\pgfpathlineto{\pgfqpoint{0.923608in}{2.997691in}}%
\pgfpathlineto{\pgfqpoint{0.913179in}{3.006863in}}%
\pgfpathlineto{\pgfqpoint{0.902751in}{3.005480in}}%
\pgfpathlineto{\pgfqpoint{0.892323in}{2.985853in}}%
\pgfpathlineto{\pgfqpoint{0.881895in}{2.985411in}}%
\pgfpathlineto{\pgfqpoint{0.871467in}{3.017953in}}%
\pgfpathlineto{\pgfqpoint{0.861039in}{2.977231in}}%
\pgfpathlineto{\pgfqpoint{0.850610in}{3.074818in}}%
\pgfpathlineto{\pgfqpoint{0.840182in}{3.051927in}}%
\pgfpathlineto{\pgfqpoint{0.829754in}{3.053069in}}%
\pgfpathlineto{\pgfqpoint{0.819326in}{3.007252in}}%
\pgfpathlineto{\pgfqpoint{0.808898in}{3.024895in}}%
\pgfpathlineto{\pgfqpoint{0.798470in}{3.021876in}}%
\pgfpathlineto{\pgfqpoint{0.788041in}{3.037788in}}%
\pgfpathlineto{\pgfqpoint{0.777613in}{3.008683in}}%
\pgfpathlineto{\pgfqpoint{0.767185in}{3.011103in}}%
\pgfpathlineto{\pgfqpoint{0.756757in}{3.065573in}}%
\pgfpathlineto{\pgfqpoint{0.746329in}{3.065097in}}%
\pgfpathlineto{\pgfqpoint{0.735900in}{3.114897in}}%
\pgfpathlineto{\pgfqpoint{0.725472in}{3.066954in}}%
\pgfpathlineto{\pgfqpoint{0.715044in}{3.091730in}}%
\pgfpathlineto{\pgfqpoint{0.704616in}{3.062161in}}%
\pgfpathlineto{\pgfqpoint{0.694188in}{3.058294in}}%
\pgfpathlineto{\pgfqpoint{0.683760in}{3.072860in}}%
\pgfpathlineto{\pgfqpoint{0.673331in}{3.046416in}}%
\pgfpathlineto{\pgfqpoint{0.662903in}{3.067411in}}%
\pgfpathlineto{\pgfqpoint{0.652475in}{3.040489in}}%
\pgfpathlineto{\pgfqpoint{0.642047in}{3.056971in}}%
\pgfpathlineto{\pgfqpoint{0.631619in}{3.058399in}}%
\pgfpathlineto{\pgfqpoint{0.621191in}{3.053478in}}%
\pgfpathlineto{\pgfqpoint{0.610762in}{3.037590in}}%
\pgfpathclose%
\pgfusepath{stroke,fill}%
\end{pgfscope}%
\begin{pgfscope}%
\pgfpathrectangle{\pgfqpoint{0.610762in}{0.961156in}}{\pgfqpoint{4.171270in}{2.577986in}} %
\pgfusepath{clip}%
\pgfsetbuttcap%
\pgfsetroundjoin%
\definecolor{currentfill}{rgb}{0.168627,0.670588,0.494118}%
\pgfsetfillcolor{currentfill}%
\pgfsetfillopacity{0.200000}%
\pgfsetlinewidth{0.301125pt}%
\definecolor{currentstroke}{rgb}{0.000000,0.000000,0.000000}%
\pgfsetstrokecolor{currentstroke}%
\pgfsetstrokeopacity{0.200000}%
\pgfsetdash{}{0pt}%
\pgfpathmoveto{\pgfqpoint{0.610762in}{3.031977in}}%
\pgfpathlineto{\pgfqpoint{0.610762in}{3.158722in}}%
\pgfpathlineto{\pgfqpoint{0.621191in}{3.151647in}}%
\pgfpathlineto{\pgfqpoint{0.631619in}{3.208030in}}%
\pgfpathlineto{\pgfqpoint{0.642047in}{3.197725in}}%
\pgfpathlineto{\pgfqpoint{0.652475in}{3.237611in}}%
\pgfpathlineto{\pgfqpoint{0.662903in}{3.176519in}}%
\pgfpathlineto{\pgfqpoint{0.673331in}{3.180530in}}%
\pgfpathlineto{\pgfqpoint{0.683760in}{3.162059in}}%
\pgfpathlineto{\pgfqpoint{0.694188in}{3.227312in}}%
\pgfpathlineto{\pgfqpoint{0.704616in}{3.172623in}}%
\pgfpathlineto{\pgfqpoint{0.715044in}{3.146269in}}%
\pgfpathlineto{\pgfqpoint{0.725472in}{3.183002in}}%
\pgfpathlineto{\pgfqpoint{0.735900in}{3.175039in}}%
\pgfpathlineto{\pgfqpoint{0.746329in}{3.191672in}}%
\pgfpathlineto{\pgfqpoint{0.756757in}{3.249743in}}%
\pgfpathlineto{\pgfqpoint{0.767185in}{3.177447in}}%
\pgfpathlineto{\pgfqpoint{0.777613in}{3.197866in}}%
\pgfpathlineto{\pgfqpoint{0.788041in}{3.211767in}}%
\pgfpathlineto{\pgfqpoint{0.798470in}{3.142975in}}%
\pgfpathlineto{\pgfqpoint{0.808898in}{3.193557in}}%
\pgfpathlineto{\pgfqpoint{0.819326in}{3.154019in}}%
\pgfpathlineto{\pgfqpoint{0.829754in}{3.163331in}}%
\pgfpathlineto{\pgfqpoint{0.840182in}{3.151280in}}%
\pgfpathlineto{\pgfqpoint{0.850610in}{3.131382in}}%
\pgfpathlineto{\pgfqpoint{0.861039in}{3.063640in}}%
\pgfpathlineto{\pgfqpoint{0.871467in}{3.094089in}}%
\pgfpathlineto{\pgfqpoint{0.881895in}{3.148978in}}%
\pgfpathlineto{\pgfqpoint{0.892323in}{3.138520in}}%
\pgfpathlineto{\pgfqpoint{0.902751in}{3.123171in}}%
\pgfpathlineto{\pgfqpoint{0.913179in}{3.152876in}}%
\pgfpathlineto{\pgfqpoint{0.923608in}{3.176411in}}%
\pgfpathlineto{\pgfqpoint{0.934036in}{3.106061in}}%
\pgfpathlineto{\pgfqpoint{0.944464in}{3.199171in}}%
\pgfpathlineto{\pgfqpoint{0.954892in}{3.201698in}}%
\pgfpathlineto{\pgfqpoint{0.965320in}{3.158247in}}%
\pgfpathlineto{\pgfqpoint{0.975748in}{3.122716in}}%
\pgfpathlineto{\pgfqpoint{0.986177in}{3.126032in}}%
\pgfpathlineto{\pgfqpoint{0.996605in}{3.133721in}}%
\pgfpathlineto{\pgfqpoint{1.007033in}{3.168330in}}%
\pgfpathlineto{\pgfqpoint{1.017461in}{3.169220in}}%
\pgfpathlineto{\pgfqpoint{1.027889in}{3.246088in}}%
\pgfpathlineto{\pgfqpoint{1.038318in}{3.165206in}}%
\pgfpathlineto{\pgfqpoint{1.048746in}{3.163324in}}%
\pgfpathlineto{\pgfqpoint{1.059174in}{3.120503in}}%
\pgfpathlineto{\pgfqpoint{1.069602in}{3.116234in}}%
\pgfpathlineto{\pgfqpoint{1.080030in}{3.114423in}}%
\pgfpathlineto{\pgfqpoint{1.090458in}{3.171580in}}%
\pgfpathlineto{\pgfqpoint{1.100887in}{3.164489in}}%
\pgfpathlineto{\pgfqpoint{1.111315in}{3.065833in}}%
\pgfpathlineto{\pgfqpoint{1.121743in}{3.098789in}}%
\pgfpathlineto{\pgfqpoint{1.132171in}{3.057557in}}%
\pgfpathlineto{\pgfqpoint{1.142599in}{3.121550in}}%
\pgfpathlineto{\pgfqpoint{1.153027in}{3.089814in}}%
\pgfpathlineto{\pgfqpoint{1.163456in}{3.108947in}}%
\pgfpathlineto{\pgfqpoint{1.173884in}{3.029849in}}%
\pgfpathlineto{\pgfqpoint{1.184312in}{2.992408in}}%
\pgfpathlineto{\pgfqpoint{1.194740in}{2.986939in}}%
\pgfpathlineto{\pgfqpoint{1.205168in}{3.018259in}}%
\pgfpathlineto{\pgfqpoint{1.215596in}{3.061526in}}%
\pgfpathlineto{\pgfqpoint{1.226025in}{3.115442in}}%
\pgfpathlineto{\pgfqpoint{1.236453in}{3.141440in}}%
\pgfpathlineto{\pgfqpoint{1.246881in}{3.119722in}}%
\pgfpathlineto{\pgfqpoint{1.257309in}{3.142775in}}%
\pgfpathlineto{\pgfqpoint{1.267737in}{3.079839in}}%
\pgfpathlineto{\pgfqpoint{1.278166in}{3.116142in}}%
\pgfpathlineto{\pgfqpoint{1.288594in}{3.081805in}}%
\pgfpathlineto{\pgfqpoint{1.299022in}{3.086928in}}%
\pgfpathlineto{\pgfqpoint{1.309450in}{3.140339in}}%
\pgfpathlineto{\pgfqpoint{1.319878in}{3.131510in}}%
\pgfpathlineto{\pgfqpoint{1.330306in}{3.155023in}}%
\pgfpathlineto{\pgfqpoint{1.340735in}{3.144044in}}%
\pgfpathlineto{\pgfqpoint{1.351163in}{3.139493in}}%
\pgfpathlineto{\pgfqpoint{1.361591in}{3.171347in}}%
\pgfpathlineto{\pgfqpoint{1.372019in}{3.097683in}}%
\pgfpathlineto{\pgfqpoint{1.382447in}{3.130410in}}%
\pgfpathlineto{\pgfqpoint{1.392875in}{3.140532in}}%
\pgfpathlineto{\pgfqpoint{1.403304in}{3.197729in}}%
\pgfpathlineto{\pgfqpoint{1.413732in}{3.203924in}}%
\pgfpathlineto{\pgfqpoint{1.424160in}{3.176666in}}%
\pgfpathlineto{\pgfqpoint{1.434588in}{3.167850in}}%
\pgfpathlineto{\pgfqpoint{1.445016in}{3.150119in}}%
\pgfpathlineto{\pgfqpoint{1.455444in}{3.188440in}}%
\pgfpathlineto{\pgfqpoint{1.465873in}{3.200262in}}%
\pgfpathlineto{\pgfqpoint{1.476301in}{3.248449in}}%
\pgfpathlineto{\pgfqpoint{1.486729in}{3.246873in}}%
\pgfpathlineto{\pgfqpoint{1.497157in}{3.205230in}}%
\pgfpathlineto{\pgfqpoint{1.507585in}{3.260092in}}%
\pgfpathlineto{\pgfqpoint{1.518014in}{3.237141in}}%
\pgfpathlineto{\pgfqpoint{1.528442in}{3.214537in}}%
\pgfpathlineto{\pgfqpoint{1.538870in}{3.211990in}}%
\pgfpathlineto{\pgfqpoint{1.549298in}{3.182497in}}%
\pgfpathlineto{\pgfqpoint{1.559726in}{3.197793in}}%
\pgfpathlineto{\pgfqpoint{1.570154in}{3.174976in}}%
\pgfpathlineto{\pgfqpoint{1.580583in}{3.106295in}}%
\pgfpathlineto{\pgfqpoint{1.591011in}{3.192279in}}%
\pgfpathlineto{\pgfqpoint{1.601439in}{3.171348in}}%
\pgfpathlineto{\pgfqpoint{1.611867in}{3.221198in}}%
\pgfpathlineto{\pgfqpoint{1.622295in}{3.264924in}}%
\pgfpathlineto{\pgfqpoint{1.632723in}{3.180002in}}%
\pgfpathlineto{\pgfqpoint{1.643152in}{3.192812in}}%
\pgfpathlineto{\pgfqpoint{1.653580in}{3.131953in}}%
\pgfpathlineto{\pgfqpoint{1.664008in}{3.086852in}}%
\pgfpathlineto{\pgfqpoint{1.674436in}{3.084732in}}%
\pgfpathlineto{\pgfqpoint{1.684864in}{3.161236in}}%
\pgfpathlineto{\pgfqpoint{1.695292in}{3.121274in}}%
\pgfpathlineto{\pgfqpoint{1.705721in}{3.161938in}}%
\pgfpathlineto{\pgfqpoint{1.716149in}{3.172165in}}%
\pgfpathlineto{\pgfqpoint{1.726577in}{3.146946in}}%
\pgfpathlineto{\pgfqpoint{1.737005in}{3.074208in}}%
\pgfpathlineto{\pgfqpoint{1.747433in}{3.034218in}}%
\pgfpathlineto{\pgfqpoint{1.757862in}{3.040265in}}%
\pgfpathlineto{\pgfqpoint{1.768290in}{3.107439in}}%
\pgfpathlineto{\pgfqpoint{1.778718in}{3.131270in}}%
\pgfpathlineto{\pgfqpoint{1.789146in}{3.131652in}}%
\pgfpathlineto{\pgfqpoint{1.799574in}{3.208288in}}%
\pgfpathlineto{\pgfqpoint{1.810002in}{3.111337in}}%
\pgfpathlineto{\pgfqpoint{1.820431in}{3.164669in}}%
\pgfpathlineto{\pgfqpoint{1.830859in}{3.174944in}}%
\pgfpathlineto{\pgfqpoint{1.841287in}{3.122445in}}%
\pgfpathlineto{\pgfqpoint{1.851715in}{3.143078in}}%
\pgfpathlineto{\pgfqpoint{1.862143in}{3.096150in}}%
\pgfpathlineto{\pgfqpoint{1.872571in}{3.095548in}}%
\pgfpathlineto{\pgfqpoint{1.883000in}{3.098785in}}%
\pgfpathlineto{\pgfqpoint{1.893428in}{3.132254in}}%
\pgfpathlineto{\pgfqpoint{1.903856in}{3.133353in}}%
\pgfpathlineto{\pgfqpoint{1.914284in}{3.173917in}}%
\pgfpathlineto{\pgfqpoint{1.924712in}{3.180845in}}%
\pgfpathlineto{\pgfqpoint{1.935141in}{3.160416in}}%
\pgfpathlineto{\pgfqpoint{1.945569in}{3.176525in}}%
\pgfpathlineto{\pgfqpoint{1.955997in}{3.188256in}}%
\pgfpathlineto{\pgfqpoint{1.966425in}{3.166104in}}%
\pgfpathlineto{\pgfqpoint{1.976853in}{3.136536in}}%
\pgfpathlineto{\pgfqpoint{1.987281in}{3.186681in}}%
\pgfpathlineto{\pgfqpoint{1.997710in}{3.122977in}}%
\pgfpathlineto{\pgfqpoint{2.008138in}{3.142631in}}%
\pgfpathlineto{\pgfqpoint{2.018566in}{3.142665in}}%
\pgfpathlineto{\pgfqpoint{2.028994in}{3.108026in}}%
\pgfpathlineto{\pgfqpoint{2.039422in}{3.124052in}}%
\pgfpathlineto{\pgfqpoint{2.049850in}{3.158520in}}%
\pgfpathlineto{\pgfqpoint{2.060279in}{3.158880in}}%
\pgfpathlineto{\pgfqpoint{2.070707in}{3.182885in}}%
\pgfpathlineto{\pgfqpoint{2.081135in}{3.160532in}}%
\pgfpathlineto{\pgfqpoint{2.091563in}{3.152295in}}%
\pgfpathlineto{\pgfqpoint{2.101991in}{3.229545in}}%
\pgfpathlineto{\pgfqpoint{2.112419in}{3.178246in}}%
\pgfpathlineto{\pgfqpoint{2.122848in}{3.166901in}}%
\pgfpathlineto{\pgfqpoint{2.133276in}{3.170845in}}%
\pgfpathlineto{\pgfqpoint{2.143704in}{3.119417in}}%
\pgfpathlineto{\pgfqpoint{2.154132in}{3.145375in}}%
\pgfpathlineto{\pgfqpoint{2.164560in}{3.151178in}}%
\pgfpathlineto{\pgfqpoint{2.174989in}{3.145547in}}%
\pgfpathlineto{\pgfqpoint{2.185417in}{3.131307in}}%
\pgfpathlineto{\pgfqpoint{2.195845in}{3.130564in}}%
\pgfpathlineto{\pgfqpoint{2.206273in}{3.134902in}}%
\pgfpathlineto{\pgfqpoint{2.216701in}{3.144179in}}%
\pgfpathlineto{\pgfqpoint{2.227129in}{3.110631in}}%
\pgfpathlineto{\pgfqpoint{2.237558in}{3.078527in}}%
\pgfpathlineto{\pgfqpoint{2.247986in}{3.079377in}}%
\pgfpathlineto{\pgfqpoint{2.258414in}{3.063497in}}%
\pgfpathlineto{\pgfqpoint{2.268842in}{3.018225in}}%
\pgfpathlineto{\pgfqpoint{2.279270in}{2.989477in}}%
\pgfpathlineto{\pgfqpoint{2.289698in}{3.014109in}}%
\pgfpathlineto{\pgfqpoint{2.300127in}{3.015636in}}%
\pgfpathlineto{\pgfqpoint{2.310555in}{3.001513in}}%
\pgfpathlineto{\pgfqpoint{2.320983in}{2.992006in}}%
\pgfpathlineto{\pgfqpoint{2.331411in}{2.990305in}}%
\pgfpathlineto{\pgfqpoint{2.341839in}{2.987006in}}%
\pgfpathlineto{\pgfqpoint{2.352267in}{2.973007in}}%
\pgfpathlineto{\pgfqpoint{2.362696in}{2.977970in}}%
\pgfpathlineto{\pgfqpoint{2.373124in}{2.955772in}}%
\pgfpathlineto{\pgfqpoint{2.383552in}{2.992775in}}%
\pgfpathlineto{\pgfqpoint{2.393980in}{2.971316in}}%
\pgfpathlineto{\pgfqpoint{2.404408in}{2.948502in}}%
\pgfpathlineto{\pgfqpoint{2.414837in}{2.951577in}}%
\pgfpathlineto{\pgfqpoint{2.425265in}{2.938257in}}%
\pgfpathlineto{\pgfqpoint{2.435693in}{2.974020in}}%
\pgfpathlineto{\pgfqpoint{2.446121in}{2.967913in}}%
\pgfpathlineto{\pgfqpoint{2.456549in}{2.985413in}}%
\pgfpathlineto{\pgfqpoint{2.466977in}{3.017833in}}%
\pgfpathlineto{\pgfqpoint{2.477406in}{3.016326in}}%
\pgfpathlineto{\pgfqpoint{2.487834in}{2.996356in}}%
\pgfpathlineto{\pgfqpoint{2.498262in}{2.991573in}}%
\pgfpathlineto{\pgfqpoint{2.508690in}{2.972480in}}%
\pgfpathlineto{\pgfqpoint{2.519118in}{2.937885in}}%
\pgfpathlineto{\pgfqpoint{2.529546in}{2.946858in}}%
\pgfpathlineto{\pgfqpoint{2.539975in}{2.901713in}}%
\pgfpathlineto{\pgfqpoint{2.550403in}{2.879590in}}%
\pgfpathlineto{\pgfqpoint{2.560831in}{2.906007in}}%
\pgfpathlineto{\pgfqpoint{2.571259in}{2.841431in}}%
\pgfpathlineto{\pgfqpoint{2.581687in}{2.829266in}}%
\pgfpathlineto{\pgfqpoint{2.592115in}{2.823468in}}%
\pgfpathlineto{\pgfqpoint{2.602544in}{2.780878in}}%
\pgfpathlineto{\pgfqpoint{2.612972in}{2.803787in}}%
\pgfpathlineto{\pgfqpoint{2.623400in}{2.786360in}}%
\pgfpathlineto{\pgfqpoint{2.633828in}{2.724253in}}%
\pgfpathlineto{\pgfqpoint{2.644256in}{2.719700in}}%
\pgfpathlineto{\pgfqpoint{2.654685in}{2.681502in}}%
\pgfpathlineto{\pgfqpoint{2.665113in}{2.629238in}}%
\pgfpathlineto{\pgfqpoint{2.675541in}{2.523582in}}%
\pgfpathlineto{\pgfqpoint{2.685969in}{2.370742in}}%
\pgfpathlineto{\pgfqpoint{2.696397in}{2.395900in}}%
\pgfpathlineto{\pgfqpoint{2.706825in}{2.262865in}}%
\pgfpathlineto{\pgfqpoint{2.717254in}{2.176507in}}%
\pgfpathlineto{\pgfqpoint{2.727682in}{2.107253in}}%
\pgfpathlineto{\pgfqpoint{2.738110in}{2.064357in}}%
\pgfpathlineto{\pgfqpoint{2.748538in}{2.055854in}}%
\pgfpathlineto{\pgfqpoint{2.758966in}{2.050984in}}%
\pgfpathlineto{\pgfqpoint{2.769394in}{2.070155in}}%
\pgfpathlineto{\pgfqpoint{2.779823in}{2.032778in}}%
\pgfpathlineto{\pgfqpoint{2.790251in}{2.015077in}}%
\pgfpathlineto{\pgfqpoint{2.800679in}{1.982633in}}%
\pgfpathlineto{\pgfqpoint{2.811107in}{1.956863in}}%
\pgfpathlineto{\pgfqpoint{2.821535in}{1.968729in}}%
\pgfpathlineto{\pgfqpoint{2.831964in}{1.952216in}}%
\pgfpathlineto{\pgfqpoint{2.842392in}{1.947312in}}%
\pgfpathlineto{\pgfqpoint{2.852820in}{1.901923in}}%
\pgfpathlineto{\pgfqpoint{2.863248in}{1.880225in}}%
\pgfpathlineto{\pgfqpoint{2.873676in}{1.885346in}}%
\pgfpathlineto{\pgfqpoint{2.884104in}{1.854066in}}%
\pgfpathlineto{\pgfqpoint{2.894533in}{1.852253in}}%
\pgfpathlineto{\pgfqpoint{2.904961in}{1.809346in}}%
\pgfpathlineto{\pgfqpoint{2.915389in}{1.820127in}}%
\pgfpathlineto{\pgfqpoint{2.925817in}{1.801911in}}%
\pgfpathlineto{\pgfqpoint{2.936245in}{1.830830in}}%
\pgfpathlineto{\pgfqpoint{2.946673in}{1.833205in}}%
\pgfpathlineto{\pgfqpoint{2.957102in}{1.806250in}}%
\pgfpathlineto{\pgfqpoint{2.967530in}{1.813192in}}%
\pgfpathlineto{\pgfqpoint{2.977958in}{1.805689in}}%
\pgfpathlineto{\pgfqpoint{2.988386in}{1.755109in}}%
\pgfpathlineto{\pgfqpoint{2.998814in}{1.765891in}}%
\pgfpathlineto{\pgfqpoint{3.009242in}{1.773788in}}%
\pgfpathlineto{\pgfqpoint{3.019671in}{1.744808in}}%
\pgfpathlineto{\pgfqpoint{3.030099in}{1.758630in}}%
\pgfpathlineto{\pgfqpoint{3.040527in}{1.750408in}}%
\pgfpathlineto{\pgfqpoint{3.050955in}{1.735805in}}%
\pgfpathlineto{\pgfqpoint{3.061383in}{1.756818in}}%
\pgfpathlineto{\pgfqpoint{3.071812in}{1.743513in}}%
\pgfpathlineto{\pgfqpoint{3.082240in}{1.710639in}}%
\pgfpathlineto{\pgfqpoint{3.092668in}{1.704508in}}%
\pgfpathlineto{\pgfqpoint{3.103096in}{1.726389in}}%
\pgfpathlineto{\pgfqpoint{3.113524in}{1.712295in}}%
\pgfpathlineto{\pgfqpoint{3.123952in}{1.675607in}}%
\pgfpathlineto{\pgfqpoint{3.134381in}{1.702225in}}%
\pgfpathlineto{\pgfqpoint{3.144809in}{1.729926in}}%
\pgfpathlineto{\pgfqpoint{3.155237in}{1.701189in}}%
\pgfpathlineto{\pgfqpoint{3.165665in}{1.675936in}}%
\pgfpathlineto{\pgfqpoint{3.176093in}{1.655297in}}%
\pgfpathlineto{\pgfqpoint{3.186521in}{1.665983in}}%
\pgfpathlineto{\pgfqpoint{3.196950in}{1.680987in}}%
\pgfpathlineto{\pgfqpoint{3.207378in}{1.654485in}}%
\pgfpathlineto{\pgfqpoint{3.217806in}{1.648445in}}%
\pgfpathlineto{\pgfqpoint{3.228234in}{1.617800in}}%
\pgfpathlineto{\pgfqpoint{3.238662in}{1.637879in}}%
\pgfpathlineto{\pgfqpoint{3.249090in}{1.668103in}}%
\pgfpathlineto{\pgfqpoint{3.259519in}{1.588658in}}%
\pgfpathlineto{\pgfqpoint{3.269947in}{1.613564in}}%
\pgfpathlineto{\pgfqpoint{3.280375in}{1.631823in}}%
\pgfpathlineto{\pgfqpoint{3.290803in}{1.608654in}}%
\pgfpathlineto{\pgfqpoint{3.301231in}{1.599008in}}%
\pgfpathlineto{\pgfqpoint{3.311660in}{1.620388in}}%
\pgfpathlineto{\pgfqpoint{3.322088in}{1.651718in}}%
\pgfpathlineto{\pgfqpoint{3.332516in}{1.663246in}}%
\pgfpathlineto{\pgfqpoint{3.342944in}{1.607653in}}%
\pgfpathlineto{\pgfqpoint{3.353372in}{1.611516in}}%
\pgfpathlineto{\pgfqpoint{3.363800in}{1.591652in}}%
\pgfpathlineto{\pgfqpoint{3.374229in}{1.593649in}}%
\pgfpathlineto{\pgfqpoint{3.384657in}{1.631827in}}%
\pgfpathlineto{\pgfqpoint{3.395085in}{1.577722in}}%
\pgfpathlineto{\pgfqpoint{3.405513in}{1.554411in}}%
\pgfpathlineto{\pgfqpoint{3.415941in}{1.612496in}}%
\pgfpathlineto{\pgfqpoint{3.426369in}{1.619178in}}%
\pgfpathlineto{\pgfqpoint{3.436798in}{1.594527in}}%
\pgfpathlineto{\pgfqpoint{3.447226in}{1.629162in}}%
\pgfpathlineto{\pgfqpoint{3.457654in}{1.614633in}}%
\pgfpathlineto{\pgfqpoint{3.468082in}{1.618507in}}%
\pgfpathlineto{\pgfqpoint{3.478510in}{1.594653in}}%
\pgfpathlineto{\pgfqpoint{3.488938in}{1.672489in}}%
\pgfpathlineto{\pgfqpoint{3.499367in}{1.611605in}}%
\pgfpathlineto{\pgfqpoint{3.509795in}{1.649315in}}%
\pgfpathlineto{\pgfqpoint{3.520223in}{1.614507in}}%
\pgfpathlineto{\pgfqpoint{3.530651in}{1.662454in}}%
\pgfpathlineto{\pgfqpoint{3.541079in}{1.622426in}}%
\pgfpathlineto{\pgfqpoint{3.551508in}{1.605075in}}%
\pgfpathlineto{\pgfqpoint{3.561936in}{1.563397in}}%
\pgfpathlineto{\pgfqpoint{3.572364in}{1.576140in}}%
\pgfpathlineto{\pgfqpoint{3.582792in}{1.629069in}}%
\pgfpathlineto{\pgfqpoint{3.593220in}{1.676295in}}%
\pgfpathlineto{\pgfqpoint{3.603648in}{1.644116in}}%
\pgfpathlineto{\pgfqpoint{3.614077in}{1.638536in}}%
\pgfpathlineto{\pgfqpoint{3.624505in}{1.628321in}}%
\pgfpathlineto{\pgfqpoint{3.634933in}{1.597862in}}%
\pgfpathlineto{\pgfqpoint{3.645361in}{1.640724in}}%
\pgfpathlineto{\pgfqpoint{3.655789in}{1.574566in}}%
\pgfpathlineto{\pgfqpoint{3.666217in}{1.564503in}}%
\pgfpathlineto{\pgfqpoint{3.676646in}{1.577329in}}%
\pgfpathlineto{\pgfqpoint{3.687074in}{1.616389in}}%
\pgfpathlineto{\pgfqpoint{3.697502in}{1.645352in}}%
\pgfpathlineto{\pgfqpoint{3.707930in}{1.559142in}}%
\pgfpathlineto{\pgfqpoint{3.718358in}{1.620439in}}%
\pgfpathlineto{\pgfqpoint{3.728787in}{1.575011in}}%
\pgfpathlineto{\pgfqpoint{3.739215in}{1.592183in}}%
\pgfpathlineto{\pgfqpoint{3.749643in}{1.643610in}}%
\pgfpathlineto{\pgfqpoint{3.760071in}{1.641533in}}%
\pgfpathlineto{\pgfqpoint{3.770499in}{1.646681in}}%
\pgfpathlineto{\pgfqpoint{3.780927in}{1.627327in}}%
\pgfpathlineto{\pgfqpoint{3.791356in}{1.605165in}}%
\pgfpathlineto{\pgfqpoint{3.801784in}{1.617358in}}%
\pgfpathlineto{\pgfqpoint{3.812212in}{1.631596in}}%
\pgfpathlineto{\pgfqpoint{3.822640in}{1.602276in}}%
\pgfpathlineto{\pgfqpoint{3.833068in}{1.592714in}}%
\pgfpathlineto{\pgfqpoint{3.843496in}{1.632832in}}%
\pgfpathlineto{\pgfqpoint{3.853925in}{1.574532in}}%
\pgfpathlineto{\pgfqpoint{3.864353in}{1.583926in}}%
\pgfpathlineto{\pgfqpoint{3.874781in}{1.616583in}}%
\pgfpathlineto{\pgfqpoint{3.885209in}{1.617425in}}%
\pgfpathlineto{\pgfqpoint{3.895637in}{1.573526in}}%
\pgfpathlineto{\pgfqpoint{3.906065in}{1.586838in}}%
\pgfpathlineto{\pgfqpoint{3.916494in}{1.584072in}}%
\pgfpathlineto{\pgfqpoint{3.926922in}{1.600355in}}%
\pgfpathlineto{\pgfqpoint{3.937350in}{1.619827in}}%
\pgfpathlineto{\pgfqpoint{3.947778in}{1.584516in}}%
\pgfpathlineto{\pgfqpoint{3.958206in}{1.627183in}}%
\pgfpathlineto{\pgfqpoint{3.968635in}{1.534755in}}%
\pgfpathlineto{\pgfqpoint{3.979063in}{1.561176in}}%
\pgfpathlineto{\pgfqpoint{3.989491in}{1.547457in}}%
\pgfpathlineto{\pgfqpoint{3.999919in}{1.622220in}}%
\pgfpathlineto{\pgfqpoint{4.010347in}{1.650147in}}%
\pgfpathlineto{\pgfqpoint{4.020775in}{1.629514in}}%
\pgfpathlineto{\pgfqpoint{4.031204in}{1.603065in}}%
\pgfpathlineto{\pgfqpoint{4.041632in}{1.553147in}}%
\pgfpathlineto{\pgfqpoint{4.052060in}{1.586005in}}%
\pgfpathlineto{\pgfqpoint{4.062488in}{1.593937in}}%
\pgfpathlineto{\pgfqpoint{4.072916in}{1.619985in}}%
\pgfpathlineto{\pgfqpoint{4.083344in}{1.634102in}}%
\pgfpathlineto{\pgfqpoint{4.093773in}{1.631051in}}%
\pgfpathlineto{\pgfqpoint{4.104201in}{1.534834in}}%
\pgfpathlineto{\pgfqpoint{4.114629in}{1.584629in}}%
\pgfpathlineto{\pgfqpoint{4.125057in}{1.549255in}}%
\pgfpathlineto{\pgfqpoint{4.135485in}{1.571180in}}%
\pgfpathlineto{\pgfqpoint{4.145913in}{1.588084in}}%
\pgfpathlineto{\pgfqpoint{4.156342in}{1.646836in}}%
\pgfpathlineto{\pgfqpoint{4.166770in}{1.543156in}}%
\pgfpathlineto{\pgfqpoint{4.177198in}{1.616708in}}%
\pgfpathlineto{\pgfqpoint{4.187626in}{1.585836in}}%
\pgfpathlineto{\pgfqpoint{4.198054in}{1.574346in}}%
\pgfpathlineto{\pgfqpoint{4.208483in}{1.582819in}}%
\pgfpathlineto{\pgfqpoint{4.218911in}{1.596200in}}%
\pgfpathlineto{\pgfqpoint{4.229339in}{1.538480in}}%
\pgfpathlineto{\pgfqpoint{4.239767in}{1.575535in}}%
\pgfpathlineto{\pgfqpoint{4.250195in}{1.548661in}}%
\pgfpathlineto{\pgfqpoint{4.260623in}{1.607634in}}%
\pgfpathlineto{\pgfqpoint{4.271052in}{1.555509in}}%
\pgfpathlineto{\pgfqpoint{4.281480in}{1.586378in}}%
\pgfpathlineto{\pgfqpoint{4.291908in}{1.617835in}}%
\pgfpathlineto{\pgfqpoint{4.302336in}{1.570797in}}%
\pgfpathlineto{\pgfqpoint{4.312764in}{1.574640in}}%
\pgfpathlineto{\pgfqpoint{4.323192in}{1.632542in}}%
\pgfpathlineto{\pgfqpoint{4.333621in}{1.618195in}}%
\pgfpathlineto{\pgfqpoint{4.344049in}{1.582569in}}%
\pgfpathlineto{\pgfqpoint{4.354477in}{1.601064in}}%
\pgfpathlineto{\pgfqpoint{4.364905in}{1.561072in}}%
\pgfpathlineto{\pgfqpoint{4.375333in}{1.576690in}}%
\pgfpathlineto{\pgfqpoint{4.385761in}{1.671444in}}%
\pgfpathlineto{\pgfqpoint{4.396190in}{1.655679in}}%
\pgfpathlineto{\pgfqpoint{4.406618in}{1.677716in}}%
\pgfpathlineto{\pgfqpoint{4.417046in}{1.642737in}}%
\pgfpathlineto{\pgfqpoint{4.427474in}{1.578187in}}%
\pgfpathlineto{\pgfqpoint{4.437902in}{1.617035in}}%
\pgfpathlineto{\pgfqpoint{4.448331in}{1.589972in}}%
\pgfpathlineto{\pgfqpoint{4.458759in}{1.577670in}}%
\pgfpathlineto{\pgfqpoint{4.469187in}{1.590607in}}%
\pgfpathlineto{\pgfqpoint{4.479615in}{1.499475in}}%
\pgfpathlineto{\pgfqpoint{4.490043in}{1.567829in}}%
\pgfpathlineto{\pgfqpoint{4.500471in}{1.511750in}}%
\pgfpathlineto{\pgfqpoint{4.510900in}{1.441281in}}%
\pgfpathlineto{\pgfqpoint{4.521328in}{1.545025in}}%
\pgfpathlineto{\pgfqpoint{4.531756in}{1.553921in}}%
\pgfpathlineto{\pgfqpoint{4.542184in}{1.455469in}}%
\pgfpathlineto{\pgfqpoint{4.552612in}{1.529820in}}%
\pgfpathlineto{\pgfqpoint{4.563040in}{1.554428in}}%
\pgfpathlineto{\pgfqpoint{4.573469in}{1.602998in}}%
\pgfpathlineto{\pgfqpoint{4.583897in}{1.646043in}}%
\pgfpathlineto{\pgfqpoint{4.594325in}{1.571658in}}%
\pgfpathlineto{\pgfqpoint{4.604753in}{1.564188in}}%
\pgfpathlineto{\pgfqpoint{4.615181in}{1.616646in}}%
\pgfpathlineto{\pgfqpoint{4.625610in}{1.639210in}}%
\pgfpathlineto{\pgfqpoint{4.636038in}{1.695926in}}%
\pgfpathlineto{\pgfqpoint{4.646466in}{1.670917in}}%
\pgfpathlineto{\pgfqpoint{4.656894in}{1.690779in}}%
\pgfpathlineto{\pgfqpoint{4.667322in}{1.598654in}}%
\pgfpathlineto{\pgfqpoint{4.677750in}{1.659682in}}%
\pgfpathlineto{\pgfqpoint{4.688179in}{1.676642in}}%
\pgfpathlineto{\pgfqpoint{4.698607in}{1.592619in}}%
\pgfpathlineto{\pgfqpoint{4.709035in}{1.589557in}}%
\pgfpathlineto{\pgfqpoint{4.719463in}{1.636818in}}%
\pgfpathlineto{\pgfqpoint{4.729891in}{1.605143in}}%
\pgfpathlineto{\pgfqpoint{4.740319in}{1.604194in}}%
\pgfpathlineto{\pgfqpoint{4.750748in}{1.642454in}}%
\pgfpathlineto{\pgfqpoint{4.761176in}{1.701381in}}%
\pgfpathlineto{\pgfqpoint{4.771604in}{1.624201in}}%
\pgfpathlineto{\pgfqpoint{4.771604in}{1.468472in}}%
\pgfpathlineto{\pgfqpoint{4.771604in}{1.468472in}}%
\pgfpathlineto{\pgfqpoint{4.761176in}{1.566826in}}%
\pgfpathlineto{\pgfqpoint{4.750748in}{1.495227in}}%
\pgfpathlineto{\pgfqpoint{4.740319in}{1.450170in}}%
\pgfpathlineto{\pgfqpoint{4.729891in}{1.462369in}}%
\pgfpathlineto{\pgfqpoint{4.719463in}{1.499207in}}%
\pgfpathlineto{\pgfqpoint{4.709035in}{1.434543in}}%
\pgfpathlineto{\pgfqpoint{4.698607in}{1.435451in}}%
\pgfpathlineto{\pgfqpoint{4.688179in}{1.526695in}}%
\pgfpathlineto{\pgfqpoint{4.677750in}{1.492655in}}%
\pgfpathlineto{\pgfqpoint{4.667322in}{1.449059in}}%
\pgfpathlineto{\pgfqpoint{4.656894in}{1.537899in}}%
\pgfpathlineto{\pgfqpoint{4.646466in}{1.519503in}}%
\pgfpathlineto{\pgfqpoint{4.636038in}{1.548297in}}%
\pgfpathlineto{\pgfqpoint{4.625610in}{1.493129in}}%
\pgfpathlineto{\pgfqpoint{4.615181in}{1.457727in}}%
\pgfpathlineto{\pgfqpoint{4.604753in}{1.408460in}}%
\pgfpathlineto{\pgfqpoint{4.594325in}{1.443015in}}%
\pgfpathlineto{\pgfqpoint{4.583897in}{1.512421in}}%
\pgfpathlineto{\pgfqpoint{4.573469in}{1.453357in}}%
\pgfpathlineto{\pgfqpoint{4.563040in}{1.419305in}}%
\pgfpathlineto{\pgfqpoint{4.552612in}{1.394012in}}%
\pgfpathlineto{\pgfqpoint{4.542184in}{1.308811in}}%
\pgfpathlineto{\pgfqpoint{4.531756in}{1.424396in}}%
\pgfpathlineto{\pgfqpoint{4.521328in}{1.424311in}}%
\pgfpathlineto{\pgfqpoint{4.510900in}{1.294701in}}%
\pgfpathlineto{\pgfqpoint{4.500471in}{1.382129in}}%
\pgfpathlineto{\pgfqpoint{4.490043in}{1.428935in}}%
\pgfpathlineto{\pgfqpoint{4.479615in}{1.361442in}}%
\pgfpathlineto{\pgfqpoint{4.469187in}{1.448132in}}%
\pgfpathlineto{\pgfqpoint{4.458759in}{1.452028in}}%
\pgfpathlineto{\pgfqpoint{4.448331in}{1.463463in}}%
\pgfpathlineto{\pgfqpoint{4.437902in}{1.473206in}}%
\pgfpathlineto{\pgfqpoint{4.427474in}{1.454852in}}%
\pgfpathlineto{\pgfqpoint{4.417046in}{1.498983in}}%
\pgfpathlineto{\pgfqpoint{4.406618in}{1.521997in}}%
\pgfpathlineto{\pgfqpoint{4.396190in}{1.538260in}}%
\pgfpathlineto{\pgfqpoint{4.385761in}{1.553898in}}%
\pgfpathlineto{\pgfqpoint{4.375333in}{1.435058in}}%
\pgfpathlineto{\pgfqpoint{4.364905in}{1.431451in}}%
\pgfpathlineto{\pgfqpoint{4.354477in}{1.461732in}}%
\pgfpathlineto{\pgfqpoint{4.344049in}{1.441640in}}%
\pgfpathlineto{\pgfqpoint{4.333621in}{1.510482in}}%
\pgfpathlineto{\pgfqpoint{4.323192in}{1.503379in}}%
\pgfpathlineto{\pgfqpoint{4.312764in}{1.441346in}}%
\pgfpathlineto{\pgfqpoint{4.302336in}{1.433499in}}%
\pgfpathlineto{\pgfqpoint{4.291908in}{1.487471in}}%
\pgfpathlineto{\pgfqpoint{4.281480in}{1.443050in}}%
\pgfpathlineto{\pgfqpoint{4.271052in}{1.430429in}}%
\pgfpathlineto{\pgfqpoint{4.260623in}{1.487447in}}%
\pgfpathlineto{\pgfqpoint{4.250195in}{1.428817in}}%
\pgfpathlineto{\pgfqpoint{4.239767in}{1.458880in}}%
\pgfpathlineto{\pgfqpoint{4.229339in}{1.443697in}}%
\pgfpathlineto{\pgfqpoint{4.218911in}{1.504341in}}%
\pgfpathlineto{\pgfqpoint{4.208483in}{1.452382in}}%
\pgfpathlineto{\pgfqpoint{4.198054in}{1.454045in}}%
\pgfpathlineto{\pgfqpoint{4.187626in}{1.468609in}}%
\pgfpathlineto{\pgfqpoint{4.177198in}{1.468846in}}%
\pgfpathlineto{\pgfqpoint{4.166770in}{1.425848in}}%
\pgfpathlineto{\pgfqpoint{4.156342in}{1.522898in}}%
\pgfpathlineto{\pgfqpoint{4.145913in}{1.458495in}}%
\pgfpathlineto{\pgfqpoint{4.135485in}{1.436749in}}%
\pgfpathlineto{\pgfqpoint{4.125057in}{1.421274in}}%
\pgfpathlineto{\pgfqpoint{4.114629in}{1.477108in}}%
\pgfpathlineto{\pgfqpoint{4.104201in}{1.416952in}}%
\pgfpathlineto{\pgfqpoint{4.093773in}{1.522905in}}%
\pgfpathlineto{\pgfqpoint{4.083344in}{1.530245in}}%
\pgfpathlineto{\pgfqpoint{4.072916in}{1.514832in}}%
\pgfpathlineto{\pgfqpoint{4.062488in}{1.485102in}}%
\pgfpathlineto{\pgfqpoint{4.052060in}{1.471918in}}%
\pgfpathlineto{\pgfqpoint{4.041632in}{1.442691in}}%
\pgfpathlineto{\pgfqpoint{4.031204in}{1.480253in}}%
\pgfpathlineto{\pgfqpoint{4.020775in}{1.511685in}}%
\pgfpathlineto{\pgfqpoint{4.010347in}{1.529415in}}%
\pgfpathlineto{\pgfqpoint{3.999919in}{1.524475in}}%
\pgfpathlineto{\pgfqpoint{3.989491in}{1.435455in}}%
\pgfpathlineto{\pgfqpoint{3.979063in}{1.447583in}}%
\pgfpathlineto{\pgfqpoint{3.968635in}{1.429317in}}%
\pgfpathlineto{\pgfqpoint{3.958206in}{1.525950in}}%
\pgfpathlineto{\pgfqpoint{3.947778in}{1.473585in}}%
\pgfpathlineto{\pgfqpoint{3.937350in}{1.495112in}}%
\pgfpathlineto{\pgfqpoint{3.926922in}{1.487149in}}%
\pgfpathlineto{\pgfqpoint{3.916494in}{1.479884in}}%
\pgfpathlineto{\pgfqpoint{3.906065in}{1.492304in}}%
\pgfpathlineto{\pgfqpoint{3.895637in}{1.457050in}}%
\pgfpathlineto{\pgfqpoint{3.885209in}{1.509464in}}%
\pgfpathlineto{\pgfqpoint{3.874781in}{1.507021in}}%
\pgfpathlineto{\pgfqpoint{3.864353in}{1.458249in}}%
\pgfpathlineto{\pgfqpoint{3.853925in}{1.458230in}}%
\pgfpathlineto{\pgfqpoint{3.843496in}{1.517844in}}%
\pgfpathlineto{\pgfqpoint{3.833068in}{1.475530in}}%
\pgfpathlineto{\pgfqpoint{3.822640in}{1.484760in}}%
\pgfpathlineto{\pgfqpoint{3.812212in}{1.521733in}}%
\pgfpathlineto{\pgfqpoint{3.801784in}{1.512399in}}%
\pgfpathlineto{\pgfqpoint{3.791356in}{1.502317in}}%
\pgfpathlineto{\pgfqpoint{3.780927in}{1.523821in}}%
\pgfpathlineto{\pgfqpoint{3.770499in}{1.549353in}}%
\pgfpathlineto{\pgfqpoint{3.760071in}{1.532646in}}%
\pgfpathlineto{\pgfqpoint{3.749643in}{1.547070in}}%
\pgfpathlineto{\pgfqpoint{3.739215in}{1.491946in}}%
\pgfpathlineto{\pgfqpoint{3.728787in}{1.480111in}}%
\pgfpathlineto{\pgfqpoint{3.718358in}{1.533378in}}%
\pgfpathlineto{\pgfqpoint{3.707930in}{1.454071in}}%
\pgfpathlineto{\pgfqpoint{3.697502in}{1.549741in}}%
\pgfpathlineto{\pgfqpoint{3.687074in}{1.509716in}}%
\pgfpathlineto{\pgfqpoint{3.676646in}{1.467218in}}%
\pgfpathlineto{\pgfqpoint{3.666217in}{1.455508in}}%
\pgfpathlineto{\pgfqpoint{3.655789in}{1.465435in}}%
\pgfpathlineto{\pgfqpoint{3.645361in}{1.521046in}}%
\pgfpathlineto{\pgfqpoint{3.634933in}{1.490251in}}%
\pgfpathlineto{\pgfqpoint{3.624505in}{1.527674in}}%
\pgfpathlineto{\pgfqpoint{3.614077in}{1.530028in}}%
\pgfpathlineto{\pgfqpoint{3.603648in}{1.531622in}}%
\pgfpathlineto{\pgfqpoint{3.593220in}{1.569937in}}%
\pgfpathlineto{\pgfqpoint{3.582792in}{1.522164in}}%
\pgfpathlineto{\pgfqpoint{3.572364in}{1.482379in}}%
\pgfpathlineto{\pgfqpoint{3.561936in}{1.464743in}}%
\pgfpathlineto{\pgfqpoint{3.551508in}{1.506624in}}%
\pgfpathlineto{\pgfqpoint{3.541079in}{1.524960in}}%
\pgfpathlineto{\pgfqpoint{3.530651in}{1.560200in}}%
\pgfpathlineto{\pgfqpoint{3.520223in}{1.523589in}}%
\pgfpathlineto{\pgfqpoint{3.509795in}{1.561711in}}%
\pgfpathlineto{\pgfqpoint{3.499367in}{1.516228in}}%
\pgfpathlineto{\pgfqpoint{3.488938in}{1.574686in}}%
\pgfpathlineto{\pgfqpoint{3.478510in}{1.499578in}}%
\pgfpathlineto{\pgfqpoint{3.468082in}{1.525193in}}%
\pgfpathlineto{\pgfqpoint{3.457654in}{1.515337in}}%
\pgfpathlineto{\pgfqpoint{3.447226in}{1.542010in}}%
\pgfpathlineto{\pgfqpoint{3.436798in}{1.501805in}}%
\pgfpathlineto{\pgfqpoint{3.426369in}{1.537587in}}%
\pgfpathlineto{\pgfqpoint{3.415941in}{1.533923in}}%
\pgfpathlineto{\pgfqpoint{3.405513in}{1.470106in}}%
\pgfpathlineto{\pgfqpoint{3.395085in}{1.494278in}}%
\pgfpathlineto{\pgfqpoint{3.384657in}{1.550054in}}%
\pgfpathlineto{\pgfqpoint{3.374229in}{1.507868in}}%
\pgfpathlineto{\pgfqpoint{3.363800in}{1.503201in}}%
\pgfpathlineto{\pgfqpoint{3.353372in}{1.529868in}}%
\pgfpathlineto{\pgfqpoint{3.342944in}{1.522996in}}%
\pgfpathlineto{\pgfqpoint{3.332516in}{1.576129in}}%
\pgfpathlineto{\pgfqpoint{3.322088in}{1.559145in}}%
\pgfpathlineto{\pgfqpoint{3.311660in}{1.536807in}}%
\pgfpathlineto{\pgfqpoint{3.301231in}{1.520741in}}%
\pgfpathlineto{\pgfqpoint{3.290803in}{1.529826in}}%
\pgfpathlineto{\pgfqpoint{3.280375in}{1.554761in}}%
\pgfpathlineto{\pgfqpoint{3.269947in}{1.531156in}}%
\pgfpathlineto{\pgfqpoint{3.259519in}{1.513549in}}%
\pgfpathlineto{\pgfqpoint{3.249090in}{1.594622in}}%
\pgfpathlineto{\pgfqpoint{3.238662in}{1.558400in}}%
\pgfpathlineto{\pgfqpoint{3.228234in}{1.540361in}}%
\pgfpathlineto{\pgfqpoint{3.217806in}{1.573577in}}%
\pgfpathlineto{\pgfqpoint{3.207378in}{1.580195in}}%
\pgfpathlineto{\pgfqpoint{3.196950in}{1.603183in}}%
\pgfpathlineto{\pgfqpoint{3.186521in}{1.585064in}}%
\pgfpathlineto{\pgfqpoint{3.176093in}{1.577218in}}%
\pgfpathlineto{\pgfqpoint{3.165665in}{1.604346in}}%
\pgfpathlineto{\pgfqpoint{3.155237in}{1.630287in}}%
\pgfpathlineto{\pgfqpoint{3.144809in}{1.654929in}}%
\pgfpathlineto{\pgfqpoint{3.134381in}{1.625249in}}%
\pgfpathlineto{\pgfqpoint{3.123952in}{1.598412in}}%
\pgfpathlineto{\pgfqpoint{3.113524in}{1.640667in}}%
\pgfpathlineto{\pgfqpoint{3.103096in}{1.658948in}}%
\pgfpathlineto{\pgfqpoint{3.092668in}{1.630898in}}%
\pgfpathlineto{\pgfqpoint{3.082240in}{1.641039in}}%
\pgfpathlineto{\pgfqpoint{3.071812in}{1.670441in}}%
\pgfpathlineto{\pgfqpoint{3.061383in}{1.681889in}}%
\pgfpathlineto{\pgfqpoint{3.050955in}{1.659702in}}%
\pgfpathlineto{\pgfqpoint{3.040527in}{1.675503in}}%
\pgfpathlineto{\pgfqpoint{3.030099in}{1.687149in}}%
\pgfpathlineto{\pgfqpoint{3.019671in}{1.668655in}}%
\pgfpathlineto{\pgfqpoint{3.009242in}{1.702344in}}%
\pgfpathlineto{\pgfqpoint{2.998814in}{1.694030in}}%
\pgfpathlineto{\pgfqpoint{2.988386in}{1.684037in}}%
\pgfpathlineto{\pgfqpoint{2.977958in}{1.728771in}}%
\pgfpathlineto{\pgfqpoint{2.967530in}{1.737694in}}%
\pgfpathlineto{\pgfqpoint{2.957102in}{1.735948in}}%
\pgfpathlineto{\pgfqpoint{2.946673in}{1.760719in}}%
\pgfpathlineto{\pgfqpoint{2.936245in}{1.763146in}}%
\pgfpathlineto{\pgfqpoint{2.925817in}{1.737979in}}%
\pgfpathlineto{\pgfqpoint{2.915389in}{1.753955in}}%
\pgfpathlineto{\pgfqpoint{2.904961in}{1.743724in}}%
\pgfpathlineto{\pgfqpoint{2.894533in}{1.787144in}}%
\pgfpathlineto{\pgfqpoint{2.884104in}{1.790623in}}%
\pgfpathlineto{\pgfqpoint{2.873676in}{1.817843in}}%
\pgfpathlineto{\pgfqpoint{2.863248in}{1.814159in}}%
\pgfpathlineto{\pgfqpoint{2.852820in}{1.833235in}}%
\pgfpathlineto{\pgfqpoint{2.842392in}{1.877722in}}%
\pgfpathlineto{\pgfqpoint{2.831964in}{1.883696in}}%
\pgfpathlineto{\pgfqpoint{2.821535in}{1.900231in}}%
\pgfpathlineto{\pgfqpoint{2.811107in}{1.893129in}}%
\pgfpathlineto{\pgfqpoint{2.800679in}{1.916104in}}%
\pgfpathlineto{\pgfqpoint{2.790251in}{1.950219in}}%
\pgfpathlineto{\pgfqpoint{2.779823in}{1.966687in}}%
\pgfpathlineto{\pgfqpoint{2.769394in}{2.007896in}}%
\pgfpathlineto{\pgfqpoint{2.758966in}{1.986667in}}%
\pgfpathlineto{\pgfqpoint{2.748538in}{1.992155in}}%
\pgfpathlineto{\pgfqpoint{2.738110in}{1.999736in}}%
\pgfpathlineto{\pgfqpoint{2.727682in}{2.039345in}}%
\pgfpathlineto{\pgfqpoint{2.717254in}{2.117308in}}%
\pgfpathlineto{\pgfqpoint{2.706825in}{2.195632in}}%
\pgfpathlineto{\pgfqpoint{2.696397in}{2.327231in}}%
\pgfpathlineto{\pgfqpoint{2.685969in}{2.303297in}}%
\pgfpathlineto{\pgfqpoint{2.675541in}{2.446874in}}%
\pgfpathlineto{\pgfqpoint{2.665113in}{2.547599in}}%
\pgfpathlineto{\pgfqpoint{2.654685in}{2.598784in}}%
\pgfpathlineto{\pgfqpoint{2.644256in}{2.639470in}}%
\pgfpathlineto{\pgfqpoint{2.633828in}{2.643568in}}%
\pgfpathlineto{\pgfqpoint{2.623400in}{2.699848in}}%
\pgfpathlineto{\pgfqpoint{2.612972in}{2.717486in}}%
\pgfpathlineto{\pgfqpoint{2.602544in}{2.695351in}}%
\pgfpathlineto{\pgfqpoint{2.592115in}{2.740724in}}%
\pgfpathlineto{\pgfqpoint{2.581687in}{2.747943in}}%
\pgfpathlineto{\pgfqpoint{2.571259in}{2.761615in}}%
\pgfpathlineto{\pgfqpoint{2.560831in}{2.824373in}}%
\pgfpathlineto{\pgfqpoint{2.550403in}{2.797960in}}%
\pgfpathlineto{\pgfqpoint{2.539975in}{2.819801in}}%
\pgfpathlineto{\pgfqpoint{2.529546in}{2.862732in}}%
\pgfpathlineto{\pgfqpoint{2.519118in}{2.849495in}}%
\pgfpathlineto{\pgfqpoint{2.508690in}{2.885951in}}%
\pgfpathlineto{\pgfqpoint{2.498262in}{2.903672in}}%
\pgfpathlineto{\pgfqpoint{2.487834in}{2.913382in}}%
\pgfpathlineto{\pgfqpoint{2.477406in}{2.933632in}}%
\pgfpathlineto{\pgfqpoint{2.466977in}{2.936142in}}%
\pgfpathlineto{\pgfqpoint{2.456549in}{2.901208in}}%
\pgfpathlineto{\pgfqpoint{2.446121in}{2.880977in}}%
\pgfpathlineto{\pgfqpoint{2.435693in}{2.887673in}}%
\pgfpathlineto{\pgfqpoint{2.425265in}{2.851712in}}%
\pgfpathlineto{\pgfqpoint{2.414837in}{2.861112in}}%
\pgfpathlineto{\pgfqpoint{2.404408in}{2.857370in}}%
\pgfpathlineto{\pgfqpoint{2.393980in}{2.881588in}}%
\pgfpathlineto{\pgfqpoint{2.383552in}{2.904186in}}%
\pgfpathlineto{\pgfqpoint{2.373124in}{2.867073in}}%
\pgfpathlineto{\pgfqpoint{2.362696in}{2.891524in}}%
\pgfpathlineto{\pgfqpoint{2.352267in}{2.882496in}}%
\pgfpathlineto{\pgfqpoint{2.341839in}{2.894554in}}%
\pgfpathlineto{\pgfqpoint{2.331411in}{2.904352in}}%
\pgfpathlineto{\pgfqpoint{2.320983in}{2.901358in}}%
\pgfpathlineto{\pgfqpoint{2.310555in}{2.911625in}}%
\pgfpathlineto{\pgfqpoint{2.300127in}{2.925213in}}%
\pgfpathlineto{\pgfqpoint{2.289698in}{2.919955in}}%
\pgfpathlineto{\pgfqpoint{2.279270in}{2.891674in}}%
\pgfpathlineto{\pgfqpoint{2.268842in}{2.926623in}}%
\pgfpathlineto{\pgfqpoint{2.258414in}{2.971094in}}%
\pgfpathlineto{\pgfqpoint{2.247986in}{2.982562in}}%
\pgfpathlineto{\pgfqpoint{2.237558in}{2.977574in}}%
\pgfpathlineto{\pgfqpoint{2.227129in}{3.010288in}}%
\pgfpathlineto{\pgfqpoint{2.216701in}{3.050419in}}%
\pgfpathlineto{\pgfqpoint{2.206273in}{3.039131in}}%
\pgfpathlineto{\pgfqpoint{2.195845in}{3.039763in}}%
\pgfpathlineto{\pgfqpoint{2.185417in}{3.048682in}}%
\pgfpathlineto{\pgfqpoint{2.174989in}{3.051831in}}%
\pgfpathlineto{\pgfqpoint{2.164560in}{3.058670in}}%
\pgfpathlineto{\pgfqpoint{2.154132in}{3.057724in}}%
\pgfpathlineto{\pgfqpoint{2.143704in}{3.029791in}}%
\pgfpathlineto{\pgfqpoint{2.133276in}{3.077977in}}%
\pgfpathlineto{\pgfqpoint{2.122848in}{3.073667in}}%
\pgfpathlineto{\pgfqpoint{2.112419in}{3.084203in}}%
\pgfpathlineto{\pgfqpoint{2.101991in}{3.126322in}}%
\pgfpathlineto{\pgfqpoint{2.091563in}{3.051235in}}%
\pgfpathlineto{\pgfqpoint{2.081135in}{3.059817in}}%
\pgfpathlineto{\pgfqpoint{2.070707in}{3.080954in}}%
\pgfpathlineto{\pgfqpoint{2.060279in}{3.053764in}}%
\pgfpathlineto{\pgfqpoint{2.049850in}{3.054144in}}%
\pgfpathlineto{\pgfqpoint{2.039422in}{3.015663in}}%
\pgfpathlineto{\pgfqpoint{2.028994in}{3.004451in}}%
\pgfpathlineto{\pgfqpoint{2.018566in}{3.031601in}}%
\pgfpathlineto{\pgfqpoint{2.008138in}{3.039027in}}%
\pgfpathlineto{\pgfqpoint{1.997710in}{3.014392in}}%
\pgfpathlineto{\pgfqpoint{1.987281in}{3.090106in}}%
\pgfpathlineto{\pgfqpoint{1.976853in}{3.042202in}}%
\pgfpathlineto{\pgfqpoint{1.966425in}{3.063015in}}%
\pgfpathlineto{\pgfqpoint{1.955997in}{3.078361in}}%
\pgfpathlineto{\pgfqpoint{1.945569in}{3.072487in}}%
\pgfpathlineto{\pgfqpoint{1.935141in}{3.038968in}}%
\pgfpathlineto{\pgfqpoint{1.924712in}{3.064689in}}%
\pgfpathlineto{\pgfqpoint{1.914284in}{3.059565in}}%
\pgfpathlineto{\pgfqpoint{1.903856in}{3.024690in}}%
\pgfpathlineto{\pgfqpoint{1.893428in}{3.021528in}}%
\pgfpathlineto{\pgfqpoint{1.883000in}{3.002203in}}%
\pgfpathlineto{\pgfqpoint{1.872571in}{2.987713in}}%
\pgfpathlineto{\pgfqpoint{1.862143in}{2.989421in}}%
\pgfpathlineto{\pgfqpoint{1.851715in}{3.021856in}}%
\pgfpathlineto{\pgfqpoint{1.841287in}{2.995315in}}%
\pgfpathlineto{\pgfqpoint{1.830859in}{3.056136in}}%
\pgfpathlineto{\pgfqpoint{1.820431in}{3.039609in}}%
\pgfpathlineto{\pgfqpoint{1.810002in}{2.993624in}}%
\pgfpathlineto{\pgfqpoint{1.799574in}{3.096171in}}%
\pgfpathlineto{\pgfqpoint{1.789146in}{3.018890in}}%
\pgfpathlineto{\pgfqpoint{1.778718in}{3.014174in}}%
\pgfpathlineto{\pgfqpoint{1.768290in}{2.982847in}}%
\pgfpathlineto{\pgfqpoint{1.757862in}{2.910320in}}%
\pgfpathlineto{\pgfqpoint{1.747433in}{2.907315in}}%
\pgfpathlineto{\pgfqpoint{1.737005in}{2.946692in}}%
\pgfpathlineto{\pgfqpoint{1.726577in}{3.020730in}}%
\pgfpathlineto{\pgfqpoint{1.716149in}{3.046667in}}%
\pgfpathlineto{\pgfqpoint{1.705721in}{3.036893in}}%
\pgfpathlineto{\pgfqpoint{1.695292in}{3.008001in}}%
\pgfpathlineto{\pgfqpoint{1.684864in}{3.042986in}}%
\pgfpathlineto{\pgfqpoint{1.674436in}{2.951942in}}%
\pgfpathlineto{\pgfqpoint{1.664008in}{2.941569in}}%
\pgfpathlineto{\pgfqpoint{1.653580in}{2.989529in}}%
\pgfpathlineto{\pgfqpoint{1.643152in}{3.045031in}}%
\pgfpathlineto{\pgfqpoint{1.632723in}{3.040576in}}%
\pgfpathlineto{\pgfqpoint{1.622295in}{3.126364in}}%
\pgfpathlineto{\pgfqpoint{1.611867in}{3.085339in}}%
\pgfpathlineto{\pgfqpoint{1.601439in}{3.031442in}}%
\pgfpathlineto{\pgfqpoint{1.591011in}{3.026757in}}%
\pgfpathlineto{\pgfqpoint{1.580583in}{2.957821in}}%
\pgfpathlineto{\pgfqpoint{1.570154in}{3.027627in}}%
\pgfpathlineto{\pgfqpoint{1.559726in}{3.047195in}}%
\pgfpathlineto{\pgfqpoint{1.549298in}{3.052812in}}%
\pgfpathlineto{\pgfqpoint{1.538870in}{3.100117in}}%
\pgfpathlineto{\pgfqpoint{1.528442in}{3.085078in}}%
\pgfpathlineto{\pgfqpoint{1.518014in}{3.123676in}}%
\pgfpathlineto{\pgfqpoint{1.507585in}{3.142304in}}%
\pgfpathlineto{\pgfqpoint{1.497157in}{3.082584in}}%
\pgfpathlineto{\pgfqpoint{1.486729in}{3.114584in}}%
\pgfpathlineto{\pgfqpoint{1.476301in}{3.128443in}}%
\pgfpathlineto{\pgfqpoint{1.465873in}{3.078979in}}%
\pgfpathlineto{\pgfqpoint{1.455444in}{3.064692in}}%
\pgfpathlineto{\pgfqpoint{1.445016in}{3.017680in}}%
\pgfpathlineto{\pgfqpoint{1.434588in}{3.027486in}}%
\pgfpathlineto{\pgfqpoint{1.424160in}{3.042588in}}%
\pgfpathlineto{\pgfqpoint{1.413732in}{3.064750in}}%
\pgfpathlineto{\pgfqpoint{1.403304in}{3.067639in}}%
\pgfpathlineto{\pgfqpoint{1.392875in}{3.001783in}}%
\pgfpathlineto{\pgfqpoint{1.382447in}{2.988746in}}%
\pgfpathlineto{\pgfqpoint{1.372019in}{2.940350in}}%
\pgfpathlineto{\pgfqpoint{1.361591in}{3.033340in}}%
\pgfpathlineto{\pgfqpoint{1.351163in}{3.015027in}}%
\pgfpathlineto{\pgfqpoint{1.340735in}{2.998782in}}%
\pgfpathlineto{\pgfqpoint{1.330306in}{3.023982in}}%
\pgfpathlineto{\pgfqpoint{1.319878in}{2.978725in}}%
\pgfpathlineto{\pgfqpoint{1.309450in}{2.987891in}}%
\pgfpathlineto{\pgfqpoint{1.299022in}{2.912950in}}%
\pgfpathlineto{\pgfqpoint{1.288594in}{2.907836in}}%
\pgfpathlineto{\pgfqpoint{1.278166in}{2.956282in}}%
\pgfpathlineto{\pgfqpoint{1.267737in}{2.913151in}}%
\pgfpathlineto{\pgfqpoint{1.257309in}{2.972092in}}%
\pgfpathlineto{\pgfqpoint{1.246881in}{2.955826in}}%
\pgfpathlineto{\pgfqpoint{1.236453in}{2.990960in}}%
\pgfpathlineto{\pgfqpoint{1.226025in}{2.966427in}}%
\pgfpathlineto{\pgfqpoint{1.215596in}{2.920047in}}%
\pgfpathlineto{\pgfqpoint{1.205168in}{2.874891in}}%
\pgfpathlineto{\pgfqpoint{1.194740in}{2.845357in}}%
\pgfpathlineto{\pgfqpoint{1.184312in}{2.842819in}}%
\pgfpathlineto{\pgfqpoint{1.173884in}{2.860991in}}%
\pgfpathlineto{\pgfqpoint{1.163456in}{2.948675in}}%
\pgfpathlineto{\pgfqpoint{1.153027in}{2.949790in}}%
\pgfpathlineto{\pgfqpoint{1.142599in}{2.973764in}}%
\pgfpathlineto{\pgfqpoint{1.132171in}{2.910364in}}%
\pgfpathlineto{\pgfqpoint{1.121743in}{2.963129in}}%
\pgfpathlineto{\pgfqpoint{1.111315in}{2.918632in}}%
\pgfpathlineto{\pgfqpoint{1.100887in}{3.022842in}}%
\pgfpathlineto{\pgfqpoint{1.090458in}{2.993995in}}%
\pgfpathlineto{\pgfqpoint{1.080030in}{2.946741in}}%
\pgfpathlineto{\pgfqpoint{1.069602in}{2.959229in}}%
\pgfpathlineto{\pgfqpoint{1.059174in}{2.954807in}}%
\pgfpathlineto{\pgfqpoint{1.048746in}{3.035958in}}%
\pgfpathlineto{\pgfqpoint{1.038318in}{3.027621in}}%
\pgfpathlineto{\pgfqpoint{1.027889in}{3.089979in}}%
\pgfpathlineto{\pgfqpoint{1.017461in}{3.019467in}}%
\pgfpathlineto{\pgfqpoint{1.007033in}{3.017061in}}%
\pgfpathlineto{\pgfqpoint{0.996605in}{2.992844in}}%
\pgfpathlineto{\pgfqpoint{0.986177in}{2.952467in}}%
\pgfpathlineto{\pgfqpoint{0.975748in}{2.975711in}}%
\pgfpathlineto{\pgfqpoint{0.965320in}{2.998496in}}%
\pgfpathlineto{\pgfqpoint{0.954892in}{3.047272in}}%
\pgfpathlineto{\pgfqpoint{0.944464in}{3.025832in}}%
\pgfpathlineto{\pgfqpoint{0.934036in}{2.948648in}}%
\pgfpathlineto{\pgfqpoint{0.923608in}{3.036846in}}%
\pgfpathlineto{\pgfqpoint{0.913179in}{3.016777in}}%
\pgfpathlineto{\pgfqpoint{0.902751in}{2.982913in}}%
\pgfpathlineto{\pgfqpoint{0.892323in}{3.000310in}}%
\pgfpathlineto{\pgfqpoint{0.881895in}{2.998409in}}%
\pgfpathlineto{\pgfqpoint{0.871467in}{2.941912in}}%
\pgfpathlineto{\pgfqpoint{0.861039in}{2.935212in}}%
\pgfpathlineto{\pgfqpoint{0.850610in}{3.003901in}}%
\pgfpathlineto{\pgfqpoint{0.840182in}{3.008040in}}%
\pgfpathlineto{\pgfqpoint{0.829754in}{3.051918in}}%
\pgfpathlineto{\pgfqpoint{0.819326in}{3.052291in}}%
\pgfpathlineto{\pgfqpoint{0.808898in}{3.083020in}}%
\pgfpathlineto{\pgfqpoint{0.798470in}{3.028119in}}%
\pgfpathlineto{\pgfqpoint{0.788041in}{3.078807in}}%
\pgfpathlineto{\pgfqpoint{0.777613in}{3.058702in}}%
\pgfpathlineto{\pgfqpoint{0.767185in}{3.063405in}}%
\pgfpathlineto{\pgfqpoint{0.756757in}{3.104818in}}%
\pgfpathlineto{\pgfqpoint{0.746329in}{3.057947in}}%
\pgfpathlineto{\pgfqpoint{0.735900in}{3.047428in}}%
\pgfpathlineto{\pgfqpoint{0.725472in}{3.045124in}}%
\pgfpathlineto{\pgfqpoint{0.715044in}{3.011665in}}%
\pgfpathlineto{\pgfqpoint{0.704616in}{3.042471in}}%
\pgfpathlineto{\pgfqpoint{0.694188in}{3.091090in}}%
\pgfpathlineto{\pgfqpoint{0.683760in}{3.027221in}}%
\pgfpathlineto{\pgfqpoint{0.673331in}{3.051635in}}%
\pgfpathlineto{\pgfqpoint{0.662903in}{3.053966in}}%
\pgfpathlineto{\pgfqpoint{0.652475in}{3.117862in}}%
\pgfpathlineto{\pgfqpoint{0.642047in}{3.068040in}}%
\pgfpathlineto{\pgfqpoint{0.631619in}{3.103805in}}%
\pgfpathlineto{\pgfqpoint{0.621191in}{3.032798in}}%
\pgfpathlineto{\pgfqpoint{0.610762in}{3.031977in}}%
\pgfpathclose%
\pgfusepath{stroke,fill}%
\end{pgfscope}%
\begin{pgfscope}%
\pgfpathrectangle{\pgfqpoint{0.610762in}{0.961156in}}{\pgfqpoint{4.171270in}{2.577986in}} %
\pgfusepath{clip}%
\pgfsetbuttcap%
\pgfsetroundjoin%
\definecolor{currentfill}{rgb}{1.000000,0.494118,0.250980}%
\pgfsetfillcolor{currentfill}%
\pgfsetfillopacity{0.200000}%
\pgfsetlinewidth{0.301125pt}%
\definecolor{currentstroke}{rgb}{0.000000,0.000000,0.000000}%
\pgfsetstrokecolor{currentstroke}%
\pgfsetstrokeopacity{0.200000}%
\pgfsetdash{}{0pt}%
\pgfpathmoveto{\pgfqpoint{0.610762in}{2.977691in}}%
\pgfpathlineto{\pgfqpoint{0.610762in}{3.017998in}}%
\pgfpathlineto{\pgfqpoint{0.621191in}{3.070257in}}%
\pgfpathlineto{\pgfqpoint{0.631619in}{3.052868in}}%
\pgfpathlineto{\pgfqpoint{0.642047in}{3.017972in}}%
\pgfpathlineto{\pgfqpoint{0.652475in}{3.031229in}}%
\pgfpathlineto{\pgfqpoint{0.662903in}{3.041554in}}%
\pgfpathlineto{\pgfqpoint{0.673331in}{3.022567in}}%
\pgfpathlineto{\pgfqpoint{0.683760in}{3.030046in}}%
\pgfpathlineto{\pgfqpoint{0.694188in}{3.062760in}}%
\pgfpathlineto{\pgfqpoint{0.704616in}{3.029235in}}%
\pgfpathlineto{\pgfqpoint{0.715044in}{3.039305in}}%
\pgfpathlineto{\pgfqpoint{0.725472in}{3.033704in}}%
\pgfpathlineto{\pgfqpoint{0.735900in}{3.034246in}}%
\pgfpathlineto{\pgfqpoint{0.746329in}{3.018001in}}%
\pgfpathlineto{\pgfqpoint{0.756757in}{3.025950in}}%
\pgfpathlineto{\pgfqpoint{0.767185in}{3.028436in}}%
\pgfpathlineto{\pgfqpoint{0.777613in}{3.016184in}}%
\pgfpathlineto{\pgfqpoint{0.788041in}{3.003535in}}%
\pgfpathlineto{\pgfqpoint{0.798470in}{3.005437in}}%
\pgfpathlineto{\pgfqpoint{0.808898in}{3.011052in}}%
\pgfpathlineto{\pgfqpoint{0.819326in}{3.022596in}}%
\pgfpathlineto{\pgfqpoint{0.829754in}{3.057471in}}%
\pgfpathlineto{\pgfqpoint{0.840182in}{3.051457in}}%
\pgfpathlineto{\pgfqpoint{0.850610in}{3.095796in}}%
\pgfpathlineto{\pgfqpoint{0.861039in}{3.034579in}}%
\pgfpathlineto{\pgfqpoint{0.871467in}{3.054291in}}%
\pgfpathlineto{\pgfqpoint{0.881895in}{3.094814in}}%
\pgfpathlineto{\pgfqpoint{0.892323in}{3.066120in}}%
\pgfpathlineto{\pgfqpoint{0.902751in}{3.079012in}}%
\pgfpathlineto{\pgfqpoint{0.913179in}{3.063598in}}%
\pgfpathlineto{\pgfqpoint{0.923608in}{3.068656in}}%
\pgfpathlineto{\pgfqpoint{0.934036in}{3.060407in}}%
\pgfpathlineto{\pgfqpoint{0.944464in}{3.062068in}}%
\pgfpathlineto{\pgfqpoint{0.954892in}{3.060646in}}%
\pgfpathlineto{\pgfqpoint{0.965320in}{3.071252in}}%
\pgfpathlineto{\pgfqpoint{0.975748in}{3.067405in}}%
\pgfpathlineto{\pgfqpoint{0.986177in}{3.059229in}}%
\pgfpathlineto{\pgfqpoint{0.996605in}{3.068122in}}%
\pgfpathlineto{\pgfqpoint{1.007033in}{3.080339in}}%
\pgfpathlineto{\pgfqpoint{1.017461in}{3.090593in}}%
\pgfpathlineto{\pgfqpoint{1.027889in}{3.077001in}}%
\pgfpathlineto{\pgfqpoint{1.038318in}{3.087178in}}%
\pgfpathlineto{\pgfqpoint{1.048746in}{3.059647in}}%
\pgfpathlineto{\pgfqpoint{1.059174in}{3.081497in}}%
\pgfpathlineto{\pgfqpoint{1.069602in}{3.079651in}}%
\pgfpathlineto{\pgfqpoint{1.080030in}{3.095003in}}%
\pgfpathlineto{\pgfqpoint{1.090458in}{3.090606in}}%
\pgfpathlineto{\pgfqpoint{1.100887in}{3.079210in}}%
\pgfpathlineto{\pgfqpoint{1.111315in}{3.072433in}}%
\pgfpathlineto{\pgfqpoint{1.121743in}{3.057485in}}%
\pgfpathlineto{\pgfqpoint{1.132171in}{3.057216in}}%
\pgfpathlineto{\pgfqpoint{1.142599in}{3.064180in}}%
\pgfpathlineto{\pgfqpoint{1.153027in}{3.064101in}}%
\pgfpathlineto{\pgfqpoint{1.163456in}{3.061251in}}%
\pgfpathlineto{\pgfqpoint{1.173884in}{3.069195in}}%
\pgfpathlineto{\pgfqpoint{1.184312in}{3.075789in}}%
\pgfpathlineto{\pgfqpoint{1.194740in}{3.043029in}}%
\pgfpathlineto{\pgfqpoint{1.205168in}{3.042973in}}%
\pgfpathlineto{\pgfqpoint{1.215596in}{3.043815in}}%
\pgfpathlineto{\pgfqpoint{1.226025in}{3.024459in}}%
\pgfpathlineto{\pgfqpoint{1.236453in}{3.044008in}}%
\pgfpathlineto{\pgfqpoint{1.246881in}{3.039734in}}%
\pgfpathlineto{\pgfqpoint{1.257309in}{3.041017in}}%
\pgfpathlineto{\pgfqpoint{1.267737in}{3.043704in}}%
\pgfpathlineto{\pgfqpoint{1.278166in}{3.032495in}}%
\pgfpathlineto{\pgfqpoint{1.288594in}{3.046160in}}%
\pgfpathlineto{\pgfqpoint{1.299022in}{3.050925in}}%
\pgfpathlineto{\pgfqpoint{1.309450in}{3.045621in}}%
\pgfpathlineto{\pgfqpoint{1.319878in}{3.058059in}}%
\pgfpathlineto{\pgfqpoint{1.330306in}{3.058075in}}%
\pgfpathlineto{\pgfqpoint{1.340735in}{3.073598in}}%
\pgfpathlineto{\pgfqpoint{1.351163in}{3.066095in}}%
\pgfpathlineto{\pgfqpoint{1.361591in}{3.074459in}}%
\pgfpathlineto{\pgfqpoint{1.372019in}{3.077038in}}%
\pgfpathlineto{\pgfqpoint{1.382447in}{3.100809in}}%
\pgfpathlineto{\pgfqpoint{1.392875in}{3.075608in}}%
\pgfpathlineto{\pgfqpoint{1.403304in}{3.055378in}}%
\pgfpathlineto{\pgfqpoint{1.413732in}{3.059294in}}%
\pgfpathlineto{\pgfqpoint{1.424160in}{3.101777in}}%
\pgfpathlineto{\pgfqpoint{1.434588in}{3.089286in}}%
\pgfpathlineto{\pgfqpoint{1.445016in}{3.079067in}}%
\pgfpathlineto{\pgfqpoint{1.455444in}{3.060477in}}%
\pgfpathlineto{\pgfqpoint{1.465873in}{3.069614in}}%
\pgfpathlineto{\pgfqpoint{1.476301in}{3.095352in}}%
\pgfpathlineto{\pgfqpoint{1.486729in}{3.087803in}}%
\pgfpathlineto{\pgfqpoint{1.497157in}{3.076944in}}%
\pgfpathlineto{\pgfqpoint{1.507585in}{3.089367in}}%
\pgfpathlineto{\pgfqpoint{1.518014in}{3.097233in}}%
\pgfpathlineto{\pgfqpoint{1.528442in}{3.108068in}}%
\pgfpathlineto{\pgfqpoint{1.538870in}{3.097031in}}%
\pgfpathlineto{\pgfqpoint{1.549298in}{3.090189in}}%
\pgfpathlineto{\pgfqpoint{1.559726in}{3.074122in}}%
\pgfpathlineto{\pgfqpoint{1.570154in}{3.058158in}}%
\pgfpathlineto{\pgfqpoint{1.580583in}{3.073614in}}%
\pgfpathlineto{\pgfqpoint{1.591011in}{3.058004in}}%
\pgfpathlineto{\pgfqpoint{1.601439in}{3.052708in}}%
\pgfpathlineto{\pgfqpoint{1.611867in}{3.067513in}}%
\pgfpathlineto{\pgfqpoint{1.622295in}{3.063262in}}%
\pgfpathlineto{\pgfqpoint{1.632723in}{3.063255in}}%
\pgfpathlineto{\pgfqpoint{1.643152in}{3.019149in}}%
\pgfpathlineto{\pgfqpoint{1.653580in}{3.042950in}}%
\pgfpathlineto{\pgfqpoint{1.664008in}{3.040223in}}%
\pgfpathlineto{\pgfqpoint{1.674436in}{3.034724in}}%
\pgfpathlineto{\pgfqpoint{1.684864in}{3.014512in}}%
\pgfpathlineto{\pgfqpoint{1.695292in}{3.021743in}}%
\pgfpathlineto{\pgfqpoint{1.705721in}{3.010748in}}%
\pgfpathlineto{\pgfqpoint{1.716149in}{3.035138in}}%
\pgfpathlineto{\pgfqpoint{1.726577in}{3.064731in}}%
\pgfpathlineto{\pgfqpoint{1.737005in}{3.067443in}}%
\pgfpathlineto{\pgfqpoint{1.747433in}{3.052542in}}%
\pgfpathlineto{\pgfqpoint{1.757862in}{3.063927in}}%
\pgfpathlineto{\pgfqpoint{1.768290in}{3.066214in}}%
\pgfpathlineto{\pgfqpoint{1.778718in}{3.081422in}}%
\pgfpathlineto{\pgfqpoint{1.789146in}{3.038023in}}%
\pgfpathlineto{\pgfqpoint{1.799574in}{3.073553in}}%
\pgfpathlineto{\pgfqpoint{1.810002in}{3.078380in}}%
\pgfpathlineto{\pgfqpoint{1.820431in}{3.057676in}}%
\pgfpathlineto{\pgfqpoint{1.830859in}{3.053262in}}%
\pgfpathlineto{\pgfqpoint{1.841287in}{3.086470in}}%
\pgfpathlineto{\pgfqpoint{1.851715in}{3.067351in}}%
\pgfpathlineto{\pgfqpoint{1.862143in}{3.074669in}}%
\pgfpathlineto{\pgfqpoint{1.872571in}{3.061571in}}%
\pgfpathlineto{\pgfqpoint{1.883000in}{3.052954in}}%
\pgfpathlineto{\pgfqpoint{1.893428in}{3.045309in}}%
\pgfpathlineto{\pgfqpoint{1.903856in}{3.055458in}}%
\pgfpathlineto{\pgfqpoint{1.914284in}{3.060643in}}%
\pgfpathlineto{\pgfqpoint{1.924712in}{3.032461in}}%
\pgfpathlineto{\pgfqpoint{1.935141in}{3.021785in}}%
\pgfpathlineto{\pgfqpoint{1.945569in}{3.037659in}}%
\pgfpathlineto{\pgfqpoint{1.955997in}{3.020635in}}%
\pgfpathlineto{\pgfqpoint{1.966425in}{3.020439in}}%
\pgfpathlineto{\pgfqpoint{1.976853in}{3.016282in}}%
\pgfpathlineto{\pgfqpoint{1.987281in}{3.017391in}}%
\pgfpathlineto{\pgfqpoint{1.997710in}{3.006204in}}%
\pgfpathlineto{\pgfqpoint{2.008138in}{3.024781in}}%
\pgfpathlineto{\pgfqpoint{2.018566in}{2.989517in}}%
\pgfpathlineto{\pgfqpoint{2.028994in}{3.016032in}}%
\pgfpathlineto{\pgfqpoint{2.039422in}{3.028739in}}%
\pgfpathlineto{\pgfqpoint{2.049850in}{3.049874in}}%
\pgfpathlineto{\pgfqpoint{2.060279in}{3.057945in}}%
\pgfpathlineto{\pgfqpoint{2.070707in}{3.058554in}}%
\pgfpathlineto{\pgfqpoint{2.081135in}{3.059350in}}%
\pgfpathlineto{\pgfqpoint{2.091563in}{3.068331in}}%
\pgfpathlineto{\pgfqpoint{2.101991in}{3.080752in}}%
\pgfpathlineto{\pgfqpoint{2.112419in}{3.098370in}}%
\pgfpathlineto{\pgfqpoint{2.122848in}{3.057693in}}%
\pgfpathlineto{\pgfqpoint{2.133276in}{3.066955in}}%
\pgfpathlineto{\pgfqpoint{2.143704in}{3.044580in}}%
\pgfpathlineto{\pgfqpoint{2.154132in}{3.042116in}}%
\pgfpathlineto{\pgfqpoint{2.164560in}{3.061884in}}%
\pgfpathlineto{\pgfqpoint{2.174989in}{3.049853in}}%
\pgfpathlineto{\pgfqpoint{2.185417in}{3.050995in}}%
\pgfpathlineto{\pgfqpoint{2.195845in}{3.024331in}}%
\pgfpathlineto{\pgfqpoint{2.206273in}{3.028485in}}%
\pgfpathlineto{\pgfqpoint{2.216701in}{3.028955in}}%
\pgfpathlineto{\pgfqpoint{2.227129in}{3.034108in}}%
\pgfpathlineto{\pgfqpoint{2.237558in}{3.037471in}}%
\pgfpathlineto{\pgfqpoint{2.247986in}{3.036826in}}%
\pgfpathlineto{\pgfqpoint{2.258414in}{3.019047in}}%
\pgfpathlineto{\pgfqpoint{2.268842in}{3.030549in}}%
\pgfpathlineto{\pgfqpoint{2.279270in}{2.998725in}}%
\pgfpathlineto{\pgfqpoint{2.289698in}{2.995021in}}%
\pgfpathlineto{\pgfqpoint{2.300127in}{3.009435in}}%
\pgfpathlineto{\pgfqpoint{2.310555in}{2.992819in}}%
\pgfpathlineto{\pgfqpoint{2.320983in}{2.993748in}}%
\pgfpathlineto{\pgfqpoint{2.331411in}{2.986436in}}%
\pgfpathlineto{\pgfqpoint{2.341839in}{3.006720in}}%
\pgfpathlineto{\pgfqpoint{2.352267in}{3.004001in}}%
\pgfpathlineto{\pgfqpoint{2.362696in}{2.992635in}}%
\pgfpathlineto{\pgfqpoint{2.373124in}{3.005111in}}%
\pgfpathlineto{\pgfqpoint{2.383552in}{3.007291in}}%
\pgfpathlineto{\pgfqpoint{2.393980in}{3.007629in}}%
\pgfpathlineto{\pgfqpoint{2.404408in}{2.994933in}}%
\pgfpathlineto{\pgfqpoint{2.414837in}{3.020817in}}%
\pgfpathlineto{\pgfqpoint{2.425265in}{3.027177in}}%
\pgfpathlineto{\pgfqpoint{2.435693in}{3.019740in}}%
\pgfpathlineto{\pgfqpoint{2.446121in}{3.035596in}}%
\pgfpathlineto{\pgfqpoint{2.456549in}{3.035837in}}%
\pgfpathlineto{\pgfqpoint{2.466977in}{3.040204in}}%
\pgfpathlineto{\pgfqpoint{2.477406in}{3.032010in}}%
\pgfpathlineto{\pgfqpoint{2.487834in}{3.026218in}}%
\pgfpathlineto{\pgfqpoint{2.498262in}{3.012459in}}%
\pgfpathlineto{\pgfqpoint{2.508690in}{3.018204in}}%
\pgfpathlineto{\pgfqpoint{2.519118in}{2.991098in}}%
\pgfpathlineto{\pgfqpoint{2.529546in}{3.029767in}}%
\pgfpathlineto{\pgfqpoint{2.539975in}{3.016778in}}%
\pgfpathlineto{\pgfqpoint{2.550403in}{3.031933in}}%
\pgfpathlineto{\pgfqpoint{2.560831in}{3.026551in}}%
\pgfpathlineto{\pgfqpoint{2.571259in}{3.026736in}}%
\pgfpathlineto{\pgfqpoint{2.581687in}{3.001493in}}%
\pgfpathlineto{\pgfqpoint{2.592115in}{3.014980in}}%
\pgfpathlineto{\pgfqpoint{2.602544in}{2.988874in}}%
\pgfpathlineto{\pgfqpoint{2.612972in}{3.005843in}}%
\pgfpathlineto{\pgfqpoint{2.623400in}{2.999410in}}%
\pgfpathlineto{\pgfqpoint{2.633828in}{2.987129in}}%
\pgfpathlineto{\pgfqpoint{2.644256in}{2.972720in}}%
\pgfpathlineto{\pgfqpoint{2.654685in}{2.980860in}}%
\pgfpathlineto{\pgfqpoint{2.665113in}{2.951836in}}%
\pgfpathlineto{\pgfqpoint{2.675541in}{2.907764in}}%
\pgfpathlineto{\pgfqpoint{2.685969in}{2.839757in}}%
\pgfpathlineto{\pgfqpoint{2.696397in}{2.828374in}}%
\pgfpathlineto{\pgfqpoint{2.706825in}{2.767206in}}%
\pgfpathlineto{\pgfqpoint{2.717254in}{2.718652in}}%
\pgfpathlineto{\pgfqpoint{2.727682in}{2.690632in}}%
\pgfpathlineto{\pgfqpoint{2.738110in}{2.683995in}}%
\pgfpathlineto{\pgfqpoint{2.748538in}{2.675902in}}%
\pgfpathlineto{\pgfqpoint{2.758966in}{2.680950in}}%
\pgfpathlineto{\pgfqpoint{2.769394in}{2.681883in}}%
\pgfpathlineto{\pgfqpoint{2.779823in}{2.669183in}}%
\pgfpathlineto{\pgfqpoint{2.790251in}{2.657842in}}%
\pgfpathlineto{\pgfqpoint{2.800679in}{2.662098in}}%
\pgfpathlineto{\pgfqpoint{2.811107in}{2.640107in}}%
\pgfpathlineto{\pgfqpoint{2.821535in}{2.634852in}}%
\pgfpathlineto{\pgfqpoint{2.831964in}{2.603445in}}%
\pgfpathlineto{\pgfqpoint{2.842392in}{2.604523in}}%
\pgfpathlineto{\pgfqpoint{2.852820in}{2.603393in}}%
\pgfpathlineto{\pgfqpoint{2.863248in}{2.597949in}}%
\pgfpathlineto{\pgfqpoint{2.873676in}{2.597961in}}%
\pgfpathlineto{\pgfqpoint{2.884104in}{2.575826in}}%
\pgfpathlineto{\pgfqpoint{2.894533in}{2.567560in}}%
\pgfpathlineto{\pgfqpoint{2.904961in}{2.529445in}}%
\pgfpathlineto{\pgfqpoint{2.915389in}{2.547329in}}%
\pgfpathlineto{\pgfqpoint{2.925817in}{2.523994in}}%
\pgfpathlineto{\pgfqpoint{2.936245in}{2.525604in}}%
\pgfpathlineto{\pgfqpoint{2.946673in}{2.536642in}}%
\pgfpathlineto{\pgfqpoint{2.957102in}{2.530953in}}%
\pgfpathlineto{\pgfqpoint{2.967530in}{2.505941in}}%
\pgfpathlineto{\pgfqpoint{2.977958in}{2.504775in}}%
\pgfpathlineto{\pgfqpoint{2.988386in}{2.503935in}}%
\pgfpathlineto{\pgfqpoint{2.998814in}{2.516338in}}%
\pgfpathlineto{\pgfqpoint{3.009242in}{2.491820in}}%
\pgfpathlineto{\pgfqpoint{3.019671in}{2.511602in}}%
\pgfpathlineto{\pgfqpoint{3.030099in}{2.535043in}}%
\pgfpathlineto{\pgfqpoint{3.040527in}{2.482362in}}%
\pgfpathlineto{\pgfqpoint{3.050955in}{2.502915in}}%
\pgfpathlineto{\pgfqpoint{3.061383in}{2.529174in}}%
\pgfpathlineto{\pgfqpoint{3.071812in}{2.473569in}}%
\pgfpathlineto{\pgfqpoint{3.082240in}{2.447319in}}%
\pgfpathlineto{\pgfqpoint{3.092668in}{2.449527in}}%
\pgfpathlineto{\pgfqpoint{3.103096in}{2.482129in}}%
\pgfpathlineto{\pgfqpoint{3.113524in}{2.448272in}}%
\pgfpathlineto{\pgfqpoint{3.123952in}{2.468592in}}%
\pgfpathlineto{\pgfqpoint{3.134381in}{2.422903in}}%
\pgfpathlineto{\pgfqpoint{3.144809in}{2.460533in}}%
\pgfpathlineto{\pgfqpoint{3.155237in}{2.448225in}}%
\pgfpathlineto{\pgfqpoint{3.165665in}{2.392927in}}%
\pgfpathlineto{\pgfqpoint{3.176093in}{2.476961in}}%
\pgfpathlineto{\pgfqpoint{3.186521in}{2.449346in}}%
\pgfpathlineto{\pgfqpoint{3.196950in}{2.455374in}}%
\pgfpathlineto{\pgfqpoint{3.207378in}{2.448148in}}%
\pgfpathlineto{\pgfqpoint{3.217806in}{2.457308in}}%
\pgfpathlineto{\pgfqpoint{3.228234in}{2.460559in}}%
\pgfpathlineto{\pgfqpoint{3.238662in}{2.477848in}}%
\pgfpathlineto{\pgfqpoint{3.249090in}{2.423852in}}%
\pgfpathlineto{\pgfqpoint{3.259519in}{2.422537in}}%
\pgfpathlineto{\pgfqpoint{3.269947in}{2.444441in}}%
\pgfpathlineto{\pgfqpoint{3.280375in}{2.419127in}}%
\pgfpathlineto{\pgfqpoint{3.290803in}{2.401711in}}%
\pgfpathlineto{\pgfqpoint{3.301231in}{2.432245in}}%
\pgfpathlineto{\pgfqpoint{3.311660in}{2.471150in}}%
\pgfpathlineto{\pgfqpoint{3.322088in}{2.472242in}}%
\pgfpathlineto{\pgfqpoint{3.332516in}{2.434395in}}%
\pgfpathlineto{\pgfqpoint{3.342944in}{2.453495in}}%
\pgfpathlineto{\pgfqpoint{3.353372in}{2.421109in}}%
\pgfpathlineto{\pgfqpoint{3.363800in}{2.417183in}}%
\pgfpathlineto{\pgfqpoint{3.374229in}{2.433648in}}%
\pgfpathlineto{\pgfqpoint{3.384657in}{2.395743in}}%
\pgfpathlineto{\pgfqpoint{3.395085in}{2.390441in}}%
\pgfpathlineto{\pgfqpoint{3.405513in}{2.386518in}}%
\pgfpathlineto{\pgfqpoint{3.415941in}{2.417363in}}%
\pgfpathlineto{\pgfqpoint{3.426369in}{2.383704in}}%
\pgfpathlineto{\pgfqpoint{3.436798in}{2.376574in}}%
\pgfpathlineto{\pgfqpoint{3.447226in}{2.381752in}}%
\pgfpathlineto{\pgfqpoint{3.457654in}{2.380854in}}%
\pgfpathlineto{\pgfqpoint{3.468082in}{2.236732in}}%
\pgfpathlineto{\pgfqpoint{3.478510in}{2.271551in}}%
\pgfpathlineto{\pgfqpoint{3.488938in}{2.247152in}}%
\pgfpathlineto{\pgfqpoint{3.499367in}{2.270998in}}%
\pgfpathlineto{\pgfqpoint{3.509795in}{2.325572in}}%
\pgfpathlineto{\pgfqpoint{3.520223in}{2.339068in}}%
\pgfpathlineto{\pgfqpoint{3.530651in}{2.350859in}}%
\pgfpathlineto{\pgfqpoint{3.541079in}{2.332553in}}%
\pgfpathlineto{\pgfqpoint{3.551508in}{2.269960in}}%
\pgfpathlineto{\pgfqpoint{3.561936in}{2.310520in}}%
\pgfpathlineto{\pgfqpoint{3.572364in}{2.261194in}}%
\pgfpathlineto{\pgfqpoint{3.582792in}{2.268869in}}%
\pgfpathlineto{\pgfqpoint{3.593220in}{2.167936in}}%
\pgfpathlineto{\pgfqpoint{3.603648in}{2.203775in}}%
\pgfpathlineto{\pgfqpoint{3.614077in}{2.159887in}}%
\pgfpathlineto{\pgfqpoint{3.624505in}{2.169715in}}%
\pgfpathlineto{\pgfqpoint{3.634933in}{2.166102in}}%
\pgfpathlineto{\pgfqpoint{3.645361in}{2.171267in}}%
\pgfpathlineto{\pgfqpoint{3.655789in}{2.169149in}}%
\pgfpathlineto{\pgfqpoint{3.666217in}{2.168607in}}%
\pgfpathlineto{\pgfqpoint{3.676646in}{2.084941in}}%
\pgfpathlineto{\pgfqpoint{3.687074in}{2.154834in}}%
\pgfpathlineto{\pgfqpoint{3.697502in}{2.121590in}}%
\pgfpathlineto{\pgfqpoint{3.707930in}{2.099550in}}%
\pgfpathlineto{\pgfqpoint{3.718358in}{2.110375in}}%
\pgfpathlineto{\pgfqpoint{3.728787in}{2.229024in}}%
\pgfpathlineto{\pgfqpoint{3.739215in}{2.069413in}}%
\pgfpathlineto{\pgfqpoint{3.749643in}{2.063412in}}%
\pgfpathlineto{\pgfqpoint{3.760071in}{2.087309in}}%
\pgfpathlineto{\pgfqpoint{3.770499in}{2.019457in}}%
\pgfpathlineto{\pgfqpoint{3.780927in}{2.149612in}}%
\pgfpathlineto{\pgfqpoint{3.791356in}{2.097493in}}%
\pgfpathlineto{\pgfqpoint{3.801784in}{2.066683in}}%
\pgfpathlineto{\pgfqpoint{3.812212in}{2.100086in}}%
\pgfpathlineto{\pgfqpoint{3.822640in}{2.039112in}}%
\pgfpathlineto{\pgfqpoint{3.833068in}{1.963533in}}%
\pgfpathlineto{\pgfqpoint{3.843496in}{1.930872in}}%
\pgfpathlineto{\pgfqpoint{3.853925in}{1.960387in}}%
\pgfpathlineto{\pgfqpoint{3.864353in}{1.985461in}}%
\pgfpathlineto{\pgfqpoint{3.874781in}{2.034368in}}%
\pgfpathlineto{\pgfqpoint{3.885209in}{2.069433in}}%
\pgfpathlineto{\pgfqpoint{3.895637in}{1.943242in}}%
\pgfpathlineto{\pgfqpoint{3.906065in}{1.844244in}}%
\pgfpathlineto{\pgfqpoint{3.916494in}{2.011870in}}%
\pgfpathlineto{\pgfqpoint{3.926922in}{2.028962in}}%
\pgfpathlineto{\pgfqpoint{3.937350in}{1.867310in}}%
\pgfpathlineto{\pgfqpoint{3.947778in}{2.068422in}}%
\pgfpathlineto{\pgfqpoint{3.958206in}{1.990672in}}%
\pgfpathlineto{\pgfqpoint{3.968635in}{2.053760in}}%
\pgfpathlineto{\pgfqpoint{3.979063in}{1.890132in}}%
\pgfpathlineto{\pgfqpoint{3.989491in}{1.971551in}}%
\pgfpathlineto{\pgfqpoint{3.999919in}{2.092485in}}%
\pgfpathlineto{\pgfqpoint{4.010347in}{1.958683in}}%
\pgfpathlineto{\pgfqpoint{4.020775in}{2.120627in}}%
\pgfpathlineto{\pgfqpoint{4.031204in}{2.025527in}}%
\pgfpathlineto{\pgfqpoint{4.041632in}{2.098917in}}%
\pgfpathlineto{\pgfqpoint{4.052060in}{2.047643in}}%
\pgfpathlineto{\pgfqpoint{4.062488in}{1.965548in}}%
\pgfpathlineto{\pgfqpoint{4.072916in}{2.023965in}}%
\pgfpathlineto{\pgfqpoint{4.083344in}{1.930286in}}%
\pgfpathlineto{\pgfqpoint{4.093773in}{1.967219in}}%
\pgfpathlineto{\pgfqpoint{4.104201in}{1.968060in}}%
\pgfpathlineto{\pgfqpoint{4.114629in}{2.001571in}}%
\pgfpathlineto{\pgfqpoint{4.125057in}{1.942182in}}%
\pgfpathlineto{\pgfqpoint{4.135485in}{2.060639in}}%
\pgfpathlineto{\pgfqpoint{4.145913in}{1.937266in}}%
\pgfpathlineto{\pgfqpoint{4.156342in}{2.023116in}}%
\pgfpathlineto{\pgfqpoint{4.166770in}{2.029603in}}%
\pgfpathlineto{\pgfqpoint{4.177198in}{2.073274in}}%
\pgfpathlineto{\pgfqpoint{4.187626in}{1.955815in}}%
\pgfpathlineto{\pgfqpoint{4.198054in}{1.895415in}}%
\pgfpathlineto{\pgfqpoint{4.208483in}{1.808971in}}%
\pgfpathlineto{\pgfqpoint{4.218911in}{2.021955in}}%
\pgfpathlineto{\pgfqpoint{4.229339in}{1.837077in}}%
\pgfpathlineto{\pgfqpoint{4.239767in}{1.988202in}}%
\pgfpathlineto{\pgfqpoint{4.250195in}{1.817847in}}%
\pgfpathlineto{\pgfqpoint{4.260623in}{1.880663in}}%
\pgfpathlineto{\pgfqpoint{4.271052in}{1.907740in}}%
\pgfpathlineto{\pgfqpoint{4.281480in}{1.843461in}}%
\pgfpathlineto{\pgfqpoint{4.291908in}{1.927830in}}%
\pgfpathlineto{\pgfqpoint{4.302336in}{2.030713in}}%
\pgfpathlineto{\pgfqpoint{4.312764in}{1.803338in}}%
\pgfpathlineto{\pgfqpoint{4.323192in}{1.784051in}}%
\pgfpathlineto{\pgfqpoint{4.333621in}{1.822140in}}%
\pgfpathlineto{\pgfqpoint{4.344049in}{1.745262in}}%
\pgfpathlineto{\pgfqpoint{4.354477in}{1.826843in}}%
\pgfpathlineto{\pgfqpoint{4.364905in}{1.956715in}}%
\pgfpathlineto{\pgfqpoint{4.375333in}{2.167693in}}%
\pgfpathlineto{\pgfqpoint{4.385761in}{1.896516in}}%
\pgfpathlineto{\pgfqpoint{4.396190in}{2.020498in}}%
\pgfpathlineto{\pgfqpoint{4.406618in}{1.805691in}}%
\pgfpathlineto{\pgfqpoint{4.417046in}{1.835098in}}%
\pgfpathlineto{\pgfqpoint{4.427474in}{1.924468in}}%
\pgfpathlineto{\pgfqpoint{4.437902in}{1.902954in}}%
\pgfpathlineto{\pgfqpoint{4.448331in}{1.942740in}}%
\pgfpathlineto{\pgfqpoint{4.458759in}{1.954469in}}%
\pgfpathlineto{\pgfqpoint{4.469187in}{1.903415in}}%
\pgfpathlineto{\pgfqpoint{4.479615in}{1.913391in}}%
\pgfpathlineto{\pgfqpoint{4.490043in}{2.030688in}}%
\pgfpathlineto{\pgfqpoint{4.500471in}{2.226562in}}%
\pgfpathlineto{\pgfqpoint{4.510900in}{2.006233in}}%
\pgfpathlineto{\pgfqpoint{4.521328in}{1.951827in}}%
\pgfpathlineto{\pgfqpoint{4.531756in}{2.140189in}}%
\pgfpathlineto{\pgfqpoint{4.542184in}{2.120573in}}%
\pgfpathlineto{\pgfqpoint{4.552612in}{2.026334in}}%
\pgfpathlineto{\pgfqpoint{4.563040in}{1.814648in}}%
\pgfpathlineto{\pgfqpoint{4.573469in}{1.925553in}}%
\pgfpathlineto{\pgfqpoint{4.583897in}{1.962852in}}%
\pgfpathlineto{\pgfqpoint{4.594325in}{1.864595in}}%
\pgfpathlineto{\pgfqpoint{4.604753in}{1.839953in}}%
\pgfpathlineto{\pgfqpoint{4.615181in}{2.048127in}}%
\pgfpathlineto{\pgfqpoint{4.625610in}{2.028213in}}%
\pgfpathlineto{\pgfqpoint{4.636038in}{1.975383in}}%
\pgfpathlineto{\pgfqpoint{4.646466in}{1.889312in}}%
\pgfpathlineto{\pgfqpoint{4.656894in}{1.938728in}}%
\pgfpathlineto{\pgfqpoint{4.667322in}{1.965193in}}%
\pgfpathlineto{\pgfqpoint{4.677750in}{1.951915in}}%
\pgfpathlineto{\pgfqpoint{4.688179in}{2.156478in}}%
\pgfpathlineto{\pgfqpoint{4.698607in}{1.981371in}}%
\pgfpathlineto{\pgfqpoint{4.709035in}{2.024890in}}%
\pgfpathlineto{\pgfqpoint{4.719463in}{1.945121in}}%
\pgfpathlineto{\pgfqpoint{4.729891in}{2.127673in}}%
\pgfpathlineto{\pgfqpoint{4.740319in}{1.934500in}}%
\pgfpathlineto{\pgfqpoint{4.750748in}{2.061491in}}%
\pgfpathlineto{\pgfqpoint{4.761176in}{2.128968in}}%
\pgfpathlineto{\pgfqpoint{4.771604in}{1.699970in}}%
\pgfpathlineto{\pgfqpoint{4.771604in}{1.440876in}}%
\pgfpathlineto{\pgfqpoint{4.771604in}{1.440876in}}%
\pgfpathlineto{\pgfqpoint{4.761176in}{1.930405in}}%
\pgfpathlineto{\pgfqpoint{4.750748in}{1.851347in}}%
\pgfpathlineto{\pgfqpoint{4.740319in}{1.684647in}}%
\pgfpathlineto{\pgfqpoint{4.729891in}{1.835331in}}%
\pgfpathlineto{\pgfqpoint{4.719463in}{1.529802in}}%
\pgfpathlineto{\pgfqpoint{4.709035in}{1.805742in}}%
\pgfpathlineto{\pgfqpoint{4.698607in}{1.660449in}}%
\pgfpathlineto{\pgfqpoint{4.688179in}{1.831522in}}%
\pgfpathlineto{\pgfqpoint{4.677750in}{1.648013in}}%
\pgfpathlineto{\pgfqpoint{4.667322in}{1.771047in}}%
\pgfpathlineto{\pgfqpoint{4.656894in}{1.770586in}}%
\pgfpathlineto{\pgfqpoint{4.646466in}{1.690682in}}%
\pgfpathlineto{\pgfqpoint{4.636038in}{1.716812in}}%
\pgfpathlineto{\pgfqpoint{4.625610in}{1.733018in}}%
\pgfpathlineto{\pgfqpoint{4.615181in}{1.767129in}}%
\pgfpathlineto{\pgfqpoint{4.604753in}{1.668657in}}%
\pgfpathlineto{\pgfqpoint{4.594325in}{1.627582in}}%
\pgfpathlineto{\pgfqpoint{4.583897in}{1.717831in}}%
\pgfpathlineto{\pgfqpoint{4.573469in}{1.650346in}}%
\pgfpathlineto{\pgfqpoint{4.563040in}{1.534452in}}%
\pgfpathlineto{\pgfqpoint{4.552612in}{1.716753in}}%
\pgfpathlineto{\pgfqpoint{4.542184in}{1.813991in}}%
\pgfpathlineto{\pgfqpoint{4.531756in}{1.845569in}}%
\pgfpathlineto{\pgfqpoint{4.521328in}{1.677712in}}%
\pgfpathlineto{\pgfqpoint{4.510900in}{1.697961in}}%
\pgfpathlineto{\pgfqpoint{4.500471in}{1.963277in}}%
\pgfpathlineto{\pgfqpoint{4.490043in}{1.811855in}}%
\pgfpathlineto{\pgfqpoint{4.479615in}{1.688198in}}%
\pgfpathlineto{\pgfqpoint{4.469187in}{1.640494in}}%
\pgfpathlineto{\pgfqpoint{4.458759in}{1.648160in}}%
\pgfpathlineto{\pgfqpoint{4.448331in}{1.687869in}}%
\pgfpathlineto{\pgfqpoint{4.437902in}{1.642116in}}%
\pgfpathlineto{\pgfqpoint{4.427474in}{1.642917in}}%
\pgfpathlineto{\pgfqpoint{4.417046in}{1.581291in}}%
\pgfpathlineto{\pgfqpoint{4.406618in}{1.462776in}}%
\pgfpathlineto{\pgfqpoint{4.396190in}{1.770741in}}%
\pgfpathlineto{\pgfqpoint{4.385761in}{1.601413in}}%
\pgfpathlineto{\pgfqpoint{4.375333in}{1.850351in}}%
\pgfpathlineto{\pgfqpoint{4.364905in}{1.666312in}}%
\pgfpathlineto{\pgfqpoint{4.354477in}{1.565095in}}%
\pgfpathlineto{\pgfqpoint{4.344049in}{1.452382in}}%
\pgfpathlineto{\pgfqpoint{4.333621in}{1.560199in}}%
\pgfpathlineto{\pgfqpoint{4.323192in}{1.537494in}}%
\pgfpathlineto{\pgfqpoint{4.312764in}{1.550348in}}%
\pgfpathlineto{\pgfqpoint{4.302336in}{1.807755in}}%
\pgfpathlineto{\pgfqpoint{4.291908in}{1.648990in}}%
\pgfpathlineto{\pgfqpoint{4.281480in}{1.571105in}}%
\pgfpathlineto{\pgfqpoint{4.271052in}{1.642808in}}%
\pgfpathlineto{\pgfqpoint{4.260623in}{1.629696in}}%
\pgfpathlineto{\pgfqpoint{4.250195in}{1.600902in}}%
\pgfpathlineto{\pgfqpoint{4.239767in}{1.761579in}}%
\pgfpathlineto{\pgfqpoint{4.229339in}{1.582176in}}%
\pgfpathlineto{\pgfqpoint{4.218911in}{1.766798in}}%
\pgfpathlineto{\pgfqpoint{4.208483in}{1.582612in}}%
\pgfpathlineto{\pgfqpoint{4.198054in}{1.692390in}}%
\pgfpathlineto{\pgfqpoint{4.187626in}{1.728214in}}%
\pgfpathlineto{\pgfqpoint{4.177198in}{1.838991in}}%
\pgfpathlineto{\pgfqpoint{4.166770in}{1.805801in}}%
\pgfpathlineto{\pgfqpoint{4.156342in}{1.759553in}}%
\pgfpathlineto{\pgfqpoint{4.145913in}{1.683071in}}%
\pgfpathlineto{\pgfqpoint{4.135485in}{1.812390in}}%
\pgfpathlineto{\pgfqpoint{4.125057in}{1.670278in}}%
\pgfpathlineto{\pgfqpoint{4.114629in}{1.732154in}}%
\pgfpathlineto{\pgfqpoint{4.104201in}{1.700201in}}%
\pgfpathlineto{\pgfqpoint{4.093773in}{1.716452in}}%
\pgfpathlineto{\pgfqpoint{4.083344in}{1.651855in}}%
\pgfpathlineto{\pgfqpoint{4.072916in}{1.798530in}}%
\pgfpathlineto{\pgfqpoint{4.062488in}{1.696277in}}%
\pgfpathlineto{\pgfqpoint{4.052060in}{1.826546in}}%
\pgfpathlineto{\pgfqpoint{4.041632in}{1.871042in}}%
\pgfpathlineto{\pgfqpoint{4.031204in}{1.769734in}}%
\pgfpathlineto{\pgfqpoint{4.020775in}{1.880806in}}%
\pgfpathlineto{\pgfqpoint{4.010347in}{1.711785in}}%
\pgfpathlineto{\pgfqpoint{3.999919in}{1.881144in}}%
\pgfpathlineto{\pgfqpoint{3.989491in}{1.731703in}}%
\pgfpathlineto{\pgfqpoint{3.979063in}{1.634437in}}%
\pgfpathlineto{\pgfqpoint{3.968635in}{1.861691in}}%
\pgfpathlineto{\pgfqpoint{3.958206in}{1.745609in}}%
\pgfpathlineto{\pgfqpoint{3.947778in}{1.844860in}}%
\pgfpathlineto{\pgfqpoint{3.937350in}{1.603296in}}%
\pgfpathlineto{\pgfqpoint{3.926922in}{1.804491in}}%
\pgfpathlineto{\pgfqpoint{3.916494in}{1.808128in}}%
\pgfpathlineto{\pgfqpoint{3.906065in}{1.613770in}}%
\pgfpathlineto{\pgfqpoint{3.895637in}{1.676109in}}%
\pgfpathlineto{\pgfqpoint{3.885209in}{1.860091in}}%
\pgfpathlineto{\pgfqpoint{3.874781in}{1.814053in}}%
\pgfpathlineto{\pgfqpoint{3.864353in}{1.748445in}}%
\pgfpathlineto{\pgfqpoint{3.853925in}{1.649902in}}%
\pgfpathlineto{\pgfqpoint{3.843496in}{1.655564in}}%
\pgfpathlineto{\pgfqpoint{3.833068in}{1.679511in}}%
\pgfpathlineto{\pgfqpoint{3.822640in}{1.734091in}}%
\pgfpathlineto{\pgfqpoint{3.812212in}{1.842412in}}%
\pgfpathlineto{\pgfqpoint{3.801784in}{1.811161in}}%
\pgfpathlineto{\pgfqpoint{3.791356in}{1.853083in}}%
\pgfpathlineto{\pgfqpoint{3.780927in}{1.942947in}}%
\pgfpathlineto{\pgfqpoint{3.770499in}{1.788593in}}%
\pgfpathlineto{\pgfqpoint{3.760071in}{1.837616in}}%
\pgfpathlineto{\pgfqpoint{3.749643in}{1.818404in}}%
\pgfpathlineto{\pgfqpoint{3.739215in}{1.836135in}}%
\pgfpathlineto{\pgfqpoint{3.728787in}{2.012715in}}%
\pgfpathlineto{\pgfqpoint{3.718358in}{1.892265in}}%
\pgfpathlineto{\pgfqpoint{3.707930in}{1.865458in}}%
\pgfpathlineto{\pgfqpoint{3.697502in}{1.912971in}}%
\pgfpathlineto{\pgfqpoint{3.687074in}{1.945921in}}%
\pgfpathlineto{\pgfqpoint{3.676646in}{1.861722in}}%
\pgfpathlineto{\pgfqpoint{3.666217in}{1.952573in}}%
\pgfpathlineto{\pgfqpoint{3.655789in}{1.955565in}}%
\pgfpathlineto{\pgfqpoint{3.645361in}{1.950618in}}%
\pgfpathlineto{\pgfqpoint{3.634933in}{1.926271in}}%
\pgfpathlineto{\pgfqpoint{3.624505in}{1.950029in}}%
\pgfpathlineto{\pgfqpoint{3.614077in}{1.935440in}}%
\pgfpathlineto{\pgfqpoint{3.603648in}{1.980559in}}%
\pgfpathlineto{\pgfqpoint{3.593220in}{1.936129in}}%
\pgfpathlineto{\pgfqpoint{3.582792in}{2.050522in}}%
\pgfpathlineto{\pgfqpoint{3.572364in}{2.035129in}}%
\pgfpathlineto{\pgfqpoint{3.561936in}{2.086132in}}%
\pgfpathlineto{\pgfqpoint{3.551508in}{2.041264in}}%
\pgfpathlineto{\pgfqpoint{3.541079in}{2.128255in}}%
\pgfpathlineto{\pgfqpoint{3.530651in}{2.146819in}}%
\pgfpathlineto{\pgfqpoint{3.520223in}{2.151957in}}%
\pgfpathlineto{\pgfqpoint{3.509795in}{2.131018in}}%
\pgfpathlineto{\pgfqpoint{3.499367in}{2.071665in}}%
\pgfpathlineto{\pgfqpoint{3.488938in}{2.049778in}}%
\pgfpathlineto{\pgfqpoint{3.478510in}{2.069371in}}%
\pgfpathlineto{\pgfqpoint{3.468082in}{2.026475in}}%
\pgfpathlineto{\pgfqpoint{3.457654in}{2.210002in}}%
\pgfpathlineto{\pgfqpoint{3.447226in}{2.204241in}}%
\pgfpathlineto{\pgfqpoint{3.436798in}{2.201437in}}%
\pgfpathlineto{\pgfqpoint{3.426369in}{2.214240in}}%
\pgfpathlineto{\pgfqpoint{3.415941in}{2.259652in}}%
\pgfpathlineto{\pgfqpoint{3.405513in}{2.209276in}}%
\pgfpathlineto{\pgfqpoint{3.395085in}{2.216003in}}%
\pgfpathlineto{\pgfqpoint{3.384657in}{2.222033in}}%
\pgfpathlineto{\pgfqpoint{3.374229in}{2.265139in}}%
\pgfpathlineto{\pgfqpoint{3.363800in}{2.247650in}}%
\pgfpathlineto{\pgfqpoint{3.353372in}{2.260044in}}%
\pgfpathlineto{\pgfqpoint{3.342944in}{2.284586in}}%
\pgfpathlineto{\pgfqpoint{3.332516in}{2.269999in}}%
\pgfpathlineto{\pgfqpoint{3.322088in}{2.317029in}}%
\pgfpathlineto{\pgfqpoint{3.311660in}{2.320018in}}%
\pgfpathlineto{\pgfqpoint{3.301231in}{2.271243in}}%
\pgfpathlineto{\pgfqpoint{3.290803in}{2.233838in}}%
\pgfpathlineto{\pgfqpoint{3.280375in}{2.260100in}}%
\pgfpathlineto{\pgfqpoint{3.269947in}{2.300209in}}%
\pgfpathlineto{\pgfqpoint{3.259519in}{2.274365in}}%
\pgfpathlineto{\pgfqpoint{3.249090in}{2.285372in}}%
\pgfpathlineto{\pgfqpoint{3.238662in}{2.342078in}}%
\pgfpathlineto{\pgfqpoint{3.228234in}{2.321841in}}%
\pgfpathlineto{\pgfqpoint{3.217806in}{2.318367in}}%
\pgfpathlineto{\pgfqpoint{3.207378in}{2.316291in}}%
\pgfpathlineto{\pgfqpoint{3.196950in}{2.316848in}}%
\pgfpathlineto{\pgfqpoint{3.186521in}{2.306136in}}%
\pgfpathlineto{\pgfqpoint{3.176093in}{2.349726in}}%
\pgfpathlineto{\pgfqpoint{3.165665in}{2.261359in}}%
\pgfpathlineto{\pgfqpoint{3.155237in}{2.326479in}}%
\pgfpathlineto{\pgfqpoint{3.144809in}{2.340792in}}%
\pgfpathlineto{\pgfqpoint{3.134381in}{2.298173in}}%
\pgfpathlineto{\pgfqpoint{3.123952in}{2.352079in}}%
\pgfpathlineto{\pgfqpoint{3.113524in}{2.330913in}}%
\pgfpathlineto{\pgfqpoint{3.103096in}{2.371354in}}%
\pgfpathlineto{\pgfqpoint{3.092668in}{2.334252in}}%
\pgfpathlineto{\pgfqpoint{3.082240in}{2.332470in}}%
\pgfpathlineto{\pgfqpoint{3.071812in}{2.362036in}}%
\pgfpathlineto{\pgfqpoint{3.061383in}{2.428339in}}%
\pgfpathlineto{\pgfqpoint{3.050955in}{2.398962in}}%
\pgfpathlineto{\pgfqpoint{3.040527in}{2.377966in}}%
\pgfpathlineto{\pgfqpoint{3.030099in}{2.438308in}}%
\pgfpathlineto{\pgfqpoint{3.019671in}{2.414820in}}%
\pgfpathlineto{\pgfqpoint{3.009242in}{2.387477in}}%
\pgfpathlineto{\pgfqpoint{2.998814in}{2.420701in}}%
\pgfpathlineto{\pgfqpoint{2.988386in}{2.404468in}}%
\pgfpathlineto{\pgfqpoint{2.977958in}{2.408970in}}%
\pgfpathlineto{\pgfqpoint{2.967530in}{2.408314in}}%
\pgfpathlineto{\pgfqpoint{2.957102in}{2.445421in}}%
\pgfpathlineto{\pgfqpoint{2.946673in}{2.445293in}}%
\pgfpathlineto{\pgfqpoint{2.936245in}{2.436330in}}%
\pgfpathlineto{\pgfqpoint{2.925817in}{2.435824in}}%
\pgfpathlineto{\pgfqpoint{2.915389in}{2.464951in}}%
\pgfpathlineto{\pgfqpoint{2.904961in}{2.445145in}}%
\pgfpathlineto{\pgfqpoint{2.894533in}{2.491399in}}%
\pgfpathlineto{\pgfqpoint{2.884104in}{2.499474in}}%
\pgfpathlineto{\pgfqpoint{2.873676in}{2.522273in}}%
\pgfpathlineto{\pgfqpoint{2.863248in}{2.522467in}}%
\pgfpathlineto{\pgfqpoint{2.852820in}{2.533260in}}%
\pgfpathlineto{\pgfqpoint{2.842392in}{2.534862in}}%
\pgfpathlineto{\pgfqpoint{2.831964in}{2.534676in}}%
\pgfpathlineto{\pgfqpoint{2.821535in}{2.570381in}}%
\pgfpathlineto{\pgfqpoint{2.811107in}{2.578760in}}%
\pgfpathlineto{\pgfqpoint{2.800679in}{2.604012in}}%
\pgfpathlineto{\pgfqpoint{2.790251in}{2.597133in}}%
\pgfpathlineto{\pgfqpoint{2.779823in}{2.612021in}}%
\pgfpathlineto{\pgfqpoint{2.769394in}{2.627546in}}%
\pgfpathlineto{\pgfqpoint{2.758966in}{2.624790in}}%
\pgfpathlineto{\pgfqpoint{2.748538in}{2.617605in}}%
\pgfpathlineto{\pgfqpoint{2.738110in}{2.628633in}}%
\pgfpathlineto{\pgfqpoint{2.727682in}{2.638788in}}%
\pgfpathlineto{\pgfqpoint{2.717254in}{2.663567in}}%
\pgfpathlineto{\pgfqpoint{2.706825in}{2.719767in}}%
\pgfpathlineto{\pgfqpoint{2.696397in}{2.781754in}}%
\pgfpathlineto{\pgfqpoint{2.685969in}{2.796134in}}%
\pgfpathlineto{\pgfqpoint{2.675541in}{2.868603in}}%
\pgfpathlineto{\pgfqpoint{2.665113in}{2.914791in}}%
\pgfpathlineto{\pgfqpoint{2.654685in}{2.942753in}}%
\pgfpathlineto{\pgfqpoint{2.644256in}{2.933582in}}%
\pgfpathlineto{\pgfqpoint{2.633828in}{2.948931in}}%
\pgfpathlineto{\pgfqpoint{2.623400in}{2.964915in}}%
\pgfpathlineto{\pgfqpoint{2.612972in}{2.970608in}}%
\pgfpathlineto{\pgfqpoint{2.602544in}{2.949735in}}%
\pgfpathlineto{\pgfqpoint{2.592115in}{2.978417in}}%
\pgfpathlineto{\pgfqpoint{2.581687in}{2.963599in}}%
\pgfpathlineto{\pgfqpoint{2.571259in}{2.991220in}}%
\pgfpathlineto{\pgfqpoint{2.560831in}{2.993797in}}%
\pgfpathlineto{\pgfqpoint{2.550403in}{2.993841in}}%
\pgfpathlineto{\pgfqpoint{2.539975in}{2.977157in}}%
\pgfpathlineto{\pgfqpoint{2.529546in}{2.995311in}}%
\pgfpathlineto{\pgfqpoint{2.519118in}{2.951655in}}%
\pgfpathlineto{\pgfqpoint{2.508690in}{2.978914in}}%
\pgfpathlineto{\pgfqpoint{2.498262in}{2.973075in}}%
\pgfpathlineto{\pgfqpoint{2.487834in}{2.984984in}}%
\pgfpathlineto{\pgfqpoint{2.477406in}{2.992393in}}%
\pgfpathlineto{\pgfqpoint{2.466977in}{3.000078in}}%
\pgfpathlineto{\pgfqpoint{2.456549in}{3.000500in}}%
\pgfpathlineto{\pgfqpoint{2.446121in}{2.994343in}}%
\pgfpathlineto{\pgfqpoint{2.435693in}{2.977658in}}%
\pgfpathlineto{\pgfqpoint{2.425265in}{2.982421in}}%
\pgfpathlineto{\pgfqpoint{2.414837in}{2.975308in}}%
\pgfpathlineto{\pgfqpoint{2.404408in}{2.945638in}}%
\pgfpathlineto{\pgfqpoint{2.393980in}{2.955672in}}%
\pgfpathlineto{\pgfqpoint{2.383552in}{2.959149in}}%
\pgfpathlineto{\pgfqpoint{2.373124in}{2.955774in}}%
\pgfpathlineto{\pgfqpoint{2.362696in}{2.945974in}}%
\pgfpathlineto{\pgfqpoint{2.352267in}{2.956814in}}%
\pgfpathlineto{\pgfqpoint{2.341839in}{2.955385in}}%
\pgfpathlineto{\pgfqpoint{2.331411in}{2.936176in}}%
\pgfpathlineto{\pgfqpoint{2.320983in}{2.939804in}}%
\pgfpathlineto{\pgfqpoint{2.310555in}{2.940425in}}%
\pgfpathlineto{\pgfqpoint{2.300127in}{2.963944in}}%
\pgfpathlineto{\pgfqpoint{2.289698in}{2.948848in}}%
\pgfpathlineto{\pgfqpoint{2.279270in}{2.948935in}}%
\pgfpathlineto{\pgfqpoint{2.268842in}{2.982374in}}%
\pgfpathlineto{\pgfqpoint{2.258414in}{2.972131in}}%
\pgfpathlineto{\pgfqpoint{2.247986in}{2.990507in}}%
\pgfpathlineto{\pgfqpoint{2.237558in}{2.982014in}}%
\pgfpathlineto{\pgfqpoint{2.227129in}{2.984972in}}%
\pgfpathlineto{\pgfqpoint{2.216701in}{2.972386in}}%
\pgfpathlineto{\pgfqpoint{2.206273in}{2.977416in}}%
\pgfpathlineto{\pgfqpoint{2.195845in}{2.967455in}}%
\pgfpathlineto{\pgfqpoint{2.185417in}{3.003103in}}%
\pgfpathlineto{\pgfqpoint{2.174989in}{2.996282in}}%
\pgfpathlineto{\pgfqpoint{2.164560in}{3.010419in}}%
\pgfpathlineto{\pgfqpoint{2.154132in}{2.989982in}}%
\pgfpathlineto{\pgfqpoint{2.143704in}{2.991647in}}%
\pgfpathlineto{\pgfqpoint{2.133276in}{3.015491in}}%
\pgfpathlineto{\pgfqpoint{2.122848in}{3.008337in}}%
\pgfpathlineto{\pgfqpoint{2.112419in}{3.053027in}}%
\pgfpathlineto{\pgfqpoint{2.101991in}{3.041191in}}%
\pgfpathlineto{\pgfqpoint{2.091563in}{3.021570in}}%
\pgfpathlineto{\pgfqpoint{2.081135in}{3.018210in}}%
\pgfpathlineto{\pgfqpoint{2.070707in}{3.015934in}}%
\pgfpathlineto{\pgfqpoint{2.060279in}{3.014186in}}%
\pgfpathlineto{\pgfqpoint{2.049850in}{3.000186in}}%
\pgfpathlineto{\pgfqpoint{2.039422in}{2.973772in}}%
\pgfpathlineto{\pgfqpoint{2.028994in}{2.962170in}}%
\pgfpathlineto{\pgfqpoint{2.018566in}{2.934137in}}%
\pgfpathlineto{\pgfqpoint{2.008138in}{2.980325in}}%
\pgfpathlineto{\pgfqpoint{1.997710in}{2.963748in}}%
\pgfpathlineto{\pgfqpoint{1.987281in}{2.968442in}}%
\pgfpathlineto{\pgfqpoint{1.976853in}{2.968506in}}%
\pgfpathlineto{\pgfqpoint{1.966425in}{2.968819in}}%
\pgfpathlineto{\pgfqpoint{1.955997in}{2.967463in}}%
\pgfpathlineto{\pgfqpoint{1.945569in}{2.984687in}}%
\pgfpathlineto{\pgfqpoint{1.935141in}{2.963727in}}%
\pgfpathlineto{\pgfqpoint{1.924712in}{2.979478in}}%
\pgfpathlineto{\pgfqpoint{1.914284in}{3.011918in}}%
\pgfpathlineto{\pgfqpoint{1.903856in}{2.994877in}}%
\pgfpathlineto{\pgfqpoint{1.893428in}{2.986575in}}%
\pgfpathlineto{\pgfqpoint{1.883000in}{2.992361in}}%
\pgfpathlineto{\pgfqpoint{1.872571in}{3.012486in}}%
\pgfpathlineto{\pgfqpoint{1.862143in}{3.015522in}}%
\pgfpathlineto{\pgfqpoint{1.851715in}{3.010469in}}%
\pgfpathlineto{\pgfqpoint{1.841287in}{3.031887in}}%
\pgfpathlineto{\pgfqpoint{1.830859in}{3.000051in}}%
\pgfpathlineto{\pgfqpoint{1.820431in}{3.008988in}}%
\pgfpathlineto{\pgfqpoint{1.810002in}{3.018658in}}%
\pgfpathlineto{\pgfqpoint{1.799574in}{3.013278in}}%
\pgfpathlineto{\pgfqpoint{1.789146in}{2.980849in}}%
\pgfpathlineto{\pgfqpoint{1.778718in}{3.023969in}}%
\pgfpathlineto{\pgfqpoint{1.768290in}{3.011032in}}%
\pgfpathlineto{\pgfqpoint{1.757862in}{3.009123in}}%
\pgfpathlineto{\pgfqpoint{1.747433in}{3.002528in}}%
\pgfpathlineto{\pgfqpoint{1.737005in}{3.024220in}}%
\pgfpathlineto{\pgfqpoint{1.726577in}{3.023660in}}%
\pgfpathlineto{\pgfqpoint{1.716149in}{2.998093in}}%
\pgfpathlineto{\pgfqpoint{1.705721in}{2.965673in}}%
\pgfpathlineto{\pgfqpoint{1.695292in}{2.968907in}}%
\pgfpathlineto{\pgfqpoint{1.684864in}{2.970763in}}%
\pgfpathlineto{\pgfqpoint{1.674436in}{2.993903in}}%
\pgfpathlineto{\pgfqpoint{1.664008in}{2.989802in}}%
\pgfpathlineto{\pgfqpoint{1.653580in}{2.983383in}}%
\pgfpathlineto{\pgfqpoint{1.643152in}{2.964374in}}%
\pgfpathlineto{\pgfqpoint{1.632723in}{2.997713in}}%
\pgfpathlineto{\pgfqpoint{1.622295in}{3.007752in}}%
\pgfpathlineto{\pgfqpoint{1.611867in}{3.020858in}}%
\pgfpathlineto{\pgfqpoint{1.601439in}{3.006765in}}%
\pgfpathlineto{\pgfqpoint{1.591011in}{2.994121in}}%
\pgfpathlineto{\pgfqpoint{1.580583in}{3.016271in}}%
\pgfpathlineto{\pgfqpoint{1.570154in}{3.002709in}}%
\pgfpathlineto{\pgfqpoint{1.559726in}{3.013563in}}%
\pgfpathlineto{\pgfqpoint{1.549298in}{3.034444in}}%
\pgfpathlineto{\pgfqpoint{1.538870in}{3.045887in}}%
\pgfpathlineto{\pgfqpoint{1.528442in}{3.050292in}}%
\pgfpathlineto{\pgfqpoint{1.518014in}{3.057659in}}%
\pgfpathlineto{\pgfqpoint{1.507585in}{3.044054in}}%
\pgfpathlineto{\pgfqpoint{1.497157in}{3.038555in}}%
\pgfpathlineto{\pgfqpoint{1.486729in}{3.046514in}}%
\pgfpathlineto{\pgfqpoint{1.476301in}{3.057199in}}%
\pgfpathlineto{\pgfqpoint{1.465873in}{3.034739in}}%
\pgfpathlineto{\pgfqpoint{1.455444in}{3.020416in}}%
\pgfpathlineto{\pgfqpoint{1.445016in}{3.045783in}}%
\pgfpathlineto{\pgfqpoint{1.434588in}{3.041595in}}%
\pgfpathlineto{\pgfqpoint{1.424160in}{3.044426in}}%
\pgfpathlineto{\pgfqpoint{1.413732in}{3.025405in}}%
\pgfpathlineto{\pgfqpoint{1.403304in}{3.010734in}}%
\pgfpathlineto{\pgfqpoint{1.392875in}{3.027042in}}%
\pgfpathlineto{\pgfqpoint{1.382447in}{3.047099in}}%
\pgfpathlineto{\pgfqpoint{1.372019in}{3.034091in}}%
\pgfpathlineto{\pgfqpoint{1.361591in}{3.028814in}}%
\pgfpathlineto{\pgfqpoint{1.351163in}{3.019157in}}%
\pgfpathlineto{\pgfqpoint{1.340735in}{3.027276in}}%
\pgfpathlineto{\pgfqpoint{1.330306in}{3.016028in}}%
\pgfpathlineto{\pgfqpoint{1.319878in}{3.011777in}}%
\pgfpathlineto{\pgfqpoint{1.309450in}{3.012847in}}%
\pgfpathlineto{\pgfqpoint{1.299022in}{2.992724in}}%
\pgfpathlineto{\pgfqpoint{1.288594in}{2.998490in}}%
\pgfpathlineto{\pgfqpoint{1.278166in}{2.976588in}}%
\pgfpathlineto{\pgfqpoint{1.267737in}{2.989747in}}%
\pgfpathlineto{\pgfqpoint{1.257309in}{2.981542in}}%
\pgfpathlineto{\pgfqpoint{1.246881in}{2.975141in}}%
\pgfpathlineto{\pgfqpoint{1.236453in}{2.990557in}}%
\pgfpathlineto{\pgfqpoint{1.226025in}{2.973089in}}%
\pgfpathlineto{\pgfqpoint{1.215596in}{3.003647in}}%
\pgfpathlineto{\pgfqpoint{1.205168in}{2.999272in}}%
\pgfpathlineto{\pgfqpoint{1.194740in}{3.003635in}}%
\pgfpathlineto{\pgfqpoint{1.184312in}{3.017667in}}%
\pgfpathlineto{\pgfqpoint{1.173884in}{3.022326in}}%
\pgfpathlineto{\pgfqpoint{1.163456in}{3.018063in}}%
\pgfpathlineto{\pgfqpoint{1.153027in}{3.034078in}}%
\pgfpathlineto{\pgfqpoint{1.142599in}{3.032417in}}%
\pgfpathlineto{\pgfqpoint{1.132171in}{3.015705in}}%
\pgfpathlineto{\pgfqpoint{1.121743in}{3.012119in}}%
\pgfpathlineto{\pgfqpoint{1.111315in}{3.029020in}}%
\pgfpathlineto{\pgfqpoint{1.100887in}{3.030955in}}%
\pgfpathlineto{\pgfqpoint{1.090458in}{3.036250in}}%
\pgfpathlineto{\pgfqpoint{1.080030in}{3.037575in}}%
\pgfpathlineto{\pgfqpoint{1.069602in}{3.030193in}}%
\pgfpathlineto{\pgfqpoint{1.059174in}{3.029295in}}%
\pgfpathlineto{\pgfqpoint{1.048746in}{2.998985in}}%
\pgfpathlineto{\pgfqpoint{1.038318in}{3.038103in}}%
\pgfpathlineto{\pgfqpoint{1.027889in}{3.030493in}}%
\pgfpathlineto{\pgfqpoint{1.017461in}{3.036684in}}%
\pgfpathlineto{\pgfqpoint{1.007033in}{3.029925in}}%
\pgfpathlineto{\pgfqpoint{0.996605in}{3.016032in}}%
\pgfpathlineto{\pgfqpoint{0.986177in}{3.013313in}}%
\pgfpathlineto{\pgfqpoint{0.975748in}{3.014814in}}%
\pgfpathlineto{\pgfqpoint{0.965320in}{3.021627in}}%
\pgfpathlineto{\pgfqpoint{0.954892in}{3.014959in}}%
\pgfpathlineto{\pgfqpoint{0.944464in}{3.017325in}}%
\pgfpathlineto{\pgfqpoint{0.934036in}{3.024756in}}%
\pgfpathlineto{\pgfqpoint{0.923608in}{3.010479in}}%
\pgfpathlineto{\pgfqpoint{0.913179in}{3.010624in}}%
\pgfpathlineto{\pgfqpoint{0.902751in}{3.015687in}}%
\pgfpathlineto{\pgfqpoint{0.892323in}{2.994652in}}%
\pgfpathlineto{\pgfqpoint{0.881895in}{3.022551in}}%
\pgfpathlineto{\pgfqpoint{0.871467in}{2.998905in}}%
\pgfpathlineto{\pgfqpoint{0.861039in}{2.988854in}}%
\pgfpathlineto{\pgfqpoint{0.850610in}{3.040102in}}%
\pgfpathlineto{\pgfqpoint{0.840182in}{3.015369in}}%
\pgfpathlineto{\pgfqpoint{0.829754in}{3.015657in}}%
\pgfpathlineto{\pgfqpoint{0.819326in}{2.975717in}}%
\pgfpathlineto{\pgfqpoint{0.808898in}{2.946310in}}%
\pgfpathlineto{\pgfqpoint{0.798470in}{2.942089in}}%
\pgfpathlineto{\pgfqpoint{0.788041in}{2.945931in}}%
\pgfpathlineto{\pgfqpoint{0.777613in}{2.932608in}}%
\pgfpathlineto{\pgfqpoint{0.767185in}{2.979231in}}%
\pgfpathlineto{\pgfqpoint{0.756757in}{2.958899in}}%
\pgfpathlineto{\pgfqpoint{0.746329in}{2.952290in}}%
\pgfpathlineto{\pgfqpoint{0.735900in}{2.984977in}}%
\pgfpathlineto{\pgfqpoint{0.725472in}{2.979268in}}%
\pgfpathlineto{\pgfqpoint{0.715044in}{2.985075in}}%
\pgfpathlineto{\pgfqpoint{0.704616in}{2.971514in}}%
\pgfpathlineto{\pgfqpoint{0.694188in}{3.015450in}}%
\pgfpathlineto{\pgfqpoint{0.683760in}{2.981188in}}%
\pgfpathlineto{\pgfqpoint{0.673331in}{2.952267in}}%
\pgfpathlineto{\pgfqpoint{0.662903in}{2.987454in}}%
\pgfpathlineto{\pgfqpoint{0.652475in}{2.969836in}}%
\pgfpathlineto{\pgfqpoint{0.642047in}{2.959750in}}%
\pgfpathlineto{\pgfqpoint{0.631619in}{3.009179in}}%
\pgfpathlineto{\pgfqpoint{0.621191in}{3.001603in}}%
\pgfpathlineto{\pgfqpoint{0.610762in}{2.977691in}}%
\pgfpathclose%
\pgfusepath{stroke,fill}%
\end{pgfscope}%
\begin{pgfscope}%
\pgfpathrectangle{\pgfqpoint{0.610762in}{0.961156in}}{\pgfqpoint{4.171270in}{2.577986in}} %
\pgfusepath{clip}%
\pgfsetbuttcap%
\pgfsetroundjoin%
\definecolor{currentfill}{rgb}{1.000000,0.694118,0.250980}%
\pgfsetfillcolor{currentfill}%
\pgfsetfillopacity{0.200000}%
\pgfsetlinewidth{0.301125pt}%
\definecolor{currentstroke}{rgb}{0.000000,0.000000,0.000000}%
\pgfsetstrokecolor{currentstroke}%
\pgfsetstrokeopacity{0.200000}%
\pgfsetdash{}{0pt}%
\pgfpathmoveto{\pgfqpoint{0.610762in}{3.007082in}}%
\pgfpathlineto{\pgfqpoint{0.610762in}{3.194902in}}%
\pgfpathlineto{\pgfqpoint{0.621191in}{3.169136in}}%
\pgfpathlineto{\pgfqpoint{0.631619in}{3.180029in}}%
\pgfpathlineto{\pgfqpoint{0.642047in}{3.232726in}}%
\pgfpathlineto{\pgfqpoint{0.652475in}{3.133693in}}%
\pgfpathlineto{\pgfqpoint{0.662903in}{3.161470in}}%
\pgfpathlineto{\pgfqpoint{0.673331in}{3.202214in}}%
\pgfpathlineto{\pgfqpoint{0.683760in}{3.266830in}}%
\pgfpathlineto{\pgfqpoint{0.694188in}{3.259220in}}%
\pgfpathlineto{\pgfqpoint{0.704616in}{3.172675in}}%
\pgfpathlineto{\pgfqpoint{0.715044in}{3.172605in}}%
\pgfpathlineto{\pgfqpoint{0.725472in}{3.185941in}}%
\pgfpathlineto{\pgfqpoint{0.735900in}{3.181828in}}%
\pgfpathlineto{\pgfqpoint{0.746329in}{3.244390in}}%
\pgfpathlineto{\pgfqpoint{0.756757in}{3.302095in}}%
\pgfpathlineto{\pgfqpoint{0.767185in}{3.358130in}}%
\pgfpathlineto{\pgfqpoint{0.777613in}{3.331220in}}%
\pgfpathlineto{\pgfqpoint{0.788041in}{3.372685in}}%
\pgfpathlineto{\pgfqpoint{0.798470in}{3.393169in}}%
\pgfpathlineto{\pgfqpoint{0.808898in}{3.429297in}}%
\pgfpathlineto{\pgfqpoint{0.819326in}{3.412329in}}%
\pgfpathlineto{\pgfqpoint{0.829754in}{3.374684in}}%
\pgfpathlineto{\pgfqpoint{0.840182in}{3.378784in}}%
\pgfpathlineto{\pgfqpoint{0.850610in}{3.289513in}}%
\pgfpathlineto{\pgfqpoint{0.861039in}{3.336130in}}%
\pgfpathlineto{\pgfqpoint{0.871467in}{3.333155in}}%
\pgfpathlineto{\pgfqpoint{0.881895in}{3.292806in}}%
\pgfpathlineto{\pgfqpoint{0.892323in}{3.284124in}}%
\pgfpathlineto{\pgfqpoint{0.902751in}{3.307427in}}%
\pgfpathlineto{\pgfqpoint{0.913179in}{3.263025in}}%
\pgfpathlineto{\pgfqpoint{0.923608in}{3.242475in}}%
\pgfpathlineto{\pgfqpoint{0.934036in}{3.288512in}}%
\pgfpathlineto{\pgfqpoint{0.944464in}{3.239727in}}%
\pgfpathlineto{\pgfqpoint{0.954892in}{3.231916in}}%
\pgfpathlineto{\pgfqpoint{0.965320in}{3.215368in}}%
\pgfpathlineto{\pgfqpoint{0.975748in}{3.247909in}}%
\pgfpathlineto{\pgfqpoint{0.986177in}{3.204734in}}%
\pgfpathlineto{\pgfqpoint{0.996605in}{3.236700in}}%
\pgfpathlineto{\pgfqpoint{1.007033in}{3.182917in}}%
\pgfpathlineto{\pgfqpoint{1.017461in}{3.169760in}}%
\pgfpathlineto{\pgfqpoint{1.027889in}{3.113538in}}%
\pgfpathlineto{\pgfqpoint{1.038318in}{3.138797in}}%
\pgfpathlineto{\pgfqpoint{1.048746in}{3.120052in}}%
\pgfpathlineto{\pgfqpoint{1.059174in}{3.150976in}}%
\pgfpathlineto{\pgfqpoint{1.069602in}{3.185650in}}%
\pgfpathlineto{\pgfqpoint{1.080030in}{3.171614in}}%
\pgfpathlineto{\pgfqpoint{1.090458in}{3.246075in}}%
\pgfpathlineto{\pgfqpoint{1.100887in}{3.139514in}}%
\pgfpathlineto{\pgfqpoint{1.111315in}{3.137127in}}%
\pgfpathlineto{\pgfqpoint{1.121743in}{3.137589in}}%
\pgfpathlineto{\pgfqpoint{1.132171in}{3.114234in}}%
\pgfpathlineto{\pgfqpoint{1.142599in}{3.129639in}}%
\pgfpathlineto{\pgfqpoint{1.153027in}{3.176607in}}%
\pgfpathlineto{\pgfqpoint{1.163456in}{3.146138in}}%
\pgfpathlineto{\pgfqpoint{1.173884in}{3.108054in}}%
\pgfpathlineto{\pgfqpoint{1.184312in}{3.074163in}}%
\pgfpathlineto{\pgfqpoint{1.194740in}{3.083291in}}%
\pgfpathlineto{\pgfqpoint{1.205168in}{3.061253in}}%
\pgfpathlineto{\pgfqpoint{1.215596in}{3.095582in}}%
\pgfpathlineto{\pgfqpoint{1.226025in}{3.061339in}}%
\pgfpathlineto{\pgfqpoint{1.236453in}{3.080553in}}%
\pgfpathlineto{\pgfqpoint{1.246881in}{3.051085in}}%
\pgfpathlineto{\pgfqpoint{1.257309in}{3.038351in}}%
\pgfpathlineto{\pgfqpoint{1.267737in}{3.085222in}}%
\pgfpathlineto{\pgfqpoint{1.278166in}{3.121173in}}%
\pgfpathlineto{\pgfqpoint{1.288594in}{3.094600in}}%
\pgfpathlineto{\pgfqpoint{1.299022in}{3.099291in}}%
\pgfpathlineto{\pgfqpoint{1.309450in}{3.158051in}}%
\pgfpathlineto{\pgfqpoint{1.319878in}{3.200313in}}%
\pgfpathlineto{\pgfqpoint{1.330306in}{3.271247in}}%
\pgfpathlineto{\pgfqpoint{1.340735in}{3.156434in}}%
\pgfpathlineto{\pgfqpoint{1.351163in}{3.157850in}}%
\pgfpathlineto{\pgfqpoint{1.361591in}{3.121936in}}%
\pgfpathlineto{\pgfqpoint{1.372019in}{3.144090in}}%
\pgfpathlineto{\pgfqpoint{1.382447in}{3.067062in}}%
\pgfpathlineto{\pgfqpoint{1.392875in}{3.066587in}}%
\pgfpathlineto{\pgfqpoint{1.403304in}{3.079523in}}%
\pgfpathlineto{\pgfqpoint{1.413732in}{3.024636in}}%
\pgfpathlineto{\pgfqpoint{1.424160in}{2.994752in}}%
\pgfpathlineto{\pgfqpoint{1.434588in}{3.078260in}}%
\pgfpathlineto{\pgfqpoint{1.445016in}{2.949785in}}%
\pgfpathlineto{\pgfqpoint{1.455444in}{3.052588in}}%
\pgfpathlineto{\pgfqpoint{1.465873in}{3.098629in}}%
\pgfpathlineto{\pgfqpoint{1.476301in}{3.168422in}}%
\pgfpathlineto{\pgfqpoint{1.486729in}{3.203142in}}%
\pgfpathlineto{\pgfqpoint{1.497157in}{3.179422in}}%
\pgfpathlineto{\pgfqpoint{1.507585in}{3.168093in}}%
\pgfpathlineto{\pgfqpoint{1.518014in}{3.165660in}}%
\pgfpathlineto{\pgfqpoint{1.528442in}{3.192519in}}%
\pgfpathlineto{\pgfqpoint{1.538870in}{3.113530in}}%
\pgfpathlineto{\pgfqpoint{1.549298in}{3.156624in}}%
\pgfpathlineto{\pgfqpoint{1.559726in}{3.166961in}}%
\pgfpathlineto{\pgfqpoint{1.570154in}{3.192653in}}%
\pgfpathlineto{\pgfqpoint{1.580583in}{3.125820in}}%
\pgfpathlineto{\pgfqpoint{1.591011in}{3.143768in}}%
\pgfpathlineto{\pgfqpoint{1.601439in}{3.136631in}}%
\pgfpathlineto{\pgfqpoint{1.611867in}{3.134609in}}%
\pgfpathlineto{\pgfqpoint{1.622295in}{3.178799in}}%
\pgfpathlineto{\pgfqpoint{1.632723in}{3.291231in}}%
\pgfpathlineto{\pgfqpoint{1.643152in}{3.251973in}}%
\pgfpathlineto{\pgfqpoint{1.653580in}{3.235409in}}%
\pgfpathlineto{\pgfqpoint{1.664008in}{3.262300in}}%
\pgfpathlineto{\pgfqpoint{1.674436in}{3.150261in}}%
\pgfpathlineto{\pgfqpoint{1.684864in}{3.190117in}}%
\pgfpathlineto{\pgfqpoint{1.695292in}{3.137037in}}%
\pgfpathlineto{\pgfqpoint{1.705721in}{3.166820in}}%
\pgfpathlineto{\pgfqpoint{1.716149in}{3.186440in}}%
\pgfpathlineto{\pgfqpoint{1.726577in}{3.140399in}}%
\pgfpathlineto{\pgfqpoint{1.737005in}{3.116014in}}%
\pgfpathlineto{\pgfqpoint{1.747433in}{3.145797in}}%
\pgfpathlineto{\pgfqpoint{1.757862in}{3.130617in}}%
\pgfpathlineto{\pgfqpoint{1.768290in}{3.165907in}}%
\pgfpathlineto{\pgfqpoint{1.778718in}{3.148920in}}%
\pgfpathlineto{\pgfqpoint{1.789146in}{3.094279in}}%
\pgfpathlineto{\pgfqpoint{1.799574in}{3.109385in}}%
\pgfpathlineto{\pgfqpoint{1.810002in}{3.121121in}}%
\pgfpathlineto{\pgfqpoint{1.820431in}{3.164364in}}%
\pgfpathlineto{\pgfqpoint{1.830859in}{3.241314in}}%
\pgfpathlineto{\pgfqpoint{1.841287in}{3.272228in}}%
\pgfpathlineto{\pgfqpoint{1.851715in}{3.246218in}}%
\pgfpathlineto{\pgfqpoint{1.862143in}{3.246165in}}%
\pgfpathlineto{\pgfqpoint{1.872571in}{3.247874in}}%
\pgfpathlineto{\pgfqpoint{1.883000in}{3.273566in}}%
\pgfpathlineto{\pgfqpoint{1.893428in}{3.209579in}}%
\pgfpathlineto{\pgfqpoint{1.903856in}{3.224736in}}%
\pgfpathlineto{\pgfqpoint{1.914284in}{3.261448in}}%
\pgfpathlineto{\pgfqpoint{1.924712in}{3.249861in}}%
\pgfpathlineto{\pgfqpoint{1.935141in}{3.237659in}}%
\pgfpathlineto{\pgfqpoint{1.945569in}{3.242779in}}%
\pgfpathlineto{\pgfqpoint{1.955997in}{3.156210in}}%
\pgfpathlineto{\pgfqpoint{1.966425in}{3.097822in}}%
\pgfpathlineto{\pgfqpoint{1.976853in}{3.059061in}}%
\pgfpathlineto{\pgfqpoint{1.987281in}{3.098793in}}%
\pgfpathlineto{\pgfqpoint{1.997710in}{3.088368in}}%
\pgfpathlineto{\pgfqpoint{2.008138in}{3.091113in}}%
\pgfpathlineto{\pgfqpoint{2.018566in}{3.105819in}}%
\pgfpathlineto{\pgfqpoint{2.028994in}{3.065865in}}%
\pgfpathlineto{\pgfqpoint{2.039422in}{3.122266in}}%
\pgfpathlineto{\pgfqpoint{2.049850in}{3.093606in}}%
\pgfpathlineto{\pgfqpoint{2.060279in}{3.087430in}}%
\pgfpathlineto{\pgfqpoint{2.070707in}{3.101727in}}%
\pgfpathlineto{\pgfqpoint{2.081135in}{3.141246in}}%
\pgfpathlineto{\pgfqpoint{2.091563in}{3.100973in}}%
\pgfpathlineto{\pgfqpoint{2.101991in}{3.139349in}}%
\pgfpathlineto{\pgfqpoint{2.112419in}{3.105566in}}%
\pgfpathlineto{\pgfqpoint{2.122848in}{3.127596in}}%
\pgfpathlineto{\pgfqpoint{2.133276in}{3.075999in}}%
\pgfpathlineto{\pgfqpoint{2.143704in}{3.110178in}}%
\pgfpathlineto{\pgfqpoint{2.154132in}{3.099872in}}%
\pgfpathlineto{\pgfqpoint{2.164560in}{3.043387in}}%
\pgfpathlineto{\pgfqpoint{2.174989in}{3.141316in}}%
\pgfpathlineto{\pgfqpoint{2.185417in}{3.072424in}}%
\pgfpathlineto{\pgfqpoint{2.195845in}{3.023168in}}%
\pgfpathlineto{\pgfqpoint{2.206273in}{3.058097in}}%
\pgfpathlineto{\pgfqpoint{2.216701in}{3.076830in}}%
\pgfpathlineto{\pgfqpoint{2.227129in}{3.042051in}}%
\pgfpathlineto{\pgfqpoint{2.237558in}{3.070205in}}%
\pgfpathlineto{\pgfqpoint{2.247986in}{3.054233in}}%
\pgfpathlineto{\pgfqpoint{2.258414in}{3.078591in}}%
\pgfpathlineto{\pgfqpoint{2.268842in}{3.074949in}}%
\pgfpathlineto{\pgfqpoint{2.279270in}{3.079109in}}%
\pgfpathlineto{\pgfqpoint{2.289698in}{3.057266in}}%
\pgfpathlineto{\pgfqpoint{2.300127in}{3.075622in}}%
\pgfpathlineto{\pgfqpoint{2.310555in}{3.032533in}}%
\pgfpathlineto{\pgfqpoint{2.320983in}{3.059650in}}%
\pgfpathlineto{\pgfqpoint{2.331411in}{3.016543in}}%
\pgfpathlineto{\pgfqpoint{2.341839in}{3.026377in}}%
\pgfpathlineto{\pgfqpoint{2.352267in}{2.929657in}}%
\pgfpathlineto{\pgfqpoint{2.362696in}{2.995998in}}%
\pgfpathlineto{\pgfqpoint{2.373124in}{3.014683in}}%
\pgfpathlineto{\pgfqpoint{2.383552in}{3.031720in}}%
\pgfpathlineto{\pgfqpoint{2.393980in}{3.012851in}}%
\pgfpathlineto{\pgfqpoint{2.404408in}{2.967635in}}%
\pgfpathlineto{\pgfqpoint{2.414837in}{2.946788in}}%
\pgfpathlineto{\pgfqpoint{2.425265in}{2.950491in}}%
\pgfpathlineto{\pgfqpoint{2.435693in}{2.972179in}}%
\pgfpathlineto{\pgfqpoint{2.446121in}{2.936206in}}%
\pgfpathlineto{\pgfqpoint{2.456549in}{2.908063in}}%
\pgfpathlineto{\pgfqpoint{2.466977in}{2.961094in}}%
\pgfpathlineto{\pgfqpoint{2.477406in}{2.940958in}}%
\pgfpathlineto{\pgfqpoint{2.487834in}{2.899636in}}%
\pgfpathlineto{\pgfqpoint{2.498262in}{2.881922in}}%
\pgfpathlineto{\pgfqpoint{2.508690in}{2.892271in}}%
\pgfpathlineto{\pgfqpoint{2.519118in}{2.879383in}}%
\pgfpathlineto{\pgfqpoint{2.529546in}{2.861770in}}%
\pgfpathlineto{\pgfqpoint{2.539975in}{2.842341in}}%
\pgfpathlineto{\pgfqpoint{2.550403in}{2.878240in}}%
\pgfpathlineto{\pgfqpoint{2.560831in}{2.863682in}}%
\pgfpathlineto{\pgfqpoint{2.571259in}{2.849512in}}%
\pgfpathlineto{\pgfqpoint{2.581687in}{2.829706in}}%
\pgfpathlineto{\pgfqpoint{2.592115in}{2.840862in}}%
\pgfpathlineto{\pgfqpoint{2.602544in}{2.816573in}}%
\pgfpathlineto{\pgfqpoint{2.612972in}{2.777119in}}%
\pgfpathlineto{\pgfqpoint{2.623400in}{2.775181in}}%
\pgfpathlineto{\pgfqpoint{2.633828in}{2.742570in}}%
\pgfpathlineto{\pgfqpoint{2.644256in}{2.731334in}}%
\pgfpathlineto{\pgfqpoint{2.654685in}{2.752884in}}%
\pgfpathlineto{\pgfqpoint{2.665113in}{2.724738in}}%
\pgfpathlineto{\pgfqpoint{2.675541in}{2.631635in}}%
\pgfpathlineto{\pgfqpoint{2.685969in}{2.531920in}}%
\pgfpathlineto{\pgfqpoint{2.696397in}{2.534229in}}%
\pgfpathlineto{\pgfqpoint{2.706825in}{2.435359in}}%
\pgfpathlineto{\pgfqpoint{2.717254in}{2.364841in}}%
\pgfpathlineto{\pgfqpoint{2.727682in}{2.286116in}}%
\pgfpathlineto{\pgfqpoint{2.738110in}{2.289092in}}%
\pgfpathlineto{\pgfqpoint{2.748538in}{2.304046in}}%
\pgfpathlineto{\pgfqpoint{2.758966in}{2.262072in}}%
\pgfpathlineto{\pgfqpoint{2.769394in}{2.289077in}}%
\pgfpathlineto{\pgfqpoint{2.779823in}{2.226185in}}%
\pgfpathlineto{\pgfqpoint{2.790251in}{2.192619in}}%
\pgfpathlineto{\pgfqpoint{2.800679in}{2.201256in}}%
\pgfpathlineto{\pgfqpoint{2.811107in}{2.184074in}}%
\pgfpathlineto{\pgfqpoint{2.821535in}{2.217882in}}%
\pgfpathlineto{\pgfqpoint{2.831964in}{2.190714in}}%
\pgfpathlineto{\pgfqpoint{2.842392in}{2.145697in}}%
\pgfpathlineto{\pgfqpoint{2.852820in}{2.136438in}}%
\pgfpathlineto{\pgfqpoint{2.863248in}{2.115606in}}%
\pgfpathlineto{\pgfqpoint{2.873676in}{2.076830in}}%
\pgfpathlineto{\pgfqpoint{2.884104in}{2.067523in}}%
\pgfpathlineto{\pgfqpoint{2.894533in}{2.095501in}}%
\pgfpathlineto{\pgfqpoint{2.904961in}{2.123472in}}%
\pgfpathlineto{\pgfqpoint{2.915389in}{2.081023in}}%
\pgfpathlineto{\pgfqpoint{2.925817in}{2.061733in}}%
\pgfpathlineto{\pgfqpoint{2.936245in}{2.047575in}}%
\pgfpathlineto{\pgfqpoint{2.946673in}{2.045791in}}%
\pgfpathlineto{\pgfqpoint{2.957102in}{2.047077in}}%
\pgfpathlineto{\pgfqpoint{2.967530in}{2.024027in}}%
\pgfpathlineto{\pgfqpoint{2.977958in}{2.008027in}}%
\pgfpathlineto{\pgfqpoint{2.988386in}{1.995435in}}%
\pgfpathlineto{\pgfqpoint{2.998814in}{1.996197in}}%
\pgfpathlineto{\pgfqpoint{3.009242in}{2.028408in}}%
\pgfpathlineto{\pgfqpoint{3.019671in}{2.019295in}}%
\pgfpathlineto{\pgfqpoint{3.030099in}{2.029195in}}%
\pgfpathlineto{\pgfqpoint{3.040527in}{2.004959in}}%
\pgfpathlineto{\pgfqpoint{3.050955in}{1.994366in}}%
\pgfpathlineto{\pgfqpoint{3.061383in}{1.984676in}}%
\pgfpathlineto{\pgfqpoint{3.071812in}{1.994961in}}%
\pgfpathlineto{\pgfqpoint{3.082240in}{1.981945in}}%
\pgfpathlineto{\pgfqpoint{3.092668in}{1.963413in}}%
\pgfpathlineto{\pgfqpoint{3.103096in}{1.951358in}}%
\pgfpathlineto{\pgfqpoint{3.113524in}{1.948484in}}%
\pgfpathlineto{\pgfqpoint{3.123952in}{1.937070in}}%
\pgfpathlineto{\pgfqpoint{3.134381in}{1.947126in}}%
\pgfpathlineto{\pgfqpoint{3.144809in}{1.982336in}}%
\pgfpathlineto{\pgfqpoint{3.155237in}{1.990287in}}%
\pgfpathlineto{\pgfqpoint{3.165665in}{2.007062in}}%
\pgfpathlineto{\pgfqpoint{3.176093in}{1.957471in}}%
\pgfpathlineto{\pgfqpoint{3.186521in}{1.996111in}}%
\pgfpathlineto{\pgfqpoint{3.196950in}{1.952409in}}%
\pgfpathlineto{\pgfqpoint{3.207378in}{1.930505in}}%
\pgfpathlineto{\pgfqpoint{3.217806in}{1.904226in}}%
\pgfpathlineto{\pgfqpoint{3.228234in}{1.929283in}}%
\pgfpathlineto{\pgfqpoint{3.238662in}{1.974288in}}%
\pgfpathlineto{\pgfqpoint{3.249090in}{1.938611in}}%
\pgfpathlineto{\pgfqpoint{3.259519in}{1.914518in}}%
\pgfpathlineto{\pgfqpoint{3.269947in}{1.947558in}}%
\pgfpathlineto{\pgfqpoint{3.280375in}{1.979618in}}%
\pgfpathlineto{\pgfqpoint{3.290803in}{1.967898in}}%
\pgfpathlineto{\pgfqpoint{3.301231in}{1.896735in}}%
\pgfpathlineto{\pgfqpoint{3.311660in}{1.884680in}}%
\pgfpathlineto{\pgfqpoint{3.322088in}{1.921315in}}%
\pgfpathlineto{\pgfqpoint{3.332516in}{1.873923in}}%
\pgfpathlineto{\pgfqpoint{3.342944in}{1.888864in}}%
\pgfpathlineto{\pgfqpoint{3.353372in}{1.913492in}}%
\pgfpathlineto{\pgfqpoint{3.363800in}{1.922978in}}%
\pgfpathlineto{\pgfqpoint{3.374229in}{1.908229in}}%
\pgfpathlineto{\pgfqpoint{3.384657in}{1.932523in}}%
\pgfpathlineto{\pgfqpoint{3.395085in}{1.934307in}}%
\pgfpathlineto{\pgfqpoint{3.405513in}{1.902263in}}%
\pgfpathlineto{\pgfqpoint{3.415941in}{1.865318in}}%
\pgfpathlineto{\pgfqpoint{3.426369in}{1.863585in}}%
\pgfpathlineto{\pgfqpoint{3.436798in}{1.884851in}}%
\pgfpathlineto{\pgfqpoint{3.447226in}{1.884679in}}%
\pgfpathlineto{\pgfqpoint{3.457654in}{1.883727in}}%
\pgfpathlineto{\pgfqpoint{3.468082in}{1.916375in}}%
\pgfpathlineto{\pgfqpoint{3.478510in}{1.867664in}}%
\pgfpathlineto{\pgfqpoint{3.488938in}{1.848105in}}%
\pgfpathlineto{\pgfqpoint{3.499367in}{1.887705in}}%
\pgfpathlineto{\pgfqpoint{3.509795in}{1.822584in}}%
\pgfpathlineto{\pgfqpoint{3.520223in}{1.902340in}}%
\pgfpathlineto{\pgfqpoint{3.530651in}{1.887985in}}%
\pgfpathlineto{\pgfqpoint{3.541079in}{1.937810in}}%
\pgfpathlineto{\pgfqpoint{3.551508in}{1.888296in}}%
\pgfpathlineto{\pgfqpoint{3.561936in}{1.916438in}}%
\pgfpathlineto{\pgfqpoint{3.572364in}{1.877173in}}%
\pgfpathlineto{\pgfqpoint{3.582792in}{1.853840in}}%
\pgfpathlineto{\pgfqpoint{3.593220in}{1.898050in}}%
\pgfpathlineto{\pgfqpoint{3.603648in}{1.874677in}}%
\pgfpathlineto{\pgfqpoint{3.614077in}{1.822838in}}%
\pgfpathlineto{\pgfqpoint{3.624505in}{1.831841in}}%
\pgfpathlineto{\pgfqpoint{3.634933in}{1.942687in}}%
\pgfpathlineto{\pgfqpoint{3.645361in}{1.836198in}}%
\pgfpathlineto{\pgfqpoint{3.655789in}{1.865576in}}%
\pgfpathlineto{\pgfqpoint{3.666217in}{1.791251in}}%
\pgfpathlineto{\pgfqpoint{3.676646in}{1.808875in}}%
\pgfpathlineto{\pgfqpoint{3.687074in}{1.865719in}}%
\pgfpathlineto{\pgfqpoint{3.697502in}{1.795810in}}%
\pgfpathlineto{\pgfqpoint{3.707930in}{1.828586in}}%
\pgfpathlineto{\pgfqpoint{3.718358in}{1.823963in}}%
\pgfpathlineto{\pgfqpoint{3.728787in}{1.792665in}}%
\pgfpathlineto{\pgfqpoint{3.739215in}{1.739031in}}%
\pgfpathlineto{\pgfqpoint{3.749643in}{1.728931in}}%
\pgfpathlineto{\pgfqpoint{3.760071in}{1.756366in}}%
\pgfpathlineto{\pgfqpoint{3.770499in}{1.810687in}}%
\pgfpathlineto{\pgfqpoint{3.780927in}{1.753070in}}%
\pgfpathlineto{\pgfqpoint{3.791356in}{1.807219in}}%
\pgfpathlineto{\pgfqpoint{3.801784in}{1.749114in}}%
\pgfpathlineto{\pgfqpoint{3.812212in}{1.821991in}}%
\pgfpathlineto{\pgfqpoint{3.822640in}{1.783432in}}%
\pgfpathlineto{\pgfqpoint{3.833068in}{1.872144in}}%
\pgfpathlineto{\pgfqpoint{3.843496in}{1.946705in}}%
\pgfpathlineto{\pgfqpoint{3.853925in}{1.784202in}}%
\pgfpathlineto{\pgfqpoint{3.864353in}{1.833839in}}%
\pgfpathlineto{\pgfqpoint{3.874781in}{1.850786in}}%
\pgfpathlineto{\pgfqpoint{3.885209in}{1.751970in}}%
\pgfpathlineto{\pgfqpoint{3.895637in}{1.810725in}}%
\pgfpathlineto{\pgfqpoint{3.906065in}{1.799436in}}%
\pgfpathlineto{\pgfqpoint{3.916494in}{1.852065in}}%
\pgfpathlineto{\pgfqpoint{3.926922in}{1.855308in}}%
\pgfpathlineto{\pgfqpoint{3.937350in}{1.819798in}}%
\pgfpathlineto{\pgfqpoint{3.947778in}{1.775894in}}%
\pgfpathlineto{\pgfqpoint{3.958206in}{1.828013in}}%
\pgfpathlineto{\pgfqpoint{3.968635in}{1.821936in}}%
\pgfpathlineto{\pgfqpoint{3.979063in}{1.779377in}}%
\pgfpathlineto{\pgfqpoint{3.989491in}{1.779550in}}%
\pgfpathlineto{\pgfqpoint{3.999919in}{1.717109in}}%
\pgfpathlineto{\pgfqpoint{4.010347in}{1.641394in}}%
\pgfpathlineto{\pgfqpoint{4.020775in}{1.749351in}}%
\pgfpathlineto{\pgfqpoint{4.031204in}{1.703568in}}%
\pgfpathlineto{\pgfqpoint{4.041632in}{1.799899in}}%
\pgfpathlineto{\pgfqpoint{4.052060in}{1.737052in}}%
\pgfpathlineto{\pgfqpoint{4.062488in}{1.848593in}}%
\pgfpathlineto{\pgfqpoint{4.072916in}{1.726694in}}%
\pgfpathlineto{\pgfqpoint{4.083344in}{1.757099in}}%
\pgfpathlineto{\pgfqpoint{4.093773in}{1.845171in}}%
\pgfpathlineto{\pgfqpoint{4.104201in}{1.853996in}}%
\pgfpathlineto{\pgfqpoint{4.114629in}{1.791565in}}%
\pgfpathlineto{\pgfqpoint{4.125057in}{1.857153in}}%
\pgfpathlineto{\pgfqpoint{4.135485in}{1.896673in}}%
\pgfpathlineto{\pgfqpoint{4.145913in}{1.807226in}}%
\pgfpathlineto{\pgfqpoint{4.156342in}{1.690830in}}%
\pgfpathlineto{\pgfqpoint{4.166770in}{1.910394in}}%
\pgfpathlineto{\pgfqpoint{4.177198in}{1.779333in}}%
\pgfpathlineto{\pgfqpoint{4.187626in}{1.681000in}}%
\pgfpathlineto{\pgfqpoint{4.198054in}{1.912881in}}%
\pgfpathlineto{\pgfqpoint{4.208483in}{1.765276in}}%
\pgfpathlineto{\pgfqpoint{4.218911in}{1.755623in}}%
\pgfpathlineto{\pgfqpoint{4.229339in}{1.703651in}}%
\pgfpathlineto{\pgfqpoint{4.239767in}{1.706975in}}%
\pgfpathlineto{\pgfqpoint{4.250195in}{1.723927in}}%
\pgfpathlineto{\pgfqpoint{4.260623in}{1.648767in}}%
\pgfpathlineto{\pgfqpoint{4.271052in}{1.561749in}}%
\pgfpathlineto{\pgfqpoint{4.281480in}{1.863128in}}%
\pgfpathlineto{\pgfqpoint{4.291908in}{1.581897in}}%
\pgfpathlineto{\pgfqpoint{4.302336in}{1.598763in}}%
\pgfpathlineto{\pgfqpoint{4.312764in}{1.700515in}}%
\pgfpathlineto{\pgfqpoint{4.323192in}{1.713626in}}%
\pgfpathlineto{\pgfqpoint{4.333621in}{1.734493in}}%
\pgfpathlineto{\pgfqpoint{4.344049in}{1.634812in}}%
\pgfpathlineto{\pgfqpoint{4.354477in}{1.606892in}}%
\pgfpathlineto{\pgfqpoint{4.364905in}{1.852912in}}%
\pgfpathlineto{\pgfqpoint{4.375333in}{1.518485in}}%
\pgfpathlineto{\pgfqpoint{4.385761in}{1.715648in}}%
\pgfpathlineto{\pgfqpoint{4.396190in}{1.598304in}}%
\pgfpathlineto{\pgfqpoint{4.406618in}{1.721433in}}%
\pgfpathlineto{\pgfqpoint{4.417046in}{1.723675in}}%
\pgfpathlineto{\pgfqpoint{4.427474in}{1.621905in}}%
\pgfpathlineto{\pgfqpoint{4.437902in}{1.700388in}}%
\pgfpathlineto{\pgfqpoint{4.448331in}{1.893377in}}%
\pgfpathlineto{\pgfqpoint{4.458759in}{1.679769in}}%
\pgfpathlineto{\pgfqpoint{4.469187in}{1.556712in}}%
\pgfpathlineto{\pgfqpoint{4.479615in}{1.708115in}}%
\pgfpathlineto{\pgfqpoint{4.490043in}{1.713446in}}%
\pgfpathlineto{\pgfqpoint{4.500471in}{1.702857in}}%
\pgfpathlineto{\pgfqpoint{4.510900in}{1.649309in}}%
\pgfpathlineto{\pgfqpoint{4.521328in}{1.910957in}}%
\pgfpathlineto{\pgfqpoint{4.531756in}{1.983784in}}%
\pgfpathlineto{\pgfqpoint{4.542184in}{2.085336in}}%
\pgfpathlineto{\pgfqpoint{4.552612in}{2.006263in}}%
\pgfpathlineto{\pgfqpoint{4.563040in}{2.097892in}}%
\pgfpathlineto{\pgfqpoint{4.573469in}{2.136839in}}%
\pgfpathlineto{\pgfqpoint{4.583897in}{2.076023in}}%
\pgfpathlineto{\pgfqpoint{4.594325in}{1.994246in}}%
\pgfpathlineto{\pgfqpoint{4.604753in}{1.858004in}}%
\pgfpathlineto{\pgfqpoint{4.615181in}{1.870313in}}%
\pgfpathlineto{\pgfqpoint{4.625610in}{1.740663in}}%
\pgfpathlineto{\pgfqpoint{4.636038in}{1.603866in}}%
\pgfpathlineto{\pgfqpoint{4.646466in}{1.518260in}}%
\pgfpathlineto{\pgfqpoint{4.656894in}{1.678796in}}%
\pgfpathlineto{\pgfqpoint{4.667322in}{1.586002in}}%
\pgfpathlineto{\pgfqpoint{4.677750in}{1.679683in}}%
\pgfpathlineto{\pgfqpoint{4.688179in}{1.738079in}}%
\pgfpathlineto{\pgfqpoint{4.698607in}{1.588037in}}%
\pgfpathlineto{\pgfqpoint{4.709035in}{1.841711in}}%
\pgfpathlineto{\pgfqpoint{4.719463in}{1.630280in}}%
\pgfpathlineto{\pgfqpoint{4.729891in}{1.912972in}}%
\pgfpathlineto{\pgfqpoint{4.740319in}{1.472720in}}%
\pgfpathlineto{\pgfqpoint{4.750748in}{1.969299in}}%
\pgfpathlineto{\pgfqpoint{4.761176in}{1.855136in}}%
\pgfpathlineto{\pgfqpoint{4.771604in}{1.851279in}}%
\pgfpathlineto{\pgfqpoint{4.771604in}{1.436368in}}%
\pgfpathlineto{\pgfqpoint{4.771604in}{1.436368in}}%
\pgfpathlineto{\pgfqpoint{4.761176in}{1.447527in}}%
\pgfpathlineto{\pgfqpoint{4.750748in}{1.525506in}}%
\pgfpathlineto{\pgfqpoint{4.740319in}{1.055488in}}%
\pgfpathlineto{\pgfqpoint{4.729891in}{1.451311in}}%
\pgfpathlineto{\pgfqpoint{4.719463in}{1.253596in}}%
\pgfpathlineto{\pgfqpoint{4.709035in}{1.480073in}}%
\pgfpathlineto{\pgfqpoint{4.698607in}{1.281131in}}%
\pgfpathlineto{\pgfqpoint{4.688179in}{1.356317in}}%
\pgfpathlineto{\pgfqpoint{4.677750in}{1.317443in}}%
\pgfpathlineto{\pgfqpoint{4.667322in}{1.212470in}}%
\pgfpathlineto{\pgfqpoint{4.656894in}{1.283917in}}%
\pgfpathlineto{\pgfqpoint{4.646466in}{1.139836in}}%
\pgfpathlineto{\pgfqpoint{4.636038in}{1.164053in}}%
\pgfpathlineto{\pgfqpoint{4.625610in}{1.312149in}}%
\pgfpathlineto{\pgfqpoint{4.615181in}{1.307127in}}%
\pgfpathlineto{\pgfqpoint{4.604753in}{1.312315in}}%
\pgfpathlineto{\pgfqpoint{4.594325in}{1.500882in}}%
\pgfpathlineto{\pgfqpoint{4.583897in}{1.607783in}}%
\pgfpathlineto{\pgfqpoint{4.573469in}{1.596775in}}%
\pgfpathlineto{\pgfqpoint{4.563040in}{1.658182in}}%
\pgfpathlineto{\pgfqpoint{4.552612in}{1.612711in}}%
\pgfpathlineto{\pgfqpoint{4.542184in}{1.607430in}}%
\pgfpathlineto{\pgfqpoint{4.531756in}{1.659399in}}%
\pgfpathlineto{\pgfqpoint{4.521328in}{1.555638in}}%
\pgfpathlineto{\pgfqpoint{4.510900in}{1.298514in}}%
\pgfpathlineto{\pgfqpoint{4.500471in}{1.414000in}}%
\pgfpathlineto{\pgfqpoint{4.490043in}{1.403680in}}%
\pgfpathlineto{\pgfqpoint{4.479615in}{1.389370in}}%
\pgfpathlineto{\pgfqpoint{4.469187in}{1.248972in}}%
\pgfpathlineto{\pgfqpoint{4.458759in}{1.366428in}}%
\pgfpathlineto{\pgfqpoint{4.448331in}{1.507668in}}%
\pgfpathlineto{\pgfqpoint{4.437902in}{1.399947in}}%
\pgfpathlineto{\pgfqpoint{4.427474in}{1.235255in}}%
\pgfpathlineto{\pgfqpoint{4.417046in}{1.445851in}}%
\pgfpathlineto{\pgfqpoint{4.406618in}{1.426060in}}%
\pgfpathlineto{\pgfqpoint{4.396190in}{1.254754in}}%
\pgfpathlineto{\pgfqpoint{4.385761in}{1.399554in}}%
\pgfpathlineto{\pgfqpoint{4.375333in}{1.153917in}}%
\pgfpathlineto{\pgfqpoint{4.364905in}{1.538279in}}%
\pgfpathlineto{\pgfqpoint{4.354477in}{1.246791in}}%
\pgfpathlineto{\pgfqpoint{4.344049in}{1.329040in}}%
\pgfpathlineto{\pgfqpoint{4.333621in}{1.429542in}}%
\pgfpathlineto{\pgfqpoint{4.323192in}{1.376474in}}%
\pgfpathlineto{\pgfqpoint{4.312764in}{1.381763in}}%
\pgfpathlineto{\pgfqpoint{4.302336in}{1.271380in}}%
\pgfpathlineto{\pgfqpoint{4.291908in}{1.286418in}}%
\pgfpathlineto{\pgfqpoint{4.281480in}{1.575221in}}%
\pgfpathlineto{\pgfqpoint{4.271052in}{1.257405in}}%
\pgfpathlineto{\pgfqpoint{4.260623in}{1.370518in}}%
\pgfpathlineto{\pgfqpoint{4.250195in}{1.364247in}}%
\pgfpathlineto{\pgfqpoint{4.239767in}{1.423767in}}%
\pgfpathlineto{\pgfqpoint{4.229339in}{1.424242in}}%
\pgfpathlineto{\pgfqpoint{4.218911in}{1.420495in}}%
\pgfpathlineto{\pgfqpoint{4.208483in}{1.485088in}}%
\pgfpathlineto{\pgfqpoint{4.198054in}{1.591960in}}%
\pgfpathlineto{\pgfqpoint{4.187626in}{1.350474in}}%
\pgfpathlineto{\pgfqpoint{4.177198in}{1.487000in}}%
\pgfpathlineto{\pgfqpoint{4.166770in}{1.637364in}}%
\pgfpathlineto{\pgfqpoint{4.156342in}{1.392481in}}%
\pgfpathlineto{\pgfqpoint{4.145913in}{1.552878in}}%
\pgfpathlineto{\pgfqpoint{4.135485in}{1.585972in}}%
\pgfpathlineto{\pgfqpoint{4.125057in}{1.586461in}}%
\pgfpathlineto{\pgfqpoint{4.114629in}{1.506427in}}%
\pgfpathlineto{\pgfqpoint{4.104201in}{1.594795in}}%
\pgfpathlineto{\pgfqpoint{4.093773in}{1.587192in}}%
\pgfpathlineto{\pgfqpoint{4.083344in}{1.507307in}}%
\pgfpathlineto{\pgfqpoint{4.072916in}{1.474716in}}%
\pgfpathlineto{\pgfqpoint{4.062488in}{1.607050in}}%
\pgfpathlineto{\pgfqpoint{4.052060in}{1.484217in}}%
\pgfpathlineto{\pgfqpoint{4.041632in}{1.533669in}}%
\pgfpathlineto{\pgfqpoint{4.031204in}{1.460069in}}%
\pgfpathlineto{\pgfqpoint{4.020775in}{1.474435in}}%
\pgfpathlineto{\pgfqpoint{4.010347in}{1.384420in}}%
\pgfpathlineto{\pgfqpoint{3.999919in}{1.485630in}}%
\pgfpathlineto{\pgfqpoint{3.989491in}{1.523926in}}%
\pgfpathlineto{\pgfqpoint{3.979063in}{1.538442in}}%
\pgfpathlineto{\pgfqpoint{3.968635in}{1.601755in}}%
\pgfpathlineto{\pgfqpoint{3.958206in}{1.610861in}}%
\pgfpathlineto{\pgfqpoint{3.947778in}{1.555512in}}%
\pgfpathlineto{\pgfqpoint{3.937350in}{1.617930in}}%
\pgfpathlineto{\pgfqpoint{3.926922in}{1.676684in}}%
\pgfpathlineto{\pgfqpoint{3.916494in}{1.627334in}}%
\pgfpathlineto{\pgfqpoint{3.906065in}{1.577574in}}%
\pgfpathlineto{\pgfqpoint{3.895637in}{1.614348in}}%
\pgfpathlineto{\pgfqpoint{3.885209in}{1.545828in}}%
\pgfpathlineto{\pgfqpoint{3.874781in}{1.626189in}}%
\pgfpathlineto{\pgfqpoint{3.864353in}{1.637574in}}%
\pgfpathlineto{\pgfqpoint{3.853925in}{1.554971in}}%
\pgfpathlineto{\pgfqpoint{3.843496in}{1.751979in}}%
\pgfpathlineto{\pgfqpoint{3.833068in}{1.680168in}}%
\pgfpathlineto{\pgfqpoint{3.822640in}{1.593993in}}%
\pgfpathlineto{\pgfqpoint{3.812212in}{1.614946in}}%
\pgfpathlineto{\pgfqpoint{3.801784in}{1.548816in}}%
\pgfpathlineto{\pgfqpoint{3.791356in}{1.617222in}}%
\pgfpathlineto{\pgfqpoint{3.780927in}{1.552763in}}%
\pgfpathlineto{\pgfqpoint{3.770499in}{1.633695in}}%
\pgfpathlineto{\pgfqpoint{3.760071in}{1.554968in}}%
\pgfpathlineto{\pgfqpoint{3.749643in}{1.526523in}}%
\pgfpathlineto{\pgfqpoint{3.739215in}{1.554326in}}%
\pgfpathlineto{\pgfqpoint{3.728787in}{1.621624in}}%
\pgfpathlineto{\pgfqpoint{3.718358in}{1.648688in}}%
\pgfpathlineto{\pgfqpoint{3.707930in}{1.638566in}}%
\pgfpathlineto{\pgfqpoint{3.697502in}{1.611346in}}%
\pgfpathlineto{\pgfqpoint{3.687074in}{1.697570in}}%
\pgfpathlineto{\pgfqpoint{3.676646in}{1.621737in}}%
\pgfpathlineto{\pgfqpoint{3.666217in}{1.609321in}}%
\pgfpathlineto{\pgfqpoint{3.655789in}{1.705792in}}%
\pgfpathlineto{\pgfqpoint{3.645361in}{1.642640in}}%
\pgfpathlineto{\pgfqpoint{3.634933in}{1.785640in}}%
\pgfpathlineto{\pgfqpoint{3.624505in}{1.669088in}}%
\pgfpathlineto{\pgfqpoint{3.614077in}{1.656147in}}%
\pgfpathlineto{\pgfqpoint{3.603648in}{1.723034in}}%
\pgfpathlineto{\pgfqpoint{3.593220in}{1.743140in}}%
\pgfpathlineto{\pgfqpoint{3.582792in}{1.698332in}}%
\pgfpathlineto{\pgfqpoint{3.572364in}{1.714763in}}%
\pgfpathlineto{\pgfqpoint{3.561936in}{1.758654in}}%
\pgfpathlineto{\pgfqpoint{3.551508in}{1.724035in}}%
\pgfpathlineto{\pgfqpoint{3.541079in}{1.781860in}}%
\pgfpathlineto{\pgfqpoint{3.530651in}{1.730349in}}%
\pgfpathlineto{\pgfqpoint{3.520223in}{1.738299in}}%
\pgfpathlineto{\pgfqpoint{3.509795in}{1.650219in}}%
\pgfpathlineto{\pgfqpoint{3.499367in}{1.718657in}}%
\pgfpathlineto{\pgfqpoint{3.488938in}{1.679369in}}%
\pgfpathlineto{\pgfqpoint{3.478510in}{1.715806in}}%
\pgfpathlineto{\pgfqpoint{3.468082in}{1.760413in}}%
\pgfpathlineto{\pgfqpoint{3.457654in}{1.709727in}}%
\pgfpathlineto{\pgfqpoint{3.447226in}{1.736314in}}%
\pgfpathlineto{\pgfqpoint{3.436798in}{1.734864in}}%
\pgfpathlineto{\pgfqpoint{3.426369in}{1.716182in}}%
\pgfpathlineto{\pgfqpoint{3.415941in}{1.708857in}}%
\pgfpathlineto{\pgfqpoint{3.405513in}{1.744878in}}%
\pgfpathlineto{\pgfqpoint{3.395085in}{1.796670in}}%
\pgfpathlineto{\pgfqpoint{3.384657in}{1.805325in}}%
\pgfpathlineto{\pgfqpoint{3.374229in}{1.765296in}}%
\pgfpathlineto{\pgfqpoint{3.363800in}{1.783679in}}%
\pgfpathlineto{\pgfqpoint{3.353372in}{1.778787in}}%
\pgfpathlineto{\pgfqpoint{3.342944in}{1.747736in}}%
\pgfpathlineto{\pgfqpoint{3.332516in}{1.727395in}}%
\pgfpathlineto{\pgfqpoint{3.322088in}{1.782608in}}%
\pgfpathlineto{\pgfqpoint{3.311660in}{1.742319in}}%
\pgfpathlineto{\pgfqpoint{3.301231in}{1.756218in}}%
\pgfpathlineto{\pgfqpoint{3.290803in}{1.837783in}}%
\pgfpathlineto{\pgfqpoint{3.280375in}{1.842327in}}%
\pgfpathlineto{\pgfqpoint{3.269947in}{1.820756in}}%
\pgfpathlineto{\pgfqpoint{3.259519in}{1.775964in}}%
\pgfpathlineto{\pgfqpoint{3.249090in}{1.813015in}}%
\pgfpathlineto{\pgfqpoint{3.238662in}{1.847301in}}%
\pgfpathlineto{\pgfqpoint{3.228234in}{1.803609in}}%
\pgfpathlineto{\pgfqpoint{3.217806in}{1.774596in}}%
\pgfpathlineto{\pgfqpoint{3.207378in}{1.803799in}}%
\pgfpathlineto{\pgfqpoint{3.196950in}{1.824285in}}%
\pgfpathlineto{\pgfqpoint{3.186521in}{1.871607in}}%
\pgfpathlineto{\pgfqpoint{3.176093in}{1.832094in}}%
\pgfpathlineto{\pgfqpoint{3.165665in}{1.885008in}}%
\pgfpathlineto{\pgfqpoint{3.155237in}{1.871367in}}%
\pgfpathlineto{\pgfqpoint{3.144809in}{1.864064in}}%
\pgfpathlineto{\pgfqpoint{3.134381in}{1.831773in}}%
\pgfpathlineto{\pgfqpoint{3.123952in}{1.818671in}}%
\pgfpathlineto{\pgfqpoint{3.113524in}{1.840865in}}%
\pgfpathlineto{\pgfqpoint{3.103096in}{1.836160in}}%
\pgfpathlineto{\pgfqpoint{3.092668in}{1.849721in}}%
\pgfpathlineto{\pgfqpoint{3.082240in}{1.870469in}}%
\pgfpathlineto{\pgfqpoint{3.071812in}{1.889217in}}%
\pgfpathlineto{\pgfqpoint{3.061383in}{1.877653in}}%
\pgfpathlineto{\pgfqpoint{3.050955in}{1.893929in}}%
\pgfpathlineto{\pgfqpoint{3.040527in}{1.906327in}}%
\pgfpathlineto{\pgfqpoint{3.030099in}{1.920910in}}%
\pgfpathlineto{\pgfqpoint{3.019671in}{1.918360in}}%
\pgfpathlineto{\pgfqpoint{3.009242in}{1.925728in}}%
\pgfpathlineto{\pgfqpoint{2.998814in}{1.890280in}}%
\pgfpathlineto{\pgfqpoint{2.988386in}{1.893083in}}%
\pgfpathlineto{\pgfqpoint{2.977958in}{1.902690in}}%
\pgfpathlineto{\pgfqpoint{2.967530in}{1.910732in}}%
\pgfpathlineto{\pgfqpoint{2.957102in}{1.941106in}}%
\pgfpathlineto{\pgfqpoint{2.946673in}{1.949102in}}%
\pgfpathlineto{\pgfqpoint{2.936245in}{1.950306in}}%
\pgfpathlineto{\pgfqpoint{2.925817in}{1.963683in}}%
\pgfpathlineto{\pgfqpoint{2.915389in}{1.976981in}}%
\pgfpathlineto{\pgfqpoint{2.904961in}{2.029422in}}%
\pgfpathlineto{\pgfqpoint{2.894533in}{1.990672in}}%
\pgfpathlineto{\pgfqpoint{2.884104in}{1.970991in}}%
\pgfpathlineto{\pgfqpoint{2.873676in}{1.978119in}}%
\pgfpathlineto{\pgfqpoint{2.863248in}{2.014707in}}%
\pgfpathlineto{\pgfqpoint{2.852820in}{2.032068in}}%
\pgfpathlineto{\pgfqpoint{2.842392in}{2.042059in}}%
\pgfpathlineto{\pgfqpoint{2.831964in}{2.094898in}}%
\pgfpathlineto{\pgfqpoint{2.821535in}{2.116388in}}%
\pgfpathlineto{\pgfqpoint{2.811107in}{2.087931in}}%
\pgfpathlineto{\pgfqpoint{2.800679in}{2.107208in}}%
\pgfpathlineto{\pgfqpoint{2.790251in}{2.095604in}}%
\pgfpathlineto{\pgfqpoint{2.779823in}{2.128374in}}%
\pgfpathlineto{\pgfqpoint{2.769394in}{2.199981in}}%
\pgfpathlineto{\pgfqpoint{2.758966in}{2.173527in}}%
\pgfpathlineto{\pgfqpoint{2.748538in}{2.209133in}}%
\pgfpathlineto{\pgfqpoint{2.738110in}{2.192993in}}%
\pgfpathlineto{\pgfqpoint{2.727682in}{2.182024in}}%
\pgfpathlineto{\pgfqpoint{2.717254in}{2.260995in}}%
\pgfpathlineto{\pgfqpoint{2.706825in}{2.336219in}}%
\pgfpathlineto{\pgfqpoint{2.696397in}{2.434297in}}%
\pgfpathlineto{\pgfqpoint{2.685969in}{2.440464in}}%
\pgfpathlineto{\pgfqpoint{2.675541in}{2.540986in}}%
\pgfpathlineto{\pgfqpoint{2.665113in}{2.628828in}}%
\pgfpathlineto{\pgfqpoint{2.654685in}{2.655002in}}%
\pgfpathlineto{\pgfqpoint{2.644256in}{2.632601in}}%
\pgfpathlineto{\pgfqpoint{2.633828in}{2.645093in}}%
\pgfpathlineto{\pgfqpoint{2.623400in}{2.676197in}}%
\pgfpathlineto{\pgfqpoint{2.612972in}{2.677815in}}%
\pgfpathlineto{\pgfqpoint{2.602544in}{2.714061in}}%
\pgfpathlineto{\pgfqpoint{2.592115in}{2.741734in}}%
\pgfpathlineto{\pgfqpoint{2.581687in}{2.734775in}}%
\pgfpathlineto{\pgfqpoint{2.571259in}{2.749591in}}%
\pgfpathlineto{\pgfqpoint{2.560831in}{2.761636in}}%
\pgfpathlineto{\pgfqpoint{2.550403in}{2.775124in}}%
\pgfpathlineto{\pgfqpoint{2.539975in}{2.737128in}}%
\pgfpathlineto{\pgfqpoint{2.529546in}{2.754666in}}%
\pgfpathlineto{\pgfqpoint{2.519118in}{2.771794in}}%
\pgfpathlineto{\pgfqpoint{2.508690in}{2.789229in}}%
\pgfpathlineto{\pgfqpoint{2.498262in}{2.775189in}}%
\pgfpathlineto{\pgfqpoint{2.487834in}{2.801286in}}%
\pgfpathlineto{\pgfqpoint{2.477406in}{2.839008in}}%
\pgfpathlineto{\pgfqpoint{2.466977in}{2.860691in}}%
\pgfpathlineto{\pgfqpoint{2.456549in}{2.800289in}}%
\pgfpathlineto{\pgfqpoint{2.446121in}{2.824032in}}%
\pgfpathlineto{\pgfqpoint{2.435693in}{2.858655in}}%
\pgfpathlineto{\pgfqpoint{2.425265in}{2.853258in}}%
\pgfpathlineto{\pgfqpoint{2.414837in}{2.842265in}}%
\pgfpathlineto{\pgfqpoint{2.404408in}{2.849325in}}%
\pgfpathlineto{\pgfqpoint{2.393980in}{2.898278in}}%
\pgfpathlineto{\pgfqpoint{2.383552in}{2.905353in}}%
\pgfpathlineto{\pgfqpoint{2.373124in}{2.895736in}}%
\pgfpathlineto{\pgfqpoint{2.362696in}{2.880044in}}%
\pgfpathlineto{\pgfqpoint{2.352267in}{2.801838in}}%
\pgfpathlineto{\pgfqpoint{2.341839in}{2.911488in}}%
\pgfpathlineto{\pgfqpoint{2.331411in}{2.895162in}}%
\pgfpathlineto{\pgfqpoint{2.320983in}{2.943176in}}%
\pgfpathlineto{\pgfqpoint{2.310555in}{2.923889in}}%
\pgfpathlineto{\pgfqpoint{2.300127in}{2.962048in}}%
\pgfpathlineto{\pgfqpoint{2.289698in}{2.943614in}}%
\pgfpathlineto{\pgfqpoint{2.279270in}{2.956457in}}%
\pgfpathlineto{\pgfqpoint{2.268842in}{2.949689in}}%
\pgfpathlineto{\pgfqpoint{2.258414in}{2.938603in}}%
\pgfpathlineto{\pgfqpoint{2.247986in}{2.936127in}}%
\pgfpathlineto{\pgfqpoint{2.237558in}{2.947380in}}%
\pgfpathlineto{\pgfqpoint{2.227129in}{2.920984in}}%
\pgfpathlineto{\pgfqpoint{2.216701in}{2.956736in}}%
\pgfpathlineto{\pgfqpoint{2.206273in}{2.930018in}}%
\pgfpathlineto{\pgfqpoint{2.195845in}{2.899736in}}%
\pgfpathlineto{\pgfqpoint{2.185417in}{2.944750in}}%
\pgfpathlineto{\pgfqpoint{2.174989in}{3.017654in}}%
\pgfpathlineto{\pgfqpoint{2.164560in}{2.935112in}}%
\pgfpathlineto{\pgfqpoint{2.154132in}{2.971401in}}%
\pgfpathlineto{\pgfqpoint{2.143704in}{2.981120in}}%
\pgfpathlineto{\pgfqpoint{2.133276in}{2.938622in}}%
\pgfpathlineto{\pgfqpoint{2.122848in}{2.995129in}}%
\pgfpathlineto{\pgfqpoint{2.112419in}{2.972669in}}%
\pgfpathlineto{\pgfqpoint{2.101991in}{3.010435in}}%
\pgfpathlineto{\pgfqpoint{2.091563in}{2.975007in}}%
\pgfpathlineto{\pgfqpoint{2.081135in}{2.988519in}}%
\pgfpathlineto{\pgfqpoint{2.070707in}{2.969822in}}%
\pgfpathlineto{\pgfqpoint{2.060279in}{2.942181in}}%
\pgfpathlineto{\pgfqpoint{2.049850in}{2.952352in}}%
\pgfpathlineto{\pgfqpoint{2.039422in}{2.977561in}}%
\pgfpathlineto{\pgfqpoint{2.028994in}{2.925789in}}%
\pgfpathlineto{\pgfqpoint{2.018566in}{2.954983in}}%
\pgfpathlineto{\pgfqpoint{2.008138in}{2.933375in}}%
\pgfpathlineto{\pgfqpoint{1.997710in}{2.942678in}}%
\pgfpathlineto{\pgfqpoint{1.987281in}{2.944096in}}%
\pgfpathlineto{\pgfqpoint{1.976853in}{2.911029in}}%
\pgfpathlineto{\pgfqpoint{1.966425in}{2.942380in}}%
\pgfpathlineto{\pgfqpoint{1.955997in}{2.994642in}}%
\pgfpathlineto{\pgfqpoint{1.945569in}{3.070185in}}%
\pgfpathlineto{\pgfqpoint{1.935141in}{3.041775in}}%
\pgfpathlineto{\pgfqpoint{1.924712in}{3.070754in}}%
\pgfpathlineto{\pgfqpoint{1.914284in}{3.078442in}}%
\pgfpathlineto{\pgfqpoint{1.903856in}{3.048575in}}%
\pgfpathlineto{\pgfqpoint{1.893428in}{3.017332in}}%
\pgfpathlineto{\pgfqpoint{1.883000in}{3.091765in}}%
\pgfpathlineto{\pgfqpoint{1.872571in}{3.055158in}}%
\pgfpathlineto{\pgfqpoint{1.862143in}{3.055415in}}%
\pgfpathlineto{\pgfqpoint{1.851715in}{3.060854in}}%
\pgfpathlineto{\pgfqpoint{1.841287in}{3.066901in}}%
\pgfpathlineto{\pgfqpoint{1.830859in}{3.051079in}}%
\pgfpathlineto{\pgfqpoint{1.820431in}{3.011908in}}%
\pgfpathlineto{\pgfqpoint{1.810002in}{2.964444in}}%
\pgfpathlineto{\pgfqpoint{1.799574in}{2.973724in}}%
\pgfpathlineto{\pgfqpoint{1.789146in}{2.948116in}}%
\pgfpathlineto{\pgfqpoint{1.778718in}{2.984124in}}%
\pgfpathlineto{\pgfqpoint{1.768290in}{3.003001in}}%
\pgfpathlineto{\pgfqpoint{1.757862in}{2.994064in}}%
\pgfpathlineto{\pgfqpoint{1.747433in}{2.994048in}}%
\pgfpathlineto{\pgfqpoint{1.737005in}{2.960590in}}%
\pgfpathlineto{\pgfqpoint{1.726577in}{2.973083in}}%
\pgfpathlineto{\pgfqpoint{1.716149in}{3.014613in}}%
\pgfpathlineto{\pgfqpoint{1.705721in}{2.992538in}}%
\pgfpathlineto{\pgfqpoint{1.695292in}{3.000004in}}%
\pgfpathlineto{\pgfqpoint{1.684864in}{3.035707in}}%
\pgfpathlineto{\pgfqpoint{1.674436in}{2.999687in}}%
\pgfpathlineto{\pgfqpoint{1.664008in}{3.085668in}}%
\pgfpathlineto{\pgfqpoint{1.653580in}{3.067706in}}%
\pgfpathlineto{\pgfqpoint{1.643152in}{3.078648in}}%
\pgfpathlineto{\pgfqpoint{1.632723in}{3.100541in}}%
\pgfpathlineto{\pgfqpoint{1.622295in}{3.045578in}}%
\pgfpathlineto{\pgfqpoint{1.611867in}{3.006924in}}%
\pgfpathlineto{\pgfqpoint{1.601439in}{2.997483in}}%
\pgfpathlineto{\pgfqpoint{1.591011in}{2.982014in}}%
\pgfpathlineto{\pgfqpoint{1.580583in}{2.960861in}}%
\pgfpathlineto{\pgfqpoint{1.570154in}{2.998454in}}%
\pgfpathlineto{\pgfqpoint{1.559726in}{2.990581in}}%
\pgfpathlineto{\pgfqpoint{1.549298in}{2.981571in}}%
\pgfpathlineto{\pgfqpoint{1.538870in}{2.965271in}}%
\pgfpathlineto{\pgfqpoint{1.528442in}{2.992225in}}%
\pgfpathlineto{\pgfqpoint{1.518014in}{2.974342in}}%
\pgfpathlineto{\pgfqpoint{1.507585in}{3.000343in}}%
\pgfpathlineto{\pgfqpoint{1.497157in}{3.020389in}}%
\pgfpathlineto{\pgfqpoint{1.486729in}{3.045343in}}%
\pgfpathlineto{\pgfqpoint{1.476301in}{3.015351in}}%
\pgfpathlineto{\pgfqpoint{1.465873in}{2.933800in}}%
\pgfpathlineto{\pgfqpoint{1.455444in}{2.874451in}}%
\pgfpathlineto{\pgfqpoint{1.445016in}{2.804119in}}%
\pgfpathlineto{\pgfqpoint{1.434588in}{2.910658in}}%
\pgfpathlineto{\pgfqpoint{1.424160in}{2.824952in}}%
\pgfpathlineto{\pgfqpoint{1.413732in}{2.867428in}}%
\pgfpathlineto{\pgfqpoint{1.403304in}{2.934311in}}%
\pgfpathlineto{\pgfqpoint{1.392875in}{2.878067in}}%
\pgfpathlineto{\pgfqpoint{1.382447in}{2.877834in}}%
\pgfpathlineto{\pgfqpoint{1.372019in}{2.923462in}}%
\pgfpathlineto{\pgfqpoint{1.361591in}{2.882109in}}%
\pgfpathlineto{\pgfqpoint{1.351163in}{2.912283in}}%
\pgfpathlineto{\pgfqpoint{1.340735in}{2.906651in}}%
\pgfpathlineto{\pgfqpoint{1.330306in}{3.015549in}}%
\pgfpathlineto{\pgfqpoint{1.319878in}{2.942392in}}%
\pgfpathlineto{\pgfqpoint{1.309450in}{2.916724in}}%
\pgfpathlineto{\pgfqpoint{1.299022in}{2.891701in}}%
\pgfpathlineto{\pgfqpoint{1.288594in}{2.882679in}}%
\pgfpathlineto{\pgfqpoint{1.278166in}{2.949643in}}%
\pgfpathlineto{\pgfqpoint{1.267737in}{2.915108in}}%
\pgfpathlineto{\pgfqpoint{1.257309in}{2.870341in}}%
\pgfpathlineto{\pgfqpoint{1.246881in}{2.893844in}}%
\pgfpathlineto{\pgfqpoint{1.236453in}{2.910425in}}%
\pgfpathlineto{\pgfqpoint{1.226025in}{2.881207in}}%
\pgfpathlineto{\pgfqpoint{1.215596in}{2.894488in}}%
\pgfpathlineto{\pgfqpoint{1.205168in}{2.857529in}}%
\pgfpathlineto{\pgfqpoint{1.194740in}{2.883755in}}%
\pgfpathlineto{\pgfqpoint{1.184312in}{2.856536in}}%
\pgfpathlineto{\pgfqpoint{1.173884in}{2.909440in}}%
\pgfpathlineto{\pgfqpoint{1.163456in}{2.931374in}}%
\pgfpathlineto{\pgfqpoint{1.153027in}{2.969709in}}%
\pgfpathlineto{\pgfqpoint{1.142599in}{2.948096in}}%
\pgfpathlineto{\pgfqpoint{1.132171in}{2.919788in}}%
\pgfpathlineto{\pgfqpoint{1.121743in}{2.925191in}}%
\pgfpathlineto{\pgfqpoint{1.111315in}{2.897270in}}%
\pgfpathlineto{\pgfqpoint{1.100887in}{2.888973in}}%
\pgfpathlineto{\pgfqpoint{1.090458in}{3.019904in}}%
\pgfpathlineto{\pgfqpoint{1.080030in}{2.949573in}}%
\pgfpathlineto{\pgfqpoint{1.069602in}{2.999821in}}%
\pgfpathlineto{\pgfqpoint{1.059174in}{2.970890in}}%
\pgfpathlineto{\pgfqpoint{1.048746in}{2.927179in}}%
\pgfpathlineto{\pgfqpoint{1.038318in}{2.914202in}}%
\pgfpathlineto{\pgfqpoint{1.027889in}{2.907997in}}%
\pgfpathlineto{\pgfqpoint{1.017461in}{2.946380in}}%
\pgfpathlineto{\pgfqpoint{1.007033in}{2.962126in}}%
\pgfpathlineto{\pgfqpoint{0.996605in}{3.035662in}}%
\pgfpathlineto{\pgfqpoint{0.986177in}{3.012931in}}%
\pgfpathlineto{\pgfqpoint{0.975748in}{3.057193in}}%
\pgfpathlineto{\pgfqpoint{0.965320in}{3.036885in}}%
\pgfpathlineto{\pgfqpoint{0.954892in}{3.037154in}}%
\pgfpathlineto{\pgfqpoint{0.944464in}{3.015906in}}%
\pgfpathlineto{\pgfqpoint{0.934036in}{3.050112in}}%
\pgfpathlineto{\pgfqpoint{0.923608in}{3.029540in}}%
\pgfpathlineto{\pgfqpoint{0.913179in}{3.053247in}}%
\pgfpathlineto{\pgfqpoint{0.902751in}{3.092864in}}%
\pgfpathlineto{\pgfqpoint{0.892323in}{3.081425in}}%
\pgfpathlineto{\pgfqpoint{0.881895in}{3.068808in}}%
\pgfpathlineto{\pgfqpoint{0.871467in}{3.139178in}}%
\pgfpathlineto{\pgfqpoint{0.861039in}{3.135850in}}%
\pgfpathlineto{\pgfqpoint{0.850610in}{3.108667in}}%
\pgfpathlineto{\pgfqpoint{0.840182in}{3.143482in}}%
\pgfpathlineto{\pgfqpoint{0.829754in}{3.141597in}}%
\pgfpathlineto{\pgfqpoint{0.819326in}{3.158641in}}%
\pgfpathlineto{\pgfqpoint{0.808898in}{3.174705in}}%
\pgfpathlineto{\pgfqpoint{0.798470in}{3.151866in}}%
\pgfpathlineto{\pgfqpoint{0.788041in}{3.131674in}}%
\pgfpathlineto{\pgfqpoint{0.777613in}{3.106019in}}%
\pgfpathlineto{\pgfqpoint{0.767185in}{3.120077in}}%
\pgfpathlineto{\pgfqpoint{0.756757in}{3.049210in}}%
\pgfpathlineto{\pgfqpoint{0.746329in}{3.009539in}}%
\pgfpathlineto{\pgfqpoint{0.735900in}{2.978121in}}%
\pgfpathlineto{\pgfqpoint{0.725472in}{3.031960in}}%
\pgfpathlineto{\pgfqpoint{0.715044in}{2.986406in}}%
\pgfpathlineto{\pgfqpoint{0.704616in}{2.980005in}}%
\pgfpathlineto{\pgfqpoint{0.694188in}{3.059983in}}%
\pgfpathlineto{\pgfqpoint{0.683760in}{3.052949in}}%
\pgfpathlineto{\pgfqpoint{0.673331in}{3.020477in}}%
\pgfpathlineto{\pgfqpoint{0.662903in}{2.988797in}}%
\pgfpathlineto{\pgfqpoint{0.652475in}{2.942503in}}%
\pgfpathlineto{\pgfqpoint{0.642047in}{3.038799in}}%
\pgfpathlineto{\pgfqpoint{0.631619in}{2.998205in}}%
\pgfpathlineto{\pgfqpoint{0.621191in}{2.959966in}}%
\pgfpathlineto{\pgfqpoint{0.610762in}{3.007082in}}%
\pgfpathclose%
\pgfusepath{stroke,fill}%
\end{pgfscope}%
\begin{pgfscope}%
\pgfpathrectangle{\pgfqpoint{0.610762in}{0.961156in}}{\pgfqpoint{4.171270in}{2.577986in}} %
\pgfusepath{clip}%
\pgfsetroundcap%
\pgfsetroundjoin%
\pgfsetlinewidth{1.756562pt}%
\definecolor{currentstroke}{rgb}{0.200000,0.427451,0.650980}%
\pgfsetstrokecolor{currentstroke}%
\pgfsetstrokeopacity{0.800000}%
\pgfsetdash{}{0pt}%
\pgfpathmoveto{\pgfqpoint{0.610762in}{3.063598in}}%
\pgfpathlineto{\pgfqpoint{0.621191in}{3.080275in}}%
\pgfpathlineto{\pgfqpoint{0.631619in}{3.086398in}}%
\pgfpathlineto{\pgfqpoint{0.642047in}{3.087175in}}%
\pgfpathlineto{\pgfqpoint{0.652475in}{3.077655in}}%
\pgfpathlineto{\pgfqpoint{0.662903in}{3.108388in}}%
\pgfpathlineto{\pgfqpoint{0.673331in}{3.080407in}}%
\pgfpathlineto{\pgfqpoint{0.683760in}{3.101690in}}%
\pgfpathlineto{\pgfqpoint{0.694188in}{3.080293in}}%
\pgfpathlineto{\pgfqpoint{0.704616in}{3.091159in}}%
\pgfpathlineto{\pgfqpoint{0.715044in}{3.121595in}}%
\pgfpathlineto{\pgfqpoint{0.725472in}{3.092841in}}%
\pgfpathlineto{\pgfqpoint{0.735900in}{3.142552in}}%
\pgfpathlineto{\pgfqpoint{0.746329in}{3.091191in}}%
\pgfpathlineto{\pgfqpoint{0.756757in}{3.089427in}}%
\pgfpathlineto{\pgfqpoint{0.767185in}{3.052778in}}%
\pgfpathlineto{\pgfqpoint{0.777613in}{3.048344in}}%
\pgfpathlineto{\pgfqpoint{0.788041in}{3.072343in}}%
\pgfpathlineto{\pgfqpoint{0.798470in}{3.055477in}}%
\pgfpathlineto{\pgfqpoint{0.808898in}{3.065177in}}%
\pgfpathlineto{\pgfqpoint{0.819326in}{3.040022in}}%
\pgfpathlineto{\pgfqpoint{0.829754in}{3.083516in}}%
\pgfpathlineto{\pgfqpoint{0.840182in}{3.080040in}}%
\pgfpathlineto{\pgfqpoint{0.850610in}{3.101890in}}%
\pgfpathlineto{\pgfqpoint{0.861039in}{3.011282in}}%
\pgfpathlineto{\pgfqpoint{0.871467in}{3.051101in}}%
\pgfpathlineto{\pgfqpoint{0.881895in}{3.021718in}}%
\pgfpathlineto{\pgfqpoint{0.892323in}{3.019368in}}%
\pgfpathlineto{\pgfqpoint{0.902751in}{3.039247in}}%
\pgfpathlineto{\pgfqpoint{0.913179in}{3.039469in}}%
\pgfpathlineto{\pgfqpoint{0.923608in}{3.035532in}}%
\pgfpathlineto{\pgfqpoint{0.934036in}{3.024021in}}%
\pgfpathlineto{\pgfqpoint{0.944464in}{3.051308in}}%
\pgfpathlineto{\pgfqpoint{0.954892in}{3.049175in}}%
\pgfpathlineto{\pgfqpoint{0.965320in}{3.054808in}}%
\pgfpathlineto{\pgfqpoint{0.986177in}{3.112387in}}%
\pgfpathlineto{\pgfqpoint{0.996605in}{3.112826in}}%
\pgfpathlineto{\pgfqpoint{1.007033in}{3.125716in}}%
\pgfpathlineto{\pgfqpoint{1.017461in}{3.098429in}}%
\pgfpathlineto{\pgfqpoint{1.027889in}{3.076370in}}%
\pgfpathlineto{\pgfqpoint{1.038318in}{3.058795in}}%
\pgfpathlineto{\pgfqpoint{1.048746in}{3.045988in}}%
\pgfpathlineto{\pgfqpoint{1.059174in}{3.073957in}}%
\pgfpathlineto{\pgfqpoint{1.069602in}{3.053630in}}%
\pgfpathlineto{\pgfqpoint{1.080030in}{3.028341in}}%
\pgfpathlineto{\pgfqpoint{1.090458in}{3.051958in}}%
\pgfpathlineto{\pgfqpoint{1.100887in}{3.015026in}}%
\pgfpathlineto{\pgfqpoint{1.111315in}{3.023461in}}%
\pgfpathlineto{\pgfqpoint{1.121743in}{3.024659in}}%
\pgfpathlineto{\pgfqpoint{1.132171in}{3.064093in}}%
\pgfpathlineto{\pgfqpoint{1.142599in}{3.062015in}}%
\pgfpathlineto{\pgfqpoint{1.153027in}{3.070668in}}%
\pgfpathlineto{\pgfqpoint{1.163456in}{3.045028in}}%
\pgfpathlineto{\pgfqpoint{1.173884in}{3.030929in}}%
\pgfpathlineto{\pgfqpoint{1.184312in}{3.033275in}}%
\pgfpathlineto{\pgfqpoint{1.194740in}{3.011007in}}%
\pgfpathlineto{\pgfqpoint{1.205168in}{3.025565in}}%
\pgfpathlineto{\pgfqpoint{1.215596in}{3.022580in}}%
\pgfpathlineto{\pgfqpoint{1.226025in}{2.995447in}}%
\pgfpathlineto{\pgfqpoint{1.246881in}{3.048364in}}%
\pgfpathlineto{\pgfqpoint{1.257309in}{3.042347in}}%
\pgfpathlineto{\pgfqpoint{1.267737in}{3.051466in}}%
\pgfpathlineto{\pgfqpoint{1.278166in}{3.023737in}}%
\pgfpathlineto{\pgfqpoint{1.288594in}{3.030307in}}%
\pgfpathlineto{\pgfqpoint{1.299022in}{3.052673in}}%
\pgfpathlineto{\pgfqpoint{1.309450in}{3.066597in}}%
\pgfpathlineto{\pgfqpoint{1.319878in}{3.063909in}}%
\pgfpathlineto{\pgfqpoint{1.330306in}{3.074480in}}%
\pgfpathlineto{\pgfqpoint{1.340735in}{3.048516in}}%
\pgfpathlineto{\pgfqpoint{1.351163in}{3.061721in}}%
\pgfpathlineto{\pgfqpoint{1.361591in}{3.068610in}}%
\pgfpathlineto{\pgfqpoint{1.372019in}{3.066371in}}%
\pgfpathlineto{\pgfqpoint{1.382447in}{3.072626in}}%
\pgfpathlineto{\pgfqpoint{1.392875in}{3.058561in}}%
\pgfpathlineto{\pgfqpoint{1.403304in}{3.048547in}}%
\pgfpathlineto{\pgfqpoint{1.413732in}{3.058139in}}%
\pgfpathlineto{\pgfqpoint{1.424160in}{3.048981in}}%
\pgfpathlineto{\pgfqpoint{1.455444in}{3.086847in}}%
\pgfpathlineto{\pgfqpoint{1.465873in}{3.063028in}}%
\pgfpathlineto{\pgfqpoint{1.476301in}{3.014317in}}%
\pgfpathlineto{\pgfqpoint{1.486729in}{3.013984in}}%
\pgfpathlineto{\pgfqpoint{1.497157in}{3.026509in}}%
\pgfpathlineto{\pgfqpoint{1.518014in}{2.970460in}}%
\pgfpathlineto{\pgfqpoint{1.528442in}{2.984607in}}%
\pgfpathlineto{\pgfqpoint{1.538870in}{2.976983in}}%
\pgfpathlineto{\pgfqpoint{1.549298in}{2.991319in}}%
\pgfpathlineto{\pgfqpoint{1.559726in}{2.999108in}}%
\pgfpathlineto{\pgfqpoint{1.570154in}{3.018518in}}%
\pgfpathlineto{\pgfqpoint{1.591011in}{3.006827in}}%
\pgfpathlineto{\pgfqpoint{1.601439in}{2.979071in}}%
\pgfpathlineto{\pgfqpoint{1.611867in}{2.988810in}}%
\pgfpathlineto{\pgfqpoint{1.632723in}{2.980079in}}%
\pgfpathlineto{\pgfqpoint{1.643152in}{3.000320in}}%
\pgfpathlineto{\pgfqpoint{1.653580in}{2.988543in}}%
\pgfpathlineto{\pgfqpoint{1.664008in}{3.017747in}}%
\pgfpathlineto{\pgfqpoint{1.674436in}{3.000310in}}%
\pgfpathlineto{\pgfqpoint{1.684864in}{3.025814in}}%
\pgfpathlineto{\pgfqpoint{1.705721in}{3.030504in}}%
\pgfpathlineto{\pgfqpoint{1.716149in}{3.040964in}}%
\pgfpathlineto{\pgfqpoint{1.726577in}{3.000844in}}%
\pgfpathlineto{\pgfqpoint{1.747433in}{3.012147in}}%
\pgfpathlineto{\pgfqpoint{1.757862in}{3.024955in}}%
\pgfpathlineto{\pgfqpoint{1.768290in}{3.023584in}}%
\pgfpathlineto{\pgfqpoint{1.778718in}{3.009641in}}%
\pgfpathlineto{\pgfqpoint{1.789146in}{3.024885in}}%
\pgfpathlineto{\pgfqpoint{1.799574in}{3.012617in}}%
\pgfpathlineto{\pgfqpoint{1.810002in}{3.019453in}}%
\pgfpathlineto{\pgfqpoint{1.820431in}{3.006392in}}%
\pgfpathlineto{\pgfqpoint{1.841287in}{3.032438in}}%
\pgfpathlineto{\pgfqpoint{1.851715in}{3.064228in}}%
\pgfpathlineto{\pgfqpoint{1.872571in}{3.032738in}}%
\pgfpathlineto{\pgfqpoint{1.893428in}{3.049886in}}%
\pgfpathlineto{\pgfqpoint{1.903856in}{3.054512in}}%
\pgfpathlineto{\pgfqpoint{1.914284in}{3.078943in}}%
\pgfpathlineto{\pgfqpoint{1.924712in}{3.071010in}}%
\pgfpathlineto{\pgfqpoint{1.935141in}{3.049620in}}%
\pgfpathlineto{\pgfqpoint{1.945569in}{3.058999in}}%
\pgfpathlineto{\pgfqpoint{1.955997in}{3.071843in}}%
\pgfpathlineto{\pgfqpoint{1.966425in}{3.078313in}}%
\pgfpathlineto{\pgfqpoint{1.976853in}{3.079384in}}%
\pgfpathlineto{\pgfqpoint{1.987281in}{3.046267in}}%
\pgfpathlineto{\pgfqpoint{1.997710in}{3.052606in}}%
\pgfpathlineto{\pgfqpoint{2.008138in}{3.056723in}}%
\pgfpathlineto{\pgfqpoint{2.018566in}{3.037098in}}%
\pgfpathlineto{\pgfqpoint{2.028994in}{3.042557in}}%
\pgfpathlineto{\pgfqpoint{2.039422in}{3.064693in}}%
\pgfpathlineto{\pgfqpoint{2.049850in}{3.022097in}}%
\pgfpathlineto{\pgfqpoint{2.060279in}{3.030313in}}%
\pgfpathlineto{\pgfqpoint{2.070707in}{3.016750in}}%
\pgfpathlineto{\pgfqpoint{2.091563in}{3.048826in}}%
\pgfpathlineto{\pgfqpoint{2.101991in}{3.027221in}}%
\pgfpathlineto{\pgfqpoint{2.112419in}{3.037744in}}%
\pgfpathlineto{\pgfqpoint{2.122848in}{3.004943in}}%
\pgfpathlineto{\pgfqpoint{2.133276in}{3.008730in}}%
\pgfpathlineto{\pgfqpoint{2.143704in}{3.022758in}}%
\pgfpathlineto{\pgfqpoint{2.164560in}{3.042381in}}%
\pgfpathlineto{\pgfqpoint{2.174989in}{2.990182in}}%
\pgfpathlineto{\pgfqpoint{2.185417in}{3.024257in}}%
\pgfpathlineto{\pgfqpoint{2.195845in}{3.018620in}}%
\pgfpathlineto{\pgfqpoint{2.206273in}{3.021158in}}%
\pgfpathlineto{\pgfqpoint{2.227129in}{3.000611in}}%
\pgfpathlineto{\pgfqpoint{2.237558in}{3.019128in}}%
\pgfpathlineto{\pgfqpoint{2.247986in}{3.028369in}}%
\pgfpathlineto{\pgfqpoint{2.258414in}{3.024485in}}%
\pgfpathlineto{\pgfqpoint{2.268842in}{3.047790in}}%
\pgfpathlineto{\pgfqpoint{2.279270in}{3.033257in}}%
\pgfpathlineto{\pgfqpoint{2.289698in}{3.045444in}}%
\pgfpathlineto{\pgfqpoint{2.300127in}{3.046775in}}%
\pgfpathlineto{\pgfqpoint{2.310555in}{3.055047in}}%
\pgfpathlineto{\pgfqpoint{2.320983in}{3.015114in}}%
\pgfpathlineto{\pgfqpoint{2.331411in}{3.037295in}}%
\pgfpathlineto{\pgfqpoint{2.341839in}{3.030307in}}%
\pgfpathlineto{\pgfqpoint{2.352267in}{3.041106in}}%
\pgfpathlineto{\pgfqpoint{2.362696in}{3.032819in}}%
\pgfpathlineto{\pgfqpoint{2.373124in}{3.022291in}}%
\pgfpathlineto{\pgfqpoint{2.383552in}{3.027024in}}%
\pgfpathlineto{\pgfqpoint{2.393980in}{3.027465in}}%
\pgfpathlineto{\pgfqpoint{2.404408in}{3.023832in}}%
\pgfpathlineto{\pgfqpoint{2.414837in}{3.015185in}}%
\pgfpathlineto{\pgfqpoint{2.425265in}{3.018710in}}%
\pgfpathlineto{\pgfqpoint{2.435693in}{2.992717in}}%
\pgfpathlineto{\pgfqpoint{2.446121in}{3.014670in}}%
\pgfpathlineto{\pgfqpoint{2.456549in}{3.008712in}}%
\pgfpathlineto{\pgfqpoint{2.466977in}{3.010780in}}%
\pgfpathlineto{\pgfqpoint{2.477406in}{3.030838in}}%
\pgfpathlineto{\pgfqpoint{2.487834in}{3.025077in}}%
\pgfpathlineto{\pgfqpoint{2.498262in}{3.022254in}}%
\pgfpathlineto{\pgfqpoint{2.508690in}{3.024399in}}%
\pgfpathlineto{\pgfqpoint{2.519118in}{3.021020in}}%
\pgfpathlineto{\pgfqpoint{2.529546in}{2.998501in}}%
\pgfpathlineto{\pgfqpoint{2.539975in}{2.988371in}}%
\pgfpathlineto{\pgfqpoint{2.550403in}{2.992448in}}%
\pgfpathlineto{\pgfqpoint{2.560831in}{2.968996in}}%
\pgfpathlineto{\pgfqpoint{2.571259in}{2.978786in}}%
\pgfpathlineto{\pgfqpoint{2.581687in}{2.935532in}}%
\pgfpathlineto{\pgfqpoint{2.592115in}{2.915143in}}%
\pgfpathlineto{\pgfqpoint{2.602544in}{2.906505in}}%
\pgfpathlineto{\pgfqpoint{2.612972in}{2.871441in}}%
\pgfpathlineto{\pgfqpoint{2.623400in}{2.851564in}}%
\pgfpathlineto{\pgfqpoint{2.633828in}{2.826498in}}%
\pgfpathlineto{\pgfqpoint{2.644256in}{2.784417in}}%
\pgfpathlineto{\pgfqpoint{2.654685in}{2.749371in}}%
\pgfpathlineto{\pgfqpoint{2.665113in}{2.696997in}}%
\pgfpathlineto{\pgfqpoint{2.675541in}{2.600080in}}%
\pgfpathlineto{\pgfqpoint{2.685969in}{2.464281in}}%
\pgfpathlineto{\pgfqpoint{2.696397in}{2.465425in}}%
\pgfpathlineto{\pgfqpoint{2.706825in}{2.336672in}}%
\pgfpathlineto{\pgfqpoint{2.717254in}{2.257768in}}%
\pgfpathlineto{\pgfqpoint{2.727682in}{2.205023in}}%
\pgfpathlineto{\pgfqpoint{2.738110in}{2.181043in}}%
\pgfpathlineto{\pgfqpoint{2.748538in}{2.124000in}}%
\pgfpathlineto{\pgfqpoint{2.758966in}{2.125581in}}%
\pgfpathlineto{\pgfqpoint{2.769394in}{2.097412in}}%
\pgfpathlineto{\pgfqpoint{2.779823in}{2.090959in}}%
\pgfpathlineto{\pgfqpoint{2.790251in}{2.082504in}}%
\pgfpathlineto{\pgfqpoint{2.800679in}{2.058049in}}%
\pgfpathlineto{\pgfqpoint{2.811107in}{2.049479in}}%
\pgfpathlineto{\pgfqpoint{2.821535in}{2.033536in}}%
\pgfpathlineto{\pgfqpoint{2.831964in}{2.014450in}}%
\pgfpathlineto{\pgfqpoint{2.842392in}{2.002857in}}%
\pgfpathlineto{\pgfqpoint{2.852820in}{2.005463in}}%
\pgfpathlineto{\pgfqpoint{2.863248in}{1.997534in}}%
\pgfpathlineto{\pgfqpoint{2.873676in}{1.979797in}}%
\pgfpathlineto{\pgfqpoint{2.884104in}{1.981332in}}%
\pgfpathlineto{\pgfqpoint{2.894533in}{1.980355in}}%
\pgfpathlineto{\pgfqpoint{2.904961in}{1.947223in}}%
\pgfpathlineto{\pgfqpoint{2.915389in}{1.961851in}}%
\pgfpathlineto{\pgfqpoint{2.925817in}{1.933898in}}%
\pgfpathlineto{\pgfqpoint{2.936245in}{1.939518in}}%
\pgfpathlineto{\pgfqpoint{2.946673in}{1.939308in}}%
\pgfpathlineto{\pgfqpoint{2.957102in}{1.917211in}}%
\pgfpathlineto{\pgfqpoint{2.967530in}{1.912813in}}%
\pgfpathlineto{\pgfqpoint{2.977958in}{1.903196in}}%
\pgfpathlineto{\pgfqpoint{2.988386in}{1.907574in}}%
\pgfpathlineto{\pgfqpoint{3.009242in}{1.876321in}}%
\pgfpathlineto{\pgfqpoint{3.019671in}{1.918916in}}%
\pgfpathlineto{\pgfqpoint{3.030099in}{1.882187in}}%
\pgfpathlineto{\pgfqpoint{3.040527in}{1.880956in}}%
\pgfpathlineto{\pgfqpoint{3.050955in}{1.913166in}}%
\pgfpathlineto{\pgfqpoint{3.061383in}{1.887205in}}%
\pgfpathlineto{\pgfqpoint{3.071812in}{1.869611in}}%
\pgfpathlineto{\pgfqpoint{3.082240in}{1.866151in}}%
\pgfpathlineto{\pgfqpoint{3.092668in}{1.857397in}}%
\pgfpathlineto{\pgfqpoint{3.103096in}{1.861975in}}%
\pgfpathlineto{\pgfqpoint{3.113524in}{1.839658in}}%
\pgfpathlineto{\pgfqpoint{3.123952in}{1.839011in}}%
\pgfpathlineto{\pgfqpoint{3.134381in}{1.831237in}}%
\pgfpathlineto{\pgfqpoint{3.144809in}{1.843333in}}%
\pgfpathlineto{\pgfqpoint{3.155237in}{1.836824in}}%
\pgfpathlineto{\pgfqpoint{3.165665in}{1.857289in}}%
\pgfpathlineto{\pgfqpoint{3.176093in}{1.838043in}}%
\pgfpathlineto{\pgfqpoint{3.186521in}{1.835804in}}%
\pgfpathlineto{\pgfqpoint{3.207378in}{1.807987in}}%
\pgfpathlineto{\pgfqpoint{3.217806in}{1.810368in}}%
\pgfpathlineto{\pgfqpoint{3.228234in}{1.820562in}}%
\pgfpathlineto{\pgfqpoint{3.238662in}{1.825604in}}%
\pgfpathlineto{\pgfqpoint{3.249090in}{1.807828in}}%
\pgfpathlineto{\pgfqpoint{3.259519in}{1.806586in}}%
\pgfpathlineto{\pgfqpoint{3.269947in}{1.780892in}}%
\pgfpathlineto{\pgfqpoint{3.280375in}{1.784828in}}%
\pgfpathlineto{\pgfqpoint{3.290803in}{1.800036in}}%
\pgfpathlineto{\pgfqpoint{3.301231in}{1.795765in}}%
\pgfpathlineto{\pgfqpoint{3.322088in}{1.799079in}}%
\pgfpathlineto{\pgfqpoint{3.332516in}{1.788056in}}%
\pgfpathlineto{\pgfqpoint{3.342944in}{1.788602in}}%
\pgfpathlineto{\pgfqpoint{3.353372in}{1.777212in}}%
\pgfpathlineto{\pgfqpoint{3.363800in}{1.773917in}}%
\pgfpathlineto{\pgfqpoint{3.374229in}{1.795271in}}%
\pgfpathlineto{\pgfqpoint{3.384657in}{1.786142in}}%
\pgfpathlineto{\pgfqpoint{3.395085in}{1.786848in}}%
\pgfpathlineto{\pgfqpoint{3.405513in}{1.792153in}}%
\pgfpathlineto{\pgfqpoint{3.415941in}{1.773229in}}%
\pgfpathlineto{\pgfqpoint{3.426369in}{1.783237in}}%
\pgfpathlineto{\pgfqpoint{3.436798in}{1.802158in}}%
\pgfpathlineto{\pgfqpoint{3.447226in}{1.787734in}}%
\pgfpathlineto{\pgfqpoint{3.457654in}{1.764140in}}%
\pgfpathlineto{\pgfqpoint{3.468082in}{1.767806in}}%
\pgfpathlineto{\pgfqpoint{3.478510in}{1.756675in}}%
\pgfpathlineto{\pgfqpoint{3.499367in}{1.757838in}}%
\pgfpathlineto{\pgfqpoint{3.509795in}{1.790942in}}%
\pgfpathlineto{\pgfqpoint{3.520223in}{1.787710in}}%
\pgfpathlineto{\pgfqpoint{3.530651in}{1.779266in}}%
\pgfpathlineto{\pgfqpoint{3.541079in}{1.752541in}}%
\pgfpathlineto{\pgfqpoint{3.551508in}{1.745052in}}%
\pgfpathlineto{\pgfqpoint{3.561936in}{1.741460in}}%
\pgfpathlineto{\pgfqpoint{3.572364in}{1.760626in}}%
\pgfpathlineto{\pgfqpoint{3.582792in}{1.768915in}}%
\pgfpathlineto{\pgfqpoint{3.593220in}{1.766495in}}%
\pgfpathlineto{\pgfqpoint{3.614077in}{1.771574in}}%
\pgfpathlineto{\pgfqpoint{3.634933in}{1.729565in}}%
\pgfpathlineto{\pgfqpoint{3.645361in}{1.737560in}}%
\pgfpathlineto{\pgfqpoint{3.666217in}{1.729054in}}%
\pgfpathlineto{\pgfqpoint{3.676646in}{1.727638in}}%
\pgfpathlineto{\pgfqpoint{3.687074in}{1.749729in}}%
\pgfpathlineto{\pgfqpoint{3.697502in}{1.731245in}}%
\pgfpathlineto{\pgfqpoint{3.707930in}{1.731592in}}%
\pgfpathlineto{\pgfqpoint{3.718358in}{1.721729in}}%
\pgfpathlineto{\pgfqpoint{3.728787in}{1.741549in}}%
\pgfpathlineto{\pgfqpoint{3.739215in}{1.712007in}}%
\pgfpathlineto{\pgfqpoint{3.749643in}{1.718964in}}%
\pgfpathlineto{\pgfqpoint{3.760071in}{1.715812in}}%
\pgfpathlineto{\pgfqpoint{3.770499in}{1.680290in}}%
\pgfpathlineto{\pgfqpoint{3.780927in}{1.716109in}}%
\pgfpathlineto{\pgfqpoint{3.791356in}{1.706700in}}%
\pgfpathlineto{\pgfqpoint{3.801784in}{1.749792in}}%
\pgfpathlineto{\pgfqpoint{3.812212in}{1.727607in}}%
\pgfpathlineto{\pgfqpoint{3.822640in}{1.744060in}}%
\pgfpathlineto{\pgfqpoint{3.843496in}{1.694985in}}%
\pgfpathlineto{\pgfqpoint{3.853925in}{1.687797in}}%
\pgfpathlineto{\pgfqpoint{3.864353in}{1.676744in}}%
\pgfpathlineto{\pgfqpoint{3.874781in}{1.647785in}}%
\pgfpathlineto{\pgfqpoint{3.885209in}{1.669537in}}%
\pgfpathlineto{\pgfqpoint{3.895637in}{1.678850in}}%
\pgfpathlineto{\pgfqpoint{3.906065in}{1.711188in}}%
\pgfpathlineto{\pgfqpoint{3.916494in}{1.677070in}}%
\pgfpathlineto{\pgfqpoint{3.937350in}{1.728916in}}%
\pgfpathlineto{\pgfqpoint{3.947778in}{1.717625in}}%
\pgfpathlineto{\pgfqpoint{3.958206in}{1.730802in}}%
\pgfpathlineto{\pgfqpoint{3.968635in}{1.720294in}}%
\pgfpathlineto{\pgfqpoint{3.979063in}{1.693422in}}%
\pgfpathlineto{\pgfqpoint{3.989491in}{1.684632in}}%
\pgfpathlineto{\pgfqpoint{3.999919in}{1.697840in}}%
\pgfpathlineto{\pgfqpoint{4.010347in}{1.718354in}}%
\pgfpathlineto{\pgfqpoint{4.020775in}{1.723132in}}%
\pgfpathlineto{\pgfqpoint{4.031204in}{1.696108in}}%
\pgfpathlineto{\pgfqpoint{4.041632in}{1.684258in}}%
\pgfpathlineto{\pgfqpoint{4.052060in}{1.656128in}}%
\pgfpathlineto{\pgfqpoint{4.062488in}{1.700233in}}%
\pgfpathlineto{\pgfqpoint{4.072916in}{1.704055in}}%
\pgfpathlineto{\pgfqpoint{4.083344in}{1.716403in}}%
\pgfpathlineto{\pgfqpoint{4.093773in}{1.686270in}}%
\pgfpathlineto{\pgfqpoint{4.104201in}{1.686051in}}%
\pgfpathlineto{\pgfqpoint{4.114629in}{1.670020in}}%
\pgfpathlineto{\pgfqpoint{4.125057in}{1.693164in}}%
\pgfpathlineto{\pgfqpoint{4.135485in}{1.672059in}}%
\pgfpathlineto{\pgfqpoint{4.145913in}{1.700722in}}%
\pgfpathlineto{\pgfqpoint{4.156342in}{1.716832in}}%
\pgfpathlineto{\pgfqpoint{4.166770in}{1.705969in}}%
\pgfpathlineto{\pgfqpoint{4.177198in}{1.692718in}}%
\pgfpathlineto{\pgfqpoint{4.187626in}{1.727825in}}%
\pgfpathlineto{\pgfqpoint{4.198054in}{1.716251in}}%
\pgfpathlineto{\pgfqpoint{4.208483in}{1.724326in}}%
\pgfpathlineto{\pgfqpoint{4.218911in}{1.676493in}}%
\pgfpathlineto{\pgfqpoint{4.229339in}{1.737802in}}%
\pgfpathlineto{\pgfqpoint{4.239767in}{1.709983in}}%
\pgfpathlineto{\pgfqpoint{4.250195in}{1.693189in}}%
\pgfpathlineto{\pgfqpoint{4.260623in}{1.712976in}}%
\pgfpathlineto{\pgfqpoint{4.271052in}{1.712790in}}%
\pgfpathlineto{\pgfqpoint{4.281480in}{1.688090in}}%
\pgfpathlineto{\pgfqpoint{4.291908in}{1.714021in}}%
\pgfpathlineto{\pgfqpoint{4.302336in}{1.675826in}}%
\pgfpathlineto{\pgfqpoint{4.312764in}{1.707200in}}%
\pgfpathlineto{\pgfqpoint{4.323192in}{1.684140in}}%
\pgfpathlineto{\pgfqpoint{4.333621in}{1.673019in}}%
\pgfpathlineto{\pgfqpoint{4.344049in}{1.671054in}}%
\pgfpathlineto{\pgfqpoint{4.354477in}{1.697477in}}%
\pgfpathlineto{\pgfqpoint{4.364905in}{1.692102in}}%
\pgfpathlineto{\pgfqpoint{4.375333in}{1.711874in}}%
\pgfpathlineto{\pgfqpoint{4.385761in}{1.665376in}}%
\pgfpathlineto{\pgfqpoint{4.396190in}{1.726802in}}%
\pgfpathlineto{\pgfqpoint{4.406618in}{1.688905in}}%
\pgfpathlineto{\pgfqpoint{4.417046in}{1.688572in}}%
\pgfpathlineto{\pgfqpoint{4.427474in}{1.693922in}}%
\pgfpathlineto{\pgfqpoint{4.437902in}{1.635457in}}%
\pgfpathlineto{\pgfqpoint{4.448331in}{1.607431in}}%
\pgfpathlineto{\pgfqpoint{4.458759in}{1.634742in}}%
\pgfpathlineto{\pgfqpoint{4.469187in}{1.656777in}}%
\pgfpathlineto{\pgfqpoint{4.479615in}{1.624408in}}%
\pgfpathlineto{\pgfqpoint{4.490043in}{1.629143in}}%
\pgfpathlineto{\pgfqpoint{4.500471in}{1.615978in}}%
\pgfpathlineto{\pgfqpoint{4.510900in}{1.663327in}}%
\pgfpathlineto{\pgfqpoint{4.521328in}{1.660458in}}%
\pgfpathlineto{\pgfqpoint{4.531756in}{1.664183in}}%
\pgfpathlineto{\pgfqpoint{4.542184in}{1.724511in}}%
\pgfpathlineto{\pgfqpoint{4.552612in}{1.715179in}}%
\pgfpathlineto{\pgfqpoint{4.563040in}{1.693509in}}%
\pgfpathlineto{\pgfqpoint{4.573469in}{1.677027in}}%
\pgfpathlineto{\pgfqpoint{4.583897in}{1.670082in}}%
\pgfpathlineto{\pgfqpoint{4.594325in}{1.642476in}}%
\pgfpathlineto{\pgfqpoint{4.604753in}{1.654851in}}%
\pgfpathlineto{\pgfqpoint{4.615181in}{1.634570in}}%
\pgfpathlineto{\pgfqpoint{4.625610in}{1.670115in}}%
\pgfpathlineto{\pgfqpoint{4.636038in}{1.651743in}}%
\pgfpathlineto{\pgfqpoint{4.646466in}{1.682824in}}%
\pgfpathlineto{\pgfqpoint{4.656894in}{1.691628in}}%
\pgfpathlineto{\pgfqpoint{4.667322in}{1.664552in}}%
\pgfpathlineto{\pgfqpoint{4.677750in}{1.678643in}}%
\pgfpathlineto{\pgfqpoint{4.688179in}{1.709890in}}%
\pgfpathlineto{\pgfqpoint{4.698607in}{1.681868in}}%
\pgfpathlineto{\pgfqpoint{4.709035in}{1.680099in}}%
\pgfpathlineto{\pgfqpoint{4.719463in}{1.722693in}}%
\pgfpathlineto{\pgfqpoint{4.729891in}{1.644648in}}%
\pgfpathlineto{\pgfqpoint{4.740319in}{1.623112in}}%
\pgfpathlineto{\pgfqpoint{4.750748in}{1.681924in}}%
\pgfpathlineto{\pgfqpoint{4.761176in}{1.672178in}}%
\pgfpathlineto{\pgfqpoint{4.771604in}{1.641539in}}%
\pgfpathlineto{\pgfqpoint{4.771604in}{1.641539in}}%
\pgfusepath{stroke}%
\end{pgfscope}%
\begin{pgfscope}%
\pgfpathrectangle{\pgfqpoint{0.610762in}{0.961156in}}{\pgfqpoint{4.171270in}{2.577986in}} %
\pgfusepath{clip}%
\pgfsetroundcap%
\pgfsetroundjoin%
\pgfsetlinewidth{1.756562pt}%
\definecolor{currentstroke}{rgb}{0.168627,0.670588,0.494118}%
\pgfsetstrokecolor{currentstroke}%
\pgfsetstrokeopacity{0.800000}%
\pgfsetdash{}{0pt}%
\pgfpathmoveto{\pgfqpoint{0.610762in}{3.095349in}}%
\pgfpathlineto{\pgfqpoint{0.621191in}{3.092222in}}%
\pgfpathlineto{\pgfqpoint{0.631619in}{3.155917in}}%
\pgfpathlineto{\pgfqpoint{0.642047in}{3.132882in}}%
\pgfpathlineto{\pgfqpoint{0.652475in}{3.177737in}}%
\pgfpathlineto{\pgfqpoint{0.662903in}{3.115242in}}%
\pgfpathlineto{\pgfqpoint{0.673331in}{3.116082in}}%
\pgfpathlineto{\pgfqpoint{0.683760in}{3.094640in}}%
\pgfpathlineto{\pgfqpoint{0.694188in}{3.159201in}}%
\pgfpathlineto{\pgfqpoint{0.704616in}{3.107547in}}%
\pgfpathlineto{\pgfqpoint{0.715044in}{3.078967in}}%
\pgfpathlineto{\pgfqpoint{0.725472in}{3.114063in}}%
\pgfpathlineto{\pgfqpoint{0.735900in}{3.111234in}}%
\pgfpathlineto{\pgfqpoint{0.746329in}{3.124810in}}%
\pgfpathlineto{\pgfqpoint{0.756757in}{3.177281in}}%
\pgfpathlineto{\pgfqpoint{0.767185in}{3.120426in}}%
\pgfpathlineto{\pgfqpoint{0.777613in}{3.128284in}}%
\pgfpathlineto{\pgfqpoint{0.788041in}{3.145287in}}%
\pgfpathlineto{\pgfqpoint{0.798470in}{3.085547in}}%
\pgfpathlineto{\pgfqpoint{0.808898in}{3.138289in}}%
\pgfpathlineto{\pgfqpoint{0.819326in}{3.103155in}}%
\pgfpathlineto{\pgfqpoint{0.829754in}{3.107625in}}%
\pgfpathlineto{\pgfqpoint{0.840182in}{3.079660in}}%
\pgfpathlineto{\pgfqpoint{0.850610in}{3.067641in}}%
\pgfpathlineto{\pgfqpoint{0.861039in}{2.999426in}}%
\pgfpathlineto{\pgfqpoint{0.871467in}{3.018001in}}%
\pgfpathlineto{\pgfqpoint{0.881895in}{3.073694in}}%
\pgfpathlineto{\pgfqpoint{0.892323in}{3.069415in}}%
\pgfpathlineto{\pgfqpoint{0.902751in}{3.053042in}}%
\pgfpathlineto{\pgfqpoint{0.913179in}{3.084826in}}%
\pgfpathlineto{\pgfqpoint{0.923608in}{3.106629in}}%
\pgfpathlineto{\pgfqpoint{0.934036in}{3.027354in}}%
\pgfpathlineto{\pgfqpoint{0.944464in}{3.112501in}}%
\pgfpathlineto{\pgfqpoint{0.954892in}{3.124485in}}%
\pgfpathlineto{\pgfqpoint{0.965320in}{3.078372in}}%
\pgfpathlineto{\pgfqpoint{0.975748in}{3.049214in}}%
\pgfpathlineto{\pgfqpoint{0.986177in}{3.039250in}}%
\pgfpathlineto{\pgfqpoint{0.996605in}{3.063282in}}%
\pgfpathlineto{\pgfqpoint{1.007033in}{3.092696in}}%
\pgfpathlineto{\pgfqpoint{1.017461in}{3.094344in}}%
\pgfpathlineto{\pgfqpoint{1.027889in}{3.168033in}}%
\pgfpathlineto{\pgfqpoint{1.038318in}{3.096414in}}%
\pgfpathlineto{\pgfqpoint{1.048746in}{3.099641in}}%
\pgfpathlineto{\pgfqpoint{1.059174in}{3.037655in}}%
\pgfpathlineto{\pgfqpoint{1.069602in}{3.037732in}}%
\pgfpathlineto{\pgfqpoint{1.080030in}{3.030582in}}%
\pgfpathlineto{\pgfqpoint{1.090458in}{3.082788in}}%
\pgfpathlineto{\pgfqpoint{1.100887in}{3.093665in}}%
\pgfpathlineto{\pgfqpoint{1.111315in}{2.992233in}}%
\pgfpathlineto{\pgfqpoint{1.121743in}{3.030959in}}%
\pgfpathlineto{\pgfqpoint{1.132171in}{2.983960in}}%
\pgfpathlineto{\pgfqpoint{1.142599in}{3.047657in}}%
\pgfpathlineto{\pgfqpoint{1.153027in}{3.019802in}}%
\pgfpathlineto{\pgfqpoint{1.163456in}{3.028811in}}%
\pgfpathlineto{\pgfqpoint{1.173884in}{2.945420in}}%
\pgfpathlineto{\pgfqpoint{1.184312in}{2.917614in}}%
\pgfpathlineto{\pgfqpoint{1.194740in}{2.916148in}}%
\pgfpathlineto{\pgfqpoint{1.205168in}{2.946575in}}%
\pgfpathlineto{\pgfqpoint{1.226025in}{3.040935in}}%
\pgfpathlineto{\pgfqpoint{1.236453in}{3.066200in}}%
\pgfpathlineto{\pgfqpoint{1.246881in}{3.037774in}}%
\pgfpathlineto{\pgfqpoint{1.257309in}{3.057434in}}%
\pgfpathlineto{\pgfqpoint{1.267737in}{2.996495in}}%
\pgfpathlineto{\pgfqpoint{1.278166in}{3.036212in}}%
\pgfpathlineto{\pgfqpoint{1.288594in}{2.994821in}}%
\pgfpathlineto{\pgfqpoint{1.299022in}{2.999939in}}%
\pgfpathlineto{\pgfqpoint{1.309450in}{3.064115in}}%
\pgfpathlineto{\pgfqpoint{1.319878in}{3.055118in}}%
\pgfpathlineto{\pgfqpoint{1.330306in}{3.089502in}}%
\pgfpathlineto{\pgfqpoint{1.340735in}{3.071413in}}%
\pgfpathlineto{\pgfqpoint{1.351163in}{3.077260in}}%
\pgfpathlineto{\pgfqpoint{1.361591in}{3.102344in}}%
\pgfpathlineto{\pgfqpoint{1.372019in}{3.019017in}}%
\pgfpathlineto{\pgfqpoint{1.382447in}{3.059578in}}%
\pgfpathlineto{\pgfqpoint{1.392875in}{3.071157in}}%
\pgfpathlineto{\pgfqpoint{1.403304in}{3.132684in}}%
\pgfpathlineto{\pgfqpoint{1.413732in}{3.134337in}}%
\pgfpathlineto{\pgfqpoint{1.424160in}{3.109627in}}%
\pgfpathlineto{\pgfqpoint{1.445016in}{3.083899in}}%
\pgfpathlineto{\pgfqpoint{1.455444in}{3.126566in}}%
\pgfpathlineto{\pgfqpoint{1.465873in}{3.139620in}}%
\pgfpathlineto{\pgfqpoint{1.476301in}{3.188446in}}%
\pgfpathlineto{\pgfqpoint{1.486729in}{3.180728in}}%
\pgfpathlineto{\pgfqpoint{1.497157in}{3.143907in}}%
\pgfpathlineto{\pgfqpoint{1.507585in}{3.201198in}}%
\pgfpathlineto{\pgfqpoint{1.518014in}{3.180408in}}%
\pgfpathlineto{\pgfqpoint{1.528442in}{3.149808in}}%
\pgfpathlineto{\pgfqpoint{1.538870in}{3.156054in}}%
\pgfpathlineto{\pgfqpoint{1.549298in}{3.117654in}}%
\pgfpathlineto{\pgfqpoint{1.559726in}{3.122494in}}%
\pgfpathlineto{\pgfqpoint{1.570154in}{3.101301in}}%
\pgfpathlineto{\pgfqpoint{1.580583in}{3.032058in}}%
\pgfpathlineto{\pgfqpoint{1.591011in}{3.109518in}}%
\pgfpathlineto{\pgfqpoint{1.601439in}{3.101395in}}%
\pgfpathlineto{\pgfqpoint{1.611867in}{3.153268in}}%
\pgfpathlineto{\pgfqpoint{1.622295in}{3.195644in}}%
\pgfpathlineto{\pgfqpoint{1.632723in}{3.110289in}}%
\pgfpathlineto{\pgfqpoint{1.643152in}{3.118922in}}%
\pgfpathlineto{\pgfqpoint{1.653580in}{3.060741in}}%
\pgfpathlineto{\pgfqpoint{1.664008in}{3.014210in}}%
\pgfpathlineto{\pgfqpoint{1.674436in}{3.018337in}}%
\pgfpathlineto{\pgfqpoint{1.684864in}{3.102111in}}%
\pgfpathlineto{\pgfqpoint{1.695292in}{3.064638in}}%
\pgfpathlineto{\pgfqpoint{1.705721in}{3.099416in}}%
\pgfpathlineto{\pgfqpoint{1.716149in}{3.109416in}}%
\pgfpathlineto{\pgfqpoint{1.726577in}{3.083838in}}%
\pgfpathlineto{\pgfqpoint{1.737005in}{3.010450in}}%
\pgfpathlineto{\pgfqpoint{1.747433in}{2.970767in}}%
\pgfpathlineto{\pgfqpoint{1.757862in}{2.975293in}}%
\pgfpathlineto{\pgfqpoint{1.768290in}{3.045143in}}%
\pgfpathlineto{\pgfqpoint{1.778718in}{3.072722in}}%
\pgfpathlineto{\pgfqpoint{1.789146in}{3.075271in}}%
\pgfpathlineto{\pgfqpoint{1.799574in}{3.152230in}}%
\pgfpathlineto{\pgfqpoint{1.810002in}{3.052481in}}%
\pgfpathlineto{\pgfqpoint{1.820431in}{3.102139in}}%
\pgfpathlineto{\pgfqpoint{1.830859in}{3.115540in}}%
\pgfpathlineto{\pgfqpoint{1.841287in}{3.058880in}}%
\pgfpathlineto{\pgfqpoint{1.851715in}{3.082467in}}%
\pgfpathlineto{\pgfqpoint{1.862143in}{3.042785in}}%
\pgfpathlineto{\pgfqpoint{1.872571in}{3.041631in}}%
\pgfpathlineto{\pgfqpoint{1.883000in}{3.050494in}}%
\pgfpathlineto{\pgfqpoint{1.893428in}{3.076891in}}%
\pgfpathlineto{\pgfqpoint{1.903856in}{3.079022in}}%
\pgfpathlineto{\pgfqpoint{1.914284in}{3.116741in}}%
\pgfpathlineto{\pgfqpoint{1.924712in}{3.122767in}}%
\pgfpathlineto{\pgfqpoint{1.935141in}{3.099692in}}%
\pgfpathlineto{\pgfqpoint{1.945569in}{3.124506in}}%
\pgfpathlineto{\pgfqpoint{1.955997in}{3.133308in}}%
\pgfpathlineto{\pgfqpoint{1.966425in}{3.114560in}}%
\pgfpathlineto{\pgfqpoint{1.976853in}{3.089369in}}%
\pgfpathlineto{\pgfqpoint{1.987281in}{3.138394in}}%
\pgfpathlineto{\pgfqpoint{1.997710in}{3.068684in}}%
\pgfpathlineto{\pgfqpoint{2.008138in}{3.090829in}}%
\pgfpathlineto{\pgfqpoint{2.018566in}{3.087133in}}%
\pgfpathlineto{\pgfqpoint{2.028994in}{3.056238in}}%
\pgfpathlineto{\pgfqpoint{2.039422in}{3.069858in}}%
\pgfpathlineto{\pgfqpoint{2.049850in}{3.106332in}}%
\pgfpathlineto{\pgfqpoint{2.060279in}{3.106322in}}%
\pgfpathlineto{\pgfqpoint{2.070707in}{3.131920in}}%
\pgfpathlineto{\pgfqpoint{2.081135in}{3.110175in}}%
\pgfpathlineto{\pgfqpoint{2.091563in}{3.101765in}}%
\pgfpathlineto{\pgfqpoint{2.101991in}{3.177933in}}%
\pgfpathlineto{\pgfqpoint{2.112419in}{3.131225in}}%
\pgfpathlineto{\pgfqpoint{2.122848in}{3.120284in}}%
\pgfpathlineto{\pgfqpoint{2.133276in}{3.124411in}}%
\pgfpathlineto{\pgfqpoint{2.143704in}{3.074604in}}%
\pgfpathlineto{\pgfqpoint{2.154132in}{3.101550in}}%
\pgfpathlineto{\pgfqpoint{2.164560in}{3.104924in}}%
\pgfpathlineto{\pgfqpoint{2.174989in}{3.098689in}}%
\pgfpathlineto{\pgfqpoint{2.185417in}{3.089994in}}%
\pgfpathlineto{\pgfqpoint{2.195845in}{3.085163in}}%
\pgfpathlineto{\pgfqpoint{2.206273in}{3.087016in}}%
\pgfpathlineto{\pgfqpoint{2.216701in}{3.097299in}}%
\pgfpathlineto{\pgfqpoint{2.237558in}{3.028051in}}%
\pgfpathlineto{\pgfqpoint{2.247986in}{3.030970in}}%
\pgfpathlineto{\pgfqpoint{2.258414in}{3.017295in}}%
\pgfpathlineto{\pgfqpoint{2.268842in}{2.972424in}}%
\pgfpathlineto{\pgfqpoint{2.279270in}{2.940576in}}%
\pgfpathlineto{\pgfqpoint{2.289698in}{2.967032in}}%
\pgfpathlineto{\pgfqpoint{2.300127in}{2.970424in}}%
\pgfpathlineto{\pgfqpoint{2.310555in}{2.956569in}}%
\pgfpathlineto{\pgfqpoint{2.320983in}{2.946682in}}%
\pgfpathlineto{\pgfqpoint{2.331411in}{2.947328in}}%
\pgfpathlineto{\pgfqpoint{2.341839in}{2.940780in}}%
\pgfpathlineto{\pgfqpoint{2.352267in}{2.927751in}}%
\pgfpathlineto{\pgfqpoint{2.362696in}{2.934747in}}%
\pgfpathlineto{\pgfqpoint{2.373124in}{2.911423in}}%
\pgfpathlineto{\pgfqpoint{2.383552in}{2.948480in}}%
\pgfpathlineto{\pgfqpoint{2.404408in}{2.902936in}}%
\pgfpathlineto{\pgfqpoint{2.414837in}{2.906345in}}%
\pgfpathlineto{\pgfqpoint{2.425265in}{2.894984in}}%
\pgfpathlineto{\pgfqpoint{2.435693in}{2.930847in}}%
\pgfpathlineto{\pgfqpoint{2.446121in}{2.924445in}}%
\pgfpathlineto{\pgfqpoint{2.456549in}{2.943311in}}%
\pgfpathlineto{\pgfqpoint{2.466977in}{2.976987in}}%
\pgfpathlineto{\pgfqpoint{2.477406in}{2.974979in}}%
\pgfpathlineto{\pgfqpoint{2.487834in}{2.954869in}}%
\pgfpathlineto{\pgfqpoint{2.498262in}{2.947622in}}%
\pgfpathlineto{\pgfqpoint{2.508690in}{2.929216in}}%
\pgfpathlineto{\pgfqpoint{2.519118in}{2.893690in}}%
\pgfpathlineto{\pgfqpoint{2.529546in}{2.904795in}}%
\pgfpathlineto{\pgfqpoint{2.539975in}{2.860757in}}%
\pgfpathlineto{\pgfqpoint{2.550403in}{2.838775in}}%
\pgfpathlineto{\pgfqpoint{2.560831in}{2.865190in}}%
\pgfpathlineto{\pgfqpoint{2.571259in}{2.801523in}}%
\pgfpathlineto{\pgfqpoint{2.581687in}{2.788605in}}%
\pgfpathlineto{\pgfqpoint{2.592115in}{2.782096in}}%
\pgfpathlineto{\pgfqpoint{2.602544in}{2.738114in}}%
\pgfpathlineto{\pgfqpoint{2.612972in}{2.760636in}}%
\pgfpathlineto{\pgfqpoint{2.623400in}{2.743104in}}%
\pgfpathlineto{\pgfqpoint{2.633828in}{2.683910in}}%
\pgfpathlineto{\pgfqpoint{2.644256in}{2.679585in}}%
\pgfpathlineto{\pgfqpoint{2.654685in}{2.640143in}}%
\pgfpathlineto{\pgfqpoint{2.665113in}{2.588418in}}%
\pgfpathlineto{\pgfqpoint{2.675541in}{2.485228in}}%
\pgfpathlineto{\pgfqpoint{2.685969in}{2.337019in}}%
\pgfpathlineto{\pgfqpoint{2.696397in}{2.361565in}}%
\pgfpathlineto{\pgfqpoint{2.706825in}{2.229249in}}%
\pgfpathlineto{\pgfqpoint{2.727682in}{2.073299in}}%
\pgfpathlineto{\pgfqpoint{2.738110in}{2.032047in}}%
\pgfpathlineto{\pgfqpoint{2.748538in}{2.024004in}}%
\pgfpathlineto{\pgfqpoint{2.758966in}{2.018826in}}%
\pgfpathlineto{\pgfqpoint{2.769394in}{2.039025in}}%
\pgfpathlineto{\pgfqpoint{2.779823in}{1.999732in}}%
\pgfpathlineto{\pgfqpoint{2.790251in}{1.982648in}}%
\pgfpathlineto{\pgfqpoint{2.800679in}{1.949369in}}%
\pgfpathlineto{\pgfqpoint{2.811107in}{1.924996in}}%
\pgfpathlineto{\pgfqpoint{2.821535in}{1.934480in}}%
\pgfpathlineto{\pgfqpoint{2.831964in}{1.917956in}}%
\pgfpathlineto{\pgfqpoint{2.842392in}{1.912517in}}%
\pgfpathlineto{\pgfqpoint{2.852820in}{1.867579in}}%
\pgfpathlineto{\pgfqpoint{2.863248in}{1.847192in}}%
\pgfpathlineto{\pgfqpoint{2.873676in}{1.851594in}}%
\pgfpathlineto{\pgfqpoint{2.884104in}{1.822344in}}%
\pgfpathlineto{\pgfqpoint{2.894533in}{1.819699in}}%
\pgfpathlineto{\pgfqpoint{2.904961in}{1.776535in}}%
\pgfpathlineto{\pgfqpoint{2.915389in}{1.787041in}}%
\pgfpathlineto{\pgfqpoint{2.925817in}{1.769945in}}%
\pgfpathlineto{\pgfqpoint{2.936245in}{1.796988in}}%
\pgfpathlineto{\pgfqpoint{2.946673in}{1.796962in}}%
\pgfpathlineto{\pgfqpoint{2.957102in}{1.771099in}}%
\pgfpathlineto{\pgfqpoint{2.967530in}{1.775443in}}%
\pgfpathlineto{\pgfqpoint{2.977958in}{1.767230in}}%
\pgfpathlineto{\pgfqpoint{2.988386in}{1.719573in}}%
\pgfpathlineto{\pgfqpoint{2.998814in}{1.729961in}}%
\pgfpathlineto{\pgfqpoint{3.009242in}{1.738066in}}%
\pgfpathlineto{\pgfqpoint{3.019671in}{1.706731in}}%
\pgfpathlineto{\pgfqpoint{3.030099in}{1.722890in}}%
\pgfpathlineto{\pgfqpoint{3.040527in}{1.712956in}}%
\pgfpathlineto{\pgfqpoint{3.050955in}{1.697753in}}%
\pgfpathlineto{\pgfqpoint{3.061383in}{1.719353in}}%
\pgfpathlineto{\pgfqpoint{3.071812in}{1.706977in}}%
\pgfpathlineto{\pgfqpoint{3.082240in}{1.675839in}}%
\pgfpathlineto{\pgfqpoint{3.092668in}{1.667703in}}%
\pgfpathlineto{\pgfqpoint{3.103096in}{1.692668in}}%
\pgfpathlineto{\pgfqpoint{3.113524in}{1.676481in}}%
\pgfpathlineto{\pgfqpoint{3.123952in}{1.637009in}}%
\pgfpathlineto{\pgfqpoint{3.144809in}{1.692427in}}%
\pgfpathlineto{\pgfqpoint{3.176093in}{1.616258in}}%
\pgfpathlineto{\pgfqpoint{3.186521in}{1.625524in}}%
\pgfpathlineto{\pgfqpoint{3.196950in}{1.642085in}}%
\pgfpathlineto{\pgfqpoint{3.207378in}{1.617340in}}%
\pgfpathlineto{\pgfqpoint{3.217806in}{1.611011in}}%
\pgfpathlineto{\pgfqpoint{3.228234in}{1.579080in}}%
\pgfpathlineto{\pgfqpoint{3.238662in}{1.598139in}}%
\pgfpathlineto{\pgfqpoint{3.249090in}{1.631363in}}%
\pgfpathlineto{\pgfqpoint{3.259519in}{1.551104in}}%
\pgfpathlineto{\pgfqpoint{3.280375in}{1.593292in}}%
\pgfpathlineto{\pgfqpoint{3.290803in}{1.569240in}}%
\pgfpathlineto{\pgfqpoint{3.301231in}{1.559875in}}%
\pgfpathlineto{\pgfqpoint{3.311660in}{1.578598in}}%
\pgfpathlineto{\pgfqpoint{3.322088in}{1.605431in}}%
\pgfpathlineto{\pgfqpoint{3.332516in}{1.619688in}}%
\pgfpathlineto{\pgfqpoint{3.342944in}{1.565325in}}%
\pgfpathlineto{\pgfqpoint{3.353372in}{1.570692in}}%
\pgfpathlineto{\pgfqpoint{3.363800in}{1.547427in}}%
\pgfpathlineto{\pgfqpoint{3.374229in}{1.550758in}}%
\pgfpathlineto{\pgfqpoint{3.384657in}{1.590941in}}%
\pgfpathlineto{\pgfqpoint{3.395085in}{1.536000in}}%
\pgfpathlineto{\pgfqpoint{3.405513in}{1.512258in}}%
\pgfpathlineto{\pgfqpoint{3.415941in}{1.573210in}}%
\pgfpathlineto{\pgfqpoint{3.426369in}{1.578383in}}%
\pgfpathlineto{\pgfqpoint{3.436798in}{1.548166in}}%
\pgfpathlineto{\pgfqpoint{3.447226in}{1.585586in}}%
\pgfpathlineto{\pgfqpoint{3.457654in}{1.564985in}}%
\pgfpathlineto{\pgfqpoint{3.468082in}{1.571850in}}%
\pgfpathlineto{\pgfqpoint{3.478510in}{1.547116in}}%
\pgfpathlineto{\pgfqpoint{3.488938in}{1.623587in}}%
\pgfpathlineto{\pgfqpoint{3.499367in}{1.563917in}}%
\pgfpathlineto{\pgfqpoint{3.509795in}{1.605513in}}%
\pgfpathlineto{\pgfqpoint{3.520223in}{1.569048in}}%
\pgfpathlineto{\pgfqpoint{3.530651in}{1.611327in}}%
\pgfpathlineto{\pgfqpoint{3.541079in}{1.573693in}}%
\pgfpathlineto{\pgfqpoint{3.551508in}{1.555850in}}%
\pgfpathlineto{\pgfqpoint{3.561936in}{1.514070in}}%
\pgfpathlineto{\pgfqpoint{3.572364in}{1.529259in}}%
\pgfpathlineto{\pgfqpoint{3.593220in}{1.623116in}}%
\pgfpathlineto{\pgfqpoint{3.603648in}{1.587869in}}%
\pgfpathlineto{\pgfqpoint{3.614077in}{1.584282in}}%
\pgfpathlineto{\pgfqpoint{3.624505in}{1.577997in}}%
\pgfpathlineto{\pgfqpoint{3.634933in}{1.544057in}}%
\pgfpathlineto{\pgfqpoint{3.645361in}{1.580885in}}%
\pgfpathlineto{\pgfqpoint{3.655789in}{1.520000in}}%
\pgfpathlineto{\pgfqpoint{3.666217in}{1.510006in}}%
\pgfpathlineto{\pgfqpoint{3.676646in}{1.522273in}}%
\pgfpathlineto{\pgfqpoint{3.687074in}{1.563053in}}%
\pgfpathlineto{\pgfqpoint{3.697502in}{1.597546in}}%
\pgfpathlineto{\pgfqpoint{3.707930in}{1.506606in}}%
\pgfpathlineto{\pgfqpoint{3.718358in}{1.576909in}}%
\pgfpathlineto{\pgfqpoint{3.728787in}{1.527561in}}%
\pgfpathlineto{\pgfqpoint{3.739215in}{1.542064in}}%
\pgfpathlineto{\pgfqpoint{3.749643in}{1.595340in}}%
\pgfpathlineto{\pgfqpoint{3.760071in}{1.587090in}}%
\pgfpathlineto{\pgfqpoint{3.770499in}{1.598017in}}%
\pgfpathlineto{\pgfqpoint{3.791356in}{1.553741in}}%
\pgfpathlineto{\pgfqpoint{3.812212in}{1.576665in}}%
\pgfpathlineto{\pgfqpoint{3.822640in}{1.543518in}}%
\pgfpathlineto{\pgfqpoint{3.833068in}{1.534122in}}%
\pgfpathlineto{\pgfqpoint{3.843496in}{1.575338in}}%
\pgfpathlineto{\pgfqpoint{3.853925in}{1.516381in}}%
\pgfpathlineto{\pgfqpoint{3.864353in}{1.521087in}}%
\pgfpathlineto{\pgfqpoint{3.874781in}{1.561802in}}%
\pgfpathlineto{\pgfqpoint{3.885209in}{1.563445in}}%
\pgfpathlineto{\pgfqpoint{3.895637in}{1.515288in}}%
\pgfpathlineto{\pgfqpoint{3.906065in}{1.539571in}}%
\pgfpathlineto{\pgfqpoint{3.916494in}{1.531978in}}%
\pgfpathlineto{\pgfqpoint{3.937350in}{1.557469in}}%
\pgfpathlineto{\pgfqpoint{3.947778in}{1.529051in}}%
\pgfpathlineto{\pgfqpoint{3.958206in}{1.576566in}}%
\pgfpathlineto{\pgfqpoint{3.968635in}{1.482036in}}%
\pgfpathlineto{\pgfqpoint{3.979063in}{1.504379in}}%
\pgfpathlineto{\pgfqpoint{3.989491in}{1.491456in}}%
\pgfpathlineto{\pgfqpoint{3.999919in}{1.573348in}}%
\pgfpathlineto{\pgfqpoint{4.010347in}{1.589781in}}%
\pgfpathlineto{\pgfqpoint{4.020775in}{1.570600in}}%
\pgfpathlineto{\pgfqpoint{4.031204in}{1.541659in}}%
\pgfpathlineto{\pgfqpoint{4.041632in}{1.497919in}}%
\pgfpathlineto{\pgfqpoint{4.052060in}{1.528961in}}%
\pgfpathlineto{\pgfqpoint{4.062488in}{1.539519in}}%
\pgfpathlineto{\pgfqpoint{4.072916in}{1.567408in}}%
\pgfpathlineto{\pgfqpoint{4.083344in}{1.582173in}}%
\pgfpathlineto{\pgfqpoint{4.093773in}{1.576978in}}%
\pgfpathlineto{\pgfqpoint{4.104201in}{1.475893in}}%
\pgfpathlineto{\pgfqpoint{4.114629in}{1.530868in}}%
\pgfpathlineto{\pgfqpoint{4.125057in}{1.485265in}}%
\pgfpathlineto{\pgfqpoint{4.145913in}{1.523289in}}%
\pgfpathlineto{\pgfqpoint{4.156342in}{1.584867in}}%
\pgfpathlineto{\pgfqpoint{4.166770in}{1.484502in}}%
\pgfpathlineto{\pgfqpoint{4.177198in}{1.542777in}}%
\pgfpathlineto{\pgfqpoint{4.198054in}{1.514196in}}%
\pgfpathlineto{\pgfqpoint{4.208483in}{1.517600in}}%
\pgfpathlineto{\pgfqpoint{4.218911in}{1.550271in}}%
\pgfpathlineto{\pgfqpoint{4.229339in}{1.491088in}}%
\pgfpathlineto{\pgfqpoint{4.239767in}{1.517208in}}%
\pgfpathlineto{\pgfqpoint{4.250195in}{1.488739in}}%
\pgfpathlineto{\pgfqpoint{4.260623in}{1.547540in}}%
\pgfpathlineto{\pgfqpoint{4.271052in}{1.492969in}}%
\pgfpathlineto{\pgfqpoint{4.281480in}{1.514714in}}%
\pgfpathlineto{\pgfqpoint{4.291908in}{1.552653in}}%
\pgfpathlineto{\pgfqpoint{4.302336in}{1.502148in}}%
\pgfpathlineto{\pgfqpoint{4.312764in}{1.507993in}}%
\pgfpathlineto{\pgfqpoint{4.323192in}{1.567961in}}%
\pgfpathlineto{\pgfqpoint{4.333621in}{1.564339in}}%
\pgfpathlineto{\pgfqpoint{4.344049in}{1.512104in}}%
\pgfpathlineto{\pgfqpoint{4.354477in}{1.531398in}}%
\pgfpathlineto{\pgfqpoint{4.364905in}{1.496262in}}%
\pgfpathlineto{\pgfqpoint{4.375333in}{1.505874in}}%
\pgfpathlineto{\pgfqpoint{4.385761in}{1.612671in}}%
\pgfpathlineto{\pgfqpoint{4.396190in}{1.596970in}}%
\pgfpathlineto{\pgfqpoint{4.406618in}{1.599857in}}%
\pgfpathlineto{\pgfqpoint{4.417046in}{1.570860in}}%
\pgfpathlineto{\pgfqpoint{4.427474in}{1.516520in}}%
\pgfpathlineto{\pgfqpoint{4.437902in}{1.545121in}}%
\pgfpathlineto{\pgfqpoint{4.448331in}{1.526718in}}%
\pgfpathlineto{\pgfqpoint{4.458759in}{1.514849in}}%
\pgfpathlineto{\pgfqpoint{4.469187in}{1.519369in}}%
\pgfpathlineto{\pgfqpoint{4.479615in}{1.430458in}}%
\pgfpathlineto{\pgfqpoint{4.490043in}{1.498382in}}%
\pgfpathlineto{\pgfqpoint{4.500471in}{1.446940in}}%
\pgfpathlineto{\pgfqpoint{4.510900in}{1.367991in}}%
\pgfpathlineto{\pgfqpoint{4.521328in}{1.484668in}}%
\pgfpathlineto{\pgfqpoint{4.531756in}{1.489158in}}%
\pgfpathlineto{\pgfqpoint{4.542184in}{1.382140in}}%
\pgfpathlineto{\pgfqpoint{4.552612in}{1.461916in}}%
\pgfpathlineto{\pgfqpoint{4.563040in}{1.486867in}}%
\pgfpathlineto{\pgfqpoint{4.573469in}{1.528177in}}%
\pgfpathlineto{\pgfqpoint{4.583897in}{1.579232in}}%
\pgfpathlineto{\pgfqpoint{4.594325in}{1.507337in}}%
\pgfpathlineto{\pgfqpoint{4.604753in}{1.486324in}}%
\pgfpathlineto{\pgfqpoint{4.615181in}{1.537187in}}%
\pgfpathlineto{\pgfqpoint{4.625610in}{1.566169in}}%
\pgfpathlineto{\pgfqpoint{4.636038in}{1.622112in}}%
\pgfpathlineto{\pgfqpoint{4.646466in}{1.595210in}}%
\pgfpathlineto{\pgfqpoint{4.656894in}{1.614339in}}%
\pgfpathlineto{\pgfqpoint{4.667322in}{1.523856in}}%
\pgfpathlineto{\pgfqpoint{4.677750in}{1.576168in}}%
\pgfpathlineto{\pgfqpoint{4.688179in}{1.601668in}}%
\pgfpathlineto{\pgfqpoint{4.698607in}{1.514035in}}%
\pgfpathlineto{\pgfqpoint{4.709035in}{1.512050in}}%
\pgfpathlineto{\pgfqpoint{4.719463in}{1.568013in}}%
\pgfpathlineto{\pgfqpoint{4.729891in}{1.533756in}}%
\pgfpathlineto{\pgfqpoint{4.740319in}{1.527182in}}%
\pgfpathlineto{\pgfqpoint{4.750748in}{1.568840in}}%
\pgfpathlineto{\pgfqpoint{4.761176in}{1.634103in}}%
\pgfpathlineto{\pgfqpoint{4.771604in}{1.546336in}}%
\pgfpathlineto{\pgfqpoint{4.771604in}{1.546336in}}%
\pgfusepath{stroke}%
\end{pgfscope}%
\begin{pgfscope}%
\pgfpathrectangle{\pgfqpoint{0.610762in}{0.961156in}}{\pgfqpoint{4.171270in}{2.577986in}} %
\pgfusepath{clip}%
\pgfsetroundcap%
\pgfsetroundjoin%
\pgfsetlinewidth{1.756562pt}%
\definecolor{currentstroke}{rgb}{1.000000,0.494118,0.250980}%
\pgfsetstrokecolor{currentstroke}%
\pgfsetstrokeopacity{0.800000}%
\pgfsetdash{}{0pt}%
\pgfpathmoveto{\pgfqpoint{0.610762in}{2.997845in}}%
\pgfpathlineto{\pgfqpoint{0.621191in}{3.035930in}}%
\pgfpathlineto{\pgfqpoint{0.631619in}{3.031024in}}%
\pgfpathlineto{\pgfqpoint{0.642047in}{2.988861in}}%
\pgfpathlineto{\pgfqpoint{0.662903in}{3.014504in}}%
\pgfpathlineto{\pgfqpoint{0.673331in}{2.987417in}}%
\pgfpathlineto{\pgfqpoint{0.683760in}{3.005617in}}%
\pgfpathlineto{\pgfqpoint{0.694188in}{3.039105in}}%
\pgfpathlineto{\pgfqpoint{0.704616in}{3.000374in}}%
\pgfpathlineto{\pgfqpoint{0.715044in}{3.012190in}}%
\pgfpathlineto{\pgfqpoint{0.725472in}{3.006486in}}%
\pgfpathlineto{\pgfqpoint{0.735900in}{3.009612in}}%
\pgfpathlineto{\pgfqpoint{0.746329in}{2.985145in}}%
\pgfpathlineto{\pgfqpoint{0.756757in}{2.992425in}}%
\pgfpathlineto{\pgfqpoint{0.767185in}{3.003834in}}%
\pgfpathlineto{\pgfqpoint{0.777613in}{2.974396in}}%
\pgfpathlineto{\pgfqpoint{0.798470in}{2.973763in}}%
\pgfpathlineto{\pgfqpoint{0.808898in}{2.978681in}}%
\pgfpathlineto{\pgfqpoint{0.819326in}{2.999157in}}%
\pgfpathlineto{\pgfqpoint{0.829754in}{3.036564in}}%
\pgfpathlineto{\pgfqpoint{0.840182in}{3.033413in}}%
\pgfpathlineto{\pgfqpoint{0.850610in}{3.067949in}}%
\pgfpathlineto{\pgfqpoint{0.861039in}{3.011716in}}%
\pgfpathlineto{\pgfqpoint{0.871467in}{3.026598in}}%
\pgfpathlineto{\pgfqpoint{0.881895in}{3.058682in}}%
\pgfpathlineto{\pgfqpoint{0.892323in}{3.030386in}}%
\pgfpathlineto{\pgfqpoint{0.902751in}{3.047349in}}%
\pgfpathlineto{\pgfqpoint{0.913179in}{3.037111in}}%
\pgfpathlineto{\pgfqpoint{0.934036in}{3.042581in}}%
\pgfpathlineto{\pgfqpoint{0.954892in}{3.037802in}}%
\pgfpathlineto{\pgfqpoint{0.965320in}{3.046439in}}%
\pgfpathlineto{\pgfqpoint{0.986177in}{3.036271in}}%
\pgfpathlineto{\pgfqpoint{0.996605in}{3.042077in}}%
\pgfpathlineto{\pgfqpoint{1.007033in}{3.055132in}}%
\pgfpathlineto{\pgfqpoint{1.017461in}{3.063638in}}%
\pgfpathlineto{\pgfqpoint{1.027889in}{3.053747in}}%
\pgfpathlineto{\pgfqpoint{1.038318in}{3.062640in}}%
\pgfpathlineto{\pgfqpoint{1.048746in}{3.029316in}}%
\pgfpathlineto{\pgfqpoint{1.059174in}{3.055396in}}%
\pgfpathlineto{\pgfqpoint{1.069602in}{3.054922in}}%
\pgfpathlineto{\pgfqpoint{1.080030in}{3.066289in}}%
\pgfpathlineto{\pgfqpoint{1.090458in}{3.063428in}}%
\pgfpathlineto{\pgfqpoint{1.100887in}{3.055083in}}%
\pgfpathlineto{\pgfqpoint{1.111315in}{3.050726in}}%
\pgfpathlineto{\pgfqpoint{1.121743in}{3.034802in}}%
\pgfpathlineto{\pgfqpoint{1.132171in}{3.036460in}}%
\pgfpathlineto{\pgfqpoint{1.142599in}{3.048299in}}%
\pgfpathlineto{\pgfqpoint{1.153027in}{3.049089in}}%
\pgfpathlineto{\pgfqpoint{1.163456in}{3.039657in}}%
\pgfpathlineto{\pgfqpoint{1.173884in}{3.045760in}}%
\pgfpathlineto{\pgfqpoint{1.184312in}{3.046728in}}%
\pgfpathlineto{\pgfqpoint{1.194740in}{3.023332in}}%
\pgfpathlineto{\pgfqpoint{1.205168in}{3.021122in}}%
\pgfpathlineto{\pgfqpoint{1.215596in}{3.023731in}}%
\pgfpathlineto{\pgfqpoint{1.226025in}{2.998774in}}%
\pgfpathlineto{\pgfqpoint{1.236453in}{3.017282in}}%
\pgfpathlineto{\pgfqpoint{1.246881in}{3.007438in}}%
\pgfpathlineto{\pgfqpoint{1.267737in}{3.016725in}}%
\pgfpathlineto{\pgfqpoint{1.278166in}{3.004541in}}%
\pgfpathlineto{\pgfqpoint{1.288594in}{3.022325in}}%
\pgfpathlineto{\pgfqpoint{1.299022in}{3.021825in}}%
\pgfpathlineto{\pgfqpoint{1.319878in}{3.034918in}}%
\pgfpathlineto{\pgfqpoint{1.330306in}{3.037052in}}%
\pgfpathlineto{\pgfqpoint{1.340735in}{3.050437in}}%
\pgfpathlineto{\pgfqpoint{1.351163in}{3.042626in}}%
\pgfpathlineto{\pgfqpoint{1.361591in}{3.051637in}}%
\pgfpathlineto{\pgfqpoint{1.372019in}{3.055565in}}%
\pgfpathlineto{\pgfqpoint{1.382447in}{3.073954in}}%
\pgfpathlineto{\pgfqpoint{1.392875in}{3.051325in}}%
\pgfpathlineto{\pgfqpoint{1.403304in}{3.033056in}}%
\pgfpathlineto{\pgfqpoint{1.413732in}{3.042350in}}%
\pgfpathlineto{\pgfqpoint{1.424160in}{3.073101in}}%
\pgfpathlineto{\pgfqpoint{1.434588in}{3.065441in}}%
\pgfpathlineto{\pgfqpoint{1.445016in}{3.062425in}}%
\pgfpathlineto{\pgfqpoint{1.455444in}{3.040446in}}%
\pgfpathlineto{\pgfqpoint{1.465873in}{3.052177in}}%
\pgfpathlineto{\pgfqpoint{1.476301in}{3.076276in}}%
\pgfpathlineto{\pgfqpoint{1.497157in}{3.057750in}}%
\pgfpathlineto{\pgfqpoint{1.518014in}{3.077446in}}%
\pgfpathlineto{\pgfqpoint{1.528442in}{3.079180in}}%
\pgfpathlineto{\pgfqpoint{1.549298in}{3.062316in}}%
\pgfpathlineto{\pgfqpoint{1.559726in}{3.043843in}}%
\pgfpathlineto{\pgfqpoint{1.570154in}{3.030433in}}%
\pgfpathlineto{\pgfqpoint{1.580583in}{3.044942in}}%
\pgfpathlineto{\pgfqpoint{1.591011in}{3.026063in}}%
\pgfpathlineto{\pgfqpoint{1.601439in}{3.029736in}}%
\pgfpathlineto{\pgfqpoint{1.611867in}{3.044185in}}%
\pgfpathlineto{\pgfqpoint{1.622295in}{3.035507in}}%
\pgfpathlineto{\pgfqpoint{1.632723in}{3.030484in}}%
\pgfpathlineto{\pgfqpoint{1.643152in}{2.991762in}}%
\pgfpathlineto{\pgfqpoint{1.653580in}{3.013167in}}%
\pgfpathlineto{\pgfqpoint{1.664008in}{3.015013in}}%
\pgfpathlineto{\pgfqpoint{1.674436in}{3.014314in}}%
\pgfpathlineto{\pgfqpoint{1.684864in}{2.992638in}}%
\pgfpathlineto{\pgfqpoint{1.695292in}{2.995325in}}%
\pgfpathlineto{\pgfqpoint{1.705721in}{2.988210in}}%
\pgfpathlineto{\pgfqpoint{1.726577in}{3.044195in}}%
\pgfpathlineto{\pgfqpoint{1.737005in}{3.045831in}}%
\pgfpathlineto{\pgfqpoint{1.747433in}{3.027535in}}%
\pgfpathlineto{\pgfqpoint{1.757862in}{3.036525in}}%
\pgfpathlineto{\pgfqpoint{1.768290in}{3.038623in}}%
\pgfpathlineto{\pgfqpoint{1.778718in}{3.052695in}}%
\pgfpathlineto{\pgfqpoint{1.789146in}{3.009436in}}%
\pgfpathlineto{\pgfqpoint{1.799574in}{3.043416in}}%
\pgfpathlineto{\pgfqpoint{1.810002in}{3.048519in}}%
\pgfpathlineto{\pgfqpoint{1.820431in}{3.033332in}}%
\pgfpathlineto{\pgfqpoint{1.830859in}{3.026656in}}%
\pgfpathlineto{\pgfqpoint{1.841287in}{3.059178in}}%
\pgfpathlineto{\pgfqpoint{1.851715in}{3.038910in}}%
\pgfpathlineto{\pgfqpoint{1.862143in}{3.045095in}}%
\pgfpathlineto{\pgfqpoint{1.872571in}{3.037029in}}%
\pgfpathlineto{\pgfqpoint{1.883000in}{3.022658in}}%
\pgfpathlineto{\pgfqpoint{1.893428in}{3.015942in}}%
\pgfpathlineto{\pgfqpoint{1.914284in}{3.036280in}}%
\pgfpathlineto{\pgfqpoint{1.924712in}{3.005969in}}%
\pgfpathlineto{\pgfqpoint{1.935141in}{2.992756in}}%
\pgfpathlineto{\pgfqpoint{1.945569in}{3.011173in}}%
\pgfpathlineto{\pgfqpoint{1.955997in}{2.994049in}}%
\pgfpathlineto{\pgfqpoint{1.966425in}{2.994629in}}%
\pgfpathlineto{\pgfqpoint{1.976853in}{2.992394in}}%
\pgfpathlineto{\pgfqpoint{1.987281in}{2.992916in}}%
\pgfpathlineto{\pgfqpoint{1.997710in}{2.984976in}}%
\pgfpathlineto{\pgfqpoint{2.008138in}{3.002553in}}%
\pgfpathlineto{\pgfqpoint{2.018566in}{2.961827in}}%
\pgfpathlineto{\pgfqpoint{2.028994in}{2.989101in}}%
\pgfpathlineto{\pgfqpoint{2.039422in}{3.001255in}}%
\pgfpathlineto{\pgfqpoint{2.049850in}{3.025030in}}%
\pgfpathlineto{\pgfqpoint{2.060279in}{3.036066in}}%
\pgfpathlineto{\pgfqpoint{2.081135in}{3.038780in}}%
\pgfpathlineto{\pgfqpoint{2.091563in}{3.044950in}}%
\pgfpathlineto{\pgfqpoint{2.112419in}{3.075699in}}%
\pgfpathlineto{\pgfqpoint{2.122848in}{3.033015in}}%
\pgfpathlineto{\pgfqpoint{2.133276in}{3.041223in}}%
\pgfpathlineto{\pgfqpoint{2.143704in}{3.018114in}}%
\pgfpathlineto{\pgfqpoint{2.154132in}{3.016049in}}%
\pgfpathlineto{\pgfqpoint{2.164560in}{3.036151in}}%
\pgfpathlineto{\pgfqpoint{2.174989in}{3.023068in}}%
\pgfpathlineto{\pgfqpoint{2.185417in}{3.027049in}}%
\pgfpathlineto{\pgfqpoint{2.195845in}{2.995893in}}%
\pgfpathlineto{\pgfqpoint{2.206273in}{3.002951in}}%
\pgfpathlineto{\pgfqpoint{2.216701in}{3.000670in}}%
\pgfpathlineto{\pgfqpoint{2.227129in}{3.009540in}}%
\pgfpathlineto{\pgfqpoint{2.237558in}{3.009743in}}%
\pgfpathlineto{\pgfqpoint{2.247986in}{3.013666in}}%
\pgfpathlineto{\pgfqpoint{2.258414in}{2.995589in}}%
\pgfpathlineto{\pgfqpoint{2.268842in}{3.006461in}}%
\pgfpathlineto{\pgfqpoint{2.279270in}{2.973830in}}%
\pgfpathlineto{\pgfqpoint{2.289698in}{2.971934in}}%
\pgfpathlineto{\pgfqpoint{2.300127in}{2.986689in}}%
\pgfpathlineto{\pgfqpoint{2.310555in}{2.966622in}}%
\pgfpathlineto{\pgfqpoint{2.320983in}{2.966776in}}%
\pgfpathlineto{\pgfqpoint{2.331411in}{2.961306in}}%
\pgfpathlineto{\pgfqpoint{2.341839in}{2.981052in}}%
\pgfpathlineto{\pgfqpoint{2.352267in}{2.980408in}}%
\pgfpathlineto{\pgfqpoint{2.362696in}{2.969305in}}%
\pgfpathlineto{\pgfqpoint{2.373124in}{2.980442in}}%
\pgfpathlineto{\pgfqpoint{2.383552in}{2.983220in}}%
\pgfpathlineto{\pgfqpoint{2.393980in}{2.981650in}}%
\pgfpathlineto{\pgfqpoint{2.404408in}{2.970285in}}%
\pgfpathlineto{\pgfqpoint{2.414837in}{2.998063in}}%
\pgfpathlineto{\pgfqpoint{2.425265in}{3.004799in}}%
\pgfpathlineto{\pgfqpoint{2.435693in}{2.998699in}}%
\pgfpathlineto{\pgfqpoint{2.446121in}{3.014970in}}%
\pgfpathlineto{\pgfqpoint{2.466977in}{3.020141in}}%
\pgfpathlineto{\pgfqpoint{2.487834in}{3.005601in}}%
\pgfpathlineto{\pgfqpoint{2.498262in}{2.992767in}}%
\pgfpathlineto{\pgfqpoint{2.508690in}{2.998559in}}%
\pgfpathlineto{\pgfqpoint{2.519118in}{2.971377in}}%
\pgfpathlineto{\pgfqpoint{2.529546in}{3.012539in}}%
\pgfpathlineto{\pgfqpoint{2.539975in}{2.996968in}}%
\pgfpathlineto{\pgfqpoint{2.550403in}{3.012887in}}%
\pgfpathlineto{\pgfqpoint{2.571259in}{3.008978in}}%
\pgfpathlineto{\pgfqpoint{2.581687in}{2.982546in}}%
\pgfpathlineto{\pgfqpoint{2.592115in}{2.996698in}}%
\pgfpathlineto{\pgfqpoint{2.602544in}{2.969304in}}%
\pgfpathlineto{\pgfqpoint{2.612972in}{2.988226in}}%
\pgfpathlineto{\pgfqpoint{2.623400in}{2.982162in}}%
\pgfpathlineto{\pgfqpoint{2.644256in}{2.953151in}}%
\pgfpathlineto{\pgfqpoint{2.654685in}{2.961807in}}%
\pgfpathlineto{\pgfqpoint{2.665113in}{2.933314in}}%
\pgfpathlineto{\pgfqpoint{2.675541in}{2.888184in}}%
\pgfpathlineto{\pgfqpoint{2.685969in}{2.817945in}}%
\pgfpathlineto{\pgfqpoint{2.696397in}{2.805064in}}%
\pgfpathlineto{\pgfqpoint{2.717254in}{2.691109in}}%
\pgfpathlineto{\pgfqpoint{2.727682in}{2.664710in}}%
\pgfpathlineto{\pgfqpoint{2.748538in}{2.646754in}}%
\pgfpathlineto{\pgfqpoint{2.758966in}{2.652870in}}%
\pgfpathlineto{\pgfqpoint{2.769394in}{2.654715in}}%
\pgfpathlineto{\pgfqpoint{2.790251in}{2.627487in}}%
\pgfpathlineto{\pgfqpoint{2.800679in}{2.633055in}}%
\pgfpathlineto{\pgfqpoint{2.811107in}{2.609433in}}%
\pgfpathlineto{\pgfqpoint{2.821535in}{2.602616in}}%
\pgfpathlineto{\pgfqpoint{2.831964in}{2.569061in}}%
\pgfpathlineto{\pgfqpoint{2.842392in}{2.569693in}}%
\pgfpathlineto{\pgfqpoint{2.852820in}{2.568326in}}%
\pgfpathlineto{\pgfqpoint{2.863248in}{2.560208in}}%
\pgfpathlineto{\pgfqpoint{2.873676in}{2.560117in}}%
\pgfpathlineto{\pgfqpoint{2.884104in}{2.537650in}}%
\pgfpathlineto{\pgfqpoint{2.894533in}{2.529480in}}%
\pgfpathlineto{\pgfqpoint{2.904961in}{2.487295in}}%
\pgfpathlineto{\pgfqpoint{2.915389in}{2.506140in}}%
\pgfpathlineto{\pgfqpoint{2.925817in}{2.479909in}}%
\pgfpathlineto{\pgfqpoint{2.936245in}{2.480967in}}%
\pgfpathlineto{\pgfqpoint{2.946673in}{2.490968in}}%
\pgfpathlineto{\pgfqpoint{2.957102in}{2.488187in}}%
\pgfpathlineto{\pgfqpoint{2.967530in}{2.457128in}}%
\pgfpathlineto{\pgfqpoint{2.977958in}{2.456872in}}%
\pgfpathlineto{\pgfqpoint{2.988386in}{2.454202in}}%
\pgfpathlineto{\pgfqpoint{2.998814in}{2.468519in}}%
\pgfpathlineto{\pgfqpoint{3.009242in}{2.439649in}}%
\pgfpathlineto{\pgfqpoint{3.030099in}{2.486676in}}%
\pgfpathlineto{\pgfqpoint{3.040527in}{2.430164in}}%
\pgfpathlineto{\pgfqpoint{3.050955in}{2.450938in}}%
\pgfpathlineto{\pgfqpoint{3.061383in}{2.478757in}}%
\pgfpathlineto{\pgfqpoint{3.071812in}{2.417803in}}%
\pgfpathlineto{\pgfqpoint{3.082240in}{2.389895in}}%
\pgfpathlineto{\pgfqpoint{3.092668in}{2.391890in}}%
\pgfpathlineto{\pgfqpoint{3.103096in}{2.426741in}}%
\pgfpathlineto{\pgfqpoint{3.113524in}{2.389592in}}%
\pgfpathlineto{\pgfqpoint{3.123952in}{2.410335in}}%
\pgfpathlineto{\pgfqpoint{3.134381in}{2.360538in}}%
\pgfpathlineto{\pgfqpoint{3.144809in}{2.400662in}}%
\pgfpathlineto{\pgfqpoint{3.155237in}{2.387352in}}%
\pgfpathlineto{\pgfqpoint{3.165665in}{2.327143in}}%
\pgfpathlineto{\pgfqpoint{3.176093in}{2.413343in}}%
\pgfpathlineto{\pgfqpoint{3.186521in}{2.377741in}}%
\pgfpathlineto{\pgfqpoint{3.196950in}{2.386111in}}%
\pgfpathlineto{\pgfqpoint{3.207378in}{2.382219in}}%
\pgfpathlineto{\pgfqpoint{3.217806in}{2.387838in}}%
\pgfpathlineto{\pgfqpoint{3.228234in}{2.391200in}}%
\pgfpathlineto{\pgfqpoint{3.238662in}{2.409963in}}%
\pgfpathlineto{\pgfqpoint{3.249090in}{2.354612in}}%
\pgfpathlineto{\pgfqpoint{3.259519in}{2.348451in}}%
\pgfpathlineto{\pgfqpoint{3.269947in}{2.372325in}}%
\pgfpathlineto{\pgfqpoint{3.280375in}{2.339613in}}%
\pgfpathlineto{\pgfqpoint{3.290803in}{2.317774in}}%
\pgfpathlineto{\pgfqpoint{3.301231in}{2.351744in}}%
\pgfpathlineto{\pgfqpoint{3.311660in}{2.395584in}}%
\pgfpathlineto{\pgfqpoint{3.322088in}{2.394635in}}%
\pgfpathlineto{\pgfqpoint{3.332516in}{2.352197in}}%
\pgfpathlineto{\pgfqpoint{3.342944in}{2.369041in}}%
\pgfpathlineto{\pgfqpoint{3.353372in}{2.340576in}}%
\pgfpathlineto{\pgfqpoint{3.363800in}{2.332416in}}%
\pgfpathlineto{\pgfqpoint{3.374229in}{2.349393in}}%
\pgfpathlineto{\pgfqpoint{3.384657in}{2.308888in}}%
\pgfpathlineto{\pgfqpoint{3.405513in}{2.297897in}}%
\pgfpathlineto{\pgfqpoint{3.415941in}{2.338508in}}%
\pgfpathlineto{\pgfqpoint{3.426369in}{2.298972in}}%
\pgfpathlineto{\pgfqpoint{3.436798in}{2.289005in}}%
\pgfpathlineto{\pgfqpoint{3.457654in}{2.295428in}}%
\pgfpathlineto{\pgfqpoint{3.468082in}{2.131604in}}%
\pgfpathlineto{\pgfqpoint{3.478510in}{2.170461in}}%
\pgfpathlineto{\pgfqpoint{3.488938in}{2.148465in}}%
\pgfpathlineto{\pgfqpoint{3.499367in}{2.171331in}}%
\pgfpathlineto{\pgfqpoint{3.509795in}{2.228295in}}%
\pgfpathlineto{\pgfqpoint{3.520223in}{2.245513in}}%
\pgfpathlineto{\pgfqpoint{3.530651in}{2.248839in}}%
\pgfpathlineto{\pgfqpoint{3.541079in}{2.230404in}}%
\pgfpathlineto{\pgfqpoint{3.551508in}{2.155612in}}%
\pgfpathlineto{\pgfqpoint{3.561936in}{2.198326in}}%
\pgfpathlineto{\pgfqpoint{3.572364in}{2.148161in}}%
\pgfpathlineto{\pgfqpoint{3.582792in}{2.159695in}}%
\pgfpathlineto{\pgfqpoint{3.593220in}{2.052032in}}%
\pgfpathlineto{\pgfqpoint{3.603648in}{2.092167in}}%
\pgfpathlineto{\pgfqpoint{3.614077in}{2.047663in}}%
\pgfpathlineto{\pgfqpoint{3.624505in}{2.059872in}}%
\pgfpathlineto{\pgfqpoint{3.634933in}{2.046186in}}%
\pgfpathlineto{\pgfqpoint{3.645361in}{2.060942in}}%
\pgfpathlineto{\pgfqpoint{3.655789in}{2.062357in}}%
\pgfpathlineto{\pgfqpoint{3.666217in}{2.060590in}}%
\pgfpathlineto{\pgfqpoint{3.676646in}{1.973332in}}%
\pgfpathlineto{\pgfqpoint{3.687074in}{2.050378in}}%
\pgfpathlineto{\pgfqpoint{3.707930in}{1.982504in}}%
\pgfpathlineto{\pgfqpoint{3.718358in}{2.001320in}}%
\pgfpathlineto{\pgfqpoint{3.728787in}{2.120869in}}%
\pgfpathlineto{\pgfqpoint{3.739215in}{1.952774in}}%
\pgfpathlineto{\pgfqpoint{3.749643in}{1.940908in}}%
\pgfpathlineto{\pgfqpoint{3.760071in}{1.962462in}}%
\pgfpathlineto{\pgfqpoint{3.770499in}{1.904025in}}%
\pgfpathlineto{\pgfqpoint{3.780927in}{2.046280in}}%
\pgfpathlineto{\pgfqpoint{3.791356in}{1.975288in}}%
\pgfpathlineto{\pgfqpoint{3.801784in}{1.938922in}}%
\pgfpathlineto{\pgfqpoint{3.812212in}{1.971249in}}%
\pgfpathlineto{\pgfqpoint{3.822640in}{1.886602in}}%
\pgfpathlineto{\pgfqpoint{3.833068in}{1.821522in}}%
\pgfpathlineto{\pgfqpoint{3.843496in}{1.793218in}}%
\pgfpathlineto{\pgfqpoint{3.853925in}{1.805144in}}%
\pgfpathlineto{\pgfqpoint{3.874781in}{1.924210in}}%
\pgfpathlineto{\pgfqpoint{3.885209in}{1.964762in}}%
\pgfpathlineto{\pgfqpoint{3.895637in}{1.809676in}}%
\pgfpathlineto{\pgfqpoint{3.906065in}{1.729007in}}%
\pgfpathlineto{\pgfqpoint{3.916494in}{1.909999in}}%
\pgfpathlineto{\pgfqpoint{3.926922in}{1.916726in}}%
\pgfpathlineto{\pgfqpoint{3.937350in}{1.735303in}}%
\pgfpathlineto{\pgfqpoint{3.947778in}{1.956641in}}%
\pgfpathlineto{\pgfqpoint{3.958206in}{1.868141in}}%
\pgfpathlineto{\pgfqpoint{3.968635in}{1.957726in}}%
\pgfpathlineto{\pgfqpoint{3.979063in}{1.762284in}}%
\pgfpathlineto{\pgfqpoint{3.989491in}{1.851627in}}%
\pgfpathlineto{\pgfqpoint{3.999919in}{1.986814in}}%
\pgfpathlineto{\pgfqpoint{4.010347in}{1.835234in}}%
\pgfpathlineto{\pgfqpoint{4.020775in}{2.000717in}}%
\pgfpathlineto{\pgfqpoint{4.031204in}{1.897630in}}%
\pgfpathlineto{\pgfqpoint{4.041632in}{1.984979in}}%
\pgfpathlineto{\pgfqpoint{4.052060in}{1.937094in}}%
\pgfpathlineto{\pgfqpoint{4.062488in}{1.830913in}}%
\pgfpathlineto{\pgfqpoint{4.072916in}{1.911248in}}%
\pgfpathlineto{\pgfqpoint{4.083344in}{1.791070in}}%
\pgfpathlineto{\pgfqpoint{4.093773in}{1.841835in}}%
\pgfpathlineto{\pgfqpoint{4.104201in}{1.834131in}}%
\pgfpathlineto{\pgfqpoint{4.114629in}{1.866863in}}%
\pgfpathlineto{\pgfqpoint{4.125057in}{1.806230in}}%
\pgfpathlineto{\pgfqpoint{4.135485in}{1.936514in}}%
\pgfpathlineto{\pgfqpoint{4.145913in}{1.810168in}}%
\pgfpathlineto{\pgfqpoint{4.156342in}{1.891334in}}%
\pgfpathlineto{\pgfqpoint{4.166770in}{1.917702in}}%
\pgfpathlineto{\pgfqpoint{4.177198in}{1.956132in}}%
\pgfpathlineto{\pgfqpoint{4.187626in}{1.842014in}}%
\pgfpathlineto{\pgfqpoint{4.198054in}{1.793902in}}%
\pgfpathlineto{\pgfqpoint{4.208483in}{1.695791in}}%
\pgfpathlineto{\pgfqpoint{4.218911in}{1.894377in}}%
\pgfpathlineto{\pgfqpoint{4.229339in}{1.709627in}}%
\pgfpathlineto{\pgfqpoint{4.239767in}{1.874890in}}%
\pgfpathlineto{\pgfqpoint{4.250195in}{1.709374in}}%
\pgfpathlineto{\pgfqpoint{4.260623in}{1.755179in}}%
\pgfpathlineto{\pgfqpoint{4.271052in}{1.775274in}}%
\pgfpathlineto{\pgfqpoint{4.281480in}{1.707283in}}%
\pgfpathlineto{\pgfqpoint{4.291908in}{1.788410in}}%
\pgfpathlineto{\pgfqpoint{4.302336in}{1.919234in}}%
\pgfpathlineto{\pgfqpoint{4.312764in}{1.676843in}}%
\pgfpathlineto{\pgfqpoint{4.323192in}{1.660773in}}%
\pgfpathlineto{\pgfqpoint{4.333621in}{1.691170in}}%
\pgfpathlineto{\pgfqpoint{4.344049in}{1.598822in}}%
\pgfpathlineto{\pgfqpoint{4.354477in}{1.695969in}}%
\pgfpathlineto{\pgfqpoint{4.364905in}{1.811513in}}%
\pgfpathlineto{\pgfqpoint{4.375333in}{2.009022in}}%
\pgfpathlineto{\pgfqpoint{4.385761in}{1.748965in}}%
\pgfpathlineto{\pgfqpoint{4.396190in}{1.895620in}}%
\pgfpathlineto{\pgfqpoint{4.406618in}{1.634233in}}%
\pgfpathlineto{\pgfqpoint{4.427474in}{1.783692in}}%
\pgfpathlineto{\pgfqpoint{4.437902in}{1.772535in}}%
\pgfpathlineto{\pgfqpoint{4.448331in}{1.815304in}}%
\pgfpathlineto{\pgfqpoint{4.458759in}{1.801315in}}%
\pgfpathlineto{\pgfqpoint{4.469187in}{1.771955in}}%
\pgfpathlineto{\pgfqpoint{4.479615in}{1.800794in}}%
\pgfpathlineto{\pgfqpoint{4.490043in}{1.921271in}}%
\pgfpathlineto{\pgfqpoint{4.500471in}{2.094919in}}%
\pgfpathlineto{\pgfqpoint{4.510900in}{1.852097in}}%
\pgfpathlineto{\pgfqpoint{4.521328in}{1.814770in}}%
\pgfpathlineto{\pgfqpoint{4.531756in}{1.992879in}}%
\pgfpathlineto{\pgfqpoint{4.542184in}{1.967282in}}%
\pgfpathlineto{\pgfqpoint{4.552612in}{1.871544in}}%
\pgfpathlineto{\pgfqpoint{4.563040in}{1.674550in}}%
\pgfpathlineto{\pgfqpoint{4.573469in}{1.787949in}}%
\pgfpathlineto{\pgfqpoint{4.583897in}{1.840342in}}%
\pgfpathlineto{\pgfqpoint{4.594325in}{1.746089in}}%
\pgfpathlineto{\pgfqpoint{4.604753in}{1.754305in}}%
\pgfpathlineto{\pgfqpoint{4.615181in}{1.907628in}}%
\pgfpathlineto{\pgfqpoint{4.625610in}{1.880616in}}%
\pgfpathlineto{\pgfqpoint{4.636038in}{1.846097in}}%
\pgfpathlineto{\pgfqpoint{4.646466in}{1.789997in}}%
\pgfpathlineto{\pgfqpoint{4.656894in}{1.854657in}}%
\pgfpathlineto{\pgfqpoint{4.667322in}{1.868120in}}%
\pgfpathlineto{\pgfqpoint{4.677750in}{1.799964in}}%
\pgfpathlineto{\pgfqpoint{4.688179in}{1.994000in}}%
\pgfpathlineto{\pgfqpoint{4.698607in}{1.820910in}}%
\pgfpathlineto{\pgfqpoint{4.709035in}{1.915316in}}%
\pgfpathlineto{\pgfqpoint{4.719463in}{1.737462in}}%
\pgfpathlineto{\pgfqpoint{4.729891in}{1.981502in}}%
\pgfpathlineto{\pgfqpoint{4.740319in}{1.809573in}}%
\pgfpathlineto{\pgfqpoint{4.750748in}{1.956419in}}%
\pgfpathlineto{\pgfqpoint{4.761176in}{2.029687in}}%
\pgfpathlineto{\pgfqpoint{4.771604in}{1.570423in}}%
\pgfpathlineto{\pgfqpoint{4.771604in}{1.570423in}}%
\pgfusepath{stroke}%
\end{pgfscope}%
\begin{pgfscope}%
\pgfpathrectangle{\pgfqpoint{0.610762in}{0.961156in}}{\pgfqpoint{4.171270in}{2.577986in}} %
\pgfusepath{clip}%
\pgfsetroundcap%
\pgfsetroundjoin%
\pgfsetlinewidth{1.756562pt}%
\definecolor{currentstroke}{rgb}{1.000000,0.694118,0.250980}%
\pgfsetstrokecolor{currentstroke}%
\pgfsetstrokeopacity{0.800000}%
\pgfsetdash{}{0pt}%
\pgfpathmoveto{\pgfqpoint{0.610762in}{3.100992in}}%
\pgfpathlineto{\pgfqpoint{0.621191in}{3.064551in}}%
\pgfpathlineto{\pgfqpoint{0.631619in}{3.089117in}}%
\pgfpathlineto{\pgfqpoint{0.642047in}{3.135763in}}%
\pgfpathlineto{\pgfqpoint{0.652475in}{3.038098in}}%
\pgfpathlineto{\pgfqpoint{0.673331in}{3.111345in}}%
\pgfpathlineto{\pgfqpoint{0.683760in}{3.159889in}}%
\pgfpathlineto{\pgfqpoint{0.694188in}{3.159602in}}%
\pgfpathlineto{\pgfqpoint{0.704616in}{3.076340in}}%
\pgfpathlineto{\pgfqpoint{0.715044in}{3.079505in}}%
\pgfpathlineto{\pgfqpoint{0.725472in}{3.108951in}}%
\pgfpathlineto{\pgfqpoint{0.735900in}{3.079975in}}%
\pgfpathlineto{\pgfqpoint{0.756757in}{3.175652in}}%
\pgfpathlineto{\pgfqpoint{0.767185in}{3.239104in}}%
\pgfpathlineto{\pgfqpoint{0.777613in}{3.218619in}}%
\pgfpathlineto{\pgfqpoint{0.788041in}{3.252180in}}%
\pgfpathlineto{\pgfqpoint{0.798470in}{3.272517in}}%
\pgfpathlineto{\pgfqpoint{0.808898in}{3.302001in}}%
\pgfpathlineto{\pgfqpoint{0.819326in}{3.285485in}}%
\pgfpathlineto{\pgfqpoint{0.829754in}{3.258140in}}%
\pgfpathlineto{\pgfqpoint{0.840182in}{3.261133in}}%
\pgfpathlineto{\pgfqpoint{0.850610in}{3.199090in}}%
\pgfpathlineto{\pgfqpoint{0.861039in}{3.235990in}}%
\pgfpathlineto{\pgfqpoint{0.871467in}{3.236167in}}%
\pgfpathlineto{\pgfqpoint{0.881895in}{3.180807in}}%
\pgfpathlineto{\pgfqpoint{0.892323in}{3.182774in}}%
\pgfpathlineto{\pgfqpoint{0.902751in}{3.200145in}}%
\pgfpathlineto{\pgfqpoint{0.913179in}{3.158136in}}%
\pgfpathlineto{\pgfqpoint{0.923608in}{3.136007in}}%
\pgfpathlineto{\pgfqpoint{0.934036in}{3.169312in}}%
\pgfpathlineto{\pgfqpoint{0.944464in}{3.127816in}}%
\pgfpathlineto{\pgfqpoint{0.954892in}{3.134535in}}%
\pgfpathlineto{\pgfqpoint{0.965320in}{3.126127in}}%
\pgfpathlineto{\pgfqpoint{0.975748in}{3.152551in}}%
\pgfpathlineto{\pgfqpoint{0.986177in}{3.108833in}}%
\pgfpathlineto{\pgfqpoint{0.996605in}{3.136181in}}%
\pgfpathlineto{\pgfqpoint{1.007033in}{3.072522in}}%
\pgfpathlineto{\pgfqpoint{1.017461in}{3.058070in}}%
\pgfpathlineto{\pgfqpoint{1.027889in}{3.010768in}}%
\pgfpathlineto{\pgfqpoint{1.038318in}{3.026499in}}%
\pgfpathlineto{\pgfqpoint{1.048746in}{3.023615in}}%
\pgfpathlineto{\pgfqpoint{1.069602in}{3.092736in}}%
\pgfpathlineto{\pgfqpoint{1.080030in}{3.060594in}}%
\pgfpathlineto{\pgfqpoint{1.090458in}{3.132990in}}%
\pgfpathlineto{\pgfqpoint{1.100887in}{3.014243in}}%
\pgfpathlineto{\pgfqpoint{1.111315in}{3.017199in}}%
\pgfpathlineto{\pgfqpoint{1.121743in}{3.031390in}}%
\pgfpathlineto{\pgfqpoint{1.132171in}{3.017011in}}%
\pgfpathlineto{\pgfqpoint{1.142599in}{3.038867in}}%
\pgfpathlineto{\pgfqpoint{1.153027in}{3.073158in}}%
\pgfpathlineto{\pgfqpoint{1.184312in}{2.965350in}}%
\pgfpathlineto{\pgfqpoint{1.194740in}{2.983523in}}%
\pgfpathlineto{\pgfqpoint{1.205168in}{2.959391in}}%
\pgfpathlineto{\pgfqpoint{1.215596in}{2.995035in}}%
\pgfpathlineto{\pgfqpoint{1.226025in}{2.971273in}}%
\pgfpathlineto{\pgfqpoint{1.236453in}{2.995489in}}%
\pgfpathlineto{\pgfqpoint{1.246881in}{2.972464in}}%
\pgfpathlineto{\pgfqpoint{1.257309in}{2.954346in}}%
\pgfpathlineto{\pgfqpoint{1.267737in}{3.000165in}}%
\pgfpathlineto{\pgfqpoint{1.278166in}{3.035408in}}%
\pgfpathlineto{\pgfqpoint{1.288594in}{2.988640in}}%
\pgfpathlineto{\pgfqpoint{1.299022in}{2.995496in}}%
\pgfpathlineto{\pgfqpoint{1.309450in}{3.037387in}}%
\pgfpathlineto{\pgfqpoint{1.319878in}{3.071353in}}%
\pgfpathlineto{\pgfqpoint{1.330306in}{3.143398in}}%
\pgfpathlineto{\pgfqpoint{1.340735in}{3.031542in}}%
\pgfpathlineto{\pgfqpoint{1.351163in}{3.035067in}}%
\pgfpathlineto{\pgfqpoint{1.361591in}{3.002023in}}%
\pgfpathlineto{\pgfqpoint{1.372019in}{3.033776in}}%
\pgfpathlineto{\pgfqpoint{1.382447in}{2.972448in}}%
\pgfpathlineto{\pgfqpoint{1.392875in}{2.972327in}}%
\pgfpathlineto{\pgfqpoint{1.403304in}{3.006917in}}%
\pgfpathlineto{\pgfqpoint{1.413732in}{2.946032in}}%
\pgfpathlineto{\pgfqpoint{1.424160in}{2.909852in}}%
\pgfpathlineto{\pgfqpoint{1.434588in}{2.994459in}}%
\pgfpathlineto{\pgfqpoint{1.445016in}{2.876952in}}%
\pgfpathlineto{\pgfqpoint{1.455444in}{2.963519in}}%
\pgfpathlineto{\pgfqpoint{1.465873in}{3.016215in}}%
\pgfpathlineto{\pgfqpoint{1.476301in}{3.091886in}}%
\pgfpathlineto{\pgfqpoint{1.486729in}{3.124242in}}%
\pgfpathlineto{\pgfqpoint{1.497157in}{3.099906in}}%
\pgfpathlineto{\pgfqpoint{1.518014in}{3.070001in}}%
\pgfpathlineto{\pgfqpoint{1.528442in}{3.092372in}}%
\pgfpathlineto{\pgfqpoint{1.538870in}{3.039400in}}%
\pgfpathlineto{\pgfqpoint{1.549298in}{3.069098in}}%
\pgfpathlineto{\pgfqpoint{1.559726in}{3.078771in}}%
\pgfpathlineto{\pgfqpoint{1.570154in}{3.095553in}}%
\pgfpathlineto{\pgfqpoint{1.580583in}{3.043340in}}%
\pgfpathlineto{\pgfqpoint{1.591011in}{3.062891in}}%
\pgfpathlineto{\pgfqpoint{1.611867in}{3.070766in}}%
\pgfpathlineto{\pgfqpoint{1.622295in}{3.112188in}}%
\pgfpathlineto{\pgfqpoint{1.632723in}{3.195886in}}%
\pgfpathlineto{\pgfqpoint{1.643152in}{3.165310in}}%
\pgfpathlineto{\pgfqpoint{1.653580in}{3.151557in}}%
\pgfpathlineto{\pgfqpoint{1.664008in}{3.173984in}}%
\pgfpathlineto{\pgfqpoint{1.674436in}{3.074974in}}%
\pgfpathlineto{\pgfqpoint{1.684864in}{3.112912in}}%
\pgfpathlineto{\pgfqpoint{1.695292in}{3.068520in}}%
\pgfpathlineto{\pgfqpoint{1.705721in}{3.079679in}}%
\pgfpathlineto{\pgfqpoint{1.716149in}{3.100526in}}%
\pgfpathlineto{\pgfqpoint{1.726577in}{3.056741in}}%
\pgfpathlineto{\pgfqpoint{1.737005in}{3.038302in}}%
\pgfpathlineto{\pgfqpoint{1.747433in}{3.069923in}}%
\pgfpathlineto{\pgfqpoint{1.757862in}{3.062341in}}%
\pgfpathlineto{\pgfqpoint{1.768290in}{3.084454in}}%
\pgfpathlineto{\pgfqpoint{1.778718in}{3.066522in}}%
\pgfpathlineto{\pgfqpoint{1.789146in}{3.021197in}}%
\pgfpathlineto{\pgfqpoint{1.799574in}{3.041554in}}%
\pgfpathlineto{\pgfqpoint{1.810002in}{3.042783in}}%
\pgfpathlineto{\pgfqpoint{1.820431in}{3.088136in}}%
\pgfpathlineto{\pgfqpoint{1.830859in}{3.146197in}}%
\pgfpathlineto{\pgfqpoint{1.841287in}{3.169564in}}%
\pgfpathlineto{\pgfqpoint{1.851715in}{3.153536in}}%
\pgfpathlineto{\pgfqpoint{1.862143in}{3.150790in}}%
\pgfpathlineto{\pgfqpoint{1.872571in}{3.151516in}}%
\pgfpathlineto{\pgfqpoint{1.883000in}{3.182665in}}%
\pgfpathlineto{\pgfqpoint{1.893428in}{3.113455in}}%
\pgfpathlineto{\pgfqpoint{1.903856in}{3.136656in}}%
\pgfpathlineto{\pgfqpoint{1.914284in}{3.169945in}}%
\pgfpathlineto{\pgfqpoint{1.924712in}{3.160307in}}%
\pgfpathlineto{\pgfqpoint{1.935141in}{3.139717in}}%
\pgfpathlineto{\pgfqpoint{1.945569in}{3.156482in}}%
\pgfpathlineto{\pgfqpoint{1.955997in}{3.075426in}}%
\pgfpathlineto{\pgfqpoint{1.966425in}{3.020101in}}%
\pgfpathlineto{\pgfqpoint{1.976853in}{2.985045in}}%
\pgfpathlineto{\pgfqpoint{1.987281in}{3.021444in}}%
\pgfpathlineto{\pgfqpoint{1.997710in}{3.015523in}}%
\pgfpathlineto{\pgfqpoint{2.008138in}{3.012244in}}%
\pgfpathlineto{\pgfqpoint{2.018566in}{3.030401in}}%
\pgfpathlineto{\pgfqpoint{2.028994in}{2.995827in}}%
\pgfpathlineto{\pgfqpoint{2.039422in}{3.049914in}}%
\pgfpathlineto{\pgfqpoint{2.049850in}{3.022979in}}%
\pgfpathlineto{\pgfqpoint{2.060279in}{3.014805in}}%
\pgfpathlineto{\pgfqpoint{2.070707in}{3.035774in}}%
\pgfpathlineto{\pgfqpoint{2.081135in}{3.064883in}}%
\pgfpathlineto{\pgfqpoint{2.091563in}{3.037990in}}%
\pgfpathlineto{\pgfqpoint{2.101991in}{3.074892in}}%
\pgfpathlineto{\pgfqpoint{2.112419in}{3.039117in}}%
\pgfpathlineto{\pgfqpoint{2.122848in}{3.061363in}}%
\pgfpathlineto{\pgfqpoint{2.133276in}{3.007310in}}%
\pgfpathlineto{\pgfqpoint{2.143704in}{3.045649in}}%
\pgfpathlineto{\pgfqpoint{2.154132in}{3.035637in}}%
\pgfpathlineto{\pgfqpoint{2.164560in}{2.989250in}}%
\pgfpathlineto{\pgfqpoint{2.174989in}{3.079485in}}%
\pgfpathlineto{\pgfqpoint{2.185417in}{3.008587in}}%
\pgfpathlineto{\pgfqpoint{2.195845in}{2.961452in}}%
\pgfpathlineto{\pgfqpoint{2.206273in}{2.994057in}}%
\pgfpathlineto{\pgfqpoint{2.216701in}{3.016783in}}%
\pgfpathlineto{\pgfqpoint{2.227129in}{2.981517in}}%
\pgfpathlineto{\pgfqpoint{2.237558in}{3.008793in}}%
\pgfpathlineto{\pgfqpoint{2.247986in}{2.995180in}}%
\pgfpathlineto{\pgfqpoint{2.258414in}{3.008597in}}%
\pgfpathlineto{\pgfqpoint{2.268842in}{3.012319in}}%
\pgfpathlineto{\pgfqpoint{2.279270in}{3.017783in}}%
\pgfpathlineto{\pgfqpoint{2.289698in}{3.000440in}}%
\pgfpathlineto{\pgfqpoint{2.300127in}{3.018835in}}%
\pgfpathlineto{\pgfqpoint{2.310555in}{2.978211in}}%
\pgfpathlineto{\pgfqpoint{2.320983in}{3.001413in}}%
\pgfpathlineto{\pgfqpoint{2.331411in}{2.955853in}}%
\pgfpathlineto{\pgfqpoint{2.341839in}{2.968932in}}%
\pgfpathlineto{\pgfqpoint{2.352267in}{2.865747in}}%
\pgfpathlineto{\pgfqpoint{2.362696in}{2.938021in}}%
\pgfpathlineto{\pgfqpoint{2.373124in}{2.955210in}}%
\pgfpathlineto{\pgfqpoint{2.383552in}{2.968537in}}%
\pgfpathlineto{\pgfqpoint{2.393980in}{2.955564in}}%
\pgfpathlineto{\pgfqpoint{2.404408in}{2.908480in}}%
\pgfpathlineto{\pgfqpoint{2.414837in}{2.894527in}}%
\pgfpathlineto{\pgfqpoint{2.425265in}{2.901874in}}%
\pgfpathlineto{\pgfqpoint{2.435693in}{2.915417in}}%
\pgfpathlineto{\pgfqpoint{2.446121in}{2.880119in}}%
\pgfpathlineto{\pgfqpoint{2.456549in}{2.854176in}}%
\pgfpathlineto{\pgfqpoint{2.466977in}{2.910893in}}%
\pgfpathlineto{\pgfqpoint{2.477406in}{2.889983in}}%
\pgfpathlineto{\pgfqpoint{2.487834in}{2.850461in}}%
\pgfpathlineto{\pgfqpoint{2.498262in}{2.828555in}}%
\pgfpathlineto{\pgfqpoint{2.508690in}{2.840750in}}%
\pgfpathlineto{\pgfqpoint{2.529546in}{2.808218in}}%
\pgfpathlineto{\pgfqpoint{2.539975in}{2.789735in}}%
\pgfpathlineto{\pgfqpoint{2.550403in}{2.826682in}}%
\pgfpathlineto{\pgfqpoint{2.581687in}{2.782241in}}%
\pgfpathlineto{\pgfqpoint{2.592115in}{2.791298in}}%
\pgfpathlineto{\pgfqpoint{2.602544in}{2.765317in}}%
\pgfpathlineto{\pgfqpoint{2.612972in}{2.727467in}}%
\pgfpathlineto{\pgfqpoint{2.623400in}{2.725689in}}%
\pgfpathlineto{\pgfqpoint{2.633828in}{2.693832in}}%
\pgfpathlineto{\pgfqpoint{2.644256in}{2.681967in}}%
\pgfpathlineto{\pgfqpoint{2.654685in}{2.703943in}}%
\pgfpathlineto{\pgfqpoint{2.665113in}{2.676783in}}%
\pgfpathlineto{\pgfqpoint{2.685969in}{2.486192in}}%
\pgfpathlineto{\pgfqpoint{2.696397in}{2.484263in}}%
\pgfpathlineto{\pgfqpoint{2.706825in}{2.385789in}}%
\pgfpathlineto{\pgfqpoint{2.727682in}{2.234070in}}%
\pgfpathlineto{\pgfqpoint{2.738110in}{2.241042in}}%
\pgfpathlineto{\pgfqpoint{2.748538in}{2.256589in}}%
\pgfpathlineto{\pgfqpoint{2.758966in}{2.217799in}}%
\pgfpathlineto{\pgfqpoint{2.769394in}{2.244529in}}%
\pgfpathlineto{\pgfqpoint{2.779823in}{2.177279in}}%
\pgfpathlineto{\pgfqpoint{2.790251in}{2.144111in}}%
\pgfpathlineto{\pgfqpoint{2.800679in}{2.154232in}}%
\pgfpathlineto{\pgfqpoint{2.811107in}{2.136003in}}%
\pgfpathlineto{\pgfqpoint{2.821535in}{2.167135in}}%
\pgfpathlineto{\pgfqpoint{2.831964in}{2.142806in}}%
\pgfpathlineto{\pgfqpoint{2.842392in}{2.093878in}}%
\pgfpathlineto{\pgfqpoint{2.852820in}{2.084253in}}%
\pgfpathlineto{\pgfqpoint{2.863248in}{2.065157in}}%
\pgfpathlineto{\pgfqpoint{2.873676in}{2.027474in}}%
\pgfpathlineto{\pgfqpoint{2.884104in}{2.019257in}}%
\pgfpathlineto{\pgfqpoint{2.894533in}{2.043086in}}%
\pgfpathlineto{\pgfqpoint{2.904961in}{2.076447in}}%
\pgfpathlineto{\pgfqpoint{2.915389in}{2.029002in}}%
\pgfpathlineto{\pgfqpoint{2.936245in}{1.998941in}}%
\pgfpathlineto{\pgfqpoint{2.946673in}{1.997446in}}%
\pgfpathlineto{\pgfqpoint{2.957102in}{1.994092in}}%
\pgfpathlineto{\pgfqpoint{2.967530in}{1.967379in}}%
\pgfpathlineto{\pgfqpoint{2.988386in}{1.944259in}}%
\pgfpathlineto{\pgfqpoint{2.998814in}{1.943238in}}%
\pgfpathlineto{\pgfqpoint{3.009242in}{1.977068in}}%
\pgfpathlineto{\pgfqpoint{3.019671in}{1.968828in}}%
\pgfpathlineto{\pgfqpoint{3.030099in}{1.975052in}}%
\pgfpathlineto{\pgfqpoint{3.040527in}{1.955643in}}%
\pgfpathlineto{\pgfqpoint{3.061383in}{1.931164in}}%
\pgfpathlineto{\pgfqpoint{3.071812in}{1.942089in}}%
\pgfpathlineto{\pgfqpoint{3.082240in}{1.926207in}}%
\pgfpathlineto{\pgfqpoint{3.092668in}{1.906567in}}%
\pgfpathlineto{\pgfqpoint{3.103096in}{1.893759in}}%
\pgfpathlineto{\pgfqpoint{3.113524in}{1.894675in}}%
\pgfpathlineto{\pgfqpoint{3.123952in}{1.877871in}}%
\pgfpathlineto{\pgfqpoint{3.134381in}{1.889450in}}%
\pgfpathlineto{\pgfqpoint{3.144809in}{1.923200in}}%
\pgfpathlineto{\pgfqpoint{3.155237in}{1.930827in}}%
\pgfpathlineto{\pgfqpoint{3.165665in}{1.946035in}}%
\pgfpathlineto{\pgfqpoint{3.176093in}{1.894782in}}%
\pgfpathlineto{\pgfqpoint{3.186521in}{1.933859in}}%
\pgfpathlineto{\pgfqpoint{3.196950in}{1.888347in}}%
\pgfpathlineto{\pgfqpoint{3.207378in}{1.867152in}}%
\pgfpathlineto{\pgfqpoint{3.217806in}{1.839411in}}%
\pgfpathlineto{\pgfqpoint{3.228234in}{1.866446in}}%
\pgfpathlineto{\pgfqpoint{3.238662in}{1.910794in}}%
\pgfpathlineto{\pgfqpoint{3.259519in}{1.845241in}}%
\pgfpathlineto{\pgfqpoint{3.269947in}{1.884157in}}%
\pgfpathlineto{\pgfqpoint{3.280375in}{1.910972in}}%
\pgfpathlineto{\pgfqpoint{3.290803in}{1.902841in}}%
\pgfpathlineto{\pgfqpoint{3.301231in}{1.826477in}}%
\pgfpathlineto{\pgfqpoint{3.311660in}{1.813499in}}%
\pgfpathlineto{\pgfqpoint{3.322088in}{1.851961in}}%
\pgfpathlineto{\pgfqpoint{3.332516in}{1.800659in}}%
\pgfpathlineto{\pgfqpoint{3.342944in}{1.818300in}}%
\pgfpathlineto{\pgfqpoint{3.353372in}{1.846140in}}%
\pgfpathlineto{\pgfqpoint{3.363800in}{1.853328in}}%
\pgfpathlineto{\pgfqpoint{3.374229in}{1.836762in}}%
\pgfpathlineto{\pgfqpoint{3.384657in}{1.868924in}}%
\pgfpathlineto{\pgfqpoint{3.395085in}{1.865489in}}%
\pgfpathlineto{\pgfqpoint{3.415941in}{1.787088in}}%
\pgfpathlineto{\pgfqpoint{3.426369in}{1.789884in}}%
\pgfpathlineto{\pgfqpoint{3.436798in}{1.809858in}}%
\pgfpathlineto{\pgfqpoint{3.447226in}{1.810497in}}%
\pgfpathlineto{\pgfqpoint{3.457654in}{1.796727in}}%
\pgfpathlineto{\pgfqpoint{3.468082in}{1.838394in}}%
\pgfpathlineto{\pgfqpoint{3.478510in}{1.791735in}}%
\pgfpathlineto{\pgfqpoint{3.488938in}{1.763737in}}%
\pgfpathlineto{\pgfqpoint{3.499367in}{1.803181in}}%
\pgfpathlineto{\pgfqpoint{3.509795in}{1.736402in}}%
\pgfpathlineto{\pgfqpoint{3.520223in}{1.820320in}}%
\pgfpathlineto{\pgfqpoint{3.530651in}{1.809167in}}%
\pgfpathlineto{\pgfqpoint{3.541079in}{1.859835in}}%
\pgfpathlineto{\pgfqpoint{3.551508in}{1.806165in}}%
\pgfpathlineto{\pgfqpoint{3.561936in}{1.837546in}}%
\pgfpathlineto{\pgfqpoint{3.572364in}{1.795968in}}%
\pgfpathlineto{\pgfqpoint{3.582792in}{1.776086in}}%
\pgfpathlineto{\pgfqpoint{3.593220in}{1.820595in}}%
\pgfpathlineto{\pgfqpoint{3.603648in}{1.798856in}}%
\pgfpathlineto{\pgfqpoint{3.614077in}{1.739493in}}%
\pgfpathlineto{\pgfqpoint{3.624505in}{1.750464in}}%
\pgfpathlineto{\pgfqpoint{3.634933in}{1.864163in}}%
\pgfpathlineto{\pgfqpoint{3.645361in}{1.739419in}}%
\pgfpathlineto{\pgfqpoint{3.655789in}{1.785684in}}%
\pgfpathlineto{\pgfqpoint{3.666217in}{1.700286in}}%
\pgfpathlineto{\pgfqpoint{3.676646in}{1.715306in}}%
\pgfpathlineto{\pgfqpoint{3.687074in}{1.781645in}}%
\pgfpathlineto{\pgfqpoint{3.697502in}{1.703578in}}%
\pgfpathlineto{\pgfqpoint{3.707930in}{1.733576in}}%
\pgfpathlineto{\pgfqpoint{3.718358in}{1.736326in}}%
\pgfpathlineto{\pgfqpoint{3.728787in}{1.707144in}}%
\pgfpathlineto{\pgfqpoint{3.739215in}{1.646678in}}%
\pgfpathlineto{\pgfqpoint{3.749643in}{1.627727in}}%
\pgfpathlineto{\pgfqpoint{3.760071in}{1.655667in}}%
\pgfpathlineto{\pgfqpoint{3.770499in}{1.722191in}}%
\pgfpathlineto{\pgfqpoint{3.780927in}{1.652917in}}%
\pgfpathlineto{\pgfqpoint{3.791356in}{1.712221in}}%
\pgfpathlineto{\pgfqpoint{3.801784in}{1.648965in}}%
\pgfpathlineto{\pgfqpoint{3.812212in}{1.718468in}}%
\pgfpathlineto{\pgfqpoint{3.822640in}{1.688712in}}%
\pgfpathlineto{\pgfqpoint{3.833068in}{1.776156in}}%
\pgfpathlineto{\pgfqpoint{3.843496in}{1.849342in}}%
\pgfpathlineto{\pgfqpoint{3.853925in}{1.669587in}}%
\pgfpathlineto{\pgfqpoint{3.864353in}{1.735706in}}%
\pgfpathlineto{\pgfqpoint{3.874781in}{1.738488in}}%
\pgfpathlineto{\pgfqpoint{3.885209in}{1.648899in}}%
\pgfpathlineto{\pgfqpoint{3.895637in}{1.712537in}}%
\pgfpathlineto{\pgfqpoint{3.906065in}{1.688505in}}%
\pgfpathlineto{\pgfqpoint{3.916494in}{1.739700in}}%
\pgfpathlineto{\pgfqpoint{3.926922in}{1.765996in}}%
\pgfpathlineto{\pgfqpoint{3.947778in}{1.665703in}}%
\pgfpathlineto{\pgfqpoint{3.958206in}{1.719437in}}%
\pgfpathlineto{\pgfqpoint{3.968635in}{1.711845in}}%
\pgfpathlineto{\pgfqpoint{3.979063in}{1.658909in}}%
\pgfpathlineto{\pgfqpoint{3.989491in}{1.651738in}}%
\pgfpathlineto{\pgfqpoint{3.999919in}{1.601370in}}%
\pgfpathlineto{\pgfqpoint{4.010347in}{1.512907in}}%
\pgfpathlineto{\pgfqpoint{4.020775in}{1.611893in}}%
\pgfpathlineto{\pgfqpoint{4.031204in}{1.581819in}}%
\pgfpathlineto{\pgfqpoint{4.041632in}{1.666784in}}%
\pgfpathlineto{\pgfqpoint{4.052060in}{1.610635in}}%
\pgfpathlineto{\pgfqpoint{4.062488in}{1.727821in}}%
\pgfpathlineto{\pgfqpoint{4.072916in}{1.600705in}}%
\pgfpathlineto{\pgfqpoint{4.083344in}{1.632203in}}%
\pgfpathlineto{\pgfqpoint{4.093773in}{1.716182in}}%
\pgfpathlineto{\pgfqpoint{4.104201in}{1.724395in}}%
\pgfpathlineto{\pgfqpoint{4.114629in}{1.648996in}}%
\pgfpathlineto{\pgfqpoint{4.125057in}{1.721807in}}%
\pgfpathlineto{\pgfqpoint{4.135485in}{1.741323in}}%
\pgfpathlineto{\pgfqpoint{4.145913in}{1.680052in}}%
\pgfpathlineto{\pgfqpoint{4.156342in}{1.541655in}}%
\pgfpathlineto{\pgfqpoint{4.166770in}{1.773879in}}%
\pgfpathlineto{\pgfqpoint{4.177198in}{1.633166in}}%
\pgfpathlineto{\pgfqpoint{4.187626in}{1.515737in}}%
\pgfpathlineto{\pgfqpoint{4.198054in}{1.752421in}}%
\pgfpathlineto{\pgfqpoint{4.208483in}{1.625182in}}%
\pgfpathlineto{\pgfqpoint{4.218911in}{1.588059in}}%
\pgfpathlineto{\pgfqpoint{4.229339in}{1.563946in}}%
\pgfpathlineto{\pgfqpoint{4.239767in}{1.565371in}}%
\pgfpathlineto{\pgfqpoint{4.250195in}{1.544087in}}%
\pgfpathlineto{\pgfqpoint{4.260623in}{1.509643in}}%
\pgfpathlineto{\pgfqpoint{4.271052in}{1.409577in}}%
\pgfpathlineto{\pgfqpoint{4.281480in}{1.719175in}}%
\pgfpathlineto{\pgfqpoint{4.291908in}{1.434158in}}%
\pgfpathlineto{\pgfqpoint{4.302336in}{1.435071in}}%
\pgfpathlineto{\pgfqpoint{4.312764in}{1.541139in}}%
\pgfpathlineto{\pgfqpoint{4.323192in}{1.545050in}}%
\pgfpathlineto{\pgfqpoint{4.333621in}{1.582017in}}%
\pgfpathlineto{\pgfqpoint{4.344049in}{1.481926in}}%
\pgfpathlineto{\pgfqpoint{4.354477in}{1.426841in}}%
\pgfpathlineto{\pgfqpoint{4.364905in}{1.695596in}}%
\pgfpathlineto{\pgfqpoint{4.375333in}{1.336201in}}%
\pgfpathlineto{\pgfqpoint{4.385761in}{1.557601in}}%
\pgfpathlineto{\pgfqpoint{4.396190in}{1.426529in}}%
\pgfpathlineto{\pgfqpoint{4.406618in}{1.573746in}}%
\pgfpathlineto{\pgfqpoint{4.417046in}{1.584763in}}%
\pgfpathlineto{\pgfqpoint{4.427474in}{1.428580in}}%
\pgfpathlineto{\pgfqpoint{4.437902in}{1.550168in}}%
\pgfpathlineto{\pgfqpoint{4.448331in}{1.700523in}}%
\pgfpathlineto{\pgfqpoint{4.458759in}{1.523099in}}%
\pgfpathlineto{\pgfqpoint{4.469187in}{1.402842in}}%
\pgfpathlineto{\pgfqpoint{4.479615in}{1.548742in}}%
\pgfpathlineto{\pgfqpoint{4.490043in}{1.558563in}}%
\pgfpathlineto{\pgfqpoint{4.500471in}{1.558429in}}%
\pgfpathlineto{\pgfqpoint{4.510900in}{1.473912in}}%
\pgfpathlineto{\pgfqpoint{4.521328in}{1.733298in}}%
\pgfpathlineto{\pgfqpoint{4.531756in}{1.821592in}}%
\pgfpathlineto{\pgfqpoint{4.542184in}{1.846383in}}%
\pgfpathlineto{\pgfqpoint{4.552612in}{1.809487in}}%
\pgfpathlineto{\pgfqpoint{4.563040in}{1.878037in}}%
\pgfpathlineto{\pgfqpoint{4.573469in}{1.866807in}}%
\pgfpathlineto{\pgfqpoint{4.583897in}{1.841903in}}%
\pgfpathlineto{\pgfqpoint{4.594325in}{1.747564in}}%
\pgfpathlineto{\pgfqpoint{4.604753in}{1.585159in}}%
\pgfpathlineto{\pgfqpoint{4.615181in}{1.588720in}}%
\pgfpathlineto{\pgfqpoint{4.625610in}{1.526406in}}%
\pgfpathlineto{\pgfqpoint{4.636038in}{1.383960in}}%
\pgfpathlineto{\pgfqpoint{4.646466in}{1.329048in}}%
\pgfpathlineto{\pgfqpoint{4.656894in}{1.481357in}}%
\pgfpathlineto{\pgfqpoint{4.667322in}{1.399236in}}%
\pgfpathlineto{\pgfqpoint{4.677750in}{1.498563in}}%
\pgfpathlineto{\pgfqpoint{4.688179in}{1.547198in}}%
\pgfpathlineto{\pgfqpoint{4.698607in}{1.434584in}}%
\pgfpathlineto{\pgfqpoint{4.709035in}{1.660892in}}%
\pgfpathlineto{\pgfqpoint{4.719463in}{1.441938in}}%
\pgfpathlineto{\pgfqpoint{4.729891in}{1.682141in}}%
\pgfpathlineto{\pgfqpoint{4.740319in}{1.264104in}}%
\pgfpathlineto{\pgfqpoint{4.750748in}{1.747403in}}%
\pgfpathlineto{\pgfqpoint{4.761176in}{1.651331in}}%
\pgfpathlineto{\pgfqpoint{4.771604in}{1.643823in}}%
\pgfpathlineto{\pgfqpoint{4.771604in}{1.643823in}}%
\pgfusepath{stroke}%
\end{pgfscope}%
\begin{pgfscope}%
\pgfpathrectangle{\pgfqpoint{0.610762in}{0.961156in}}{\pgfqpoint{4.171270in}{2.577986in}} %
\pgfusepath{clip}%
\pgfsetbuttcap%
\pgfsetroundjoin%
\pgfsetlinewidth{1.756562pt}%
\definecolor{currentstroke}{rgb}{0.501961,0.501961,0.501961}%
\pgfsetstrokecolor{currentstroke}%
\pgfsetdash{{6.000000pt}{6.000000pt}}{0.000000pt}%
\pgfpathmoveto{\pgfqpoint{2.696397in}{0.961156in}}%
\pgfpathlineto{\pgfqpoint{2.696397in}{3.539143in}}%
\pgfusepath{stroke}%
\end{pgfscope}%
\begin{pgfscope}%
\pgfsetrectcap%
\pgfsetmiterjoin%
\pgfsetlinewidth{1.254687pt}%
\definecolor{currentstroke}{rgb}{0.150000,0.150000,0.150000}%
\pgfsetstrokecolor{currentstroke}%
\pgfsetdash{}{0pt}%
\pgfpathmoveto{\pgfqpoint{0.610762in}{0.961156in}}%
\pgfpathlineto{\pgfqpoint{0.610762in}{3.539143in}}%
\pgfusepath{stroke}%
\end{pgfscope}%
\begin{pgfscope}%
\pgfsetrectcap%
\pgfsetmiterjoin%
\pgfsetlinewidth{1.254687pt}%
\definecolor{currentstroke}{rgb}{0.150000,0.150000,0.150000}%
\pgfsetstrokecolor{currentstroke}%
\pgfsetdash{}{0pt}%
\pgfpathmoveto{\pgfqpoint{0.610762in}{0.961156in}}%
\pgfpathlineto{\pgfqpoint{4.782032in}{0.961156in}}%
\pgfusepath{stroke}%
\end{pgfscope}%
\begin{pgfscope}%
\pgfsetroundcap%
\pgfsetroundjoin%
\pgfsetlinewidth{1.756562pt}%
\definecolor{currentstroke}{rgb}{0.200000,0.427451,0.650980}%
\pgfsetstrokecolor{currentstroke}%
\pgfsetstrokeopacity{0.800000}%
\pgfsetdash{}{0pt}%
\pgfpathmoveto{\pgfqpoint{4.894532in}{3.382893in}}%
\pgfpathlineto{\pgfqpoint{5.144532in}{3.382893in}}%
\pgfusepath{stroke}%
\end{pgfscope}%
\begin{pgfscope}%
\definecolor{textcolor}{rgb}{0.150000,0.150000,0.150000}%
\pgfsetstrokecolor{textcolor}%
\pgfsetfillcolor{textcolor}%
\pgftext[x=5.244532in,y=3.339143in,left,base]{\color{textcolor}\rmfamily\fontsize{9.000000}{10.800000}\selectfont WT + Vehicle}%
\end{pgfscope}%
\begin{pgfscope}%
\pgfsetroundcap%
\pgfsetroundjoin%
\pgfsetlinewidth{1.756562pt}%
\definecolor{currentstroke}{rgb}{0.168627,0.670588,0.494118}%
\pgfsetstrokecolor{currentstroke}%
\pgfsetstrokeopacity{0.800000}%
\pgfsetdash{}{0pt}%
\pgfpathmoveto{\pgfqpoint{4.894532in}{3.208593in}}%
\pgfpathlineto{\pgfqpoint{5.144532in}{3.208593in}}%
\pgfusepath{stroke}%
\end{pgfscope}%
\begin{pgfscope}%
\definecolor{textcolor}{rgb}{0.150000,0.150000,0.150000}%
\pgfsetstrokecolor{textcolor}%
\pgfsetfillcolor{textcolor}%
\pgftext[x=5.244532in,y=3.164843in,left,base]{\color{textcolor}\rmfamily\fontsize{9.000000}{10.800000}\selectfont WT + TAT-GluA2\textsubscript{3Y}}%
\end{pgfscope}%
\begin{pgfscope}%
\pgfsetroundcap%
\pgfsetroundjoin%
\pgfsetlinewidth{1.756562pt}%
\definecolor{currentstroke}{rgb}{1.000000,0.494118,0.250980}%
\pgfsetstrokecolor{currentstroke}%
\pgfsetstrokeopacity{0.800000}%
\pgfsetdash{}{0pt}%
\pgfpathmoveto{\pgfqpoint{4.894532in}{3.034294in}}%
\pgfpathlineto{\pgfqpoint{5.144532in}{3.034294in}}%
\pgfusepath{stroke}%
\end{pgfscope}%
\begin{pgfscope}%
\definecolor{textcolor}{rgb}{0.150000,0.150000,0.150000}%
\pgfsetstrokecolor{textcolor}%
\pgfsetfillcolor{textcolor}%
\pgftext[x=5.244532in,y=2.990544in,left,base]{\color{textcolor}\rmfamily\fontsize{9.000000}{10.800000}\selectfont Tg + Vehicle}%
\end{pgfscope}%
\begin{pgfscope}%
\pgfsetroundcap%
\pgfsetroundjoin%
\pgfsetlinewidth{1.756562pt}%
\definecolor{currentstroke}{rgb}{1.000000,0.694118,0.250980}%
\pgfsetstrokecolor{currentstroke}%
\pgfsetstrokeopacity{0.800000}%
\pgfsetdash{}{0pt}%
\pgfpathmoveto{\pgfqpoint{4.894532in}{2.859994in}}%
\pgfpathlineto{\pgfqpoint{5.144532in}{2.859994in}}%
\pgfusepath{stroke}%
\end{pgfscope}%
\begin{pgfscope}%
\definecolor{textcolor}{rgb}{0.150000,0.150000,0.150000}%
\pgfsetstrokecolor{textcolor}%
\pgfsetfillcolor{textcolor}%
\pgftext[x=5.244532in,y=2.816244in,left,base]{\color{textcolor}\rmfamily\fontsize{9.000000}{10.800000}\selectfont Tg + TAT-GluA2\textsubscript{3Y}}%
\end{pgfscope}%
\end{pgfpicture}%
\makeatother%
\endgroup%

    \caption[Normalized distance to the \gls{gsvm} classification boundary.]{The normalized distance to the \gls{gsvm} classification boundary. A positive distance represents the network state of not freezing, and a negative distance represent the network state of freezing. We found during freezing, the network state of the \gls{wt} groups are deep in the freezing state, while the \gls{tg} mice show a network state close to the classification boundary. This suggests that the freezing state in the \gls{tg} \gls{ca1} network is more likely to be disrupted by small perturbations in the network. \tglu{} treatment during training is able to rescue this deficit in \gls{tg} mice. \label{f.ad.cls-distance}}
\end{figure}

The result is shown in Figure~\ref{f.ad.cls-distance}. We found at the initiation of freezing behaviour, the neural state of \gls{wt} mice fell into the freezing classification boundary, as is shown with a significant negative distance. However in the \gls{tg} mice, the network state stayed close to the classification boundary. This result suggests that a small perturbation is more likely to shift the \gls{tg} mouse network state out of freezing, and this may result in a corresponding change of behavioural state from freezing to not freezing. This result suggests that the \gls{ca1} network state in \gls{tg} mice is unstable. \tglu{} treatment is able to rescue this deficit. 

\subsection{Memory deficits in \Gls{tg} mice are not a result of forgetting}

Recent studies have suggested that the memory deficit in early \gls{ad}, may not be a result of forgetting, as a previously learned memory in a mouse model of \gls{ad} can be artificially reactivated \citep{roy16}. Given that we are able to detect a neuronal signature for freezing in our \gls{tg} mice, and they are able to initiate freezing as often as \gls{wt} mice, we hypothesized that they still retain a representation of the fear memory, and their memory deficit is not a result of forgetting.

To test this hypothesis, we trained \gls{wt} and \gls{tg} mice treatment-free, and \SI{3}{\day} later, treated mice with \tglu{} or saline vehicle when they were briefly exposed to the training context as a reminder. The memory test occurred \SI{24}{\hour} later. The result is summarized in Figure~\ref{f.ad.reminder1}. We found no significant interaction between \textit{Genotype} and \textit{Treatment} (F\tsb{1,43}=1.3, p=0.2), but significant main effects of \textit{Genotype} (F\tsb{1,43}=9.02, p=0.004) and \textit{Treatment} (F\tsb{1,55}=7.3, p=0.010). \textit{Post hoc} tests showed that Tg-Veh had a significant memory deficit (Tg-Veh vs WT-Veh, T=2.92, p=0.006), however this deficit was fully rescued by \tglu{} treatment (Tg-\glu{} vs Tg-Veh, T=2.61, p=0.012; WT-Veh vs Tg-\glu{}, T=0.11, p=0.91). This result suggests that \gls{tg} mice retain a neural representation of the fear memory for at least \SI{3}{\day}, therefore the memory deficit \SI{1}{\day} after training is not a result of forgetting. In addition, the memory deficit can also be rescued with \tglu{} treatment if the mouse is exposed to a reminder.

\begin{figure}[h]
    \begin{subfigure}[h]{\textwidth}
        %% Creator: Matplotlib, PGF backend
%%
%% To include the figure in your LaTeX document, write
%%   \input{<filename>.pgf}
%%
%% Make sure the required packages are loaded in your preamble
%%   \usepackage{pgf}
%%
%% Figures using additional raster images can only be included by \input if
%% they are in the same directory as the main LaTeX file. For loading figures
%% from other directories you can use the `import` package
%%   \usepackage{import}
%% and then include the figures with
%%   \import{<path to file>}{<filename>.pgf}
%%
%% Matplotlib used the following preamble
%%   \usepackage[utf8]{inputenc}
%%   \usepackage[T1]{fontenc}
%%   \usepackage{siunitx}
%%
\begingroup%
\makeatletter%
\begin{pgfpicture}%
\pgfpathrectangle{\pgfpointorigin}{\pgfqpoint{6.000873in}{2.614199in}}%
\pgfusepath{use as bounding box, clip}%
\begin{pgfscope}%
\pgfsetbuttcap%
\pgfsetmiterjoin%
\definecolor{currentfill}{rgb}{1.000000,1.000000,1.000000}%
\pgfsetfillcolor{currentfill}%
\pgfsetlinewidth{0.000000pt}%
\definecolor{currentstroke}{rgb}{1.000000,1.000000,1.000000}%
\pgfsetstrokecolor{currentstroke}%
\pgfsetdash{}{0pt}%
\pgfpathmoveto{\pgfqpoint{0.000000in}{0.000000in}}%
\pgfpathlineto{\pgfqpoint{6.000873in}{0.000000in}}%
\pgfpathlineto{\pgfqpoint{6.000873in}{2.614199in}}%
\pgfpathlineto{\pgfqpoint{0.000000in}{2.614199in}}%
\pgfpathclose%
\pgfusepath{fill}%
\end{pgfscope}%
\begin{pgfscope}%
\pgfsetbuttcap%
\pgfsetmiterjoin%
\definecolor{currentfill}{rgb}{1.000000,1.000000,1.000000}%
\pgfsetfillcolor{currentfill}%
\pgfsetlinewidth{0.000000pt}%
\definecolor{currentstroke}{rgb}{0.000000,0.000000,0.000000}%
\pgfsetstrokecolor{currentstroke}%
\pgfsetstrokeopacity{0.000000}%
\pgfsetdash{}{0pt}%
\pgfpathmoveto{\pgfqpoint{0.536697in}{0.161328in}}%
\pgfpathlineto{\pgfqpoint{4.244492in}{0.161328in}}%
\pgfpathlineto{\pgfqpoint{4.244492in}{2.452871in}}%
\pgfpathlineto{\pgfqpoint{0.536697in}{2.452871in}}%
\pgfpathclose%
\pgfusepath{fill}%
\end{pgfscope}%
\begin{pgfscope}%
\pgfsetbuttcap%
\pgfsetroundjoin%
\definecolor{currentfill}{rgb}{0.150000,0.150000,0.150000}%
\pgfsetfillcolor{currentfill}%
\pgfsetlinewidth{1.003750pt}%
\definecolor{currentstroke}{rgb}{0.150000,0.150000,0.150000}%
\pgfsetstrokecolor{currentstroke}%
\pgfsetdash{}{0pt}%
\pgfsys@defobject{currentmarker}{\pgfqpoint{0.000000in}{0.000000in}}{\pgfqpoint{0.041667in}{0.000000in}}{%
\pgfpathmoveto{\pgfqpoint{0.000000in}{0.000000in}}%
\pgfpathlineto{\pgfqpoint{0.041667in}{0.000000in}}%
\pgfusepath{stroke,fill}%
}%
\begin{pgfscope}%
\pgfsys@transformshift{0.536697in}{0.161328in}%
\pgfsys@useobject{currentmarker}{}%
\end{pgfscope}%
\end{pgfscope}%
\begin{pgfscope}%
\definecolor{textcolor}{rgb}{0.150000,0.150000,0.150000}%
\pgfsetstrokecolor{textcolor}%
\pgfsetfillcolor{textcolor}%
\pgftext[x=0.439475in,y=0.161328in,right,]{\color{textcolor}\rmfamily\fontsize{10.000000}{12.000000}\selectfont \(\displaystyle 0\)}%
\end{pgfscope}%
\begin{pgfscope}%
\pgfsetbuttcap%
\pgfsetroundjoin%
\definecolor{currentfill}{rgb}{0.150000,0.150000,0.150000}%
\pgfsetfillcolor{currentfill}%
\pgfsetlinewidth{1.003750pt}%
\definecolor{currentstroke}{rgb}{0.150000,0.150000,0.150000}%
\pgfsetstrokecolor{currentstroke}%
\pgfsetdash{}{0pt}%
\pgfsys@defobject{currentmarker}{\pgfqpoint{0.000000in}{0.000000in}}{\pgfqpoint{0.041667in}{0.000000in}}{%
\pgfpathmoveto{\pgfqpoint{0.000000in}{0.000000in}}%
\pgfpathlineto{\pgfqpoint{0.041667in}{0.000000in}}%
\pgfusepath{stroke,fill}%
}%
\begin{pgfscope}%
\pgfsys@transformshift{0.536697in}{0.447771in}%
\pgfsys@useobject{currentmarker}{}%
\end{pgfscope}%
\end{pgfscope}%
\begin{pgfscope}%
\definecolor{textcolor}{rgb}{0.150000,0.150000,0.150000}%
\pgfsetstrokecolor{textcolor}%
\pgfsetfillcolor{textcolor}%
\pgftext[x=0.439475in,y=0.447771in,right,]{\color{textcolor}\rmfamily\fontsize{10.000000}{12.000000}\selectfont \(\displaystyle 10\)}%
\end{pgfscope}%
\begin{pgfscope}%
\pgfsetbuttcap%
\pgfsetroundjoin%
\definecolor{currentfill}{rgb}{0.150000,0.150000,0.150000}%
\pgfsetfillcolor{currentfill}%
\pgfsetlinewidth{1.003750pt}%
\definecolor{currentstroke}{rgb}{0.150000,0.150000,0.150000}%
\pgfsetstrokecolor{currentstroke}%
\pgfsetdash{}{0pt}%
\pgfsys@defobject{currentmarker}{\pgfqpoint{0.000000in}{0.000000in}}{\pgfqpoint{0.041667in}{0.000000in}}{%
\pgfpathmoveto{\pgfqpoint{0.000000in}{0.000000in}}%
\pgfpathlineto{\pgfqpoint{0.041667in}{0.000000in}}%
\pgfusepath{stroke,fill}%
}%
\begin{pgfscope}%
\pgfsys@transformshift{0.536697in}{0.734213in}%
\pgfsys@useobject{currentmarker}{}%
\end{pgfscope}%
\end{pgfscope}%
\begin{pgfscope}%
\definecolor{textcolor}{rgb}{0.150000,0.150000,0.150000}%
\pgfsetstrokecolor{textcolor}%
\pgfsetfillcolor{textcolor}%
\pgftext[x=0.439475in,y=0.734213in,right,]{\color{textcolor}\rmfamily\fontsize{10.000000}{12.000000}\selectfont \(\displaystyle 20\)}%
\end{pgfscope}%
\begin{pgfscope}%
\pgfsetbuttcap%
\pgfsetroundjoin%
\definecolor{currentfill}{rgb}{0.150000,0.150000,0.150000}%
\pgfsetfillcolor{currentfill}%
\pgfsetlinewidth{1.003750pt}%
\definecolor{currentstroke}{rgb}{0.150000,0.150000,0.150000}%
\pgfsetstrokecolor{currentstroke}%
\pgfsetdash{}{0pt}%
\pgfsys@defobject{currentmarker}{\pgfqpoint{0.000000in}{0.000000in}}{\pgfqpoint{0.041667in}{0.000000in}}{%
\pgfpathmoveto{\pgfqpoint{0.000000in}{0.000000in}}%
\pgfpathlineto{\pgfqpoint{0.041667in}{0.000000in}}%
\pgfusepath{stroke,fill}%
}%
\begin{pgfscope}%
\pgfsys@transformshift{0.536697in}{1.020656in}%
\pgfsys@useobject{currentmarker}{}%
\end{pgfscope}%
\end{pgfscope}%
\begin{pgfscope}%
\definecolor{textcolor}{rgb}{0.150000,0.150000,0.150000}%
\pgfsetstrokecolor{textcolor}%
\pgfsetfillcolor{textcolor}%
\pgftext[x=0.439475in,y=1.020656in,right,]{\color{textcolor}\rmfamily\fontsize{10.000000}{12.000000}\selectfont \(\displaystyle 30\)}%
\end{pgfscope}%
\begin{pgfscope}%
\pgfsetbuttcap%
\pgfsetroundjoin%
\definecolor{currentfill}{rgb}{0.150000,0.150000,0.150000}%
\pgfsetfillcolor{currentfill}%
\pgfsetlinewidth{1.003750pt}%
\definecolor{currentstroke}{rgb}{0.150000,0.150000,0.150000}%
\pgfsetstrokecolor{currentstroke}%
\pgfsetdash{}{0pt}%
\pgfsys@defobject{currentmarker}{\pgfqpoint{0.000000in}{0.000000in}}{\pgfqpoint{0.041667in}{0.000000in}}{%
\pgfpathmoveto{\pgfqpoint{0.000000in}{0.000000in}}%
\pgfpathlineto{\pgfqpoint{0.041667in}{0.000000in}}%
\pgfusepath{stroke,fill}%
}%
\begin{pgfscope}%
\pgfsys@transformshift{0.536697in}{1.307099in}%
\pgfsys@useobject{currentmarker}{}%
\end{pgfscope}%
\end{pgfscope}%
\begin{pgfscope}%
\definecolor{textcolor}{rgb}{0.150000,0.150000,0.150000}%
\pgfsetstrokecolor{textcolor}%
\pgfsetfillcolor{textcolor}%
\pgftext[x=0.439475in,y=1.307099in,right,]{\color{textcolor}\rmfamily\fontsize{10.000000}{12.000000}\selectfont \(\displaystyle 40\)}%
\end{pgfscope}%
\begin{pgfscope}%
\pgfsetbuttcap%
\pgfsetroundjoin%
\definecolor{currentfill}{rgb}{0.150000,0.150000,0.150000}%
\pgfsetfillcolor{currentfill}%
\pgfsetlinewidth{1.003750pt}%
\definecolor{currentstroke}{rgb}{0.150000,0.150000,0.150000}%
\pgfsetstrokecolor{currentstroke}%
\pgfsetdash{}{0pt}%
\pgfsys@defobject{currentmarker}{\pgfqpoint{0.000000in}{0.000000in}}{\pgfqpoint{0.041667in}{0.000000in}}{%
\pgfpathmoveto{\pgfqpoint{0.000000in}{0.000000in}}%
\pgfpathlineto{\pgfqpoint{0.041667in}{0.000000in}}%
\pgfusepath{stroke,fill}%
}%
\begin{pgfscope}%
\pgfsys@transformshift{0.536697in}{1.593542in}%
\pgfsys@useobject{currentmarker}{}%
\end{pgfscope}%
\end{pgfscope}%
\begin{pgfscope}%
\definecolor{textcolor}{rgb}{0.150000,0.150000,0.150000}%
\pgfsetstrokecolor{textcolor}%
\pgfsetfillcolor{textcolor}%
\pgftext[x=0.439475in,y=1.593542in,right,]{\color{textcolor}\rmfamily\fontsize{10.000000}{12.000000}\selectfont \(\displaystyle 50\)}%
\end{pgfscope}%
\begin{pgfscope}%
\pgfsetbuttcap%
\pgfsetroundjoin%
\definecolor{currentfill}{rgb}{0.150000,0.150000,0.150000}%
\pgfsetfillcolor{currentfill}%
\pgfsetlinewidth{1.003750pt}%
\definecolor{currentstroke}{rgb}{0.150000,0.150000,0.150000}%
\pgfsetstrokecolor{currentstroke}%
\pgfsetdash{}{0pt}%
\pgfsys@defobject{currentmarker}{\pgfqpoint{0.000000in}{0.000000in}}{\pgfqpoint{0.041667in}{0.000000in}}{%
\pgfpathmoveto{\pgfqpoint{0.000000in}{0.000000in}}%
\pgfpathlineto{\pgfqpoint{0.041667in}{0.000000in}}%
\pgfusepath{stroke,fill}%
}%
\begin{pgfscope}%
\pgfsys@transformshift{0.536697in}{1.879985in}%
\pgfsys@useobject{currentmarker}{}%
\end{pgfscope}%
\end{pgfscope}%
\begin{pgfscope}%
\definecolor{textcolor}{rgb}{0.150000,0.150000,0.150000}%
\pgfsetstrokecolor{textcolor}%
\pgfsetfillcolor{textcolor}%
\pgftext[x=0.439475in,y=1.879985in,right,]{\color{textcolor}\rmfamily\fontsize{10.000000}{12.000000}\selectfont \(\displaystyle 60\)}%
\end{pgfscope}%
\begin{pgfscope}%
\pgfsetbuttcap%
\pgfsetroundjoin%
\definecolor{currentfill}{rgb}{0.150000,0.150000,0.150000}%
\pgfsetfillcolor{currentfill}%
\pgfsetlinewidth{1.003750pt}%
\definecolor{currentstroke}{rgb}{0.150000,0.150000,0.150000}%
\pgfsetstrokecolor{currentstroke}%
\pgfsetdash{}{0pt}%
\pgfsys@defobject{currentmarker}{\pgfqpoint{0.000000in}{0.000000in}}{\pgfqpoint{0.041667in}{0.000000in}}{%
\pgfpathmoveto{\pgfqpoint{0.000000in}{0.000000in}}%
\pgfpathlineto{\pgfqpoint{0.041667in}{0.000000in}}%
\pgfusepath{stroke,fill}%
}%
\begin{pgfscope}%
\pgfsys@transformshift{0.536697in}{2.166428in}%
\pgfsys@useobject{currentmarker}{}%
\end{pgfscope}%
\end{pgfscope}%
\begin{pgfscope}%
\definecolor{textcolor}{rgb}{0.150000,0.150000,0.150000}%
\pgfsetstrokecolor{textcolor}%
\pgfsetfillcolor{textcolor}%
\pgftext[x=0.439475in,y=2.166428in,right,]{\color{textcolor}\rmfamily\fontsize{10.000000}{12.000000}\selectfont \(\displaystyle 70\)}%
\end{pgfscope}%
\begin{pgfscope}%
\pgfsetbuttcap%
\pgfsetroundjoin%
\definecolor{currentfill}{rgb}{0.150000,0.150000,0.150000}%
\pgfsetfillcolor{currentfill}%
\pgfsetlinewidth{1.003750pt}%
\definecolor{currentstroke}{rgb}{0.150000,0.150000,0.150000}%
\pgfsetstrokecolor{currentstroke}%
\pgfsetdash{}{0pt}%
\pgfsys@defobject{currentmarker}{\pgfqpoint{0.000000in}{0.000000in}}{\pgfqpoint{0.041667in}{0.000000in}}{%
\pgfpathmoveto{\pgfqpoint{0.000000in}{0.000000in}}%
\pgfpathlineto{\pgfqpoint{0.041667in}{0.000000in}}%
\pgfusepath{stroke,fill}%
}%
\begin{pgfscope}%
\pgfsys@transformshift{0.536697in}{2.452871in}%
\pgfsys@useobject{currentmarker}{}%
\end{pgfscope}%
\end{pgfscope}%
\begin{pgfscope}%
\definecolor{textcolor}{rgb}{0.150000,0.150000,0.150000}%
\pgfsetstrokecolor{textcolor}%
\pgfsetfillcolor{textcolor}%
\pgftext[x=0.439475in,y=2.452871in,right,]{\color{textcolor}\rmfamily\fontsize{10.000000}{12.000000}\selectfont \(\displaystyle 80\)}%
\end{pgfscope}%
\begin{pgfscope}%
\definecolor{textcolor}{rgb}{0.150000,0.150000,0.150000}%
\pgfsetstrokecolor{textcolor}%
\pgfsetfillcolor{textcolor}%
\pgftext[x=0.231141in,y=1.307099in,,bottom,rotate=90.000000]{\color{textcolor}\rmfamily\fontsize{10.000000}{12.000000}\selectfont \textbf{\% Freezing}}%
\end{pgfscope}%
\begin{pgfscope}%
\pgfpathrectangle{\pgfqpoint{0.536697in}{0.161328in}}{\pgfqpoint{3.707795in}{2.291544in}} %
\pgfusepath{clip}%
\pgfsetbuttcap%
\pgfsetmiterjoin%
\definecolor{currentfill}{rgb}{0.200000,0.427451,0.650980}%
\pgfsetfillcolor{currentfill}%
\pgfsetlinewidth{1.505625pt}%
\definecolor{currentstroke}{rgb}{0.200000,0.427451,0.650980}%
\pgfsetstrokecolor{currentstroke}%
\pgfsetdash{}{0pt}%
\pgfpathmoveto{\pgfqpoint{0.669118in}{0.161328in}}%
\pgfpathlineto{\pgfqpoint{1.331224in}{0.161328in}}%
\pgfpathlineto{\pgfqpoint{1.331224in}{1.176768in}}%
\pgfpathlineto{\pgfqpoint{0.669118in}{1.176768in}}%
\pgfpathclose%
\pgfusepath{stroke,fill}%
\end{pgfscope}%
\begin{pgfscope}%
\pgfpathrectangle{\pgfqpoint{0.536697in}{0.161328in}}{\pgfqpoint{3.707795in}{2.291544in}} %
\pgfusepath{clip}%
\pgfsetbuttcap%
\pgfsetmiterjoin%
\definecolor{currentfill}{rgb}{0.168627,0.670588,0.494118}%
\pgfsetfillcolor{currentfill}%
\pgfsetlinewidth{1.505625pt}%
\definecolor{currentstroke}{rgb}{0.168627,0.670588,0.494118}%
\pgfsetstrokecolor{currentstroke}%
\pgfsetdash{}{0pt}%
\pgfpathmoveto{\pgfqpoint{1.596067in}{0.161328in}}%
\pgfpathlineto{\pgfqpoint{2.258173in}{0.161328in}}%
\pgfpathlineto{\pgfqpoint{2.258173in}{1.571200in}}%
\pgfpathlineto{\pgfqpoint{1.596067in}{1.571200in}}%
\pgfpathclose%
\pgfusepath{stroke,fill}%
\end{pgfscope}%
\begin{pgfscope}%
\pgfpathrectangle{\pgfqpoint{0.536697in}{0.161328in}}{\pgfqpoint{3.707795in}{2.291544in}} %
\pgfusepath{clip}%
\pgfsetbuttcap%
\pgfsetmiterjoin%
\definecolor{currentfill}{rgb}{1.000000,0.494118,0.250980}%
\pgfsetfillcolor{currentfill}%
\pgfsetlinewidth{1.505625pt}%
\definecolor{currentstroke}{rgb}{1.000000,0.494118,0.250980}%
\pgfsetstrokecolor{currentstroke}%
\pgfsetdash{}{0pt}%
\pgfpathmoveto{\pgfqpoint{2.523016in}{0.161328in}}%
\pgfpathlineto{\pgfqpoint{3.185122in}{0.161328in}}%
\pgfpathlineto{\pgfqpoint{3.185122in}{0.227267in}}%
\pgfpathlineto{\pgfqpoint{2.523016in}{0.227267in}}%
\pgfpathclose%
\pgfusepath{stroke,fill}%
\end{pgfscope}%
\begin{pgfscope}%
\pgfpathrectangle{\pgfqpoint{0.536697in}{0.161328in}}{\pgfqpoint{3.707795in}{2.291544in}} %
\pgfusepath{clip}%
\pgfsetbuttcap%
\pgfsetmiterjoin%
\definecolor{currentfill}{rgb}{1.000000,0.694118,0.250980}%
\pgfsetfillcolor{currentfill}%
\pgfsetlinewidth{1.505625pt}%
\definecolor{currentstroke}{rgb}{1.000000,0.694118,0.250980}%
\pgfsetstrokecolor{currentstroke}%
\pgfsetdash{}{0pt}%
\pgfpathmoveto{\pgfqpoint{3.449965in}{0.161328in}}%
\pgfpathlineto{\pgfqpoint{4.112071in}{0.161328in}}%
\pgfpathlineto{\pgfqpoint{4.112071in}{1.141822in}}%
\pgfpathlineto{\pgfqpoint{3.449965in}{1.141822in}}%
\pgfpathclose%
\pgfusepath{stroke,fill}%
\end{pgfscope}%
\begin{pgfscope}%
\pgfpathrectangle{\pgfqpoint{0.536697in}{0.161328in}}{\pgfqpoint{3.707795in}{2.291544in}} %
\pgfusepath{clip}%
\pgfsetbuttcap%
\pgfsetroundjoin%
\pgfsetlinewidth{1.505625pt}%
\definecolor{currentstroke}{rgb}{0.200000,0.427451,0.650980}%
\pgfsetstrokecolor{currentstroke}%
\pgfsetdash{}{0pt}%
\pgfpathmoveto{\pgfqpoint{1.000171in}{1.176768in}}%
\pgfpathlineto{\pgfqpoint{1.000171in}{1.385585in}}%
\pgfusepath{stroke}%
\end{pgfscope}%
\begin{pgfscope}%
\pgfpathrectangle{\pgfqpoint{0.536697in}{0.161328in}}{\pgfqpoint{3.707795in}{2.291544in}} %
\pgfusepath{clip}%
\pgfsetbuttcap%
\pgfsetroundjoin%
\pgfsetlinewidth{1.505625pt}%
\definecolor{currentstroke}{rgb}{0.168627,0.670588,0.494118}%
\pgfsetstrokecolor{currentstroke}%
\pgfsetdash{}{0pt}%
\pgfpathmoveto{\pgfqpoint{1.927120in}{1.571200in}}%
\pgfpathlineto{\pgfqpoint{1.927120in}{1.806656in}}%
\pgfusepath{stroke}%
\end{pgfscope}%
\begin{pgfscope}%
\pgfpathrectangle{\pgfqpoint{0.536697in}{0.161328in}}{\pgfqpoint{3.707795in}{2.291544in}} %
\pgfusepath{clip}%
\pgfsetbuttcap%
\pgfsetroundjoin%
\pgfsetlinewidth{1.505625pt}%
\definecolor{currentstroke}{rgb}{1.000000,0.494118,0.250980}%
\pgfsetstrokecolor{currentstroke}%
\pgfsetdash{}{0pt}%
\pgfpathmoveto{\pgfqpoint{2.854069in}{0.227267in}}%
\pgfpathlineto{\pgfqpoint{2.854069in}{0.250956in}}%
\pgfusepath{stroke}%
\end{pgfscope}%
\begin{pgfscope}%
\pgfpathrectangle{\pgfqpoint{0.536697in}{0.161328in}}{\pgfqpoint{3.707795in}{2.291544in}} %
\pgfusepath{clip}%
\pgfsetbuttcap%
\pgfsetroundjoin%
\pgfsetlinewidth{1.505625pt}%
\definecolor{currentstroke}{rgb}{1.000000,0.694118,0.250980}%
\pgfsetstrokecolor{currentstroke}%
\pgfsetdash{}{0pt}%
\pgfpathmoveto{\pgfqpoint{3.781018in}{1.141822in}}%
\pgfpathlineto{\pgfqpoint{3.781018in}{1.378137in}}%
\pgfusepath{stroke}%
\end{pgfscope}%
\begin{pgfscope}%
\pgfpathrectangle{\pgfqpoint{0.536697in}{0.161328in}}{\pgfqpoint{3.707795in}{2.291544in}} %
\pgfusepath{clip}%
\pgfsetbuttcap%
\pgfsetroundjoin%
\definecolor{currentfill}{rgb}{0.200000,0.427451,0.650980}%
\pgfsetfillcolor{currentfill}%
\pgfsetlinewidth{1.505625pt}%
\definecolor{currentstroke}{rgb}{0.200000,0.427451,0.650980}%
\pgfsetstrokecolor{currentstroke}%
\pgfsetdash{}{0pt}%
\pgfsys@defobject{currentmarker}{\pgfqpoint{-0.111111in}{-0.000000in}}{\pgfqpoint{0.111111in}{0.000000in}}{%
\pgfpathmoveto{\pgfqpoint{0.111111in}{-0.000000in}}%
\pgfpathlineto{\pgfqpoint{-0.111111in}{0.000000in}}%
\pgfusepath{stroke,fill}%
}%
\begin{pgfscope}%
\pgfsys@transformshift{1.000171in}{1.176768in}%
\pgfsys@useobject{currentmarker}{}%
\end{pgfscope}%
\end{pgfscope}%
\begin{pgfscope}%
\pgfpathrectangle{\pgfqpoint{0.536697in}{0.161328in}}{\pgfqpoint{3.707795in}{2.291544in}} %
\pgfusepath{clip}%
\pgfsetbuttcap%
\pgfsetroundjoin%
\definecolor{currentfill}{rgb}{0.200000,0.427451,0.650980}%
\pgfsetfillcolor{currentfill}%
\pgfsetlinewidth{1.505625pt}%
\definecolor{currentstroke}{rgb}{0.200000,0.427451,0.650980}%
\pgfsetstrokecolor{currentstroke}%
\pgfsetdash{}{0pt}%
\pgfsys@defobject{currentmarker}{\pgfqpoint{-0.111111in}{-0.000000in}}{\pgfqpoint{0.111111in}{0.000000in}}{%
\pgfpathmoveto{\pgfqpoint{0.111111in}{-0.000000in}}%
\pgfpathlineto{\pgfqpoint{-0.111111in}{0.000000in}}%
\pgfusepath{stroke,fill}%
}%
\begin{pgfscope}%
\pgfsys@transformshift{1.000171in}{1.385585in}%
\pgfsys@useobject{currentmarker}{}%
\end{pgfscope}%
\end{pgfscope}%
\begin{pgfscope}%
\pgfpathrectangle{\pgfqpoint{0.536697in}{0.161328in}}{\pgfqpoint{3.707795in}{2.291544in}} %
\pgfusepath{clip}%
\pgfsetbuttcap%
\pgfsetroundjoin%
\definecolor{currentfill}{rgb}{0.168627,0.670588,0.494118}%
\pgfsetfillcolor{currentfill}%
\pgfsetlinewidth{1.505625pt}%
\definecolor{currentstroke}{rgb}{0.168627,0.670588,0.494118}%
\pgfsetstrokecolor{currentstroke}%
\pgfsetdash{}{0pt}%
\pgfsys@defobject{currentmarker}{\pgfqpoint{-0.111111in}{-0.000000in}}{\pgfqpoint{0.111111in}{0.000000in}}{%
\pgfpathmoveto{\pgfqpoint{0.111111in}{-0.000000in}}%
\pgfpathlineto{\pgfqpoint{-0.111111in}{0.000000in}}%
\pgfusepath{stroke,fill}%
}%
\begin{pgfscope}%
\pgfsys@transformshift{1.927120in}{1.571200in}%
\pgfsys@useobject{currentmarker}{}%
\end{pgfscope}%
\end{pgfscope}%
\begin{pgfscope}%
\pgfpathrectangle{\pgfqpoint{0.536697in}{0.161328in}}{\pgfqpoint{3.707795in}{2.291544in}} %
\pgfusepath{clip}%
\pgfsetbuttcap%
\pgfsetroundjoin%
\definecolor{currentfill}{rgb}{0.168627,0.670588,0.494118}%
\pgfsetfillcolor{currentfill}%
\pgfsetlinewidth{1.505625pt}%
\definecolor{currentstroke}{rgb}{0.168627,0.670588,0.494118}%
\pgfsetstrokecolor{currentstroke}%
\pgfsetdash{}{0pt}%
\pgfsys@defobject{currentmarker}{\pgfqpoint{-0.111111in}{-0.000000in}}{\pgfqpoint{0.111111in}{0.000000in}}{%
\pgfpathmoveto{\pgfqpoint{0.111111in}{-0.000000in}}%
\pgfpathlineto{\pgfqpoint{-0.111111in}{0.000000in}}%
\pgfusepath{stroke,fill}%
}%
\begin{pgfscope}%
\pgfsys@transformshift{1.927120in}{1.806656in}%
\pgfsys@useobject{currentmarker}{}%
\end{pgfscope}%
\end{pgfscope}%
\begin{pgfscope}%
\pgfpathrectangle{\pgfqpoint{0.536697in}{0.161328in}}{\pgfqpoint{3.707795in}{2.291544in}} %
\pgfusepath{clip}%
\pgfsetbuttcap%
\pgfsetroundjoin%
\definecolor{currentfill}{rgb}{1.000000,0.494118,0.250980}%
\pgfsetfillcolor{currentfill}%
\pgfsetlinewidth{1.505625pt}%
\definecolor{currentstroke}{rgb}{1.000000,0.494118,0.250980}%
\pgfsetstrokecolor{currentstroke}%
\pgfsetdash{}{0pt}%
\pgfsys@defobject{currentmarker}{\pgfqpoint{-0.111111in}{-0.000000in}}{\pgfqpoint{0.111111in}{0.000000in}}{%
\pgfpathmoveto{\pgfqpoint{0.111111in}{-0.000000in}}%
\pgfpathlineto{\pgfqpoint{-0.111111in}{0.000000in}}%
\pgfusepath{stroke,fill}%
}%
\begin{pgfscope}%
\pgfsys@transformshift{2.854069in}{0.227267in}%
\pgfsys@useobject{currentmarker}{}%
\end{pgfscope}%
\end{pgfscope}%
\begin{pgfscope}%
\pgfpathrectangle{\pgfqpoint{0.536697in}{0.161328in}}{\pgfqpoint{3.707795in}{2.291544in}} %
\pgfusepath{clip}%
\pgfsetbuttcap%
\pgfsetroundjoin%
\definecolor{currentfill}{rgb}{1.000000,0.494118,0.250980}%
\pgfsetfillcolor{currentfill}%
\pgfsetlinewidth{1.505625pt}%
\definecolor{currentstroke}{rgb}{1.000000,0.494118,0.250980}%
\pgfsetstrokecolor{currentstroke}%
\pgfsetdash{}{0pt}%
\pgfsys@defobject{currentmarker}{\pgfqpoint{-0.111111in}{-0.000000in}}{\pgfqpoint{0.111111in}{0.000000in}}{%
\pgfpathmoveto{\pgfqpoint{0.111111in}{-0.000000in}}%
\pgfpathlineto{\pgfqpoint{-0.111111in}{0.000000in}}%
\pgfusepath{stroke,fill}%
}%
\begin{pgfscope}%
\pgfsys@transformshift{2.854069in}{0.250956in}%
\pgfsys@useobject{currentmarker}{}%
\end{pgfscope}%
\end{pgfscope}%
\begin{pgfscope}%
\pgfpathrectangle{\pgfqpoint{0.536697in}{0.161328in}}{\pgfqpoint{3.707795in}{2.291544in}} %
\pgfusepath{clip}%
\pgfsetbuttcap%
\pgfsetroundjoin%
\definecolor{currentfill}{rgb}{1.000000,0.694118,0.250980}%
\pgfsetfillcolor{currentfill}%
\pgfsetlinewidth{1.505625pt}%
\definecolor{currentstroke}{rgb}{1.000000,0.694118,0.250980}%
\pgfsetstrokecolor{currentstroke}%
\pgfsetdash{}{0pt}%
\pgfsys@defobject{currentmarker}{\pgfqpoint{-0.111111in}{-0.000000in}}{\pgfqpoint{0.111111in}{0.000000in}}{%
\pgfpathmoveto{\pgfqpoint{0.111111in}{-0.000000in}}%
\pgfpathlineto{\pgfqpoint{-0.111111in}{0.000000in}}%
\pgfusepath{stroke,fill}%
}%
\begin{pgfscope}%
\pgfsys@transformshift{3.781018in}{1.141822in}%
\pgfsys@useobject{currentmarker}{}%
\end{pgfscope}%
\end{pgfscope}%
\begin{pgfscope}%
\pgfpathrectangle{\pgfqpoint{0.536697in}{0.161328in}}{\pgfqpoint{3.707795in}{2.291544in}} %
\pgfusepath{clip}%
\pgfsetbuttcap%
\pgfsetroundjoin%
\definecolor{currentfill}{rgb}{1.000000,0.694118,0.250980}%
\pgfsetfillcolor{currentfill}%
\pgfsetlinewidth{1.505625pt}%
\definecolor{currentstroke}{rgb}{1.000000,0.694118,0.250980}%
\pgfsetstrokecolor{currentstroke}%
\pgfsetdash{}{0pt}%
\pgfsys@defobject{currentmarker}{\pgfqpoint{-0.111111in}{-0.000000in}}{\pgfqpoint{0.111111in}{0.000000in}}{%
\pgfpathmoveto{\pgfqpoint{0.111111in}{-0.000000in}}%
\pgfpathlineto{\pgfqpoint{-0.111111in}{0.000000in}}%
\pgfusepath{stroke,fill}%
}%
\begin{pgfscope}%
\pgfsys@transformshift{3.781018in}{1.378137in}%
\pgfsys@useobject{currentmarker}{}%
\end{pgfscope}%
\end{pgfscope}%
\begin{pgfscope}%
\pgfpathrectangle{\pgfqpoint{0.536697in}{0.161328in}}{\pgfqpoint{3.707795in}{2.291544in}} %
\pgfusepath{clip}%
\pgfsetroundcap%
\pgfsetroundjoin%
\pgfsetlinewidth{1.756562pt}%
\definecolor{currentstroke}{rgb}{0.627451,0.627451,0.643137}%
\pgfsetstrokecolor{currentstroke}%
\pgfsetdash{}{0pt}%
\pgfpathmoveto{\pgfqpoint{1.000171in}{1.463710in}}%
\pgfpathlineto{\pgfqpoint{1.000171in}{2.014989in}}%
\pgfusepath{stroke}%
\end{pgfscope}%
\begin{pgfscope}%
\pgfpathrectangle{\pgfqpoint{0.536697in}{0.161328in}}{\pgfqpoint{3.707795in}{2.291544in}} %
\pgfusepath{clip}%
\pgfsetroundcap%
\pgfsetroundjoin%
\pgfsetlinewidth{1.756562pt}%
\definecolor{currentstroke}{rgb}{0.627451,0.627451,0.643137}%
\pgfsetstrokecolor{currentstroke}%
\pgfsetdash{}{0pt}%
\pgfpathmoveto{\pgfqpoint{1.000171in}{2.014989in}}%
\pgfpathlineto{\pgfqpoint{2.854069in}{2.014989in}}%
\pgfusepath{stroke}%
\end{pgfscope}%
\begin{pgfscope}%
\pgfpathrectangle{\pgfqpoint{0.536697in}{0.161328in}}{\pgfqpoint{3.707795in}{2.291544in}} %
\pgfusepath{clip}%
\pgfsetroundcap%
\pgfsetroundjoin%
\pgfsetlinewidth{1.756562pt}%
\definecolor{currentstroke}{rgb}{0.627451,0.627451,0.643137}%
\pgfsetstrokecolor{currentstroke}%
\pgfsetdash{}{0pt}%
\pgfpathmoveto{\pgfqpoint{2.854069in}{2.014989in}}%
\pgfpathlineto{\pgfqpoint{2.854069in}{0.407206in}}%
\pgfusepath{stroke}%
\end{pgfscope}%
\begin{pgfscope}%
\pgfpathrectangle{\pgfqpoint{0.536697in}{0.161328in}}{\pgfqpoint{3.707795in}{2.291544in}} %
\pgfusepath{clip}%
\pgfsetroundcap%
\pgfsetroundjoin%
\pgfsetlinewidth{1.756562pt}%
\definecolor{currentstroke}{rgb}{0.627451,0.627451,0.643137}%
\pgfsetstrokecolor{currentstroke}%
\pgfsetdash{}{0pt}%
\pgfpathmoveto{\pgfqpoint{2.854069in}{2.093114in}}%
\pgfpathlineto{\pgfqpoint{2.854069in}{2.223323in}}%
\pgfusepath{stroke}%
\end{pgfscope}%
\begin{pgfscope}%
\pgfpathrectangle{\pgfqpoint{0.536697in}{0.161328in}}{\pgfqpoint{3.707795in}{2.291544in}} %
\pgfusepath{clip}%
\pgfsetroundcap%
\pgfsetroundjoin%
\pgfsetlinewidth{1.756562pt}%
\definecolor{currentstroke}{rgb}{0.627451,0.627451,0.643137}%
\pgfsetstrokecolor{currentstroke}%
\pgfsetdash{}{0pt}%
\pgfpathmoveto{\pgfqpoint{2.854069in}{2.223323in}}%
\pgfpathlineto{\pgfqpoint{3.781018in}{2.223323in}}%
\pgfusepath{stroke}%
\end{pgfscope}%
\begin{pgfscope}%
\pgfpathrectangle{\pgfqpoint{0.536697in}{0.161328in}}{\pgfqpoint{3.707795in}{2.291544in}} %
\pgfusepath{clip}%
\pgfsetroundcap%
\pgfsetroundjoin%
\pgfsetlinewidth{1.756562pt}%
\definecolor{currentstroke}{rgb}{0.627451,0.627451,0.643137}%
\pgfsetstrokecolor{currentstroke}%
\pgfsetdash{}{0pt}%
\pgfpathmoveto{\pgfqpoint{3.781018in}{2.223323in}}%
\pgfpathlineto{\pgfqpoint{3.781018in}{1.534387in}}%
\pgfusepath{stroke}%
\end{pgfscope}%
\begin{pgfscope}%
\pgfsetrectcap%
\pgfsetmiterjoin%
\pgfsetlinewidth{1.254687pt}%
\definecolor{currentstroke}{rgb}{0.150000,0.150000,0.150000}%
\pgfsetstrokecolor{currentstroke}%
\pgfsetdash{}{0pt}%
\pgfpathmoveto{\pgfqpoint{0.536697in}{0.161328in}}%
\pgfpathlineto{\pgfqpoint{0.536697in}{2.452871in}}%
\pgfusepath{stroke}%
\end{pgfscope}%
\begin{pgfscope}%
\pgfsetrectcap%
\pgfsetmiterjoin%
\pgfsetlinewidth{1.254687pt}%
\definecolor{currentstroke}{rgb}{0.150000,0.150000,0.150000}%
\pgfsetstrokecolor{currentstroke}%
\pgfsetdash{}{0pt}%
\pgfpathmoveto{\pgfqpoint{0.536697in}{0.161328in}}%
\pgfpathlineto{\pgfqpoint{4.244492in}{0.161328in}}%
\pgfusepath{stroke}%
\end{pgfscope}%
\begin{pgfscope}%
\definecolor{textcolor}{rgb}{0.150000,0.150000,0.150000}%
\pgfsetstrokecolor{textcolor}%
\pgfsetfillcolor{textcolor}%
\pgftext[x=2.854069in,y=0.299784in,,]{\color{textcolor}\rmfamily\fontsize{15.000000}{18.000000}\selectfont \textbf{*}}%
\end{pgfscope}%
\begin{pgfscope}%
\definecolor{textcolor}{rgb}{0.150000,0.150000,0.150000}%
\pgfsetstrokecolor{textcolor}%
\pgfsetfillcolor{textcolor}%
\pgftext[x=3.781018in,y=1.426965in,,]{\color{textcolor}\rmfamily\fontsize{15.000000}{18.000000}\selectfont \textbf{*}}%
\end{pgfscope}%
\begin{pgfscope}%
\pgfsetbuttcap%
\pgfsetmiterjoin%
\definecolor{currentfill}{rgb}{0.200000,0.427451,0.650980}%
\pgfsetfillcolor{currentfill}%
\pgfsetlinewidth{1.505625pt}%
\definecolor{currentstroke}{rgb}{0.200000,0.427451,0.650980}%
\pgfsetstrokecolor{currentstroke}%
\pgfsetdash{}{0pt}%
\pgfpathmoveto{\pgfqpoint{4.344492in}{2.269558in}}%
\pgfpathlineto{\pgfqpoint{4.455603in}{2.269558in}}%
\pgfpathlineto{\pgfqpoint{4.455603in}{2.347336in}}%
\pgfpathlineto{\pgfqpoint{4.344492in}{2.347336in}}%
\pgfpathclose%
\pgfusepath{stroke,fill}%
\end{pgfscope}%
\begin{pgfscope}%
\definecolor{textcolor}{rgb}{0.150000,0.150000,0.150000}%
\pgfsetstrokecolor{textcolor}%
\pgfsetfillcolor{textcolor}%
\pgftext[x=4.544492in,y=2.269558in,left,base]{\color{textcolor}\rmfamily\fontsize{8.000000}{9.600000}\selectfont WT + Vehicle (8)}%
\end{pgfscope}%
\begin{pgfscope}%
\pgfsetbuttcap%
\pgfsetmiterjoin%
\definecolor{currentfill}{rgb}{0.168627,0.670588,0.494118}%
\pgfsetfillcolor{currentfill}%
\pgfsetlinewidth{1.505625pt}%
\definecolor{currentstroke}{rgb}{0.168627,0.670588,0.494118}%
\pgfsetstrokecolor{currentstroke}%
\pgfsetdash{}{0pt}%
\pgfpathmoveto{\pgfqpoint{4.344492in}{2.102918in}}%
\pgfpathlineto{\pgfqpoint{4.455603in}{2.102918in}}%
\pgfpathlineto{\pgfqpoint{4.455603in}{2.180696in}}%
\pgfpathlineto{\pgfqpoint{4.344492in}{2.180696in}}%
\pgfpathclose%
\pgfusepath{stroke,fill}%
\end{pgfscope}%
\begin{pgfscope}%
\definecolor{textcolor}{rgb}{0.150000,0.150000,0.150000}%
\pgfsetstrokecolor{textcolor}%
\pgfsetfillcolor{textcolor}%
\pgftext[x=4.544492in,y=2.102918in,left,base]{\color{textcolor}\rmfamily\fontsize{8.000000}{9.600000}\selectfont WT + TAT-GluA2\textsubscript{3Y} (7)}%
\end{pgfscope}%
\begin{pgfscope}%
\pgfsetbuttcap%
\pgfsetmiterjoin%
\definecolor{currentfill}{rgb}{1.000000,0.494118,0.250980}%
\pgfsetfillcolor{currentfill}%
\pgfsetlinewidth{1.505625pt}%
\definecolor{currentstroke}{rgb}{1.000000,0.494118,0.250980}%
\pgfsetstrokecolor{currentstroke}%
\pgfsetdash{}{0pt}%
\pgfpathmoveto{\pgfqpoint{4.344492in}{1.936279in}}%
\pgfpathlineto{\pgfqpoint{4.455603in}{1.936279in}}%
\pgfpathlineto{\pgfqpoint{4.455603in}{2.014057in}}%
\pgfpathlineto{\pgfqpoint{4.344492in}{2.014057in}}%
\pgfpathclose%
\pgfusepath{stroke,fill}%
\end{pgfscope}%
\begin{pgfscope}%
\definecolor{textcolor}{rgb}{0.150000,0.150000,0.150000}%
\pgfsetstrokecolor{textcolor}%
\pgfsetfillcolor{textcolor}%
\pgftext[x=4.544492in,y=1.936279in,left,base]{\color{textcolor}\rmfamily\fontsize{8.000000}{9.600000}\selectfont Tg + Vehicle (6)}%
\end{pgfscope}%
\begin{pgfscope}%
\pgfsetbuttcap%
\pgfsetmiterjoin%
\definecolor{currentfill}{rgb}{1.000000,0.694118,0.250980}%
\pgfsetfillcolor{currentfill}%
\pgfsetlinewidth{1.505625pt}%
\definecolor{currentstroke}{rgb}{1.000000,0.694118,0.250980}%
\pgfsetstrokecolor{currentstroke}%
\pgfsetdash{}{0pt}%
\pgfpathmoveto{\pgfqpoint{4.344492in}{1.769639in}}%
\pgfpathlineto{\pgfqpoint{4.455603in}{1.769639in}}%
\pgfpathlineto{\pgfqpoint{4.455603in}{1.847417in}}%
\pgfpathlineto{\pgfqpoint{4.344492in}{1.847417in}}%
\pgfpathclose%
\pgfusepath{stroke,fill}%
\end{pgfscope}%
\begin{pgfscope}%
\definecolor{textcolor}{rgb}{0.150000,0.150000,0.150000}%
\pgfsetstrokecolor{textcolor}%
\pgfsetfillcolor{textcolor}%
\pgftext[x=4.544492in,y=1.769639in,left,base]{\color{textcolor}\rmfamily\fontsize{8.000000}{9.600000}\selectfont Tg + TAT-GluA2\textsubscript{3Y} (9)}%
\end{pgfscope}%
\end{pgfpicture}%
\makeatother%
\endgroup%

        \caption{\label{f.ad.reminder1.paradigm}}
    \end{subfigure}
    \begin{subfigure}[h]{\textwidth}
        %% Creator: Matplotlib, PGF backend
%%
%% To include the figure in your LaTeX document, write
%%   \input{<filename>.pgf}
%%
%% Make sure the required packages are loaded in your preamble
%%   \usepackage{pgf}
%%
%% Figures using additional raster images can only be included by \input if
%% they are in the same directory as the main LaTeX file. For loading figures
%% from other directories you can use the `import` package
%%   \usepackage{import}
%% and then include the figures with
%%   \import{<path to file>}{<filename>.pgf}
%%
%% Matplotlib used the following preamble
%%   \usepackage[utf8]{inputenc}
%%   \usepackage[T1]{fontenc}
%%   \usepackage{siunitx}
%%
\begingroup%
\makeatletter%
\begin{pgfpicture}%
\pgfpathrectangle{\pgfpointorigin}{\pgfqpoint{6.000873in}{2.614199in}}%
\pgfusepath{use as bounding box, clip}%
\begin{pgfscope}%
\pgfsetbuttcap%
\pgfsetmiterjoin%
\definecolor{currentfill}{rgb}{1.000000,1.000000,1.000000}%
\pgfsetfillcolor{currentfill}%
\pgfsetlinewidth{0.000000pt}%
\definecolor{currentstroke}{rgb}{1.000000,1.000000,1.000000}%
\pgfsetstrokecolor{currentstroke}%
\pgfsetdash{}{0pt}%
\pgfpathmoveto{\pgfqpoint{0.000000in}{0.000000in}}%
\pgfpathlineto{\pgfqpoint{6.000873in}{0.000000in}}%
\pgfpathlineto{\pgfqpoint{6.000873in}{2.614199in}}%
\pgfpathlineto{\pgfqpoint{0.000000in}{2.614199in}}%
\pgfpathclose%
\pgfusepath{fill}%
\end{pgfscope}%
\begin{pgfscope}%
\pgfsetbuttcap%
\pgfsetmiterjoin%
\definecolor{currentfill}{rgb}{1.000000,1.000000,1.000000}%
\pgfsetfillcolor{currentfill}%
\pgfsetlinewidth{0.000000pt}%
\definecolor{currentstroke}{rgb}{0.000000,0.000000,0.000000}%
\pgfsetstrokecolor{currentstroke}%
\pgfsetstrokeopacity{0.000000}%
\pgfsetdash{}{0pt}%
\pgfpathmoveto{\pgfqpoint{0.536697in}{0.161328in}}%
\pgfpathlineto{\pgfqpoint{4.244492in}{0.161328in}}%
\pgfpathlineto{\pgfqpoint{4.244492in}{2.452871in}}%
\pgfpathlineto{\pgfqpoint{0.536697in}{2.452871in}}%
\pgfpathclose%
\pgfusepath{fill}%
\end{pgfscope}%
\begin{pgfscope}%
\pgfsetbuttcap%
\pgfsetroundjoin%
\definecolor{currentfill}{rgb}{0.150000,0.150000,0.150000}%
\pgfsetfillcolor{currentfill}%
\pgfsetlinewidth{1.003750pt}%
\definecolor{currentstroke}{rgb}{0.150000,0.150000,0.150000}%
\pgfsetstrokecolor{currentstroke}%
\pgfsetdash{}{0pt}%
\pgfsys@defobject{currentmarker}{\pgfqpoint{0.000000in}{0.000000in}}{\pgfqpoint{0.041667in}{0.000000in}}{%
\pgfpathmoveto{\pgfqpoint{0.000000in}{0.000000in}}%
\pgfpathlineto{\pgfqpoint{0.041667in}{0.000000in}}%
\pgfusepath{stroke,fill}%
}%
\begin{pgfscope}%
\pgfsys@transformshift{0.536697in}{0.161328in}%
\pgfsys@useobject{currentmarker}{}%
\end{pgfscope}%
\end{pgfscope}%
\begin{pgfscope}%
\definecolor{textcolor}{rgb}{0.150000,0.150000,0.150000}%
\pgfsetstrokecolor{textcolor}%
\pgfsetfillcolor{textcolor}%
\pgftext[x=0.439475in,y=0.161328in,right,]{\color{textcolor}\rmfamily\fontsize{10.000000}{12.000000}\selectfont \(\displaystyle 0\)}%
\end{pgfscope}%
\begin{pgfscope}%
\pgfsetbuttcap%
\pgfsetroundjoin%
\definecolor{currentfill}{rgb}{0.150000,0.150000,0.150000}%
\pgfsetfillcolor{currentfill}%
\pgfsetlinewidth{1.003750pt}%
\definecolor{currentstroke}{rgb}{0.150000,0.150000,0.150000}%
\pgfsetstrokecolor{currentstroke}%
\pgfsetdash{}{0pt}%
\pgfsys@defobject{currentmarker}{\pgfqpoint{0.000000in}{0.000000in}}{\pgfqpoint{0.041667in}{0.000000in}}{%
\pgfpathmoveto{\pgfqpoint{0.000000in}{0.000000in}}%
\pgfpathlineto{\pgfqpoint{0.041667in}{0.000000in}}%
\pgfusepath{stroke,fill}%
}%
\begin{pgfscope}%
\pgfsys@transformshift{0.536697in}{0.447771in}%
\pgfsys@useobject{currentmarker}{}%
\end{pgfscope}%
\end{pgfscope}%
\begin{pgfscope}%
\definecolor{textcolor}{rgb}{0.150000,0.150000,0.150000}%
\pgfsetstrokecolor{textcolor}%
\pgfsetfillcolor{textcolor}%
\pgftext[x=0.439475in,y=0.447771in,right,]{\color{textcolor}\rmfamily\fontsize{10.000000}{12.000000}\selectfont \(\displaystyle 10\)}%
\end{pgfscope}%
\begin{pgfscope}%
\pgfsetbuttcap%
\pgfsetroundjoin%
\definecolor{currentfill}{rgb}{0.150000,0.150000,0.150000}%
\pgfsetfillcolor{currentfill}%
\pgfsetlinewidth{1.003750pt}%
\definecolor{currentstroke}{rgb}{0.150000,0.150000,0.150000}%
\pgfsetstrokecolor{currentstroke}%
\pgfsetdash{}{0pt}%
\pgfsys@defobject{currentmarker}{\pgfqpoint{0.000000in}{0.000000in}}{\pgfqpoint{0.041667in}{0.000000in}}{%
\pgfpathmoveto{\pgfqpoint{0.000000in}{0.000000in}}%
\pgfpathlineto{\pgfqpoint{0.041667in}{0.000000in}}%
\pgfusepath{stroke,fill}%
}%
\begin{pgfscope}%
\pgfsys@transformshift{0.536697in}{0.734213in}%
\pgfsys@useobject{currentmarker}{}%
\end{pgfscope}%
\end{pgfscope}%
\begin{pgfscope}%
\definecolor{textcolor}{rgb}{0.150000,0.150000,0.150000}%
\pgfsetstrokecolor{textcolor}%
\pgfsetfillcolor{textcolor}%
\pgftext[x=0.439475in,y=0.734213in,right,]{\color{textcolor}\rmfamily\fontsize{10.000000}{12.000000}\selectfont \(\displaystyle 20\)}%
\end{pgfscope}%
\begin{pgfscope}%
\pgfsetbuttcap%
\pgfsetroundjoin%
\definecolor{currentfill}{rgb}{0.150000,0.150000,0.150000}%
\pgfsetfillcolor{currentfill}%
\pgfsetlinewidth{1.003750pt}%
\definecolor{currentstroke}{rgb}{0.150000,0.150000,0.150000}%
\pgfsetstrokecolor{currentstroke}%
\pgfsetdash{}{0pt}%
\pgfsys@defobject{currentmarker}{\pgfqpoint{0.000000in}{0.000000in}}{\pgfqpoint{0.041667in}{0.000000in}}{%
\pgfpathmoveto{\pgfqpoint{0.000000in}{0.000000in}}%
\pgfpathlineto{\pgfqpoint{0.041667in}{0.000000in}}%
\pgfusepath{stroke,fill}%
}%
\begin{pgfscope}%
\pgfsys@transformshift{0.536697in}{1.020656in}%
\pgfsys@useobject{currentmarker}{}%
\end{pgfscope}%
\end{pgfscope}%
\begin{pgfscope}%
\definecolor{textcolor}{rgb}{0.150000,0.150000,0.150000}%
\pgfsetstrokecolor{textcolor}%
\pgfsetfillcolor{textcolor}%
\pgftext[x=0.439475in,y=1.020656in,right,]{\color{textcolor}\rmfamily\fontsize{10.000000}{12.000000}\selectfont \(\displaystyle 30\)}%
\end{pgfscope}%
\begin{pgfscope}%
\pgfsetbuttcap%
\pgfsetroundjoin%
\definecolor{currentfill}{rgb}{0.150000,0.150000,0.150000}%
\pgfsetfillcolor{currentfill}%
\pgfsetlinewidth{1.003750pt}%
\definecolor{currentstroke}{rgb}{0.150000,0.150000,0.150000}%
\pgfsetstrokecolor{currentstroke}%
\pgfsetdash{}{0pt}%
\pgfsys@defobject{currentmarker}{\pgfqpoint{0.000000in}{0.000000in}}{\pgfqpoint{0.041667in}{0.000000in}}{%
\pgfpathmoveto{\pgfqpoint{0.000000in}{0.000000in}}%
\pgfpathlineto{\pgfqpoint{0.041667in}{0.000000in}}%
\pgfusepath{stroke,fill}%
}%
\begin{pgfscope}%
\pgfsys@transformshift{0.536697in}{1.307099in}%
\pgfsys@useobject{currentmarker}{}%
\end{pgfscope}%
\end{pgfscope}%
\begin{pgfscope}%
\definecolor{textcolor}{rgb}{0.150000,0.150000,0.150000}%
\pgfsetstrokecolor{textcolor}%
\pgfsetfillcolor{textcolor}%
\pgftext[x=0.439475in,y=1.307099in,right,]{\color{textcolor}\rmfamily\fontsize{10.000000}{12.000000}\selectfont \(\displaystyle 40\)}%
\end{pgfscope}%
\begin{pgfscope}%
\pgfsetbuttcap%
\pgfsetroundjoin%
\definecolor{currentfill}{rgb}{0.150000,0.150000,0.150000}%
\pgfsetfillcolor{currentfill}%
\pgfsetlinewidth{1.003750pt}%
\definecolor{currentstroke}{rgb}{0.150000,0.150000,0.150000}%
\pgfsetstrokecolor{currentstroke}%
\pgfsetdash{}{0pt}%
\pgfsys@defobject{currentmarker}{\pgfqpoint{0.000000in}{0.000000in}}{\pgfqpoint{0.041667in}{0.000000in}}{%
\pgfpathmoveto{\pgfqpoint{0.000000in}{0.000000in}}%
\pgfpathlineto{\pgfqpoint{0.041667in}{0.000000in}}%
\pgfusepath{stroke,fill}%
}%
\begin{pgfscope}%
\pgfsys@transformshift{0.536697in}{1.593542in}%
\pgfsys@useobject{currentmarker}{}%
\end{pgfscope}%
\end{pgfscope}%
\begin{pgfscope}%
\definecolor{textcolor}{rgb}{0.150000,0.150000,0.150000}%
\pgfsetstrokecolor{textcolor}%
\pgfsetfillcolor{textcolor}%
\pgftext[x=0.439475in,y=1.593542in,right,]{\color{textcolor}\rmfamily\fontsize{10.000000}{12.000000}\selectfont \(\displaystyle 50\)}%
\end{pgfscope}%
\begin{pgfscope}%
\pgfsetbuttcap%
\pgfsetroundjoin%
\definecolor{currentfill}{rgb}{0.150000,0.150000,0.150000}%
\pgfsetfillcolor{currentfill}%
\pgfsetlinewidth{1.003750pt}%
\definecolor{currentstroke}{rgb}{0.150000,0.150000,0.150000}%
\pgfsetstrokecolor{currentstroke}%
\pgfsetdash{}{0pt}%
\pgfsys@defobject{currentmarker}{\pgfqpoint{0.000000in}{0.000000in}}{\pgfqpoint{0.041667in}{0.000000in}}{%
\pgfpathmoveto{\pgfqpoint{0.000000in}{0.000000in}}%
\pgfpathlineto{\pgfqpoint{0.041667in}{0.000000in}}%
\pgfusepath{stroke,fill}%
}%
\begin{pgfscope}%
\pgfsys@transformshift{0.536697in}{1.879985in}%
\pgfsys@useobject{currentmarker}{}%
\end{pgfscope}%
\end{pgfscope}%
\begin{pgfscope}%
\definecolor{textcolor}{rgb}{0.150000,0.150000,0.150000}%
\pgfsetstrokecolor{textcolor}%
\pgfsetfillcolor{textcolor}%
\pgftext[x=0.439475in,y=1.879985in,right,]{\color{textcolor}\rmfamily\fontsize{10.000000}{12.000000}\selectfont \(\displaystyle 60\)}%
\end{pgfscope}%
\begin{pgfscope}%
\pgfsetbuttcap%
\pgfsetroundjoin%
\definecolor{currentfill}{rgb}{0.150000,0.150000,0.150000}%
\pgfsetfillcolor{currentfill}%
\pgfsetlinewidth{1.003750pt}%
\definecolor{currentstroke}{rgb}{0.150000,0.150000,0.150000}%
\pgfsetstrokecolor{currentstroke}%
\pgfsetdash{}{0pt}%
\pgfsys@defobject{currentmarker}{\pgfqpoint{0.000000in}{0.000000in}}{\pgfqpoint{0.041667in}{0.000000in}}{%
\pgfpathmoveto{\pgfqpoint{0.000000in}{0.000000in}}%
\pgfpathlineto{\pgfqpoint{0.041667in}{0.000000in}}%
\pgfusepath{stroke,fill}%
}%
\begin{pgfscope}%
\pgfsys@transformshift{0.536697in}{2.166428in}%
\pgfsys@useobject{currentmarker}{}%
\end{pgfscope}%
\end{pgfscope}%
\begin{pgfscope}%
\definecolor{textcolor}{rgb}{0.150000,0.150000,0.150000}%
\pgfsetstrokecolor{textcolor}%
\pgfsetfillcolor{textcolor}%
\pgftext[x=0.439475in,y=2.166428in,right,]{\color{textcolor}\rmfamily\fontsize{10.000000}{12.000000}\selectfont \(\displaystyle 70\)}%
\end{pgfscope}%
\begin{pgfscope}%
\pgfsetbuttcap%
\pgfsetroundjoin%
\definecolor{currentfill}{rgb}{0.150000,0.150000,0.150000}%
\pgfsetfillcolor{currentfill}%
\pgfsetlinewidth{1.003750pt}%
\definecolor{currentstroke}{rgb}{0.150000,0.150000,0.150000}%
\pgfsetstrokecolor{currentstroke}%
\pgfsetdash{}{0pt}%
\pgfsys@defobject{currentmarker}{\pgfqpoint{0.000000in}{0.000000in}}{\pgfqpoint{0.041667in}{0.000000in}}{%
\pgfpathmoveto{\pgfqpoint{0.000000in}{0.000000in}}%
\pgfpathlineto{\pgfqpoint{0.041667in}{0.000000in}}%
\pgfusepath{stroke,fill}%
}%
\begin{pgfscope}%
\pgfsys@transformshift{0.536697in}{2.452871in}%
\pgfsys@useobject{currentmarker}{}%
\end{pgfscope}%
\end{pgfscope}%
\begin{pgfscope}%
\definecolor{textcolor}{rgb}{0.150000,0.150000,0.150000}%
\pgfsetstrokecolor{textcolor}%
\pgfsetfillcolor{textcolor}%
\pgftext[x=0.439475in,y=2.452871in,right,]{\color{textcolor}\rmfamily\fontsize{10.000000}{12.000000}\selectfont \(\displaystyle 80\)}%
\end{pgfscope}%
\begin{pgfscope}%
\definecolor{textcolor}{rgb}{0.150000,0.150000,0.150000}%
\pgfsetstrokecolor{textcolor}%
\pgfsetfillcolor{textcolor}%
\pgftext[x=0.231141in,y=1.307099in,,bottom,rotate=90.000000]{\color{textcolor}\rmfamily\fontsize{10.000000}{12.000000}\selectfont \textbf{\% Freezing}}%
\end{pgfscope}%
\begin{pgfscope}%
\pgfpathrectangle{\pgfqpoint{0.536697in}{0.161328in}}{\pgfqpoint{3.707795in}{2.291544in}} %
\pgfusepath{clip}%
\pgfsetbuttcap%
\pgfsetmiterjoin%
\definecolor{currentfill}{rgb}{0.200000,0.427451,0.650980}%
\pgfsetfillcolor{currentfill}%
\pgfsetlinewidth{1.505625pt}%
\definecolor{currentstroke}{rgb}{0.200000,0.427451,0.650980}%
\pgfsetstrokecolor{currentstroke}%
\pgfsetdash{}{0pt}%
\pgfpathmoveto{\pgfqpoint{0.669118in}{0.161328in}}%
\pgfpathlineto{\pgfqpoint{1.331224in}{0.161328in}}%
\pgfpathlineto{\pgfqpoint{1.331224in}{1.176768in}}%
\pgfpathlineto{\pgfqpoint{0.669118in}{1.176768in}}%
\pgfpathclose%
\pgfusepath{stroke,fill}%
\end{pgfscope}%
\begin{pgfscope}%
\pgfpathrectangle{\pgfqpoint{0.536697in}{0.161328in}}{\pgfqpoint{3.707795in}{2.291544in}} %
\pgfusepath{clip}%
\pgfsetbuttcap%
\pgfsetmiterjoin%
\definecolor{currentfill}{rgb}{0.168627,0.670588,0.494118}%
\pgfsetfillcolor{currentfill}%
\pgfsetlinewidth{1.505625pt}%
\definecolor{currentstroke}{rgb}{0.168627,0.670588,0.494118}%
\pgfsetstrokecolor{currentstroke}%
\pgfsetdash{}{0pt}%
\pgfpathmoveto{\pgfqpoint{1.596067in}{0.161328in}}%
\pgfpathlineto{\pgfqpoint{2.258173in}{0.161328in}}%
\pgfpathlineto{\pgfqpoint{2.258173in}{1.571200in}}%
\pgfpathlineto{\pgfqpoint{1.596067in}{1.571200in}}%
\pgfpathclose%
\pgfusepath{stroke,fill}%
\end{pgfscope}%
\begin{pgfscope}%
\pgfpathrectangle{\pgfqpoint{0.536697in}{0.161328in}}{\pgfqpoint{3.707795in}{2.291544in}} %
\pgfusepath{clip}%
\pgfsetbuttcap%
\pgfsetmiterjoin%
\definecolor{currentfill}{rgb}{1.000000,0.494118,0.250980}%
\pgfsetfillcolor{currentfill}%
\pgfsetlinewidth{1.505625pt}%
\definecolor{currentstroke}{rgb}{1.000000,0.494118,0.250980}%
\pgfsetstrokecolor{currentstroke}%
\pgfsetdash{}{0pt}%
\pgfpathmoveto{\pgfqpoint{2.523016in}{0.161328in}}%
\pgfpathlineto{\pgfqpoint{3.185122in}{0.161328in}}%
\pgfpathlineto{\pgfqpoint{3.185122in}{0.227267in}}%
\pgfpathlineto{\pgfqpoint{2.523016in}{0.227267in}}%
\pgfpathclose%
\pgfusepath{stroke,fill}%
\end{pgfscope}%
\begin{pgfscope}%
\pgfpathrectangle{\pgfqpoint{0.536697in}{0.161328in}}{\pgfqpoint{3.707795in}{2.291544in}} %
\pgfusepath{clip}%
\pgfsetbuttcap%
\pgfsetmiterjoin%
\definecolor{currentfill}{rgb}{1.000000,0.694118,0.250980}%
\pgfsetfillcolor{currentfill}%
\pgfsetlinewidth{1.505625pt}%
\definecolor{currentstroke}{rgb}{1.000000,0.694118,0.250980}%
\pgfsetstrokecolor{currentstroke}%
\pgfsetdash{}{0pt}%
\pgfpathmoveto{\pgfqpoint{3.449965in}{0.161328in}}%
\pgfpathlineto{\pgfqpoint{4.112071in}{0.161328in}}%
\pgfpathlineto{\pgfqpoint{4.112071in}{1.141822in}}%
\pgfpathlineto{\pgfqpoint{3.449965in}{1.141822in}}%
\pgfpathclose%
\pgfusepath{stroke,fill}%
\end{pgfscope}%
\begin{pgfscope}%
\pgfpathrectangle{\pgfqpoint{0.536697in}{0.161328in}}{\pgfqpoint{3.707795in}{2.291544in}} %
\pgfusepath{clip}%
\pgfsetbuttcap%
\pgfsetroundjoin%
\pgfsetlinewidth{1.505625pt}%
\definecolor{currentstroke}{rgb}{0.200000,0.427451,0.650980}%
\pgfsetstrokecolor{currentstroke}%
\pgfsetdash{}{0pt}%
\pgfpathmoveto{\pgfqpoint{1.000171in}{1.176768in}}%
\pgfpathlineto{\pgfqpoint{1.000171in}{1.385585in}}%
\pgfusepath{stroke}%
\end{pgfscope}%
\begin{pgfscope}%
\pgfpathrectangle{\pgfqpoint{0.536697in}{0.161328in}}{\pgfqpoint{3.707795in}{2.291544in}} %
\pgfusepath{clip}%
\pgfsetbuttcap%
\pgfsetroundjoin%
\pgfsetlinewidth{1.505625pt}%
\definecolor{currentstroke}{rgb}{0.168627,0.670588,0.494118}%
\pgfsetstrokecolor{currentstroke}%
\pgfsetdash{}{0pt}%
\pgfpathmoveto{\pgfqpoint{1.927120in}{1.571200in}}%
\pgfpathlineto{\pgfqpoint{1.927120in}{1.806656in}}%
\pgfusepath{stroke}%
\end{pgfscope}%
\begin{pgfscope}%
\pgfpathrectangle{\pgfqpoint{0.536697in}{0.161328in}}{\pgfqpoint{3.707795in}{2.291544in}} %
\pgfusepath{clip}%
\pgfsetbuttcap%
\pgfsetroundjoin%
\pgfsetlinewidth{1.505625pt}%
\definecolor{currentstroke}{rgb}{1.000000,0.494118,0.250980}%
\pgfsetstrokecolor{currentstroke}%
\pgfsetdash{}{0pt}%
\pgfpathmoveto{\pgfqpoint{2.854069in}{0.227267in}}%
\pgfpathlineto{\pgfqpoint{2.854069in}{0.250956in}}%
\pgfusepath{stroke}%
\end{pgfscope}%
\begin{pgfscope}%
\pgfpathrectangle{\pgfqpoint{0.536697in}{0.161328in}}{\pgfqpoint{3.707795in}{2.291544in}} %
\pgfusepath{clip}%
\pgfsetbuttcap%
\pgfsetroundjoin%
\pgfsetlinewidth{1.505625pt}%
\definecolor{currentstroke}{rgb}{1.000000,0.694118,0.250980}%
\pgfsetstrokecolor{currentstroke}%
\pgfsetdash{}{0pt}%
\pgfpathmoveto{\pgfqpoint{3.781018in}{1.141822in}}%
\pgfpathlineto{\pgfqpoint{3.781018in}{1.378137in}}%
\pgfusepath{stroke}%
\end{pgfscope}%
\begin{pgfscope}%
\pgfpathrectangle{\pgfqpoint{0.536697in}{0.161328in}}{\pgfqpoint{3.707795in}{2.291544in}} %
\pgfusepath{clip}%
\pgfsetbuttcap%
\pgfsetroundjoin%
\definecolor{currentfill}{rgb}{0.200000,0.427451,0.650980}%
\pgfsetfillcolor{currentfill}%
\pgfsetlinewidth{1.505625pt}%
\definecolor{currentstroke}{rgb}{0.200000,0.427451,0.650980}%
\pgfsetstrokecolor{currentstroke}%
\pgfsetdash{}{0pt}%
\pgfsys@defobject{currentmarker}{\pgfqpoint{-0.111111in}{-0.000000in}}{\pgfqpoint{0.111111in}{0.000000in}}{%
\pgfpathmoveto{\pgfqpoint{0.111111in}{-0.000000in}}%
\pgfpathlineto{\pgfqpoint{-0.111111in}{0.000000in}}%
\pgfusepath{stroke,fill}%
}%
\begin{pgfscope}%
\pgfsys@transformshift{1.000171in}{1.176768in}%
\pgfsys@useobject{currentmarker}{}%
\end{pgfscope}%
\end{pgfscope}%
\begin{pgfscope}%
\pgfpathrectangle{\pgfqpoint{0.536697in}{0.161328in}}{\pgfqpoint{3.707795in}{2.291544in}} %
\pgfusepath{clip}%
\pgfsetbuttcap%
\pgfsetroundjoin%
\definecolor{currentfill}{rgb}{0.200000,0.427451,0.650980}%
\pgfsetfillcolor{currentfill}%
\pgfsetlinewidth{1.505625pt}%
\definecolor{currentstroke}{rgb}{0.200000,0.427451,0.650980}%
\pgfsetstrokecolor{currentstroke}%
\pgfsetdash{}{0pt}%
\pgfsys@defobject{currentmarker}{\pgfqpoint{-0.111111in}{-0.000000in}}{\pgfqpoint{0.111111in}{0.000000in}}{%
\pgfpathmoveto{\pgfqpoint{0.111111in}{-0.000000in}}%
\pgfpathlineto{\pgfqpoint{-0.111111in}{0.000000in}}%
\pgfusepath{stroke,fill}%
}%
\begin{pgfscope}%
\pgfsys@transformshift{1.000171in}{1.385585in}%
\pgfsys@useobject{currentmarker}{}%
\end{pgfscope}%
\end{pgfscope}%
\begin{pgfscope}%
\pgfpathrectangle{\pgfqpoint{0.536697in}{0.161328in}}{\pgfqpoint{3.707795in}{2.291544in}} %
\pgfusepath{clip}%
\pgfsetbuttcap%
\pgfsetroundjoin%
\definecolor{currentfill}{rgb}{0.168627,0.670588,0.494118}%
\pgfsetfillcolor{currentfill}%
\pgfsetlinewidth{1.505625pt}%
\definecolor{currentstroke}{rgb}{0.168627,0.670588,0.494118}%
\pgfsetstrokecolor{currentstroke}%
\pgfsetdash{}{0pt}%
\pgfsys@defobject{currentmarker}{\pgfqpoint{-0.111111in}{-0.000000in}}{\pgfqpoint{0.111111in}{0.000000in}}{%
\pgfpathmoveto{\pgfqpoint{0.111111in}{-0.000000in}}%
\pgfpathlineto{\pgfqpoint{-0.111111in}{0.000000in}}%
\pgfusepath{stroke,fill}%
}%
\begin{pgfscope}%
\pgfsys@transformshift{1.927120in}{1.571200in}%
\pgfsys@useobject{currentmarker}{}%
\end{pgfscope}%
\end{pgfscope}%
\begin{pgfscope}%
\pgfpathrectangle{\pgfqpoint{0.536697in}{0.161328in}}{\pgfqpoint{3.707795in}{2.291544in}} %
\pgfusepath{clip}%
\pgfsetbuttcap%
\pgfsetroundjoin%
\definecolor{currentfill}{rgb}{0.168627,0.670588,0.494118}%
\pgfsetfillcolor{currentfill}%
\pgfsetlinewidth{1.505625pt}%
\definecolor{currentstroke}{rgb}{0.168627,0.670588,0.494118}%
\pgfsetstrokecolor{currentstroke}%
\pgfsetdash{}{0pt}%
\pgfsys@defobject{currentmarker}{\pgfqpoint{-0.111111in}{-0.000000in}}{\pgfqpoint{0.111111in}{0.000000in}}{%
\pgfpathmoveto{\pgfqpoint{0.111111in}{-0.000000in}}%
\pgfpathlineto{\pgfqpoint{-0.111111in}{0.000000in}}%
\pgfusepath{stroke,fill}%
}%
\begin{pgfscope}%
\pgfsys@transformshift{1.927120in}{1.806656in}%
\pgfsys@useobject{currentmarker}{}%
\end{pgfscope}%
\end{pgfscope}%
\begin{pgfscope}%
\pgfpathrectangle{\pgfqpoint{0.536697in}{0.161328in}}{\pgfqpoint{3.707795in}{2.291544in}} %
\pgfusepath{clip}%
\pgfsetbuttcap%
\pgfsetroundjoin%
\definecolor{currentfill}{rgb}{1.000000,0.494118,0.250980}%
\pgfsetfillcolor{currentfill}%
\pgfsetlinewidth{1.505625pt}%
\definecolor{currentstroke}{rgb}{1.000000,0.494118,0.250980}%
\pgfsetstrokecolor{currentstroke}%
\pgfsetdash{}{0pt}%
\pgfsys@defobject{currentmarker}{\pgfqpoint{-0.111111in}{-0.000000in}}{\pgfqpoint{0.111111in}{0.000000in}}{%
\pgfpathmoveto{\pgfqpoint{0.111111in}{-0.000000in}}%
\pgfpathlineto{\pgfqpoint{-0.111111in}{0.000000in}}%
\pgfusepath{stroke,fill}%
}%
\begin{pgfscope}%
\pgfsys@transformshift{2.854069in}{0.227267in}%
\pgfsys@useobject{currentmarker}{}%
\end{pgfscope}%
\end{pgfscope}%
\begin{pgfscope}%
\pgfpathrectangle{\pgfqpoint{0.536697in}{0.161328in}}{\pgfqpoint{3.707795in}{2.291544in}} %
\pgfusepath{clip}%
\pgfsetbuttcap%
\pgfsetroundjoin%
\definecolor{currentfill}{rgb}{1.000000,0.494118,0.250980}%
\pgfsetfillcolor{currentfill}%
\pgfsetlinewidth{1.505625pt}%
\definecolor{currentstroke}{rgb}{1.000000,0.494118,0.250980}%
\pgfsetstrokecolor{currentstroke}%
\pgfsetdash{}{0pt}%
\pgfsys@defobject{currentmarker}{\pgfqpoint{-0.111111in}{-0.000000in}}{\pgfqpoint{0.111111in}{0.000000in}}{%
\pgfpathmoveto{\pgfqpoint{0.111111in}{-0.000000in}}%
\pgfpathlineto{\pgfqpoint{-0.111111in}{0.000000in}}%
\pgfusepath{stroke,fill}%
}%
\begin{pgfscope}%
\pgfsys@transformshift{2.854069in}{0.250956in}%
\pgfsys@useobject{currentmarker}{}%
\end{pgfscope}%
\end{pgfscope}%
\begin{pgfscope}%
\pgfpathrectangle{\pgfqpoint{0.536697in}{0.161328in}}{\pgfqpoint{3.707795in}{2.291544in}} %
\pgfusepath{clip}%
\pgfsetbuttcap%
\pgfsetroundjoin%
\definecolor{currentfill}{rgb}{1.000000,0.694118,0.250980}%
\pgfsetfillcolor{currentfill}%
\pgfsetlinewidth{1.505625pt}%
\definecolor{currentstroke}{rgb}{1.000000,0.694118,0.250980}%
\pgfsetstrokecolor{currentstroke}%
\pgfsetdash{}{0pt}%
\pgfsys@defobject{currentmarker}{\pgfqpoint{-0.111111in}{-0.000000in}}{\pgfqpoint{0.111111in}{0.000000in}}{%
\pgfpathmoveto{\pgfqpoint{0.111111in}{-0.000000in}}%
\pgfpathlineto{\pgfqpoint{-0.111111in}{0.000000in}}%
\pgfusepath{stroke,fill}%
}%
\begin{pgfscope}%
\pgfsys@transformshift{3.781018in}{1.141822in}%
\pgfsys@useobject{currentmarker}{}%
\end{pgfscope}%
\end{pgfscope}%
\begin{pgfscope}%
\pgfpathrectangle{\pgfqpoint{0.536697in}{0.161328in}}{\pgfqpoint{3.707795in}{2.291544in}} %
\pgfusepath{clip}%
\pgfsetbuttcap%
\pgfsetroundjoin%
\definecolor{currentfill}{rgb}{1.000000,0.694118,0.250980}%
\pgfsetfillcolor{currentfill}%
\pgfsetlinewidth{1.505625pt}%
\definecolor{currentstroke}{rgb}{1.000000,0.694118,0.250980}%
\pgfsetstrokecolor{currentstroke}%
\pgfsetdash{}{0pt}%
\pgfsys@defobject{currentmarker}{\pgfqpoint{-0.111111in}{-0.000000in}}{\pgfqpoint{0.111111in}{0.000000in}}{%
\pgfpathmoveto{\pgfqpoint{0.111111in}{-0.000000in}}%
\pgfpathlineto{\pgfqpoint{-0.111111in}{0.000000in}}%
\pgfusepath{stroke,fill}%
}%
\begin{pgfscope}%
\pgfsys@transformshift{3.781018in}{1.378137in}%
\pgfsys@useobject{currentmarker}{}%
\end{pgfscope}%
\end{pgfscope}%
\begin{pgfscope}%
\pgfpathrectangle{\pgfqpoint{0.536697in}{0.161328in}}{\pgfqpoint{3.707795in}{2.291544in}} %
\pgfusepath{clip}%
\pgfsetroundcap%
\pgfsetroundjoin%
\pgfsetlinewidth{1.756562pt}%
\definecolor{currentstroke}{rgb}{0.627451,0.627451,0.643137}%
\pgfsetstrokecolor{currentstroke}%
\pgfsetdash{}{0pt}%
\pgfpathmoveto{\pgfqpoint{1.000171in}{1.463710in}}%
\pgfpathlineto{\pgfqpoint{1.000171in}{2.014989in}}%
\pgfusepath{stroke}%
\end{pgfscope}%
\begin{pgfscope}%
\pgfpathrectangle{\pgfqpoint{0.536697in}{0.161328in}}{\pgfqpoint{3.707795in}{2.291544in}} %
\pgfusepath{clip}%
\pgfsetroundcap%
\pgfsetroundjoin%
\pgfsetlinewidth{1.756562pt}%
\definecolor{currentstroke}{rgb}{0.627451,0.627451,0.643137}%
\pgfsetstrokecolor{currentstroke}%
\pgfsetdash{}{0pt}%
\pgfpathmoveto{\pgfqpoint{1.000171in}{2.014989in}}%
\pgfpathlineto{\pgfqpoint{2.854069in}{2.014989in}}%
\pgfusepath{stroke}%
\end{pgfscope}%
\begin{pgfscope}%
\pgfpathrectangle{\pgfqpoint{0.536697in}{0.161328in}}{\pgfqpoint{3.707795in}{2.291544in}} %
\pgfusepath{clip}%
\pgfsetroundcap%
\pgfsetroundjoin%
\pgfsetlinewidth{1.756562pt}%
\definecolor{currentstroke}{rgb}{0.627451,0.627451,0.643137}%
\pgfsetstrokecolor{currentstroke}%
\pgfsetdash{}{0pt}%
\pgfpathmoveto{\pgfqpoint{2.854069in}{2.014989in}}%
\pgfpathlineto{\pgfqpoint{2.854069in}{0.407206in}}%
\pgfusepath{stroke}%
\end{pgfscope}%
\begin{pgfscope}%
\pgfpathrectangle{\pgfqpoint{0.536697in}{0.161328in}}{\pgfqpoint{3.707795in}{2.291544in}} %
\pgfusepath{clip}%
\pgfsetroundcap%
\pgfsetroundjoin%
\pgfsetlinewidth{1.756562pt}%
\definecolor{currentstroke}{rgb}{0.627451,0.627451,0.643137}%
\pgfsetstrokecolor{currentstroke}%
\pgfsetdash{}{0pt}%
\pgfpathmoveto{\pgfqpoint{2.854069in}{2.093114in}}%
\pgfpathlineto{\pgfqpoint{2.854069in}{2.223323in}}%
\pgfusepath{stroke}%
\end{pgfscope}%
\begin{pgfscope}%
\pgfpathrectangle{\pgfqpoint{0.536697in}{0.161328in}}{\pgfqpoint{3.707795in}{2.291544in}} %
\pgfusepath{clip}%
\pgfsetroundcap%
\pgfsetroundjoin%
\pgfsetlinewidth{1.756562pt}%
\definecolor{currentstroke}{rgb}{0.627451,0.627451,0.643137}%
\pgfsetstrokecolor{currentstroke}%
\pgfsetdash{}{0pt}%
\pgfpathmoveto{\pgfqpoint{2.854069in}{2.223323in}}%
\pgfpathlineto{\pgfqpoint{3.781018in}{2.223323in}}%
\pgfusepath{stroke}%
\end{pgfscope}%
\begin{pgfscope}%
\pgfpathrectangle{\pgfqpoint{0.536697in}{0.161328in}}{\pgfqpoint{3.707795in}{2.291544in}} %
\pgfusepath{clip}%
\pgfsetroundcap%
\pgfsetroundjoin%
\pgfsetlinewidth{1.756562pt}%
\definecolor{currentstroke}{rgb}{0.627451,0.627451,0.643137}%
\pgfsetstrokecolor{currentstroke}%
\pgfsetdash{}{0pt}%
\pgfpathmoveto{\pgfqpoint{3.781018in}{2.223323in}}%
\pgfpathlineto{\pgfqpoint{3.781018in}{1.534387in}}%
\pgfusepath{stroke}%
\end{pgfscope}%
\begin{pgfscope}%
\pgfsetrectcap%
\pgfsetmiterjoin%
\pgfsetlinewidth{1.254687pt}%
\definecolor{currentstroke}{rgb}{0.150000,0.150000,0.150000}%
\pgfsetstrokecolor{currentstroke}%
\pgfsetdash{}{0pt}%
\pgfpathmoveto{\pgfqpoint{0.536697in}{0.161328in}}%
\pgfpathlineto{\pgfqpoint{0.536697in}{2.452871in}}%
\pgfusepath{stroke}%
\end{pgfscope}%
\begin{pgfscope}%
\pgfsetrectcap%
\pgfsetmiterjoin%
\pgfsetlinewidth{1.254687pt}%
\definecolor{currentstroke}{rgb}{0.150000,0.150000,0.150000}%
\pgfsetstrokecolor{currentstroke}%
\pgfsetdash{}{0pt}%
\pgfpathmoveto{\pgfqpoint{0.536697in}{0.161328in}}%
\pgfpathlineto{\pgfqpoint{4.244492in}{0.161328in}}%
\pgfusepath{stroke}%
\end{pgfscope}%
\begin{pgfscope}%
\definecolor{textcolor}{rgb}{0.150000,0.150000,0.150000}%
\pgfsetstrokecolor{textcolor}%
\pgfsetfillcolor{textcolor}%
\pgftext[x=2.854069in,y=0.299784in,,]{\color{textcolor}\rmfamily\fontsize{15.000000}{18.000000}\selectfont \textbf{*}}%
\end{pgfscope}%
\begin{pgfscope}%
\definecolor{textcolor}{rgb}{0.150000,0.150000,0.150000}%
\pgfsetstrokecolor{textcolor}%
\pgfsetfillcolor{textcolor}%
\pgftext[x=3.781018in,y=1.426965in,,]{\color{textcolor}\rmfamily\fontsize{15.000000}{18.000000}\selectfont \textbf{*}}%
\end{pgfscope}%
\begin{pgfscope}%
\pgfsetbuttcap%
\pgfsetmiterjoin%
\definecolor{currentfill}{rgb}{0.200000,0.427451,0.650980}%
\pgfsetfillcolor{currentfill}%
\pgfsetlinewidth{1.505625pt}%
\definecolor{currentstroke}{rgb}{0.200000,0.427451,0.650980}%
\pgfsetstrokecolor{currentstroke}%
\pgfsetdash{}{0pt}%
\pgfpathmoveto{\pgfqpoint{4.344492in}{2.269558in}}%
\pgfpathlineto{\pgfqpoint{4.455603in}{2.269558in}}%
\pgfpathlineto{\pgfqpoint{4.455603in}{2.347336in}}%
\pgfpathlineto{\pgfqpoint{4.344492in}{2.347336in}}%
\pgfpathclose%
\pgfusepath{stroke,fill}%
\end{pgfscope}%
\begin{pgfscope}%
\definecolor{textcolor}{rgb}{0.150000,0.150000,0.150000}%
\pgfsetstrokecolor{textcolor}%
\pgfsetfillcolor{textcolor}%
\pgftext[x=4.544492in,y=2.269558in,left,base]{\color{textcolor}\rmfamily\fontsize{8.000000}{9.600000}\selectfont WT + Vehicle (8)}%
\end{pgfscope}%
\begin{pgfscope}%
\pgfsetbuttcap%
\pgfsetmiterjoin%
\definecolor{currentfill}{rgb}{0.168627,0.670588,0.494118}%
\pgfsetfillcolor{currentfill}%
\pgfsetlinewidth{1.505625pt}%
\definecolor{currentstroke}{rgb}{0.168627,0.670588,0.494118}%
\pgfsetstrokecolor{currentstroke}%
\pgfsetdash{}{0pt}%
\pgfpathmoveto{\pgfqpoint{4.344492in}{2.102918in}}%
\pgfpathlineto{\pgfqpoint{4.455603in}{2.102918in}}%
\pgfpathlineto{\pgfqpoint{4.455603in}{2.180696in}}%
\pgfpathlineto{\pgfqpoint{4.344492in}{2.180696in}}%
\pgfpathclose%
\pgfusepath{stroke,fill}%
\end{pgfscope}%
\begin{pgfscope}%
\definecolor{textcolor}{rgb}{0.150000,0.150000,0.150000}%
\pgfsetstrokecolor{textcolor}%
\pgfsetfillcolor{textcolor}%
\pgftext[x=4.544492in,y=2.102918in,left,base]{\color{textcolor}\rmfamily\fontsize{8.000000}{9.600000}\selectfont WT + TAT-GluA2\textsubscript{3Y} (7)}%
\end{pgfscope}%
\begin{pgfscope}%
\pgfsetbuttcap%
\pgfsetmiterjoin%
\definecolor{currentfill}{rgb}{1.000000,0.494118,0.250980}%
\pgfsetfillcolor{currentfill}%
\pgfsetlinewidth{1.505625pt}%
\definecolor{currentstroke}{rgb}{1.000000,0.494118,0.250980}%
\pgfsetstrokecolor{currentstroke}%
\pgfsetdash{}{0pt}%
\pgfpathmoveto{\pgfqpoint{4.344492in}{1.936279in}}%
\pgfpathlineto{\pgfqpoint{4.455603in}{1.936279in}}%
\pgfpathlineto{\pgfqpoint{4.455603in}{2.014057in}}%
\pgfpathlineto{\pgfqpoint{4.344492in}{2.014057in}}%
\pgfpathclose%
\pgfusepath{stroke,fill}%
\end{pgfscope}%
\begin{pgfscope}%
\definecolor{textcolor}{rgb}{0.150000,0.150000,0.150000}%
\pgfsetstrokecolor{textcolor}%
\pgfsetfillcolor{textcolor}%
\pgftext[x=4.544492in,y=1.936279in,left,base]{\color{textcolor}\rmfamily\fontsize{8.000000}{9.600000}\selectfont Tg + Vehicle (6)}%
\end{pgfscope}%
\begin{pgfscope}%
\pgfsetbuttcap%
\pgfsetmiterjoin%
\definecolor{currentfill}{rgb}{1.000000,0.694118,0.250980}%
\pgfsetfillcolor{currentfill}%
\pgfsetlinewidth{1.505625pt}%
\definecolor{currentstroke}{rgb}{1.000000,0.694118,0.250980}%
\pgfsetstrokecolor{currentstroke}%
\pgfsetdash{}{0pt}%
\pgfpathmoveto{\pgfqpoint{4.344492in}{1.769639in}}%
\pgfpathlineto{\pgfqpoint{4.455603in}{1.769639in}}%
\pgfpathlineto{\pgfqpoint{4.455603in}{1.847417in}}%
\pgfpathlineto{\pgfqpoint{4.344492in}{1.847417in}}%
\pgfpathclose%
\pgfusepath{stroke,fill}%
\end{pgfscope}%
\begin{pgfscope}%
\definecolor{textcolor}{rgb}{0.150000,0.150000,0.150000}%
\pgfsetstrokecolor{textcolor}%
\pgfsetfillcolor{textcolor}%
\pgftext[x=4.544492in,y=1.769639in,left,base]{\color{textcolor}\rmfamily\fontsize{8.000000}{9.600000}\selectfont Tg + TAT-GluA2\textsubscript{3Y} (9)}%
\end{pgfscope}%
\end{pgfpicture}%
\makeatother%
\endgroup%

        \caption{\label{f.ad.reminder1.res}}
    \end{subfigure}
    \caption[\tglu{} treatment during a brief reminder rescues memory deficit.]{\gls{tg} mice retained a representation of fear memory for at least \SI{3}{\day}. \gls{wt} and \gls{tg} mice were contextual fear conditioned, given a reminder \SI{3}{\day} later with treatment and tested on the following day. \gls{tg} mice showed significantly less freezing. This effect was rescued by \tglu{} treatment. \label{f.ad.reminder1}}
\end{figure}

We then investigated whether the effect of the \tglu{} was memory specific. In a separate cohort of mice, we used a protocol similar to  Figure~\ref{f.ad.reminder1}, except the mice received \tglu{} in their home-cage instead of the reminder context (Figure~\ref{f.ad.reminder2}). In this case, the \tglu{} treatment did not have any effect on memory recall. There was no significant interaction between \textit{Genotype} and \textit{Treatment} (F\tsb{1,43}=0.2, p=0.66), however there was a significant main effect of \textit{Genotype} (F\tsb{1,43}=45.1, p<0.001), but no significant effect of \textit{Treatment} (F\tsb{1,43}=0.11, p=0.74). This suggests that \tglu{} treatment has no effect without the presence of the reminder. Together, these two results suggest that \gls{tg} mice can retain a neural representation of the fear memory, can therefore the memory deficit is not due to forgetting. The \tglu{} treatment during exposure to a reminder can rescue this deficit. 


\begin{figure}[h]
    \begin{subfigure}[h]{\textwidth}
        %% Creator: Matplotlib, PGF backend
%%
%% To include the figure in your LaTeX document, write
%%   \input{<filename>.pgf}
%%
%% Make sure the required packages are loaded in your preamble
%%   \usepackage{pgf}
%%
%% Figures using additional raster images can only be included by \input if
%% they are in the same directory as the main LaTeX file. For loading figures
%% from other directories you can use the `import` package
%%   \usepackage{import}
%% and then include the figures with
%%   \import{<path to file>}{<filename>.pgf}
%%
%% Matplotlib used the following preamble
%%   \usepackage[utf8]{inputenc}
%%   \usepackage[T1]{fontenc}
%%   \usepackage{siunitx}
%%
\begingroup%
\makeatletter%
\begin{pgfpicture}%
\pgfpathrectangle{\pgfpointorigin}{\pgfqpoint{6.059681in}{2.614199in}}%
\pgfusepath{use as bounding box, clip}%
\begin{pgfscope}%
\pgfsetbuttcap%
\pgfsetmiterjoin%
\definecolor{currentfill}{rgb}{1.000000,1.000000,1.000000}%
\pgfsetfillcolor{currentfill}%
\pgfsetlinewidth{0.000000pt}%
\definecolor{currentstroke}{rgb}{1.000000,1.000000,1.000000}%
\pgfsetstrokecolor{currentstroke}%
\pgfsetdash{}{0pt}%
\pgfpathmoveto{\pgfqpoint{0.000000in}{0.000000in}}%
\pgfpathlineto{\pgfqpoint{6.059681in}{0.000000in}}%
\pgfpathlineto{\pgfqpoint{6.059681in}{2.614199in}}%
\pgfpathlineto{\pgfqpoint{0.000000in}{2.614199in}}%
\pgfpathclose%
\pgfusepath{fill}%
\end{pgfscope}%
\begin{pgfscope}%
\pgfsetbuttcap%
\pgfsetmiterjoin%
\definecolor{currentfill}{rgb}{1.000000,1.000000,1.000000}%
\pgfsetfillcolor{currentfill}%
\pgfsetlinewidth{0.000000pt}%
\definecolor{currentstroke}{rgb}{0.000000,0.000000,0.000000}%
\pgfsetstrokecolor{currentstroke}%
\pgfsetstrokeopacity{0.000000}%
\pgfsetdash{}{0pt}%
\pgfpathmoveto{\pgfqpoint{0.536697in}{0.161328in}}%
\pgfpathlineto{\pgfqpoint{4.244492in}{0.161328in}}%
\pgfpathlineto{\pgfqpoint{4.244492in}{2.452871in}}%
\pgfpathlineto{\pgfqpoint{0.536697in}{2.452871in}}%
\pgfpathclose%
\pgfusepath{fill}%
\end{pgfscope}%
\begin{pgfscope}%
\pgfsetbuttcap%
\pgfsetroundjoin%
\definecolor{currentfill}{rgb}{0.150000,0.150000,0.150000}%
\pgfsetfillcolor{currentfill}%
\pgfsetlinewidth{1.003750pt}%
\definecolor{currentstroke}{rgb}{0.150000,0.150000,0.150000}%
\pgfsetstrokecolor{currentstroke}%
\pgfsetdash{}{0pt}%
\pgfsys@defobject{currentmarker}{\pgfqpoint{0.000000in}{0.000000in}}{\pgfqpoint{0.041667in}{0.000000in}}{%
\pgfpathmoveto{\pgfqpoint{0.000000in}{0.000000in}}%
\pgfpathlineto{\pgfqpoint{0.041667in}{0.000000in}}%
\pgfusepath{stroke,fill}%
}%
\begin{pgfscope}%
\pgfsys@transformshift{0.536697in}{0.161328in}%
\pgfsys@useobject{currentmarker}{}%
\end{pgfscope}%
\end{pgfscope}%
\begin{pgfscope}%
\definecolor{textcolor}{rgb}{0.150000,0.150000,0.150000}%
\pgfsetstrokecolor{textcolor}%
\pgfsetfillcolor{textcolor}%
\pgftext[x=0.439475in,y=0.161328in,right,]{\color{textcolor}\rmfamily\fontsize{10.000000}{12.000000}\selectfont \(\displaystyle 0\)}%
\end{pgfscope}%
\begin{pgfscope}%
\pgfsetbuttcap%
\pgfsetroundjoin%
\definecolor{currentfill}{rgb}{0.150000,0.150000,0.150000}%
\pgfsetfillcolor{currentfill}%
\pgfsetlinewidth{1.003750pt}%
\definecolor{currentstroke}{rgb}{0.150000,0.150000,0.150000}%
\pgfsetstrokecolor{currentstroke}%
\pgfsetdash{}{0pt}%
\pgfsys@defobject{currentmarker}{\pgfqpoint{0.000000in}{0.000000in}}{\pgfqpoint{0.041667in}{0.000000in}}{%
\pgfpathmoveto{\pgfqpoint{0.000000in}{0.000000in}}%
\pgfpathlineto{\pgfqpoint{0.041667in}{0.000000in}}%
\pgfusepath{stroke,fill}%
}%
\begin{pgfscope}%
\pgfsys@transformshift{0.536697in}{0.543251in}%
\pgfsys@useobject{currentmarker}{}%
\end{pgfscope}%
\end{pgfscope}%
\begin{pgfscope}%
\definecolor{textcolor}{rgb}{0.150000,0.150000,0.150000}%
\pgfsetstrokecolor{textcolor}%
\pgfsetfillcolor{textcolor}%
\pgftext[x=0.439475in,y=0.543251in,right,]{\color{textcolor}\rmfamily\fontsize{10.000000}{12.000000}\selectfont \(\displaystyle 10\)}%
\end{pgfscope}%
\begin{pgfscope}%
\pgfsetbuttcap%
\pgfsetroundjoin%
\definecolor{currentfill}{rgb}{0.150000,0.150000,0.150000}%
\pgfsetfillcolor{currentfill}%
\pgfsetlinewidth{1.003750pt}%
\definecolor{currentstroke}{rgb}{0.150000,0.150000,0.150000}%
\pgfsetstrokecolor{currentstroke}%
\pgfsetdash{}{0pt}%
\pgfsys@defobject{currentmarker}{\pgfqpoint{0.000000in}{0.000000in}}{\pgfqpoint{0.041667in}{0.000000in}}{%
\pgfpathmoveto{\pgfqpoint{0.000000in}{0.000000in}}%
\pgfpathlineto{\pgfqpoint{0.041667in}{0.000000in}}%
\pgfusepath{stroke,fill}%
}%
\begin{pgfscope}%
\pgfsys@transformshift{0.536697in}{0.925175in}%
\pgfsys@useobject{currentmarker}{}%
\end{pgfscope}%
\end{pgfscope}%
\begin{pgfscope}%
\definecolor{textcolor}{rgb}{0.150000,0.150000,0.150000}%
\pgfsetstrokecolor{textcolor}%
\pgfsetfillcolor{textcolor}%
\pgftext[x=0.439475in,y=0.925175in,right,]{\color{textcolor}\rmfamily\fontsize{10.000000}{12.000000}\selectfont \(\displaystyle 20\)}%
\end{pgfscope}%
\begin{pgfscope}%
\pgfsetbuttcap%
\pgfsetroundjoin%
\definecolor{currentfill}{rgb}{0.150000,0.150000,0.150000}%
\pgfsetfillcolor{currentfill}%
\pgfsetlinewidth{1.003750pt}%
\definecolor{currentstroke}{rgb}{0.150000,0.150000,0.150000}%
\pgfsetstrokecolor{currentstroke}%
\pgfsetdash{}{0pt}%
\pgfsys@defobject{currentmarker}{\pgfqpoint{0.000000in}{0.000000in}}{\pgfqpoint{0.041667in}{0.000000in}}{%
\pgfpathmoveto{\pgfqpoint{0.000000in}{0.000000in}}%
\pgfpathlineto{\pgfqpoint{0.041667in}{0.000000in}}%
\pgfusepath{stroke,fill}%
}%
\begin{pgfscope}%
\pgfsys@transformshift{0.536697in}{1.307099in}%
\pgfsys@useobject{currentmarker}{}%
\end{pgfscope}%
\end{pgfscope}%
\begin{pgfscope}%
\definecolor{textcolor}{rgb}{0.150000,0.150000,0.150000}%
\pgfsetstrokecolor{textcolor}%
\pgfsetfillcolor{textcolor}%
\pgftext[x=0.439475in,y=1.307099in,right,]{\color{textcolor}\rmfamily\fontsize{10.000000}{12.000000}\selectfont \(\displaystyle 30\)}%
\end{pgfscope}%
\begin{pgfscope}%
\pgfsetbuttcap%
\pgfsetroundjoin%
\definecolor{currentfill}{rgb}{0.150000,0.150000,0.150000}%
\pgfsetfillcolor{currentfill}%
\pgfsetlinewidth{1.003750pt}%
\definecolor{currentstroke}{rgb}{0.150000,0.150000,0.150000}%
\pgfsetstrokecolor{currentstroke}%
\pgfsetdash{}{0pt}%
\pgfsys@defobject{currentmarker}{\pgfqpoint{0.000000in}{0.000000in}}{\pgfqpoint{0.041667in}{0.000000in}}{%
\pgfpathmoveto{\pgfqpoint{0.000000in}{0.000000in}}%
\pgfpathlineto{\pgfqpoint{0.041667in}{0.000000in}}%
\pgfusepath{stroke,fill}%
}%
\begin{pgfscope}%
\pgfsys@transformshift{0.536697in}{1.689023in}%
\pgfsys@useobject{currentmarker}{}%
\end{pgfscope}%
\end{pgfscope}%
\begin{pgfscope}%
\definecolor{textcolor}{rgb}{0.150000,0.150000,0.150000}%
\pgfsetstrokecolor{textcolor}%
\pgfsetfillcolor{textcolor}%
\pgftext[x=0.439475in,y=1.689023in,right,]{\color{textcolor}\rmfamily\fontsize{10.000000}{12.000000}\selectfont \(\displaystyle 40\)}%
\end{pgfscope}%
\begin{pgfscope}%
\pgfsetbuttcap%
\pgfsetroundjoin%
\definecolor{currentfill}{rgb}{0.150000,0.150000,0.150000}%
\pgfsetfillcolor{currentfill}%
\pgfsetlinewidth{1.003750pt}%
\definecolor{currentstroke}{rgb}{0.150000,0.150000,0.150000}%
\pgfsetstrokecolor{currentstroke}%
\pgfsetdash{}{0pt}%
\pgfsys@defobject{currentmarker}{\pgfqpoint{0.000000in}{0.000000in}}{\pgfqpoint{0.041667in}{0.000000in}}{%
\pgfpathmoveto{\pgfqpoint{0.000000in}{0.000000in}}%
\pgfpathlineto{\pgfqpoint{0.041667in}{0.000000in}}%
\pgfusepath{stroke,fill}%
}%
\begin{pgfscope}%
\pgfsys@transformshift{0.536697in}{2.070947in}%
\pgfsys@useobject{currentmarker}{}%
\end{pgfscope}%
\end{pgfscope}%
\begin{pgfscope}%
\definecolor{textcolor}{rgb}{0.150000,0.150000,0.150000}%
\pgfsetstrokecolor{textcolor}%
\pgfsetfillcolor{textcolor}%
\pgftext[x=0.439475in,y=2.070947in,right,]{\color{textcolor}\rmfamily\fontsize{10.000000}{12.000000}\selectfont \(\displaystyle 50\)}%
\end{pgfscope}%
\begin{pgfscope}%
\pgfsetbuttcap%
\pgfsetroundjoin%
\definecolor{currentfill}{rgb}{0.150000,0.150000,0.150000}%
\pgfsetfillcolor{currentfill}%
\pgfsetlinewidth{1.003750pt}%
\definecolor{currentstroke}{rgb}{0.150000,0.150000,0.150000}%
\pgfsetstrokecolor{currentstroke}%
\pgfsetdash{}{0pt}%
\pgfsys@defobject{currentmarker}{\pgfqpoint{0.000000in}{0.000000in}}{\pgfqpoint{0.041667in}{0.000000in}}{%
\pgfpathmoveto{\pgfqpoint{0.000000in}{0.000000in}}%
\pgfpathlineto{\pgfqpoint{0.041667in}{0.000000in}}%
\pgfusepath{stroke,fill}%
}%
\begin{pgfscope}%
\pgfsys@transformshift{0.536697in}{2.452871in}%
\pgfsys@useobject{currentmarker}{}%
\end{pgfscope}%
\end{pgfscope}%
\begin{pgfscope}%
\definecolor{textcolor}{rgb}{0.150000,0.150000,0.150000}%
\pgfsetstrokecolor{textcolor}%
\pgfsetfillcolor{textcolor}%
\pgftext[x=0.439475in,y=2.452871in,right,]{\color{textcolor}\rmfamily\fontsize{10.000000}{12.000000}\selectfont \(\displaystyle 60\)}%
\end{pgfscope}%
\begin{pgfscope}%
\definecolor{textcolor}{rgb}{0.150000,0.150000,0.150000}%
\pgfsetstrokecolor{textcolor}%
\pgfsetfillcolor{textcolor}%
\pgftext[x=0.231141in,y=1.307099in,,bottom,rotate=90.000000]{\color{textcolor}\rmfamily\fontsize{10.000000}{12.000000}\selectfont \textbf{\% Freezing}}%
\end{pgfscope}%
\begin{pgfscope}%
\pgfpathrectangle{\pgfqpoint{0.536697in}{0.161328in}}{\pgfqpoint{3.707795in}{2.291544in}} %
\pgfusepath{clip}%
\pgfsetbuttcap%
\pgfsetmiterjoin%
\definecolor{currentfill}{rgb}{0.200000,0.427451,0.650980}%
\pgfsetfillcolor{currentfill}%
\pgfsetlinewidth{1.505625pt}%
\definecolor{currentstroke}{rgb}{0.200000,0.427451,0.650980}%
\pgfsetstrokecolor{currentstroke}%
\pgfsetdash{}{0pt}%
\pgfpathmoveto{\pgfqpoint{0.669118in}{0.161328in}}%
\pgfpathlineto{\pgfqpoint{1.331224in}{0.161328in}}%
\pgfpathlineto{\pgfqpoint{1.331224in}{1.807038in}}%
\pgfpathlineto{\pgfqpoint{0.669118in}{1.807038in}}%
\pgfpathclose%
\pgfusepath{stroke,fill}%
\end{pgfscope}%
\begin{pgfscope}%
\pgfpathrectangle{\pgfqpoint{0.536697in}{0.161328in}}{\pgfqpoint{3.707795in}{2.291544in}} %
\pgfusepath{clip}%
\pgfsetbuttcap%
\pgfsetmiterjoin%
\definecolor{currentfill}{rgb}{0.168627,0.670588,0.494118}%
\pgfsetfillcolor{currentfill}%
\pgfsetlinewidth{1.505625pt}%
\definecolor{currentstroke}{rgb}{0.168627,0.670588,0.494118}%
\pgfsetstrokecolor{currentstroke}%
\pgfsetdash{}{0pt}%
\pgfpathmoveto{\pgfqpoint{1.596067in}{0.161328in}}%
\pgfpathlineto{\pgfqpoint{2.258173in}{0.161328in}}%
\pgfpathlineto{\pgfqpoint{2.258173in}{1.661525in}}%
\pgfpathlineto{\pgfqpoint{1.596067in}{1.661525in}}%
\pgfpathclose%
\pgfusepath{stroke,fill}%
\end{pgfscope}%
\begin{pgfscope}%
\pgfpathrectangle{\pgfqpoint{0.536697in}{0.161328in}}{\pgfqpoint{3.707795in}{2.291544in}} %
\pgfusepath{clip}%
\pgfsetbuttcap%
\pgfsetmiterjoin%
\definecolor{currentfill}{rgb}{1.000000,0.494118,0.250980}%
\pgfsetfillcolor{currentfill}%
\pgfsetlinewidth{1.505625pt}%
\definecolor{currentstroke}{rgb}{1.000000,0.494118,0.250980}%
\pgfsetstrokecolor{currentstroke}%
\pgfsetdash{}{0pt}%
\pgfpathmoveto{\pgfqpoint{2.523016in}{0.161328in}}%
\pgfpathlineto{\pgfqpoint{3.185122in}{0.161328in}}%
\pgfpathlineto{\pgfqpoint{3.185122in}{0.304931in}}%
\pgfpathlineto{\pgfqpoint{2.523016in}{0.304931in}}%
\pgfpathclose%
\pgfusepath{stroke,fill}%
\end{pgfscope}%
\begin{pgfscope}%
\pgfpathrectangle{\pgfqpoint{0.536697in}{0.161328in}}{\pgfqpoint{3.707795in}{2.291544in}} %
\pgfusepath{clip}%
\pgfsetbuttcap%
\pgfsetmiterjoin%
\definecolor{currentfill}{rgb}{1.000000,0.694118,0.250980}%
\pgfsetfillcolor{currentfill}%
\pgfsetlinewidth{1.505625pt}%
\definecolor{currentstroke}{rgb}{1.000000,0.694118,0.250980}%
\pgfsetstrokecolor{currentstroke}%
\pgfsetdash{}{0pt}%
\pgfpathmoveto{\pgfqpoint{3.449965in}{0.161328in}}%
\pgfpathlineto{\pgfqpoint{4.112071in}{0.161328in}}%
\pgfpathlineto{\pgfqpoint{4.112071in}{0.347706in}}%
\pgfpathlineto{\pgfqpoint{3.449965in}{0.347706in}}%
\pgfpathclose%
\pgfusepath{stroke,fill}%
\end{pgfscope}%
\begin{pgfscope}%
\pgfpathrectangle{\pgfqpoint{0.536697in}{0.161328in}}{\pgfqpoint{3.707795in}{2.291544in}} %
\pgfusepath{clip}%
\pgfsetbuttcap%
\pgfsetroundjoin%
\pgfsetlinewidth{1.505625pt}%
\definecolor{currentstroke}{rgb}{0.200000,0.427451,0.650980}%
\pgfsetstrokecolor{currentstroke}%
\pgfsetdash{}{0pt}%
\pgfpathmoveto{\pgfqpoint{1.000171in}{1.807038in}}%
\pgfpathlineto{\pgfqpoint{1.000171in}{1.996472in}}%
\pgfusepath{stroke}%
\end{pgfscope}%
\begin{pgfscope}%
\pgfpathrectangle{\pgfqpoint{0.536697in}{0.161328in}}{\pgfqpoint{3.707795in}{2.291544in}} %
\pgfusepath{clip}%
\pgfsetbuttcap%
\pgfsetroundjoin%
\pgfsetlinewidth{1.505625pt}%
\definecolor{currentstroke}{rgb}{0.168627,0.670588,0.494118}%
\pgfsetstrokecolor{currentstroke}%
\pgfsetdash{}{0pt}%
\pgfpathmoveto{\pgfqpoint{1.927120in}{1.661525in}}%
\pgfpathlineto{\pgfqpoint{1.927120in}{1.913595in}}%
\pgfusepath{stroke}%
\end{pgfscope}%
\begin{pgfscope}%
\pgfpathrectangle{\pgfqpoint{0.536697in}{0.161328in}}{\pgfqpoint{3.707795in}{2.291544in}} %
\pgfusepath{clip}%
\pgfsetbuttcap%
\pgfsetroundjoin%
\pgfsetlinewidth{1.505625pt}%
\definecolor{currentstroke}{rgb}{1.000000,0.494118,0.250980}%
\pgfsetstrokecolor{currentstroke}%
\pgfsetdash{}{0pt}%
\pgfpathmoveto{\pgfqpoint{2.854069in}{0.304931in}}%
\pgfpathlineto{\pgfqpoint{2.854069in}{0.397357in}}%
\pgfusepath{stroke}%
\end{pgfscope}%
\begin{pgfscope}%
\pgfpathrectangle{\pgfqpoint{0.536697in}{0.161328in}}{\pgfqpoint{3.707795in}{2.291544in}} %
\pgfusepath{clip}%
\pgfsetbuttcap%
\pgfsetroundjoin%
\pgfsetlinewidth{1.505625pt}%
\definecolor{currentstroke}{rgb}{1.000000,0.694118,0.250980}%
\pgfsetstrokecolor{currentstroke}%
\pgfsetdash{}{0pt}%
\pgfpathmoveto{\pgfqpoint{3.781018in}{0.347706in}}%
\pgfpathlineto{\pgfqpoint{3.781018in}{0.451208in}}%
\pgfusepath{stroke}%
\end{pgfscope}%
\begin{pgfscope}%
\pgfpathrectangle{\pgfqpoint{0.536697in}{0.161328in}}{\pgfqpoint{3.707795in}{2.291544in}} %
\pgfusepath{clip}%
\pgfsetbuttcap%
\pgfsetroundjoin%
\definecolor{currentfill}{rgb}{0.200000,0.427451,0.650980}%
\pgfsetfillcolor{currentfill}%
\pgfsetlinewidth{1.505625pt}%
\definecolor{currentstroke}{rgb}{0.200000,0.427451,0.650980}%
\pgfsetstrokecolor{currentstroke}%
\pgfsetdash{}{0pt}%
\pgfsys@defobject{currentmarker}{\pgfqpoint{-0.111111in}{-0.000000in}}{\pgfqpoint{0.111111in}{0.000000in}}{%
\pgfpathmoveto{\pgfqpoint{0.111111in}{-0.000000in}}%
\pgfpathlineto{\pgfqpoint{-0.111111in}{0.000000in}}%
\pgfusepath{stroke,fill}%
}%
\begin{pgfscope}%
\pgfsys@transformshift{1.000171in}{1.807038in}%
\pgfsys@useobject{currentmarker}{}%
\end{pgfscope}%
\end{pgfscope}%
\begin{pgfscope}%
\pgfpathrectangle{\pgfqpoint{0.536697in}{0.161328in}}{\pgfqpoint{3.707795in}{2.291544in}} %
\pgfusepath{clip}%
\pgfsetbuttcap%
\pgfsetroundjoin%
\definecolor{currentfill}{rgb}{0.200000,0.427451,0.650980}%
\pgfsetfillcolor{currentfill}%
\pgfsetlinewidth{1.505625pt}%
\definecolor{currentstroke}{rgb}{0.200000,0.427451,0.650980}%
\pgfsetstrokecolor{currentstroke}%
\pgfsetdash{}{0pt}%
\pgfsys@defobject{currentmarker}{\pgfqpoint{-0.111111in}{-0.000000in}}{\pgfqpoint{0.111111in}{0.000000in}}{%
\pgfpathmoveto{\pgfqpoint{0.111111in}{-0.000000in}}%
\pgfpathlineto{\pgfqpoint{-0.111111in}{0.000000in}}%
\pgfusepath{stroke,fill}%
}%
\begin{pgfscope}%
\pgfsys@transformshift{1.000171in}{1.996472in}%
\pgfsys@useobject{currentmarker}{}%
\end{pgfscope}%
\end{pgfscope}%
\begin{pgfscope}%
\pgfpathrectangle{\pgfqpoint{0.536697in}{0.161328in}}{\pgfqpoint{3.707795in}{2.291544in}} %
\pgfusepath{clip}%
\pgfsetbuttcap%
\pgfsetroundjoin%
\definecolor{currentfill}{rgb}{0.168627,0.670588,0.494118}%
\pgfsetfillcolor{currentfill}%
\pgfsetlinewidth{1.505625pt}%
\definecolor{currentstroke}{rgb}{0.168627,0.670588,0.494118}%
\pgfsetstrokecolor{currentstroke}%
\pgfsetdash{}{0pt}%
\pgfsys@defobject{currentmarker}{\pgfqpoint{-0.111111in}{-0.000000in}}{\pgfqpoint{0.111111in}{0.000000in}}{%
\pgfpathmoveto{\pgfqpoint{0.111111in}{-0.000000in}}%
\pgfpathlineto{\pgfqpoint{-0.111111in}{0.000000in}}%
\pgfusepath{stroke,fill}%
}%
\begin{pgfscope}%
\pgfsys@transformshift{1.927120in}{1.661525in}%
\pgfsys@useobject{currentmarker}{}%
\end{pgfscope}%
\end{pgfscope}%
\begin{pgfscope}%
\pgfpathrectangle{\pgfqpoint{0.536697in}{0.161328in}}{\pgfqpoint{3.707795in}{2.291544in}} %
\pgfusepath{clip}%
\pgfsetbuttcap%
\pgfsetroundjoin%
\definecolor{currentfill}{rgb}{0.168627,0.670588,0.494118}%
\pgfsetfillcolor{currentfill}%
\pgfsetlinewidth{1.505625pt}%
\definecolor{currentstroke}{rgb}{0.168627,0.670588,0.494118}%
\pgfsetstrokecolor{currentstroke}%
\pgfsetdash{}{0pt}%
\pgfsys@defobject{currentmarker}{\pgfqpoint{-0.111111in}{-0.000000in}}{\pgfqpoint{0.111111in}{0.000000in}}{%
\pgfpathmoveto{\pgfqpoint{0.111111in}{-0.000000in}}%
\pgfpathlineto{\pgfqpoint{-0.111111in}{0.000000in}}%
\pgfusepath{stroke,fill}%
}%
\begin{pgfscope}%
\pgfsys@transformshift{1.927120in}{1.913595in}%
\pgfsys@useobject{currentmarker}{}%
\end{pgfscope}%
\end{pgfscope}%
\begin{pgfscope}%
\pgfpathrectangle{\pgfqpoint{0.536697in}{0.161328in}}{\pgfqpoint{3.707795in}{2.291544in}} %
\pgfusepath{clip}%
\pgfsetbuttcap%
\pgfsetroundjoin%
\definecolor{currentfill}{rgb}{1.000000,0.494118,0.250980}%
\pgfsetfillcolor{currentfill}%
\pgfsetlinewidth{1.505625pt}%
\definecolor{currentstroke}{rgb}{1.000000,0.494118,0.250980}%
\pgfsetstrokecolor{currentstroke}%
\pgfsetdash{}{0pt}%
\pgfsys@defobject{currentmarker}{\pgfqpoint{-0.111111in}{-0.000000in}}{\pgfqpoint{0.111111in}{0.000000in}}{%
\pgfpathmoveto{\pgfqpoint{0.111111in}{-0.000000in}}%
\pgfpathlineto{\pgfqpoint{-0.111111in}{0.000000in}}%
\pgfusepath{stroke,fill}%
}%
\begin{pgfscope}%
\pgfsys@transformshift{2.854069in}{0.304931in}%
\pgfsys@useobject{currentmarker}{}%
\end{pgfscope}%
\end{pgfscope}%
\begin{pgfscope}%
\pgfpathrectangle{\pgfqpoint{0.536697in}{0.161328in}}{\pgfqpoint{3.707795in}{2.291544in}} %
\pgfusepath{clip}%
\pgfsetbuttcap%
\pgfsetroundjoin%
\definecolor{currentfill}{rgb}{1.000000,0.494118,0.250980}%
\pgfsetfillcolor{currentfill}%
\pgfsetlinewidth{1.505625pt}%
\definecolor{currentstroke}{rgb}{1.000000,0.494118,0.250980}%
\pgfsetstrokecolor{currentstroke}%
\pgfsetdash{}{0pt}%
\pgfsys@defobject{currentmarker}{\pgfqpoint{-0.111111in}{-0.000000in}}{\pgfqpoint{0.111111in}{0.000000in}}{%
\pgfpathmoveto{\pgfqpoint{0.111111in}{-0.000000in}}%
\pgfpathlineto{\pgfqpoint{-0.111111in}{0.000000in}}%
\pgfusepath{stroke,fill}%
}%
\begin{pgfscope}%
\pgfsys@transformshift{2.854069in}{0.397357in}%
\pgfsys@useobject{currentmarker}{}%
\end{pgfscope}%
\end{pgfscope}%
\begin{pgfscope}%
\pgfpathrectangle{\pgfqpoint{0.536697in}{0.161328in}}{\pgfqpoint{3.707795in}{2.291544in}} %
\pgfusepath{clip}%
\pgfsetbuttcap%
\pgfsetroundjoin%
\definecolor{currentfill}{rgb}{1.000000,0.694118,0.250980}%
\pgfsetfillcolor{currentfill}%
\pgfsetlinewidth{1.505625pt}%
\definecolor{currentstroke}{rgb}{1.000000,0.694118,0.250980}%
\pgfsetstrokecolor{currentstroke}%
\pgfsetdash{}{0pt}%
\pgfsys@defobject{currentmarker}{\pgfqpoint{-0.111111in}{-0.000000in}}{\pgfqpoint{0.111111in}{0.000000in}}{%
\pgfpathmoveto{\pgfqpoint{0.111111in}{-0.000000in}}%
\pgfpathlineto{\pgfqpoint{-0.111111in}{0.000000in}}%
\pgfusepath{stroke,fill}%
}%
\begin{pgfscope}%
\pgfsys@transformshift{3.781018in}{0.347706in}%
\pgfsys@useobject{currentmarker}{}%
\end{pgfscope}%
\end{pgfscope}%
\begin{pgfscope}%
\pgfpathrectangle{\pgfqpoint{0.536697in}{0.161328in}}{\pgfqpoint{3.707795in}{2.291544in}} %
\pgfusepath{clip}%
\pgfsetbuttcap%
\pgfsetroundjoin%
\definecolor{currentfill}{rgb}{1.000000,0.694118,0.250980}%
\pgfsetfillcolor{currentfill}%
\pgfsetlinewidth{1.505625pt}%
\definecolor{currentstroke}{rgb}{1.000000,0.694118,0.250980}%
\pgfsetstrokecolor{currentstroke}%
\pgfsetdash{}{0pt}%
\pgfsys@defobject{currentmarker}{\pgfqpoint{-0.111111in}{-0.000000in}}{\pgfqpoint{0.111111in}{0.000000in}}{%
\pgfpathmoveto{\pgfqpoint{0.111111in}{-0.000000in}}%
\pgfpathlineto{\pgfqpoint{-0.111111in}{0.000000in}}%
\pgfusepath{stroke,fill}%
}%
\begin{pgfscope}%
\pgfsys@transformshift{3.781018in}{0.451208in}%
\pgfsys@useobject{currentmarker}{}%
\end{pgfscope}%
\end{pgfscope}%
\begin{pgfscope}%
\pgfpathrectangle{\pgfqpoint{0.536697in}{0.161328in}}{\pgfqpoint{3.707795in}{2.291544in}} %
\pgfusepath{clip}%
\pgfsetroundcap%
\pgfsetroundjoin%
\pgfsetlinewidth{1.756562pt}%
\definecolor{currentstroke}{rgb}{0.627451,0.627451,0.643137}%
\pgfsetstrokecolor{currentstroke}%
\pgfsetdash{}{0pt}%
\pgfpathmoveto{\pgfqpoint{1.000171in}{2.083278in}}%
\pgfpathlineto{\pgfqpoint{1.000171in}{2.227954in}}%
\pgfusepath{stroke}%
\end{pgfscope}%
\begin{pgfscope}%
\pgfpathrectangle{\pgfqpoint{0.536697in}{0.161328in}}{\pgfqpoint{3.707795in}{2.291544in}} %
\pgfusepath{clip}%
\pgfsetroundcap%
\pgfsetroundjoin%
\pgfsetlinewidth{1.756562pt}%
\definecolor{currentstroke}{rgb}{0.627451,0.627451,0.643137}%
\pgfsetstrokecolor{currentstroke}%
\pgfsetdash{}{0pt}%
\pgfpathmoveto{\pgfqpoint{1.000171in}{2.227954in}}%
\pgfpathlineto{\pgfqpoint{2.854069in}{2.227954in}}%
\pgfusepath{stroke}%
\end{pgfscope}%
\begin{pgfscope}%
\pgfpathrectangle{\pgfqpoint{0.536697in}{0.161328in}}{\pgfqpoint{3.707795in}{2.291544in}} %
\pgfusepath{clip}%
\pgfsetroundcap%
\pgfsetroundjoin%
\pgfsetlinewidth{1.756562pt}%
\definecolor{currentstroke}{rgb}{0.627451,0.627451,0.643137}%
\pgfsetstrokecolor{currentstroke}%
\pgfsetdash{}{0pt}%
\pgfpathmoveto{\pgfqpoint{2.854069in}{2.227954in}}%
\pgfpathlineto{\pgfqpoint{2.854069in}{0.570968in}}%
\pgfusepath{stroke}%
\end{pgfscope}%
\begin{pgfscope}%
\pgfsetrectcap%
\pgfsetmiterjoin%
\pgfsetlinewidth{1.254687pt}%
\definecolor{currentstroke}{rgb}{0.150000,0.150000,0.150000}%
\pgfsetstrokecolor{currentstroke}%
\pgfsetdash{}{0pt}%
\pgfpathmoveto{\pgfqpoint{0.536697in}{0.161328in}}%
\pgfpathlineto{\pgfqpoint{0.536697in}{2.452871in}}%
\pgfusepath{stroke}%
\end{pgfscope}%
\begin{pgfscope}%
\pgfsetrectcap%
\pgfsetmiterjoin%
\pgfsetlinewidth{1.254687pt}%
\definecolor{currentstroke}{rgb}{0.150000,0.150000,0.150000}%
\pgfsetstrokecolor{currentstroke}%
\pgfsetdash{}{0pt}%
\pgfpathmoveto{\pgfqpoint{0.536697in}{0.161328in}}%
\pgfpathlineto{\pgfqpoint{4.244492in}{0.161328in}}%
\pgfusepath{stroke}%
\end{pgfscope}%
\begin{pgfscope}%
\definecolor{textcolor}{rgb}{0.150000,0.150000,0.150000}%
\pgfsetstrokecolor{textcolor}%
\pgfsetfillcolor{textcolor}%
\pgftext[x=2.854069in,y=0.451610in,,]{\color{textcolor}\rmfamily\fontsize{15.000000}{18.000000}\selectfont \textbf{*}}%
\end{pgfscope}%
\begin{pgfscope}%
\pgfsetbuttcap%
\pgfsetmiterjoin%
\definecolor{currentfill}{rgb}{0.200000,0.427451,0.650980}%
\pgfsetfillcolor{currentfill}%
\pgfsetlinewidth{1.505625pt}%
\definecolor{currentstroke}{rgb}{0.200000,0.427451,0.650980}%
\pgfsetstrokecolor{currentstroke}%
\pgfsetdash{}{0pt}%
\pgfpathmoveto{\pgfqpoint{4.344492in}{2.269558in}}%
\pgfpathlineto{\pgfqpoint{4.455603in}{2.269558in}}%
\pgfpathlineto{\pgfqpoint{4.455603in}{2.347336in}}%
\pgfpathlineto{\pgfqpoint{4.344492in}{2.347336in}}%
\pgfpathclose%
\pgfusepath{stroke,fill}%
\end{pgfscope}%
\begin{pgfscope}%
\definecolor{textcolor}{rgb}{0.150000,0.150000,0.150000}%
\pgfsetstrokecolor{textcolor}%
\pgfsetfillcolor{textcolor}%
\pgftext[x=4.544492in,y=2.269558in,left,base]{\color{textcolor}\rmfamily\fontsize{8.000000}{9.600000}\selectfont WT + Vehicle (11)}%
\end{pgfscope}%
\begin{pgfscope}%
\pgfsetbuttcap%
\pgfsetmiterjoin%
\definecolor{currentfill}{rgb}{0.168627,0.670588,0.494118}%
\pgfsetfillcolor{currentfill}%
\pgfsetlinewidth{1.505625pt}%
\definecolor{currentstroke}{rgb}{0.168627,0.670588,0.494118}%
\pgfsetstrokecolor{currentstroke}%
\pgfsetdash{}{0pt}%
\pgfpathmoveto{\pgfqpoint{4.344492in}{2.102918in}}%
\pgfpathlineto{\pgfqpoint{4.455603in}{2.102918in}}%
\pgfpathlineto{\pgfqpoint{4.455603in}{2.180696in}}%
\pgfpathlineto{\pgfqpoint{4.344492in}{2.180696in}}%
\pgfpathclose%
\pgfusepath{stroke,fill}%
\end{pgfscope}%
\begin{pgfscope}%
\definecolor{textcolor}{rgb}{0.150000,0.150000,0.150000}%
\pgfsetstrokecolor{textcolor}%
\pgfsetfillcolor{textcolor}%
\pgftext[x=4.544492in,y=2.102918in,left,base]{\color{textcolor}\rmfamily\fontsize{8.000000}{9.600000}\selectfont WT + TAT-GluA2\textsubscript{3Y} (11)}%
\end{pgfscope}%
\begin{pgfscope}%
\pgfsetbuttcap%
\pgfsetmiterjoin%
\definecolor{currentfill}{rgb}{1.000000,0.494118,0.250980}%
\pgfsetfillcolor{currentfill}%
\pgfsetlinewidth{1.505625pt}%
\definecolor{currentstroke}{rgb}{1.000000,0.494118,0.250980}%
\pgfsetstrokecolor{currentstroke}%
\pgfsetdash{}{0pt}%
\pgfpathmoveto{\pgfqpoint{4.344492in}{1.936279in}}%
\pgfpathlineto{\pgfqpoint{4.455603in}{1.936279in}}%
\pgfpathlineto{\pgfqpoint{4.455603in}{2.014057in}}%
\pgfpathlineto{\pgfqpoint{4.344492in}{2.014057in}}%
\pgfpathclose%
\pgfusepath{stroke,fill}%
\end{pgfscope}%
\begin{pgfscope}%
\definecolor{textcolor}{rgb}{0.150000,0.150000,0.150000}%
\pgfsetstrokecolor{textcolor}%
\pgfsetfillcolor{textcolor}%
\pgftext[x=4.544492in,y=1.936279in,left,base]{\color{textcolor}\rmfamily\fontsize{8.000000}{9.600000}\selectfont Tg + Vehicle (4)}%
\end{pgfscope}%
\begin{pgfscope}%
\pgfsetbuttcap%
\pgfsetmiterjoin%
\definecolor{currentfill}{rgb}{1.000000,0.694118,0.250980}%
\pgfsetfillcolor{currentfill}%
\pgfsetlinewidth{1.505625pt}%
\definecolor{currentstroke}{rgb}{1.000000,0.694118,0.250980}%
\pgfsetstrokecolor{currentstroke}%
\pgfsetdash{}{0pt}%
\pgfpathmoveto{\pgfqpoint{4.344492in}{1.769639in}}%
\pgfpathlineto{\pgfqpoint{4.455603in}{1.769639in}}%
\pgfpathlineto{\pgfqpoint{4.455603in}{1.847417in}}%
\pgfpathlineto{\pgfqpoint{4.344492in}{1.847417in}}%
\pgfpathclose%
\pgfusepath{stroke,fill}%
\end{pgfscope}%
\begin{pgfscope}%
\definecolor{textcolor}{rgb}{0.150000,0.150000,0.150000}%
\pgfsetstrokecolor{textcolor}%
\pgfsetfillcolor{textcolor}%
\pgftext[x=4.544492in,y=1.769639in,left,base]{\color{textcolor}\rmfamily\fontsize{8.000000}{9.600000}\selectfont Tg + TAT-GluA2\textsubscript{3Y} (5)}%
\end{pgfscope}%
\end{pgfpicture}%
\makeatother%
\endgroup%

        \caption{\label{f.ad.reminder2.paradigm}}
    \end{subfigure}
    \begin{subfigure}[h]{\textwidth}
        %% Creator: Matplotlib, PGF backend
%%
%% To include the figure in your LaTeX document, write
%%   \input{<filename>.pgf}
%%
%% Make sure the required packages are loaded in your preamble
%%   \usepackage{pgf}
%%
%% Figures using additional raster images can only be included by \input if
%% they are in the same directory as the main LaTeX file. For loading figures
%% from other directories you can use the `import` package
%%   \usepackage{import}
%% and then include the figures with
%%   \import{<path to file>}{<filename>.pgf}
%%
%% Matplotlib used the following preamble
%%   \usepackage[utf8]{inputenc}
%%   \usepackage[T1]{fontenc}
%%   \usepackage{siunitx}
%%
\begingroup%
\makeatletter%
\begin{pgfpicture}%
\pgfpathrectangle{\pgfpointorigin}{\pgfqpoint{6.059681in}{2.614199in}}%
\pgfusepath{use as bounding box, clip}%
\begin{pgfscope}%
\pgfsetbuttcap%
\pgfsetmiterjoin%
\definecolor{currentfill}{rgb}{1.000000,1.000000,1.000000}%
\pgfsetfillcolor{currentfill}%
\pgfsetlinewidth{0.000000pt}%
\definecolor{currentstroke}{rgb}{1.000000,1.000000,1.000000}%
\pgfsetstrokecolor{currentstroke}%
\pgfsetdash{}{0pt}%
\pgfpathmoveto{\pgfqpoint{0.000000in}{0.000000in}}%
\pgfpathlineto{\pgfqpoint{6.059681in}{0.000000in}}%
\pgfpathlineto{\pgfqpoint{6.059681in}{2.614199in}}%
\pgfpathlineto{\pgfqpoint{0.000000in}{2.614199in}}%
\pgfpathclose%
\pgfusepath{fill}%
\end{pgfscope}%
\begin{pgfscope}%
\pgfsetbuttcap%
\pgfsetmiterjoin%
\definecolor{currentfill}{rgb}{1.000000,1.000000,1.000000}%
\pgfsetfillcolor{currentfill}%
\pgfsetlinewidth{0.000000pt}%
\definecolor{currentstroke}{rgb}{0.000000,0.000000,0.000000}%
\pgfsetstrokecolor{currentstroke}%
\pgfsetstrokeopacity{0.000000}%
\pgfsetdash{}{0pt}%
\pgfpathmoveto{\pgfqpoint{0.536697in}{0.161328in}}%
\pgfpathlineto{\pgfqpoint{4.244492in}{0.161328in}}%
\pgfpathlineto{\pgfqpoint{4.244492in}{2.452871in}}%
\pgfpathlineto{\pgfqpoint{0.536697in}{2.452871in}}%
\pgfpathclose%
\pgfusepath{fill}%
\end{pgfscope}%
\begin{pgfscope}%
\pgfsetbuttcap%
\pgfsetroundjoin%
\definecolor{currentfill}{rgb}{0.150000,0.150000,0.150000}%
\pgfsetfillcolor{currentfill}%
\pgfsetlinewidth{1.003750pt}%
\definecolor{currentstroke}{rgb}{0.150000,0.150000,0.150000}%
\pgfsetstrokecolor{currentstroke}%
\pgfsetdash{}{0pt}%
\pgfsys@defobject{currentmarker}{\pgfqpoint{0.000000in}{0.000000in}}{\pgfqpoint{0.041667in}{0.000000in}}{%
\pgfpathmoveto{\pgfqpoint{0.000000in}{0.000000in}}%
\pgfpathlineto{\pgfqpoint{0.041667in}{0.000000in}}%
\pgfusepath{stroke,fill}%
}%
\begin{pgfscope}%
\pgfsys@transformshift{0.536697in}{0.161328in}%
\pgfsys@useobject{currentmarker}{}%
\end{pgfscope}%
\end{pgfscope}%
\begin{pgfscope}%
\definecolor{textcolor}{rgb}{0.150000,0.150000,0.150000}%
\pgfsetstrokecolor{textcolor}%
\pgfsetfillcolor{textcolor}%
\pgftext[x=0.439475in,y=0.161328in,right,]{\color{textcolor}\rmfamily\fontsize{10.000000}{12.000000}\selectfont \(\displaystyle 0\)}%
\end{pgfscope}%
\begin{pgfscope}%
\pgfsetbuttcap%
\pgfsetroundjoin%
\definecolor{currentfill}{rgb}{0.150000,0.150000,0.150000}%
\pgfsetfillcolor{currentfill}%
\pgfsetlinewidth{1.003750pt}%
\definecolor{currentstroke}{rgb}{0.150000,0.150000,0.150000}%
\pgfsetstrokecolor{currentstroke}%
\pgfsetdash{}{0pt}%
\pgfsys@defobject{currentmarker}{\pgfqpoint{0.000000in}{0.000000in}}{\pgfqpoint{0.041667in}{0.000000in}}{%
\pgfpathmoveto{\pgfqpoint{0.000000in}{0.000000in}}%
\pgfpathlineto{\pgfqpoint{0.041667in}{0.000000in}}%
\pgfusepath{stroke,fill}%
}%
\begin{pgfscope}%
\pgfsys@transformshift{0.536697in}{0.543251in}%
\pgfsys@useobject{currentmarker}{}%
\end{pgfscope}%
\end{pgfscope}%
\begin{pgfscope}%
\definecolor{textcolor}{rgb}{0.150000,0.150000,0.150000}%
\pgfsetstrokecolor{textcolor}%
\pgfsetfillcolor{textcolor}%
\pgftext[x=0.439475in,y=0.543251in,right,]{\color{textcolor}\rmfamily\fontsize{10.000000}{12.000000}\selectfont \(\displaystyle 10\)}%
\end{pgfscope}%
\begin{pgfscope}%
\pgfsetbuttcap%
\pgfsetroundjoin%
\definecolor{currentfill}{rgb}{0.150000,0.150000,0.150000}%
\pgfsetfillcolor{currentfill}%
\pgfsetlinewidth{1.003750pt}%
\definecolor{currentstroke}{rgb}{0.150000,0.150000,0.150000}%
\pgfsetstrokecolor{currentstroke}%
\pgfsetdash{}{0pt}%
\pgfsys@defobject{currentmarker}{\pgfqpoint{0.000000in}{0.000000in}}{\pgfqpoint{0.041667in}{0.000000in}}{%
\pgfpathmoveto{\pgfqpoint{0.000000in}{0.000000in}}%
\pgfpathlineto{\pgfqpoint{0.041667in}{0.000000in}}%
\pgfusepath{stroke,fill}%
}%
\begin{pgfscope}%
\pgfsys@transformshift{0.536697in}{0.925175in}%
\pgfsys@useobject{currentmarker}{}%
\end{pgfscope}%
\end{pgfscope}%
\begin{pgfscope}%
\definecolor{textcolor}{rgb}{0.150000,0.150000,0.150000}%
\pgfsetstrokecolor{textcolor}%
\pgfsetfillcolor{textcolor}%
\pgftext[x=0.439475in,y=0.925175in,right,]{\color{textcolor}\rmfamily\fontsize{10.000000}{12.000000}\selectfont \(\displaystyle 20\)}%
\end{pgfscope}%
\begin{pgfscope}%
\pgfsetbuttcap%
\pgfsetroundjoin%
\definecolor{currentfill}{rgb}{0.150000,0.150000,0.150000}%
\pgfsetfillcolor{currentfill}%
\pgfsetlinewidth{1.003750pt}%
\definecolor{currentstroke}{rgb}{0.150000,0.150000,0.150000}%
\pgfsetstrokecolor{currentstroke}%
\pgfsetdash{}{0pt}%
\pgfsys@defobject{currentmarker}{\pgfqpoint{0.000000in}{0.000000in}}{\pgfqpoint{0.041667in}{0.000000in}}{%
\pgfpathmoveto{\pgfqpoint{0.000000in}{0.000000in}}%
\pgfpathlineto{\pgfqpoint{0.041667in}{0.000000in}}%
\pgfusepath{stroke,fill}%
}%
\begin{pgfscope}%
\pgfsys@transformshift{0.536697in}{1.307099in}%
\pgfsys@useobject{currentmarker}{}%
\end{pgfscope}%
\end{pgfscope}%
\begin{pgfscope}%
\definecolor{textcolor}{rgb}{0.150000,0.150000,0.150000}%
\pgfsetstrokecolor{textcolor}%
\pgfsetfillcolor{textcolor}%
\pgftext[x=0.439475in,y=1.307099in,right,]{\color{textcolor}\rmfamily\fontsize{10.000000}{12.000000}\selectfont \(\displaystyle 30\)}%
\end{pgfscope}%
\begin{pgfscope}%
\pgfsetbuttcap%
\pgfsetroundjoin%
\definecolor{currentfill}{rgb}{0.150000,0.150000,0.150000}%
\pgfsetfillcolor{currentfill}%
\pgfsetlinewidth{1.003750pt}%
\definecolor{currentstroke}{rgb}{0.150000,0.150000,0.150000}%
\pgfsetstrokecolor{currentstroke}%
\pgfsetdash{}{0pt}%
\pgfsys@defobject{currentmarker}{\pgfqpoint{0.000000in}{0.000000in}}{\pgfqpoint{0.041667in}{0.000000in}}{%
\pgfpathmoveto{\pgfqpoint{0.000000in}{0.000000in}}%
\pgfpathlineto{\pgfqpoint{0.041667in}{0.000000in}}%
\pgfusepath{stroke,fill}%
}%
\begin{pgfscope}%
\pgfsys@transformshift{0.536697in}{1.689023in}%
\pgfsys@useobject{currentmarker}{}%
\end{pgfscope}%
\end{pgfscope}%
\begin{pgfscope}%
\definecolor{textcolor}{rgb}{0.150000,0.150000,0.150000}%
\pgfsetstrokecolor{textcolor}%
\pgfsetfillcolor{textcolor}%
\pgftext[x=0.439475in,y=1.689023in,right,]{\color{textcolor}\rmfamily\fontsize{10.000000}{12.000000}\selectfont \(\displaystyle 40\)}%
\end{pgfscope}%
\begin{pgfscope}%
\pgfsetbuttcap%
\pgfsetroundjoin%
\definecolor{currentfill}{rgb}{0.150000,0.150000,0.150000}%
\pgfsetfillcolor{currentfill}%
\pgfsetlinewidth{1.003750pt}%
\definecolor{currentstroke}{rgb}{0.150000,0.150000,0.150000}%
\pgfsetstrokecolor{currentstroke}%
\pgfsetdash{}{0pt}%
\pgfsys@defobject{currentmarker}{\pgfqpoint{0.000000in}{0.000000in}}{\pgfqpoint{0.041667in}{0.000000in}}{%
\pgfpathmoveto{\pgfqpoint{0.000000in}{0.000000in}}%
\pgfpathlineto{\pgfqpoint{0.041667in}{0.000000in}}%
\pgfusepath{stroke,fill}%
}%
\begin{pgfscope}%
\pgfsys@transformshift{0.536697in}{2.070947in}%
\pgfsys@useobject{currentmarker}{}%
\end{pgfscope}%
\end{pgfscope}%
\begin{pgfscope}%
\definecolor{textcolor}{rgb}{0.150000,0.150000,0.150000}%
\pgfsetstrokecolor{textcolor}%
\pgfsetfillcolor{textcolor}%
\pgftext[x=0.439475in,y=2.070947in,right,]{\color{textcolor}\rmfamily\fontsize{10.000000}{12.000000}\selectfont \(\displaystyle 50\)}%
\end{pgfscope}%
\begin{pgfscope}%
\pgfsetbuttcap%
\pgfsetroundjoin%
\definecolor{currentfill}{rgb}{0.150000,0.150000,0.150000}%
\pgfsetfillcolor{currentfill}%
\pgfsetlinewidth{1.003750pt}%
\definecolor{currentstroke}{rgb}{0.150000,0.150000,0.150000}%
\pgfsetstrokecolor{currentstroke}%
\pgfsetdash{}{0pt}%
\pgfsys@defobject{currentmarker}{\pgfqpoint{0.000000in}{0.000000in}}{\pgfqpoint{0.041667in}{0.000000in}}{%
\pgfpathmoveto{\pgfqpoint{0.000000in}{0.000000in}}%
\pgfpathlineto{\pgfqpoint{0.041667in}{0.000000in}}%
\pgfusepath{stroke,fill}%
}%
\begin{pgfscope}%
\pgfsys@transformshift{0.536697in}{2.452871in}%
\pgfsys@useobject{currentmarker}{}%
\end{pgfscope}%
\end{pgfscope}%
\begin{pgfscope}%
\definecolor{textcolor}{rgb}{0.150000,0.150000,0.150000}%
\pgfsetstrokecolor{textcolor}%
\pgfsetfillcolor{textcolor}%
\pgftext[x=0.439475in,y=2.452871in,right,]{\color{textcolor}\rmfamily\fontsize{10.000000}{12.000000}\selectfont \(\displaystyle 60\)}%
\end{pgfscope}%
\begin{pgfscope}%
\definecolor{textcolor}{rgb}{0.150000,0.150000,0.150000}%
\pgfsetstrokecolor{textcolor}%
\pgfsetfillcolor{textcolor}%
\pgftext[x=0.231141in,y=1.307099in,,bottom,rotate=90.000000]{\color{textcolor}\rmfamily\fontsize{10.000000}{12.000000}\selectfont \textbf{\% Freezing}}%
\end{pgfscope}%
\begin{pgfscope}%
\pgfpathrectangle{\pgfqpoint{0.536697in}{0.161328in}}{\pgfqpoint{3.707795in}{2.291544in}} %
\pgfusepath{clip}%
\pgfsetbuttcap%
\pgfsetmiterjoin%
\definecolor{currentfill}{rgb}{0.200000,0.427451,0.650980}%
\pgfsetfillcolor{currentfill}%
\pgfsetlinewidth{1.505625pt}%
\definecolor{currentstroke}{rgb}{0.200000,0.427451,0.650980}%
\pgfsetstrokecolor{currentstroke}%
\pgfsetdash{}{0pt}%
\pgfpathmoveto{\pgfqpoint{0.669118in}{0.161328in}}%
\pgfpathlineto{\pgfqpoint{1.331224in}{0.161328in}}%
\pgfpathlineto{\pgfqpoint{1.331224in}{1.807038in}}%
\pgfpathlineto{\pgfqpoint{0.669118in}{1.807038in}}%
\pgfpathclose%
\pgfusepath{stroke,fill}%
\end{pgfscope}%
\begin{pgfscope}%
\pgfpathrectangle{\pgfqpoint{0.536697in}{0.161328in}}{\pgfqpoint{3.707795in}{2.291544in}} %
\pgfusepath{clip}%
\pgfsetbuttcap%
\pgfsetmiterjoin%
\definecolor{currentfill}{rgb}{0.168627,0.670588,0.494118}%
\pgfsetfillcolor{currentfill}%
\pgfsetlinewidth{1.505625pt}%
\definecolor{currentstroke}{rgb}{0.168627,0.670588,0.494118}%
\pgfsetstrokecolor{currentstroke}%
\pgfsetdash{}{0pt}%
\pgfpathmoveto{\pgfqpoint{1.596067in}{0.161328in}}%
\pgfpathlineto{\pgfqpoint{2.258173in}{0.161328in}}%
\pgfpathlineto{\pgfqpoint{2.258173in}{1.661525in}}%
\pgfpathlineto{\pgfqpoint{1.596067in}{1.661525in}}%
\pgfpathclose%
\pgfusepath{stroke,fill}%
\end{pgfscope}%
\begin{pgfscope}%
\pgfpathrectangle{\pgfqpoint{0.536697in}{0.161328in}}{\pgfqpoint{3.707795in}{2.291544in}} %
\pgfusepath{clip}%
\pgfsetbuttcap%
\pgfsetmiterjoin%
\definecolor{currentfill}{rgb}{1.000000,0.494118,0.250980}%
\pgfsetfillcolor{currentfill}%
\pgfsetlinewidth{1.505625pt}%
\definecolor{currentstroke}{rgb}{1.000000,0.494118,0.250980}%
\pgfsetstrokecolor{currentstroke}%
\pgfsetdash{}{0pt}%
\pgfpathmoveto{\pgfqpoint{2.523016in}{0.161328in}}%
\pgfpathlineto{\pgfqpoint{3.185122in}{0.161328in}}%
\pgfpathlineto{\pgfqpoint{3.185122in}{0.304931in}}%
\pgfpathlineto{\pgfqpoint{2.523016in}{0.304931in}}%
\pgfpathclose%
\pgfusepath{stroke,fill}%
\end{pgfscope}%
\begin{pgfscope}%
\pgfpathrectangle{\pgfqpoint{0.536697in}{0.161328in}}{\pgfqpoint{3.707795in}{2.291544in}} %
\pgfusepath{clip}%
\pgfsetbuttcap%
\pgfsetmiterjoin%
\definecolor{currentfill}{rgb}{1.000000,0.694118,0.250980}%
\pgfsetfillcolor{currentfill}%
\pgfsetlinewidth{1.505625pt}%
\definecolor{currentstroke}{rgb}{1.000000,0.694118,0.250980}%
\pgfsetstrokecolor{currentstroke}%
\pgfsetdash{}{0pt}%
\pgfpathmoveto{\pgfqpoint{3.449965in}{0.161328in}}%
\pgfpathlineto{\pgfqpoint{4.112071in}{0.161328in}}%
\pgfpathlineto{\pgfqpoint{4.112071in}{0.347706in}}%
\pgfpathlineto{\pgfqpoint{3.449965in}{0.347706in}}%
\pgfpathclose%
\pgfusepath{stroke,fill}%
\end{pgfscope}%
\begin{pgfscope}%
\pgfpathrectangle{\pgfqpoint{0.536697in}{0.161328in}}{\pgfqpoint{3.707795in}{2.291544in}} %
\pgfusepath{clip}%
\pgfsetbuttcap%
\pgfsetroundjoin%
\pgfsetlinewidth{1.505625pt}%
\definecolor{currentstroke}{rgb}{0.200000,0.427451,0.650980}%
\pgfsetstrokecolor{currentstroke}%
\pgfsetdash{}{0pt}%
\pgfpathmoveto{\pgfqpoint{1.000171in}{1.807038in}}%
\pgfpathlineto{\pgfqpoint{1.000171in}{1.996472in}}%
\pgfusepath{stroke}%
\end{pgfscope}%
\begin{pgfscope}%
\pgfpathrectangle{\pgfqpoint{0.536697in}{0.161328in}}{\pgfqpoint{3.707795in}{2.291544in}} %
\pgfusepath{clip}%
\pgfsetbuttcap%
\pgfsetroundjoin%
\pgfsetlinewidth{1.505625pt}%
\definecolor{currentstroke}{rgb}{0.168627,0.670588,0.494118}%
\pgfsetstrokecolor{currentstroke}%
\pgfsetdash{}{0pt}%
\pgfpathmoveto{\pgfqpoint{1.927120in}{1.661525in}}%
\pgfpathlineto{\pgfqpoint{1.927120in}{1.913595in}}%
\pgfusepath{stroke}%
\end{pgfscope}%
\begin{pgfscope}%
\pgfpathrectangle{\pgfqpoint{0.536697in}{0.161328in}}{\pgfqpoint{3.707795in}{2.291544in}} %
\pgfusepath{clip}%
\pgfsetbuttcap%
\pgfsetroundjoin%
\pgfsetlinewidth{1.505625pt}%
\definecolor{currentstroke}{rgb}{1.000000,0.494118,0.250980}%
\pgfsetstrokecolor{currentstroke}%
\pgfsetdash{}{0pt}%
\pgfpathmoveto{\pgfqpoint{2.854069in}{0.304931in}}%
\pgfpathlineto{\pgfqpoint{2.854069in}{0.397357in}}%
\pgfusepath{stroke}%
\end{pgfscope}%
\begin{pgfscope}%
\pgfpathrectangle{\pgfqpoint{0.536697in}{0.161328in}}{\pgfqpoint{3.707795in}{2.291544in}} %
\pgfusepath{clip}%
\pgfsetbuttcap%
\pgfsetroundjoin%
\pgfsetlinewidth{1.505625pt}%
\definecolor{currentstroke}{rgb}{1.000000,0.694118,0.250980}%
\pgfsetstrokecolor{currentstroke}%
\pgfsetdash{}{0pt}%
\pgfpathmoveto{\pgfqpoint{3.781018in}{0.347706in}}%
\pgfpathlineto{\pgfqpoint{3.781018in}{0.451208in}}%
\pgfusepath{stroke}%
\end{pgfscope}%
\begin{pgfscope}%
\pgfpathrectangle{\pgfqpoint{0.536697in}{0.161328in}}{\pgfqpoint{3.707795in}{2.291544in}} %
\pgfusepath{clip}%
\pgfsetbuttcap%
\pgfsetroundjoin%
\definecolor{currentfill}{rgb}{0.200000,0.427451,0.650980}%
\pgfsetfillcolor{currentfill}%
\pgfsetlinewidth{1.505625pt}%
\definecolor{currentstroke}{rgb}{0.200000,0.427451,0.650980}%
\pgfsetstrokecolor{currentstroke}%
\pgfsetdash{}{0pt}%
\pgfsys@defobject{currentmarker}{\pgfqpoint{-0.111111in}{-0.000000in}}{\pgfqpoint{0.111111in}{0.000000in}}{%
\pgfpathmoveto{\pgfqpoint{0.111111in}{-0.000000in}}%
\pgfpathlineto{\pgfqpoint{-0.111111in}{0.000000in}}%
\pgfusepath{stroke,fill}%
}%
\begin{pgfscope}%
\pgfsys@transformshift{1.000171in}{1.807038in}%
\pgfsys@useobject{currentmarker}{}%
\end{pgfscope}%
\end{pgfscope}%
\begin{pgfscope}%
\pgfpathrectangle{\pgfqpoint{0.536697in}{0.161328in}}{\pgfqpoint{3.707795in}{2.291544in}} %
\pgfusepath{clip}%
\pgfsetbuttcap%
\pgfsetroundjoin%
\definecolor{currentfill}{rgb}{0.200000,0.427451,0.650980}%
\pgfsetfillcolor{currentfill}%
\pgfsetlinewidth{1.505625pt}%
\definecolor{currentstroke}{rgb}{0.200000,0.427451,0.650980}%
\pgfsetstrokecolor{currentstroke}%
\pgfsetdash{}{0pt}%
\pgfsys@defobject{currentmarker}{\pgfqpoint{-0.111111in}{-0.000000in}}{\pgfqpoint{0.111111in}{0.000000in}}{%
\pgfpathmoveto{\pgfqpoint{0.111111in}{-0.000000in}}%
\pgfpathlineto{\pgfqpoint{-0.111111in}{0.000000in}}%
\pgfusepath{stroke,fill}%
}%
\begin{pgfscope}%
\pgfsys@transformshift{1.000171in}{1.996472in}%
\pgfsys@useobject{currentmarker}{}%
\end{pgfscope}%
\end{pgfscope}%
\begin{pgfscope}%
\pgfpathrectangle{\pgfqpoint{0.536697in}{0.161328in}}{\pgfqpoint{3.707795in}{2.291544in}} %
\pgfusepath{clip}%
\pgfsetbuttcap%
\pgfsetroundjoin%
\definecolor{currentfill}{rgb}{0.168627,0.670588,0.494118}%
\pgfsetfillcolor{currentfill}%
\pgfsetlinewidth{1.505625pt}%
\definecolor{currentstroke}{rgb}{0.168627,0.670588,0.494118}%
\pgfsetstrokecolor{currentstroke}%
\pgfsetdash{}{0pt}%
\pgfsys@defobject{currentmarker}{\pgfqpoint{-0.111111in}{-0.000000in}}{\pgfqpoint{0.111111in}{0.000000in}}{%
\pgfpathmoveto{\pgfqpoint{0.111111in}{-0.000000in}}%
\pgfpathlineto{\pgfqpoint{-0.111111in}{0.000000in}}%
\pgfusepath{stroke,fill}%
}%
\begin{pgfscope}%
\pgfsys@transformshift{1.927120in}{1.661525in}%
\pgfsys@useobject{currentmarker}{}%
\end{pgfscope}%
\end{pgfscope}%
\begin{pgfscope}%
\pgfpathrectangle{\pgfqpoint{0.536697in}{0.161328in}}{\pgfqpoint{3.707795in}{2.291544in}} %
\pgfusepath{clip}%
\pgfsetbuttcap%
\pgfsetroundjoin%
\definecolor{currentfill}{rgb}{0.168627,0.670588,0.494118}%
\pgfsetfillcolor{currentfill}%
\pgfsetlinewidth{1.505625pt}%
\definecolor{currentstroke}{rgb}{0.168627,0.670588,0.494118}%
\pgfsetstrokecolor{currentstroke}%
\pgfsetdash{}{0pt}%
\pgfsys@defobject{currentmarker}{\pgfqpoint{-0.111111in}{-0.000000in}}{\pgfqpoint{0.111111in}{0.000000in}}{%
\pgfpathmoveto{\pgfqpoint{0.111111in}{-0.000000in}}%
\pgfpathlineto{\pgfqpoint{-0.111111in}{0.000000in}}%
\pgfusepath{stroke,fill}%
}%
\begin{pgfscope}%
\pgfsys@transformshift{1.927120in}{1.913595in}%
\pgfsys@useobject{currentmarker}{}%
\end{pgfscope}%
\end{pgfscope}%
\begin{pgfscope}%
\pgfpathrectangle{\pgfqpoint{0.536697in}{0.161328in}}{\pgfqpoint{3.707795in}{2.291544in}} %
\pgfusepath{clip}%
\pgfsetbuttcap%
\pgfsetroundjoin%
\definecolor{currentfill}{rgb}{1.000000,0.494118,0.250980}%
\pgfsetfillcolor{currentfill}%
\pgfsetlinewidth{1.505625pt}%
\definecolor{currentstroke}{rgb}{1.000000,0.494118,0.250980}%
\pgfsetstrokecolor{currentstroke}%
\pgfsetdash{}{0pt}%
\pgfsys@defobject{currentmarker}{\pgfqpoint{-0.111111in}{-0.000000in}}{\pgfqpoint{0.111111in}{0.000000in}}{%
\pgfpathmoveto{\pgfqpoint{0.111111in}{-0.000000in}}%
\pgfpathlineto{\pgfqpoint{-0.111111in}{0.000000in}}%
\pgfusepath{stroke,fill}%
}%
\begin{pgfscope}%
\pgfsys@transformshift{2.854069in}{0.304931in}%
\pgfsys@useobject{currentmarker}{}%
\end{pgfscope}%
\end{pgfscope}%
\begin{pgfscope}%
\pgfpathrectangle{\pgfqpoint{0.536697in}{0.161328in}}{\pgfqpoint{3.707795in}{2.291544in}} %
\pgfusepath{clip}%
\pgfsetbuttcap%
\pgfsetroundjoin%
\definecolor{currentfill}{rgb}{1.000000,0.494118,0.250980}%
\pgfsetfillcolor{currentfill}%
\pgfsetlinewidth{1.505625pt}%
\definecolor{currentstroke}{rgb}{1.000000,0.494118,0.250980}%
\pgfsetstrokecolor{currentstroke}%
\pgfsetdash{}{0pt}%
\pgfsys@defobject{currentmarker}{\pgfqpoint{-0.111111in}{-0.000000in}}{\pgfqpoint{0.111111in}{0.000000in}}{%
\pgfpathmoveto{\pgfqpoint{0.111111in}{-0.000000in}}%
\pgfpathlineto{\pgfqpoint{-0.111111in}{0.000000in}}%
\pgfusepath{stroke,fill}%
}%
\begin{pgfscope}%
\pgfsys@transformshift{2.854069in}{0.397357in}%
\pgfsys@useobject{currentmarker}{}%
\end{pgfscope}%
\end{pgfscope}%
\begin{pgfscope}%
\pgfpathrectangle{\pgfqpoint{0.536697in}{0.161328in}}{\pgfqpoint{3.707795in}{2.291544in}} %
\pgfusepath{clip}%
\pgfsetbuttcap%
\pgfsetroundjoin%
\definecolor{currentfill}{rgb}{1.000000,0.694118,0.250980}%
\pgfsetfillcolor{currentfill}%
\pgfsetlinewidth{1.505625pt}%
\definecolor{currentstroke}{rgb}{1.000000,0.694118,0.250980}%
\pgfsetstrokecolor{currentstroke}%
\pgfsetdash{}{0pt}%
\pgfsys@defobject{currentmarker}{\pgfqpoint{-0.111111in}{-0.000000in}}{\pgfqpoint{0.111111in}{0.000000in}}{%
\pgfpathmoveto{\pgfqpoint{0.111111in}{-0.000000in}}%
\pgfpathlineto{\pgfqpoint{-0.111111in}{0.000000in}}%
\pgfusepath{stroke,fill}%
}%
\begin{pgfscope}%
\pgfsys@transformshift{3.781018in}{0.347706in}%
\pgfsys@useobject{currentmarker}{}%
\end{pgfscope}%
\end{pgfscope}%
\begin{pgfscope}%
\pgfpathrectangle{\pgfqpoint{0.536697in}{0.161328in}}{\pgfqpoint{3.707795in}{2.291544in}} %
\pgfusepath{clip}%
\pgfsetbuttcap%
\pgfsetroundjoin%
\definecolor{currentfill}{rgb}{1.000000,0.694118,0.250980}%
\pgfsetfillcolor{currentfill}%
\pgfsetlinewidth{1.505625pt}%
\definecolor{currentstroke}{rgb}{1.000000,0.694118,0.250980}%
\pgfsetstrokecolor{currentstroke}%
\pgfsetdash{}{0pt}%
\pgfsys@defobject{currentmarker}{\pgfqpoint{-0.111111in}{-0.000000in}}{\pgfqpoint{0.111111in}{0.000000in}}{%
\pgfpathmoveto{\pgfqpoint{0.111111in}{-0.000000in}}%
\pgfpathlineto{\pgfqpoint{-0.111111in}{0.000000in}}%
\pgfusepath{stroke,fill}%
}%
\begin{pgfscope}%
\pgfsys@transformshift{3.781018in}{0.451208in}%
\pgfsys@useobject{currentmarker}{}%
\end{pgfscope}%
\end{pgfscope}%
\begin{pgfscope}%
\pgfpathrectangle{\pgfqpoint{0.536697in}{0.161328in}}{\pgfqpoint{3.707795in}{2.291544in}} %
\pgfusepath{clip}%
\pgfsetroundcap%
\pgfsetroundjoin%
\pgfsetlinewidth{1.756562pt}%
\definecolor{currentstroke}{rgb}{0.627451,0.627451,0.643137}%
\pgfsetstrokecolor{currentstroke}%
\pgfsetdash{}{0pt}%
\pgfpathmoveto{\pgfqpoint{1.000171in}{2.083278in}}%
\pgfpathlineto{\pgfqpoint{1.000171in}{2.227954in}}%
\pgfusepath{stroke}%
\end{pgfscope}%
\begin{pgfscope}%
\pgfpathrectangle{\pgfqpoint{0.536697in}{0.161328in}}{\pgfqpoint{3.707795in}{2.291544in}} %
\pgfusepath{clip}%
\pgfsetroundcap%
\pgfsetroundjoin%
\pgfsetlinewidth{1.756562pt}%
\definecolor{currentstroke}{rgb}{0.627451,0.627451,0.643137}%
\pgfsetstrokecolor{currentstroke}%
\pgfsetdash{}{0pt}%
\pgfpathmoveto{\pgfqpoint{1.000171in}{2.227954in}}%
\pgfpathlineto{\pgfqpoint{2.854069in}{2.227954in}}%
\pgfusepath{stroke}%
\end{pgfscope}%
\begin{pgfscope}%
\pgfpathrectangle{\pgfqpoint{0.536697in}{0.161328in}}{\pgfqpoint{3.707795in}{2.291544in}} %
\pgfusepath{clip}%
\pgfsetroundcap%
\pgfsetroundjoin%
\pgfsetlinewidth{1.756562pt}%
\definecolor{currentstroke}{rgb}{0.627451,0.627451,0.643137}%
\pgfsetstrokecolor{currentstroke}%
\pgfsetdash{}{0pt}%
\pgfpathmoveto{\pgfqpoint{2.854069in}{2.227954in}}%
\pgfpathlineto{\pgfqpoint{2.854069in}{0.570968in}}%
\pgfusepath{stroke}%
\end{pgfscope}%
\begin{pgfscope}%
\pgfsetrectcap%
\pgfsetmiterjoin%
\pgfsetlinewidth{1.254687pt}%
\definecolor{currentstroke}{rgb}{0.150000,0.150000,0.150000}%
\pgfsetstrokecolor{currentstroke}%
\pgfsetdash{}{0pt}%
\pgfpathmoveto{\pgfqpoint{0.536697in}{0.161328in}}%
\pgfpathlineto{\pgfqpoint{0.536697in}{2.452871in}}%
\pgfusepath{stroke}%
\end{pgfscope}%
\begin{pgfscope}%
\pgfsetrectcap%
\pgfsetmiterjoin%
\pgfsetlinewidth{1.254687pt}%
\definecolor{currentstroke}{rgb}{0.150000,0.150000,0.150000}%
\pgfsetstrokecolor{currentstroke}%
\pgfsetdash{}{0pt}%
\pgfpathmoveto{\pgfqpoint{0.536697in}{0.161328in}}%
\pgfpathlineto{\pgfqpoint{4.244492in}{0.161328in}}%
\pgfusepath{stroke}%
\end{pgfscope}%
\begin{pgfscope}%
\definecolor{textcolor}{rgb}{0.150000,0.150000,0.150000}%
\pgfsetstrokecolor{textcolor}%
\pgfsetfillcolor{textcolor}%
\pgftext[x=2.854069in,y=0.451610in,,]{\color{textcolor}\rmfamily\fontsize{15.000000}{18.000000}\selectfont \textbf{*}}%
\end{pgfscope}%
\begin{pgfscope}%
\pgfsetbuttcap%
\pgfsetmiterjoin%
\definecolor{currentfill}{rgb}{0.200000,0.427451,0.650980}%
\pgfsetfillcolor{currentfill}%
\pgfsetlinewidth{1.505625pt}%
\definecolor{currentstroke}{rgb}{0.200000,0.427451,0.650980}%
\pgfsetstrokecolor{currentstroke}%
\pgfsetdash{}{0pt}%
\pgfpathmoveto{\pgfqpoint{4.344492in}{2.269558in}}%
\pgfpathlineto{\pgfqpoint{4.455603in}{2.269558in}}%
\pgfpathlineto{\pgfqpoint{4.455603in}{2.347336in}}%
\pgfpathlineto{\pgfqpoint{4.344492in}{2.347336in}}%
\pgfpathclose%
\pgfusepath{stroke,fill}%
\end{pgfscope}%
\begin{pgfscope}%
\definecolor{textcolor}{rgb}{0.150000,0.150000,0.150000}%
\pgfsetstrokecolor{textcolor}%
\pgfsetfillcolor{textcolor}%
\pgftext[x=4.544492in,y=2.269558in,left,base]{\color{textcolor}\rmfamily\fontsize{8.000000}{9.600000}\selectfont WT + Vehicle (11)}%
\end{pgfscope}%
\begin{pgfscope}%
\pgfsetbuttcap%
\pgfsetmiterjoin%
\definecolor{currentfill}{rgb}{0.168627,0.670588,0.494118}%
\pgfsetfillcolor{currentfill}%
\pgfsetlinewidth{1.505625pt}%
\definecolor{currentstroke}{rgb}{0.168627,0.670588,0.494118}%
\pgfsetstrokecolor{currentstroke}%
\pgfsetdash{}{0pt}%
\pgfpathmoveto{\pgfqpoint{4.344492in}{2.102918in}}%
\pgfpathlineto{\pgfqpoint{4.455603in}{2.102918in}}%
\pgfpathlineto{\pgfqpoint{4.455603in}{2.180696in}}%
\pgfpathlineto{\pgfqpoint{4.344492in}{2.180696in}}%
\pgfpathclose%
\pgfusepath{stroke,fill}%
\end{pgfscope}%
\begin{pgfscope}%
\definecolor{textcolor}{rgb}{0.150000,0.150000,0.150000}%
\pgfsetstrokecolor{textcolor}%
\pgfsetfillcolor{textcolor}%
\pgftext[x=4.544492in,y=2.102918in,left,base]{\color{textcolor}\rmfamily\fontsize{8.000000}{9.600000}\selectfont WT + TAT-GluA2\textsubscript{3Y} (11)}%
\end{pgfscope}%
\begin{pgfscope}%
\pgfsetbuttcap%
\pgfsetmiterjoin%
\definecolor{currentfill}{rgb}{1.000000,0.494118,0.250980}%
\pgfsetfillcolor{currentfill}%
\pgfsetlinewidth{1.505625pt}%
\definecolor{currentstroke}{rgb}{1.000000,0.494118,0.250980}%
\pgfsetstrokecolor{currentstroke}%
\pgfsetdash{}{0pt}%
\pgfpathmoveto{\pgfqpoint{4.344492in}{1.936279in}}%
\pgfpathlineto{\pgfqpoint{4.455603in}{1.936279in}}%
\pgfpathlineto{\pgfqpoint{4.455603in}{2.014057in}}%
\pgfpathlineto{\pgfqpoint{4.344492in}{2.014057in}}%
\pgfpathclose%
\pgfusepath{stroke,fill}%
\end{pgfscope}%
\begin{pgfscope}%
\definecolor{textcolor}{rgb}{0.150000,0.150000,0.150000}%
\pgfsetstrokecolor{textcolor}%
\pgfsetfillcolor{textcolor}%
\pgftext[x=4.544492in,y=1.936279in,left,base]{\color{textcolor}\rmfamily\fontsize{8.000000}{9.600000}\selectfont Tg + Vehicle (4)}%
\end{pgfscope}%
\begin{pgfscope}%
\pgfsetbuttcap%
\pgfsetmiterjoin%
\definecolor{currentfill}{rgb}{1.000000,0.694118,0.250980}%
\pgfsetfillcolor{currentfill}%
\pgfsetlinewidth{1.505625pt}%
\definecolor{currentstroke}{rgb}{1.000000,0.694118,0.250980}%
\pgfsetstrokecolor{currentstroke}%
\pgfsetdash{}{0pt}%
\pgfpathmoveto{\pgfqpoint{4.344492in}{1.769639in}}%
\pgfpathlineto{\pgfqpoint{4.455603in}{1.769639in}}%
\pgfpathlineto{\pgfqpoint{4.455603in}{1.847417in}}%
\pgfpathlineto{\pgfqpoint{4.344492in}{1.847417in}}%
\pgfpathclose%
\pgfusepath{stroke,fill}%
\end{pgfscope}%
\begin{pgfscope}%
\definecolor{textcolor}{rgb}{0.150000,0.150000,0.150000}%
\pgfsetstrokecolor{textcolor}%
\pgfsetfillcolor{textcolor}%
\pgftext[x=4.544492in,y=1.769639in,left,base]{\color{textcolor}\rmfamily\fontsize{8.000000}{9.600000}\selectfont Tg + TAT-GluA2\textsubscript{3Y} (5)}%
\end{pgfscope}%
\end{pgfpicture}%
\makeatother%
\endgroup%

        \caption{\label{f.ad.reminder2.res}}
    \end{subfigure}
    \caption[\tglu{} treatment does not rescue memory recall without reminder.]{\tglu{} does not rescue memory recall without reminder. \gls{wt} and \gls{tg} mice were contextual fear conditioned, given treatment \SI{3}{\day} later in their home cages and tested on the following day. \tglu{} treatment without reminder is unable to rescue recall deficit in \gls{tg} mice. \label{f.ad.reminder2}}
\end{figure}

\section{Discussion}


In the current project, we took advantage of our custom built miniature microscope to record \gls{ca1} cells in WT and a mouse model of early \gls{ad}, TgCRND8 mice, during contextual fear conditioning. We found a significant memory deficit in TgCRND8 mice. Correlated with the memory deficit, we also discovered that the TgCRND8 mice have hyperactive \gls{ca1} neurons. Compared to the WT mice, the \gls{ca1} neurons in TgCRND8 mice contain reduced information content about the mouse's behavioural state of memory recall. This reduction of information content is independent of spatial encoding and the strength of memory (as measured by proportion of time the mouse spent freezing), suggesting that the reduction in information content is related to a specific memory deficit in the TgCRND8 mice.

To investigate ensemble encoding of the fear memory, we trained machine learning classifiers to predict, at each time point, the behavioural state of memory recall using the calcium activity of the neurons. To differentiate the cellular and network activity, we used \gls{nbc} and \gls{gsvm} and compared results obtained from both classifiers. \Gls{nbc} is only able to detect signals when each neuron is treated as independent from others, while \gls{gsvm}, being a general classifier, is able to utilize all available information, including those both at the individual cell level, and also information stored in the higher order relationship between the activity of cells. 

The first finding from the prediction accuracy of the classifiers is that the classifiers are able to decode at similar accuracy level between TgCRND8 and WT mice. This is in contrast of our mutual information content result, which suggests that individual cells in the \gls{tg} mice have a deficit in encoding memory recall. These two results suggest that although the physiology of individual cells are significantly impacted by the TgCRND8 phenotype, neurons as a network can largely compensate and perform information encoding. This is also supported by the finding that the \gls{gsvm} performs significantly better than the \gls{nbc}, suggesting that a significant portion of the information for memory recall is stored in the cooperative activity of neurons. 

Previous studies have established that the neural activity in \gls{ca1} is both necessary and sufficient for contextual fear conditioning. Early studies has shown that inhibition or lesion \gls{ca1} hippocampus, either before or after training, is detrimental for contextual fear conditioning \citep{maren01}. The sufficiency of \gls{ca1} activity in contextual fear conditioning has recently been demonstrated \citep{ryan15, roy16}, where the authors labeled \gls{ca1} neurons active during contextual fear conditioning, and shows that reactivating the same set of neurons is able to reactivate the contextual fear memory. These results suggest that a neural signature for the fear memory is present in the \gls{ca1} neural activity, and therefore can be detected by a universal classifier. 

However while the \gls{nbc} and \gls{gsvm} are able to detect neural signature of the fear memory, the neural signature the classifiers learned may not be exclusively about fear memory signature. Neural activity which correlates to physiological and other undetected behavioural processes, as long as it is predictive to the freezing behaviour of the mice, may also be learn by the classifiers. Inclusion of these extra patterns does not affect our conclusion, since these patterns are still results of fear memory recall, and just like freezing behaviour, can be regarded as part of the neural signature forthe \textit{expression} of fear memory. 

To investigate how the degradation of cell firing in \gls{ca1} neurons of TgCRND8 mice contributes to their memory deficit, we then investigated whether the important circuit function of hippocampus, pattern completion, which is necessary for memory recall, is affected in the TgCRND8 mice. The Marr model and more recent empirical evidence \citep{rolls13, neunuebel14} suggest that while \gls{ca3} is the major site in hippocampus where pattern completion occur, however the input--output difference in \gls{ca1} is relative linear \citep{neunuebel14, knierim16}. Therefore, the pattern completion process should also be reflected in \gls{ca1} activity, and can be detected in our recording.

To observe the pattern completion process, we have aligned the classifier performances to the time when mice begin to freeze. In \gls{nbc}, we observed a decrease in prediction accuracy just before the mice start to freeze. The decrease of prediction accuracy is a result of the classifier misclassifying non-freezing to freezing. This shows that the classifier starts to predict the mice freezing just before the behavioural change. There is no between-group difference in the \gls{nbc} prediction accuracy, however while the WT group shows similar prediction precedence in \gls{gsvm}, this precedence is missing for \gls{gsvm} prediction in the TgCRND8 mice. 

It is worth noting that, here we used trained classifiers as detectors for the freezing neural signature, and our goal is to investigate whether neural signature of freezing appears before behaviour onset. Although we have found that the machine learning classifiers predict freezing ahead of time, we have not investigated whether the neural signature lead any other behavioural changes such as exiting freezing state. This is primarily because the mice showed a range of different behaviour upon the end of freezing period, including moving, grooming, rearing and sniffing, and these different behaviours may have different dynamics in the \gls{ca1} neural activity. However, if there is a global precedence of \gls{ca1} activity on behaviour change, it can be detected by training and testing the machine learning classifier with different time lags. The distribution of classifier accuracy across different time lags can reveal information of any global precedence of neural activity signature for behaviour. 

Since the classifiers are trained to recognize specific neural activity patterns for the fear memory, the precedence of classifier prediction to behaviour suggest that the \gls{ca1} activity pattern for the fear memory leads behaviour. The gradual rise of the classifier's prediction before freezing also suggests that patterns of fear memory arise dynamically: the fear memory pattern starts as a partial pattern which can only barely detected by the classifier, and this gradually develops into a full pattern which biases the prediction of the classifier. 

The difference of temporal dynamics of the ``pattern completion'' process also hints at the underlying dynamic mechanism of the process. Since the \gls{nbc} is only able to detect activity patterns at the individual cell level, the rise of fear prediction in \gls{nbc} suggests that individual cells start show the firing pattern of the fear memory. On the other hand, the \gls{gsvm} detects network pattern, which is composed of synchronized activity of individual cells. We have found the \gls{nbc} prediction precedes the \gls{gsvm} prediction, suggesting that the fear memory pattern in the \gls{ca1} starts as an incomplete pattern composed of unsynchronized individual cell activity, and as the cellular pattern completes, it leads to synchronized network activity, and is detected by the \gls{gsvm} later on. 

The difference between the TgCRND8 mice and WT mice can also be interpreted in this framework. The TgCRND8 mice display similar dynamics in \gls{nbc}, suggesting that the cellular signals are not affected. However, the absence of a dynamic pattern completion process in the \gls{gsvm} suggests that while individual cells are able to respond to the initiation of a fear memory, the cells are unable to coordinate to form a network pattern. 

The inability of the TgCRND8 mice to coordinate and form a network pattern is better shown in Figure~\ref{f.ad.cls-distance}, where we investigated the relative distance of network state to the \gls{gsvm} classifier boundary. Given the high accuracy of the \gls{gsvm} prediction, the \gls{gsvm} classifier boundary can approximate the actual behaviour state transition boundary. We found that TgCRND8 mice approach the fear memory state boundary later, and also more shallowly than \gls{wt} mice, further confirming a deficit in the pattern completion process. 

Interestingly, treating the TgCRND8 mice with \tglu{} during memory formation is able to rescue the cellular, network, and behavioural deficits we have found. TgCRND8 mice with \tglu{} treatment are able to show normal expression of fear memory. The \tglu{} treatment is able to decrease the cell excitability of TgCRND8 mice, and allow the activity of the cells to be as informative as \gls{wt} mice about the behavioural state of the memory recall. Moreover, the \tglu{} treatment is also able to rescue the pattern completion deficit. 

Some of the effects of the \tglu{} treatment are also observed on \gls{wt} mice. While the \tglu{} treatment has no effect on the overall cell activity in \gls{wt} mice, when we control movement of the mice and only compare cell activity during freezing, we found \tglu{} decreases cell activity even in \gls{wt} mice. Moreover, \tglu{} treatment in \gls{wt} mice also leads the classifier to predict freezing earlier than vehicle-treated \gls{wt} mice, which is evidence for an enhancement of the pattern completion process. 

The deficits seen in TgCRND8 mice could be the result of a deficit in memory, or a pathologically enhanced forgetting. To test this hypothesis, we trained \gls{wt} and TgCRND8 mice, and only treated the mice with vehicle or \tglu{} three days later when mice were briefly exposed to a reminder. The memory was tested on the following day. We found that while TgCRND8 mice show a memory deficit, \tglu{} treatment during reminder is able to completely rescue the memory deficit. However interestingly, the presence of the reminder is crucial for the rescuing effect of \tglu, as \tglu{} treatment when mice are in their home cage does not have an effect on memory recall. 

These two results suggest that first, TgCRND8 mice are still able to form a stable representation of the contextual fear memory which last for at least 3 days. Since if this were not the case, \tglu{} treatment with reminder should not have any rescuing effect, as the memory would not be present at that time point. Therefore, these results suggest that the memory deficit in the TgCRND8 mice is not a result of pathologically enhanced forgetting, but that they were unable to recall the encoded memory representation.

This conclusion suggests two possibilities. First, it is possible that the TgCRND8 mice has a deficit in memory encoding. These mice may only be able to encode a memory representation which is too weak to be recalled. \tglu{} treatment during training or reminder strengthens this memory representation, and therefore rescues the deficit. A second possibility is that the TgCRND8 mice has a deficit in memory recall. In this case, even if they are able to encode memory unaffected by \gls{ad}, they may not be able to recall them, potentially due to a deficit in the pattern completion process. In this case, \tglu{} treatment during training and reminder sessions makes the memory representation even stronger, and this additional strength of memory representation in \gls{ca1} compensates for a lack of pattern completion process. 

An acute mouse model of \gls{ad} will be required to tease these two possibilities out, for example, by viral expression or direct infusion of \abeta{} in \gls{ca1}. If the mice can recall any memory formed before \abeta{} infusion, it suggests that the memory deficit in \gls{ad} is due to memory encoding deficit. On the other hand if there is a deficit in recalling a normally encoded memory, it suggests that the disruption of memory recall process leads to the memory deficit in \gls{ad}.

In conclusion, we have found that TgCRND8 mice are able to form a contextual fear memory, however are unable to recall it. This deficit is accompanied by a hyperactive \gls{ca1} and decreased information content in \gls{ca1} cells. While behaviour information can still be decoded using activity of a neural population, cells in TgCRND8 mice are unable to pattern complete a contextual fear memory. This deficit may explain the recall deficit in these mice. Moreover, all deficits can be rescued by \tglu{} treatment during training. Given the effect of \tglu{} on synaptic strengthening, the current study is able to link synaptic functions to neural circuits in \gls{ca1}, and ultimately behaviour. While the mechanism by which synaptic function creates changes in \gls{ca1} neural circuit function is still unclear, this study creates a potential link of between circuit function and analysis at the molecular and behavioural scale. It further demonstrates the importance of understanding circuit function in the study of \gls{ad}, and suggests a potential novel treatment target for \gls{ad} at a neural circuit level. 

