\Gls{ad} is characterized by progressive memory loss. While \abeta is widely implicated in the pathophysiology of \gls{ad}, how impairment of neural function on the molecular and cellular level creates cognitive deficits in \gls{ad} is still unknown. Recent development of imaging techniques provides methods to simultaneously imaging neural ensembles in freely behaving mice. 

To investigate the circuitry deficit in the \gls{ad}, I first developed a miniature flourescent microscope for \textit{in vivo} calcium imaging. We have showed that the mini-microscope is able to image hundreds of neurons in hippocampal \gls{ca1} simultaneously in freely behaving mice. Moreover, we have shown that the mini-microscope is able to imaging deep brain regions such as \gls{la}, and in two different colour channels. 

We then used the mini-microscope to image in a mouse model of early \gls{ad}, TgCRND8, and hypothesized that an impairment of circuitry function underlies the cognitive deficit in these mice, and that rescuing the behavioural deficit by enhancing \gls{ltp} is able to restore the circuitry function of the \gls{ad} mice. We trained the \gls{ad} and \gls{wt} mice in contextual fear conditioning while imaging dorsal hippocampal \gls{ca1} neurons using the mini-microscope. We found while the \gls{ad} mice shows a deficit during memeory test, the memory deficit is present as a significantly shortened freezing bouts, but without a significant decrease of the number of freezing bouts. Mini-microscope recording data have shown that the \gls{ad} mice have shown significant hyperexcitability and less information content about memory recall. Moreover, we have found the pattern completion process, which is important for recalling hippocampal-dependent memory, is impaired in the \gls{ad} mice, represented as a lack of coordination between \gls{ca1} neuron acitivities. 

Moreover, we found that blocking \gls{ampa} endocytosis using \tglu is able to rescue both the behaviour and circuitry deficit in the \gls{ad} mice. Moreover, we found that the memory deficit in the \gls{ad} mice is a deficit of memory recall, since \tglu treatment during a reminder one day ofter training is able to rescue the memory deficit, suggesting \gls{ad} mice can still retain the memory one day after training. 

In this thesis, I have shown that new development in calcium imaging is able to help study the circuitry function of the brain. In the \gls{ad} project, the results suggest that abberation of neural circuitry function is important in producing the cognictive symptoms in \gls{ad}. Our findings also provide a link from cellular deficit in \gls{ad}, to neural circuitry and finally behavioural deficit. 


