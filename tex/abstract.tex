\Gls{ad} is characterized by progressive memory loss. While \abeta{} is widely implicated in the pathophysiology of AD, the mechanisms by which \abeta{} impairs memory, especially in the early stages of AD is unknown. The results of many experiments suggest that in its early stages, AD is a disease of synaptic dysfunction. However, how deficits in synapses translate into circuit dysfunction is entirely unknown. To address the circuit-level dysfunction in AD, I developed a miniature fluorescent microscope that could be mounted on the heads of behaving mice. This mini-microscope is capable of imaging hundreds of neurons in the hippocampus \gls{ca1} region of mice. Moreover, this mini-microscope can also image deeper brain structures in two color channels. 
To investigate circuit dysfunction in \gls{ad}, we used transgenic mice designed to model \gls{ad} in humans. These TgCRND8 mice (referred to here as \gls{ad} mice) express human genes that result in familial \gls{ad} in humans, show high levels of \abeta{} and have pronounced memory deficits compared to their \gls{wt} littermate controls. To examine memory, we trained \gls{ad} and \gls{wt} littermate mice in contextual fear conditioning, in which an aversive stimulus is paired with a specific context. As expected, the \gls{ad} mice showed lower freezing that \gls{wt} mice when replaced back in that context. We noted, though, that the number of freezing bouts did not differ between the groups, but the length of each freezing bout was lower in the \gls{ad} mice. To examine the circuit dysfunction in \gls{ad} mice, we imaged neuronal activity during training and testing. CA1 cells in \gls{ad} mice showed greater excitability than in \gls{wt} mice. Furthermore, cells in \gls{ad} mice showed less information content related to memory recall than \gls{wt} mice. Finally, the cells from \gls{ad} mice showed less coordinated activity than cells from \gls{wt} mice, suggesting that \gls{ad} mice are impaired in pattern completion, a process important in recalling hippocampal-dependent memories.
In addition to the circuit dysfunction, we also showed that \gls{ad} mice have excessive endocytosis of GluA2-containing \gls{ampar}, thus accounting for the synaptic dysfunction. This finding is consistent with results from previous work in cultured neurons. We found that disrupting GluA2 endocytosis using an interfering peptide (\tglu{}) before training was sufficient to rescue both the memory deficit and the circuitry deficit in \gls{ad} mice. Finally, we showed that disrupting GluA2 endocytosis during a reminder one-day following training also restored memory. These results suggest that \gls{ad} mice still retain some aspect of the memory one day after training, and that strengthening this weakened memory allowed mice to recall it.
In this thesis, I developed and used a novel calcium imaging method to investigate the circuit dysfunction in a mouse model of \gls{ad}. Using this tool, I revealed the important link between the synaptic and circuit deficit in \gls{ad}. The findings from this thesis may have important implications for designing the next generation of tools and therapies for identifying and treating \gls{ad}.
