\Gls{ad} is characterized by progressive memory loss. Results from many experiments suggest \gls{ad} begins as a disease of synaptic dysfunction. However, how synaptic dysfunction translates into circuit dysfunction and, ultimately, memory impairment is entirely unknown. To address the circuit-level dysfunction in \gls{ad}, I developed a miniature fluorescent microscope that can be mounted on the head of a behaving mouse. We showed this mini-microscope is capable of imaging hundreds of neurons in the \gls{ca1} region of the hippocampus, and with minor modifications, also in deeper brain structures and in two colour channels. 

To investigate the circuit dysfunction in \gls{ad}, we used transgenic mice designed to model \gls{ad} in humans. These TgCRND8 mice (referred to as \gls{ad} mice) express human familial \gls{ad} genes and similarly develop memory impairments. As expected, we showed that \gls{ad} mice have robust memory deficits in a contextual fear conditioning paradigm, in which mice were trained to associate an aversive stimulus with a context. Imaging data from our mini-microscope showed that compared to control mice, \gls{ca1} cells in \gls{ad} mice were hyperexcitable and contained less information related to memory recall. Moreover, cells from \gls{ad} mice showed a deficit in coordinated activity, suggesting that these mice are impaired in pattern completion, a process important in recalling hippocampal-dependent memories.

Previous studies suggest that \gls{ad} mice have excessive endocytosis of GluA2-containing AMPAR, which accounts for their synaptic dysfunction. We found disrupting GluA2 endocytosis using an interfering peptide (\tglu{}) before training was sufficient to rescue the memory and circuitry deficits in \gls{ad} mice. Moreover, we showed disrupting GluA2 endocytosis during a reminder three days after training also restored memory. These results argue that \gls{ad} mice still retain some aspect of the memory days after training, and that strengthening this weakened memory allowed mice to recall it.

In conclusion, I developed and used a novel calcium imaging method to investigate the circuit dysfunction in a mouse model of \gls{ad}. Using this tool, I revealed an important link between the synaptic and circuit deficits in \gls{ad} mice. These findings may have important implications for designing the next generation of tools and therapies for \gls{ad}.
